\begin{tabular}{|p{1cm}|p{6.5cm}|p{6.5cm}|}

\hline
1.1.3 & njn guardar \textbf{ syn la gera de dios conviene de cada vn omne } e mayormente prinçipe o Rey & absque diuina gratia obseruari non possunt , \textbf{ decet quemlibet hominem , } et maxime regiam maiestatem \\\hline
1.1.9 & e muestre algua bondat de fuera \textbf{ por la qual cosa commo al Rey conuenga ser todo diuinal e semeiante a dios } si non es cosa conuenible & quod exterius bona praetendat . \textbf{ Quare cum Regem deceat } esse totum diuinum , \\\hline
1.2.10 & assi conmosi quisiere auer mas de aquellos bienes \textbf{ de quanto le conuiene auer } por esta razon viene danno alos otros çibdadanos & ut quod velit habere plus de iis , \textbf{ quam eum deceat : } ex hoc infertur nocumentum aliis ciuibus : \\\hline
1.2.18 & que menos dan de quanto les conuiene ᷤ dar \textbf{ Et menos fazen de quanto les conuiene de fazer . } Et desto puede bien paresçer & Semper ergo cogitare debent , \textbf{ quod minora faciunt , | quam deceat . } Ex hoc autem apparere potest \\\hline
1.2.18 & que si espendiere \textbf{ do non le conuiene espender } Mas los Reyes e los prinçipes de suranse & ubi oportet , \textbf{ quam si expendat | ubi non oportet . } Deuiant autem a liberalitate Reges , \\\hline
1.3.6 & por ende en este primero libro \textbf{ conuiene de tractar delas costunbres de lons Reyes } uniuersalmente & ut dicitur 1 Physicorum , \textbf{ deo in hoc primo de moribus Regum oportet } pertransire uniuersaliter typo : \\\hline
1.3.7 & Ca nos podemos natanlmente querer mal a todos los ladrones . \textbf{ pero non conuiene de temer quetsteza se aconpanne a esta mal querençia . } ¶ La septima diferençia es & uniuersaliter omnes fures : \textbf{ non tamen oportet , | quod tristitia committetur huiusmodi odium . } Septima differentia est : \\\hline
1.4.5 & Onde el philosofo dize en el quarto libro de la rectorica \textbf{ que conuiene de ser los nobles magranimos } e de grandes coraçones e magnificos & Unde Philos’ 4 Eth’ ait , \textbf{ quod magnanimos et magnificos decet } esse nobiles et gloriosos . \\\hline
2.1.4 & todas las cosas neçessarias ala uida \textbf{ conuiene de dar comunidat ala çibdat } sobre la comunidat deluarrio . & omnia necessaria ad vitam , \textbf{ praeter communitatem vici | oportuit } dare communitatem ciuitatis . \\\hline
2.2.7 & e en las sçiençias liberales \textbf{ quanto mas les conuiene de ser mas entendudos } e mas sabios que los otros & insudare literalibus disciplinis , \textbf{ quanto decet eos intelligentiores et prudentiores esse , } ut possint naturaliter dominari . \\\hline
2.2.21 & ostrado que non conuiene alas moças de andar uagarosas a quande e allende \textbf{ nin les conuiene de beuir ociosas } finca que agora lo terçero mostremos & quod non decet puellas esse vagabundas , \textbf{ nec decet eas viuere otiose : } restat ut nunc tertio ostendamus , \\\hline
2.3.16 & en el gouernamiento delas casas de los Reyes \textbf{ En las quales por la grandeza de los offiçios conuiene de } acomne dar vn ofiçio a muchos seruientes & in gubernatione domorum regalium , \textbf{ ubi propter magnitudinem officiorum oportet } idem ministerium committi ministris multis , \\\hline
2.3.20 & Mas podemos mostrar por dos razones \textbf{ que non conuiene de fablar mucho en las mesas de los Reyes } nin de los prinçipes & Possumus autem duplici via ostendere , \textbf{ quod non decet | in mensis Regum et Principum } et uniuersaliter omnium nobilium \\\hline
3.1.8 & assi commo de andar e de tanner e de oyr e deuer . \textbf{ por ende conuiene de dar . } y departidos mienbros & ut ambulatione , tactu , visione , \textbf{ et auditus ideo oportet } ibi dare diuersa membra exercentia diuersos actus : \\\hline
3.1.12 & Et por ende por que los lidiadores non se enflaquezcan en las batallas \textbf{ conuiene de echar dela batalla } e dela fazienda alos de flaco & ne igitur reddantur bellantes pusillanimes , \textbf{ quos constat esse timidos oportet } ab exercitu expelli . \\\hline
3.1.14 & que sienpre los çibdadanos \textbf{ non les conuenga de lidiar } por defendimiento de su tierra & ab artificibus et ab aliis ciuibus , \textbf{ quod ciues alii pro defensione patriae bellare non oporteat } melius est ergo dicere in ciuitate \\\hline
3.1.17 & por que podrian los çibdadanos auer tan pocas possessiones \textbf{ que les conuenia de beuir } assi es casamente & possent enim ciues adeo modicas possessiones habere , \textbf{ quod oporteret eos ita parce viuere } quod opera liberalitatis de facili exercere non valerent . \\\hline
3.2.1 & entp̃o dela paz \textbf{ por las leyes conuiene de fazer tractado destas quatro cosas sobredichͣs en este gouernamiento ¶ } La segunda razon para prouar & Quare si considerentur quae requiruntur ad hoc quod tempore pacis per leges bene gubernetur ciuitas , \textbf{ oportet in huiusmodi regimine | de praedictis quatuor considerationem facere . } Secunda via ad inuestigandum hoc idem sumitur ex fine \\\hline
3.2.13 & segund que dize el philosofo \textbf{ ca conujene de dar a entender } que estos tales non han cuydado de saluar su vida ¶ & ( ut ait Philos’ ) \textbf{ sunt paucissimi numero , | supponi oportet } eos nihil curare , \\\hline
3.2.17 & por la quel cosa commo muchs mas cosas ayan prouadas \textbf{ que vno solo conuiene de llamar otros } para los negoçios . por que por el conseio dellos pueda ser escogida la meior carrera & Quare cum plures plura experti sint , \textbf{ quam unus solus : | decet ad huiusmodi negocia alios aduocare , } ut per eorum consilium possit \\\hline
3.2.21 & La primera seqma par aquello que tales palabras han de to terçeres desegualar eliez \textbf{ el qual conuiene de ser } assi commo regla derecha en & obligare habent iudicem , \textbf{ quem esse oportet } quasi regulam in iudicando . \\\hline
3.2.26 & en el quarto libro delas politicas \textbf{ que non conuiene de apropar las comunidades } delas çibdades alas leyes . & Ideo dicitur 4 Politicorum \textbf{ quod non oportet } adaptare politias legibus , \\\hline
3.3.8 & e fazer muy apriessa . \textbf{ Mas conuiene de poner algunos maestros } para costruyr los castiellos & debet celeriter castra construere . \textbf{ Oportet autem semper construendis castris , } et faciendis fossis aliquos magistros praestitui , \\\hline
3.3.23 & de la batalla de las naues . \textbf{ enpero non conuiene de nos } de tener çerca esto tanto . & volumus aliqua de nauali bello : \textbf{ non tamen oportet } circa hoc tantum insistere , \\\hline

\end{tabular}
