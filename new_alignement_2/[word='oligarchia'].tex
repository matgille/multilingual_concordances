\begin{tabular}{|p{1cm}|p{6.5cm}|p{6.5cm}|}

\hline
3.2.2 & et politia sunt principatus boni : tyrannides , \textbf{ oligarchia , } et democratia sunt mali . Docet enim idem ibidem discernere bonum principatum a malo . & e la poliçia que quiere dezer pueblo bien enssenoreante son bueons prinçipados . La thirama que quiere dezer sennorio malo \textbf{ e la obligaçia que quiere dezer sennorio duro . | Et la democraçia que quiere dez maldat del pueblo } enssennoreante son malos prinçipados . Et alli muestra el pho departir el buen prinçipado del malo . \\\hline
3.2.12 & est rectus principatus , et vocatur aristocratia siue principatus bonorum . Si vero dominentur non quia boni , \textbf{ sed quia diuites , est peruersus et vocatur oligarchia . } Sed si dominatur totus populus et intendat bonum omnium tam insignium quam aliorum , est principatus rectus , et vocatur regimen populi . Si vero populus tyrannizet & Mas si enssennorear en pocos non por que son buenos \textbf{ mas por que son ricos es llamado obligarçia | que quiere dezer señorio tuerto . } Mas quando enssennore a todo el pueblo si entienda al bien comun de todos \\\hline

\end{tabular}
