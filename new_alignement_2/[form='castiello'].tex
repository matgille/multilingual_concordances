\begin{tabular}{|p{1cm}|p{6.5cm}|p{6.5cm}|}

\hline
1.2.6 & Verbigera . ¶ \textbf{ Si nos quesieremos tomar algun castiello } segunt manera de guerra . Pmeramente deuemos buscar carreras & secundum inuenta et iudicata : \textbf{ ut si vellemus } secundum opera bellica castrum aliquod capere . \\\hline
1.2.6 & escodrinnar maneras e sotilezas \textbf{ por que podamos tomar aquel castiello ¶ } Lo segundo deuemos iudgar de aquellas carreras e maneras que fablamos & et cogitandi essent modi , \textbf{ per quos castrum istud capi posset . } Secundo iudicandum esset \\\hline
3.3.16 & si por auentura fueren çercados de los enemigos . \textbf{ Et tal manera de batalla en la qual defiende cada vno su villa o su castiello } es dicha batalla defenssiua . & si contingat eas ab hostibus impugnari . \textbf{ Tale genus pugnae | quo quis defendit munitiones et castra , } dicitur defensiuum . \\\hline
3.3.16 & Otrossi si por fanbre se ha de tomar la çibdat cercada \textbf{ o el castiello } meior es de cercar la en el tienpo del estiuo & Rursus si per famen est castrum , \textbf{ vel ciuitas obsessa obtinenda , } melius est obsessionem facere aestiuo tempore , \\\hline
3.3.17 & deuen fincar las tiendas \textbf{ e el real alueñe de la çibdat o del castiello cercado } quanto podrie lançar la vallesta o el dardo & longe a munitione obsessa saltem \textbf{ per ictum teli } vel iaculi debent castrametari , \\\hline
3.3.17 & por las cueuas soterrañas \textbf{ assi que por ellas puedan entrar a la çibdat o al castiello . } Et estas cosas todas deuense fazer muy encubiertamente & diuertendo vias subterraneas , \textbf{ ut per eas possit | haberi ingressus ad ciuitatem et castrum : } quae omnia latenter fieri possunt \\\hline
3.3.17 & Et por aquellas cosas soterrañas \textbf{ ayan entrada al castiello o a la çibdat } e por la entrada que se faze & et munitiones suffossae : \textbf{ et per vias subterraneas fiat ingressus ad castrum , vel ad ciuitatem : } et per aditum factum ex muris cadentibus reliqui obsidentes ingrediantur castrum , \\\hline
3.3.17 & en la çibdato \textbf{ en el castiello çercado } e assi podran ganar aquellas fortalezas . & et per aditum factum ex muris cadentibus reliqui obsidentes ingrediantur castrum , \textbf{ vel ciuitatem obsessam : } et sic poterunt obtinere illam . \\\hline
3.3.18 & o por castiellos que se pueden enpuxar \textbf{ fasta las menas del castiello o de la çibdat cercada . } Conuiene de vsar de tales armadijas o de tales armamientos & vel per aedificia propulsa usque ad moenia castri , \textbf{ vel ciuitatis obsessae , } oportet talibus uti argumentis \\\hline
3.3.18 & Et pues que assi es aquel \textbf{ que çerca algun castiello o alguna çibdat } si la quiere tomar & Illae igitur \textbf{ qui obsidet castrum aut ciuitatem aliquam , } si vult eam impugnare \\\hline
3.3.18 & o en todas aquellas maneras de lançar \textbf{ que dichas son o en algunas o en alguna dellas podra acomter el castiello o la çibdat cercada . } Ca si conplidamente sopiere todas estas maneras de engennios & vel omnibus praefatis modis proiiciendi , \textbf{ vel aliquibus | sine aliqua praedictarum machinarum , castrum , vel ciuitatem obsessam poterit impugnare . } Si enim plena notitia habeatur de machinis , \\\hline
3.3.19 & por los artifiçios de madera enpuxados \textbf{ e allegados a los muros del castiello o de la çibdat . } Et estos artifiçios pueden se adozir a quatro maneras . & per aedificia impulsa ad muros , \textbf{ vel ad moenia castri , | vel ciuitatis obsessae . } Huiusmodi autem aedificia \\\hline
3.3.19 & en tres maneras puede acometer la fortaleza . \textbf{ ca que el castiello assi fecho } para conbatir la fortaleza & tripliciter impugnanda est munitio . \textbf{ Nam in castro sic aedificato } ad munitionem impugnandam , \\\hline
3.3.19 & deuemos penssar tres cosas \textbf{ Conuiene de saber la parte del castiello } mas alta que los muros & est tria considerare , \textbf{ videlicet partem superiorum excedentem muros ; } et curriculas munitionis capiendae partem quasi mediam , \\\hline
3.3.19 & que traen o enpuxan los castiellos de madera . \textbf{ Et quando aquel castiello o castiellos se llegaren } quanto deuen a los muros de la fortaleza los cercados & vel impellentes castrum . \textbf{ Cum ergo castrum illud appropinquauit } quantum debuit \\\hline
3.3.21 & osi se pueden traer \textbf{ e non son muy prouechosas al castiello } o a la çibdat cercada & ( vel si deferretur \textbf{ non multum esset utile castro } vel ciuitati obsessae ) \\\hline
3.3.21 & entre todas las otras cosas de que deuen basteçer la fortaleza \textbf{ e el castiello deuenla basteçer mayormente de mijo . Ca el mijo menos se podresçe } e mas dura que tedos los otros granos . & vel castrum obsessum milio : \textbf{ nam milium inter cetera minus putrefit , | et plus durare perhibetur . } Copia etiam carnium salitarum non est praetermittenda . \\\hline
3.3.21 & e presta en la fortaleza çercada . \textbf{ Lo segundo en basteciendo el castiello o la cibdat } que teme de ser cercada & eo quod ad multa sit utilis . \textbf{ Secundo in muniendo castrum } vel ciuitatem aliquam obsidendam , \\\hline
3.3.21 & por que la fortaleza cercada non se pueda vençer por batalla . \textbf{ Et pues que assi es deuense traer a la çibdat o al castiello } que teme de ser cercado piedra sufre pez e oleo e rasina en grand anbondança & ne per pugnam obsessa munitio deuincatur . \textbf{ Debent ergo ad ciuitatem , | vel ad castrum obsessum deportari } in magna copia sulphur , \\\hline
3.3.22 & que se puede de ligero cauar e estonçe es de \textbf{ enfortaleçer el castiello o la çibdat } afondando mucho las carcauas & quae de facili fodi potest : \textbf{ et tunc per profundas foueas est fortificanda munitio , } ne per cuniculos deuincatur . \\\hline
3.3.22 & por que puedan passar \textbf{ allende del castiello o de la villa çercada . } la qual tierra cauada & unde debet transire castrum ; \textbf{ qua suffossa , et castro demerso in ipsam propter magnitudinem ponderis , } oportet castrum iterum construi , \\\hline
3.3.22 & la qual tierra cauada \textbf{ conuiene de apoyar bien el castiello o la çerca } por que se non funda & qua suffossa , et castro demerso in ipsam propter magnitudinem ponderis , \textbf{ oportet castrum iterum construi , } eo quod non possit \\\hline

\end{tabular}
