\begin{tabular}{|p{1cm}|p{6.5cm}|p{6.5cm}|}

\hline
1.1.7 & cuius nomen erat Mida , \textbf{ qui cum nimis esset auidus auri } ( ut fabulose dicitur ) & a que dezian meda \textbf{ el qual ero muy codiçioso de auerors } E asi commo dla fabliella gano de dios \\\hline
1.1.7 & Erat ergo ei magna copia auri , \textbf{ cum tamen fame periret . } Quod esse non posset , & grand Raqueza e grand cunplimiento de oro \textbf{ Enpero muria de fanbre } la qual cosa non pudiera ser \\\hline
1.2.14 & Nam , cum nauigiis , \textbf{ et cum toto suo exercitu transfretaret , } ne aliquis de suo exercito haberet materiam fugiemdi , & Ca commo el estudiese en sus naues \textbf{ e con todas sus naues passase la mar } por que ninguno de sus conpannas non ouiese manera de fuyr \\\hline
1.2.15 & nomine Phyloxenus , \textbf{ qui , cum esset pultiuorax , orauit , } ut guttur eius longius quam gruis fieret . & e tomaua muy grant delectaçion en ellas g̃rago a dios \textbf{ quel feziese la garganta } mas luenga que garganta de grulla \\\hline
1.2.16 & in Rege Sardanapallo , \textbf{ qui cum esset totus muliebris , } et deditus intemperantiae ( ut recitat Iustinus Historicus , libro 1 abbreuiationis Trogi Pompeii ) & enxienplo en el Rey sardan \textbf{ apalo que por que era todo mugeril | e dado a mugers } e era muy \\\hline
1.2.32 & quod \textbf{ cum quaedam praegnans esset , } et non potuisset parere , & entre las quales dize \textbf{ que commo fuesse vna mugier prenada } e non pudiesse parir desto conçibio tan grant dolor \\\hline
1.3.3 & Unde et Valerius Maximus de Dionysio Ciciliano recitat , \textbf{ qui cum esset tyrannus , } erat amator proprii commodi , & Onde ualerio maximo cuenta de dionsio seziliano \textbf{ que commo fuesse tyrano } e amador de propio prouecho despoblaua \\\hline
1.3.5 & quod eos esse decet humiles et magnanimos : \textbf{ cum ergo humilitas moderet spem , } quia humiles cognoscentes defectum proprium , & que conuenia alos Reyes de ser humildosos e de ser mag̃nimos . \textbf{ Et por ende por que la humildat tienpra la esperança } ca los humildosos conosçiendo su propre o fallesçemiento non esperan \\\hline
2.1.11 & et apud nullas gentes permissum inuenimus , \textbf{ ut quis cum matre contraheret . } Nam cum uxor debeat & e nin lo fallamos consentido \textbf{ que ninguno cassase con su madre } por que la muger deue ser subiecta al uaron . \\\hline
2.1.19 & Unde et aliquos Philosophos legimus sic fecisse , \textbf{ qui cum essent impeditae linguae , } accipientes specialem conatum & Ende leemos que algunos philosofos lo fizieron \textbf{ assi los quales commo ouiessen las lenguas enbargadas } tomaron especial esfuerço çerca aquellas letras \\\hline
2.3.12 & qui primo philosophari coeperunt . \textbf{ Ipse enim cum esset pauper , } et improperaretur sibi a multis cur philosopharetur , & que primeramente comneçara a philosofo far . \textbf{ Este mille sio commo fuesse muy pobre } e le denostassen sus amigos \\\hline
2.3.12 & et ad quid valeret Philosophia sua , \textbf{ cum semper in egestate viueret . } Ipse non denariorum cupidus , & diziendol que por que se daua tanto ala ph̃ia \textbf{ e aquel aprouechaua suph̃ia pues siengͤ biue en pobreza e en mengua . } Et el por este denuesto \\\hline
3.1.14 & nisi bellare , \textbf{ cum adesset oportunitas : } et onerosius et quasi omnino importabile esset sustentare sic quinque milia : & si non lidiar \textbf{ quando fuesse me este } Et muy mayor carga e peor de sofrir serie \\\hline
3.2.12 & Legitur enim de quodam tyranno , \textbf{ qui cum a fratre suo cotidie increparetur , } quare ipse semper tristis existeret , & ca leemos de vn tirano \textbf{ que cada dia era denostado de vn su hͣrmano } por que sienpre andaua triste \\\hline
3.2.12 & stare faciebat . \textbf{ Et cum frater eius timore horribili inuaderetur , } timens ab acuto gladio vulnerari , & poñuallesteros con ballestas armadas contrael . \textbf{ Et estonçe commo aquel su hͣrmano tomasse grant espanto } e grant temor \\\hline
3.3.7 & Legitur enim de Africano Scipione , \textbf{ qui cum pro populo Romano certare deberet , } non aliter contra hostes se obtinere credebat , & que quando auie de lidiar \textbf{ por el pueblo de roma non cuydaua vençer en otra manera a los enemigos } si non poniendo arqueros \\\hline
3.3.7 & ut matres nullum cibum eis exhiberent , \textbf{ quem non primo cum funda percuterent . } Est enim hoc exercitium utile , & que las madres nunca les querien dar de comer \textbf{ fasta que ferien con la fonda en logar çierto . } Et este uso es muy prouechoso \\\hline
3.3.20 & murus constitutus ex terra quasi absque laesione susciperet ictus machinarum : \textbf{ quia cum lapis eiectus a machina perueniret ad huiusmodi murum , } propter mollitiem eius cederet terra , & sin grand danno fuyo . \textbf{ Ca quando la piedra del engennio firiere en el muro de tierra . } por la blandura de la tierra \\\hline

\end{tabular}
