\begin{tabular}{|p{1cm}|p{6.5cm}|p{6.5cm}|}

\hline
1.1.5 & deue entender al bien dela gente \textbf{ e al bien comun } que es mas conuenible & quia in operibus suis debet \textbf{ intendere bonum gentis et commune , } quod est magis expediens et diuinius , \\\hline
1.1.7 & assy commo demostrͣemos enl terçero libro quando determinaremos del gouernamiento del regno \textbf{ Ca el Rey es aquel que propiamente parara mientes al bien del regno e al bien comun¶ } Mas si alguas vezes para mientes & cum determinabitur de regimine Regni . \textbf{ Nam Rex proprie est , | qui intendit bonum Regni , } et bonum commune : \\\hline
1.1.7 & por el su bien propio \textbf{ esto es por que del bien comun se le siguͤ bien ala persona } Mas al tirano faze todo el contrario . & si autem intendit bonum proprium , \textbf{ hoc est ex consequenti . } Tyrannus vero econuerso , \\\hline
1.1.7 & Ca prinçipalmente para mientes al su bien propio . \textbf{ Et si algunas vezes para mientes al bien comun } esto non es si non por que del bien comun se le sigͤa el algun bien propio ¶ & principaliter intendit bonum priuatum : \textbf{ si autem intendit bonum publicum , } hoc est ex consequenti , \\\hline
1.1.7 & Et si algunas vezes para mientes al bien comun \textbf{ esto non es si non por que del bien comun se le sigͤa el algun bien propio ¶ } Pues que asi es commo la fin & principaliter intendit bonum priuatum : \textbf{ si autem intendit bonum publicum , } hoc est ex consequenti , \\\hline
1.1.7 & que el non es Rey mas es tiranno . \textbf{ Ca non entiende prinçipalmente el bien comun . } Mas entiende prinçipalmente al su bien propio & ø \\\hline
1.1.12 & Mas el bien del entendimiento \textbf{ e delan razon es bien comun e entelligible } Conuiene ala Real magestad & quod est bonum secundum sensum , \textbf{ sed bonum rationis sit bonum uniuersale , et intelligibile , } decet Regiam maiestatem \\\hline
1.1.12 & Ca el que gouienna a muchos deue tener \textbf{ mientesal bien comun de todos . } Et por ende deue poner la su feliçidat & nam regens multitudinem debet \textbf{ intendere commune bonum . } In eo ergo debet \\\hline
1.1.12 & Et lo otro por que ha entendimiento . \textbf{ Et lo otro por que deue te çier mientes al bien comun } deue pener la su feliçidat & tum etiam quia habet intellectum , \textbf{ et tum quia intendit bonum commune , } debet suam felicitatem ponere in Deo , \\\hline
1.1.13 & gualardon de los otros \textbf{ Por que los Reyes trabaian en el bien comun } si non trispassaren los mandamientos de dios & quandam magnitudinem habere videtur : \textbf{ Reges enim vacantes communi bono , } si non transgrediantur , \\\hline
1.1.13 & trispassar son de mayor meresçimiento . \textbf{ Et dezimos que los reys trabaian en el bien comun . } Ca si non trabaiassen en el bien comun & maioris meriti esse videntur . \textbf{ Dicimus autem | ( vacantes communi bono ) } quia si bono communi non vacarent , \\\hline
1.1.13 & Et dezimos que los reys trabaian en el bien comun . \textbf{ Ca si non trabaiassen en el bien comun } non les conuernia de ser de mayor meresçimiento & ( vacantes communi bono ) \textbf{ quia si bono communi non vacarent , } non oportet eos esse maioris meriti \\\hline
1.1.13 & por que ellos pueden passar los mandamientos \textbf{ e non los passan mas penssando el bien comun } por que se ponen a peligro han mayor meresçiminto . & ex hoc quod transgredi possent , \textbf{ quia non considerato communi bono , } si quis periculo se exponat , \\\hline
1.1.13 & si non se pusiessen a peligro \textbf{ por el bien comun menguarian } enlo que han de fazer & dato quod non transgrediatur , \textbf{ quia indiscrete agit , } suum meritum non augmentatur . \\\hline
1.1.13 & Ca por el bien dela gente \textbf{ e por el bien comun se ponen a grandes trabaios } e a grandes periglos¶ & quia pro bono gentis . \textbf{ Et pro bono totali , } totaliter se exponunt . \\\hline
1.1.13 & si ama alguna ꝑson a singular . \textbf{ pues que assi es commo el bien comun } e el bien dela gente sea mas diuinal & si dirigat personam aliquam singularem . \textbf{ Cum ergo bonum gentis sit diuinius , } quam bonum aliquod singulare , \\\hline
1.2.5 & por que la fortaleza es mas ordenada al bien dela gente \textbf{ e al bien comun que la tenperança . } por ende esta es la orden que deuemos tener . & cum fortitudo magis ordinetur ad bonum gentis , \textbf{ et ad bonum commune quam ipsa temperantia : } ideo hic ordo est tenendus . \\\hline
1.2.10 & Pues que assi es en quanto los çibdadanos han orden assi mismos \textbf{ e al bien comun e al fazedor dela ley . } al Rey en tanto ha de seer en ellos estas dos iustiçias legal e ygual . & ad se inuicem , \textbf{ et ad rempublicam , | et ad legislatores seu ad ipsum Regem , } habet esse in eis Iustitia legalis , et aequalis . \\\hline
1.2.10 & Ca enlos çibdadanos en quanto han esta orden a otro \textbf{ o quieren bien comun o bien espeçial e suyo propio . } Ca si quieren bien comun & ex eo quod habent huiusmodi ordinem , \textbf{ vel quaeritur bonum commune , | vel bonum speciale et proprium . } Si quaeritur commune bonum : \\\hline
1.2.10 & o quieren bien comun o bien espeçial e suyo propio . \textbf{ Ca si quieren bien comun } assi es en ello la iustiçia legal . & vel bonum speciale et proprium . \textbf{ Si quaeritur commune bonum : } sic est in eis Iustitia legalis . \\\hline
1.2.10 & Mas si quieren algun bien espeçial \textbf{ e proprio assi es enllos iustiçia ygual . Ca el bien comun nasçe de todos los bienes de los çibdadanos . } Ca el bien comun e toda la çibdat es meior & in ipsis aliquod bonum priuatum : \textbf{ erit in eis Iustitia aequalis . | Bonum enim commune resultat } ex omni bono ciuium : \\\hline
1.2.10 & e proprio assi es enllos iustiçia ygual . Ca el bien comun nasçe de todos los bienes de los çibdadanos . \textbf{ Ca el bien comun e toda la çibdat es meior } que ningun bien espeçial & Bonum enim commune resultat \textbf{ ex omni bono ciuium : | res enim publica , } et tota ciuitas melior est , \\\hline
1.2.10 & porque las leyes entienden \textbf{ sienpre en el bien comun Manda } e ordenan toda manera de bondat Et por ende seer el omne iusto segunt la ley & Leges igitur \textbf{ quia intendunt commune bonum , } praecipiunt omnem modum bonitatis . \\\hline
1.2.10 & en qual quier manera que lo sea \textbf{ e donde se leunato el bien comun } e por ende sea la çibdat meior & qualitercunque ciuis bonus sit , \textbf{ ex hoc resultet commune bonum , } et sit inde melior ciuitas , \\\hline
1.2.15 & que la tenperança \textbf{ por que la fortaleze es uirtud mas ordenada a bien comun } que la tenꝑanca & quia Fortitudo magis ordinatur \textbf{ ad bonum commune , } ut ad tuitionem regni , \\\hline
1.2.15 & Et nos sienpre dezimos \textbf{ que el bien comun es meior } que el bien propio & Bonum autem commune \textbf{ diuinius est bono proprio . } Postquam ergo diximus de Prudentia , \\\hline
1.2.15 & muchas vegadas es dichon desuso \textbf{ si el bien comunes mas diuinal } que el bien propreo paresçe & Si ergo ( ut pluries dictum est ) \textbf{ bonum commune diuinius est quam proprium , } rationabiliter videtur \\\hline
1.2.15 & e a guarda dela persona commo en el matrimomo \textbf{ que es ordenado a bien comun } e aguarda dela genera conn humanal . & sicut in matrimonio , \textbf{ quod ordinatur ad bonum alterius , } et ad conseruationem speciei . \\\hline
1.2.21 & assi commo quando alguno granadamente sea en la honrra de dios \textbf{ e en el bien comun } e en las personas dignas de honrra . & erga cultum diuinum , \textbf{ et erga rempublicam , } et circa personas dignas , \\\hline
1.2.23 & que por el bien de dios \textbf{ e por el bien comun sea apareiado } de esponer su uida & et magna nimus , \textbf{ ut pro bono diuino et communi , paratus sit vitam exponere . } Secundo decet Reges , \\\hline
1.2.30 & destenp̃damente e desonestamente de las delecta connes de los iuegos \textbf{ en quanto el bien comun dela gente } que pertenesçe de cuydar al prinçipe es mas alto et muy mayor & vel inhoneste uti delectationibus ludorum , \textbf{ quanto cura boni communis , } quae spectat ad Principem excellentior est , \\\hline
1.3.3 & por que cada vno deue ser amadores \textbf{ que primero e prinçipalmente ame el bien diuinal e el bien comun . } Ca en el bien diuinal & quo quilibet debet esse amatiuus , \textbf{ est ut primo et principaliter | diligat } bonum diuinum et commune . \\\hline
1.3.3 & e mas noblemente es guardado el su bien en dios que en ssi mismo . \textbf{ Et por que el bien comun es mas diuinal } que el bien singular e el personal & quam in seipso . \textbf{ Et quia commune bonum est } diuinius quam singulare , \\\hline
1.3.3 & en el primero libro delas ethicas . \textbf{ Et otrosi por que el bien comun es ençerrado el bien propio de cada vno } sienpredeuemos ante poner el bien comun & ut dicitur 1 Ethic’ \textbf{ et quia in communi bono | includitur bonum priuatum , } semper bono priuato praeponendum est commune bonum . \\\hline
1.3.3 & Et otrosi por que el bien comun es ençerrado el bien propio de cada vno \textbf{ sienpredeuemos ante poner el bien comun } e dela comunidat al bien propio e personal de cada vno . & includitur bonum priuatum , \textbf{ semper bono priuato praeponendum est commune bonum . } Naturaliter enim videmus \\\hline
1.3.3 & por que los çibdadanos non duda una de se poner ala muerte \textbf{ por el bien comun de todos . } Ca el amor que auian los romanos al bien comun & se morti exponere . \textbf{ Dilectatio enim quam habebant Romani } ad Rempublicam fecit Romam esse principantem et monarcham . \\\hline
1.3.3 & por el bien comun de todos . \textbf{ Ca el amor que auian los romanos al bien comun } e publicofizo a Roma ser sennora & se morti exponere . \textbf{ Dilectatio enim quam habebant Romani } ad Rempublicam fecit Romam esse principantem et monarcham . \\\hline
1.3.3 & e assi commo el philosofo lo praeua en las politicas difetençia e deꝑtimiento es entre el Rey e el tyra non \textbf{ por que el Rey prinçipalmente entiende el bien comun de todos . } Et entendiendo en el bien comun & differentia est inter Regem , et tyrannum : \textbf{ quia Rex principaliter intendit bonum commune : } et intendendo bonum commune , \\\hline
1.3.3 & por que el Rey prinçipalmente entiende el bien comun de todos . \textbf{ Et entendiendo en el bien comun } entiende en el su bien proprio & quia Rex principaliter intendit bonum commune : \textbf{ et intendendo bonum commune , } intendit bonum proprium : \\\hline
1.3.3 & ca prinçipalmente entiende en el su bien propio mas despues desto \textbf{ e assi conmo por açidente entiende en el bien comun } en quanto del bien comun se leunata a el algun bien ppreo . & ex consequenti autem \textbf{ et quasi per accidens intendit bonum commune , | inquantum ex bono communi } consurgit \\\hline
1.3.3 & e assi conmo por açidente entiende en el bien comun \textbf{ en quanto del bien comun se leunata a el algun bien ppreo . } Mas si los Reyes e los tyranos se han en manera contraria & inquantum ex bono communi \textbf{ consurgit | sibi aliquod bonum priuatum . } Si ergo modo opposito se habent , \\\hline
1.3.3 & mas el bien propio \textbf{ que el bien comun . } Et la manera del amor del Rey deue ser & cum modus amoris tyrannici sit \textbf{ ut bonum priuatum praeponat bono communi , } modus amoris regis esse debet \\\hline
1.3.3 & que ante ponga e preçie \textbf{ mas el bien comun } que el bien propre o & modus amoris regis esse debet \textbf{ ut bonum commune praeponat priuato bono . } Immo quia speciali modo Rex \\\hline
1.3.3 & mas el bien diuianl \textbf{ e el bien comun } que el su bien propio & et Principes bonum diuinum \textbf{ et commune praeponere cuilibet priuato bono . } Secundo hoc idem patet , \\\hline
1.3.3 & que pueden los Reyes \textbf{ e los prinçipes aduzir a uirtudes es que amen prinçipalmente el bien diuianl e el bien comun } por que si el Rey prinçipallmente entendiere & quae inducere possent alios ad virtutes , \textbf{ est , ut bonum diuinum | et commune principaliter diligant . } Nam si Rex principaliter bonum commune intendat , \\\hline
1.3.3 & por que si el Rey prinçipallmente entendiere \textbf{ e estudiare en el bien comun estudiara . } por que aya memoria delas cosas passadas & Nam si Rex principaliter bonum commune intendat , \textbf{ studebit } ut habeat memoriam praeteritorum , \\\hline
1.3.3 & por las quales pue da meior gouernar su pueblo . \textbf{ Mas si ante pusiere e preçiare mas el bien comun } que el bien propio tanto mas estudiara & per quam possit melius suum populum regere . \textbf{ Immo si bonum commune praeponat bono priuato , } tanto magis studebit \\\hline
1.3.3 & quanto mayor pradençia e mayor sabiduria es meester \textbf{ para guardar el bien comun } que el bien propreo . & quanto maior prudentia requiritur \textbf{ ad custodiendum bonum commune , } quam proprium . \\\hline
1.3.3 & Et avn en ella milma maneral era iusto \textbf{ por que el bien comun es guardado mayormente por la iustiçia . } e sera magnanimo & si etiam erit iustus : \textbf{ quia bonum commune potissime | per iustitiam conseruatur . } Erit magnanimus ; \\\hline
1.3.3 & prinçipalmente ha de ser çerca los bienes diuinales e comunes . \textbf{ Otrosi sera fuerte por que ante pone el bien comunal bien propreo } e avn non dubdara de poner la persona a muerte siuiere & esse circa diuina , et communia . \textbf{ Erit fortis ; quia cum bonum cumune proponat bono priuato , } non dubitabit etiam personam exponere , \\\hline
1.3.3 & por alguna manera especial sobre todos los otros deuen amar el bien diuinal \textbf{ e el bien comunal en alguna manera espeçial sobre todos los otros deuen aborresçer todas aquellas cosas } que son contrarias al bien diuinal e comunal . & diligere bonum diuinum et commune , \textbf{ et quodam speciali modo | prae alios odire debent } quae contrariantur bono diuino et communi : \\\hline
1.3.3 & si por auentra a non pueden en otra manera destroyr los males \textbf{ nin puede en otra manera durar el bien comun } si non destruiendo & aliter vitia extirpari , \textbf{ nec potest aliter durare commune bonum , } nisi exterminando maleficos homines , \\\hline
1.3.3 & prinçipalmente deuen amar el bien \textbf{ diuica el bien comun . } Et pues que assi es amar el bien comun & ne pereat commune bonum . \textbf{ Amare ergo commune bonum , } et odire malefica , \\\hline
1.3.3 & diuica el bien comun . \textbf{ Et pues que assi es amar el bien comun } e querer mal a las colas malfechas & ne pereat commune bonum . \textbf{ Amare ergo commune bonum , } et odire malefica , \\\hline
1.3.3 & e querer mal a las colas malfechas \textbf{ que son contrarias al bien comun } Tanto mas conuiene alos Reyes & et odire malefica , \textbf{ quae ei contrariantur : } tanto magis decet Reges et Principes , \\\hline
1.3.4 & Conuiene alos Reyes e alos prinçipes entender e amar \textbf{ prinçipalmente el bien del regno e el bien comun . } Por la qual cosa si el desseo deue tomar mesura del amor & intendere \textbf{ et amare bonum regni et commune . } Quare si desiderium debet \\\hline
1.3.4 & e prinçipalmente amar el bien diuinal \textbf{ e el bien comun delas gentes . } Mas las otras cosas deuen amar & quia sicut debent amare primo \textbf{ et principaliter bonum diuinum et commune , } alia autem sunt desideranda \\\hline
1.3.4 & quanto mas conuiene a ellos \textbf{ de auer cuydado del regno e del bien comun . } Mas quales cosas son aquellas que guardan el regno en buen estado & tanto magis decet Reges et Principes , quanto magis eos decet \textbf{ habere curam de bono regni et communi . } Quae sunt autem illa \\\hline
1.4.2 & por iustiçia \textbf{ e por el bien comun de todos ¶ } Lo quinto es mucho de elquiuar la mentira alos reyes & sed propter iustitiam , \textbf{ et propter commune bonum . } Quinto maxime a Regibus et Principibus , \\\hline
2.1.3 & non solamente en quanto este gouernamiento es bien propreo suyo \textbf{ mas avn en quanto tal gouernamiento es ordenado a bien comun } assi cenmo al bien del regno e dela çibdat & non solum inquantum huiusmodi regimen est bonum proprium , \textbf{ sed etiam prout tale regimen ordinatur | ad bonum commune , } ut ad bonum regni et ciuitatis . \\\hline
2.1.7 & que conuiene mucho al \textbf{ mantene miento del humanal linage e al bien comun . } Mas eluarrio e la çibdat & quae maxime expedit conseruationi speciei , \textbf{ et bono communi . } Vicus autem , ciuitas , \\\hline
2.1.7 & assi commo dize el philosofo en el viij ̊ libro delas ethicas . \textbf{ Ca ordenar assi los bienes propreos al bien comun fazen avn } abastamientode uida . & ut dicitur 8 Ethic’ . \textbf{ Nam sic propria ordinare | ad bonum commune , } facit ad quandam sufficientiam vitae . \\\hline
2.1.8 & Empero si fuere y el bien de los fijos \textbf{ por que los fijos es vn bien comun } en el qual se ay unta el marido e la muger . & attamen si adsit ibi bonum prolis , \textbf{ quia proles est | quoddam commune bonum } in quo coniungitur vir et uxor , \\\hline
2.1.8 & e el uaron ala mugnỉ sin departimiento ninguno . \textbf{ Ca veemos que sienpre es bien comun } por que es bien comunal a todos & indiuisibiliter adhaerere . \textbf{ Videmus enim quod semper commune bonum } ( eo quod commune est ) \\\hline
2.1.8 & Ca veemos que sienpre es bien comun \textbf{ por que es bien comunal a todos } ayunta los ꝑtiçipantes & Videmus enim quod semper commune bonum \textbf{ ( eo quod commune est ) } coniungit participantes bono illo . \\\hline
2.1.8 & ayunta los ꝑtiçipantes \textbf{ en aquel bien comunal de todos . } Et por ende & ( eo quod commune est ) \textbf{ coniungit participantes bono illo . } Sicut ergo ciuitas coniungit \\\hline
2.1.8 & e retiene los çibdadanos \textbf{ que non se partan del bien comun dela çibdat } por que es bien comunal dellos & et continet ciues ipsos , \textbf{ ne a ciuilitate recedant , eo quod sit } quoddam commune bonum ipsorum : \\\hline
2.1.8 & que non se partan del bien comun dela çibdat \textbf{ por que es bien comunal dellos } assi los fijos ayuntan & ne a ciuilitate recedant , eo quod sit \textbf{ quoddam commune bonum ipsorum : } sic filii coniungunt \\\hline
2.1.8 & que non se pueden partir vno de otro \textbf{ por que son bien comunal dellos . } Ca sienpre es de la razon del bien comun que tenga e ayunte amistança & ne ab inuicem recedant , \textbf{ eo quod sint | quoddam commune bonum ipsorum : } semper enim de ratione communis , \\\hline
2.1.8 & por que son bien comunal dellos . \textbf{ Ca sienpre es de la razon del bien comun que tenga e ayunte amistança } assi commo dela razon del bien propreo es que ayunte e desayunte el vno del otro . & quoddam commune bonum ipsorum : \textbf{ semper enim de ratione communis , | est quod contineat , uniat , et coniungat , } sicut de ratione proprii , \\\hline
2.1.8 & Et esta razon pone el philosofo en el viii̊ libro delas ethicas \textbf{ do dize que por que el bien comunal contiene } e ayunta los omes . & Hanc autem rationem tangit Philosophus 8 Ethic’ dicens , \textbf{ quod quia commune continet } et coniungit , \\\hline
2.1.8 & Por ende los fijos \textbf{ por que son bien comunal del marido } e dela muger paresçen ser razon & et coniungit , \textbf{ filii eo quod sint commune bonum utrorumque , } coniugium videtur esse causa , \\\hline
2.1.8 & por la generacion de los fijos \textbf{ que es bien comun dellos . } Ca quando alguons son amigos de vno & inclinantur ex ipsa prole , \textbf{ quae est ipsorum commune bonum immo } eo ipso quod aliqui sunt amici unius , \\\hline
2.1.13 & quanto ellos deuen auer mayor cuydado de sus fijos propreos \textbf{ por que dellos cuelga el bien comun } e la salud del regno ¶ & eo quod ex eis dependeat \textbf{ bonum commune et salus regni , | plus solicitari debent , } quam alii . \\\hline
2.3.14 & ssi commo sin el derecho natraal \textbf{ por el bien comun } connino de dar & Sicut praeter ius naturale \textbf{ propter commune bonum oportuit } dare leges aliquas positiuas , \\\hline
2.3.14 & por defendimiento de su çibdat \textbf{ o de su regno fueles otorgado en fauor del bien comun } e del defendimiento dela tierra & pro defensione proprie ciuitatis et regni , \textbf{ in fauorem communis boni } et defensionis patriae inductum est , \\\hline
2.3.14 & non contradigan al ordenamiento dela ley . \textbf{ Ca el bien comun deue ser ante puesto al bien propreo ¶ } La terçera razon se toma dela salud de los vençidos . & non resistere ordinationi legali : \textbf{ quia commune bonum bono priuato est praeponendum . } Tertia congruitas sumitur \\\hline
2.3.19 & de ser acuçiosos \textbf{ çerca aquellas cosas que derechamente parte nesçen al bien comun e al gouernamiento del regno } e quales cosas son aquellas en el terçero libro paresçca . & circa ea quae directe spectant \textbf{ ad bonum commune , | et ad regnum regni : } quae autem sunt illa \\\hline
3.1.14 & La primera razon paresçe \textbf{ assi ca sienpre el bien comun deue ser antepuesto al bien propra o ca natural cosaes } que la parte se ponga al peligro & Prima via sit patet . \textbf{ Nam semper bonum commune praeponendum est bono priuato : } naturale enim est partem se exponere periculo pro toto , \\\hline
3.1.15 & Et pues que assi es en esta manera deuen ser todas las cosas comunes alos çibdadanos \textbf{ por que cada vno entienda el bien comun } e el bien de todos & Hoc ergo modo ciuibus omnia debent esse communia , \textbf{ ut quilibet intendat bonum commune et bonum omnium , } et sit solicitus \\\hline
3.1.19 & e esta llama una comun \textbf{ por que los lidiadores deuian entender el bien comun } assi commo al defendimiento dela tr̃ra . & quam appellabat communem , \textbf{ eo quod bellatores communi bono } ut defensioni patriae vacare debeant : \\\hline
3.2.2 & Et alli muestra el pho departir el buen prinçipado del malo . \textbf{ ca si en algun señorio o prinçipado es entendido el bien comun } e el bien de todos los çibdadanos & bonum principatum a malo . \textbf{ Nam si in aliquo dominio aut principatu | intenditur bonum commune } et omnium ciuium \\\hline
3.2.2 & enssennorea vno o pocos o muchos \textbf{ Si vnio o aquel entiende bien comun } e bien de los subdictos & vel pauci , vel multi . \textbf{ Si unus vel intendit bonum commune et subditorum , } vel intendit bonum proprium . \\\hline
3.2.2 & e estonçe es dicho tal sennorio monarch̃ia o e egno \textbf{ ca al Rey parte nesçe de enteder el bien comun . } Et li aquel vno assi & tunc dicitur Monarchia siue Regnum : \textbf{ regis autem est intendere commune bonum . } Si vero ille unus dominans \\\hline
3.2.2 & Et li aquel vno assi \textbf{ enssennoreante non entiende el bien comun } Mas entiende por poderio çiuil apremiar los otros & Si vero ille unus dominans \textbf{ non intendit commune bonum , } sed per ciuilem potentiam opprimens alios , \\\hline
3.2.2 & e assi paresçe que dos principados se le una tan del sennono de vno prinçipado de derechas \textbf{ si commo quando por el bien comun en } llennore a vno este es dicho Rey & ex dominio unius unus rectus , \textbf{ ut cum propter bonum commune dominatur Rex : } et alius peruersus , \\\hline
3.2.2 & mas por alguons pocos estonçe aquellos pocos o son uirtuosos o buenos \textbf{ que entienden al bien comun } e tal prinçipado es dicħa ristrocaçia & vel sunt virtuosi et boni , \textbf{ et intendunt commune bonum ; } et tunc talis principatus \\\hline
3.2.2 & Mas si aquellos pocos non son uirtuosos \textbf{ nin entienden el bien comun } mas son ricos e apremadores de los otros & Sed si illi pauci non sunt virtuosi , \textbf{ nec intendunt commune bonum , } sed sunt diuites , \\\hline
3.2.2 & que son uirtuosos \textbf{ e entienden enł bien comun . } Et otro malo assi commo quando enssenore an alguons & ut cum dominantur aliqui , \textbf{ quia sunt virtuosi et intendentes commune bonum : } et alius peruersus , \\\hline
3.2.4 & por que estonçe ha derecha entençion \textbf{ si non entiende abien propo mas a bien comun . } Et pues que assi es quanto menos entiede en el bien comun & Tunc enim principans rectam habet intentionem , \textbf{ si non intendat bonum proprium sed commune : quanto igitur minus intenditur commune bonum , } tanto peior principatus : \\\hline
3.2.4 & si non entiende abien propo mas a bien comun . \textbf{ Et pues que assi es quanto menos entiede en el bien comun } tanto peoras el prinçipe e el prinçipado & Tunc enim principans rectam habet intentionem , \textbf{ si non intendat bonum proprium sed commune : quanto igitur minus intenditur commune bonum , } tanto peior principatus : \\\hline
3.2.4 & dado que entiendan el bien propre \textbf{ o non se parten del todo dela entençion del bien comun . } Mas si & intendendo sic bonum proprium , \textbf{ non omnino recedunt | ab intentione communis boni . } Sed si dominaretur unus solus , \\\hline
3.2.4 & que assi entiende el bien propre \textbf{ o apartasse much del bien comun . } Et pues que assi es meior cosa es de prinçipar muchs que vno¶ & sic intendens recedit \textbf{ quasi omnino a communi bono . } Peius est igitur principari unum , \\\hline
3.2.4 & que muchs non se pueden \textbf{ assi artedrar dela entencion del bien comun } commo por auentra a se arredraria vno solo . & et multi non sic deuiare possunt \textbf{ ab intentione communis boni , } sicut forte faceret unus solus : \\\hline
3.2.4 & que assi ouo aconpannado fuessen tristornados e corronpidos \textbf{ por que tales prinçipalmente entienden el bien comun . } Por ende si esto fuere & contingeret esse peruersos : \textbf{ talis enim maxime intendit commune bonum ; } ideo si sic se habeat , \\\hline
3.2.5 & a qual de los fijos pertenesçra el regno . \textbf{ Por ende por que el bien comun non sea puesto a peligro de todos los fios deuen auer los padres grant cuydado } pho en el quanto libro delas poluenta tres cosas & et nescitur cui filiorum succedat regnum . \textbf{ Ideo ne periclitetur bonum commune , | de omnibus filiis Regis cura diligens est habenda . } Philosophus 5 Politic’ \\\hline
3.2.6 & e de los uirtuosos es de amar \textbf{ mas el bien comun } que el bien propreo . & ab excessu virtuosarum actionum : \textbf{ nam quia bonorum virtuosorum est diligere bonum commune potius quam priuatum , } ideo reputatur dignus \\\hline
3.2.6 & Otrossi si ouiere auataia de los otros \textbf{ en obras uirtuosas procurara el bien comun . } ca si la uirtud parte nesçede se estender a mayor bien & Si excellat in actionibus virtuosis , \textbf{ procurabit commune bonum : } quia si virtutis est , \\\hline
3.2.6 & ca si la uirtud parte nesçede se estender a mayor bien \textbf{ e en mayor bien dela gente es el bien comun } que es mas diuinal & tendere in bonum , \textbf{ eius erit magis tendere in maius bonum , | bonum ergo gentis et commune } quod est diuinius \\\hline
3.2.6 & que otro bien singular \textbf{ por ende el omne bueno e uirtuoso mas procurara el bien comun } que el su bien pro ̉o priuado . & quam aliquod bonum singulare , \textbf{ magis procurabit vir bonus et virtuosus , | quam bonum aliquod proprium et priuatum . } Tertio expedit eum abundare \\\hline
3.2.6 & que son entre el thirano e el Rey . \textbf{ ¶ El primero es que el rey cata al bien comun } mas el thirano cata el bien propreo . & inter tyrannum et regem . \textbf{ Prima est , quia Rex respicit bonum commune : } tyrannus vero bonum proprium . \\\hline
3.2.6 & por que el regno es prinçipado derech . \textbf{ mas la tirania es sennorio tuerto e malo . Et pues que assi es commo el bien comun delas gentes sea mas diuinal que el bien de vno . } malamente e desigualmente & Nam regnum est principatus rectus , \textbf{ tyrannis vero est dominium peruersum . | Cum ergo bonum gentis sit } diuinius bono unius , \\\hline
3.2.6 & malamente e desigualmente \textbf{ enssennorea aquel que despreçiando el bien comun } entiende el bien propreo & diuinius bono unius , \textbf{ peruerse dominatur | qui spreto bono communi } intendit bonum proprium . \\\hline
3.2.6 & e quanto es digno de honrra \textbf{ el que entiende en el bien comun . } por la qual cosa si el thirano entiende el su bien propreo siguese & quantus honor sequitur , \textbf{ et quanto honore est dignius intendens commune bonum . } Quare si tyrannus intendit \\\hline
3.2.6 & ca non es cerca el bien honrrado e de honira \textbf{ por que el non entiende enl bien comun } veste departimiento & circa bonum honorificum , \textbf{ eo quod ipse intendat commune bonum . } Ex hac autem secunda differentia sequitur tertia ; \\\hline
3.2.6 & Ca el tirano \textbf{ por que despreçia el bien comun } non ha cuydado & videlicet quod intentio tyrannica est circa pecuniam . \textbf{ Tyrannus quia spreto communi bono non curat } nisi de delectationibus propriis , \\\hline
3.2.6 & e esta en que los çibdadanos sean buenos e uirtuosos \textbf{ por que desto nasçe prençipalmente el bien honrrado e el bien comun . } Et deste departimiento terçero se sigue el quarto . & et ut ciues sint virtuosi : \textbf{ quia ex hoc maxime resultat | bonum honorificum , et commune . } Ex hac autem differentia tertia \\\hline
3.2.6 & Ca el tiranno \textbf{ por que menospreçia el bien comun e non ha cuydado } si non delas sus delecta connes proprias . & et a genitis in proprio regno . \textbf{ Nam tyrannus eo quod spreto communi bono non curat } nisi de delectationibus propriis , \\\hline
3.2.6 & por que vee que ha muy grant cuydado del regno \textbf{ e del bien comun fia muchͣ } de aquellos que son en el su regno . & sed Rex econuerso eo \textbf{ quod videat se maximam curam habere de bono regni et communi , } maxime confidit de his \\\hline
3.2.7 & ¶La primera se toma \textbf{ por razon que tal sennorio mucho se arriedra dela entençion del bien comun . } La segunda se toma & tyrannidem esse pessimum principatum . \textbf{ Prima sumitur ex eo quod tale dominium maxime recedit | ab intentione communis boni . } Secunda , ex eo quod est maxime innaturale . \\\hline
3.2.7 & non se arriedra dela \textbf{ entençiondel bien comun } por que el Rey non es rey si non fuere uirtuoso & non receditur \textbf{ ab intentione communis boni ; } eo quod Rex non est , \\\hline
3.2.7 & por que el Rey non es rey si non fuere uirtuoso \textbf{ e si non entendiere en el bien comun . } Onde en el terçero libro delas politicas & nisi sit virtuosus , \textbf{ et nisi intendat commune bonum . } Unde 3 Polit’ dicitur , \\\hline
3.2.7 & puesto que los que assi enssenno rean non entiendan \textbf{ si non el bien propo enpero non se arriedran del todo dela entençion del bien comun . } por razo que muchs son los que son assi prinçipantes . & nisi bonum proprium , \textbf{ non omnino recederent | ab intentione communis boni , } eo quod plures essent sic principantes : \\\hline
3.2.7 & por razo que muchs son los que son assi prinçipantes . \textbf{ ca todo el bien de muchos es en alguna manera bien comun . } Mas si el tirano & eo quod plures essent sic principantes : \textbf{ omne autem bonum plurium , | est quodammodo bonum commune . } Sed si tyrannus dominetur , \\\hline
3.2.7 & si non el bien propreo \textbf{ en toda manera se arriedra dela entençion del bien comun . } Pues que assi es el prinçipado & et non intendat nisi bonum proprium ; \textbf{ omnino receditur ab intentione communis boni . } Principatus ergo quia non est rectus \\\hline
3.2.7 & si non fuere en alguͣ manera cosa diuinal \textbf{ e por que en el prinçipalmente se deue entender el bien comun } que es mas diuinal & quodammodo quid diuinum , \textbf{ et quia in eo principaliter debet | intendi bonum commune } quod est diuinius \\\hline
3.2.7 & por ende tanto la tirania es peor prinçipado \textbf{ quanto por ella mas se arriedra el tirano dela entençion del bien comun . } Et esta razon tanne el philosofo çerca el comienço del quarto libro delas politicas . & tanto tyrannis est principatus peior , \textbf{ quanto in eo plus receditur | ab intentione communis boni . } Hanc autem rationem tangit Philosophus \\\hline
3.2.7 & assi commo y dize el pho \textbf{ por quela tirama much se arriedra dela poliçia e del bien comun . } la segunda manera para prouar esto mismo se toma . & ( ut ibi dicitur ) \textbf{ quia tyrannis plurimum distat a politia , | idest a communi bono . } Secunda via ad inuestigandum hoc idem , sumitur ex eo quod tale dominium maxime est naturale . \\\hline
3.2.8 & que es de enderesçar \textbf{ e gara la fin e al bien comun . } Et pues que assi estes cosas pertenesçen al offiçio del Rey . & populus vero , est quasi sagitta quaedam dirigenda \textbf{ in finem et in bonum . } Tria igitur spectant ad regis officium . \\\hline
3.2.9 & que las rentas del regno se pongan \textbf{ enł bien comun } e en el bien del regno & et regni redditus studeat \textbf{ expendere in bonum commune , } vel in bonum regni : \\\hline
3.2.9 & e las donaçiones \textbf{ en el bien comun } e en el bien del regno . & et oblationes ordinare \textbf{ in bonum commune regni , } sed etiam bona communia \\\hline
3.2.9 & e guarnesçer las çibdades e los castiellos que son en el su regno \textbf{ assi que parezca mas ser procurador del bien comun } que tirano que quiere sienpre supro . & et castra existentia in regno , \textbf{ ut appareat magis esse procurator communis boni , } quam tyrannus quaerens utilitatem propriam . \\\hline
3.2.10 & Mas el uerdadero Rey faze todo el contrario \textbf{ entendiendo en el bien comun } e conosçiendo que el es amado de todos & quod signum est tyrannidis pessimae . \textbf{ Verus autem Rex econuerso intendens commune bonum , } et cognoscens se diligi ab ipsis \\\hline
3.2.10 & contra razon derech̃tu eyendo \textbf{ que ellos non entienden enl bien comun } mas en el su bien propreo . & Vident enim se contra dictamen rectae rationis agere , \textbf{ et non intendere bonum commune sed proprium : } ideo vellent omnes suos subditos \\\hline
3.2.10 & e mantiene le ueyendo \textbf{ que por el bien comun } e el buen estado del regno & et conseruat , \textbf{ videns quod per ipsum , bonum commune , } et bonus status regni , \\\hline
3.2.10 & que auia buen Rey e uerdadero \textbf{ e que amaua el bien comun } si el esso mismo non amasse mucho al Rey & si cognosceret se habere verum Regem , \textbf{ et diligere commune bonum , } si viceuersa non diligeret ipsum Regem . \\\hline
3.2.10 & e en otra manera non sia uerdadero Rey \textbf{ ca non entendrie en el bien comun . } La vij ͣ̊ cautela del tirano es fazer los subditos pobres & aliter enim non esset verus Rex , \textbf{ quia non intenderet commune bonum . } Septima , est pauperes facere subditos adeo \\\hline
3.2.12 & enssennoreasse \textbf{ por el bien comun } este prinçipado es derech & Dicebatur autem \textbf{ quod si dominatur unus propter bonum commune , } est principatus rectus , \\\hline
3.2.12 & Mas quando enssennore a todo el pueblo \textbf{ si entienda al bien comun de todos tan bien de los nobles conmo de los otros es senorio derech } e es llamado gouernamiento de pueblo . & Sed si dominatur totus populus \textbf{ et intendat bonum omnium | tam insignium quam aliorum , } est principatus rectus , \\\hline
3.2.12 & que los ricos mas enssennorean \textbf{ e non entienden en el bien comun } mas en las riquezas entienden en las delectaçonnes corporales & sequitur diuites inique dominantes , \textbf{ et intendentes non commune bonum , } sed pecuniam , \\\hline
3.2.12 & enssennore antes mayormente han acuçia eti guarda de su cuerpo . \textbf{ Ca ninguno despreciando el bien comun } non entiende a riquezas & circa custodiam corporis . \textbf{ Nam nullus spreto communi bono intendit } ad pecuniam et voluptates corporis , \\\hline
3.2.12 & e el tirano \textbf{ assi commo diches de suso non entienden enel bien comun } mas en riquezas & Nam tyrannus ut supra dicebatur \textbf{ non intendit bonum commune , sed pecuniam . } Rursus ut supra dicebatur \\\hline
3.2.13 & e fazer tuerto alos subditos \textbf{ e non entender al bien comun } commo quier & et iniuriari subditis , \textbf{ et non intendere commune bonum ; } licet pluribus viis ostenderimus \\\hline
3.2.13 & por que se fazen despreçiados de los pueblos \textbf{ ca por qua non qͥeren el bien comun } mas quieren delectaçiones del su cuerpo & per quae se contemptibiles reddunt . \textbf{ Nam cum non quaerant bonum commune , } sed delectationes corporis , \\\hline
3.2.13 & que cupana mas dela garganta \textbf{ que del bien comun era despreçiado de los sbraditos } por que sienpre lo tienen por enbriago . & et curans magis de gula \textbf{ quam de bono communi , | contemnebatur a subditis , } eo quod quasi semper esset ebrius : \\\hline
3.2.13 & e non quiere honrrar los subditos \textbf{ njn quiere el bien comun } quariendo algunos alcançar la gloriar la honrra & et non honorare subditos , \textbf{ et non quaerere commune bonum , } volentes adipisci honorem \\\hline
3.2.15 & que estonçe entendra much \textbf{ en el bien comun del regno ¶ } La xͣ cosa que salua la poliçia es & si Rex sit bonus et virtuosus , \textbf{ quia intendet bono regni et communi . } Decimum , est Regem non ignorare qualis sit illa politia \\\hline
3.2.17 & por que cada vno de los consseieros tirando de ssi el \textbf{ amortan solamente tenga mientes al bien comun } e al prouecho del regno & ut unusquisque consiliarius \textbf{ adiecta dilectione priuati boni , | solum aspiciat ad communem profectum : } et ut regni profectus impediri non possit , \\\hline
3.2.19 & por todo su ponder los derechs del Rey . \textbf{ por que tales biens deuen ser ordenados a bien comun } e a defendimiento del regno & pro viribus saluare iura Regis , \textbf{ eo quod huiusmodi bona ordinanda sunt | ad bonum commune , } ut ad defensionem regni , \\\hline
3.2.23 & non podria estar nin ser guardado . \textbf{ Mas saluado el bien comun } e la paz del regno & et bonus status ciuium non potest consistere ; \textbf{ sed saluato communi bono } et pace regni \\\hline
3.2.24 & e qual si quier de los malos \textbf{ que turban la paz de los çibdadanos desfazen el bien comun . } Por ende el de recħnatra al manda & sic , quia fur et maleficus , \textbf{ et quilibet malefactor turbat pacem ciuium insidiatur communi bono , } ideo naturale ius exigit \\\hline
3.2.24 & e sea muerto \textbf{ por que non sea puesto a periglo el bien comun } nin sea enbargado . & et exterminetur , \textbf{ ne exponatur periculo , } et ne impediatur commune bonum : \\\hline
3.2.26 & al derecho natural o ala ley dela natura dela qual tomo Rays e fuidamiento . \textbf{ Et puede se conparar al bien comun } que se entiende enl ła & a qua suscipit fundamentum : \textbf{ ad bonum commune quod in ea intenditur : } et ad gentem ad quam applicatur , \\\hline
3.2.26 & conuiene que sea derechͣ . \textbf{ Et en quanto es conparada al bien comun } conuiene que sea aprouechosa . & oportet quod sit iusta : \textbf{ ut comparatur ad bonum commune , } necesse est quod sit utilis : \\\hline
3.2.26 & e çiuil deue ser prouechosa \textbf{ en quanto es conparada al bien comun . } Ca si enla ley non fuere entendido el bien comun & Secundo lex humana et ciuilis debet esse utilis \textbf{ ut comparatur ad bonum commune : } nam si in lege non intenditur bonum commune , \\\hline
3.2.26 & en quanto es conparada al bien comun . \textbf{ Ca si enla ley non fuere entendido el bien comun } non sera la ley derechͣ nin de Rey . & ut comparatur ad bonum commune : \textbf{ nam si in lege non intenditur bonum commune , } tunc non est recta et regularis , \\\hline
3.2.26 & que gouierna derechamente \textbf{ e entiende enel bien comun . } Et tytano es dicha & qui recte agit \textbf{ et intendit commune bonum , } tyrannus vero \\\hline
3.2.26 & Assi la ley derechͣ e real \textbf{ enla qual es entendido el bien comun } se departe dela ley mala e dela ley del tyrano & sic lex recta et regia \textbf{ in qua intenditur bonum commune differt } a peruersa et tyrannica \\\hline
3.2.26 & si derechͣs fueren \textbf{ que sea entendido el bien comun . } Ca sienpre la fin es regla de todas las nuestras obras . & ( si rectae sint ) \textbf{ intendi commune bonum : } nam semper finis est regula omnium nostrorum agibilium . \\\hline
3.2.26 & e la fin que entendemos . \textbf{ Por la qual cosa commo el bien comun sea mas diuinal que el bien propreo } e aya mas razon de bien e de fin & et finis intentus : \textbf{ quare cum bonum commune sit | diuinius } quam bonum priuatum , \\\hline
3.2.26 & que el bien propio . \textbf{ porque el bien propra o es ordenado al bien comun . } Conuiene que las leyes tales sean non & quam bonum priuatum , \textbf{ et bonum priuatum ordinetur ad ipsum , } oportet tales leges fieri \\\hline
3.2.26 & o \textbf{ mas quales demanda el bien comun . } por que el bien comun ha & non quales requirit bonum priuatum , \textbf{ sed quales requirit commune bonum , } eo quod hoc habet magis rationem finis , \\\hline
3.2.26 & mas quales demanda el bien comun . \textbf{ por que el bien comun ha } mas razon de bien e de fin . & sed quales requirit commune bonum , \textbf{ eo quod hoc habet magis rationem finis , } et per consequens \\\hline
3.2.26 & Et por ende se sigue \textbf{ que segunt el bien comun } mas sean puestas las leyes e las reglas delas nuestras obras & et per consequens \textbf{ secundum ipsum magis sumendae } sunt leges et regulae agibilium . \\\hline
3.2.26 & Ante porque el bien propreo non ha razon de bien \textbf{ si non en quanto es ordenado al bien comun . } Ca mala e torpe es la parte & Immo quia bonum priuatum non habet rationem boni \textbf{ nisi ut ordinatur ad bonum commune ; } quia turpis est illa pars , \\\hline
3.2.26 & e segunt que es bien propio \textbf{ Mas en quanto ello es ordenado al bien comun } ¶ & et secundum se , \textbf{ sed prout ordinatur ad bonum commune . } Tertio lex prout comparatur ad populum \\\hline
3.2.27 & que auemos de fazer \textbf{ que nos ordenan a bien comun } e han poder de nos costrennir . & sunt quaedam regulae agibilium , \textbf{ ordinantes nos in commune bonum , } habentes coactiuam potentiam . \\\hline
3.2.27 & La primera se toma de aqual \textbf{ lo que las leyes nos ordenan a bien comun . } La segunda de aquello & Prima via sumitur \textbf{ ex eo quod leges nos ordinant ad commune bonum . } Secundo ex eo quod habent potentiam coactiuam . \\\hline
3.2.27 & por las quales leyes ymosa aquel bien . \textbf{ Por la qual cosa commo el bien comun sea entendido } prinçipalmente de toda la comunidat & secundum quas intendimus in bonum illud : \textbf{ quare cum bonum commune principaliter intendatur a tota communitate } ut a toto populo , vel a principante , \\\hline
3.2.27 & las leyes comunales \textbf{ que ordenananos al bien comunal } deuen ser establesçidas del prinçipe & omnes leges , \textbf{ quae ordinant nos in commune bonum , } condendae sunt a Principe \\\hline
3.2.27 & a quien parte nesçe \textbf{ de enderesçar los omes al bien comun . } Ca si es ley diuinal e natural establesçida es de dios & quae non sit edita \textbf{ ab eo cuius est dirigere in bonum commune : } nam si est lex diuina et naturalis , \\\hline
3.2.27 & a quienꝑtenesçe enderesçar todas las cosas asimesmo . \textbf{ El qual mayormente es bien comun } por que es bien de todo bien . & cuius est omnia dirigere in seipsum , \textbf{ qui maxime est commune bonum , } quia est bonum omnis boni , \\\hline
3.2.27 & quando enssennorea ha de ordenar \textbf{ e de enderesçar todos los otros al bien comun . } Et pues que assi es cada vna & Princeps enim aut totus populus cum principatur , \textbf{ habet dirigere et ordinare alios in commune bonum . } Quaelibet ergo persona particularis , \\\hline
3.2.27 & del bien propreo nin dela casa . \textbf{ Mas del bien comun } que es entendido en el regno & sunt leges et regulae agibilium , \textbf{ sequitur quod non a bono priuato et domestico sed a bono } quod intenditur in regno et ciuitate sumendae \\\hline
3.2.28 & alos quales conuiene ser muy acuçiosos \textbf{ çerca del bien comun } e çerca del gouernamiento del regno & quorum interest solicitari \textbf{ circa bonum commune : } circa regimen regni , \\\hline
3.2.31 & por el prouecho comun \textbf{ e por el bien comun delos omes } assi commo mostramos suso en el segudo libro & et quia propter utilitatem \textbf{ et propter commune bonum hominum est seruitus introducta , } ut supra in lib’ 2 \\\hline
3.2.36 & si fuere menester a toda cosa \textbf{ por el bien comun } e por el defendemiento del regno . & ponentes ( si oporteat ) \textbf{ seipsos pro bono communi , } et defensione regni . \\\hline
3.2.36 & e por el amor \textbf{ que han al bien comun } e al Rey son mas uirtuosos & et ex dilectione \textbf{ quam habent | ad bonum commune et ad Regem , } sint magis boni et virtuosi , \\\hline
3.2.36 & e tan acabados \textbf{ que por el amorsolo del bien honesto e del bien comun } e por amor del prinçipe ponedor dela ley & Nam non omnes sunt adeo boni et perfecti \textbf{ quod solo amore honesti et boni communis , } et ex dilectione legislatoris , \\\hline
3.2.36 & e por amor del prinçipe ponedor dela ley \textbf{ cuya entençiones de tener mientes al bien comun } que por ende queden los omes de mal fazer . & et ex dilectione legislatoris , \textbf{ cuius est intendere commune bonum , } quiescant male agere : \\\hline
3.3.1 & sea otro que el bien de vna perssona apartada et singular . \textbf{ assi el bien comun es otro e departido de . } algun bien singular . & a bono alicuius singularis personae , \textbf{ sicut bonum commune est aliud } a bono aliquo singulari , \\\hline
3.3.1 & por la qual se pueden uençer los enemigos \textbf{ e los que enbarguan el bien comun } a el bien de la çibdat . & per quam superantur hostes \textbf{ et prohibentes bonum ciuile et commune . } Ex hoc autem apparet \\\hline
3.3.1 & assi commo dicho es de suso \textbf{ principalmente catan al bien comun } assy la caualleria & ( ut supra ostendebatur ) \textbf{ principaliter respiciunt commune bonum , } sic et militia principaliter instituta est \\\hline
3.3.1 & assy la caualleria \textbf{ prinçipalmente es establesçida e ordenada a defendemiento del bien comun } e de la çibdat a e del regno . & principaliter respiciunt commune bonum , \textbf{ sic et militia principaliter instituta est | ad defensionem communis boni , } ut ciuitatis , aut regni . \\\hline
3.3.1 & por las quales cosas se puede turbar la paz . \textbf{ el assessiego de los çibdadanos e el bien comun . } Et esto deue fazer los caualleros & qui sunt in regno , \textbf{ per quas turbari potest tranquillitas ciuium et commune bonum . } Hanc autem prudentiam videlicet militarem , \\\hline
3.3.1 & la essecuçion de las batallas \textbf{ e tirar e arredrar los enbargos del bien comun . } Et tales cosas commo estas pertenezcan a aquellos & maxime decet habere Regem . \textbf{ Nam licet executio bellorum , et remouere impedimenta ipsius communis boni , } spectet ad ipsos milites , \\\hline
3.3.1 & e en qual manera se pueden sabiamente tirar e arredrar los enbargos \textbf{ que son contra el bien comun esto pertenesçe prinçipalmente al rey o al prinçipe . } Et pues que assi es desto puede paresçer & et qualiter caute remoueri possint \textbf{ impedientia commune bonum , | maxime spectat ad principantem . } Ex hoc ergo patere potest , \\\hline
3.3.1 & Ca la caualleria paresçe ser alguna sabiduria \textbf{ de la obra de la batalla ordenada a bien comun } por que paresçe que assi se deuen auer los caualleros & quaedam prudentia operis bellici , \textbf{ ordinata ad commune bonum : } videntur enim se habere milites in opere bellico , \\\hline
3.3.1 & si non fueren çiertos \textbf{ que el ama el bien del regno e el bien comun } e si non ouieren esperança & ad dignitatem militarem , \textbf{ nisi constet ipsum diligere bonum regni et commune , } et nisi spes habeatur \\\hline
3.3.1 & quales se quier cosas \textbf{ que enbarguen el bien comun . } Et desto puede parescer & remouere quaecunque \textbf{ impedire possunt commune bonum . } Ex hoc etiam patere potest \\\hline
3.3.4 & Lo quinto \textbf{ que por razon de la iustiçia e del bien comun despreçien la uida corporal . } Lo sexto que non teman & Quarto non curare de incommoditate iacendi et standi . \textbf{ Quinto quasi non appretiare corporalem vitam . } Sexto non horrere sanguinis effusionem . \\\hline
3.3.4 & de non preçiar la vianda corporal \textbf{ por la iustiçia e por el bien comun . } Ca commo toda la hueste sea puesta & et commune bonum \textbf{ quasi non appretiari corporalem vitam . } Nam cum tota operatio bellica exposita sit periculis mortis , \\\hline
3.3.4 & Ca por defendimiento de la iustiçia \textbf{ e por defendimiento del bien comun } es de poner la vida corporal a periglo de muerte & Nam pro defensione iustitiae \textbf{ et pro communi bono exponenda est periculo corporalis vita , } non est cauenda effusio sanguinis , \\\hline
3.3.4 & por las quales la iustiçia \textbf{ e el bien comun se pueda defender Et destas cosas paresçe llanamente quales lidiadores } e quales omnes & per quae iustitia et commune bonum defendi potest . \textbf{ Ex his autem plane patet , | quales bellatores , } et quos viros pugnatiuos \\\hline
3.3.23 & a paz e assossiego de los omnes . \textbf{ e al bien comun . } ca assi se deuen auer las batallas & et ad quietem hominum , \textbf{ et ad commune bonum . } Nam sic se debent habere bella \\\hline
3.3.23 & assi por las batallas son los enemigos de taiar e de cortar . \textbf{ por los quales se enbarga el bien comun } e la paz de los çibdadanos & sic per bella sunt hostes conculcandi , et occidendi , \textbf{ per quos impeditur commune bonum , } et pax ciuium , \\\hline
3.3.23 & et los sus enemigos turben la paz \textbf{ e el bien comun a tuerto non es cosa sin guisa } que nos ensseñemos e demos doctrina a todos los prinçipes & per quem possint suos hostes vincere , \textbf{ quod totum ordinare debent | ad commune bonum , } et pacem ciuium . \\\hline

\end{tabular}
