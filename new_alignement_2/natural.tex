\begin{tabular}{|p{1cm}|p{6.5cm}|p{6.5cm}|}

\hline
1.1.1 & en tanto demandar çertidunbre de cada cosa \textbf{ en quanto la naturaleza dessa mismͣ cosa lo demanda } Ca semeja la naturaleza & intantum certitudinem inquirere \textbf{ secundum unumquodque genus , } inquantum natura rei recipit . \\\hline
1.1.1 & en quanto la naturaleza dessa mismͣ cosa lo demanda \textbf{ Ca semeja la naturaleza } Dela sçiençia moral del todo ser contraria ala sçiençia matematica & secundum unumquodque genus , \textbf{ inquantum natura rei recipit . } Videtur enim natura rei moralis \\\hline
1.1.2 & E deuedes saber \textbf{ que esta orden es de rrazon e de naturaleza } primeramente es de rrazon & Secundo Oeconomica . Tertio Politica . \textbf{ Est autem hic ordo , rationalis , et naturalis . } Rationalis quidem , \\\hline
1.1.2 & asi enlas sçiençias speculatiuas \textbf{ por orden natural } sienpre ymos de mengua a conplimiento & quam perfectus et vir . \textbf{ In ipsis etiam speculabilibus ordine naturali } semper ex imperfecto ad perfectum procedimus , \\\hline
1.1.2 & para gouerna mj̊ de çibdado den rregno \textbf{ Conuiene Segund orden natural ala rreal magestad } primeramente que el Ruy sepa gouernar asy mesmo ¶ & quanta in gubernatione ciuitatis et regni : \textbf{ ordine naturali decet regiam maiestatem } primo scire se ipsum regere , \\\hline
1.1.2 & son natraalmente mal creyentes e auarientos \textbf{ Mas los mançebos son naturalmente liƀales e francos e bien creyentes } pues que asy es paresçe & sunt naturaliter increduli , et auari : \textbf{ iuuenes vero sunt naturaliter liberales , et creditiui . } Videntur autem haec quatuor \\\hline
1.1.3 & Et estos son sotileza del entendemiento \textbf{ e engennjo natural } e los poderios del alma & huiusmodi sunt industria mentis , \textbf{ ingenium naturale , } potentiae animae : \\\hline
1.1.3 & que prinçipe ¶ \textbf{ Et naturalmente mas es sieruo } que señor ¶ & subiici quam principari , \textbf{ et magis est naturaliter seruus quam dominus . } Quarto obseruatis his , \\\hline
1.1.4 & que son mundanales \textbf{ Ca el omes es naturalmente medianero entre las bestias } sobre las quales es mayor & est autem homo naturaliter medius \textbf{ inter bruta } quibus est superior , \\\hline
1.1.4 & Mas sy es omne \textbf{ por que el omne es natural . } naturalmente aina la conpanable çibdadano & Si autem est homo , \textbf{ quia homo | ( ut ibi probatur ) } est naturaliter animal sociale , ciuile , et politicum , \\\hline
1.1.4 & por que el omne es natural . \textbf{ naturalmente aina la conpanable çibdadano } e ordenado & ( ut ibi probatur ) \textbf{ est naturaliter animal sociale , ciuile , et politicum , } sequitur quod regatur secundum prudentiam , \\\hline
1.1.4 & ca creyeron \textbf{ que qual quier ome naturalmente } syn otra ayuda . & crediderunt enim , \textbf{ quod ex puris naturalibus } absque alio auxilio gratiae posset \\\hline
1.1.10 & Mas assi commo el fuego \textbf{ que es naturalmente caliente } e sienpre escalienta assi el omne & Immo sicut ignis , \textbf{ qui est naturae calidae , } naturaliter agit , \\\hline
1.1.11 & Et el alma mas es de razon del omne que el cuerpo . \textbf{ Et sienpre la forma es mas natural dela cosa que la materia . } Et pues que assi es aquellas cosas son grandes bienes & quam corpus : \textbf{ et semper forma magis dicit naturam rei , quam materia . } Illa ergo sunt bona maxima interiora nostra , \\\hline
1.1.12 & por partiçipaçion \textbf{ e non lo ha conplidamente es instrumento de aquel que la ha naturalmente e conplidamente . } Et pues que assi es commo dios solo aya poderio de regnar & Nam illud quod habet aliquid per participationem , \textbf{ et imperfecte , | est instrumentum et organum eius , } quod habet illud essentialiter et perfecte , \\\hline
1.1.12 & dize \textbf{ que el amor tan bien natural } commo aquel commo humanal o angelical o diuinal ha muy grant uirtud de ayuntar al que ama con lo que ama . & Unde Dionysius 4 De diuinis nominibus ait , \textbf{ quod amorem , | siue diuinum , } siue angelicum , \\\hline
1.2.1 & assi departir \textbf{ ca alguons destos poderios del alma son naturales } e algunos son poderios senssitiuos conosçedores . & sic distingui possunt , \textbf{ quia potentiae animae | quaedam sunt naturales , } quaedam cognitiuae sensitiuae , \\\hline
1.2.1 & e algunos intellectiuos e entendedores ¶ \textbf{ Naturales poderios son aquellos } enlos quales partiçipamos con los arboles & et quaedam intellectiuae . \textbf{ Naturales potentiae sunt illae , } in quibus communicamus cum vegetabilibus , et plantis , \\\hline
1.2.1 & o en todas estos o en alguons dellos . \textbf{ Mas que en los poderios naturales non pueden ser las uirtudes podemos lo prouar } por tres razons ¶ & vel in aliquibus horum . \textbf{ In potentiis autem naturalibus | esse non possunt , } quod tripliciter patet . \\\hline
1.2.1 & por tres razons ¶ \textbf{ La primera es que por las cosas naturales } non somos alabados nin denostados . & quod tripliciter patet . \textbf{ Primo , quia ex naturalibus nec laudamur , } nec vituperamur nullus enim dicitur bonus homo \\\hline
1.2.1 & e se faze grande en el cuerpo ¶ \textbf{ La segunda razon por que en tales poderios naturales } non han de seer las uirtudeses . & vel bonam augmentatiuam . \textbf{ Secundo , in talibus non habent esse virtutes : } quia virtus est aliquid \\\hline
1.2.1 & en el segundo libro delas ethicas . \textbf{ Mas los poderios segund que son naturales non obedescena la razon . } Ca si alguno razonare o quisiere alguna cosa en su uoluntad & ut probari habet 2 Ethic’ . \textbf{ Potentiae autem naturales | ( secundum quod huiusmodi ) } rationi non obediunt : \\\hline
1.2.1 & e su poder fazen sienpre sus obras ¶ \textbf{ La terçera razon por que enlos poderios naturales } non pueden seer las disposiçonnes & suas actiones efficiunt . \textbf{ Tertio , in talibus potentiis } non sunt habitus , et virtutes , \\\hline
1.2.1 & ¶ Et pues que assi es commo la natura sea determimada a vna cosa \textbf{ e los poderios naturales sean determinandos conplidamente para obrar } segund su natura & cum ergo natura sit determinata ad unum , \textbf{ et potentiae naturales sufficienter determinentur ad agendum , } ex natura sua \\\hline
1.2.1 & assi commo aqui fablamos dela maldat e dela uirtud \textbf{ por la qual cosa no pueden ser las uirtudes en los poderios naturales } Et pues que assi es & et virtute loquimur : \textbf{ quare in eis huiusmodi virtutes | esse non possunt . } His ergo tribus rationibus , \\\hline
1.2.1 & por las quales es prouado \textbf{ que las uirtudes non son en los poderios naturales } se puede prouar & per quas probatum est virtutes \textbf{ non esse in potentiis naturalibus , } probari potest eas \\\hline
1.2.1 & nin denostados \textbf{ por los poderios naturales } assi non somos alabadosi & Nam sicut non laudamur , \textbf{ nec vituperamur ex potentiis naturalibus : } sic non laudamur , \\\hline
1.2.1 & non deuen ser puestas enlos poderios senssibles . \textbf{ porque assi commo los poderios naturales } non obedesçen a la razon . & Secundo in sensibus non est ponenda virtus moralis , \textbf{ quia sicut potentiae naturales } non obediunt rationi , \\\hline
1.2.1 & non deuen ser puestas en los sesos es esta \textbf{ Ca assi commo los poderios naturales son conplidamente determinados a sus obras por naturaleza . } Bien assi los sesos et los poderios & in sensibus poni non debet . Tertio in talibus non est moralis virtus , \textbf{ quia sicut potentiae naturales } sufficienter determinantur \\\hline
1.2.1 & senssiblessor determinados a sus oƀras \textbf{ por su naturaleza . } Ca assi commo el fuego tanto escalienta quato puede escalentar & sufficienter determinantur \textbf{ ad actiones suas per suam naturam : | sic et sensus . } Nam sicut ignis tantum calefacit , \\\hline
1.2.1 & el poderio es determinado para obrar . \textbf{ Et estos poderios naturales } e senssibles son determinados & Nam per virtutem moralem determinatur potentia ad agendum : \textbf{ haec autem sufficienter determinantur } ad actiones proprias per naturam : \\\hline
1.2.1 & e senssibles son determinados \textbf{ por su naturaleza a sus obras } por ende non ha mester ninguna uirtud moral & haec autem sufficienter determinantur \textbf{ ad actiones proprias per naturam : } non ergo indigent virtute morali . \\\hline
1.2.1 & Pues que assi es si las uirtudes morales \textbf{ non son en los poderios naturales } nin en los sesos naturales . & non ergo indigent virtute morali . \textbf{ Si ergo nec in potentiis naturalibus , } nec in sensibus est virtus moralis , \\\hline
1.2.1 & non son en los poderios naturales \textbf{ nin en los sesos naturales . } Commo sin estos poderios naturales & Si ergo nec in potentiis naturalibus , \textbf{ nec in sensibus est virtus moralis , } cum praeter potentias naturales , \\\hline
1.2.1 & nin en los sesos naturales . \textbf{ Commo sin estos poderios naturales } e sesos naturales & nec in sensibus est virtus moralis , \textbf{ cum praeter potentias naturales , } et sensus non sit nisi intellectus , \\\hline
1.2.1 & Commo sin estos poderios naturales \textbf{ e sesos naturales } non aya en el omne otro poderio & nec in sensibus est virtus moralis , \textbf{ cum praeter potentias naturales , } et sensus non sit nisi intellectus , \\\hline
1.2.2 & ues ya es mostrado \textbf{ que las uirtudes ¶ morales non son en los poderios naturales } nin en los sesos . & Ostenso , \textbf{ quod nec in potentiis naturalibus , } nec in sensibus habent esse virtutes : \\\hline
1.2.2 & que esta enl entendimiento \textbf{ assi commo la ph̃ia natural e la geometria e la methaphisica } e las otras sçiençias tałs¶ & cuiusmodi sunt scientiae speculatiuae , \textbf{ ut naturalis Philosophia , Geometria , Metaphysica , } et caetera talia . \\\hline
1.2.2 & por que es apareiado e inclinado de obedesçer al entendimiento \textbf{ e ala razon mas los poderios naturales } e los sesos non son razonabłs por enençia . & quia est aptus natus rationi obedire . \textbf{ Potentiae autem naturales , et sensus , } nec sunt rationales per essentiam , \\\hline
1.2.2 & Et otro del seso . \textbf{ Ca assi commo el appetito natural sigue a su forma naturalmente auida . } Assi el appetito conoscedor sigue la forma resçebidao aujda & In nobis autem duplex est appetitus , intellectiuus , et sensitiuus . \textbf{ Nam sicut appetitus naturalis | sequitur formam naturaliter adeptam , } sic appetitus cognitiuus \\\hline
1.2.2 & En tanto se sigue aellas vn appetito \textbf{ e vna inclinaçion natural } por que dessean natural . & sequitur ea quaedam naturalis inclinatio , \textbf{ et quidam appetitus naturalis , } ut naturaliter desiderent esse deorsum : \\\hline
1.2.2 & e vna inclinaçion natural \textbf{ por que dessean natural . } mente descender ayuso ¶ & et quidam appetitus naturalis , \textbf{ ut naturaliter desiderent esse deorsum : } sic habentia cognitionem \\\hline
1.2.2 & qua non han alma . \textbf{ Ca ueemos nos que el fuego naturalmente es caliente e liuiano } e por la liuiandat sube arriba a su logar propio et a su folgura . & quae sunt perfectiora illis . \textbf{ Videmus enim quod ignis naturaliter est calidus , } et leuis per leuitatem autem tendit in locum proprium , \\\hline
1.2.7 & ¶Lo terçero deue estudiar \textbf{ que en sennor ee natural . mente . } ¶ & Tertio studere debet , \textbf{ ut naturaliter dominetur . } Triplici ergo via inuestigare possumus , \\\hline
1.2.7 & ¶ la terçera \textbf{ que sin la pradençia non puede seer senor naturalmente¶ } La primera manera se declara assi . & Secundo , quia sine ea de facili vertitur in tyrannum . \textbf{ Tertio , quia sine ipsa non potest naturaliter dominari . } Prima via sic patet . \\\hline
1.2.7 & nin \textbf{ enssennorear natural mente . } Ca assi commo dize el philosofo en el primero libro delas politicas & quia sine ea non possunt naturaliter dominari . \textbf{ Nam ( ut declarari habet 1 Polit’ ) } ex hoc est aliquis naturaliter seruus , \\\hline
1.2.7 & Ca assi commo dize el philosofo en el primero libro delas politicas \textbf{ que por esso es dicho alguno naturalmente sieruo } por que es menguado de entendimiento & Nam ( ut declarari habet 1 Polit’ ) \textbf{ ex hoc est aliquis naturaliter seruus , } quia deficit intellectu , \\\hline
1.2.7 & e non sabe gouernar a ssi mismo . \textbf{ Et por ende es dicho alguno naturalmente señor } por que es conplido de entendimiento e de sabiduria . & et nescit seipsum regere . \textbf{ Ex hoc autem naturaliter est Dominus , } quia viget intellectu et prudentia , \\\hline
1.2.7 & mas ahun alabanla \textbf{ e confirman la todos los gouernamientos naturales . } Ca uehemos que los omes son naturalmente & non solum approbant physica dicta , \textbf{ sed etiam confirmant singula regimina naturalia . } Videmus enim naturaliter homines dominari bestiis , \\\hline
1.2.7 & e confirman la todos los gouernamientos naturales . \textbf{ Ca uehemos que los omes son naturalmente } sennores delas bestias & sed etiam confirmant singula regimina naturalia . \textbf{ Videmus enim naturaliter homines dominari bestiis , } Viros foeminis , \\\hline
1.2.7 & Et los uieios de los moços \textbf{ Ca los omes naturalmente sen } sennores delas bestias & Senes pueris . \textbf{ Homines naturaliter dominantur bestiis , } quia hominum genus viget prudentia : \\\hline
1.2.7 & que los omes esto contesçer alamente e pocas vezes . \textbf{ Et por ende por la mayor parte la muger deue ser naturalmente subietta al ome } por que naturalmente fallesçe dela sabiduria del omne ¶ahun & et in paucioribus ut plurimum . \textbf{ Ergo foemina viro naturaliter debet esse subiecta , } eo quod naturaliter deficiat a viri prudentia . \\\hline
1.2.7 & Et por ende por la mayor parte la muger deue ser naturalmente subietta al ome \textbf{ por que naturalmente fallesçe dela sabiduria del omne ¶ahun } en esta misma gusa las moços & Ergo foemina viro naturaliter debet esse subiecta , \textbf{ eo quod naturaliter deficiat a viri prudentia . } Hoc etiam modo iuuenes naturaliter decet \\\hline
1.2.7 & e los mançebos \textbf{ conuiene que naturalmente sean subiectos de los mas antigos } por que non son espiertos & Hoc etiam modo iuuenes naturaliter decet \textbf{ antiquioribus esse subiectos , } quia inexperti agibilium \\\hline
1.2.7 & nin son assi conplidos de sabiduria commo los vieios . \textbf{ pues que assi es do quier que ay mengua de sabiduria ay naturalmente seruidunbre . } Et do quier que ay sabiduria ha naturalmente sennorio . & sicut alii . \textbf{ Ubicunque igitur hoc naturaliter seruit , } et illud natureliter dominatur , \\\hline
1.2.7 & pues que assi es do quier que ay mengua de sabiduria ay naturalmente seruidunbre . \textbf{ Et do quier que ay sabiduria ha naturalmente sennorio . } Por ende sienpre el prinçipe deue auer sabiduria . & Ubicunque igitur hoc naturaliter seruit , \textbf{ et illud natureliter dominatur , } semper principans pollet prudentia , a qua deficit \\\hline
1.2.7 & por la qual sera sennor \textbf{ por cuya mengua el sieruo esta naturalmente en su seruidunbre . } Pues que assi es el Rey & semper principans pollet prudentia , a qua deficit \textbf{ qui naturaliter seruus existit . } Ut igitur Rex naturaliter dominetur oportet \\\hline
1.2.7 & Pues que assi es el Rey \textbf{ por que naturalmente sea sennor } conuiene que florescaen sabiduria e en entendimiento & qui naturaliter seruus existit . \textbf{ Ut igitur Rex naturaliter dominetur oportet } quod polleat prudentia , et intellectu . \\\hline
1.2.8 & Et pues que assi es por razon desta manera de conos \textbf{ çer que es enxerida naturalmente alos omes . } El que quiere alos otros guiar & Ratione igitur huiusmodi cognoscendi , \textbf{ qui est inditus hominibus , } volens alios dirigere , \\\hline
1.2.9 & La primera es esta para que ellos gouiernen el regno \textbf{ naturalmente deuen cuydar primero en los tpons passados } en los quales meior se gouerno el regno . & ut naturaliter regnum regant : \textbf{ excogitando primo tempora retroacta , } sub quibus temporibus \\\hline
1.2.11 & Ca cada vno de los regnos \textbf{ e canda vna comunidat semeia avn cuerpo natural . } Ca assi commo veemos & Quodlibet enim regnum , \textbf{ et quaelibet congregatio assimilatur cuidam corpori naturali . } Sicut enim videmus corpus animalis constare \\\hline
1.2.11 & Et si los pies non traxiessen la cabeça e la cabeça \textbf{ non enderecasse los pies el cuerpo natural } non podria estat nin durar¶ Vien assi por que ningun omne non es conpludo del todo & et caput dirigeret pedes : \textbf{ corpus naturale non posset subsistere . } Sic quia nullus homo sufficit sibi ad viuendum , \\\hline
1.2.11 & o es en ellos vna iustiçia mudadora e acorredora \textbf{ por sabiduria natural } sin la qual el cuerpo nal non podria durar . & est in eis quaedam commutatiua Iustitia , \textbf{ sine qua corpus naturale durare non posset : } sic prout ciues eiusdem ciuitatis , \\\hline
1.2.11 & segunt sus uirtudes e sus dignidades ¶ \textbf{ pues que assi es assi conmo el cuerpo natural } no podria estar & et bona tribuere . \textbf{ Sicut ergo corpus naturale non subsisteret , } nisi in eo reseruaretur \\\hline
1.2.11 & que assi commo cada vna desigualdat \textbf{ non tuelle la uida del cuerpo natural } maguer que cada vna desigualdat le enflaquezca & quod sicut non quaelibet inaequalitas \textbf{ tollit vitam corporis naturalis , } tamen quaelibet inaequalitas aegrotat , \\\hline
1.2.13 & en la mayor parte son tristes \textbf{ Et naturalmente cada vno fuye dela tristoza } assi commo naturalmente sigue las cosas delectables . & ut plurimum tristia sunt . \textbf{ Tristia autem naturaliter } quilibet fugit , sicut naturaliter sequitur delectabilia . \\\hline
1.2.13 & Et naturalmente cada vno fuye dela tristoza \textbf{ assi commo naturalmente sigue las cosas delectables . } Et pues que assi es commo nos natural mente fuyamos dela tristeza & Tristia autem naturaliter \textbf{ quilibet fugit , sicut naturaliter sequitur delectabilia . } Cum ergo naturaliter tristia fugiamus , \\\hline
1.2.13 & assi commo naturalmente sigue las cosas delectables . \textbf{ Et pues que assi es commo nos natural mente fuyamos dela tristeza } graue cosa es de repmir los temores & quilibet fugit , sicut naturaliter sequitur delectabilia . \textbf{ Cum ergo naturaliter tristia fugiamus , } difficile est reprimere timores , \\\hline
1.2.17 & non es cosa fuerte por si . \textbf{ Ca cada hun omne es naturalmente inclinado a amar asi mismo } e aguardar los sus biens propos & secundum se non est difficile : \textbf{ quia unusquisque naturaliter inclinatur | ut se diligat , } et ut sua bona custodiat . \\\hline
1.2.17 & cosaque parte nesçen a nos mismos \textbf{ e naturalmente amamos lo que pertenesçe anos . } Mas los auarientos tanto aman los bienes tenporales de fuera & quia propria bona sunt aliquid ad nos pertinens , \textbf{ et naturaliter afficimur ad illa . Immo auari adeo afficiuntur } ad ista exteriora bona , \\\hline
1.2.18 & e que non les conuiene en ninguna manera delo ser . \textbf{ Ca si el regno de cada vno dios Reyes deue ser natural } e bien ordenado & eos esse auaros . \textbf{ Nam si regnum alicuius debet } esse naturale , \\\hline
1.2.18 & que nos veemos en la nataleza \textbf{ e en las cosas naturales } ca en las cosas natraales & ø \\\hline
1.2.18 & non deue ser ninguna cosa ociosa nin baldia . \textbf{ por la qual razon commo la naturaleza de los omes se tenga } por pagada de pocas cosas & aliquid ociosum esse debet . \textbf{ Quare cum natura humana modicis contenta sit , } quia uni personae modica sufficiunt : si una aliqua persona multitudine diuitiarum superabundat , \\\hline
1.2.19 & Mas commo en cada cosa \textbf{ mas e menos non fagan departimiento en la naturaleza } e en la semeiança delas cosas & Sed cum magis , \textbf{ et minus non videantur | diuersificare speciem , } et naturam rerum , \\\hline
1.2.27 & que es nunca se enssanar \textbf{ por lo que ha razon de se ensannar . Ca natural cosa es anos } que por los males & et primo intendit reprimere iras , \textbf{ ex consequenti autem intendit moderare passiones oppositas irae . } Nam naturale est nobis \\\hline
1.2.27 & assi pequano que non paresca anos grande . \textbf{ Et non solamente naturalmente somos inclinados } para querer ser vengados & quin videatur nobis multum , \textbf{ non solum naturaliter inclinamur , } ut velimus puniri \\\hline
1.2.27 & a aquellos que nos fazen alguons males . \textbf{ Mas avn en alguna manera natural cosa esa nos de dessear de ser vengados dellos mas que deuemos . } Ca por que el mal que ellos nos fazen paresçe a nos & inferentes nobis aliqua mala , \textbf{ sed etiam quodammodo naturale est nobis | appetere punitionem ultra condignum . } Nam quia malum nobis illatum videtur nobis maius esse , \\\hline
1.2.28 & çerca las palauras en quanto son ordenadas a buena conuerssaçion en la uida del omne . \textbf{ Ca si el omne es naturalmente animalia aconpanable } por que se aconpanna a otro & ut ordinantur ad debitam conuersationem in uita . \textbf{ Si enim homo est naturaliter animal sociale , } ut probari habet 1 Politicorum , \\\hline
1.2.29 & La segunda de parte de los otros ¶ \textbf{ La primera por que cada vno naturalmente es Inclinado } a querer su bien propio & Secunda ex parte aliorum . \textbf{ Quilibet enim ita naturaliter | afficitur } ad proprium bonum , \\\hline
1.2.31 & que las uirtudes pueden se tomar en dos maneras . \textbf{ O en quanto son naturales } e non acabadas nin conplidas & ait , quod virtutes dupliciter considerari possunt : \textbf{ vel ut sunt naturales , et imperfectae : } vel ut sunt principales , et completae . \\\hline
1.2.31 & o en quanto son prinçipales e conplidas . \textbf{ Mas lasuirtudes naturales } e non conplidas & vel ut sunt principales , et completae . \textbf{ Virtutes autem naturales et imperfectae , } non oportet esse connexas . \\\hline
1.2.31 & nin ayuntadas vna a otra \textbf{ por que veemos alguons naturalmente auer alguna n industria } e alguna sotileza de entendimiento . & non oportet esse connexas . \textbf{ Videmus enim aliquos naturaliter habere | quandam industriam , } et quandam subtilitatem mentis : \\\hline
1.2.31 & Mas veemos a otros fazer el contrario desto \textbf{ por que han algpradençia natural } empero non son liberales & aliqui vero econtrario habent \textbf{ quandam naturalem pudicitiam , } non tamen liberales existunt . \\\hline
1.2.31 & e la industria moral \textbf{ que es aꝑcebimiento natural . } la qual industria el philosofo llama de motica & Differt enim prudentia , et industria , \textbf{ quam Philosophus appellat Denoteta . } Ille enim dicitur Denos , et industris , \\\hline
1.2.33 & que dixieron \textbf{ que por prinçipios puros naturales } podriemos escusar todos los males & violentium \textbf{ quod ex puris naturalibus possemus omnia mala vitare , } et perfectam bonitatem acquirere . \\\hline
1.3.2 & por que por todas estas cosas sera mas conosçida \textbf{ anos la naturaleza delas passiones } la qual catada e conosçida & innotescit \textbf{ nobis natura ipsarum passionum , } qua inspecta cognoscere possumus \\\hline
1.3.2 & que dichͣs son en alguna manera se nos demuestra la nataleza dellas superfiçialmente \textbf{ e en figua a la qual naturaleza conosçida } poremos conosçer & aliquo modo figuraliter \textbf{ et typo innotescit nobis natura ipsarum : } qua cognita , cognoscere possumus quomodo sint prosequendae , \\\hline
1.3.3 & assi mismo fazer bueno o guardar \textbf{ assymismo en bondat . La razon natural muestra } que mas deue amar el omne el bien diuinal que assi mismo . & vel se in bonitate conseruare , \textbf{ dictat naturalis ratio } ut magit diligat Deum quam seipsum : \\\hline
1.3.3 & por todo el cuerpo \textbf{ por inclinacion natural } quando alguno ha de ser ferido . & periculo pro corpore : \textbf{ ex naturali enim instinctu } cum quis vult percuti , \\\hline
1.3.3 & enlos quales esta la salud comun de todo el cuerpo prinçipal mente . \textbf{ Et esto por inclinaçion natural pone el braço a periglo } por que todo el cuerpo non ꝑesca . & ø \\\hline
1.3.4 & son semeiables \textbf{ en alguna manera alas cosas naturales . } Ca assi commo los cuerpos naturales & Nam gesta moralia \textbf{ quodammodo rebus naturalibus sunt similia . } Nam sicut corpora naturalia \\\hline
1.3.4 & en alguna manera alas cosas naturales . \textbf{ Ca assi commo los cuerpos naturales } por sus formas . & quodammodo rebus naturalibus sunt similia . \textbf{ Nam sicut corpora naturalia } per suas formas , \\\hline
1.3.4 & Empero deue resçebir medida e mesura dela mal querençia . \textbf{ Ca assi commo en las cosas naturales } beemos & tamen ex odio debet mensuram suscipere . \textbf{ Sic enim in naturalibus videmus , } quod licet motus deorsum \\\hline
1.3.4 & que maguera el mouimiento ayuso non sea vna cosa con la forma dela cosa passada . \textbf{ Empero naturalmente el mouimiento ayuso toma medida e manera dela pesadura } assi que quanto algunos cuerpos son mas pesados tanto & non sit idem quod forma grauis : \textbf{ naturaliter tamen motus deorsum sumit mensuram | et modum ex grauitate , } ut quanto aliqua sunt grauiora , \\\hline
1.3.4 & assi commo en el arte dela fisica \textbf{ prinçipalmente es entendida la salud del cuerpo natural . } Conuiene alos Reyes e alos prinçipes entender e amar & sicut in arte medicandi principaliter \textbf{ intenditur sanitas corporis : } naturaliter decet Reges et Principes \\\hline
1.3.5 & por la descenpranca de alguna passion . \textbf{ Ca por que abonda en ellos mucho la calentura natural } vienen ayna atentar algunas cosas & Contingit etiam hoc ex immoderatione passionis : \textbf{ nam quia nimis abundat in eis calor , } prorumpunt \\\hline
1.3.6 & que non puede obrar \textbf{ por que quando alguno teme la calentura natural tornasse alos mienbros de dentro . } Ca segunt la manera que nos veemos en todas las cosas podemos & Quarto facit eum inoperatiuum . \textbf{ Cum enim quis timet , | calor ad interiora progreditur ; } modum enim , \\\hline
1.3.6 & Ca segunt la manera que nos veemos en todas las cosas podemos \textbf{ lo ueer en la calentura del cuerpo natural . } Ca quando algunos omes que estan en los canpos & quem videmus in hominibus , \textbf{ aspicere possumus | in calore corporis naturalis . } Cum enim homines existentes \\\hline
1.3.6 & En essa misma manera \textbf{ quando alguno teme la calentura natural } que esta en los mienbros de fuera & sic cum quis timet , \textbf{ calor existens in exterioribus membris , } statim confugit ad interiora ; \\\hline
1.3.6 & e sin razon faze al ome tremuliento e tremedor \textbf{ Ca por el temor la calentura natural } tornase alos mienbros de dentro & Tertio immoderatus timor reddit hominem tremulentum . \textbf{ Nam propter timorem calore progrediente ad interiora , } exteriora membra frigida manent . \\\hline
1.4.1 & e avn son de buena esꝑança \textbf{ por que en ellos abonda mucho calentura natural . } Et por ende el coraçon & Sunt etiam bonae spei , \textbf{ quia in eis multum abundat calor : } corde ergo et aliis membris inflammatis \\\hline
1.4.1 & e los otros mienbros escalentados \textbf{ e entlamados por la calentura natural } que es en los mançebos & corde ergo et aliis membris inflammatis \textbf{ ex calore existente in ipsis iuuenibus , } fiunt iuuenes bonae spei , \\\hline
1.4.1 & Otrossi los mançebos biuieron poco en el tienpo passado \textbf{ e segunt cursso natural deuen much beuir en el tienpo } que ha de uenir . & Rursus iuuenes parum vixerunt in praeterito , \textbf{ et secundum cursum naturalem debent multum viuere in futuro . } Cum ergo memoria sit respectu praeteritorum , \\\hline
1.4.1 & Et pues que assi es por que los mançebos \textbf{ por la calentura natural desse an } sobrepuiar los otros temen & Cum ergo iuuenes , \textbf{ qui percalidi nimis affectent excellere , } timent inglorificari , \\\hline
1.4.2 & e desseo de luxia . \textbf{ Por ende la natural disposiçion del cueꝑpo mueue } e abiua los mançebos a cobdiçia de luyia . & et corpore calefacto fiat venereorum appetitus , \textbf{ naturalis dispositio corporis } incitat iuuenes ad concupiscentias venereorum . \\\hline
1.4.2 & e por la su sinpleza mesuran alos otros . \textbf{ Et pues que assi es commo natural cosa sea } que qual quier omne de ligero cree a aquel que cuyda que es bueno . & sed sua innocentia alios mensurant . \textbf{ Cum ergo naturale sit , } quod quis de facili credat ei , \\\hline
1.4.3 & Ca segunt el philosofo \textbf{ qual quier que es inclinado naturalmente a alguna passion } assi commo aquel que esta puesto en aquella passion & Nam secundum Philosophum , \textbf{ Quicunque naturaliter sic disponitur , } prout disponitur , \\\hline
1.4.3 & Et pues que assi es commo los temerosos sean esfriados . \textbf{ Et qual si quier que naturalmente es frio naturalmente es temeroso } Et por ende siguese & Cum ergo timidi efficiantur frigidi , \textbf{ quicunque est naturaliter frigidus , } sequitur quod sit naturaliter formidolosus . \\\hline
1.4.3 & Et por ende siguese \textbf{ que los uieios son naturalmente temerosos . Ca fallesçe enellos la calentura natural } e han los mienbs naturalmente frios ¶ & sequitur quod sit naturaliter formidolosus . \textbf{ Sequitur ergo senes esse naturaliter timidos , | quia deficit in eis naturalis calor , } et habent membra naturaliter frigida . \\\hline
1.4.3 & que los uieios son naturalmente temerosos . Ca fallesçe enellos la calentura natural \textbf{ e han los mienbs naturalmente frios ¶ } Lo quarto los uie ios son estallos & quia deficit in eis naturalis calor , \textbf{ et habent membra naturaliter frigida . } Quarto sunt illiberales : \\\hline
1.4.3 & e enlos humores \textbf{ e en la calentura natural } assi fallesçen en el coraçon & et in humoribus , \textbf{ et in calore naturali : } sic deficiunt in animo , \\\hline
1.4.4 & e van a ella . \textbf{ Et esto es contra la razon natural del frio . } Ca el frio en quanto esfrio non se estiende propreamente ala luxia mas & in alia se extendit . \textbf{ Hoc autem est contra rationem frigidi . } Nam frigidi ( secundum quod huiusmodi ) \\\hline
1.4.4 & e alguna inclinaçion \textbf{ por que segunt cuisu natural } e segunt la orden de natura & et inclinationem , \textbf{ quia secundum cursum naturalem , } et secundum ordinem quem videmus , \\\hline
1.4.4 & que veemos en las cosas los mançebos e los vieios \textbf{ e aquellos que son en estado medianero han alguna inclinacion natural alas costunbres } que les conuienen & iuuenes , et senes , \textbf{ et illi qui sunt in statu , | quandam pronitatem , } et inclinationem habent \\\hline
1.4.4 & por que assi commo los vieios e los mançebos maguera ayan disposiconn \textbf{ e indinacion natural ha costunbres malas e de deno star . } Empero pueden fazer contra aquella disposiconn & et iuuenes habent \textbf{ quandam pronitatem naturalem , | et inclinationem ad mores vituperabiles : } possunt tamen contra illam pronitatem facere \\\hline
1.4.4 & Empero pueden fazer contra aquella disposiconn \textbf{ e inclina conn natural } e segnir bueans costunbrs e de loar . & possunt tamen contra illam pronitatem facere \textbf{ consequi laudabiles mores . } Sic et illi \\\hline
1.4.4 & e indinaçion a costunbres bueans e de loar \textbf{ enpero pueden uenir e fazer contra esta disposiçion natural } assi que por la corrupçion del appetito & ad mores laudabiles , \textbf{ possunt tamen contra istam pronitatem facere , } ut per corruptionem appetitus sequantur \\\hline
1.4.5 & e sean de grant coraçon . \textbf{ Ca natural cosa es } que sienpre la fechura quiera semeiar a su fazedor & et sint magnanimi . \textbf{ Naturale est enim , } quod semper effectus vult assimilari causae : \\\hline
1.4.5 & que sienpre la fechura quiera semeiar a su fazedor \textbf{ por que los fijos son fechuras de los padron natural cosa es que los fuos semeien alos paradres . } Et por ende los nobles teniendo mientes & cum filii sint \textbf{ quidam effectus parentum , | naturale est filios imitari parentes . } Nobiles ergo aduertentes \\\hline
1.4.5 & La segunda son despreçiadores de aquellos que los engendran . \textbf{ Ca natural cosa es que cada vno quiere ayuntar } e a montonar alguna cosa & Secundus , sunt progenitorum despectores . \textbf{ Naturale est enim , | quod quilibet vult accumulare } ad id quod habet : \\\hline
1.4.5 & Et pues que assi es commo los Reyes \textbf{ e los prinçipes non puedan naturalmente ensseñorear } si non fueren bueons & Reges ergo et Principes , \textbf{ cum non possint naturaliter dominari , } nisi sint boni et virtuosi , \\\hline
2.1.1 & Et por ende de aquella cosa \textbf{ que naturalmente es fecha todas aquellas cosas le son naturales } sin las quales non se puede bien guardar en su ser . & ei ergo , quod naturaliter fit , \textbf{ naturalia sunt ea , } sine quibus non potest \\\hline
2.1.1 & Ca la nata en vano faria las cosas \textbf{ si las cosas naturales en ninguna manera non se pudiessen guardar en si mesmas e en su ser . } Et en vano & Frustra enim natura ageret , \textbf{ si res naturales nullo modo conseruarentur in esse , } sed statim , \\\hline
2.1.1 & assi mesmo en la uida \textbf{ son colas naturales al omne } Mas entre las o triscosas & sibi in vita sufficere , \textbf{ sunt homini naturalia : } inter alia autem , \\\hline
2.1.1 & que fazen ala bondamiento dela uida del ome es la conpannia . \textbf{ Et por ende naturalmente el omne } esaianlia aconpanable e conpanera Mas que la conpannia mucho faga a & ad sufficientiam vitae humanae , \textbf{ est societas , } naturaliter ergo homo est animal sociabile . \\\hline
2.1.1 & e las o tristales dales gouernamiento delas \textbf{ otrasaian lias que naturalmen te son engendradas . } Et pues que assi es la natura & administratur nutrimentum \textbf{ ex caeteris animalibus , | quae naturaliter producuntur . } Quodammodo ergo omnibus aliis animalibus natura sufficienter \\\hline
2.1.1 & que de parte dela uianda que auemos mester \textbf{ El omne es naturalmente con panno e ainalia aconpannable ¶ } La segunda manera para prouar esto mesmo se toma & quo indigemus , \textbf{ homo est naturaliter animal sociale . } Secunda via ad inuestigandum hoc idem , \\\hline
2.1.1 & Ca las bestias e las aues veemos \textbf{ que han natural uestido } assi commo las bestias han la lana e las aues las pennolas . & et aues , \textbf{ quasi naturale indumentum , } habere videntur lanam , vel pennas . \\\hline
2.1.1 & para estas cosas siguese \textbf{ que el o omne ha natural inclinamiento } para ser conp̃anero e ainal aconpannable . & sine societate alterius , \textbf{ sequitur quod homo naturalem impetum habeat } ut sit animal sociale ; \\\hline
2.1.1 & Mas si estas cosas son neçessarias \textbf{ para guardar la uida natural } delos omes & ad hoc quod homines perfecte sibi in vita sufficiant . \textbf{ Sed si haec sunt necessaria ad conseruandam hominis naturalem vitam , } viuere in communitate et in societate est quodammodo homini naturale . \\\hline
2.1.1 & siguese que beuir en conpannia \textbf{ e en comunidat es en alguna manera natural alos omes } ¶ & Sed si haec sunt necessaria ad conseruandam hominis naturalem vitam , \textbf{ viuere in communitate et in societate est quodammodo homini naturale . } Tertia via ad inuestigandum hoc idem , \\\hline
2.1.1 & para nuestro defendemiento . \textbf{ Par la qual cosa si natural cosa es al omne de dessear conseruaçion e guarda de su uida } commo el omne & fabricare valemus . \textbf{ Quare si naturale est homini desiderare conseruationem vitae , } cum homo solitarius non sufficiat sibi \\\hline
2.1.1 & por los quales se pueda defender de los enemigos . \textbf{ por ende natural cosa es ael } que & per quae a contrariis defendatur : \textbf{ naturale est ei , } ut desideret viuere in communitate , \\\hline
2.1.1 & Ca tondas las otras aian lias se inclinan conplidamente alas obras \textbf{ que deuen por inclinaçion natural sin ninguno otro ensseñamiento primero } assi commo el aranna & ad opera sibi debita \textbf{ ex instinctu naturae | absque introductione aliqua praecedente : } ut aranea ex instinctu naturae \\\hline
2.1.1 & Et pues que assi es por que el omne non es inclinado conplidamente \textbf{ por inclinaçion natural } alas obras & Quia ergo homo non sufficienter \textbf{ ex instinctu naturae inclinatur } ad opera sibi debita , \\\hline
2.1.1 & si non biuieremos en vno con los otros omes . \textbf{ Por ende natural cosa es al omne } de beuir con los otros omes & nisi simul cum aliis conuiuamus : \textbf{ naturale est homini } simul conuiuere cum aliis , \\\hline
2.1.1 & que tanne \textbf{ por las quales praeua el omne es naturalmente aial aconpannable } prinçipalmente se firma en esta razon la qual es & quas tangit , \textbf{ probantes hominem naturaliter | esse sociale animal , } potissime innititur huic rationi , videlicet , \\\hline
2.1.1 & e fabla la qual cosa non dio alas o trisaian lias siguese \textbf{ que el omne mas naturalmente es aial aconpanable } que ningunan delas otras aian lias & sequitur hominem \textbf{ magis naturaliter esse animal sociale , } quam animalia cetera . \\\hline
2.1.1 & por la qual cosa \textbf{ si assi es cosa natural al omne de ser aian la conpanable } los que fuyen la conpannia & quam animalia cetera . \textbf{ Quare si sic naturale est , | hominem esse animal sociale : } recusantes societatem , \\\hline
2.1.2 & las quales llaman algunos nietos e fijos e fijos de fijos . \textbf{ Et pues que assi es natural fazimiento } e comienço del uarrio & et pueros puerorum . \textbf{ Naturalis ergo origo vici , } est ex conuicinia domorum , \\\hline
2.1.2 & lo que dicho es \textbf{ que esta generaciones muy natural } que viene de gene raçion humanal & ad praesens vero in tantum dictum sit , \textbf{ quod huiusmodi generatio est maxime generalis : } quia procedit ex genere \\\hline
2.1.3 & Visto en qual manera la comunidat dela casa es primero en alguna manera que las otras comuni dades de ligero puede paresçer \textbf{ en alguna manera esta comunidat dela cała es natural } Ca commo la natura non presupone & quam communitates aliae : \textbf{ de leui videri potest , | quomodo sit huiusmodi communitas naturalis . } Nam cum natura non praesupponat artem , \\\hline
2.1.3 & ante pone la natura . \textbf{ Qual si quier cosa que dela cosa naturales presunpuesta } e antepuesta non puede ser propreamente cosa artifiçial & sed ars naturam : \textbf{ quicquid arte naturali supponitur , } non proprie quid artificiale erit , \\\hline
2.1.3 & e antepuesta non puede ser propreamente cosa artifiçial \textbf{ mas conuiene que aquello en quanto es tal sea cosa natural } Por la qual cosa si el omne es naturalmente & non proprie quid artificiale erit , \textbf{ sed oportet illud | ( secundum quod huiusmodi est ) naturale esse . } Quare si homo est \\\hline
2.1.3 & mas conuiene que aquello en quanto es tal sea cosa natural \textbf{ Por la qual cosa si el omne es naturalmente } ai al comiuncable e aconpanable & ( secundum quod huiusmodi est ) naturale esse . \textbf{ Quare si homo est } naturaliter animal communicatiuum et sociale , \\\hline
2.1.3 & e ante pongan la comunidat dela casa \textbf{ conuiene quala comunidat dela casa o la casa sea cosa natural . } Et por ende conuiene alos Reyes e alos prinçipes & praesupponat communitatem domus , \textbf{ oportet communitatem domesticam siue domum | quid naturale esse . } Reges ergo et Principes decet \\\hline
2.1.3 & e qual es la comunidat dela casa \textbf{ ca es comunidat en alguna manera natural } Et en algunan manera esta comunidat se ha al regno & ut sciant domum propriam gubernare , \textbf{ et ut cognoscant quae et qualis est communitas domus } ut se habet ad regnum et ciuitatem , \\\hline
2.1.4 & Ca ya mostrado es \textbf{ que el omne es naturalmente ainalia domestica e de casa } e quela comunidat dela casa es en alguna manera natural . & cum ostensum sit \textbf{ quod homo est naturaliter animal domesticum , } et quod communitas domus est quodammodo naturalis . \\\hline
2.1.4 & que el omne es naturalmente ainalia domestica e de casa \textbf{ e quela comunidat dela casa es en alguna manera natural . } Empero por que esto non auemos & quod homo est naturaliter animal domesticum , \textbf{ et quod communitas domus est quodammodo naturalis . } Attamen quia per hoc non sufficienter habetur \\\hline
2.1.4 & Ca que la casa sea comunidat \textbf{ segunt natura e natural de suso es prouado gruesamente e figuaalmente } e avn adelante lo mostraremos mas claramente & Nam quod domus sic communitas secundum naturam , \textbf{ superius grosse et figuraliter probabatur , } et infra clarius ostendetur ; \\\hline
2.1.4 & e departiremos todas las ꝑtes dela casa \textbf{ e prouaremos que cada vna ꝑtetal dela casa es cosa natural . } Pues que assi es finça de declarar en la difiniçion sobredichͣ & ubi distinguentur omnes partes domus , \textbf{ et probabitur quod quaelibet | talis pars est aliquid naturale . } Restat ergo declarare \\\hline
2.1.4 & Et por ende paresçe qual es la comunidat dela casa . \textbf{ Ca es comunidat natural } e establesçida para las obras cotidianas & Patet ergo qualis sit communitas domus , \textbf{ quia est communitas naturalis constituita } propter opera diurnalia et quotidiana . \\\hline
2.1.5 & e assi commo mas conplidamente se declara \textbf{ mas adelante es cosa natural . } Ca prinçipalmente cosa natural es la generaçion delas cosas & Nam domus ( ut superius dicebatur , \textbf{ et ut in prosequendo melius declarabitur ) est quid naturale . } Maxime autem quid naturale esse videtur , \\\hline
2.1.5 & mas adelante es cosa natural . \textbf{ Ca prinçipalmente cosa natural es la generaçion delas cosas } e la conseruaçiondellas & et ut in prosequendo melius declarabitur ) est quid naturale . \textbf{ Maxime autem quid naturale esse videtur , | rerum generatio , } et earum conseruatio . \\\hline
2.1.5 & commo la generaçion sea camino e carrera en la natura \textbf{ et commo las cosas naturales resçiban su naturaleza } propra a por la generaçion & sit via in naturam , \textbf{ et cum res naturales } per generationem propriam naturam accipiant , \\\hline
2.1.5 & a qual lo que dize damasçeno \textbf{ que la generaçion es cosa natural } e es obra de natura . & quod Damascenus ait , \textbf{ generationem esse quid naturale , } et esse opus naturae . \\\hline
2.1.5 & e la mantenençia delas cosas es cosanatal . \textbf{ Ca en vano seria algua cosa engendrada naturalmente } si non fuesse conseruada e guardada & quid naturale est . \textbf{ Nam frustra esset | aliquid naturaliter generatum , } si non posset in esse conseruari . \\\hline
2.1.5 & Et pues que assi es en esta manera estas dos comuindades \textbf{ fazen ser la casa cosa natural . } Ca la comunidat del uaron e dela mugnies ordenada ala generacion . & in esse conseruari . \textbf{ Hoc ergo modo hae duae communitates faciunt domum esse quid naturale : } quia communitas viri et uxoris ordinatur ad generationem , \\\hline
2.1.5 & Mas la comunidat del sennor e del sieruo es ordenada ala \textbf{ conseruaçique por la qual cosa si la generaçion e la conseruaçion es cosa natural } conuiene & communitas vero domini \textbf{ et serui ad conseruationem . Quare si generatio et conseruatio est quid naturale , } oportet domum quid naturale esse . \\\hline
2.1.5 & conuiene \textbf{ que la casa sea cosa natural ¶ } Otrossi por que la generaçion e la conseruaçion non pueden ser apartadas la vna dela otra & et serui ad conseruationem . Quare si generatio et conseruatio est quid naturale , \textbf{ oportet domum quid naturale esse . } Amplius , quia generatio et conseruatio \\\hline
2.1.5 & que sea seruido del sieruo . \textbf{ Por que naturalmente los sennores han mayor sabiduria e mayor entendemiento } por la qual cosa commo los que han mayor entendemiento & ut ei seruiatur a seruo : \textbf{ nam naturaliter domini vigent prudentia , et intellectu . } Cum ergo vigentes intellectu , \\\hline
2.1.6 & que es de padre e de fijo . \textbf{ Ca ueemos en las cosas naturales } que luego que son engendradas las cosas dla natura & scilicet patris et filii . \textbf{ Videmus enim in naturalibus rebus } quod statim quum generatae sunt , \\\hline
2.1.6 & assi conpado alas cosas natraales \textbf{ por que non puede la cosa natural } luego que es fecha fazer otra semeiante & non sic comparatur ad res naturales : \textbf{ quia non statim | cum est res naturalis , } potest sibi simile producere , \\\hline
2.1.6 & mas conuiene que primeramente el sea acabado . \textbf{ Et pues que assi es engendrar su semeiante non pertenesçe a cosa natural tomada en qual quier manera mas pertenesçe a cosa natural en quanto ella es acabada . } Et pues que assi es si la casa es cosa natural & producere ergo sibi similem , \textbf{ non est de ratione rei naturalis | quocumque modo , sumptae ; sed est de ratione eius , } ut habet esse perfectum . \\\hline
2.1.6 & Et pues que assi es engendrar su semeiante non pertenesçe a cosa natural tomada en qual quier manera mas pertenesçe a cosa natural en quanto ella es acabada . \textbf{ Et pues que assi es si la casa es cosa natural } e las cosas que veemos en la casa queremos traer & quocumque modo , sumptae ; sed est de ratione eius , \textbf{ ut habet esse perfectum . | Si ergo domus est quid naturale , } et ea quae videmus in domo , \\\hline
2.1.6 & e las cosas que veemos en la casa queremos traer \textbf{ a razones naturales diremos que las dos comuidades } que son de varon e de muger e de señor e de sieruo & et ea quae videmus in domo , \textbf{ reducere volumus in naturales causas , | dicemus duas communitates , } videlicet , viri et uxoris , et domini et serui , \\\hline
2.1.6 & que veemos \textbf{ segunt orden natural } enla muchedunbre dela casa & Nam semper multitudo ab unitate procedit , \textbf{ propter quod ordine naturali } quae videmus in multitudine domestica , \\\hline
2.1.6 & ¶Lo primero de parte dela generaçion \textbf{ e dela fructificaçion natural } ¶ & primo possumus probare \textbf{ ex parte generationis et fructificationis naturalis . } Secundo ex parte perpetuitatis . \\\hline
2.1.6 & en que non ay engendraçion de fijos . \textbf{ ¶ La segunda razon para prouar esto mesmo se toma de acabamiento de comunindat natural duradera por sienpre . } Ca commo los omes non pueden & ubi non est pollulatio filiorum . \textbf{ Secunda via ad inuestigandum hoc idem , | sumitur ex parte naturalis perpetuitatis . } Nam cum homines non possunt \\\hline
2.1.6 & Por que la morada continuada dela casa \textbf{ en vna manera es natural } assi commos por generaçion de fijos aquella casa sea continuadamente morada & Continua enim habitatio domus \textbf{ uno modo est naturalis , } ut si per creationem filiorum domus \\\hline
2.1.6 & en otra manera es \textbf{ assi comm̃ natural casual } e auentura assi commo si los primeros moradores son muertos oydos dela casa & ista continue habitetur . \textbf{ Alio modo est quasi casualis , } ut si prioribus habitatoribus defunctis , \\\hline
2.1.6 & e viene nueuamente otra conpanna en aquella casa . \textbf{ Et pues que assi es la morada natural dela casa } non puede durar naturalmente para sienpre & noua familia inhabitet domum illam . \textbf{ Naturalis ergo habitatio domestica naturaliter perpetuari non potest , } nisi per generationem , \\\hline
2.1.6 & Et pues que assi es la morada natural dela casa \textbf{ non puede durar naturalmente para sienpre } si non quando por generaçion & noua familia inhabitet domum illam . \textbf{ Naturalis ergo habitatio domestica naturaliter perpetuari non potest , } nisi per generationem , \\\hline
2.1.6 & fallesçen los fijos \textbf{ por los quales contesçe que la morada dela casa naturalmente } en alguna manera puede durar para sienpre¶ & ubi desunt filii , \textbf{ per quos habitationem domesticam naturaliter } quodammodo contingit perpetuari . \\\hline
2.1.7 & pmeramente nos conuiene de declarar en qual manera el mater moino es alguna cosa segunt natura . \textbf{ Et que el omne naturalmente } esaianl coniugable e ayuntable en mater moino . & coniugium esse aliquid secundum naturam , \textbf{ et quod homo naturaliter est animal coniugale . } Sciendum ergo quod Philosophus 8 Ethic’ volens \\\hline
2.1.7 & Et aduze para esto tres razones \textbf{ que el omne sea naturalmente } aianl conuigable¶ & adducens triplicem rationem \textbf{ quod homo sit naturaliter animal coniugale . } Prima ratio sumitur \\\hline
2.1.7 & Ca prouado es en el primero deste segundo libro \textbf{ que el omne es naturalmente } aina l aconpannable e comun incatiuo & Tertia ex parte operum . \textbf{ Probabatur enim in primo capitulo huius secundi libri , } hominem esse naturaliter animal sociale et communicatiuum . \\\hline
2.1.7 & e que el regno . \textbf{ Et el omne naturalmente es mas aianl domestico e de casa } que çiuil e de çibdat . & homo naturaliter \textbf{ magis est animal de mesticum , } quam ciuile : et communitas domus \\\hline
2.1.7 & Et la comunidat dela casa mas paresçe \textbf{ que es natural al omne } que la comunidat del uarrio & quam ciuile : et communitas domus \textbf{ magis videtur esse naturalis ipsi homini , } quam communitas vici , ciuitatis , et regni . \\\hline
2.1.7 & cosasa que es ordenada la comunindat dela casa non ay dubda que la comuidat dela casa \textbf{ mas es natural al omne } que la comunindat del uarrio & non videtur habere dubium , \textbf{ quin communitas domus magis sit naturalis homini , } quam vici , ciuitatis , et regni . \\\hline
2.1.7 & Ca aquella \textbf{ cosaparesçe ser mayormente natural } ala qual el omne ha natural inclinaçion & ex parte procreationis prolis . \textbf{ Nam illud maxime videtur naturale , } ad quod homo habet naturalem impetum : \\\hline
2.1.7 & cosaparesçe ser mayormente natural \textbf{ ala qual el omne ha natural inclinaçion } e natra al apetito & Nam illud maxime videtur naturale , \textbf{ ad quod homo habet naturalem impetum : } quare cum homo et omnia animalia naturaliter inclinentur , \\\hline
2.1.7 & e natra al apetito \textbf{ por la qual cosa commo todas las ainalias naturalmente sean inclinadas } para querer engendrar otro semeiable & ad quod homo habet naturalem impetum : \textbf{ quare cum homo et omnia animalia naturaliter inclinentur , } ut velint producere sibi simile , quia in hominibus hoc debite sit per coniugium , \\\hline
2.1.7 & assi por que entre los omes esto se faze conueniblemente por el casamiento . \textbf{ por ende el omne es naturalmente } aianl conuuigable e ayuntable por casamiento . & ut velint producere sibi simile , quia in hominibus hoc debite sit per coniugium , \textbf{ homo naturaliter est animal coniugale . } Hanc autem rationem tangit Philosophus 1 Politicorum , et 8 Ethicorum , \\\hline
2.1.7 & que el casamiento conuiene alos omes segunt natura \textbf{ por que natural cola es al omne } e atondas las aianlias & ubi probat coniugium competere homini secundum naturam , \textbf{ quia naturale est homini , } et omnibus animalibus , \\\hline
2.1.7 & e atondas las aianlias \textbf{ auer natural inclinaçion e appetito } para engendrar cosa semeiable & et omnibus animalibus , \textbf{ habere naturalem impetum } ad producendum sibi simile . \\\hline
2.1.7 & abastamientode uida . \textbf{ Por la qual cosa si natural cosa es al omne de auer inclinaçion e appetito al abastamiento dela uida natural cosa es a el de querer ser a i al conuigable e ayuntable a su muger } mas si el ma termonio es cosa natural siguese & habere impetum ad sufficientiam vitae : \textbf{ naturale est ei , | quod velit esse animal coniugale . } Sed si coniugium est \\\hline
2.1.7 & Por la qual cosa si natural cosa es al omne de auer inclinaçion e appetito al abastamiento dela uida natural cosa es a el de querer ser a i al conuigable e ayuntable a su muger \textbf{ mas si el ma termonio es cosa natural siguese } que la fornicaçion & quod velit esse animal coniugale . \textbf{ Sed si coniugium est | quid naturale , } sequitur quod fornicatio , \\\hline
2.1.7 & es generalmente de esquiuar alos çibdadanos \textbf{ assi conmo aquella que es contraria ala cosa natural . } La qual fornicaçion e general mente todo vso de luxuria non conueinble tanto & sit uniuersaliter a ciuibus vitanda , \textbf{ tanquam aliquid contrarium rei naturali : } quam videlicet fornicationem , \\\hline
2.1.7 & assi paresçe nasçer vna dubda delas cosas sobredichas . \textbf{ Ca si el casamiento es al omne natural } por ende de reprehenderes & ex dictis oriri . \textbf{ Nam si coniugium est homini naturale , } reprehensibilis est igitur \\\hline
2.1.7 & alo que ya dicho es de ligero se puede soluer . \textbf{ Ca si natural cosa es al omne de ser aianlia conuigable } qualquier que esqua de tomar mugni & de leui refellitur . \textbf{ Nam si naturale est homini esse animal coniugale , } quicunque renuit coniugem ducere , \\\hline
2.1.8 & ¶ La primera razon se toma de parte dela fe \textbf{ o de parte dela amistança natural } que deue ser entre el marido e la muger . & Prima via sumitur ex parte fidei , \textbf{ vel ex parte amicitiae naturalis , } quae debet esse inter virum et uxorem . \\\hline
2.1.8 & e esso mismo la muger al uaron . \textbf{ Ca commo entre el uaron e la muger sea amistança natural } assi commo se prueua en el viij̊ libro delas ethicas & et econuerso . \textbf{ Cum enim inter virum et uxorem sit amicitia naturalis , } ut probatur 8 Ethicorum , \\\hline
2.1.8 & assi commo se prueua en el viij̊ libro delas ethicas \textbf{ commo non sea natural amistan } ca entre algunos & ut probatur 8 Ethicorum , \textbf{ cum non fit naturalis amicitia } inter aliquos nisi obseruent sibi debitam fidem ; \\\hline
2.1.8 & que sea segunt natura \textbf{ e para que entre el uaron e la muger sea amistança natural conuiene que guarden vno a otro fe e lealtad } assi que non se puedan partir vno de otro . & ad hoc quod coniugium sit secundum naturam , \textbf{ et ad hoc quod inter uxorem et virum sit amicitia naturalis , | oportet quod sibi inuicem seruent fidem , } ita quod ab inuicem non discedant . \\\hline
2.1.8 & Et por ende el padre e la madre \textbf{ por que naturalmente aman a sus fijos } por el amor natural que han con ellos & ut sint amici inter se parentes , \textbf{ qui naturaliter diligunt suam prolem , } ex dilectione naturali \\\hline
2.1.8 & por que naturalmente aman a sus fijos \textbf{ por el amor natural que han con ellos } acresçientase entre ellos amorio natural & qui naturaliter diligunt suam prolem , \textbf{ ex dilectione naturali | quam habent ad ipsam , } augmentatur eorum amicitia naturalis . \\\hline
2.1.8 & por el amor natural que han con ellos \textbf{ acresçientase entre ellos amorio natural } Mas coͣtra odo amor aya alguna fuerça & quam habent ad ipsam , \textbf{ augmentatur eorum amicitia naturalis . } Sed cum omnis amor vim quandam unitiuam dicat , \\\hline
2.1.9 & ca assi commo dize el philosofo en el octauo libro delas ethicas \textbf{ entre ellos es amistança muy grande e muy natural . } Mas commo el grand amor non pueda ser departido amuchͣs partes & ( ut probatur 8 Ethicorum ) \textbf{ est amicitia excellens et naturalis . } Sed cum excellens amor non possit esse ad plures , \\\hline
2.1.9 & cerazon delos fijos . \textbf{ Ca commo el matermonio sea cosa natural } en qual manera se deua fazer & sumitur ex nutritione filiorum . \textbf{ Nam cum coniugium sit quid naturale : } quomodo debito modo fieri debeat \\\hline
2.1.9 & non abaste la fenbra sola \textbf{ natural cosa es alos omes } que vn uaron case con vna muger . & non sufficiat sola foemina , \textbf{ naturale est hominibus } ut unus vir uni mulieri nubat . \\\hline
2.1.9 & que vn uaron case con vna muger . \textbf{ Ca nos deuemos iudgar las cosas naturales } segunt que son en la mayor parte & ut unus vir uni mulieri nubat . \textbf{ Ea enim naturalia iudicare debemus } quae sunt ut in pluribus , \\\hline
2.1.9 & assi commo dezimos \textbf{ que natural cosa es al omne de ser diestro } commo quier que contezca a algunos de ser esquierdos en essa manera & quae sunt ut in pluribus , \textbf{ ut naturale est homini | quod sit dexter , } licet contingat aliquos esse sinistros . \\\hline
2.1.9 & para dar conuenible nudermiento alos fijos . \textbf{ Empero por que non es iudgar la cosa natural } segunt aquello que es en pocas cosas & ad praestandum filiis debitum nutrimentum , \textbf{ quia tamen naturale non est iudicandum illud quod est in paucioribus , } sed quod est ut in pluribus : \\\hline
2.1.9 & que iudguemos los omes \textbf{ que es cosa natural } que tan bien el mas o commo la fenbra & ut in hominibus iudicetur \textbf{ quid naturale , } ut tam mas quam foemina supportent onera filiorum . \\\hline
2.1.9 & e mientra los fijos han mester ayuda del padre \textbf{ e dela madre sea cosa natural } que vn mal lo se ayunte a vna fenbra . & quam diu filii indigent parentibus , \textbf{ naturale sit } ut unus masculus uni adhaereat foeminae , \\\hline
2.1.9 & Por en de liguele \textbf{ que en los omes le acola natural } que mientra que los fijos han menester ayuda del padre & ut unus masculus uni adhaereat foeminae , \textbf{ sequitur in hominibus esse quid naturale , } ut quam diu filii indigent parentibus , \\\hline
2.1.9 & commo es delas otras ainalias . \textbf{ Ca alas otras ain alias conplidamente la naturales apareia su uianda } assi commo es dichon mas conplidamente ençima en el primero capitulo deste segundo libro ¶ & sicut de animalibus aliis , \textbf{ quia eis natura sufficienter parat victum , } ut supra in 1 capit’ \\\hline
2.1.9 & Et tanto esto mas pertenesçe a los Reyes \textbf{ e alos prinçipes de segnir mas orden natural } quanto mas conuiene aellos de ser meiores & una sola uxore esse contentos . \textbf{ Et tanto magis hoc decet Reges et Principes , } quanto decet eos meliores esse aliis , \\\hline
2.1.10 & Ca en el mater moino \textbf{ primeramente es guardada la orden natural . } Canatra al cosa es que la fenbrasea subiecta aluaron & esse pluribus viris . \textbf{ In coniugio enim primo reseruatur ordo naturalis : } nam naturale est foeminam \\\hline
2.1.10 & ca non solamente el casamiento es ordenado a conseruaçion \textbf{ e aguarda dela orden natural } e apaz conueinble mas avn es ordenado a generacion de los fijos . & non solum \textbf{ ad conseruationem ordinis naturalis , } et ad debitam pacem , \\\hline
2.1.10 & Ca por esto se enbargarian le todas las quatro cosas sobredichͣs \textbf{ e por esto se tolleria la orden natural } e por esto se tolleria la concordia & quia per hoc omnia praedicta quatuor impediuntur . \textbf{ Tolletur enim ex hoc naturalis ordo ; } non resultabit inde pax et concordia , \\\hline
2.1.10 & nin les seria dado alos fijos conuenible nudermiento \textbf{ Et que por esto se tire la orden natural } esto non es guaue de prouar . & non tribuetur filiis debitum nutrimentum . \textbf{ Quod autem ex hoc tollatur naturalis ordo , } videre non est difficile . \\\hline
2.1.10 & esto non es guaue de prouar . \textbf{ Ca segunt la orden natural } assi commo paresçe & videre non est difficile . \textbf{ Nam secundum ordinem naturalem } ( ut patet per Philosophum in Polit’ ) \\\hline
2.1.10 & e la mugni deue ser subiecta e obediente al uaron ¶ \textbf{ Otrossi segunt la orden natural } en essas mismas obras & mulier vero debet esse subiecta . \textbf{ Rursus secundum ordinem naturalem } in eisdem operibus nullus aeque per se duobus \\\hline
2.1.10 & prinoste sea ordenado al Rey e sea so el . \textbf{ Et por ende contradize ala orden natural } que vno sea subiecto a dos egualmente . & oportet Propositum illum ad Regem ordinari , \textbf{ et esse sub ipso repugnat | ergo ordini naturali } eundem duobus esse subiectum . \\\hline
2.1.10 & Enpero que vno obedezca a muchos prinçipantes \textbf{ segunt que son muchos non puede ser segunt orden natural . } Por la qual cosa si cosa de denostares & ut plures sunt , \textbf{ secundum naturalem ordinem esse non potest . Quare et si detestabile est } plures foeminas coniuges esse unius viri , \\\hline
2.1.10 & por que en el casamiento \textbf{ dellos conuiene de guardar la orden natural mas que en otro ninguno . } ¶ Lo segundo esso mismo pue de ser mostrada & coniuges Regum et Principum , \textbf{ quia in eorum coniugio magis quam in alio decet | naturalem ordinem conseruare . } Secundo hoc idem inuestigari potest \\\hline
2.1.11 & La primera razon se declara assi . \textbf{ Ca commo por la orden natural deuamos auer } subiectiuo al padre e ala madre & Prima via sic patet . \textbf{ Nam cum ex naturali ordine debeamus parentibus debitam subiectionem , } et consanguineis debitam reuerentiam , \\\hline
2.1.11 & para en mater moion . \textbf{ Mas en tanto esto paresçe conuenible a razon natural } que apenas ay gentes ningunas & immo adeo videtur \textbf{ hoc naturali rationi consentaneum , } quod vix sint aliquae gentes \\\hline
2.1.11 & Onde el philosofo en las politicas \textbf{ mouiendo se con razon natural saca algunas perssonas } que non son conuenibles a mater momo . & Unde et Philosophus 2 Polit’ \textbf{ sola ratione naturali ductus | exceptuat personas aliquas a contractione connubii : } nunquam enim fuit licitum alicui , \\\hline
2.1.11 & Enpero tanto mas esto conuiene alos Reyes e alos prinçipes \textbf{ quanto mas conuiene a ellos de guardar la orden natural } ¶La segunda razon para prouar esto mesmo se toma del bien & tanto tamen hoc magis decet Reges , et Principes , \textbf{ quanto magis eos obseruare | decet ordinem naturalem . } Secunda via ad inuestigandum hoc idem , \\\hline
2.1.11 & parentescoparesca de ser amistança grande \textbf{ Por ende la razon natural dize } que los matermonios non son de fazer & ex ipsa proximitate carnis sufficiens amicitia esse videatur , \textbf{ dictat naturalis } ratio coniugia contrahenda esse inter illos \\\hline
2.1.11 & e de entendemiento \textbf{ que non den grand obra adelectaçiones dela catue . Et pues que assi es commo ayan natural amor las perssonas } que son ayuntadas en grand parentesco & vigere ratione et intellectu , \textbf{ non nimiam operam dare venereis . | Cum ergo ad personas nimia affinitate coniunctas habeatur naturalis amor , } si supra amorem illum \\\hline
2.1.12 & que el casamiento deue ser segunt natura \textbf{ por que el omne naturalmente es aian la conpannable ama termoino } ca la primera con pama natural & coniugium esse secundum naturam , \textbf{ eo quod homo naturaliter esse animal sociale : } prima autem naturalis societas ( ut patet per Philosophum in Polit’ ) \\\hline
2.1.12 & por que el omne naturalmente es aian la conpannable ama termoino \textbf{ ca la primera con pama natural } assi commo paresçe por el philosofo enlas politicas es del maslo e dela fenbra e del uaron e dela mugni . & eo quod homo naturaliter esse animal sociale : \textbf{ prima autem naturalis societas ( ut patet per Philosophum in Polit’ ) } est maris et foeminae , viri , et uxoris . \\\hline
2.1.12 & assi commo paresçe por el philosofo enlas politicas es del maslo e dela fenbra e del uaron e dela mugni . \textbf{ Mas esto non si asi el casamiento non fuesen ordenado a algua conpanna conuenible e natural . } ¶ Et pues que assi es commo deuidamente & est maris et foeminae , viri , et uxoris . \textbf{ Hoc autem non esset , | nisi coniugium ordinaretur } in quandam societatem debitam et naturalem . \\\hline
2.1.13 & ostrado es de sus qual deue ser el casamiento . \textbf{ Ca es cosa natural } e es cosa & quale debet esse coniugium , \textbf{ quia est quid naturale , } et est quid indiuisibile , \\\hline
2.1.14 & departese el gouernamiento paternal del gouernamiento matermoni al . \textbf{ Ca veemos que el gouernamiento real es mayor entado e mas natural . } Mas el gouernamiento çiuiles mas particular e por election . & differt regimen coniugale a regimine paternali . \textbf{ Videmus enim quod regimen regale est magis totale et naturale : } regimen vero politicum est magis paternale et ex electione . \\\hline
2.1.14 & si es \textbf{ derechurero sea natural } Enpero el gouernamiento çiuil & Rursus , licet omne regimen \textbf{ ( si sit rectum ) sit naturale , } attamen regimen politicum \\\hline
2.1.14 & assi con el padre a cosas eguales \textbf{ nin escogen el padre para si mas naturalmente son egendrades del padre . } Et por ende el sennorio del padre es dicho mas segina natura & ad paria eum patre , \textbf{ nec eligunt sibi patrem , | sed naturaliter producuntur ab ipso . } Dicitur dominium paternale esse \\\hline
2.1.14 & e por el ectiuo \textbf{ assi commo el omne naturalmente es despuesto a fablar } Enpero que fable en este & secundum placitum et ex electione : \textbf{ sicut homo naturaliter est aptus ad loquendum , } tamen quod loquatur hoc idiomate vel illo , \\\hline
2.1.14 & que la orden \textbf{ e la razon natural muestra . } a dixiemos de suso que en la casa ay tres gouernamientos departidos & quanto ipsi plus obseruare debent \textbf{ quae dictat ordo et ratio naturalis . } Dicebatur superius \\\hline
2.1.15 & Ende el sh̃o en el libro dela buena uentura dize \textbf{ que los mouimientos naturales } que nos auemos en el alma & et a substantiis separatis : \textbf{ unde et Philosophus in libro De bona fortuna ait , } Impetus naturales quos habemus in anima esse in nobis a Deo \\\hline
2.1.15 & non conuiene que sea ordenada a seruir . \textbf{ ¶ Et pues que assi es non es orden natural } que el marido en ssennore e a la muger & quod ordinetur ad seruiendum . \textbf{ Non est ergo naturalis ordo , } virum praeesse uxori eo regimine quo praeest dominus seruis . \\\hline
2.1.15 & por que entre los barbaros non era ninguno natrealmente sennor \textbf{ mas vn omne era naturalmente barbaro e sieruo . } Ca ser barbaro de alguno este es ser estranno del & quia inter Barbaros nullus est naturaliter principans , \textbf{ sed idem est | esse naturaliter barbarum et seruum ; } esse enim barbarum ab aliquo , \\\hline
2.1.15 & si non fuese priuado de vso de razon e de entendimiento . \textbf{ Mas commo aquel que es priuado de vso de razon e de entendemiento sea naturalmente sieruo } por que non sabe gniar assi mismo & nisi careat usu rationis et intellectus . \textbf{ Sed cum carens rationis usu sit naturaliter seruus , } quia nescit seipsum dirigere , \\\hline
2.1.15 & e conuiene que sea gado de otro este \textbf{ tal es naturalmente barbaro e sieruo } Por la qual cosa sient los barbaros han vna orden la muger & et expedit ei quod ab aliquo alio dirigatur , \textbf{ idem est esse natura barbarum et seruum . } Quare si apud Barbaros eundem habent ordinem uxor et seruus , \\\hline
2.1.15 & Et por ende si conuiene alos çibdadanos de ser sabidores \textbf{ e conosçer la manerar la orden natural } cosa muy desconuenble es a ellos de vsar delas muger & Quare si decet ciues esse industres , \textbf{ et cognoscere modum | et ordinem naturalem ; } indecens est eos uti uxoribus tanquam seruis . \\\hline
2.1.15 & e de ser menguados de razon e de encendemiento . \textbf{ Et pues que assi es de parte de la orden natural paresçe que otra cosa es el gouernamiento del marido ala mug̃r } que del señor al sieruo . & et carere ratione et intellectu . \textbf{ Ex parte igitur ordinis naturalis | patet aliud esse regimen coniugale quam seruile : } et non esse utendum uxoribus tanquam seruis . \\\hline
2.1.16 & por que assi commo veemos \textbf{ en las otras cosas naturales } que quando para fazimiento de alguna nobra es menester alguna cosa & Sic enim videmus \textbf{ in aliis rebus naturalibus , } quod quando ad productionem alicuius effectus \\\hline
2.1.17 & en la he perdat de grand mançebia demanda en quet pon deuen dar mas obra ala generaçion delos fijos . \textbf{ Et dizen que esto otorgan tan bien los naturales } commo los fisicos & magis insistendum est procreationi filiorum , \textbf{ et ait , quod tam a naturalibus , } quam a medicis conceditur , \\\hline
2.1.17 & Otrossi los poros abiertos \textbf{ salle la calentura natural . } La qual sallida fincan los cuerpos de dentro frios & Rursus , apertis poris \textbf{ exalat naturalis calor , } quo exalante corpora intrinsecus \\\hline
2.1.17 & en que vienta el cierço meior muelle la uianda \textbf{ por la calentura natural se ençierran de dentro } por el frio qual çerca de fuera & quod tempore frigido flante borea melius digerit , \textbf{ quia calor eius interius } propter frigus circunstans \\\hline
2.1.17 & por que non salle dellos \textbf{ la cal entra a natural } e sono mas humidos & tempore hyemali sunt calidiores , \textbf{ quia non exalat inde calor ; } et humidiores , \\\hline
2.1.18 & por el temor del coraçon . \textbf{ Ca commo las muger ssean naturalmente temerosas } en tanto que semeia & ex timiditate cordis . \textbf{ Nam cum mulieres sint naturaliter adeo timidae , } quod quasi omnia expauescunt ; \\\hline
2.1.22 & Ca assi conuiene a cada vn marido de auer çelo ordenado de su mugni \textbf{ por que sea entre ellos amistança natural delectable e honesta } L consseio delas mugers & erga suam coniugem ornatum habere zelum , \textbf{ ut sit inter eos amicitia naturalis delectabilis , et honesta . } Consilium mulierum , \\\hline
2.1.24 & que la conpannia del uaron \textbf{ e dela mug̃ sea natural } et en qual manera la conpannia dela casa & cum ostensum sit , \textbf{ quomodo communitas viri et uxoris sit naturalis , } et quomodo se habeant \\\hline
2.2.1 & por aquello que conuiene entre ellos \textbf{ de ser amistança natural ¶ } La primera razon se praeua assi . & ex eo quod inter eos debet \textbf{ esse amicitia naturalis . } Prima via sic patet . \\\hline
2.2.1 & La primera razon se praeua assi . \textbf{ Ca assi commo veemos en las cosas naturales } que nunca la natura da sera algunan cosa & Prima via sic patet . \textbf{ Nam sicut in naturalibus rebus aspicimus } quod nunquam natura dat esse alicui , \\\hline
2.2.1 & e razon de los fijos \textbf{ e los fijos naturalmente han el ser de los padres . Conuiene alos padres de auer cuydado de los fijos } e ser cuydadosos dellos & et filii naturaliter \textbf{ a patribus esse habent , | decet patres habere curam filiorum , } et solicitari erga eos , \\\hline
2.2.1 & e deuen enssennorear a ellos \textbf{ por que naturalmente las cosas de suso } enbian su uirtudalas de yuso & et debent praeesse eis . \textbf{ Naturaliter enim semper superiora in inferiora influunt , } et ea regulant et conseruant : \\\hline
2.2.1 & Por la qual \textbf{ cosasi natural cosa es } que los cuerpos de suso & et ea regere , et conseruare . \textbf{ Quare si naturale est , } ut superiora et praeeminentia in inferiora influant , \\\hline
2.2.1 & e grand prouidençia de todo el mundo . \textbf{ ¶ Et pues que assi es por que los padres enssenore a naturalmente } alos fijos deuen sor muy & et prouidentiam totius Uniuersi . \textbf{ Patres ergo eo ipso quod naturaliter praesunt filiis , } debent circa eorum regimen esse soliciti . \\\hline
2.2.1 & assi commo se praeua en el viij delas . ethicas . \textbf{ Conuiene que los padres por amor natural } que han alos fijos sean cuy dadosos della & ut probatur 8 Ethicorum , \textbf{ decet patres ex ipso amore naturali , } quem habent ad filios , \\\hline
2.2.2 & mas ha cuydado de sus fijos \textbf{ Ca natural cosa es que cada vno ame sus obras } assi commo dize el philosofo en las ethicas & magis habet solicitudinem circa filios : \textbf{ naturale est enim quemlibet diligere sua opera , } ut Philosophus in Ethicorum \\\hline
2.2.2 & assi commo dize el philosofo en las ethicas \textbf{ Onde los padres naturalmente aman los fijos } e los poetas sus ditados & ut Philosophus in Ethicorum \textbf{ unde et patres naturaliter diligunt filios , } et poetae sua poemata tanquam proprium opus . \\\hline
2.2.2 & que los Reyes e los prinçipes \textbf{ e generalmente todos los señores sy de una naturalmente ensseñorear } conuiene les que ayan sabidia e entendimiento . & et Principes et uniuersaliter omnes dominantes , \textbf{ si debeant naturaliter dominari , } oportet quod polleant prudentia et intellectu : \\\hline
2.2.3 & qł padre toma comie y de amor¶ \textbf{ La primera razon se torna de la orden natural } ¶La segunda da partertian del padre¶ & paternale regimen trahere originem ex amore . \textbf{ Prima via sumitur ex ordine naturali . } Secunda ex ipsa perfectione patris . \\\hline
2.2.3 & Por la qual cosa si el gonernamiento del padre desto tora a comienco \textbf{ por que el fijo naturalmente es vria semerança que desçende del cadre . } Canmo segunr nacsta & ex hoc sumit originem , \textbf{ quia filius naturaliter est | quaedam similitudo procedens a patre : } cum secundum naturam ad huiusmodi similia fit dilectio , \\\hline
2.2.3 & e para cerar sus fijos \textbf{ assi en ellos es natural appetito } para los amar & et ad filios procreandum : \textbf{ sic est in eis naturalis impetus } ad eos diligendum , \\\hline
2.2.4 & que los fiios alos padres \textbf{ Ca si entre los padres e los fijos es amor natural } tanto este amor es mas guande & quam econuerso . \textbf{ Nam si inter parentes et filios est amor naturalis , } tanto huiusmodi amor est validior , \\\hline
2.2.4 & e que los mantengan . \textbf{ Et non es natural cosa } que las de diyuso enbien su uirtud alas de suso & ø \\\hline
2.2.6 & que los moços son de enssennar en su ninnes en bueans costunbres ¶ \textbf{ La primera se toma dela naturaleza dela delectacion } ¶La segunda del fallesçimiento de razon . & quod ab ipsa puerilitate instruendi sunt pueri ad bonos mores . \textbf{ Prima via sumitur | ex naturalitate delectationis . } Secunda , ex rationis defectu . \\\hline
2.2.6 & Ca segunt dize el philosofo en las ethicas \textbf{ en tanto es natural } anos de nos delectar enla ninnes & in Ethic’ adeo \textbf{ connaturale est nobis delectari , } quod ab ipsa infantia delectari incipimus : \\\hline
2.2.6 & luego en la moçedat deuemos poner freno e contradezir ala cobdiçia enla nr̃a moçedat . \textbf{ Et paues que assi es dela naturaleza } e dela delectaçion paresçe & ab ipsa infantia est tali concupiscentiae resistendum : \textbf{ ex ipsa ergo connaturalitate delectationis , } statim cum pueri sunt sermonum capaces , \\\hline
2.2.7 & e alas sçiençias liberales . \textbf{ Ca assi conmodicho es de suso ninguno non es dich̃ sennor naturalmente } si non fuere enoblesçido & ab ipsa infantia eos tradere literalibus disciplinis . \textbf{ Nam ( ut superius dicebatur ) | nullus est naturaliter dominus , } nisi vigeat prudentia et intellectu . \\\hline
2.2.7 & por que puedan \textbf{ enssennorear mas sabiamente e mas natural mente . } Mas podriemos para prouar esto mismo & quanto decet eos intelligentiores et prudentiores esse , \textbf{ ut possint naturaliter dominari . } Posset autem ad hoc idem alia ratio adduci . \\\hline
2.2.8 & por ende segunt que dize este mismo philosofo \textbf{ la musica es conueinble ala naturaleza de los mançebos } por que les muestra & quare ( secundum eundem Philosophum ) \textbf{ musica est consentanea naturae iuuenum , } quia habent innocuas delectationes . \\\hline
2.2.8 & qua non estas . \textbf{ Ca la natural ph̃ia } que muestra conosçer las naturas delas cosas & longe nobiliores istis . \textbf{ Nam Naturalis Philosophia docens } cognoscere naturas rerum , \\\hline
2.2.8 & La qual sçiençia es sola geometera . \textbf{ Et la sçiençia dela fisica es sola ph̃ia natural . } Et las leyes e los derechos & est sub Geometria . \textbf{ Medicina vero est | sub naturali Philosophia . } Leges et iura , \\\hline
2.2.8 & delas otras sçiençias falladas por los omes . \textbf{ Et en pos estos deuen ser mas honrrados los philosofos naturales } por que lph̃ia natural & metaphysica primatum tenet . \textbf{ Post hos quidem honorari debent naturales Philosophi : } quia naturalis Philosophia licet \\\hline
2.2.8 & Et en pos estos deuen ser mas honrrados los philosofos naturales \textbf{ por que lph̃ia natural } commo quier que sea a quande dela methasisica . & Post hos quidem honorari debent naturales Philosophi : \textbf{ quia naturalis Philosophia licet } sit infra metaphysicam , \\\hline
2.2.11 & por que con ellos mascassen bien la uianda \textbf{ por que mas liga mente passasse la calentura natural ala bianda } e dende sen signiria & Ordinauit enim natura animalibus dentes , ut per eos cibus debite tritus , \textbf{ facilius pateretur a calore naturali , } et per consequens facilius conuerteretur in nutrimentum : \\\hline
2.2.11 & e se conuierte en fuego . \textbf{ Mas esta orden natural en la mayor parte non la guardan } los que toman la uianda muy & et conuertuntur in ignem . \textbf{ Hunc autem ordinem naturalem , | ut plurimum non obseruant } sumentes cibum auide . \\\hline
2.2.11 & Ca si la vianda se ouiere bien a cozer \textbf{ conuiene que sea bien proporçionada ala calentura natural } Por la qual cosa si en tan grand quantia se & Si enim cibus digeri debeat , \textbf{ oportet | ipsum esse proportionatum calori naturali . } Quare si in tanta quantitate sumatur , \\\hline
2.2.11 & Por la qual cosa si en tan grand quantia se \textbf{ tomaque la calentura natural non pueda } enssennorar sobrella non se puede bien moler nin cozer . & Quare si in tanta quantitate sumatur , \textbf{ quod calor naturalis } ei dominari non possit , non bene digeritur , \\\hline
2.2.11 & e destenprados e avn danan sus cuerpos . \textbf{ Ca todas las obras naturales } segunt el philosofo son mesuradas & et etiam laeditur secundum corpus . \textbf{ Nam omnes actiones naturales } secundum Philosophum \\\hline
2.2.13 & e el entendimiento non siente \textbf{ assi los mouimientos naturales ni obra } assi por natra al inclinaçion & non sic percipit naturales impetus , \textbf{ nec sic agit ex naturali instinctu , } ut aues et bestiae . \\\hline
2.2.13 & ca la hedat dela vegez \textbf{ por que ha menos de calentura natural de dentro } mas ha menester calentraa de fuera . & quia senilis aetas , \textbf{ eo quod magis caret calore naturali intrinseco , } magis indiget de calore exteriori . \\\hline
2.3.2 & que el que non ha alma . \textbf{ Ca si la conpania dela casa es cosa natural } e en la natura todas las cosas son ordenadas & ut organum animatum ante inanimatum . \textbf{ Si enim societas domus est quid naturale , } et in natura ordinata sunt omnia , \\\hline
2.3.5 & maspodemos prouar \textbf{ por tres razons la possession delas cosas es natural en algua manera al omne } ¶La primera se toma dela neçessidat dela uida . & Possumus autem triplici via venari , \textbf{ quod rerum possessio est quodammodo naturalis . } Prima sumitur ex necessitate vitae . \\\hline
2.3.5 & La primera se prueua \textbf{ assi casi los omes biuen naturalmente } e la conpanna dela çibdat es en alguna manera natural al omne & Tertia , ex actione naturae . \textbf{ Si enim homines naturaliter viuunt , } et societas politica est quodammodo homini naturalis , \\\hline
2.3.5 & assi casi los omes biuen naturalmente \textbf{ e la conpanna dela çibdat es en alguna manera natural al omne } assi commo es prouado & Si enim homines naturaliter viuunt , \textbf{ et societas politica est quodammodo homini naturalis , } ut in prima parte huius secundi libri diffusius probabatur : \\\hline
2.3.5 & conuiene en algunan manera \textbf{ que las cosas naturales sean neçessarias enla uida politica . } mas segunt el philosofo & oportet aliquomodo naturalia esse \textbf{ quae sunt necessaria in vita politica ; } sed secundum Philosophum primo Polit’ \\\hline
2.3.5 & sirue ala neçessidat dela uida \textbf{ en alguna manera es natural al omne } ¶ & eo ergo ipso quod rerum possessio deseruit necessitati vitae , \textbf{ est quodammodo homini naturalis . } Secunda via ad inuestigandum \\\hline
2.3.5 & conparaconn de las cosas corporales e sensibles \textbf{ por ende han señorio natural sobrellas . } por la qual cosa natural cosa es al ome & et sensibilium est creaturarum dignissima , \textbf{ habet naturale dominium super ipsa : } quare naturale est homini \\\hline
2.3.5 & por ende han señorio natural sobrellas . \textbf{ por la qual cosa natural cosa es al ome } que enssennore e a estas cosas senssibles & habet naturale dominium super ipsa : \textbf{ quare naturale est homini } quod dominetur istis sensibilibus , \\\hline
2.3.5 & enł primer libro delas politicas do prueua \textbf{ que la possession de tales cosas es natural . } dize que naturalmente es batalla derecha de los omes contra las bestias & Unde et Philosop’ 1 Politic’ \textbf{ ubi probat possessionem talium naturalem esse , } ait , quod hominum ad bestias naturaliter est iustum bellum : \\\hline
2.3.5 & que la possession de tales cosas es natural . \textbf{ dize que naturalmente es batalla derecha de los omes contra las bestias } por que las bestias deuen ser suiebtas del omne & ubi probat possessionem talium naturalem esse , \textbf{ ait , quod hominum ad bestias naturaliter est iustum bellum : } eo enim quod bestiae naturaliter homini debent esse subiectae , \\\hline
2.3.5 & Et pues que assi es \textbf{ assi commo el omne naturalmente } enssennorea alas bestias avn en essa misma manera naturalmente & quibus naturaliter dominabitur . \textbf{ Sicut ergo homo naturaliter dominatur bestiis , } sic et naturaliter dominatur aliis exterioribus rebus ; \\\hline
2.3.5 & assi commo el omne naturalmente \textbf{ enssennorea alas bestias avn en essa misma manera naturalmente } enssennorea a todas las otras cosas de fuera . & Sicut ergo homo naturaliter dominatur bestiis , \textbf{ sic et naturaliter dominatur aliis exterioribus rebus ; } quod non esset , \\\hline
2.3.5 & la qual cosa non podrie ser \textbf{ si la possession delas cosas en algua manera non fuesse natural al omne } ¶ & quod non esset , \textbf{ nisi rerum possessio | quodammodo naturalis esset . } Tertia sumitur \\\hline
2.3.5 & non fallesçe alas ainalias \textbf{ que non son acabadas mas naturalmente apareia el nudmiento conuenible a ellas mas conueible cosa es } que les non fallesca & non deficit animalibus , \textbf{ sed naturaliter praeparat eis debitum nutrimentum ; | congruum est } ut non deficiat eis iam perfectis . \\\hline
2.3.5 & Et pues que assi es natal cosa es a nos de auer las cosas de fuera \textbf{ e por ende el sennorio delas cosas de fuera es en algua manera natural al omne . } por que la natan engendro & habere res exteriores . \textbf{ Habere ergo dominium rerum exteriorum est quodammodo homini naturale : } quia natura produxit \\\hline
2.3.5 & e engendrar otros semeiables \textbf{ assi es cosa natural } enpero son muchs & et per generationem producere sibi similia , \textbf{ est homini naturale : } nihilominus tamen multi sunt \\\hline
2.3.6 & caueemos que los hͣrmaros fijos de vn padre entre los quales seg̃tel philosofo enł . viij̊ \textbf{ libro delas ethicas es amistança natural } en la mayor parte veemos los contender e uaraiar sobre la heredat & videmus enim ipsos fratres ex eodem patre natos , \textbf{ inter quos secundum Philosophum 8 Ethicorum est amicitia naturalis , } ut plurimum bellare ad inuicem , \\\hline
2.3.7 & ca los omes barbaros e siluesttes e montesmos \textbf{ por que fallesçen de uso de razon naturalmente } deuen ser subietos alos omes & homines enim barbari syluestres , \textbf{ quia ab usu rationis deficiunt , } naturaliter debent \\\hline
2.3.7 & que los sabios de una ser \textbf{ sennorsnaturalmente de los non sabios } e por ende han batalla decha contra ellos & Videtur enim velle , \textbf{ quia sapientes naturaliter debent dominari insipientibus , } iustum habere bellum contra ipsos , \\\hline
2.3.8 & mas si quiere gouernar su casa \textbf{ segunt manera natural } non deue dessear possessiones & sed vult suam domum regere \textbf{ secundum modum et ordinem naturalem , } non debet infinitas diuitias \\\hline
2.3.10 & e abondar en vino \textbf{ e entgo es abondar en riquezas e possessiones naturales } mas abondar endmeros & et abundare vino et frumento , \textbf{ est abundare in diuitiis naturalibus : } sed abundare in denariis et numismatibus , \\\hline
2.3.10 & despues del tractado delas possessiones \textbf{ delas quales nasçen las riquezas naturales } se sigue el tractado delas monedas & congrue post tractatum de possessionibus , \textbf{ ex quibus oriuntur diuitiae naturales , } annectitur tractatur de numismatibus , \\\hline
2.3.10 & pone quatro maneras de dineros conuiene saber . \textbf{ Natural . } Et canssoria de canbio . & in Poli’ \textbf{ quatuor species pecuniatiuae : } videlicet naturalem , campsoriam , obolostaticam , \\\hline
2.3.10 & La primera manera dela pecunia es dichͣ \textbf{ por aquello que las cosas naturales se mudan en dineros . } Et por ende si alguno abondasse en vino & Prima ergo species ipsius pecuniatiuae \textbf{ dicitur esse quasi naturalis : | quae fit ex eo quod res naturales commutantur in pecuniam . } Si quis ergo abundans in vino et frumento , \\\hline
2.3.10 & Et por ende si alguno abondasse en vino \textbf{ e entrago el qual resçibiesse naturalmente destas cosas dineros . } tales dineros e tal riqueza serie dichͣ & Si quis ergo abundans in vino et frumento , \textbf{ quae naturaliter producuntur , | ex eis pecuniam susciperet , } talis pecuniatiua quasi naturalis diceretur , \\\hline
2.3.10 & tales dineros e tal riqueza serie dichͣ \textbf{ assi commo natural } por razon que ha comienco delas cosas naturales . & ex eis pecuniam susciperet , \textbf{ talis pecuniatiua quasi naturalis diceretur , } quia a rebus naturalibus inciperet . \\\hline
2.3.10 & assi commo natural \textbf{ por razon que ha comienco delas cosas naturales . } ¶ la segunda manera de los dineros es dichͣ camiadora . & talis pecuniatiua quasi naturalis diceretur , \textbf{ quia a rebus naturalibus inciperet . } Secunda species pecuniatiuae \\\hline
2.3.10 & por razon de ganer ardiños \textbf{ e Mas esta arte pecumatiua de dineros non deue ser dicha natural } por que non com . & esset causa lucrandi pecuniam . \textbf{ Haec enim pecuniatiua , | naturalis dici non debet ; } quia nec a rebus naturalibus incipit , \\\hline
2.3.10 & por que non com . \textbf{ iença de cosas naturales } nin se termina en cosas naturales & naturalis dici non debet ; \textbf{ quia nec a rebus naturalibus incipit , } nec ad naturalia terminatur . \\\hline
2.3.10 & iença de cosas naturales \textbf{ nin se termina en cosas naturales } mas en esta arte & quia nec a rebus naturalibus incipit , \textbf{ nec ad naturalia terminatur . } Sed in ea \\\hline
2.3.10 & la primera que es assi commo \textbf{ yconomica es natural } e es de loarmas las & secundum Philosophum in Polit’ sola prima , \textbf{ quae est quasi oeconomica et naturaliter , } est laudabilis . \\\hline
2.3.10 & que es i conomica \textbf{ e assi commo natural } ca conuiene alos Rey & Nam primam speciem pecuniatiuae , \textbf{ quae est oeconomica et quasi naturalis , decet . } Decet enim ipsos abundare \\\hline
2.3.11 & ca parir e engendrar e amuchiguar se las cosas en si mismas \textbf{ es cosa proprea en las cosas naturales } e es contra natura en las cosas artifiçiales & et multiplicari in seipsis , \textbf{ est proprium naturalibus , } et est contra naturam artificialium . \\\hline
2.3.11 & Mas las aianlias \textbf{ por que son cosas naturales } quando estan en vno engendran & nunquam se ipsas multiplicant . \textbf{ Animalia vero quia sunt res naturales , } simul manentia generant , \\\hline
2.3.11 & que non se fagan \textbf{ por que son contrael derecho natural } erca la fin del primero libro delas politicas & ne fiant eo \textbf{ quod iuri naturali contradicant . } Circa finem primi Polit’ distinguit Philosophus diuersos modos , \\\hline
2.3.13 & e nos mostraremos primeramente \textbf{ que alguna suidunbre es dichͣ natural } e que conuiene que alg ssean subietos naturalmente a algunos otros & ut de seruis . \textbf{ Ostendemus enim primo seruitutem aliquam naturalem esse , } et quod naturaliter expedit aliquibus aliis esse subiectos : \\\hline
2.3.13 & que alguna suidunbre es dichͣ natural \textbf{ e que conuiene que alg ssean subietos naturalmente a algunos otros } la qual cosa praeua el philosofo & Ostendemus enim primo seruitutem aliquam naturalem esse , \textbf{ et quod naturaliter expedit aliquibus aliis esse subiectos : } quod probat Philosophus primo Polit’ quadruplici via , \\\hline
2.3.13 & que nunca algunas cosas \textbf{ muchas fazen naturalmente alguna cosa } que sea vna si non & quod nunquam aliqua plura constituunt \textbf{ naturaliter aliquid unum , } nisi ibi naturaliter aliud sit praedominans : \\\hline
2.3.13 & Et commo la conpannia de los omes sean a falpor el ome es aianlia \textbf{ aconpannable naturalmente } assi commo es prouado de suso & cum societas hominum sit naturalis , \textbf{ quia homo est naturaliter animal sociale , } ut superius diffusius probabatur , \\\hline
2.3.13 & assi commo es prouado de suso \textbf{ mas conplidamente nunca de mucho omes se faria naturalmente } vna conpannia o vna poliçia & ut superius diffusius probabatur , \textbf{ numquam ex pluribus hominibus fieret naturaliter una societas vel una politia , } nisi naturale esset \\\hline
2.3.13 & vna conpannia o vna poliçia \textbf{ si naturalmente non fuesse a ellos conuenible que algunos fuessen sennores } e alguons sieruos . & numquam ex pluribus hominibus fieret naturaliter una societas vel una politia , \textbf{ nisi naturale esset | aliquos principari } et aliquos seruire . \\\hline
2.3.13 & e alguons sieruos . \textbf{ Et pues que assi es algunos son natraalmente sieruos e alguas naturalmente senores ¶ } La segunda razon para prouar esto mismo se toma de aquellas cosas & et aliquos seruire . \textbf{ Sunt ergo aliqui naturaliter domini , | et aliqui naturaliter serui . } Secunda via ad inuestigandum \\\hline
2.3.13 & assi commo el cuerpo al alma siguesse \textbf{ que aquellos sean naturalmente sieruos } Et por que algunos son menguados de entendimiento e de sabideria & quasi corpus ad animam , \textbf{ sequitur eos esse naturaliter seruos . } Sunt enim aliqui carentes prudentia et intellectu , \\\hline
2.3.13 & por que el omne \textbf{ assi como es diche de suso naturalmente } enssennorea alas & homo enim \textbf{ ( ut supra dicebatur ) | naturaliter dominatur bestiis . } Videmus enim multas bestias domesticas , \\\hline
2.3.13 & Et pues que assi es conuieneleᷤ \textbf{ e es cosa natural } aellos de ser subietos al omne & Expedit ergo eis , \textbf{ et naturale est ipsis subiici homini ; } eo quod per hominum prudentiam \\\hline
2.3.13 & nin razon assi commo las bestias non saben enderesçar nin gniar assi mismos . \textbf{ Por ende assi commo natural cosa es alas bestias } de seruir alos omes & nesciant seipsos dirigere : \textbf{ sicut naturale est bestias seruire hominibus , } sic naturale est ignorantes \\\hline
2.3.13 & de seruir alos omes \textbf{ assi cosa naturales alos non sabios } de ser suiebtos alos sabios . & sicut naturale est bestias seruire hominibus , \textbf{ sic naturale est ignorantes } subiici prudentibus \\\hline
2.3.13 & por ende se sigue \textbf{ que algunos sean naturalmente subietos e sieruos } Pot la qual cosa la piudunbre es en alguna manera cosa natural & a rationis usu quam foeminae a viris , \textbf{ sequitur eos naturaliter esse subiectos . } Quare seruitus est \\\hline
2.3.13 & que algunos sean naturalmente subietos e sieruos \textbf{ Pot la qual cosa la piudunbre es en alguna manera cosa natural } Et natural mente conuiene ala conpannia human & sequitur eos naturaliter esse subiectos . \textbf{ Quare seruitus est | aliquo modo quid naturale , } et naturaliter expedit societati humanae aliquos seruire , \\\hline
2.3.13 & Pot la qual cosa la piudunbre es en alguna manera cosa natural \textbf{ Et natural mente conuiene ala conpannia human } al quaalgers sean sieruos e alguas sennors assi comms es dicho en el comienço del capitulo & aliquo modo quid naturale , \textbf{ et naturaliter expedit societati humanae aliquos seruire , } et aliquos principari , \\\hline
2.3.14 & alos estables çedores delas leyes \textbf{ que lin la hudunbre natural } segunt la qual los nesçios & sic visum fuit conditoribus legum , \textbf{ quod praeter seruitutem naturalem , } secundum quam ignorantes debent seruire sapientibus , \\\hline
2.3.14 & o segunt el alma o segunt el cuerpo . \textbf{ Ca cada vno de los omes es naturalmente establesçido e conpuesto destas dos partes . } Mas el auentaia que es & vel secundum corpus . \textbf{ Nam homo quilibet naturaliter | ex his duabus partibus est compositus . } Excessus autem secundum bona animae , \\\hline
2.3.14 & e bienes de fuera \textbf{ non fazen sennorio natural sinplemente . } Mas mas fazen sennorio legal & quae sunt bona corporalia et exteriora , \textbf{ non faciunt dominium simpliciter naturale , } sed magis faciunt ipsum legale et positiuum . \\\hline
2.3.14 & a aquellos que vençio \textbf{ e esto non es derech natural } mas es derech segunt prigon de ley . & dominetur iis quos debellauit , \textbf{ non est iustum naturale , } sed secundum promulgacionem legis . \\\hline
2.3.15 & sdemos dezip que son quatro manas de aministradors e de sirmientes . \textbf{ ca algunos son tałs naturalmente . } mas otros son tales & Possumus dicere quatuor esse maneries ministrorum . \textbf{ Nam aliqui sunt tales naturaliter , } aliqui vero ex lege , \\\hline
2.3.15 & por que fallesçen en los bienes del alma \textbf{ segunt el sobrepinamiento de los quales bienes conuiene a algunos de ser senno res dellos natural mente . } Mas aquellos que non son poderosos & Deficiunt enim in bonis animae , \textbf{ secundum quorum excessum contingit | aliquos naturaliter dominari . } Impotentes vero contingit \\\hline
2.3.15 & tenporal esto deue ser despues de aquel bien que entiende . \textbf{ Mas conuiene de dar a ministraçion de alquiler e de amor sin la ministt̃ion natural et segunt ley . } Ca por que en nos es el appetito corrupto & hoc debet esse ex consequenti . \textbf{ Oportuit autem dare ministrationem conductam et dilectiuam | praeter ministrationem naturalem } et secundum legem : \\\hline
2.3.15 & non guardamos \textbf{ sienpre la orden natural . } Et por ende en la mayor parte los prinçipados e los sennorios son malos & nam quia est in nobis corruptio appetitus , \textbf{ et non semper reseruamus ordinem naturalem , } ut plurimum principatus sunt peruersi : \\\hline
2.3.19 & que son departidas maneras delos seruientes \textbf{ ca algunos son sieruos naturalmente } e algersson seruientes por ley & diuersa esse maneries seruientium , \textbf{ quia quidam sunt serui naturaliter , } quidam ex lege , \\\hline
2.3.19 & alguaque ende espaauer . \textbf{ Et pues que assi es commo alos que son sieruos naturalmente } non londe descobrar las poridades & magis quam merces aliqua quam inde habituri essent . \textbf{ Cum seruis ergo naturaliter non sunt communicanda secreta neque consilia : } quia ( ut supra dicebatur ) \\\hline
2.3.19 & por que assi commo es dicho \textbf{ dessuso nunca alguno es dicho naturalmente sieruo } si non aquel que non es sabio & quia ( ut supra dicebatur ) \textbf{ nunquam est | quis naturaliter seruus , } nisi sit inscius , \\\hline
2.3.20 & que tal cosa commo esta \textbf{ contradize ala orden natural . } ¶ La segunda de aquello que contradize ala bondat delas buenas costunbres . & Prima via sumitur \textbf{ ex eo quod hoc repugnat ordini naturali . } Secunda ex eo quod contradicit bonitati morum . \\\hline
2.3.20 & e por que el vno non enbargue al otro \textbf{ contra natural ordenes } quando por aquel estrumento entendemos fazen vna de aquellas cosas & et ne unum impediat aliud , \textbf{ contra naturalem ordinem est } cum per illud organum intendimus \\\hline
2.3.20 & assi conmoparagostar e para fablar . \textbf{ Et por ende contra orden natural es } que quando nos usamos deste estrumento & ut in gustum , \textbf{ et locutionem contra naturalem ordinem est , } cum huiusmodi organum exercemus \\\hline
2.3.20 & que es el fablar \textbf{ contradize ala orden natural . } Lo segundo contradize ala bondat delas costunbres . & quod est loqui , \textbf{ repugnat ergo hoc ordini naturali . } Secundo repugnat bonitati morum . \\\hline
2.3.20 & e los prinçipeᷤ alos quales conuiene ser muy tenprados \textbf{ e guardar la orden natural en toda } meranera deuen ordenar en sus mesas & quos decet maxime temperatos esse , \textbf{ et obseruare ordinem naturalem } omnino in suis mensis , \\\hline
2.3.20 & conuiene de escusar muchedunbre de palabras \textbf{ por que non sea tirada la ordenn natural } e por que non parezcan destenprados & Sed si recumbentes , \textbf{ ne tollatur naturalis ordo , } et ne intemperati appareant , \\\hline
3.1.1 & a algun bien alguas uezes \textbf{ a inclinaçion e amouemiento naturala aquel bien . } Et alguas uezes somos inclina dos a aquel bien & ad aliquod bonum , \textbf{ aliquando ad bonum illud habemus impetum a natura , } aliquando quasi ex corruptione naturae . \\\hline
3.1.1 & que establescen la çibdat \textbf{ por que han natural inclinaçion } han establesçimiento della & ad homines ciuitatem constituentes , \textbf{ eo quod habent naturalem impetum ad constitutionem eius , } ciuitas non solum constituta est gratia eius \\\hline
3.1.1 & por conparaçion alas otras comuidades . \textbf{ Ca commo quier que toda comun dar natural sea ordenada a bien } enpero mayormente es ordenada aquel bien la comunidat & per comparationem ad ciuitates alias . \textbf{ Nam licet omnis communitas naturalis ordinetur ad bonum , } maxime tamen ordinatur \\\hline
3.1.3 & udaron alguons \textbf{ si la çibdat es cosa natural } e si el omne es natraalmente aian l . politicas . & Dubitant nonnulli , \textbf{ An ciuitas sit aliquid | secundum naturalem ? } et An homo sit naturaliter \\\hline
3.1.3 & e do quier que es sienpre faze obra de escalentamiento \textbf{ por la qual cosa si la çibdat fuesse alguna cosa natural } e el omne naturalmente fuesse aianl çiuil nunca & habet exercere calefactionis actum . \textbf{ Quare si ciuitas | quid naturale esset , } et homo naturaliter esset animal ciuile , \\\hline
3.1.3 & por la qual cosa si la çibdat fuesse alguna cosa natural \textbf{ e el omne naturalmente fuesse aianl çiuil nunca } si e fallado omne & quid naturale esset , \textbf{ et homo naturaliter esset animal ciuile , } nullus reperiretur homo non ciuilis . \\\hline
3.1.3 & aianlçiuil \textbf{ ca non es esto assi natural al omne } conmoes cosa natural al fuego de escalentar & quomodo naturale est homini esse animal ciuile . \textbf{ Non enim hoc est sic homini naturale , } sicut est naturale igni calefacere , \\\hline
3.1.3 & ca non es esto assi natural al omne \textbf{ conmoes cosa natural al fuego de escalentar } e ala piedra de desçender ayuso & Non enim hoc est sic homini naturale , \textbf{ sicut est naturale igni calefacere , } et lapidi deorsum tendere : \\\hline
3.1.3 & por que estas cosas tales \textbf{ assi parte nesçen naturalmente al fuego e ala piedra } que non se pueden acostunbrar al contrario . & et lapidi deorsum tendere : \textbf{ quia talia sic eis naturaliter competunt } quod ad contrarium assuefieri non possunt , \\\hline
3.1.3 & mas el oen non es \textbf{ assi naturalmente aianl çiuil } mas es dicho qual conuiene naturalmente de seraianl ciuil & et semper ignis calefacit , \textbf{ homo ergo non sic naturaliter est animal ciuile , } sed dicitur ei hoc naturaliter conuenire , \\\hline
3.1.3 & assi naturalmente aianl çiuil \textbf{ mas es dicho qual conuiene naturalmente de seraianl ciuil } por que ha alguna inclinaçion & homo ergo non sic naturaliter est animal ciuile , \textbf{ sed dicitur ei hoc naturaliter conuenire , } quia habet quendam impetum \\\hline
3.1.3 & por que ha alguna inclinaçion \textbf{ e algun appetito natural } para benir ciuilmente e en conpanna e las cosas & quia habet quendam impetum \textbf{ et quandam aptitudinem naturalem , } ut ciuiliter viuat . \\\hline
3.1.3 & para benir ciuilmente e en conpanna e las cosas \textbf{ que assi son naturales } por algun caso . & ut ciuiliter viuat . \textbf{ Quae autem sic sunt naturalia , } ex casu vel ex aliquo impedimento \\\hline
3.1.3 & assi commo dezimos \textbf{ que maguera natural cosa sea al omne de ser diestro } enpero much sson fallados simestris & siue ex aliqua causa impediri possunt , \textbf{ ut licet naturale sit | homini esse dextrum , } multi tamen ex aliquo impedimento \\\hline
3.1.3 & por algun enbargo o por algua otra cosa bien \textbf{ assi maguera natural cosa sea al omne de beuir çiuil mente . } Enpero muchos son fallados canpesinos e montanneses & vel ex aliqua causa reperiuntur abdextri : \textbf{ sic licet naturale sit homini viuere ciuiliter , } reperiuntur tamen multi campestre viuentes . \\\hline
3.1.4 & por las quales se prouaua \textbf{ que la çibdat non era cosa natural } e que el oen non era naturalmente aianlçiuil . & per quas probari uidebatur , \textbf{ ciuitatem non esse aliquid secundum naturam , } et hominem non esse naturaliter animal ciuile . \\\hline
3.1.4 & que la çibdat non era cosa natural \textbf{ e que el oen non era naturalmente aianlçiuil . } Et pues que assi es commo non cunpla soluer la & ciuitatem non esse aliquid secundum naturam , \textbf{ et hominem non esse naturaliter animal ciuile . } Cum ergo non satis sit \\\hline
3.1.4 & que muestren \textbf{ que la çibdat es cosa natural } e que el omne es naturalmente aianlçiuil & intendimus in hoc capitulo adducere rationes ostendentes ciuitatem esse quid naturale , \textbf{ et hominem esse naturaliter animal ciuile . } Possumus autem duplici uia ostendere \\\hline
3.1.4 & que la çibdat es cosa natural \textbf{ e que el omne es naturalmente aianlçiuil } e podemos por dos razones mostrat & intendimus in hoc capitulo adducere rationes ostendentes ciuitatem esse quid naturale , \textbf{ et hominem esse naturaliter animal ciuile . } Possumus autem duplici uia ostendere \\\hline
3.1.4 & assi ca fue prouado dessuso \textbf{ que beuir es cosa natural al omne } e por que la nafa non pueda fallesçer en las cosas neçessarias & Probabatur enim supra , \textbf{ quod uiuere erat homini secundum naturam ; } ut natura non deficiat in necessariis , \\\hline
3.1.4 & conuiene \textbf{ que sea cosa natural todo aquello } que sirue a conplimiento de uida & ut natura non deficiat in necessariis , \textbf{ oportet quid naturale esse } quicquid secundum se deseruit \\\hline
3.1.4 & e por ende la comunidat dela casa \textbf{ e avn del uartio son cosas naturales } por que siruen al conplimiento dela uida humanal & Communitas ergo domestica \textbf{ et etiam uici naturalia sunt , } quia deseruiunt ad sufficientiam uitae humanae . \\\hline
3.1.4 & lo que es fin dela generaçion \textbf{ delas cosas naturales es cosa natural } e es nata delas cosas engendradas & quod est finis generationis naturalium , \textbf{ est quid naturale , } et est natura ipsorum generatorum : \\\hline
3.1.4 & assi commo lo que es fin dela generaçio del omne \textbf{ es cosa natural al omne } e es nata del omne & ut quod est finis generationis hominis \textbf{ est quid naturale , } et est natura eius . \\\hline
3.1.4 & la qual forma por sobrepuiança es cosa natraal e es essa misma natura . \textbf{ mas ueemos que la comuidat dela casa es cosa natural } non solamente por que sirue al conplimiento dela uida & et est ipsa natura . \textbf{ Videmus autem quod communitas domus est quid naturale , } non solum quia deseruit \\\hline
3.1.4 & mas avn por que aquellas comuindades \textbf{ que acaban la casa son cosa natural . } ca la casa se faze de comuidat de omne e de su muger e de sennor e de sieruo e de padre e de fijos . Et cada vna destas comuidades es cosa segunt natura bien & sed etiam quia communitates illae , \textbf{ quae perficiunt domum , | sunt quid naturale : } constat enim domus \\\hline
3.1.4 & ca la casa se faze de comuidat de omne e de su muger e de sennor e de sieruo e de padre e de fijos . Et cada vna destas comuidades es cosa segunt natura bien \textbf{ assi avn la comunidat del uarrio es cosa natural } non solamente por que sirue a conplimiento dela uida & ex communitate viri et uxoris , domini , et serui , patris et filii , quarum quaelibet est secundum naturam . \textbf{ Sic etiam communitas vici est | quid naturale , } non solum quia deseruit \\\hline
3.1.4 & por la qual cosa \textbf{ si tal acresçentamiento es cosa natural } siguese & ut vult Philosophus 1 Polit’ \textbf{ propter quod si tale crementum est naturale , } vicus ipse quid naturale erit . \\\hline
3.1.4 & siguese \textbf{ que el uarriosa cosa natural } e por ende la çibdat & propter quod si tale crementum est naturale , \textbf{ vicus ipse quid naturale erit . } Ciuitas ergo , \\\hline
3.1.4 & e por ende la çibdat \textbf{ que es fin de la generaçion dela casa e del vairio sera cosa natural } por que es fin dela generaçion de cosas naturales & Ciuitas ergo , \textbf{ quae est finis generationis domus | et vici erit } quid secundum naturam , \\\hline
3.1.4 & que es fin de la generaçion dela casa e del vairio sera cosa natural \textbf{ por que es fin dela generaçion de cosas naturales } ca la generaçion dela casa & et vici erit \textbf{ quid secundum naturam , | eo quod sit finis generationis naturalium : } generatio enim domus , \\\hline
3.1.4 & Et pues que assi es iusto \textbf{ que la çibdat es cosa natural . } finca de demostrar & tanquam ad finem et complementum , \textbf{ ordinatur ad ciuitatem . Viso , ciuitatem esse aliquid secundum naturam : } reliquum est ostendere , \\\hline
3.1.4 & finca de demostrar \textbf{ que el omne es naturalmente } aianl politicas & reliquum est ostendere , \textbf{ hominem esse naturaliter animal politicum et ciuile , } quod etiam duplici via inuestigare possumus . \\\hline
3.1.4 & La primera se toma de parte dela palabra ¶ \textbf{ La segunda de parte dela inclinaçion natural } que ha el omne & ex parte sermonis . \textbf{ Secunda ex parte impetus naturalis . } Probatur enim in principio secundi libri , \\\hline
3.1.4 & en el comienço del segundo libro delas politicas de parte dela palabra \textbf{ que el omne es naturalmente } aian la conpannable & ex parte sermonis \textbf{ hominem esse naturaliter animal sociale , } eo quod per sermonem acquirimus instructionem et disciplinam . \\\hline
3.1.4 & que la comunidat dela casa \textbf{ e la comunidat dela çibdat sean cosas naturales } ca si la natura dio al omne palabra natural aquella comunidat & oportet communitatem domesticam \textbf{ et ciuilem esse quid naturale . } Nam si natura dedit homini sermonem , \\\hline
3.1.4 & e la comunidat dela çibdat sean cosas naturales \textbf{ ca si la natura dio al omne palabra natural aquella comunidat } que es ordenada a aquellas cosas & et ciuilem esse quid naturale . \textbf{ Nam si natura dedit homini sermonem , | naturalis est illa communitas } quae ordinatur ad illa , \\\hline
3.1.4 & por la palabra conuiene \textbf{ que sea natural } ca la cosa iusta e la cosa & ø \\\hline
3.1.4 & co muindades es natra al tan bien la dela casa conmola dela çibdat \textbf{ por quela palabra anos en dada naturalmente demostrar } qual cosa nos es delectable & tam domestica quam ciuilis , \textbf{ eo quod per sermonem nobis | datum a natura repraesentatur } conferens et nociuum , \\\hline
3.1.4 & La segunda razon para prouaͬes \textbf{ pose toma de parte dela inclinacion natural } ca todas las ainalias han natural inclinaçon & Secunda via ad inuestigandum hoc idem , \textbf{ sumitur ex parte impetus naturalis . } Nam omnia animalia habent \\\hline
3.1.4 & pose toma de parte dela inclinacion natural \textbf{ ca todas las ainalias han natural inclinaçon } para guarda aquellas cosas & sumitur ex parte impetus naturalis . \textbf{ Nam omnia animalia habent | naturalem impetum ad conseruandum } ea quae sunt eis a natura tributa : \\\hline
3.1.4 & el beuir \textbf{ diol natural inclinaçion } para fazer aquellas cosas & quare si natura dedit homini viuere , \textbf{ dedit ei naturalem impetum ad faciendum ea } per quae possit \\\hline
3.1.4 & para la vida del omne \textbf{ e por ende en todos los omes es inclinaçion natural } para beuir politicas miente en çibdat & quae ad vitam sufficiunt . \textbf{ Inerit ergo hominibus impetus naturalis } ad viuendum politice , \\\hline
3.1.4 & e para fazer çibdat \textbf{ mas commo aquello a que auemos inclinaçion natural sea cosa natural } conuiene quela çibdat sea cosa natural & et ad constituendum ciuitatem . \textbf{ Sed cum id , | ad quod habemus impetum naturalem , } sit secundum naturam , \\\hline
3.1.4 & mas commo aquello a que auemos inclinaçion natural sea cosa natural \textbf{ conuiene quela çibdat sea cosa natural } e sea alguna cosa segunt natura & ad quod habemus impetum naturalem , \textbf{ sit secundum naturam , | oportet ciuitatem } esse quid naturale , \\\hline
3.1.6 & os podemos sennalar dos maneras del fazemiento dela çibdat e del regno \textbf{ e cada vna dlłas es en alguna manera natural } enpero la vna es mas natural que la otra & Generationis ciuitatis et regni duos modos possumus assignare , \textbf{ quorum quilibet est aliquo modo naturalis , } aliter tamen est naturalior altero . \\\hline
3.1.6 & e cada vna dlłas es en alguna manera natural \textbf{ enpero la vna es mas natural que la otra } ¶La primera manera es aquella dela qual en el segundo libro feziemos mençion desuso do dixiemos & quorum quilibet est aliquo modo naturalis , \textbf{ aliter tamen est naturalior altero . } Primus est ille de quo supra \\\hline
3.1.6 & e cada vna desta dos maneras esnatraal \textbf{ mas la primera es mas natural } que la segunda & Uterque autem horum modorum est naturalis : \textbf{ sed primus est naturalior secundo . } Prima enim constitutio ciuitatis et regni est dupliciter naturalis . \\\hline
3.1.6 & por que la primera manera de establesçimiento de la çibdat \textbf{ e del regno es natural en dos maneras } ¶La primera es natural & sed primus est naturalior secundo . \textbf{ Prima enim constitutio ciuitatis et regni est dupliciter naturalis . } Primo enim naturalis est , \\\hline
3.1.6 & e del regno es natural en dos maneras \textbf{ ¶La primera es natural } por que es establesçida & Prima enim constitutio ciuitatis et regni est dupliciter naturalis . \textbf{ Primo enim naturalis est , } quia constituitur ex generatione \\\hline
3.1.6 & que es obra de natura . \textbf{ ¶ la segunda es natural } por que los omes naturalmente han inclinaçion & quae est opus naturae . \textbf{ Secundo naturalis existit , } quia homines naturalem habent impetum ad constituendam ciuitatem et regnum . \\\hline
3.1.6 & ¶ la segunda es natural \textbf{ por que los omes naturalmente han inclinaçion } a establesçer çibdat e regno & Secundo naturalis existit , \textbf{ quia homines naturalem habent impetum ad constituendam ciuitatem et regnum . } Si enim per generationem \\\hline
3.1.6 & e de çibdat es muynatra al non solamente \textbf{ por que los omes han natural inclinacion } atal establesçimiento mas avn & talis constitutio regni et ciuitatis est maxime naturalis , \textbf{ non solum quia homines habent naturalem impetum } ad talem constitutionem , \\\hline
3.1.6 & establesçi miento del regno \textbf{ e dela çibdat es natural } assi commo quando se faze tal establesçimiento & Secundus autem modus constitutionis regni et ciuitatis , \textbf{ ut cum ex concordia hominum } talis constitutio habet esse , \\\hline
3.1.6 & tal commo la primera \textbf{ enpero es natural } por que los omes han natal inclinaçion & licet non sit adeo naturalis \textbf{ ut prima attamen naturalis est , } quia homines propter viuere habent naturalem impetum , \\\hline
3.1.6 & que cunplen para la uida . \textbf{ Avn en essa misma manera han natural inclinaçion } para establesçer prinçipado e regno & in qua reperiuntur sufficientia ad vitam . \textbf{ Sic etiam habent naturalem impetum , } ut constituant principatum et regnum ; \\\hline
3.1.6 & que les quieren mal fazer \textbf{ et esta tal inclinaçion es natural } que assi commo los omes han naturͣal inclinaçion & et magis resistere hostibus volentibus impugnare ipsos . \textbf{ Est enim huius impetus naturalis : } nam sicut homines naturalem habent impetum ut viuant , \\\hline
3.1.6 & e del tegno delas \textbf{ quales cada vna puede ser dichͣ natural Podemos eñader la terçera manera que es sinplemente } assi commo manera forcada & Enumeratis autem duobus modis generationis ciuitatis et regni , \textbf{ quorum quilibet dici potest naturalis : | possumus addere modum tertium , } qui quasi est simpliciter violentus . \\\hline
3.1.7 & as socrates commo ouiesse phophado luengo tienpo çerca las naturas delas cosas \textbf{ ueyendo muy grant guaueza cerca la sciençia natural } assi commo cuenta el pho & circa naturas rerum , \textbf{ videns circa naturalem scientiam | magnam difficultatem esse , } ut narrat Philosophus in Metaphysica sua , \\\hline
3.1.7 & que non sola mente batallan los mas los mas avn las fenbras . \textbf{ por ende paresçe que segunt la orden natural } en la qual partiçipamos con las otras aianlias & quia maxime videtur esse \textbf{ secundum ordinem naturalem } in quo communicamus \\\hline
3.1.7 & si ymaginaremos o signieremos las obras dela natura . \textbf{ Et mayormente la çibdat sera bien gouernada si semeiare ala orden natural } ca la vena del oro es & si imitamur actiones naturae , \textbf{ et maxime bene regitur ciuitas | si imitatur ordinem naturalem : } vena auri ut principantes maiori principatu , \\\hline
3.1.14 & La primera razon paresçe \textbf{ assi ca sienpre el bien comun deue ser antepuesto al bien propra o ca natural cosaes } que la parte se ponga al peligro & Prima via sit patet . \textbf{ Nam semper bonum commune praeponendum est bono priuato : } naturale enim est partem se exponere periculo pro toto , \\\hline
3.1.14 & por defendimiento del cuerpo \textbf{ Pues que assi es sin orden natural } biue todo çibdadano & ut brachium statim exponit se periculo pro defensione corporis , \textbf{ praeter ergo ordinem naturalem agit quilibet ciuis , } si non exponat se periculo pro defensione patriae : \\\hline
3.1.19 & ¶Conuienea saber comer e beuer \textbf{ por la calentura natural } que consume el humido radical & videlicet victu , potu , et cibo \textbf{ propter calorem naturalem consumentem huiusmodi radicale . } Secundo indiget domo , et vestitu , \\\hline
3.2.3 & atraalmente siguen a vn rey \textbf{ Et pues que assi es si cada vna delas cosas naturales } fueren penssadas en ssi sienpre veemos & naturaliter sunt sub uno rege . \textbf{ Si igitur singula naturalia considerentur , } semper videmus multitudinem quamlibet reduci \\\hline
3.2.5 & nin vanno . \textbf{ ca naturalmente ha cada vno amor assi mismo } e por ende natra al cosa es & ociosum et vanum esse non potest ; \textbf{ naturaliter autem quilibet habet amicitiam ad seipsum : } naturale est igitur tanto regem magis solicitari \\\hline
3.2.5 & e faze el sennorio \textbf{ assi commo natural . } Lo primero tira las uaraias et las discordias & tollit tyrannidem , \textbf{ efficit quasi dominium naturale . } Litigia enim sedat , \\\hline
3.2.5 & por electon mas ligerament ethiranizan Avn faze el hedamiento \textbf{ que el ssennorio sea natural . } ca el pueblo inclinase naturalmente & tales facilius tyrannizant . \textbf{ Facit etiam hoc dominium naturale , } quia populus quasi naturaliter inclinatur \\\hline
3.2.5 & que el ssennorio sea natural . \textbf{ ca el pueblo inclinase naturalmente } a obedesçer los mandamientos de tal Rey & Facit etiam hoc dominium naturale , \textbf{ quia populus quasi naturaliter inclinatur } ut obediat iussionibus talis Regis . \\\hline
3.2.5 & que muchs males nasçen enlas çibdades \textbf{ e en los regnos do non ay algun sennor natural } ca contesçe & plura mala oriri in ciuitatibus et regnis , \textbf{ ubi non est dominus aliquis naturalis : } nam contingit ea aliquando diu carere gubernatore , \\\hline
3.2.7 & por razon que tal sennorio es \textbf{ muchodes natural ¶ } La terçera se toma & ø \\\hline
3.2.7 & la segunda manera para prouar esto mismo se toma . \textbf{ por razon que tal sennorio es muy desnatural } por que aquella es obra natural & idest a communi bono . \textbf{ Secunda via ad inuestigandum hoc idem , sumitur ex eo quod tale dominium maxime est naturale . } Nam illa est naturalis operatio erga aliquid , \\\hline
3.2.7 & por razon que tal sennorio es muy desnatural \textbf{ por que aquella es obra natural } quando cada cosa se faze & Secunda via ad inuestigandum hoc idem , sumitur ex eo quod tale dominium maxime est naturale . \textbf{ Nam illa est naturalis operatio erga aliquid , } quando sic agitur \\\hline
3.2.7 & assi commo se deue fazer \textbf{ por la qual cosa el regno estonçe es naturalmente gouernado } quando los omes que son en el son & ut est aptum natum agi : \textbf{ quare tunc regnum naturaliter agitur , } quando homines existentes in ipso sic reguntur \\\hline
3.2.7 & Mas el omne por que ha libre aluedrio \textbf{ e ha razon estonçe es naturalmente gouernado } assi commo se deue gouernar & ut sunt apti nati regi : \textbf{ homo autem quia libero arbitrio et ratione participat , } tunc naturaliter regitur \\\hline
3.2.7 & alguons es mas contra uoluntad . \textbf{ mas deue ser dicho desnatural . } por que quanto el sennorio de alguon ses mas contra uoluntad de los omes & quando voluntarie seruit \textbf{ et libere obedit : } quare quanto dominium aliquorum \\\hline
3.2.7 & por que quanto el sennorio de alguon ses mas contra uoluntad de los omes \textbf{ tanto mas deue ser dich des natural . } Et por ende la tirania es muy mala & magis est inuoluntarium , \textbf{ magis debet dici in naturale : } tyrannis igitur est pessima , \\\hline
3.2.8 & deue penssar con grant acuçia \textbf{ en las cosas naturales } ca si toda la natura es gouernada & diligenter considerare debet \textbf{ in naturalibus rebus . } Nam si natura tota administratur per ipsum Deum , \\\hline
3.2.8 & Por ende del gouernamiento \textbf{ que veemos enlas cosas naturales } deue descender el gouernamiento & quare a regimine , \textbf{ quod videmus in naturalibus , } deriuari debet regimen , \\\hline
3.2.8 & Ca el arte semeia mucha la natura . \textbf{ Et en las cosas naturales } assi lo veemos & est enim ars imitatrix naturae . \textbf{ In naturalibus autem sic videmus , } quod natura primo dat rebus ea per quae possunt \\\hline
3.2.8 & por que pueden arredrar dessi las cosas que les enpesçen . \textbf{ lo terçero las cosas naturales } por estas cosas que les da la natura . & per quae possunt prohibentia remouere . \textbf{ Tertio per huiusmodi collata naturaliter intendunt } in suos fines siue in suos terminos . \\\hline
3.2.8 & por estas cosas que les da la natura . \textbf{ naturalmente una a sus terminos o a ssus fines } assi commo paresçe por este exenplo & Tertio per huiusmodi collata naturaliter intendunt \textbf{ in suos fines siue in suos terminos . } Ut natura dat igni leuitatem , \\\hline
3.2.8 & por aquellas cosas \textbf{ que resçibe dela natura naturalmente sube arriba . } Et pues que assi es para que el gouernamiento sea bueno & Tertio ignis per ea quae accepit a natura , \textbf{ naturaliter tendit sursum . } Ergo ad hoc quod regimen sit bonum et naturale , \\\hline
3.2.8 & Et pues que assi es para que el gouernamiento sea bueno \textbf{ e natural tres cosas son meciester ¶ } Lo primero que en tal manera sea el pueblo apareiado e ordenado por que pue da alcançar su fin que entiende . & Ergo ad hoc quod regimen sit bonum et naturale , \textbf{ tria requiruntur . } Primo , ut populus taliter disponatur et ordinetur , \\\hline
3.2.8 & lo primero se prueua assi . \textbf{ Ca lo omes son en ssi mismos naturalmente corruptibles } e por ende por que en ssi mismos non pueden durar & Primum sic patet . \textbf{ Nam homines in seipsis sunt naturaliter corruptibiles : } inde est ergo quod quia in seipsis durare non possunt , \\\hline
3.2.8 & e por ende por que en ssi mismos non pueden durar \textbf{ dessean naturalmente de durar en sus fijos . } si quier sean naturales & inde est ergo quod quia in seipsis durare non possunt , \textbf{ naturaliter appetunt perpetuari } in suis filiis siue sint naturales siue adoptiui . \\\hline
3.2.8 & dessean naturalmente de durar en sus fijos . \textbf{ si quier sean naturales } siquier por fuados . & naturaliter appetunt perpetuari \textbf{ in suis filiis siue sint naturales siue adoptiui . } Videtur enim homini quasi post mortem viuere , \\\hline
3.2.10 & e que fien vnos de otros . \textbf{ Ca commo el entienda enl bien de los çibdadanos natural } cosaes que sea amado dellos . & et de se confidere ; \textbf{ nam cum intendat bonum ipsorum ciuium et subditorum , | naturale est } ut diligatur ab eis : \\\hline
3.2.10 & non por aquellos que son del regno \textbf{ por que non fia delos sus naturales } mas por los estrannos . & qui sunt in eo regno , \textbf{ eo quod diffidat de illis , } sed per extraneos . \\\hline
3.2.15 & mas si el regnado fuere en estado de antiguedat \textbf{ Et aquell sennor fuere natural de antiguo tienpo } assi que non sea memoria delos omes & sed si regnum diu in statu perstiterit , \textbf{ et dominus ille sit naturalis , } ita quod quasi non sit \\\hline
3.2.16 & aquellas cosas \textbf{ que se fazen muchͣs uezos li le fazen naturalmente } e por ende delas luuias & Tertio non sunt consiliabilia \textbf{ etiam quae fiunt frequenter , | si fiunt a natura . } Ideo de imbribus \\\hline
3.2.16 & por que tales cosas \textbf{ commo estas son naturales } e non nasçen de las nuestras obras . Et por ende dellas non tomamos conseio & non habet esse consilium : \textbf{ quia talia naturalia sunt , } et non dependent ex operibus nostris , \\\hline
3.2.24 & e alguno es proprao . \textbf{ Et alguon es natural . } Et alguno es legal e positiuo que es puesto por omes . & quoddam proprium ; \textbf{ quoddam est naturale , } quoddam legale siue positiuum . \\\hline
3.2.24 & Et pues que assi es en aquella man \textbf{ era que los iuristas apartan el derecho natural del derecho delas } gentes podemos nos apartar el derecho natural del derecho delas animalias & et quoddam ciuile . \textbf{ Illo ergo modo quo iuristae separant | ius naturale a iure gentium , } possemus separare nos ius naturale \\\hline
3.2.24 & era que los iuristas apartan el derecho natural del derecho delas \textbf{ gentes podemos nos apartar el derecho natural del derecho delas animalias } e darla quanta distinçion & ius naturale a iure gentium , \textbf{ possemus separare nos ius naturale | a iure animalium : } et dare quintam distinctionem iuris , \\\hline
3.2.24 & diziendo que en quatro maneras se departe el derech . \textbf{ Conuiene a saber ende recħ natural } e en derecho delas ainalias & et dare quintam distinctionem iuris , \textbf{ dicendo quod quadruplex est ius , } videlicet naturale , animalium , gentium , et ciuile . \\\hline
3.2.24 & Et la ley es en dos maneras . \textbf{ Conuiene a saber natural e positiua puesta por omne . } Ca los & quod duplex est iustum , \textbf{ vel duplex est lex , naturalis , et positiua . } Dicuntur enim iusta naturalia , \\\hline
3.2.24 & Ca los \textbf{ derechsson dichos naturales } por que son proporçionados e ygualados por su natura . & vel duplex est lex , naturalis , et positiua . \textbf{ Dicuntur enim iusta naturalia , } quae sunt adaequata et proportionata ex natura sua , \\\hline
3.2.24 & O son dichos natales \textbf{ por que la razon natural los muestra de ser tales } o por que auemos natural apetito o natural inclinaçion & vel dicuntur iusta naturaliter \textbf{ quae dictat esse talia ratio naturalis , } vel ad quae habemus naturalem impetum et inclinationem . \\\hline
3.2.24 & por que la razon natural los muestra de ser tales \textbf{ o por que auemos natural apetito o natural inclinaçion } a ellos aas los derechos positiuos & quae dictat esse talia ratio naturalis , \textbf{ vel ad quae habemus naturalem impetum et inclinationem . } Iusta vero positiua dicuntur , \\\hline
3.2.24 & Et por ende se sigue \textbf{ que el derecho natural es departido del derechpo sitiuo . } ca el derech natural & et edicta principum non sunt eadem apud omnes . \textbf{ Inde est quod ius naturale dicitur | differre a positiuo : } quia naturale ut traditur 5 Ethicor’ \\\hline
3.2.24 & que el derecho natural es departido del derechpo sitiuo . \textbf{ ca el derech natural } assi commo dize el philosofo & differre a positiuo : \textbf{ quia naturale ut traditur 5 Ethicor’ } ubique eandem habet potentiam ; \\\hline
3.2.24 & de obligar alos omes . \textbf{ Mas la razon por que al derech natural conuinio anneder derecho positiuo es esta } por que muchas cosas son derechas naturalmente & habere ligandi efficaciam . \textbf{ Ratio autem , | quare iuri naturali oportuit } superaddere positiuum , \\\hline
3.2.24 & Mas la razon por que al derech natural conuinio anneder derecho positiuo es esta \textbf{ por que muchas cosas son derechas naturalmente } assi commo natural cosa es al ome de fablar & quare iuri naturali oportuit \textbf{ superaddere positiuum , | est : quia multa sunt sic iusta naturaliter , } sicut est naturale homini loqui : \\\hline
3.2.24 & por que muchas cosas son derechas naturalmente \textbf{ assi commo natural cosa es al ome de fablar } ca auemos natural apetito & est : quia multa sunt sic iusta naturaliter , \textbf{ sicut est naturale homini loqui : } habemus enim naturalem impetum et naturalem inclinationem \\\hline
3.2.24 & assi commo natural cosa es al ome de fablar \textbf{ ca auemos natural apetito } e natural inclinacion para fablar & sicut est naturale homini loqui : \textbf{ habemus enim naturalem impetum et naturalem inclinationem } ut loquamur , \\\hline
3.2.24 & ca auemos natural apetito \textbf{ e natural inclinacion para fablar } e para manifestar a otri lo que conçebimos en la uoluntad & habemus enim naturalem impetum et naturalem inclinationem \textbf{ ut loquamur , } et ut per sermonem manifestemus \\\hline
3.2.24 & Et pues que assi es \textbf{ assi commo fablar es cosa natural alos omes } assi fablar tal & sermonem nobis esse datum a natura . \textbf{ Sicut ergo loqui est naturale , } sic autem loqui vel sic , est positiuum et ad placitum . \\\hline
3.2.24 & e o tristales cosas son de \textbf{ derechnatural } por que la razon natural muestra & et cetera huiusmodi sunt , \textbf{ de iure naturali , } quia haec esse fienda \\\hline
3.2.24 & derechnatural \textbf{ por que la razon natural muestra } que se deuen fazer . & et cetera huiusmodi sunt , \textbf{ de iure naturali , } quia haec esse fienda \\\hline
3.2.24 & que se deuen fazer . \textbf{ Et auemos natural inclinaçion } que estas cosas se fagan . & dictat ratio naturalis , \textbf{ et habemus naturalem impetum } ut haec fiant . \\\hline
3.2.24 & ca estas cosas sele una tan dela natura dela cosa . \textbf{ Ca natural cosa es que espongamos a periglo el bien de vna parte } por salud del bien del todo & ex ipsa natura rei , \textbf{ naturale est enim | quod exponatur periculo bonum partis } pro salute boni totius , \\\hline
3.2.24 & nin sea enbargado . \textbf{ Et pues que assi es natural cosa es } que tales cosas sean castigadas & et ne impediatur commune bonum : \textbf{ naturale est ergo talia punire . } Sed ea punire sic , \\\hline
3.2.24 & Et pues que assi es en aquel logar \textbf{ ose termina el derecho natural } alli comiença a naçer & sed non apud omnes eadem maleficia corriguntur eisdem poenis . \textbf{ Ubi ergo terminatur ius naturale , } ibi incipit oriri ius positiuum : \\\hline
3.2.24 & e por sabiduria delos omes \textbf{ presurone el derecho natural } e fundasse en el & et industriam hominum adinuentum \textbf{ praesupponit ius naturale , } sicut ea quae sunt artis praesupponunt \\\hline
3.2.24 & que son dela natura \textbf{ Por la qual cosa si el derecho natural manda } que los ladrones & quae sunt naturae . \textbf{ Quare si ius naturale dictat fures et maleficos esse puniendos , } hoc praesupponens ius positiuum procedit ulterius , \\\hline
3.2.24 & e dos diferençias \textbf{ entre el derech natural e el positiuo . } La primera es que el derech natural & assignare possumus \textbf{ inter ius naturale , et positiuum . } Prima est , \\\hline
3.2.24 & entre el derech natural e el positiuo . \textbf{ La primera es que el derech natural } luego en la primera faz se ofresçe al entendimiento & inter ius naturale , et positiuum . \textbf{ Prima est , } quia ius naturale prima facie se offert intellectui : \\\hline
3.2.24 & Et por que el derecho natre alas si se ofresçe al nuestro entendimiento . \textbf{ Por ende las leyes naturales } e aquello que es del derecho natural & intellectui nostro naturales leges , \textbf{ quod est de iure naturali simpliciter } dicitur esse scriptum in cordibus nostris . \\\hline
3.2.24 & Por ende las leyes naturales \textbf{ e aquello que es del derecho natural } sinplemente es dicho seer escpto en los nros coraçones & intellectui nostro naturales leges , \textbf{ quod est de iure naturali simpliciter } dicitur esse scriptum in cordibus nostris . \\\hline
3.2.24 & Ca las gentes \textbf{ que non han ley natural } fazen aquellas cosas & dicitur esse scriptum in cordibus nostris . \textbf{ Nam gentes quae legem non habent , } naturaliter ea quae legis sunt faciunt , \\\hline
3.2.24 & conuiene de ser escerpto en algun libro . \textbf{ Enpero cada vno destos dos derechs tan bien el natural commo el positiuo se puede escͥuir en algun libro } Enpero non es tanto meester de se escuir el natural commo el positiuo & Potest itaque \textbf{ utrunque ius scribi | in aliqua exteriori substantia } tam naturale quam positiuum : \\\hline
3.2.24 & Enpero cada vno destos dos derechs tan bien el natural commo el positiuo se puede escͥuir en algun libro \textbf{ Enpero non es tanto meester de se escuir el natural commo el positiuo } por que non pue de & in aliqua exteriori substantia \textbf{ tam naturale quam positiuum : | naturale tamen non sic indiget } ut scribatur sicut positiuum , \\\hline
3.2.24 & por que non pue de \textbf{ assi caer dela memoria el natural commo el positiuo ¶ } La segunda diferençia es & nam non sic potest \textbf{ a memoria recedere | sicut illud . } Secunda differentia est , \\\hline
3.2.24 & en el primero libro de la \textbf{ rectorica al derecho natural ayre o fuego } que continuadamente se estiende en claridat & ut recitat Philosoph’ 1 Rhet’ appellat \textbf{ ius naturale aetherem siue ignem , } qui continuat protendere \\\hline
3.2.24 & mas que los otros helementos \textbf{ e es mas claro que los otros helementos semeia al derecho natural } que es mas esparzido & quam alia elementa , \textbf{ et distantior est omnibus illis , | assimilatur iuri naturali , } quod est diffusiuum et communius \\\hline
3.2.25 & de los quales el vn mienbro se contenie \textbf{ so el derecho natural } e el otro so el derecho & quarum unum membrum continebatur \textbf{ sub iure naturali , } et aliud sub ciuili siue sub positiuo . \\\hline
3.2.25 & Et pues que assi es \textbf{ si aquellas cosas son del derech natural . } a que auemos natal inclinaçion & Si igitur ea sunt \textbf{ de iure naturali , ad quae habemus naturalem impetum et inclinationem : } huiusmodi naturalis impetus \\\hline
3.2.25 & a que auemos natal inclinaçion \textbf{ esta inclinaçion natural del apeti } too ligue lanr̃a natura & de iure naturali , ad quae habemus naturalem impetum et inclinationem : \textbf{ huiusmodi naturalis impetus } vel sequitur naturam nostram , \\\hline
3.2.25 & siguiere la nuestra natura en quanto auemos conueniençia con las otras aian lias \textbf{ assi es dich derech natural . } Et por ende en la instituta del derecho natural & ut conuenimus cum animalibus aliis : \textbf{ sic dicitur esse ius naturale . } Ideo in Instituta , \\\hline
3.2.25 & assi es dich derech natural . \textbf{ Et por ende en la instituta del derecho natural } do estas cosas son puestas & sic dicitur esse ius naturale . \textbf{ Ideo in Instituta , } ubi haec sunt tradita , \\\hline
3.2.25 & do estas cosas son puestas \textbf{ dize el derecho natural } enssenna a todas las ainalias . & ubi haec sunt tradita , \textbf{ dicitur , | quod ius naturale , } est quod natura omnia animalia docuit . \\\hline
3.2.25 & e enla tierra e enla mar . \textbf{ Et pues que assi es segunt esto de derecho natural } es el ayuntamiento de linaslo e dela fenbra & et quae in mari nascuntur . \textbf{ Secundum hoc ergo est de iure naturali } coniunctio maris et foeminae , \\\hline
3.2.25 & Et pues que assi es el derecho delas gentes es \textbf{ assi commo vn derecho natural mas espeçial . } Et por ende aquel derecho & Ius ergo gentium est \textbf{ quoddam ius naturale contractum . } Ius itaque illud quod natura omnia animalia docuit , \\\hline
3.2.25 & que la natura enssenno a todas las otras nanlas \textbf{ e que sigue la nuestra inclinaçion natural } en quanto participamos con las otras aianlas & Ius itaque illud quod natura omnia animalia docuit , \textbf{ et quod sequitur inclinationem nostram naturalem } ut communicamus cum animalibus aliis , \\\hline
3.2.25 & en quanto participamos con las otras aianlas \textbf{ en conparacion del derecho delas gentes es dicho derecho natural . } Ca si penssaremos los dichos del capitulo & ut communicamus cum animalibus aliis , \textbf{ respectu iuris gentium dicitur esse naturale . } Nam si considerentur dicta in praecedenti capitulo , \\\hline
3.2.25 & Ca si penssaremos los dichos del capitulo \textbf{ sobredicho el derecho natural es algun an cosa comun } e es algunan cosa conosçida & Nam si considerentur dicta in praecedenti capitulo , \textbf{ ius naturale est | quid commune , } quid notum , \\\hline
3.2.25 & e menos se puede mudar tanto \textbf{ mas meresçe de auer nonbre derech natural . } Et por ende aquel derecho & quam ius gentium : \textbf{ tanto magis meretur nomen iuris naturalis . } Ius ergo , \\\hline
3.2.25 & e resçiben mayor mudamiento . \textbf{ Et pues que assi es con grant razon el derecho delas aianlias es dicho ser derecho natural en conparaçion del derecho delas gentes } ¶ & et maiorem mutationem suscipiunt : \textbf{ merito igitur huiusmodi ius , | naturale dicitur , } respectu iuris gentium . \\\hline
3.2.25 & en quanto el omne es aianl \textbf{ conuiene con las naturales delas otras aianlias . } assi en quanto omne biue es alguna cosa conuiene con las plantas & in quantum animal est , \textbf{ conuenit cum naturis aliorum animalium , | sic in quantum viuit } et est quoddam ens , \\\hline
3.2.25 & e con todas las cosas que han ser . \textbf{ Et por ende la inclinacion natural } puede seguir la natura del ome & et cum entibus omnibus . \textbf{ Poterit ergo inclinatio naturalis } sequi naturam hominis \\\hline
3.2.25 & Ca el omne \textbf{ naturalnse te dessea ser guardado en su ser Ra qual cola avn del sean todas las otras cosas que son . } avn el omne natutalmente dessea de auer fijos & vel ut conuenit cum omnibus entibus . \textbf{ Nam homo naturaliter appetit conseruari in esse , | quod et omnia entia alia appetunt : } naturaliter appetit producere filios , educare prolem , \\\hline
3.2.25 & delas obras de los omes se fundaren sobre esto \textbf{ que el ome naturalmente dessea ser . } assi tales reglas podran ser de derecho natural & Si ergo regulae agibilium fundentur super hoc , \textbf{ quod homo naturaliter appetit esse : } sic huiusmodi regulae poterunt esse de iure naturali , \\\hline
3.2.25 & que el ome naturalmente dessea ser . \textbf{ assi tales reglas podran ser de derecho natural } en quanto la natura humanal es alguna sub̃a e haser & quod homo naturaliter appetit esse : \textbf{ sic huiusmodi regulae poterunt esse de iure naturali , } prout natura humana est quaedam entitas , \\\hline
3.2.25 & e con todas las sustançias . \textbf{ Mas si aquellas reglas se tomaren en quanto el omne naturalmente } dessea fazer fiios e carlos & et conuenit cum entibus omnibus . \textbf{ Si vero regulae illae sumantur | ex eo quod homo naturaliter } appetit filios producere et educare : \\\hline
3.2.25 & dessea fazer fiios e carlos \textbf{ assi podria ser de derecho natural } en quanto el derecho naturales dicho ser & appetit filios producere et educare : \textbf{ sic esse poterunt de iure naturali , } prout ius naturale dicitur esse , \\\hline
3.2.25 & assi podria ser de derecho natural \textbf{ en quanto el derecho naturales dicho ser } tal qual la natura mostro a todas las ainalias . & sic esse poterunt de iure naturali , \textbf{ prout ius naturale dicitur esse , } quod natura omnia animalia docuit . \\\hline
3.2.25 & Mas si fueren tomadas las reglas del derecho \textbf{ en quanto el omne naturalmente } dessea beuir en conpannia & Sed si sumantur , \textbf{ prout homo naturaliter appetit } in societate viuere \\\hline
3.2.25 & establesçimientos e posturas cs̃uenbles \textbf{ assi seran de derecho natural } en quanto el derecho natural es traydo al derecho delas gentes & secundum debitas conuentiones et pacta ; \textbf{ sic erit de iure naturali , } prout ius naturale contractum est \\\hline
3.2.25 & assi seran de derecho natural \textbf{ en quanto el derecho natural es traydo al derecho delas gentes } el qual derecho esppreo solamente al linage humanal . & sic erit de iure naturali , \textbf{ prout ius naturale contractum est | ad ius gentium , } quod est proprium soli humano generi . \\\hline
3.2.25 & que assi commo el derecho delas gentes \textbf{ non es dicho assi derecho natural commo el derecho } que la natura enssenno a todas las asanlias . & quod ius gentium non dicitur \textbf{ ita ius naturale , } sicut ius quod nam omnia animalia docuit : \\\hline
3.2.25 & assi este derecho tal que la natura enssenno a todas las aianlias non es \textbf{ assi natural commo es aquel derecho } que sigue la inclinaçion del anr̃a natura & quod nam omnia animalia docuit , \textbf{ non est ita naturale , | sicut ius illud } quod sequitur \\\hline
3.2.25 & e foyr del non seres \textbf{ mas de derech natural } que dessear de engendrar fijos e criar los . & et non esse , \textbf{ est plus de iure naturali , } quam appetere procreare filios , \\\hline
3.2.25 & que el derecho que ligue lanr̃a natura en quanto desseamos ser \textbf{ e desseamos bien es natural en conparaçion del derech delas aianlias } o en conparaçion del derech & quod ius consequens naturam nostram \textbf{ prout appetimus esse et bonum , | est naturale respectu iuris animalium , } siue respectu iuris \\\hline
3.2.25 & o en conparaçion del derech \textbf{ que la natura demostro a todas las aian lias ¶ Avn en essa misma este dereches dicho natural } en conparaçion del derech delas gentes . & siue respectu iuris \textbf{ quod natura omnia animalia docuit : } sic etiam huiusmodi ius est naturale \\\hline
3.2.25 & en conparaçion del derech delas gentes . \textbf{ Et el detecho delas gentes es natural } en conparaçion del derech ciuil & quod natura omnia animalia docuit : \textbf{ sic etiam huiusmodi ius est naturale } respectu iuris ciuilis , \\\hline
3.2.25 & que es sinplementeposituo . \textbf{ Et pues que assi es tres cosas son en alguna manera del derecho natural Lo primero es que el sea ygualado proporçionado ala natura humanal } o que lo diga la razon nata & quod est simpliciter positiuum . \textbf{ Tria ergo sunt aliquo modo de iure naturali , } secundum quod inclinatio sequitur naturam nostram : \\\hline
3.2.25 & o que lo diga la razon nata \textbf{ lo que ayamos a ello inclinaçion natural . } Mas segunt que la intlinaçion sigue lanr̃a natura & Tria ergo sunt aliquo modo de iure naturali , \textbf{ secundum quod inclinatio sequitur naturam nostram : } Nam si inclinatio illa sequitur naturam nostram \\\hline
3.2.25 & assi se toma aquel derech \textbf{ que es dicho natural pora un ataia de los otros derechos . } Por que dessear el bien e el ser & sic habet esse ius illud , \textbf{ quod per antonomasiam dicitur esse naturale . } Appetere enim esse et bonum , \\\hline
3.2.25 & e foyr el non ser \textbf{ e el mal es aquello que desseamos naturalmente } en quanto la nuestra nata conuiene con todas las substançias . & et fugere non esse et malum , \textbf{ quod naturaliter appetimus , } prout natura nostra conuenit \\\hline
3.2.25 & en quanto la nuestra nata conuiene con todas las substançias . \textbf{ Et assi avn es el derech natural } que todas las otras reglas & cum omnibus entibus , \textbf{ sic est de iure naturali : } a quo caeterae aliae regulae , \\\hline
3.2.25 & e todas las otras leyes \textbf{ si quier sean naturales } si quier cuuileᷤ en el tomaron rays & et caeterae leges , \textbf{ siue sint naturales , } siue ciueles , \\\hline
3.2.25 & e el derech delas aianlias \textbf{ e avn el derechçiuil se departe del derecho natural . } Saresçe que derecho çiuil o el derecho humanal & et ius animalium , \textbf{ et etiam ius ciuile differt | a iure naturali . } Videtur autem ius ciuile , \\\hline
3.2.26 & Con uiene a saber . \textbf{ al derecho natural o ala ley dela natura dela qual tomo Rays e fuidamiento . } Et puede se conparar al bien comun & videlicet ad ius naturale \textbf{ siue ad legem naturalem , | a qua suscipit fundamentum : } ad bonum commune quod in ea intenditur : \\\hline
3.2.26 & Lo primero conuiene que la ley humanal o positiua sea derecha \textbf{ en quanto es conparada ala razon natural o ala ley de natura . } Ca si derechͣ non fuere non es ley mas es & siue positiuam esse iustam \textbf{ ut comparatur ad rationem naturalem | siue ad legem naturalem : } quoniam si iusta non sit , \\\hline
3.2.26 & tomare comienco dela ley de natura . \textbf{ Et si en alguna manera la razon natural non iudgare } que aquello deue ser establesçido . & ex lege naturali , \textbf{ et nisi aliquo modo ratio naturalis dictet illud statuendum esse . } Secundo lex humana et ciuilis debet esse utilis \\\hline
3.2.27 & de enderesçar los omes al bien comun . \textbf{ Ca si es ley diuinal e natural establesçida es de dios } a quienꝑtenesçe enderesçar todas las cosas asimesmo . & ab eo cuius est dirigere in bonum commune : \textbf{ nam si est lex diuina et naturalis , | condita est a Deo } cuius est omnia dirigere in seipsum , \\\hline
3.2.27 & conuiene que sea publicada e pregonada . \textbf{ Mas commo otra sea la ley natural e otra la positiua en vna manera se deue publicar la vna } e en otra manera la otra . & oportet eam promulgatam esse . \textbf{ Sed cum alia sit lex naturalis , | alia positiua : } aliter propalatur haec , \\\hline
3.2.27 & e en otra manera la otra . \textbf{ Ca la ley naturales en tanto fincada enlos nuestros coraçons } que en cada vn omne es publicada e manifestada & aliter illa . \textbf{ Nam lex naturalis est a deo indita in cordibus nostris : } ideo in quolibet homine haec promulgatur et propalatur , \\\hline
3.2.27 & e qual cosa ha de foyr e de escusar \textbf{ segunt que esto pertenesçe al derecho natural . } Mas la ley humanal e positiua estonçe es publicada & quid sequendum et quid fugiendum , \textbf{ secundum quod haec pertinent ad ius naturale . } Sed lex humana tunc promulgatur , \\\hline
3.2.27 & e tomadas \textbf{ por las leyes naturales } este la salut del regno e dela çibdat . & si sint rectae et iustae \textbf{ et a legibus naturalibus determinatae , } consistat salus regni et ciuitatis ; \\\hline
3.2.29 & conuiene de saber que el rey \textbf{ e qual se quier sennor deue ser medianero entre la ley natural e la ley positiua . } Ca ninguno non iudga derechamente nin & et quemlibet principantem \textbf{ esse medium | inter legem naturalem et positiuam ; } nam nullus recte principatur , \\\hline
3.2.29 & Et assi se sigue \textbf{ que sigua la ley natural } la qual se leunata de razon derecha e de entendumento derecho . & oportet Regem in regendo alios \textbf{ sequi rectam rationem , } et per consequens sequi naturalem legem , \\\hline
3.2.29 & Et por ende el rey en gouernando es a \textbf{ quande dela ley natural } por que en tanto gouierna derechamente & sequi rectam rationem , \textbf{ et per consequens sequi naturalem legem , } quia in tantum recte regit , \\\hline
3.2.29 & por que en tanto gouierna derechamente \textbf{ en quanto non se parte nin se arriedra dela ley natural . } Enpero es sobre la ley positiua & quia in tantum recte regit , \textbf{ in quantum a lege naturali non deuiat : } est tamen supra legem positiuam , \\\hline
3.2.29 & Et pues que assi es assi commo el rey nunca gouierna derechamente \textbf{ sy non se esforçare en la ley natural } et si non feziere & Itaque sicut Rex nunquam recte regit , \textbf{ nisi innitatur lege naturali , } et agat ut recta ratio dictat : \\\hline
3.2.29 & que la manda fazer \textbf{ assi commo la ley naturales sobre el señor } por que la non puede mudar . & lex est infra principantem , \textbf{ sicut lex naturalis est supra . Et si dicatur legem } aliquam positiuam esse supra principantem , \\\hline
3.2.29 & esto non es en quanto es ley positua \textbf{ mas en quanto enella es guardada la uirtud del derecho e dela ley natural . } Et pues que assi es quando es demandado enla quastiuo & hoc non est ut positiua est , \textbf{ sed ut in ea reseruatur | virtus iuris naturalis . } Cum ergo quaeritur \\\hline
3.2.29 & que es mas prinçipal en \textbf{ gouernando la ley natural } que el rey & Si loquamur de lege naturali , \textbf{ patet hanc principaliorem esse in regendo , } quam sit ipse Rex : \\\hline
3.2.29 & por que ninguon non es derecho rey \textbf{ si non en quanto se esfuerça enla ley natural . } Por la qual cosa bien dicho es & eo quod nullus sit rectus Rex \textbf{ nisi in quantum innititur illi legi . } Propterea bene dictum est \\\hline
3.2.29 & nin se arriedra de derecha razon \textbf{ nin dela ley natural } la qual dios puso en el entendimiento de cada vno . & a ratione recta , \textbf{ et a lege naturali , } quam Deus indidit intellectui cuiuscumque . \\\hline
3.2.29 & Por ende conuiene que el Rey o otro prinçipe \textbf{ por razon derechͣo por ley natural . } la qual dios puso en voluntad de cada vn omne & aut alium principantem per rationem rectam , \textbf{ aut per legem naturalem , } quam Deus impressit \\\hline
3.2.30 & e se mandan fazer todas las uirtudes \textbf{ Mas que sin la ley natural e humanal fue menester de dar ley e un aglical e diuinal . } podemos lo prouar por tres razones & et omnes virtutes praecipere . \textbf{ Sed quod praeter legem naturalem | et humanam fuerit } expediens \\\hline
3.2.30 & e ningun bien non fincasse sia gualardon . \textbf{ Mas para esto fazer non cunple la ley natural } assi commo paresçra adelante . & et nullum bonum irremuneratum . \textbf{ Ad hoc autem faciendum non sufficit lex naturalis , } ut in prosequendo patebit : \\\hline
3.2.30 & Ca commo este tal bien sea sobre el poderio dela nuestra natura . \textbf{ la ley natural e la } humanal que nos ayudan a alcançar este bien . & supra facultates nostrae naturae , \textbf{ lex naturalis et humana iuuantes nos } ad consecutionem illius boni \\\hline
3.2.30 & humanal que nos ayudan a alcançar este bien . \textbf{ el qual non podemos natural mente alcançar non cunplen para alcançar este bien } que es sobre natural . & lex naturalis et humana iuuantes nos \textbf{ ad consecutionem illius boni | quod possumus naturaliter adipisci , } non sufficiunt ad consequendum illud bonum supernaturale ; \\\hline
3.2.30 & el qual non podemos natural mente alcançar non cunplen para alcançar este bien \textbf{ que es sobre natural . } Et pues que assi es . & quod possumus naturaliter adipisci , \textbf{ non sufficiunt ad consequendum illud bonum supernaturale ; } ergo necessaria fuit lex euangelica et diuina , \\\hline
3.2.31 & si fuere derecha conuiene que se raygͤ \textbf{ e se funde enla ley natural . } Et conuiene que determine las obras & sciendum quod lex positiua si recta sit , \textbf{ oportet quod innitatur legi naturali , } et quod determinet gesta particularia hominum . \\\hline
3.2.31 & Lo primero si \textbf{ fuerecontraria ala ley natural . } Lo segundo si non determinar e conplidamente los fechos e las obras particulares & Primum si sit \textbf{ contraria legi naturali . } Secundo si non sufficienter determinaret particularia gesta . \\\hline
3.2.31 & si en la primera manera fallesçieron las leyes positiuas dela tierra \textbf{ siendo contra la ley natural non son leyes } mas son corronpimiento de leyes . & Si primo modo deficiant leges paternae et positiuae , \textbf{ non sunt leges sed corruptiones legum , } propter hoc obseruari non debent . \\\hline
3.2.31 & quierque sean ennadidas \textbf{ alas leyes naturales } enpero non son contrarias dellas . & Nam leges positiuae licet \textbf{ sint additae naturalibus legibus , } non tamen sunt contrariae illis , \\\hline
3.2.31 & saluo si quisiessemos dezir \textbf{ que contrario del derecho natural } es aquello que non es enduzido dela natura . & non tamen sunt contrariae illis , \textbf{ nisi vellemus appellare contrarium iuri naturali } quod non est a natura inductum , \\\hline
3.2.31 & por arte de los o omes \textbf{ segunt la qual manera de fablar el omne es desnudo natural mente . } Ca desnudo nasçe & et per artem omnium adinuentum : \textbf{ secundum quem modum loquendi | homo est nudus naturaliter , } et vestimentum est contra naturam : \\\hline
3.2.31 & Ca ser el omne uestido paresçe contrario a aquello que es ser de sudo . \textbf{ Et segunt esta manera de fablar fablan los iuristas del derecho natural } assi commo paresçe en la & ei quod est esse nudum . \textbf{ Secundum hunc modum loquendi loquuntur Iuristae , } ut patet ex Institutis de iure naturali , \\\hline
3.2.31 & cuyo contrario dize \textbf{ e manda la razon natural e el entendimiento . } Et por que la seruidunbre es puesta & Nam illud proprie est contra naturam , \textbf{ cuius contrarium dictat ratio naturalis : } et quia propter utilitatem \\\hline
3.2.31 & e obedesçer alos otros \textbf{ maguer que sea ennadido e sobrepuesto al derecho natural } assi commo la uestidura es en nadida & et obedire licet \textbf{ sit iuri naturali additum et appositum , } sicut vestimentum est additum \\\hline
3.2.31 & e esto semeia contra natura . \textbf{ Erpero non es contraria al derech natural } nin es contra el derecho natural . & et adductum corpori nudo \textbf{ quod est natura productum : } non tamen est contra ius naturale , \\\hline
3.2.31 & Erpero non es contraria al derech natural \textbf{ nin es contra el derecho natural . } Ca la razon natural non contradize atal seruidunbre . & quod est natura productum : \textbf{ non tamen est contra ius naturale , } quia huic naturalis ratio non contradicit . \\\hline
3.2.31 & nin es contra el derecho natural . \textbf{ Ca la razon natural non contradize atal seruidunbre . } ¶ Et pues que assi es las leyes que fallesçen & non tamen est contra ius naturale , \textbf{ quia huic naturalis ratio non contradicit . } Leges ergo deficientes \\\hline
3.2.31 & por que son malas \textbf{ e son contra derecho natural } assi commo emaquella ley que dizie & quia sunt prauae \textbf{ et contra ius naturale , } cuiusmodi erat lex illa , \\\hline
3.2.31 & e non renouar las leyes dela tierra saluo \textbf{ si fuessen contrarias ala razon natural } e ala razon derecha & et non innouare patrias leges , \textbf{ nisi fuerit rectae rationi contrariae . } Circa regimen regni et ciuitatis \\\hline
3.2.32 & que commo quier que la çibdat \textbf{ en alguna manera sea cosa natural } por que auemos natural inclinaçion & quod cum ciuitas \textbf{ sit aliquo modo } quid naturale , eo quod naturalem habemus \\\hline
3.2.32 & en alguna manera sea cosa natural \textbf{ por que auemos natural inclinaçion } e desseo para establesçer e fazer la çibdat . & sit aliquo modo \textbf{ quid naturale , eo quod naturalem habemus } impetum ad ciuitatem constituendam : \\\hline
3.2.34 & quanto alguno mas se allega ala natura bestial \textbf{ tanto mas es natural miente sieruo } e de natura seruil . & ad naturam bestialem , \textbf{ tanto est magis naturaliter seruus . } Esse quidem sceleratum et affectatorem belli , \\\hline
3.3.2 & ca estos tales \textbf{ naturalmente han miedo de las feridas . } Ca por que naturalmente han poca sangre & et propterea non habent constantiam pugnandi neque fiduciam , \textbf{ quia naturaliter metuunt vulnera . } Nam cum naturaliter habeant modicum sanguinis , \\\hline
3.3.2 & naturalmente han miedo de las feridas . \textbf{ Ca por que naturalmente han poca sangre } naturalmente temen de perder la sangre . & quia naturaliter metuunt vulnera . \textbf{ Nam cum naturaliter habeant modicum sanguinis , } naturaliter timent sanguinis amissionem : \\\hline
3.3.2 & Ca por que naturalmente han poca sangre \textbf{ naturalmente temen de perder la sangre . } Et por ende estos non son apareiados & Nam cum naturaliter habeant modicum sanguinis , \textbf{ naturaliter timent sanguinis amissionem : } non ergo sunt prompta ad bella , \\\hline
3.3.7 & ca la costunbre es \textbf{ assi conmo vna naturaleza } Et por ende quando alguno es acostunbrado a leuar mayor carga & Nam consuetudo est \textbf{ quasi natura quaedam . } Cum ergo quis assuetus \\\hline
3.3.23 & ca si la naue se faze de madera verde \textbf{ quando el humor natural dellos se va e se seca . } los maderos se encogen & Nam si ex lignis viridibus construatur nauis , \textbf{ quando naturalis eorum humor expirauerit , } contrahuntur ligna , \\\hline

\end{tabular}
