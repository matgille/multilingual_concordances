\begin{tabular}{|p{1cm}|p{6.5cm}|p{6.5cm}|}

\hline
3.2.7 & e esta razon tanne el philosofo en el quinto libro delas politicas \textbf{ do dize que la tirnia es la postrimera obligarçia } que quiere dezer muy mala obligacion por que es muy enpesçedera alos subditos ¶ & multa mala efficere . Hanc autem rationem tangit Philosophus quinto Politicorum ubi ait , \textbf{ tyrannidem esse oligarchiam } extremam idest pessimam : quia est maxime nociua subditis . \\\hline
3.2.12 & Mas si enssennorear en pocos non por que son buenos \textbf{ mas por que son ricos es llamado obligarçia } que quiere dezer señorio tuerto . Mas quando enssennore a todo el pueblo & siue principatus bonorum . Si vero dominentur non quia boni , \textbf{ sed quia diuites , } est peruersus et vocatur oligarchia . \\\hline

\end{tabular}
