\begin{tabular}{|p{1cm}|p{6.5cm}|p{6.5cm}|}

\hline
1.2.27 & et moderantem superfluitates . In vindicando autem exteriora mala ab aliis facta , aliqui superabundant , \textbf{ ut iracundi nimis , } facientes vindictam . & por los otros algunos sobrepuian . \textbf{ Assi conmoson los muy sañudos } que dessean la vengança . \\\hline
1.2.27 & et de quolibet vindictam exposcere , \textbf{ quod faciunt iracundi , } vituperabile est . Rursum , & e sobre qual cosa dessear uengança \textbf{ la qual cosa fazen los sannudos } esto es de denostar . \\\hline
1.2.27 & Nam cum ira peruertat iudicium rationis , \textbf{ non decet Reges et Principes esse iracundos , } cum in eis maxime vigere debeat ratio & e del entendimiento \textbf{ non conuiene alos Reyes et alos prinçipes de seer sannudos } por que en ellos mayormente deue seer apoderada la razon e el entendemiento \\\hline
1.2.27 & et qui debet esse regula agendorum , \textbf{ inconueniens est quod sit iracundus , } ne per iram peruertatur & et que deue seer regla de todas las cosas fazederas . \textbf{ non es cosa conuenible al Rey de ser sañudo . } por que por la saña non sea tristornado nin torçido . \\\hline
1.2.27 & et conseruatores Reipublicae . \textbf{ Quare si Reges non debent esse iracundi , } et debent moueri & et mantenedores dela comunidat . \textbf{ Por la qual cosa | si alos Reyes non conuiene de seer sannudos } e deuen se mouer \\\hline
3.2.35 & ne Reges prouocentur \textbf{ ad iracundiam contra ipsos . } Ita autem & en qual manera se de una auer los que son en el regno \textbf{ por que los Reyes non se mue una asanna contra ellos . } Ca la sanna \\\hline

\end{tabular}
