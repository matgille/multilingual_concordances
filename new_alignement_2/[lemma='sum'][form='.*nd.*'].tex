\begin{tabular}{|p{1cm}|p{6.5cm}|p{6.5cm}|}

\hline
1.1.1 & et obseruentur , \textbf{ quae in hoc opere sunt dicenda , } totus ergo populus auditor & e non guardare todas aquellas cosas \textbf{ que sean de dezjr en este libro , } ¶ E pues que asi es todo el pueblo deue seer Oydor en alguna manera Deste libro \\\hline
1.1.3 & et malis : \textbf{ huiusmodi sunt industria mentis , } ingenium naturale , & e ño paresçen los quales pueden ser comunales alos buenos e alos malos \textbf{ Et estos son sotileza del entendemiento } e engennjo natural \\\hline
1.1.4 & restat dicere seriatim \textbf{ quae in hoc opere sunt dicenda . } Verum quia finis est & que auemos de dezer fincanos de dezer ordenadamente \textbf{ que cosas son de dezir en esta obra | e en este libro ¶ } Mas por que la fin es comjenço mas prinçipal de todas las obras \\\hline
1.1.4 & et ostendendum est , \textbf{ quomodo in eis felicitas est ponenda . } Distinxerunt autem Philosophi & e mostrͣemos commo enllas es de poner la bien andança \textbf{ que es la fin de los buenos ¶ } Mas los philosofos departieron tres uidas asy como paresçe \\\hline
1.1.4 & quod in vita voluptuosa \textbf{ non est quaerenda felicitas , } ut infra clarius ostendetur : & Ca mager que dissiese nudat \textbf{ que en la vida seliçonsa non es de poner bien andança } asy como adelante lo mostraremos mas claramente \\\hline
1.1.5 & Nam qui casu vel fortuitu bene agit , \textbf{ ex hoc non est laudandus , } nec debetur ei & Ca aqual \textbf{ que faze bien acaso e auentura por esto non es de alabar } njn por esto non le es deujda buean fin njn buena ventura . \\\hline
1.1.6 & quod insatiabilis est delectabilis appetitus . \textbf{ Secundo in talibus non est ponenda felicitas , } quia non sunt bona secundum rationem , & non se puede fartar delas delectaçones \textbf{ ¶ La segunda Razon es esta | que en estas delecta con nes } non es de poner feliçidat nin bien andança \\\hline
1.1.6 & cognitionem idest rationem percutiunt . \textbf{ Tertio non est ponenda felicitas } in voluptatibus sensibilibus , & e nol dexan entender lo que . \textbf{ cunple ¶ | La terçera razon por que non es de poner la feliçidat } o la bien andança \\\hline
1.1.7 & In neutris autem diuitiis \textbf{ est ponenda felicitas . } Tangit enim 1 Politicor’ tria , & enpero son riquezas artifiçiales \textbf{ Mas en njngunas destas riquezas non es de poner la bien andança } Ca el philosofo tanne tres razones \\\hline
1.1.7 & in artificialibus diuitiis \textbf{ felicitatem non esse ponendam . } Primo , quia artificiales diuitiae & por las quales nos pondemos prouar \textbf{ que la feliçidat e la bien andança non es de poner en les riquezas artifiçiales ¶ } La primera razon es por que las riquezas artifiçiales son orderandas alas riquezas natraales ¶la segunda \\\hline
1.1.7 & rationem felicitatis habere non possunt . \textbf{ Secundo in talibus non est ponenda felicitas , } quia non habent & por razon que son ordenadas alas riquezas natra a les non pueden auer razon nin manera e bien andanças \textbf{ la segunda razon por que non auemos de estableçer | nin de poner lanr̃a feliçidat ni } lanr̃abine andança en las riquezas artifiçiales \\\hline
1.1.7 & et dispositio utentium eis . \textbf{ Tertio in talibus non est ponenda felicitas , } quia aurum , et argentum , & La terçera Razon \textbf{ por que en las riquezas artifiçiales | non es de poner la feliçidat e la bien andança es esta } Ca el oro e la plata \\\hline
1.1.7 & per se non sufficiunt , \textbf{ in eis non est ponenda felicitas . } Quod autem in naturalibus diuitiis , & nin conplir \textbf{ por si alas menguas corporales . | Et por ende non es de poner la bien andança enellas . } Otrosy que nos non auemos de poner la nuestra feliçidat \\\hline
1.1.7 & et ea quae per se indigentiae corporali satisfaciunt , \textbf{ non sit ponenda felicitas } de leui patet . & Et ahun aquellas que abondan e cunplen por si las menguas corporales \textbf{ esto ligeramente lo podemos prouar . } ¶ Por que commo la feliçidat \\\hline
1.1.8 & quare si honor est bonum ordinatum ad virtutem , \textbf{ in honore non est ponenda felicitas , } sed potius in ipsis virtutibus , vel in actibus earum . & que es ordenado a uirtud . \textbf{ la bien andança non es de poner en la honrra } mas es de poner enlas uirtudes \\\hline
1.1.8 & vel in ipso beatificato , \textbf{ in honoribus non est ponenda felicitas . } Indecens est ergo cuilibet homini & que es mas en el bien andante \textbf{ que non en otro ninguno non es de poner en las honrras | ¶ } Pues que assi es muy sin Razon \\\hline
1.1.8 & si ab hominibus honoratur . \textbf{ Maxime tamen hoc est indecens regiae maiestati : } quod etiam triplici via venari potest . & si los omes le honrran \textbf{ Et muy mas sin razones | que la real magestad } ponga la su bien andança en las honrras . \\\hline
1.1.8 & suam felicitatem in honoribus ponere . \textbf{ Tertio hoc est indicens ei , } ne sit iniustus et inaequalis : & nin se ensoƀuezca mucho nol conuiene de poner su bien andança en las honrras ¶ \textbf{ Lo terçero se demuestra | assi porque non conuiene al prinçipe en ninguna manera } que sea iniusto nin desegual ¶ \\\hline
1.1.9 & in gloria et in fama . \textbf{ Tertio in fama non est ponenda felicitas , } quia magis innititur exterioribus signis , & e en eglesia ¶ \textbf{ Lo terçero en la fama non es de poner la feliçidat } nin la bienandança por que mas paresçe en las sennales de fuera \\\hline
1.1.9 & nequaquam tamen in fama , \textbf{ et in gloria hominum felicitas est ponenda . } Quod vero dicebatur , & Enpo en ninguna manera non es de poner la bienandança de lons omes \textbf{ en la fama de los omes | nin en la eglesia del mundo . } Mas aquello que dizian alguons de suso \\\hline
1.1.9 & et hunc esse honorem et gloriam . \textbf{ Non est intelligendus textus Philosophi , } quod Reges principaliter pro suo merito quaerere debeant gloriam , & auer era en eglesia e en honrra segunt el philosofo dezie . \textbf{ El testo del philosofo non se deue | assi entender } que los Reys prinçipalmente por su meresçimiento deuen demandar \\\hline
1.1.9 & et quoddam testimonium bonitatis , \textbf{ ut patet , non est condigna retributio in vita . } Attamen ut huiusmodi honor procedit & segunt que dicho es . \textbf{ Et por ende non es | gualardon ygual nin digno al su meresçimiento . } Mas enpero teniendo mientes ala honrra \\\hline
1.1.10 & sine bonitate vitae . \textbf{ Tertia vero , ex quod est indignus . } Quarta autem , ex eo quod per huiusmodi principatum ciues ordinantur & sin bondat deuida ¶ \textbf{ La terçera razon daquello que tal prinçipado pue de seer non digno¶ } La quarta razon es \\\hline
1.1.10 & diu durare non potest : \textbf{ felicitas enim non est ponenda in aliquo transitorio , } sed magis in aliquo sempiterno . & La segunda razon se declara \textbf{ assi que la feliçidat } e la bien andança non es de poner en poderio \\\hline
1.1.10 & Secundo in ciuili potentia \textbf{ non est ponenda felicitas , } quia hoc potest & assi que la feliçidat \textbf{ e la bien andança non es de poner en poderio } çiuilca tal poderio puede ser en \\\hline
1.1.10 & si abiiciat bene viuere . \textbf{ Non ergo in ciuili potentia est ponenda felicitas , } quae sine bonitate vitae inesse potest . & auentraado si despreçia el bien beuir . \textbf{ Et pues que assi es non es de poner la feliçidat | e la bien andança en poderio çiuil } que puede seer o non ser sin bondat deuida \\\hline
1.1.10 & Tertio in huiusmodi potentia \textbf{ non est ponenda felicitas , } quia huiusmodi Principatus non est optimus , & ¶ la terçera razon muestra \textbf{ que la feliçidat | et la bien andança non es de poner en este poderio çiuil . } Por que este sennorio non es muy bueno nin muy digno . \\\hline
1.1.10 & idest dominaliter . \textbf{ Quarto non est ponenda felicitas } in ciuili potentia : & seruilmente e sobre los sieruos ¶ \textbf{ La quarta razon es | que la feliçidat } e la bien andança \\\hline
1.1.11 & quod in bonis corporalibus \textbf{ non est ponenda felicitas . } Sunt tamen tria bona corporis , & que son dichas quanones \textbf{ de poner la bien andança en los bienes corporales } Empero tres son los bienes corporales \\\hline
1.1.11 & et robur corporalia esse dicuntur ; \textbf{ non ergo in eis est ponenda felicitas . } Secundo in talibus felicitas poni non debet , & Et por ende en tales bienes \textbf{ commo estos non denue seer puesta la feliçidat | nin la bien } andança¶ \\\hline
1.1.11 & quod testis est nobis Deus , \textbf{ quod felicitas in bonis interioribus est ponenda . Testificatur enim hoc Deus per seipsum , } ut idem ibidem innuit : & que la feliçidat \textbf{ e la bien andança es de poner en los bienes de dentro del alma . | Et esto testigua dios } por si mesmo \\\hline
1.1.11 & Tertio in talibus bonis \textbf{ non est ponenda felicitas , } quia sunt valde mutabilia . & La terçera razon es \textbf{ que en tales bienes non es de poner la feliçidat | nin la bien andança } por que son bienes muy mouibles \\\hline
1.1.12 & Voluit autem felicitatem \textbf{ non esse ponendam in viribus , } siue in potentiis animae , & que la feliçidat \textbf{ e la bien andança non se deue poner en las fuerças corporales } nin en las potençias del alma senssetuias \\\hline
1.1.12 & sed etiam mali participant . \textbf{ Nec etiam voluit esse ponendam eam in habitibus , } quia habens habitum , & en las disposiconnes \textbf{ nin en las scinas | que son en el alma } que el que ha las scians \\\hline
1.1.12 & siue in operatione animae \textbf{ est ponenda felicitas : } non in operatione vitii , & en las obras del alma \textbf{ e non en las obras de pecados } mas en obras de uirtud . \\\hline
1.1.12 & quandam dicimus esse virtutem . \textbf{ In amore ergo diuino est ponenda felicitas . } Sed cum probatio dilectionis & Pues que assi es en el amor de dios \textbf{ es de poner la feliçidat en la bien andança } e por que la praeua del amor \\\hline
1.1.12 & per quam immediatius coniungimur ipsi Deo , \textbf{ magis est ponenda felicitas , } quam in actu prudentiae : & por la qual nos somos ayuntados con dios \textbf{ sin ningun medio enlła deuemos poner la feliçidat e la bien andança } mas que en las obras dela pradençia \\\hline
1.1.12 & actu modo quo dictum est , \textbf{ aliqualiter felicitas sit ponenda . } Magnum autem esse praemium Regis , & que en estas obras dela perdençia \textbf{ es de poner en alguna manera la feliçidat et la bien andança | segunt dicho es } or çinco razones podemos prouar \\\hline
1.1.13 & si referatur ad ipsum Regem , \textbf{ cui huiusmodi praemium est reddendum : } nam omnis actus & gualaidon fuer conparado al rey mesmo \textbf{ a quien deue ser dado . } Por que todas las obras resçiben su bondat dela graueza que es en ellas . \\\hline
1.1.13 & Ex hoc autem est bonus , et virtuosus , \textbf{ inquantum est secundum naturam , } et ordinem rationis : & Et por tanto es dicha la obrar buena e uertuosa \textbf{ en quanto es segunt natura } e segunt orden de razon . \\\hline
1.1.13 & quanto ergo actus \textbf{ magis est secundum naturam , } et ordinem rationis , & e segunt orden de razon . \textbf{ Et pues que assi es quanto la obra es mas segunt natura } e segunt orden de razon \\\hline
1.2.1 & Consequenter autem manifestabitur , \textbf{ quomodo virtutes sunt distinguendae . } Postea vero ostendemus , & Et despues desto mostremos \textbf{ en qual manera se depart̃ las uirtudes . } Et despues desto mostr̉emos quantas son \\\hline
1.2.1 & et cibo moderate . \textbf{ Secundo in sensibus non est ponenda virtus moralis , } quia sicut potentiae naturales & tenpradamente del comer e del beuer ¶ \textbf{ La segunda razon es por que las uirtudes morales | non deuen ser puestas enlos poderios senssibles . } porque assi commo los poderios naturales \\\hline
1.2.2 & ut dimissis scientiis speculatiuis , de prudentia , \textbf{ et de virtutibus moralibus est tractandum . } Suscepimus enim & Aqui solamente auemos de fablar dela pradençia \textbf{ e de las uirtudes morales | que son en el entendimiento pratico o en el apetito . } Ca assi commo muchas vezes auemos dicho tomamos esta obra presente \\\hline
1.2.3 & vel virtutes annexas , \textbf{ singulariter est dicendum . } Numerus autem earum sic potest accipi . & de todo esto \textbf{ fablaremos mas conplidamente adelante . } Mas el cuento dellas \\\hline
1.2.4 & Sed primo de ipsis uirtutibus \textbf{ est dicendum . } Enumerauimus supra duodecim virtutes : & Mas primeramente diremos e fablaremos delas uirtudes . \textbf{ ¶ } a contamos de suso que son doze las uirtudes \\\hline
1.2.5 & et cardinales respectu aliarum , \textbf{ primo de his quatuor est dicendum . } Rursus quia prudentia est & e caddinales en conparaçonn delas otras . \textbf{ primeramente auemos dellas | de dezir que delas otras } ¶Otrosi por que la prudençia es mas prinçipal que todas las otras \\\hline
1.2.5 & et ad bonum commune quam ipsa temperantia : \textbf{ ideo hic ordo est tenendus . } Primo dicemus & e al bien comun que la tenperança . \textbf{ por ende esta es la orden que deuemos tener . } Ca primeramente diremos que cosa es la pradença . \\\hline
1.2.8 & Ratione igitur huiusmodi cognoscendi , \textbf{ qui est inditus hominibus , } volens alios dirigere , & Et pues que assi es por razon desta manera de conos \textbf{ çer que es enxerida naturalmente alos omes . } El que quiere alos otros guiar \\\hline
1.2.8 & ut tantam gentem regere habeat , \textbf{ oportet quod sit industris , et solers , } ut sciat ex se inuenire bona gentis sibi commissae . & que es puesto para gouernar tanta gente e tanto pueblo . \textbf{ Conuiene le que sea engennoso e sotil | por que sepa } por si buscar e fallar aquellos bienes \\\hline
1.2.9 & ex quibus scire potest , \textbf{ quid in quolibet negotio sit agendum . } Quarto saepe saepius excogitare debet , & por las quales puede saber \textbf{ que ha de fazer en cada negoçio¶ } La quarta manera es que muchas e muchas uezes deue cuydar en qual manera ahun \\\hline
1.2.9 & iudicat enim esse agendum \textbf{ quod est fugiendum , } et e conuerso . & iudgaque ha de fazer aquello que deuia escusar \textbf{ e alas vezes el contrario¶ pues que assi es bien dicho es } e con razon aquello \\\hline
1.2.9 & ne propter malitiam appetitus , imprudenter agant , \textbf{ et iudicent esse agenda , } quae sunt fugienda . & Et que non yerren en el iuyzio \textbf{ Judgando | que han de fazer aquello } que deuien escusar \\\hline
1.2.9 & et iudicent esse agenda , \textbf{ quae sunt fugienda . } Philosophus in 5 Ethicorum distinguit & que han de fazer aquello \textbf{ que deuien escusar } lphilosofo \\\hline
1.2.10 & ex hoc resultet commune bonum , \textbf{ et sit inde melior ciuitas , } et qualitercunque malus sit , & e donde se leunato el bien comun \textbf{ e por ende sea la çibdat meior } et en qual quier manera que el çibdadano sea malo \\\hline
1.2.13 & circa bonam mortem , \textbf{ et maxime circa eam quae est secundum bellum . } Sed adhuc et in mari , & aquel que non es temeroso cerca la buena muerte \textbf{ Et mayormente cerca aquella muerte | que es enlas batallas . } Mas ahun en la mar e enlas enfermedades . \\\hline
1.2.16 & Nam quanto magis aliquis voluntarie peccat , \textbf{ tanto magis est increpandus . } Rursus quanto aliquis facilius potest benefacere , & ¶La primera es que por que quanto cada vno peca mas de voluntad \textbf{ tanto es mas de denostar } Otrosi quando alguno mas ligeramente puede fazer bien \\\hline
1.2.16 & si non benefaciat , \textbf{ magis est detestandus , } et reprehensibilis . & Otrosi quando alguno mas ligeramente puede fazer bien \textbf{ e non lo faze mas es de denostar e de reprehender . } Et por ende el que no es tenprado es mas de denostar e de reprehender \\\hline
1.2.16 & et reprehensibilis . \textbf{ Intemperatus ergo magis est detestandus , } et reprehensibilis : & e non lo faze mas es de denostar e de reprehender . \textbf{ Et por ende el que no es tenprado es mas de denostar e de reprehender } que el temeroso¶ \\\hline
1.2.16 & quia sicut puer debet regi per paedagogum , \textbf{ sic vis concupiscibilis est regenda , } et regulanda per rationem . & por su ayo o por su maestro \textbf{ assi el apetito cobdiciador } e desseador se deue gouernar e reglar \\\hline
1.2.17 & respicere magnos sumptus ; \textbf{ quod quomodo sit intelligendum , } in prosequendo patebit . & Mas la magnificençia es tal uirtud que cata alas grandes despenssas . \textbf{ Et esto en qual manera se deue entender adelante lo mostrͣemos¶ } pues que assi es en faziendo espenssas \\\hline
1.2.17 & Uti autem pecunia , \textbf{ est expendere eam } et tribuere eam aliis . & en uso conueinble de espender del auer . \textbf{ Ca husar del auer es en espender lo } e partir lo alos otros \\\hline
1.2.20 & quod remouere a se pecuniam , \textbf{ sit abscindere membra a proprio corpore . } Ideo sicut dato & Ca paresçe leal paruifico \textbf{ que tirar el auer de ssi | estaiarle los mienbros de su cuerpo . } Por ende assi commo si fuesse menester \\\hline
1.2.20 & ( ut dictum est ) \textbf{ regia persona debet esse reuerenda et honore digna , } spectat ad Regem magnifice se habere erga personam propriam , & Otrosi por que assi commo dicho es . \textbf{ La persona del Rey deue ser de grand reuerençia } e digna de grand honrra parte nesçe mucho al Rey de se auer granadamente \\\hline
1.2.21 & Idem est enim esse magnificum , \textbf{ quod esse abundanter liberalem . } Nam ( ut infra patebit ) & por que essa misma cosa es ser magnifico \textbf{ que ser conplidamente libal e franço . } Ca assi commo paresçera adelante la \\\hline
1.2.26 & Non tamen aequae principaliter operatur utrunque : \textbf{ nam cum magnanimi sit tendere in magnum , } magnanimitas magis est & e el entendimiento muestran . Enpero estas dos cosas non las obra egualmente nin prinçipalmente \textbf{ Ca commo almagranimo pertenesca de yr | e entender en cosas grandes la magranimidat } mas es uirtud \\\hline
1.2.27 & praeter ordinem rationis . \textbf{ Ratio enim dictat punitiones aliquas esse faciendas , } et quod est irascendum , & fuera de orden de razon e de entendimiento \textbf{ por que la razon demanda | que algunas penas sean dadas } e algunas venganças sean fechas \\\hline
1.2.27 & Ratio enim dictat punitiones aliquas esse faciendas , \textbf{ et quod est irascendum , } cui debet , & que algunas penas sean dadas \textbf{ e algunas venganças sean fechas } e que cada vno se enssanne \\\hline
1.2.27 & quod nullo modo velit alios puniri , \textbf{ sed est condonatiuus , } et punitiuus & que los otros ayan pena \textbf{ por los males qual fezieton mas escondenador } e dador de pena \\\hline
1.2.27 & Nam cum ira peruertat iudicium rationis , \textbf{ non decet Reges et Principes esse iracundos , } cum in eis maxime vigere debeat ratio et intellectus . Sicut enim videmus & tristorna el iuyzio dela razon \textbf{ e del entendimiento non conuiene alos Reyes | et alos prinçipes de seer sannudos } por que en ellos mayormente deue seer apoderada la razon e el entendemiento \\\hline
1.2.27 & inconueniens est \textbf{ quod sit iracundus , } ne per iram peruertatur et obliquatur . & et que deue seer regla de todas las cosas fazederas . \textbf{ non es cosa conuenible al Rey de ser sañudo . } por que por la saña non sea tristornado nin torçido . \\\hline
1.2.27 & sed propter amorem \textbf{ et zelum est irascendum , } et sunt punitiones appetendae . & por rancor mas por amor \textbf{ e por zelo de bien se deue enssannar . } Et son de dessear las uenganças e las penas \\\hline
1.2.27 & et conseruatores Reipublicae . \textbf{ Quare si Reges non debent esse iracundi , } et debent moueri & et mantenedores dela comunidat . \textbf{ Por la qual cosa si alos Reyes non conuiene de seer sannudos } e deuen se mouer \\\hline
1.2.28 & veraces , et amicabiles , \textbf{ de quibus omnibus est dicendum . } Sed primo de amicabilitate . & e uerdaderos e amigables \textbf{ delas quales cosas todas auemos aqui de dozir } mas primero diremos dela amistan ca por que ueemos que en partiçipando \\\hline
1.2.28 & quia se ostendunt nimis amicabiles , \textbf{ cuiusmodi sunt blanditores et placidi . } Hi enim adeo se ostendunt communicabiles et sociales , & algunos sobrepuian por que se muestran mucho amigables \textbf{ quales son los que fablan palauras blandas e plazenteras } Ca estos en tanto se muestran conpanneros \\\hline
1.2.28 & quod secundum diuersitatem personarum \textbf{ diuersimode sit conuersandum : } licet omnes homines uolentes viuere politice & que segt el departimiento \textbf{ delons omes ha omne de conuerssar departidamente con ellos } commo quier que todos los omes \\\hline
1.2.29 & non tamen semper \textbf{ et ubique quaelibet veritas est dicenda : } loco enim et tempore & Ca commo quier que nin guon non deua mentir \textbf{ enpero non es de dezir cada vna uerdat sienpre } e en todo logar \\\hline
1.2.29 & non apertos et veraces , \textbf{ nullo modo in talibus est excedendum in plus , } sed magis declinandum in minus : & e non manifiestos nin uerdaderos . \textbf{ Et por ende en ninguna manera en tales cosas non deuemos sobrepuiar en lo } mas \\\hline
1.2.29 & quia ( ut dictum est ) \textbf{ in talibus magis est declinandum in minus dicendo de se minora quam sint , } quam sit excedendum & de repremir los alabamientos que tenprar los escarnesçimientos por que assi commo dicho es \textbf{ mas deuemos de elinar en tales cosas alo menos . . | deziendo cada vno dessi menores cosas } que sean en el que sobrepiuaren mas afirmando \\\hline
1.2.29 & quare in talibus \textbf{ semper est declinandum in minus . } Prima sumitur ex parte sui . & por dos razones que en tales cosas sienp̊ͤ \textbf{ deuemos declinar alo menos ¶ } Lo primera se toma de parte de ssi mismo¶ \\\hline
1.2.29 & declinandum esse in minus \textbf{ propter onerosas esse superabundantias . } Ostenso quid est veritas & declinaralo menos \textbf{ por la sobrepuiança de carga . } ¶ Visto que cosa es la uerdat \\\hline
1.2.30 & immo dicentibus sunt molesti . \textbf{ Patet ergo quid est iocunditas , } vel eutrapelia , & que fazen los solazes e los iuegos \textbf{ Et pues que assi es paresçe } que cosa es el alegria o la entropolia \\\hline
1.2.30 & et Principes decet \textbf{ esse iocundos . } Ut enim ex habitis patere potest , & e alos prinçipes \textbf{ de ser alegres e iugadores . } Ca assi commo puede paresçer delas cosas \\\hline
1.2.30 & passiones et motus , \textbf{ non est accipiendum secundum rem , } sed quo ad nos . & que repreme las passiones \textbf{ e los mouemientos non es de tomar | commo esta en la cosa } mas es de tomar \\\hline
1.2.33 & Circa diuina autem aiunt est duplex gradus . \textbf{ Nam aliqui sunt tendentes , } et euntes in diuina similitudine : & assi commo dizen algunos son dos guados \textbf{ Ca algers son que van en semeiança diunal } e tales son dichos auer uirtudes pgatorias . \\\hline
1.2.34 & est dispositio ad virtutem , \textbf{ et quomodo est conditio sequens virtutem , } non est praesentis speculationis . & Mas declarar en qual manera la perseuerança es disposicion ala uirtud . \textbf{ Et en qual manera es condicion segniente ala uirtud } non es deste presente negoçio \\\hline
1.3.1 & Sed cum constat de re , \textbf{ de verbis minime est curandum . } Enumeratis ergo passionibus , & non bͤapio a cada vna cosa podia lo fazer \textbf{ mas quando nos somos çiertos dela cosa non deuemos auer cuydado delas palauras . } Et pues que assi es contadas las passiones \\\hline
1.3.2 & Quia nullus bene seipsum regere potest , \textbf{ nisi sciat quae passiones sunt fugiendae , } et quae prosequendae : & orque niguno non puede bien gor̉inar \textbf{ assi mismo | si non sopiere quals passiones son de fuyr } e quales son de leguir \\\hline
1.3.2 & et quae sunt vituperabiles , \textbf{ et quae sunt sequendae , } et quae sunt fugiendae . & qual sson de loar e quales son de denostar . \textbf{ Et quales son de seguir } e quales de fuyr ¶ \\\hline
1.3.2 & et quae sunt sequendae , \textbf{ et quae sunt fugiendae . } Viso ergo quot sunt passiones , et quomodo accipitur earum numerus : & Et quales son de seguir \textbf{ e quales de fuyr ¶ | Et pues que assi es visto } quantas son las passiones \\\hline
1.3.2 & et typo innotescit nobis natura ipsarum : \textbf{ qua cognita , cognoscere possumus quomodo sint prosequendae , } et quomodo vitandae passiones praedictae . & e en figua a la qual naturaleza conosçida \textbf{ poremos conosçer | en qual manera auemos de segiuir } e de esquiuar las pasiones sobredichas \\\hline
1.3.3 & nisi ea extirpent et exterminent . \textbf{ Per se enim homines non sunt exterminandi , et odiendi : } sed quia vitia sunt extirpanda et odienda & e los non tollieren . \textbf{ Ca los omes | por si non son de destroyr } njn de aborresçer \\\hline
1.3.3 & Per se enim homines non sunt exterminandi , et odiendi : \textbf{ sed quia vitia sunt extirpanda et odienda } si non possunt & njn de aborresçer \textbf{ mas por que las malas obras de petado son de destroyr | e de aborresçer } si por auentra a non pueden en otra manera destroyr los males \\\hline
1.3.5 & quod decet Reges et Principes aliquid aggredi ultra vires , \textbf{ et sperare ultra quam sit sperandum . } Prima via sumitur & e alos prinçipes de acometer ninguna cosa \textbf{ mas que la fuerca suya } nin la su uirtud demanda . \\\hline
1.3.5 & Secunda , ex parte gentis sibi commissae . \textbf{ Sperare enim ultra quam sit sperandum , } et aggredi opus ultra vires suas , & Reyeᷤ¶ la segunda de parte dela gente del pueblo quales acomnedado . \textbf{ Por que es parmas | que deue omne esparar } e acometer alguna obra \\\hline
1.3.6 & indecens est ipsum timere timore immoderato . \textbf{ Secundo hoc est indecens , } quia immoderatus timor facit hominem inconciliatiuum . & e de auer temor destenprado e sin razon¶ \textbf{ Lo segundo esto es cosa desconuenible | por que el } temordestenprado \\\hline
1.3.6 & toti regno praeiudicium gignitur : \textbf{ quare si hoc est indecens , } non decet & Esto tal faze periuyzio a todo el regno \textbf{ por la qual cosa | si esto es cosa desconuenible } non conuiene alos Reyes de temer \\\hline
1.3.6 & Si vero quis nullam audaciam habeat , \textbf{ omnino est indecens : } quia tunc nihil aggreditur . & Mas si alguno non ouiesse ninguna osadia \textbf{ en toda manera | seria cosa desconuenible } por que estonçe non acometria ninguna cosa¶ Et pues \\\hline
1.3.7 & quia aliter , \textbf{ non esset vindicta . } Sed odienti quidem nihil differt : & contra quien ha saña aquel mal \textbf{ ca en otra manera non seria uengança . } Mas por çierto el que quiere mal a otro non cura desto . \\\hline
1.3.7 & Si enim ira rationem praecedat , \textbf{ dupliciter est cauenda . } Primo , quia non perfecte rationem audit . & e ante el entendimiento \textbf{ en dos maneras es de esquiuar e fuyr } ¶Lo primero por que non oye acabadamente la razon ¶ \\\hline
1.3.7 & quia non perfecte perceperunt \textbf{ quomodo exequendum sit mandatum illud . } Sic etiam et canes statim & por que non entendieron conplidamente \textbf{ en qual manera se auia de conplir aquel mandamiento . } En essa misma manera avn los canes \\\hline
1.3.7 & statim enim cum ratio dicit \textbf{ vindictam esse fiendam , } statim vult currere , & ento dize \textbf{ que sea techa vengança } luego quiere correr \\\hline
1.3.8 & quae est delectationi contraria . \textbf{ Nam si tristitia quaelibet est fugienda , } et habet rationem mali : & que es contraria dela delectacion . \textbf{ Et dizia assi que si qual si quier tristeza era de foyr } por que auia razon de mal \\\hline
1.3.8 & et habet rationem mali : \textbf{ quaelibet delectatio est prosequenda , } et habet rationem boni . & por que auia razon de mal \textbf{ qual si quier delectacion era de seguir } por que auia razon de bien . \\\hline
1.3.8 & Alii autem econtrario , \textbf{ dicebant omnem delectationem esse fugiendam . } Sed hi omnem delectationem condemnantes , & Mas otros dizian todo el contrario desto \textbf{ diziendo | que toda delectaçion era mala de foyr e de esquiuar } Mas todos estos que despreçia un a todas las delectaçiones \\\hline
1.3.8 & ( ut patet per Philos 4 Metaphy’ ) \textbf{ sic ponens omnem delectationem esse fugiendam , } ponit aliquam delectationem esse prosequendam . & en el quarto libro delas ethicas \textbf{ En essa misma manera el que pone que toda delectaçiones de esquiuar e de foyr pone que alguna delectaciones de segnir . } Ca assy commo la fabla non puede ser negada sinon por la fabla . \\\hline
1.3.8 & sic ponens omnem delectationem esse fugiendam , \textbf{ ponit aliquam delectationem esse prosequendam . } Nam cum loquela non possit negari , & en el quarto libro delas ethicas \textbf{ En essa misma manera el que pone que toda delectaçiones de esquiuar e de foyr pone que alguna delectaciones de segnir . } Ca assy commo la fabla non puede ser negada sinon por la fabla . \\\hline
1.3.8 & ut homines boni , et virtuosi . \textbf{ Sicut ergo non sunt dicenda vere dulcia , } quae videntur dulcia infirmis , & Et algunos lo han bien ordenado assi commo . \textbf{ los omes bueons e uirtuosos ¶ Et pues que assi es assi commo non son de dezir } uerdaderamente cosas dulçes \\\hline
1.3.8 & et habentibus linguam bene dispositam . \textbf{ Sic non sunt dicenda vere delectabilia , } quae sunt delectabilia vitiosis , & e en buena disposicion . \textbf{ Et essa misma manera non son de dezir uerdaderamente cosas delectables } aquellas que son delectables alos uiçiosos e alos malos \\\hline
1.3.8 & Si autem homines talibus delectationibus uti debent : \textbf{ hoc non est secundum se et simpliciter , } sed prout habent ordinem , & de tales delecta connes \textbf{ esto non deue ser sinplemente | e segunt si mas en quanto tal es delecta } connes han orden \\\hline
1.3.8 & quomodo se habere debent ad tristitias . \textbf{ Tristitia autem nunquam est assumenda , } nec est laudabilis , & finça deuer en qual maneras \textbf{ e de una auer alas tristezas . | Mas la tristeza nunca es de tomar } nin de loar \\\hline
1.3.8 & debet dolere et tristari . \textbf{ De turpibus igitur est tristandum , } sed omnis alia tristitia est moderanda , et vitanda ; & et entsteçer se dello . \textbf{ ¶ pues que assi es delas cosas torpes se deuen los omes entristeçer } e auer tristeza \\\hline
1.3.8 & De turpibus igitur est tristandum , \textbf{ sed omnis alia tristitia est moderanda , et vitanda ; } ut ergo huiusmodi tristitia moderetur , & ¶ pues que assi es delas cosas torpes se deuen los omes entristeçer \textbf{ e auer tristeza | e toda otra tristeza es destenpda e es escusadera . } Et por ende por que esta tal tristeza sea tenprada \\\hline
1.3.8 & nec videtur usque quaque vera . \textbf{ Nam cum de dolore amicorum sit dolendum , } cum nos dolemus , & que en todas maneras sea uerdadera \textbf{ Ca quando nos dolemos del dolor de los amigos dolemos nos nos } e veemos los amigos doler . \\\hline
1.3.8 & est consideratio veritatis . \textbf{ Nam licet de turpibus sit dolendum , } de bonis tamen fortunae , & e el penssamiento dela uerdat . \textbf{ Ca commo quier que nos deuemos doler delas } cosastorpes empero de los bienes dela uentura \\\hline
1.3.8 & eorum impedimur \textbf{ ab operibus virtuosis . Patet ergo non esse dolendum , } nisi de turpibus , & quanto por perdimiento de aquellos bienes somos enbargados delas obras uirtuosas . \textbf{ Et pues que assi es paresçe | que non nos deuemos doler } sinon delas cosas torpes \\\hline
1.3.11 & ipsum oportere operari , \textbf{ in quibus est verecundia . } Sic etiam ibidem dicitur , & que conuenga aellos de obrar ninguna cosa \textbf{ en que caya uerguença . } En essa misma manera ahun es dicho \\\hline
1.3.11 & Sic etiam ibidem dicitur , \textbf{ quod studiosi non est verecundari , } quia verecundia est in prauis : & En essa misma manera ahun es dicho \textbf{ y que los estudiosos non deuen ser | uirgon cosos . } por que la uerguença es delas cosas malas . Mas al estudioso non conuiene obrar ningunas cosas malas . \\\hline
1.3.11 & nisi esset vitiorum : \textbf{ nam vitia sunt odienda , } et sunt pro viribus extirpanda . & si non fuesse de pecados o de males . \textbf{ Ca los pecados londe mal querer } e aborresçer e son de deraygar por toda su fuerça \\\hline
1.3.11 & delectentur , et tristentur , \textbf{ ut est delectandum , et tristandum : } et ad haec omnia se habeant , & e se entristezcan en las cosas \textbf{ que deuen delectar e entsteçer . } Et a todas estas cosas se deuen auer \\\hline
1.4.1 & Sexto , et ultimo , \textbf{ quia sunt verecundi et erubescitiui . } Sunt enim iuuenes liberales , & Lo sexto e lo postrimero \textbf{ por que son uergo non sos | e toma uerguenña del mal ¶ } Lo primero los mançebos son liberales \\\hline
1.4.1 & est laudabile simpliciter : \textbf{ uidemus enim quod esse furibundum , } est laudabile in cane , & Mas qual si quier cosa \textbf{ que sea de loar en vno | e non en otro o es de loar } por alguna condicion non es de loar sinple mente . \\\hline
1.4.1 & Sextum autem , \textbf{ videlicet , esse verecundos , } non decet simpliciter competere Regibus et Principibus . Decet enim Reges et Principes esse liberales : & e de ser mibicordiosos . \textbf{ Mas la sexta condicion conuiene a saber ser uergoncosos . } Esta non conuiene nin pertenesçe \\\hline
1.4.2 & Quarto sunt contumeliosi . \textbf{ Quinto sunt mendaces , } omnia quodammodo pertinaciter asserentes . & que creen much̃e ayna ¶ \textbf{ Lo quarto son peleadores ¶ } Lo quanto son mintrosos afirmando \\\hline
1.4.2 & iis quae eis dicuntur , \textbf{ sed diligenter inspicerent utrum illa essent credenda . } Quarto sunt contumeliosi . & que a ellos son dichͣs \textbf{ mas catarian con grand acuçia | si aquellas cosas deuen ser creydas o non¶ } Lo quanto son peleadores . \\\hline
1.4.2 & et sic de facili sunt contumeliosi . \textbf{ Quinto sunt mendaces , } et quodammodo omnia pertinaciter asserunt . & por ende son peleadores de ligero ¶ \textbf{ Lo quinto son mintrosos } e afirman todas las cosas \\\hline
1.4.2 & et uniuersaliter ab omnibus dominantibus , \textbf{ est mendacium fugiendum , } ne seipsos contemptibiles reddant : & e generalmente a todos los omes \textbf{ que ha sennorio } por que non fagan assi mismos seer despreçiados . \\\hline
1.4.2 & in actionibus suis : \textbf{ quia cum alia sint moderanda per mensuram , } maxime decet mensuram & Ca commo todas las sus obras \textbf{ de una ser tenp̃das | por me lura } e por regla mucho \\\hline
1.4.3 & et pauca se facere sperant . \textbf{ Sexto senex sunt inuerecundi , } et inerubescitiui . & e esperan de fazer pocas cosas \textbf{ ¶ Lo sexto los uieios son desuergonçados } e non toman uerguenca delas cosas . \\\hline
1.4.3 & et non curant reputari : \textbf{ quare contingit eos esse inuerecundos . } Posset autem una causa assignari , & e non han cuydado \textbf{ que los otros les tengan en much̃ . | Et por esta razon aceesçe alos uieios de ser desuergoncados } Mas puede aqui ser fallada vna razon que es a comun a todas estas cosas de suso dichͣs . \\\hline
1.4.4 & sunt timidi \textbf{ ubi est timendum ; } et audaces , & Mas tienen el medio entre los mançebos e los uieios . \textbf{ Et por ende son temerosos do lo han de ser } e osados do lo han de ser . \\\hline
1.4.4 & et audaces , \textbf{ ubi est audendum . } Sic etiam , & Et por ende son temerosos do lo han de ser \textbf{ e osados do lo han de ser . } ¶ En essa misma manera ahun por que non son del todo \\\hline
1.4.4 & et iuuenum \textbf{ aliqui mores sunt imitandi , } aliqui fugiendi . & e de los \textbf{ mançebosson bueans } e de segnir \\\hline
1.4.4 & Sed eorum qui sunt in statu , \textbf{ quodammodo omnes mores sunt imitandi . } Dicimus autem , quodammodo : & Mas todas las costunbres de aquellos que son en estado medianero \textbf{ en alguna manera son bueans e de seguir . } Et dezimos en alguna manera \\\hline
1.4.6 & Hoc autem dictum Philosophicum \textbf{ diligenter est considerandum . } Nam diuitias , & Et por ende este dicho tan sotil del philosofo \textbf{ deuemos le estudiar con grand acuçia . } Ca las riquezas \\\hline
2.1.4 & ad cognoscendum domum , \textbf{ et ad sciendum qualiter sit regenda . } Nam ( ut superius tangebatur ) & siruen el en alguna manera al conosçimiento dela casa . \textbf{ Et para saber en qual manera se ha de gouernar la } casaca assi commo es dicho de suso \\\hline
2.1.5 & non debent ab inuicem separari . \textbf{ Nam non est separanda conseruatio a generatione : } quia haec illam praesupponit . & dellas non deuen ser departidas vna de otra . \textbf{ Ca non es de departir la conseruaçion dela generaçion } por que la conleruaçion presupone \\\hline
2.1.5 & est quasi coecus , \textbf{ et nescit qualiter sit eundum . } Qui ergo est huiusmodi , & assi commo çiego \textbf{ e non sabe en que manera deua andar . } Et por ende aquel que esta \\\hline
2.1.7 & Dicebatur in praecedenti capitulo , \textbf{ tria esse determinanda in hoc secundo libro , } in quo agitur de regimine domus , & icho es en el capitulo sobredicho \textbf{ que tres cosas son de determinar | en este segundo libro } en el qual tractaremos del gouernamiento dela casa \\\hline
2.1.7 & qualis amicitia sit viri ad uxorem , \textbf{ probat amicitiam illam esse secundum naturam : } adducens triplicem rationem & qual es el amistança del uaron \textbf{ a la muger prueua | que aquella amistança es segunt natura . } Et aduze para esto tres razones \\\hline
2.1.7 & consurgit bonitas ipsius domus : \textbf{ ipsa enim domus est inde melior , } sic est in bono vico , & dellas se leunata bondat dela casa \textbf{ por que la casa es por ende meior } si es en buen uarrio o en buena çibdat o en buen regno ¶ Et pues que assi es si el omne \\\hline
2.1.7 & Opera enim uiri uidentur esse in agendo , \textbf{ quae sunt fienda extra domum : } opera uero uxoris in conseruando suppellectilia , & mugnỉca las obras del uaron son en fazer aquellas cosas \textbf{ que son de fazer fuera de casa . } Mas las obras dela muger son en guardando las alfaias dela casa \\\hline
2.1.8 & inter aliquos nisi obseruent sibi debitam fidem ; \textbf{ ad hoc quod coniugium sit secundum naturam , } et ad hoc quod inter uxorem et virum sit amicitia naturalis , & assi mesmos fe conuenible \textbf{ para el casamiento | que sea segunt natura } e para que entre el uaron e la muger sea amistança natural conuiene que guarden vno a otro fe e lealtad \\\hline
2.1.8 & a tempore , \textbf{ quo urbs Romana fuit condita , } usque ad vigesimum et quingentesimum annum , nullum intercessit . & que ningun repoyo entre el marido e la muger non ouo del tienpo \textbf{ que la çibdat de Roma fue fecha } fasta çiento et cinquanta a nons . \\\hline
2.1.8 & non solum ex fide , \textbf{ quae in coniugio est seruanda , } inclinantur coniuges , & non solamente por la fe que es de guardar \textbf{ enel casamiento son inclinados los casados } por que se non departan de en vno . \\\hline
2.1.9 & ad praestandum filiis debitum nutrimentum , \textbf{ quia tamen naturale non est iudicandum illud quod est in paucioribus , } sed quod est ut in pluribus : & para dar conuenible nudermiento alos fijos . \textbf{ Empero por que non es iudgar la cosa natural | segunt aquello que es en pocas cosas } mas segunt aquello que es en las muchͣssiguese \\\hline
2.1.10 & secundum quascunque leges viuentes \textbf{ nunquam hoc fuisse indultum , } ut simul una mulier pluribus viris & si quier leyes nunca leemos \textbf{ que esto fuese otorgado | que vna mug fuesse ayuntada } a muchos uarones en vno por casamiento . \\\hline
2.1.11 & nimia consanguineitate coniunctis \textbf{ non sit ineundum coniugium , } triplici via venari possumus . & por grand parentesco . \textbf{ Esto podemos prouar por tres razones ¶ } La primera se toma dela grand reuerençia \\\hline
2.1.11 & cum huiusmodi reuerentia debita non reseruetur \textbf{ inter virum et uxorem propter ea quae inter eos mutuo sunt agenda , } dicta naturalis ratio , & entre la muger e el uaron \textbf{ por aquellas cosas | que son fazederas entre ellos vno con otro } por ende la razon natra al muestra \\\hline
2.1.12 & quasi primo et per se : \textbf{ sed pluralitas diuitiarum est intendenda quasi ex consequenti . } Decet enim eos talem uxorem acceptare , & entre todos los bienes de fuera . \textbf{ Et desi deuen enten dera muchedunbre de rriquezas | assi commo a cosa que se sigue . } Conuiene a ellos de tomar tal muger \\\hline
2.1.12 & quam multitudo diuitiarum . \textbf{ Omnia tamen haec tria aliquo modo sunt attendenda . } Ordinantur enim coniugium & que ala muchedunbre delas rrianzas . \textbf{ Enpero a todas estas tres cosas deuen entender en algua manera . Ca el mater moion es ordenado } assi commo paresçe en las cosas \\\hline
2.1.12 & Probabatur enim supra , \textbf{ coniugium esse secundum naturam , } eo quod homo naturaliter esse animal sociale : & Ca prouado es de suso \textbf{ que el casamiento deue ser segunt natura } por que el omne naturalmente es aian la conpannable ama termoino \\\hline
2.1.12 & in omni coniugio \textbf{ nimia imparitas videtur esse vitanda . } Nam imparitas in excessu , & e conpannia digna en todo casamiento \textbf{ deue ser esquiuada la grand desigualeza del maridor dela muger . } Ca la desigualeza en sobrepuiança \\\hline
2.1.12 & Nam imparitas in excessu , \textbf{ siue sit secundum nobilitatem , } siue secundum aetatem , & Ca la desigualeza en sobrepuiança \textbf{ si quier sea segunt nobleza | si quier sea segunt desnobleza } si quier sea segunt hedat \\\hline
2.1.13 & Ulterius autem declarauimus , \textbf{ qualia bona exteriora sunt quaerenda in coniuge . } Reliquum est ut dicamus , & e avn adelante declararemos \textbf{ quales bienes de fuera son de demandar enla muger . } Esto uisto finca nos de dezir \\\hline
2.1.13 & sed quantum ad bona animae , \textbf{ maxime videtur esse quaerendum in foemina } quod sit temperata , & Mas quanto los bienes del alma mayor ment en paresçe \textbf{ que deue ser demandada en la fenbra tenprança } e que sea tenprada \\\hline
2.1.13 & Illud enim bonum maxime videtur \textbf{ esse quaerendum in foemina , } ad cuius oppositum maxime incitatur . & e la fermosura finca de veer en qual manera quanto alos bienes del alma son de demandar en ella tenprança e amor de bien obrar . \textbf{ Ca aquel bien paresçe de ser demandado prinçipalmente en la fenbra } a cuyo contrario son las mugers mas inclinadas . \\\hline
2.1.14 & Rursus , dominium paternale \textbf{ magis est secundum naturam , } quam coniugale : & e sinplemente en toda manera que el marido ala muger . \textbf{ Otrossi el sennorio del padre mas es segunt natura } que el mater moinal . \\\hline
2.1.14 & tamen quod loquatur hoc idiomate vel illo , \textbf{ hoc est secundum placitum et ex electione . } Coniugale ergo regimen non est sic naturale , & lenguaie o en aquel estos es segunt uoluntad \textbf{ e por elecçion Et } pues que assi es el aruernamiento matermonial soes \\\hline
2.1.14 & cum sint adulti : \textbf{ ad quae non sunt instruendae uxores , } quia talibus vacare non debent . & quando fueren criados \textbf{ e mayores alas quales cosas non son de enssennar las mugers } por que non deuen entender atales cosas . \\\hline
2.1.15 & inter regimen coniugale , et seruile . \textbf{ Quare si decet ciues esse industres , } et cognoscere modum & e entre el del señor e del sieruo . \textbf{ Et por ende si conuiene alos çibdadanos de ser sabidores } e conosçer la manerar la orden natural \\\hline
2.1.15 & patet aliud esse regimen coniugale quam seruile : \textbf{ et non esse utendum uxoribus tanquam seruis . } Secunda via ad inuestigandum hoc idem , & Et pues que assi es de parte de la orden natural paresçe que otra cosa es el gouernamiento del marido ala mug̃r \textbf{ que del señor al sieruo . | Et paresçe que non deuen vsar los omes delas mugers } assi commo de sieruas ¶ \\\hline
2.1.16 & Determinandum est ergo particulariter , \textbf{ in qua aetate sit utendum coniugio . } Tangit enim Philosophus 7 Polit’ & que los generales . Et por ende deuemos determiuar particulariente \textbf{ en qual hedat | deua ser vsado el casamiento . } Ca el philosofo tanne en el . \\\hline
2.1.16 & quatuor rationes probantes \textbf{ quod in aetate nimis iuuenili non est utendum coniugio . } Prima ratio sumitur & vi̊ libro delas politicas quatro razones \textbf{ por que praeua que enla hedat de grand moçedat | non deuamos vsar del casamiento ¶ } La primera razon se toma de parte del dannamiento delos fijos ¶ \\\hline
2.1.17 & postquam probauit per rationes plurimas , \textbf{ non esse dandam operam coniugio in aetate nimis iuuenili : } inquirit quo tempore & por muchos razones \textbf{ que los omes non deue dar obra al casamiento } en la he perdat de grand mançebia demanda en quet pon deuen dar mas obra ala generaçion delos fijos . \\\hline
2.1.18 & timor de inglorificatione et de amissione laudis , \textbf{ mulieres communiter sunt uerecundae , } quia timent inglorificari & que los omes por la qual cosa commo la uirguença sea temor de non auer eglesia o de ꝑder alabança \textbf{ e las mugers son comunalmente uergonçosas } por que temen de non ser gliadas \\\hline
2.1.18 & quicquid tamen sit de eius causis , \textbf{ laudabile est in ipsis esse uerecundas : } quia propter uerecundiam multa turpia dimittunt & Enpero que quier que sea destas razones \textbf{ mucho es de alabar en ellas ser uergon cosas } ca por la uerguença dexan de fazer muchs cosas torpes \\\hline
2.1.18 & sunt valde crudeles : \textbf{ et cum sunt inuerecundae , } sunt nimis inuerecundae . Postquam enim mulieres audaciam capiunt , & Et quando son crueles son muy crueles \textbf{ Et quando son desuergonçedas son muy sin uerguença . } Ca del pues que las mugers toman osadia \\\hline
2.1.20 & nisi utantur ea discrete . \textbf{ Nam in omnibus operibus discretio est adhibenda , } ut fiant tempore debito , & tenpradamente si non vsaren del sabia mente . \textbf{ Ca en todas las obras la sabiduria es menester } por que las obras se fagan en tienpo conuenible \\\hline
2.1.22 & ostendentes nimis \textbf{ zelotypos non esse laudandos . } Primum est , quia viri in seipsis nimia turbatione vexantur . & para prouar \textbf{ que los muy çelosos non son de loar ¶ } La primera se toma de esto \\\hline
2.1.23 & Secundum vero , \textbf{ quod est attendendum , } est , quia est velox et citum . & es que el es muy flaco . \textbf{ Mas lo segundo a que deuemos parar mientes } es que el es apressurado e arrebatado . \\\hline
2.2.4 & ut possint discernere \textbf{ quod sit diligendum . } Immo quia pueri mox nati nesciunt & por que luego non son de tan grand conosçimiento \textbf{ que pueden conoscer } qual cosa deuen amar . Ca los moços luego que nasçen non saben apartar los padres de los otros omes \\\hline
2.2.5 & Secundo ea , quae sunt fidei , \textbf{ simpliciter sunt credenda . } Tertio eis est firmiter adhaerendum . & que son dela fe deuen ser \textbf{ en toda manera creydas | ¶ } Lo terçero deuen se los omes llegar firmemente \\\hline
2.2.5 & quia supra rationem sunt , \textbf{ ideo eis simpliciter est credendum . } Acquiescendum est autem & por que son sobre razon \textbf{ por ende las deuemos creer sinplemente } e deuemos estar por ellas \\\hline
2.2.5 & Si ergo ea quae sunt fidei \textbf{ simpliciter sunt credenda , } conuenienter talia in illa aetate proponuntur , & ¶ Pues que assi es si aquellas cosas \textbf{ que son de fe son de creer suplemente . | Conueinble cosa es que aquellas cosas tales } que pertenesçen ala fe sean enssennadas \\\hline
2.2.6 & statim cum pueri sunt sermonum capaces , \textbf{ sunt instruendi ad bonos mores , } et debent eis fieri monitiones debitae . & que han entendemiento para tomar razon \textbf{ estonçe son de enssennar en bueans costunbres } e deuen les ser fechos amonestamientos conuenibles . \\\hline
2.2.6 & et passionum insecutores : \textbf{ ergo tunc maxime est subueniendum , } ut per monitiones debitas , & e siguen sus passiones e sus desseos . \textbf{ Por la qual cosa estonçe los deuemos mucho mas acorrer } assi que por moniconnes conueinbles \\\hline
2.2.6 & Ne ergo iuuenes informentur habitibus vitiosis , \textbf{ statim ab ipsa infantia sunt monendi et corrigendi ; } ut per monitiones et correctiones debitas & Pues que assi es por que los moços non sean enformados en malas costunbres \textbf{ e de pecado | luego en su moçedat son de amonestar e de castigar } assi que por bueons amonestamientos \\\hline
2.2.8 & ideo grammatica \textbf{ inter liberales scientias est computanda , } quia filii liberorum et nobilium instruebantur in illa . & Et por que esto sepamos por la g̃matica . \textbf{ Por ende la g̃matica es contada entre las sçiençias libales } por que los fijos de los liberales \\\hline
2.2.8 & Ex hoc autem apparere potest , \textbf{ qui scientes magis sint honorandi . } Nam primo honorandi sunt diuini & Et avn desto puede paresçer \textbf{ que los sabios en la theologia son mas de honrrar . } Ca primero son de honrrar los omes diuinales \\\hline
2.2.8 & inter omnes alias scientias inuentas ab homine . \textbf{ Sic ergo scientes gradatim sunt honorandi . } His ergo scientiis sic diuisis , & Enpero tiene señono entre todas las otras sçiençias falladas por los omes . \textbf{ Et pues que assi es assi los omes sabios | son de honrrar cada vno en su grado . } Et pues que assi es estas sçiençias \\\hline
2.2.9 & iudicare \textbf{ quae sunt tenenda , } et quae respuenda . & commo delas falladas \textbf{ quales cosas son de retener } e quales de refusar . \\\hline
2.2.9 & obliquati fuerint \textbf{ qui ab eo sunt dirigendi : } sic debet prouidere futura , & por los tienpos passados \textbf{ los omes fueron torçidos e errados los quales deuen ser guardados por el ¶ Bien } assi deue proueer las cosas \\\hline
2.2.9 & sicut in cognoscendo et speculando \textbf{ est adhibenda cautela , } ne falsa admisceantur veris : & e en \textbf{ contenplando es de tomar cautela } por que las cosas falssas non sean mezcladas alas uerdaderas . \\\hline
2.2.10 & iuuenes sunt insecutores passionum , \textbf{ et ad lasciuiam proni . Quare cum semper sit adhibenda cautela } ubi periculum imminet , & los moços alos mançebos son segnidores de passiones \textbf{ e son inclinados a orgullos e aloçania . | Por la qual cosa commo sienpre deuemos dar } algunan cautellado paresçe el peligro . \\\hline
2.2.10 & et a sermonibus turpibus : \textbf{ et sunt increpandi et etiam corrigendi , } si eos talia loqui contingat . & e las palauras torpes . \textbf{ Et son mucho de denostar | e avn de castigar } por ello \\\hline
2.2.10 & a sermonibus turpibus prohibendi , \textbf{ est secundum Philosophum , } quia ex talibus locutionibus & Mas la razon por que les deuen ser defendidas las palauras torpes \textbf{ es esta segunt el philosofo } por que por tales palabras de ligero son inclinados a obras torpes . \\\hline
2.2.10 & ne loquantur lasciua . \textbf{ Secundo sunt cohibendi et corrigendi ; } ne loquantur falsa . & Otrossi deuen ser costrinudos \textbf{ e castigados los moços | e los mançebos } que non fablen cosas falssas . \\\hline
2.2.10 & Restat videre , \textbf{ quomodo sunt instruendi , } ut se habeant circa visum . & enla fabla finca de ver \textbf{ en qual manera son de enssennar } e commo se deuen auer en uista . \\\hline
2.2.10 & ut se habeant circa visum . \textbf{ In visione autem iuuenum duplex cautela est adhibenda . } Primo quantum ad visibilia . & e commo se deuen auer en uista . \textbf{ Ca en la uista de los moços | e de los mançebos dos cautellas son de tomar . } ¶ La primera quanto alas cosas uisibles \\\hline
2.2.10 & quod in Regibus , \textbf{ et Principibus valde est indecens , } si habeant oculos vagabundos . & por la uista \textbf{ e han los oios vagabundos | e por esto son iudgados } por liuianos de coraçon \\\hline
2.2.10 & restat ostendere , \textbf{ quomodo sunt instruendi , } ut se habeant ad auditum . & e quanto ala iusta finca de demostrar \textbf{ en qual manera son de enssennar e de castigar } e en commo se deuen auer çerca las cosas \\\hline
2.2.10 & Circa quem ( quantum ad praesens spectat ) \textbf{ etiam duplex cautela est adhibenda . } Primo , quantum ad res auditas . & que oyen e çerca el oyr . \textbf{ Et quanto pertenesçe alo presente dos cautellas son de tomar } ¶La primera quanto alas cosas \\\hline
2.2.12 & maxime est prona ad intemperantiam , \textbf{ quare cum semper sit adhibenda cautela , } ubi maius periculum imminet , & Ca do mayor es el periglo \textbf{ alli deue omne poner mayor remedio } Et por ende en la hedat de los moços deuemos guardar \\\hline
2.2.12 & ne iuuenes efficiantur intemperati . \textbf{ Temperantia autem circa tria est adhibenda : } circa cibum , potum , et venerea . & que non se fagan destenprados . \textbf{ Ca la tenprança ha de ser puesta çerca de tres cosas . } Conuiene de saber . \\\hline
2.2.12 & cum uxore iam ducta , \textbf{ et quae sunt consideranda in uxore ducenda : } supra , cum egimus de regimine coniugali , diffusius diximus . & que ya tienen tomadas \textbf{ e quales cosas son aquellas | que deuen cuydar en las mugers } que deuen tomar de suso lo dixiemos mas largamente \\\hline
2.2.13 & Disciplina autem , \textbf{ quae est danda in gestibus , } est , ut quodlibet membrum ordinetur & Et la doctrina e la enssenança \textbf{ que es de dar en los gestos alos moços } es tal que cada vn mienbro sea ordenado \\\hline
2.2.13 & tenent ora aperta : \textbf{ sic sunt indisciplinati secundum gestus , } qui cum volunt loqui , & e tienen las bocas abiertas \textbf{ para esto son | desenssennandos en los gestos . } En essa misma manera son desenssennados segunt los gestos \\\hline
2.2.14 & et de facili persuadetur eis , \textbf{ ut credant bona sensibilia esse sequenda . } Quia sermones particulares valde videntur & por que ciean \textbf{ que los bienes senssibłs son de segnir | mas que los otros } or que los sermones particulares son muy aprouechosos ala sçiençia de costunbres . \\\hline
2.2.15 & ostendemus qualis cura \textbf{ circa filios sit gerenda . } Primo enim declarabimus & mostraremos qual cuydado deua ser \textbf{ tomado çerca los fijos . } Et pues que assi es¶ \\\hline
2.2.15 & Primum est , quia ad septimum debent pasci mollibus , \textbf{ ita tamen quod a principio sint alendi lacte . } Secundum , quia sunt prohibendi a vino . & fasta los siete años de cosas humidas . \textbf{ Enpero assy que en el comienço dela nasçençia | mayormente sean cados con leche ¶ } La segunda es que les deuen defender el vino¶ \\\hline
2.2.15 & ita tamen quod a principio sint alendi lacte . \textbf{ Secundum , quia sunt prohibendi a vino . } Tertium , sunt assuescendi ad frigora . & mayormente sean cados con leche ¶ \textbf{ La segunda es que les deuen defender el vino¶ } La terçera es que los deuen acostunbrar alos frios \\\hline
2.2.15 & Secundum , quia sunt prohibendi a vino . \textbf{ Tertium , sunt assuescendi ad frigora . } Quartum , sunt assuescendi ad conuenientes et temperatos motus , & La segunda es que les deuen defender el vino¶ \textbf{ La terçera es que los deuen acostunbrar alos frios } ¶la quarta es que son de acostunbrar a mouimientos conuenibles e tenpdos . \\\hline
2.2.15 & Tertium , sunt assuescendi ad frigora . \textbf{ Quartum , sunt assuescendi ad conuenientes et temperatos motus , } quod in omni aetate videtur esse proficuum . & La terçera es que los deuen acostunbrar alos frios \textbf{ ¶la quarta es que son de acostunbrar a mouimientos conuenibles e tenpdos . } Et esto es prouechoso en todas las hedades ¶ \\\hline
2.2.15 & quod in omni aetate videtur esse proficuum . \textbf{ Quintum , sunt recreandi per debitos ludos , } et sunt eis recitandae aliquae historiae , & Et esto es prouechoso en todas las hedades ¶ \textbf{ La quinta es que lon de recrear | por trebeios conuenibles . } Et deuen rezar ante ellos algunas bueans estorias . \\\hline
2.2.15 & cum incipiunt percipere significationes verborum . \textbf{ Sextum , a ploratu sunt cohibendi . } Iuuenes ergo usque ad septennium alendi sunt mollibus ; & quando comiençan a entender las significa connes delas palabras . \textbf{ ¶ La sexta es que deuen ser guardados de llorar . } Et pues que assi es los moços \\\hline
2.2.15 & esse proportionatum proprio filio . \textbf{ Secundo pueri sunt prohibendi a vino , } et maxime illo tempore quo lac sumunt : & que otra \textbf{ ning¶lo segundo alos moços es de defendeᷤ el vino mayormente en aquel tienpo } que maman la leche . \\\hline
2.2.15 & disponuntur ad lepram . \textbf{ Tertio pueri sunt assuescendi ad frigora : } unde Philosophus septimo Politi’ ait , & vino resçiben ende apareiamiento para ser gafos . \textbf{ ¶ Lo terçero los mocos deuen ser acostunbrados al frio . } Onde el philosofo en el septimo libro delas politicas \\\hline
2.2.15 & et ut requirit conditio personarum . \textbf{ Quarto pueri sunt assuescendi ad conuenientes , } et temperatos motus . & assi commo demanda la condiçion dela perssona ¶ \textbf{ Lo quarto los mocos deuen ser acostunbrados } a mouimientos conuenibles e tenprados \\\hline
2.2.15 & ad aliquos motus modicos \textbf{ et temperatos sunt assuescendi , } ut membra eorum solidantur & por que han los mienbros muy tiernos \textbf{ deuen los acostunbrar a algers mouimientos pequanos et tenpdos } por que los mienbros dellos sean mas firmes . \\\hline
2.2.15 & et innocuas delectationes . \textbf{ Sexto sunt cohibendi a ploratu . } Nam cum pueri a ploratu cohibentur , & que les non enpezcan \textbf{ ¶ Loserto deuen ser castigados | que non lloren . } Ca quando alos moços defienden \\\hline
2.2.16 & ut cum dicimus , \textbf{ usque ad septem annos sic esse regendos : } a septimo usque ad decimumquartum sic esse instruendos , & assi deuian ser gouernados los moços . \textbf{ Et de los siete años | fasta los que torze } assi deuian ser enssennados . \\\hline
2.2.16 & usque ad septem annos sic esse regendos : \textbf{ a septimo usque ad decimumquartum sic esse instruendos , } huiusmodi septennia sunt abbreuianda et elonganda & fasta los que torze \textbf{ assi deuian ser enssennados . } Et estos setenarios son de encortar o de al ougar \\\hline
2.2.16 & a septimo usque ad decimumquartum sic esse instruendos , \textbf{ huiusmodi septennia sunt abbreuianda et elonganda } secundum diuersitatem personarum . & assi deuian ser enssennados . \textbf{ Et estos setenarios son de encortar o de al ougar } segunt el departimiento delas ꝑssonas . \\\hline
2.2.16 & eo quod nimis sit tenera , \textbf{ non sunt assumenda opera militaria nec opera ardua . } Unde Philosophus 8 Polit’ ait , & por que tal hedat es muy tierna \textbf{ non deuen tomar obras de caualłia nin otras obras fuertes . } Et por ende el pho enł viij̊ libro delas politicas dize \\\hline
2.2.16 & idest usque ad decimumquartum annum , \textbf{ leuiora quaedam exercitia sunt assumenda , } ne impediatur incrementum . & que quiere dezer \textbf{ fasta la hedat de los xiuf ͤ años deuen vsar de los vsos et mouimientos mas ligeros } por que non enbarguen la cresçençia delos mienbros ¶ \\\hline
2.2.16 & sed sint abstinentes et sobrii , \textbf{ ne sint mendaces sed veridici , } nec omnia agant valde & nin sus appetitos mas sean guardados e mesurados \textbf{ por que non sean mintrosos mas uerdaderos } nin fagan todas las cosas \\\hline
2.2.16 & Ne tamen cum incipiunt habere rationis usum , \textbf{ omnino sint indispositi ad scientiam , } assuescendi sunt ad alias artes , & Enpero por que quando comiençan a auer \textbf{ vso de razon non seanda todo mal apareiados ala sçiençia deuen ser acostunbrados alas otras artes delas } quales ya fiziemos mençion . \\\hline
2.2.17 & ut per eos respublica possit defendi . \textbf{ Qui autem magis et qui minus sunt assuescendi ad tales labores , } in prosequendo patebit . & e pueda defender la tierra . \textbf{ Mas quales son de acostunbrar } mas o menos atales trabaios adelante paresçra \\\hline
2.2.17 & ut instruantur in ulterioribus scientiis . \textbf{ Quae autem ulteriores scientiae eis sunt proponendae , } patet etiam per iam dicta . & en las sciençias mas altas . \textbf{ Mas quales sçiençias mas altas deuen sera ellos demostradas . } abn esto paresçe \\\hline
2.2.17 & quae durat usque ad decimumquartum annum , \textbf{ tales scientiae non sunt proponendae illis : } sed a decimoquarto anno ultra & que dura fasta los . xiiii̊ años tałs ̃ sçiençias altas \textbf{ non son de proponer } nin de enssennar a ellos Mas del . xiiiij̊ . año adelante \\\hline
2.2.18 & ad huiusmodi labores corporales \textbf{ minus sunt exercitandi quam alii , } et adhuc primogeniti & tra lasios corporales \textbf{ merenos son doma e devlar que los otros . } Et avn el primero gento que deue regnar \\\hline
2.2.18 & qui regnare debent , \textbf{ minus sunt assuescendi ad corporales labores quam alii , } ne propter huiusmodi labores & que deuen regnar \textbf{ por ende menos son de acostunbrar alos trabaios del cuerpo | que los otros } por que por tales trabaios la carne dellos non se faga dura \\\hline
2.2.18 & et a suis haeredibus \textbf{ magis est vitanda inertia , } quam per laborem , & e sus herederos \textbf{ e todos gouernadores meior escusar la peza et el uagar | que non por el vso corporal } nin por el trabaio del cuerpo . \\\hline
2.2.19 & ne prorumpant in turpia , \textbf{ videtur esse verecundia . } Decens ergo est cohibere puellas & por que non puedan sallir a fazer cosas torpes . \textbf{ Et pues que assi es cosa conuenible es de defender es alas mocas } que non corran \\\hline
2.2.20 & ut se circa talia exercitaret : \textbf{ adhuc non esset dimittendum , } ut non viueret ociosa , & que se trabaiasse en tales obras \textbf{ commo estas avn non la | deurien dexar } que beuiesse baldia \\\hline
2.2.21 & Ostenso , \textbf{ quod non decet puellas esse vagabundas , } nec decet eas viuere otiose : & que las mugers fuesen acuçiosas . \textbf{ ostrado que non conuiene alas moças de andar uagarosas a quande e allende } nin les conuiene de beuir ociosas \\\hline
2.3.2 & et organa inferiora , \textbf{ ut inanimata , sunt mouenda , } et administranda per organa superiora , & e que los instrumentos mas baxos \textbf{ assi commo los que non han alma sean mouidos e gados } por los uistrumentos mayores \\\hline
2.3.2 & nam animata sunt ante inanimata , \textbf{ et haec per illa sunt administranda et mouenda , } ut possint propria opera adimplere . & que non han alma . \textbf{ Et estas que non han alma deuen ser gouernadas e mouidas | por aquellas } que ban alma \\\hline
2.3.3 & ut tradit Pallad’ in lib de Agric’ \textbf{ est industria operis , } ut aedificium sit subtiliter & assi commo dize paladio enel libro dela \textbf{ agnitul traa es la sotileza dela obra } por que la morada sea sotilmente et conueiblemente fecho \\\hline
2.3.4 & et dispositio terrarum . \textbf{ Quantum ad conditionem caelestem duo sunt attendenda . } Primo ut hyeme debita claritate illustretur . & e la disposiconn delas tierras . \textbf{ Mas quanto ala condiconn del çielo | segunt dize paladio son de penssar dos cosas . } ¶ La primera que enł yuier no sean alunbradas \\\hline
2.3.8 & et putant ipsum finem \textbf{ in diuitiis esse ponendum , } appetunt eas in infinitum . & que han de poner su fin \textbf{ e su bien andança en las riquezas } dessean las sin mesura e sin fin . \\\hline
2.3.10 & et abundare vino et frumento , \textbf{ est abundare in diuitiis naturalibus : } sed abundare in denariis et numismatibus , & e abondar en vino \textbf{ e entgo es abondar en riquezas e possessiones naturales } mas abondar endmeros \\\hline
2.3.10 & ut patet per Philos’ 1 Pol’ \textbf{ est abundare in diuitiis artificialibus . } Si ergo ars naturam praesupponit , & mas abondar endmeros \textbf{ e en riquezas es abondar en | riquezasartifiçiales } e por ende si el arte presupone la natura \\\hline
2.3.10 & et ad pecuniam terminantur , \textbf{ uituperandae sunt secundum Philosophum : } nimis enim videtur esse denariorum cupidus , & e terminansse en dineros son de denostar \textbf{ segunt dize el philosofo } ca paresçe ser muy cudicioso de dineros \\\hline
2.3.12 & quae magis ad vinearum plantationem , \textbf{ et qualis cura circa arbores sit gerenda , } disposuimus silentio pertransire , & e qual mas para plantar las uinnas . \textbf{ Et qual cuydado deuen tomar çerca los arboles } que plantan propusiemos de pasar todas estas cosas en silençio e callando \\\hline
2.3.13 & Dicebatur supra in hac tertia parte huius secundi libri , \textbf{ de quatuor esse dicendum : } videlicet de aedificiis , possessionibus , numismatibus , et ministris siue seruis . & auemos de tractar \textbf{ e de dezir de quatro cosas . | Conuiene a saber } Delos hedifiçios e delas moradas . \\\hline
2.3.14 & secundum quas regentur regna et ciuitates : \textbf{ sic visum fuit conditoribus legum , } quod praeter seruitutem naturalem , & legunt las quales se gouernassen los regnos e las çibdades \textbf{ assi paresçio | alos estables çedores delas leyes } que lin la hudunbre natural \\\hline
2.3.14 & non resistere ordinationi legali : \textbf{ quia commune bonum bono priuato est praeponendum . } Tertia congruitas sumitur & non contradigan al ordenamiento dela ley . \textbf{ Ca el bien comun deue ser ante puesto al bien propreo ¶ } La terçera razon se toma dela salud de los vençidos . \\\hline
2.3.16 & quantum ad praesens spectat , \textbf{ tria sunt attendenda , } videlicet ordo ministrandi , & Enpero quanto pertenesçe alo psente . \textbf{ tres cosas deuemos pessar en esto . | Conuiene de saber . } la orden del ministrͣ \\\hline
2.3.16 & Primo enim attendenda sunt officia domestica \textbf{ sic esse committenda ministris , } ut reseruetur ibi debitus ordo ministrandi : & que los ofiçios dela casa sean \textbf{ assi acomnedados alos seruientes } por que sea y guardada la orden conuenible del seruiçio \\\hline
2.3.16 & sic se habet magna domus ad paruam . \textbf{ In magna enim ciuitate non sunt congreganda officia et principatus , } ita quod eidem committantur diuersa officia , & assi commo el dize \textbf{ çerca la fin del quarto | librdelas politicas non son de ayuntar } muchsofiçios \\\hline
2.3.16 & Tertio in commissione officiorum \textbf{ consideranda est conditio ministrantium . } Communiter enim ministri in duobus consueuerunt deficere : & o el seruiçio del ofiçio ¶ \textbf{ Lo terçero en acomne dar los o siçios es de penssar la con diconn de los seruientes . } por que comunalmente los ofiçiales acostunbraron de fallesçer en dos cosas \\\hline
2.3.16 & et in diuersis officiis commissis fideliter se gessit , \textbf{ fidelis est iudicandus . } Prudentia vero cognosci habet & qual fueren acomne dados \textbf{ si fielmente se ouiere en ellos tal deue ser iudgado por fiel . } Mas la sabidina se puede conosçer \\\hline
2.3.16 & et alia quae ibi diximus , \textbf{ prudens est reputandus : } et secundum magis et minus , & que dinemos alli este \textbf{ tal deue ser tendo por sabio } e esto segunt \\\hline
2.3.17 & qualiter circa hoc se habeant honorifice et prudenter : \textbf{ videndum est qualiter sunt exhibenda indumenta ministris . } Ad cuius euidentiam sciendum quod circa hoc & e sabrar \textbf{ en qual manera son de dar | e departir las vestiduras alos seruientes . } e para conosçimiento desto \\\hline
2.3.17 & ( quantum ad praesens spectat ) \textbf{ quinque sunt attendenda , } videlicet regis magnificentia , & quanto pertenesçe alo presente \textbf{ que cinco cosas son de penssar en esto . Conuiene a saber la magnificençia et gran dia del Rey . } La ordenança e semeiança de los siruientes . \\\hline
2.3.17 & Nam licet non ad inanem gloriam , \textbf{ nec ad ostentationem talia sint fienda : } tamen ut Reges et Principes conseruent se & ca commo quier que non por uana eglesia \textbf{ nin por aparesçençia uanas | e de una fazer tales cosas . } Enpero por que los Reyes e los prinçipes sean guardados en su estado granado \\\hline
2.3.17 & vel si non multum ad inuicem distant , \textbf{ eodem modo sunt induendi , } ut ex conformitate indumentorum & o que non son alongados mucho de vn grado \textbf{ deuen ser uestidos todos en vna manera e de vn panno . } por que por la semeiança delas uestidas sea conosçidos \\\hline
2.3.17 & Tertio circa prouisionem indumentorum \textbf{ consideranda est conditio personarum . } Nam non omnes decet & que son seruientes de vn prinçipe¶ \textbf{ Lo terçero çerca la prouision delas uestidas es de penssar la condiçion delas personas } por que non conuiene que todos sean uestidos \\\hline
2.3.17 & talia deseruientia ad indigentiam vitae , \textbf{ sunt varianda . } Videtur autem curialitas se habere & e las otras cosas \textbf{ que siruen ala mengua dela uida } paresçe que la curialidat \\\hline
2.3.19 & Secundo , ut ad officia commissa debite solicitentur . \textbf{ Tertio , ut sciatur qualiter cum eis est conuersandum . } Quarto , qualiter ipsis communicanda consilia & conueinblemente cerca los ofiçios \textbf{ que les son acomnedados . | ¶ Lo terçero que lep̃a los prinçipes } en qual manera han de beuir con ellos \\\hline
2.3.19 & Quinto et ultimo oportet cognoscere , \textbf{ qualiter sunt beneficiandi , } et quomodo sunt eis gratiae impendendae . & e lo postrimero conuiene de saber \textbf{ en qual manera los señores les han de fazer bien } e en qual manera les deuen fazer grans ¶ \\\hline
2.3.19 & ad officia nimis alta , \textbf{ sed gradatim et per diuturna tempora est habenda de ipsis experientia , } prius quam ad aliud altum ascendant . & para otros ofiçios mas altos \textbf{ mas deuen sobir de guado enguado | e deue ser tomada experiençia } e prueua dellos por luengo stp̃os \\\hline
2.3.19 & videlicet qualiter \textbf{ cum ipsis sit conuersandum . } Quod aliquomodo tradit Philosophus & ¶ Esto iusto finça de demostrar lo terçero \textbf{ que es en qual manera han de beuir los sennores con sus ofiçiales } la qual cosa muestra el philosofo en alguna manera \\\hline
2.3.19 & cum ipsis conuersandum : \textbf{ reliquum est ostendere , } qualiter sunt eis communicanda consilia , & e en qual manera deuen conuersar e veuir con ellos \textbf{ ¶ finca de demostrar } en qual manera deuen tractar sus conseios con ellos \\\hline
2.3.19 & magis quam merces aliqua quam inde habituri essent . \textbf{ Cum seruis ergo naturaliter non sunt communicanda secreta neque consilia : } quia ( ut supra dicebatur ) & alguaque ende espaauer . \textbf{ Et pues que assi es commo alos que son sieruos naturalmente | non londe descobrar las poridades } nin los conseios \\\hline
2.3.19 & Sic etiam nec seruis \textbf{ ex lege communiter sunt communicanda consilia : } quia tales & ca essa misma manera avn a los sieruos \textbf{ por ley non deuen ser descubiertos los conseios } por que tales por la mayor parte mas siruen por teraor que por amor . \\\hline
2.3.19 & quam ex amore . \textbf{ Nec etiam mercenariis sunt communicanda secreta : } quia tales non personam Principis , & por que tales por la mayor parte mas siruen por teraor que por amor . \textbf{ Otrossi uin a los que siruen por gualardon e por merçed | que atienden non son de descobrir los conseios } nin las poridades \\\hline
2.3.19 & quod ultimo dicebatur \textbf{ esse tractandum , } de leui potest patere . & mas en qual manera por los trabaios son de dar a los seruientes los bñfiçios . \textbf{ la qual cosa era postrimera de tractar de ligero } puede \\\hline
3.1.1 & quod videtur bonum ; \textbf{ nec sic est intelligendum , } ciuitatem constitutam esse gratia eius & que paresçe buena \textbf{ ni esto es de entender } assi que la çibdat sea establesçida por grande aquello que paresçe bueno \\\hline
3.1.4 & quam communitates illae , \textbf{ oportet eam esse secundum naturam . } Secunda uia ad inuestigandum hoc idem , & que estas dos comuidades \textbf{ por ende conuiene | que la çibdat lea comuidat natraal ¶ } La segunda razon para prouar \\\hline
3.1.4 & constat enim domus \textbf{ ex communitate viri et uxoris , domini , et serui , patris et filii , quarum quaelibet est secundum naturam . } Sic etiam communitas vici est & que acaban la casa son cosa natural . \textbf{ ca la casa se faze de comuidat de omne e de su muger e de sennor e de sieruo e de padre e de fijos . Et cada vna destas comuidades es cosa segunt natura bien } assi avn la comunidat del uarrio es cosa natural \\\hline
3.1.4 & ordinatur ad ciuitatem . Viso , ciuitatem esse aliquid secundum naturam : \textbf{ reliquum est ostendere , } hominem esse naturaliter animal politicum et ciuile , & que la çibdat es cosa natural . \textbf{ finca de demostrar } que el omne es naturalmente \\\hline
3.1.4 & ad quod habemus impetum naturalem , \textbf{ sit secundum naturam , } oportet ciuitatem & mas commo aquello a que auemos inclinaçion natural sea cosa natural \textbf{ conuiene quela çibdat sea cosa natural } e sea alguna cosa segunt natura \\\hline
3.1.9 & Diu ergo sunt principia pertractanda , \textbf{ et longis sermonibus sunt excutienda , } ne circa ipsa contingat error . & Et pues que assi es luengamente son los comienços de terctar \textbf{ e por luengos sermones son de } escodrinnar por que çerca ellos non contezca yerro . \\\hline
3.1.9 & Unde et Philosophus ait in Politicis ciues \textbf{ non esse applicandos legibus , } sed leges ciuibus . & Onde el pho dize en el terçero libro delas politicas \textbf{ que los çibdadanos non deuen ser llegados alas leyes } mas las leyes alos çibdadanos \\\hline
3.1.10 & et notitiam consanguineitatis , \textbf{ non est commendanda : } quia ex hoc consequitur aliquos & tire la cercidunbre de los fijos e el conosçimiento del parentesco \textbf{ non es de ella bartal comunidat } ca della se sigue \\\hline
3.1.12 & et prouidae sicut viri , \textbf{ non sunt ordinandae ad opera bellica . } Nam in bellis magna cautela & nin tan sabias commo los omes . \textbf{ Et por ende non son de poner en las | batallasca grant cautela } e grant sabiduria es meester en las batallas \\\hline
3.1.12 & Nam in bellis magna cautela \textbf{ et industria est adhibenda , } quia secundum Vegetium in De re militari , & batallasca grant cautela \textbf{ e grant sabiduria es meester en las batallas } segunt dizeuegeçio en el libro del negoçio dela caualleria \\\hline
3.1.14 & clarius apparebit \textbf{ quid circa huiusmodi regimen sit censendum . } Quare cum patefactum sit in praecedentibus , & mas claramente paresçra \textbf{ que auemos de iudgar en este gouernamiento } por la quel cosa commo sea manifiesto \\\hline
3.1.15 & quod Socrates et discipulus eius Plato dixerunt , \textbf{ ciuitatem sic esse regendam et gubernandam , } ut ciuibus communes essent uxores , et filii , et possessiones . & ontado es de suso que socrates e su disçipnlo platon dixieron \textbf{ que la çibdat deuia | assi ser gouernada } que todos los çibdadanos deuian auer las possessionos e las mugers \\\hline
3.1.17 & sed quia aliqua connubia sunt omnino sterilia , \textbf{ aliqua vero sunt foecundiora quam alia , } non est possibile statuere in ciuitate & Mas por que algunos casamientos son en toda manera manentos \textbf{ e alguons abondan mas en fijos } que otros por ende non se puede poner ley en la çibdat \\\hline
3.1.18 & vel quia existimant eis posse tristitiam inferre . \textbf{ non ergo solum propter possessiones sunt instituendae leges , } ut statuebat Phaleas : & o porque cuydan que les pueden fazer tristeza . \textbf{ Et pues que assi es non solamente son de esta | blesçer las leyes } por las possessiones \\\hline
3.1.20 & vel per haereditatem , \textbf{ et utrum leges sint innouandae dato } quod aliquem defectum contineant , & por elecçion o por suçession de heredamiento . \textbf{ Et si las leyes son assi de renouar | commo el dixo } puesto que algunan mengua aya en sus dichos \\\hline
3.1.20 & quod aliquem defectum contineant , \textbf{ et alia quae circa istam materiam sunt quaerenda , } infra diffusius tractabuntur : & puesto que algunan mengua aya en sus dichos \textbf{ e las otras cosas } que en esta materia son de dezer adelante lo tractaremos mas conplidamente . \\\hline
3.2.2 & tunc dicitur Monarchia siue Regnum : \textbf{ regis autem est intendere commune bonum . } Si vero ille unus dominans & e estonçe es dicho tal sennorio monarch̃ia o e egno \textbf{ ca al Rey parte nesçe de enteder el bien comun . } Et li aquel vno assi \\\hline
3.2.4 & simpliciter fuisse de intentione Philosophi , \textbf{ dominium plurium esse comendabilius dominio unius , } dum tamen utrunque sit rectum , & que fue la entençion del philosofo \textbf{ que el ssennorio de muchos es mas de alabar | que el ssennorio de vno } Puesto que amos los ssennorios sean derechs \\\hline
3.2.5 & Ideo ne periclitetur bonum commune , \textbf{ de omnibus filiis Regis cura diligens est habenda . } Philosophus 5 Politic’ & a qual de los fijos pertenesçra el regno . \textbf{ Por ende por que el bien comun non sea puesto a peligro de todos los fios deuen auer los padres grant cuydado } pho en el quanto libro delas poluenta tres cosas \\\hline
3.2.8 & Haec etiam tria sunt . \textbf{ Nam primo commissa sunt supplenda : } ut si viderint aliquid deesse & Et para esto ver son menester tres cosas . \textbf{ ¶ Ca lo primero las cosas | que menguna a buen gouernamiento } son de ennader \\\hline
3.2.8 & ad bonum regimen ciuitatis , \textbf{ illud est supplendum : } hoc autem fieri potest & para buen gouernamiento dela çibdat \textbf{ aquello deuen cunplir } e esto se puede fazer \\\hline
3.2.8 & Secundo , bonae ordinationes , \textbf{ et bona statuta debent esse obseruanda . } Tertio , bene operantes & Lo segundo los bueons ordenamientos \textbf{ e los bueons | establesçimientos son de guardar } ¶ Lo terçero los que bien obran \\\hline
3.2.8 & per quae resultat commune bonum , \textbf{ sunt remunerandi et praemiandi : } nam licet commissa supplere , & por que viene grant bien al comun son de gualardonar \textbf{ e de les fazer bien . Ca commo quier que las cosas menguadas se cunplan } e los bienes ordenados se guarden . \\\hline
3.2.11 & Ex hoc autem manifeste patet , \textbf{ tyrannidem maxime esse fugiendam a regibus : } quia pessimum de se , & e desto paresçe manifiesta miente \textbf{ que la tirama es much de escusar alos Reyes | e mucho han de foyr della } por que es . muy mala . \\\hline
3.2.12 & et perimere diuites et insignes . \textbf{ Summe ergo est cauenda tyrannis a Rege , } in qua tot mala congregantur . & e de matar los nobles e los rricos \textbf{ E pues que asi es muncho es de esquar alos rreyes la tiranja } enla qual son ayuntados tantos males commo son dichos \\\hline
3.2.15 & paulatim defluit in maiora . \textbf{ Principiis ergo est obstandum , } et inhibendae sunt obliquitates , & pequanos males ayna cahe en los grandes . \textbf{ Et pues que assi es deuemos contradezir en los comientos } e deuen avn ser defendidos los males pequanos . \\\hline
3.2.16 & quantum spectat ad praesens negocium , \textbf{ sufficienter tractauimus quae circa Principem sunt dicenda . } Restat ergo de consilio pertransire & Et quanto parte nesçe a este negoçio presente \textbf{ conplidamente dixiemos aquellas cosas | que eran de dezer } para enformaçion del prinçipe . \\\hline
3.2.16 & Si autem circa talia cadit consilium ; \textbf{ hoc non est secundum se , } sed prout deseruiunt actionibus nostris : & sobre tales cosas commo estas . \textbf{ mas si en tales cosas cae algun conse io esto non es } por si mas en quanto siruen algunas obras nr̃as . \\\hline
3.2.16 & ut sciamus quo tempore \textbf{ quae opera sunt fienda . } Tertio non sunt consiliabilia & por que sepamos en \textbf{ quetp̃o son de fazer algunas obras | e en que tp̃o orͣ̃s . } L tercero non cae so consseio \\\hline
3.2.17 & quia est quaestio agibilium humanorum : \textbf{ restat videre qualiter est consiliandum , } et quem modum in consiliis habere debemus . & ca es question delas obras \textbf{ que pueden fazer los omes finca de ver | en qual manera es de tomar el conseio } e qual manera deuemos tener en los conseios \\\hline
3.2.17 & ( quantum ad praesens spectat ) sex obseruanda , \textbf{ ut sciamus qualiter sit consiliandum . } Primum est , & Mas quanto pertenesçe alo presente seys \textbf{ cosasson de guardar para que sepamos en qual manera auemos de tomar conseio . } La primera cosa es que quanto alguna cosa es mas deteranada \\\hline
3.2.17 & et quanto minus habet certas et determinatas vias , \textbf{ tanto per plus tempus est consiliandum , } ut de illis viis facilior et melior eligatur . & e determinadas carreras \textbf{ para se fazer tanto mayor tienpo ha menester omne | para tomar consseio dello . } Porque de aquellas carreras escoia omne la meior e la mas ligera ¶ \\\hline
3.2.18 & Haec autem sunt tria , \textbf{ secundum quod in omni locutione tria sunt consideranda , } videlicet dicentem & e estas son tres cosas conuiene saber . \textbf{ ¶ el dezidor que fabla } Et el oydor a quien fabla \\\hline
3.2.18 & quia cognoscent negocia agibilia , \textbf{ et scient qualiter sit agendum . } Haec ergo tria quaerenda sunt in consiliariis : & que han de fazer \textbf{ e saben en qual manera los han de fazer . } Et pues que assi es estas tres cosas \\\hline
3.2.19 & Primo enim contingit esse Regis consilium circa prouentus , \textbf{ in quo duo sunt attendenda . } Primo , ne maiestas regia & sea cerca las sus rentas \textbf{ en la qual cosa dos cosas conuiene de penssar } ¶Lo primero couiene que el Rey non tome ningunas rentas \\\hline
3.2.19 & quod esse non posset , si bona eorum quae sunt in regno usurparet iniuste . \textbf{ Rursus est attendendum , } ne in suis prouentibus defraudetur : & la qual cosa non podria ser \textbf{ si tomasse los bienes } de aquellos que son en el su regno sin derech \\\hline
3.2.19 & Rursus circa custodiam ciuitatis et regni \textbf{ non solum sunt adhibenda consilia } propter ipsos ciues vel propter eos & Otrossi çerca la guarda dela çibdat \textbf{ e del regno non solamente son de tomar consseios } por essos mismos çibdadanos o por aquellos que son en el regno \\\hline
3.2.19 & et de hoc quod principaliter intenditur , \textbf{ nullus dubitat ipsum esse prosequendum . } De eius autem opposito & e de aquello que prinçipalmente omne entiende ninguno \textbf{ non dubda delo segnir . } Et del contrario della cada vno sabe \\\hline
3.2.19 & quilibet cognoscit \textbf{ ipsum esse fugiendum . } Ideo pax ciuium pro viribus est prosequenda , & Et del contrario della cada vno sabe \textbf{ que es de foyr . } Por la qual cosa la paz de los çibdadanos \\\hline
3.2.19 & ipsum esse fugiendum . \textbf{ Ideo pax ciuium pro viribus est prosequenda , } et eorum dissensiones & que es de foyr . \textbf{ Por la qual cosa la paz de los çibdadanos | con todo poder es de segnir . } Et las discordias dellos \\\hline
3.2.19 & Rector ciuitatis aut regni , \textbf{ qualiter circa lationem legum sit consiliandum . } Quae autem leges sunt ferendae , ut saluetur principatus eius , & podra ser enssennado el gouernador dela çibdat e del regno . \textbf{ en qual manera se deue consseiar çerca el establesçimiento delas leyes } Et quales leyes son de establesçer e de poner \\\hline
3.2.19 & qualiter circa lationem legum sit consiliandum . \textbf{ Quae autem leges sunt ferendae , ut saluetur principatus eius , } intendimus in capitulis sequentibus de legibus multa inquirere , & en qual manera se deue consseiar çerca el establesçimiento delas leyes \textbf{ Et quales leyes son de establesçer e de poner | por que se salue el prinçipado } e el sennorio del Rey . \\\hline
3.2.20 & aut per arbitrium , \textbf{ aut per utrunque priusquam ostendamus qualiter sit iudicandum , } declarare volumus quod quantum possibile est & por las leyes o por aluedrio o por amas estas cosas . \textbf{ ante que mostremos en qual manera deuemos iudgar . | queremos declarar } que quanto puede ser todas las cosas son de determinar \\\hline
3.2.20 & tamen quantum possibile est \textbf{ omnia legibus sunt determinanda , } et quanto pauciora possunt , & Enpero quanto pudieren ser \textbf{ todas las cosas deuen ser determinadas por las leyes e muy pocas cosas se deuen dexar en aluedrio de los mezes } assi commo dize el pho en el vij̊ . \\\hline
3.2.21 & et ad benignitatem et misericordiam erga seipsos . \textbf{ Sed quod tales sermones sint prohibendi , } triplici via inuestigare possumus . & e pugnan por lo mouer amanssedunbre e amiscderia contra si mismos . \textbf{ Mas que estas palabras malas sean de defender | e de deue dar } ante los alcalłs rodemos lo prouar por tres razones ¶ \\\hline
3.2.23 & ideo dicitur 1 Rhetor’ \textbf{ quod epiikis est indulgere humanis : } dicitur enim esse epiikiis & que mas que iusto es el \textbf{ que perdona alas obras humanales | e por ende el que perdona alas obras humanales } llama el pho \\\hline
3.2.23 & Philosophus igitur parcentem humanis appellat supra iustum , \textbf{ quia clementia est extollenda supra veritatem , } et supra iustitiam . & Ca la piadat es mas de alabar \textbf{ e mas de enssalças | que el castigo afincado } nin que la iustiçia afincada ¶ \\\hline
3.2.24 & quae sunt naturae . \textbf{ Quare si ius naturale dictat fures et maleficos esse puniendos , } hoc praesupponens ius positiuum procedit ulterius , & que son dela natura \textbf{ Por la qual cosa si el derecho natural manda | que los ladrones } e los mas fechores sean castigados \\\hline
3.2.25 & tanto plures defectus contrahunt , \textbf{ et in pluribus casibus non sunt obseruandae , } et maiorem mutationem suscipiunt : & quanto mas se allegan a algua materia espeçial tanto mas trahe consigo alguons desfallesçimientos \textbf{ e en muchos casos non son de guardar } e resçiben mayor mudamiento . \\\hline
3.2.26 & et ad gentem ad quam applicatur , \textbf{ quae per illam legem est regulanda . } Tria igitur lex habere debet , & e puede se conparar ala gente ala que es dada . \textbf{ la qual gente ha de ser reglada | por aquella ley . } Et pues que al sy es tres cosas \\\hline
3.2.26 & Tertio lex prout comparatur ad populum \textbf{ cui est imponenda , } debet esse compossibilis & en quanto es conparada al pueblo \textbf{ al quien es dada deue ser tal que se pueda guardar } e sea conuenible ala tierra \\\hline
3.2.27 & et dirigere aliquos in aliquod bonum , \textbf{ eiusdem est condere leges , } et regulas agibilium & Ca aquel cuyo es de ordenar e enderesçar a alguos en algun bien atlgun aquel parte nesçe \textbf{ establesçer leyes e reglas delas nuestras obras . } por las quales leyes ymosa aquel bien . \\\hline
3.2.28 & aliquem populum regere . \textbf{ Ideo non solum permittenda sunt indifferentia , } sed etiam quae modicam malitiam habent & o por quales quier culpas pequanas apenas o nunca poder e gouernar ningun pueblo . \textbf{ Por ende non solamente son de conssentir aquellas cosas | que nin son malas nin bueans . } Mas avn aquellas cosas \\\hline
3.2.28 & antequam fiant , \textbf{ sunt prohibenda : } sed postquam iam facta sunt , & ø \\\hline
3.2.28 & sed postquam iam facta sunt , \textbf{ sunt punienda . } Opera vero notabiliter bona , & assi que las malas obras ante que se fagan son mucho de defender . \textbf{ mas despues que son ya fechas son de castigar . } Et las obras buen asante \\\hline
3.2.28 & antequam fiant , \textbf{ per leges sunt praecipienda et consulenda : } facta vero sunt praemianda . & que se fagan son demandar \textbf{ e de consseiar por las leyes . } Mas despues que son fechͣs son de gualardonar . \\\hline
3.2.28 & per leges sunt praecipienda et consulenda : \textbf{ facta vero sunt praemianda . } Sed opera indifferentia , & e de consseiar por las leyes . \textbf{ Mas despues que son fechͣs son de gualardonar . } Mas las que non son buenas nin malas o nin son muy buean so muy malas . \\\hline
3.2.28 & et suos consiliarios discutiant \textbf{ quae bona sunt praecipienda et praemianda , } et quae mala sunt prohibenda , & asi que con grant acuçia \textbf{ por si e por sus consseieros examun en quales bueans obras son demandar } e de poner so mandamiento . \\\hline
3.2.28 & quae bona sunt praecipienda et praemianda , \textbf{ et quae mala sunt prohibenda , } et quae dissimulanda , & por si e por sus consseieros examun en quales bueans obras son demandar \textbf{ e de poner so mandamiento . | Et quales malas son de vedar e poner so pena } et quales son de desseneiar e de sofrir \\\hline
3.2.29 & quod eligibilius est principari lege , \textbf{ quia Reges aut Principes ita sunt instituendi , } ut seruatores legis et ministri . & que la ley enssennore \textbf{ e que los Reyes e los prinçipes } que son establesçidos paser guardadores dela ley o seruidores delas leyes . \\\hline
3.2.29 & Itaque ut appareat \textbf{ quid circa hanc materiam sit dicendum , } sciendum est regem & Et por ende por que parezca meior \textbf{ que auemos de dezir e de sentir en esta tal materia . } conuiene de saber que el rey \\\hline
3.2.29 & et non obseruare legem , \textbf{ ubi non est obseruanda . } Ex hoc autem patere potest , & e qua non guarde la ley positiua \textbf{ do non la deue guardar . } Et desto puede paresçer en qual manera el rigor \\\hline
3.2.31 & vel quaecunque aliae leges sic prauae et iniustae , \textbf{ non sunt obseruandae , } sed extirpandae . & o otras quales si quier leyes malas \textbf{ e non derechas non son de guardar } mas de tirar e de derraygar . \\\hline
3.2.31 & dato quod occurrant leges meliores et magis sufficientes , \textbf{ non est assuescendum innouare leges . } Primo , quia aliquando contingit & que sean falladas leyes meiores e mas conplidas . \textbf{ Enpero non nos auemos a acostunbrar a renouar las leyes . } Lo primero por que algunans vegadas contesçe que se engannan los omes \\\hline
3.2.33 & ex aequalitate et iustitia , \textbf{ quae est seruanda in ciuitate : } Nam ( ut ait Philosophus ) & la terçera razon se toma dela egualdat e dela iustiçia \textbf{ que se deue guardar en la çibdat . } Ca assi commo dize el philosofo \\\hline
3.2.33 & Qui autem secundum excessum \textbf{ sunt indigentes et pauperes , } nesciunt principari . & Mas los otros que sobrepuian en grand mengua \textbf{ e en grand pobreza } non saben \\\hline
3.2.34 & Nam ( ut dicebatur in praecedentibus ) \textbf{ intentio legislatoris est inducere ciues ad virtutem . } In recta enim Politia & dich̃en los capitulos \textbf{ sobredichos la entençion del ponedor dela ley es enduzer los çibdadanos o uirtud . } Ca en la derecha poliçia \\\hline
3.2.36 & et ex dilectione legislatoris , \textbf{ cuius est intendere commune bonum , } quiescant male agere : & e por amor del prinçipe ponedor dela ley \textbf{ cuya entençiones de tener mientes al bien comun } que por ende queden los omes de mal fazer . \\\hline
3.3.1 & et ad quid sit instituta . \textbf{ Sciendum igitur militiam esse quandam prudentiam , } siue quandam speciem prudentiae . & Et pues que assi es deuedes saber \textbf{ que la caualleria es vna prudençia o vna manera de sabiduria . . | Mas podemos quanto pertenesçe a lo presente } departir çinco maneras de prudençia e de sabiduria . \\\hline
3.3.2 & In partibus igitur nimis propinquis soli , \textbf{ non sunt eligendi bellantes : } quia in eis deficit strenuitas & que son muy cercanas al sol \textbf{ non son de escoger los lidiadores } por que en aquellas tierrras \\\hline
3.3.2 & et nimis a sole remotis , \textbf{ non sunt eligendi bellantes : } quia et si illis est sanguinis copia , & e muy arredradas del sol \textbf{ non son de escoger los lidiadores por que commo quier que en aquellas abonde mucho la sangre } assi que non teman las feridas . \\\hline
3.3.2 & in hominibus bellicosis , \textbf{ ut sciamus quales homines sunt eligendi ad bellum , } et ex quibus artibus sunt assumendi bellantes . & que son menester en los omnes lidiadores \textbf{ por que sepamos quales omnes son de escoger para la batalla } e de quales artes son de tomar los lidiadores . \\\hline
3.3.2 & ut sciamus quales homines sunt eligendi ad bellum , \textbf{ et ex quibus artibus sunt assumendi bellantes . } Sciendum ergo quod cum bellantes debeant & por que sepamos quales omnes son de escoger para la batalla \textbf{ e de quales artes son de tomar los lidiadores . } Et pues que assi es conuiene de saber \\\hline
3.3.2 & Aucupes , et Piscatores \textbf{ non sunt eligendi ad huiusmodi opera : } eo quod non habeant artem conformem operibus bellicosis . & e los bretadores \textbf{ que toman los paxaros con el brete | e los pescadores no son do escoger } para las obras de la batalla \\\hline
3.3.4 & et feriendi valde est expediens bellatoribus . \textbf{ Qualiter , autem talis industria habeatur , et qualiter sit feriendum hostem , } et quae adminiculantur ad ista , & mucho es conuenible a los lidiadores \textbf{ Mas en qual manera tal sabiduria se puede auer | e en qual manera han de ferir alos enemigos } e otras cosas que se ayuntan a estas adelante paresçra Lo . viij° \\\hline
3.3.4 & et pro communi bono exponenda est periculo corporalis vita , \textbf{ non est cauenda effusio sanguinis , } et caetera alia sunt fienda , & es de poner la vida corporal a periglo de muerte \textbf{ e non deue foyr | nin auer miedo de esparzer la sangre } e todas las otras cosas dichas \\\hline
3.3.4 & non est cauenda effusio sanguinis , \textbf{ et caetera alia sunt fienda , } per quae iustitia et commune bonum defendi potest . & nin auer miedo de esparzer la sangre \textbf{ e todas las otras cosas dichas } que son de fazer \\\hline
3.3.6 & propter inexercitium armorum \textbf{ non sunt industres in bellis . } Exercitium enim in quolibet negocio praebet audaciam , & por non auer vso de las armas \textbf{ non son sabidores en las faziendas . } Ca el vso en cada vn negocio \\\hline
3.3.7 & Rursus , non solum arma , \textbf{ sed etiam plura alia sunt ferenda in bello : } ideo etiam ad maiora pondera & Otrossi non solamente son de vsar los lidiadores a leuar las armas . \textbf{ mas avn a otras muchas cosas | que son de leuar en las batallas . } Et por ende prouechosa cosa es de se acostunbrar los lidiadores a leuar grandes pesos . \\\hline
3.3.8 & quis superabundare cautelis . \textbf{ In pugna enim omnino est eligendum , } maiorem diligentiam habuisse & que mas non sean menester . \textbf{ Ca en la batalla sienpre auemos de tomar mayor acuçia } de que demandan las batallas . \\\hline
3.3.8 & quod contra suos victores vix aut nunquam audent bella committere . \textbf{ Quare si in bellis omnino est superabundandum cautelis , } non est praetermittendum & e apenas o nunca osan acometer batalla contra ellos . \textbf{ Por la qual cosa si en las batallas auemos de auer muchas cautellas } non deuemos dexar qual si quier cosa que sea aprouechosa a la hueste en qualquier caso . \\\hline
3.3.8 & quomodo huiusmodi monitiones \textbf{ et castra sunt construenda . } Nam si hostes sunt absentes & finca de demostrar en qual manera las tales guarniciones \textbf{ et los tales castiellos se deuen fazer . } Ca si los enemigos non estudieren cerca de ligero pueden fazer carcauas çerca de la hueste \\\hline
3.3.8 & inter quorum spatium est exercitus collocandus , \textbf{ tria sunt consideranda , } videlicet situs , forma ; & deue estar assentada la hueste toda . \textbf{ Et por ende son de penssar tres cosas . } Conuiene a saber el assentamiento \\\hline
3.3.9 & quantum ad praesens spectat \textbf{ sex sunt attendenda : } sicut etiam ex parte auxiliorum & que lidian quanto pertenesçe a lo presente \textbf{ son seys cosas de penssar . } Assi commo avn de parte de las ayudas \\\hline
3.3.9 & sex alia enumerari possunt , \textbf{ quae etiam sunt attendenda . } In uniuerso igitur rex , & otras seys cosas \textbf{ las quales avn son de penssar . } Et pues que assi es en general el rey o el prinçipe \\\hline
3.3.9 & et duriores corpore . \textbf{ Quinto , qui sunt industriores , } et sagaciores mente . & e mas duros en el cuerpo . \textbf{ Lo quinto quales son mas sabidores } e mas arteros para lidiar . \\\hline
3.3.9 & ut dimicare non possint . \textbf{ Sexto est attendendum , } qui plures auxiliatores expectant . & contra si resçiben daño de los oios en manera que non pueden lidiar . \textbf{ Lo vi° es de penssar quales esperan mayores ayudas . } ca si los enemigos esperan mayores ayudas \\\hline
3.3.9 & Nam si hostes plura expectant auxilia , \textbf{ vel non est bellandum , } vel acceleranda est pugna . & ca si los enemigos esperan mayores ayudas \textbf{ o non conuiene de } lidiaro conuiene de apressurar la batalla . \\\hline
3.3.10 & et ignorantes ad quid deberent attendere : propter quod si in debellatione vita multorum hominum periculis mortis exponitur , \textbf{ cum magna diligentia vexillifer est quaerendus . } Ex dictis etiam patere potest , & es puesta a periglos de muerte \textbf{ con grant acuçia | e con grant diligençia } deue ser escogido el alferes . \\\hline
3.3.11 & de quorum consilio agat \textbf{ quicquid viderit ipse dux belli esse fiendum . } Nam ubi tantum currit periculum , & de cuyo conseio faga todas aquellas cosas \textbf{ que ouiere de fazer . } Ca do puede contesçer tan grant periglo \\\hline
3.3.11 & Itaque cum pericula visa minus noceant , \textbf{ per velocissimos equites sunt detegendae insidiae , } ne exercitus circa aliquam partem ex improuiso patiatur molestias . & Et por ende por que los periglos que son ante vistos menos enpeesçen . \textbf{ por caualleros muy ligeros son de descobrir las çeladas } por que la hueste non aya de resçebir a desora en alguna parte algunos daños . \\\hline
3.3.12 & ut se defendant : \textbf{ tunc est construenda acies } secundum rotundam formam ; & Mas cunpleles que se puedan defender . \textbf{ Estonçe es de establesçer el az } segunt forma redonda . \\\hline
3.3.12 & secundum formam quam appellant conum , \textbf{ id est secundum figuram pyramidalem et acutam , } ut possit hostes scindere et diuidere . & segunt forma que llaman cuño \textbf{ o segunt forma de pera e aguda . } por que puedan fender \\\hline
3.3.13 & Ostenso qualiter sunt acies ordinandae et construendae , \textbf{ reliquum est ostendere , } qualiter pugnantes percutere debeant , & m mostrado en qual manera son de establesçer \textbf{ e de ordenar las azes fincanos de mostrar en qual manera los lidiadores deuen ferir } e si es meior de ferir cortando o ferir de punta o estocando . \\\hline
3.3.14 & cum modo opposito se habent , \textbf{ sunt inuadendi et debellandi . } Primo igitur dux belli per insidias & quando son las maneras contrarias \textbf{ de aquellas siete son de acometer e de ferir . } Et pues que assi es lo primero el señor de la batalla por ascuchas \\\hline
3.3.14 & ponat eam Vegetius , \textbf{ non multum est appretianda : } quia repugnaret bonis moribus . & Mas esta cautela commo quier que la ponga vegeçio \textbf{ non es mucho de preçiar } por que es contraria a buenas costunbres . \\\hline
3.3.15 & Inde est quod laudatur Scipionis sententia , dicentis : \textbf{ Nunquam sic esse claudendos hostes , } quod non pateat eis aditus fugiendi . & Et por ende es alabada la sentençia de çipion \textbf{ por la qual dizia que nunca eran de encerrar los enemigos } assi que les non fincasse logar para foyr . \\\hline
3.3.15 & Restat nunc tertio declarare , \textbf{ qualiter sit declinandum a pugna , } si non habeatur consilium , & finca nos agora lo terçero de mostrar \textbf{ en qual manera se deuan arredrar de la batalla } si non ouieren conseio de lidiar \\\hline
3.3.15 & Nam et si dux consilium habeat \textbf{ non esse pugnandum , } debet hoc valde paucis patefacere , & La primera es de parte de la su hueste propria . \textbf{ Ca si el cabdiello ouiere conseio de non lidiar . } esto deue dezir en su poridat a muy pocos \\\hline
3.3.16 & Amplius si post bellum \textbf{ et pugnam munitiones sunt obtinendae , } melius est hoc agere aestiuo tempore . & por fanbre e por mengua . \textbf{ Otrossi si por batalla o por lid son de ganar las fortalezas } meior es de acometerlas en este tienpo . \\\hline
3.3.17 & quod viae subterraneae \textbf{ semper sunt muniendae tabulis et aliis artificiis , } ne cadat terra & que las cueuas soterrañas \textbf{ deuen ser sotenidas de tablas | e de otros artifiçios } e de otros apoyamientos \\\hline
3.3.18 & et eleuans virgam machinae , \textbf{ ad quam coniuncta est funda , } qua lapides iaciuntur . & e leuante el pertegal del engennio \textbf{ al qual pertegal esta } atada la fonda en que enbia las piedras . \\\hline
3.3.20 & Nam in munitione fienda \textbf{ non solum est quaerenda bonitas situs , } et angularitas murorum , & Ca en la fortaleza que es de fazer \textbf{ non solamente es de catar la bondat do esta assentada } e que sean los muros fechos de esquinas . \\\hline
3.3.20 & circa munitionem illam , \textbf{ vel est aliunde terra apportanda , } et ponenda in illo spatio intermedio . & las quales carcauas son de fazer enderredor de la fortaleza e de los muros \textbf{ o ssi alli non ouiesse tierra deue se traer de otra parte } e deuesse echar entre el espaçio \\\hline
3.3.21 & nisi sciatur \textbf{ quomodo sunt muniendae , } ut non de facili vinci possint . & e commo deuen ser assentadas \textbf{ sil non sopieremos commo son de bastecer } por que non puedan de ligero ser tomadas . \\\hline
3.3.21 & Ne enim fame deuincantur , \textbf{ tria sunt attendenda , } videlicet ut frumenta , auena , ordeum , & e por que non sean tomadas nin vençidas por fanbre . \textbf{ tres cosas son de penssar e de proueer . } Lo primero que el trigo e el auena e el ordio \\\hline
3.3.21 & et plus durare perhibetur . \textbf{ Copia etiam carnium salitarum non est praetermittenda . } Salis etiam multitudo multum est & e el castiello deuenla basteçer mayormente de mijo . Ca el mijo menos se podresçe \textbf{ e mas dura que tedos los otros granos . } Avn basteçer de grand conplimiento de carnes saladas e de mucha sal . \\\hline
3.3.21 & commode abstinere , \textbf{ sunt occidendae , et comedendae , vel saliendae , } si esui aptae sunt : & las que pueden bien escusar \textbf{ son de matar para comer | o son de salar } si son conuenibles para comer . \\\hline
3.3.22 & quae de facili fodi potest : \textbf{ et tunc per profundas foueas est fortificanda munitio , } ne per cuniculos deuincatur . & que se puede de ligero cauar e estonçe es de \textbf{ enfortaleçer el castiello o la çibdat | afondando mucho las carcauas } por que non puedan passar \\\hline
3.3.22 & et est custodia negligenda . \textbf{ Immo inuestigandae sunt conditiones hostium : } ut quod palam habere non potuerunt , & mas deuen poner en ella guarda \textbf{ e ante deuen escudriñar las condiçiones de los enemigos | si son ydos o si non o si estan en çelada . } assi que lo que non pueden auer \\\hline
3.3.23 & ex eis fabricanda nauis : \textbf{ sed primo arbores sunt diuidendae per tabulas ; } et per aliquod tempus dimittendae , & luego de fazer la naue dellos . \textbf{ Mas primero los arboles deuen ser serrados | e partidos por tablas } e por algun tienpo son de dexar que esten \\\hline
3.3.23 & quae omnia sunt cum stupa conuoluenda . \textbf{ Haec enim vasa sic repleta sunt succendenda , } et proiicienda ad nauem hostium . & las quales cosas todas son de enboluer con estopa . \textbf{ Et estos belhezos tales | assi llenos son de ençender } e de lançar a las naues de los enemigos . \\\hline
3.3.23 & ut de calce alba puluerizata habeant multa vasa plena , \textbf{ quae ex alto sunt proiicienda in naues hostium , } quibus ex impetu proiectis , & Lo . viij̇° . es de tomar esta cautela en la batalla dela naue que conuiene que se fincan muchas cantaras de cal poluorizada . \textbf{ Et quando fuere la batalla | que lançen esta cal en la naue de los enemigos } e sean lançados de alto \\\hline
3.3.23 & et unus non iniuriatur alteri ; \textbf{ non sunt committenda bella . } Quare sicut per phlebotomiam , & e el vno non faze tuerto al otro non ay \textbf{ porque auer batalla ninguna . } Por la qual cosa assi commo por la sangria \\\hline

\end{tabular}
