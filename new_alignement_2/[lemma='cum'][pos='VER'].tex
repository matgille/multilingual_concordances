\begin{tabular}{|p{1cm}|p{6.5cm}|p{6.5cm}|}

\hline
1.1.1 & tangere Philosophus 1 Ethicorum , \textbf{ cum ait , } quod dicetur sufficienter de morali negocio , & eL segundo libro delas ethicas quando \textbf{ dizeque conplidamente se dize dela } moralph̃ia si fuer fecha manifestaçion \\\hline
1.1.7 & tum quia sunt diuitiae \textbf{ ex institutione Hominum , tum quia cum sint corporalia , } ipsi indigentiae corporali & Lo segundo que por que estas Riquezas son riquezas \textbf{ por ordenamiento e estableçemiento | delons omes } e non en otra gnisa¶ \\\hline
1.1.7 & ut patebit in 3 lib’ \textbf{ cum determinabitur de regimine Regni . } Nam Rex proprie est , & ca ay grant diferençia entre el Rey e tirano \textbf{ assy commo demostrͣemos enl terçero libro quando determinaremos del gouernamiento del regno } Ca el Rey es aquel que propiamente parara mientes al bien del regno e al bien comun¶ \\\hline
1.1.8 & Honor ergo habet rationem boni extrinseci , \textbf{ cum sit reuerentia exhibita } per quaedam exteriora signa . & que esta de fuera \textbf{ por que es reuerençia fecha } por señales mostradas de fuera¶ \\\hline
1.1.8 & Nam si Princeps suam felicitatem in honoribus ponat , \textbf{ cum sufficiat ad hoc } quod quis honoretur , & si pusiere la su bien andança en honrras \textbf{ conmoabaste acanda vno } para que sea honrrado \\\hline
1.1.10 & naturaliter agit , \textbf{ cum calefacit : } sic homo , qui est naturaliter liber arbitrio , & que es naturalmente caliente \textbf{ e sienpre escalienta assi el omne } por que es natraalmente franco \\\hline
1.1.10 & quia tale dominium \textbf{ cum sit violentum , } et contra naturam , & Mas es de poner en aquello \textbf{ que sienpre ha de durar ¶ } La segunda razon se declara \\\hline
1.1.10 & ait , turpe esse , \textbf{ cum bellamus , participare bonis , } cum vero vacamus et sumus in pace , fieri vitiosus . & Et dize assi que torpe cosa es \textbf{ que nos quando lidiamos o estamos en la batalla seamos bueons } Et despues quando fueremos en paz \\\hline
1.1.10 & cum bellamus , participare bonis , \textbf{ cum vero vacamus et sumus in pace , fieri vitiosus . } Quare si inconueniens est & que nos quando lidiamos o estamos en la batalla seamos bueons \textbf{ Et despues quando fueremos en paz | que seamos uiçiosos e malos . } por las quales cosas ya dichas \\\hline
1.1.11 & Aequatio enim humorum , \textbf{ cum subsunt motui supercoelestium corporum , } variationi aeris , & e se puerden perder de ligero . \textbf{ Ca la egualança de los humores | en que esta la salut } commo sea subiecta al \\\hline
1.1.13 & si non transgrediantur , \textbf{ cum possint transgredi , } maioris meriti esse videntur . & si non trispassaren los mandamientos de dios \textbf{ conmolos podiessen } trispassar son de mayor meresçimiento . \\\hline
1.2.2 & remanet indiuisus . \textbf{ Nam cum intellectus uniuersaliori modo } respiciat suum obiectum quam sensus , & que es dich̃o uoluntad non es departido en ningunas partes . \textbf{ Ca commo el entendimiento sea mas general que el seso . } Mas generalmente cata aquello en que ha de obrar que el seso . \\\hline
1.2.3 & Numerus autem earum sic potest accipi . \textbf{ Nam cum subiectum virtutis sit , } vel intellectus , vel voluntas , & assi se puede tomar . \textbf{ Ca commo el subiecto delas uirtudes sea o el entendimiento o la uoluntad o el appetito senssitiuo . } toda uirtud moral o es en el entendimiento \\\hline
1.2.3 & sunt in concupiscibili . \textbf{ Patet ergo quod cum quatuor potentiae animae sint } susceptibiles virtutum de quibus loquimur , & guaue son en el appetito desseador ¶ \textbf{ Pues que assi es paresçe | que commo sean quatro poderios del alma } que pueden resçebir las uirtudes \\\hline
1.2.5 & per quam de ipsis agibilibus rectas rationes faciamus . \textbf{ Rursus cum contingat operari recte et non recte , } sic ut est dare virtutem , & que fazemos fagamos razones derechas ¶ \textbf{ Otrosi commo contesca de obrar derechamente | e non derechamente } assi commo auemos a dar uirtud . \\\hline
1.2.5 & principalior omnibus aliis , \textbf{ cum sit directiua omnium aliarum , } et iustitia sit principalior & ¶Otrosi por que la prudençia es mas prinçipal que todas las otras \textbf{ por que es endereçadora e regladora de todas las otras ¶ } Et la iustiçia en pos ella es mas prinçipal que la fortaleza e la tenperança . \\\hline
1.2.6 & sed per prudentiam \textbf{ ( cum habemus ) } dirigimur in fines illos . & Mas por la pradençia somos reglados \textbf{ en qual manera pondemos alcançar aquellas fines . } Mas la pradençia toma aquellas fines delas uirtudes morałs \\\hline
1.2.7 & vel plumbeus positus in computo mercatorum , \textbf{ Mercatores enim cum ratiocinando computant , } aliquando unum denarium aeneum & puesto en el cuento delos mercadores . \textbf{ Ca los mercandores su mando | e razoñado cuentan algunas vezes } vn dinero de cobre de plomo \\\hline
1.2.12 & et ut obseruent Iustitiam : \textbf{ cum sine ea ciuitates , } et regna durare non possint . & e que guarden iustiçia \textbf{ sin la qual las çibdades e los regnos non pueden durar } Empero por que el coraçon del noble ome \\\hline
1.2.12 & quod unumquodque perfectum est , \textbf{ cum potest sibi simile producere , } et cum actio sua ad alios se extendit : & que cada vna cosa es acabada \textbf{ enssi quando puede fazer otra tal commo si . } Et quando la su obra se estiende alos otros \\\hline
1.2.13 & aggrediuntur bellorum pericula : \textbf{ sed , cum aggressi sunt ea , } si inuenerint resistentiam , & Ca alguonsligeramente acomneten los periglos delas faziendas e delas batallas . \textbf{ Mas quando los han acometidos } si fallan fortaleza \\\hline
1.2.13 & Quidam autem non de leui aggrediuntur bellum , \textbf{ sed cum aggressi fuerint , } sustinent , & Et otrosi hay otros que non deligero acometen la fazienda e la batalla . \textbf{ mas quando la han acometida estan } e susten los periglos dela batalla e dela fazienda . \\\hline
1.2.13 & Non enim sic aperte apprehendimus mortem , \textbf{ cum aegrotamur : } quia aegritudines intrinsecus latent ; & manifiestamente la muerte quando enfermamos \textbf{ por que las enfermedades yazen ascondidamente de dentro nin sentimos . | los periglos dela mar } commo sentimos delas batallas \\\hline
1.2.15 & cum cibus aut potus attingit guttur , \textbf{ quam cum coniungitur linguae . } Credendum est enim in talibus iudicio gulosorum . & e el beuer llega ala garganta \textbf{ que quando llega ala lengua . } Et estas cosas deuemos creer al iuizio de los golosos . \\\hline
1.2.15 & nomine Phyloxenus , \textbf{ qui , cum esset pultiuorax , orauit , } ut guttur eius longius quam gruis fieret . & e tomaua muy grant delectaçion en ellas g̃rago a dios \textbf{ quel feziese la garganta } mas luenga que garganta de grulla \\\hline
1.2.16 & in Rege Sardanapallo , \textbf{ qui cum esset totus muliebris , } et deditus intemperantiae ( ut recitat Iustinus Historicus , libro 1 abbreuiationis Trogi Pompeii ) & enxienplo en el Rey sardan \textbf{ apalo que por que era todo mugeril | e dado a mugers } e era muy \\\hline
1.2.18 & ( quia senes auariores sunt iuuenibus ) \textbf{ cum veniunt ad senectutem , } ut plurimum curatur eorum prodigalitas , & Et por ende son mas auarientos \textbf{ que quando eran mançebos } e assi los mas dellos \\\hline
1.2.18 & de leui \textbf{ quis cum sit prodigus , } fieri poterit liberalis . & desbien commo el libal non lo es de ligero se puede fazer \textbf{ qual quier gastador liberal e franco ¶ } pues que assi es si es conueinble al Rey \\\hline
1.2.23 & ut esse veridicos ; \textbf{ cum sint regula aliorum , } quae obliquari , & de seer manifiestos e claros e seer uerdaderos \textbf{ por que son regla de los otros } La qual regla non se deue torcer nin falssar \\\hline
1.2.26 & ut supra diffusius dicebatur . \textbf{ Quare cum distincta sit virtus haec ab illa , } videndum est & assi commo dicho es de ssuso mas conplidamente \textbf{ por la qual cosa commo esta uirtud | que es dicha humildança sea apartada dela magnanimidat } conuienenos de veer \\\hline
1.2.26 & Est enim hoc notabiliter attendendum , \textbf{ quod cum virtus magis sit retrahens quam impellens , } principaliter opponitur superabundantiae , & que quando la uirtud . \textbf{ mas nos trahe e tira | que nos allega e esfuerca . } Estonçe prinçipalmente \\\hline
1.2.29 & in cap’ praetacto , \textbf{ cum ait , } quod prudentis est declinare in minus . & que dicha es \textbf{ quando dize } que pertenesçe al sabio de declinar alo menos . \\\hline
1.2.31 & nisi sit prudens . \textbf{ Nam cum virtus moralis sit } habitus bonus , & si non fuere prudente e sabio \textbf{ Ca commo la uirtud moral sea habito e disposiçion firme de alma e buena escogedora } e acaba a aquel que la ha \\\hline
1.2.32 & Tales ergo nihil difficile sustinere volentes , \textbf{ statim cum passionantur , } vel cum tentantur , cadunt . & Et por ende estos tales non quariendo sofrir ninguna cosa guaue \textbf{ luego que padelçen o son passionados por alguna passion } e quando son tentados \\\hline
1.2.32 & statim cum passionantur , \textbf{ vel cum tentantur , cadunt . } In alio gradu dicuntur & luego que padelçen o son passionados por alguna passion \textbf{ e quando son tentados | luego } ca en estos tales son dichͣs muelles . \\\hline
1.3.1 & passionem oppositam irae . \textbf{ Sed cum sit quaedam virtus } inter iram et mansuetudinem , & Ca la manssedunbre nonbra propriamente passion contraria ala saña . \textbf{ Mas por que ha de ser alguna uirtud entre la sanna e la mansedunbre } la qual uirtud non podemos nonbrar \\\hline
1.3.1 & inuenire nomen proprium . \textbf{ Sed cum constat de re , } de verbis minime est curandum . & non bͤapio a cada vna cosa podia lo fazer \textbf{ mas quando nos somos çiertos dela cosa non deuemos auer cuydado delas palauras . } Et pues que assi es contadas las passiones \\\hline
1.3.2 & Abominatio uero immediate innititur odio : \textbf{ quia statim cum odimus , } aliquid abominamur illud . & Mas la aborrençia sin ningun medio se ayunta ala mal querençia \textbf{ por que luego commo queremos mal a alguna persona . } luego la aborresçemos . \\\hline
1.3.2 & Nam spes , et desperatio , \textbf{ cum sumantur respectu boni , } praecedunt timorem , et audaciam , iram , et mansuetudinem , & Maen el tercero logar son de poner la esperançar la desesꝑança . \textbf{ Ca por que son tomadas } por razon de bien son puestas primero que el temor e la oladia \\\hline
1.3.3 & Unde et Valerius Maximus de Dionysio Ciciliano recitat , \textbf{ qui cum esset tyrannus , } erat amator proprii commodi , & Onde ualerio maximo cuenta de dionsio seziliano \textbf{ que commo fuesse tyrano } e amador de propio prouecho despoblaua \\\hline
1.3.5 & in tertio libro diffusius ostendetur . \textbf{ Cum determinauimus de ordine passionum animae , } diximus quod amor et odium erant passiones primae , & en el terçero libro lo mostraremos mas conplidamente \textbf{ uando determinamos dela ança orden delas passiones del alma dixiemos } que el amor e la malqreçia eran las primeras passiones \\\hline
1.3.7 & Statim enim , \textbf{ cum scimus aliquem esse malum , } ut cum scimus aliquem esse furem , & que parte nesçen asi mismo o a otro . \textbf{ Por que luego quando sabemos } que alguno es ladron podemos le mal querer \\\hline
1.3.7 & cum scimus aliquem esse malum , \textbf{ ut cum scimus aliquem esse furem , } possumus ipsum odire , & Por que luego quando sabemos \textbf{ que alguno es ladron podemos le mal querer } si quiera aya fecho mal a nos o a \\\hline
1.3.7 & ei sed odium pro nullo miserebitur , \textbf{ cum sit quid insatiabile . } Octaua differentia est : & Mas la mal querençia de ninguno non se apiada por que es cosa \textbf{ que se non farta . } ¶ La octaua diferençia es \\\hline
1.3.7 & Serui enim veloces statim \textbf{ cum audiunt verbum Domini , } antequam plene percipiant praeceptum eius , & Ca los sieruos ligeros \textbf{ luego commo oyen la palaura del sennor } ante que entiendan conplidamente el mandamiento del corren \\\hline
1.3.7 & Sic etiam et canes statim \textbf{ cum audiunt sonitum venientis , } latrant , non distinguentes , & En essa misma manera avn los canes \textbf{ luego que oy en el sueno | de aquel que viene } luego ladran \\\hline
1.3.8 & minus grauamur : \textbf{ sic cum videmus multitudinem amicorum } condolore nobis , & quando muchos nos ayudan a leuar aquel peso menos nos aguauiamos del peso . \textbf{ en essa misma manera quando ueemos muchedunbre de amigos que se duelen } connusco o se duelen de nos \\\hline
1.3.8 & Possumus ergo dicere \textbf{ quod cum videmus eos dolere } de dolore nostro , & Et pues que assi es podemos dezir \textbf{ que quando nos veemos los amigos doler se del nuestro dolor } non es menguado el nuestro dolor \\\hline
1.3.9 & maxime bonum arduum , \textbf{ cum est futurum et speratur : } et malum arduum , & por muy bueno e muy alto \textbf{ quando es futuro e ha de uenir e es esparado . } Et el mal es alto e grande \\\hline
1.3.9 & et malum arduum , \textbf{ cum est futurum et timetur : } spes et timor sunt principales passiones respectu irascibilis . & Et el mal es alto e grande \textbf{ quando es futuro que es de uenir e es tenido . } Et por ende la esperança e el temor son passiones prinçipales \\\hline
1.3.10 & Si sunt corporalia , \textbf{ quia talia cum habentur ab uno , } non habentur ab alio , & o son spun al ssi son corporales \textbf{ por que tales quando son auidas de vno non son auidas de otro . } Por ende se suele difinir \\\hline
1.4.1 & Iuuenes ergo , \textbf{ cum sint liberales , et cum sint animosi et bonae spei , } non habent unde retrahantur & e entremetesse de fazer grandes cosas . \textbf{ Et pues que assi es commo los mancebos } non ayan ninguna cosa \\\hline
1.4.3 & Verecundia ergo , \textbf{ cum sit timor inhonorationis , } non competit senibus ; & en el segundo libro dela Rectorica \textbf{ Ca por que la uerguença es temor de desonera non pertenesçe alos uieios } por que may orcuidado han del prouecho \\\hline
1.4.4 & sed quia quilibet \textbf{ cum est in imbecillitate , } vel in defectu , & o por que ellos sean amadores de amistanças . \textbf{ Mas porque cada vno quando es en flaqueza } e en fallestimiento \\\hline
1.4.6 & Recitat enim Philosophus 2 Rhetoricorum , \textbf{ quod cum quaesitum fuisset } a muliere quadam , & en el segundo libro de la rectorica \textbf{ que fue demandado a vna muger qual cosa era meior ser rico o ser sabio . } Et ella respondio \\\hline
1.4.7 & retrahitur ab ocio : \textbf{ et cum dat se uni operi , } retrahitur ab alio . & tyrase de ocçi olidat \textbf{ e quando se da a vna obra } tyrase dela otra . \\\hline
2.1.1 & esse desinerent ; \textbf{ quare cum viuere sit homini naturale , } omnia illa , & si ansi luego que son fechͣs començassen afallesçer . \textbf{ Por la qual cosa commo el beuir sean | atuer tal cosa al omne } todas aquellas cosas \\\hline
2.1.1 & magis habet offendi , quam illa ; \textbf{ quare cum habere victum et vestitum congruat } ad vitam humanam , & ala uida humanal de auer uianda \textbf{ e vestida conuenible } e ninguno non abaste assi mismo sin conpannia de otro \\\hline
2.1.1 & ideo statim \textbf{ cum audiunt strepitum , } fugam arripiunt . & si non por lignieza del su cuerpo e por foyr . \textbf{ Et por ende luego que oy en algun roydo } luego comiençana foyr . \\\hline
2.1.1 & si nunquam vidisset canes alias peperisse . \textbf{ Mulier autem cum parit , nescit qualiter se debeat habere in partu , } nisi per obstetrices sit sufficienter edocta . & avn que nunca uiesse o trisperras parir . \textbf{ Mas la mug̃r quan do pare non labe | en qual manera se deua auer en el parto } si non fuere enssennada conuenibłmente por las parteras . \\\hline
2.1.3 & aliquid primo operatum , \textbf{ cum adepto fine cesset operatio , } nunquam operaremur ea , & por que si la fin fuesse primeramente alguna cosa obrada \textbf{ quando ouiessemos ganada la | finçessarie la obra } e nunca obrariemos nada de aquellas cosas \\\hline
2.1.4 & qualis sit communitas domus : \textbf{ cum ostensum sit } quod homo est naturaliter animal domesticum , & sobredicho qual es la comunidat dela casa . \textbf{ Ca ya mostrado es } que el omne es naturalmente ainalia domestica e de casa \\\hline
2.1.5 & quia haec illam praesupponit . \textbf{ Nam cum generata non possint conseruari in esse } nisi prius per generationem acceperint esse , & e antepone la generaçion . \textbf{ Ca commo las cosas engendradas | non pueden ser conseruadas } nin guardadas en su ser si primeramente non resçibieren el su ser por generaçion . \\\hline
2.1.5 & Sicut enim caecus corporaliter , \textbf{ nisi ( cum pergit ) dirigatur ab aliquo , } de leui obuiat alicui offensiuo : & que si alguno es çiego corporalmente \textbf{ si quando anda non fue regua ado } por alguno otro de ligero \\\hline
2.1.6 & quia non statim \textbf{ cum est res naturalis , } potest sibi simile producere , & assi conpado alas cosas natraales \textbf{ por que non puede la cosa natural } luego que es fecha fazer otra semeiante \\\hline
2.1.6 & statim enim , \textbf{ cum natus est homo , } solicitatur natura circa conseruationem ipsius : & enssi \textbf{ ca luego quando nasçe el omne } la natura es \\\hline
2.1.6 & Tunc unumquodque perfectum est , \textbf{ cum potest sibi simile producere . } Ad hoc enim quod aliquid sit perfectum , & estonçe toda cosa es acabada \textbf{ quan do puede fazer | e engendrar su semeiante } Ca para ser la cosa acabada \\\hline
2.1.6 & Impotens autem ad agendum dicitur aliquid , \textbf{ cum praesente proprio passiuo , } non producat sibi simile ; & que non es poderoso de obrar \textbf{ quando tiene su materia proprea presente en que puede obrar } e non puede fazer su semeiante . \\\hline
2.1.9 & se habent masculus et foemina tempore partus . \textbf{ Sed cum dictum sit , } quod toto tempore partus in huiusmodi auibus masculus & en el tienpo del parto . \textbf{ Mas commo dicho es } que en todo el tp̃o del parto \\\hline
2.1.12 & homines libenter iniustificant , \textbf{ cum possunt . } Qui ergo caret ciuili potentia & Ca segunt el philosofo en la rectorica los omes de buenamente fazen iniustiçias e tuertos \textbf{ quando pue den . } Et pues que assi es aquel \\\hline
2.1.13 & secundum modum eis congruum foeminas pollere deceat , \textbf{ tamen cum tradenda est aliqua nuptui , } potissime inquirendum est , & segunt la manera que les conuiene . \textbf{ Enpero quando la fenbra es de dar a algun marido mayormente deuemos tener } mientessi resplandesçe por tenprança \\\hline
2.1.14 & Dicitur autem quis praeesse regali dominio , \textbf{ cum praeest secundum arbitrium et secundum leges , } quas ipse instistuit . & Mas alguon es dicho ser adelantado en sennorio real \textbf{ quando es adelantado segunt aluedrio | e segunt las leyes } que el mismo establesçio \\\hline
2.1.14 & et dicitur regale . \textbf{ Sed cum leges non instituuntur } a principante sed a ciuibus , & e es dicho gouernamiento real . \textbf{ as quando las leyes non son establesçidas } por el prinçipe \\\hline
2.1.14 & quibus vacare debeant \textbf{ cum sint adulti : } ad quae non sunt instruendae uxores , & e alas obras çiuiles alas quales deuen entender \textbf{ quando fueren criados } e mayores alas quales cosas non son de enssennar las mugers \\\hline
2.1.15 & nisi careat usu rationis et intellectus . \textbf{ Sed cum carens rationis usu sit naturaliter seruus , } quia nescit seipsum dirigere , & si non fuese priuado de vso de razon e de entendimiento . \textbf{ Mas commo aquel que es priuado de vso de razon e de entendemiento sea naturalmente sieruo } por que non sabe gniar assi mismo \\\hline
2.1.18 & ideo statim miserentur , \textbf{ cum vident aliquos dura pati . } Tertio considerandum est in mulieribus , & e por ende luego que veen a algunos \textbf{ sofrir cosas duras han piadat sobre ellos ¶ } Lo terçero deuemos penssar en las mugers \\\hline
2.1.18 & quia communiter nimis excedunt . \textbf{ Unde cum sunt piae , } sunt valde piae : & e sobrepuian entondo . \textbf{ Onde quando son piadosas son muy piadosas . } Et quando son crueles son muy crueles \\\hline
2.1.18 & sunt valde piae : \textbf{ et cum sunt crudeles , } sunt valde crudeles : & Onde quando son piadosas son muy piadosas . \textbf{ Et quando son crueles son muy crueles } Et quando son desuergonçedas son muy sin uerguença . \\\hline
2.1.18 & sunt valde crudeles : \textbf{ et cum sunt inuerecundae , } sunt nimis inuerecundae . Postquam enim mulieres audaciam capiunt , & Et quando son crueles son muy crueles \textbf{ Et quando son desuergonçedas son muy sin uerguença . } Ca del pues que las mugers toman osadia \\\hline
2.1.18 & Primo ergo foeminae , \textbf{ cum possunt , } ut plurimum sunt intemperatae , & Pues que assi es lo primero las mugers \textbf{ quando pueden } por la mayor parte son destenpdas \\\hline
2.1.18 & ex verecundia quam ex ratione . \textbf{ Quare cum motae sunt , } nesciunt se moderare , & esto fazen mas por uerguença que por razon . \textbf{ Por la qual cosa quando se mueuen non } sabenertenprar assy mesmas . \\\hline
2.1.19 & Unde et aliquos Philosophos legimus sic fecisse , \textbf{ qui cum essent impeditae linguae , } accipientes specialem conatum & Ende leemos que algunos philosofos lo fizieron \textbf{ assi los quales commo ouiessen las lenguas enbargadas } tomaron especial esfuerço çerca aquellas letras \\\hline
2.1.20 & redundat in persona ipsius viri . \textbf{ Immo cum ostensum sit supra uxorem } non se habere & tornase en la perssona del marido . \textbf{ Mas por que fue mostrado de suso que la mugni non se deue auer al marido } assi commo sierua mas assi commo conpanera . \\\hline
2.1.22 & quod in domo consurgit . \textbf{ Nam cum uidetur uxoribus } quod sine causa calumnientur , et quod earum uiri sine culpa suspicentur de ipsis mala , & que se leunata en la casa . \textbf{ Ca quando veen las mugers } que sus maridos se acallonan sin razon \\\hline
2.1.23 & elegibilius esset consilium muliebre quam virile . \textbf{ Natura enim cum moueatur ab intelligentiis , et a Deo , } in quo est suprema prudentia ; & e la razon es esta \textbf{ por que la natura toda es mouida delos angeles | e de dios } en que es conplimiento de sabiduria . Et por ende conuiene que aquellas cosas \\\hline
2.1.24 & ut operemur illud . \textbf{ Quare cum ponere aliquid in praecepto , } sit prohibere , & para obrar aquella cosa . \textbf{ Por la qual razon commo poner alguna cosa } en poridat se a uedar \\\hline
2.1.24 & si sciant secreta ipsorum ; \textbf{ cum conglorientur , } si possint se laudari & por que parescan ser amadas de sus maridos \textbf{ si sopieren las poridades dellos . } Et por ende se glorian \\\hline
2.1.24 & reuelare secreta . \textbf{ Nam cum dicimus hos esse mores iuuenum , } hos mulierum , hos senum . & en qual manera los maridos de una descobrir a sus mugieres los sus secretos . \textbf{ Ca quando nos dezimos | que estas son las costunbres de los mançebos } e estas las de los uieios \\\hline
2.1.24 & His visis , \textbf{ cum ostensum sit , } quomodo communitas viri et uxoris sit naturalis , & ¶ vistas estas cosas \textbf{ commo seaya prouado } que la conpannia del uaron \\\hline
2.2.3 & et propter bonum ipsorum : \textbf{ cum amare aliquod , } idem sit quod velle ei bonum , & enssennorear alos fiios realmente \textbf{ e por el bien dollos commo amar a alguno sea esso mismo } que querer qual bien . \\\hline
2.2.4 & Immo filii , \textbf{ cum possunt , furantur , } et rapiunt bona parentum . & Mas los fijos non allegan para los padres . \textbf{ Mas los fijos quando pueden furtan } e cobdician los bienes de los padres \\\hline
2.2.4 & quam econuerso ; \textbf{ cum diligere aliquem , } idem sit quod velle ei bonum , & que los fijos alos padres \textbf{ commo amara alguno sea essa misma cosa } que querer bien \\\hline
2.2.4 & in honore et reuerentia : \textbf{ cum honorari et reuereri alium sit } quodammodo subiici illi ; & Commo honrrar \textbf{ e auer reuerençia a otro sea en alguna manera ser subiecto a el . } Por ende assi commo por el amor que han los padres alos fijos los deuen gouernar ben \\\hline
2.2.5 & Nam ut in primo libro diximus \textbf{ cum tractauimus de moribus iuuenum , } Iuuenes sunt simpliciter creditiui : & Ca assi commo dixiemos en el primero libro \textbf{ quando tractauamos delas costunbres de los mançebos } los mançebos son sinples en creer . \\\hline
2.2.6 & nam et pueri statim delectantur , \textbf{ cum incipiunt suggere mammas . } Si ergo sic ab ipsa infantia nobiscum & assi que los moços luego se delectan \textbf{ e quaeçentes | si que los mocos luero de mamar . } Et pues que assi es assi commo dela moçedat cresçe \\\hline
2.2.7 & et Principum \textbf{ cum ponuntur in aliquo dominio tyrannizent , } decet ipsos etiam ab ipsa infantia insudare literis , & e de los prinçipes \textbf{ quando son puestos en algun sennorio non tiraniçen | nin sean tirannos } Conuiene les avn de trabaiar \\\hline
2.2.11 & sed maior est , \textbf{ cum attingit guttur . } Gulosi ergo , & mas mayor delectaçiones \textbf{ quando la uianda llega ala garganta . } Et por ende los golosos que con grand cobdiçia \\\hline
2.2.12 & et quae sunt consideranda in uxore ducenda : \textbf{ supra , cum egimus de regimine coniugali , diffusius diximus . } Ostenso , & que deuen cuydar en las mugers \textbf{ que deuen tomar de suso lo dixiemos mas largamente | quando dixiemos del gouernamiento del casamiento } ostrado en qual manera los as . moços deuen ser guardados enla vianda \\\hline
2.2.13 & Sicut ergo habent indisciplinatos gestus , \textbf{ qui cum volunt audire alios , } tenent ora aperta : & Et pues que assi es assi commo aquellos \textbf{ que quieren oyr alos otros } e tienen las bocas abiertas \\\hline
2.2.13 & sic sunt indisciplinati secundum gestus , \textbf{ qui cum volunt loqui , } extendunt pedes et crura , & desenssennandos en los gestos . \textbf{ En essa misma manera son desenssennados segunt los gestos | aquellos que quando que eren fablar estienden los pies } e las prinas o mueuen los braços \\\hline
2.2.15 & et hoc maxime , \textbf{ cum incipiunt percipere significationes verborum . } Sextum , a ploratu sunt cohibendi . & Et esto les es prouechoso mayormente \textbf{ quando comiençan a entender las significa connes delas palabras . } ¶ La sexta es que deuen ser guardados de llorar . \\\hline
2.2.15 & Obseruandum est tamen in iuuenibus \textbf{ cum aluntur lacte , } quod si contingat eos suggere aliud lac quam maternum , & Enpero deuemos guardar \textbf{ que quando los mocos maman la leche | si contesçe } que ayan de mamar otra leche \\\hline
2.2.15 & Attendendum est tamen , \textbf{ quod cum dicimus pueros paruos } assuescendos esse & Enpero deuemos entender \textbf{ que quando dezimos | que los moços pequanos son de acostunbrara esto } o aquello deue se entender tenpradamente \\\hline
2.2.15 & ut retineant spiritum et anhelitum . \textbf{ Nam sicut cum plorare permittuntur , } emittunt spiritum et anhelitum : & que retengan en ssi el spun e el eneldo . \textbf{ Ca assy commo quando los dexan llorar } enbian el spuer e el eneldo . \\\hline
2.2.15 & emittunt spiritum et anhelitum : \textbf{ sic cum plorare cohibentur , } spiritum et anhelitum tenent . & enbian el spuer e el eneldo . \textbf{ assi quando les defienden | que non lloren } retienen en ssi el spun e el eneldo \\\hline
2.2.16 & sunt a ploratu illo cohibendi . \textbf{ Cum distinguimus aetates filiorum per septennia , } ut cum dicimus , & ¶ \textbf{ ommo nos ayamos departido las hedades de los fijos | por setenarios quando dixiemos que fastaliente a nons } assi deuian ser gouernados los moços . \\\hline
2.2.16 & Cum distinguimus aetates filiorum per septennia , \textbf{ ut cum dicimus , } usque ad septem annos sic esse regendos : & por setenarios quando dixiemos que fastaliente a nons \textbf{ assi deuian ser gouernados los moços . } Et de los siete años \\\hline
2.2.16 & ad aliquos motus . \textbf{ Sed cum impleuerunt septennium } usque ad annum decimum quartum , & a algunos mouimientos conuenibles . \textbf{ Mas quando ouieren conplido el vii̊ año fasta el xiiij̊ . } deuen se acostun brar poco a poco \\\hline
2.2.16 & et omnia faciunt valde , \textbf{ ita quod cum amant nimis amant , } cum incipiunt ludere nimis ludunt , & que fazen fazen las mucho \textbf{ mas que deuen asi que quando aman am̃a much̃ . Etrͣndo } comiençan de trebeiar trebeian much̃ . \\\hline
2.2.16 & ita quod cum amant nimis amant , \textbf{ cum incipiunt ludere nimis ludunt , } et in caeteris aliis semper excessum faciunt , & mas que deuen asi que quando aman am̃a much̃ . Etrͣndo \textbf{ comiençan de trebeiar trebeian much̃ . } Et assi que todas las cosas que fazen \\\hline
2.2.16 & perfecte scire non possunt . \textbf{ Ne tamen cum incipiunt habere rationis usum , } omnino sint indispositi ad scientiam , & fallesçe de vso de razon non pueden saber las sçiençias acabada mente . \textbf{ Enpero por que quando comiençan a auer } vso de razon non seanda todo mal apareiados ala sçiençia deuen ser acostunbrados alas otras artes delas \\\hline
2.2.19 & Sed hoc breui tractatu indiget : \textbf{ quia cum determinauimus de regimine coniugali , } et ostendimus qualiter regendae sunt foeminae ; & mas esto ha menester muy pequano tractado \textbf{ ca quando dixiemos | e determinamos del gouer namiento del casamiento } e mostramos en qual manera son de gouernar las muger s casadas \\\hline
2.2.19 & ut plurimum male faciant , \textbf{ cum possunt . } Maxima ergo cautela & que los omes en la mayor parte fazen mal \textbf{ quando pueden . } Et pues que assi es muy grant cautela es de poter \\\hline
2.2.20 & quod superius dimisimus , \textbf{ cum tractauimus de regimine coniugali . } Dicebatur enim , & Et por esto que dixiemos es declarado lo que dixiemos de ssuso \textbf{ quando rͣctauamos del gouernamiento del } casamien toca y dixiemos que adelante serie de declarar cerca quales obras conuenia \\\hline
2.3.9 & et talia quibus indigemus ad vitam , \textbf{ cum sint magni ponderis , } commode ad partes longinquas portari non possunt . & que auemos menester para la uida \textbf{ por que son de grand peso } non las poderemos leuar conueniblemente a luengas tierras . \\\hline
2.3.12 & qui primo philosophari coeperunt . \textbf{ Ipse enim cum esset pauper , } et improperaretur sibi a multis cur philosopharetur , & que primeramente comneçara a philosofo far . \textbf{ Este mille sio commo fuesse muy pobre } e le denostassen sus amigos \\\hline
2.3.14 & si scirent se ex eis nullam utilitatem consecuturos ; \textbf{ sed cum cogitant eos acquirere in seruos , } reseruant ipsos propter utilitatem & soperiessen que nigunt pro non aurian de tal uençimiento . \textbf{ Mas quando pienssan que aquellos a quien vençe } que los gana \\\hline
3.1.2 & nisi viuant bene et virtuose , \textbf{ cum sine lege et iustitia } constituta ciuitas stare non posset , & si non biuiessen bien \textbf{ e uirtuosamente commo la çibdat establesçida sini ley } e sin iustiçia non pueda estar \\\hline
3.1.4 & Canis enim eo quod latrat , \textbf{ aliter latrat cum delectatur : } et cum tristatur potest & por que ladra en otra manera \textbf{ quando se delecta | e en otra manera } quando se trista puede a otro can de mostrar \\\hline
3.1.4 & aliter latrat cum delectatur : \textbf{ et cum tristatur potest } alteri cani per suum latratum & e en otra manera \textbf{ quando se trista puede a otro can de mostrar } por su ladrado sutsteza o su delecta conn que ha mas al omne \\\hline
3.1.5 & Prima via sic patet . \textbf{ Nam cum ait Philosophus primo Polit’ } quod communitas perfecta , & La primera paresçe \textbf{ assi ca segunt que dize el philosofo | en el primero libro delas politicas } que la comunidat acabada \\\hline
3.1.5 & et bonum statum aliorum ciuium . \textbf{ Quare cum peruersi in ciuitate aliqua } non audeant insurgere contra principem , & e el bue estado de los otros çibdadanos \textbf{ por la qual cosa commo los malos en alguna çibdat } non se osenle unatat \\\hline
3.1.7 & et maxima coniunctio in ciuitate . \textbf{ Nam cum sit maxima unitas , } et maxima coniunctio patrum ad filios , & e muy grant ayuntamiento enla çibdat . \textbf{ Ca commo sea muy grant vnidat } e grant ayuntamiento de los padres alos fijos los mas antiguos \\\hline
3.1.11 & tanto magis ad inuicem conuersantur : \textbf{ sed cum esse non possit , } aliquos valde & mas han de beuir en vno \textbf{ mas commo non pueda ser } que alguons \\\hline
3.1.11 & tamen propter liberalitatem quantum ad usum erant illis ciuibus communes serui , et equi , et canes : \textbf{ quilibet enim ciuium cum indigebat , } absque alia requisitione utebatur alterius equis , canibus , et seruis . & assi commo los cauallos e los sieruos e los canes \textbf{ ca cada vno de los çibdadanos | quando auia meester alguna cosa sin la demandar al otro } vsauad los cauallos \\\hline
3.1.12 & ex parte fortitudinis corporalis . \textbf{ Nam cum bellantes oporteat } diu & que son temerosos ¶ \textbf{ La terçera razon se toma de parte dela fortaleza corporal } ca commo los lidiadores ayan de sofrir el peso delas armas \\\hline
3.1.13 & vel ad aliquem magistratum assumitur . \textbf{ Quare cum deceat regia maiestatem } et uniuersaliter omnem ciuem , & commo se conosçe despues que esle un atada en alguna dignidat o en algun maestradgo o en algun poderio \textbf{ por la qual razon commo venga ala real magestad } e generalmente a qual quier que ha de dar \\\hline
3.1.13 & tangit Philosophus 2 Poli’ \textbf{ cum ait . } Socrates semper facit eosdem Principes , & en el segundo libro delas politicas \textbf{ quando } dizeque socrates \\\hline
3.1.14 & quid circa huiusmodi regimen sit censendum . \textbf{ Quare cum patefactum sit in praecedentibus , } non expedire ciuitati possessiones , & que auemos de iudgar en este gouernamiento \textbf{ por la quel cosa commo sea manifiesto | por las cosas dichͣs de suso } que non conuiene ala çibdat \\\hline
3.1.14 & nisi bellare , \textbf{ cum adesset oportunitas : } et onerosius et quasi omnino importabile esset sustentare sic quinque milia : & si non lidiar \textbf{ quando fuesse me este } Et muy mayor carga e peor de sofrir serie \\\hline
3.1.15 & et sit solicitus \textbf{ ( cum adest facultas ) } de rebus aliorum , & e sea cuydados \textbf{ o quanto pudiere delas cosas de los otros } assi commo si fuessen suyas . \\\hline
3.1.15 & sic etiam saluare possumus dictum eius quantum ad unitatem ciuitatis . \textbf{ Nam cum dixit ciuitatem debere esse maxime unam , } forte non intellexit de unitate habitationis , & del quanto ala vnidat dela çibdat \textbf{ ca quando dixo | que deuia ser la çibdat much vna } por auentura non entendio de vnidat dela morada \\\hline
3.1.17 & et iurgia in ciuitate . \textbf{ Primo quia pauperes cum ditantur nesciunt fortunas ferre , } ut plane ostendit & contesçerien iniurias e tuertos e uaraias en la çibdat . \textbf{ Lo primero quando los pobres se fazen ricos | non saben sofrir la su buena uentura } assi commo el philosofo muestra llanamente en el segundo libro de la rectoriça \\\hline
3.1.20 & non poterat \textbf{ cum statuto de electione principis . } Si enim ciuitas & non puede estar \textbf{ con el establesçimiento dela elecçion | oł prinçipe } por que si la çibdat \\\hline
3.1.20 & et homines libenter iniustificant \textbf{ cum possunt ; } hoc posito artifices , & sienpre quieren ser senors en las çibdades \textbf{ e los omes de grado fazen tuerto quando puede . } Puesto que el dizie siguese que los menestrales \\\hline
3.1.20 & stare non potest \textbf{ cum statuto de bellatoribus , } ut quod ipsi sint potentiores aliis , & non puede estar con el \textbf{ establesçimiento que fizo de los lidiadores } que ellos fuessen mas poderosos que los otros \\\hline
3.2.2 & unus rectus , \textbf{ ut cum dominantur aliqui , } quia sunt virtuosi et intendentes commune bonum : & se leuna tan del sennorio de po cos vno derecho \textbf{ assi commo quando enslennorean alguons } que son uirtuosos \\\hline
3.2.2 & et alius peruersus , \textbf{ ut cum dominantur aliqui , } non quia sunt boni , & e entienden enł bien comun . \textbf{ Et otro malo assi commo quando enssenore an alguons } non por que son bueons \\\hline
3.2.6 & quia materia tunc ignitur \textbf{ cum calefit et rarefit , } oportet raritatem et calorem perfectius reperiri & e por ca lentura . \textbf{ por que la materia estonçe es puesta e tornada en fuego } quando es muy escalentada \\\hline
3.2.10 & et de se confidere ; \textbf{ nam cum intendat bonum ipsorum ciuium et subditorum , } naturale est & e que fien vnos de otros . \textbf{ Ca commo el entienda enl bien de los çibdadanos natural } cosaes que sea amado dellos . \\\hline
3.2.10 & populum cum insignibus , \textbf{ insignes cum seipsis . } Vident autem quod quandiu ciues discordant a ciuibus , & e el pueblo con los nobles \textbf{ e los nobles con si mismos . } Caueen que mientra los çibdadanos desacuerdan entre si mismos \\\hline
3.2.12 & pendentem tenuissimo filo apponi fecit : \textbf{ circa ipsum quosdam homines cum ballistis , sagittis appositis , } stare faciebat . & e fizol colgar una espada sobre su cabesça muy aguda de vn filo muy delgado \textbf{ e fizo | poñuallesteros con ballestas armadas contrael . } Et estonçe commo aquel su hͣrmano tomasse grant espanto \\\hline
3.2.12 & Priuatur ergo tyrannus a maxima delectatione , \textbf{ cum videat se esse populis odiosum . } Viso tyrannidem cauendam esse , & e por ende el tirano es pri uado de grant delectaçion \textbf{ quando bee | que es aborresçido delos pueblos } Disto que la tirauja es de esquiuar e de aborresçer \\\hline
3.2.13 & ut recte et debite gubernent populum sibi commissum : \textbf{ cum deuiare a recto regimine sit tyrannizare , } et iniuriari subditis , & e commo desinarse \textbf{ e arredrarse los rreyes del | gouernemjento derecho sea tiranizar } e fazer tuerto alos subditos \\\hline
3.2.16 & quae tractanda sunt circa ipsum . \textbf{ Sed , cum dicat Philosophus } 3 Ethic’ & quales cosas son de trattrar çerca el . \textbf{ Mas commo el pho diga en el segundo libro delas ethicas } que por çierto alguno tomara consseio non de aquellas cosas \\\hline
3.2.17 & ut de magnis consilietur negotiis . \textbf{ Tertio cum consiliari volumus , } debemus alios assumere nobiscum , & por que de grandes cosas tomemos consseio . \textbf{ la tercera cosa es que quando queremos tomar consseio deuemos tomar } connusco otros con los quales ayamos acuerdo delas cosas \\\hline
3.2.17 & cito in opere exequamur . \textbf{ Nam cum adest opportunitas operandi , } et si recte volumus & que luego lo pongamosen obra . \textbf{ ca quando viene el tp̃on coueinble para obrar } si derechͣmente queremos obrar \\\hline
3.2.18 & apparenter saltem . \textbf{ Itaque cum dictum sit } quod qui bene persuadens , & que parezca tal . \textbf{ Et por ende commo sea dicho } que el que es buen amonestador e razonador \\\hline
3.2.19 & et pondera vendentium : \textbf{ et cum expedit taxandum } est pretium venditionis , & e los pesos de los vendedores e de los conpradores . \textbf{ Et quando fuere menester tassar } e poner presçio alas cosas \\\hline
3.2.24 & infra patebit . \textbf{ Cum leges sunt quaedam regulae iuris , } per quas in agibilibus regulamur , & Mas en qual manera con la piadat pueda estar la iustiçia adelante parezcra \textbf{ ommo las leyes sean vnas reglas de derech . } por las quales nos somos reglados en las nuestras obras \\\hline
3.2.25 & et cum substantiis aliis , \textbf{ et cum entibus omnibus . } Poterit ergo inclinatio naturalis & e con las otras sustançias \textbf{ e con todas las cosas que han ser . } Et por ende la inclinacion natural \\\hline
3.2.25 & prout natura humana est quaedam entitas , \textbf{ et conuenit cum entibus omnibus . } Si vero regulae illae sumantur & en quanto la natura humanal es alguna sub̃a e haser \textbf{ e conuiene con todas las cosas | que son } e con todas las sustançias . \\\hline
3.2.25 & cum animalibus aliis , \textbf{ sed ut conuenimus cum entibus omnibus : } nam huiusmodi ius est notius & en quanto non solamente auemos conueniençia con las otras aian las . \textbf{ mas en quanto auemos conueniençia con todas las sustançias . } Ca este derecho tales mas conosçido \\\hline
3.2.27 & si totus populus principetur ; \textbf{ Princeps enim aut totus populus cum principatur , } habet dirigere et ordinare alios in commune bonum . & enssennoreare . \textbf{ Ca el prinçipe o avn todo el pueblo } quando enssennorea ha de ordenar \\\hline
3.2.29 & quam legem . \textbf{ Nam Rex cum sit homo } non dicit intellectum bonum tantum , & corconper el rey que la ley . \textbf{ Ca el Rey por que es omne } non dize entendemiento tan solamente mas dize entendimiento con cobdiçia . \\\hline
3.2.29 & non dicit intellectum bonum tantum , \textbf{ sed dicit intellectum cum concupiscentia : } dato ergo quod Rex non peruertatur & Ca el Rey por que es omne \textbf{ non dize entendemiento tan solamente mas dize entendimiento con cobdiçia . } Et pues que assi es puesto \\\hline
3.2.29 & quia non est ulterius optimus . \textbf{ Rex itaque dicit intellectum cum concupiscentia , } sed lex & Ca de ally adelante non es muy bueno . \textbf{ Et pues que assi es el Rey | dize entendimiento con cobdiçia . } mas la ley por que es alguna cosa \\\hline
3.2.29 & Oportet igitur aliquando legem plicare ad partem unam , \textbf{ et agere mitius cum delinquente , } quam lex dictat : & Et por ende conuiene quela ley que se ençorue \textbf{ e se allegue algunas vezes ala vna parte e que obre mas manssamente con el que peca } quela ley demanda o que la ley nidga . \\\hline
3.2.29 & et tunc iuste \textbf{ et secundum rationem clementer agitur cum delinquente . } Aliquando tales circumstantiae aggrauant : & e estonçe derechͣmente \textbf{ e segunt razon obra el prinçipe pudosamente con el que peca . } Mas algunas uegadas tales cercunstançias agcauian el pecado \\\hline
3.2.30 & per Philosophum 2 Politicorum \textbf{ cum disputat contra Socratem , } secundum hunc modum magis prohibetur concupiscentia & en el segundo libro delas politicas \textbf{ do disputa contra socrates . } Et segunt esta manera mas es defendida la cobdiçia del coraçon \\\hline
3.2.30 & et ad vitium si sint mali , \textbf{ cum procedant ex interiori appetitu . } Sed si consideretur lex humana & e a pecado si son malas . \textbf{ quando uieñe del apetito | et ple desseo del coraçon . } Mas si fuere penssa para la ley \\\hline
3.2.31 & Quaerit Philos’ 2 Polit’ \textbf{ cum disputat contra Hippodamum , } utrum sit expediens ciuitatibus & e manda el pho en el segundo libro delas politicas \textbf{ quando disputa contra ypodomio } si es cosa conuenible alas çibdades \\\hline
3.2.34 & inducere alios ad virtutem , \textbf{ cum virtus faciat habentem bonum ; } et opus bonum , & por que la su entençion es enduzir los otros a uirtud . \textbf{ Et la uirtud faze al que la ha buenon } Esta buean obra conuiene que sea en el gouernamiento derech \\\hline
3.3.2 & et ex quibus artibus sunt assumendi bellantes . \textbf{ Sciendum ergo quod cum bellantes debeant } habere membra apta & e de quales artes son de tomar los lidiadores . \textbf{ Et pues que assi es conuiene de saber } que commo los lidiadores deuan auer los mienbros apareiados \\\hline
3.3.4 & et erubescere turpem fugam . \textbf{ Aduertendum autem quod cum dicimus , } bellatores non habere effusionem sanguinis , & torpemente de la batalla . \textbf{ Mas deuemos parar mientes | que quando dezimos } que los lidiadores non deuen aborresçer el esparzimiento de la sangre \\\hline
3.3.5 & Si in bello terga vertam , \textbf{ Hector cum concionabitur inter Troianos , } dicet , & Ca dizie que si boluiesse las espaldas en la batalla \textbf{ que ector | quando razonasse en las cortes de troya } dirie que diomedes era su vençido \\\hline
3.3.5 & A me deuictus est Diomedes . \textbf{ Quare cum velle honorari } et erubescere de aliquo turpi facto , & dirie que diomedes era su vençido \textbf{ Por la qual cosa commo querer auer honrra de la batalla } e tomar uergueña de torpe fecho \\\hline
3.3.6 & ac si deberent pugnam committere . \textbf{ Et cum viderit magister bellorum } aliquem non tenere ordinem debitum in acie , & assi commo si ouiessen de acometer la batalla . \textbf{ Et quando vieren los caudiellos maestros de las batallas } que alguno non guarda orden en la az \\\hline
3.3.7 & Quarto ad iaciendum sagittas . \textbf{ Quinto ad proiiciendum lapides cum fundis . } Sexto ad percutiendum cum plumbatis . & lo quarto a a lançar saetas . \textbf{ lo quinto a a lançar piedras con fondas . } lo sexto a ferir con pellas de plomo o de fierro \\\hline
3.3.7 & Quinto ad proiiciendum lapides cum fundis . \textbf{ Sexto ad percutiendum cum plumbatis . } Septimo ad ascendendum equos . & lo quinto a a lançar piedras con fondas . \textbf{ lo sexto a ferir con pellas de plomo o de fierro } Lo vij° . \\\hline
3.3.7 & ad iaciendum sagittas , \textbf{ vel cum arcubus , vel cum ballistis . } Nam quia contingit & Lo quarto son de vsarlos lidiadores \textbf{ a alançar saetas con arcos e con ballestas . } ca quando contesçe que non podemos \\\hline
3.3.7 & sunt bellatores exercitandi \textbf{ ad iaciendum lapides cum fundis . } Hic enim modus bellandi & e ballesteros mucho escogidos en todas las azes . \textbf{ Lo quinto son los lidiadores de usar a a lançar piedras con fondas . } Ca esta manera de lidiar fue fallada \\\hline
3.3.7 & ut matres nullum cibum eis exhiberent , \textbf{ quem non primo cum funda percuterent . } Est enim hoc exercitium utile , & que las madres nunca les querien dar de comer \textbf{ fasta que ferien con la fonda en logar çierto . } Et este uso es muy prouechoso \\\hline
3.3.7 & non inutile est lapides \textbf{ cum fundis eiicere . } Sexto , exercitandi sunt bellantes & ca non es trabaio ninguno leuar fondas . \textbf{ Ca algunas uezes es lançar piedras con fondas . } Lo . vi° son de usar los lidiadores a ferir con pellas de fierro o de plomo . \\\hline
3.3.7 & Sexto , exercitandi sunt bellantes \textbf{ ad percutiendum cum plumbatis . } Nam pila plumbea vel ferrea & Ca algunas uezes es lançar piedras con fondas . \textbf{ Lo . vi° son de usar los lidiadores a ferir con pellas de fierro o de plomo . } Ca las pellas de plomo o de fierro \\\hline
3.3.7 & est proprium equitibus : \textbf{ proiicere lapides cum funda , } videtur esse proprium peditibus . & Ca sobir en los cauallos pertenesçe a los caualleros . \textbf{ Et lançar piedas con fondas pertenesçe a los peones . } Mas las otras cosas en alguna manera puenden pertenesçer a todos . \\\hline
3.3.8 & Debet enim exercitus secum ferre munitiones congruas , \textbf{ ut cum castrametari voluerit , } quasi quandam munitam ciuitatem secum portasse videatur . & Ca deue sienpre la hueste leuar consigo guarniciones conuenibles . \textbf{ por que quando quisiere la hueste folgar en algun logar parezca } que lieuan consigo assi commo vna çibdat guarnida . \\\hline
3.3.10 & portare scutum ad se protegendum : \textbf{ et cum debeant esse vigilantes , agiles , sobrii , } habentes armorum experientiam : & para encobrirse meior \textbf{ e avn que ayan los oios bien espiertos | e que sean ligeros e mesurados en beuer e gerrdados de vino } e avn que ayan vso de las armas . \\\hline
3.3.13 & et conuertuntur in fugam . \textbf{ Quare cum percutiendo caesim } propter magnum motum brachiorum insurgat & e tornarsse han a foyr . \textbf{ Por la qual cosa commo feriendo cortando } por el grand mouimiento de los braços \\\hline
3.3.13 & minor laesio ei potest accidere . \textbf{ Quare cum percutiendo punctim } etiam tecto corpore possit & por que assi feriendo menor daño le puede contesçer . \textbf{ Por la qual cosa commo feriendo de punta } avn que este el cuerpo cubierto \\\hline
3.3.15 & qualiter debeant stare \textbf{ cum volunt hostes percutere . } Percussionis autem hostium duplex est modus . & en qual manera deuen estar \textbf{ quando quisieren ferir los enemigos . } Et ay dos maneras de ferir los enemigos . \\\hline
3.3.15 & Unus a remotis , \textbf{ ut cum iaciendo iacula , } vel missilia aduersarios feriunt . & La vna es de lueñe \textbf{ assi conmo quando fieren los enemigos } lançando piedras e dardos e saetas . \\\hline
3.3.15 & tenere pedem sinistrum immobiliter : \textbf{ et cum volunt percutere , } cum pede dextro debent se antefacere , & quando vienen a las manos tener el pie esquierdo firme \textbf{ e quando quieren ferir } deuen se fazer adelante con el pie derecho \\\hline
3.3.15 & cum pede dextro debent se antefacere , \textbf{ et cum volunt ictus fugere , } cum eodem pede debent se retrahere ; & deuen se fazer adelante con el pie derecho \textbf{ e quando dieren los colpes } deuense arredrar con el pie derecho . \\\hline
3.3.16 & nauales dicuntur . \textbf{ Quare cum sint quatuor genera pugnarum , } postquam diximus de campestri , & son dichas batallas nauales e de naues . \textbf{ Por la qual cosa commo sean quatro maneras da batallas } despues que dixiemos de la batalla de la tierra fincanos de dezir de las otras tres . \\\hline
3.3.16 & Tertius modus obtinendi munitiones est per pugnam : \textbf{ ut cum itur ad muros , } et cum per pugnam dimicatur & La terçera manera de ganar las fortalezas es por batalla \textbf{ assi commo quando van a los nivros } e los acometen \\\hline
3.3.17 & ab obsessis molestari poterunt . \textbf{ Nam cum contingat obsessiones } per multa aliquando durare tempora , & de los que estan çercados \textbf{ Ca commo contezca } que las çercas puedan durar algunas vezes \\\hline
3.3.17 & contingit quod existentes in castris \textbf{ ( cum fuerint occupati obsidentes somno , } vel ludo , vel ocio , & o en las çibdades cercadas \textbf{ quando los que çercan durmieren o comieren o estudieren de vagar o fueren derramados } por alguna neçessidat a desora pueden dar en ellos \\\hline
3.3.17 & statim impugnant eos \textbf{ cum ballistis et arcubus : } iaciunt contra ipsos lapides & luego los acometen los de fuera con ballestas \textbf{ e con arcos } e lançan contra ellos piedras con las manos o con fondas \\\hline
3.3.17 & iaciunt contra ipsos lapides \textbf{ cum manibus vel cum fundis ; } apponunt scalas ad muros , & e con arcos \textbf{ e lançan contra ellos piedras con las manos o con fondas } e ponen escaleras a los muros \\\hline
3.3.19 & usque ad munitionem obsessam : \textbf{ quod cum factum est , } tripliciter impugnanda est munitio . & fasta la fortaleza que tienen cercada . \textbf{ Et esto quando assi fuere fecho } en tres maneras puede acometer la fortaleza . \\\hline
3.3.22 & obsidentes inchoare cuniculos : \textbf{ quod cum perceperint , } statim debent viam aliam subterraneam & que los que cercan comiençan a fazer cueuas coneieras . \textbf{ Et quando esto entendieren } luego sin detenimiento \\\hline
3.3.22 & iuxta inchoationem viae subterraneae habere magnas tinnas plenas aquis vel etiam urinis : \textbf{ et cum bellant contra obsidentes , } debent se fingere fugere , & auer tiñas lleñas de agua o de oriñas . \textbf{ En quando lidian contra los que los çercan deuen fingir que fuyen } e deuen salir de aquella cueua \\\hline
3.3.23 & ut ex multis partibus possit nauis succendi ; \textbf{ et cum proiiciuntur talia , } tunc est contra nautas & Et deuen echar muchos tales cantaros en la naue de los enemigos . \textbf{ por que de muchas partes se pueda quemar la naue . } Et entonçe deuen acometer muy fuerte batalla contra los enemigos \\\hline
3.3.23 & e iam nautae habere uncos ferreos fortes , \textbf{ ut cum vident se esse plures hostibus , } cum illis uncis capiunt eorum naues , & auer coruos de fierro muy fuertes \textbf{ e quando veen que son mas } que los enemigos con aquellos coruos prenden las naues dellos \\\hline

\end{tabular}
