\begin{tabular}{|p{1cm}|p{6.5cm}|p{6.5cm}|}

\hline
3.3.16 & Otrossi si por fanbre se ha de tomar la çibdat cercada \textbf{ o el castiello } meior es de cercar la en el tienpo del estiuo & Rursus si per famen est castrum , \textbf{ vel ciuitas obsessa obtinenda , } melius est obsessionem facere aestiuo tempore , \\\hline
3.3.17 & en la çibdato \textbf{ en el castiello çercado } e assi podran ganar aquellas fortalezas . & et per aditum factum ex muris cadentibus reliqui obsidentes ingrediantur castrum , \textbf{ vel ciuitatem obsessam : } et sic poterunt obtinere illam . \\\hline
3.3.18 & o en todas aquellas maneras de lançar \textbf{ que dichas son o en algunas o en alguna dellas podra acomter el castiello o la çibdat cercada . } Ca si conplidamente sopiere todas estas maneras de engennios & vel omnibus praefatis modis proiiciendi , \textbf{ vel aliquibus | sine aliqua praedictarum machinarum , castrum , vel ciuitatem obsessam poterit impugnare . } Si enim plena notitia habeatur de machinis , \\\hline
3.3.19 & en tres maneras puede acometer la fortaleza . \textbf{ ca que el castiello assi fecho } para conbatir la fortaleza & tripliciter impugnanda est munitio . \textbf{ Nam in castro sic aedificato } ad munitionem impugnandam , \\\hline
3.3.21 & entre todas las otras cosas de que deuen basteçer la fortaleza \textbf{ e el castiello deuenla basteçer mayormente de mijo . Ca el mijo menos se podresçe } e mas dura que tedos los otros granos . & vel castrum obsessum milio : \textbf{ nam milium inter cetera minus putrefit , | et plus durare perhibetur . } Copia etiam carnium salitarum non est praetermittenda . \\\hline
3.3.21 & e presta en la fortaleza çercada . \textbf{ Lo segundo en basteciendo el castiello o la cibdat } que teme de ser cercada & eo quod ad multa sit utilis . \textbf{ Secundo in muniendo castrum } vel ciuitatem aliquam obsidendam , \\\hline
3.3.22 & que se puede de ligero cauar e estonçe es de \textbf{ enfortaleçer el castiello o la çibdat } afondando mucho las carcauas & quae de facili fodi potest : \textbf{ et tunc per profundas foueas est fortificanda munitio , } ne per cuniculos deuincatur . \\\hline
3.3.22 & la qual tierra cauada \textbf{ conuiene de apoyar bien el castiello o la çerca } por que se non funda & qua suffossa , et castro demerso in ipsam propter magnitudinem ponderis , \textbf{ oportet castrum iterum construi , } eo quod non possit \\\hline

\end{tabular}
