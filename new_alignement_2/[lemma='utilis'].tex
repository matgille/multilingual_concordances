\begin{tabular}{|p{1cm}|p{6.5cm}|p{6.5cm}|}

\hline
1.1.3 & nam quilibet attente audit , \textbf{ si sperat se utilia auditurum . Ex dicendis autem , } si & por que cada vno acuçiosamente oye \textbf{ sy es para quel diran cosas prouechosas . } ¶ pues que asy es delas cosas \\\hline
1.1.3 & quam etiam in Ethicis , \textbf{ quia quaedam sunt delectabilia , quaedam utilia , quaedam honesta . Bona autem honesta , sunt bona per excellentiam : } nam in his bonis & Et algunos bienes son prouechosos ¶ \textbf{ Et algunos bienes son honestos | Mas los bienes honestos son bienes de grant aunataja } Ca enestos bienes honestos \\\hline
1.1.3 & ( \textbf{ secundum ipsum ) includuntur bona delectabilia , et utilia . Bona enim , } si honesta sint , & Ca enestos bienes honestos \textbf{ segunt el philosofo se ençierran los bienes deletables | e los bines prouechosos . } Ca los bienes sy son honestos \\\hline
1.1.3 & habent in se magnam delectationem , \textbf{ et includunt bonitatem utilium bonorum . } Cum ergo in hoc libro intendatur , & e an en sy grant deletaçion \textbf{ e ençierran en sy bondat de tondos los bienes prouechosos ¶ } Pues que asy es commo en este libro \\\hline
1.2.3 & et talia : \textbf{ quoddam vero utile , } ut pecunia : & e de o tristales cosas ¶ \textbf{ Ay otro bien prouechoso } asi commo son dineros o rriquezas . \\\hline
1.2.3 & ut Temperantia , \textbf{ quae est in concupiscibili . Circa passiones vero ortas ex utili sumuntur duae virtutes . } Nam huiusmodi bonum , vel est mediocre , et commune , & que es en el appetito desseador . \textbf{ Cerca delas passiones | que nasçe del bien prouechoso se toman dos uirtudes . } Ca este bien prouechoso o es medianero e comunal \\\hline
1.2.3 & Ex quo patet , \textbf{ quod quia bona delectabilia non sic possunt habere rationem ardui sicut utilia , } et honesta , & Et daqui paresçe \textbf{ que por que los bienes delectables non pueden auer | assi manera de guaueza } commo han los bienes prouechosos e los honestos maguera que tan bien de los bienes prouechosos \\\hline
1.2.3 & et honesta , \textbf{ licet tam ex bonis utilibus quam ex honestis sumantur duae virtutes , } quarum una est in concupiscibili , & assi manera de guaueza \textbf{ commo han los bienes prouechosos e los honestos maguera que tan bien de los bienes prouechosos } commo delos honestos sean tomadas dos uirtun des delas quales la vna es en el appetito cobdiçiador ¶ Et la otra en el enssannador . \\\hline
1.2.3 & et Magnanimitas sunt circa bona ardua , aliter et aliter : \textbf{ quia Magnificentia est circa magna bona utilia , } ut circa magnos sumptus : & Enpero de departidas maneras . \textbf{ Ca la magnifiçençia es çerca de los bienes grandes e prouechosos } assi commo en fazer grandes espenssas . \\\hline
1.2.3 & nam quaedam sunt delectabilis , \textbf{ circa quae est Temperantia ; quaedam utilia , } circa quae est Liberalitas : & Ca alguons bienes son delectables \textbf{ e en estos ha de seer la tenprança ¶ | Otros bienes son prouechosos } e enestos ha de seer la franqueza . \\\hline
1.2.8 & bona gentis sibi commissae . \textbf{ Verum quia nullus homo sufficit ad excogitandum omnia quae possunt esse utilia toti regno , } cum hoc quod Regem expedit esse solertem ex se , & que conuiene a su pueblo e asu gente ¶ \textbf{ Mas porque ningun omne non puede conplidamente penssar aquellas cosas | que son aprouechables a todo el regno . } Enpero con esto \\\hline
1.2.8 & cum hoc quod Regem expedit esse solertem ex se , \textbf{ quae bona sunt regno utilia excogitando , } oportet ipsum esse docilem , & que conuiene al Rey de ser sotil e agudo \textbf{ de si penssando los bienes | que son aprouechables a su regno } ahun conuiene le de ser doctrinable resçebiendo e tomando coseio de bueons \\\hline
1.2.17 & secundum se . Postea determinabimus de virtutibus respicientibus exteriora bona in ordine ad aliud . \textbf{ Bona autem exteriora vel sunt utilia , } ut pecunia , & en quanto son ordenados a otra cola . \textbf{ Et deuedes laber que los bienes tenporales de fuera o son aprouechosos } assi conmo los dineros o las riquezas \\\hline
1.2.17 & ut honores . \textbf{ Primo ergo dicemus de virtutibus respicientibus bona utilia : } et postea de respicientibus bona honesta . & assi commo son las honrras . \textbf{ Et pues que assi es primeramente diremos delas uirtudes | que caran alos bienes aprouechosos } e despues diremos delas uirtudes \\\hline
1.2.17 & Incipit enim nostra cognitio a sensu . \textbf{ Cum ergo bona utilia sensibiliora sint honestis , } prius determinandum est & e en las cosas que sentimos . \textbf{ Et por ende commo los bienes aprouechosos sintamos nos | mas que los bienes honestos . } primera mendiremos delas \\\hline
1.2.17 & prius determinandum est \textbf{ de virtutibus respicientibus bona utilia . } Circa autem bona utilia & primera mendiremos delas \textbf{ uirtudesque catan dos bienes prouechosos . } Mas assi commo dize el philosofo \\\hline
1.2.17 & de virtutibus respicientibus bona utilia . \textbf{ Circa autem bona utilia } ( & uirtudesque catan dos bienes prouechosos . \textbf{ Mas assi commo dize el philosofo | en el quarto libro delas ethicas } çerca \\\hline
1.2.17 & et circa non accipere alienos , \textbf{ est ex consequenti . Usurpans enim bona utilia , } et non accipiens ea & Ca aquel que vsurpa \textbf{ e toma los bienes prouechosos agenos } malamente commo non deue . \\\hline
1.2.22 & ( ut supra dicebatur ) \textbf{ quaedam sunt utilia , } ut pecuniae , & assi commo es auer et dineros . \textbf{ Et general mientre qual si quier cosa } que puede ser mesurada por dineros . \\\hline
1.2.22 & cuiusmodi sunt honores . \textbf{ Sicut igitur circa ipsa bona utilia est duplex virtus una respiciens magnos sumptus , } ut magnificentia , & assy commo son las honrras . \textbf{ Por enl de assi commo çerca los bienes aprouechosos son dos uirtudes . | La vna que cata alas grandes espenssas } assi commo es la magnificençia . \\\hline
1.4.3 & et pauca se facere sperant . Sexto senex sunt inuerecundi , \textbf{ et inerubescitiui . Nam senes quia illiberales sunt , magis curant de utili quam de honesto . } Magis enim student ad utilitatem , & Lo sexto los uieios son desuergonçados \textbf{ e non toman uerguenca delas cosas . | Ca por que los vieios son escassos } mayor cuydado han del pro \\\hline
1.4.3 & non competit senibus ; \textbf{ quia magis curant de utili , } quam de honore . Tota enim causa , & Ca por que la uerguença es temor de desonera non pertenesçe alos uieios \textbf{ por que may orcuidado han del prouecho } que dela honrra Ca toda la razon \\\hline
1.4.3 & ut quod non curent de honoris statu , \textbf{ et quod magis studeant circa utilia , quam circa opera honore digna . } Non decet tamen eos verecundari : & assi que non ayan cuydado de estado de honrra \textbf{ Et que mas estudien çerca el prouecho | que çerca las obras } que son dignas de honrra . \\\hline
2.1.4 & quia contingit ciuitates habere guerras , \textbf{ utile est uni ciuitati ad expugnandam ciuitatem } aliam confoederare se alteri ciuitati ; & Otrossi . por que contesçe que las çibdades han entressi guerras prouechosa cosa es a vna çibdat \textbf{ para que pueda lidiar con otra çibdat | que aya conpanna } e amistança con otra çibdat \\\hline
2.1.4 & aliam confoederare se alteri ciuitati ; \textbf{ quare cum confoederatio ciuitatum utilis sit ad bellandum hostes , } et ad remouendum prohibentia corruptiua , & que la pueda ayudar . \textbf{ Por la qual consa el amistança delas çibdades es prouechosa | para vençer los enemigos } e para tirar e arredrar todas las cosas \\\hline
2.2.2 & et dominantur in regno . \textbf{ Utile est ergo toti regno habere bonos ciues , } sed utilius est habere bonos principantes , & e son sennors en el regno ¶ \textbf{ pues que assi es prouechosa cosa es a todo el regno | de auer bueon sçibdadanos . } Mas mas prouechosa cosa es de auer bueons prinçipes \\\hline
2.2.2 & Utile est ergo toti regno habere bonos ciues , \textbf{ sed utilius est habere bonos principantes , } eo quod principantis sit alios regere et gubernare : & de auer bueon sçibdadanos . \textbf{ Mas mas prouechosa cosa es de auer bueons prinçipes } por que alos prinçipes parte nesçe de gouernar \\\hline
2.2.4 & distinguendum est de ipso bono . \textbf{ Nam si loqueris de bono utili , } ut de numismate & para el deuemos departir deste bien . \textbf{ Ca si fablaremos del bien aprouechable } assi commo son las riquezas e los dineros e delas otras cosas \\\hline
2.2.4 & debent eis obedire et esse subiecti . \textbf{ Patet igitur quod quantum ad bonum quod est utile , } patres magis diligunt filios , & e ser subiectos a ellos . \textbf{ Et puos que assi es paresçe | que quanto al bien que es apuechable } mas aman los padres alos fijos \\\hline
2.2.5 & et ea quae sunt fidei ratione comprehendi non possunt : \textbf{ utile est ut in illa aetate proponantur ea quae sunt fidei , } in qua ratio non quaeritur dictorum , & e aquellas cosas que son de fe non se pueden prouar \textbf{ por razon prouechosa cosa es que en aquella hedat sean enssennadas las cosas | que son de fe } en la qual hedat non se demanda razon delos dichs \\\hline
2.2.5 & omnem perspicaciam humani generis superare . \textbf{ Quare utilius auctoritati diuinae simpliciter creditur , } quam acquiescatur rationibus & e la su auctoridat sobrepiua toda sotileza de engennio humanal . por la qual cosa mas prouechosa cosa es de creer \textbf{ sinplemente la auctoridat de dios } que alas razones de los omes . \\\hline
2.2.6 & Tanto tamen hoc magis decet Reges et Principes , \textbf{ quanto bonitas filiorum est utilior ipsi regno , } et quanto ex eorum malitia potest in regno maius periculum imminere . & mas conuiene alos Reyes e alos prinçipes \textbf{ quanto la bondat de sus fijos es mas prouechosa al regno . } Et quanto dela maliçia dellos vernie mayor periglo a todo el regno \\\hline
2.2.8 & et Politica , \textbf{ quae est de regimine ciuitatis et regni , valde sunt utiles } et necessariae filiis liberorum et nobilium . Immo & que es del gouernamiento dela casa e dela conparatid̃ . Et la politica \textbf{ que es del gouernamiento de las çibdades | e del regno } mucho son aprouechables \\\hline
2.2.13 & et honorem . Omne enim bonum vel est delectabile , \textbf{ vel utile , } vel honestum siue honorabile . & por que todo el bien o es delectable \textbf{ o es aprouechable o honesto e honrrable } por que el bien honesto puede ser dich̃o honrrable . \\\hline
2.2.13 & et mollia . \textbf{ Si propter utile : } sic quaeruntur calida ad repellendum frigus tempore hyemali , & por bien delectable deuemos las querer delicadas e muelles . \textbf{ Et si las queremos } por el bien prouechoso deuemos las querertales \\\hline
2.2.13 & et aetatum , diuersificanda sunt , \textbf{ ut deseruiunt ad bonum utile . } Sed ut deseruiunt ad bonum honorabile , attendenda est consuetudo patriae , & e delas hedades las vestiduras son de departir \textbf{ en quanto siruen al bien aprouechable . } Mas en quanto siruen al bien de honrra es de catar la costunbre dela tierra \\\hline
2.2.15 & ait , \textbf{ quod mox expedit pueris paruis consuescere ad frigora . Assuescere enim pueros ad frigora utile est ad duo . Primo ad sanitatem , } unde idem Philosophus ait , & Onde el philosofo en el septimo libro delas politicas \textbf{ dizeque luego conuiene alos mocos pequanos de acostunbrar los alos frios | por que es aprouechable a dos cosas ¶ } La primera a sanidat . Onde el pho dize \\\hline
2.2.15 & unde idem Philosophus ait , \textbf{ quod exercitium ad frigora facit bonum habitudinem in pueris propter caliditatem existentem in ipsis . Secundo exercitium ad frigora paruis pueris utile est ad bellicas actiones . } Nam frigus membra consolidat & que el vso \textbf{ que toman en el frio faze buena disposiconn en los moços | por la calentura que es en ellos } ¶La segunda \\\hline
2.2.16 & quod Philosophus 5 Polit’ ait , \textbf{ quod pessimum est non instruere pueros ad virtutem , et ad obseruantiam legum utilium . Inquirit enim Philosophus 8 Polit’ } utrum prius curandum sit de pueris , & que el philosofo enl vij̊ . \textbf{ libro delas politicas dizeque muy mala cosa es de non enssennar | e de non enduzir los mocos a uirtudes } e aguardar las leyes bueans e aprouechosas . \\\hline
2.3.6 & Fuit opinio Socratis et Platonis , \textbf{ ut recitat Philosophus 2 Polit’ quod esset utile } et expediens ciuitati quod ciues propriis possessionibus non gauderent , & ne opinion de socrates e de platon \textbf{ assi commo cuenta el philosofo enł segundo libro delas politicas | que cola aprouechosa } e conueniente serie ala çibdat \\\hline
2.3.6 & ut nunc , \textbf{ utile est ciuitati ciues gaudere possessionibus propriis , } eo quod vita ciuilis & e reluze mas claramente enpero las cosas estando assi como agoraes tan cosa aprouechosa es ala çibdat \textbf{ que los çibdadanos . ayan sus possessiones proprias } por que la uida dela çibdat es uida de los omes en comun \\\hline
2.3.6 & sibi in vita sufficere ; \textbf{ aliquid igitur non est utile simpliciter , } quod est utile in casu . & assi enla uida \textbf{ e por ende en alguna cosa que non es aprouechosa } sinplemente es proprouechosa en algun caso . \\\hline
2.3.6 & aliquid igitur non est utile simpliciter , \textbf{ quod est utile in casu . } In rebus ergo sic se habentibus , & e por ende en alguna cosa que non es aprouechosa \textbf{ sinplemente es proprouechosa en algun caso . } Et pues que assi es estando las cosas \\\hline
2.3.6 & In rebus ergo sic se habentibus , \textbf{ utile est ciuitati ciues habere possessiones proprias , } ne propter ignauiam circa communia , domus ciuium patiantur inopiam . & assi commo dicho es \textbf{ pro prouechosa cosa es ala çibdat | que los çibdadanos ayan possessionspropreas } por que non auiendo cuy dado çerca las cosas comunes dela casa uernien los omes amengua \\\hline
2.3.6 & ut ergo tollatur confusio et inordinatio circa expedientia ciuitati , \textbf{ utile est ciues gaudere possessionibus propriis , } ut magis solicite , & e la desordenaçion çerca las cosas \textbf{ que conuienen ala çibdat cosa aprouechosa es | que los çibdadanos ayan possessiones propreas } por que con mayor cuy dado e mas ordenadamente \\\hline
2.3.9 & et quod esset pulchrum , \textbf{ et utile , } pro quo inueniri possent victualia . & que se podiesse leuar \textbf{ e que fues fermosa e aprouechable } por que se podiessen fallar las uiandas . \\\hline
2.3.9 & quae inter caetera metalla sunt pulchriora , \textbf{ et sunt utilia , } et honorabilia : & por que entre todos los otros metalles son mas fermosos \textbf{ e mas aprouechables e mas honrrados | que dellos se pueden fazer basos } que son aprouechables \\\hline
2.3.9 & et honorabilia : \textbf{ ex eis enim possunt fieri vasa , quae sunt hominibus utilia , } quibus factis videtur utens illis esse in honore et gloria . & e llos omes \textbf{ los quales | assi fechs paresçen alos omes } que los que vsan dellos son en eglesia e en honrra . \\\hline
2.3.12 & et qualis cura circa arbores sit gerenda , disposuimus silentio pertransire , \textbf{ eo quod alii de talibus sufficienter tradidisse videntur . Palladius enim multa huiusmodi enarrauit . Secunda via utilis ad pecuniam acquirendam , dicitur esse mercatiua , cum quis per mare aut per terram defert mercationes aliquas , } vel assistit deferentibus mercationes ipsas . Diuiditur autem ( secundum Philosophum ) mercatoria in tres partes , & para ganar las riquezas \textbf{ es dichͣ mercaduria | assi commo quando alguno por la mar o por la tierra lieuna algunas nicadurias o esta con aquellos que lieun a las mercadurias . } Ca segunt el philosofo la mercaduria se parte en tres partes . \\\hline
2.3.12 & et Principes \textbf{ inter vias tactas solae duae viae videntur esse utiles : } videlicet possessoria , & Mas alos Reyes e alos prinçipes entre las otras maneras dichͣs paresçen dos maneras tan solamente conueinbles \textbf{ para auer dineros . | Conuiene a saber . } La possessoria \\\hline
2.3.12 & et via experimentalis ; \textbf{ utilis est Regibus et Principibus in acquisitione pecuniae . } Sic etiam utilis est via possessionalis non solum in possessionibus immobilibus , & Et la manera dela esperiençia \textbf{ e dela prueua son aprouechables alos Reyes | e alos prinçipes en ganar riquezas } avn en esta misma manera es aprouechable la manera de possessiones \\\hline
2.3.12 & utilis est Regibus et Principibus in acquisitione pecuniae . \textbf{ Sic etiam utilis est via possessionalis non solum in possessionibus immobilibus , } cuiusmodi sunt agri , et vineae , & e alos prinçipes en ganar riquezas \textbf{ avn en esta misma manera es aprouechable la manera de possessiones | non solamente en las possessiones } que son rayzes \\\hline
2.3.20 & Immo si ad mensas Regum \textbf{ et Principum aliqua utilia legerentur , } ut simul , & Mas si alas mesas de los Reyes \textbf{ e de los prinçipes | sienpre se leyessen algunas cosas proprouechosas } assi que quando ellos comne oyessen alguas palabras de bueans costunbres \\\hline
2.3.20 & quomodo est principibus obediendum . Haec ergo , \textbf{ vel alia utilia tradita } secundum vulgare idioma , & Et pues que \textbf{ assi es esto | e o triscosas aprouechosas deuen leer alas mesas de los Reyes } e de los prinçipes dela trra \\\hline
3.1.1 & de qua suo loco dicetur : \textbf{ ostendemus enim communitatem regni utilem esse in vita humana , } et esse principaliorem communitate ciuitatis . Videtur enim suo modo communitas regni se habere ad communitatem ciuitatis , sicut haec communitas se habet ad domum , et vicum . Nam ciuitas sicut complectitur domum , et vicum ; & dela qual diremos en su logar \textbf{ ca mostraremos que la comunidat del regno es prouechosa en la uida humanal | e es mas prinçipal } que la comunidat dela çibdat \\\hline
3.1.5 & quod praeter communitatem ciuitatis , \textbf{ utile est humanae vitae statuere communitatem regni . Prima via sumitur ex parte sufficientiae vitae . } Secunda ex parte adeptionis virtutis . & e podemos mostrar por tres razones \textbf{ que sin la comunidat dela çibdat cosa aprouechosa fue ala uida humanal de establesçer comunidat de regno ¶ } La primera razon se toma de parte del conplimiento dela uida \\\hline
3.1.5 & oportet ciuitatem unam indigere auxilio alterius . \textbf{ Quare sicut utile est vitae humanae in eadem ciuitate congregari diuersos vicos , } ut facilius habeantur quae requiruntur ad vitam : sic utile est ciuitates plures congregari sub uno principatu aut sub uno regno , & que la vna çibdat aya acerto del otra \textbf{ por la qual cosa | assi commo cosa apuechosa es ala uida humanal } que en vna çibdat sean ayuntados muchos uarrios \\\hline
3.1.5 & Quare sicut utile est vitae humanae in eadem ciuitate congregari diuersos vicos , \textbf{ ut facilius habeantur quae requiruntur ad vitam : sic utile est ciuitates plures congregari sub uno principatu aut sub uno regno , } ut facilius & assi commo cosa apuechosa es ala uida humanal \textbf{ que en vna çibdat sean ayuntados muchos uarrios } por que mas ligeramente \\\hline
3.1.5 & et unum indiget alterius opere , \textbf{ propter quod utile est ipsis membris congregari in uno corpore , } ut sibi inuicem & mienbros del cuerpo del omne non han vna obra \textbf{ e el vn mienbro ha mester seruiçio del otro por la qual cosa . cosa aprouechosa es alos mienbros de ser ayuntados en vn cuerpo } por que se acorran los bnos alos otros bien assi \\\hline
3.1.5 & non omnes ciuitates abundant in eisdem , \textbf{ utile est eis congregari } sub uno regno , & por que todas las çibdades non abondan en todas las cosas \textbf{ prouechosa cosa es alas çibdades de ser ay uirtadas so vn regno | por que meior se puedan acorrer las vnas alas otras } en aquellas cosas que son meester \\\hline
3.1.5 & propter faciliorem defensionem \textbf{ et tuitionem utile fuit } ex pluribus communitatibus politicis constituere communitatem unam regni . & por mas ligero defendemiento \textbf{ e mas seguro cosa aprouechosa fue } que de muchͣs comuidades publicas \\\hline
3.1.8 & et alterum est ciuitas , \textbf{ nam hoc quidem scilicet compugnatio utilis est } secundum quantitatem , & e otra cosa es la çibdat \textbf{ ca este defendimiento es prouechable seg̃t la quantidat mayor de los omes } commo quier que sean ellos todos vna cosa en espeçie \\\hline
3.1.9 & quia impossibile est omnes ciues aequaliter esse prudentes \textbf{ et bonos , et esse aequaliter utiles ciuitati , } statim insurgeret rixa inter ciues , & que todos los çibdadanos sean egualmente sabios \textbf{ e todos sean bien egualmente prouechosos ala çibdat } por ende sele unataria luego la contienda \\\hline
3.1.14 & et filios esse communes , \textbf{ ut Socrates statuebat ; nec esse decens , mulieres ordinari ad opera bellica ; nec esse utile , } eosdem semper in eisdem magistratibus praefici , & nin avn es conueinble \textbf{ que las mugieres sean ordenadas alas obras de batalla | nin es prouechoso } que sienpre vnos ofiçialon sean puestos en essos mismos ofiçios \\\hline
3.1.15 & secundum \textbf{ rei veritatem non possibile neque utile : } tamen & segunt uerdat . \textbf{ Enpero si dixieremos | que estas cosas son comunes } por amor \\\hline
3.2.12 & Adeo sensus hominum sunt ad malum proni , \textbf{ ut valde utile sit in via morum multis viis ostendere , } et multis rationibus probare malum , & a tanto los sesos de los omes son enclinados a mal \textbf{ que mucho es prouechoso en la carrera de buenans costunbres mostrar | por muchͣs maneras } e prouar por muchͣs razones \\\hline
3.2.15 & intra seipsos bellare coeperunt . \textbf{ Est autem haec cautela utilis solum duobus principatibus : } ut principatui constituto ex hominibus assuetis ad bella , & en los quales depues que les fallesçieron las guerras de fuera comneçaron a auer guerra entre ssi . mismos . \textbf{ Mas esta cautela es propro prouechable en dos prinçipados } assi commo en el prinçipado \\\hline
3.2.15 & ut plurimum corrumpunt mentes hominum , \textbf{ ut fiant transgressores iustitiae . Est autem haec cautela maxime utilis ad homines , } de quibus Rex certam et diuturnam experientiam non accepit . & por la mayor parte corronpen las uoluntades de los oens \textbf{ por que se fagan traspassadores dela iustiçia | Mas esta cautela es muy aprouechosa } quanto aquellos omes \\\hline
3.2.17 & Bene ergo se habet diligenter quodlibet negocium discutere arduum , \textbf{ an utile sit illud facere : } sed post quam per diuturnum consilium est recte cognitum & pues que assi es muy bien es de escodrinnar con grant acuçia todo negoçio alto e noble \textbf{ si es prouechoso delo fazer . } mas despues que fuere conosçido derechamente \\\hline
3.2.26 & necesse est \textbf{ quod sit utilis : } sed ut refertur ad populum ad quem debet applicari & Et en quanto es conparada al bien comun \textbf{ conuiene que sea aprouechosa . } Mas en quanto es conparada al pueblo \\\hline
3.2.26 & debet \textbf{ esse utilis } ut comparatur ad bonum commune : & Lo segundo la ley humanal e çiuil deue ser prouechosa \textbf{ en quanto es conparada al bien comun . } Ca si enla ley non fuere entendido el bien comun \\\hline
3.2.26 & quae sunt iustae , \textbf{ utiles } et competens populo , & e los prinçipes deuen poner Ca deuen poner buenas e aprouechables \textbf{ e parte nesçientes al pueblo } al qual son puestas de ligero puede paresçer \\\hline
3.2.28 & quales debent esse leges condendae a Regibus et Principibus \textbf{ quia debent esse iustae utiles , } et conuenientes populo cui imponuntur leges . & por los Reyes e por los prinçipes \textbf{ ca dixiemos | que deuen ser derechas e prouechosas } e conueninbles al pueblo \\\hline
3.2.31 & inducebantur multi \textbf{ ut inuenientes consuetudines nouas , dicentes eas esse utiles } et proficuas ciuitati , soluerent leges patrias & muchos eran enduzidos \textbf{ para fallar costunbres nueuas | diziendo que aquellas eran prouechosas ala çibdat . } Et en esto desfazien \\\hline
3.2.34 & triplici via venari possumus , \textbf{ quantum sit utile et expediens populo obedire Regibus et Principibus , et obseruare leges . } Primo enim & por tres razones \textbf{ quanto es prouechoso e conuenible al pueblo de obedesçer alos Reyes | e guardar las leyes . } Ca lo primero desto alçança el pueblo uirtudes e grandes bienes \\\hline
3.2.35 & non instruere pueros ad virtutem , \textbf{ et obseruantiam legum utilium : } et ad obseruandum ea quae requirit politia , & non enssennar los mocos auertudes \textbf{ e aguarda delas leyes prouechosas | e aguardar aquellas cosas } que demanda la poliçia o el gouernamiento del regno \\\hline
3.3.2 & et impetum , sunt quasi furibundi \textbf{ et imprudentes . Ideo non omnino sunt utiles operibus bellicis , } quia consilium et prudentia in dimicando non est modicum utilis . Experimento enim videmus , & Enpero por la grant abondança de la sangre son sañudos e arrebatados \textbf{ e non son sabios en la batalla . | Et por ende no son del todo a prouechosos en las batallas } por que e el conseio e la sabiduria en la batalla es muy prouechosa \\\hline
3.3.2 & et imprudentes . Ideo non omnino sunt utiles operibus bellicis , \textbf{ quia consilium et prudentia in dimicando non est modicum utilis . Experimento enim videmus , } et plane hoc vult Philosophus 7 Politicorum & Et por ende no son del todo a prouechosos en las batallas \textbf{ por que e el conseio e la sabiduria en la batalla es muy prouechosa | ca esto veemos } por prueua \\\hline
3.3.2 & quam etiam prudentia sit necessaria , \textbf{ magis tamen animositas est utilis . } Ideo & commo la sabiduria sean cosas neçessarias . \textbf{ Empero mas aprouechable es la fortaleza del coraçon } que la sabiduria . Et por ende si la gente que es açercana del sol \\\hline
3.3.2 & et si gentes omnino propinquae soli , \textbf{ et omnino remotae non sunt penitus utiles actibus bellicis : } magis tamen inter medias regiones eligendi sunt ad opera bellica remotiores a sole , & e la que es muy arredrada \textbf{ non son aprouechables en toda manera | a las obras de la batalla } Enpero mas aprouechable es la gente de la tierra medianera \\\hline
3.3.2 & dicere possumus quod Fabriferrarii , \textbf{ et carpentarii utiles sunt ad opera bellica : } quia ex arte sua habent brachia apta & e deuan ser poderosos para sofrir los trabaios . \textbf{ Podemos dezir que los ferreros e los carpenteros son aprouechables a las obras de la batalla } por que por la su arte han los braços acostunbrados e apareiados para ferir . \\\hline
3.3.2 & et assueta ad percutiendum . Sic \textbf{ etiam utiles sunt Macellarii : } quia non horrent sanguinis effusionem , cum assueti sint ad occisionem animalium , & por que por la su arte han los braços acostunbrados e apareiados para ferir . \textbf{ Avn en essa misma manera son aprouechables los carniceros } por que non aborresçen el derramamiento de la sangre \\\hline
3.3.2 & ad opus bellicum , \textbf{ aliqua genera artium diximus utilia } ad actiones bellicas , & e en quanto el arte faze al omne apareiado o desapareiado a la obra de la batalla \textbf{ dezimos que algunas maneras de artes son aprouechosas a las obras de la batalla } e algunas non son aprouechosas . \\\hline
3.3.3 & ex quibus signis cognosci habeant homines bellicosi . \textbf{ Sciendum igitur viros audaces et cordatos utiliores esse ad bellum , } quam timidos . Rursus , & que los omnes osados eatreuidos \textbf{ e de grandes coraçones | son mas prouechosos para la batalla } que los temerosos e de flacos coraçones . \\\hline
3.3.3 & quia potentiores sunt viribus , sunt magis eligendi ad opus bellicum . Amplius cum videamus aliqua animalia bellicosa , aliqua vero timida : homines similiores animalibus bellicosis , \textbf{ utiliores videntur esse ad bellum . } Tribus igitur generibus signorum & que son semeiantes masa las animalias lidiadoras \textbf{ paresçe ser mas prouechosos para la batalla . } Et pues que asy es \\\hline
3.3.6 & et facta hostium . \textbf{ Secundo hoc est utile ad obtinendum meliorem locum . } Nam et locus multum facit ad pugnam . & que vayan e escuchar e a saber las condiciones e el estado delos enemigos \textbf{ Lo segundo esto es prouechoso } para ganar meior lugar en la batalla . por que el logar mucho ayuda a la batalla . \\\hline
3.3.6 & facilius obtinebunt aptiorem locum ad pugnandum . Est \textbf{ etiam hoc utile ad prosequendum hostes fugientes . } Nam non de facili quis potest euadere manus agilium & Lo terçero esto es aprouechoso \textbf{ para seguir | e alcançar los enemigos quando fuyen . } Ca no puede ninguno de ligero foyr de las manos de aquellos \\\hline
3.3.6 & vel per saltum incedere . \textbf{ Quod etiam ad tria est utile . Primo ad remouendum impedimenta . } Secundo ad terrendum aduersarios . Tertio ad infligendum maiores plagas . & por que sepan andar saltando e por saltos . \textbf{ la qual cosa es prouechosa a tres cosas | Lo primero para tirar los enbargos } Lo segundo para espantar los enemigos . \\\hline
3.3.6 & quae sine saltu in via transire non possunt : \textbf{ quare utile est ad remouenda impedimenta , } ut equites sic sint docti , & que sin salto non lo pueden saltar nin passar \textbf{ por la qual cosa prouechosa cosa es el saltar | para tirar estos enbargos . } Por que los caualleros \\\hline
3.3.7 & Nam quia contingit quod ipsos hostes non possumus immediate attingere , \textbf{ utile est eos sagittis impugnare : } immo dato quod pugnantes se cum hostibus possint coniungere , & que non podemos de tan çerca llegar a los enemigos \textbf{ para ferirlos . | prouechosa cosa es lançar las saetas } mas puesto que los lidiadores se puedan ayuntar con los enemigos \\\hline
3.3.7 & quem non primo cum funda percuterent . \textbf{ Est enim hoc exercitium utile , } quia fundam portare , & fasta que ferien con la fonda en logar çierto . \textbf{ Et este uso es muy prouechoso } ca non es trabaio ninguno leuar fondas . \\\hline
3.3.8 & quasi quandam munitam ciuitatem secum portasse videatur . \textbf{ Viso utile esse circa exercitum facere fossas } et construere castra : & assi commo vna çibdat guarnida . \textbf{ Visto commo es cosa prouechable a la hueste fazer carcauas } e costruir guarniçiones e castiellos . \\\hline
3.3.8 & quod ipsum oporteat facere . \textbf{ Ostenso utile esse castra construere , } et qualiter & e manden a cada vno qual cosa deua fazer . \textbf{ Mostrado que prouechosa cosa es de fazer los castiellos . } avn en qual manera los enemigos presentes son de fazer los castiellos \\\hline
3.3.9 & secundum quod huiusmodi sunt victoriam obtinere debent : \textbf{ nam ut dicitur 2 Polit’ quantitas in compugnatione est utilis , } sicut maius pondus magis trahit . Secundo , & segunt razon deuen auer uictoria . \textbf{ Ca assi commo dize el philosofo en el segundo libro de las . | la quantia en la batalla es prouechosa } assi commo el mayor peso de la ualança trae al menor . \\\hline
3.3.10 & ut si contingeret aliquem bellatorem deuiare a propria acie , \textbf{ de facili rediret ad illam ; utile ergo fuit in bellis insignia et vexilla deferre , } ne confunderetur exercitus . & para aquella \textbf{ señalPor ende prouechosa cosa fue | e es en las batallas de leuar pendones e sobreseñales } por que se non desordenasse la hueste . \\\hline
3.3.12 & Formae autem acierum \textbf{ secundum se utiles ad bellandum , } sunt pyramidalis , rotunda , & Mas las formas de las azes \textbf{ que son prouechosas | para lidiar son estas . } La . piramidalEt la tiiaral . \\\hline
3.3.12 & Nam diuisis hostibus facilius debellantur . Acies ergo constructa in forma rotunda , \textbf{ utilis est ad sustinendum . In forma vero forficulari , est utilis ad circum dandum et concludendum , } cum hostes sunt pauci . Sed in forma acuta & Mas en forma de tigeras es prouechosa \textbf{ para çercar } e ençerrar los enemigos \\\hline
3.3.12 & et pyramidali , \textbf{ utilis est ad scindendum et diuidendum , } cum hostes sunt plures . & Mas la forma aguda en manera de pera ens prouechosa \textbf{ para fender e departir los enemigos } quando son muchos . \\\hline
3.3.19 & sub quo homines existentes fodiunt muros illos . \textbf{ Est autem hoc aedificium utile , } cum talis est munitio obsessa , & e en castiella a estos artifiçios llaman gatas . \textbf{ Et es este artifizo muy prouechoso } quando tal es la fortaleza \\\hline
3.3.21 & ( \textbf{ vel si deferretur non multum esset utile castro vel ciuitati obsessae ) totum est igni comburendum ; } ne obsidentes superuenientes inde capiant emolumentum , et ex bonis propriis munitionis obsessae inpugnent ipsam . & osi se pueden traer \textbf{ e non son muy prouechosas al castiello o a la çibdat cercada | todas son de quemar por fuego . } por que los cercadores quanda vinieren a cercar \\\hline
3.3.21 & etiam multitudo multum est expediens munitioni obsessae , \textbf{ eo quod ad multa sit utilis . } Secundo in muniendo castrum & Ca la muchedunbre de la sal mucho es prouechosa \textbf{ e a muchas cosas vale } e presta en la fortaleza çercada . \\\hline
3.3.21 & et non videntes percuti possint . Neruorum etiam copia , \textbf{ et funium utilis est munitioni obsessae , } propter ballistas , et arcus , & Et avn es meester grand conplimiento de neruios \textbf{ e de sogas de cañamo en la fortaleza | que teme de ser çercada } para las ballestas e para los arcos \\\hline

\end{tabular}
