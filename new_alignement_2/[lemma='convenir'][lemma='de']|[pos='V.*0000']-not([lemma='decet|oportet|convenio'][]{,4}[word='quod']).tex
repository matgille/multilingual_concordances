\begin{tabular}{|p{1cm}|p{6.5cm}|p{6.5cm}|}

\hline
1.1.2 & mas ligeramente se podra auer \textbf{ Pues que asi es conuiene de saber } que todo este libro entendemos partir & intellectus dicendorum facilius capietur . \textbf{ Sciendum ergo , } quod hunc totalem librum intendimus \\\hline
1.1.2 & el que quiere tractar del gouernaiento \textbf{ e fablarde sy mesmo conujene de tractar } e de dar conosçimiento de todas aquellas cosas & volens tractare de regimine sui , \textbf{ oportet ipsum notitiam tradere de omnibus his } quae diuersificant mores et actiones . \\\hline
1.1.3 & njn guardar \textbf{ syn la gera de dios conviene de cada vn omne } e mayormente prinçipe o Rey & absque diuina gratia obseruari non possunt , \textbf{ decet quemlibet hominem , } et maxime regiam maiestatem \\\hline
1.1.4 & departidanmente establescan e orden en asy mesmos a departidas fines \textbf{ por ende conujene de contar las maneras de beujr } e mostrͣemos commo enllas es de poner la bien andança & diuersi diuersimode sibi finem praestituant , \textbf{ narrandi sunt modi viuendi , } et ostendendum est , \\\hline
1.1.4 & por el primero libro delas ethicas ¶ \textbf{ Conuien de saber ujda delectosa et plazentera ¶ } vida politica e çiuil¶ & ( ut patet ex 1 Ethic’ ) \textbf{ triplicem vitam , } videlicet , voluptuosam , politicam , et contemplatiuam . \\\hline
1.1.5 & e cuenta \textbf{ as conuiene de notar e de saber acuçiosamente } que asy commo la materia & Quod maxime expedit regiae maiestati \textbf{ Est autem diligenter notandum , } quod sicut materia \\\hline
1.1.7 & son riquezas ordenandas a otra cosa . \textbf{ conuiene saber a los riquezas natraales ¶ } Lo segundo que por que estas Riquezas son riquezas & in ordine ad aliud , \textbf{ tum quia sunt diuitiae } ex institutione Hominum , tum quia cum sint corporalia , \\\hline
1.1.8 & non presuma \textbf{ nin se ensoƀuezca mucho nol conuiene de poner su bien andança en las honrras ¶ } Lo terçero se demuestra & non expedit ei \textbf{ suam felicitatem in honoribus ponere . } Tertio hoc est indicens ei , \\\hline
1.1.9 & e muestre algua bondat de fuera \textbf{ por la qual cosa commo al Rey conuenga ser todo diuinal e semeiante a dios } si non es cosa conuenible & quod exterius bona praetendat . \textbf{ Quare cum Regem deceat } esse totum diuinum , \\\hline
1.1.11 & por que non sea menospreçiada la Real magestad . \textbf{ Et por ende le conuiene de auer poderio çeuil . } Ca por el & est Rex dignus honore , \textbf{ et expedit ei habere ciuilem potentiam : } nam propter paruipensionem Principis , \\\hline
1.2.1 & e la su bien andança . \textbf{ Et que non los conuiene poner la su fin en riquezas } nin en poderio çiuil & suam felicitatem debeant ponere , \textbf{ quia non decet | eos suum finem ponere in diuitiis , } nec in ciuili potentia , \\\hline
1.2.3 & de poner meatado ygualdat o derechura . \textbf{ Et estos son tres conuiene de saber quales . } Ca son razones derechas & vel aequalitatem , siue rectitudinem : \textbf{ huiusmodi autem , | tria sunt , } scilicet , rationes , passiones , et operationes exteriores : \\\hline
1.2.3 & departidas me \textbf{ atadespues que assi es por que nos conuenga de passar } por figurança & et aliud medium reperitur . \textbf{ Ut ergo liceat typo , } et superficialiter pertransire , \\\hline
1.2.3 & qua nasçendel bien . \textbf{ Entonçe conuiene de departir del bien . } Ca ay vn bien del omne en si . & quae oriuntur ex bono , \textbf{ tunc distinguendum est de bono ; } quia quoddam est bonum hominis in se , \\\hline
1.2.5 & Ca commo contesca de razonar derechamente \textbf{ e non derechamente conuiene de dar alguna uirtud } que sea razon derecha . & ratiocinari recte et non recte , \textbf{ oportet dare virtutem aliquam , } quae sit recta ratio , \\\hline
1.2.5 & assi commo son las passiones dela saña . \textbf{ Conuiene dar alguna uirtud en las passiones } por la qual las passiones non nos pueden mouer & ut passiones irascibiles : \textbf{ circa passiones oportet | dare virtutem aliquam , } ne passiones nos impellant \\\hline
1.2.5 & nin inclinar a aquelo que uieda la razon ¶ \textbf{ Et otrosi nos conuiene de dar otra uirtud } por la qual las passiones non nos pueden arredrar & ad id quod ratio vetat : \textbf{ et oportet dare virtutem aliam , } ne passiones retrahant nos ab eo , \\\hline
1.2.8 & conuieney de penssar . \textbf{ Conuiene de saber los bienes } aque guia la pradençia¶ & Quatuor ergo est ibi considerare , \textbf{ videlicet , bona , } ad quae dirigit : \\\hline
1.2.8 & e por aquellos prinçipios \textbf{ que es aquello que le conuiene de fazer . } pues que assi es & speculando ex illis regulis \textbf{ quid agere congruit . } Sicut ergo ratione bonorum \\\hline
1.2.8 & aque ha de guiar el Rey su pueblo \textbf{ le conuiene de ser acordable e prouisor . } Assi por razon de la manera & ad quae dirigit , \textbf{ oportet Regem esse memorem , | et prouidum : } sic ratione modi per quem dirigit , \\\hline
1.2.8 & por la qual ha de guiar el pueblo \textbf{ le conuiene de ser entendido e razonable . } Mas por razon dela su persona propia & sic ratione modi per quem dirigit , \textbf{ oportet ipsum esse intelligentem , et rationalem . } Sed ratione propriae personae \\\hline
1.2.10 & por que manda fazer las obras de todas las uirtudes . \textbf{ Enpero conuiene de saber } que la iustiçia legal non es dicha toda uirtud & quodammodo omnis virtus , quia exercet opera omnium virtutum . \textbf{ Non est autem simpliciter legalis Iustitia omnis virtus , } quia est virtus distincta \\\hline
1.2.10 & assi conmosi quisiere auer mas de aquellos bienes \textbf{ de quanto le conuiene auer } por esta razon viene danno alos otros çibdadanos & ut quod velit habere plus de iis , \textbf{ quam eum deceat : } ex hoc infertur nocumentum aliis ciuibus : \\\hline
1.2.11 & e cada vna delas comunidades es conpuesta de de pattidas personas ayuntadas e ordenadas a vna cosa . \textbf{ Et por ende pues nos conuiene de fablar } en semeiança de los mienbros dezimos & et ordinatis ad unum aliquid . \textbf{ Ut ergo liceat figuraliter loqui , } in membris eiusdem corporis est \\\hline
1.2.13 & e pecar en obrando . \textbf{ Conuiene de dar e de ponetur algua uirtud } por la qual seamos reglados en las obras & et bene agere , \textbf{ oportet dare virtutem aliquam , } per quam regulentur in agendo . \\\hline
1.2.13 & e non derechamente en los temores e en las osadias . \textbf{ Conuiene de dar alguna uirtud medianera en los temores } e en las osadias & et audacias contingat \textbf{ aliquem se habere recte , } et non recte , \\\hline
1.2.13 & Et por ende si quisieremos fazer fuertes a nos mismos \textbf{ conuiene de inclinar nos ante ala osadia } que al temor ¶ & quam timor , \textbf{ si volumus nos ipsos facere fortes . } Declarata ergo sunt illa tria , \\\hline
1.2.17 & de dezir delas otras och̃o uirtudes . \textbf{ Onde conuiene saber } que esta s ochon uirtudes o catan alos bienes tenporales de fuera & Virtutes autem aliae \textbf{ vel respiciunt } exteriora bona , \\\hline
1.2.17 & escontra regla derecha de razon e de entendimiento . \textbf{ Conuiene de dar alguna uirtud medianera } entre la auariçia e el gastamiento . & contra rectam regulam rationis , \textbf{ oportet dare virtutem aliquam mediam } inter auaritiam , et prodigalitatem : \\\hline
1.2.18 & que menos dan de quanto les conuiene ᷤ dar \textbf{ Et menos fazen de quanto les conuiene de fazer . } Et desto puede bien paresçer & Semper ergo cogitare debent , \textbf{ quod minora faciunt , | quam deceat . } Ex hoc autem apparere potest \\\hline
1.2.18 & e deue descender todo el gouernamiento del regno \textbf{ non le conuiene de enfermar en las sus costunbres } de tal enfermedat & a quo totum regnum dirigi debet , \textbf{ indecens est aegrotare } secundum mores morbo incurabili , \\\hline
1.2.18 & los que son en el su regno \textbf{ mucho les conuiene de ser liberales e francos } Mas par tenesce al libal e alstan ço de catar tres cosas ¶ & qui sunt in Regno , \textbf{ maxime decet eos liberales esse . } Spectat autem ad liberalem primo \\\hline
1.2.18 & que si espendiere \textbf{ do non le conuiene espender } Mas los Reyes e los prinçipes de suranse & ubi oportet , \textbf{ quam si expendat | ubi non oportet . } Deuiant autem a liberalitate Reges , \\\hline
1.2.18 & e a otros semeiantes \textbf{ a quien non conuiene de dar . } por que aquestos tales mas les conuiene de ser pobres & vel aliis , \textbf{ quibus non oportet dare : } quia magis deceret \\\hline
1.2.18 & a quien non conuiene de dar . \textbf{ por que aquestos tales mas les conuiene de ser pobres } que non ser ricos . & quibus non oportet dare : \textbf{ quia magis deceret | eos esse pauperes , } quam diuites . \\\hline
1.2.19 & que en las espenssas medianas e mesuradas . \textbf{ Conuiene de dezir que la magnifiçençia } que es en las grandes & in mediocribus sumptibus , \textbf{ dici potest magnanimitatem } quae est circa magnos sumptus , \\\hline
1.2.19 & que las espenssas se pueden conparar a dos cosas . \textbf{ Conuiene de saber que se pueden conparar a las riquezas } de aquel & quod sumptus ad duo comparari possunt . \textbf{ Videlicet , ad facultates eius } qui facit sumptus , \\\hline
1.2.19 & en quales cosas ha de ser . \textbf{ Onde conuiene saber } que el omne quanto pertenesçe alo presente & restat videre circa quae habet esse . \textbf{ Homo enim } ( quantum ad praesens ) \\\hline
1.2.21 & Et pues que assi es todas las propiedades del magnifico \textbf{ conuiene auer a los Reyes } e ales prinçipes mas conplidamente e cabada mente . & et semper facere magnifica opera . \textbf{ Omnes igitur proprietates magnifici per amplius , } et perfectius decet \\\hline
1.2.23 & Mas si acaesçiere algun caso tan alto \textbf{ por que al Rey conuenga de poner su gente o avn assi mismo } aperigłsᷤen & casus adeo arduus , \textbf{ quod Rex gentem suam , | vel etiam seipsum debeat } exponere periculis : \\\hline
1.2.24 & E en quanto son dignas de grant honrra . \textbf{ Onde assi cerca las despenssas son dos uirtudes conuiene saber La liƀalidat Et franqza } que cata alas despenssas & Vel ut sunt digna magno honore . \textbf{ Sicut enim circa sumptus sunt duae virtutes , | videlicet , liberalitas , } quae respicit sumptus , \\\hline
1.2.24 & e alos prinçipes de seer magnificos e liberales \textbf{ assi en essa misma manera les conuiene de seer magranimos } e amado res de honrra . & et Principes esse magnificos , et liberales : \textbf{ sic decet eos esse magnanimos , } et honoris amatiuos . Reges enim et Principes decet honores diligere modo quo dictum est ; \\\hline
1.2.24 & en la manera que dich̃ones de suso . \textbf{ Conuiene saber que amen e cobdicien fazer lobras } que sean dignas de honrra . & et honoris amatiuos . Reges enim et Principes decet honores diligere modo quo dictum est ; \textbf{ videlicet , ut diligant et cupiant facere opera , } quae sint honore digna . \\\hline
1.2.25 & por que sin la humildat las otras uirtudes non se pueden auer Et pues que assi es para auer conplida declaraçion de la uerdat \textbf{ conuiene de demostrar } que sin la humildat ninguno non puede ser magnanimo . & Ad plenam igitur \textbf{ declarationem veritatis ostendendum est , } quod sine humilitate nullus potest \\\hline
1.2.25 & e nos allega a aquello que la razon manda o uieda . \textbf{ Conuiene de dar en aquella cosa dos uirtudes ¶ La vna que nos allegue . } Et la otra qua nos arriedre dello . & si unum et idem aliter et aliter acceptum nos retrahit et impellit , \textbf{ oportebit circa illud dare duas virtutes , | unam impellentem , } et aliam retrahentem . \\\hline
1.2.26 & la qual llamamos . humildança . \textbf{ Et pues que assi es conuiene saber } que assi commo cerca la magnanimidat & quam humilitatem vocamus . \textbf{ Sciendum igitur } quod sicut circa magnanimitatem conuenit \\\hline
1.2.27 & e bien obrar o çerca qual quier cosa en que podemos sobre puiar \textbf{ e tal sesçer conuiene de dar y . } alguna uirtud por la qual seamos enderesçados & vel contingit abundare , \textbf{ et deficere , } oportet ibi dare virtutem aliquam , \\\hline
1.2.27 & e uenganças del contesçe de sobrepiuar e de fallesçer . \textbf{ Conuiene de dar y alguna uirtud } que reprima las sobrepuianças & contingit superabundare et deficere : \textbf{ oportet ibi dare virtutem } aliquam reprimentem superabundantias , \\\hline
1.2.27 & et mantenedores dela comunidat . \textbf{ Por la qual cosa si alos Reyes non conuiene de seer sannudos } e deuen se mouer & et conseruatores Reipublicae . \textbf{ Quare si Reges non debent esse iracundi , } et debent moueri \\\hline
1.2.28 & siruen a nos en tres cosas . \textbf{ Conuiene saber . } a amistanca ¶ Et a uerdat ¶ & ad tria nobis deseruiunt , \textbf{ videlicet , ad amicabilitatem , ad veritatem , } et ad iocosas delectationes . \\\hline
1.2.28 & e çerca las obras en las quales partiçipamos con los otros han de ser tres uirtudes \textbf{ conuiene saber ¶amistança } que en otro nonbre puede ser dicha afabilidat & habet esse triplex virtus , \textbf{ videlicet , amicabilitas , } quae alio nomine affabilitas \\\hline
1.2.28 & cerca la qual contesçe de sobrepuiar e de fallesçer . \textbf{ Conuiene de dar uirtud alguna } que reprima las sobrepuianças & circa quam contingit abundare et deficere , \textbf{ oportet dare uirtutem } aliquam reprimentem superabundantias , \\\hline
1.2.29 & que quiere dezir escarnidores e despreçiadores dessi mismos . \textbf{ Et pues que assi es conuiene de dar alguna uirtud medianera } por la qual sean tenpradas las cosas menguadas & idest irrisores , et despectores . \textbf{ Oportet ergo dare aliquam virtutem mediam , } per quam moderentur diminuta , \\\hline
1.2.29 & traca quales cosas ha de seer . \textbf{ Et pues que assi es conuiene saber } que maguera firmar cada vno de ser en ssi aquello que non es o negar & restat videre circa quae habet esse . \textbf{ Sciendum ergo quod licet } affirmare in se esse quod non est , \\\hline
1.2.29 & mas de ser honrrados . \textbf{ nin conuiene de seer pesados } nin guaues mas guatiosos e amables . & sed reuerendos : \textbf{ nec onerosos , } sed gratiosos et amabiles : \\\hline
1.2.31 & e non conplidas \textbf{ non conuiene de ser conexas } nin ayuntadas vna a otra & Virtutes autem naturales et imperfectae , \textbf{ non oportet esse connexas . } Videmus enim aliquos naturaliter habere \\\hline
1.2.33 & que assi commo contesçe de dar guados de ptidos de bueons \textbf{ assi conuiene de dar den parti dos linages de uirtudes . } Conuiene a saber & dare diuersos gradus bonorum , \textbf{ sic est dare diuersa virtutum genera , } ita quod secundum quod aliquis est excellentior bonus , \\\hline
1.2.33 & paresçe por plotino el philosofo do dize \textbf{ que las primeras uirtudes conuiene saber . } Las politicas amollesçen & patet per Plotinum dicentem , \textbf{ quod primae virtutes , } scilicet politicae , \\\hline
1.2.33 & Et las uirtudes segundas \textbf{ conuiene saber las pgatorias tiran } e fazen oluidar las passiones & molliunt idest ad medium reducunt . \textbf{ Secundae , scilicet purgatoriae , } auferunt . \\\hline
1.2.33 & que propusiemos en este capitulo \textbf{ conuiene saber } que son departidos quados de uirtudes . & quod in principio capituli proponebatur , \textbf{ videlicet diuersos esse gradus virtutum . } Declarare vero secundum , \\\hline
1.3.1 & e los prinçipes poner su fin e su bien andança . \textbf{ Et otrosi mostrado es en commo les conuiene de ser uirtuosos } ¶ Agora finca de dezir dela tercera parte deste libro & in quo Reges et Principes suum finem ponere debeant , \textbf{ et quomodo oportet | eos virtuosos esse . } Restat exequi de tertia parte huius primi libri , \\\hline
1.3.1 & assi podemos dezinr que las passiones son doze \textbf{ conuiene saber amor e mal querençia e desseo . } e aborrençia er delectacion . & sicut dicebamus esse duodecim virtutes , \textbf{ sic dicere possumus quod sunt duodecim passiones : } videlicet , amor , odium , desiderium , abominatio , delectatio , tristitia , spes , desperatio , timor , audacia , ira , et mansuetudo . \\\hline
1.3.1 & assi se departen . \textbf{ Ca las primeras seys conuiene saber ¶ } El amor ¶ & Praedictae ergo passiones sic distinguuntur , \textbf{ quia primae sex videlicet , } amor , odium , desiderium , abominatio , delectatio , et tristitia , \\\hline
1.3.2 & en que entendemos de determinar del gouernamiento del omne en si mismo . \textbf{ en quanto es omne conuiene de ueer quantas son las passiones } e que orden han entressi & et quia in hoc primo libro determinare intendimus de regimine sui , \textbf{ videndum est quot sunt passiones , } et quem ordinem habent adinuicem , \\\hline
1.3.2 & e en qual manera se toma el cuento dellas . \textbf{ Conuiene de veer } que orden han entres si vnas aotras & Viso ergo quot sunt passiones , et quomodo accipitur earum numerus : \textbf{ videndum est quem ordinem habeant } ad se inuicem . \\\hline
1.3.3 & quanto parte nesçe alo presente puede ser conparada a tres cosas . \textbf{ Conuiene saber ala tirania . } del tirano ala qual es contraria la real magestad . & ad tria comparari potest \textbf{ scilicet ad tyrannidem , } cui contrariatur : \\\hline
1.3.5 & ninguno non ha esperança mas ha des̃esperança . \textbf{ ¶ Et estas quatro cosas conuiene saber El bien . } Et el bien alto e guaue & sed desperat . \textbf{ Haec autem quatuor , } videlicet , bonum , arduum , futurum , \\\hline
1.3.5 & e alos prinçipes de entender en el bien \textbf{ Mas avn les conuiene de entender } en bien alto e grande e guaue de fazer De mas desto & et Principes tendere in bonum , \textbf{ sed etiam decet eos tendere in bonum arduum . } Amplius quanto maior est communitas , \\\hline
1.3.5 & e alos prinçipes de penssar los bienes \textbf{ non solamente en quanto son altos e grandes mas avn les conuiene de los penssar } en quanto son bienes & Decet ergo Reges et Principes \textbf{ considerare bona non solum } ut sunt ardua , \\\hline
1.3.5 & que han de venir \textbf{ Et avn les conuiene de penssar tales bienes } en quanto pueden ser . & sed ut sunt futura . \textbf{ Congruit etiam eos considerare talia , } ut possibilia . \\\hline
1.3.5 & mas que la su fuerça demanda . \textbf{ Otrossi les conuiene de non esparar alguas cosas } que non son de esparar ¶ & non aggredi aliquid ultra vires , \textbf{ et non sperare aliqua non speranda . } Secundo hoc decet \\\hline
1.3.6 & por ende en este primero libro \textbf{ conuiene de tractar delas costunbres de lons Reyes } uniuersalmente & ut dicitur 1 Physicorum , \textbf{ deo in hoc primo de moribus Regum oportet } pertransire uniuersaliter typo : \\\hline
1.3.6 & descendremos mas alas cercunstançias particulares de cada vno . \textbf{ Empero conuiene de tractar primero estas cosas uniuersales et generales } por que el conosçimiento dellas faze mucho al conosçimiento delas cosas que se siguen . & et maxime in tertio plus descendemus ad particulares circumstantias . \textbf{ Expedit tamen haec uniuersalia pertransire , } quia horum cognitio faciet \\\hline
1.3.6 & en el primero libͤ de los grandes morales . \textbf{ Et pues que assi es conuiene deuer } en qual manera conuiene alos Reyes de sertemosos & non est fortis , sed satuus . \textbf{ Oportet ergo videre } quo modo eos esse deceat timidos , et audaces . \\\hline
1.3.7 & e los prinçipesse de una auer çerca la sanna \textbf{ e cerca la mansedunbre conuiene de saber que la sanna } algunans vezes va ante la razon & circa iram , \textbf{ et mansuetudinem : | sciendum quod ira aliquando rationem praecedit , } et tunc est inordinata et cauenda , \\\hline
1.3.7 & de non ser enbargados en el vso dela razon . \textbf{ Et otrossi quanto mas les conuiene de segnir } esforcadamente & eos non impediri in usu rationis , \textbf{ et viriliter exequi , } quae ratio iudicabit . \\\hline
1.3.8 & por los quales la tristeza se puede esquiuar . \textbf{ Conuiene saber las uirtudes . } los amigos e la consideracion & per quae tristitia vitatur ; \textbf{ videlicet , virtutes , amicos , } et considerationem veritatis . \\\hline
1.3.10 & rectorica cuenta otras seys passiones . \textbf{ Conuiene saber Relo . } gera Njemesim que quiere dezir tanto & 2 Rhetor’ \textbf{ sex alias passiones enumerare videtur , } videlicet , zelum , gratiam , nemesin \\\hline
1.3.10 & por que dos cosas pue de cada vno temer \textbf{ Conuiene saber . } Corrinpimientos e desonrras . & Sed verecundia reducitur ad timorem . \textbf{ Dupliciter autem quis timere potest , } videlicet , corruptiones , et inhonorationes , \\\hline
1.3.11 & e en quanto son de loar tienen el medio . \textbf{ ¶ Estas cosas iustas conuiene de veer } en qual manera los Reyes & ut sunt laudabiles , tenent medium . \textbf{ His visis , videndum est , } quomodo Reges et Principes \\\hline
1.3.11 & uirgon cosos . \textbf{ por que la uerguença es delas cosas malas . Mas al estudioso non conuiene obrar ningunas cosas malas . } Por la qual cosa si conuiene alos Reyes & quod studiosi non est verecundari , \textbf{ quia verecundia est in prauis : | eius autem non est praua operari . } Quare si decet Reges esse studiosos , \\\hline
1.4.1 & Enpero esto non es de loar en los uieios nin en los Reyes . \textbf{ por que los Reyes e los prinçipes alos quales conuiene de ser } assi commo medios dioses & de quibus decet eos uerecundari : \textbf{ Reges tamen et Principes , } quos decet esse quasi semideos , \\\hline
1.4.1 & assi commo medios dioses \textbf{ non solamente non les conuiene de fazer cosas torpes } mas avn deuen aborresçer delas oyr nonbrar & quos decet esse quasi semideos , \textbf{ non solum quod turpia committant , } sed abominabile eis esse debet \\\hline
1.4.1 & por la qual son puestas las penas \textbf{ Mas alos Reyes e alos nobles non les conuiene ser uergonnosos } por si mas por la mala obra & qua infliguntur poenae , \textbf{ dicet miseratiuos esse . } Esse autem verecundos \\\hline
1.4.3 & e alos prinçipes de ser francos faziendo espenssas medianas \textbf{ mas ahun les conuiene de sern magnificos } e granados fazie do grandes cosas ¶ & faciendo mediocres sumptus : \textbf{ sed etiam congruit eos esse magnificos , } magnifica faciendo . \\\hline
1.4.3 & e assi non les conuiene aellos de auer uerguenna \textbf{ por que non les conuiene de obrar cosas torꝑes } delas quales se le unata la uerguença . & Non decet tamen eos verecundari : \textbf{ quia indecens est ipsos operari turpia , } ex quibus verecundia consurgit . \\\hline
1.4.4 & e confonden el entendemiento . \textbf{ Otrosy les conuiene de ser misconiosos non por fallesçimiento nin por llaqueza de } coraçon quales en los vieios . & rationem percutiunt . \textbf{ Decet etiam eos esse miseratiuos , | non propter defectum , } vel propter imbecillitatem : \\\hline
1.4.5 & enla qual antigua miente fueron muchos prinçipes e muchos nobles . \textbf{ Et por ende assi nos conuiene de sentir desta nobleza } Enpero por que segunt la comun opinion de los omes todas las cosas son mesuradas & et multi insignes , \textbf{ sic ergo sentiendum est de nobilitate . } Verum quia secundum communem opinionem hominum omnia mensurantur numismate , \\\hline
1.4.5 & Onde el philosofo dize en el quarto libro de la rectorica \textbf{ que conuiene de ser los nobles magranimos } e de grandes coraçones e magnificos & Unde Philos’ 4 Eth’ ait , \textbf{ quod magnanimos et magnificos decet } esse nobiles et gloriosos . \\\hline
1.4.5 & e escodrinnadores sotilmente de todo aquello \textbf{ que les conuiene de fazer } por que las sus obras & subtiliter inuestigantes \textbf{ quid decet eos facere , } ne opera eorum , \\\hline
1.4.5 & e muy sabios e bien razonados . \textbf{ Otrossi les conuiene de foyr malas costunbres } por que non senas obrauios e deipreçiadores de los otros . & ut sint magnanimi et magnifici , prudentes et affabiles : \textbf{ et fugere malos mores , } ut non sint elati , \\\hline
1.4.7 & Mas los poderosos et los prinçipes \textbf{ por que les conuiene de entender } e auer cuydados de muchͣs cosas retrahen se & Potentes vero et principantes , \textbf{ quia oportet eos intendere exterioribus curis , } retrahuntur , \\\hline
2.1.1 & para ser comunal con todos e conpanenro . \textbf{ Et por ende conuiene de saber } que el omne sobre todas las otras ainalias ha menester quatro cosas & videndum est quomodo se habeat homo adesse communicatiuum , et sociale . \textbf{ Sciendum igitur , } quod homo \\\hline
2.1.3 & or que non trabaiemos en vano fablando dela casa \textbf{ conuiene de saber que la casa algunas uezes } puede ser dicha costruymiento fech̃o de paredes e de techo e de & cum de domo loquimur , \textbf{ sciendum quod domus nominari potest } aedificium constitutum \\\hline
2.1.4 & non cunplie la comunidat de vna casa \textbf{ mas conuiene de dar comunidat de varrio . } Por que commo el uarrio sea fech̃ de muchas casas & non sufficiebat communitas domestica , \textbf{ sed oportuit dare communitatem vici , } ita quod cum vicus constet \\\hline
2.1.4 & todas las cosas neçessarias ala uida \textbf{ conuiene de dar comunidat ala çibdat } sobre la comunidat deluarrio . & omnia necessaria ad vitam , \textbf{ praeter communitatem vici | oportuit } dare communitatem ciuitatis . \\\hline
2.1.4 & non solamente la casa es vna comiundat \textbf{ mas en la casa conuiene de dar muchͣs comunidades } la qual cosa non puede ser sin muchͣs perssonas . & non solum domus est communitas quaedam , \textbf{ sed in domo oportet | dare plures communitates : } quod sine pluralitate personarum \\\hline
2.1.6 & e dela mu ger e del sennor e del lieruo . \textbf{ Emposi la casa fuere acabada conuiene de dar } y la terçera comunidat & Sed tamen , \textbf{ si domus debet esse perfecta , } oportet ibi dare communitatem tertiam , \\\hline
2.1.7 & en la comunidat dela casa \textbf{ primeramente conuiene de ayuntar el uaron con la mugni } e esta orden es muy con razon . & in communitate domestica , \textbf{ primum oportet | congregare marem , et foeminam . } Est autem hic ordo rationabilis . \\\hline
2.1.7 & Mas en demostrando quales el ayuntamiento del uaron e dela muger \textbf{ pmeramente nos conuiene de declarar en qual manera el mater moino es alguna cosa segunt natura . } Et que el omne naturalmente & In ostendendo quidem quale sit ipsum coniugium , \textbf{ primo declarandum occurrit , | coniugium esse aliquid secundum naturam , } et quod homo naturaliter est animal coniugale . \\\hline
2.1.8 & e alos prinçipes \textbf{ quanto a ellos conuiene de auer } may or cuydado de sus fijos & Tanto tamen hoc magis decet Reges et Principes , \textbf{ quanto de prole suscepta } prae omnibus aliis \\\hline
2.1.9 & que deuen tomar en el gouernar aiento del regno \textbf{ non les conuiene de auer muchͣs } mugiers¶ & ab huiusmodi cura , \textbf{ indecens est eos plures habere uxores . } Secunda via sumitur \\\hline
2.1.9 & guaadhuso dela luxia el marido non sea enbargado en el cuydado \textbf{ quel conuiene de auer . } En essa misma manera esto es desconueinble de parte dela muger & ne propter nimiam operam venereorum \textbf{ vir a cura debita retrahatur : } si hoc indecens est parte uxoris , \\\hline
2.1.10 & por que en el casamiento \textbf{ dellos conuiene de guardar la orden natural mas que en otro ninguno . } ¶ Lo segundo esso mismo pue de ser mostrada & coniuges Regum et Principum , \textbf{ quia in eorum coniugio magis quam in alio decet | naturalem ordinem conseruare . } Secundo hoc idem inuestigari potest \\\hline
2.1.11 & e les son tenudos de fazer . \textbf{ Avn essa misma manera non les conuiene de casar con los parientes } que les son muy çercanos & quam sibi inuicem debent . \textbf{ Sic etiam non licet | eis contrahere cum consanguineis aliis , } si sint eis nimia consanguineitate coniuncti , \\\hline
2.1.15 & assi que vn cuchiello sirue a muchos ofiçios . \textbf{ Conuiene saber que por que los pobres non podian auer } muchos instrumentos fazian fazer vn instrͤde & ita quod unus gladius deseruiebat pluribus officiis : \textbf{ utputa pauperes non valentes plura habere instrumenta , } faciebant aliquod instrumentum fabricari , \\\hline
2.1.16 & e en qual manera de una vsar del . \textbf{ Et pues que assi es conuiene de desçender } mas en particular & et quomodo utendum sit eo . \textbf{ Oportet ergo magis in particulari descendere , } qualiter omnes ciues \\\hline
2.1.18 & por qual gouernamiento son de gouerenar las mugieres . \textbf{ Conuiene de contar breuemente } e en suma quales costunbres son de loar & ut sciamus quo regimine regendae sint coniuges , \textbf{ narrandum est sub compendio et succincte , } quae sunt laudabilia , \\\hline
2.1.19 & e then a sus maridos a mayor amor . \textbf{ Et por ende les conuiene de ser calladas } e en essa misma manera avn les conuiene de ser estables e firmes & et ad maiorem amorem viros inducunt : \textbf{ decet ergo eas esse taciturnas . } Sic etiam decet esse stabiles : \\\hline
2.1.19 & Et por ende les conuiene de ser calladas \textbf{ e en essa misma manera avn les conuiene de ser estables e firmes } que quanto la mug̃res mas firme e mas estable & decet ergo eas esse taciturnas . \textbf{ Sic etiam decet esse stabiles : } quia quanto uxor est magis constans , \\\hline
2.1.20 & de guardar tienpo conuenible \textbf{ e avn assi les conuiene de guardar logar conuenible en manera conuenible } por que sea entre los casados & Sic etiam obseruandus est locus congruus \textbf{ et modus conueniens , } ut sit inter coniuges \\\hline
2.1.21 & quanto a todas las otras cosas \textbf{ en que pueden las muger serrar ¶ Et pues que assi es conuiene de sabra } que el conponimiento delas mugers & et etiam quantum ad omnia alia , \textbf{ circa quae foeminae consueuerunt | delinquere debent eas debite admonere . } Sciendum ergo ornatum foemineum \\\hline
2.1.21 & Mas alli commo parelçe en el sobrepiuamiento \textbf{ conuiene de ser tres uirtudes } las quales tanne andronico peri patetico & ( ut videtur ) oportet \textbf{ ibi triplicem virtutem concurrere , } quam tangit Andromicus Peripateticus in libello \\\hline
2.1.21 & Et avn ala mugni del Rey o del prinçipe \textbf{ conuiene de ser mas honrrada } e mas conpuesta que otra & vel etiam Regis decet \textbf{ magis ornatam esse . } Dato igitur quod uxor alicuius viri esset \\\hline
2.2.2 & e en algun sennorio \textbf{ en quel conuiene gouernar los otros } mucho les conuiene de ser sabios e buenos . & et in aliquo dominio , \textbf{ in quo oportet eos alios gubernare ; } maxime decet eos esse prudentes et bonos . \\\hline
2.2.2 & en quel conuiene gouernar los otros \textbf{ mucho les conuiene de ser sabios e buenos . } Mas commo los fijos bengan a mayor bondat e a mayor sabiduria & in quo oportet eos alios gubernare ; \textbf{ maxime decet eos esse prudentes et bonos . } Et cum filii perueniunt \\\hline
2.2.3 & atrattardel gouernamiento paternal . \textbf{ Conuienne de uer onde toma comienço el gouernamiento paternal . } Et por qual gduernamiento son de gouernar los fijos ¶ & determinandum est de regimine paternali : \textbf{ videndum est , | unde sumit originem regimen paternum , } et quo regimine regendi sunt filii . \\\hline
2.2.3 & Et por qual gduernamiento son de gouernar los fijos ¶ \textbf{ pues que assi es conuiene de saber } que tres son los gouernamientos . & et quo regimine regendi sunt filii . \textbf{ Sciendum ergo triplex esse regimen . } Nam omnis regens alios , \\\hline
2.2.7 & e en las sçiençias liberales \textbf{ quanto mas les conuiene de ser mas entendudos } e mas sabios que los otros & insudare literalibus disciplinis , \textbf{ quanto decet eos intelligentiores et prudentiores esse , } ut possint naturaliter dominari . \\\hline
2.2.8 & i anifestamos nr̃a uoluntad e nr̃a entençion . \textbf{ Et por ende conuiene de fallar algua sçiençia } que nos mostrasse & et per debitas rationes manifestemus propositum . \textbf{ Oportuit ergo inuenire aliquam scientiam docentem modum , } quo formanda sunt argumenta , et rationes . \\\hline
2.2.8 & en quanto ella sirue alas bueans costunbres . \textbf{ Et pues que assi es si conuiene de saber la sçiençia moral } a aquellos que dessean enssennorear & inquantum deseruit ad bonos mores . \textbf{ Sic ergo morale negocium scire expedit } ab iis \\\hline
2.2.10 & ¶ La primera quanto alas cosas uisibles \textbf{ que assi commo non les conuiene de fablar cosas torpes } Et la razon desto pone el philosofo en łvij̊ libro delas ethicas & Quantum ad visibilia quidem , \textbf{ quia sicut non decet | eos turpia sequi : } sic indecens est eos turpia videre . \\\hline
2.2.10 & co¶la segunda \textbf{ quanto ala manera dt veret conuiene de saber } quanto alas cosas iusibło sas & quare magis sumus attenti circa illa , \textbf{ et per consequens ea magis memoriter retinemus . } Iuuenes igitur , \\\hline
2.2.11 & e los mançebos çerta el comer \textbf{ Mas conuiene saber } que cerça el comer & et qualiter se debeant \textbf{ habere iuuenes circa ipsum . Circa cibum autem contingit } sex modis peccare , vel delinquere . \\\hline
2.2.12 & Ca la tenprança ha de ser puesta çerca de tres cosas . \textbf{ Conuiene de saber . } Cerca el comer . & Temperantia autem circa tria est adhibenda : \textbf{ circa cibum , potum , et venerea . } Nam non solum cibus indebite \\\hline
2.2.12 & non solamente que se non fagan gollosos por el comͣ \textbf{ mas avn les conuiene de ser mesurados } que non se fagan beodos & ex sumptione cibi : \textbf{ sed etiam decet eos esse sobrios , } ut non efficiantur ebrii \\\hline
2.2.18 & Et avn el primero gento que deue regnar \textbf{ conuiene de tomar menores trabaios } ca segunt el pho & qui regere debent decet \textbf{ minores labores assumere . } Nam secundum Philosophum 8 Politicorum labor corporalis , \\\hline
2.2.18 & Et aquellos que deuen gouernar el regno \textbf{ mas les conuiene de ser sabios } que non lidiadores & Eos autem qui debent regere regnum , \textbf{ magis expedit esse prudentes , } quam bellicosos : \\\hline
2.2.18 & nin por otra uentra a qual si quier non osen tomar armas . \textbf{ Enpero por que mas conuiene de ser sabios } que lidiadores alos fijos de los Reyes & nec pro alio casu audeant arma assumere ; \textbf{ attamen quia decet | eos esse magis prudentes quam bellatores , } filii Regum et Principum \\\hline
2.2.20 & si non nos delectaremos en alguas cosas \textbf{ conuiene de tomar alguas obras conuenibles e honestas } çerca las quales entendamos & nisi in aliquibus delectemur : \textbf{ decet nos assumere | aliqua opera licita et honesta , } circa quae vacantes , \\\hline
2.2.20 & Mas si alguno demandare \textbf{ de que se deuen trabaiar las mugers conuiene de fablar en tales cosas departidamente } segunt el departimiento delas perssonas & Si autem quaeratur \textbf{ circa qualia opera solicitari debent : | oportet in talibus differenter loqui } secundum diuersitatem personarum . \\\hline
2.2.21 & ostrado que non conuiene alas moças de andar uagarosas a quande e allende \textbf{ nin les conuiene de beuir ociosas } finca que agora lo terçero mostremos & quod non decet puellas esse vagabundas , \textbf{ nec decet eas viuere otiose : } restat ut nunc tertio ostendamus , \\\hline
2.3.1 & e çerca el gouernamiento dela conpanna e de los sirmientes . \textbf{ Mas que conuenga de cuydar todas estas cosas } al sabio padre familias & et ministrorum . \textbf{ Quod autem omnia haec considerare } deceat prudentem patremfamilias , \\\hline
2.3.3 & que anings de los otros nobles \textbf{ ca ellos conuiene de ser nobles prinçipalmente } e magnificos en todas sus cosas . & qui debent esse nobiles et praeclari , \textbf{ potissime decet esse magnificos . } Alii enim moderatas possessiones habentes , \\\hline
2.3.9 & que hades es en todo el regno \textbf{ conuiene de poner m̃udaçion delas cosas alos dineros } e de los diueros alas cosas & quod habetur in toto regno , \textbf{ oportuit introduci commutationem rerum ad denarios , } et econuerso . \\\hline
2.3.9 & e de departidas prouinçias \textbf{ conuiene de poner non sola mente mudaçion delas cosas alas cosas } o delas cosas alos dineros & et prouinciarum , \textbf{ oportuit introduci | non solum commutationem rerum ad res , } vel rerum ad denarios ; \\\hline
2.3.9 & o delas cosas alos dineros \textbf{ mas avn conuiene de auer mudaçion } e canbio de diueros a dineros & ø \\\hline
2.3.9 & non las poderemos leuar conueniblemente a luengas tierras . \textbf{ Et pues que assi es conuiene de fablar alguna cosa } que se podiesse leuar & commode ad partes longinquas portari non possunt . \textbf{ Oportuit ergo inuenire aliquid } quod esset portabile , \\\hline
2.3.10 & Et el philosofo en las politicas \textbf{ pone quatro maneras de dineros conuiene saber . } Natural . & in Poli’ \textbf{ quatuor species pecuniatiuae : } videlicet naturalem , campsoriam , obolostaticam , \\\hline
2.3.10 & Enpero alos Reyes e alos prinçipes los quales deuen ser medios dioses \textbf{ non los conuiene de usar dellas } saluo dela primera manera pecumatiua & quod decet esse quasi semideos , \textbf{ exercere non congruit . } Nam primam speciem pecuniatiuae , \\\hline
2.3.11 & Et para esto entender \textbf{ conuiene de saber } que otra cosa es la cosa & qui non est suus . \textbf{ Ad cuius euidentiam sciendum , } quod licet aliud sit \\\hline
2.3.13 & de algun cuerpo mezclado \textbf{ conuiene de dar } ay algun helemento & Sic etiam si plura elementa concurrunt \textbf{ ad constitutionem eiusdem corporis mixti , } oportet aliquod elementum praedominans , \\\hline
2.3.13 & de ser suiebtos alos sabios . \textbf{ Et por ende les conuiene de ser assi subietos } por que por la sabiduria de los sabios sean enderescados e sean sabios & subiici prudentibus \textbf{ expedit enim eis sic esse subiectos , } ut per eorum industriam dirigantur \\\hline
2.3.15 & tenporal esto deue ser despues de aquel bien que entiende . \textbf{ Mas conuiene de dar a ministraçion de alquiler e de amor sin la ministt̃ion natural et segunt ley . } Ca por que en nos es el appetito corrupto & hoc debet esse ex consequenti . \textbf{ Oportuit autem dare ministrationem conductam et dilectiuam | praeter ministrationem naturalem } et secundum legem : \\\hline
2.3.16 & tres cosas deuemos pessar en esto . \textbf{ Conuiene de saber . } la orden del ministrͣ & quantum ad praesens spectat , \textbf{ tria sunt attendenda , } videlicet ordo ministrandi , \\\hline
2.3.16 & e non se faga confusamente \textbf{ e desordenadamente conuiene de poner y vn } mayoral que sea ordenador e mandador de todos los seruientes & ne illud negligatur , \textbf{ et ne fiat confuse et inordinate , } praeficiendus est unus architector ministris illis , \\\hline
2.3.16 & en el gouernamiento delas casas de los Reyes \textbf{ En las quales por la grandeza de los offiçios conuiene de } acomne dar vn ofiçio a muchos seruientes & in gubernatione domorum regalium , \textbf{ ubi propter magnitudinem officiorum oportet } idem ministerium committi ministris multis , \\\hline
2.3.16 & que ellos sean fieles e sabios \textbf{ conuiene saber } que sean fieles & ut sint fideles , et prudentes : \textbf{ fideles quidem quantum } ad rectitudinem voluntatis , \\\hline
2.3.18 & e de los no nobles omes alos \textbf{ que les conuienne de ser dadores } e cobidadores & sed quia volunt retinere mores curiae et nobilium , \textbf{ quos decet datiuos esse ; } propter quod tales curiales dici debent . \\\hline
2.3.19 & Lo quinto \textbf{ e lo postrimero conuiene de saber } en qual manera los señores les han de fazer bien & et aperienda sunt secreta . \textbf{ Quinto et ultimo oportet cognoscere , } qualiter sunt beneficiandi , \\\hline
2.3.19 & e los prinçipes \textbf{ alos quales conuiene de ser magn animos } deuen se mostrar & Reges ergo et Principes , \textbf{ quos decet esse magnanimos } ad proprios ministros , \\\hline
2.3.20 & Mas podemos mostrar por dos razones \textbf{ que non conuiene de fablar mucho en las mesas de los Reyes } nin de los prinçipes & Possumus autem duplici via ostendere , \textbf{ quod non decet | in mensis Regum et Principum } et uniuersaliter omnium nobilium \\\hline
2.3.20 & Et pues que assi es los Reyes \textbf{ e los prinçipeᷤ alos quales conuiene ser muy tenprados } e guardar la orden natural en toda & Reges ergo et Principes , \textbf{ quos decet maxime temperatos esse , } et obseruare ordinem naturalem \\\hline
2.3.20 & Mas alos que son assentados en las mesas \textbf{ conuiene de escusar muchedunbre de palabras } por que non sea tirada la ordenn natural & etiam \textbf{ et ipsos participare virtutes et bonos mores . | Sed si recumbentes , } ne tollatur naturalis ordo , \\\hline
3.1.1 & por nonbre comunal çibdat . \textbf{ Enpero conuiene de saber } que la comuidat dela çibdat & quae communi nomine vocatur ciuitas . \textbf{ Aduertendum tamen , } communitatem ciuitatis esse principalissimam \\\hline
3.1.8 & quales si acaesçiere logar nos podremos dellas fazer mençion . \textbf{ on conuiene de demandar en todas las cosas } grant egualdat cosas fuessen & Maximam unitatem et aequalitatem \textbf{ non oportet | quaerere in omnibus rebus . } Nam si omnia essent aequalia , \\\hline
3.1.8 & nin en vna semeiança \textbf{ conuiene de dar } y deꝑ tidas espeçies & reseruari in una specie , \textbf{ oportet ibi dare species diuersas ; } ut in pluribus speciebus entium reseruetur maior perfectio , \\\hline
3.1.8 & para que aya ser acabada \textbf{ conuiene de dar ay algun departimiento } nin conuiene de ser & esse perfectum , \textbf{ oportet dare diuersitatem aliquam , nec oportet ibi esse } omnimodam conformitatem et aequalitatem , \\\hline
3.1.8 & conuiene de dar ay algun departimiento \textbf{ nin conuiene de ser } y en toda manera confirmada egualdat & esse perfectum , \textbf{ oportet dare diuersitatem aliquam , nec oportet ibi esse } omnimodam conformitatem et aequalitatem , \\\hline
3.1.8 & assi commo de andar e de tanner e de oyr e deuer . \textbf{ por ende conuiene de dar . } y departidos mienbros & ut ambulatione , tactu , visione , \textbf{ et auditus ideo oportet } ibi dare diuersa membra exercentia diuersos actus : \\\hline
3.1.8 & auemos mester casas e uestid̃as e viandas e otras cosas tales \textbf{ por ende conuiene de dar algun departimiento en la çibdat por que en ella sean falladas todas las cosas } que cunplen ala uida . & et aliis huiusmodi ; \textbf{ oportet in ciuitate | dare diuersitatem aliqua , } ut in ea reperiatur sufficientia ad vitam . \\\hline
3.1.8 & delos çibdadanos a algun prinçipe o algun sennor \textbf{ e commo en la çibdat conuenga de dar alguons ofiçioso } alguons maestradgos o algunas alcaldias & ad aliquem principantem vel dominantem , \textbf{ ut cum in ciuitate oporteat | dare aliquos magistratus , } et aliquas praeposituras , \\\hline
3.1.8 & por ende commo estas cosas demanden departimiento \textbf{ conuiene de dar en la çibdat algun departimiento . } La quanta razon se toma & Quare cum hoc diuersitatem requirat , \textbf{ oportet in ciuitate | dare diuersitatem aliquam . } Quinta uia sumitur \\\hline
3.1.8 & mas ala derecha consonançia delas bozes \textbf{ conuiene de dar y departimiento de los tonos } assi commo la pintura non es bien ordenada & sed ad rectam consonantiam oportet \textbf{ ibi dare diuersitatem tonorum . } Sic pictura nunquam est bene ordinata , \\\hline
3.1.8 & si non sopiere en qual manera es establesçida la çibdat \textbf{ e si non sopiere en qual manera conuiene de auer en ella departimiento de ofiçios e de ofiçiales } l sermon en los comienços deueser luengo & nisi sciuerit qualiter constituitur ; \textbf{ et nisi cognoscat | quod oportet in ea diuersitatem esse . } Sermo in principiis debet esse longus , \\\hline
3.1.9 & e de escodrinnar \textbf{ en qual manera la çibdat conuiene de ser vna } e qual departimiento deue auer enlła & diu inuestigandum est , \textbf{ qualiter ciuitatem oportet esse unam , } et quam diuersitatem habere debet , \\\hline
3.1.11 & en los fechos particulares \textbf{ conuiene de venir ala prueua } ca veemos e prouamos & in actibus particularibus oportet \textbf{ ad experientiam recurrere : experti enim sumus } quod habentes aliqua communia , \\\hline
3.1.11 & e nos enssannamos contra ellos muchͣs uezes \textbf{ por que nos conuiene de fablar muchͣs uezes con ellos } e de beuir conellos & et indignamur erga illos , \textbf{ quia oportet nos habere | ad illos multa colloquia , } et diu conuersari cum illis . \\\hline
3.1.12 & Et por ende por que los lidiadores non se enflaquezcan en las batallas \textbf{ conuiene de echar dela batalla } e dela fazienda alos de flaco & ne igitur reddantur bellantes pusillanimes , \textbf{ quos constat esse timidos oportet } ab exercitu expelli . \\\hline
3.1.14 & que sienpre los çibdadanos \textbf{ non les conuenga de lidiar } por defendimiento de su tierra & ab artificibus et ab aliis ciuibus , \textbf{ quod ciues alii pro defensione patriae bellare non oporteat } melius est ergo dicere in ciuitate \\\hline
3.1.17 & por que podrian los çibdadanos auer tan pocas possessiones \textbf{ que les conuenia de beuir } assi es casamente & possent enim ciues adeo modicas possessiones habere , \textbf{ quod oporteret eos ita parce viuere } quod opera liberalitatis de facili exercere non valerent . \\\hline
3.2.1 & e del regno entr̃o de guerra . \textbf{ Pues que assi es conuiene de saber } que assi commo entp̃o de guerra & et de huiusmodi regimine tempore belli . \textbf{ Sciendum igitur , } quod tempore belli defendenda est ciuitas per arma ; \\\hline
3.2.1 & entp̃o dela paz \textbf{ por las leyes conuiene de fazer tractado destas quatro cosas sobredichͣs en este gouernamiento ¶ } La segunda razon para prouar & Quare si considerentur quae requiruntur ad hoc quod tempore pacis per leges bene gubernetur ciuitas , \textbf{ oportet in huiusmodi regimine | de praedictis quatuor considerationem facere . } Secunda via ad inuestigandum hoc idem sumitur ex fine \\\hline
3.2.3 & a departidos ofiçios \textbf{ e departidos mouimientos conuiene de dar algun mienbro vno } assi commo es el coraçon & ad diuersa officia et diuersos motus , \textbf{ est dare aliquod unum membrum } ut cor , \\\hline
3.2.3 & Otrossi si a conposicion de vn cuerpo vienen departidos helementos \textbf{ conuiene de dar } y alguna cosa vna & Rursus si ad constitutionem eiusdem concurrunt diuersa elementa , \textbf{ est dare ibi unum aliquid , } ut animam regentem \\\hline
3.2.5 & linage donde ha de ser tomado el sennor . \textbf{ Mas avn conuiene de determinar la perssona . } Ca assi commo nasçen discordias & ex qua praeficiendus est dominus , \textbf{ sed etiam oportet determinare personam . } Nam sicut oriuntur dissentiones et lites , \\\hline
3.2.5 & mas aman alos primogenitos \textbf{ por que el Rey aya mayor cuy dado del bien del regno conuiene de establesçer } que el regno pertenesca al primo genito & Immo quia patres plus communiter primogenitos diligunt ; \textbf{ ut magis sit curae regi de bono regni , } expedit statuere regnum succedere primogenito : \\\hline
3.2.5 & Mas aqual lo que dessuso fue dich \textbf{ conuiene saber } que quando va el regno & Quod vero superius tangebatur , \textbf{ videlicet quod ire per haereditatem , dignitatem regiam , } est exponere fortunae , \\\hline
3.2.6 & segundo se sigue el terçero . \textbf{ Conuiene de saber } que la entencion del tiran no es en auer riquezas o dineros & eo quod ipse intendat commune bonum . \textbf{ Ex hac autem secunda differentia sequitur tertia ; } videlicet quod intentio tyrannica est circa pecuniam . \\\hline
3.2.6 & Et deste departimiento terçero se sigue el quarto . \textbf{ Conuiene de saber } que el tiranno non ha cuydado & Ex hac autem differentia tertia \textbf{ sequitur quarta videlicet quod tyrannus non curat custodiri a ciuibus , } sed ab extraneis : \\\hline
3.2.7 & que puna de enbargar el tirano . \textbf{ Conuiene de saber . } Paz ¶ Virtudes . & quae satagit impedire tyrannus , \textbf{ videlicet pacem , virtutes , et scientias . } Tyranni enim nolunt ciues habere pacem \\\hline
3.2.8 & e pueda bien beuir son estas . \textbf{ Conuiene de saber ¶ Las uirtudes e las sçiençias } e los bienes de fuera . & et bene viuere , \textbf{ sunt tria , } videlicet , virtutes , scientia , et bona exteriora . \\\hline
3.2.13 & segund que dize el philosofo \textbf{ ca conujene de dar a entender } que estos tales non han cuydado de saluar su vida ¶ & ( ut ait Philos’ ) \textbf{ sunt paucissimi numero , | supponi oportet } eos nihil curare , \\\hline
3.2.16 & Reyr \textbf{ quales cosas le conuiene de fazer } para que derechamente gouierne el pueblo qual es acomendado . & manifestauimus item quod sit Regis officium , \textbf{ et quae oporteat ipsum facere } ut recte regat populum sibi commissum : \\\hline
3.2.16 & que delas securas e delas luuias \textbf{ non conuiene de tomar conseio ¶ La . iii in . avn non caen so conseio } en aquellas cosas & quod de siccitatibus , \textbf{ et imbribus non est consilium . | Quarto non sunt consiliabilia } quae etiam fiunt raro , \\\hline
3.2.17 & por la quel cosa commo muchs mas cosas ayan prouadas \textbf{ que vno solo conuiene de llamar otros } para los negoçios . por que por el conseio dellos pueda ser escogida la meior carrera & Quare cum plures plura experti sint , \textbf{ quam unus solus : | decet ad huiusmodi negocia alios aduocare , } ut per eorum consilium possit \\\hline
3.2.17 & mas obramos en poco tienpo \textbf{ e luego e que conuiene de touiar conseio prolongadamente } mas conuiene de fazerl cosas conseiadas mucho ayna . & operamur autem prompte : \textbf{ et quod oportet consiliari tarde , } sed facere consiliata velociter . \\\hline
3.2.17 & e luego e que conuiene de touiar conseio prolongadamente \textbf{ mas conuiene de fazerl cosas conseiadas mucho ayna . } mar ala Real magestado das aquellas cosas & et quod oportet consiliari tarde , \textbf{ sed facere consiliata velociter . } Omnia autem illa quae habere debet bene persuadens \\\hline
3.2.18 & quales consseieros deue auer la real magestad e quales e quantas cosas son menester en los consseios \textbf{ conuiene de saber } en quantas maneras los omes son de endozir & et quae et quot sunt in consiliis requirenda : \textbf{ scire expedit } quot modis persuadetur hominibus , \\\hline
3.2.18 & que oyen \textbf{ e estas son tres cosas conuiene saber . } ¶ el dezidor que fabla & et inclinantur ad credendum sermones auditos . \textbf{ Haec autem sunt tria , } secundum quod in omni locutione tria sunt consideranda , \\\hline
3.2.19 & sea cerca las sus rentas \textbf{ en la qual cosa dos cosas conuiene de penssar } ¶Lo primero couiene que el Rey non tome ningunas rentas & Primo enim contingit esse Regis consilium circa prouentus , \textbf{ in quo duo sunt attendenda . } Primo , ne maiestas regia \\\hline
3.2.21 & La primera seqma par aquello que tales palabras han de to terçeres desegualar eliez \textbf{ el qual conuiene de ser } assi commo regla derecha en & obligare habent iudicem , \textbf{ quem esse oportet } quasi regulam in iudicando . \\\hline
3.2.22 & demos contar quatro cosas \textbf{ que conuiene de auer alos iuezes } para que den uerdaderos iuisios & Possumus autem quatuor enumerare , \textbf{ quae oportet habere iudices , } ut vera iudicia proferant , \\\hline
3.2.22 & quanto parte nesçe alo presente quatro cosas podemos peussar . \textbf{ Conuiene de saber las partes que contienden . } Et el negoçio de que contienden . & quatuor est considerare , \textbf{ videlicet partes litigantes , } negocium de quo litigant , \\\hline
3.2.23 & mas conuiene alos Reyes \textbf{ e alos prinçipes alos quales conuiene de resplandesçer } por mayor bondat . & multo magis decet Reges et Principes , \textbf{ quibus congruit ampliori bonitate pollere . } Decet itaque eos esse clementes et benignos , \\\hline
3.2.24 & Et por que estos departimientos adugamos a concordia . \textbf{ conuiene de saber que la cosa derechurera es en dos maneras . } Et la ley es en dos maneras . & facere possumus de ipsa lege . Ut ergo haec omnia melius patefiant , \textbf{ et ut has diuersitates ad concordiam reducamus sciendum } quod duplex est iustum , \\\hline
3.2.24 & por que non se pierda dela memoria de los omes \textbf{ conuiene de ser escerpto en algun libro . } Enpero cada vno destos dos derechs tan bien el natural commo el positiuo se puede escͥuir en algun libro & ne a memoria recederet , \textbf{ oportuit ipsum scribi | in aliqua exteriori substantia . } Potest itaque \\\hline
3.2.25 & que es derecho delas aian lias . \textbf{ Et para declaraçion desto conuiene de saber } que el omne & ut ius animalium . \textbf{ Ad cuius euidentiam sciendum quod homo } ut est homo et secudum propriam rationem consideratus differt \\\hline
3.2.26 & en el quarto libro delas politicas \textbf{ que non conuiene de apropar las comunidades } delas çibdades alas leyes . & Ideo dicitur 4 Politicorum \textbf{ quod non oportet } adaptare politias legibus , \\\hline
3.2.26 & Mas las leyes alas comunidades \textbf{ de las çibdades las quales leyes conuiene de ser departidas } segunt el departimiento delas comunidades . & sed leges politiae , \textbf{ quas leges oportet diuersas esse } secundum diuersitatem politiarum . \\\hline
3.2.28 & e quales e quantas obras deuen contener estas leyes \textbf{ e conuiene de sabra } que çinco son los fechos & continere huiusmodi leges . \textbf{ Dicuntur autem quinque esse effectus legum , } vel quinque esse opera legalia , \\\hline
3.2.28 & que son de fazer \textbf{ podemos apodar alas leyes tres cosas conuiene de saber . } Mandar e Consentir e vedar . & aliqui vero respectu operum iam factorum . \textbf{ Respectu fiendorum quidem tria possumos attribuere legibus , } videlicet praecipere , permittere , et prohibere . \\\hline
3.2.28 & Mas en conparaçion delas obras que son ya fechas dos cosas podemos a podar alas leyes . \textbf{ Conuiene de saber gualardonar e dar pena } ca todas las obras de los omes & Respectu factorum vero duo legibus attribuimus , \textbf{ videlicet punire , praemiare . } Nam actiones humanae \\\hline
3.2.28 & e a proprear tres cosas \textbf{ alas leyes conuiene de saber . } Mandar quanto alas buenas obras . & tria legibus attribuimus , \textbf{ videlicet praecipere , } quantum ad opera bona : \\\hline
3.2.28 & e alos prinçipes \textbf{ alos quales conuiene ser muy acuçiosos } çerca del bien comun & dicamus quod decet Reges et Principes , \textbf{ quorum interest solicitari } circa bonum commune : \\\hline
3.2.29 & que auemos de dezir e de sentir en esta tal materia . \textbf{ conuiene de saber que el rey } e qual se quier sennor deue ser medianero entre la ley natural e la ley positiua . & quid circa hanc materiam sit dicendum , \textbf{ sciendum est regem } et quemlibet principantem \\\hline
3.2.30 & assi commo paresçra adelante . \textbf{ Et por ende conuiene de dar ley diuinal } e e un agłical segunt la qual fuessen vedados los pecados todos . & ut in prosequendo patebit : \textbf{ oportuit igitur dare legem euangelicam et diuinam , } secundum quam prohiberentur \\\hline
3.2.30 & e estonfico de suso de declarar . \textbf{ Et para esto conuiene de saber } que si fuere penssada la entençion del & hoc declarandum esse . \textbf{ Sciendum ergo quod } si consideretur intentio legislatoris , \\\hline
3.2.31 & Et por ende non sin razon dubdauna si la opinion de ypodomio era buena \textbf{ e si conuinie de renouar } e mudar las leyes muchͣ suegadas & an positio Hippodami esset bona , \textbf{ et an expediat saepe saepius immutare leges : } dato etiam quod occurrant leges aliquae \\\hline
3.2.31 & que muestra \textbf{ que conuiene de renouar las leyes ¶ } La primera se toma de parte delas sçiençias e delas artes . & per quas videtur ostendi , \textbf{ quod expediat innouare leges . } Prima sumitur ex parte scientiarum et artium . \\\hline
3.2.31 & e qual es la soluçion della . \textbf{ Conuiene de saber que la ley politica sitiua } si fuere derecha conuiene que se raygͤ & de quae sito , \textbf{ sciendum quod lex positiua si recta sit , } oportet quod innitatur legi naturali , \\\hline
3.2.32 & assi commo dixiemos de suso quatro cosas eran de tranctar . \textbf{ Conuiene saber qual deue ser el Rey o el } prinçipe¶Lo segundo quales deuen ser los consseieros & erant quatuor pertractanda , \textbf{ videlicet qualis debet esse | Rex siue Princeps , } quales consiliarii , \\\hline
3.2.32 & Et por ende desenbargadas las tres cosas finca de dezer de la quarta . \textbf{ Conuiene de saber del pueblo . } Mas commo para saber qual deua ser el puebło & restat dicere de quarto , \textbf{ scilicet de populo . } Sed cum ad sciendum qualis debet esse populus , \\\hline
3.2.32 & e commo se deua auer al prinçipe \textbf{ e conuenga de saber } que cosa es çibdat & se habere ad principantem , \textbf{ non modicum amminiculetur | scire } quid sit ciuitas , \\\hline
3.2.32 & que cosa es çibdat \textbf{ conuiene de contar todos aquellos bienes } aque sirue la fechura dela çibdat & quid est ciuitas , \textbf{ enumeranda sunt bona illa } ad quae deseruit constitutio ciuitatis , \\\hline
3.2.36 & en qual manera puede esto ser . \textbf{ Et por ende conuiene de saber } que para que los Reyes e los prinçipes & volumus hic exequi qualiter fieri hoc contingat . \textbf{ Sciendum itaque } quod ut Reges et Principes communiter amentur a populo , \\\hline
3.2.36 & La terçera cosa para que los Reyes sean amados del pueblo \textbf{ es quales conuiene de ser derechureros e eguales . } Ca el pueblo mayormente se le una taria a mal querençia del Rey & ut Reges diligantur a populo , \textbf{ decet eos esse iustos , et aequales . } Nam maxime prouocatur populus ad odium Regis , \\\hline
3.3.1 & departir çinco maneras de prudençia e de sabiduria . \textbf{ Conuiene saber prudençia singular } para gouernar cada vno assi mesmo . & Sciendum igitur militiam esse quandam prudentiam , \textbf{ siue quandam speciem prudentiae . } Possumus autem , \\\hline
3.3.2 & o en quales tierras son meiores lidiadores . \textbf{ Conuiene de tener mientes en estas dos cosas sobredichas . } Et pues que asy es en las partes & in quibus regionibus meliores sunt bellatores , \textbf{ oportet attendere circa praedicta duo . } In partibus igitur nimis propinquis soli , \\\hline
3.3.2 & e de quales artes son de tomar los lidiadores . \textbf{ Et pues que assi es conuiene de saber } que commo los lidiadores deuan auer los mienbros apareiados & et ex quibus artibus sunt assumendi bellantes . \textbf{ Sciendum ergo quod cum bellantes debeant } habere membra apta \\\hline
3.3.3 & finca de ver por quales señales se han de conosçer los buenos lidiadores . \textbf{ Et para esto conuiene de saber } que los omnes osados eatreuidos & ex quibus signis cognosci habeant homines bellicosi . \textbf{ Sciendum igitur viros audaces et cordatos } utiliores esse ad bellum , \\\hline
3.3.5 & aconpañada con la uerguenca del foyr . \textbf{ Enpero conuiene de saber } que para que los nobles de toda parte sean fechos estremados & sagacitas sociata erubescentiae fugiendi . \textbf{ Sciendum tamen quod } ut nobiles ex omni parte efficiantur strenui bellatores , \\\hline
3.3.7 & assi que pudiessen ferir aquel palo o lançar cerca del \textbf{ Mas conuiene de saber } que en lançar dardo o lança & vel saltem prope ipsum proiicere . \textbf{ Est autem attendendum } quod in proiiciendo telum , \\\hline
3.3.7 & que en lançar dardo o lança \textbf{ conuiene de auer maestria } ca primeramente es de esgrimir el dardo o la lança & ø \\\hline
3.3.7 & e avn los caualleros usauan a nadar \textbf{ Enpero conuiene de tener mientes } que algunos destos usos sobredichos pertenesçen mas propriamente a los caualleros & et etiam ipsos equos ad natandum exercebant . \textbf{ Aduertendum autem | quod praedictorum exercitiorum } quaedam sunt magis propria equitibus , quaedam peditibus , \\\hline
3.3.8 & e fazer muy apriessa . \textbf{ Mas conuiene de poner algunos maestros } para costruyr los castiellos & debet celeriter castra construere . \textbf{ Oportet autem semper construendis castris , } et faciendis fossis aliquos magistros praestitui , \\\hline
3.3.8 & penssadas çerca de los assentamientos de los castiellos . \textbf{ Conuiene de declarar } qual deue ser la folgura e las guarnicoñes de las carcauas & circa situm castrorum : \textbf{ declarandum est , } qualis debeat esse eorum forma . \\\hline
3.3.8 & e son de fazer mas anchas carcauas . \textbf{ mas solamente quieren y estar vna noche o por poco tienpo non conuiene de fazer tantas guarniçiones . } Mas la manera e la quantidat de las carcauas pone la vegeçio & aut ibi debet \textbf{ per modicum tempus existere , | non oportet tantas munitiones expetere . } Modum autem , \\\hline
3.3.8 & la carcaua deue ser muy ancha de nueue pies e alta de siete . \textbf{ Mas si la fuerça de los enemigos paresciere mas fuerte conuiene de fazer las carcauas mas anchas et mas fondas } si han uagar para las fazer & fossa debet esse lata pedes nouem , alta septem . \textbf{ Sed si aduersariorum vis acrior imminet , | contingit fossam ampliorem } et altiorem facere ita , \\\hline
3.3.8 & assi que sea la carcaua ancha de doze pies e alta de nueue . \textbf{ Enpero conuiene de saber } que si la carcaua fuere fonda de & et alta nouem . \textbf{ Est tamen aduertendum } quod si fossa sit alta pedum nouem , \\\hline
3.3.9 & que en la batalla son dos cosas . \textbf{ Conuiene saber . } Los omnes lidiadores & Videmus autem in bello duo existere , \textbf{ videlicet viros pugnantes , } et auxilia alia \\\hline
3.3.9 & ca si los enemigos esperan mayores ayudas \textbf{ o non conuiene de } lidiaro conuiene de apressurar la batalla . & Nam si hostes plura expectant auxilia , \textbf{ vel non est bellandum , } vel acceleranda est pugna . \\\hline
3.3.9 & o non conuiene de \textbf{ lidiaro conuiene de apressurar la batalla . } Mas si ellos esperan mayores ayudas & vel non est bellandum , \textbf{ vel acceleranda est pugna . } Si autem ipsi plures auxiliatores expectant , \\\hline
3.3.9 & conplidamente puede entender sil \textbf{ conuiene de acometer batalla publica o non . } Ca segunt que viere & prudens dux exercitus sufficienter aduertere potest , \textbf{ utrum debeat publicam pugnam committere . } Nam prout viderit se \\\hline
3.3.10 & para guiar los lidiadores . \textbf{ Mas conuiene de dar otras seña les manifiestas . por que cada vno viendo aquellas señales } se sepa tener ordenadamente en su az & non sufficiunt ad dirigendum bellantes , \textbf{ sed oportet dare euidentia signa ; | ut quilibet solo intuitu sciat } se tenere ordinate in acie , \\\hline
3.3.12 & Et vistas estas cosas \textbf{ conuiene de saber } que entre todas las otras formas de la az la quadrada es mas sin prouecho . & His visis sciendum quadrangularem formam aciei \textbf{ inter caeteras formas esse magis inutilem : } ideo secundum hanc formam nunquam formanda est acies simpliciter , \\\hline
3.3.12 & assi puede establesçer muchas o pocas azes . \textbf{ Otrossi conuiene de saber } que sienpre ençima del as & poterit plures aut pauciores acies construere . \textbf{ Sciendum etiam , } quod semper in cornu aciei \\\hline
3.3.12 & por que non pueda ser cofondida el az . \textbf{ Otrosi conuiene de tener mientes } que en cada vna de las azes & qui possint virilius dimicare . \textbf{ Est etiam aduertendum , } quod in qualibet acie \\\hline
3.3.13 & por que quanto aquellos aniellos mas son ayuntados . \textbf{ tanto conuiene de cortar mas dellos } para que los colpes enpeescan . & quia quanto illi annuli magis sunt compacti , \textbf{ tanto oportet plures ex eis frangere } ut vulnera noceant . \\\hline
3.3.13 & Mas en feriendo cortando . \textbf{ por que conuiene de fazer grand mouimiento de los braços } ante que se de el colpe el enemigo & In percutiendo autem caesim , \textbf{ quia oportet fieri magnum brachiorum motum prius quam infligatur plaga , } aduersarius ex longinquo potest prouidere vulnus , \\\hline
3.3.13 & por que firiendo taiando \textbf{ conuiene de leuantar el braço derecho e diestro . } Et leuantando el braço derecho paresçe descubierto el costado derecho & Percutiendo enim caesim oportet \textbf{ eleuare brachium dextrum : } quo eleuato dextrum latus nudatur et discooperitur , \\\hline
3.3.14 & en qual manera deuen lidiar . \textbf{ Por la qual cosa les conuiene de foyr . } Lo quarto el señor de la hueste se deue tenprar & qualiter debeant dimicare : \textbf{ propter quod oportebit eos fugam eligere . } Quarto dux exercitus sic se temperare debet : \\\hline
3.3.15 & finca nos de ver qual manera son los enemigos de ençerrar e de çercar . \textbf{ pues que assi es conuiene de saber } que pocas vezes o nunca son los enemigos de ençerar & quomodo sunt hostes includendi et circumdandi . \textbf{ Sciendum igitur , } quod raro aut nunquam sic circumdandi sunt hostes in pugna publica , \\\hline
3.3.15 & Et ellos ydos los caualleros pueden meior despues escusar los colpes de los enemigos . \textbf{ Avn conuiene de saber } que quando se assi escusa la batalla & vitare hostium percussiones . \textbf{ Est etiam aduertendum } quod quando sic declinatur pugna , \\\hline
3.3.16 & las quales son estas . \textbf{ Conuiene de saber . } Batalla canpal . & Videntur omnia bella \textbf{ ad quatuor genera reduci , } videlicet ad campestre , obsessiuum , defensiuum , et nauale . \\\hline
3.3.16 & Et avn algunas vezes contesçe que algunos otros çercan sus villas o sus castiellos . \textbf{ Por la qual cosa les conuiene de vsar de batalla defenssiua para se defender . } Otrossi contesçe que en el prinçipado & inuadere aliquas munitiones eorum ; \textbf{ propter quod eos oportet | uti pugna defensiua . } Amplius in principatu et regno contingit \\\hline
3.3.16 & en quantas maneras tales fortalezas pueden ser vençidas . \textbf{ Et conuiene de saber } que son tres maneras de ganar las fortalezas e los castiellos . & ø \\\hline
3.3.16 & que son tres maneras de ganar las fortalezas e los castiellos . \textbf{ Conuiene saber . } por sed e por fanbre e por batalla . & munitiones et castra , \textbf{ videlicet , per sitim , famem , et pugnam . } Contingit enim aliquando obsessos carere aqua : \\\hline
3.3.16 & Ca contesçe algunas vegadas que los cercados non han agua . \textbf{ e por ende o les conuiene de peresçer o de morir } de sedo de dar las fortalezas . & Contingit enim aliquando obsessos carere aqua : \textbf{ ideo vel oportet eos siti perire , } vel munitiones reddere . \\\hline
3.3.16 & finca de demostrar en que tienpo es meior de çercar las çibdades e las castiellos . \textbf{ Et por ende conuiene de saber } que en el tienpo del uerano & obsidere ciuitates et castra . \textbf{ Sciendum itaque quod tempore aestiuo } antequam sint \\\hline
3.3.18 & fasta las menas del castiello o de la çibdat cercada . \textbf{ Conuiene de vsar de tales armadijas o de tales armamientos } por que puedan ganar el logar & vel ciuitatis obsessae , \textbf{ oportet talibus uti argumentis } ut habeatur intentum . \\\hline
3.3.18 & e auer lo que entienden . \textbf{ Et pues que assi es conuiene de ver } e de saber & ut habeatur intentum . \textbf{ Videndum est igitur , } quot sunt genera machinarum lapidariarum , \\\hline
3.3.18 & o tomar puede rayz o comienço de aquellas sobredichas . \textbf{ Et avn conuiene de saber } que tan bien de noche commo de dia se pueden acometer las fortalezas cercadas & vel potest originem sumere ex praedictis . \textbf{ Est etiam aduertendum } quod die et nocte per lapidarias machinas impugnari possunt munitiones obsessae . \\\hline
3.3.19 & Et estos artifiçios pueden se adozir a quatro maneras . \textbf{ Conuiene de saber a carneros . } Et a vinnas . & quasi ad quatuor genera reducuntur , \textbf{ videlicet , ad arietes , vineas , turres , et musculos . } Vocatur enim Aries , \\\hline
3.3.19 & deuemos penssar tres cosas \textbf{ Conuiene de saber la parte del castiello } mas alta que los muros & est tria considerare , \textbf{ videlicet partem superiorum excedentem muros ; } et curriculas munitionis capiendae partem quasi mediam , \\\hline
3.3.21 & Et dicho fue dessuso que tres maneras ay para tomar las fortalezas . \textbf{ Conuiene de saber . } Por fanbre e por sed e por batalla . & triplicem esse modum deuincendi munitiones : \textbf{ videlicet per famem , sitim , et pugnam . } Sic ergo muniendae sunt munitiones obsessae , \\\hline
3.3.21 & que passa por los foradillos menudos de la çera toda se torna dulçe . \textbf{ Et avn conuiene de traer } e de acarrear vinagre & Deferendum est etiam ad munitionem obsidendem \textbf{ in magna copia acerum , } et vinum , \\\hline
3.3.21 & mucha cal fecha poluo \textbf{ e conuiene de la traer en grand abondança } donde quier que la podieren fazer a la fortaleza . & Calcem etiam puluerizatam deferendum est \textbf{ ad ipsam munitionem in magna abundantia , } et ex ea replenda sunt multa vasa ; \\\hline
3.3.21 & donde quier que la podieren fazer a la fortaleza . \textbf{ Et conuiene de finchir della } muchas vasigas de tierra assi commo tinaias e cantaros & ad ipsam munitionem in magna abundantia , \textbf{ et ex ea replenda sunt multa vasa ; } ut cum obsidentes \\\hline
3.3.22 & la qual tierra cauada \textbf{ conuiene de apoyar bien el castiello o la çerca } por que se non funda & qua suffossa , et castro demerso in ipsam propter magnitudinem ponderis , \textbf{ oportet castrum iterum construi , } eo quod non possit \\\hline
3.3.23 & de la batalla de las naues . \textbf{ enpero non conuiene de nos } de tener çerca esto tanto . & volumus aliqua de nauali bello : \textbf{ non tamen oportet } circa hoc tantum insistere , \\\hline
3.3.23 & por pequena batalla de los enemigos de ligero peresçe . \textbf{ Et pues que assi es conuiene de saber } que segunt que dize vegeçio que los maderos que se deue fazer la & ex modica impugnatione hostium de facili perit . \textbf{ Sciendum ergo , } quod secundum Vegetium , \\\hline
3.3.23 & lo terçero en la batalla de la mar \textbf{ conuiene de tener mientes } que los que lidian sobre mar & eos facilius vincant . \textbf{ Tertio est circa marinum bellum attendendum , } ut semper pugnantes nauem suam faciant \\\hline
3.3.23 & aquellos que se lleguan a la tierra . \textbf{ Lo quarto conuiene de colgar al maste de la } naue vn madero luengo e ferrado de a mas partes . & qui detrahuntur ad terram . \textbf{ Quarto ad arborem nauis suspendendum est lignum quoddam longum } ex utraque parte ferratum , \\\hline
3.3.23 & en la batalla de la mar \textbf{ conuiene de auer grand conplimiento de saetas anchas . } con las quales se pueden & Quinto in bello nauali habenda est \textbf{ copia ampliarum sagittarum , } cum quibus scindenda sunt vela hostium . \\\hline
3.3.23 & commo las de la tierra . \textbf{ Et para esto saber conuiene de notar } que segunt el philosofo non lidiamos & ad quid bella omnia ordinantur . \textbf{ Sciendum ergo quod } secundum philosophum non bellamus , \\\hline

\end{tabular}
