\begin{tabular}{|p{1cm}|p{6.5cm}|p{6.5cm}|}

\hline
1.1.13 & Et commo el amor sienpre sean los semeiables e acordables con el . \textbf{ Conuiene que aquel que es para de ser } gualardonado de dios & cum semper amor sit ad similes , et conformes , \textbf{ oportet esse similem , } et conformem Deo , \\\hline
1.2.6 & en las cosas singulares . \textbf{ Conuiene que la pradençia sea cerca las cosas singulares } e particulares & et agibilia sint singularia , \textbf{ oportet prudentiam esse circa particularia , } applicando uniuersales regulas \\\hline
1.2.31 & commo los p̃h̃osacuerdan en esta sentençia \textbf{ que conuiene que todas las uirtudes sean ayinntadas la vna con la vna con la otra . } Ca dixieron que aquel que ha vna uirtud & in hanc sententiam conuenerunt , \textbf{ quod oportet virtutes connexas esse . } Dixerunt enim \\\hline
1.4.7 & li dellos contamos algunas malas costunbres \textbf{ ca non conuiene que todos seantales . } Mas abasta que aquellas costunbres sean falladas en muchos por que non & aliquos malos mores : \textbf{ quia non oportet omnes esse tales , } sed sufficit reperiri illud in pluribus : \\\hline
2.1.2 & e por si vale a conplimiento dela uida . \textbf{ Conuiene que la comunidat dela casa sea mas neçessaria } Et pues que assi es los Reyes e los prinçipes & ad per se sufficientiam vitae , \textbf{ oportet communitatem domus necessariam esse . } Reges ergo et Principes , \\\hline
2.1.6 & luego que es fecha fazer otra semeiante \textbf{ assi mas conuiene que ella primeramente sea acabada } enssi & potest sibi simile producere , \textbf{ sed oportet prius ipsam esse perfectam . } statim enim , \\\hline
2.1.6 & luego otro su semeinante \textbf{ mas conuiene que primeramente el sea acabado . } Et pues que assi es engendrar su semeiante non pertenesçe a cosa natural tomada en qual quier manera mas pertenesçe a cosa natural en quanto ella es acabada . & nec statim potest sibi simile producere , \textbf{ sed oportet prius ipsum esse perfectum : | producere ergo sibi similem , } non est de ratione rei naturalis \\\hline
2.1.6 & Et por ende paresçe que para que la casa sea acabada \textbf{ que conuiene que sean enlla tres comuundades . } ¶ La vna del uaron e dela muger ¶ & Patet ergo quod ad hoc quod domus habeat esse perfectum , \textbf{ oportet ibi esse tres communitates : } unam viri et uxoris , aliam domini et serui , \\\hline
2.1.6 & Ca commo en la casa acabada sean tres gouernamientos . \textbf{ Ca conuiene que este libro sea partido en tres partes . } ¶ En la primera delas quales tractaremos del gouernamiento mater moianl . & Nam cum in domo perfecta sint tria regimina , \textbf{ oportet hunc librum tres habere partes ; } in quarum prima tractetur primo de regimine coniugali : \\\hline
2.1.10 & assi commo si algun çibdadano es subiecto al preuoste e al Rey . \textbf{ Conuiene que el } prinoste sea ordenado al Rey e sea so el . & ut si quis subiicitur Proposito et Regi , \textbf{ oportet Propositum illum ad Regem ordinari , } et esse sub ipso repugnat \\\hline
2.1.15 & por la natura \textbf{ conuiene que sea muy ordenado . } Ca aquel gnia la natura de que viene todo ordenamiento & et quicquid natura praeparatur , \textbf{ oportet ordinatissimum esse : } quia ille naturam dirigit , \\\hline
2.2.6 & Ca quando alguno es inclinado a alguacosa . \textbf{ Conuiene que el vse mucho en el contrario } por que non sea inclinado a aquella cosa . & Nam cum aliquis est pronus ad aliquid , \textbf{ oportet ipsum multum assuescere in contrarium , } ne inclinetur ad illud : \\\hline
2.2.9 & e que entienda los dichͣs de los otros . \textbf{ ¶ Lo terçero conuiene que sea iudgador } e que aya razon para iudgar . & et intelligens aliorum dicta . \textbf{ Tertio oportet ipsum esse iudicatiuum : } nam perfectio scientiae potissime \\\hline
2.2.11 & Ca si la vianda se ouiere bien a cozer \textbf{ conuiene que sea bien proporçionada ala calentura natural } Por la qual cosa si en tan grand quantia se & Si enim cibus digeri debeat , \textbf{ oportet | ipsum esse proportionatum calori naturali . } Quare si in tanta quantitate sumatur , \\\hline
2.3.16 & En essa misma manera cada vna muchedunbre si bien ordenada es \textbf{ conuiene que sea aduchͣa vn ordenador } de quien ella sea ordenada . & si debet esse ordinata , \textbf{ oportet reduci in unum aliquem , } a quo ordinetur . \\\hline
3.1.1 & commo toda comunidat sea por graçia de algun bien . \textbf{ Conuiene que la çibdat sea establesçida por algun bien } Ca pruena el pho & gratia alicuius boni , \textbf{ oportet ciuitatem ipsam constitutam esse propter aliquod bonum . } Probat autem Philosophus primo Polit’ duplici via , \\\hline
3.1.6 & niguno non abasta assi mismo en fallar algunan arte \textbf{ mas conuiene que sea ayuda de } por ayuda de los que passaronante & in inueniendo artem aliquam , \textbf{ sed oportet ad hoc iuuari } per auxilium praecedentium \\\hline
3.1.8 & por que nos auemos me estermuchͣs cosas departidas para abastamiento dela uida \textbf{ conuiene que enla çibdat sea algun departimiento . } La tercera razon que declara e manifiesta las razones & Quia ergo diuersis indigemus ad vitam , \textbf{ oportet in ciuitate diuersitatem esse . } Tertia via declarans \\\hline
3.2.6 & quando es muy escalentada \textbf{ e muy enraleçida conuiene que la raledat e la calentura mas acabadamente sea fallada en el fuego } despues que fuere engendrado e ençendido . & cum calefit et rarefit , \textbf{ oportet raritatem et calorem perfectius reperiri } in igne iam generato \\\hline
3.2.16 & ca el nuestro consseio non es dela fu . \textbf{ por que conuiene que en el conseio sorongamos la fin } e que non tomemos consseio della & sed de his quae sunt ad finem : \textbf{ oportet enim in consilio | praesupponere finaliter intentum , } et non consiliari de ipso , \\\hline
3.2.22 & por abortençia o por mal querençia . \textbf{ conuiene que el uiez judgue mal e desigual mente . } Ca entonçe el uuzio non salle de zelo de iustiçia & ab alia vero recedit per odium , \textbf{ oportet ipsum iudicare inique : } quia tunc iudicium non procedit \\\hline
3.2.26 & ues que assi es . \textbf{ Lo primero conuiene que la ley humanal o positiua sea derecha } en quanto es conparada ala razon natural o ala ley de natura . & aliquam diuersitatem existere . \textbf{ Primo igitur oportet legem humanam | siue positiuam esse iustam } ut comparatur ad rationem naturalem \\\hline
3.2.26 & e esto queremos alcançar \textbf{ conuiene que fagamos estas cosas . } Et pues que assi estales deuen ser las leyes & et hoc sequi volumus , \textbf{ oportet hoc agere . } Tales ergo debent esse leges , \\\hline
3.2.26 & porque el bien propra o es ordenado al bien comun . \textbf{ Conuiene que las leyes tales sean non } quales demanda el bien propre & et bonum priuatum ordinetur ad ipsum , \textbf{ oportet tales leges fieri } non quales requirit bonum priuatum , \\\hline
3.2.27 & Poque la ley aya uirtud e fuerça de obligar \textbf{ conuiene que sea publicada e pregonada . } Mas commo otra sea la ley natural e otra la positiua en vna manera se deue publicar la vna & ad hoc quod lex habeat vim obligandi , \textbf{ oportet eam promulgatam esse . } Sed cum alia sit lex naturalis , \\\hline
3.2.29 & quela ley demanda o que la ley nidga . \textbf{ Et algunas vezes conuiene que la regla se encorue } ala parte contraria & quam lex dictat : \textbf{ aliquando etiam oportet eam plicare ad partem oppositam , } et rigidius punire peccantem , \\\hline
3.2.30 & comunalmente non puede alcançar forma de beuir en punto . \textbf{ Por ende conuiene que } dessemeie alguons pecados & attingere punctalem formam viuendi , \textbf{ ideo oportet aliqua peccata dissimulare } et non punire lege humana , \\\hline
3.2.32 & que es en el regno e enla çibdat . \textbf{ conuiene que sea atal que biuna bien e uirtuosamente . } Et por ende assi conmo dize el philosofo en el terçero libro delas politicas & in ciuitate et regno , \textbf{ oportet esse talem , | quod viuat bene et virtuose . } Inde est ergo quod ait Philosop’ 3 Politicorum quod magis est ciuis abundans \\\hline
3.2.33 & uenta el philosofo en el quarto libro delas politicas \textbf{ que conuiene que sean tres partes dela çibdat . } Ca alguons son muy ricos . & Quarto Politicorum ait Philosophus , \textbf{ quod tres oportet | esse partes ciuitatis . } Nam alii quidem sunt opulenti valde , \\\hline
3.3.1 & por la qual cada vno sabe gouernar la casa e la conpaña . \textbf{ Conuiene que sea otra e departida de la sabiduria } por la qual cada vno sabe gouernar a ssi mismo . & per quam quis scit regere domum et familiam , \textbf{ oportet esse aliam a prudentia , } qua quis nouit seipsum regere . \\\hline
3.3.9 & ca los que estan en las huestes \textbf{ conuiene que sufran muchos males . } por la qual cosa si fueren y algunos muelles e mugerilles & erga necessitates corporis . \textbf{ Nam existentes in exercitu oportet multa incommoda tolerare : } quare si sint ibi aliqui molles , \\\hline

\end{tabular}
