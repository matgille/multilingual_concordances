\begin{tabular}{|p{1cm}|p{6.5cm}|p{6.5cm}|}

\hline
1.1.2 & aliter , et aliter operantur . \textbf{ Videmus enim quod quia timor , } et desperatio & e ende partidas gnisas \textbf{ Ca asy lo veemos | que por que el temor } e la esꝑança son departidas pasiones \\\hline
1.1.2 & et aggrediuntur hostes . \textbf{ Quod ergo dictum est de spe , et timore , } intelligendum est de aliis passionibus : & e entra en łlos rreziamente \textbf{ pues que asy es aquella | que dicho es dela esperança } e del temor esso mesmo se deue entender \\\hline
1.2.3 & in nobis passiones , \textbf{ timor , et audacia : } timor cum ab eo refugimus , & pues que assi es si de mal de futuro \textbf{ assi se leuantan en nos passiones de temor e de osadia . } temoͬ quando fuymos del mal futur . \\\hline
1.2.3 & timor , et audacia : \textbf{ timor cum ab eo refugimus , } audacia cum illud aggredimur . & assi se leuantan en nos passiones de temor e de osadia . \textbf{ temoͬ quando fuymos del mal futur . } Osadia quando acometemos algun mal futuro ¶ \\\hline
1.2.3 & erit ergo fortitudo \textbf{ ( quae est medietas timoris , et audaciae ) in irascibili habent esse . } Si vero huiusmodi passiones oriuntur & Pues que assi es sera la fortaleza \textbf{ que es meatad del temor | e dela osadia en el appetito enssannador } Ca el temor e la osadia han de seer \\\hline
1.2.13 & per quam regulentur in agendo . \textbf{ Cum igitur circa timores , } et audacias contingat & por la qual seamos reglados en las obras \textbf{ que auemos de fazer ¶ | pues que assi es commo los omes alguas vezes puedan } e les contezca de se auer derechamente \\\hline
1.2.13 & oportet dare virtutem aliquam \textbf{ circa timores , et audacias . } Accidit enim aliquos timere timenda , & por que contesce que algunos remen algunas cosas \textbf{ que han de temer e alas uegadas temen alguas cosas } que non han de temer . \\\hline
1.2.13 & Erit igitur Fortitudo virtus \textbf{ quaedam reprimens timores , } et moderans audacias . & Et por ende la fortaleza es vna uirtud \textbf{ que repreme los temores } e tienpra las osadias . \\\hline
1.2.13 & et moderans audacias . \textbf{ Reprimit enim Fortitudo timores , } ne per eos quis retrahatur ab eo , & e tienpra las osadias . \textbf{ Ca la fortaleza repreme los temores } por que por ellos non sea el omne rerenido \\\hline
1.2.13 & et etiam quia in periculis bellicis \textbf{ difficilius est reprimere timores , } quam moderare audacias : & Et ahun por que en los periglos delas batallas \textbf{ mas fuerte cosa es de repremer los temores } que de restenar las osadias . \\\hline
1.2.13 & principalius est in reprimendo \textbf{ timores contingentes } circa pericula talia , & que en los otros periglos . \textbf{ Et mas prinçipalmente esta esta uirtud | en repremir los temores } que acaesçen en los periglos delas faziendas \\\hline
1.2.13 & circa pericula bellica , \textbf{ reprimendo timores , } et moderando audacias : & cerca los periglos delas batallas \textbf{ por que son mas graues de sofrir | que los orros } Otrosi maguera que la fortaleza sea cerca los periglos delas batallas \\\hline
1.2.13 & principalius tamen est \textbf{ circa repressionem timorum , } quam circa moderationem audaciarum . & e refrenando las osadias . Enpero mas prinçipalmente es cerca aquellas cosas \textbf{ que repremen los temores } que cerca aquellas cosas \\\hline
1.2.13 & Cum ergo naturaliter tristia fugiamus , \textbf{ difficile est reprimere timores , } per quos tristia fugimus . & Et pues que assi es commo nos natural mente fuyamos dela tristeza \textbf{ graue cosa es de repmir los temores } por los quales fuyamos dela tristeza . \\\hline
1.2.13 & Fortitudinem principalius esse \textbf{ circa repressionem timorum , } quam circa moderantiam audaciarum . & mas prinçipalmente es cercas las cosas mas guaues bien dicho es \textbf{ que la fortaleza mas prinçipalmente es cerca del repremiento de los temores } que cerca el refrenamiento delas osadias ¶ \\\hline
1.2.13 & quod Fortitudo est \textbf{ circa timores , et audacias . } Magis tamen est & en el capitulo dela fortaleza \textbf{ dize } que la fortaleza es cerca los temores repremiendo los \\\hline
1.2.13 & Magis tamen est \textbf{ circa timores reprimendo eos , } quam circa audacias moderando ipsas & dize \textbf{ que la fortaleza es cerca los temores repremiendo los } e cerca las osadias res renando las \\\hline
1.2.13 & Sed quia difficilius est \textbf{ reprimere timores , } quam moderare audacias : & assi commo mas \textbf{ guaue cosa es de repremir los temores } que refrenar las osadias . \\\hline
1.2.13 & Fortitudo magis insistit \textbf{ ut reprimat timores , } quam ut moderet audacias : & la fortaleza mas es \textbf{ e mas esta en repremir los temores } que en reftenar las osadias \\\hline
1.2.13 & et plus repugnat , \textbf{ et contradicit timor fortitudini , } quam faciat audacia . & que en reftenar las osadias \textbf{ Et mas contradize el temor ala fortaleza que la osadia } ¶Pues que assi es porque non podemos en punto alcançar el medio entre la osadia e el temor . \\\hline
1.2.13 & attingere medium inter audaciam , \textbf{ et timorem : } declinandum est ad audaciam , & ¶Pues que assi es porque non podemos en punto alcançar el medio entre la osadia e el temor . \textbf{ por ende auemos de inclinar nos mas ala osadia } por que menos contradize ala fortaleza que el temor \\\hline
1.2.13 & quae minus repugnat Fortitudini , \textbf{ quam timor , } si volumus nos ipsos facere fortes . & por ende auemos de inclinar nos mas ala osadia \textbf{ por que menos contradize ala fortaleza que el temor } Et por ende si quisieremos fazer fuertes a nos mismos \\\hline
1.2.13 & quid est Fortitudo : \textbf{ quia est virtus reprimens timores , } et moderans audacias . & que cosa es la fortaleza \textbf{ ca es uirtud | que te pree me los temores } e refrena las osadias . \\\hline
1.2.13 & circa pericula bellica , \textbf{ et in reprimendo timores , } et in sustinendo pugnam . & Ca prinçipalmente es en las batallas \textbf{ e es en reprimiendo los temores . } Et en sufriendo \\\hline
1.2.13 & quae non tantum repugnat fortitudini , \textbf{ sicut timor . } Distinguit Philosophus 3 Ethicorum cap’ & Ca la osadia non contradize tanto ala fortaleza \textbf{ commo el temor } euedes saber \\\hline
1.2.14 & vel ut consequatur honores : \textbf{ sed timore poenae , } vel aliqua necessitate ductus aggreditur pugnam . & o por gauar honrras \textbf{ mas por temor de pena o enduzido } por alguna neçesidat acomete la fazienda . \\\hline
1.2.15 & quae possunt nos retrahere a bono rationis , \textbf{ sunt timores belli , } et pericula mortis , & que non siguamos el bien de tazon \textbf{ son los temores dela batalla } e los periglos dela muerte \\\hline
1.2.15 & quod sicut Fortitudo media est \textbf{ inter timores , et audacias : } quia qui omnia timet , & que assi commo la fortaleza es medianera \textbf{ entre los temores e las osadias . } Por que aquel que teme en todas las cosas \\\hline
1.2.15 & Nam sicut fortitudo est \textbf{ reprimens timores , } et moderans audacias , & Ca assi commo la fortaleza \textbf{ reprime los temotes } e refrena las osadias \\\hline
1.2.16 & magis voluntarie male agit , \textbf{ quam qui peccat ex timore . } Secundo hoc idem patet : & destenpranca mas de uoluntad \textbf{ fazemal que el peca con temor ¶ } Lo segundo esto mesmo se puede prouar \\\hline
1.2.16 & quia ( ut ait Philosophus ) \textbf{ timor obstupefacit , } et reddit naturam immobilem , et attonitam : & Ca asi commo dize el philosofo el temor faze al omne acometido e faze al omne \textbf{ que se non puede mouer } e que \\\hline
1.2.16 & nec voluntarie et deliberate agit quod agit . \textbf{ Tolerabilius est igitur peccare per timorem , } quam per intemperantiam : & e esta fuera de ssi non faz aquello que faze por uoluntad ñcon delibramiento . \textbf{ Et por ende mas de foyr | e de escusares de pecar } por temor o por miedo \\\hline
1.2.17 & quia est media \textbf{ inter timores et audacias , ideo est virtus reprimens timores , } et moderans audacias : & e por E ende es uirtud \textbf{ que repreme los temo eres } e tienpra las osadias . \\\hline
1.2.17 & Nam sicut quia fortitudo plus opponitur \textbf{ timori quam audaciae , } facimus nosipsos fortes , & Ca bien conmo la fortaleza mas \textbf{ comtradize al miedo que ala osadia . } Et nos fazemos a nos mismos fuertes declinando ala osadia \\\hline
1.2.27 & Haec autem est mansuetudo : \textbf{ quia sicut fortitudo est media inter timores et audacias , } sic mansuetudo est media inter iram , & que cosaes la mansedunbre . \textbf{ Ca assi commo la fortaleza es medianera | entre los miedos e las osadias } assi en essa misma manera es la manssedunbre medianera entre la sanna \\\hline
1.2.27 & secundum ordinem rationis . \textbf{ Quare sicut fortitudo reprimit timores , } et moderat audacias , & segunt orden de razon e de entendimiento . \textbf{ por la qual cosa assi commo la fortaleza | repreme los miedos } e tienpra las osadias en essa misma manera la manssedunbre repreme las sannas \\\hline
1.3.1 & sic dicere possumus quod sunt duodecim passiones : \textbf{ videlicet , amor , odium , desiderium , abominatio , delectatio , tristitia , spes , desperatio , timor , audacia , ira , et mansuetudo . } Computabatur enim supra mansuetudo inter virtutes : & conuiene saber amor e mal querençia e desseo . \textbf{ e aborrençia er delectacion . | e tristeza e esperança e desesperança e temor e osadia . } e sanna . e mansedunbre . \\\hline
1.3.1 & vel ut ab eo refugimus , \textbf{ et sic est timor . } Si vero malum sit praesens , & O en quanto ymos ael e le acometemos e assi es osadia . \textbf{ O en quanto foymos del e assi es temor . } Mas si el mal fuere presente esto es avn en dos maneras . \\\hline
1.3.2 & In tertio vero , spes , et desperatio . \textbf{ In quarto autem , timor , et audacia . } In quinto autem , ira , et mansuetudo . & Et en el terçero son la esperança e la desesperança . \textbf{ En el quarto son el temos e la osadia . } Et en el quanto son la sana et la manssedunbre . \\\hline
1.3.2 & cum sumantur respectu boni , \textbf{ praecedunt timorem , et audaciam , iram , et mansuetudinem , } quae sumuntur respectu mali . & Ca por que son tomadas \textbf{ por razon de bien son puestas primero que el temor e la oladia | e que la sanna e la monssedunbre } que son tomadas \\\hline
1.3.2 & quae sumuntur respectu mali . \textbf{ Timor autem , et audacia praecedunt iram , } et mansuetudinem . & por razon de mal . \textbf{ Mas el temor e la osadia son primero } que sasanna e la manssedunbre . \\\hline
1.3.2 & quam praesens : \textbf{ timor , et audacia , } quae sumuntur respectu mali futuri , & ante que sea presente . \textbf{ Et el temor e la osadia } que son tomadas \\\hline
1.3.2 & spes desperatione : \textbf{ timor audacia : } ira mansuetudine : & que la desesparaçion ¶ \textbf{ Et el temor primero | que la osadia¶ } Et la saña primero \\\hline
1.3.2 & per quam deficimus ab ipso . \textbf{ Timor autem praecedit audaciam . } Nam sicut quia appetitus per se & por la qual fallesçemos del bien \textbf{ Mas el temor es primero que la osadia . } Ca assi commo el apetito \\\hline
1.3.2 & sic quia refugere malum habet rationem boni , \textbf{ ideo timor per quem refugimus malum , } prior est audacia & por que fuyr del mal ha razon de bien \textbf{ por ende el temor } por el qual fuymos del mal es primero que la osadia \\\hline
1.3.3 & qui eam tollit . \textbf{ Timor ergo et odium , } et breuiter omnis passio & que tuelle la uida ¶ Et \textbf{ pues que assi es el temor | e la mal querençia } e breue mençe toda pasion \\\hline
1.3.5 & Spes enim primo est de bono : \textbf{ nam de malo non est spes , sed timor . } Secundo de arduo : & Lo primero que la espança es de algun bien \textbf{ ca si fuesse de mal | non seria esperança mas temor ¶ } Lo segundo la esperança es de algun bien abto e guaue . \\\hline
1.3.6 & quomodo se habere debeant \textbf{ contra timorem , et audaciam , } quae respiciunt futurum malum . & en qual manera se de una auer çerca la osadia \textbf{ e çerca el temor } que catan al mal \\\hline
1.3.6 & quo modo eos esse deceat timidos , et audaces . \textbf{ Timor autem si moderatus sit , } expediens est Regibus et Principibus . & e de ser osados \textbf{ por que el temor si fuere tenprado es conuenible alos Reyes e alos prinçipes . } Ca por temor tenprado todos los prinçipes deuen temer \\\hline
1.3.6 & expediens est Regibus et Principibus . \textbf{ Moderato enim timore omnes principantes timere debent , } ne aliquid insurgat in regno , & por que el temor si fuere tenprado es conuenible alos Reyes e alos prinçipes . \textbf{ Ca por temor tenprado todos los prinçipes deuen temer } que alguna cosa non se leunate \\\hline
1.3.6 & duplici via inuestigare , \textbf{ quod moderatus timor necessarius sit Regibus . } Prima via sumitur & nos podemos declarar en dos maneras \textbf{ que el | temortenprado es neçessario alos Reyes e alos prinçipes } ¶La primera se toma de parte del \\\hline
1.3.6 & Prima via sic patet : \textbf{ nam , ut dicitur 2 Rhetoric’ cap’ de timore , } Timor consiliatiuos facit , & la primera manera se puede assi mostrar . \textbf{ Ca assi commo es dicho en el segundo libro delas ethicas | en el capitulo del temor } que el temor nos faze auer consseio . \\\hline
1.3.6 & nam , ut dicitur 2 Rhetoric’ cap’ de timore , \textbf{ Timor consiliatiuos facit , } ex eo , quod aliquis ei timet , & en el capitulo del temor \textbf{ que el temor nos faze auer consseio . } Ca por que alguno teme alguna cosa \\\hline
1.3.6 & ut consiliatiui reddantur , \textbf{ habere aliquem moderatum timorem . } Secundo hoc idem inuestigare possumus & conuiene alos Reyes e alos prinçipes de auer algun temor tenprado \textbf{ sienpre tomado conseio sobrello ¶ } Lo segundo podemos esso mismo mostrar \\\hline
1.3.6 & ut opera diligentius operemur . \textbf{ Nam si moderatus adsit timor , } diligentius agimus opera , & temortenprado non solamente faze alos Reyes tomadores de conseio mas faze avn que fagan las obras mas acuçiosa mente . \textbf{ Ca si nos viniere algun temor tenprado mas } acuciosamente fazemos las obras \\\hline
1.3.6 & diligentius agimus opera , \textbf{ per quae fugere credimus timorem illum . } Ostensum est ergo , & acuciosamente fazemos las obras \textbf{ por las quales queremos foyr de aquel temor } Et por ende mostrado es que los Reyes e los prinçipes deuen auer temor tenprado . \\\hline
1.3.6 & quod decet Reges , \textbf{ et Principes moderatum habere timorem . } Attamen immoderate timere & por las quales queremos foyr de aquel temor \textbf{ Et por ende mostrado es que los Reyes e los prinçipes deuen auer temor tenprado . } Enpero temer destenpradamente en ninguna manera \\\hline
1.3.6 & nullo modo decet eos . \textbf{ Immoderatus enim timor quatuor habere videtur , } quae omnino derogant regno . & non conuiene alos Reyes . \textbf{ t te paresçe que el temor | destenprado ha quatro cosas } que del todo ponen mengua en el gouernamiento del regno . \\\hline
1.3.6 & quae omnino derogant regno . \textbf{ Nam timor immoderatus primo } reddit hominem immobilem , et contractum . & que del todo ponen mengua en el gouernamiento del regno . \textbf{ Ca el temor destenprado | e sin razon primero faze } al ome ser encogido \\\hline
1.3.6 & et redditur immobilis . Quare si indecens est caput regni siue Regem esse immobilem et contractum , \textbf{ indecens est ipsum timere timore immoderato . } Secundo hoc est indecens , & que la cabeca del regno o el Rey \textbf{ sea tal que se non mueua | e sea encogido cosa muy desconuenjble es al Rey de temer } e de auer temor destenprado e sin razon¶ \\\hline
1.3.6 & Secundo hoc est indecens , \textbf{ quia immoderatus timor facit hominem inconciliatiuum . } Cum enim quis immoderate timet , & por que el \textbf{ temordestenprado | e sin razon fazen al omne sin conseio . } Ca quando alguno teme destenpradamente \\\hline
1.3.6 & et ut Rex sit inconsiliatiuus ; \textbf{ indecens est ipsum timere immoderato timore . } Tertio immoderatus timor reddit hominem tremulentum . & e que el Rey sea sin conseio . \textbf{ Cosa desconuenible es mucho al regno de temer | por temor } destenprado et sin razon ¶ \\\hline
1.3.6 & indecens est ipsum timere immoderato timore . \textbf{ Tertio immoderatus timor reddit hominem tremulentum . } Nam propter timorem calore progrediente ad interiora , & destenprado et sin razon ¶ \textbf{ Lo terçero el temor deste prado | e sin razon faze al ome tremuliento e tremedor } Ca por el temor la calentura natural \\\hline
1.3.6 & Tertio immoderatus timor reddit hominem tremulentum . \textbf{ Nam propter timorem calore progrediente ad interiora , } exteriora membra frigida manent . & e sin razon faze al ome tremuliento e tremedor \textbf{ Ca por el temor la calentura natural | tornase alos mienbros de dentro } e los mienbros de fuera fincan frios . \\\hline
1.3.6 & qui debet esse virilis et constans , \textbf{ inconueniens est ipsum timere immoderato timore . } Quarto immoderatus timor & la qual cosa deue seruaron costante e firme desconuenible cosa es ael de temer \textbf{ por temor destenpdo | e sin razon¶ } Lo quarto el temor destenprado \\\hline
1.3.6 & reddit hominem inoperatiuum . \textbf{ Nam homo propter timorem immoderatum tremens } et obstupefactus immobilitatur , & que non obre . \textbf{ Ca el omne por el temor destenprado | e sin razon } trieme e esta atomeçido \\\hline
1.3.6 & si Rex sit inoperatiuus , \textbf{ et imperare non valeat propter immoderatum timorem , } toti regno praeiudicium gignitur : & si el rey fuere tal que no nobre \textbf{ e non pueda mandar | por el temor destenprado } e sin razon \\\hline
1.3.6 & non decet \textbf{ Regem immoderato timore timere . } Viso quomodo Reges se habere debeant ad timorem , & non conuiene alos Reyes de temer \textbf{ e por temor deste prado | e sin razon . } ¶ visto en qual manera los Reyes se deuen auer al temor \\\hline
1.3.6 & Regem immoderato timore timere . \textbf{ Viso quomodo Reges se habere debeant ad timorem , } quia difficilius est reprimere timorem , & e sin razon . \textbf{ ¶ visto en qual manera los Reyes se deuen auer al temor } por que cosa mas guaue es de repremir el temor que tenprar la osadia \\\hline
1.3.6 & Viso quomodo Reges se habere debeant ad timorem , \textbf{ quia difficilius est reprimere timorem , } quam moderare audaciam , & ¶ visto en qual manera los Reyes se deuen auer al temor \textbf{ por que cosa mas guaue es de repremir el temor que tenprar la osadia } assi commo fue dicho de suso \\\hline
1.3.6 & quia tunc nihil aggreditur . \textbf{ Moderate ergo se habere ad timorem , } et ad audaciam Regibus et Principibus omnino congruit . & por que estonçe non acometria ninguna cosa¶ Et pues \textbf{ que assi es auersse tenpradamente al temor } e ala osadia es cosa en todo en todo conuenible alos Reyes e alos prinçipes . \\\hline
1.3.9 & quod sunt quatuor principales ; \textbf{ ut spes , timor , gaudium , et tristitia . } Has autem esse magis principales , & signiendo la doctrina de nuestros anteçessores podemos dezer que son las quatro prinçipales . \textbf{ assi commo la esperança Et el temor . | Et el gozo . } Et la tristeza . \\\hline
1.3.9 & sumptae autem respectu mali , \textbf{ ordinari videntur ad timorem , et tristitiam . } Nam passio sumpta respectu boni , & en conparaçion de algun mal \textbf{ son ordenadas al temor e ala tristeza . } Ca la passion tomada en conparaçion de algun bien . \\\hline
1.3.9 & vel abominationem , \textbf{ et terminatur in timorem , } si malum illud sit futurum : & e para lo aborrescer \textbf{ e terminasse en temor } si aquel mal fuerefuturo que ha de uenir . \\\hline
1.3.9 & et iam adeptum . \textbf{ Timor autem , et tristitia sunt passiones principales : } quia ad eas ordinantur & ¶ Et pues que assi es el temor \textbf{ e la tristeza | son avn passiones prinçipales } por que son ordenadas aellas las o tris passiones . que son tomadas en conparaçion de mal \\\hline
1.3.9 & de praesenti est gaudium : \textbf{ de malo futuro est timor , } de praesenti vero est tristitia . & e del presente es el gozo . \textbf{ Et del mal futuro | que es de venir es el temor . } Mas del presente es latsteza¶ \\\hline
1.3.9 & esse principales passiones respectu concupiscibilis . \textbf{ Spes autem et timor sunt principales passiones respectu irascibilis . } Nam cum irascibilis tendit & en conparaçion del appetito cobdiçiador . \textbf{ Mas la esperança e el temor son passiones prinçipales | en conparacion del appetito enssañador . } Por que el appetito enssannador va a bien o a mal en quanto es alto e grande . \\\hline
1.3.9 & cum est futurum et timetur : \textbf{ spes et timor sunt principales passiones respectu irascibilis . } Sed cum ex passionibus diuersificari habeant opera nostra , & quando es futuro que es de uenir e es tenido . \textbf{ Et por ende la esperança e el temor son passiones prinçipales | en conparacion del appetito enssannador . } Mas commo las nr̃as obras ayan de ser departidas \\\hline
1.3.9 & circa delectationem et tristitiam , \textbf{ et circa spem et timorem , } patefactum est per Capitula supra dicta . & e cerca la tristeza \textbf{ e cerca la esperança | e cerca el temor mostrado es } en los capitulos ya dichos de suso . \\\hline
1.3.10 & Enumerabantur supra duodecim passiones , \textbf{ videlicet , amor , odium , desiderium , abominatio , delectatio , tristitia , spes , desperatio , timor , audacia , ira , et mansuetudo . } Sed praeter omnes has passiones Philosop’ & Et reçia Delectacion . \textbf{ Et tristeza . | Esperança mal querençia . } Desseo . \\\hline
1.3.10 & et gratia reducuntur ad amorem : \textbf{ verecundia ad timorem : } inuidia , et misericordia , et nemesis siue indignatio & Ca el zelo e la gera se pueden adozir al amor . \textbf{ Et la uerguença se puede adozir al temor e al esꝑanto . } La inuidia e la miscderia en \\\hline
1.3.10 & Zelus ergo et gratia reducuntur ad amorem . \textbf{ Sed verecundia reducitur ad timorem . } Dupliciter autem quis timere potest , & e la gera son aducho sal amor \textbf{ Mas la uerguença reduze se al temor } por que dos cosas pue de cada vno temer \\\hline
1.3.10 & Nihil est aliud verecundia , \textbf{ quam timor inhonorationis , vel inglorificationis . } unde verecundia erubescentia nominari consueuit , & Et poͬende non es otra cosa uerguença \textbf{ si non | temorde desonrra o temor de non glorificacion } qua non aya eglesia . \\\hline
1.3.10 & et membra remanent pallida : \textbf{ quia timor consurgit ex eo , } quod quis se credit & e los mienbros de fuera fincan amariellos \textbf{ por que el temor se leunata de aquesto } que cuyda alguno \\\hline
1.3.10 & ex eo quod quis se credit amittere exteriora bona . \textbf{ Duplex ergo est timor , } unus amittendi vitam , & que pierde los bienes de fuera . \textbf{ Et pues que assi es dos son los temores . } El vno es que te meꝑ̃der la uida \\\hline
1.3.10 & per quae quis erubescit . \textbf{ Timor ergo corruptiui , } et amittendi bona interiora & por el qual temor alguno se enbermeieçe ¶ \textbf{ Et pues que assi es el temor | coirupartiuo } que es temor de perder los bienes de dentro \\\hline
1.3.10 & retinet sibi nomen commune , \textbf{ et dicitur timor , } sed timor amittendi gloriam , & Retiene en ssi nonbre comun \textbf{ e es dichon temor . } Mas el temor que es temor de perder la eglesia \\\hline
1.3.10 & et dicitur timor , \textbf{ sed timor amittendi gloriam , } et honorem , & e es dichon temor . \textbf{ Mas el temor que es temor de perder la eglesia } e la honrra ha nonbre espeçial \\\hline
1.3.10 & et dicitur verecundia , vel erubescentia . \textbf{ Verecundia ergo est quidam timor , } et reducitur ad timorem . & e es dicha uerguença o herubesçençia \textbf{ que quiere dezir en bermegecimiento . | Et pues que assi es la uergunença es temor espeçial } e reduzesse al tenmor general \\\hline
1.3.10 & Verecundia ergo est quidam timor , \textbf{ et reducitur ad timorem . } Viso , quomodo zelus et gratia reducuntur ad amorem : & Et pues que assi es la uergunença es temor espeçial \textbf{ e reduzesse al tenmor general } Et pues que assi es iusto \\\hline
1.3.10 & Viso , quomodo zelus et gratia reducuntur ad amorem : \textbf{ et quomodo verecundia reducitur ad timorem . } Restat videre , & Et pues que assi es iusto \textbf{ en qual manera el zelo | que es amor grande } e la grason aduchas al amor \\\hline
1.3.11 & Per fortitudinem vero debite se habebunt \textbf{ circa audaciam , et timorem . } Nam ille est fortis , & Mas por la fortaleza se aur̃a conueniblemente los Reyes cerca la osadia \textbf{ e cerca el temor . } por que aquel es dicho fuerte \\\hline
1.3.11 & infra diffusius tractabitur . \textbf{ Agetur enim in tertio de timore , amore , misericordia , et de aliis , } ut rei cognoscentia postulabit . & lo tractaremos mas conplida mente . \textbf{ Ca diremos en el terçero libro del temor et del amor e dela miscd̃ia | e delanso tris passiones } segunt que conuiene a cada cosa . \\\hline
1.4.1 & ( ut dicebatur ) \textbf{ est timor inglorificationis . } Cum ergo iuuenes , & Ca la uerguença \textbf{ assi commo es dicho es temor de non auergłia nin honra . } Et pues que assi es por que los mançebos \\\hline
1.4.1 & quae important ignominiam et inhonorationem : \textbf{ et quia erubescentia est timor inglorificationis , } iuuenes de facili erubescunt . & que trahen a denuesto e a desonira . \textbf{ Et por que la uerguença es temor de non auer eglesia } nin honrra los mançebos de ligero toman uerguença \\\hline
1.4.3 & Verecundia ergo , \textbf{ cum sit timor inhonorationis , } non competit senibus ; & en el segundo libro dela Rectorica \textbf{ Ca por que la uerguença es temor de desonera non pertenesçe alos uieios } por que may orcuidado han del prouecho \\\hline
2.1.18 & cum uerecundia sit \textbf{ timor de inglorificatione et de amissione laudis , } mulieres communiter sunt uerecundae , & qen la bondat \textbf{ que los omes por la qual cosa commo la uirguença sea temor de non auer eglesia o de ꝑder alabança } e las mugers son comunalmente uergonçosas \\\hline
2.1.18 & sequitur eas uerecundas esse : \textbf{ quia uerecundia est quidam timor , } ut superius dicebatur . & Siguese que ellas sean uergon cosas \textbf{ por que la uerguença es algun temor } assi commo dicho es de suso . \\\hline
3.1.12 & nam ( ut dicebatur supra ) \textbf{ frigiditas viam timori praeparat ; } frigidi enim est costringere et retrahere ; & commo dela frialdat dela conplission \textbf{ ca assi commo es dicho de suso la frialdat apareia carrera al temor } por que el frio a restrennir e apretar \\\hline
3.2.12 & stare faciebat . \textbf{ Et cum frater eius timore horribili inuaderetur , } timens ab acuto gladio vulnerari , & poñuallesteros con ballestas armadas contrael . \textbf{ Et estonçe commo aquel su hͣrmano tomasse grant espanto } e grant temor \\\hline
3.2.13 & quare subditi tyrannis insidiantur . \textbf{ Prima est propter timorem . } Nam multi pusillanimes existentes , & por que los sbraditos asecha alos tisanos ¶ \textbf{ La primera es por temorça munchons } que son de flaco coraçon \\\hline
3.2.13 & bestiae ergo \textbf{ ut plurimum timore compulsae insidiantur homini et inuadunt eum : } hoc ergo modo multotiens subditi insidiantur tyranno , & e pues que asi es las bestias enla mayor parte cos cringidas \textbf{ por temor assecha al omne | e acometenlo } e por ende en esta misma manera \\\hline
3.2.15 & nam corruptiones longe secundum rem , \textbf{ prope autem secundum timorem politiam saluant : } ciues enim magis sunt subiecti Principi & que es puesto enlos omans guardan \textbf{ e saluna lo poliçia | e el gouernamiento dela çibdat } ca los çibdadanos son mas subiectos al prinçipe \\\hline
3.2.15 & et plus ei obediunt , \textbf{ si ex timoribus corruptionem timeant . } Guerra enim exterius tollit seditiones intrinsecas , & e meior le obedesçen \textbf{ e se temerien de los de auer danno de los enemigos de fuera } por quela guerra de fuera tira las discordias \\\hline
3.2.15 & quare talibus ut debite se habeant \textbf{ semper incutiendi sunt timores bellici , } et correptiones extrinsecae . & por que se ayan conuenible \textbf{ sienpre les son de poner temores de guerras } e temores de dannos de fuera . \\\hline
3.2.15 & ne aliquod inconueniens accidat circa amatum . \textbf{ Timor autem consiliatiuos facit , } ut dicitur 2 Rhet’ & e mala çerca aquello que ama \textbf{ ca el temor faze alos omes tomar conseio | e ser sabios } assi commo dize el philosofo \\\hline
3.2.17 & Dicebatur enim supra , \textbf{ quod timor consiliatiuos facit : } qui ergo consiliatur , & Ca dicho fue desuso \textbf{ que el temor faze alos omes tomar conseio . } Et por ende paresçe \\\hline
3.2.20 & quantam si iudicaret eos arbitrio proprio , \textbf{ ne iudex timore inimicitiae inclinatus differat } debito fini iudicia mancipare , & por su aluedrio ppreo . \textbf{ por que el uiez non se incline . | por temor de enemistança a alongar los iuyzios } e non los traer a su fin . \\\hline
3.2.32 & Inde est igitur \textbf{ quod timore poenae multi desinunt malefacere , } et assuescunt ad operationes bonas : & Et dende viene que \textbf{ por temor de pena | muchs dexan de fazermal } e acostunbran se a fazer buenas obras . \\\hline
3.2.34 & ab operibus sceleratis , \textbf{ saltem timore poenae retrahatur ab illis . } Expediens enim fuit regno et ciuitati & por El xx castigos de sus padres e de lus amigos \textbf{ si al que non por temor de pena se quite de aquellas malas obras . } Et por ende cosa conuenible fue al regno \\\hline
3.2.36 & sint magis boni et virtuosi , \textbf{ quam si hoc facerent timore poenae , } et ne punirentur ; & e al Rey son mas uirtuosos \textbf{ e mas bueon ssi esto assi fazen que si lo fiziessen | por temor de pena } e por que non fuessen condep̃nados . \\\hline
3.2.36 & quam timeri ab eis , \textbf{ et quod timore poenae cauere sibi ab actibus malis . } Utrumque enim est necessarium , & que por temor dellos \textbf{ nin que por temor de pena se escusen de fazer malas obras . } Et pues que assi es estas dos cosas ser temidos \\\hline
3.2.36 & oportuit ergo aliquos inducere ad bonum , \textbf{ et retrahere a malo timore poenae . } Elegibilius tamen est amari , & enduxiessemosa bien \textbf{ e arredrassemos del mal | por temor de pena . } Enpero mas de escoger es alos Reyes \\\hline

\end{tabular}
