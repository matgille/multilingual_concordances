\begin{tabular}{|p{1cm}|p{6.5cm}|p{6.5cm}|}

\hline
1.4.5 & si ab antiquo affluebat diuitiis . \textbf{ Cum ergo semper sit dare initium , } in quo genitores alicuius ditari inceperunt : & si de antigo tienpo abondo en riquezas . \textbf{ Et pues que assi es comm sienpre ayamos de dar comienço } en que los padres de alguons comne caron de se enrriqueçer \\\hline
3.2.31 & ubi dicitur quod leges humanae contrariae sunt iuri naturali ; \textbf{ quia iure naturali ab initio homines liberi nascebantur . Seruitus ergo est contra naturam , } inquantum natura non induxit seruitutem ; & que las leyes humanales contrarias son al derecho natraal . \textbf{ Ca de comienço todos los omes nascian forros e libres . } Et por ende la seruidunbre escontra la natura \\\hline

\end{tabular}
