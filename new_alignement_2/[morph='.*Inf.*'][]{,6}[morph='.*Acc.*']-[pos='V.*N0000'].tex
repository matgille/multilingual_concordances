\begin{tabular}{|p{1cm}|p{6.5cm}|p{6.5cm}|}

\hline
1.1.1 & suis subditis imperare , \textbf{ oportet doctrinam hanc extendere usque ad populum , } ut sciat qualiter debeat & E si por qual manera Deuen mandar a los sus Subditos \textbf{ conujene esta doctrina | e esta sçiençia estender la fasta el pueblo } por que Sepa commo ha de obedesçer a sus prinçipes \\\hline
1.1.6 & nulla tamen delectatio est essentialiter ipsa felicitas , \textbf{ licet possit esse aliquid felicitatem consequens , } sed hoc declarare non est praesentis negocii . & por si mesma \textbf{ maguera que se pueda conseguir | ala feliçidat e ala bien andança . } Mas declarar esto non parte nesçe a esta arte presente . \\\hline
1.1.9 & vel si sit in populis gloriosus . \textbf{ Non igitur debet Rex se credere esse beatum , } si sit in gloria apud homines : & o si es głioso en los pueblos ¶ \textbf{ Et pues que assi es el rey | non deue creer } que es bien auenturado \\\hline
1.1.12 & quomodo deceat regiam maiestatem \textbf{ ponere suam felicitatem } in actu prudentiae , & en qual manera conuenga ala Real magestad \textbf{ de poner la primera feliçidat } en las obras de pradençia . \\\hline
1.2.5 & ad id quod ratio vetat : \textbf{ et oportet dare virtutem aliam , } ne passiones retrahant nos ab eo , & nin inclinar a aquelo que uieda la razon ¶ \textbf{ Et otrosi nos conuiene de dar otra uirtud } por la qual las passiones non nos pueden arredrar \\\hline
1.2.8 & quia nulli agenti hoc est possibile , \textbf{ sed decet Regem habere praeteritorum memoriam , } ut possit ex praeteritis cognoscere , & Ca esto ninguno non lo pie de fazer . \textbf{ Mas conuiene al Rey de auer memoria delans cosas passadas | por que pue da } por las cosas passadas conosçer e tomar \\\hline
1.2.10 & sed quia ea lex praecipit , \textbf{ et vult implere legem , } iustus legalis est . & mas en quanto las manda fazer la ley \textbf{ e el quiere conplir la ley es dicho iusto legal . } Et pues que assi es el iusto legal \\\hline
1.2.13 & et in aegritudinibus , \textbf{ et in aliis circa quae conuenit esse pericula . } Rursus in periculis bellorum homines diuersimode se habent . & e en las enfermedades \textbf{ e en los otros negoçios | en los quales pueden conteçer periglos . } Otrosi en los periglos delas faziendas \\\hline
1.2.16 & et acquirere temperantiam , \textbf{ quam sit acquirere fortitudinem . } Quod autem magis voluntarie peccet intemperatus & Mas que el \textbf{ destenprado pequemas de voluntad que el temeroso puede se demostrar } por dos razons¶ \\\hline
1.2.17 & esse magis circa expendere \textbf{ et circa tribuere pecuniam aliis , } quam circa proprios redditus custodire . & que la franqueza es mas en espender \textbf{ e partir el auer alos otros que en guardar las rentas propias } ¶ \\\hline
1.2.19 & ista decenter se habere debet : \textbf{ non tamen aeque principaliter intendere debet circa omnia ista . } Nam principaliter et primo , & conueniblemente çerca estas quatro cosas . \textbf{ Mas enpero non deue entender egualmente nin prinçipalmente cerca estas quatro cosas . } Ca primero e prinçipalmente deue seer el omne magnifico \\\hline
1.2.22 & Ad pusillanimem enim pertinet \textbf{ nescire fortunas ferre . } Ideo Andron’ Perip’ ait : & Mas al pusill animo \textbf{ e de flaco coraçon pertenesçe non saber sofrir buenas uenturas . } Por ende dize andronico el sabio philosofo \\\hline
1.2.24 & et honoris amatiuos . Reges enim et Principes decet honores diligere modo quo dictum est ; \textbf{ videlicet , ut diligant et cupiant facere opera , } quae sint honore digna . & e alos prinçipes amar las honrras \textbf{ en la manera que dich̃ones de suso . | Conuiene saber que amen e cobdicien fazer lobras } que sean dignas de honrra . \\\hline
1.2.27 & Nam cum ira peruertat iudicium rationis , \textbf{ non decet Reges et Principes esse iracundos , } cum in eis maxime vigere debeat ratio et intellectus . Sicut enim videmus & tristorna el iuyzio dela razon \textbf{ e del entendimiento non conuiene alos Reyes | et alos prinçipes de seer sannudos } por que en ellos mayormente deue seer apoderada la razon e el entendemiento \\\hline
1.2.29 & vel promittendo aliis maiora quam faciant . \textbf{ Immo tanto magis decet Reges et Principes cauere iactantiam , } quanto plures habent incitantes ipsos ad iactantiam , & nin prometiendo alos otros mayores cosas que faran . \textbf{ Mas por tanto conuiene alos Reyes | e alos prinçipes de escusar } e de foyr el alabança \\\hline
1.2.32 & Incontinere ergo est aggredi pugnam , \textbf{ et in pugna non posse se tenere , } sed deficere . & Et por enerde el non contener se es acometer algua batalla \textbf{ e en aquella batalla non se poder tener mas fallesçer en el ła¶ } En el terçero guado de malos son los destenprados . \\\hline
1.3.3 & ideo necessarium est ostendere \textbf{ quomodo nos habere debeamus ad illas . } Oportebat ergo enumerare omnes passiones , & por ende escoła neçesaria de mostrar \textbf{ en qual manera nos deuemos auer a aquellas passiones } Et por ende conuena de contar tondas las passiones \\\hline
1.3.4 & tanto magis decet Reges et Principes , quanto magis eos decet \textbf{ habere curam de bono regni et communi . } Quae sunt autem illa & quanto mas conuiene a ellos \textbf{ de auer cuydado del regno e del bien comun . } Mas quales cosas son aquellas que guardan el regno en buen estado \\\hline
1.3.6 & Ergo si inconueniens est \textbf{ Regem esse tremulentum , } qui debet esse virilis et constans , & e viene les luego el tremer ¶ \textbf{ Et pues que assi es si cosa desconuenible es al Rey de ser tremuliento } la qual cosa deue seruaron costante e firme desconuenible cosa es ael de temer \\\hline
1.3.8 & Viso , quomodo Reges , \textbf{ et Principes se habere debeant ad delectationes : } videre restat , & ¶ Visto en qual manera los Reyes \textbf{ e los prinçipes se deuen auer alas delectaçiones } finça deuer en qual maneras \\\hline
1.4.1 & est laudabile simpliciter : \textbf{ uidemus enim quod esse furibundum , } est laudabile in cane , & Mas qual si quier cosa \textbf{ que sea de loar en vno | e non en otro o es de loar } por alguna condicion non es de loar sinple mente . \\\hline
1.4.3 & et incurrerent maliuolentiam subditorum . \textbf{ Tertio non decet eos esse timidos et pusillanimes , } immo fortes et magnanimos : & e caerien en malquerençia de los sus subditos ¶ \textbf{ Lo terçero non conuiene a ellos de ser tem̃osos e de flacos coraçones } mas conuiene les de ser fuertes e de grandes coraçones \\\hline
1.4.5 & quia hoc faciunt elati et superbi : \textbf{ sed debemus appetere opera honore digna , } quod faciunt virtuosi et magnanimi . & Ca esto fazen los orgullolos et los sob̃uios . \textbf{ Mas deuemos dessear las obras | que son dignas de honrra } la qual cosa fazen los uirtuosos \\\hline
1.4.7 & et magnam pronitatem habent , \textbf{ ut sequantur praedictos mores . } Iuuenes ergo et senes non indignentur , & e han grand disposiçion \textbf{ para segnir las costunbres sobredichͣs . } Et por ende non se deuen enssonnar los mançebos \\\hline
2.1.3 & intendere bonum regni : \textbf{ spectat ad unumquemque ciuem scire regere domum suam , } non solum inquantum huiusmodi regimen est bonum proprium , & entenderal bien del regno . \textbf{ Por ende pertenesçe a cada vn | çibdadano saber gouernar su casa } non solamente en quanto este gouernamiento es bien propreo suyo \\\hline
2.1.8 & vel ex parte amicitiae naturalis , \textbf{ quae debet esse inter virum et uxorem . } Secunda vero ex parte prolis . & o de parte dela amistança natural \textbf{ que deue ser entre el marido e la muger . } ¶ La segunda razon se toma de parte dela generaçion de los fijos \\\hline
2.1.12 & quae sint ex nobili genere . \textbf{ Secundo propter esse pacificum quaerenda est amicorum multitudo . } Nam pax inter homines se habet & que sean de noble linage¶ \textbf{ Lo segundo por el bien dela paz es de querer en el mater moino la muchedunbe de los amigos . } Ca la paz se ha entre los omes \\\hline
2.1.15 & patet aliud esse regimen coniugale quam seruile : \textbf{ et non esse utendum uxoribus tanquam seruis . } Secunda via ad inuestigandum hoc idem , & Et pues que assi es de parte de la orden natural paresçe que otra cosa es el gouernamiento del marido ala mug̃r \textbf{ que del señor al sieruo . | Et paresçe que non deuen vsar los omes delas mugers } assi commo de sieruas ¶ \\\hline
2.1.19 & et ciuili potentia , \textbf{ decet inquirere matronas } aliquas boni testimonii & e en poderio çiuil \textbf{ conuiene les de bulcar buenas mugers } e antiguas de buen testimoino prouadas \\\hline
2.2.1 & dare animalibus ora et alia organa , \textbf{ per quae possunt sumere cibum et nutrimentum . } Quare si patres sunt causa filiorum , & e todos los organos e instrumentos \textbf{ por los quales puedan tomar la uianda qual les conuiene . } Por la qual cosa sy los padres son comienço \\\hline
2.2.5 & qui vero mala , in ignem aeternum . \textbf{ Debent ergo omnes ciues solicitari circa proprios filios , } ut ab infantia instruantur in hac fide ; & assi que aquellos que bien fezieron yran ala uida perdurable . \textbf{ Mas aquellos que fizieron mal yran al fuegon del infierno ¶ Et pues que assi es todos los çibdadanos deuen ser acuçiosos de sus fiios } por que enla moçedat sean enssennados en esta fe \\\hline
2.2.9 & Secundo decet \textbf{ ipsum esse prouidum futurorum . } Nam sicut aliorum director debet & Lo segundo le conuiene \textbf{ que sea prouiso en las cosas | que han de uenir . } Ca assi commo el que ha degniar los otros \\\hline
2.2.13 & Frustra ergo , \textbf{ cum quis vult audire alium , } retinet os apertum . & Ca el omne non oye con la boca mas por el oreia . \textbf{ Et pues que assi es quando alguno quiere oyr al otro } en vano tiene la boca abierta . \\\hline
2.2.17 & Restat videre , \textbf{ quomodo solicitari debeant circa eos , } ut habeant ordinatum appetitum . & por que ayan el cuerpo bien ordenado \textbf{ finca de ver en qual manera de una auer cuydado dellos } por que ayan el appetito bien ordenado . \\\hline
2.3.1 & talia instrumenta cognoscere . \textbf{ Quare volens tradere notitiam } de arte gubernationis domus , & e pertenesçe al texedor de conosçer tales estrumentos . \textbf{ Por la qual cosa el que quisiere dar conosçimiento del arte del gouernamiento dela casa } deue determinar de los hedifiçios \\\hline
2.3.8 & detestabilius est in Rege \textbf{ non habere ueram aestimationem } de fine quam in populo , & mas de denostares enl Rey \textbf{ de non auer uerdadera } estimaçonn dela fin \\\hline
2.3.11 & si volunt naturaliter Dominari , \textbf{ prohibere usuras , } ne fiant eo & si quisieren ser señors natalmente \textbf{ de defender las usuras } que non se fagan \\\hline
2.3.16 & non magnam curam habent annexam , \textbf{ congregari possunt officia et magistratus , } ita quod eidem diuersa officia committantur . & non han grand cura anexa \textbf{ pueden se muchos ofiçios | e muchos maestradgos ayuntar en vno . } Assi que avna perssona sean acomnedados departidos ofiçios \\\hline
2.3.20 & etiam \textbf{ et ipsos participare virtutes et bonos mores . } Sed si recumbentes , & Mas alos que son assentados en las mesas \textbf{ conuiene de escusar muchedunbre de palabras } por que non sea tirada la ordenn natural \\\hline
3.1.7 & est , quia dixerunt mulieres \textbf{ instruendas esse ad opera bellica , } et debere bellare & es que dixieron \textbf{ que las mugers deuian ser enssennadas | alas obras dela batalla } e que deuian batallar e guerrear \\\hline
3.1.9 & et quasi quaedam praeambula ad sequentia . Volumus autem in hoc capitulo ostendere , \textbf{ quod non expedit ciuitati habere omnia communia } ut Socrates ordinauit : & mas nos queremos en este capitulo mostrar primeramente \textbf{ que conuiene ala çibdat | quer todas las cosas comunes } assi commo socrates ordeno . \\\hline
3.1.12 & et dare magnos ictus , \textbf{ expedit eos habere magnos humeros et renes } ad sustinendum armorum grauedinem , & e ayan de dar grandes colpes \textbf{ conuieneles de auer fuertes honbros e fuertes rennes } para sofrir la pesadura delas armas \\\hline
3.1.15 & sicut seipsum : \textbf{ sic debent diligere uxores , filios , } et possessiones aliorum , & assy commo assi mesmo . \textbf{ En essa misma manera deue amar los fijos e las mugers } e las possessiones de los otros çibdadanos \\\hline
3.1.20 & tangentes diuersa genera personatum . \textbf{ Primo enim dictus Phil’ deferre fecit statuendo impossibilia . } Nam statutum de distinctione ciuium stare & tanniendo departidos linages de perssonas . \textbf{ Ca lo primero el dicho philosofo fallesçio | establesçien do establesçimientos que non podian ser nin estar en vno . } ca el establesçimiento del departimiento de los çibdadanos \\\hline
3.2.5 & quare si Rex videat \textbf{ debere se principari super regnum non solum ad vitam , } sed etiam per haereditatem in propriis filiis , & Por la qual cosa si el Rey viere \textbf{ que deue regnar sobre el regno | non solamente en su uida } mas avn por heredat en sus fijos . \\\hline
3.2.7 & sed etiam satagit \textbf{ impedire eorum maxima bona . } Tangit autem Philosophus 5 Polit’ tria maxima bona , & de aquellos que son en el regno \textbf{ mas avn esfuercasse para enbargar los bienes dellos } e tanne espho en el quinto libro delas politicas muy grandes tres bienes \\\hline
3.2.10 & Octaua , est procurare bella , \textbf{ mittere bellatores ad partes extraneas , } et semper facere bellare & ¶ La . viij n . cautela del tirano \textbf{ es procurar guerras e enbiar | guerrasa partes estrannas } e sienpre faze lidiar sus çibdadanos \\\hline
3.2.14 & dicens , \textbf{ Tyrannidem corrumpi a se , } a tyrannide alia , et a regno . & Ca cuenta el phon enel quinto libro delas politicas tres maneras dela corrupçion dela tiranja \textbf{ e dize que la tiranja corrope de si mismar coronpese desta çirana } e corronpese por El regno ¶ \\\hline
3.2.16 & utrum debeat \textbf{ sanare egrum } sed hoc accipit & por su fin non toma conseio \textbf{ si deua sanar el doliente } mas esto toma \\\hline
3.2.19 & quam alia : \textbf{ debet ergo adhiberi consilium , } ut circa talia maior custodia praebeatur . & para fazer mal que los otros . \textbf{ e por ende deue ser tomado consseio } por que en tales sea puesta mayor guarda . \\\hline
3.2.21 & et quod teneant supremum gradum in iudicando , \textbf{ quae debent tenere infimum . } Peruertitur ibi talis ordo , & e tener elguado primero en iudgar \textbf{ aquellos que deuen tener el postrimero } e assi se trastorna \\\hline
3.2.25 & Poterit ergo inclinatio naturalis \textbf{ sequi naturam hominis } vel ut homo est , & Et por ende la inclinacion natural \textbf{ puede seguir la natura del ome } o en quanto es omne o en quanto conuiene con todas las \\\hline
3.2.28 & si vero eleuando eam vellet purgare domum \textbf{ vel facere aliquod aliud opus pium , } propter bonam intentionem operantis , & para alinpiat la casa \textbf{ o para fazer alguna otra obra buean } por la buena entençion \\\hline
3.2.31 & innouare patrias leges , \textbf{ et inducere nouas consuetudines . } Ordinauerat enim Hippodamus & de renouar las leyes dela tierra \textbf{ e de enduzir nueuas costunbres } por que ypodomio ordenara \\\hline
3.2.34 & obedire Regibus et Principibus , \textbf{ et obseruare leges . } Primo enim ex hoc consequitur populus virtutes , & quanto es prouechoso e conuenible al pueblo de obedesçer alos Reyes \textbf{ e guardar las leyes . } Ca lo primero desto alçança el pueblo uirtudes e grandes bienes \\\hline
3.2.35 & non solum non forefacere in ipsum Regem , \textbf{ sed etiam non forefacere in cognatos , } uxorem , filios , & non solamente de non fazer ningun tuerto contra el rey en su perssona . \textbf{ Mas avn de non fazer contra sus parientes } nin contra su muger \\\hline
3.3.1 & adhuc oporteret \textbf{ ipsum habere aliqualem prudentiam } qua sciret se regere et gubernare : & e morasse solo avn conuenir le \textbf{ ya de auer alguna sabiduria } por la qual se sopiesse gouernar . \\\hline
3.3.3 & Tribus igitur generibus signorum \textbf{ cognoscere possumus bellicosos viros . } Primo quidem per signa , & por tres maneras de señales \textbf{ podemos conosçer los omnes lidiadores } Lo primero por aquellas señales \\\hline
3.3.5 & sequitur hos meliores esse pugnantes . \textbf{ Videntur enim haec duo maxima esse ad obtinendam victoriam , } videlicet erubescentia fugiendi , & siguese que son meiores lidiadores . \textbf{ Ca paresçe que estas dos cosas son prinçipales | para auer victoria . } Conuiene a saber uerguença de foyr . \\\hline
3.3.9 & Considerato enim bello in uniuersali , \textbf{ omnes volunt esse boni bellatores , } sed postquam veniunt & Ca penssada la batalla en general \textbf{ todos quieren ser buenos lidiadores } mas despues que vienen a la prueua de los fechos particulares \\\hline
3.3.13 & Inde est quod bellorum experti dicunt pugnantes \textbf{ semper debere habere loricas amplas ita , } ut annuli loricarum se constringant : & que los que son prouados en las batallas . \textbf{ dizen que los lidiadores sienpre deuen auer las lorigas anchas . } assi que les aniellos de las lorigas se ayunten \\\hline
3.3.16 & Inde est quod multotiens obsidentes \textbf{ volentes citius opprimere munitiones , } si contingat eos capere aliquos de obsessis , & Et por ende contesçe que muchas uegadas \textbf{ los que çercan queriendo | mas ayna ganar las fortalezas } si contezca \\\hline
3.3.18 & Est etiam aduertendum \textbf{ quod die et nocte per lapidarias machinas impugnari possunt munitiones obsessae . } Tamen , ut videatur qualiter in nocte percutiunt lapides emissi a machinis , & Et avn conuiene de saber \textbf{ que tan bien de noche commo de dia se pueden acometer las fortalezas cercadas | por los engeñios } que lançan piedras . \\\hline
3.3.22 & debent se fingere fugere , \textbf{ et exire foueam illam } quo facto totam aquam aut urinam congregatam & En quando lidian contra los que los çercan deuen fingir que fuyen \textbf{ e deuen salir de aquella cueua } la qual cosa fecho toda aquella agua o aquella orina \\\hline

\end{tabular}
