\begin{tabular}{|p{1cm}|p{6.5cm}|p{6.5cm}|}

\hline
1.1.1 & Prima via sic patet . \textbf{ Cum enim doctrina de regimine Principum } sit de actibus humanis , & ¶ la primera rrazon es tal¶ Commo la doctrina del \textbf{ gouernamjentodel gouernamiento | delos prinçipes } sea delas obras de los omes \\\hline
1.1.1 & tangere Philosophus 1 Ethicorum , \textbf{ cum ait , } quod dicetur sufficienter de morali negocio , & eL segundo libro delas ethicas quando \textbf{ dizeque conplidamente se dize dela } moralph̃ia si fuer fecha manifestaçion \\\hline
1.1.1 & nec est veritas , sed bonum . \textbf{ Cum ergo rationes subtiles } magis illuminent intellectum , & por saber la bondad Dellas \textbf{ E pues que asi es commo las Razones sotiles } mas alunbren el entendimjento \\\hline
1.1.1 & ut ostendit in 1 Rhetoricorum . \textbf{ Cum igitur totus populus subtilia comprehendere non possit , } incedendum est in morali negocio figuraliter et grosse . & enel primero libro dela recthorica¶ \textbf{ E pues que asi es commo todo el pueblo pueda entender las cosas sotiles } deuemosyr en este libro por enxenplos \\\hline
1.1.2 & esse grossum et figuralem . \textbf{ Cum omnis doctrina } et omnis disciplina ex praeexistenti fiat cognitione , & que la manera que deuemos tener enesta obra sea gruesa e figural e exenplar \textbf{ a asi commo dize el philosopho | en el primer libro delos posteriores toda doctrina } e toda disçiplina desçiende \\\hline
1.1.2 & sic in operabilibus perfectam industriam praecedit astutia imperfecta . \textbf{ Cum ergo non requiratur tanta industria ad regendum seipsum , } quanta requiritur in gubernatione familiae : & ante que la sabiduria conplida donde se sigue \textbf{ que commo non sea demandada | tanta sabiduria ya gouerna mj̊ } asi mismo qunata es demandada \\\hline
1.1.2 & Quarto quos mores debeat imitari . \textbf{ Nam cum bene vivere , } et bene regere seipsum , & e quales non \textbf{ Ca commo bien beuir e bien gouernar } asy mesmo non pue da ser \\\hline
1.1.2 & videlicet , ex finibus , habitibus , passionibus , et moribus . \textbf{ Nam cum finis sit operationum nostrarum principium , } secundum quod quis sibi alium et alium finem praestituit , & Lo terçero de parte delas pasiones ¶ \textbf{ Lo quarto de parte delas costunbres ¶ | Ca commo la fin sea comjenço delas nr̃as obras } segunt que cada vno ordena asy mjsmo a \\\hline
1.1.2 & et desperatio \textbf{ non sunt eadem passio cum spe , } timentes , & que por que el temor \textbf{ e la esꝑança son departidas pasiones } quando los omes temen \\\hline
1.1.3 & opus istud suscepimus gratia eruditionis Principum : \textbf{ cum nunquam quis plene erudiatur , } nisi sit beniuolus , docilis , et attentus : & et commo nunca el prinçipen ̃j otro \textbf{ njnguno conplidamente puenda ser enssennado | sy non fueᷤ begniuolo } e uolenteroso para a prinder doçible e engennoso \\\hline
1.1.3 & et includunt bonitatem utilium bonorum . \textbf{ Cum ergo in hoc libro intendatur , } quomodo maiestas regia fiat virtuosa , & e ençierran en sy bondat de tondos los bienes prouechosos ¶ \textbf{ Pues que asy es commo en este libro ente damos demostrͣ } commo la magesad Real aya de ser uertuosa \\\hline
1.1.4 & ideo a fine et felicitate inchoandum est . \textbf{ Cum ergo secundum diuersos modos viuendi } diuersi diuersimode sibi finem praestituant , & e en la bien andança de todos los bienes obrantes . \textbf{ ¶ pues que Asy es commo departidos omes | segunt departidos maneras de beuir } departidanmente establescan e orden en asy mesmos a departidas fines \\\hline
1.1.4 & Tripliciter igitur poterit considerari homo : \textbf{ Primo ut communicat cum brutis : } Secundo ut est aliquid in se : & et es sobre todas las bestias ¶pues que asy es en tres maneras podemos pensar e fablar del omne ¶ \textbf{ primeramente en quanto partiçipa con las bestias en algunas cosas | que son comunes tan bien alos omes } commo alas bestias ¶ \\\hline
1.1.4 & Secundo ut est aliquid in se : \textbf{ Tertio ut participat cum angelis , } siue cum substantiis separatis . & quanto es alguno cosa \textbf{ que sy mismo¶ | Lo terçero en quanto partiçipa con los angeles } o con las sustançias apartadas ¶ \\\hline
1.1.4 & Tertio ut participat cum angelis , \textbf{ siue cum substantiis separatis . } Secundum has tres considerationes sumptae sunt & Lo terçero en quanto partiçipa con los angeles \textbf{ o con las sustançias apartadas ¶ } Segunt estos tres pensamjentos tomaron los philosofos \\\hline
1.1.4 & tres vitae voluerunt enim quod homini \textbf{ ut communicat cum brutis , } competit vita voluptuosa ; & que al omne conuiene la uida delectosa en \textbf{ quanto partiçipara con las bestias } Et en quanto es alguna cosa en sy mjsmo dizen qual coujene la uida politica e ordenada \\\hline
1.1.4 & ut est aliquid in seipso , vita politica : \textbf{ sed ut participat cum substantiis separatis , } competit ei vita contemplatiua . & Et en quanto es alguna cosa en sy mjsmo dizen qual coujene la uida politica e ordenada \textbf{ Mas en quanto partiçipa con las sb̃as apartadas | e con los angeles } dizen qual conujene la ujda contenplatina \\\hline
1.1.4 & Nam agere et communicare \textbf{ in actionibus cum aliis , } competit homini & Ca fazer e partiçipar enlas obras \textbf{ con los otros omes } conviene al omne \\\hline
1.1.4 & qui est aliquid diuinum , \textbf{ et secundum quem communicamus cum Deo , } et cum substantiis separatis . & e escodrinador que es alg̃cos e diujnal . \textbf{ Et segunt esto partiçipamos con dios con los angeles } ¶ Onde dize el philosofo en el primero libro delas politicas \\\hline
1.1.4 & et secundum quem communicamus cum Deo , \textbf{ et cum substantiis separatis . } Unde ex 1 Politicorum patet , & Et segunt esto partiçipamos con dios con los angeles \textbf{ ¶ Onde dize el philosofo en el primero libro delas politicas } que cada vno de los omes o es omne o es peor que omne \\\hline
1.1.4 & sed ut communicat \textbf{ cum substantiis separatis . } Sapientes igitur , & Et llamaron bien auenturado politico \textbf{ aquel que biue | çibdadanamente segunt omne } pues que asy es los sabios e los varones especulatinos \\\hline
1.1.5 & consequitur suam perfectionem et felicitatem . \textbf{ Cum ergo nunquam contingat recte agere , } ut requirit consecutio finis , & e su bien andança acabada¶ \textbf{ pues que asy es com̃ nunca pueda omne bien | e derechamente obrar asy } commo demanda la fin que ha de seguir . \\\hline
1.1.5 & non consequimur beatitudinem , et felicitatem . \textbf{ Immo cum ex operibus virtuosis felicitatem consequamur } ( quia virtus est & njn buena ventura de los . \textbf{ ¶ Mas commo nos alcançamos la buena ventura | por obras uirtuosas } ca la uirtud co sabidiria \\\hline
1.1.5 & consequi finem vel felicitatem . \textbf{ Cum ergo ista tria contingunt , } quod agamus bona , & nin bue an uentura ¶ \textbf{ Et quando estas tres cosas todas uienen en vno } que nos fagamos buean sobras \\\hline
1.1.6 & et delectationes , \textbf{ cum tamen ( simpliciter loquendo ) } delectationes intelligibiles , & llaman tan solamente plazenterias e delecta connes . \textbf{ Enpero fablando mas altamente las delectaçonnes del entendimiento } e las spuerales sin conparaçion \\\hline
1.1.6 & et ut ratio dictat . \textbf{ Cum enim bona secundum rationem , } et secundum intellectum & e muestra la gazon . \textbf{ Ca los bienes segunt la Razon } e segunt el entendimiento \\\hline
1.1.6 & non in bonis corporis . \textbf{ Cum enim corpus ordinetur ad animam , } sicut materia ad formam , & e non en los bienes del cuerpo . \textbf{ Ca el | cuerpoes ordenado al alma } bien commo la materia ala forma . \\\hline
1.1.6 & ut declarari habet 1 Ethic’ . \textbf{ Cum igitur voluptates , } et delectationes sensibiles & en el primero libro delas ethicas ¶ \textbf{ pues que asi es | que las plazenterias } e las delecta con nessensibles \\\hline
1.1.6 & quam animae . \textbf{ Cum enim corpus ordinetur ad animam , } et non econuerso : & son bienes del cuerpo \textbf{ mas quedtalma . ¶ pues que asi es commo el cuerpo sea ordenado al alma } e non el alma al cuerpo \\\hline
1.1.6 & Dicebatur enim supra , \textbf{ quod in vita voluptuosa conuenimus cum brutis , } et desiderantes sic viuere sunt & en la uida delectosa \textbf{ e uiçonsa dela carne | auemos conueiençia e partiçipaçon con las bestias . } Et los que asy dessean de beuir en delectes \\\hline
1.1.6 & et ebrii uti ratione non possunt . \textbf{ Cum ergo secundum eundem Philosophum } non facile sit contemptibilis & nin de entendimiento¶ \textbf{ pues que asi es commo | seg̃t ese mismo philosofo } non deua ninguno de ligeros \\\hline
1.1.7 & cuius nomen erat Mida , \textbf{ qui cum nimis esset auidus auri } ( ut fabulose dicitur ) & a que dezian meda \textbf{ el qual ero muy codiçioso de auerors } E asi commo dla fabliella gano de dios \\\hline
1.1.7 & fieret aurum . \textbf{ Cum ergo tactus reseruetur } in singulis partibus corporis , & que quanto el tanxiesse que todo se le tornasse oro \textbf{ asi que quanto el tannia todo se le tornaua oro } Et por que el seso del tanner es en todas las partes del cuerpo \\\hline
1.1.7 & Erat ergo ei magna copia auri , \textbf{ cum tamen fame periret . } Quod esse non posset , & grand Raqueza e grand cunplimiento de oro \textbf{ Enpero muria de fanbre } la qual cosa non pudiera ser \\\hline
1.1.7 & tum quia sunt diuitiae \textbf{ ex institutione Hominum , tum quia cum sint corporalia , } ipsi indigentiae corporali & Lo segundo que por que estas Riquezas son riquezas \textbf{ por ordenamiento e estableçemiento | delons omes } e non en otra gnisa¶ \\\hline
1.1.7 & de leui patet . \textbf{ Nam cum felicitas sit bonum optimum , } in optimo nostro quaeri debet . & esto ligeramente lo podemos prouar . \textbf{ ¶ Por que commo la feliçidat | e la bien andança sea muy gñdvien } e deua ser puesta por nos \\\hline
1.1.7 & in optimo nostro quaeri debet . \textbf{ Cum ergo anima sit potior corpore , } felicitas non est ponenda & en el mayor bien que nos podemos dessear . \textbf{ siguese que commo el alma sea meior | que el cuerpo la feliçidat } e la bien andança non es de poner en tales riquezas \\\hline
1.1.7 & nihil magnum attentabit . \textbf{ Immo ( cum ille sit Magnanimus , } cui nihil corporale est magnum , & Ca temiendo deꝑder los des e las riquezas nunca acometra grandes cosas \textbf{ Et la razon es esta | que aquel es magnamimo } e de grant coraçon \\\hline
1.1.7 & ut patebit in 3 lib’ \textbf{ cum determinabitur de regimine Regni . } Nam Rex proprie est , & ca ay grant diferençia entre el Rey e tirano \textbf{ assy commo demostrͣemos enl terçero libro quando determinaremos del gouernamiento del regno } Ca el Rey es aquel que propiamente parara mientes al bien del regno e al bien comun¶ \\\hline
1.1.7 & ei aliquod bonum proprium . \textbf{ Cum ergo finis maxime diligatur , } ponens suam felicitatem in numismate , & e la bien andança de los omes \textbf{ deua ser muy desseada | a aquel que pone las un feliçidat } e la su bien andança en las riquezas \\\hline
1.1.7 & sed Tyrannus , \textbf{ cum non intendat principaliter bonum publicum , sed priuatum . } Tertio hoc posito sequitur & ¶Lo terçero se declara \textbf{ asi que poniendo el prinçipe la su feliçidat | e la su bien andança en las riquezas corporales } dende se sigue \\\hline
1.1.8 & Honor ergo habet rationem boni extrinseci , \textbf{ cum sit reuerentia exhibita } per quaedam exteriora signa . & que esta de fuera \textbf{ por que es reuerençia fecha } por señales mostradas de fuera¶ \\\hline
1.1.8 & quam in honorato . \textbf{ Sed cum felicitas sit } maxime in ipso felicitato , & nin en aquel que la resçibe¶ \textbf{ Onde se sigue | que la bien andança } que es mas en el bien andante \\\hline
1.1.8 & Nam si Princeps suam felicitatem in honoribus ponat , \textbf{ cum sufficiat ad hoc } quod quis honoretur , & si pusiere la su bien andança en honrras \textbf{ conmoabaste acanda vno } para que sea honrrado \\\hline
1.1.8 & quia ex hoc efficietur periclitator Populi , et praesumptuosus : \textbf{ nam cum finis maxime diligatur , } si Princeps suam felicitatem in honoribus ponat , & e pornia los pueblos en peligro¶ \textbf{ Ca commo cada vn omne mucho ame la su fin } en que pone la su bien andança \\\hline
1.1.9 & quia fama est quaedam clara \textbf{ cum laude notitia . } Si tamen vellemus aliquo modo distinguere & Et esso mesmo es la fama \textbf{ Ca la fama es vn claro conosçimiento con loor . } Enpero si quisieremos fazer depart ineto entre la fama e la eglesia \\\hline
1.1.9 & esse magnae latitudinis , \textbf{ cum ad diuersas partes diuulgare possit : } et magnae diuturnitatis , & por que ella es de grant anchura \textbf{ e se estiende a muchas partes } Et otrosi por que es de grant durança det pon \\\hline
1.1.9 & et magnae diuturnitatis , \textbf{ cum per multa tempora contingat } ipsam indelebilem esse . & Et otrosi por que es de grant durança det pon \textbf{ por que dura por mucho stp̃os } e non se puede desfazer . \\\hline
1.1.9 & et clara notitia , \textbf{ cum scientia nostra } non sit ipsa res , & por que deles auido entre los omes algun loando e claro conosçimiento¶ \textbf{ Como el nuestro connosçimiento non sea aque|p{1cm}|p{6.5cm}|p{6.5cm}|la cosa de que es } nin sea razon dellas \\\hline
1.1.9 & quod exterius bona praetendat . \textbf{ Quare cum Regem deceat } esse totum diuinum , & e muestre algua bondat de fuera \textbf{ por la qual cosa commo al Rey conuenga ser todo diuinal e semeiante a dios } si non es cosa conuenible \\\hline
1.1.9 & quod gloria et fama , \textbf{ est quaedam clara cum laude notitia . } Dei autem notitia & e la fama es vn connosçimiento claro ton alabança \textbf{ mas quanto parte nesçe alo presente en tres cosas | se departe el conosçimiento de dios del nuestro conosçimiento . } Ca el conosçimiento de dios fizo \\\hline
1.1.9 & Rursus circa bonitatem nostram notitia Dei non fallit , \textbf{ cum scientia sua falli non possit . } Amplius bonitatem , & conosçimientode dios no puede ser engannado en lanr̃a bondat . \textbf{ Ca lascian de dios non puede resçebir enganno . | Et ahun dezimos mas adelante } que dios mas claramente vee \\\hline
1.1.9 & vel sit apud ipsum \textbf{ in clara notitia cum laude , } nisi sit bonus , et beatus , & nin puede ser ante \textbf{ e en claro conosçimiento con alabança } si non fuere bueno en uerdat e de fecho \\\hline
1.1.9 & Quomodo ergo fama unius hominis \textbf{ per uniuersam terram se extendet ? Sed , cum tota terra sit } quasi punctus respectu caeli , & nin eglesia de vir omne non se puede estender \textbf{ por todo el mundo . | Et commo tondo el mundo sea } asi common punto en conparaçion del çielo . \\\hline
1.1.9 & Rursus homini fama est breuis non diuturna , \textbf{ cum totum tempus vitae praesentis sit } quasi punctale respectu Dei aeternitatis : & delos omes es muy breue e non duradera . \textbf{ Ca todo el tpon dela uida presente de los omeses } assi commo vn momneto \\\hline
1.1.9 & quasi punctale respectu Dei aeternitatis : \textbf{ cum enim Homo } secundum intellectum sit immortalis , & e vn punto en conparaçion del anuida perdurable . \textbf{ Ca commo el omne segunt el alma } e el entendemiento seam mortal en aquella \\\hline
1.1.9 & Honor autem in se , \textbf{ cum non sit } nisi quoddam signum , & La otra manera es teniendo mientes al talante de aquellos que la fazen . \textbf{ Ca la honrra en si non es } si non vna sennal \\\hline
1.1.10 & Violentia autem perpetuitatem nescit . \textbf{ Cum igitur violenta non diu durent , } talis principatus diu durare non potest . & Mas ninguna cosa que es por fuerça \textbf{ non puede mucho durar¶ | pues que assi es que las cosas forçadas non pueden mucho durar } tal prinçipado non puede mucho durar \\\hline
1.1.10 & naturaliter agit , \textbf{ cum calefacit : } sic homo , qui est naturaliter liber arbitrio , & que es naturalmente caliente \textbf{ e sienpre escalienta assi el omne } por que es natraalmente franco \\\hline
1.1.10 & quia tale dominium \textbf{ cum sit violentum , } et contra naturam , & Mas es de poner en aquello \textbf{ que sienpre ha de durar ¶ } La segunda razon se declara \\\hline
1.1.10 & per coactionem , et violentiam . \textbf{ Cum ergo Principatus se extendant adinuicem , } secundum eos quibus aliquis principatus , & por costrinimiento e por fuerça¶ \textbf{ pues que assi es commo los prinçipados | e los sennorios se estiendan } segunt aquellos a quien \\\hline
1.1.10 & et ciuilem potentiam , \textbf{ cum non sit liberorum , } sed seruorum , & e mas digno que ensseñorear a los sieruos . \textbf{ Et por que el señorio por fuerça e por poderio çiuil commo non sea delons libres } mas de los sieruos non puede ser muy bueno nin digno¶ \\\hline
1.1.10 & quod principatus liberorum , \textbf{ qui est cum virtute , } est melior , & que es sobre los libres \textbf{ que es con uirtud es muy meior } que ensennorear despotice \\\hline
1.1.10 & ut plurimum nocumentum . \textbf{ Nam cum felicitas sit finis omnium operatorum , } quilibet totam vitam suam , & Ca commo la feliçidat \textbf{ e la bien andança | sea fin de todas las nuestras obras } ¶ Cada vno porna toda su uida \\\hline
1.1.10 & tempore tamen pacis nesciet bene viuere . \textbf{ Nam , cum ut plurimum studuerit , } nisi in exercitiis bellicis , & Enpero non sabra beuir bien entp̃o de paz . \textbf{ Ca commo en la mayor parte non aya estudiando } si non en vsos de armas e de batallas . \\\hline
1.1.10 & ait , turpe esse , \textbf{ cum bellamus , participare bonis , } cum vero vacamus et sumus in pace , fieri vitiosus . & Et dize assi que torpe cosa es \textbf{ que nos quando lidiamos o estamos en la batalla seamos bueons } Et despues quando fueremos en paz \\\hline
1.1.10 & cum bellamus , participare bonis , \textbf{ cum vero vacamus et sumus in pace , fieri vitiosus . } Quare si inconueniens est & que nos quando lidiamos o estamos en la batalla seamos bueons \textbf{ Et despues quando fueremos en paz | que seamos uiçiosos e malos . } por las quales cosas ya dichas \\\hline
1.1.11 & Robur debita proportio ossium et neruorum . \textbf{ Quare cum humores , membra , nerui , } et ossa sint corporalia , & e delons nieruos . \textbf{ Por la qual cosa commo los humores e los mienbros e los neruios } e los huessos sean cosas corporales . \\\hline
1.1.11 & Aequatio enim humorum , \textbf{ cum subsunt motui supercoelestium corporum , } variationi aeris , & e se puerden perder de ligero . \textbf{ Ca la egualança de los humores | en que esta la salut } commo sea subiecta al \\\hline
1.1.11 & siue propter conseruationem propriae personae : \textbf{ nam , cum ipse sit caput Regni , } ex defectu eius posset & para conseruaçion e guarda dela su persona . \textbf{ Ca commo el Rey sea cabesça de su Regno } por fallesçimiento dela su persona \\\hline
1.1.11 & talis est Populus . \textbf{ Cum ergo Princeps , } bonus praedicatur , & que qual es el prinçipe tal es el pueblo . \textbf{ Pues que assi es quando el prinçipe es alabado } por bueno los subditos toman manera para fazer bien . \\\hline
1.1.12 & secundum virtutem perfectam . \textbf{ Cum igitur perfecta virtus } secundum Philosophum & segund el philosofo es pridençia . \textbf{ Et la uirtud acabada en la | uidaçon tenplatiua es sapiençia o methaphisica } segund esse mismo philosofo ¶ Et \\\hline
1.1.12 & participat intellectum , et rationem . \textbf{ Cum ergo bonum rationis } non sit bonum aliquod particulare , & e razon ¶ \textbf{ Et pues que assi es commo al bien del entendimiento } e dela razon non sea bien particular nin personal \\\hline
1.1.12 & In amore ergo diuino est ponenda felicitas . \textbf{ Sed cum probatio dilectionis } sit exhibitio operis , & es de poner la feliçidat en la bien andança \textbf{ e por que la praeua del amor } e dela caridat es en la . obra . \\\hline
1.1.13 & nisi ex amore , \textbf{ cum semper amor sit ad similes , et conformes , } oportet esse similem , & por amor qual ha . \textbf{ Et commo el amor sienpre sean los semeiables e acordables con el . } Conuiene que aquel que es para de ser \\\hline
1.1.13 & qualis homo sit , \textbf{ cum in principatu existens , in quo potest bene et male facere , } cogitat qualiter se habeat . & por que estonçe paresçe qual es el omne \textbf{ quando es puesto en señorio | en que pueda fazer bien e mal . } Et aquella hora entiendan los omes \\\hline
1.1.13 & si non transgrediantur , \textbf{ cum possint transgredi , } maioris meriti esse videntur . & si non trispassaren los mandamientos de dios \textbf{ conmolos podiessen } trispassar son de mayor meresçimiento . \\\hline
1.1.13 & sed etiam totum regnum . \textbf{ Cum ergo magnae virtuti debeatur magna merces , } magnum erit meritum & mas ahun a todo el regno . \textbf{ Et pues que assi es commo grant uirtud deua auer grant merçed } e gm̃t gualardon gerad sera el meresçimiento e el gualardon de los Reyes . \\\hline
1.1.13 & si dirigat personam aliquam singularem . \textbf{ Cum ergo bonum gentis sit diuinius , } quam bonum aliquod singulare , & si ama alguna ꝑson a singular . \textbf{ pues que assi es commo el bien comun | e el bien dela gente sea mas diuinal } que nigun bien singular paresçe quela materia \\\hline
1.2.1 & Naturales potentiae sunt illae , \textbf{ in quibus communicamus cum vegetabilibus , et plantis , } ut potentia nutritiua , augmentatiua , generatiua , & Naturales poderios son aquellos \textbf{ enlos quales partiçipamos con los arboles | e con las plantas } e con todas las cosas \\\hline
1.2.1 & et talia , \textbf{ in quibus communicamus cum brutis . } Appetitiuae vero distinguuntur : & e el tannimiento o el veer \textbf{ e el oyr e el gostar e el oler e el tanner | Et estos tales son comunes tan bien alas bestias commo anos ¶ } Mas los poderios desseadores se departen en esta gusa . \\\hline
1.2.1 & nam quidam appetitus est in homine , \textbf{ in quo non communicat cum brutis , } ut est appetitus sequens intellectum : & Ca alguno dellos es propio al omne \textbf{ en el qual non partiçipa con las bestias . } assi commo es el appetito \\\hline
1.2.1 & ut est appetitus sequens intellectum : \textbf{ quidam vero in quo communicat cum eis , } ut appetitus sequens sensum . & que sigue al entendimiento . \textbf{ Et alguon sson comunes tan bien alas bestias | commo alos omes } assi commo el a etito que sigue alos sesos \\\hline
1.2.1 & per virtuosos ad agendum bene : \textbf{ cum ergo natura sit determinata ad unum , } et potentiae naturales sufficienter determinentur ad agendum , & e uirtudes se apareian a bien obrar . \textbf{ ¶ Et pues que assi es commo la natura sea determimada a vna cosa } e los poderios naturales sean determinandos conplidamente para obrar \\\hline
1.2.1 & nec in sensibus est virtus moralis , \textbf{ cum praeter potentias naturales , } et sensus non sit nisi intellectus , & nin en los sesos naturales . \textbf{ Commo sin estos poderios naturales | e sesos naturales } non aya en el omne otro poderio \\\hline
1.2.2 & computari tamen potest \textbf{ cum virtutibus moralibus : } nam Prudentia non est nisi in hominibus bonis , & e entre las inellectuales . \textbf{ Enpero puede se contar con las uirtudes morales } por que la pradençia non es sinon en los buenos omes . \\\hline
1.2.2 & Appetitus autem sensitiuus duplex est . \textbf{ Nam cum animalia sint supra inanimata , } si natura elementis & Mas el appetito senssitiuo es en dos maneras \textbf{ ca commo las aina las sean sobre las cosas | que non han alma } si la natura de los helementos dio a todas las cosas \\\hline
1.2.2 & et malum inquantum habent rationem difficilis , et ardui . \textbf{ Nam cum bonum secundum se dicat prosequendum , } malum vero quid fugiendum : & en quanto han razon de cosa guaue e fuerte . \textbf{ Ca commo el bien | por si diga tal cosa } que deue el omne seguir \\\hline
1.2.2 & remanet indiuisus . \textbf{ Nam cum intellectus uniuersaliori modo } respiciat suum obiectum quam sensus , & que es dich̃o uoluntad non es departido en ningunas partes . \textbf{ Ca commo el entendimiento sea mas general que el seso . } Mas generalmente cata aquello en que ha de obrar que el seso . \\\hline
1.2.3 & Numerus autem earum sic potest accipi . \textbf{ Nam cum subiectum virtutis sit , } vel intellectus , vel voluntas , & assi se puede tomar . \textbf{ Ca commo el subiecto delas uirtudes sea o el entendimiento o la uoluntad o el appetito senssitiuo . } toda uirtud moral o es en el entendimiento \\\hline
1.2.3 & timor , et audacia : \textbf{ timor cum ab eo refugimus , } audacia cum illud aggredimur . & assi se leuantan en nos passiones de temor e de osadia . \textbf{ temoͬ quando fuymos del mal futur . } Osadia quando acometemos algun mal futuro ¶ \\\hline
1.2.3 & timor cum ab eo refugimus , \textbf{ audacia cum illud aggredimur . } Inter has autem duas passiones sumitur & temoͬ quando fuymos del mal futur . \textbf{ Osadia quando acometemos algun mal futuro ¶ } Mas entre estas dos passiones se toma vna uirtud medianera \\\hline
1.2.3 & quae oriuntur ex bonis , \textbf{ ut communicamus cum aliis , } sic ( ut dicitur secundo Ethicorum ) sumuntur tres virtutes . & que nasçen de lons bienes \textbf{ en quanto los auemos comunes con los otros | assi commo dize el philosofo } en el segundo libro delas ethicas . \\\hline
1.2.3 & sic ( ut dicitur secundo Ethicorum ) sumuntur tres virtutes . \textbf{ Nam cum aliis communicamus in verbis , et operibus : } opera autem , et verba , & pueden se tomar tres uirtudes . \textbf{ Ca con los otros omes partiçipamos en palauras e en obras . } Mas las palauras e las obras en quanto partiçipamos con los otros en lłas siruen nos para la uirdat \\\hline
1.2.3 & opera autem , et verba , \textbf{ ut communicamus cum aliis deseruiunt nobis ad veritatem , vitam , et ludum . } Erit ergo triplex virtus ; & Ca con los otros omes partiçipamos en palauras e en obras . \textbf{ Mas las palauras e las obras en quanto partiçipamos con los otros en lłas siruen nos para la uirdat | e para la uida } e para el trebeio ¶ \\\hline
1.2.3 & sunt in concupiscibili . \textbf{ Patet ergo quod cum quatuor potentiae animae sint } susceptibiles virtutum de quibus loquimur , & guaue son en el appetito desseador ¶ \textbf{ Pues que assi es paresçe | que commo sean quatro poderios del alma } que pueden resçebir las uirtudes \\\hline
1.2.5 & vel circa operationes . \textbf{ Cum enim contingat } ratiocinari recte et non recte , & o para reglar las obras ¶ \textbf{ Ca commo contesca de razonar derechamente } e non derechamente conuiene de dar alguna uirtud \\\hline
1.2.5 & per quam de ipsis agibilibus rectas rationes faciamus . \textbf{ Rursus cum contingat operari recte et non recte , } sic ut est dare virtutem , & que fazemos fagamos razones derechas ¶ \textbf{ Otrosi commo contesca de obrar derechamente | e non derechamente } assi commo auemos a dar uirtud . \\\hline
1.2.5 & per quas modificentur in ipsis passionibus . \textbf{ Cum ergo passiones quaedam impellant nos ad malum , } ut passiones concupiscibiles , & por las quales seamos tenprados e reglados en aquellas passiones ¶ \textbf{ Et pues que assi es commo algunas delas passiones nos mueuen a mal } assi commo las passiones dela cobdiçia \\\hline
1.2.5 & videlicet , in intellectu , in voluntate , in irascibili , in concupiscibili : \textbf{ cum ergo in intellectu practico } principalior virtus sit prudentia , in voluntate & en el appetito desseador ¶ \textbf{ Et pues que assi es commo en el entendimiento pratico la mas prinçipal uirtud son la pradençia . } Et en la uoluntad la prinçipal uirtud sea la iustiçia fablando delas uirtudes \\\hline
1.2.5 & principalior omnibus aliis , \textbf{ cum sit directiua omnium aliarum , } et iustitia sit principalior & ¶Otrosi por que la prudençia es mas prinçipal que todas las otras \textbf{ por que es endereçadora e regladora de todas las otras ¶ } Et la iustiçia en pos ella es mas prinçipal que la fortaleza e la tenperança . \\\hline
1.2.5 & quod fortitudo et temperantia , \textbf{ cum iustitia sit } circa commutationes rerum , & Et la iustiçia en pos ella es mas prinçipal que la fortaleza e la tenperança . \textbf{ Por que la iusticia es cerca de aquellas cosas } que se canbian vna por otra . \\\hline
1.2.5 & et quia fortitudo est principalior quam temperantia , \textbf{ cum fortitudo magis ordinetur ad bonum gentis , } et ad bonum commune quam ipsa temperantia : & en que prinçipalmente se trabaia lanr̃auida¶ Et otrosi por que semeiablemente la fortaleza es mas prinçipal que la tenꝑança . \textbf{ por que la fortaleza es mas ordenada al bien dela gente } e al bien comun que la tenperança . \\\hline
1.2.6 & sed per prudentiam \textbf{ ( cum habemus ) } dirigimur in fines illos . & Mas por la pradençia somos reglados \textbf{ en qual manera pondemos alcançar aquellas fines . } Mas la pradençia toma aquellas fines delas uirtudes morałs \\\hline
1.2.6 & quam faciat virtus inuentiua et iudicatiua . \textbf{ Cum ergo in moralibus actus } et opera dicantur esse potiora , & que la uirtud buscadora e falladora . \textbf{ Et pues que assi es commo en las uirtudes morales las obras } e los fechos sean dichos mayores e meiors . \\\hline
1.2.6 & circa quam versatur . \textbf{ Cum enim Prudentia sit circa agibilia , } et agibilia sint singularia , & ala materia en que obra . \textbf{ Ca commo la pradençia aya de ser en las obras . } Et las obras ayan de ser \\\hline
1.2.7 & vel plumbeus positus in computo mercatorum , \textbf{ Mercatores enim cum ratiocinando computant , } aliquando unum denarium aeneum & puesto en el cuento delos mercadores . \textbf{ Ca los mercandores su mando | e razoñado cuentan algunas vezes } vn dinero de cobre de plomo \\\hline
1.2.7 & et regia dignitate fungatur , \textbf{ cum ipse parui valoris sic , } est loco magni precii : & e husare de dignidat de Rey \textbf{ commo el sea de poco ualor esta en logar de grant preçio . } Pues que assi es mas ha señal de rey \\\hline
1.2.8 & esse utilia toti regno , \textbf{ cum hoc quod Regem expedit } esse solertem ex se , & que son aprouechables a todo el regno . \textbf{ Enpero con esto que conuiene al Rey de ser sotil e agudo de si penssando los bienes } que son aprouechables a su regno \\\hline
1.2.9 & et Principes volunt esse prudentes , \textbf{ cum hoc quod debent esse memores , prouidi , solertes , et dociles , } et alia , & por la qual cosa si los Reyes e los prinçipes quieren ser sabios con esto \textbf{ que deuen ser acordables prouisores engennosos e doctrinables } e auer las otras cosas \\\hline
1.2.10 & Ex ista autem differentia sequitur secunda : \textbf{ nam cum fortis , } et temperatus delectetur & Et desta diferenços se sigue la segunda . \textbf{ Ca commo el fuerte e el tenprado se deleite } segunt \\\hline
1.2.10 & et aequalitati intendit : \textbf{ cum bona exteriora } dupliciter ciues inaequaliter participare possint , & que cada vno sea señor de lo suyo ¶ Pues que assi es si esta iustiçia sp̃al es dicha igual por que entiende a egualdat . \textbf{ Como los çibdadanos pue dan estos biens de fuera partiçipar en dos maneras } desigualmente siguese \\\hline
1.2.11 & et quidam principatus . \textbf{ Cum igitur ordo , } et principatus sit ipsorum subditorum & Ca el regno e toda comunidat es vna orden e vn prinçipado . \textbf{ ¶ pues que assi es como la orden } e el principado sea de los subditos por conparaçion alas leyes \\\hline
1.2.12 & et ut obseruent Iustitiam : \textbf{ cum sine ea ciuitates , } et regna durare non possint . & e que guarden iustiçia \textbf{ sin la qual las çibdades e los regnos non pueden durar } Empero por que el coraçon del noble ome \\\hline
1.2.12 & ut possit ipsas leges dirigere : \textbf{ cum aliquo casu leges obseruari non debeant , } ut infra patebit . & e de tan grant egualdat por que pueda enderesçar e egualar las leyes . \textbf{ Ca algun caso ay en que se non deuen guardar las leyes } assi commo paresçra adelante . \\\hline
1.2.12 & nihil regulatum erit : \textbf{ cum omnia per regulam regulentur . } Sic si Reges sint iniusti , & Ca çierta cosa es \textbf{ que por la regla se reglan | e se egualan todas las cosas . } Et en essa misma gnisa \\\hline
1.2.12 & ut sint Reges . \textbf{ Cum enim deceat regulam esse rectam et aequalem , } Rex quia est quaedam animata lex , & enpero non son dignos de seer Reyes . \textbf{ Ca commo conuenga ala regla de ser derecha } e egual e el Rey sea vna ley animada e vna regla . \\\hline
1.2.12 & quod unumquodque perfectum est , \textbf{ cum potest sibi simile producere , } et cum actio sua ad alios se extendit : & que cada vna cosa es acabada \textbf{ enssi quando puede fazer otra tal commo si . } Et quando la su obra se estiende alos otros \\\hline
1.2.12 & cum potest sibi simile producere , \textbf{ et cum actio sua ad alios se extendit : } ut tunc aliquid est perfecte calidum , & enssi quando puede fazer otra tal commo si . \textbf{ Et quando la su obra se estiende alos otros | assi commo paresçe por la calentura . } Ca estonçe es dicha alguna cosa \\\hline
1.2.13 & per quam regulentur in agendo . \textbf{ Cum igitur circa timores , } et audacias contingat & por la qual seamos reglados en las obras \textbf{ que auemos de fazer ¶ | pues que assi es commo los omes alguas vezes puedan } e les contezca de se auer derechamente \\\hline
1.2.13 & aggrediuntur bellorum pericula : \textbf{ sed , cum aggressi sunt ea , } si inuenerint resistentiam , & Ca alguonsligeramente acomneten los periglos delas faziendas e delas batallas . \textbf{ Mas quando los han acometidos } si fallan fortaleza \\\hline
1.2.13 & Quidam autem non de leui aggrediuntur bellum , \textbf{ sed cum aggressi fuerint , } sustinent , & Et otrosi hay otros que non deligero acometen la fazienda e la batalla . \textbf{ mas quando la han acometida estan } e susten los periglos dela batalla e dela fazienda . \\\hline
1.2.13 & et tolerant pericula bellorum . \textbf{ Cum igitur virtus sit } circa bonum , & e susten los periglos dela batalla e dela fazienda . \textbf{ pues que assi es commo toda uirtud . } sea çerca algun bien \\\hline
1.2.13 & Non enim sic aperte apprehendimus mortem , \textbf{ cum aegrotamur : } quia aegritudines intrinsecus latent ; & manifiestamente la muerte quando enfermamos \textbf{ por que las enfermedades yazen ascondidamente de dentro nin sentimos . | los periglos dela mar } commo sentimos delas batallas \\\hline
1.2.13 & Non enim sic per fugam vitare possumus aegritudines : \textbf{ quia cum aegritudo sit aliquid in nobis existens , } per fugam eam vitare non possumus . & assi por foyr escapar las enfermedades \textbf{ por que la enfermedat es alguna cosa | que esta en nos } e por foyr non la podemos escusar . \\\hline
1.2.13 & sicut pericula belli . \textbf{ Cum ergo difficilius sit durare , et sustinere pericula illa } quae per fugam vitare possumus , & assi escusar commo los periglos delas batallas . \textbf{ Et pues que assi es commo sea mas | guaue cosa de endurar } e de sufrir aquellos periglos que podemos escusar \\\hline
1.2.13 & quia per ea maxime apprehendimus mortem violentam , \textbf{ cum mors ibi illata sit } per mutilationem membrorum , & ymaginamos la muerte forcada \textbf{ Ca la muete y dada en la batalla fazese } por taiamiento de mienbros \\\hline
1.2.13 & quilibet fugit , sicut naturaliter sequitur delectabilia . \textbf{ Cum ergo naturaliter tristia fugiamus , } difficile est reprimere timores , & assi commo naturalmente sigue las cosas delectables . \textbf{ Et pues que assi es commo nos natural mente fuyamos dela tristeza } graue cosa es de repmir los temores \\\hline
1.2.14 & quam prima : \textbf{ ut cum aliquis non ut vitet opprobria , } vel ut consequatur honores : & que la primera alsi commo \textbf{ quando alguno non | por esquiuar denuestos } o por gauar honrras \\\hline
1.2.14 & coegisse ad Fortitudinem . \textbf{ Nam , cum nauigiis , } et cum toto suo exercitu transfretaret , & por que fuesen fuertes . \textbf{ Ca commo el estudiese en sus naues } e con todas sus naues passase la mar \\\hline
1.2.14 & Nam , cum nauigiis , \textbf{ et cum toto suo exercitu transfretaret , } ne aliquis de suo exercito haberet materiam fugiemdi , & Ca commo el estudiese en sus naues \textbf{ e con todas sus naues passase la mar } por que ninguno de sus conpannas non ouiese manera de fuyr \\\hline
1.2.14 & Non tamen propter hoc proprie fortes sunt , \textbf{ nam cum adeo inualescit bellum , } quod excedat eorum experientiam , & Mas enpero por esto non son fuertes propia mente . \textbf{ Ca quando la batalla es afincada e continuada en tal manera } que es mayor la batalla \\\hline
1.2.14 & Sexta fortitudo dicitur esse bestialis , \textbf{ ut cum aliquis ignorans fortitudinem aduersarii , bellatur . } Ut puta si habitantes in septentrione sunt fortes , et audaces , & La sexta fortaleza es testial e de bestia \textbf{ assi commo quando alguno comiença de lidiar non sabiendo | nin conosçiendo la fortaleza del su contrario . } Enxient lo desto . \\\hline
1.2.14 & et timidi , \textbf{ aggredientes pugnam cum septentrionalibus , } credentes eos esse meridionales , & e fuesen flacos e temerosos . \textbf{ Et si alguons acometiesen la batalla con los setenteronales } que son fuertes cuydando \\\hline
1.2.14 & Septima Fortitudo dicitur virtuosa : \textbf{ ut cum aliquis non coactione , } vel furore , vel propter experientiam , & ¶La septima fortaleza es uirtuosa e de uirtud \textbf{ assi commo si alguno non por primia nin por sanna nin otrossi por praeua que aya auido de batalla } o por non saber el poder de sus enemigos acomete la batalla . \\\hline
1.2.14 & et quomodo possunt \textbf{ cum aduersariis bellare : } ipsi tamen debent esse fortes fortitudine virtuosa , & por que sepan en qual manera han de ser fuertes . \textbf{ Et en commo pueden lidiar con sus enemigos . } Enpero ellos deuen ser fuertes de fortaleza uirtuosa \\\hline
1.2.15 & et a licitis delectationibus , \textbf{ quod cum tali abstinentia eius natura stare non posset , } quia huius contrarium recta ratio dictat , & e delas otras delecta connes corporales \textbf{ en tal manera que non puede estar la natura con tanta abstinençia } por que el contrario desto manda la razon derecha \\\hline
1.2.15 & magis delectamur , \textbf{ cum cibus aut potus attingit guttur , } quam cum coniungitur linguae . & mas nos delectamos quando el comer \textbf{ e el beuer llega ala garganta } que quando llega ala lengua . \\\hline
1.2.15 & cum cibus aut potus attingit guttur , \textbf{ quam cum coniungitur linguae . } Credendum est enim in talibus iudicio gulosorum . & e el beuer llega ala garganta \textbf{ que quando llega ala lengua . } Et estas cosas deuemos creer al iuizio de los golosos . \\\hline
1.2.15 & nomine Phyloxenus , \textbf{ qui , cum esset pultiuorax , orauit , } ut guttur eius longius quam gruis fieret . & e tomaua muy grant delectaçion en ellas g̃rago a dios \textbf{ quel feziese la garganta } mas luenga que garganta de grulla \\\hline
1.2.15 & Sicut ergo Fortitudo \textbf{ magis conuenit cum audacia ; } et si volumus esse fortes , & que son delectables . \textbf{ Et pues que assi es assi conmo la fortaleza mas conuiene con la osadi } Et si nos quisieremos fazer nos fuertes \\\hline
1.2.15 & quam timidi : \textbf{ sic Temperantia plus conuenit cum insensibilitate . } Si ergo volumus nosipsos facere temperatos , & mas auemos aser osados e temerosos . \textbf{ assi en essa misma manera la tenpranca mas conuiene con el non sentimiento | que con la senssiblidat de los sesos . } Et por ende si nos quisieremos fazer a nos mismos tenprados deuemos \\\hline
1.2.16 & Magis quis voluntarie agit \textbf{ quod facit cum delectatione , } quam quod facit cum tristitia . & Et mas de uoluntad faze cada vno \textbf{ lo que faze con } delectaçion que lo que faze con tͥsteza . \\\hline
1.2.16 & quod facit cum delectatione , \textbf{ quam quod facit cum tristitia . } Peccans igitur per intemperantiam , & lo que faze con \textbf{ delectaçion que lo que faze con tͥsteza . } Et por ende el que peca por \\\hline
1.2.16 & quam per intemperantiam : \textbf{ cum hoc sit magis voluntarium , } quam sit illud . & que non por \textbf{ destenpranca por que esto es mas de uoluntad } que aquello otro ¶ \\\hline
1.2.16 & Valde est ergo increpandus carens tempesantia , \textbf{ cum eam sine periculo possit acquirere : } non autem adeo increpandus est & que non ha \textbf{ tenpranca | commo puede ganar la tenpranca sin ningun periglo . } Mas non es tanto de denostas el \\\hline
1.2.16 & quia virtus illa est difficilis , \textbf{ et cum maiori periculo acquiritur . } Rursus facilius acquiri potest temperantia , quam fortitudo : & por que aquella uirtud dela fortaleza es mas guaue \textbf{ e se gana con mayor periglo . } Otrosi mas ligeramente se puede ganar la tenpranca que la fortaleza \\\hline
1.2.16 & in Rege Sardanapallo , \textbf{ qui cum esset totus muliebris , } et deditus intemperantiae ( ut recitat Iustinus Historicus , libro 1 abbreuiationis Trogi Pompeii ) & enxienplo en el Rey sardan \textbf{ apalo que por que era todo mugeril | e dado a mugers } e era muy \\\hline
1.2.16 & ut haberet colloquia \textbf{ cum baronibus regni sui ; } sed omnes collocutiones eius erant & a auer fabla con los Ricos omes \textbf{ e cauałłos de su regno . } mas todas sus fablas eran en las camaras con las mugieres \\\hline
1.2.16 & Accidit autem , \textbf{ quod , cum quidam Dux exercitus diu ei seruiuisset , } et fideliter , & lo que auian de fazer \textbf{ Et acaesçio que vn prinçipe mucho su priuado que grant t p̃o le auia seruido e fiel mente . } Et el Rey que tiendo fazer plazer a aquel \\\hline
1.2.16 & clausit se in quadam domo , \textbf{ et cum toto thesauro , } et omnibus supellectilibus suis , & de aquel duque ençerrosse en vna casa con todo su tesoro \textbf{ e con todas sus alfaias } e que mosse con todo \\\hline
1.2.17 & Incipit enim nostra cognitio a sensu . \textbf{ Cum ergo bona utilia } sensibiliora sint honestis , & e en las cosas que sentimos . \textbf{ Et por ende commo los bienes aprouechosos } sintamos nos mas que los bienes honestos . \\\hline
1.2.17 & Quinto hoc idem patet : \textbf{ quia cum liberales maxime amentur , } circa illud maxime consistit liberalitas , & Lo quinto se puede prouar esso mismo \textbf{ por que los liberales | e los francos son mas amados sobre todos los otros . } Et por ende la franqueza esta prinçipalmente en aquello \\\hline
1.2.18 & aliquid ociosum esse debet . \textbf{ Quare cum natura humana modicis contenta sit , } quia uni personae modica sufficiunt : si una aliqua persona multitudine diuitiarum superabundat , & non deue ser ninguna cosa ociosa nin baldia . \textbf{ por la qual razon commo la naturaleza de los omes se tenga | por pagada de pocas cosas } por que a cada vna delas personas pocas cosas le abastan \\\hline
1.2.18 & ( quia senes auariores sunt iuuenibus ) \textbf{ cum veniunt ad senectutem , } ut plurimum curatur eorum prodigalitas , & Et por ende son mas auarientos \textbf{ que quando eran mançebos } e assi los mas dellos \\\hline
1.2.18 & et cuius gratia debet . \textbf{ Quare cum prodigus non sit amator pecuniae , } sicut nec liberalis , & nin por la razon que deue . \textbf{ por la qual razon commo el gastador non sea amador de los } desbien commo el libal non lo es de ligero se puede fazer \\\hline
1.2.18 & de leui \textbf{ quis cum sit prodigus , } fieri poterit liberalis . & desbien commo el libal non lo es de ligero se puede fazer \textbf{ qual quier gastador liberal e franco ¶ } pues que assi es si es conueinble al Rey \\\hline
1.2.18 & quae continet . \textbf{ Cum ergo tanto deceat fontem habere os largius , } quanto ex eo plures participare debent : & Ca ha . manera daua so ancho e largo e da conplidamente lo que tiene \textbf{ ¶pues que assi es conmo tanto conuenga ala fuente auer la boca | mas ancha } quanto della deuen \\\hline
1.2.19 & quam magnificentiam nominant . \textbf{ Sed cum magis , } et minus non videantur & que quiere dezir grandeza en despender . \textbf{ Mas commo en cada cosa } mas e menos non fagan departimiento en la naturaleza \\\hline
1.2.19 & et difficile . \textbf{ quare cum in maioribus sumptibus reperiatur } specialis ratio bonitatis et difficultatis , & que se a çerca bien grande e guaue de fazer . \textbf{ por la qual cosa commo en las mayores espenssas sea fallada . } mas espeçial razon de bondat e de guaueza \\\hline
1.2.19 & et proportionati diuitibus . \textbf{ Cum ergo liberalitas non respiciat sumptus secundum se , } sed ut proportionantur facultatibus , & e conuenibles alos ricos \textbf{ ¶Pues que assi es commo la libalidat | non cate alas espenssas } segunt \\\hline
1.2.20 & Quinta proprietas eius est , \textbf{ ut semper expendat cum tristitia , et dolore . } Dicebatur enim , & La quinto propiedat del pariufico es \textbf{ que sienpre espienda aquello que espendiere con tristeza e con dolor . } Ca assi commo dicho es el \\\hline
1.2.20 & sine tristitia , et dolore . \textbf{ Participat enim in hoc paruificus cum auaro : } quia omnis paruificus auarus est , & assi conmo el mienbro non se podria partir del cuerpo lindolor . \textbf{ Et en esto el partu fico partiçipa con el auariento } por que todos los paruificos son auarientos . \\\hline
1.2.20 & Sexta est , \textbf{ quia cum paruificus nihil faciat , } videtur tamen ei quod semper agat maiora , & La sexta propiendat es \textbf{ que quando el parufico faze alguna cosa } aparesçe ael \\\hline
1.2.20 & et non appreciatur ea . \textbf{ Cum igitur vix possit } aliquis dare & si non el auer ¶ \textbf{ Pues que assi es commo abes pueda el } parufico dar vna cosa muy pequana dela cosa que mucho ama . \\\hline
1.2.20 & semper facere sumptum \textbf{ cum tristitia et dolore ; et cum nihil facit , } credere se magna operari , & mas penssar sienpre en commo espienda poco e fazer sienpre espenssa con tristeza e con dolor . \textbf{ Et quando non faze ningunan cosa cree el } que faze grandescosas e grandes obras . \\\hline
1.2.20 & et circa seipsum . \textbf{ Cum ergo Rex sit caput regni , } et sit persona honorabilis , reuerenda , et publica , & e erca si mismo \textbf{ Et pues que assi es commo el Rey sea } ca besça del regno e leaꝑlona honrrada \\\hline
1.2.21 & esse proprium magnificentiae . \textbf{ Nam cum in operibus magnificis , } ut cum aliquis magnifice se habet & espeçialmente esto es propio ala magnifiçençia . \textbf{ Ca commo en las grandes obras | sienpre deua tener omne nayentes a buena fin } assi commo quando alguno granadamente sea en la honrra de dios \\\hline
1.2.21 & Nam cum in operibus magnificis , \textbf{ ut cum aliquis magnifice se habet } erga cultum diuinum , & sienpre deua tener omne nayentes a buena fin \textbf{ assi commo quando alguno granadamente sea en la honrra de dios } e en el bien comun \\\hline
1.2.21 & et quidam ornatus omnium virtutum . \textbf{ Cum enim ille sit magnificus , } qui in magnis operibus facit decentes sumptus : & assi commo la magnanimidat es vna ꝑfectiuo e vn honrramiento de todas las uirtudes \textbf{ Ca commo aquel sea magnifico } que faze conuenibles espenssas enlas grandes obras \\\hline
1.2.22 & non extollemur , \textbf{ cum non reputamus } talia esse simpliciter maxima bona . & non nos le unataremos en vanagłina \textbf{ por que non cuydaremos } que tales bienes son los mayores bienes . \\\hline
1.2.22 & decenter tolerabimus , \textbf{ cum non multum reputemus ipsa . } Recitat autem Philosophus & tenporales sofrir la hemos conueniblemente \textbf{ por que los non tenemos en mucho } e dedes saber \\\hline
1.2.23 & consurgere magna utilitas . \textbf{ Cum autem sic se periculis exponit , } adeo debet esse constans in suis negociis , & e muy grand prouecho \textbf{ Et quando assi el mangnanimo se pusiere alos periglos . } deue ser ta firme en sus negoçios e en sus obras . \\\hline
1.2.23 & nec quod alii vituperentur . \textbf{ Nam cum talia inter exteriora bona computentur , } ipse non multum curat de eis , & ni de seer denostado . \textbf{ Ca porque tales cosas commo estas son contadas | entroͤ los bienes de fuera } non faze grant fuerça dellas . \\\hline
1.2.23 & ut esse veridicos ; \textbf{ cum sint regula aliorum , } quae obliquari , & de seer manifiestos e claros e seer uerdaderos \textbf{ por que son regla de los otros } La qual regla non se deue torcer nin falssar \\\hline
1.2.24 & et ea faciet excellenter . \textbf{ Nam cum honor inter exteriora bona sit bonum excellens , } faciens opera virtutum , & Ca commo la honrra \textbf{ entre los bienes de fuera | sea mas alto e meior bien } el que faze obras de uirtudes \\\hline
1.2.25 & non appellat magnanimum , sed temperatum . \textbf{ Cum igitur habere temperantiam in honoribus , } sit idem , & mas llamale tenprado . \textbf{ Et por ende auer algun tenpramiente en las honrras } es esso mismo \\\hline
1.2.25 & quia docuissemus eos esse sine virtutibus , \textbf{ cum absque humilitate virtutes haberi non possint . } Ad plenam igitur & que serian sin uirtudes \textbf{ si non ouiessen humildat } por que sin la humildat las otras uirtudes non se pueden auer Et pues que assi es para auer conplida declaraçion de la uerdat \\\hline
1.2.25 & ab eo quod ratio dictat . \textbf{ Quare cum passiones nos inclinant } ad aliquid contra rationem , & de aquello que la razon e el entendimiento manda . \textbf{ por la qual cosa quando las passiones nos inclinan . } a alguna cosa contra razon e contra entendemiento auemos mester uirtud \\\hline
1.2.25 & a delectationibus sensibilibus . \textbf{ Sed cum passiones nos retrahunt } ab eo quod ratio dictat , & delecta connes de los sesos \textbf{ mas quando las passiones nos tiran } de aquella cosa \\\hline
1.2.25 & quam retrahat nos ab illis . \textbf{ Cum ergo retrahere et impellere sint quodammodo opposita , } et formaliter differant , & nin nos arriedra dellas . \textbf{ Et por ende commo tirar nos | de aquello que la razon manda } e allegarnos a esso \\\hline
1.2.25 & Omnino ergo magnanimus humilis est . \textbf{ Nam cum impossibile sit } esse magnanimum non existentem bonum , & Et por ende todo magnanimo es humildoso \textbf{ Ca commo non pueda seer } que alguno sea magnanimo \\\hline
1.2.25 & de qua loquitur Philosophus , \textbf{ non esse per omnem modum idem cum humilitate : } quia illa de qua Philosophus loquitur , & mostrariamos que la uirtud de que fabla el philosofo \textbf{ non es en toda manera vna cosa misma con la humildat } por que aquella de que fabla el philosofo \\\hline
1.2.26 & Non tamen aequae principaliter operatur utrunque : \textbf{ nam cum magnanimi sit tendere in magnum , } magnanimitas magis est & e el entendimiento muestran . Enpero estas dos cosas non las obra egualmente nin prinçipalmente \textbf{ Ca commo almagranimo pertenesca de yr | e entender en cosas grandes la magranimidat } mas es uirtud \\\hline
1.2.26 & ut supra diffusius dicebatur . \textbf{ Quare cum distincta sit virtus haec ab illa , } videndum est & assi commo dicho es de ssuso mas conplidamente \textbf{ por la qual cosa commo esta uirtud | que es dicha humildança sea apartada dela magnanimidat } conuienenos de veer \\\hline
1.2.26 & Est enim hoc notabiliter attendendum , \textbf{ quod cum virtus magis sit retrahens quam impellens , } principaliter opponitur superabundantiae , & que quando la uirtud . \textbf{ mas nos trahe e tira | que nos allega e esfuerca . } Estonçe prinçipalmente \\\hline
1.2.27 & ostendere non est difficile . \textbf{ Nam cum ira peruertat iudicium rationis , } non decet Reges et Principes esse iracundos , & esto non es cosa guaue mas ligera . \textbf{ Ca por que la yr a | tristorna el iuyzio dela razon } e del entendimiento non conuiene alos Reyes \\\hline
1.2.27 & non decet Reges et Principes esse iracundos , \textbf{ cum in eis maxime vigere debeat ratio et intellectus . Sicut enim videmus } quod lingua infecta per coleram , & et alos prinçipes de seer sannudos \textbf{ por que en ellos mayormente deue seer apoderada la razon e el entendemiento | que en otros ningunos } Ca assi commo veemos \\\hline
1.2.27 & ut fiant punitiones et vindictae , \textbf{ cum hoc faciat mansuetudo , } decet eos mansuetos esse . & para fazer uenganças e dar penas . \textbf{ Et commo esto faga la manssedunbre } conuiene a ellos de ser manssos \\\hline
1.2.28 & de virtutibus respicientibus bona exteriora , \textbf{ ut communicamus cum aliis . } Communicamus autem cum aliis , & que catan alos bienes de fuera \textbf{ segunt que partiçipamos con los otros . } Mas nos partiçipamos con los otros en palauras e en obras . \\\hline
1.2.28 & ut communicamus cum aliis . \textbf{ Communicamus autem cum aliis , } verbis , et operibus . & segunt que partiçipamos con los otros . \textbf{ Mas nos partiçipamos con los otros en palauras e en obras . } Et las palauras et las obras \\\hline
1.2.28 & et operibus debite \textbf{ conuersamur cum aliis , } honorando eos , & Ca en quanto por las palauras e por las obras \textbf{ conueniblemente nos auemos con los otros honrrando los } e resçibiendo los \\\hline
1.2.28 & de qua hic determinare intendimus , \textbf{ nisi recte conuersari cum hominibus , } et ordinare opera , & entendemos aqui determinar \textbf{ si non derechamente beuir con todos } e ordenar las nr̃as palauras \\\hline
1.2.28 & circa verba et opera , \textbf{ in quibus cum aliis communicamus , } habet esse triplex virtus , & Et pues que assi es assi commo es dicho de suso çerca las palauras \textbf{ e çerca las obras en las quales partiçipamos con los otros han de ser tres uirtudes } conuiene saber ¶amistança \\\hline
1.2.28 & quam eutrapeliam vocat . \textbf{ Communicando igitur cum aliis , } si bene conuersari volumus , & que quiere dezir buena conpanina . \textbf{ Et pues que assi es si quisieremos bien couerssar partiçipando con los otros } deuemos seer alegres conueniblemente \\\hline
1.2.28 & Sed primo de amicabilitate . \textbf{ Videmus enim quod conuersando cum aliis , } aliqui superabundant , & mas primero diremos dela amistan ca por que ueemos que en partiçipando \textbf{ e en conuerssando con los otros } algunos sobrepuian por que se muestran mucho amigables \\\hline
1.2.28 & et agrestes , \textbf{ non valentes cum aliis conuersari . } Uterque autem a recta ratione deficiunt , & e montesmos \textbf{ que no saben beuir con los otros . } Mas cada vno destos fallesçen en cada vna destas razo nes \\\hline
1.2.28 & ut uideatur discolus , et litigiosus . \textbf{ Cum igitur uirtus sit } quid medium inter superfluum et diminutum , & por que sea visto desacordable e vara ador . \textbf{ Et pues que assi es commo la uirtud sea medianera } entre la cosa sobrepuiante e la . \\\hline
1.2.28 & et opera , \textbf{ in quibus communicat cum aliis , } dare uirtutem aliquam , & e çerca las obras \textbf{ en las quales el omne partiçipa con los otros } de dar alguna uirtud \\\hline
1.2.28 & per quam debite conseruetur . \textbf{ Quare cum recta ratio dictet , } quod secundum diuersitatem personarum & por la qual conueniblemente sepa conuerssar e beuir con los otros . \textbf{ Por la qual cosa commo la razon derecha } e el entendemiento muestre \\\hline
1.2.29 & a veritate recedit ratione defectus . \textbf{ Quare cum mendacium sit semper fugiendum , } ut dicitur 4 Ethicorum , & en razon de fallesçimiento . \textbf{ Por la qual cosa commo la mentira | por si misma sea mala deuemos foyr della } assi commo dize aristotiles \\\hline
1.2.29 & Concedunt enim de se aliqui magnas bonitates , \textbf{ cum tamen illis careant : } et promittunt amicis & Ca otorgan alguons de ssi mismos grandes bondades \textbf{ commo quier que en ellos nen sean } e prometen alos amigos conosçidos grandes bienes e grandes ayudas . \\\hline
1.2.29 & in cap’ praetacto , \textbf{ cum ait , } quod prudentis est declinare in minus . & que dicha es \textbf{ quando dize } que pertenesçe al sabio de declinar alo menos . \\\hline
1.2.30 & et laboribus nostris . \textbf{ Quare cum in talibus contingat peccare , } et bene facere , & e non baldias nin en vano . \textbf{ Por la qual cosa sientales cosas | contesçe de pecar } e de bien fazer \\\hline
1.2.30 & quorum curam habere debemus . \textbf{ Quare cum iocus immoderatus , vel inhonestus distrahat nos a bonis operibus , } et a debitis curis : & por la qual cosa commo el iuego non te prado \textbf{ nin honesto nos parta | e nos tire delas buenas obras } e de los cuydados conuenibles \\\hline
1.2.31 & omnibus caret . \textbf{ Cum ergo omnino manifestum sit , } quod decet Reges & non ha ningunan de todas las otras \textbf{ Et pues que assi es commo en todo en todo es manifiesto e prouado } que conuiene alos Reyes \\\hline
1.2.31 & Prudentia vero rectificat viam . \textbf{ Sed cum non sit perfecta via } nisi ordinetur in bonum finem et terminum , & e enderesça el camino a aquellas cosas \textbf{ que son ordenadas a aquella fin . | Mas commo non puede seer camino } e carrera acabada \\\hline
1.2.31 & nisi sit prudens . \textbf{ Nam cum virtus moralis sit } habitus bonus , & si non fuere prudente e sabio \textbf{ Ca commo la uirtud moral sea habito e disposiçion firme de alma e buena escogedora } e acaba a aquel que la ha \\\hline
1.2.31 & et opus suum bonum reddat : \textbf{ cum ad bene eligere , } et ad bonum opus , & e faga la su obra buena . \textbf{ Por ende commo havien escoger } e a buena obra fazer \\\hline
1.2.31 & quia perfecte una virtus sine aliis haberi non potest . Immo expedit Regibus et Principibus , \textbf{ cum non possint se excusare } per defectum exteriorum bonorum , & mas ante conuiene alos Reyes \textbf{ e alos prinçipes | commo ellos non se puedan escusar } por mengua de los bienes \\\hline
1.2.32 & Tales ergo nihil difficile sustinere volentes , \textbf{ statim cum passionantur , } vel cum tentantur , cadunt . & Et por ende estos tales non quariendo sofrir ninguna cosa guaue \textbf{ luego que padelçen o son passionados por alguna passion } e quando son tentados \\\hline
1.2.32 & statim cum passionantur , \textbf{ vel cum tentantur , cadunt . } In alio gradu dicuntur & luego que padelçen o son passionados por alguna passion \textbf{ e quando son tentados | luego } ca en estos tales son dichͣs muelles . \\\hline
1.2.32 & quod \textbf{ cum quaedam praegnans esset , } et non potuisset parere , & entre las quales dize \textbf{ que commo fuesse vna mugier prenada } e non pudiesse parir desto conçibio tan grant dolor \\\hline
1.2.32 & praestabant sibi filios inconuiuiis . \textbf{ Cum enim qui alios conuiuare volebat , } si filius suus domi non erat , & que presta un a los sus fiios en los conbites a sus vezinos que los comiessen . \textbf{ Ca quando alguno quaria conbidar a } otrossi el su fiio non era en casa tomaua prestado el fijo de otro su vezino \\\hline
1.2.33 & nefas est turpia nominari : \textbf{ sed cum tantae bonitatis nullus esse possit } absque Dei gratia , & assi commo dize plotino aquel philosofo grand pecado es de nonbrar cosas torpes . \textbf{ Et commo ninguno non pueda ser | de tan grand bondat } sin la gera de dios \\\hline
1.2.34 & et immobiliter operari . \textbf{ Quare cum cabulia consilietur , } synesis iudicet , & Et lo terçero que firmemente e si mouemiento obre . \textbf{ Por la qual cosa commo esta uirtud | que es dicha eubolia conseie } Et esta otra que llaman sinesis \\\hline
1.2.34 & quam virtus . \textbf{ Sed cum continentia potior sit , } quam perseuerantia & ala uirtud que uirtud . \textbf{ Mas commo la continençia sea meior } e mayor que la perseuerança tomando la perseueraça \\\hline
1.2.34 & sed quid honorabilius virtute . \textbf{ Cum ergo prudentia , iustitia , et alia , } de quibus superius tractauimus , & mas es alguna cosa mas honrrada que uirtud . \textbf{ Et pues que assi es commo la pradençia e la iustiçia e las otras } uirtudesde que tractamos de suso \\\hline
1.3.1 & et quos motus animi Reges et Principes debeant imitari . \textbf{ Sed cum hoc sciri non possit , } nisi prius sciuerimus & deuen seguir los Reyes e los prinçipes . \textbf{ Mas por que esto non se puede saber } si non sopieremos primero \\\hline
1.3.1 & passionem oppositam irae . \textbf{ Sed cum sit quaedam virtus } inter iram et mansuetudinem , & Ca la manssedunbre nonbra propriamente passion contraria ala saña . \textbf{ Mas por que ha de ser alguna uirtud entre la sanna e la mansedunbre } la qual uirtud non podemos nonbrar \\\hline
1.3.1 & eo quod illa virtus \textbf{ plus communicat cum mansuetudine , } quam cum ira . & por su nonbre propreo nonbramos la por nonbre de mansedunbre \textbf{ por que aquella uirtud mas conuie ne e partiçipa con la manledunbre que conla sana . } Et pues que assi es sera la \\\hline
1.3.1 & plus communicat cum mansuetudine , \textbf{ quam cum ira . } Erit mansuetudo aequiuocum ad virtutem , & por que aquella uirtud mas conuie ne e partiçipa con la manledunbre que conla sana . \textbf{ Et pues que assi es sera la } manssedunbre nonbre \\\hline
1.3.1 & inuenire nomen proprium . \textbf{ Sed cum constat de re , } de verbis minime est curandum . & non bͤapio a cada vna cosa podia lo fazer \textbf{ mas quando nos somos çiertos dela cosa non deuemos auer cuydado delas palauras . } Et pues que assi es contadas las passiones \\\hline
1.3.1 & et sic est mansuetudo . \textbf{ Cum ergo non possint } pluribus modis variari nostri motus et nostrae affectiones , & e assi es manssedunbre . \textbf{ ¶ Et pues que assi es conmo los nuestros mouimientos del alma } et las nr̃as afectiones e passiones \\\hline
1.3.2 & Nam desiderium immediate innititur amori , \textbf{ cum enim amamus aliquid , } uel desideramus ipsum habere , & Ca el desseo sin ningun medio se ayunta luego al amor . \textbf{ Ca quando amamos alguna cosa } luego desseamos auer aquella cosa . \\\hline
1.3.2 & Abominatio uero immediate innititur odio : \textbf{ quia statim cum odimus , } aliquid abominamur illud . & Mas la aborrençia sin ningun medio se ayunta ala mal querençia \textbf{ por que luego commo queremos mal a alguna persona . } luego la aborresçemos . \\\hline
1.3.2 & Nam spes , et desperatio , \textbf{ cum sumantur respectu boni , } praecedunt timorem , et audaciam , iram , et mansuetudinem , & Maen el tercero logar son de poner la esperançar la desesꝑança . \textbf{ Ca por que son tomadas } por razon de bien son puestas primero que el temor e la oladia \\\hline
1.3.2 & et mansuetudinem . \textbf{ Nam cum aliquid prius sit futurum , } quam praesens : & que sasanna e la manssedunbre . \textbf{ Ca commo algunan cosa primero sea futura de uenir } ante que sea presente . \\\hline
1.3.3 & ut sciremus quo ordine determinaremus de illis . \textbf{ Quare cum amor , } et odium sint passiones primae , & por qual orden determinariemos dellas . \textbf{ Por la qual cosa commo el amor e la mal querençia } sean las primeras passiones \\\hline
1.3.3 & posset illud reficere . \textbf{ Quare cum nullus homo } sine diuino auxilio possit & e perdido dios donde el quisiese lo podria refazer \textbf{ por la qual cosa ningun omne sin ayuda de dios non pue da } assi mismo fazer bueno o guardar \\\hline
1.3.3 & ex naturali enim instinctu \textbf{ cum quis vult percuti , } ne vulnerentur membra & por inclinacion natural \textbf{ quando alguno ha de ser ferido . } por que non sean los mienbros llagados \\\hline
1.3.3 & Regnum , et tyrannides : \textbf{ cum modus amoris tyrannici sit } ut bonum priuatum praeponat bono communi , & Mas si los Reyes e los tyranos se han en manera contraria \textbf{ por que la manera del amor del tirano es ante poner e preçiar } mas el bien propio \\\hline
1.3.3 & esse circa diuina , et communia . \textbf{ Erit fortis ; quia cum bonum cumune proponat bono priuato , } non dubitabit etiam personam exponere , & prinçipalmente ha de ser çerca los bienes diuinales e comunes . \textbf{ Otrosi sera fuerte por que ante pone el bien comunal bien propreo } e avn non dubdara de poner la persona a muerte siuiere \\\hline
1.3.3 & Unde et Valerius Maximus de Dionysio Ciciliano recitat , \textbf{ qui cum esset tyrannus , } erat amator proprii commodi , & Onde ualerio maximo cuenta de dionsio seziliano \textbf{ que commo fuesse tyrano } e amador de propio prouecho despoblaua \\\hline
1.3.3 & et uniuersaliter omnia vitia . \textbf{ Quare cum de ratione odii sit exterminare , } et nunquam satiari nisi exterminet , & Et generalmente todos males e todos pecados \textbf{ por la qual cosa commo de razon dela mal querençia sea matar } e nunca se fartar \\\hline
1.3.4 & est tria considerare . \textbf{ Nam cum bonum aliquod apprehendimus , } primo per amorem & deuemos penssar estas tres cosas . \textbf{ Ca quando aprendemos algun bien primeramente por el amor } e por algun plazer somos confirmados e ayuntados a el \\\hline
1.3.4 & et mensuram sanitatis . \textbf{ Cum ergo in arte regnandi et principandi principaliter } et finaliter intendatur salus Regni et Principatus , & segunt medida e manera de salud . \textbf{ Et pues que assi es commo en la arte del regnar | e de enssennorear prinçipalmente } e finalmente se entiende la salud del Regno \\\hline
1.3.5 & in tertio libro diffusius ostendetur . \textbf{ Cum determinauimus de ordine passionum animae , } diximus quod amor et odium erant passiones primae , & en el terçero libro lo mostraremos mas conplidamente \textbf{ uando determinamos dela ança orden delas passiones del alma dixiemos } que el amor e la malqreçia eran las primeras passiones \\\hline
1.3.5 & quod eos esse decet humiles et magnanimos : \textbf{ cum ergo humilitas moderet spem , } quia humiles cognoscentes defectum proprium , & que conuenia alos Reyes de ser humildosos e de ser mag̃nimos . \textbf{ Et por ende por que la humildat tienpra la esperança } ca los humildosos conosçiendo su propre o fallesçemiento non esperan \\\hline
1.3.5 & potissime competere debent Regibus et Principibus . \textbf{ Nam cum Reges et Principes sint latores legum , } quia secundum Philosophum proprie spectat & e nesçera los Reyes \textbf{ e alos prinçipes . | Ca conmolos Rayes sean fazedores e conponedores delas leyes . } Ca segunt el philosofo propriamente pertenesçe alos Reyes \\\hline
1.3.5 & debet esse bonum diuinum et commune , \textbf{ cum talia sint bona excellentia et ardua , } non solum spectat ad Reges & assi commo mostramos de suso deue ser el bien diuinal e comunal \textbf{ por que tales bienes son bienes mas sobrepiunates | e mas altos que los otros } Por ende non solamente parte nesçe alos Reyes \\\hline
1.3.5 & ideo maiori indiget prouidentia et saniori consilio : \textbf{ cum ergo prouidentia et consilium non sit } nisi de rebus futuris , & e de mas sano consseio \textbf{ Et pues que assi es commo la prouidençia | e el consseio non sean sinon de las cosas } que han de venir . \\\hline
1.3.5 & et magno honore digna . \textbf{ Quare cum Reges et Principes } tendere debeant in bona ardua , & e tan dignos de grand honrra . \textbf{ Por la qual cosa commo los Reyes | e los prinçipes } de una entender alos bienes altos e grandes \\\hline
1.3.5 & Decet enim eos \textbf{ cum magna diligentia inuestigare , } quid sperent , & Ca conuiene alos Reyes de cuydar \textbf{ e escodinar con grand diligençia } que cosa es aquello que deuen esparar \\\hline
1.3.5 & quae non valent perficere . \textbf{ Cum ergo regium officium requirat hominem prudentem } et non passionatum immoderata passione , & ¶ \textbf{ Et pues que assi es conmo el ofiçio de los Reyes | demande omne sabio } e non passionado \\\hline
1.3.6 & quod dubitat . \textbf{ Cum ergo unum totum regnum } absque magno consilio debite gubernari non possit , & de que dubda e teme . \textbf{ Et pueᷤ que assi es commo todo vn regno non pueda ser conueniblemente gouernado sin grand conseio } conuiene alos Reyes e alos prinçipes de auer algun temor tenprado \\\hline
1.3.6 & Quarto facit eum inoperatiuum . \textbf{ Cum enim quis timet , } calor ad interiora progreditur ; & que non puede obrar \textbf{ por que quando alguno teme la calentura natural tornasse alos mienbros de dentro . } Ca segunt la manera que nos veemos en todas las cosas podemos \\\hline
1.3.6 & in calore corporis naturalis . \textbf{ Cum enim homines existentes } in campis timent , & lo ueer en la calentura del cuerpo natural . \textbf{ Ca quando algunos omes que estan en los canpos } temen \\\hline
1.3.6 & statim confugiunt ad castrum , vel ad arcem : \textbf{ sic cum quis timet , } calor existens in exterioribus membris , & luego fuyen alos castiellos e alas torres . \textbf{ En essa misma manera } quando alguno teme la calentura natural \\\hline
1.3.6 & quia immoderatus timor facit hominem inconciliatiuum . \textbf{ Cum enim quis immoderate timet , } totus obstupescit , & e sin razon fazen al omne sin conseio . \textbf{ Ca quando alguno teme destenpradamente } e sin razon todo finca atomeçido \\\hline
1.3.6 & et nescit quid faciat . \textbf{ Cum ergo totum regnum sit } in Rege tanquam in mouente , & e non se puede mouer \textbf{ e non sabe que se faza ¶ Et pues que assi es commo todo el regno sea en el Rey } assi commo en su mouedor \\\hline
1.3.7 & Quia ira maximam affinitatem videtur \textbf{ habere cum odio , } Prius quam ostendamus , & que ha grant ayuntamiento \textbf{ e grand vezindat con la mal querençia . } primeronte que mostremos en qual manera los reyes \\\hline
1.3.7 & Statim enim , \textbf{ cum scimus aliquem esse malum , } ut cum scimus aliquem esse furem , & que parte nesçen asi mismo o a otro . \textbf{ Por que luego quando sabemos } que alguno es ladron podemos le mal querer \\\hline
1.3.7 & cum scimus aliquem esse malum , \textbf{ ut cum scimus aliquem esse furem , } possumus ipsum odire , & Por que luego quando sabemos \textbf{ que alguno es ladron podemos le mal querer } si quiera aya fecho mal a nos o a \\\hline
1.3.7 & nisi alicui speciali . \textbf{ Nam cum homo in communi non iniurietur nobis , } sed semper committatur iniuria & si non a alguno en espeçial . \textbf{ Ca commo el omne en comun non faga iniuria nin tuerto a nos } mas sienpre el tuerto o la imiuria es acometida \\\hline
1.3.7 & Sed odienti quidem nihil differt : \textbf{ nam cum odium sit mali } secundum se et absolute , & Mas por çierto el que quiere mal a otro non cura desto . \textbf{ Ca commo la mal querençia sea algun mal segunt si } e sueltamente abasta el mal quariente \\\hline
1.3.7 & Sexta differentia est , \textbf{ quia ira semper est cum tristitia : } tanta enim est anxietas irati & La sexta diferençia es \textbf{ que la saña es sienpra contsteza } por que tanta es la quexura del coraçon del sañudo \\\hline
1.3.7 & non autem odio . \textbf{ Nam cum ira satietur , } si multa mala inferantur alteri , & Mas la mal querençia non \textbf{ Porque commo la sanna se pueda fartar } si muchos males fueren fechos al otro \\\hline
1.3.7 & ei sed odium pro nullo miserebitur , \textbf{ cum sit quid insatiabile . } Octaua differentia est : & Mas la mal querençia de ninguno non se apiada por que es cosa \textbf{ que se non farta . } ¶ La octaua diferençia es \\\hline
1.3.7 & sed vult eum interimi et non esse . \textbf{ Cum ergo conditiones odii sint multo peiores , } quam conditiones irae , & e non sea . \textbf{ Et pues que assi es commo las condiconnes dela mal querençia | sean mucho peores } que las condiconnes dela saña . Mas nos deuemos guardar dela mal querençia \\\hline
1.3.7 & Serui enim veloces statim \textbf{ cum audiunt verbum Domini , } antequam plene percipiant praeceptum eius , & Ca los sieruos ligeros \textbf{ luego commo oyen la palaura del sennor } ante que entiendan conplidamente el mandamiento del corren \\\hline
1.3.7 & Sic etiam et canes statim \textbf{ cum audiunt sonitum venientis , } latrant , non distinguentes , & En essa misma manera avn los canes \textbf{ luego que oy en el sueno | de aquel que viene } luego ladran \\\hline
1.3.7 & Sic et ira facit : \textbf{ statim enim cum ratio dicit } vindictam esse fiendam , & Ca luego que la razon \textbf{ e el entendimi | ento dize } que sea techa vengança \\\hline
1.3.7 & impedimur ab usu rationis , \textbf{ quare cum per iram accendatur sanguis circa cor , } corpus redditur intemperatum , & Ca el cuerpo non estando en tenpramiento conuenible somos enbargados en el vso dela razon . \textbf{ Por la qual cosa commo por la saña se ençienda la sangre cerca el coraçon tornasse el cuerpo destenprado } e non podemos conueniblemente vsar de la razon . \\\hline
1.3.7 & potest esse ordinata , et imitanda . \textbf{ Nam cum ira est rationis organum , } et agit secundum imperium rationis , & e es de segnir . \textbf{ Ca quando la sana es organo } et estrumento de la razon e del entendimiento e obra \\\hline
1.3.8 & ponit aliquam delectationem esse prosequendam . \textbf{ Nam cum loquela non possit negari , } nisi per loquelam , & En essa misma manera el que pone que toda delectaçiones de esquiuar e de foyr pone que alguna delectaciones de segnir . \textbf{ Ca assy commo la fabla non puede ser negada sinon por la fabla . } Ca nengando la fabla fabla el omne en fablando otorga e pone la fabla . \\\hline
1.3.8 & Rursus quia delectatio contingit \textbf{ ex coniunctione conuenientis cum conuenienti : } cum ergo alia conueniant bestiis , alia hominibus : & e segunt alguna parte . \textbf{ Otrossi por que la delectacion se faze por ayuntamiento dela cosa conuenible con cosa conuenible } Por ende commo algunas cosas conuengan alas bestias \\\hline
1.3.8 & ex coniunctione conuenientis cum conuenienti : \textbf{ cum ergo alia conueniant bestiis , alia hominibus : } aliquae delectationes sunt conuenientes bestiis , & Otrossi por que la delectacion se faze por ayuntamiento dela cosa conuenible con cosa conuenible \textbf{ Por ende commo algunas cosas conuengan alas bestias | e algunos alos omes } algunans delectaçiones seran conuenientes alas bestias \\\hline
1.3.8 & ad ipsas delectationes . \textbf{ Nam cum detestabile sit } cuilibet quod sit vitiosus , & Pues que assi es paresçe en qual manera nos deuemos auer a estas delecta connes . \textbf{ Ca commo cosa denostable sea a cada vn } çibdadano \\\hline
1.3.8 & et aliud passione agunt . \textbf{ Quare cum in seipsis pacem non habeant , } de seipsis non gaudent . & e lo al fazen despues por la obra \textbf{ Por la qual cosa commo ellos non ayan paz en si mismos non gozan de ssi mismos . } Et por ende grand remedio es anos \\\hline
1.3.8 & Sicut ergo in pondere corporali \textbf{ cum multi iuuant nos } ad portandum illud , & Et por ende assi commo en el peso corporal \textbf{ quando muchos nos ayudan a leuar aquel peso menos nos aguauiamos del peso . } en essa misma manera quando ueemos muchedunbre de amigos que se duelen \\\hline
1.3.8 & minus grauamur : \textbf{ sic cum videmus multitudinem amicorum } condolore nobis , & quando muchos nos ayudan a leuar aquel peso menos nos aguauiamos del peso . \textbf{ en essa misma manera quando ueemos muchedunbre de amigos que se duelen } connusco o se duelen de nos \\\hline
1.3.8 & nec videtur usque quaque vera . \textbf{ Nam cum de dolore amicorum sit dolendum , } cum nos dolemus , & que en todas maneras sea uerdadera \textbf{ Ca quando nos dolemos del dolor de los amigos dolemos nos nos } e veemos los amigos doler . \\\hline
1.3.8 & Nam cum de dolore amicorum sit dolendum , \textbf{ cum nos dolemus , } et videmus amicos dolere , & que en todas maneras sea uerdadera \textbf{ Ca quando nos dolemos del dolor de los amigos dolemos nos nos } e veemos los amigos doler . \\\hline
1.3.8 & Possumus ergo dicere \textbf{ quod cum videmus eos dolere } de dolore nostro , & Et pues que assi es podemos dezir \textbf{ que quando nos veemos los amigos doler se del nuestro dolor } non es menguado el nuestro dolor \\\hline
1.3.8 & et talia quae tristitiam fugare solent . \textbf{ Cum ergo tales tristitiae } impediant operationes virtuosas , & cales que suelen alongar \textbf{ e arredrar la tristeza Et pues que assi es commo tales obras de tristeza } enbarguen las obras de uirtudes tanto \\\hline
1.3.9 & ut habet rationem ardui . \textbf{ Cum ergo aliquid maxime sit bonum , } cum iam est adeptum , & Mas el appetito enssannador va a bien en quanto aquel bien ha razon de ser alto e grande . \textbf{ Et por ende commo alguna cosa | entonçe sea dicha muy buean } quando ya es ganada \\\hline
1.3.9 & Cum ergo aliquid maxime sit bonum , \textbf{ cum iam est adeptum , } circa quod est delectatio et gaudium ; et maxime sit malum , & entonçe sea dicha muy buean \textbf{ quando ya es ganada } çerca la qual es la delectaçion e el gozo . \\\hline
1.3.9 & circa quod est delectatio et gaudium ; et maxime sit malum , \textbf{ cum iam est adeptum , } circa quod est dolor et tristitia & Et otrossi alguna cosa es muy mala \textbf{ quando ya es ganada } terca la qual es el dolor e la tristeza . \\\hline
1.3.9 & Spes autem et timor sunt principales passiones respectu irascibilis . \textbf{ Nam cum irascibilis tendit } in bonum et malum & en conparacion del appetito enssañador . \textbf{ Por que el appetito enssannador va a bien o a mal en quanto es alto e grande . } Et por que estonçe es tenido el bien \\\hline
1.3.9 & maxime bonum arduum , \textbf{ cum est futurum et speratur : } et malum arduum , & por muy bueno e muy alto \textbf{ quando es futuro e ha de uenir e es esparado . } Et el mal es alto e grande \\\hline
1.3.9 & et malum arduum , \textbf{ cum est futurum et timetur : } spes et timor sunt principales passiones respectu irascibilis . & Et el mal es alto e grande \textbf{ quando es futuro que es de uenir e es tenido . } Et por ende la esperança e el temor son passiones prinçipales \\\hline
1.3.9 & spes et timor sunt principales passiones respectu irascibilis . \textbf{ Sed cum ex passionibus diuersificari habeant opera nostra , } decet nos diligenter intendere , & en conparacion del appetito enssannador . \textbf{ Mas commo las nr̃as obras ayan de ser departidas | por estas passiones . } Cconuiene a nos de acuçiosamente entender \\\hline
1.3.10 & Si sunt corporalia , \textbf{ quia talia cum habentur ab uno , } non habentur ab alio , & o son spun al ssi son corporales \textbf{ por que tales quando son auidas de vno non son auidas de otro . } Por ende se suele difinir \\\hline
1.3.10 & amittere interiora bona . \textbf{ Sed cum quis verecundatur , } sanguis fluit ad exteriora , & para esforçar los mienbros de dentro . \textbf{ Mas quando alguno ha uerguença } la sangre corre alos mienbros de fuera \\\hline
1.4.1 & non acquisiuerunt proprio labore . \textbf{ Nam quilibet cum maiori diligentia retinet facultates suas , } quando propter indigentiam passus est aliqua mala , & por su trabaio propreo . \textbf{ Ca cada vno con mayor acuçia guarda las sus riquezas | e el su auer } quando ha sofrido alguons males \\\hline
1.4.1 & et proprio labore . \textbf{ Nam quod cum labore acquiritur , } diligentius custoditur , & por su sabiduria propia o por su trabaio proprao . \textbf{ Ca aquella cosa que es ganada con trabaio } con mayor acuçia es guardada e retenida . \\\hline
1.4.1 & et secundum cursum naturalem debent multum viuere in futuro . \textbf{ Cum ergo memoria sit respectu praeteritorum , } et spes respectu futurorum : & que ha de uenir . \textbf{ Et pues que assi es commo la memoria sea en conparaçion del tienpo passado | por que es recordaçion delas cosas } que passaron e la esperança es en conparacion \\\hline
1.4.1 & Iuuenes ergo , \textbf{ cum sint liberales , et cum sint animosi et bonae spei , } non habent unde retrahantur & e entremetesse de fazer grandes cosas . \textbf{ Et pues que assi es commo los mancebos } non ayan ninguna cosa \\\hline
1.4.1 & Posset etiam ad hoc specialis ratio assignari . \textbf{ Nam cum iuuenes sint percalidi , } et calidi sit superferri : & Et avn podemos aesto adozir otra razon espeçial . \textbf{ Ca por que los mançebos son muy calientes } e la calentura quiere sienpre sobir arriba e sobrepuiar . \\\hline
1.4.1 & videlicet , terrae et aquae . \textbf{ Cum ergo inter caetera , } per quae quis videtur & assi commo es la tierra e el agua . \textbf{ Et pues que assi es conmo entre todas las cosas } por las quales cada vno quiere sobir \\\hline
1.4.1 & est timor inglorificationis . \textbf{ Cum ergo iuuenes , } qui percalidi nimis affectent excellere , & assi commo es dicho es temor de non auergłia nin honra . \textbf{ Et pues que assi es por que los mançebos } por la calentura natural desse an \\\hline
1.4.1 & per quas iustruuntur pueri . \textbf{ Cum ergo matres semper moneant } suos filios ad honesta ; & por los quales los mocos sson ensseñados e doctrinados . \textbf{ Et pues que assi es commo las muger } ssienpre amonestan asus fijos a honestad \\\hline
1.4.2 & quod duplici de causa contingit . \textbf{ Nam cum iuuenes sint percalidi , } et corpore calefacto fiat venereorum appetitus , & la qual cosa les contesçe por dos razones ¶ \textbf{ La primera razon es que por que los mancebos son muy calient s̃ } e el cuerpo escalentado faze appetito \\\hline
1.4.2 & sed sua innocentia alios mensurant . \textbf{ Cum ergo naturale sit , } quod quis de facili credat ei , & e por la su sinpleza mesuran alos otros . \textbf{ Et pues que assi es commo natural cosa sea } que qual quier omne de ligero cree a aquel que cuyda que es bueno . \\\hline
1.4.2 & quia non sunt multorum experti , \textbf{ statim cum eis aliquod proponitur negocium , } non valentes ad multa respicere , & Et pues que assi es por que los mançebos non son prouados nin han esperiença de muchͣs cosas \textbf{ luego que les dizen algo | e les proponen algun negoçio } non pueden catar a muchͣs cosas \\\hline
1.4.2 & eos esse permutabiles , et vertibiles . \textbf{ Nam cum inconueniens sit } regulam esse obliquam , & alos reys de ser mudables e trastornables . \textbf{ Porque cosa desconuenible es que la regla sea tuerta } e ellos son commo regla . \\\hline
1.4.2 & eos esse nimis creditiuos . \textbf{ Nam cum multos habeant adulatores , } et plurimi sint in eorum auribus susurrantes , & e alos prinçipes de çreer de ligero . \textbf{ Ca commo ellos ayan muchos lisongeros } e muchos les estenruyendo alas oreias \\\hline
1.4.2 & et plurimi sint in eorum auribus susurrantes , \textbf{ cum maxima diligentia cogitare debent , } qui sunt qui loquuntur , & e muchos les estenruyendo alas oreias \textbf{ deuen penssar con grand acuçia | commo les fabla cada vno } o que les son aquellos que les fablan \\\hline
1.4.2 & in actionibus suis : \textbf{ quia cum alia sint moderanda per mensuram , } maxime decet mensuram & Ca commo todas las sus obras \textbf{ de una ser tenp̃das | por me lura } e por regla mucho \\\hline
1.4.3 & naturaliter passionatur passione illa . \textbf{ Cum ergo timidi efficiantur frigidi , } quicunque est naturaliter frigidus , & natraalmente es passionado de aquella passion \textbf{ Et pues que assi es commo los temerosos sean esfriados . } Et qual si quier que naturalmente es frio naturalmente es temeroso \\\hline
1.4.3 & Videmus autem quod \textbf{ cum senes adinuicem congregantur , } semper recitant res gestas , & en que se delecta . \textbf{ Ca nos veemos que quando los uieios se ayuntan en vno } sienpre cuentan las cosas passadas \\\hline
1.4.3 & Verecundia ergo , \textbf{ cum sit timor inhonorationis , } non competit senibus ; & en el segundo libro dela Rectorica \textbf{ Ca por que la uerguença es temor de desonera non pertenesçe alos uieios } por que may orcuidado han del prouecho \\\hline
1.4.3 & immo fortes et magnanimos : \textbf{ quia cum negocia respicientia totum regnum , } circa quae Reges et Principes insudare debent , & mas conuiene les de ser fuertes e de grandes coraçones \textbf{ Por que conmo los negoçios e los fech̃d | que pertenesçen a todo el regno } çerca las quales los Reyes \\\hline
1.4.3 & ipsos esse illiberales . \textbf{ Nam supra cum de virtutibus tractabatur , } sufficienter ostensum fuit , & ca por esto serien denostados . \textbf{ Ca assi commo dicho es de suso quando tractamos delas uirtudes suficientemente fue mostrado } que non solamente conuiene alos Reyes \\\hline
1.4.4 & quomodo senes sunt illiberales . \textbf{ Nam cum per illiberalitatem } contingit peccare dupliciter . & en qual manera los uieios son escassos . \textbf{ Ca commo contezca alos omes de pecar } por la escasseza en dos maneras ¶ \\\hline
1.4.4 & quam se extendant ad alia . \textbf{ Cum enim nulla sit actio animae , } in qua non utatur & que non se estiendan alas otras cosas que non han . \textbf{ Ca commo non sea ninguna obra del alma en el cuerpo } en la qual non vse el alma \\\hline
1.4.4 & sed quia quilibet \textbf{ cum est in imbecillitate , } vel in defectu , & o por que ellos sean amadores de amistanças . \textbf{ Mas porque cada vno quando es en flaqueza } e en fallestimiento \\\hline
1.4.4 & et nihil fixe pronunciant ; \textbf{ cum enim ab eis quaeritur } de aliquo negocio , & nin sentençian ninguna cosa firmemente . \textbf{ Ca quando alguno les demanda de algun fecho responden } que por auentra a assi es o que en algua manera \\\hline
1.4.4 & et inuiriles ut senes : \textbf{ sed sunt viriles cum temperantia , } et temperati cum virilitate . & en el segundo sibro de la rectorica \textbf{ son esforçados contenprànça } e tenprados con uirtud . \\\hline
1.4.4 & sed sunt viriles cum temperantia , \textbf{ et temperati cum virilitate . } Ut ergo sit ad unum dicere , & son esforçados contenprànça \textbf{ e tenprados con uirtud . } Por ende se \\\hline
1.4.4 & eos habere decet . \textbf{ Nam cum Reges , et Principes magis debeant } viuere ratione quam passione , & en quanto talos costunbres son de loar . \textbf{ Ca commo los Reyes | e los prinçipes mas de una beuir } por razon que por passion dela carne \\\hline
1.4.5 & quod semper effectus vult assimilari causae : \textbf{ cum filii sint } quidam effectus parentum , & Ca natural cosa es \textbf{ que sienpre la fechura quiera semeiar a su fazedor } por que los fijos son fechuras de los padron natural cosa es que los fuos semeien alos paradres . \\\hline
1.4.5 & si ab antiquo affluebat diuitiis . \textbf{ Cum ergo semper sit dare initium , } in quo genitores alicuius ditari inceperunt : & si de antigo tienpo abondo en riquezas . \textbf{ Et pues que assi es comm sienpre ayamos de dar comienço } en que los padres de alguons comne caron de se enrriqueçer \\\hline
1.4.5 & in filiis quam in parentibus : \textbf{ quare cum nobilitas semper inclinet animum nobilium } ut faciant magna , & antiguadas las riquezas en los fijos que en los padres . \textbf{ Por la qual razon commo la nobleza | sienpre incline el coraçon de los nobles } para fazer grandes cosas siguese \\\hline
1.4.5 & et societate quadam aliorum . \textbf{ Cum enim nobiles } cum magna diligentia nutriantur , & ¶la primera razon le prueua alsi . \textbf{ Ca por que los nobles son cados con grand acuçia } e con grand cura \\\hline
1.4.5 & Cum enim nobiles \textbf{ cum magna diligentia nutriantur , } et cum magna cura proprium corpus custodiant : & ¶la primera razon le prueua alsi . \textbf{ Ca por que los nobles son cados con grand acuçia } e con grand cura \\\hline
1.4.5 & cum magna diligentia nutriantur , \textbf{ et cum magna cura proprium corpus custodiant : } rationabile est , & Ca por que los nobles son cados con grand acuçia \textbf{ e con grand cura | e con grand guarda delos sus cuerpos } con razon es \\\hline
1.4.5 & et bene complexionatum . \textbf{ Cum ergo molles carne aptos mente dicamus , } ut vult Philos’ 2 de Anima : & que ellos ayan los cuerpos bien ordenados e bien conplissionados . \textbf{ Et pues que assi es conmolos bien conplissionados | que son muelles en las carnes sean mas engennosos e sotiles en las almas } assi commo dize el philosofo \\\hline
1.4.5 & omnia aliorum facta commendantes . \textbf{ Cum enim nobiles non reprehenduntur , } sed ab adulatoribus & que alaban falsamente los fechos de los senores . \textbf{ Ca commo los nobles non sean reprehendidos destos lisongos } mas los sus malos fechos sean alabados \\\hline
1.4.5 & Diximus enim supra , \textbf{ cum de virtute tractauimus , } quomodo decet Reges , & Ca ya dixiemos de ssuso \textbf{ quando tractamos delas uirtudes } en qual manera conuiene alos Reyes de ser magranimos \\\hline
1.4.5 & Reges ergo et Principes , \textbf{ cum non possint naturaliter dominari , } nisi sint boni et virtuosi , & Et pues que assi es commo los Reyes \textbf{ e los prinçipes non puedan naturalmente ensseñorear } si non fueren bueons \\\hline
1.4.6 & et volunt videri esse excellentes : \textbf{ cum ex hoc quis excellere videatur , } si potest aliis contumelias inferre : & mas altos que todos los otros . \textbf{ Et commo por tal razon commo esta alguno parezca de ser mas alto } si puede fazer tuertos alos otros \\\hline
1.4.6 & Recitat enim Philosophus 2 Rhetoricorum , \textbf{ quod cum quaesitum fuisset } a muliere quadam , & en el segundo libro de la rectorica \textbf{ que fue demandado a vna muger qual cosa era meior ser rico o ser sabio . } Et ella respondio \\\hline
1.4.7 & et habet multos sub suo dominio ; \textbf{ quare cum multos nobiles videamus esse impotentes , } et non posse principari , & e ha muchos so su sennorio . \textbf{ por la qual cosa commo nos veamos muchos ser nobles | qua non son poderosos } e non pueden ser prinçipes \\\hline
1.4.7 & magis temperati quam diuites . \textbf{ Nam cum quis dat se exercitio , } retrahitur ab ocio : & Lo segundo los poderosos son mas tenprados que los ricos \textbf{ e la razon desto es | que quando alguno se da a algun huso o̊ alguna obra } tyrase de ocçi olidat \\\hline
1.4.7 & retrahitur ab ocio : \textbf{ et cum dat se uni operi , } retrahitur ab alio . & tyrase de ocçi olidat \textbf{ e quando se da a vna obra } tyrase dela otra . \\\hline
1.4.7 & minus reputantur . \textbf{ Quare cum ditati ab antiquo , } magis assueti sint in diuitiis , & tanto menos son presçiados . \textbf{ Por la qual cosa los que son enrriqueçidos de luengo tienpo } mas son acostunbrados en las rriquezas \\\hline
2.1.1 & In hoc ergo secundo libro determinabitur de regimine domus . \textbf{ Sed cum familia domus } sit communitas quaedam , & determinaremos del gouernamiento de la casa \textbf{ mas commo la conpanna o la casa sea vna comunidat } e sea comunidat natraal \\\hline
2.1.1 & esse desinerent ; \textbf{ quare cum viuere sit homini naturale , } omnia illa , & si ansi luego que son fechͣs començassen afallesçer . \textbf{ Por la qual cosa commo el beuir sean | atuer tal cosa al omne } todas aquellas cosas \\\hline
2.1.1 & intelligendum est de cibariis aliis . \textbf{ numquam cum homo existens } solus sufficit sibi & de todas las otras uiandas \textbf{ por que nunca el omne estando señero puede conplir assy mismo } para auer viandas conuenibles \\\hline
2.1.1 & Homini autem non sufficienter prouidet natura in vestitu : \textbf{ cum enim homo sit nobilioris complexionis } quam animalia alia , & Mas la natura non prouee al omne tan conplidamente en uestidura \textbf{ por que el omne es de mas nobł conplission } que las otras aianlias e mas ayna resçibedano \\\hline
2.1.1 & magis habet offendi , quam illa ; \textbf{ quare cum habere victum et vestitum congruat } ad vitam humanam , & ala uida humanal de auer uianda \textbf{ e vestida conuenible } e ninguno non abaste assi mismo sin conpannia de otro \\\hline
2.1.1 & et ab hostibus defendimur . \textbf{ Natura cum aliquibus animalibus } ad sui tuitionem dedit cornua , & para se guardar e defender de los enemi gos . \textbf{ Ca por que la natura dio a algunas anmalias } para su defendemiento cuernos \\\hline
2.1.1 & ideo statim \textbf{ cum audiunt strepitum , } fugam arripiunt . & si non por lignieza del su cuerpo e por foyr . \textbf{ Et por ende luego que oy en algun roydo } luego comiençana foyr . \\\hline
2.1.1 & Quare si naturale est homini desiderare conseruationem vitae , \textbf{ cum homo solitarius non sufficiat sibi } ad habendum congruum victum et vestitum , & Par la qual cosa si natural cosa es al omne de dessear conseruaçion e guarda de su uida \textbf{ commo el omne | que biue solo non abaste assi mismo } para auer uianda conueinble nin uestidura \\\hline
2.1.1 & si nunquam vidisset canes alias peperisse . \textbf{ Mulier autem cum parit , nescit qualiter se debeat habere in partu , } nisi per obstetrices sit sufficienter edocta . & avn que nunca uiesse o trisperras parir . \textbf{ Mas la mug̃r quan do pare non labe | en qual manera se deua auer en el parto } si non fuere enssennada conuenibłmente por las parteras . \\\hline
2.1.1 & Et quia hoc fieri non potest , \textbf{ nisi simul cum aliis conuiuamus : } naturale est homini & Et por que esto non se puede fazer \textbf{ si non biuieremos en vno con los otros omes . } Por ende natural cosa es al omne \\\hline
2.1.1 & naturale est homini \textbf{ simul conuiuere cum aliis , } et esse animal sociale . & Por ende natural cosa es al omne \textbf{ de beuir con los otros omes } e de ser aia la conpannable¶ \\\hline
2.1.2 & quomodo communitas domestica se habet ad communitates alias : \textbf{ cum quaelibet communitas } includat communitatem domesticam , & otrascomuidades \textbf{ commo cada vna delas otras comundades } ençierre en ssi la comunidat dela casa \\\hline
2.1.2 & necessariam esse communitatem domesticam : \textbf{ cum omnis alia communitas communitatem illam praesupponat . } Aduertendum ergo quod & que la comunidat dela casa es neçessaria \textbf{ por que todas las comuni dades ençierran en ssi | e ante ponen esta comunidat dela casa ¶ Et } pues que assy es deuedes saber \\\hline
2.1.3 & Ne laboremus in aequiuoco , \textbf{ cum de domo loquimur , } sciendum quod domus nominari potest & e los regnos siruen al conplimiento dela uida del omne . \textbf{ or que non trabaiemos en vano fablando dela casa } conuiene de saber que la casa algunas uezes \\\hline
2.1.3 & aliquid primo operatum , \textbf{ cum adepto fine cesset operatio , } nunquam operaremur ea , & por que si la fin fuesse primeramente alguna cosa obrada \textbf{ quando ouiessemos ganada la | finçessarie la obra } e nunca obrariemos nada de aquellas cosas \\\hline
2.1.3 & omnes uero aliae sunt perfectiores ipsa ; \textbf{ cum enim omnis alia communitas } includat communitatem domus , & e las otras son mas conplidas que ella \textbf{ por que todas las otras comunidades ençierran en ssi la comunidat dela casa } e ennaden alguna cosa sobre ella . \\\hline
2.1.3 & de prioritate generationis vel temporis , \textbf{ cum ipsemet dicat ciuitatem procedere ex multiplicatione vici , } sicut et vicus procedit ex multiplicatione domorum , & Et esto non se deue entender que es primera por generaçion \textbf{ e por tienpo commo el mismo | diga } que la çibdat se faze de muchos uarrios \\\hline
2.1.3 & et complementi . \textbf{ Amplius cum communitas domus ad communitates alias } non solum se habeat & por manera de perfecçion e de cunplimiento . \textbf{ Otrossi commo la comunidat dela casa } non solamente se aya alas otras comuindades \\\hline
2.1.3 & quomodo sit huiusmodi communitas naturalis . \textbf{ Nam cum natura non praesupponat artem , } sed ars naturam : & en alguna manera esta comunidat dela cała es natural \textbf{ Ca commo la natura non presupone | nin antepone arte } mas el arte presunpone \\\hline
2.1.3 & naturaliter animal communicatiuum et sociale , \textbf{ cum omnis communitas } praesupponat communitatem domus , & ai al comiuncable e aconpanable \textbf{ commo todas las comunidades presupongan } e ante pongan la comunidat dela casa \\\hline
2.1.4 & qualis sit communitas domus : \textbf{ cum ostensum sit } quod homo est naturaliter animal domesticum , & sobredicho qual es la comunidat dela casa . \textbf{ Ca ya mostrado es } que el omne es naturalmente ainalia domestica e de casa \\\hline
2.1.4 & sed oportuit dare communitatem vici , \textbf{ ita quod cum vicus constet } ex pluribus domibus , & mas conuiene de dar comunidat de varrio . \textbf{ Por que commo el uarrio sea fech̃ de muchas casas } aquello que non es fallado en vna casa \\\hline
2.1.4 & confoederare se alteri ciuitati ; \textbf{ quare cum confoederatio ciuitatum utilis sit } ad bellandum hostes , & que aya conpanna e amistança con otra çibdat \textbf{ que la pueda ayudar . | Por la qual consa el amistança delas çibdades es prouechosa } para vençer los enemigos \\\hline
2.1.4 & sit communitas quaedam et societas personarum : \textbf{ cum non sit proprie communitas nec societas ad seipsum , } si in domo communitatem saluare volumus , & e vna conpannia de muchͣs ꝑssonas . \textbf{ Et commo non sea propreamente comunidat | nin conpannia de vno } assi commo si queremos saluar la comuidat dela casa conuiene que ella sea establesçida de muchͣs perssonas \\\hline
2.1.4 & His sic pertractatis , \textbf{ cum communitas domus sit } tam necessaria in vita ciuili , & Et estas cosas assi tractadas \textbf{ commo la comiundat dela casa } sea tan neçessaria \\\hline
2.1.5 & et earum conseruatio . \textbf{ Cum enim generatio } ( ut dicitur 2 Physicorum ) & e la conseruaçiondellas \textbf{ por que assi commo es dicho en el segundo libro de los fisicos } commo la generaçion sea camino e carrera en la natura \\\hline
2.1.5 & sit via in naturam , \textbf{ et cum res naturales } per generationem propriam naturam accipiant , & commo la generaçion sea camino e carrera en la natura \textbf{ et commo las cosas naturales resçiban su naturaleza } propra a por la generaçion \\\hline
2.1.5 & quia haec illam praesupponit . \textbf{ Nam cum generata non possint conseruari in esse } nisi prius per generationem acceperint esse , & e antepone la generaçion . \textbf{ Ca commo las cosas engendradas | non pueden ser conseruadas } nin guardadas en su ser si primeramente non resçibieren el su ser por generaçion . \\\hline
2.1.5 & Sicut enim caecus corporaliter , \textbf{ nisi ( cum pergit ) dirigatur ab aliquo , } de leui obuiat alicui offensiuo : & que si alguno es çiego corporalmente \textbf{ si quando anda non fue regua ado } por alguno otro de ligero \\\hline
2.1.5 & nam naturaliter domini vigent prudentia , et intellectu . \textbf{ Cum ergo vigentes intellectu , } et existentes apti mente , & Por que naturalmente los sennores han mayor sabiduria e mayor entendemiento \textbf{ por la qual cosa commo los que han mayor entendemiento } o son mas sotiles en el alma \\\hline
2.1.5 & Videretur enim forte alicui , \textbf{ quod cum domus prima constet } ex duabus communitatibus , & alo menos son menester y . tres linages de perssonas . \textbf{ Ca por auentra a paresçrie alguno que commo la casa primera } sea establesçida de dos comunidades \\\hline
2.1.6 & Sic ergo saluatio comparatur ad rem generatam : \textbf{ quia statim cum res est genita , } solicitatur natura circa salutem eius ; & assi la saluaçion es conpada ala cosa engendrada \textbf{ que luego que la cosa es } engendrada la natura es acuçiosa çerca de su salud . \\\hline
2.1.6 & quia non statim \textbf{ cum est res naturalis , } potest sibi simile producere , & assi conpado alas cosas natraales \textbf{ por que non puede la cosa natural } luego que es fecha fazer otra semeiante \\\hline
2.1.6 & statim enim , \textbf{ cum natus est homo , } solicitatur natura circa conseruationem ipsius : & enssi \textbf{ ca luego quando nasçe el omne } la natura es \\\hline
2.1.6 & est de ratione hominis iam perfecti . \textbf{ Cum enim primo homo est , } oportet quod sit genitus : & Mas fazer e engendrar semeiante desi es de natura de omne ya acabado . \textbf{ Ca quando el ome es primero } conuiene que sea engendrado . \\\hline
2.1.6 & nisi sit iam perfectus . \textbf{ Quare cum communitas patris ad filium sumat originem } ex eo quod parentes sibi simile produxerunt : & por la qual cosa \textbf{ commo la comunidat del padre al fijo tome nasçençia e comienço de aquello que el padre e la madre } engendran su semeiança esta tal comunidat non es dicha de razon dela primera casa \\\hline
2.1.6 & unde et Phil’ 1 Poli’ \textbf{ cum prius dixisset } communitatem viri et vxoris , domini et serui facere communitatem primam : & ¶ Onde el philosofo en el primero libro delas politicas \textbf{ commo ouiesse dicho primeramente } que la comiundat del omne e dela muger e del sennor e del sieruo fazen la primera casa . \\\hline
2.1.6 & Tunc unumquodque perfectum est , \textbf{ cum potest sibi simile producere . } Ad hoc enim quod aliquid sit perfectum , & estonçe toda cosa es acabada \textbf{ quan do puede fazer | e engendrar su semeiante } Ca para ser la cosa acabada \\\hline
2.1.6 & Impotens autem ad agendum dicitur aliquid , \textbf{ cum praesente proprio passiuo , } non producat sibi simile ; & que non es poderoso de obrar \textbf{ quando tiene su materia proprea presente en que puede obrar } e non puede fazer su semeiante . \\\hline
2.1.6 & non producat sibi simile ; \textbf{ quare cum proprium actiuum generationis sit masculus , } et proprium susceptiuum sit foemina : & e non puede fazer su semeiante . \textbf{ Por la qual razon commo el propre o fazedor en la generaçion sea el mas lo . } Et la propre a materia para \\\hline
2.1.6 & vel foemine , vel utriusque . \textbf{ Sed cum mas et foemina , } vir et uxor sit & o por mengua dela fenbra \textbf{ o por mengua de amos } Mas commo el maslo e la fenbra e el marido e la muger sea la primera parte dela casar la primera comunidat \\\hline
2.1.6 & sumitur ex parte naturalis perpetuitatis . \textbf{ Nam cum homines non possunt } per seipsos perpetuari in vita , & ¶ La segunda razon para prouar esto mesmo se toma de acabamiento de comunindat natural duradera por sienpre . \textbf{ Ca commo los omes non pueden } por si mesmos durar \\\hline
2.1.6 & quia non habet felicitatem politicam \textbf{ cum omni sua claritate . } Patet ergo quod ad hoc quod domus habeat esse perfectum , & por que non ha la bien andança \textbf{ çiuil con todas las cosas } que cunplen para esto . \\\hline
2.1.6 & et aliquid obsequens . \textbf{ Quare cum in communitate maris et foeminae , } mas debet esse principans , & e alguno que faga su mandado . \textbf{ Por la qual cosa commo en la comunindat del uaron } et dela fenbra el uaron deua ser prinçipal et ordenador \\\hline
2.1.6 & in quo tractatur de regimine domus . \textbf{ Nam cum in domo perfecta sint tria regimina , } oportet hunc librum tres habere partes ; & eł gouertiamiento dela casa \textbf{ Ca commo en la casa acabada sean tres gouernamientos . } Ca conuiene que este libro sea partido en tres partes . \\\hline
2.1.7 & praesupponunt communitatem domesticam . \textbf{ Cum ergo domus sit prior vico , ciuitate , et regno : } homo naturaliter & ante ponen la comunidat dela casa . \textbf{ Et pues que assi es commo la casa sea primero | que el uarrio } e que la çibdat \\\hline
2.1.7 & quam politicum : \textbf{ cum prima communitas ipsius domus sit coniunctio viri et uxoris , } sequitur ex parte ipsius communitatis humanae , & que de çibdat \textbf{ commo la primera comunidat dela casa sea | ayuntamientode uaron } e de muger siguese de parte desta conñçion humanal \\\hline
2.1.7 & ad quod homo habet naturalem impetum : \textbf{ quare cum homo et omnia animalia naturaliter inclinentur , } ut velint producere sibi simile , quia in hominibus hoc debite sit per coniugium , & ala qual el omne ha natural inclinaçion \textbf{ e natra al apetito | por la qual cosa commo todas las ainalias naturalmente sean inclinadas } para querer engendrar otro semeiable \\\hline
2.1.7 & Ponentes ergo propria ad commune , \textbf{ ut cum uxor propria sua ordinat } in bonum uiri uel & Et por ende poniendo ellos las cosas propreas al comun \textbf{ assi commo quando la muger orden a las sus obras propreas al bien de su marido } o al bien de toda la casa . \\\hline
2.1.8 & Prima via sic patet . \textbf{ Nam cum nunquam aliquis fideliter } amicetur alicui , & ¶ la primera se prueua assi . \textbf{ Ca commo . nunca ninguno fiel } e leal se ayunte fielmente a otro \\\hline
2.1.8 & et econuerso . \textbf{ Cum enim inter virum et uxorem sit amicitia naturalis , } ut probatur 8 Ethicorum , & e esso mismo la muger al uaron . \textbf{ Ca commo entre el uaron e la muger sea amistança natural } assi commo se prueua en el viij̊ libro delas ethicas \\\hline
2.1.8 & ut probatur 8 Ethicorum , \textbf{ cum non fit naturalis amicitia } inter aliquos nisi obseruent sibi debitam fidem ; & assi commo se prueua en el viij̊ libro delas ethicas \textbf{ commo non sea natural amistan | ca entre algunos } si non guardaren \\\hline
2.1.8 & augmentatur eorum amicitia naturalis . \textbf{ Sed cum omnis amor vim quandam unitiuam dicat , } augmentato amore propter prolem genitam , & acresçientase entre ellos amorio natural \textbf{ Mas coͣtra odo amor aya alguna fuerça | para ayuncar los omes } el actes çentamiento del amor \\\hline
2.1.9 & ad nimiam concupiscentiam venereorum : \textbf{ Quare cum huiusmodi concupiscentiae } ( si fortes sint ) & assi la muchedunbre de las mugieres trahe al omne a grand cobdiçia de luyuria . \textbf{ Por la qual cosa commo estas tales cobdiçias de luxias } si fueren fuertes \\\hline
2.1.9 & est amicitia excellens et naturalis . \textbf{ Sed cum excellens amor non possit esse ad plures , } ut vult Philosophus 9 Ethicor’ , & entre ellos es amistança muy grande e muy natural . \textbf{ Mas commo el grand amor non pueda ser departido amuchͣs partes } assi conmo dize el philosofo en elix̊ . \\\hline
2.1.9 & sumitur ex nutritione filiorum . \textbf{ Nam cum coniugium sit quid naturale : } quomodo debito modo fieri debeat & cerazon delos fijos . \textbf{ Ca commo el matermonio sea cosa natural } en qual manera se deua fazer \\\hline
2.1.9 & et in pluribus aliis animalibus . \textbf{ Quare cum coniunctio maris , et foeminae in omnibus animalibus ordinetur } ad bonum prolis , & e en muchͣs otras aianlias . \textbf{ Por la qual cola commo la conuiction | e el ayuntamiento del mallo e dela fenbra sea ordenado en todas las ainalias abien delos fiios . } Et en aquellas aianlias \\\hline
2.1.9 & et partem masculus . \textbf{ Cum igitur ad supportandum onera coniugii } in hominibus & e parte los mas los . \textbf{ Et pues que assi es commo para sofrir las cargas del matermoion en los omes } non abaste la fenbra sola \\\hline
2.1.9 & ut tam mas quam foemina supportent onera filiorum . \textbf{ Sed cum in aliis animalibus , } in quibus tam mas quam foemina supportant onera filiorum , & sufran las cargas de los fijos . \textbf{ Mas commo en las otras aianlias } en las quales tan bien el \\\hline
2.1.9 & se habent masculus et foemina tempore partus . \textbf{ Sed cum dictum sit , } quod toto tempore partus in huiusmodi auibus masculus & en el tienpo del parto . \textbf{ Mas commo dicho es } que en todo el tp̃o del parto \\\hline
2.1.10 & ad quam coniugium ordinatur . \textbf{ Nam cum quilibet moleste ferat , } si in usu suae rei delectabilis impeditur ; & ala qual es ordenado el casamiento . \textbf{ Ca commo qual si quier sufra } guauemente si le enbargaren del vso de aquella cosa \\\hline
2.1.11 & contra rationis dictamen : \textbf{ et quod cum parentibus et consanguineis } nimia consanguineitate coniunctis & mas que esto sea contra razon \textbf{ e que non deua ser fecho casamiento | nin ayuntado con padre e madre } nin con parientes ayuntados \\\hline
2.1.11 & Prima via sic patet . \textbf{ Nam cum ex naturali ordine debeamus parentibus debitam subiectionem , } et consanguineis debitam reuerentiam , & La primera razon se declara assi . \textbf{ Ca commo por la orden natural deuamos auer | subiectiuo al padre e ala madre } e reuerençia conueible alos parientes \\\hline
2.1.11 & et consanguineis debitam reuerentiam , \textbf{ cum huiusmodi reuerentia debita non reseruetur } inter virum et uxorem propter ea quae inter eos mutuo sunt agenda , & e reuerençia conueible alos parientes \textbf{ e commo esta reuerençia conueinble non sea guardada } entre la muger e el uaron \\\hline
2.1.11 & et apud nullas gentes permissum inuenimus , \textbf{ ut quis cum matre contraheret . } Nam cum uxor debeat & e nin lo fallamos consentido \textbf{ que ninguno cassase con su madre } por que la muger deue ser subiecta al uaron . \\\hline
2.1.11 & ut quis cum matre contraheret . \textbf{ Nam cum uxor debeat } esse subiecta viro , & que ninguno cassase con su madre \textbf{ por que la muger deue ser subiecta al uaron . } Et cosa desconueinente \\\hline
2.1.11 & Sic etiam non licet \textbf{ eis contrahere cum consanguineis aliis , } si sint eis nimia consanguineitate coniuncti , & e les son tenudos de fazer . \textbf{ Avn essa misma manera non les conuiene de casar con los parientes } que les son muy çercanos \\\hline
2.1.11 & non contrahere coniugia \textbf{ cum quibuscunque personis ; } tanto tamen hoc magis decet Reges , et Principes , & Et pues que assi es conuiene a todos los çibdadanos de non fazer matermonios \textbf{ con quales quier perssonas . } Enpero tanto mas esto conuiene alos Reyes e alos prinçipes \\\hline
2.1.11 & inter ipsos contrahentes oritur pax , et concordia . \textbf{ Sed cum inter consanguineos } ex ipsa proximitate carnis sufficiens amicitia esse videatur , & sobredicho que del casamiento se leunata paz e concordia \textbf{ entre los omes que casan . } Mas commo entre los parientes por razon del \\\hline
2.1.11 & ab intemperantia retrahantur . \textbf{ Cum enim concupiscentiae carnis , } si nimiae sint , & e de soltura de luxͣia . \textbf{ Ca commo las cobdiçias dela carne } si grandes fueren \\\hline
2.1.11 & non nimiam operam dare venereis . \textbf{ Cum ergo ad personas nimia affinitate coniunctas habeatur naturalis amor , } si supra amorem illum & e de entendemiento \textbf{ que non den grand obra adelectaçiones dela catue . Et pues que assi es commo ayan natural amor las perssonas } que son ayuntadas en grand parentesco \\\hline
2.1.11 & Decet ergo omnes ciues non inire connubia \textbf{ cum personis nimia consanguinitate coniunctis ; } ne dando nimis operam venereis , & de non fazer casamientos entre perssonas \textbf{ que son muy ayuntadas en parentesço } por que non sea la su razon menguada \\\hline
2.1.12 & ut patet per Philosophum primo Rhetoricorum . \textbf{ Cum ergo Reges , } et Principes volunt alicui per coniugium copulari , & por el philosofo en el primero libro delas ethicas . \textbf{ Et pues que assi es quando los Reyes } e los prinçipes \\\hline
2.1.12 & in quandam societatem debitam et naturalem . \textbf{ Cum ergo debite et congrue nobili societur : } Reges et Principes , & e conueniblemente \textbf{ el noble deua ser aconpannado } ala noble los Reyes \\\hline
2.1.12 & homines libenter iniustificant , \textbf{ cum possunt . } Qui ergo caret ciuili potentia & Ca segunt el philosofo en la rectorica los omes de buenamente fazen iniustiçias e tuertos \textbf{ quando pue den . } Et pues que assi es aquel \\\hline
2.1.13 & secundum modum eis congruum foeminas pollere deceat , \textbf{ tamen cum tradenda est aliqua nuptui , } potissime inquirendum est , & segunt la manera que les conuiene . \textbf{ Enpero quando la fenbra es de dar a algun marido mayormente deuemos tener } mientessi resplandesçe por tenprança \\\hline
2.1.13 & Decet eas etiam amare operositatem : \textbf{ quia cum aliqua persona ociosa existat , } leuius inclinatur & Et avn les conuiene aellas de amar fazer buenas obras . \textbf{ Ca quando alguna persona esta de uagar mas ligeramente es inclinada a aquellas cosas } que la razon iueda \\\hline
2.1.13 & nescit ociosa esse ; \textbf{ statim ergo cum aliquis non dat se bonis et licitis exercitiis , } eius mens vagatur circa alia & non sabe ser uagarosa . \textbf{ Et pues que assy es luego | que alguno non se da alas buean sobras } e conuenibles la uoluntad del \\\hline
2.1.14 & quod debet esse in uno homine : \textbf{ cum ciuitas sit pars uniuersi , } regimen totius ciuitatis multo magis reseruabitur in una domo . & que deua ser en vn omne \textbf{ commo la çibdat } se aparte de todo el mundo el gouernamiento de teda la çibdat mucho mas es fallada e nonacasa . \\\hline
2.1.14 & Dicitur autem quis praeesse regali dominio , \textbf{ cum praeest secundum arbitrium et secundum leges , } quas ipse instistuit . & Mas alguon es dicho ser adelantado en sennorio real \textbf{ quando es adelantado segunt aluedrio | e segunt las leyes } que el mismo establesçio \\\hline
2.1.14 & sed secundum eas quas ciues instituerunt . \textbf{ Cum enim principans in ciuitate ipse } secundum seipsum principatur , & que establesçieron los çibdadanos . \textbf{ Ca quando el que | enssennorea en la çibdat el mesmo ensseñorea por si e el } establesçe leyes aquel gouernamiento toma nenbre \\\hline
2.1.14 & et dicitur regale . \textbf{ Sed cum leges non instituuntur } a principante sed a ciuibus , & e es dicho gouernamiento real . \textbf{ as quando las leyes non son establesçidas } por el prinçipe \\\hline
2.1.14 & quodammodo ex electione , \textbf{ cum in tali regimine ciues sibi dominum eligant . } Ex hoc autem maxime patet differentia & realante aquel gouernamiento es en alguna manera por election . \textbf{ Ca en tal gouernamiento los çibdadanos escogen | assi sennor qual les conuiene . } Mas desto paresçe manifiestamente el departimiento \\\hline
2.1.14 & aliquo enim modo uxor iudicatur \textbf{ ad paria cum viro , } et eligit sibi virum . & que el mater moinal . \textbf{ Ca en algua manera la muger es iudgada con el uaron a cosas eguales } e escoge para si el uaron \\\hline
2.1.14 & quibus vacare debeant \textbf{ cum sint adulti : } ad quae non sunt instruendae uxores , & e alas obras çiuiles alas quales deuen entender \textbf{ quando fueren criados } e mayores alas quales cosas non son de enssennar las mugers \\\hline
2.1.15 & a quo est omnis ordo . \textbf{ Cum ergo tunc sit } aliquid ordinatum maxime , & Ca aquel gnia la natura de que viene todo ordenamiento \textbf{ Pues que assi es conmo } estonçe sea alguna cosa bien ordenada \\\hline
2.1.15 & quando unum ordinatur ad unum officium : \textbf{ cum uxor naturaliter sit ordinata ad generandum , } non erit ordinata ad seruiendum . & çquano cada cosa es ordenada a vn ofiçio \textbf{ commo la muger sea | ordenadanatraalmente ala generaçion de los fijos } non deue ser ordenada a seruiçio . \\\hline
2.1.15 & si non multis operibus sit seruiens , sed uni . \textbf{ Quare cum natura ordinauerit } coniugem ad generationem , & que qual si quier de los instrumentos fara conplidamente su obra si non siruiere en muchos obras mas en vna . \textbf{ Por la qual cosa commo la natura aya ordenada la mugr } para la generaçino de los fijos \\\hline
2.1.15 & nisi careat usu rationis et intellectus . \textbf{ Sed cum carens rationis usu sit naturaliter seruus , } quia nescit seipsum dirigere , & si non fuese priuado de vso de razon e de entendimiento . \textbf{ Mas commo aquel que es priuado de vso de razon e de entendemiento sea naturalmente sieruo } por que non sabe gniar assi mismo \\\hline
2.1.16 & ipsis tamen uniuersalibus sermonibus sunt particularia addenda , \textbf{ quia cum negocium morale circa particularia consistat } ( secundum doctrinam Philosophi 2 Ethicorum ) & e alos ymones generales deuemos añader los sermones particulares . \textbf{ Ca commo el negoçio moral | e de costunbres se açerca las cosas singulares } segunt la doctrina del philosofo \\\hline
2.1.16 & sed etiam quantum ad animam : \textbf{ quia cum anima sequatur complexiones corporis } ( nam cum aliquis non est bene proportionatus in corpore , & non solamente quanto al cuerpo mas avn quanto al alma \textbf{ ca el alma sigue las conplissiones del cuerpo . } Ca quando alguno non es bien conplissionado en el cuerpo \\\hline
2.1.16 & quia cum anima sequatur complexiones corporis \textbf{ ( nam cum aliquis non est bene proportionatus in corpore , } non est bonae complexionis ) & ca el alma sigue las conplissiones del cuerpo . \textbf{ Ca quando alguno non es bien conplissionado en el cuerpo } nin es de buean conplission el alma es enbargada \\\hline
2.1.16 & Nam si per totum tempus augmenti nociuum est masculis uti coniugio , \textbf{ cum ad tempus augmenti communiter } in hominibus requirantur tria septennia , & mas los vsar de casamiento . \textbf{ Como el tp̃o dela cresçençia demande communalmente } en los omes tres setenas de años . \\\hline
2.1.18 & gaudere de hominum opinione . \textbf{ Quare cum mulieres communiter } non tanta bonitate polleant sicut uiri ; & Por la qual cosa non han grant cuydado de se gozar dela opinion de los omes . \textbf{ por ende commo las mugers comunalmente non sean ennoblesçidas } por tanta bondat \\\hline
2.1.18 & a perfectione uirorum . \textbf{ Quare cum ad augmentum perfectionis bonitatis minuatur } cupiditas laudis et desiderium reputationis , & Por la qual cosa comunalmente much fallesçe en ellas la perfecçion de los uarones . \textbf{ Et por ende conmo el acresçentamiento dela perfecçion } e dela bondat sea menguada la cobdiçia dela alabança \\\hline
2.1.18 & Propter quod , \textbf{ cum uerecundia sit } timor de inglorificatione et de amissione laudis , & qen la bondat \textbf{ que los omes por la qual cosa commo la uirguença sea temor de non auer eglesia o de ꝑder alabança } e las mugers son comunalmente uergonçosas \\\hline
2.1.18 & ex timiditate cordis . \textbf{ Nam cum mulieres sint naturaliter adeo timidae , } quod quasi omnia expauescunt ; & por el temor del coraçon . \textbf{ Ca commo las muger ssean naturalmente temerosas | en tanto que semeia } que de todas las cosas se espantan . \\\hline
2.1.18 & Senes uero miseratiui existunt : \textbf{ quia cum ipsi in corpore } et in uita deficiant , & por otra razon \textbf{ ca por que ellos fallesçen en el cuerpo } e enla uida quieren \\\hline
2.1.18 & uolunt aliis misereri et compati ipsis : \textbf{ quare cum de facili quis inclinetur } ad faciendum aliis , & e piadat dellos \textbf{ por que cada vno de ligero se inclina a fazer alos otros lo que el quarne } que los otros fiziessen a el . \\\hline
2.1.18 & ideo statim miserentur , \textbf{ cum vident aliquos dura pati . } Tertio considerandum est in mulieribus , & e por ende luego que veen a algunos \textbf{ sofrir cosas duras han piadat sobre ellos ¶ } Lo terçero deuemos penssar en las mugers \\\hline
2.1.18 & quia communiter nimis excedunt . \textbf{ Unde cum sunt piae , } sunt valde piae : & e sobrepuian entondo . \textbf{ Onde quando son piadosas son muy piadosas . } Et quando son crueles son muy crueles \\\hline
2.1.18 & sunt valde piae : \textbf{ et cum sunt crudeles , } sunt valde crudeles : & Onde quando son piadosas son muy piadosas . \textbf{ Et quando son crueles son muy crueles } Et quando son desuergonçedas son muy sin uerguença . \\\hline
2.1.18 & sunt valde crudeles : \textbf{ et cum sunt inuerecundae , } sunt nimis inuerecundae . Postquam enim mulieres audaciam capiunt , & Et quando son crueles son muy crueles \textbf{ Et quando son desuergonçedas son muy sin uerguença . } Ca del pues que las mugers toman osadia \\\hline
2.1.18 & Primo ergo foeminae , \textbf{ cum possunt , } ut plurimum sunt intemperatae , & Pues que assi es lo primero las mugers \textbf{ quando pueden } por la mayor parte son destenpdas \\\hline
2.1.18 & ex verecundia quam ex ratione . \textbf{ Quare cum motae sunt , } nesciunt se moderare , & esto fazen mas por uerguença que por razon . \textbf{ Por la qual cosa quando se mueuen non } sabenertenprar assy mesmas . \\\hline
2.1.19 & Unde et aliquos Philosophos legimus sic fecisse , \textbf{ qui cum essent impeditae linguae , } accipientes specialem conatum & Ende leemos que algunos philosofos lo fizieron \textbf{ assi los quales commo ouiessen las lenguas enbargadas } tomaron especial esfuerço çerca aquellas letras \\\hline
2.1.19 & Hoc ergo modo et circa opera se habet . \textbf{ Cum enim quis se vel alium videt } circa aliqua deficere , & Et pues que assi es en essa misma manera deuemos fazer çerca las obras \textbf{ por que quando alguno vee } assi o a otro fallesçer en algunas cosas \\\hline
2.1.19 & ut pater sit certus de sua prole . \textbf{ Cum ergo signa inhonesta , } et impudica & padresea çierto de su fijo . \textbf{ Et pues que assi es por que las señales desonestas } e malas \\\hline
2.1.20 & Secundo debent eas honorifice tractare . \textbf{ Tertio debent cum eis debite conuersari . } Decet enim eos suis coniugibus & Lo segundo deuen las tractar honrradamente¶ \textbf{ Lo terçero deuen beuir con ellas conueniblemente . | ¶ Lo primero se praeua } assi que conuien e alos uarones de vsar con sus muger \\\hline
2.1.20 & ei necessaria debita tribuendo . \textbf{ Nam cum uxor sit } persona valde coniuncta , & dandol conueniblemente lo que ha menester . \textbf{ Ca commo la mug̃r sea perssona . } muy ayuntada a su marido \\\hline
2.1.20 & redundat in persona ipsius viri . \textbf{ Immo cum ostensum sit supra uxorem } non se habere & tornase en la perssona del marido . \textbf{ Mas por que fue mostrado de suso que la mugni non se deue auer al marido } assi commo sierua mas assi commo conpanera . \\\hline
2.1.20 & restat ostendere , \textbf{ qualiter cum eis debeant conuersari . } Tunc autem viri ad uxorem est conuersatio congrua , & honrradamente finca de demostrar \textbf{ en qual manera deuen beuir conellas } mas estonçe es dicha la conuersaçion \\\hline
2.1.20 & utrum sint prudentes aut fatuae . \textbf{ Nam sic conuersandum est cum uxoribus , } quod plura signa amicitiae ostendenda sunt eis , & o si son homildosas o si son sabias o si son locas . \textbf{ Ca assi deuemos beuir con las mugers } que les deuemos mostrar muchͣs señales de amistança \\\hline
2.1.20 & ut velint etiam viris propriis dominari . \textbf{ Rursus sic conuersandum est cum eis , } quod aliter instruendae sunt prudentes , & que avn quieren enssenorear a sus maridos . \textbf{ ¶ Otrossi en tal manera deuemos beuir con ella } sperando mientes \\\hline
2.1.21 & suas coniuges debite se habere . \textbf{ Nam cum vir suam uxorem regere debeat , } eam dirigendo ad actiones honestas , & de se auer conueniblemente en el conponimiento e honrramiento de sus cuerpos . \textbf{ Ca quando el marido gouierna e castiga a su muger } deue la castigar a obras honestas \\\hline
2.1.21 & et quae illicita . \textbf{ Cum ergo mulieres } ut plurimum appetant videri pulchrae , & e que desconueibls¶ \textbf{ Et pues que assi es commo las mugieres } por la mayor parte \\\hline
2.1.21 & quod infirmior magis gloriatur , \textbf{ quia credit quod in cum plures aspiciant , } et sperat se plures eleemosynas accepturum : & contesçe que el mas enfermose eglesia \textbf{ mas por que cree que muchos catan ael } e es para que resçibra mas helemosinas que los otros . \\\hline
2.1.22 & triplici via ostendere possumus . \textbf{ Nam cum quis erga suam coniugem est nimis zelotypus , } ex nimio zelo quem erga illam gerit , & por tres razones . \textbf{ Ca quando alguno es muy çeloso de su muger } por el grand çelo que ha della sospecha todas las cosas \\\hline
2.1.22 & et per consequens semper sunt in anxietate cordis : \textbf{ quare cum una cura impediat aliam , } oportet sic zelantes & que sienpreson en grand angostura de su çoraçon . \textbf{ Por la qual cosa commo el vn cuydado enbargue el otro . } Conuiene alos tales çelosos de ser enbargados enlos cuydados \\\hline
2.1.22 & et quod uidemus nobis deficere . \textbf{ Sed cum res prohibita , } eo ipso quod prohibetur & e aquello que veemos que nos fallesçe¶ Lo segundo la \textbf{ cosaue dada por que es mas uedada paresçe } que nos mengua e nos fallesçe \\\hline
2.1.22 & quod in domo consurgit . \textbf{ Nam cum uidetur uxoribus } quod sine causa calumnientur , et quod earum uiri sine culpa suspicentur de ipsis mala , & que se leunata en la casa . \textbf{ Ca quando veen las mugers } que sus maridos se acallonan sin razon \\\hline
2.1.23 & elegibilius esset consilium muliebre quam virile . \textbf{ Natura enim cum moueatur ab intelligentiis , et a Deo , } in quo est suprema prudentia ; & e la razon es esta \textbf{ por que la natura toda es mouida delos angeles | e de dios } en que es conplimiento de sabiduria . Et por ende conuiene que aquellas cosas \\\hline
2.1.23 & ad corpus cuius habet \textbf{ esse perfectum quam vir . Quare cum anima sequatur complexionem corporis , } sicut ipsum corpus muliebre & que el uaron . \textbf{ Por la qual cosa | commo el alma sigua ala conplission del cuerpo . } A assi commo el cuerpo dela muzer \\\hline
2.1.24 & ut operemur illud . \textbf{ Quare cum ponere aliquid in praecepto , } sit prohibere , & para obrar aquella cosa . \textbf{ Por la qual razon commo poner alguna cosa } en poridat se a uedar \\\hline
2.1.24 & ut plurimum sunt molles et ductibiles , \textbf{ statim cum aliqua persona eis applaudet , } et ridet in facie earum , & e mas de ligero son mouibles \textbf{ luego que algunas ꝑssonas les comiençan a lisongar } e a Reyr en su faz dellas \\\hline
2.1.24 & si sciant secreta ipsorum ; \textbf{ cum conglorientur , } si possint se laudari & por que parescan ser amadas de sus maridos \textbf{ si sopieren las poridades dellos . } Et por ende se glorian \\\hline
2.1.24 & reuelare secreta . \textbf{ Nam cum dicimus hos esse mores iuuenum , } hos mulierum , hos senum . & en qual manera los maridos de una descobrir a sus mugieres los sus secretos . \textbf{ Ca quando nos dezimos | que estas son las costunbres de los mançebos } e estas las de los uieios \\\hline
2.1.24 & His visis , \textbf{ cum ostensum sit , } quomodo communitas viri et uxoris sit naturalis , & ¶ vistas estas cosas \textbf{ commo seaya prouado } que la conpannia del uaron \\\hline
2.1.24 & ad alias communitates domus : \textbf{ cum etiam declaratum sit , } quomodo Reges et Principes , & se aya alas otras conpannias . \textbf{ Et avn commo sea declarado } en qual manera los Reyes e los prinçipes \\\hline
2.1.24 & et uniuersaliter omnes ciues se habere debeant ad suas coniuges , \textbf{ et quomodo cum eis debeant conuersari , } imponatur finis primae parti huius secundi Libri , & e generalmente todos los çibdad a uos se de una auer alus mugers \textbf{ e como de una beuir con ellas . } Pongamos fin e acabamiento a esta primera parte deste segundo libro \\\hline
2.2.1 & Sciendum igitur , \textbf{ quod cum communitas viri et uxoris , } et domini et serui pertineant & Et pues que assi es \textbf{ deuedessaber que commo la comunidat } e la conpannia del uaron e dela muger e del sennor e del sieruo part enescan ala casa primera \\\hline
2.2.1 & quilibet enim solicitatur circa dilectum : \textbf{ quare cum inter patrem et filium sit amor naturalis , } ut probatur 8 Ethicorum , & Ca cada vno ha cuydado de su amor . \textbf{ por la qual cosa commo entre el padre | e el fijo sea amor natraal } assi commo se praeua en el viij delas . ethicas . \\\hline
2.2.2 & et quanto maiori intelligentia vigent . \textbf{ Sed cum habitum sit } quod Reges , & e quanto mayor entendimiento ha en ellos . \textbf{ Mas conmo sea dicho } que los Reyes e los prinçipes \\\hline
2.2.2 & maxime decet eos esse prudentes et bonos . \textbf{ Et cum filii perueniunt } ad maiorem bonitatem et prudentiam , & mucho les conuiene de ser sabios e buenos . \textbf{ Mas commo los fijos bengan a mayor bondat e a mayor sabiduria } si los padres ouieron cuydado dellos mas \\\hline
2.2.3 & ut uxores viros . \textbf{ Quare cum regimen filiorum sit ex arbitrio , } et sit propter bonum ipsorum filiorum ; & assis padres \textbf{ commo las muger suarones Por la qual razon como el gouernamiento de los fijos } sea por aluedrio e sea por el bien de los fijos . \\\hline
2.2.3 & et propter bonum ipsorum : \textbf{ cum amare aliquod , } idem sit quod velle ei bonum , & enssennorear alos fiios realmente \textbf{ e por el bien dollos commo amar a alguno sea esso mismo } que querer qual bien . \\\hline
2.2.3 & quaedam similitudo procedens a patre : \textbf{ cum secundum naturam ad huiusmodi similia fit dilectio , } ex ipso ordine naturali arguere possumus , & por que el fijo naturalmente es vria semerança que desçende del cadre . \textbf{ Canmo segunr nacsta } semeianca sea ardenado el amor per la ore enna tal podemos i prouar \\\hline
2.2.3 & quando potest sibi simile generare . \textbf{ Quare cum quilibet suam perfectionem diligat , } naturaliter pater diligit filium , & quando ꝑuede engendrar su semeiante . \textbf{ Et commo quier que cada vno ame su perfecçion . } Emperona traalmente el padre ama el fijo \\\hline
2.2.3 & naturaliter pater diligit filium , \textbf{ cum proprie filius sit } quidam testis perfectionis eius . & Emperona traalmente el padre ama el fijo \textbf{ commo el fiio sea vn testigo dela perfecçion del padre ¶ } pues que assi es por que el padre ha natura l \\\hline
2.2.4 & quam filiorum ad parentes . \textbf{ Nam statim cum filii nascuntur , } parentes diligunt eos : & mayortpon que de los fijos alos padres . \textbf{ Ca luego que los fijos nasçen los aman los padres . } Enpero los fijos \\\hline
2.2.4 & quam econuerso . \textbf{ Quare cum amor quandam unionem importet , } filii tanquam magis uniti et magis propinqui parentibus , & que los padres alos fijos . \textbf{ por la qual razon commo el amor faga algun ayuntamiento los fijos } assi conmomas ayuntados \\\hline
2.2.4 & Immo filii , \textbf{ cum possunt , furantur , } et rapiunt bona parentum . & Mas los fijos non allegan para los padres . \textbf{ Mas los fijos quando pueden furtan } e cobdician los bienes de los padres \\\hline
2.2.4 & quam econuerso ; \textbf{ cum diligere aliquem , } idem sit quod velle ei bonum , & que los fijos alos padres \textbf{ commo amara alguno sea essa misma cosa } que querer bien \\\hline
2.2.4 & ad filios debent eos regere et gubernare . \textbf{ Sed cum filii afficiantur ad parentes , } tanquam ad eos , & que han los padres alos fijos los deuen gouernar \textbf{ Mas commo los fijos sean inclinados alos padres . } assi commo aquellos que quieren auer en honrra e en reuerençia . \\\hline
2.2.4 & in honore et reuerentia : \textbf{ cum honorari et reuereri alium sit } quodammodo subiici illi ; & Commo honrrar \textbf{ e auer reuerençia a otro sea en alguna manera ser subiecto a el . } Por ende assi commo por el amor que han los padres alos fijos los deuen gouernar ben \\\hline
2.2.5 & Infantes enim , \textbf{ cum eis a matre vel a patre proponuntur aliqua credenda , } non quaerunt rationem dictorum , & Ca los uinnos \textbf{ quando los padres o las madres les proponen | e les dan castigos alguons } o les muestra a las cosas \\\hline
2.2.5 & Nam ut in primo libro diximus \textbf{ cum tractauimus de moribus iuuenum , } Iuuenes sunt simpliciter creditiui : & Ca assi commo dixiemos en el primero libro \textbf{ quando tractauamos delas costunbres de los mançebos } los mançebos son sinples en creer . \\\hline
2.2.5 & et tanto feruentius adhaeremus illi . \textbf{ Cum ergo magis simus assuefacti ad ea , } circa quae in ipsa infantia insudamus , & e con mayor acuçianos llegamos a ella . \textbf{ Et pues que assi es commo mas somos usados a aquellas cosas } en que trabaiamos enla moçedat \\\hline
2.2.6 & nam et pueri statim delectantur , \textbf{ cum incipiunt suggere mammas . } Si ergo sic ab ipsa infantia nobiscum & assi que los moços luego se delectan \textbf{ e quaeçentes | si que los mocos luero de mamar . } Et pues que assi es assi commo dela moçedat cresçe \\\hline
2.2.6 & ex ipsa ergo connaturalitate delectationis , \textbf{ statim cum pueri sunt sermonum capaces , } sunt instruendi ad bonos mores , & e dela delectaçion paresçe \textbf{ que quando los moços son tales | que han entendemiento para tomar razon } estonçe son de enssennar en bueans costunbres \\\hline
2.2.6 & retrahantur a lasciuiis . \textbf{ Quare cum rationis sit concupiscentias refraenare et lasciuias , } quanto aliquis magis a ratione deficit , & por que de la razon \textbf{ e del entendimiento | es de refrenar los desseos e las locanias . } Et por ende quanto alguon mas fallesçe en razon \\\hline
2.2.6 & quam habemus ad malum . \textbf{ Nam cum aliquis est pronus ad aliquid , } oportet ipsum multum assuescere in contrarium , & que auemos a mal . \textbf{ Ca quando alguno es inclinado a alguacosa . } Conuiene que el vse mucho en el contrario \\\hline
2.2.7 & et attentus circa studium . \textbf{ Cum ergo consuetudo sit } quasi altera natura , & si non fuere muy cuydado so çerca el estudio . \textbf{ Et pues que assi es conmo la costunbre sea } assi commo otra natraa \\\hline
2.2.7 & quasi altera natura , \textbf{ cum secundum Philosophum in Polit’ nobis } magis placeant illa opera , & assi commo otra natraa \textbf{ Et commo segunt el philosofo en las politicas } aquellas obras \\\hline
2.2.7 & et Principum \textbf{ cum ponuntur in aliquo dominio tyrannizent , } decet ipsos etiam ab ipsa infantia insudare literis , & e de los prinçipes \textbf{ quando son puestos en algun sennorio non tiraniçen | nin sean tirannos } Conuiene les avn de trabaiar \\\hline
2.2.9 & valde deberent esse soliciti , \textbf{ et cum magna diligentia attendere , } qualem magistrum proponerent in regimine filiorum . & que delas otras cosas . \textbf{ por cuydadolos e tener mientes con grand acuçia } qual maestro deuen poner en gouernamiento de sus fijos \\\hline
2.2.10 & iuuenes sunt insecutores passionum , \textbf{ et ad lasciuiam proni . Quare cum semper sit adhibenda cautela } ubi periculum imminet , & los moços alos mançebos son segnidores de passiones \textbf{ e son inclinados a orgullos e aloçania . | Por la qual cosa commo sienpre deuemos dar } algunan cautellado paresçe el peligro . \\\hline
2.2.10 & iuuenes sunt de facili mentitiui : \textbf{ cum ergo consuetudo sit } quasi altera natura , & e los mançebos son de ligero mintrolos . \textbf{ Et por ende commo la costunbre sea } assi commo otra nafa \\\hline
2.2.10 & quia quae primo aspiciuntur , \textbf{ cum maiori admiratione videntur , } quare magis sumus attenti circa illa , & que primeramente catamos con mayor a miraçion \textbf{ e con mayor marauilla las veemos . | Por la qual } cosamas somos acuçiosos cerca aquellas \\\hline
2.2.10 & ut instruantur quod palpebras oculorum \textbf{ cum maturitate eleuent , } ut non habeant oculos vagabundos . & quanto ala manera de ver \textbf{ assi que alçen las palpebras de los oios con grand madureza } e que non echen los oios \\\hline
2.2.11 & modica delectatio est , \textbf{ cum cibus attingit linguam : } sed maior est , & pequana delectaçiones \textbf{ quando la lengua alcança la uianda } mas mayor delectaçiones \\\hline
2.2.11 & sed maior est , \textbf{ cum attingit guttur . } Gulosi ergo , & mas mayor delectaçiones \textbf{ quando la uianda llega ala garganta . } Et por ende los golosos que con grand cobdiçia \\\hline
2.2.11 & qui nimis auide \textbf{ et cum magna delectatione cibum sumunt , } non multum delectantur , & Et por ende los golosos que con grand cobdiçia \textbf{ e con grand delectaçion toman } sauianda non se delectan mucho \\\hline
2.2.11 & si contingat ex inordinatione animae . \textbf{ Quare cum turpis modus sumendi cibum signum sit cuiusdam gulositatis , } vel inordinationis mentis & por desordenamiento del alma . \textbf{ Por la qual cosa commo la manera torpe de resçebir la vianda | sea señal de golosina } de aquel que la toma o de desordenaçion del alma \\\hline
2.2.11 & quasi altera natura , \textbf{ ideo cum quis assuescit , } sumere cibum in aliqua hora , & assi commo otra nata \textbf{ por ende quando alguno se acostunbra a tomar la uianda } en algua ora desordenada \\\hline
2.2.11 & non comedere ut viuant , \textbf{ cum nimium studium } et nimiam curam apponant & por que coma et non quieren comer \textbf{ por que biuna por que ponen grand estudio e grant cuydado } çerca los apareiamientos delas viandas . \\\hline
2.2.11 & Sufficit autem eos paulatim et pedetentim instruere , \textbf{ ut cum ad debitam aetatem peruenerint , } sint sufficienter instructi , & que poco a poco sean enformados e enssennados \textbf{ por que quando venieren a hedat } conueinble e acabada puedan ser enssennados \\\hline
2.2.12 & maxime est prona ad intemperantiam , \textbf{ quare cum semper sit adhibenda cautela , } ubi maius periculum imminet , & Ca do mayor es el periglo \textbf{ alli deue omne poner mayor remedio } Et por ende en la hedat de los moços deuemos guardar \\\hline
2.2.12 & Primo , quia venerea prouocat . \textbf{ Cum enim corpore calefacto maior fiat } incitatio ad actus venereos , & ¶ El primer mal es que abiua el omne asa lux̉ia . \textbf{ Ca escalentado el cuerpo faze se enł omne mayor inclinaçion alas obras de luxia . } Onde el vino tomado \\\hline
2.2.12 & ne sint lasciui . \textbf{ Cum ergo omnis actus venereus , excepto matrimonio , } sit contra rationis dictamen , & por qua non sean locanos e orgullosos nin luxiosos . \textbf{ Et pues que assi es commo todas las obras de lux̃ia sacado el matrimonio } sean contra ordenamiento de \\\hline
2.2.12 & Qualiter autem se debeant habere iuuenes \textbf{ cum uxore iam ducta , } et quae sunt consideranda in uxore ducenda : & Mas en qual manera se de una auer los mançebos \textbf{ çerca las mugers | que ya tienen tomadas } e quales cosas son aquellas \\\hline
2.2.12 & et quae sunt consideranda in uxore ducenda : \textbf{ supra , cum egimus de regimine coniugali , diffusius diximus . } Ostenso , & que deuen cuydar en las mugers \textbf{ que deuen tomar de suso lo dixiemos mas largamente | quando dixiemos del gouernamiento del casamiento } ostrado en qual manera los as . moços deuen ser guardados enla vianda \\\hline
2.2.13 & et modeste se habere \textbf{ cum uxore iam ducta . } Restat ostendere , & e commo se deuen auer \textbf{ tenpradamente con sus mugers } que han ya tomadas finca de demostrar \\\hline
2.2.13 & Prima via sic patet . \textbf{ Nam mens humana nescit ociosa esse : cum ergo quis vacat ocio , } et non intendit aliquibus delectationibus licitis , & La primera razon paresçe assi . \textbf{ Ca la uoluntad del omne non sabe ser ocçiosa | nin estar de vagar } Et pues que assi es quando alguon se da adagar \\\hline
2.2.13 & Frustra ergo , \textbf{ cum quis vult audire alium , } retinet os apertum . & Ca el omne non oye con la boca mas por el oreia . \textbf{ Et pues que assi es quando alguno quiere oyr al otro } en vano tiene la boca abierta . \\\hline
2.2.13 & Sicut ergo habent indisciplinatos gestus , \textbf{ qui cum volunt audire alios , } tenent ora aperta : & Et pues que assi es assi commo aquellos \textbf{ que quieren oyr alos otros } e tienen las bocas abiertas \\\hline
2.2.13 & sic sunt indisciplinati secundum gestus , \textbf{ qui cum volunt loqui , } extendunt pedes et crura , & desenssennandos en los gestos . \textbf{ En essa misma manera son desenssennados segunt los gestos | aquellos que quando que eren fablar estienden los pies } e las prinas o mueuen los braços \\\hline
2.2.13 & Secundo , nimia mollicies vestium reddit hominem timidum . \textbf{ Nam cum arma ferrea in se } quandam duriciem habeant , & ¶ Lo segundo la grand blandura delas vestiduras faze al omne temeroso \textbf{ por que las armas del fierro han en ssi alguna dureza } por ende aquellos que han cuydado çerca tales vestiduras muelles \\\hline
2.2.13 & et efficiuntur timidi . \textbf{ Iuuenes , maxime cum ad aliam aetatem venerint , } ad hoc quod sint habiles ad vacandum & e fazen semedrosos . \textbf{ Et por que mucho conuiene alos mançebos | quando vinieren a aquella hedat } que sean bien ordenados e bien apareiados \\\hline
2.2.14 & nisi sciuerint \textbf{ cum quibus sociis debeant conuersari . } Quanto autem ad praesens spectat , & si non sopieren con quales conpanneros \textbf{ e en qual conpannia deuen beuir . } Mas quanto pertenesçe alo presente quatro cosas paresçe \\\hline
2.2.14 & secundum appetitum . \textbf{ Quare cum mollia et ductilia } facilius recipiant impressionem & e mas tristornable segunt el appetito \textbf{ e el desseo de los sesos . | Por la qual cosa commo las cosas muelles } e tristornables \\\hline
2.2.14 & Si enim illa aetas maxime delectabilia prosequitur : \textbf{ cum valde delectabile sit } conuiuere amicis , & delectabłs \textbf{ Como mas delectable sea beuir con los amigos } aquella hedat goza de beuir en conpannia \\\hline
2.2.14 & Delectatio enim est \textbf{ ex coniunctione conuenientis cum conuenienti . } Nullus ergo gaudet in societate viuere , & Ca la delectaçiones \textbf{ por ayuntamiento de cosa conuenible con cosa qual conuiene } Et por ende ninguon non goza de beuir en conpanna \\\hline
2.2.15 & et hoc maxime , \textbf{ cum incipiunt percipere significationes verborum . } Sextum , a ploratu sunt cohibendi . & Et esto les es prouechoso mayormente \textbf{ quando comiençan a entender las significa connes delas palabras . } ¶ La sexta es que deuen ser guardados de llorar . \\\hline
2.2.15 & Obseruandum est tamen in iuuenibus \textbf{ cum aluntur lacte , } quod si contingat eos suggere aliud lac quam maternum , & Enpero deuemos guardar \textbf{ que quando los mocos maman la leche | si contesçe } que ayan de mamar otra leche \\\hline
2.2.15 & Attendendum est tamen , \textbf{ quod cum dicimus pueros paruos } assuescendos esse & Enpero deuemos entender \textbf{ que quando dezimos | que los moços pequanos son de acostunbrara esto } o aquello deue se entender tenpradamente \\\hline
2.2.15 & quoddam proficuum ad augmentum . \textbf{ Nam cum augmentum fiat } ex ipso alimento faciente corpus bene dispositum , & que sea aprouechoso alacresçentamiento del cuerpo . \textbf{ Ca commo el acresçentamiento se faga del nudrimiento } aquellas cosas \\\hline
2.2.15 & Sexto sunt cohibendi a ploratu . \textbf{ Nam cum pueri a ploratu cohibentur , } ex ipsa prohibitione fit , & que non lloren . \textbf{ Ca quando alos moços defienden | que non lloren } por esse mismo defendimiento se faze \\\hline
2.2.15 & ut retineant spiritum et anhelitum . \textbf{ Nam sicut cum plorare permittuntur , } emittunt spiritum et anhelitum : & que retengan en ssi el spun e el eneldo . \textbf{ Ca assy commo quando los dexan llorar } enbian el spuer e el eneldo . \\\hline
2.2.15 & emittunt spiritum et anhelitum : \textbf{ sic cum plorare cohibentur , } spiritum et anhelitum tenent . & enbian el spuer e el eneldo . \textbf{ assi quando les defienden | que non lloren } retienen en ssi el spun e el eneldo \\\hline
2.2.16 & sunt a ploratu illo cohibendi . \textbf{ Cum distinguimus aetates filiorum per septennia , } ut cum dicimus , & ¶ \textbf{ ommo nos ayamos departido las hedades de los fijos | por setenarios quando dixiemos que fastaliente a nons } assi deuian ser gouernados los moços . \\\hline
2.2.16 & Cum distinguimus aetates filiorum per septennia , \textbf{ ut cum dicimus , } usque ad septem annos sic esse regendos : & por setenarios quando dixiemos que fastaliente a nons \textbf{ assi deuian ser gouernados los moços . } Et de los siete años \\\hline
2.2.16 & ad aliquos motus . \textbf{ Sed cum impleuerunt septennium } usque ad annum decimum quartum , & a algunos mouimientos conuenibles . \textbf{ Mas quando ouieren conplido el vii̊ año fasta el xiiij̊ . } deuen se acostun brar poco a poco \\\hline
2.2.16 & et omnia faciunt valde , \textbf{ ita quod cum amant nimis amant , } cum incipiunt ludere nimis ludunt , & que fazen fazen las mucho \textbf{ mas que deuen asi que quando aman am̃a much̃ . Etrͣndo } comiençan de trebeiar trebeian much̃ . \\\hline
2.2.16 & ita quod cum amant nimis amant , \textbf{ cum incipiunt ludere nimis ludunt , } et in caeteris aliis semper excessum faciunt , & mas que deuen asi que quando aman am̃a much̃ . Etrͣndo \textbf{ comiençan de trebeiar trebeian much̃ . } Et assi que todas las cosas que fazen \\\hline
2.2.16 & perfecte scire non possunt . \textbf{ Ne tamen cum incipiunt habere rationis usum , } omnino sint indispositi ad scientiam , & fallesçe de vso de razon non pueden saber las sçiençias acabada mente . \textbf{ Enpero por que quando comiençan a auer } vso de razon non seanda todo mal apareiados ala sçiençia deuen ser acostunbrados alas otras artes delas \\\hline
2.2.17 & esse non potest . \textbf{ Cum ergo omnes volentes viuere vita politica , } oporteat aliquando sustinere fortes labores & la qual cosa non puede ser sin fuerte trabaio de su cuerpo . \textbf{ ¶ Et pues que assi es commo todos aquellos | que quieren beuir uida çiuil } conuiene les de sofrir alguas uegadas fuertes trabaios \\\hline
2.2.17 & Primo quantum ad elationem , \textbf{ quia cum ex tunc incipiant } habere perfectum rationis usum , & Conuiene a saber quanto al orgullo e ala locama . \textbf{ paresce que estonçe comiençan a auer vso de razon acabada } paresçe les que son dignos de enssennorear e de ser senneres \\\hline
2.2.17 & Secunda est ; \textbf{ quia cum filii venerint } ad aetatem perfectam & ¶ La segunda razon es \textbf{ por que los fiios } quando venieren ahedat acabada sean sennores . Et por ende ellos deuen mientra son enla hedat dela mançebia \\\hline
2.2.19 & et exercitium corporalem . \textbf{ Cum ex usu coniugii } non solum oriantur filii et mares , & nin por el trabaio del cuerpo . \textbf{ ommo por el vso del casamiento } non sola mente nazcanfuos \\\hline
2.2.19 & Sed hoc breui tractatu indiget : \textbf{ quia cum determinauimus de regimine coniugali , } et ostendimus qualiter regendae sunt foeminae ; & mas esto ha menester muy pequano tractado \textbf{ ca quando dixiemos | e determinamos del gouer namiento del casamiento } e mostramos en qual manera son de gouernar las muger s casadas \\\hline
2.2.19 & ut plurimum male faciant , \textbf{ cum possunt . } Maxima ergo cautela & que los omes en la mayor parte fazen mal \textbf{ quando pueden . } Et pues que assi es muy grant cautela es de poter \\\hline
2.2.19 & est non assuescere eas inter gentes . \textbf{ Quare cum puellae circemeundo , } et vagando per patriam & nin entre las gentes . \textbf{ por la qual cosa las moças uagando } e andando por la tierra \\\hline
2.2.20 & nescit ociosa esse , \textbf{ statim cum quis non dat se licitis exercitiis , } vagatur eius mens & non labe estar de uagar \textbf{ luego commo alguno non se da a vsos | conueinbł stanto la su uoluntad } andauagando aquende \\\hline
2.2.20 & quod superius dimisimus , \textbf{ cum tractauimus de regimine coniugali . } Dicebatur enim , & Et por esto que dixiemos es declarado lo que dixiemos de ssuso \textbf{ quando rͣctauamos del gouernamiento del } casamien toca y dixiemos que adelante serie de declarar cerca quales obras conuenia \\\hline
2.2.21 & Prima via sic patet \textbf{ nam cum desiderium sit } eius quod abest , & ¶ La primera razon paresçe \textbf{ assi ca commo el desseo del omne sea de aquella cosa | que non ha } e dessea de auer \\\hline
2.2.21 & ubi consueuit maior defectus consurgere . \textbf{ Cum ergo ex hoc } quis loquatur prudenter , & y mayor cautelao se acostunbro de se leunatar mayor fallesçimiento o mayor pecado . \textbf{ Et pues que assi es commo } por esto cada vno fable sabiamente e cuerdamente \\\hline
2.2.21 & tanto magis contingit ipsum incaute loqui ; \textbf{ cum ergo mulieres magis deficiant } a rationis usu quam viri , & tanto mas fabla sin sabiduria . \textbf{ Et pues que assi es commo las mugers | mas } fallezcan de vso de razon \\\hline
2.2.21 & nisi prius ipsum diligenter examinet . \textbf{ Cum ergo diligens examinatio } cum loquacitate stare non possit , & palabrasi primero non la examinare con grand renssamiento ¶ \textbf{ La terçera razon | para prouar esto se } tomadesto \\\hline
2.2.21 & Cum ergo diligens examinatio \textbf{ cum loquacitate stare non possit , } ut foeminae etiam a puellari aetate discant & para prouar esto se \textbf{ tomadesto } que las mugrͣ̃s non sean prestas avaraias e apeleas \\\hline
2.2.21 & ad iurgia et ad lites : \textbf{ nam cum foeminae , } et potissime puellae deficiant a rationis usu , & si non fueren callantias \textbf{ en manera que les conuiene } e si non examinaten con grand cordura las palabras \\\hline
2.3.6 & sicut accidit , \textbf{ cum aliquid committitur pluribus ministris : } cum enim hoc fit , & commo contesçe \textbf{ quando alguna cosa es mandada a muchs siruientes que la fagan . } Ca quando esto se manda \\\hline
2.3.6 & cum aliquid committitur pluribus ministris : \textbf{ cum enim hoc fit , } quilibet ministrorum retrahitur , & quando alguna cosa es mandada a muchs siruientes que la fagan . \textbf{ Ca quando esto se manda } assi en comun cada vno de los siruientes se retrahe \\\hline
2.3.8 & ut viuant secundum corporis voluptatem ; \textbf{ cum diuitiae maxime videantur hoc efficere , } ut per eas quilibet consequi possit & que biuna segunt los delectes del cuerpo \textbf{ e por que las riquezas prinçipalmente fazen esto } assi que por ellas cada vno cuyda \\\hline
2.3.8 & quantum sufficiat ad nutrimentum animalis generati . \textbf{ Cum ergo diuitiae et possessiones ordinentur ad nutrimentum } et ad sufficientiam vitae , & para el nudermiento dela aian lia que es engendrada . \textbf{ ¶ Et pues que assi es las riquezas | e las possessiones sean ordenadas } para nudermiento e abastamiento dela uida \\\hline
2.3.8 & et possessiones appetere ; \textbf{ sed cum tot habet } quot & nin riquezas sin mesura e sin fin . \textbf{ Mas quando ha tantas riquezas } qual abastan segunt el mester de su estado deue ser pagado \\\hline
2.3.9 & Tertia est numismatum ad numismata : \textbf{ ut cum denarii argentei } commutantur in aureos , & ¶La terçera es mudamiento de dineros admeros \textbf{ assi conmodinos de plata se mudan en dineros de oro } o dineros de oro se mudan en dineros de plata \\\hline
2.3.9 & nam ipsius patrisfamilias est totam indigentiam subleuare domesticam . \textbf{ Sed cum eiusdem ad seipsum non sit } nec emptio nec uenditio nec commutatio , & Ca al padre familias parte nesçe dereleuar toda la menguadela casa . \textbf{ Mas por que non puede ser conpra } nin vendida ni mudaçion de vno assi mismo . \\\hline
2.3.9 & et talia quibus indigemus ad vitam , \textbf{ cum sint magni ponderis , } commode ad partes longinquas portari non possunt . & que auemos menester para la uida \textbf{ por que son de grand peso } non las poderemos leuar conueniblemente a luengas tierras . \\\hline
2.3.9 & ne nimis grauarentur homines commorantes in ipso , \textbf{ cum ex una parte regni oportebat } eos accedere ad aliam , & Et pues que assi es por que non fuessen muy agua uiados los omes \textbf{ que moran en vn regno | e que estan en vn logar del regno } commo contesçe alas vezes \\\hline
2.3.11 & et quia hoc est contra naturam artificialium , \textbf{ cum denarius sit quid artificiale , } bene dictum est & e por que esto es contra natura delas cosas artifiçiales \textbf{ e como los diueros sean cosas artifiçiales } bien dicho es lo que dize el ph̃co \\\hline
2.3.11 & id est , rapina usus . \textbf{ Cum ergo usus ipsius domus sit domum inhabitare , } non domum alienare ; & tomassemos serie y usura que quieredezer robo de uso . \textbf{ Et por ende commo el uso dela casa sea morar en la casa } et non enagenar la casa \\\hline
2.3.11 & nisi concedatur eius substantia : \textbf{ cum ad talem usum oporteat } ipsam substantiam alienare . & si non se otorgare la sustançia \textbf{ dellos commo atal uso pertenezca de enagenar la sustançia . } Et pues que assi es por que el açidente desçende dela sustançia del subieto \\\hline
2.3.11 & de eo quod non spectat ad ipsum , \textbf{ cum non ulterius spectet } ad eum usus denarii , & Por la qual cosa el que resçibe ganançia del uso del dinero vende lo que non e suyo o tomagat saçia de aquello que non parte nesçe ael \textbf{ por que dende adelante non pertenesçe ael el uso del diuero } despues que otorgae la sustaçia del . \\\hline
2.3.11 & habent coram se multitudinem pecuniae . \textbf{ Cum ergo quis possit } concedere denarios & dellos sus dineros ponen ante ssi muchedunbre de dineros . \textbf{ Et pues que assi es commo alguon pueda otorgar los dineros para tal uso . } Conuiene a sabra para paresçer con ellos \\\hline
2.3.12 & dicitur esse mercatiua , \textbf{ cum quis per mare aut per terram defert mercationes aliquas , } vel assistit deferentibus mercationes ipsas . & es dichͣ mercaduria \textbf{ assi commo quando alguno por la mar o por la tierra } lieuna algunas nicadurias o esta con aquellos que lieun a las mercadurias . \\\hline
2.3.12 & dicitur esse mercenaria vel conducta : \textbf{ ut cum quis spe mercedis , } vel precio conductus aliqua operatur . & es llamada merçendera o logadera \textbf{ assi commo quando alguno por esꝑança de merçed o de preçio se aluega } para obrar alguas cosas \\\hline
2.3.12 & experimentum enim particularium est . \textbf{ cum ergo quis nouit particularia facta aliquorum , } quibus pecuniam sunt lucrati , & la quarta manera es dicho esperimental de praeua . por quela praeua es delas cosas particulares . \textbf{ Et por ende quando alguno conosçelos fechs particulares de algunos omes } por los quales fechos ganaron alguas riquezas \\\hline
2.3.12 & qui primo philosophari coeperunt . \textbf{ Ipse enim cum esset pauper , } et improperaretur sibi a multis cur philosopharetur , & que primeramente comneçara a philosofo far . \textbf{ Este mille sio commo fuesse muy pobre } e le denostassen sus amigos \\\hline
2.3.12 & et ad quid valeret Philosophia sua , \textbf{ cum semper in egestate viueret . } Ipse non denariorum cupidus , & diziendol que por que se daua tanto ala ph̃ia \textbf{ e aquel aprouechaua suph̃ia pues siengͤ biue en pobreza e en mengua . } Et el por este denuesto \\\hline
2.3.12 & quasi ad pecuniam ordinantur , \textbf{ cum ex opere vel ex arte facto pecuniam intendunt . } Medici enim , fabri , domificatores , & por algun acçidente o por alguna mengua \textbf{ que han aquellos | que vsan de aquella seteᷤ } Ca los fisicos e los ferreros e los carpenteros \\\hline
2.3.12 & et etiam ipsi milites , \textbf{ cum stipendiarii fiunt , pecuniam intendunt . } Decet ergo quemlibet & Et avn los caualleros \textbf{ e quantos toman soldadas | todos entienden admeros . } Et pues que assi es conuiene a cada vno \\\hline
2.3.12 & colligitur multus fructus \textbf{ cum paruis expensis . } Ut ergo sit ad unum dicere , & en algunas trras se cogen muy grand fructo \textbf{ con pequanans despenssas en los logares } que son conueinbles a ellas . Et pueᷤ que assi es por que ayuntemos todos los dichs en vno dezimos \\\hline
2.3.13 & nisi ibi naturaliter aliud sit praedominans : \textbf{ cum societas hominum sit naturalis , } quia homo est naturaliter animal sociale , & fuereynatraalmente alguna cosa \textbf{ que en ssennore e a todas aquellas muchͣs . } Et commo la conpannia de los omes sean a falpor el ome es aianlia \\\hline
2.3.13 & per virtutem animae : \textbf{ cum ergo multi hominum comparentur } ad alias & por uirtud del ala \textbf{ Et pues que assi es commo muchs omes sean conparados aots } assi commo el cuerpo al alma siguesse \\\hline
2.3.13 & diriguntur et saluantur . \textbf{ Quare cum insipientes comparentur } ad industres & por que por la sabiduria de los omes son enderesçados e saluos . \textbf{ Por la qual cosa conmolos non sabios sean conparados a a industria e ala sabiduria de los omes } ¶t commo la bestia alos omes \\\hline
2.3.13 & quod habet consilium inualidum : \textbf{ cum ergo videamus aliquos homines respectu aliorum } plus deficere & que ha conseio muy flaco \textbf{ Et pues que assi es commo nos ueamos | que algs omes en conparaçion de los otros } pueden \\\hline
2.3.14 & si considerentur legum conditores . \textbf{ Nam cum legum latores sint homines , } quibus magis sunt nota bona corporis et exteriora , & si pararemos mientes alos establesçedores delas leyes . \textbf{ Ca commo los establesçedores sołas leyes sean omes } alos quales mas son conosçidos los bienes del cuerpo \\\hline
2.3.14 & si scirent se ex eis nullam utilitatem consecuturos ; \textbf{ sed cum cogitant eos acquirere in seruos , } reseruant ipsos propter utilitatem & soperiessen que nigunt pro non aurian de tal uençimiento . \textbf{ Mas quando pienssan que aquellos a quien vençe } que los gana \\\hline
2.3.15 & ampliora beneficia tribuenda : \textbf{ cum ergo virtuosus seruiens } ex amore honesti , & Ca sienpre alos mas digunos son de dar los mayores benefiçios . \textbf{ Et por ende commo el seruiente uirtuoso } que sirue por amor de bien e de honestad sea mas digno que el que sirue \\\hline
2.3.16 & et amplius honorari , et praemiari a principante . \textbf{ Cum in domibus Regum et Principum ministris officia dispensantur , } ut aliqui praesint mensis , & qua los orros \textbf{ Omo en las casas delos Reyes | e delos prinçipes de una ser partidos los ofiçios a los siruientes . } assi que agunos sean ppuestos sobre las mesas . \\\hline
2.3.16 & multi ministrantes quam pauci . \textbf{ Nam cum multis idem ministerium committitur ; } saepe illud negligitur : & do dize que alguas vezes peor siruen los muchs seruidores que los pocos . \textbf{ Ca quando vn seruiçio es a comnedado a muchs } muchͣs vezes aquel seruiçio es mal fech o perdido \\\hline
2.3.17 & et congruentia temporum . \textbf{ Cum enim deceat Regem esse magnificum , } ut supra in primo libro diffusius probabatur , & La conueniençia de los tiepos . \textbf{ Ca commo conuenga alos Reyes | e alos prinçipes ser magnificos } assi commo es prouado mas conplidamente en el primero libro \\\hline
2.3.17 & in ordine Uniuersi , \textbf{ quod cum totum uniuersum sit } quasi una domus summi Principis , & Ca assi commo ueemos en la orden de todo el mundo \textbf{ que todo el mundo es | assi es } assi commo vna casa del muy alto prinçipe \\\hline
2.3.17 & Quinto circa hoc considerandum occurrit congruentia temporum . \textbf{ Nam cum haec inferiora corpora per super caelestia regantur } prout variantur tempora , & que son las cosas conuemientes alos tienpos . \textbf{ Ca commo estos cuerpos de aqui deyuso sean gouernados | por los cuerpos del çielo } segunt que se mudan e desuarian los tienpos \\\hline
2.3.18 & nobiles ergo homines , \textbf{ quia cum pluribus conuersantur , } quasi experti & e por ende los nobles \textbf{ por que han conuersaçion con muchs } e por que han prouado muchas cosas \\\hline
2.3.18 & Sic etiam si hylariter et affabiliter \textbf{ quis cum aliis conuersetur , } si hoc agit & si \textbf{ algbiuiere et morare con los otros alegremente } e amigablemente si esto faze \\\hline
2.3.19 & Secundo , ut ad officia commissa debite solicitentur . \textbf{ Tertio , ut sciatur qualiter cum eis est conuersandum . } Quarto , qualiter ipsis communicanda consilia & conueinblemente cerca los ofiçios \textbf{ que les son acomnedados . | ¶ Lo terçero que lep̃a los prinçipes } en qual manera han de beuir con ellos \\\hline
2.3.19 & videlicet qualiter \textbf{ cum ipsis sit conuersandum . } Quod aliquomodo tradit Philosophus & ¶ Esto iusto finça de demostrar lo terçero \textbf{ que es en qual manera han de beuir los sennores con sus ofiçiales } la qual cosa muestra el philosofo en alguna manera \\\hline
2.3.19 & et qualitater est \textbf{ cum ipsis conuersandum : } reliquum est ostendere , & e en qual manera deuen ser acuçiosos çerca ellos \textbf{ e en qual manera deuen conuersar e veuir con ellos } ¶ finca de demostrar \\\hline
2.3.19 & magis quam merces aliqua quam inde habituri essent . \textbf{ Cum seruis ergo naturaliter non sunt communicanda secreta neque consilia : } quia ( ut supra dicebatur ) & alguaque ende espaauer . \textbf{ Et pues que assi es commo alos que son sieruos naturalmente | non londe descobrar las poridades } nin los conseios \\\hline
2.3.20 & tam ipsi Reges et Principes \textbf{ et omnes recumbentes cum eis , } quam etiam ministrantes . & e los prinçipes \textbf{ e aquellos que estan assentados con ellos } commo avn aquellos que los siruen . \\\hline
2.3.20 & esse confusio , \textbf{ cum opera eius sint } ab ipso deo et intelligentiis ordinata : & ca en las obras dela natura non deue ser confusion \textbf{ por que las obras dela natura son ordenadas de dios } e de los angeles . \\\hline
2.3.20 & contra naturalem ordinem est \textbf{ cum per illud organum intendimus } circa unum illorum operum , & contra natural ordenes \textbf{ quando por aquel estrumento entendemos fazen vna de aquellas cosas } si en aquel tienpo non quedaremos dela otra obra . \\\hline
2.3.20 & si per illud tempus non cessemus ab alio . \textbf{ Quare cum secundum Philosophum } in tertio de Anima , & si en aquel tienpo non quedaremos dela otra obra . \textbf{ por la qual cosa segunt el philosofo } en el terçero libro del alma \\\hline
2.3.20 & et locutionem contra naturalem ordinem est , \textbf{ cum huiusmodi organum exercemus } ad illud opus & Et por ende contra orden natural es \textbf{ que quando nos usamos deste estrumento } que es la lengua a aquella obra que es gostar . \\\hline
2.3.20 & ut simul , \textbf{ cum fauces recumbentium cibum sumunt , } earum aures doctrinam perciperent ; & sienpre se \textbf{ leyessen algunas cosas proprouechosas } assi que quando ellos comne oyessen \\\hline
3.1.1 & esse communitatem quandam , \textbf{ cum omnis communitas fit } gratia alicuius boni , & or que toda çibdat conuiene que sea alguna comunindat \textbf{ commo toda comunidat sea por graçia de algun bien . } Conuiene que la çibdat sea establesçida por algun bien \\\hline
3.1.1 & Si ergo omnes homines ordinant sua opera in id quod videtur bonum , \textbf{ cum ciuitas sit opus humanum , } ex parte hominum constituentium ciuitatem oportet & a aquello que paresçe bien \textbf{ o commo la çibdat | sea obra de los omes } de parte de los omes \\\hline
3.1.1 & quod existit bonum . \textbf{ Nam cum opera nostra ordinamus } ad aliquod bonum , & e que non es bueno en ssica \textbf{ commo las nuestras obras nos ordenamos } a algun bien alguas uezes \\\hline
3.1.2 & nisi viuant bene et virtuose , \textbf{ cum sine lege et iustitia } constituta ciuitas stare non posset , & si non biuiessen bien \textbf{ e uirtuosamente commo la çibdat establesçida sini ley } e sin iustiçia non pueda estar \\\hline
3.1.4 & et hominem non esse naturaliter animal ciuile . \textbf{ Cum ergo non satis sit } remouere omnes obiectiones & e que el oen non era naturalmente aianlçiuil . \textbf{ Et pues que assi es commo non cunpla soluer la } sobiecconnes contrarias \\\hline
3.1.4 & quia deseruiunt ad sufficientiam uitae humanae . \textbf{ Quare cum communitas ciuilis has communitates comprehendat , } et perfectius deseruiat & por que siruen al conplimiento dela uida humanal \textbf{ por la qual cosa commo la comuidat çiuil | o la çibdat conprehenda estas dos comuindades } e mas acabadamente sirua al conplimiento dela uida humanal \\\hline
3.1.4 & Canis enim eo quod latrat , \textbf{ aliter latrat cum delectatur : } et cum tristatur potest & por que ladra en otra manera \textbf{ quando se delecta | e en otra manera } quando se trista puede a otro can de mostrar \\\hline
3.1.4 & aliter latrat cum delectatur : \textbf{ et cum tristatur potest } alteri cani per suum latratum & e en otra manera \textbf{ quando se trista puede a otro can de mostrar } por su ladrado sutsteza o su delecta conn que ha mas al omne \\\hline
3.1.4 & et ad constituendum ciuitatem . \textbf{ Sed cum id , } ad quod habemus impetum naturalem , & e para fazer çibdat \textbf{ mas commo aquello a que auemos inclinaçion natural sea cosa natural } conuiene quela çibdat sea cosa natural \\\hline
3.1.5 & Prima via sic patet . \textbf{ Nam cum ait Philosophus primo Polit’ } quod communitas perfecta , & La primera paresçe \textbf{ assi ca segunt que dize el philosofo | en el primero libro delas politicas } que la comunidat acabada \\\hline
3.1.5 & et bonum statum aliorum ciuium . \textbf{ Quare cum peruersi in ciuitate aliqua } non audeant insurgere contra principem , & e el bue estado de los otros çibdadanos \textbf{ por la qual cosa commo los malos en alguna çibdat } non se osenle unatat \\\hline
3.1.5 & Videmus enim quod \textbf{ cum aliqua ciuitas impugnatur } confoederat se ciuitati alii , & e dela redramiento del enbargo de los enemigos . \textbf{ ca veemos que quando alguna çibdat es conbatida de los } enemigos \\\hline
3.1.5 & resistere impugnationem hostium : \textbf{ cum ergo regnum sit } quasi quaedam confederatio plurium ciuitatum , & que la conbaten . \textbf{ ¶ Et pues que assy es commo el regno sea } assi commo vna amistança de muchͣs çibdades \\\hline
3.1.6 & Secundus autem modus constitutionis regni et ciuitatis , \textbf{ ut cum ex concordia hominum } talis constitutio habet esse , & establesçi miento del regno \textbf{ e dela çibdat es natural } assi commo quando se faze tal establesçimiento \\\hline
3.1.6 & qui quasi est simpliciter violentus . \textbf{ Cum enim homines dispersi morabantur , } poterat & assi commo manera forcada \textbf{ por que quando los omes mora una esꝑzidos } e cada vno podie le una tarse \\\hline
3.1.7 & et maxima coniunctio in ciuitate . \textbf{ Nam cum sit maxima unitas , } et maxima coniunctio patrum ad filios , & e muy grant ayuntamiento enla çibdat . \textbf{ Ca commo sea muy grant vnidat } e grant ayuntamiento de los padres alos fijos los mas antiguos \\\hline
3.1.7 & habere maximam unitatem \textbf{ cum iunioribus , } et econuerso , & que auian muy grant vnidat con los moços \textbf{ e esso mismo los mocos cuydarian } que auian grant vnidat \\\hline
3.1.7 & in quo communicamus \textbf{ cum animalibus aliis , } videtur ciuitas maxime naturaliter ordinata & por ende paresçe que segunt la orden natural \textbf{ en la qual partiçipamos con las otras aianlias } que las mugeres deuen lidiar tan bien commo los omes . \\\hline
3.1.8 & ad principantes et subiectos . \textbf{ Nam cum ciuitas sit ordo ciuium } ad aliquem principantem vel dominantem , & por conparaicion de los prinçipes e de los subditos \textbf{ ca commo la çibdat sea orden } delos çibdadanos a algun prinçipe o algun sennor \\\hline
3.1.8 & ad aliquem principantem vel dominantem , \textbf{ ut cum in ciuitate oporteat } dare aliquos magistratus , & delos çibdadanos a algun prinçipe o algun sennor \textbf{ e commo en la çibdat conuenga de dar alguons ofiçioso } alguons maestradgos o algunas alcaldias \\\hline
3.1.8 & et aliqui subiecti . \textbf{ Quare cum hoc diuersitatem requirat , } oportet in ciuitate & e alguons que fuessen subditos \textbf{ por ende commo estas cosas demanden departimiento } conuiene de dar en la çibdat algun departimiento . \\\hline
3.1.9 & ne circa ipsa contingat error . \textbf{ Quare cum in regimine ciuitatis primo sit politia ordinanda , } diu inuestigandum est , & escodrinnar por que çerca ellos non contezca yerro . \textbf{ por la qual cosa commo en el gouernamiento dela çibdat | primeramente se ha de ordenar la poliçia } muy luengamente es de buscar \\\hline
3.1.10 & inter seipsos iniurias et contumelias inferant . \textbf{ Quare cum communitas uxorum , } quam Socrates ordinauit , & nin tuertosa sus padres e a sus parientes \textbf{ por la qual cosa commo la comuidat delas mugieres } que ordeno socrates \\\hline
3.1.10 & et aequale amicitiam soluit . \textbf{ Cum ergo boni et nobiles sint } supra viles et ignobiles , & e saluna la amistança . \textbf{ Et pues que assi es commo sean muchos buenos e nobles en la çibdat sobre los uiles } e sobre los que non son nobles \\\hline
3.1.10 & habere curam et diligentiam , \textbf{ ne filii coirent cum matribus , } et patres cum filiabus . & que al prinçipe dela çibdat pertenesçia de auer cuydado e acuçia \textbf{ por que los fijos non yoguiessen con sus madres } nin los padres con sus fuas . \\\hline
3.1.10 & ne filii coirent cum matribus , \textbf{ et patres cum filiabus . } Sed ait Philosophus , & por que los fijos non yoguiessen con sus madres \textbf{ nin los padres con sus fuas . } Mas assi commo el pho dize esto non abasta \\\hline
3.1.10 & ex quo non manifestabatur eis parentela illa . \textbf{ Quare cum in personis tam coniunctis } non solum detestabilis sit actualis commistio , & o el su parentesco \textbf{ por la qual cosa commo en las personas tan ayuntadas } non solamente es de denostar el ayuntamiento carnal \\\hline
3.1.11 & superficietenus considerata valde expediens ciuitati : \textbf{ cum enim quis audit ciuitatem } sic ordinatam esse & e liuianamente paresçe muy conueinble a la cibdat \textbf{ ca quando algud vee } que la çibdat es es si ordenada \\\hline
3.1.11 & Prima via sic patet . \textbf{ nam cum aliquid est commune aliquibus , } cum unus ab usu et fructu illius rei communis & La primera razon paresçe \textbf{ assi ca quando alguna cosa es comuna } alguons \\\hline
3.1.11 & nam cum aliquid est commune aliquibus , \textbf{ cum unus ab usu et fructu illius rei communis } propter alium impeditur , & assi ca quando alguna cosa es comuna \textbf{ alguons } quando el vno enbarga al otro del uso \\\hline
3.1.11 & multa orirentur litigia , \textbf{ cum ipsi multi sint } et diuersarum voluntatum , & que muchͣs contiendas \textbf{ por que ellos son muchs } e de departidas uoluntades \\\hline
3.1.11 & tanto magis ad inuicem conuersantur : \textbf{ sed cum esse non possit , } aliquos valde & mas han de beuir en vno \textbf{ mas commo non pueda ser } que alguons \\\hline
3.1.11 & ad illos multa colloquia , \textbf{ et diu conuersari cum illis . } Quare si inter dominos et famulos quos habent & por que nos conuiene de fablar muchͣs uezes con ellos \textbf{ e de beuir conellos | non los podiendo escusar } que si entre los sennoron e los sus sieruos \\\hline
3.1.11 & quae et Socrates concedebat . \textbf{ Cum ergo custodes ciuitatis nobiliores sint agricolis , } tanquam meliores et nobiliores estimarent se plus esse accepturos & e esto otorgauas ocrates . \textbf{ Et pues que assi es commo las guardas | e los defendedores dela çibdat sean mas nobles que los labradores } assi commo meiores \\\hline
3.1.11 & tamen propter liberalitatem quantum ad usum erant illis ciuibus communes serui , et equi , et canes : \textbf{ quilibet enim ciuium cum indigebat , } absque alia requisitione utebatur alterius equis , canibus , et seruis . & assi commo los cauallos e los sieruos e los canes \textbf{ ca cada vno de los çibdadanos | quando auia meester alguna cosa sin la demandar al otro } vsauad los cauallos \\\hline
3.1.12 & quam eos in societate habere \textbf{ nam cum humanum sit timere mortem , } viriles etiam et animosi trepidant & que auerlos en su conpannia \textbf{ ca commo todos los omes | teman la muerte los esforçados } e de grandes coraçones temen \\\hline
3.1.12 & ex parte fortitudinis corporalis . \textbf{ Nam cum bellantes oporteat } diu & que son temerosos ¶ \textbf{ La terçera razon se toma de parte dela fortaleza corporal } ca commo los lidiadores ayan de sofrir el peso delas armas \\\hline
3.1.13 & vel ad aliquem magistratum assumitur . \textbf{ Quare cum deceat regia maiestatem } et uniuersaliter omnem ciuem , & commo se conosçe despues que esle un atada en alguna dignidat o en algun maestradgo o en algun poderio \textbf{ por la qual razon commo venga ala real magestad } e generalmente a qual quier que ha de dar \\\hline
3.1.13 & tangit Philosophus 2 Poli’ \textbf{ cum ait . } Socrates semper facit eosdem Principes , & en el segundo libro delas politicas \textbf{ quando } dizeque socrates \\\hline
3.1.14 & quid circa huiusmodi regimen sit censendum . \textbf{ Quare cum patefactum sit in praecedentibus , } non expedire ciuitati possessiones , & que auemos de iudgar en este gouernamiento \textbf{ por la quel cosa commo sea manifiesto | por las cosas dichͣs de suso } que non conuiene ala çibdat \\\hline
3.1.14 & nisi bellare , \textbf{ cum adesset oportunitas : } et onerosius et quasi omnino importabile esset sustentare sic quinque milia : & si non lidiar \textbf{ quando fuesse me este } Et muy mayor carga e peor de sofrir serie \\\hline
3.1.14 & de conditionibus bellantium . \textbf{ Quare cum ars et scientia non possit } esse circa particularia signata , & departidamente se deue tomar el cuento de los lidiadores \textbf{ por la qual cosa la arte e la sçiençia non pueden ser } cerca las cosas particulares e senñaladas . \\\hline
3.1.15 & ipse Socrates habebat , \textbf{ cum Plato eius discipulus fuisset . } Si volumus non ut verba sonant & por auentra a auia socrates \textbf{ por que platon fue su disçipulo . } Si quisieremos entender los dichos de socrates \\\hline
3.1.15 & et sit solicitus \textbf{ ( cum adest facultas ) } de rebus aliorum , & e sea cuydados \textbf{ o quanto pudiere delas cosas de los otros } assi commo si fuessen suyas . \\\hline
3.1.15 & sic etiam saluare possumus dictum eius quantum ad unitatem ciuitatis . \textbf{ Nam cum dixit ciuitatem debere esse maxime unam , } forte non intellexit de unitate habitationis , & del quanto ala vnidat dela çibdat \textbf{ ca quando dixo | que deuia ser la çibdat much vna } por auentura non entendio de vnidat dela morada \\\hline
3.1.15 & quorum non est manibus operari . \textbf{ Cum enim in ciuitate } quidam manibus opererentur & ningunan otra cosa con sus manos \textbf{ ca commo en la çibdat alguon sobren por manos } assi commo los que labran la trarra e los menestrales \\\hline
3.1.16 & ad aequalitatem mediantibus dotibus statuendo \textbf{ quod pauperes contrahant cum diuitibus : } et in contrahendo accipiant dotes , & que se dan en los casamientos \textbf{ ca establesçiendo que los pobrescasen con las ricas } e en casamiento resçiban grandes arras \\\hline
3.1.17 & Quare non habebunt possessiones aequales . \textbf{ Cum ergo quanto aliquid } in plures partes diuiditur , & que non aurien las possessiones yguales . \textbf{ Pues que assi es commo quanto alguna cosa es partida en mas partes } tanto menos resçibe cada vna parte de aquella cosa \\\hline
3.1.17 & et iurgia in ciuitate . \textbf{ Primo quia pauperes cum ditantur nesciunt fortunas ferre , } ut plane ostendit & contesçerien iniurias e tuertos e uaraias en la çibdat . \textbf{ Lo primero quando los pobres se fazen ricos | non saben sofrir la su buena uentura } assi commo el philosofo muestra llanamente en el segundo libro de la rectoriça \\\hline
3.1.17 & non esse insolentes . \textbf{ Quare cum filii diuitum } ut plurimum sint magnanimi et magni cordis & nin turbadores de paz \textbf{ ca porque los fijnos de los ricos } por la mayor parte son magnanimos \\\hline
3.1.19 & eo quod talibus alii de facili iniuriantur , \textbf{ cum non possint defendere iura sua . } Multa bona consequimur & por que tales perssonas los otros de ligero les fazen tuerto \textbf{ por que non pueden defender su derecho } uchos bienes se nos siguen delas opiniones de los phos antigos \\\hline
3.1.20 & non poterat \textbf{ cum statuto de electione principis . } Si enim ciuitas & non puede estar \textbf{ con el establesçimiento dela elecçion | oł prinçipe } por que si la çibdat \\\hline
3.1.20 & et homines libenter iniustificant \textbf{ cum possunt ; } hoc posito artifices , & sienpre quieren ser senors en las çibdades \textbf{ e los omes de grado fazen tuerto quando puede . } Puesto que el dizie siguese que los menestrales \\\hline
3.1.20 & stare non potest \textbf{ cum statuto de bellatoribus , } ut quod ipsi sint potentiores aliis , & non puede estar con el \textbf{ establesçimiento que fizo de los lidiadores } que ellos fuessen mas poderosos que los otros \\\hline
3.2.2 & ex dominio unius unus rectus , \textbf{ ut cum propter bonum commune dominatur Rex : } et alius peruersus , & e assi paresçe que dos principados se le una tan del sennono de vno prinçipado de derechas \textbf{ si commo quando por el bien comun en | llennore a vno este es dicho Rey } Et otro prinçipado malo \\\hline
3.2.2 & et alius peruersus , \textbf{ ut cum propter bonum priuatum principatur Tyrannus . } Contingit tamen aliquando ciuitatem & Et otro prinçipado malo \textbf{ quando por el bien propreo | enssennorea vno } e este es dich tiranno . Enpero contesçe alguas uezes \\\hline
3.2.2 & unus rectus , \textbf{ ut cum dominantur aliqui , } quia sunt virtuosi et intendentes commune bonum : & se leuna tan del sennorio de po cos vno derecho \textbf{ assi commo quando enslennorean alguons } que son uirtuosos \\\hline
3.2.2 & et alius peruersus , \textbf{ ut cum dominantur aliqui , } non quia sunt boni , & e entienden enł bien comun . \textbf{ Et otro malo assi commo quando enssenore an alguons } non por que son bueons \\\hline
3.2.3 & quam si dominentur plures . \textbf{ Immo cum plures principantur , } nunquam potest esse pax in huiusmodi principatu , & que quando \textbf{ enssennorea much sante | quando enssennorean muchs } nunca puede ser paz en tal prinçipado \\\hline
3.2.3 & nisi iuuantur in tractu , \textbf{ ut cum unus trahit , } alius trahat ; & si non fueren ayuntados en vn traymiento \textbf{ assi que quando el vno tira } que tire el otro nunca se traeria la naue \\\hline
3.2.4 & dum tamen utrunque sit rectum , \textbf{ cum ipse pluries dicat in eisdem politicis , } regnum esse dignissimum principatum : & Puesto que amos los ssennorios sean derechs \textbf{ ca el dize muchͣs uezeᷤ | en esse mismo libro delas politicas } que el regno es prinçipado muy digno \\\hline
3.2.4 & quam dominari unum ; \textbf{ cum nunquam plures recte dominari possint , } nisi inquantum tenent locum unius , & que si enssennoreas se vno . \textbf{ ca nunca pueden much | senssennorear } derechͣmente \\\hline
3.2.5 & ut voluntarie obediant : \textbf{ quare cum omne voluntarium sit minus onerosum et difficile , } ut libentius et facilius obediat populus mandatis regis , & assi commo a cosanatal \textbf{ ca commo toda cosa uoluntaria | sea de menor carga e menos graue } por que mas de grado \\\hline
3.2.5 & nam contingit ea aliquando diu carere gubernatore , \textbf{ et cum gubernatorem habent } ut plurimum tyrannizant : & que alguons regnos algunas uezes non han gouernador en algunt p̃o . \textbf{ Et quando han gouernador muchͣs uezes t hiranzan . } Onde muchs males auemos visto en tales gouernamientos \\\hline
3.2.5 & et bonis moribus sint imbuti ; \textbf{ cum bonum commune et totius regni in hoc consistat . } Nec sufficit quod quia solus primogenitus regnare debet , & e bien enformados en todas buenas costunbres \textbf{ por que el bien de todo el regno esta en esto . } Et non cunple \\\hline
3.2.6 & quia materia tunc ignitur \textbf{ cum calefit et rarefit , } oportet raritatem et calorem perfectius reperiri & e por ca lentura . \textbf{ por que la materia estonçe es puesta e tornada en fuego } quando es muy escalentada \\\hline
3.2.6 & tyrannis vero est dominium peruersum . \textbf{ Cum ergo bonum gentis sit } diuinius bono unius , & por que el regno es prinçipado derech . \textbf{ mas la tirania es sennorio tuerto e malo . Et pues que assi es commo el bien comun delas gentes sea mas diuinal que el bien de vno . } malamente e desigualmente \\\hline
3.2.7 & Sed si tyrannus dominetur , \textbf{ cum unus sit dominans , } et non intendat nisi bonum proprium ; & Mas si el tirano \textbf{ enssennoreare commo vno sea el señor } e non entienda \\\hline
3.2.9 & Tyranni autem hoc non faciunt : \textbf{ nam cum non intendant } nisi commodum proprium & Mas los tiranos esto non fazen \textbf{ ca commo ellos non entienden } si non el su bien propo \\\hline
3.2.9 & Recitat autem Philosophus 5 Polit’ \textbf{ quod cum quidam Rex partem sui regni dimisisset , } quia eam forte iniuste tenebat : & en el quanto libro delas politicas \textbf{ que commo vn Rey dexasse vna parte de su regno . } por que por auentura non la tenie \\\hline
3.2.10 & est excellentes perimere . \textbf{ Cum enim tyrannus non diligat } nisi bonum proprium , & La primera cautela del tirano es matar los grandes omes e los poderosos . \textbf{ Ca commo el tirano non ame sinon el su bien propreo los poderosos e los nobłs } ̃en el su regno \\\hline
3.2.10 & nisi quomodo possit excellentes perimere . \textbf{ Immo cum aliqui tyrannizare cupiunt , } non solum excellentes alios , & si non commo podra matar los grandes e los nobles . \textbf{ Ante quando alguons cobdician de fazer tiranias } non solamente matan los grandes \\\hline
3.2.10 & et de se confidere ; \textbf{ nam cum intendat bonum ipsorum ciuium et subditorum , } naturale est & e que fien vnos de otros . \textbf{ Ca commo el entienda enl bien de los çibdadanos natural } cosaes que sea amado dellos . \\\hline
3.2.10 & quicquid a ciuibus agitur . \textbf{ Cum enim tyranni sciant se non diligi a populo , } eo quod in multis offendant ipsum , & delo que fazen los çibdadanos . \textbf{ Ca commo los tyranos sepan | que non lon amados del pueblo . } por que en muchͣs cosas le aguauian quieren auer muchs assechadores \\\hline
3.2.10 & Volunt enim tyranni turbare \textbf{ amicos cum amicis , } populum cum insignibus , & que ya son fechos turbar las e destroyr las . \textbf{ Ca quieren los tyranos turbar los amigos con los amigos } e el pueblo con los nobles \\\hline
3.2.10 & amicos cum amicis , \textbf{ populum cum insignibus , } insignes cum seipsis . & Ca quieren los tyranos turbar los amigos con los amigos \textbf{ e el pueblo con los nobles } e los nobles con si mismos . \\\hline
3.2.10 & populum cum insignibus , \textbf{ insignes cum seipsis . } Vident autem quod quandiu ciues discordant a ciuibus , & e el pueblo con los nobles \textbf{ e los nobles con si mismos . } Caueen que mientra los çibdadanos desacuerdan entre si mismos \\\hline
3.2.10 & et partes in regno , \textbf{ cum una parte affligit aliam } ut clauum clauo retundat . & pues que ha puesto vandos e departimientos en el regno \textbf{ que con vn unado atormente al otro assi que con vn clauo atenaçe el otro . } Mas el buen Rey faze todo el contrario \\\hline
3.2.12 & ad quam potio ordinatur . \textbf{ Cum ergo homines communiter } et maxime habentes voluntatem peruersam , & a aquello que el quiere \textbf{ e por ende commo los omes comunalmente } e mayormente los que han la uoluntad desordenada orden en sus riquezas \\\hline
3.2.12 & Legitur enim de quodam tyranno , \textbf{ qui cum a fratre suo cotidie increparetur , } quare ipse semper tristis existeret , & ca leemos de vn tirano \textbf{ que cada dia era denostado de vn su hͣrmano } por que sienpre andaua triste \\\hline
3.2.12 & pendentem tenuissimo filo apponi fecit : \textbf{ circa ipsum quosdam homines cum ballistis , sagittis appositis , } stare faciebat . & e fizol colgar una espada sobre su cabesça muy aguda de vn filo muy delgado \textbf{ e fizo | poñuallesteros con ballestas armadas contrael . } Et estonçe commo aquel su hͣrmano tomasse grant espanto \\\hline
3.2.12 & stare faciebat . \textbf{ Et cum frater eius timore horribili inuaderetur , } timens ab acuto gladio vulnerari , & poñuallesteros con ballestas armadas contrael . \textbf{ Et estonçe commo aquel su hͣrmano tomasse grant espanto } e grant temor \\\hline
3.2.12 & ne conuertatur in tyrannum . \textbf{ Nam tyranni cum hoc quod perdunt aeternam vitam , } in hac temporali vita vix unam diem bonam habent , & que non se torne en tyrano . \textbf{ ca los tiranios maguera pierdan la uida perdirable } avn en esta uida tenporal apenas ha vn buen dia \\\hline
3.2.12 & Priuatur ergo tyrannus a maxima delectatione , \textbf{ cum videat se esse populis odiosum . } Viso tyrannidem cauendam esse , & e por ende el tirano es pri uado de grant delectaçion \textbf{ quando bee | que es aborresçido delos pueblos } Disto que la tirauja es de esquiuar e de aborresçer \\\hline
3.2.12 & mala peruersi principatus populi . \textbf{ Cum enim populus principatur peruerse , } non intendit quodlibet seruare in suo statu , & por que en ella son ayuntados los males del mal priͥnçipado del pueblo \textbf{ por que quando el pueblo enseñorea malamente non entiende guaedar a } njnguno en su estado mas esfuercas \\\hline
3.2.13 & ut recte et debite gubernent populum sibi commissum : \textbf{ cum deuiare a recto regimine sit tyrannizare , } et iniuriari subditis , & e commo desinarse \textbf{ e arredrarse los rreyes del | gouernemjento derecho sea tiranizar } e fazer tuerto alos subditos \\\hline
3.2.13 & Nam multi pusillanimes existentes , \textbf{ cum nimis timent , } et non credunt se posse euadere , & La primera es por temorça munchons \textbf{ que son de flaco coraçon } quando temen muncto non creyendo \\\hline
3.2.13 & per quae se contemptibiles reddunt . \textbf{ Nam cum non quaerant bonum commune , } sed delectationes corporis , & por que se fazen despreçiados de los pueblos \textbf{ ca por qua non qͥeren el bien comun } mas quieren delectaçiones del su cuerpo \\\hline
3.2.13 & aut propter lucrum adipiscendum . \textbf{ Nam cum honor et gloria } inter bona exteriora sint bonum maximum , & por ganar honrra o auer ganançia , \textbf{ Ca commo la honrra e la gloria deste mundo } sean muy grandes bienes \\\hline
3.2.14 & iustitia enim urbanitates seruat . \textbf{ Quare cum a iustitia receditur , } praeparatur via & e todas las leyes . \textbf{ por la qual cosa quando algͤse arriedra dela iustiçia apareia en ssi carrera } por que sea corrun pido el su sennorio \\\hline
3.2.14 & et una monarchia tyrannica contrariatur alii . \textbf{ Cum enim aliquis monarcha } vel aliquis unus Princeps tyrannizet in populum , & escontrana ala tirama del mal prinçipado \textbf{ Et vn prinçipado tiranico es contrario a otro prinçipado tiranico e malo } ca quando algun \\\hline
3.2.14 & Debent ergo cauere Reges et Principes ne tyranizent , \textbf{ cum tot modis dissoluatur tyrannicus principatus . } Tertio dissoluitur tyrannis & e los prinçipes \textbf{ que non tiraniz en commo en tantas maneras | segunt dicho es se aya de destroyr el prinçipado tiranico . } Lo terçero se desfaze la tirani a non solamente \\\hline
3.2.15 & Talia autem maxime sciri poterunt per experientiam : \textbf{ nam cum quis diu expertus est regni negocia , } de leui arbitrari & e por prueua \textbf{ ca quando alguno ha prouado por luengo tienpo los negoçios del regno de ligero puede penssar qual cosa } corronpe el buen estado del regno \\\hline
3.2.16 & quae tractanda sunt circa ipsum . \textbf{ Sed , cum dicat Philosophus } 3 Ethic’ & quales cosas son de trattrar çerca el . \textbf{ Mas commo el pho diga en el segundo libro delas ethicas } que por çierto alguno tomara consseio non de aquellas cosas \\\hline
3.2.17 & ut de magnis consilietur negotiis . \textbf{ Tertio cum consiliari volumus , } debemus alios assumere nobiscum , & por que de grandes cosas tomemos consseio . \textbf{ la tercera cosa es que quando queremos tomar consseio deuemos tomar } connusco otros con los quales ayamos acuerdo delas cosas \\\hline
3.2.17 & in consiliis hunc habere modum : \textbf{ ut cum aliis conferamus quid agendum , } quod ex duobus patet . & ca de grant sabiduria es en los consseios tener esta manera \textbf{ que con los otros ayamos acuerdo } delo que auemos de fazer la qual cosa paresçe por dos cosas . \\\hline
3.2.17 & plus proficit expertus , quam artifex . \textbf{ Quare cum plures plura experti sint , } quam unus solus : & que el que ha el arte della \textbf{ por la quel cosa commo muchs mas cosas ayan prouadas } que vno solo conuiene de llamar otros \\\hline
3.2.17 & cito in opere exequamur . \textbf{ Nam cum adest opportunitas operandi , } et si recte volumus & que luego lo pongamosen obra . \textbf{ ca quando viene el tp̃on coueinble para obrar } si derechͣmente queremos obrar \\\hline
3.2.18 & nam quia dicens creditur esse bonus , \textbf{ cum tales mentiri nolint , } de facili creditur eorum dictis . & que es bueno \textbf{ e los bueons non quieren mentir } de ligero creen los omes asodichos . \\\hline
3.2.18 & apparenter saltem . \textbf{ Itaque cum dictum sit } quod qui bene persuadens , & que parezca tal . \textbf{ Et por ende commo sea dicho } que el que es buen amonestador e razonador \\\hline
3.2.19 & et pondera vendentium : \textbf{ et cum expedit taxandum } est pretium venditionis , & e los pesos de los vendedores e de los conpradores . \textbf{ Et quando fuere menester tassar } e poner presçio alas cosas \\\hline
3.2.19 & et in hoc non est quaestio nec consilium . \textbf{ Sed utrum cum extraneis debeamus habere pacem } vel bella dubitabile esse potest , & questiuo nin consseio . \textbf{ Mas si deuemos auer paz con los estrannos o guerra pue de ser cosa dubdosa } e çerca esto pueden ser tomados consseios çerca la qual \\\hline
3.2.20 & et alia quae circa istam occurrunt materiam . \textbf{ Sed cum iudicium fiat per leges , } aut per arbitrium , & que pueden acaesçer çerca desta materia \textbf{ mas commo el iuyzio se deua fazer } por las leyes o por aluedrio o por amas estas cosas . \\\hline
3.2.20 & statim iudicatiuam sententiam proferre . \textbf{ Itaque cum legum conditores multo tempore } et magno consilio deliberare possint & que si luego man ama no fuesse dadas m̃a difinitiua . \textbf{ por ende commo los fazedores delas leyes en muchtp̃o } e con grant conseio puedan et de una determinar quales leyes de una poner . \\\hline
3.2.20 & finem executione iudiciorum . \textbf{ Quare cum Iudex iudicando reos } secundum leges & e dar fin a los iuyzios \textbf{ e por la qual cosa commo el iues | iudgando los culpados } segunt las leyes \\\hline
3.2.20 & sufficienter enim iudex excusatur , \textbf{ cum secundum positas leges aliquid iudicat ; } quia non videtur ipse & e muy pocas cosas son de dexar en aluedrio de los mueze ᷤ \textbf{ quando iudga alguna cosa | segunt las leyes puestas } ca non paresçe \\\hline
3.2.21 & inquantum allegant leges conditas a legumlatore . \textbf{ Quare cum sermones passionabiles } inclinent voluntatem , & por el ponedor dela ley . \textbf{ por la qual cosa conmo las palabras desiguales } enclinan la uoluntad de los omes \\\hline
3.2.21 & et sunt extra negocium iudicandi . \textbf{ Nam cum lis fit de aliquo facto vel de aliquare , } nihil oportet dici in iudicio & que ha de ser iudgado . \textbf{ Ca commo la contienda sea de algun fech̃ | o de alguna cosa } non se deue dezir ninguna cosa en iuizio \\\hline
3.2.23 & pro clementia delinquentis . \textbf{ Nam cum natura humana de se sit debilis , } et mutabilis , et prona ad malum , & por la piadat del que peca . \textbf{ ca commo la natura humanal } dessi sea flaca e mouible \\\hline
3.2.23 & magis debeat agere \textbf{ cum eo misericorditer quam crudeliter . } Ideo dicitur 1 Rheto’ & Mas deue el iues tener manera de mibicordia \textbf{ que de crueldat contra el culpado . } Et por ende dize el pho enel primero libro de la rectorica \\\hline
3.2.23 & respiciens diuturnitatem temporis , \textbf{ non est idem cum quinto , } quod respicit multitudinem operum . & catando alongamiento de tp̃o non es vna \textbf{ nin es essa misma | que la quinta } que cata ala muchedunbre delas obras \\\hline
3.2.23 & qui toto tempore se bene habuit praecedenti , \textbf{ est cum ipso misericorditer agendum , } et magis respiciendum est ad multum et ad totum tempus praecedens , & elt pon passado \textbf{ deue el iuez con el passar mibicordiosamente . } Et en tal cosa commo esta mas deue omne tener mientes alo much \\\hline
3.2.23 & et non murmurat in poena sibi imposita , \textbf{ est cum illo magis misericorditer agendum . } Ideo dicitur 1 Rhet’ & qual es puesta es de passar \textbf{ contra el mas misericordi osamente que con otro . } Et por ende dize el pho \\\hline
3.2.23 & et ponit se totaliter in arbitrio iudicantis , \textbf{ est cum eo mitius agendum . } Ideo dicitur 1 Rhet’ & e se homilla e se pone todo en el aluedrio del iuez \textbf{ deuen passar contra el mas manssamente . } Et por ende dize el philosofo \\\hline
3.2.23 & omnino enim contra rationem est humilitati non parcere , \textbf{ cum bestiae hoc agant : } canes quidem non offendunt humiliantes se , & ca las bestias perdonan \textbf{ a aquellos que se les homillan | e esto prouamos } por que los canes non fazen mal a aquellos que se les homillan \\\hline
3.2.23 & Qualiter autem clementia possit \textbf{ stare cum iustitia , } infra patebit . & deuen se inclinar a mi bicordia \textbf{ Mas en qual manera con la piadat pueda estar la iustiçia adelante parezcra } ommo las leyes sean vnas reglas de derech . \\\hline
3.2.24 & infra patebit . \textbf{ Cum leges sunt quaedam regulae iuris , } per quas in agibilibus regulamur , & Mas en qual manera con la piadat pueda estar la iustiçia adelante parezcra \textbf{ ommo las leyes sean vnas reglas de derech . } por las quales nos somos reglados en las nuestras obras \\\hline
3.2.24 & vel ex institutione iusta iudicantur . \textbf{ Itaque cum naturae rerum } sint eaedem ubique , & por establesçimiento dellos son iudgados derechs . \textbf{ Et pues que assi es commo las naturas delas cosas sean vnas en todos logares } e a todos los omes \\\hline
3.2.25 & et secundum rationem communem \textbf{ acceptus conuenit cum illis . } Si igitur ea sunt & que es ainalia \textbf{ e ha conueiençia conlas o trisaian lias non se deꝑte dellas . } Et pues que assi es \\\hline
3.2.25 & Si vero inclinatio illa sequatur naturam nostram , \textbf{ ut conuenimus cum animalibus aliis : } sic dicitur esse ius naturale . & Mas si esta inclinaçion \textbf{ siguiere la nuestra natura en quanto auemos conueniençia con las otras aian lias } assi es dich derech natural . \\\hline
3.2.25 & sed ut animalis est , \textbf{ et ut conuenit cum animalibus aliis : } nam et animalia naturaliter inclinantur , & mas en quanto es natura de aianl \textbf{ e en quanto conuiene con otras aian lias . } Ca las otras aiquelias natraalmente son inclinadas \\\hline
3.2.25 & et quod sequitur inclinationem nostram naturalem \textbf{ ut communicamus cum animalibus aliis , } respectu iuris gentium dicitur esse naturale . & e que sigue la nuestra inclinaçion natural \textbf{ en quanto participamos con las otras aianlas } en conparacion del derecho delas gentes es dicho derecho natural . \\\hline
3.2.25 & in quantum animal est , \textbf{ conuenit cum naturis aliorum animalium , } sic in quantum viuit & en quanto el omne es aianl \textbf{ conuiene con las naturales delas otras aianlias . } assi en quanto omne biue es alguna cosa conuiene con las plantas \\\hline
3.2.25 & et est quoddam ens , \textbf{ conuenit cum plantis , } et cum substantiis aliis , & conuiene con las naturales delas otras aianlias . \textbf{ assi en quanto omne biue es alguna cosa conuiene con las plantas } e con las otras sustançias \\\hline
3.2.25 & conuenit cum plantis , \textbf{ et cum substantiis aliis , } et cum entibus omnibus . & assi en quanto omne biue es alguna cosa conuiene con las plantas \textbf{ e con las otras sustançias } e con todas las cosas que han ser . \\\hline
3.2.25 & et cum substantiis aliis , \textbf{ et cum entibus omnibus . } Poterit ergo inclinatio naturalis & e con las otras sustançias \textbf{ e con todas las cosas que han ser . } Et por ende la inclinacion natural \\\hline
3.2.25 & vel ut homo est , \textbf{ vel ut conuenit cum animalibus aliis , } vel ut conuenit cum omnibus entibus . & o en quanto es omne o en quanto conuiene con todas las \textbf{ otrasaian lias } e en quanto conuiene con todas las cosas que son e han ser . \\\hline
3.2.25 & vel ut conuenit cum animalibus aliis , \textbf{ vel ut conuenit cum omnibus entibus . } Nam homo naturaliter appetit conseruari in esse , & otrasaian lias \textbf{ e en quanto conuiene con todas las cosas que son e han ser . | Ca el omne } naturalnse te dessea ser guardado en su ser Ra qual cola avn del sean todas las otras cosas que son . \\\hline
3.2.25 & prout natura humana est quaedam entitas , \textbf{ et conuenit cum entibus omnibus . } Si vero regulae illae sumantur & en quanto la natura humanal es alguna sub̃a e haser \textbf{ e conuiene con todas las cosas | que son } e con todas las sustançias . \\\hline
3.2.25 & prout non solum communicamus \textbf{ cum animalibus aliis , } sed ut conuenimus cum entibus omnibus : & que sigue la inclinaçion del anr̃a natura \textbf{ en quanto non solamente auemos conueniençia con las otras aian las . } mas en quanto auemos conueniençia con todas las sustançias . \\\hline
3.2.25 & cum animalibus aliis , \textbf{ sed ut conuenimus cum entibus omnibus : } nam huiusmodi ius est notius & en quanto non solamente auemos conueniençia con las otras aian las . \textbf{ mas en quanto auemos conueniençia con todas las sustançias . } Ca este derecho tales mas conosçido \\\hline
3.2.25 & sumitur ius gentium . \textbf{ Si ut conuenit cum animalibus aliis , } sic habet esse ius illud , & assi se toma el derecho delas getes . \textbf{ Mas en quanto conuiene el omne con todas las otras } ai alias \\\hline
3.2.25 & Sed si ut conuenit \textbf{ cum omnibus entibus , } sic habet esse ius illud , & Mas en quanto el omne ha conueniençia \textbf{ con todas las sustançias } assi se toma aquel derech \\\hline
3.2.25 & prout natura nostra conuenit \textbf{ cum omnibus entibus , } sic est de iure naturali : & e el mal es aquello que desseamos naturalmente \textbf{ en quanto la nuestra nata conuiene con todas las substançias . } Et assi avn es el derech natural \\\hline
3.2.26 & et finis intentus : \textbf{ quare cum bonum commune sit } diuinius & e la fin que entendemos . \textbf{ Por la qual cosa commo el bien comun sea mas diuinal que el bien propreo } e aya mas razon de bien e de fin \\\hline
3.2.27 & secundum quas intendimus in bonum illud : \textbf{ quare cum bonum commune principaliter intendatur a tota communitate } ut a toto populo , vel a principante , & por las quales leyes ymosa aquel bien . \textbf{ Por la qual cosa commo el bien comun sea entendido | prinçipalmente de toda la comunidat } assi commo de todo el pueblo o del prinçipe \\\hline
3.2.27 & si totus populus principetur ; \textbf{ Princeps enim aut totus populus cum principatur , } habet dirigere et ordinare alios in commune bonum . & enssennoreare . \textbf{ Ca el prinçipe o avn todo el pueblo } quando enssennorea ha de ordenar \\\hline
3.2.27 & nisi sit promulgata . \textbf{ Nam cum lex sit } quoddam mandatum superioris , & ningnon si non fuere obligada e prigo nada . \textbf{ Ca commo la ley sea vn mandamiento del sennor mayor . } por el qual somos e reglados e ligados en las nr̃as obras \\\hline
3.2.27 & oportet eam promulgatam esse . \textbf{ Sed cum alia sit lex naturalis , } alia positiua : & conuiene que sea publicada e pregonada . \textbf{ Mas commo otra sea la ley natural e otra la positiua en vna manera se deue publicar la vna } e en otra manera la otra . \\\hline
3.2.27 & et prolata in scripto redigitur . \textbf{ Quare cum in legibus , } si sint rectae et iustae & e es puesta en esc̀pto . \textbf{ Por la qual cosa conmo en las leyes } si fueren iustas e derechas \\\hline
3.2.28 & duplex cura esse potest : \textbf{ una cum opera sunt futura , } priusquam sint effectui mancipata : & ¶ \textbf{ El vno es delas obras que son de fazer ante que sean fechͣs . } El otro es delas obras \\\hline
3.2.28 & alia vero opera \textbf{ cum iam sunt in esse producta , } utroque modo debet & El otro es delas obras \textbf{ que son ya fechas . } Et en estas dos maneras deuen auer cuydado \\\hline
3.2.29 & quam legem . \textbf{ Nam Rex cum sit homo } non dicit intellectum bonum tantum , & corconper el rey que la ley . \textbf{ Ca el Rey por que es omne } non dize entendemiento tan solamente mas dize entendimiento con cobdiçia . \\\hline
3.2.29 & non dicit intellectum bonum tantum , \textbf{ sed dicit intellectum cum concupiscentia : } dato ergo quod Rex non peruertatur & Ca el Rey por que es omne \textbf{ non dize entendemiento tan solamente mas dize entendimiento con cobdiçia . } Et pues que assi es puesto \\\hline
3.2.29 & tamen quantum ad esse optimum : \textbf{ quia cum optimus homo incipit furire } et concupiscere peruersa , & enpero matase quanto al ser muy bueno . \textbf{ por que quando el muy bueno en se en comiença de enssennar e de cobdiçiar las cosas malas } si se non mata \\\hline
3.2.29 & quia non est ulterius optimus . \textbf{ Rex itaque dicit intellectum cum concupiscentia , } sed lex & Ca de ally adelante non es muy bueno . \textbf{ Et pues que assi es el Rey | dize entendimiento con cobdiçia . } mas la ley por que es alguna cosa \\\hline
3.2.29 & virtus iuris naturalis . \textbf{ Cum ergo quaeritur } utrum melius sit & mas en quanto enella es guardada la uirtud del derecho e dela ley natural . \textbf{ Et pues que assi es quando es demandado enla quastiuo } si es meior \\\hline
3.2.29 & Nam tunc principatur bestia , \textbf{ cum quis non innititur } regere alios ratione & que la bestia en ssennorea \textbf{ quando alguno non se esfuerça de gouernar los otros } por razon e por entendimiento mas por passion \\\hline
3.2.29 & sed passione et concupiscentia , \textbf{ in quibus communicamus cum bestiis . } Tunc vero principatur Deus , & e por çobdiçia \textbf{ en las quales cosas partiçipamos con las bestias . } Mas estonçe \\\hline
3.2.29 & quomodo seueritas et clementia possunt \textbf{ simul stare cum iustitia . } Nam humani actus & e la clemençia o la piedat \textbf{ pueden estar en vno con la iustiçia . } Ca las obras hmanales \\\hline
3.2.29 & Oportet igitur aliquando legem plicare ad partem unam , \textbf{ et agere mitius cum delinquente , } quam lex dictat : & Et por ende conuiene quela ley que se ençorue \textbf{ e se allegue algunas vezes ala vna parte e que obre mas manssamente con el que peca } quela ley demanda o que la ley nidga . \\\hline
3.2.29 & et tunc iuste \textbf{ et secundum rationem clementer agitur cum delinquente . } Aliquando tales circumstantiae aggrauant : & e estonçe derechͣmente \textbf{ e segunt razon obra el prinçipe pudosamente con el que peca . } Mas algunas uegadas tales cercunstançias agcauian el pecado \\\hline
3.2.29 & Et quia haec omnia iuste \textbf{ et rationabiliter fieri possunt , clementia et seueritas simul cum iustitia possunt existere . } Fuerunt enim aliqui de suo ingenio praesumentes , & Et por que todas estas cosas se pueden fazer derechamente \textbf{ e con razon la reziedunbre dela iustiçia | e la piedat pueden estar en vno con la iustiçia } e mandan algunos presunptuosos presumiendo de su engennio . \\\hline
3.2.30 & quam virtuosiores potest . \textbf{ Sed cum ad perfectam virtutem nullus inducatur } nisi intendat omnia peccata vitare : & si non entendiere fazer a todos los sus çib \textbf{ dadanos los mas uirtuosos que pudiere ¶as commo ninguno non pueda venir a acatadas uirtudes sim̃o entendiere escusar todos los pecados } si fuer penssada la ley \\\hline
3.2.30 & per Philosophum 2 Politicorum \textbf{ cum disputat contra Socratem , } secundum hunc modum magis prohibetur concupiscentia & en el segundo libro delas politicas \textbf{ do disputa contra socrates . } Et segunt esta manera mas es defendida la cobdiçia del coraçon \\\hline
3.2.30 & et ad vitium si sint mali , \textbf{ cum procedant ex interiori appetitu . } Sed si consideretur lex humana & e a pecado si son malas . \textbf{ quando uieñe del apetito | et ple desseo del coraçon . } Mas si fuere penssa para la ley \\\hline
3.2.30 & quod secundum aliorum iudicium est iniustum . \textbf{ Quare cum in humanis iudiciis cadere possit dubieras et error , } expediens fuit lex euangelica & la qual segunt el iuyzio de los otros non es derechͣ . \textbf{ Por la qual cosa commo en los iuyzios humanales pueda caer dubda e yerro . | cosa muy aprouechable } e muy neçessaria fue la ley e una gelical e diuinal \\\hline
3.2.30 & quod intendimus adipisci . \textbf{ Nam cum huiusmodi bonum sit } supra facultates nostrae naturae , & la terçera razon se toma de parte del grant bien fin al que nos entendemos de aleançar . \textbf{ Ca commo este tal bien sea sobre el poderio dela nuestra natura . } la ley natural e la \\\hline
3.2.31 & Quaerit Philos’ 2 Polit’ \textbf{ cum disputat contra Hippodamum , } utrum sit expediens ciuitatibus & e manda el pho en el segundo libro delas politicas \textbf{ quando disputa contra ypodomio } si es cosa conuenible alas çibdades \\\hline
3.2.32 & scilicet de populo . \textbf{ Sed cum ad sciendum qualis debet esse populus , } et quomodo debeat & Conuiene de saber del pueblo . \textbf{ Mas commo para saber qual deua ser el puebło } e commo se deua auer al prinçipe \\\hline
3.2.32 & Dicebatur enim supra \textbf{ cum de legibus tractabamus , } quod facere commutationes , & Ca dixiemos dessuso \textbf{ quando fablauamos delas leyes } que fazer mudaçiones \\\hline
3.2.34 & ( quantum ad praesens spectat ) tria , \textbf{ si cum magna diligentia obediat regibus , et principibus , } et obseruet leges regias . & uanto alo presente parte nesçe el pueblo alcança tres bienes \textbf{ si obedesçiere alos Reyes | e alos prinçipes con grand acuçia . } Et si guardare las leyes de los Reyes \\\hline
3.2.34 & inducere alios ad virtutem , \textbf{ cum virtus faciat habentem bonum ; } et opus bonum , & por que la su entençion es enduzir los otros a uirtud . \textbf{ Et la uirtud faze al que la ha buenon } Esta buean obra conuiene que sea en el gouernamiento derech \\\hline
3.2.34 & eo quod virtutes sunt maxima bona , \textbf{ cum summa diligentia studere debet , } ut Regi obediatur , & por que las uirtudes son muy grandes bienes . \textbf{ Et avn con grant diligençia deue estudiar cada vn çibdadano } por que obedezca al Rey \\\hline
3.2.34 & esse seruitutem . \textbf{ Cum enim bestiae sint naturae seruilis : } quanto quis magis accedit & e obedesçer alos Reyes es seruidunbre . \textbf{ Ca commo las bestias sean de natura seruil } quanto alguno mas se allega ala natura bestial \\\hline
3.2.34 & et obseruationem legum : \textbf{ cum ex hoc consurgat tantum bonum , } quantum bona est pax & e guardar las leyes . \textbf{ Commo desto se leunate tan grant bien } quanto es la paz e el assessiego de los que son en el regno \\\hline
3.2.35 & vel ea quae aliquo modo ordinantur ad ipsum . \textbf{ Cum ergo quis aliquo dictorum modorum manifeste forefacit in alium , } ille tristatus ex appetitu punitionis , & que en alguna manera son ordenadas a el \textbf{ ¶ Et pues que assi es quando alguno en alguna destas dichas maneras | faze tuerto a otro manifiesta mente . } aquel a quien es fech tu \\\hline
3.2.35 & obseruando eius leges et mandata . \textbf{ Cum enim haec duo , } honor videlicet , & guardando sus leyes e sus mandamientos . \textbf{ Ca quando estas dos cosas . } Conuiene a saber la honrra \\\hline
3.2.36 & quod fieri contingit , \textbf{ cum ad eorum iudices } et praepositos latenter et caute se gerunt & la qual cosa puede contesçer \textbf{ quando assi se han los sus uiezes } e los sus mjnos ascondidamente e cautelosamente en dar las penas \\\hline
3.2.36 & debet esse magis intentum a legislatore . \textbf{ Cum ergo ciues et existentes in regno } si bene agant , & prinçipalmente quarido e entendido del ponedor dela ley . \textbf{ Et pues que assi es quando los çibdadanos | e los que son en el regno } si fazen bien \\\hline
3.3.1 & et alia species prudentiae : \textbf{ quare cum bonum domesticum , } et bonum totius familiae & alli son falladas departidas maneras de sabiduria . \textbf{ Por la qual cosa commo el bien de la casa } e el bien de toda la conpaña \\\hline
3.3.1 & ut ciuitatis , aut regni . \textbf{ Quare cum commune bonum directe videatur } impediri per impugnationem hostium , & e de la çibdat a e del regno . \textbf{ Por la qual cosa commo el bie comun | derechamente parezca de ser enbargado } por la guerra de los enemigos \\\hline
3.3.2 & quia naturaliter metuunt vulnera . \textbf{ Nam cum naturaliter habeant modicum sanguinis , } naturaliter timent sanguinis amissionem : & naturalmente han miedo de las feridas . \textbf{ Ca por que naturalmente han poca sangre } naturalmente temen de perder la sangre . \\\hline
3.3.2 & et ex quibus artibus sunt assumendi bellantes . \textbf{ Sciendum ergo quod cum bellantes debeant } habere membra apta & e de quales artes son de tomar los lidiadores . \textbf{ Et pues que assi es conuiene de saber } que commo los lidiadores deuan auer los mienbros apareiados \\\hline
3.3.2 & quia non horrent sanguinis effusionem , \textbf{ cum assueti sint ad occisionem animalium , } et ad effundendum sanguinem Venatores & por que non aborresçen el derramamiento de la sangre \textbf{ por que son acostunbrados a matar las animalias } e a esparzer la sangre ellas . \\\hline
3.3.2 & Imo forte non minus periculosum est \textbf{ bellare cum apro , } quam pugnare cum hoste . & e estremados para la batalla ante \textbf{ por auentura non es menor peligro lidiar con el puerco montes que lidiar con el enemigo . } Por ende los que non temen los periglos de los puercos monteses \\\hline
3.3.3 & ut ad labores militares et bellicosos . \textbf{ Quod concordat cum Vegetio de re militari dicente , } quod a tempore pubertatis & e de las batallas \textbf{ el qual dicho concuerda con vegeçio del Fecho de la caualleria } e dize que del tienpo de la mançebia se deuen los mançebos a los trabaios de la caualleria \\\hline
3.3.4 & quasi non appretiari corporalem vitam . \textbf{ Nam cum tota operatio bellica exposita sit periculis mortis , } nunquam quis est fortis animo & por la iustiçia e por el bien comun . \textbf{ Ca commo toda la hueste sea puesta | a periglos de muerte en la batalla } nunca ninguno es fuerte de coraçon \\\hline
3.3.4 & exponit se mortis periculis . \textbf{ Cum autem quis , } ultra quam debeat , & o por otro alguno muy grant bien derechamente \textbf{ e a osadas se pone a los periglos de la muerte . } Mas quando alguno mas que deue ama la vida corporal \\\hline
3.3.4 & Finis militaris , est victoria . \textbf{ Sed cum omnis bellica operatio contineatur sub militari , } ut supra ostendebatur , & la fin de la caualleria es victoria e vençer . \textbf{ Mas conmo todas las obras de la batalla sean contenidas so la caualleria } assi commo es mostrado desuso \\\hline
3.3.4 & victoria esse finis . \textbf{ Quare cum maxime contingat bellantes vincere , } si bene sciant se protegere & la uictoria es dicha fin de todas las obras de la batalla . \textbf{ Por la qual cosa commo mayormente contezca a los lidiadores vençer } si bien se sopieren cobrir \\\hline
3.3.4 & et erubescere turpem fugam . \textbf{ Aduertendum autem quod cum dicimus , } bellatores non habere effusionem sanguinis , & torpemente de la batalla . \textbf{ Mas deuemos parar mientes | que quando dezimos } que los lidiadores non deuen aborresçer el esparzimiento de la sangre \\\hline
3.3.5 & Si in bello terga vertam , \textbf{ Hector cum concionabitur inter Troianos , } dicet , & Ca dizie que si boluiesse las espaldas en la batalla \textbf{ que ector | quando razonasse en las cortes de troya } dirie que diomedes era su vençido \\\hline
3.3.5 & A me deuictus est Diomedes . \textbf{ Quare cum velle honorari } et erubescere de aliquo turpi facto , & dirie que diomedes era su vençido \textbf{ Por la qual cosa commo querer auer honrra de la batalla } e tomar uergueña de torpe fecho \\\hline
3.3.5 & quam corporis fortitudo . \textbf{ Quare cum communiter nobiles homines industriores sint rusticis , } sequitur hos meliores esse pugnantes . & que la fortaleza del pueblo . \textbf{ Por la qual cosa commo los nobles omnes sean mas sotiles comunalmente } e mas arteros que los aldeanos \\\hline
3.3.6 & impedietur ad percutiendum . \textbf{ Nam cum bellator a suo consocio nimis comprimitur , } sua impediuntur brachia , & e muy espessa enbargan se los vnos a los otros para ferir . \textbf{ Ca quando el lidiador esta muy apretado de su conpañon } enbargansele los braços para ferir \\\hline
3.3.6 & ac si deberent pugnam committere . \textbf{ Et cum viderit magister bellorum } aliquem non tenere ordinem debitum in acie , & assi commo si ouiessen de acometer la batalla . \textbf{ Et quando vieren los caudiellos maestros de las batallas } que alguno non guarda orden en la az \\\hline
3.3.7 & Secundo ad inuadendum \textbf{ et percutiendum cum claua . } Tertio ad emittendum tela siue iacula , & Lo primero se deuen vsar aleuar grandes pesos . \textbf{ Lo segundo a acometer e a ferir con maças . } Lo terçero son de vsar a lançar dardos \\\hline
3.3.7 & Tertio ad emittendum tela siue iacula , \textbf{ et ad percutiendum cum lancea . } Quarto ad iaciendum sagittas . & Lo terçero son de vsar a lançar dardos \textbf{ e a ferir con llan ças . } lo quarto a a lançar saetas . \\\hline
3.3.7 & Quarto ad iaciendum sagittas . \textbf{ Quinto ad proiiciendum lapides cum fundis . } Sexto ad percutiendum cum plumbatis . & lo quarto a a lançar saetas . \textbf{ lo quinto a a lançar piedras con fondas . } lo sexto a ferir con pellas de plomo o de fierro \\\hline
3.3.7 & Quinto ad proiiciendum lapides cum fundis . \textbf{ Sexto ad percutiendum cum plumbatis . } Septimo ad ascendendum equos . & lo quinto a a lançar piedras con fondas . \textbf{ lo sexto a ferir con pellas de plomo o de fierro } Lo vij° . \\\hline
3.3.7 & quomodo exercitandi sunt bellatores \textbf{ ad percutiendum cum gladiis et ensibus . } Sed de hoc speciale capitulum faciemus . & en qual manera se auian de vsar los lidiadores \textbf{ a ferir con espadas e con cuchiellos | por el arte del esgrima } mas desto faremos espeçial capitulo . \\\hline
3.3.7 & quasi natura quaedam . \textbf{ Cum ergo quis assuetus } ad portandum maius pondus , & assi conmo vna naturaleza \textbf{ Et por ende quando alguno es acostunbrado a leuar mayor carga } que la carga de las armas semeial \\\hline
3.3.7 & Secundo exercitandi sunt bellantes \textbf{ adinuadendum et percutiendum cum claua . } Recitat enim Vegetius , & Lo segundo se deuen usar los lidiadores \textbf{ a acometer | e a ferir con porras e con maças . } Ca cuenta vegeçio que antiguamente los romanos fincauan muchos palos en los canpos . \\\hline
3.3.7 & ac si contra hostem dimicaret : \textbf{ et cum diu mane et sero iuuenes sic exercitati essent , } cum postea veniebant ad bellum , & commo si lidiassen con sus enemigos . \textbf{ Et quando se assi usauan los mancebos prolongadamente en la mañana | e en la tarde } quando despues venien a la batalla non resçibien trabaio en ferir con la maca \\\hline
3.3.7 & et cum diu mane et sero iuuenes sic exercitati essent , \textbf{ cum postea veniebant ad bellum , } non grauabantur in percutiendo cum claua , & e en la tarde \textbf{ quando despues venien a la batalla non resçibien trabaio en ferir con la maca } nin en sofrir quales quier otros trabaios de la batalla . \\\hline
3.3.7 & cum postea veniebant ad bellum , \textbf{ non grauabantur in percutiendo cum claua , } vel in sustinendo quoscunque labores bellicos . & e en la tarde \textbf{ quando despues venien a la batalla non resçibien trabaio en ferir con la maca } nin en sofrir quales quier otros trabaios de la batalla . \\\hline
3.3.7 & admittendum tela et iacula , \textbf{ et ad percutiendum cum lancea : } quod etiam ad defixum palum fieri habet . & a alcançar dardos e azconetas \textbf{ e a ferir con lanças } la qual cosa avn fazien les mançebos al palo fincado . \\\hline
3.3.7 & Fiebat enim antiquitus \textbf{ ut cum iuuenes exercitati erant } ad percutiendum palos infixos cum claua , & ca assi fazien antiguamente \textbf{ ca quando los mançebos eran usados a ferir en los palos fincados con las maças } usauanse a ferir con las azconetas e con las lanças \\\hline
3.3.7 & ut cum iuuenes exercitati erant \textbf{ ad percutiendum palos infixos cum claua , } quod exercitabantur ad percutiendum & ca assi fazien antiguamente \textbf{ ca quando los mançebos eran usados a ferir en los palos fincados con las maças } usauanse a ferir con las azconetas e con las lanças \\\hline
3.3.7 & quod exercitabantur ad percutiendum \textbf{ cum telo vel cum iaculo , } siue cum lancea . & ca quando los mançebos eran usados a ferir en los palos fincados con las maças \textbf{ usauanse a ferir con las azconetas e con las lanças } e estauan alongados \\\hline
3.3.7 & cum telo vel cum iaculo , \textbf{ siue cum lancea . } Stabant enim a remotis & usauanse a ferir con las azconetas e con las lanças \textbf{ e estauan alongados } e usauan las braços \\\hline
3.3.7 & ad iaciendum sagittas , \textbf{ vel cum arcubus , vel cum ballistis . } Nam quia contingit & Lo quarto son de vsarlos lidiadores \textbf{ a alançar saetas con arcos e con ballestas . } ca quando contesçe que non podemos \\\hline
3.3.7 & immo dato quod pugnantes \textbf{ se cum hostibus possint coniungere , } antequam coniungantur & prouechosa cosa es lançar las saetas \textbf{ mas puesto que los lidiadores se puedan ayuntar con los enemigos } ante que se apunte con ellos \\\hline
3.3.7 & Legitur enim de Africano Scipione , \textbf{ qui cum pro populo Romano certare deberet , } non aliter contra hostes se obtinere credebat , & que quando auie de lidiar \textbf{ por el pueblo de roma non cuydaua vençer en otra manera a los enemigos } si non poniendo arqueros \\\hline
3.3.7 & sunt bellatores exercitandi \textbf{ ad iaciendum lapides cum fundis . } Hic enim modus bellandi & e ballesteros mucho escogidos en todas las azes . \textbf{ Lo quinto son los lidiadores de usar a a lançar piedras con fondas . } Ca esta manera de lidiar fue fallada \\\hline
3.3.7 & ut matres nullum cibum eis exhiberent , \textbf{ quem non primo cum funda percuterent . } Est enim hoc exercitium utile , & que las madres nunca les querien dar de comer \textbf{ fasta que ferien con la fonda en logar çierto . } Et este uso es muy prouechoso \\\hline
3.3.7 & non inutile est lapides \textbf{ cum fundis eiicere . } Sexto , exercitandi sunt bellantes & ca non es trabaio ninguno leuar fondas . \textbf{ Ca algunas uezes es lançar piedras con fondas . } Lo . vi° son de usar los lidiadores a ferir con pellas de fierro o de plomo . \\\hline
3.3.7 & Sexto , exercitandi sunt bellantes \textbf{ ad percutiendum cum plumbatis . } Nam pila plumbea vel ferrea & Ca algunas uezes es lançar piedras con fondas . \textbf{ Lo . vi° son de usar los lidiadores a ferir con pellas de fierro o de plomo . } Ca las pellas de plomo o de fierro \\\hline
3.3.7 & Nam pila plumbea vel ferrea \textbf{ cum cathena aliqua coniuncta } manubrio ligneo vehementem ictum reddit . & Ca las pellas de plomo o de fierro \textbf{ atadas con alguna cadena a mango de madero da muy fuentes colpes . } Ca por el mouimiento muy grande del ayre la pella \\\hline
3.3.7 & vehementius percutit pila \textbf{ cum cathena hastae infixa , } quam si ipsi hastae , & Ca por el mouimiento muy grande del ayre la pella \textbf{ con la cadena fincada al asta fiere muy mas fuertemente } que si estudiesse ayuntada con el mango o con aste . \\\hline
3.3.7 & est proprium equitibus : \textbf{ proiicere lapides cum funda , } videtur esse proprium peditibus . & Ca sobir en los cauallos pertenesçe a los caualleros . \textbf{ Et lançar piedas con fondas pertenesçe a los peones . } Mas las otras cosas en alguna manera puenden pertenesçer a todos . \\\hline
3.3.8 & Debet enim exercitus secum ferre munitiones congruas , \textbf{ ut cum castrametari voluerit , } quasi quandam munitam ciuitatem secum portasse videatur . & Ca deue sienpre la hueste leuar consigo guarniciones conuenibles . \textbf{ por que quando quisiere la hueste folgar en algun logar parezca } que lieuan consigo assi commo vna çibdat guarnida . \\\hline
3.3.9 & ut quilibet virilius et expeditius \textbf{ et cum minori labore } et poena faciat opera consueta . & e mas prouadamente \textbf{ e sin mayor trabaio } e pena faze las obras que ha acostunbradas . \\\hline
3.3.10 & Nam vexillo confracto totus exercitus est confusus . \textbf{ Cum magna igitur diligentia est vexillifer eligendus , } ut sit corpore fortis , & ca la señal ronpida o tomada toda la hueste es confondida e desordenada . \textbf{ Et pues que assi es con grant sabiduria es de escoger el alferez } assi que sea fuerte de cuerpo \\\hline
3.3.10 & et ignorantes ad quid deberent attendere : propter quod si in debellatione vita multorum hominum periculis mortis exponitur , \textbf{ cum magna diligentia vexillifer est quaerendus . } Ex dictis etiam patere potest , & es puesta a periglos de muerte \textbf{ con grant acuçia | e con grant diligençia } deue ser escogido el alferes . \\\hline
3.3.10 & in iis quae requiruntur ad pugnam . \textbf{ Quare cum pedites , } si debent boni bellatores existere & que son menester para la batalla \textbf{ por la qual cosa commo los peones } si quisieren ser buenos lidiadores deuan ser fuertes e rezios en los sus cuerpos \\\hline
3.3.10 & portare scutum ad se protegendum : \textbf{ et cum debeant esse vigilantes , agiles , sobrii , } habentes armorum experientiam : & para encobrirse meior \textbf{ e avn que ayan los oios bien espiertos | e que sean ligeros e mesurados en beuer e gerrdados de vino } e avn que ayan vso de las armas . \\\hline
3.3.10 & scire fortiter equitare , \textbf{ cum lancea percutere , } eiicere iacula , & e sepa bien caualgar e ferir \textbf{ fuertemiente con la lança } e lançar lança e dardo \\\hline
3.3.10 & eiicere iacula , \textbf{ cum scuto se protegere , } cum claua et ense dimicare , & e lançar lança e dardo \textbf{ e sepa cobrirse del escudo } et sepa ferir con la maca \\\hline
3.3.10 & cum scuto se protegere , \textbf{ cum claua et ense dimicare , } habere omnium armorum exercitium , & e sepa cobrirse del escudo \textbf{ et sepa ferir con la maca | e lidiar bien con el espada } e que aya uso en todas las armas \\\hline
3.3.11 & Secunda cautela est , \textbf{ ut simul cum hoc quod habet vias } et qualitates viarum conscriptas et depictas , & La segunda cautela es que el señor de la batalla con esto \textbf{ que dicho es | que deue auer las carreras } e las qualidades de los caminos escriptas e pintadas . \\\hline
3.3.11 & quod et hoc posset , ad aures hostium peruenire \textbf{ Itaque cum pericula visa minus noceant , } per velocissimos equites sunt detegendae insidiae , & que avn este podria venir en las oreias de los enemigos . \textbf{ Et por ende por que los periglos que son ante vistos menos enpeesçen . } por caualleros muy ligeros son de descobrir las çeladas \\\hline
3.3.12 & est utilis ad circum dandum et concludendum , \textbf{ cum hostes sunt pauci . Sed in forma acuta et pyramidali , } utilis est ad scindendum et diuidendum , & quando son pocos . \textbf{ Mas la forma aguda en manera de pera ens prouechosa para fender } e departir los enemigos \\\hline
3.3.12 & utilis est ad scindendum et diuidendum , \textbf{ cum hostes sunt plures . } Sciendum est ergo , & Mas la forma aguda en manera de pera ens prouechosa para fender \textbf{ e departir los enemigos | quando son muchos . } Et pues que assi es las maneras de las azes son de establescer \\\hline
3.3.13 & et conuertuntur in fugam . \textbf{ Quare cum percutiendo caesim } propter magnum motum brachiorum insurgat & e tornarsse han a foyr . \textbf{ Por la qual cosa commo feriendo cortando } por el grand mouimiento de los braços \\\hline
3.3.13 & minor laesio ei potest accidere . \textbf{ Quare cum percutiendo punctim } etiam tecto corpore possit & por que assi feriendo menor daño le puede contesçer . \textbf{ Por la qual cosa commo feriendo de punta } avn que este el cuerpo cubierto \\\hline
3.3.14 & si sint in acie debite ordinati . \textbf{ Nam cum ipsa virtus unita , } ( ut etiam supra tangebatur ) & Lo primero es si fueren las azes ordenadas \textbf{ commo deuen . | Ca commo la uertud ayuntada } assi commo dicho es \\\hline
3.3.14 & suos hostes inuadere debeant . \textbf{ Nam cum septem modis enumeratis hostes fortiores existant ; } cum modo opposito se habent , & e en qual manera los lidiadores deuen acometer sus enemigos . \textbf{ Ca commo en las siete maneras contadas | sean los enemigos mas fuertes } quando son las maneras contrarias \\\hline
3.3.14 & Nam cum septem modis enumeratis hostes fortiores existant ; \textbf{ cum modo opposito se habent , } sunt inuadendi et debellandi . & sean los enemigos mas fuertes \textbf{ quando son las maneras contrarias } de aquellas siete son de acometer e de ferir . \\\hline
3.3.15 & qualiter debeant stare \textbf{ cum volunt hostes percutere . } Percussionis autem hostium duplex est modus . & en qual manera deuen estar \textbf{ quando quisieren ferir los enemigos . } Et ay dos maneras de ferir los enemigos . \\\hline
3.3.15 & Unus a remotis , \textbf{ ut cum iaciendo iacula , } vel missilia aduersarios feriunt . & La vna es de lueñe \textbf{ assi conmo quando fieren los enemigos } lançando piedras e dardos e saetas . \\\hline
3.3.15 & vel missilia aduersarios feriunt . \textbf{ Alius autem cum adeo appropinquant , } quod manu ad manum se percutiunt . & lançando piedras e dardos e saetas . \textbf{ Et otra manera ay | quando tanto se llegan los enemigos } quando vienen a las manos \\\hline
3.3.15 & Aliter autem debent stare bellatores viri , \textbf{ cum a remotis iacula iaciunt , } et aliter cum ex propinquo se feriunt . & e en otra manera deuen estar los lidiadores \textbf{ quando lançan los dardos de lueñe . } Et en otra manera quando se fieren de çerca . \\\hline
3.3.15 & cum a remotis iacula iaciunt , \textbf{ et aliter cum ex propinquo se feriunt . } Nam iaciendo iacula a remotis , & quando lançan los dardos de lueñe . \textbf{ Et en otra manera quando se fieren de çerca . } Ca lançando dardos de lueñe \\\hline
3.3.15 & oportet quod pars dextra innitatur parti sinistrae quiescenti . \textbf{ Cum igitur pes sinister anteponitur , } et latus dextrum elongatur , & que se afirme sobre la siniestra \textbf{ que se non mueue . | Et por ende quando el pie esquierdo se pone delante } e el costado derecho se aluenga estonçe esta el omne bien apareiado \\\hline
3.3.15 & super sinistrum pedem ante missum , \textbf{ et elongare nos cum dextro , } ut possimus vehementius impellere , & deuemos folgar sobre el pie esquierdo puesto delante \textbf{ e alongarnos con el derecho } por que podamos mas fuertemente esgremir e lançar el dardo . \\\hline
3.3.15 & Debent enim bellatores , \textbf{ cum ad manum pugnant , } tenere pedem sinistrum immobiliter : & Ca deuen los lidiadores \textbf{ quando vienen a las manos tener el pie esquierdo firme } e quando quieren ferir \\\hline
3.3.15 & tenere pedem sinistrum immobiliter : \textbf{ et cum volunt percutere , } cum pede dextro debent se antefacere , & quando vienen a las manos tener el pie esquierdo firme \textbf{ e quando quieren ferir } deuen se fazer adelante con el pie derecho \\\hline
3.3.15 & et cum volunt percutere , \textbf{ cum pede dextro debent se antefacere , } et cum volunt ictus fugere , & e quando quieren ferir \textbf{ deuen se fazer adelante con el pie derecho } e quando dieren los colpes \\\hline
3.3.15 & cum pede dextro debent se antefacere , \textbf{ et cum volunt ictus fugere , } cum eodem pede debent se retrahere ; & deuen se fazer adelante con el pie derecho \textbf{ e quando dieren los colpes } deuense arredrar con el pie derecho . \\\hline
3.3.15 & et cum volunt ictus fugere , \textbf{ cum eodem pede debent se retrahere ; } sic itaque tenendo pedem sinistrum immobilem , & e quando dieren los colpes \textbf{ deuense arredrar con el pie derecho . } Et pues que assi es temiendo \\\hline
3.3.15 & sic itaque tenendo pedem sinistrum immobilem , \textbf{ et cum dextro se mouendo , } poterunt fortius hostes percutere , & assi el pie esquierto firme \textbf{ e mouiendo se con el esquierdo pie } podrian mas fuertemente ferir los enemigos e mas ligeramente foyr los colpes dellos . \\\hline
3.3.15 & et in fuga periclitantur multi absque nocumento persequentium ; \textbf{ sed cum se vident inclusos , } quasi coacti feriunt includentes . & de aquellos que los persiguen . \textbf{ Mas quando se veen ençerrados han se de tornar } assi commo costrennidos de la muerte a ferir \\\hline
3.3.15 & quasi coacti feriunt includentes . \textbf{ Cum ergo supra diximus , } formandam aliquando esse aciem sub forma forficulari , & a aquellos que los ençierran . \textbf{ Et pues que assi es quando dixiemos dessuso que } alguans vezes el az es de formar so forma de tiieras e esto quando los enemigos son pocos \\\hline
3.3.16 & Nam etsi in omni pugna est aliquo modo inuasio et defensio : \textbf{ attamen cum quis obsedit munitiones , et castra , } magis dicitur alios inuadere & Ca si en toda batalla es algun acometemiento en alguna manera . \textbf{ Enpero quando alguno o algunos çercan villas o castiellos o fortallezas } mas dezimos que aquellos acometen que se defienden . \\\hline
3.3.16 & nauales dicuntur . \textbf{ Quare cum sint quatuor genera pugnarum , } postquam diximus de campestri , & son dichas batallas nauales e de naues . \textbf{ Por la qual cosa commo sean quatro maneras da batallas } despues que dixiemos de la batalla de la tierra fincanos de dezir de las otras tres . \\\hline
3.3.16 & et de cautelis bellorum multa discutimus ; \textbf{ cum per iam dicta } circa omne bellum possint & muchas cosas derminamos . \textbf{ Commo por las cosas ya dichas puedan ser } avn dos cautelas en toda lid \\\hline
3.3.16 & de pugna obsessiua : \textbf{ cum per huiusmodi pugnam } contingat obtineri et deuinci munitiones et urbanitates : & que se faz por çercar \textbf{ por que por tal lid } contesçe tomar e vençer las villas \\\hline
3.3.16 & ad munitiones obsessas , \textbf{ ut ibi una cum aliis comedentes } apud ipsos obsessos & e despues enbianlos a las fortalezas cercadas \textbf{ por que y con los otros comedores cercados } fagan mayor fanbre \\\hline
3.3.16 & Tertius modus obtinendi munitiones est per pugnam : \textbf{ ut cum itur ad muros , } et cum per pugnam dimicatur & La terçera manera de ganar las fortalezas es por batalla \textbf{ assi commo quando van a los nivros } e los acometen \\\hline
3.3.16 & ut cum itur ad muros , \textbf{ et cum per pugnam dimicatur } contra obsessos . & assi commo quando van a los nivros \textbf{ e los acometen } por batalla a los que son cercados . \\\hline
3.3.17 & ab obsessis molestari poterunt . \textbf{ Nam cum contingat obsessiones } per multa aliquando durare tempora , & de los que estan çercados \textbf{ Ca commo contezca } que las çercas puedan durar algunas vezes \\\hline
3.3.17 & contingit quod existentes in castris \textbf{ ( cum fuerint occupati obsidentes somno , } vel ludo , vel ocio , & o en las çibdades cercadas \textbf{ quando los que çercan durmieren o comieren o estudieren de vagar o fueren derramados } por alguna neçessidat a desora pueden dar en ellos \\\hline
3.3.17 & statim impugnant eos \textbf{ cum ballistis et arcubus : } iaciunt contra ipsos lapides & luego los acometen los de fuera con ballestas \textbf{ e con arcos } e lançan contra ellos piedras con las manos o con fondas \\\hline
3.3.17 & iaciunt contra ipsos lapides \textbf{ cum manibus vel cum fundis ; } apponunt scalas ad muros , & e con arcos \textbf{ e lançan contra ellos piedras con las manos o con fondas } e ponen escaleras a los muros \\\hline
3.3.17 & ne statim cadant . \textbf{ Et cum omnes muros , } vel maximam partem murorum sic suffosserunt et subpunctauerunt , & nin fazer daño a los que cauan \textbf{ e quando todos los muros o grant parte dellos } assi ouieren socauados \\\hline
3.3.17 & ne videatur ab obsessis . \textbf{ Et rursus cum ignis apponitur } ad ipsa ligna sustinentia murum , & es de asconder en tal manera por que non la vean los que estan cercados . \textbf{ Otrossi quando se pone el fuego a la madera } que sotiene los muros \\\hline
3.3.17 & apponens huiusmodi ignem et existentes \textbf{ cum eo debent } ad locum tutum fugere , & el que pone el fuego \textbf{ e los que estan con el } deuen se poner en saluo \\\hline
3.3.17 & in impugnatione per cuniculos , \textbf{ cum ad munitionem obtinendam } sufficit sola murorum ruina . & que es por cueuas conegeras \textbf{ e esto quanto para ganar la fortaleza } cunple la cayda de los muros . \\\hline
3.3.17 & sufficit sola murorum ruina . \textbf{ Sed cum hoc creditur non sufficere , } muris existentibus subfossis et subpunctatis , & cunple la cayda de los muros . \textbf{ Mas quando cuydan que non pueden entrar el logar } estando socauados los muros \\\hline
3.3.18 & ubi agetur de defensiua pugna . \textbf{ Cum enim tractabimus qualiter } obsessi se defendere debeant , declarabitur qualiter obsessi per cuniculos & de los que se defienden . \textbf{ Ca quando tractaremos en qual manera se deuen auer | los que cercan declararemos en qual manera } los cercados deuen ver \\\hline
3.3.18 & quae semper faciliori via res ad effectum producit : \textbf{ cum per viculos non ita de facili munitiones impugnari possunt , } sicut per machinas lapidarias , & que puede las cosas a su fin . \textbf{ Commo por cueuas conegeras non puedan tan de ligero tomar las fortalezas | commo se pueden tomar } por los engeñios \\\hline
3.3.18 & aliquando autem non sufficit contrapondus , \textbf{ sed ulterius cum funibus eleuatur virga machinae , } qua eleuata iaciuntur lapides . & e algunas vegadas non abasta el contrapeso . \textbf{ Mas avn leuantanle con cuerdas el partegal del engeñio } guindandol el qual leuantando \\\hline
3.3.18 & ideo semper eodem modo impellit , \textbf{ cum hac enim machina } quasi acus percuti posset . & e por ende sienpre en vna manera enbia la piedra . \textbf{ e con este engeñio pueden ferir tan açierto } commo si lançassen el aguia . \\\hline
3.3.18 & quasi acus percuti posset . \textbf{ Nam cum aliquod signum percutiendum est per ipsam , } si nimis proiicit ad dextram , & commo si lançassen el aguia . \textbf{ Ca quando quieren lançar con el engenio a alguna señal } si lançan mucho a la mano \\\hline
3.3.18 & Tamen , ut videatur qualiter in nocte percutiunt lapides emissi a machinis , \textbf{ semper cum lapide alligandus est ignis , } vel ticio ignitus . & Empero por que vean en qual manera han de lançar las piedras de noche \textbf{ que enbian con los engeñios | sienpre deuen atar algun fuego o algun tizon ençendido con la piedra } que enbian de noche . \\\hline
3.3.19 & et postea fortiter muros munitionis obsessae percutit et disrumpit . \textbf{ Cum enim per huiusmodi trabem sic ferratam } multis ictibus percussus est murus ita , & assi que los ronpen et los quebrantan . \textbf{ Ca quando con esta viga tal } assi ferrada den muchos colpes en el muro \\\hline
3.3.19 & Fit autem hoc , \textbf{ cum tabulae grossae } et fortes optime conligantur , & e fazese este artificio \textbf{ quando las tablas gruessas e fuertes son bien iuntadas e dobladas } o se fazen dos tablados \\\hline
3.3.19 & Est autem hoc aedificium utile , \textbf{ cum talis est munitio obsessa , } quod usque ad muros eius potest & Et es este artifizo muy prouechoso \textbf{ quando tal es la fortaleza } a cada que sasta los muros se pueden enpucar tal artifiçio \\\hline
3.3.19 & ne succendantur ab igne . \textbf{ Cum his quidem ligneis castris } dupliciter impugnantur munitiones obsessae . & por que non gelos quemen con fuego . \textbf{ Et con estos castiellos de madera } en dos maneras conbaten las fortalezas cercadas . \\\hline
3.3.19 & trahat se magis prope aedificium illud , \textbf{ si vero visus protendatur magis basse , cum tabula sit existente ad pedes , } et sic iacens in terra , & e si la vista del oio descendiere \textbf{ e fuere mas baxa aluengesse | mas con la tabla en tal manera } que catando \\\hline
3.3.19 & usque ad munitionem obsessam : \textbf{ quod cum factum est , } tripliciter impugnanda est munitio . & fasta la fortaleza que tienen cercada . \textbf{ Et esto quando assi fuere fecho } en tres maneras puede acometer la fortaleza . \\\hline
3.3.19 & vel impellentes castrum . \textbf{ Cum ergo castrum illud appropinquauit } quantum debuit & que traen o enpuxan los castiellos de madera . \textbf{ Et quando aquel castiello o castiellos se llegaren } quanto deuen a los muros de la fortaleza los cercados \\\hline
3.3.20 & murus constitutus ex terra quasi absque laesione susciperet ictus machinarum : \textbf{ quia cum lapis eiectus a machina perueniret ad huiusmodi murum , } propter mollitiem eius cederet terra , & sin grand danno fuyo . \textbf{ Ca quando la piedra del engennio firiere en el muro de tierra . } por la blandura de la tierra \\\hline
3.3.21 & nisi parce , \textbf{ et cum temperamento dispensetur . } Tertio est in talibus attendendum , & Ca non aprouecha nada traer muchas viandas \textbf{ si non fueren partidas con tenpramiento et escassamente . } Lo terçero es de proueer en tales cosas \\\hline
3.3.21 & Recitat enim Vegetius , \textbf{ quod cum Romanis neruorum copia defecisset , } et non possent eorum machinas reparare & Ca cuenta vegeçio \textbf{ que quando a los romanos fallesçieron los neruios } e non podien adobar los engeñios \\\hline
3.3.21 & illae pudicissimae foeminae \textbf{ cum maritis conuiuere deformato capite , } quam seruire hostibus integris crinibus . & Ca dize vegeçio \textbf{ que mas quisieron aquellas buenas mugeres muy castas beuir con sus maridos trasquiladas } que non yr con sus enemigos con cabellos . \\\hline
3.3.22 & obsidentes inchoare cuniculos : \textbf{ quod cum perceperint , } statim debent viam aliam subterraneam & que los que cercan comiençan a fazer cueuas coneieras . \textbf{ Et quando esto entendieren } luego sin detenimiento \\\hline
3.3.22 & iuxta inchoationem viae subterraneae habere magnas tinnas plenas aquis vel etiam urinis : \textbf{ et cum bellant contra obsidentes , } debent se fingere fugere , & auer tiñas lleñas de agua o de oriñas . \textbf{ En quando lidian contra los que los çercan deuen fingir que fuyen } e deuen salir de aquella cueua \\\hline
3.3.22 & ex oleo , sulphure , et pice , et resina : \textbf{ quem ignem cum stupa conuolutum bellatores antiqui Incendiarium vocauerunt . } Huiusmodi autem sagitta & El qual fuego buelto en \textbf{ estopa | llamaronle los lidiadores antigos ençendemiento . } Et esta saeta enuiada \\\hline
3.3.22 & et acutis , et ligatum funibus , \textbf{ cum quo capitur caput arietis , } vel caput illius trabis ferratae : & e atado con fuertes cuerdas \textbf{ con el qual prenden a la cabeça del carnero o a la cabesça de la viga ferrada . } la qual cabesça presa o en todo en todo \\\hline
3.3.22 & per huiusmodi aedificia perforari muros munitionis obsessae : \textbf{ cum de hoc dubitatur , } antequam hoc fiat , & que por estos artifiçios fueren foradados los muros de la fortaleza çercada . \textbf{ quando desto dubdaren ante que los muros sean foradados } çerca de aquellos muros deuen algunos castiellos de madera \\\hline
3.3.23 & ad aliqua similem modum bellandi \textbf{ cum ipsa pugna terrestri . } Nam sicut terrestri pugna oportet & ca la batalla de las naues \textbf{ a semeiança de lidiar en algunas cosas con la batalla de la tierra . } Ca assi commo en la batalla de la tierra . \\\hline
3.3.23 & homines melius esse armatos , quam in terrestri : \textbf{ quia cum pugnatores marini quasi fixi stent in naui , } et quasi modicum se moueant , & meior armados que en la de la tierra \textbf{ por que los lidiadores de la mar esten firmes } e mueuen se muy poco . \\\hline
3.3.23 & Expedit enim eis habere multa vasa plena pice , sulphure , rasina , oleo ; \textbf{ quae omnia sunt cum stupa conuoluenda . } Haec enim vasa sic repleta sunt succendenda , & e de rasina e de olio . \textbf{ las quales cosas todas son de enboluer con estopa . } Et estos belhezos tales \\\hline
3.3.23 & ut ex multis partibus possit nauis succendi ; \textbf{ et cum proiiciuntur talia , } tunc est contra nautas & Et deuen echar muchos tales cantaros en la naue de los enemigos . \textbf{ por que de muchas partes se pueda quemar la naue . } Et entonçe deuen acometer muy fuerte batalla contra los enemigos \\\hline
3.3.23 & quam nautas se habeat quasi aries , \textbf{ cum quo teruntur muri ciuitatis obsessae . } Debet autem sic ordinari lignum illud , & naue commo en los marineros \textbf{ que sea tal commo el carnero con el qual suelen quebrar los muros de la çibdat çercada . } Et deue este madero \\\hline
3.3.23 & quia hoc facto maior habetur commoditas , \textbf{ ut cum ipso percuti possit tam nauis , } quam etiam existentes in ipsa . & ca esto echo siguiesse meior prouecho del \textbf{ ca pueden ferir tan bien en la } naue commo en los que estan en ella \\\hline
3.3.23 & copia ampliarum sagittarum , \textbf{ cum quibus scindenda sunt vela hostium . } Nam velis eorum perforatis , & conuiene de auer grand conplimiento de saetas anchas . \textbf{ con las quales se pueden | ronper las uelas } e los treos de las naues de los enemigos . \\\hline
3.3.23 & e iam nautae habere uncos ferreos fortes , \textbf{ ut cum vident se esse plures hostibus , } cum illis uncis capiunt eorum naues , & auer coruos de fierro muy fuertes \textbf{ e quando veen que son mas } que los enemigos con aquellos coruos prenden las naues dellos \\\hline
3.3.23 & ut cum vident se esse plures hostibus , \textbf{ cum illis uncis capiunt eorum naues , } ut non permittant eos discedere . & e quando veen que son mas \textbf{ que los enemigos con aquellos coruos prenden las naues dellos } et non los dexan foyr . \\\hline
3.3.23 & ex molli sapone , \textbf{ quae cum impetu proiicienda sunt } ad naues hostium ; & dexabon muelle \textbf{ que lançen de rezio en las naues de los enemigos . } Et esto sobre aquellos logares \\\hline
3.3.23 & quae foramina ab hostibus reperiri non poterunt , \textbf{ cum per ipsa coeperit abundare aqua , } qua abundante , et hostes , & los quales non podran los enemigos çercar \textbf{ quando entrare mucha agua dentro en la } naue sabullira a los enemigos \\\hline
3.3.23 & quae quasi lapides iaciuntur , \textbf{ cum quibus hostes nimium offenduntur . } Sed caetera talia & que lançen assi commo piedras \textbf{ de las quales resçiben grand daño los enemigos . } Mas estas otras tales cosas \\\hline

\end{tabular}
