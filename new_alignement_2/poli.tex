\begin{tabular}{|p{1cm}|p{6.5cm}|p{6.5cm}|}

\hline
1.1.1 & E gruesamente \textbf{ Ca Segund dize el philosopho enlas politicas } que aquellas cosas & incedendum est in morali negocio figuraliter et grosse . \textbf{ Immo quia ( secundum Philosophum in Politicis ) } quae oportet dominum scire praecipere , \\\hline
1.1.2 & sirue lascin a quell manyconomica \textbf{ que quiere dezjr gouernamjento de conpanans de casa ¶ al terçero libro sirue la politica } que es sçiençia de gouernamjento delas çibdades e del rreyno & ø \\\hline
1.1.4 & Conuien de saber ujda delectosa et plazentera ¶ \textbf{ vida politica e çiuil¶ } Et uida contenplatina e acabada¶ & ( ut patet ex 1 Ethic’ ) \textbf{ triplicem vitam , } videlicet , voluptuosam , politicam , et contemplatiuam . \\\hline
1.1.4 & quanto partiçipara con las bestias \textbf{ Et en quanto es alguna cosa en sy mjsmo dizen qual coujene la uida politica e ordenada } Mas en quanto partiçipa con las sb̃as apartadas & ut communicat cum brutis , \textbf{ competit vita voluptuosa ; | ut est aliquid in seipso , vita politica : } sed ut participat cum substantiis separatis , \\\hline
1.1.4 & ca pusieron feliçidat \textbf{ e bien andança politica e ordenada } e bien andança contenplatina & quod et Theologi negant : \textbf{ posuerunt enim felicitatem politicam , et contemplatiuam : } ut dicatur quis felix politice ; \\\hline
1.1.4 & Et segunt esto partiçipamos con dios con los angeles \textbf{ ¶ Onde dize el philosofo en el primero libro delas politicas } que cada vno de los omes o es omne o es peor que omne & et secundum quem communicamus cum Deo , \textbf{ et cum substantiis separatis . | Unde ex 1 Politicorum patet , } quod quilibet vel est homo , \\\hline
1.1.4 & segunt sabiduria e rrazon derecha \textbf{ Et de beuir vida politica e ordenada } Mas sy el omne es bestia e peor & sequitur quod regatur secundum prudentiam , \textbf{ et viuat vita politica . } Si autem est bestia , \\\hline
1.1.4 & Et el acabado ente las sçiençias especulatians quanta es \textbf{ entre el que biue vida humanal e vida politica } que es vida ordenada . & et perfectum in speculabilibus , \textbf{ quanta est inter viuentem vita humana et politica , } et viuentem uita contemplatiua et angelica . \\\hline
1.1.4 & ¶ \textbf{ Et llamaron bien auenturado politico } aquel que biue & et ut hominem , \textbf{ sed ut communicat } cum substantiis separatis . \\\hline
1.1.4 & ¶ \textbf{ Enpero del an vida politica e ordenada la qual los theologos llaman vida actiua } que quiere dezir vida de bien obrar & ut infra clarius ostendetur : \textbf{ de vita tamen politica , | quam Theologi vocant vitam actiuam , } et de vita contemplatiua \\\hline
1.1.4 & Mas en lo que ellos dixieron \textbf{ que la vida contenplatian es mejor que la vida politica e actiua } que esta en las obras en esto non descordaron de los cheologos njn dela uerdat . & Quod autem vitam contemplatiuam dixerint \textbf{ esse potiorem , | quam vitam politicam et actiuam , } a Theologis et a veritate catholica non discordant . \\\hline
1.1.6 & Ca asi commo dize el philosofo \textbf{ en el quinto libro delas politicas el prinçipado } e el señorio deue responder ala grandeza e ala dignidat de la persona & et totaliter diuinum . \textbf{ Nam ( ut dicitur 5 Politicorum principatus debet } respondere magnitudini , et dignitati , \\\hline
1.1.6 & e farie que los otros le touiese en pos . \textbf{ ¶ Onde el philosofo en el quinto libro delas politicas } dize & Secundo hoc est detestabile Regi , quia seipsum contemptibilem reddit . \textbf{ unde Philosophus 5 Politicorum ait , } quod maxime expedit Principibus , \\\hline
1.1.7 & deue menospreçiar las delecta connes desmesuradas e carnales \textbf{ philosofo en el primero libro delas politicas } departe dos maneras de riquezas . & Quod non decet regiam maiestatem , \textbf{ Philosophus 1 Politicor’ distinguit } duo genera diuitiarum \\\hline
1.1.7 & Ca el philosofo tanne tres razones \textbf{ enl primero libro delas politicas } por las quales nos pondemos prouar & est ponenda felicitas . \textbf{ Tangit enim 1 Politicor’ tria , } propter quae venari possumus , \\\hline
1.1.7 & por el ordenamiento de los omes \textbf{ ¶ Onde el philosofo enl primero libro delans politicas dize que trismudados los usadores } e trismudado el ordenamiento de los usadores & nisi ex ordinatione Hominum . \textbf{ Unde 1 Politicorum dicitur , | transmutatis utentibus , } idest transmutata dispositione utentium , \\\hline
1.1.7 & Ca puede contesçer asy commo dizeel philosofo \textbf{ en el primero libro delas politicas contado vna buena fabla } que alguno podia ser rico de mucho oro & Potest enim contingere \textbf{ ( ut dicitur 1 Politicorum ) } quod quis diues pecunia , fame moriatur . \\\hline
1.1.8 & e la su bien andança en las riquezas corporales . \textbf{ ora uentra a muchos biuen uida politica } e creen que es de poner la feliçidat & in diuitiis ponere . \textbf{ Forte multi viuentes vita politica credunt } felicitatem ponendam esse in honoribus , \\\hline
1.1.8 & superfiçialmente \textbf{ Onde esto praeua el philosofo en el primero libro delas politicas } do dize & sed apparenter , et superficialiter : \textbf{ unde et Philosophus in Politicis , } curantes de honore tantum , \\\hline
1.1.10 & la qual cosa es falsa . \textbf{ Por que el philosofo praeua en el septimo libro delas politicas } por çinco razones & quod est falsum : \textbf{ probat enim Philosophus | in 7 Pol’ quinque rationibus felicitatem } non esse ponendam \\\hline
1.1.10 & que muchos muy malos tiranos ouieron muy grant poderio çiuil \textbf{ assi commo dize el philosofo en las politicas } que dionisio siracusano o dionisios çiçiliano ouo muy grant poderio çiuil & Nam Dionysius Syracusanus , siue Sicilianus , \textbf{ ut recitat Philosophus in politicis , } maxime abundauit in ciuili potentia , \\\hline
1.1.10 & mas de grand cruel dat ¶ \textbf{ Et por ende el philosofo dize en el se partimo libro delas politicas } que cosa de escarnio es cuydar & nec clementia aliqua . \textbf{ Unde Philosophus 7 Politicorum ait , } quod ridiculum est \\\hline
1.1.10 & enssennorean \textbf{ assi commo quiere el philosofo en las politicas } Et quanto los libres son meiores que los sieruos . & secundum eos quibus aliquis principatus , \textbf{ ut vult Philosophus in Politicis , } quanto liberi sunt meliores seruis : tanto principari liberis , \\\hline
1.1.10 & asi commo dize el philosofo \textbf{ en el septimo libro delas politicas } es mayor bien que la fortaleza ¶ & ut dicit Philosophus \textbf{ 7 Politic’ est maius bonum , } quam sit fortitudo . \\\hline
1.1.10 & por la qual cosa dize el philosofo \textbf{ en el septimo libro delas politicas } deno stando alos griegos & et incurret nocumentum \textbf{ secundum animam . Propter quod Philosophus 7 Politicorum vituperans Lacedaemones , } ponentes felicitatem \\\hline
1.1.11 & Et esto paresçe por el philosofo \textbf{ en el septimo libro delas politicas } que dize & Quod autem in bonis interioribus sit proprie felicitas , \textbf{ patet per Philosophum 7 Politicorum dicentem , } quod testis est nobis Deus , \\\hline
1.1.12 & e dos bien andanças \textbf{ la vna es politica e actiua } que es en las obras & Duas autem felicitates Philosophus posuit , \textbf{ unam politicam , aliam contemplatiuam . } Voluit autem felicitatem \\\hline
1.1.12 & e la bien andança es obra del alma segunt uirtud acabada Et por que la uertud acabada \textbf{ en la uida politica e çibdadana } segund el philosofo es pridençia . & est operatio animae \textbf{ secundum virtutem perfectam . } Cum igitur perfecta virtus \\\hline
1.1.12 & que han de fazer \textbf{ es bien auentado en la uidapolitica . } Et qual quier que sepa bien entender & secundum Prudentiam , \textbf{ est felix politice : } qui vero scit bene speculari \\\hline
1.2.3 & La nona famil yaridat \textbf{ ¶La . x̊ . . eutropolia } que es uirtud & Igitur computata Iustitia , \textbf{ et Prudentia duodecim sunt virtutes morales ; } de quibus omnibus quid sunt , \\\hline
1.2.3 & la otra es bien fablança . \textbf{ La otra es eutropolia } que quiere dezir buena conuerssaçion o buena manera de beuir . & Erit ergo triplex virtus ; \textbf{ videlicet , Veritas , Affabilitas , et Eutrapelia , } quae potest dici bona versio . \\\hline
1.2.3 & mas es bien fablante e curial¶ \textbf{ Mas entropolia que quiere dezir buena conpanma } o buena manera de beuir en conpanna es & sed est affabilis , et curialis . \textbf{ Eutrapelia vero siue bona versio , } est , quando aliquis sic se habet in ludis , \\\hline
1.2.3 & assi commo conuiene en los trebeios . \textbf{ Et la su uirtud es eutropolia } que ̀ere dezir buena conpanma . & ut se habeat circa ludos \textbf{ prout expedit . } Omnes autem hae tres virtutes , \\\hline
1.2.3 & v̉dat de uida ¶a fabilidato bien fablança . \textbf{ Et eutropolia } que es buen a conpania . & ø \\\hline
1.2.3 & Et assi es la uirtud \textbf{ que llaman eutropolia } que es buena conpannia & quot sunt huiusmodi virtutes : \textbf{ et quomodo distinguuntur . } Quod bonarum dispositionum , \\\hline
1.2.7 & enssennorear natural mente . \textbf{ Ca assi commo dize el philosofo en el primero libro delas politicas } que por esso es dicho alguno naturalmente sieruo & quia sine ea non possunt naturaliter dominari . \textbf{ Nam ( ut declarari habet 1 Polit’ ) } ex hoc est aliquis naturaliter seruus , \\\hline
1.2.7 & Ca assi commo dize el philosofo \textbf{ en el primero libro delas politicas } la hmuger ha poco de sabiduria & Sic etiam viri dominantur foeminis , \textbf{ quia ( ut declarari habet 1 Politic’ ) } foemina habet consilium inualidum . \\\hline
1.2.11 & e al prinçipe que las da . \textbf{ assi commo lo muestra el philosofo enlas polititas . } Si los çibdadanos non guardaren la iustiçia legal . & et ad Principem , cuius est leges ferre , \textbf{ ut declarari habet in Politicis : } si ciues non participarent legalem Iustitiam , \\\hline
1.2.14 & enxienplo que ector era fuerte en esta manera \textbf{ que temie ser denostado de polimas . } Et por ende & Hector fortis erat , \textbf{ qui timens increpationes Polydamantis , } aggrediebatur terribilia . \\\hline
1.2.19 & assi commo dize el philosofo \textbf{ en el primero libro delas politicas . } por ende semeiaria a alguons & et naturam rerum , \textbf{ ut vult Philosophus 1 Politicorum , } non videtur sufficienter \\\hline
1.2.28 & assi commo se puede prouar \textbf{ en el primero libro delas politicas . } Conuiene cerca las palauras & Si enim homo est naturaliter animal sociale , \textbf{ ut probari habet 1 Politicorum , } oportet circa uerba , \\\hline
1.2.28 & commo quier que todos los omes \textbf{ que quieren beuir uida politica } e de çibdat deuen seer & diuersimode sit conuersandum : \textbf{ licet omnes homines uolentes viuere politice } debeant esse amicabiles et affabiles , \\\hline
1.2.28 & que los otro . \textbf{ Onde el philosofo enel quarto libro delas politicas } dando captelas alos Reyes & quam alii . \textbf{ Unde Philosophus 5 Politicorum } dando cautelas Regum et Principum , \\\hline
1.2.30 & Et pues que assi es paresçe \textbf{ que cosa es el alegria o la entropolia } en quanto es uirtud . & Patet ergo quid est iocunditas , \textbf{ vel eutrapelia , } prout est virtus , \\\hline
1.2.33 & macrobio commo plotino quatro guados de uirtudes . \textbf{ Ca algunas uirt Sudes son politicas . } Et algunas son purgatorias . & quam Plotinus quatuor gradus virtutum . \textbf{ Nam quaedam sunt politicae , } quaedam purgatoriae , \\\hline
1.2.33 & que son en dios . \textbf{ Et las uirtudes politicas son uirtudes g̃nadas } que ganan los . & virtutes exemplares esse in ipso Deo . \textbf{ Virtutes autem politicas , } esse virtutus acquisitas , \\\hline
1.2.33 & Et por ende diremos \textbf{ que los persseuerantes han uirtudes politicas } e los continentes han uirtudes pragatorias . & demus suum ordinem virtutum . \textbf{ Dicemus ergo quod perseuerantes habent virtutes politicas : } continentes , purgatorias : \\\hline
1.2.33 & son mas altas que las pragatorias . \textbf{ Et las p̃gatorias son mas altas que las politicas ¶ } Et que estos linages destas & excellunt purgatorias : \textbf{ et purgatoriae politicas . } Quod autem haec genera virtutum adaptari debeant \\\hline
1.2.33 & que las primeras uirtudes conuiene saber . \textbf{ Las politicas amollesçen } e desponen el coraçon a bien fazer e reduzen lo a medio¶ & quod primae virtutes , \textbf{ scilicet politicae , } molliunt idest ad medium reducunt . \\\hline
1.2.33 & por la qual cosa bien dicho es \textbf{ que las uirtudes politicas parten } e lçen alos ꝑlseuerantes & Quare bene dictum est , \textbf{ quod perseuerantibus competunt virtutes politicae , } quia perseuerantes \\\hline
1.2.33 & e non salga allende nin a quande¶ Et \textbf{ pues que assi es las uirtudes politicas } que amollesçen e ordenan el coraçon abien fazer & quod teneat se in medio . \textbf{ Virtutes ergo politicae } quae molliunt , \\\hline
1.2.34 & e podriemos ganar bondat acabada . \textbf{ ssi commo es dicho dsuso commo quier que largamente tomando las uirtudes toda buean dispoliçion del alma } puede ser dicha alguna uirtud . & et perfectam bonitatem acquirere . \textbf{ Dicebatur enim supra , | quod licet largo modo accipiendo virtutes , } omnis bona dispositio mentis possit \\\hline
1.2.34 & Por ende bien dichones \textbf{ que delas bueans dispoliçiones del alma } algunas son uirtudes & benedictum est , \textbf{ quod bonarum dispositionum } quaedam sunt virtutes , \\\hline
1.3.3 & Ca assi commo es dicho de suso \textbf{ e assi commo el philosofo lo praeua en las politicas difetençia e deꝑtimiento es entre el Rey e el tyra non } por que el Rey prinçipalmente entiende el bien comun de todos . & nam ( ut superius dicebatur , \textbf{ et ut Philosophus in Polit’ probat ) | differentia est inter Regem , et tyrannum : } quia Rex principaliter intendit bonum commune : \\\hline
1.3.4 & Ca nos veemos en cada vna delas obras \textbf{ assi commo es prouado en el primero delas politicas } que la fin es desseada & Videmus autem in singulis artibus , \textbf{ ut probatus primo Politicorum , } quod finis appetitur in infinitum : \\\hline
2.1.1 & Et por ende el philosofo \textbf{ en el primero libro delas politicas entre las otras razones } que tanne & Unde et Philosophus 1 Politicorum \textbf{ inter alias rationes , } quas tangit , \\\hline
2.1.1 & Et por ende dize el philosofo \textbf{ en el primero libro delas politicas el que toma e escoge de beuir uida sola } e apartada non es parte dela çibdat & Ideo dicitur primo Politicorum \textbf{ quod eligens solitariam vitam , } non est pars ciuitatis : \\\hline
2.1.2 & Ca segunt dize el philosofo \textbf{ en el primero libro de las politicas } la comunindat dela casa & sed ciuilis : \textbf{ quia secundum Philosophum 1 Politicorum , } Communitas domus , \\\hline
2.1.2 & que si nos cuydamos con grant acuçia los dichos del philosofo \textbf{ en las politicas paresçra } que son quatro las comuindades Conuiene a saber . & Aduertendum ergo quod \textbf{ si dicta Politica diligenter consideremus , } apparebit quadruplicem esse communitatem ; \\\hline
2.1.2 & Ca assi commo paresçe por el philosofo \textbf{ en el primero libro delas politicas } ca uatural nasçemiento & Naturalis enim origo ciuitatis \textbf{ ut patet per Philosophum 1 Politicorum , } hoc modo existit , \\\hline
2.1.2 & e assi en esta manera fue secho de muchas casas vn \textbf{ uarrio¶ Onde dize el philosofo en el primero libro delas politicas } que el uarrio es vezindat de casas & et sic ex multis domibus factus fuit vicus . \textbf{ Unde et 1 Politicorum , | dicitur , } quod vicus est vicinia domorum , \\\hline
2.1.3 & en quanto ha orden ala \textbf{ ca la que es comuidat delas perssonas assy commo parte nesçe al politico } que quiere dezir ordenador de çibdat & prout habet ordinem ad domum , \textbf{ quae est communitas personarum ; | sicut spectat ad Politicum determinare } de ordine domorum , \\\hline
2.1.3 & por perfeccion e por conplimiento . \textbf{ Onde el philosofo en el primero libro delas politicas } conparando la çibdat aluarrio & illis perfectione et complemento . \textbf{ Unde et Philosophus 1 Politicorum } comparans ciuitatem \\\hline
2.1.4 & que el philosofo en el primero libro delas \textbf{ politicasasse declara } e difine la comunidat dela casa & Sciendum ergo , \textbf{ Philosophum 1 Politicorum sic describere communitatem domus : } videlicet , quod domus est communitas secundum naturam , \\\hline
2.1.4 & que assi comm̃ departe el philosofo el philosofo \textbf{ en el primero libro delas politicas } que algunas delas obras de los omes & quomodo domus sit communitas constituta in omnem diem . \textbf{ Ad cuius euidentiam aduertendum , } quod humanorum operum , \\\hline
2.1.4 & Et por ende dize el philosofo \textbf{ en el primero libro delas politicas } que assi commo la comunidat dela casa es establesçida & reperiatur in alia . \textbf{ Propter quod Philosophus 1 Politicorum ait , } quod sicut communitas domus \\\hline
2.1.4 & e assi commo dize el philosofo \textbf{ en el primero libro delas politicas } non solamente la casa es vna comiundat & immo ( ut infra patebit , \textbf{ et ut vult Philosophus 1 Polit’ ) } non solum domus est communitas quaedam , \\\hline
2.1.5 & por mal gouernamiento de los otros . \textbf{ l philosofo en el primero libro delas politicas } dize que la casa primera es establesçida de dos comuindades . & quam ex incuria aliorum . \textbf{ Philosophus 1 Politic’ vult , } quod ex duabus communitatibus , \\\hline
2.1.5 & Mas aqueles propraamente sieruo segunt dize el philosofo \textbf{ en el primero libro delas politicas } que fallesçe en el entondemiento & ille vero est proprie seruus \textbf{ ( ut patet per Philosophus 1 Politic’ ) } qui deficiens intellectu , \\\hline
2.1.5 & mas sienpre ay alguna cosa en logar de sieruo . \textbf{ assi commo dize el philosofo en el primero libro delas politicas . } Ca assi commo alli dize los omes pobres & ibi aliquid loco serui , \textbf{ ut vult Philosophus 1 Politic’ . } Nam ut ait , pauperes homines , \\\hline
2.1.5 & las quales son uaron e muger e sieruo o alguna cosa en logar de sieruo . \textbf{ Onde el philosofo en el primero libro delas politicas alaba a } esiodo que dixo & videlicet , ex viro , uxore , seruo vel aliquo loco serui . \textbf{ Unde et Philosophus 1 Politicorum commendat Hesiodum , } dicentem domum constare ex tribus , \\\hline
2.1.5 & que dize el philosofo \textbf{ enel primero libro delas politicas } non solamente sirue a gouernamiento dela casa & et quot communitates requiruntur ad domum : \textbf{ secundum Philosophum 1 Politicorum , } non solum deseruit regimini domestico , \\\hline
2.1.6 & mas es de razon dela casa ya acabada \textbf{ ¶ Onde el philosofo en el primero libro delas politicas } commo ouiesse dicho primeramente & sed est de ratione domus perfectae ; \textbf{ unde et Phil’ 1 Poli’ } cum prius dixisset \\\hline
2.1.6 & Siguese que en la casa acabada \textbf{ assi commo dize el philosofo en el primero libro delas politicas tres son los gouernamientos . } Ca el vno matrimonial de marido e de muger & in domo perfecta \textbf{ ( ut vult Philosophus 1 Politicorum ) | sunt tria regimina , unum coniugale , } secundum quod vir praeest uxori : \\\hline
2.1.7 & Ca segunt dize el philosofo \textbf{ en el primero libro delas politicas } en la comunidat dela casa & primo agendum est de regimine coniugali : \textbf{ quia secundum Philosophum 1 Politic’ } in communitate domestica , \\\hline
2.1.7 & por natura es aianl mas coniuigable \textbf{ que politico } que quier dezer & Homo enim natura \textbf{ magis est coniugale animal quam politicum , } quanto domus est prior \\\hline
2.1.7 & aianl conuuigable e ayuntable por casamiento . \textbf{ Et esta razon tanne el philosofo en el primero libro delas politicas } e en el octauo delas ethicas do prueua & homo naturaliter est animal coniugale . \textbf{ Hanc autem rationem tangit Philosophus 1 Politicorum , et 8 Ethicorum , } ubi probat coniugium competere homini secundum naturam , \\\hline
2.1.7 & Por la qual cosa assi \textbf{ conmodiziemos dela uida politica e de çibdat . } Conuiene a saber que el que escoge beuir solo & Quare sicut dicebamus \textbf{ de societate politica , } videlicet quod eligens solitudinem , \\\hline
2.1.10 & assi commo paresçe \textbf{ por el philosofo en las politicas } sienpre el uaron deue ser mayor & Nam secundum ordinem naturalem \textbf{ ( ut patet per Philosophum in Polit’ ) } semper vir debet esse praeeminens , \\\hline
2.1.11 & que non saquan algunas perssonas del mater moino . \textbf{ Onde el philosofo en las politicas } mouiendo se con razon natural saca algunas perssonas & non exceptuantes personas aliquas a coniugio contrahendo . \textbf{ Unde et Philosophus 2 Polit’ } sola ratione naturali ductus \\\hline
2.1.12 & ca la primera con pama natural \textbf{ assi commo paresçe por el philosofo enlas politicas es del maslo e dela fenbra e del uaron e dela mugni . } Mas esto non si asi el casamiento non fuesen ordenado a algua conpanna conuenible e natural . & prima autem naturalis societas ( ut patet per Philosophum in Polit’ ) \textbf{ est maris et foeminae , viri , et uxoris . } Hoc autem non esset , \\\hline
2.1.13 & assi commo dize el philosofo \textbf{ en el septimo delas politicas . } non sabe ser uagarosa . & Nam mens humana \textbf{ ( ut innuit Philosophus 7 Politicorum ) } nescit ociosa esse ; \\\hline
2.1.14 & falladasen vna casa . \textbf{ Onde el philosofo en el primero delas politicas } conpara los gouernamientos de vna casa & per quandam similitudinem reseruantur in domo . \textbf{ Unde et Philosophus primo Politicorum , } regimina unius domus \\\hline
2.1.14 & e de los lus çibdadanos \textbf{ e es dicho tal gouernamiento politico e çiuil . } Et por ende tal regno sera dicho politico e çiuil . & sed magis ab ipsa politia et ab ipsis ciuibus . \textbf{ Dicitur ergo tale regimen politicum vel ciuile . } His autem duobus regiminibus in ciuitate \\\hline
2.1.14 & e es dicho tal gouernamiento politico e çiuil . \textbf{ Et por ende tal regno sera dicho politico e çiuil . } Mas estos dos gouernamientos que son en la çibdat & sed magis ab ipsa politia et ab ipsis ciuibus . \textbf{ Dicitur ergo tale regimen politicum vel ciuile . } His autem duobus regiminibus in ciuitate \\\hline
2.1.14 & segunt el philosofo \textbf{ enlaspoliticas son semeiantes dos gouernamientos dela casa de } lons quales el vno es paternal & His autem duobus regiminibus in ciuitate \textbf{ secundum Philosophum in Polit’ } assimilantur \\\hline
2.1.14 & en qual manera el marido se deua auer çerca la muger . \textbf{ Et por ende es dicho tal gouernamiento politico e çiuil por que es semeiado a aquel gouernamiento } en el qual los çibdadanos llamando a su señor muestran le los pleitos & quomodo vir habere se debeat circa ipsam . \textbf{ Dicitur ergo tale regimen politicum : | quia assimilatur illi regimini , } quo ciues vocantes dominum , \\\hline
2.1.15 & Et por ende dize el philosofo \textbf{ en el primero libro delas politicas } que na falmente se departe la muger del sieruo . & non erit ordinata ad seruiendum . \textbf{ Ideo dicitur 1 Politicorum , } quod naturaliter distinguuntur \\\hline
2.1.15 & Assi conmo dize el philosofo \textbf{ en el primero libro delas politicas do dize que entre los barbaros la fenbra } e el sieruo han vna orden & ut recitat Philosophus 1 Politicorum dicens , \textbf{ quod inter Barbaros foemina et seruus eundem habent ordinem . } Utebantur enim illi coniugibus tanquam seruis . \\\hline
2.1.15 & Onde dize el philosofo \textbf{ en el sexto delas politicas } que los pobres & si uxor et seruus habeant eundem ordinem . \textbf{ Unde dicitur 6 Politicorum , } quod pauperes , \\\hline
2.1.16 & Ca el philosofo tanne en el . \textbf{ vi̊ libro delas politicas quatro razones } por que praeua que enla hedat de grand moçedat & Tangit enim Philosophus 7 Polit’ \textbf{ quatuor rationes probantes } quod in aetate nimis iuuenili non est utendum coniugio . \\\hline
2.1.16 & deguatid moçedat se ayuntare el uaron ala muger \textbf{ assi commo praeua el philosofo en las politicas } dende sallen los fijos dannados quanto al cuerpo . & Nam si in aetate nimis iuuenili contingantur vir et uxor \textbf{ ( ut probat Philosophus in Polit’ ) } laeduntur inde filii quantum ad corpus , \\\hline
2.1.16 & Onde en el . \textbf{ vij̊ delas politicas dize el philosofo } que aquellas mugersmas son destenpradas que fueron vsadas seyendo moças ¶ La terçera razon se toma del periglo delas mugers & assueta ad illud . \textbf{ Unde et 7 Politic’ dicitur , } quod intemperatiores videntur \\\hline
2.1.16 & ca assi commo dize el philosofo \textbf{ en las politicas las mugers moças } mas se duelen en el parto que las otras & Tertia via sumitur \textbf{ ex periculo mulierum . Nam ut dicitur in Polit’ | in partu iuuenculae } magis dolent et periclitantur plures . \\\hline
2.1.16 & si en grand moçedat vsaren de casamiento . \textbf{ Onde en las politicas dize el philosofo } que los cuerpos de los mas los resçiben & si in nimia iuuentute utantur coniugio . \textbf{ Unde in Politicis , } dicitur quod masculorum corpora laeduntur , \\\hline
2.1.17 & que cunple . \textbf{ L philosofo en el . vij̊ libro dellas politicas } despues que prouo & si videbitur expedire . \textbf{ Philosophus 7 Polit’ } postquam probauit per rationes plurimas , \\\hline
2.1.18 & que estan en hedat acabada . \textbf{ Ca legunt el philosofo enlas politicas } los mas los mas son ennoblesçidos & ad homines existentes in aetate perfecta . \textbf{ Nam secundum eundem Philosophum } in Polit’ mares \\\hline
2.1.19 & por que sea callada conueniblemente . \textbf{ Ca assi commo dize el philosofo en el pramer libro delas politicas } grand conponimiento es delas mugers el silençio . & quomodo regenda sit , ut debite sit taciturna . \textbf{ Nam , ut scribitur 1 Polit’ ornamentum mulieris est taciturnitas . } Si enim mulieres debite se habeant , \\\hline
2.1.23 & assi conmo dize el philosofo \textbf{ en el primero delas politicas es flaco . } Ca assi comm̃el moço ha consseio menguado por que & Consilium mulierum , \textbf{ ut dicitur 1 Politicorum est inualidum : } nam sicut puer habet consilium imperfectum , \\\hline
2.2.1 & deuiemos determinar del gouernamiento de los sieruos . \textbf{ Enpero assi commo dize el philosofo en el primero delas politicas en el gouernamiento dela } casa mayor deue ser el cuydado de los omes & determinandum esse de regimine seruorum . Verum quia , \textbf{ ut dicitur primo Politicorum , } oeconomiae amplior est solicitudo \\\hline
2.2.2 & e mayor nobleza que los otros . \textbf{ Ca segunt el philosofo enlas politicas . } segunt que algunos son en mayorestado e en mas alta dignidat & maiori bonitate pollere quam alios : \textbf{ quia secundum Philosophum in Politic’ | secundum } quod aliqui sunt \\\hline
2.2.3 & Et tal gouer namiento \textbf{ segunt dicho es de lulo es nonbrado politico o çiuil } o los gouenena segunt albedrio . & Et tale regimen \textbf{ ( ut superius dicebatur ) | nominatur politicum vel ciuile . } Vel regit eos secundum arbitrium , \\\hline
2.2.3 & Este gouernamiento non es semeiante al gouernamiento çiuil mas al Real . \textbf{ Onde el philosofo en el primo libro delas politicas } dize & huiusmodi regimen non assimilatur regimini politico , sed regali . \textbf{ Unde et Philosophus 1 Politicorum ait , } virum praeesse mulieri , \\\hline
2.2.3 & que non puede auer conplimiento de sieruos \textbf{ Ende el philosofo en el sexto libro delas politicas dize } que los robres han menester de vsar de sus mugers e de sus fiios & quod seruorum copiam habere non potest . \textbf{ Unde et Philosophus 6 Polit’ ait , } quod egenis necesse est \\\hline
2.2.7 & assi commo otra natraa \textbf{ Et commo segunt el philosofo en las politicas } aquellas obras & quasi altera natura , \textbf{ cum secundum Philosophum in Polit’ nobis } magis placeant illa opera , \\\hline
2.2.8 & La quarta sçiençia libal es dichͣ musica . \textbf{ Et esta segunt que dize el philosofo en el octauo delas politicas } conuiene alos mançebos & dicitur esse Musica . \textbf{ Haec secundum Philos’ | 8 Polit’ } conuenit ipsis iuuenibus , \\\hline
2.2.8 & La segunda razon para prouar esto mismo puede ser esta \textbf{ que pone el philosofo en el vii i̊ libro delas politicas } do dize que el entendimiento del omne non sabe ser vagaroso nin estar de ualde . & Secunda ratio ad hoc idem esse potest , \textbf{ quia ( ut Philosophus innuit | in eodem 8 Poli’ ) } mens humana nescit ociosa esse . \\\hline
2.2.8 & que son conueinbles e honestos . \textbf{ Mas el philosofo tanne muchͣs razones en las politicas } por las quales se podrie mostrar & quae sunt licitae et honestae . \textbf{ Tangit enim Philosophus | multas rationes in Politicis , } per quas ostendi posset , \\\hline
2.2.8 & que es del gouernamiento dela casa e dela conparatid̃ . \textbf{ Et la politica } que es del gouernamiento de las çibdades e del regno & et Oeconomica , quae est de regimine familiae : \textbf{ et Politica , quae est de regimine ciuitatis et regni , } valde sunt utiles \\\hline
2.2.8 & e de los prinçipes \textbf{ li quisieren beuir uida politica e çiuil } e quasieren gouernar los otro & et maxime filii Regum et Principum , \textbf{ si velint politice viuere , } et velint alios regere et gubernare , \\\hline
2.2.8 & que son de las obras de los omes \textbf{ sonsola politica quees del gouernamiento delas çibdades . } Ca assi con mocios dixiemos en otro logar & quae sunt de actibus hominum , \textbf{ sub politica , | quae est de regimine ciuitatum . } Nam ( ut alibi nos dixisse meminimus ) \\\hline
2.2.8 & que todos los legistas son \textbf{ assi commo vnos nesçios politicos . } Ca assi commo los legos & omnes legistae sunt \textbf{ quasi quidam idiotae politici . } Nam sicut laici et vulgares , \\\hline
2.2.8 & en essa misma manera los legistas \textbf{ por que aquellas cosas delas quales es la sçiençia politica } digen los legistas contando & appellantur a Philosopho idiotae dialectici : \textbf{ sic Legistae , quia ea de quibus est politica , } dicunt narratiue et sine ratione , \\\hline
2.2.8 & e non demostrando las \textbf{ por razon por ende pueden ser llamados nesçios politicos . } Et desto puede parescer en qual manera & dicunt narratiue et sine ratione , \textbf{ appellari possunt idiotae politici . } Ex hoc autem patere potest \\\hline
2.2.8 & mas son de honrrar \textbf{ los que saben la politica } e las sçiençias morales & quod magis honorandi sunt \textbf{ scientes politicam } et morales scientias , \\\hline
2.2.8 & maguera que entiendan ser caualleros \textbf{ e entender en el fecho politico dela çibdat deuen trabaiar } por que sepan el lenguage delas letras . & quantumcunque intendant esse milites , \textbf{ et vacare negocio politico , | debent insudare , } ut sciant idioma literale . \\\hline
2.2.8 & por que sepan gouernar assi e alo . s otros . \textbf{ Otrossi segunt el philosofo en las politicas conuiene les } alguacosado saber dela musica & Nam et de musica \textbf{ secundum Philosophum in politicis } eos scire decet , \\\hline
2.2.9 & Por que seg̃t que dize el philosofo \textbf{ en el primero libro delas politicas } mayor cuydado deuemos auer sienpre delas cosas & possessionibus , et rebus inanimatis : \textbf{ quia secundum Philosophum 1 Politicorum , } semper de animatis amplior cura habenda \\\hline
2.2.10 & qual maestro deuen poner en gouernamiento de sus fijos \textbf{ erca la fin del septimo delas politicas } muestra el philosofo & qualem magistrum proponerent in regimine filiorum . \textbf{ Circa finem 7 Politicor’ docet Philosophus } iuuenes cohibendos esse \\\hline
2.2.10 & si les fuer defendido de oyr cosas torpes . \textbf{ Ca segunt el philosofo en el vi̊ libro delas politicas } do fabla desta materia es & si prohibeantur ab auditione turpium . \textbf{ Nam secundum philosophum vii Polit’ } ubi de hac materia loquitur , \\\hline
2.2.12 & muestra lo el philosofo \textbf{ en el septimo libro delas politicas } o dize & In qua autem aetate debeant uti coniugio , \textbf{ ostendit Philosophus 7 Poli’ dicens , } quod in muliere requiritur aetas decem et octo annorum , \\\hline
2.2.12 & e de los nançebos \textbf{ assi commo dize el philosofo en esse mismo libro dela o politicas } Et pues que assi es en esta misma manera & et impeditur eorum augmentum , \textbf{ ut vult Philosophus in eisdem Poli’ . } Sic ergo utendum est coniugio , \\\hline
2.2.13 & assi commo prueua el philosofo \textbf{ en el viij̊ libro delas politicas } es neçessario enla vida & ut probat Philosophus \textbf{ 8 Poli’ } est necessarius in vita quod \\\hline
2.2.13 & luego comiença a andar vagando cuydando enlas cosas desconueibles . \textbf{ Onde el philosofo enłviiij libro delas politicas dize } que el trebeio es vn reduzimiento & statim incipit vagari cogitando de illicitis : \textbf{ unde Philosophus 8 Polit’ ait , } quod ludus est quaedam deductio \\\hline
2.2.13 & assi que en esto resçibiendo alguno folgua a puedan mas trabaiar para alcançar su fin . \textbf{ Onde el philosofo enłviij̊ delas politicas dize } que por que el ome non puede sienpre folgar en la fin ganada . & magis possint laborare in consecutione finis . \textbf{ Unde et Philosophus 8 Politicorum ait , } quod quia homo non potest \\\hline
2.2.13 & assi commo dize el philosofo enł . viij̊ \textbf{ delas politicas ¶ } Visto en qual manera los moços se deuen auer çerca los trebeios finca de ver & et deductiones inhonestae prohibendae sunt a iuuenibus , \textbf{ ut vult Philosophus 7 Politicorum . } Viso qualiter iuuenes se habere debeant circa ludos . \\\hline
2.2.15 & e dende adelante \textbf{ Mas el pho tanne enł septimo libro delas politicas seys cosas } que se deuen guardar çerca los moços enla primera hedat . & Postea a decimoquarto et deinceps . \textbf{ Tangit autem Philosophus 7 Polit’ sex } circa ipsos pueros , \\\hline
2.2.15 & mayormente sean cerados de leche \textbf{ Onde el philosofo en el septimo libro delas politicas } dize que el humor dela leche es muy conuenible & ita tamen quod a principio maxime alendi sunt lacte : \textbf{ unde et Philosophus 7 Politic’ ait , } quod nutrimentum lactis maxime videtur \\\hline
2.2.15 & ¶ Lo terçero los mocos deuen ser acostunbrados al frio . \textbf{ Onde el philosofo en el septimo libro delas politicas } dizeque luego conuiene alos mocos pequanos & Tertio pueri sunt assuescendi ad frigora : \textbf{ unde Philosophus septimo Politi’ ait , } quod mox expedit pueris paruis consuescere ad frigora . \\\hline
2.2.15 & por que los mienbros dellos sean mas firmes . \textbf{ Onde el philosofo en el septimo libro delas politicas } dize & ut membra eorum solidantur \textbf{ unde Philosophus 7 Poli’ ait , } quod expedit in pueris \\\hline
2.2.15 & segunt dize el philosofo \textbf{ en el septimo libro delas politicas } faze a fortaleza del cuerpo . & Detinere autem spiritum et anhelitum \textbf{ secundum Philosophum septimo Politicorum , } facit ad robur corporis . \\\hline
2.2.16 & non deuen tomar obras de caualłia nin otras obras fuertes . \textbf{ Et por ende el pho enł viij̊ libro delas politicas dize } que fasta la hedat dela pubeçençia & non sunt assumenda opera militaria nec opera ardua . \textbf{ Unde Philosophus 8 Polit’ ait , } quod usque ad pubescentiam , \\\hline
2.2.16 & que el philosofo enl vij̊ . \textbf{ libro delas politicas } dizeque muy mala cosa es de non enssennar & Sciendum ergo , \textbf{ quod Philosophus 5 Polit’ ait , } quod pessimum est \\\hline
2.2.16 & Ca el philosofo enłvij̊ libͤ \textbf{ delas politicas } demanda si primeramente deuemos auer cuydado de los moços . & et ad obseruantiam legum utilium . \textbf{ Inquirit enim Philosophus 8 Polit’ } utrum prius curandum sit de pueris , \\\hline
2.2.16 & Assi segunt dize el philosofo en el . \textbf{ vij̊ libro delas politicas } primeramente deuemos auer cuydado de los mocos & et postea illi anima infunditur : \textbf{ sic secundum Philosophum 8 Politicorum , } prius curandum est \\\hline
2.2.17 & por que fasta el xiiij ano \textbf{ segunt el philosofo en las politicas } mas son de enduzir los moços a bien & usque enim ad quartumdecimum annum \textbf{ secundum Philosophum in Polit’ } magis inducendi sunt pueri \\\hline
2.2.17 & Mas la buena disposiçonn del cuerpo \textbf{ aquellos que quieren beuir uida politica e de çibdat es esta que del . xiiij ̊ . } año adelante t omne trabaios mayores que tomaron ante . & Bona autem dispositio corporis , \textbf{ ut volentibus viuere vita politica , | a quartodecimo anno et deinceps , } est ut assumant fortiores labores quam hactenus . \\\hline
2.2.17 & Et esto dize elpho \textbf{ enł viij̊ libro delas politicas } que fasta los . xiiij años los moços deuen ser acostunbrados a trabaios ligeros & ex tunc autem assumendi sunt labores fortiores . \textbf{ Nam et Philosophus 8 Poli’ ait , } quod usque ad quartumdecimum annum pueri \\\hline
2.2.17 & Mas el pho pone en el vii̊ . \textbf{ libro delas politicas tres } razono breues & 8 Poli’ \textbf{ tres breues rationes , } quare decet filios esse subiectos , \\\hline
2.2.17 & por las cosas ya dichos . \textbf{ Ca si quieren beuir uida politica } e de çibdat e de caualłia & patet etiam per iam dicta . \textbf{ Nam si volunt viuere vita politica et militari , } potissime studere debent \\\hline
2.2.18 & o dos los mançebos \textbf{ que quieren beuiruida politica } e de çibdat en alguna manera se deuen bsar en los trabaios corporałs & qualiter seipsos regant . \textbf{ Omnes iuuenes volentes viuere vita politica , } aliquo modo exercitandi sunt \\\hline
2.2.18 & ca segunt el pho \textbf{ en łviij̊ libro delas politicas los trabaios corporales } e el cuydar del entendimiento & minores labores assumere . \textbf{ Nam secundum Philosophum 8 Politicorum labor corporalis , } et consideratio per intellectum , \\\hline
2.2.21 & mauera se torna despreçiada \textbf{ e por ende el philosofo en el primero libro delas poliricas } dizeque el honrramiento delas mugers es callar & et quodammodo se contemptibilem reddit ; \textbf{ unde et Philosophus primo Politicorum ait , } quod ornamentum mulierum est silentium . \\\hline
2.3.1 & Esto muestra el philosofo muy conplidament en \textbf{ enldmero libro delas politicas } demuestra dot dos razones & ut in principio capituli proponebatur , \textbf{ sufficienter ostendit Philosophus 1 Poli’ . } Ostendit autem duabus rationibus , \\\hline
2.3.1 & por la qual \textbf{ cosasi para beuir uida politica } e de çibdat son menester casas e possesiones et dineros & et quae requiruntur ad sufficientiam vitae : \textbf{ quare si ad vinendum politice requiruntur domus , possessiones , et numismata , } spectat ad gubernatorem domus considerare de talibus . \\\hline
2.3.1 & segt̃ dize el philosofo \textbf{ enł primero libro delas politicas } Ca los estrumentos del arte del gouernamiento dela casa son actiuos & Differunt tamen haec ab illis \textbf{ secundum Philosophum primo Poli’ } quia organa gubernationis sunt actiua , \\\hline
2.3.2 & e el sieruo estas dos materias conueniblemente son ayuntadas en vno¶ \textbf{ lpho enl primero libro delas politicas } reduze todos los estrumentos & congrue hae duae materiae connectuntur . \textbf{ Philosophus primo Politicorum omnia organa gubernationis domus } ad bimembrem distinctionem reducit . \\\hline
2.3.2 & assi segunt el philosofo \textbf{ enl primero libro delas politicas } deue ser cerca el gouernamiento dela casa . & Sicut enim vidimus in aliis artibus , \textbf{ sic ( secundum Philosophum ) } circa gubernationem domus esse habet . \\\hline
2.3.2 & la qual cosa declara el philosofo \textbf{ enł primero libro delas politicas } por cosa semeiable en las otras artes & ut per animata ; \textbf{ quod declarat Philosophus 1 Polit’ } per simile in aliis artibus , \\\hline
2.3.2 & commo era la ymagen de dalo \textbf{ dela qual dize el philosofo en las politicas } en manera de fabli ella & qualis erat statua Daedali , \textbf{ de qua Philosophus fabulose recitat in Politicis ; } quod per se implebat opus debitum : \\\hline
2.3.3 & e esta razon tanne el philosofo \textbf{ enł vi̊ libro delas politicas } do dize que alos Reyes & sumitur ex parte ipsius populi : \textbf{ et hanc tangit Philosophus 6 Politicorum , } ubi ait , \\\hline
2.3.5 & conuiene en algunan manera \textbf{ que las cosas naturales sean neçessarias enla uida politica . } mas segunt el philosofo & oportet aliquomodo naturalia esse \textbf{ quae sunt necessaria in vita politica ; } sed secundum Philosophum primo Polit’ \\\hline
2.3.5 & mas segunt el philosofo \textbf{ en el primero libro delas politicas } la possession de las cosas es neçessaria en el & quae sunt necessaria in vita politica ; \textbf{ sed secundum Philosophum primo Polit’ } necessaria est rerum possessio in gubernatione domus , \\\hline
2.3.5 & la possession de las cosas es neçessaria en el \textbf{ gouernamientode la casa si alguno quisiere beuir uida politica . } Et pues que assi es por esta razon misma & necessaria est rerum possessio in gubernatione domus , \textbf{ si quis debeat politice viuere ; } eo ergo ipso quod rerum possessio deseruit necessitati vitae , \\\hline
2.3.5 & omenatanlonde el philosofo \textbf{ enł primer libro delas politicas do prueua } que la possession de tales cosas es natural . & ideo eorum possessio est ei naturalis . \textbf{ Unde et Philosop’ 1 Politic’ } ubi probat possessionem talium naturalem esse , \\\hline
2.3.5 & mas esto faze e deue fazer alas aianlias acabadas . \textbf{ Et por ende el philosofo dize en el primero libro delas politicas } que estas cosas de que & multo magis hoc facit animalibus perfectis . \textbf{ Ideo ait Philosophus primo Polit’ } quod haec , \\\hline
2.3.5 & ca assi commo es dicho \textbf{ enł primero libro delas politicas } nos somos en alguna manera fin de todas las cosas Et pues que assi es aquellos & ø \\\hline
2.3.5 & segunt dize el philosofo \textbf{ enl primero libro delas politicas de auer possession } e sennorio de algers cosas de fuera & ut homo est , \textbf{ ut vult Philosophus primo Polit’ habere possessionem , } et dominium aliquarum rerum exteriorum \\\hline
2.3.6 & assi commo cuenta el philosofo \textbf{ enł segundo libro delas politicas } que cola aprouechosa & Fuit opinio Socratis et Platonis , \textbf{ ut recitat Philosophus 2 Polit’ } quod esset utile \\\hline
2.3.6 & de departidos logares \textbf{ enł libro delas politicas tres cosas } por las quales podemos prouar & Possumus autem ex diuersis locis \textbf{ in libro Polit’ accipere tria , } per quae triplici via venari possumus , \\\hline
2.3.6 & Ca segunt dize el pho \textbf{ enł segundo libro delas politicas } assi contescria estonçe & ex remotione inordinationis et confusionis : \textbf{ nam secundum Philosophum 2 Politicorum , } sic accideret tunc , \\\hline
2.3.7 & que fazen fructo sean labrados \textbf{ el pho enł primero libro delas politicas } muestra & et absque litigio , campi et terrae et alia fructifera excolantur . \textbf{ Philosophus primo Politicorum ostendit , } quod ex alio et alio usu exteriorum rerum , \\\hline
2.3.7 & e non han vna manera de beuir . \textbf{ Et el philosofo enł primero libro delas politicas } departe quatro vidas sinples & et non eodem modo viuendi viuunt . \textbf{ Distinguit enim Philosophus primo Politicorum quatuor vitas simplices , } vel quatuor modos viuendi , \\\hline
2.3.7 & es aquello que es triuio el philosofo \textbf{ en el primero libro delas politicas } que la natura dio anos tales cosas & Propter quod bene dictum est \textbf{ quod scribitur 1 Polit’ } quod natura dedit nobis talia , \\\hline
2.3.7 & lias batalla derecha \textbf{ por la qual cosa el philosofo enl ꝑ̀mo libro delas politicas } dize & habet contra talia iustum bellum \textbf{ propter quod Philosophus 1 Politic’ vult venatiuam et piscatiuam } esse vitas licitas . \\\hline
2.3.8 & que non quisiessen ante muchͣs mas ¶ \textbf{ Mas el philosofo da en el primero libro delas politicas dos razones } por las quales la cobdiçia de las riquezas es sin fin e sin mesura¶ & quin plura velint . \textbf{ Assignat autem Philosophus 1 Politicor’ duplicem causam , } quare concupiscentia diuitiarum est infinita . \\\hline
2.3.8 & ca assi commo departe el philosofo \textbf{ enł primero libro delas politicaͤ } e assi commo dixiemos dessuso enl primero & Nam ut distinguit Philosophus primo Polit’ \textbf{ et ut supra in primo libro diffusius diximus , } aliter appetitur finis , \\\hline
2.3.8 & que dize el philosofo \textbf{ enł primero libro delas politicas . } Et pues que assi es nin el arte del gouernamiento dela casa non deue demandar possessiones et riquezas sin mesura e sin fin . & nulla , autem ars , \textbf{ ut ait Philosophus 1 Politicor’ } habet organa infinita , \\\hline
2.3.9 & n fueren penssados los dichs del philosofo \textbf{ eñł primero libro delas politicas } todas las muda connes son aduchos a tres linages & Si considerentur dicta Philosophi \textbf{ in 1 Politicor’ } omnes commutationes \\\hline
2.3.9 & bien dicho es lo que se dize \textbf{ enł primero libro delas politicas } que en la primera comunidat & bene dictum est \textbf{ quod dicitur 1 Politicorum } quod in prima communitate quae est domus , \\\hline
2.3.9 & assi commo da a entender el philosofo \textbf{ en el prim̃o libro delas politicas } buuendo en su sinpliçidat & sed etiam denariorum ad denarios . Antiquitus enim homines \textbf{ ( ut satis innuit Philosophus primo Politicorum ) } in simplicitate viuentes \\\hline
2.3.10 & e los dineros finca de dezer quantas son las maneras de los dineros . \textbf{ Et el philosofo en las politicas } pone quatro maneras de dineros conuiene saber . & quot sunt species pecuniatiuae . \textbf{ Distinguit autem Philos’ | in Poli’ } quatuor species pecuniatiuae : \\\hline
2.3.10 & Et esta segunt el philosofo \textbf{ enel primero libro delas politicas } primeramente fue fallada sinplemente e auentura . & dicitur esse campsoria : \textbf{ haec enim ( secundum Philosophum 1 Politicorum ) } forte primitus casu \\\hline
2.3.10 & segunt el pho \textbf{ en el primero libro delas politicas } e el dinero es comienço e fin por que esta e arte comiencaen erl dinero & Sed in ea \textbf{ ( secundum Philosophum primo Politicorum ) } denarius est elementum et terminus , \\\hline
2.3.10 & ¶ Mas destas quatro maneras \textbf{ segunt que dize el p̃h̃o en las politicas } la primera que es assi commo & Harum autem quatuor species \textbf{ secundum Philosophum in Polit’ sola prima , } quae est quasi oeconomica et naturaliter , \\\hline
2.3.11 & Ca primeramente la llamamos parto de dineros \textbf{ del qual nobre el philosofo la rephede en el primero libro delas politicas } diziendo que es contra natura & Vocatur enim primo denariorum partus , \textbf{ ex quo nomine arguit Philosophus 1 Polit’ } eam contra naturam esse . \\\hline
2.3.11 & bien dicho es lo que dize el ph̃co \textbf{ en el primero libro delas politicas } que la usuraes de denostar & bene dictum est \textbf{ quod dicitur 1 Poli’ usuram esse } quid detestabile et contra naturam . \\\hline
2.3.11 & Enpero deuedes saber que assi commo dize el pho \textbf{ enl primero delas politicas de cada cosa } ay dos usos vn uso propo & ( ut ait Philos’ 1 Politicorum ) \textbf{ quasi cuiuslibet rei est duplex usus : | unus proprius , } et alius non proprius . \\\hline
2.3.12 & por que son contrael derecho natural \textbf{ erca la fin del primero libro delas politicas } pone el philosofo departidas maneras & quod iuri naturali contradicant . \textbf{ Circa finem primi Polit’ distinguit Philosophus diuersos modos , } quibus numismata acquiruntur . \\\hline
2.3.12 & entre todas las cosas \textbf{ que acresçientan las riquezas es fazer monopolia } que quiere dezer vendiconn de vno solo . & ( secundum Philos’ ) \textbf{ est facere monopoliam , } idest facere vendationem unius : \\\hline
2.3.12 & que quiere proueer ala mengua dela casa \textbf{ segunt uida politica de auer cuydado de ganar dineros segunt que requiere } e demanda el su estado de cada vno . & secundum vitam politicam volentem prouidere indigentiae domesticae , \textbf{ habere curam de acquisitione pecuniae , } secundum quod exigit suus status : \\\hline
2.3.12 & ¶ Avn en essa misma manera \textbf{ segunt el philosofo dize enlas politicas deuemos cuydar cerca el labramiento delas tr̃ras } por que deuemos cuydar & Sic etiam , \textbf{ ut Philosophus in Polit’ ait , | insistendum est circa apum culturam , } si partes illae aptae essent \\\hline
2.3.13 & la qual cosa praeua el philosofo \textbf{ en el primero libro delas politicas por quatro razones . } ¶ La primera razon se toma dela semeiança & et quod naturaliter expedit aliquibus aliis esse subiectos : \textbf{ quod probat Philosophus primo Polit’ quadruplici via , } sumpta ex quadruplici similitudine . \\\hline
2.3.13 & mas conplidamente nunca de mucho omes se faria naturalmente \textbf{ vna conpannia o vna poliçia } si naturalmente non fuesse a ellos conuenible que algunos fuessen sennores & ut superius diffusius probabatur , \textbf{ numquam ex pluribus hominibus fieret naturaliter una societas vel una politia , } nisi naturale esset \\\hline
2.3.13 & e bien ordena de el alma en ssennorea \textbf{ e el cuerpo obedesçe assi en la poliçia } e enla çibdat bien ordenadlos sabios deuen enssennorear & anima dominatur , \textbf{ et corpus obedit : } sic in politia bene ordinata sapientes debent dominari , \\\hline
2.3.13 & Mas esto contesçe \textbf{ por desordenança dela poliçia e dela çibdat . } Ca assi commo en el omne pestilençial e malo & Sed hoc accidit \textbf{ ex peruersitate politiae : } nam sicut in homine pestilente , \\\hline
2.3.13 & nin la razon \textbf{ assi enlas poliçias } delas çibdades pestilençiosas e desordenadas & magis quam anima vel ratio : \textbf{ sic in politiis pestilentibus , } et corruptis magis dominantur ignorantes , \\\hline
2.3.13 & ssennorea ala feribra dela qual dize el philosofo \textbf{ en el primero libro delas politicas } que ha conseio muy flaco & naturaliter dominari foeminae , \textbf{ de qua dicitur primo Politicorum } quod habet consilium inualidum : \\\hline
2.3.14 & que dize el philosofo \textbf{ enl primero libro delas politicas } que los uençidos en la fazienda deuen seruir alos uençedores . & Est enim iustum legale , \textbf{ ut recitat Philosophus 1 Politicorum superatos } in bello seruire superantibus . \\\hline
2.3.14 & que dize el philosofo \textbf{ en las politicas } aya algua auentaia sobre el su sieruo . & oportet enim dominans \textbf{ ( ut dicitur in Politic’ ) } habere aliquem excessum respectu serui . \\\hline
2.3.14 & Por la qual cosa bien dicho es \textbf{ lo que dize el pho enel primero libro delas politicas } que si dizumos & propter quod bene dictum est , \textbf{ quod dicitur 1 Politicor’ } quod si debemus aliquos digne dominari \\\hline
2.3.16 & Et la razon desto pone el philosofo \textbf{ en el segundo libro delas politicas } do dize que alguas vezes peor siruen los muchs seruidores que los pocos . & Ratio autem assignatur 2 Polit’ \textbf{ ubi dicitur , } quod aliquando deterius seruiunt \\\hline
2.3.16 & que son dichͣs \textbf{ en el quarto libro delas politicas . } Ca deuemoos assi ymaginar que comm̃o se ha la guand çibdat ala pequeña . & Ratio autem huius haberi potest \textbf{ ex iis quae dicuntur 4 Polit’ . } Debemus enim sic imaginari \\\hline
2.3.16 & çerca la fin del quarto \textbf{ librdelas politicas non son de ayuntar } muchsofiçios & sic se habet magna domus ad paruam . \textbf{ In magna enim ciuitate non sunt congreganda officia et principatus , } ita quod eidem committantur diuersa officia , \\\hline
2.3.17 & assi commo prueua el philosofo \textbf{ en el sesto libro delas politicas¶ } Lo segundo çerca las uestidas es de penssar la semeiança de los siruientes . & decet eos magnifica facere , \textbf{ ut probat Philosophus 7 Poli’ . } Secundo circa vestitum consideranda est uniformitas ministrantium . \\\hline
2.3.17 & assi commo dize el philosofo çerca la fin del . viij̊ . \textbf{ libro delas politicas } que sienpre la persona & nimis afficimur ad illa . \textbf{ Ideo dicitur circa finem 7 Polit’ } quod semper prima magis amamus . \\\hline
2.3.18 & non puede ser falso del todo \textbf{ assi commo dize el philosofo en el viij̊ . libro delas politicas . } Et esta tal opinion del pue blo ha de ssi alguna prueua & impossibile est esse falsum secundum totum , \textbf{ ut videtur velle Philosophus 7 Ethicorum , } huiusmodi vulgaris opinio alicui probabilitati innititur . \\\hline
2.3.18 & Ca assi commo dize el pho \textbf{ enl primero libro delas politicas . } Assi commo de omne nasçe omne & ut plurimum quod nobiles genere sunt nobiliorum morum quam alii : \textbf{ nam sicut ex homine nascitur homo , } et ex bestiis bestia : \\\hline
2.3.18 & ay falssedat \textbf{ ca assi commo dize el philosofo en las politicas la natura quiere sienpte fazer alguna cosa . } Enpero muchͣs uezes non la puede fazer & Huic autem probabilitati aliquando subest falsitas , \textbf{ quia ( ut dicitur in Politicis ) | natura vult } qui de hoc facere multotiens , \\\hline
2.3.19 & ca assi conmo conuiene alos çibdadanos de ser iustos e legales \textbf{ para guardar su poliçia conueniblemente } assi conuiene alos sermient & Nam sicut decet ciues \textbf{ ut debitam politiam seruent } esse iustos legales , \\\hline
2.3.19 & ca assi commo dize el pho \textbf{ enł primero libro delas politicas } que este fecho & nec Reges et Principes . \textbf{ Nam ( ut dicitur primo Polit’ ) } et hoc non habet aliquod magnum \\\hline
2.3.19 & que son dichͣs \textbf{ en el quinto libro delas politicas } do dize que la persona del prinçipe & sumi potest \textbf{ ex iis quae dicuntur 5 Polit’ } ubi dicitur , \\\hline
2.3.19 & cerca la fin del prim̃o libro \textbf{ delas politicas dize } que la fenbra por cierto ha conseio & De his autem loquens Philosophus \textbf{ circa finem primi Politicorum ait , } quod foemina quidem habet consilium inualidum , \\\hline
2.3.20 & ¶ La primera razon paresçe \textbf{ assi ca assi commo es dicho en el primero libro delas politicas } estonçe cada vna cosa es ordenada a vna obra segunt natura & Prima via sic patet . \textbf{ Nam , ut dicitur primo Poli’ } tunc secundum naturam unumquodque perficitur , \\\hline
3.1.1 & Ca pruena el pho \textbf{ enl primero libro delas politicas } por dos razones & oportet ciuitatem ipsam constitutam esse propter aliquod bonum . \textbf{ Probat autem Philosophus primo Polit’ duplici via , } ciuitatem constitutam esse gratia alicuius boni . \\\hline
3.1.1 & que toda obra e toda electiuo dessea de auer algun bien . \textbf{ Et en el primero libro delas politicas } dize & quoddam appetere videtur , \textbf{ et primo Politicorum scribitur , } quod gratia eius quod videtur bonum , \\\hline
3.1.1 & ca assi commo dize el philosofo \textbf{ enl primero libro delas politicas } vna inclinaçion de natura es en todos los orans & quod tamen existit bonum : \textbf{ huius autem est constitutio ciuitatis . Nam ( ut dicitur primo Poli’ ) } natura quidem impetus in omnibus inest \\\hline
3.1.1 & e esto es lo que dize el philosofo \textbf{ en el primero libro delas politicas } que si nos dizimos & maxime ordinatur ad bonum . \textbf{ Hoc est ergo quod dicitur primo Polit’ } quod si communitatem omnem gratia alicuius boni dicimus constitutam , \\\hline
3.1.1 & por grande algun bien \textbf{ e esta es comunidat politica } que es llamada & et omnes alias circumplectens , potissime gratia boni constitutam esse contingit : \textbf{ haec autem est communitas politica , } quae communi nomine vocatur ciuitas . \\\hline
3.1.2 & porque por ella alcançan los omes ser acabados \textbf{ e beuir en comunidat politica e de çibdat } ca sin ella la uida del omne non puede ser & quia per eam homines consequuntur \textbf{ omnia tria praedicta bona . | Nam ipsum viuere consequuntur homines ex communitate politica : } quia sine ea vita hominis \\\hline
3.1.2 & en quanto es omne . \textbf{ Et por ende estendiendo el beuir politico } segunt alguas leys de loar & ut homo est . \textbf{ Ostendendo ergo viuere politicum secundum aliquas leges } et secundum aliquas laudabiles ordinationes , \\\hline
3.1.2 & por que el ome \textbf{ esnatraalmente aianl politico } e çiuil & homo enim est \textbf{ naturaliter politicum animal et ciuile , } ut infra patebit . \\\hline
3.1.2 & commo omne los omes alcançan tal beuir \textbf{ por partiçipamiento politico o por establestimiento dela çibdat } ¶ Lo segundo por tal establesçimiento de çibdat los omes & ( loquendo de viuere ut homo ) \textbf{ consequuntur homines excommunicatione politica | siue ex constitutione ciuitatis . } Secundo ex tali constitutione homines \\\hline
3.1.2 & lo que dize el pho \textbf{ en el primero libro delas politicas } que la comunidat & Bene ergo dictum est , \textbf{ quod scribitur primo Politicorum } quod communitas , \\\hline
3.1.2 & Et por ende dize el pho \textbf{ en el primero libro delas politicas } que fue fecha la çibdat & secundum leges et virtuose . \textbf{ Ideo dicitur primo Politicorum } quod facta \\\hline
3.1.2 & nin durar \textbf{ otdenaron la comunidat politica } que era fechͣ & constituta ciuitas stare non posset , \textbf{ ordinauerunt communitatem politicam , } quae facta erat ad viuere , \\\hline
3.1.2 & aquello que dize el pho \textbf{ en el primero libro delas politicas } que el primo & bene dictum est \textbf{ quod scribitur primo Politicorum , } quod \\\hline
3.1.3 & si la çibdat es cosa natural \textbf{ e si el omne es natraalmente aian l . politicas . } e çiuilca aque łłas cosas & secundum naturalem ? \textbf{ et An homo sit naturaliter } animal politicum et ciuile ? \\\hline
3.1.3 & mas veemos \textbf{ que muchos fuyen la conpanna politicas } e çiuil & nullus reperiretur homo non ciuilis . \textbf{ Videmus autem multos societatem politicam retinentes , } eligere solitariam vitam , et campestrem . \\\hline
3.1.3 & que non biuen çiuilmente . \textbf{ Mas el pho tanne enl primero libro delas politicas tres razones } por las quales contesçe & reperiuntur tamen multi campestre viuentes . \textbf{ Tangit autem Philosophus 1 Polit’ tria , } quare contingit alienos \\\hline
3.1.3 & assi commo dize el philosofo \textbf{ en el primero libro delas politicas } do dize que maldicho es el & Hi autem sunt illi , \textbf{ quos ( ut recitat Philosophus 1 Poli’ ) } maledicebat Homerus , dicens , \\\hline
3.1.3 & por la qual cosa dize el philosofo \textbf{ en el primero libro delas politicas } que los que non pueden beuir en conpanna con los otros & eligens altiorem vitam : \textbf{ propter quod scribitur primo Poli’ } quod non potens aliis communicare , \\\hline
3.1.4 & e podemos por dos razones mostrat \textbf{ que la comiundat politicas } o la çibdat es algunan cosa segunt natura . & Possumus autem duplici uia ostendere \textbf{ communitatem politicam } siue ciuitatem esse aliquid secundum naturam . \\\hline
3.1.4 & ca assi commo prueua el pho \textbf{ en el primero libro delas politicas } lo que es fin dela generaçion & illarum communitatum finis et complementum . \textbf{ Nam ut arguit Philosophus primo Politicorum } quod est finis generationis naturalium , \\\hline
3.1.4 & assi commo dize el philosofo \textbf{ en el primero libro delas politicas } por la qual cosa & et nepotum , \textbf{ ut vult Philosophus 1 Polit’ } propter quod si tale crementum est naturale , \\\hline
3.1.4 & que el omne es naturalmente \textbf{ aianl politicas } e ciuilla qual cosa podemos demostrar & reliquum est ostendere , \textbf{ hominem esse naturaliter animal politicum et ciuile , } quod etiam duplici via inuestigare possumus . \\\hline
3.1.4 & assi capuado fue \textbf{ en el comienço del segundo libro delas politicas de parte dela palabra } que el omne es naturalmente & Probatur enim in principio secundi libri , \textbf{ ex parte sermonis } hominem esse naturaliter animal sociale , \\\hline
3.1.4 & que el omne es natraalmente \textbf{ aianl politicas e ciuil } por que la boz del omne & Hoc autem ex parte sermonis ostendere possumus \textbf{ hominem esse naturaliter animal politicum et ciuile , } ex eo quod vox humana , \\\hline
3.1.4 & que la bos delas bestias . \textbf{ Onde el philosofo dize en el primero libro delas politicas } que en las o trisaian lias apartadas del omne & quam vox brutorum . \textbf{ Unde Philos’ ait 1 Pol’ } quod in aliis animalibus ab homine , \\\hline
3.1.4 & e por ende en todos los omes es inclinaçion natural \textbf{ para beuir politicas miente en çibdat } e para fazer çibdat & Inerit ergo hominibus impetus naturalis \textbf{ ad viuendum politice , } et ad constituendum ciuitatem . \\\hline
3.1.5 & assi ca segunt que dize el philosofo \textbf{ en el primero libro delas politicas } que la comunidat acabada & Prima via sic patet . \textbf{ Nam cum ait Philosophus primo Polit’ } quod communitas perfecta , \\\hline
3.1.6 & la otra manera es por concordia de los que establesçen la çibdat e el regno \textbf{ ca assi commo cuenta el philosofo en el segundo libro delas politicas } enel tienpo antigo los omes mora una esparzidos & Alius modus est ex concordia constituentium ciuitatem vel regnum . \textbf{ Nam ut recitat Philosophus 2 Politicorum , } antiquitus homines morabantur dispersi , \\\hline
3.1.6 & en qual manera de una gouernar las çibdades e los regnos ¶ \textbf{ Lo segundo mostraremos qual es la muy buean politica o çibdat o muy vuen regno } e de quales cautelas deuen usar los prinçipes e los . Reyes & regere ciuitates et regna . \textbf{ Secundo ostendetur , | quae sit optima politia siue optimum regnum , } et quibus cautelis uti debeant principantes , \\\hline
3.1.7 & assi commo cuenta el philosofo \textbf{ enł primero libro delas politicas } por aquellas cosas & sicut et viros . Inducebantur enim ad hoc \textbf{ ( ut Philosophus recitat 2 Polit’ ) } ex iis quae videbant in bestiis \\\hline
3.1.8 & e platon dixieron \textbf{ ca el philosofo pone etł segundo libro delas politicas } seys razones & ut Socrates et Plato credebant . \textbf{ Tangit autem Philosophus | 2 Polit’ } quasi sex rationes , \\\hline
3.1.8 & alli commo dize el pho \textbf{ enł segundo libro delas politicas } otra cosa es defendimiento de çibdat & vel ad pugnationem . \textbf{ Nam ut dicitur 2 Polit’ } alterum est compugnatio , \\\hline
3.1.9 & por la qual cosa commo en el gouernamiento dela çibdat \textbf{ primeramente se ha de ordenar la poliçia } muy luengamente es de buscar & ne circa ipsa contingat error . \textbf{ Quare cum in regimine ciuitatis primo sit politia ordinanda , } diu inuestigandum est , \\\hline
3.1.9 & losoens puedan beuir comunal mente . \textbf{ Onde el pho dize en el terçero libro delas politicas } que los çibdadanos non deuen ser llegados alas leyes & secundum quam homines communiter possint uiuere . \textbf{ Unde et Philosophus ait in Politicis ciues } non esse applicandos legibus , \\\hline
3.1.9 & que lo semeiaua . \textbf{ Onde el pho cuenta en el segundo libro delas politicas } que entre algunas gentes & quem videret sibi esse similem . \textbf{ Unde et Philosophus narrat 2 Politicor’ } quod apud quasdam gentes , \\\hline
3.1.9 & assi commo dize el pho \textbf{ en el segundo libro delas politicas } por que vno ama a otro & Nam plus est modo de dilectione in ciuitate , \textbf{ ut Philosophus innuit 2 Politicor’ , } quia unus diligit alium tanquam filium , \\\hline
3.1.10 & assi commo conuiene ala comunidat de los çibdadanos \textbf{ lpho prueua en el primero libro delas politicas } que muchos males se siguen en la çibdat & ut expedit communitati ciuium . \textbf{ Philosophus 2 Polit’ probat multa mala sequi in ciuitate , } si uxores et filii ponantur esse communes . \\\hline
3.1.10 & muy de coraçon por aquellos dos otres \textbf{ Mas esto reprahende el philosofo enel segundo libro delas politicas } ca por dos o por tres o por pocos mocos querera mar grant muchedunbre de moços & quod omnes alios intime diligerent propter illos . \textbf{ Sed ut arguit Philosophus 2 Politic’ | propter duos vel tres } vel propter paucos pueros velle magnam multitudinem diligere puerorum tanquam proprios filios , \\\hline
3.1.11 & assi commo dize el philosofo \textbf{ en el segundo libro delas politicas } en tres maneras se puede entender & Esse res communes , \textbf{ ut ait Philosophus 2 Politic’ , } tripliciter potest intelligi . \\\hline
3.1.11 & mas assi commo dize el philosofo \textbf{ en el segundo libro delas politicas } en los fechos particulares & inter quos tanta communitas obseruatur . \textbf{ Sed ut dicitur secundo Politicorum } in actibus particularibus oportet \\\hline
3.1.11 & que por la mayor parte han contiendas e uaraias por la qual cosa dize el philosofo \textbf{ en el segundo libro de las politicas } que de los siruientes & ostenditur ut plurimum homines habere lites et iurgia \textbf{ propter quod Philosophus ait 2 Polit’ } quod ab ipsis famulis , \\\hline
3.1.11 & assi commo cuenta el philosofo \textbf{ en el segundo libro delas politicas } tanta era la franqueza & Unde et apud Lacedaemones , \textbf{ ut recitatur secundo Politi’ } tanta erat liberalitas , \\\hline
3.1.12 & por que segunt el philosofo \textbf{ en el segundo libro delas politicas } alas bestias & insufficiens est : \textbf{ quia secundum Philosophum 2 Poli’ } bestiis nihil attinet oeconomice , \\\hline
3.1.13 & assi commo dize elpho \textbf{ en el segundo libro delas politicas } ca si menospreçiando alos vnos sienpre los otros fueren puestos en los ofiçios & et pacificum statum ciuium . \textbf{ Nam si spretis aliis semper iidem in magistratibus et praeposituris praeficiantur , } alii videntes se esse despectos \\\hline
3.1.13 & pone el pho \textbf{ en el segundo libro delas politicas } quando & Hanc autem tertiam rationem improbantem ordinationem Socraticam \textbf{ tangit Philosophus 2 Poli’ } cum ait . \\\hline
3.1.14 & por la qual cosa el philosofo \textbf{ en el libro delas politicas } reprehende a socrates deste & ø \\\hline
3.1.14 & ca segunt dize el philosofo \textbf{ en el segundo libro delas politicas } el que quiere poner leyes o fazer ordenaçion alguna en la çibdat a tres & in constituendo determinatum numerum bellatorum . \textbf{ Nam secundum Philosophum secundo Politicorum , } volens ponere leges \\\hline
3.1.15 & esquariasocrates \textbf{ que algua poliçia non deuiesse ser llamada çibdat } si a lo de menos non ouiesse en ella minłłomes nobles & Volebat ergo Socrates politiam \textbf{ aliquam non debere nominari ciuitatem , } nisi saltem contineret mille nobiles , \\\hline
3.1.16 & assi commo cuenta el philosofo \textbf{ en el segundo libro delas politicas } que se entremi tio del ordenamiento dela çibdat & ut narrat Philosophus \textbf{ 2 Politicorum intromisit se de ordine ciuitatis , } statuens quomodo posset \\\hline
3.1.16 & que se entremi tio del ordenamiento dela çibdat \textbf{ establesçiendo en qual manera se podria ordenas muy bien la poliçia e la çibdat } ca dizia & 2 Politicorum intromisit se de ordine ciuitatis , \textbf{ statuens quomodo posset | optime politia ordinari . } Dicebat autem , \\\hline
3.1.16 & La terçera por aquellas cosas \textbf{ que ueya en las otras poliçias . } ¶ La primera razon paresçe & Tertio , ex his quae videbat \textbf{ in politiis aliis . } Prima via sic patet . \\\hline
3.1.17 & i fueren penssados los dichos del philosofo \textbf{ en el segundo libro delas politicas } quanto pertenesçe alo presente & volens eos aequatas possessiones habere . \textbf{ Si considerentur dicta Philosophi 2 Politicorum , } quantum ad praesens spectat , \\\hline
3.1.17 & que se non puede guardar \textbf{ ca segunt elpho en el quarto libro delas politicas } non solamente los principeᷤ deuen auer cuydado & quae conseruari non potest . \textbf{ Nam secundum Philosophum 4 Politicorum } non solum debet esse cura \\\hline
3.1.17 & ca assi commo dize el philosofo \textbf{ en el segundo libro delas politicas meesteres ala pazer dela çibdat } que los fijos de los ricos & quia ut dicitur 2 Polit’ \textbf{ opus est } ad pacem ciuitatis filios diuitum \\\hline
3.1.18 & assi commo cuenta el philosofo \textbf{ en el segundo libro delas politicas era ley } que para que las suertes antiguas fuessen guardadas & Sic etiam apud Locros , \textbf{ ut recitat Philosophus 2 Politic’ lex erat , } quod ad hoc ut antiquae sortes seruarentur illesae , \\\hline
3.1.18 & assi commo prueua el philosofo muy llanamente \textbf{ en el segundo libro delas politicas . } Mas desto diremos adelante mas conplidamente & circa reprimendas concupiscentias quam circa alia , \textbf{ ut plane probat Philosophus 2 Polit’ . } Sed de hoc inferius diffusius dicemus . \\\hline
3.1.18 & que fiziemos vn tractado del partimiento \textbf{ que es entre la ethica e la rectorica e la politica } en el qual si los dichs fueren penssados superfiçialmeᷤte & nos edidisse quendam tractatum . \textbf{ De differentia Ethicae Rhetoricae et Politicae , } ubi dicta superficialiter considerata contradicere videntur \\\hline
3.1.19 & que y podo mio \textbf{ establesciendo su poliçia } primero se entremetio dela muchedunbre & diuersa genera personarum . \textbf{ Hippodamus autem statuens suam politiam , } primo intromisit se de multitudine \\\hline
3.1.19 & assi commo dize el philosofo \textbf{ en el segundo libro delas politicas } deuiese de vieios sabios escogidos & Volebat quidem principale praetorium , \textbf{ ut narrat Philosophus 2 Politicorum , } debere esse ex senibus electis : \\\hline
3.1.19 & e en algunan manera de soltar deuie lo traer determinado por su estriptura . \textbf{ Mas el philosofo en el segundo libro delas politicas } do pone la opinion del dicho & per scripturam determinaret illud . \textbf{ Philosophus autem 2 Politicor’ | ubi positionem dicti Philosophi narrat , } assignat , \\\hline
3.1.20 & e por ende contamos la opinion de ipodomio \textbf{ por que el en la su poliçia manifesto muchͣs bueanssmans . } Empo algunas cosas establesçio non conuenible mente . & Hippodami ergo opinionem recitauimus , \textbf{ quia in sua politia multas bonas sententias promulgauit : } aliqua tamen incongrue statuit . \\\hline
3.1.20 & signiendo los dichos del philosofo \textbf{ enel segundo libro delas politicas rephender a } ipodomio & Possumus autem quantum ad praesens spectat , \textbf{ sequendo dicta Philos’ 2 Pol’ increpare } Hippodamum quantum ad tria . \\\hline
3.2.1 & e rezando opiniones de departidos philosofos \textbf{ que establesçieron poliçias } e dieron arte del gouernamiento dela çibdar & praemittendo quaedam praeambula ad propositum , \textbf{ et recitando opinionem diuersorum Philosophorum instituentium politiam , } et tradentium artem \\\hline
3.2.1 & en el gouernamiento del regno e dela çibdat \textbf{ Mas el philosofo en el terçero libro delas politicas } tanne quatro cosas & in tali regimine . \textbf{ Videtur autem Philos’ 3 Polit’ tangere , } quatuor quae consideranda sunt \\\hline
3.2.2 & Enpero primero diremos del prinçipado . \textbf{ el terçero libro delas politicas } departe el philosofo seys linaies de prinçipados & primo tamen dicemus de ipso principatu . \textbf{ Tertio Politicorum distinguit Philosophus } sex modos principantium , \\\hline
3.2.2 & que quiere dezer sennorio de buenos \textbf{ e la poliçia } que quiere dezer pueblo bien & Nam regnum aristocratia , \textbf{ et politia sunt principatus boni : } tyrannides , oligarchia , et democratia sunt mali . \\\hline
3.2.2 & llamalle el philosofo nonbre comun \textbf{ e diz el poliçia } por que poliçia es & vocat eum Philosophus nomine communi , \textbf{ et dicit ipsum esse Politiam . } Politia enim quasi idem est , \\\hline
3.2.2 & e diz el poliçia \textbf{ por que poliçia es } assi commo ordenamiento bueno de çibdat & et dicit ipsum esse Politiam . \textbf{ Politia enim quasi idem est , } quod ordinatio ciuitatis quantum \\\hline
3.2.2 & enssennorea a todos los otros \textbf{ ca la poliçia esta mayormente en el ordenamiento del grant prinçipado } que es en la çibdat & qui dominantur omnibus aliis . \textbf{ Politia enim consistit | maxime in ordine summi principatus , } qui est in ciuitate . \\\hline
3.2.2 & Pues que assi es todo ordenamiento de çibdat \textbf{ puede ser dicħ poliçia . } Enpero el prinçipado del pueblo si derecho es & Omnis ergo ordinatio , \textbf{ ciuitatis Politia dici potest . } Principatus tamen populi si rectus sit , \\\hline
3.2.2 & Enpero el prinçipado del pueblo si derecho es \textbf{ por que non ha nonbre comun es dich poliçia } e nos podemos llamar atal prinçipado gouernamiento del pueblo & Principatus tamen populi si rectus sit , \textbf{ eo quod non habeat commune nomen , | Politia dicitur . } Nos autem talem principatum appellare possumus gubernationem populi , \\\hline
3.2.3 & ir que entendemos mostrar \textbf{ qual es la muy buena poliçia } e que cosa es el muy buen prinçipado & Quia intendimus ostendere \textbf{ quae sit optima politia , } et quis sit optimus principatus . \\\hline
3.2.4 & e han cunplimiento delas cosas \textbf{ lpho en el terçero libro delas politicas tanne tres razons } por las quales paresçe que se puede prouar & abundantia florent . \textbf{ Philosophus 3 Politicorum videtur | tangere tres rationes , } per quas probari videtur , \\\hline
3.2.4 & que vno solo si fuesse vn prinçipe . \textbf{ Onde el pho en el terçero libro delas politicas } dize & quam si principaretur unus tantum : \textbf{ unde Philosophus 3 Politicorum ait , } quod plures homines sic principantes quasi constituunt \\\hline
3.2.4 & que esta duda \textbf{ que el pho pone en el terçero libro delas politicas dudado } e poniendo muchͣs razones para esto & quod hanc dubitationem \textbf{ quam Philosophus 3 Politicorum venatur , | dubitando , } assignans rationes multas , \\\hline
3.2.4 & ca el dize muchͣs uezeᷤ \textbf{ en esse mismo libro delas politicas } que el regno es prinçipado muy digno & dum tamen utrunque sit rectum , \textbf{ cum ipse pluries dicat in eisdem politicis , } regnum esse dignissimum principatum : \\\hline
3.2.4 & segunt \textbf{ que dize el philosofo enel terçero libro delas politicas } que deue aconpannar assi e tomar consigo muchos sabios & ille Princeps vel ille Rex \textbf{ ( secundum Philosophum 3 Politicor’ ) } debet sibi associare multos sapientes , \\\hline
3.2.5 & La primera razon paresçe \textbf{ assi ca segunt elpho en el segundo libro delas politicas } non se puede contar & Prima via sic patet . \textbf{ Nam secundum Philosophum 2 Politic’ } inenarrabile est \\\hline
3.2.5 & si penssare que el regno ha de venir a señorio de los fijos \textbf{ enpero el pho enel terçero libro delas politicas } do fabla destamateria dize & ad dominium filiorum . \textbf{ Philosophus tamen 3 Politic’ | ubi de hac materia determinat , } ait , quod hoc est difficile , \\\hline
3.2.5 & que alos otros segunt \textbf{ que dize el pho en las politicas } conuiene & tribuere magis primogenito quam aliis : \textbf{ quia ( ut ait Philosophus in Politiis ) } decet iuniores senioribus obedire . \\\hline
3.2.7 & e si non entendiere en el bien comun . \textbf{ Onde en el terçero libro delas politicas } dize el philosofo & et nisi intendat commune bonum . \textbf{ Unde 3 Polit’ dicitur , } quod principari talem , \\\hline
3.2.7 & quanto por ella mas se arriedra el tirano dela entençion del bien comun . \textbf{ Et esta razon tanne el philosofo çerca el comienço del quarto libro delas politicas . } do dize que assi conmo el regno es muy buena et muy digna poliçia . & ab intentione communis boni . \textbf{ Hanc autem rationem tangit Philosophus | circa principium 4 Politicorum ubi ait , } quod sicut Regnum est optima \\\hline
3.2.7 & Et esta razon tanne el philosofo çerca el comienço del quarto libro delas politicas . \textbf{ do dize que assi conmo el regno es muy buena et muy digna poliçia . } assi la tirania es muy mala & circa principium 4 Politicorum ubi ait , \textbf{ quod sicut Regnum est optima | et dignissima politia , } sic tyrannis est pessima : \\\hline
3.2.7 & assi commo y dize el pho \textbf{ por quela tirama much se arriedra dela poliçia e del bien comun . } la segunda manera para prouar esto mismo se toma . & ( ut ibi dicitur ) \textbf{ quia tyrannis plurimum distat a politia , | idest a communi bono . } Secunda via ad inuestigandum hoc idem , sumitur ex eo quod tale dominium maxime est naturale . \\\hline
3.2.7 & Et esta razon tanne avn el philosofo \textbf{ en el quarto libro delas politicas } do dize que la tirania es muy mal prinçipado & Hanc autem rationem tangit Philosophus \textbf{ in eodem 4 Politicorum ubi ait , } tyrannidem esse pessimum principatum , \\\hline
3.2.7 & e esta razon tanne el philosofo \textbf{ en el quinto libro delas politicas } do dize que la tirnia es la postrimera obligarçia & multa mala efficere . \textbf{ Hanc autem rationem tangit Philosophus quinto Politicorum ubi ait , } tyrannidem esse oligarchiam \\\hline
3.2.7 & mas avn esfuercasse para enbargar los bienes dellos \textbf{ e tanne espho en el quinto libro delas politicas muy grandes tres bienes } que puna de enbargar el tirano . & impedire eorum maxima bona . \textbf{ Tangit autem Philosophus 5 Polit’ tria maxima bona , } quae satagit impedire tyrannus , \\\hline
3.2.9 & Mas avn assi commo dize el philosofo \textbf{ en el terçero libro delas politicas deue enduziras Ꝯmugres propraas } por que sean familiares e bien querençiosas alas mugers & per quos bonus status regni conseruari potest , \textbf{ sed etiam ut ait Philosophus in Polit’ inducere debent uxores proprias } ut sint familiares et beniuolae uxoribus praedictorum : \\\hline
3.2.9 & ca assi conmo dize el pho \textbf{ en el quinto libro delas politicas } nunca es menospreçiado el mesurado . & ne a subditis habeatur in contemptu : \textbf{ nam ut dicitur 5 Polit’ } non contemnitur \\\hline
3.2.9 & assi commo dize el philosofo \textbf{ en el quinto libro delas politicas . } muchos de los tiranos non solamente non son tenpdos mas quieren paresçertenprados & Immo ( quod peius est ) \textbf{ ut narrat Philosophus 5 Politic’ } multi tyrannorum \\\hline
3.2.9 & Ca assi commo dize el philosofo \textbf{ en el tercero libro delas politicas } mas durable es regnar sobre pocos que sobre muchos . & ø \\\hline
3.2.9 & Ca cuenta el philosofo \textbf{ en el quanto libro delas politicas } que commo vn Rey dexasse vna parte de su regno . & per usurpationem et iniustitiam . \textbf{ Recitat autem Philosophus 5 Polit’ } quod cum quidam Rex partem sui regni dimisisset , \\\hline
3.2.10 & e por auentura de paresçer tal . \textbf{ nchas cautelas tanne el philosofo en el quinto libro delas politicas delas quales quanto par tenesçe alo presente podemos tomar diez . } por las qualose esfuerça el tiranno de se mantener en su sennorio . & sed simulat se talem esse . \textbf{ Multas cautelas tangit Philosophus 5 Polit’ | ex quibus quantum ad praesens spectat , } possumus enumerare decem \\\hline
3.2.12 & que ninguno de los otros . \textbf{ por que assi commo prueua el philosofo en el quanto libro delas politicas . } qual quier cosa de maldat es & Tyrannis tamen est peruersior principatus : \textbf{ quia ut probat Philosophus 5 Polit’ } quicquid peruersitatis est \\\hline
3.2.13 & Ca cuenta el philosofo en el \textbf{ quanto libro delos politicas seis cosas } por que los sbraditos asecha alos tisanos ¶ & Narrat autem Philosophus \textbf{ 5 Polit’ sex causas , } quare subditi tyrannis insidiantur . \\\hline
3.2.13 & enxienplo desto pone el philosofo \textbf{ enel quinto libro delas polititas } do dice & de facili inuaduntur . \textbf{ Exemplum horum recitat Philosophus 5 Politicor’ } ubi habemus de Sardinapalo rege , \\\hline
3.2.13 & por quel tomne los tesoos Oude Dize el philosofo \textbf{ enel quinto libro delas politicas } que algunos veyendo las grandes ganançias & et accipiunt thesauros eius . \textbf{ Unde dicitur 5 Polit’ } quod quidam tyrannos inuadunt , \\\hline
3.2.14 & nin el prinçipado rreal \textbf{ Ca cuenta el phon enel quinto libro delas politicas tres maneras dela corrupçion dela tiranja } e dize que la tiranja corrope de si mismar coronpese desta çirana & quam rectus et regius principatus . \textbf{ Narrat autem Philosophus 5 Politicorum | tres modos corruptionis tyrannidis , } dicens , \\\hline
3.2.14 & E por ende dize el pho \textbf{ en el terçero libro delas politicas } que la tirama quanto mayor es tanto menos dura & corrumpitur et durare non potest . \textbf{ Ideo dicitur in Politi’ } quod tyrannis quanto intensior , \\\hline
3.2.15 & por que non se pongan alos peligros sobredichos \textbf{ anne el pho en el quànto libro delas politicas diez cosas } que saluna la poliçia & ne praedictis periculis exponatur . \textbf{ Tangit autem Philosophus 5 Polit’ decem } quae politiam saluant , \\\hline
3.2.15 & anne el pho en el quànto libro delas politicas diez cosas \textbf{ que saluna la poliçia } e el gouernamiento del regno & Tangit autem Philosophus 5 Polit’ decem \textbf{ quae politiam saluant , } et quae oportet facere Regem ad hoc \\\hline
3.2.15 & egualan se a vn grant mal \textbf{ assi commo dize el philosofo en las politicas bien } assi commo muchas pequanans despenssas se ygualan a vna & Nam multae modicae transgressiones \textbf{ ( ut ait Philos’ ) | aequantur uni magnae , } sicut multae paruae expensae \\\hline
3.2.15 & e deuen avn ser defendidos los males pequanos . \textbf{ La segunda cosa que guarda la poliçia } e el gouernamiento del regno es bien vsar & etiam modicae . \textbf{ Secundum praeseruans politiam } et regnum regium , \\\hline
3.2.15 & pol . \textbf{ bien vsar de los çibdadanos non solamente guarda la poliçia } e el gouernamiento derecho . & bene uti ciuibus \textbf{ non solum praeseruat politiam rectam , } sed etiam principatus ex hoc durabilior redditur , \\\hline
3.2.15 & que es puesto enlos omans guardan \textbf{ e saluna lo poliçia } e el gouernamiento dela çibdat & nam corruptiones longe secundum rem , \textbf{ prope autem secundum timorem politiam saluant : } ciues enim magis sunt subiecti Principi \\\hline
3.2.15 & uirtudessi non de fortaleza . \textbf{ Estos tales segunt el philosofo en el vii̊ libro delas politicas semeian al fierro } que mientra es en vso continuado esta luzio e claro . & de virtutibus aliis nisi de fortitudine , \textbf{ secundum Philosophum 7 Politicorum assimilantur ferro , } quod dum est in continuo exercitio claritatem habet , \\\hline
3.2.15 & que puede ser muy prouechosa la quarta cosa \textbf{ que salua la poliçia } es escusar las discordias & et antecessores sui obtinuerunt huiusmodi principatum , tanta cautela non magnam utilitatem habere videtur . \textbf{ Quartum autem quod politiam saluare videtur , } est cauere seditiones et contentiones nobilium ; \\\hline
3.2.15 & La quinta cosa \textbf{ que guardan la poliçia es catar con grant acuçia } en qual manera se han aquellos que honrro la real maiestad & ut dissoluatur regius principatus . \textbf{ Quintum , est diligenter aspicere , } quomodo se habeant , \\\hline
3.2.15 & ca ninguna cosa non salua tanto el regno \textbf{ e la poliçia } commo poner los bueons e los uirtuosos en las dignidades & Nihil enim adeo regnum conseruat \textbf{ et politiam saluat , } sicut praeficere homines bonos et virtuosos , \\\hline
3.2.15 & por la qual \textbf{ cosalo que mucho salua la poliçia } es que el Rey piensse con grant acuçia & et conferre eis dominia et principatus . \textbf{ Quare maxime saluatiuum politiae est , } regiam maiestatem considerare diligenter \\\hline
3.2.15 & e de su unar de muerte ¶ \textbf{ La vjͣ cosa que guarda la poliçia . } es non dara ninguer muy grant señorio & vel etiam capitali sententia condemnandi . \textbf{ Sextum est , nulli valde magnum dominium conferre . } Nam magna dominia \\\hline
3.2.15 & La vi jncosa que salua el regno \textbf{ e la poliçia es } que el Rey & ne repente constituatur aliquis in maximo principatu . \textbf{ Septimum saluans regnum et politiam , } est Regem siue principantem habere dilectionem \\\hline
3.2.15 & cosa que salua el regno \textbf{ e la poliçia es auer poderio } çiuilca & Octauum saluans regnum et politiam , \textbf{ est habere ciuilem potentiam . } Nam ( ut dicitur in Magnis moralibus ) \\\hline
3.2.15 & ca assi commo dize el philosofo \textbf{ en el quinto libro delas politicas } mayor uirtud es menester & est esse regem bonum et virtuosum . \textbf{ Nam ut dicitur 5 Politicorum , } maior virtus requiritur \\\hline
3.2.15 & e esto es lo que much salua el regno \textbf{ e la poliçia } si el Rey fuere bueno e uirtuoso & sic eos bonitate superet : \textbf{ hoc enim maxime saluabit regnum et politiam , } si Rex sit bonus et virtuosus , \\\hline
3.2.15 & en el bien comun del regno ¶ \textbf{ La xͣ cosa que salua la poliçia es } que el Rey sepa aquella poliçia e gouernamiento & quia intendet bono regni et communi . \textbf{ Decimum , est Regem non ignorare qualis sit illa politia } secundum quam principatur , \\\hline
3.2.15 & La xͣ cosa que salua la poliçia es \textbf{ que el Rey sepa aquella poliçia e gouernamiento } segunt el qual el & quia intendet bono regni et communi . \textbf{ Decimum , est Regem non ignorare qualis sit illa politia } secundum quam principatur , \\\hline
3.2.20 & en el primero libro de la rectorica \textbf{ e la quat catanne en el sexto libro delas politicas } cos son los fazedores delas leyes & quarum tres tanguntur 1 Rhet’ \textbf{ quarta vero tangitur 1 Polit’ . } Prima via sic patet . \\\hline
3.2.20 & assi commo dize el pho en el vij̊ . \textbf{ libro de las politicas } pues que assi es & sunt arbitrio iudicum committenda : \textbf{ quia ( ut dicitur 6 Politicorum ) } quanto utique minor inimicitia fuerit exequentibus iudicia , \\\hline
3.2.24 & que las palabras e las smones son a uoluntad . \textbf{ Et esse mismo pho dize en el primero delas politicas } que la palabra non es dada por natura . & voces et sermones dicit esse ad placitum , \textbf{ qui primo Politicorum ait , } sermonem nobis esse datum a natura . \\\hline
3.2.26 & Et por ende dize el philosofo \textbf{ en el quarto libro delas politicas } que non conuiene de apropar las comunidades & Ideo dicitur 4 Politicorum \textbf{ quod non oportet } adaptare politias legibus , \\\hline
3.2.27 & assi commo cuenta el philosofo \textbf{ en el primero libro delas politicas } dizia omero que cada vno podia fazer leyes a sus fijos e a so mugers . & dicebat Homerus \textbf{ quod unusquisque statuit } legis pueris et uxoribus : \\\hline
3.2.27 & Ca segunt dize el philosofo \textbf{ en el quarto libro delas politicas . } çerca las leyes deuen ser tomados dos cuydados . & et promulgatas custodire et obseruaret : \textbf{ quia secundum Philosophum 4 Politicorum } circa leges duplex cura esse debet : \\\hline
3.2.28 & que se contienen enla fisi ca . \textbf{ assi la sçiençia politica } que es del gouernamiento del regno & vult regulare et aequare humanos humores : \textbf{ sic scientia politica } quae est de regimine regni et ciuitatis , \\\hline
3.2.29 & o quales son de conssentir \textbf{ L philosofo en el terçero delas politicas demanda } si es meior de ser gouernado el Regno o la çibdat & et permittenda . \textbf{ Philosophus 3 Politicorum inquirit , } utrum regnum aut ciuitas sit \\\hline
3.2.29 & Et esto es lo que dize el philosofo \textbf{ en el tercero delas politicas } que mas de escoger es & quam Regi optimo Rege . \textbf{ Hoc est ergo quod ait Philosophus 3 Politicorum , } quod eligibilius est principari lege , \\\hline
3.2.29 & que diga entendimiento solo . \textbf{ Et por ende dize el pho en el terçero delas politicas } que aquel que manda & videtur dicere intellectum solum : \textbf{ ideo dicitur 3 Polit’ } quod qui iubet principari intellectum , \\\hline
3.2.29 & lo que dize el philosofo \textbf{ en el tercero delas politicas } que en el derecho gouernamiento & Propterea bene dictum est \textbf{ quod innuit Philosophus tertio Politicorum } quod in recto regimine principari \\\hline
3.2.30 & assi commo paresçe por el pho \textbf{ en el segundo libro delas politicas } do disputa contra socrates . & Immo ut patet \textbf{ per Philosophum 2 Politicorum } cum disputat contra Socratem , \\\hline
3.2.31 & e en uirtud \textbf{ e manda el pho en el segundo libro delas politicas } quando disputa contra ypodomio & sic eos superent bonitate et virtute . \textbf{ Quaerit Philos’ 2 Polit’ } cum disputat contra Hippodamum , \\\hline
3.2.31 & que paresçiessen ser mas aprouechosas e meiores . \textbf{ Mas el philosofo enel quarto libro delas politicas pone quatro razones } por las quales para pesçe & quae videantur esse magis proficuae et meliores . \textbf{ Adducit autem Phil’ | in Polit’ quatuor vias , } per quas videtur ostendi , \\\hline
3.2.31 & Et en la sçiençia del luchar o del torneamiento \textbf{ assi commo dize el pho en las politicas } muchas cosas & siue in arte luctatiua \textbf{ quae docet luctari , | ut innuit Philosophus } in Polit’ multa quae priores tradiderunt , \\\hline
3.2.31 & assi commo dize elpho \textbf{ en el segundo libro delas politicas . } es acostunbrar sea non obedesçer alas leyes & Nam assuescere inducere nouas leges \textbf{ ( ut innuit Philosophus 2 Pol’ ) } est assuescere non obedire legibus . \\\hline
3.2.31 & e qual es la soluçion della . \textbf{ Conuiene de saber que la ley politica sitiua } si fuere derecha conuiene que se raygͤ & de quae sito , \textbf{ sciendum quod lex positiua si recta sit , } oportet quod innitatur legi naturali , \\\hline
3.2.32 & qual de aquellos bienes es el meior . \textbf{ Et cuenta el philosofo enel terçero libro delas politicas } quariendo de el arar & quod bonorum illorum sit potius . \textbf{ Narrat quidem Philosophus 3 Politic’ volens diffinire } quid sit ciuitas , \\\hline
3.2.32 & Bien diches lo que dize el philosofo \textbf{ en el terçero libro delas politicas . } que la morada del lo gar & accipienda est eius notitia , \textbf{ benedictum est quod dicitur 3 Polit’ } quod unitas loci , communicatio connubiorum , compugnationis gratia , commutatio rerum , \\\hline
3.2.32 & Ca segunt el philosofo \textbf{ en las politicas } vno es el bien & et in toto regno . \textbf{ Nam secundum Philos’ in Polit’ } Idem finis est unius ciuis , \\\hline
3.2.32 & Mas si cada vn çibdadano deue ser uirtuoso \textbf{ segunt que dize el philosofo en las politicas . } segunt que alguno sobrepuia alos otros en poderio & Sed si quilibet ciuis debet virtuose se habere ; \textbf{ quia secundum Philos’ in Polit’ } secundum \\\hline
3.2.32 & conuiene que sea atal que biuna bien e uirtuosamente . \textbf{ Et por ende assi conmo dize el philosofo en el terçero libro delas politicas } mas es çibdadano & quod viuat bene et virtuose . \textbf{ Inde est ergo quod ait Philosop’ 3 Politicorum quod magis est ciuis abundans } in bonis operibus virtuosis , \\\hline
3.2.33 & e muchedunbre de bienes tenporales \textbf{ uenta el philosofo en el quarto libro delas politicas } que conuiene que sean tres partes dela çibdat . & et multitudinem ipsorum bonorum exteriorum . \textbf{ Quarto Politicorum ait Philosophus , } quod tres oportet \\\hline
3.2.33 & que nin sean muy ricos nin muy pobres . \textbf{ Et pone el pho en el quarto libro delas politicas quatro cosas } delas quales se pueden tomar quatro razonnes & ex multis personis mediis constitutus . \textbf{ Tangit autem Philosophus | 4 Politicorum , quatuor , } ex quibus sumi possunt quatuor viae , \\\hline
3.2.33 & delas quales se pueden tomar quatro razonnes \textbf{ que muestran que meior es la poliçia } o meior es el regno o la çibdat & ex quibus sumi possunt quatuor viae , \textbf{ ostendentes meliorem esse politiam , } vel melius esse regnum et ciuitatem , \\\hline
3.2.33 & e de ligero los omes obedescran ala razon . \textbf{ Et esto es lo que dize el philosofo en el quarto libro delas politicas } que por que lo medianero es muy bueno la possesion medianera es muy buena . & et de facili rationi obediunt . \textbf{ Hoc est quod dicitur 4 Politicorum } quod quoniam mediocre est optimum , \\\hline
3.2.33 & Ca assi commo dize el philosofo \textbf{ enł quarto libro delas politicas . } los pobres han grant enuidia alos ricos & Nam pauperes \textbf{ ( ut dicitur 4 Polit’ ) } maxime inuident diuitibus , \\\hline
3.2.34 & sobredichos la entençion del ponedor dela ley es enduzer los çibdadanos o uirtud . \textbf{ Ca en la derecha poliçia } assi commo dize el philosofo çerca el comienço del quarto libro delas polticas vna misma es la uirtud del buen çibdadano e del buen uaron . & intentio legislatoris est inducere ciues ad virtutem . \textbf{ In recta enim Politia } ( ut vult Philosophus ) \\\hline
3.2.34 & e obedesçer el Rey lea algunasiudunbre . \textbf{ Enpero leguntel pho en el quanto libro delas politicas } esto non es suidunbre mas es libertad . & sit quaedam seruitus . \textbf{ Sed secundum Philosoph’ 5 Politic’ } hoc non est seruitus , \\\hline
3.2.34 & si fueren penssadas las palabras del philosofo \textbf{ en el quato libro delas politicas . } El qual philosofo con para el Rey al regno & si considerentur verba Philosophi \textbf{ 4 Polit’ qui comparat Regem ad regnum , } sicut animam ad corpus . \\\hline
3.2.35 & e aguardar aquellas cosas \textbf{ que demanda la poliçia } o el gouernamiento del regno & et ad obseruandum \textbf{ ea quae requirit politia , } vel regimen regni , \\\hline
3.2.36 & por iustiçia non deue perdonar a ninguno . \textbf{ Por ende dize el philosofo en el vij̊ . libro delas politicas . } que el que bien obra & Nam iustus pro iustitia nulli parcere debet . \textbf{ Ideo dicitur 7 Politicorum } quod bene operans nulli parcit : \\\hline
3.3.1 & para gouernar el regno . \textbf{ Et prudencia politica o çiuil } para gouernar la çibdat . & distinguere quinque species prudentiae : \textbf{ videlicet prudentiam singularem , oeconomicam , regnatiuam , politicam siue ciuilem , et militarem . } Dicitur enim aliquis habere singularem \\\hline
3.3.1 & Et la regnatiua que muestra gouernar el regno . \textbf{ la quarta manera de sabiduria es dicha politica o çiuil . } Ca assi commo en el principe es neçessaria mayor sabiduria & videlicet particularem , oeconomicam et regnatiuam . \textbf{ Quarta species prudentiae dicitur | esse politica siue ciuilis . } Nam sicut in principante requiritur \\\hline
3.3.5 & que si el fuyesse de la batalla \textbf{ que polimidias otro cauallero le denostaria muy mal . Avn en essa misma manera } diomedes fue fecho atreuido cauallero . & Si ex pugna fugiam , \textbf{ Polydamas mihi redargutiones ponet . } Sic etiam et Diomedes \\\hline

\end{tabular}
