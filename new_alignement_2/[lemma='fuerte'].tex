\begin{tabular}{|p{1cm}|p{6.5cm}|p{6.5cm}|}

\hline
1.1.5 & que quiere dezer en aquellas faziendas \textbf{ o es aquellas batallas non son coronados los muy fuertes } mas los bien lidiantes & quod in Olimpidiadibus , \textbf{ idest in illis bellis et agonibus non coronantur fortissimi , } sed agonizantes : \\\hline
1.1.5 & mas los bien lidiantes \textbf{ ca los que son muy fuertes pueden lidiar . } Enpero si non lidiaren de fech̃o & sed agonizantes : \textbf{ qui enim fortissimi sunt , potentes agonixare , attamen si non actu agonizant , } non debetur eis corona . Oportet igitur \\\hline
1.1.5 & e la buena uentraa \textbf{ que omne ha por las buean s obras las obras guaues e fuertes de fazer se fazen muy delectables e plazenteras ¶ } pues que asi e sacanda hun omne conuiene ante connosçerlo su fin & et felicitate , \textbf{ quam ex ipsis consequimur . Cuilibet ergo homini , } ut agat bene , ex electione , et delectabiliter , \\\hline
1.1.6 & e enbargan la razon \textbf{ quando son grandes e fuertes . } Ca asi lo dize el philosofo en el terçero libro delas ethicas & sed huiusmodi delectationes sensibiles \textbf{ si vehementes sint , } rationem obnubilant , \\\hline
1.1.11 & o ouiere buena proporçion de los huessos e de los neruios \textbf{ e si fuere fuerte en el cuerpo . } Mas si ouiere las potençias & vel habeat proportionem ossium et neruorum , \textbf{ et sit robustus corporaliter . } Sed si habeat aequatas potentias , \\\hline
1.1.11 & assi que las potençias mas baxas sean subiectas ala razon e al entendemiento \textbf{ Et fuere sano e fuerte en la uoluntad } Et si estas potençias del alma fueren conpuestas e enfermosadas e honrradas de uirtudes e de buenas obras & ut quod inferiores potentiae subsint rationi ; \textbf{ et sit sanus , et fortis mente : } et si huiusmodi potentias habeat ornatas virtutibus , \\\hline
1.2.2 & por el bien e por el mal \textbf{ en quanto han razon de cosa guaue e fuerte . } Ca commo el bien & secundum se : irascibilis vero respicit bonum , \textbf{ et malum inquantum habent rationem difficilis , | et ardui . } Nam cum bonum \\\hline
1.2.2 & que siente non tornando los bienes e los males segt̃ \textbf{ si mas en quanto son fuertes e guaues Ca la fortaleza e la graueza son cosas } que nos enbargan mucho & sed ut habent rationem ardui , et difficilis : \textbf{ nam arduitas , } et difficultas potissime sunt repugnantia , et prohibentia , \\\hline
1.2.2 & segund el qual segnimos los bienes delectables por el entendimiento . \textbf{ Et acometemos los bienes fuertes de alcançar } assi commo es otro & Non ergo est alius appetitus intellectiuus , \textbf{ secundum quem prosequimur bona delectabilia per intellectum , et aggredimur bona ardua : sicut est alius appetitus sensitiuus , } secundum quem prosequimur \\\hline
1.2.3 & que son grandeza e alteza de coraçon \textbf{ son çerca de los bienes guaues e fuertes de alcançar . } Enpero de departidas maneras . & ut Fortitudo est circa passiones ortas ex malis futuris , Mansuetudo circa passiones ortas ex malis praesentibus . Magnificentia vero , \textbf{ et Magnanimitas sunt circa bona ardua , aliter et aliter : } quia Magnificentia est circa magna bona utilia , \\\hline
1.2.4 & Mas continente es dicho aquel que es tentado e passionado \textbf{ e ha fuertes passiones } e tentaçiones en si . & qui passionatur \textbf{ et habet passiones fortes , } tamen continet se , \\\hline
1.2.4 & Ca por las tenticonnes \textbf{ e por las passiones fuertes } conmo quier que bien faga . & non ergo continens est plene virtuosus , \textbf{ quia propter passiones fortes licet bene agat , } non tamen est ei delectabile bene agere . Continentia ergo , et Perseuerantia sic accepta , quamuis sint quaedam bona dispositio , \\\hline
1.2.5 & conuiene que se faga sabiamente \textbf{ e iusta mente . fuerte mente . } e tenprada mente . & oportet quod fiat prudenter , \textbf{ iuste , | fortiter , } et temperate : \\\hline
1.2.5 & Et desi diremos dela fortaleza e dela tenprança mostrando e declarando \textbf{ que conuiene alos Reyes e alos prinçipes de seer fuertes e tenprados . } Mas enpos esto todo tractaremos dela magnanimidat e dela magnfiçençia & declarantes \textbf{ quod contingit Reges et Principes esse fortes et temperatos . } Consequenter vero tractabimus de Magnitudine , et Magnificentia , \\\hline
1.2.10 & La ley manda fazer las obras de todas las uirtudes . \textbf{ Ca manda la ley obrar obras fuertes e obras tenpradas . } Et generalmente todas las obras & lex praecipit actus omnium virtutum . \textbf{ Praecipit enim lex operari fortia et temperata , } et uniuersaliter omnia quae dicuntur \\\hline
1.2.10 & Ca commo quier que el iusto legal faga essas mismas obras \textbf{ que faze el fuerte e el tenprado . } Enpero non las faze & nam licet eadem opera agat Iustus legalis , \textbf{ quae agit fortis , | et temperatus : } non tamen aget \\\hline
1.2.10 & en que las fazen los otros . \textbf{ Ca aquel que faze las obras fuertes } en quanto se delecta en ellas es dicho fuerte . & vel \textbf{ secundum eandem rationem formalem . Nam qui agit opera fortia , } quia delectatur in talibus , fortis est , et agens temperata , \\\hline
1.2.10 & Ca aquel que faze las obras fuertes \textbf{ en quanto se delecta en ellas es dicho fuerte . } Et el que faze las obras tenpradas & secundum eandem rationem formalem . Nam qui agit opera fortia , \textbf{ quia delectatur in talibus , fortis est , et agens temperata , } quia delectatur in ipsis , temperatus est . \\\hline
1.2.10 & por la ley . \textbf{ Mas el fuerte e el tenprado } e el acabado & ø \\\hline
1.2.10 & por las quales es honrrado¶ \textbf{ Mas si el fuerte e el tenprado se deleytan } en conplimiento de la ley & per se et primo delectatur in operibus conuenientibus virtutibus , quibus ornatur . \textbf{ Si autem delectatur in impletione legis , } hoc est ex consequenti , \\\hline
1.2.10 & Et desta diferenços se sigue la segunda . \textbf{ Ca commo el fuerte e el tenprado se deleite } segunt & Ex ista autem differentia sequitur secunda : \textbf{ nam cum fortis , | et temperatus delectetur in operibus talium virtutum , } secundum se virtutes illae perficiunt habentem eas , \\\hline
1.2.13 & que a dios non tema este \textbf{ non es fuerte } mas es loco e landio . & quod Deum non timeat , \textbf{ non est Fortis , } sed insanus . Spectat igitur ad fortem timere timenda , \\\hline
1.2.13 & mas es loco e landio . \textbf{ Et pues que assi es al fuerte pertenesçe temer las cosas } que ha de temer & non est Fortis , \textbf{ sed insanus . Spectat igitur ad fortem timere timenda , } et audere audenda . \\\hline
1.2.13 & sea çerca algun bien \textbf{ e cerca alguna cosa fuerte e graue } por que los periglos de las batallas son mas fuertes e mas espantables & Cum igitur virtus sit circa bonum , \textbf{ et difficile } ( quia difficiliora , \\\hline
1.2.13 & e cerca alguna cosa fuerte e graue \textbf{ por que los periglos de las batallas son mas fuertes e mas espantables } que los otros periglos . & et difficile \textbf{ ( quia difficiliora , } et terribiliora sunt pericula bellica , \\\hline
1.2.13 & que los otros periglos . \textbf{ Et ahun por que en los periglos delas batallas mas fuerte cosa es de repremer los temores } que de restenar las osadias . & ( quia difficiliora , \textbf{ et terribiliora sunt pericula bellica , } quam alia : \\\hline
1.2.13 & que de restenar las osadias . \textbf{ Otrosi por que en auiendo osadia non es tan fuerte } nin tan graue cosa acometer la batalla & et terribiliora sunt pericula bellica , \textbf{ quam alia : } et etiam quia in periculis bellicis difficilius est reprimere timores , \\\hline
1.2.13 & que en acometiendo los . \textbf{ Mas que los periglos delas batallas sean mas fuertes } e mas graues de sofrir & quae est virtus circa pericula , principalius est in reprimendo timores contingentes circa pericula talia , quam in moderando audacias circa ipsa . Sic etiam principalius est huiusmodi virtus in sustinendo pugnantes , \textbf{ quam in aggrediendo eos . Quod autem pericula bellica sint difficiliora ad sustinendum , } quam pericula alia : \\\hline
1.2.13 & en el tercero libro delas ethicas \textbf{ que prinçipalmente es dicho fuerte } aquel que non es temeroso cerca la buena muerte & Propter quod dicitur 3 Ethicorum , \textbf{ quod principaliter dicetur utique fortis , } qui impauidus est circa bonam mortem , \\\hline
1.2.13 & e enlas enfermedades . \textbf{ aquel es fuerte } que non es temeroso . & Sed adhuc et in mari , \textbf{ et in aegritudinibus intimidus est , } qui est fortis . Amplius licet Fortitudo sit circa pericula bellica , \\\hline
1.2.13 & que non es temeroso . \textbf{ Et por ende al fuerte parte nesçe non temer } qual si quier periglo & ø \\\hline
1.2.13 & por tres razones ¶ \textbf{ La primera por que acometer parte nesçe al mas fuerte . } Mas sufrir parte nesçe al mas fiaco & triplici via venari potest . Primo , \textbf{ quia aggrediendum , est fortioris : sustinere autem , debilioris est . Aggrediens autem comparatur ad alios , sicut ad debiliores } sed sustinens , \\\hline
1.2.13 & Mas el que sufre es conparado alos otros \textbf{ assi commo amas fuertes . } Et por ende mas guaue cosa es de esforçarse contra los mas fuertes & sicut ad fortiores : \textbf{ Difficilius est autem inniti contra fortiores , } quam contra debiliores . \\\hline
1.2.13 & assi commo amas fuertes . \textbf{ Et por ende mas guaue cosa es de esforçarse contra los mas fuertes } que contra los mas flacos . & sicut ad fortiores : \textbf{ Difficilius est autem inniti contra fortiores , } quam contra debiliores . \\\hline
1.2.13 & Pues que assi es fincanos de declarar \textbf{ en qual manera podemos fazer anos mismos fuertes } Pues que assi es deuen dos notar & restat ergo declarandum , \textbf{ quomodo possumus facere nos ipsos fortes . Notandum ergo , } quod licet virtus opponatur duabus malitiis , \\\hline
1.2.13 & que el temor \textbf{ Et por ende si quisieremos fazer fuertes a nos mismos conuiene de inclinar nos ante ala osadia } que al temor ¶ & inter audaciam , \textbf{ et timorem : declinandum est ad audaciam , quae minus repugnat Fortitudini , } quam timor , \\\hline
1.2.13 & ¶ Lo terçero ya declaramos \textbf{ en qual manera podemos fazer a nos mismos fuertes . } Ca mayormente nos podemos fazer fuertes & est circa pericula alia , \textbf{ et in moderando audacias , et in aggrediendo pugnantes . Tertio declaratum fuit , quomodo possumus facere nos ipsos fortes : } quia maxime hoc faciemus , declinando magis ad audaciam , \\\hline
1.2.13 & en qual manera podemos fazer a nos mismos fuertes . \textbf{ Ca mayormente nos podemos fazer fuertes } si mas nos inclinaremos ala osadia & et in moderando audacias , et in aggrediendo pugnantes . Tertio declaratum fuit , quomodo possumus facere nos ipsos fortes : \textbf{ quia maxime hoc faciemus , declinando magis ad audaciam , } quae non tantum repugnat fortitudini , \\\hline
1.2.14 & quando alguno temiendo uerguença \textbf{ e quariendo ganar honrra acomete alguna cosa fuerte e espantable . } Onde dize el philosofo & quando aliquis timens verecundiam , \textbf{ et volens honorem adipisci , | aggreditur aliquod terribile , } unde ait Philosophus , \\\hline
1.2.14 & que segunt esta manera de fortaleza \textbf{ aquellos son dichos muy fuertes que quieren gauar honrra entre aquellas gentes . } Entre las quales los temerosos son desonrrados & secundum hanc Fortitudinem fortissimi videntur esse \textbf{ apud illas gentes , } apud quas timidi inhonorati sunt , fortes vero honorantur . Hoc modo \\\hline
1.2.14 & alli do pone tal enxienplo \textbf{ que ector era fuerte en esta manera } que temie ser denostado de polimas . & ø \\\hline
1.2.14 & Et por ende acometie cosas espatables \textbf{ e estaua fuerte enla fazienda } por que dezie si el fuxiese Polimas & qui timens increpationes Polydamantis , \textbf{ aggrediebatur terribilia . } Dicebat enim quod si fugeret , Polydamas , \\\hline
1.2.14 & para le dezir muchos denuestos . \textbf{ Et por ende temiendo qual denostaria su contrario era fuerte . } Et ahun pone otro enxienplo el philosofo en esse mismo logar & qui erat ex parte aduersa , \textbf{ primum sibi increpationes imponeret . Sic etiam ( ut idem Philosophus recitat ) Diomedes , } hoc modo fortis erat . \\\hline
1.2.14 & e dize \textbf{ que di omnedes enesta manera de fortaleza era fuerte . } Ca dize que si non lidiase reziamente su contrario ector & primum sibi increpationes imponeret . Sic etiam ( ut idem Philosophus recitat ) Diomedes , \textbf{ hoc modo fortis erat . } Dicebat enim quod si non strenue bellaret , \\\hline
1.2.14 & Et en esta manera muchos de los troyanos temiendo a ector \textbf{ que era cabdiello dela caualłia eran fuertes . } Ca dize el philosofo & sed timore poenae , \textbf{ vel aliqua necessitate ductus aggreditur pugnam . Hoc modo multi Troianorum timentes Hectorem fortes erant . } Nam Philosophus ait Hectorem statuisse , \\\hline
1.2.14 & que costrino sus conpannas a fortaleza . \textbf{ por que fuesen fuertes . } Ca commo el estudiese en sus naues & Hoc autem modo quidam Dux dicitur exercitum suum coegisse ad Fortitudinem . Nam , \textbf{ cum nauigiis , } et cum toto suo exercitu transfretaret , \\\hline
1.2.14 & e delas faziendas acometen muchas cosas \textbf{ que semeian espantables e fuertes . } Ca assi commo dize vegeçio & multa aggrediuntur , \textbf{ quae videntur esse terribilia . } Nam ( ut dicit Vegetius in libro De re militari ) , \\\hline
1.2.14 & mas reçiamente las cosas guaues \textbf{ por que sean iudgados e tenidos por fuertes . } Mas enpero por esto non son fuertes propia mente . & aggrediuntur aliqua terribilia , \textbf{ ut iudicentur esse fortes . } Non tamen propter hoc proprie fortes sunt , \\\hline
1.2.14 & por que sean iudgados e tenidos por fuertes . \textbf{ Mas enpero por esto non son fuertes propia mente . } Ca quando la batalla es afincada e continuada & ut iudicentur esse fortes . \textbf{ Non tamen propter hoc proprie fortes sunt , } nam cum adeo inualescit bellum , \\\hline
1.2.14 & en que estan algunas vezes acometen la batalla \textbf{ por que sean iudgados por fuertes . } Enpero tales non son fuertes propiamente . & ita \textbf{ ut iudicentur fortes : } non tamen proprie tales fortes existunt , \\\hline
1.2.14 & por que sean iudgados por fuertes . \textbf{ Enpero tales non son fuertes propiamente . } Ca la fortaleza udadera acomete la batalla & ut iudicentur fortes : \textbf{ non tamen proprie tales fortes existunt , } quia fortitudo vera aggreditur pugnam \\\hline
1.2.14 & e toman osadia para lidiar \textbf{ Pues que assi es tales semeian fuertes } por que acometen la batalla esperando dela victoria & quandam spem victoriae , \textbf{ et accipiunt quandam audaciam bellandi . Tales ergo fortes esse videntur , } quia aggrediuntur pugnam , sperantes de victoria , \\\hline
1.2.14 & que sufran alguna cosa de mal \textbf{ Enpero non son propreamente fuertes . } Ca si fallaren alguna resistençia o fortaleza & et non credentes aliquid mali pati : \textbf{ non tamen vere fortes sunt , } quia si inueniant resistentiam , \\\hline
1.2.14 & assi commo si algunos morasen en setenturon \textbf{ e fuesen fuertes e osados . } Et otros morasen en el meridiano & ut cum aliquis ignorans fortitudinem aduersarii , bellatur . \textbf{ Ut puta si habitantes in septentrione sunt fortes , } et audaces , \\\hline
1.2.14 & Et si alguons acometiesen la batalla con los setenteronales \textbf{ que son fuertes cuydando } que son meridionales & et audaces , \textbf{ in meridiano vero sunt debiles , } et timidi , aggredientes pugnam \\\hline
1.2.14 & que son dichas \textbf{ por que sepan en qual manera han de ser fuertes . } Et en commo pueden lidiar con sus enemigos . & Reges ergo et Principes licet has maneries fortitudinum scire debeant , \textbf{ ut cognoscant qualiter populus suus fortis est , } et quomodo possunt cum aduersariis bellare : \\\hline
1.2.14 & Et en commo pueden lidiar con sus enemigos . \textbf{ Enpero ellos deuen ser fuertes de fortaleza uirtuosa } por que non pongan la su gente & et quomodo possunt cum aduersariis bellare : \textbf{ ipsi tamen debent esse fortes fortitudine virtuosa , } ut non exponant suam gentem periculis bellicis , \\\hline
1.2.15 & que assi commo la fortaleza es medianera entre los temores e las osadias . \textbf{ Por que aquel que teme en todas las cosas non es fuerte } nin aquel que es osado en todas las cosas & et audacias : \textbf{ quia qui omnia timet , | non est fortis , } nec qui omnia audet : \\\hline
1.2.15 & nin aquel que es osado en todas las cosas \textbf{ otrosi non es fuerte . } Mas aquel es fuerte & non est fortis , \textbf{ nec qui omnia audet : } sed qui timet timenda , \\\hline
1.2.15 & otrosi non es fuerte . \textbf{ Mas aquel es fuerte } que teme & nec qui omnia audet : \textbf{ sed qui timet timenda , } et audet audenda . \\\hline
1.2.15 & Et deuedes saber \textbf{ que alguas delas delectaconnes corporales son fuertes } e algunan sson flacas . & Si ergo Temperantia reprimit delectationes sensibiles , videndum est circa quas delectationes sensibiles habet esse . \textbf{ Delectationum autem sensibilium quaedam sunt fortes , } quaedam sunt debiles , \\\hline
1.2.15 & de todos los sesos . \textbf{ Empero las mas fuertes delecta connes son en el gusto e en el tannimiento } que del veer e del oyr e del oler la qual cosa podemos prouar & et circa sensibilia omnium sensuum contingat nos delectari : \textbf{ tamen fortiores delectationes sunt | secundum gustum et tactum , } quam secundum visum , \\\hline
1.2.15 & assi conmo la fortaleza mas conuiene con la osadi \textbf{ Et si nos quisieremos fazer nos fuertes } mas auemos aser osados e temerosos . & Sicut ergo Fortitudo magis conuenit cum audacia ; \textbf{ et si volumus esse fortes , } debemus magis esse audaces , \\\hline
1.2.16 & mas non es assi dela fortaleza . \textbf{ por que cada vn acometemiento de batallas non nos faze fuertes } saluo si fuessen en aquellas batallas derechureras . & ut simus temperati . Non autem sic est de fortitudine : \textbf{ nam non quaelibet aggressio bellorum facit nos fortes , } nisi bella ista sint iusta . \\\hline
1.2.16 & por non ser tenprados \textbf{ que por non ser fuertes . } Et por ende si el Rey non fuere fuerte & quare exprobrabilius est nos esse intemperatos , \textbf{ quam non esse fortes . } Si ergo Regem non esse virilem , \\\hline
1.2.16 & que por non ser fuertes . \textbf{ Et por ende si el Rey non fuere fuerte } e non fuer firme en el coraçon es de deno star & quam non esse fortes . \textbf{ Si ergo Regem non esse virilem , } et non esse constantem animo est exprobrabile , \\\hline
1.2.17 & o que non tomar los bienes agenos . \textbf{ Ca guardar omne lo suyo propio non es cosa fuerte por si . } Ca cada hun omne es naturalmente inclinado a amar asi mismo & vel quam aliena non surripere . Nam custodire propria \textbf{ secundum se non est difficile : } quia unusquisque naturaliter inclinatur \\\hline
1.2.17 & que ala osadia . \textbf{ Et nos fazemos a nos mismos fuertes declinando ala osadia } assi que seamos mas osados que temerosos . & Nam sicut quia fortitudo plus opponitur timori quam audaciae , \textbf{ facimus nosipsos fortes , declinando ad audaciam ; } ita quod potius plus audeamus , \\\hline
1.2.22 & en tal manera \textbf{ que avn el fuerte se ha conueniblemente en los otros periglos . } En essa misma manera la magranimidat prinçipalmente es cerca las honrras & ex consequenti autem erat circa pericula alia , \textbf{ quod fortis | etiam in aliis periculis decenter se habeat . } Sic magnanimitas principaliter est circa honores , \\\hline
1.2.24 & si esto faze \textbf{ por que se delecta en tales obras este es dicho fuerte . } Mas si esto faze & si hoc facit , \textbf{ quia delectatur in talibus actibus , fortis est . } Si vero hoc agit , \\\hline
1.2.29 & Et desto pongamos enxenplo . \textbf{ Ca si alguno en tanto fuese fuerte } e estremado sobre los otros en fortaleza en tal manera & quia si notabiliter a medio recederet , non in hoc appareret verax , sed derisor . \textbf{ Ut si aliquis adeo esset fortis et strenuus , } quod constaret aliis quod contra centum bellare posset : \\\hline
1.2.31 & Et pues que assi es sera aquel acebadamente te prado \textbf{ empero non sera fuerte mas temeroso } nin sera largo nin liberal & Erit ergo ille perfecte temperatus , \textbf{ non tamen erit fortis , | sed timidus : } nec erit largus , \\\hline
1.2.32 & la qual cosafazen los persseuerates \textbf{ o los que se tienen contra las fuertes passiones } la qual cosa faz en los continentes . Mas son assi castigados en el appetito e en el desseo & quod faciunt perseuerantes : \textbf{ vel contra fortes passiones se tenent , quod faciunt continentes : } sed sunt ita castigati in appetitu , \\\hline
1.2.34 & que assi es en quanto alguno es continente \textbf{ e en quanto ha passiones fuertes } non ha acabado uso de razon nin de uirtud . & quia contra passiones se tenet , ratione pugnae quam sentit , non est ei delectabile benefacere . Quandiu ergo aliquis est continens , et quandiu habet passiones fortes , \textbf{ non habet perfectum usum rationis et virtutis : } attamen vincendo passiones illas , disponitur \\\hline
1.3.3 & prinçipalmente ha de ser çerca los bienes diuinales e comunes . \textbf{ Otrosi sera fuerte por que ante pone el bien comunal bien propreo } e avn non dubdara de poner la persona a muerte & secundum Philosophum 4 Ethic’ magnificentia potissime habet esse circa diuina , \textbf{ et communia . Erit fortis ; quia cum bonum cumune proponat bono priuato , } non dubitabit \\\hline
1.3.6 & e non son uerdaderos \textbf{ por que conuiene alos Reyes de ser fuertes } e non ser locos . & non veridici : \textbf{ docent enim Reges non esse fortes . } Nam qui omnia audent , \\\hline
1.3.6 & e non teme ninguna cosa \textbf{ este non es dicho fuerte } mas es dicho loco & et nihil timent , \textbf{ ut dicitur 1 Magnorum moralium , } non est fortis , \\\hline
1.3.11 & e cerca el temor . \textbf{ por que aquel es dicho fuerte } que es osado en las cosas que deue e es temeroso en las cosas & et timorem . \textbf{ Nam ille est fortis , } qui audet audenda , \\\hline
1.4.3 & de ser tem̃osos e de flacos coraçones \textbf{ mas conuiene les de ser fuertes e de grandes coraçones } Por que conmo los negoçios e los fech̃d & Tertio non decet eos esse timidos \textbf{ et pusillanimes , | immo fortes et magnanimos : } quia cum negocia respicientia totum regnum , \\\hline
1.4.3 & son grandes e altos \textbf{ por ende conuiene les aellos de ser fuertes e de grandes coraçones ¶ } Lo quarto non conuiene a ellos de ser escassos & sint magna et ardua , \textbf{ oportet eos esse fortes | et magnanimos . } Quarto detestabile est ipsos esse illiberales . Nam supra cum de virtutibus tractabatur , \\\hline
2.1.4 & para las obras cotidianas e de cada dia \textbf{ mas non es cosa fuerte de veer } que la casa sea establesçida de muchas perssonas . & quia est communitas naturalis constituita propter opera diurnalia et quotidiana . \textbf{ Quod autem oporteat domum ex pluribus constare personis , videre non est difficile . } Nam cum domus \\\hline
2.1.9 & Por la qual cosa commo estas tales cobdiçias de luxias \textbf{ si fueren fuertes entenebrezcan la uoluntad e çiegun e la razon e el entendemiento } si non es cosa conuenible a todos los çibdadanos & Quare cum huiusmodi concupiscentiae \textbf{ ( si fortes sint ) obnubilent mentem , } et rationem percutiant ; si indecens est omnibus ciuibus nimis vacare venereis , \\\hline
2.1.12 & para ser el omne sano conuiene \textbf{ que aya la nata fuerte } por que pueda arredrar dessi las cosas quel enpesçen & quare sicut ad esse sanum requiritur , \textbf{ quod quis habeat naturam fortem , } ut possit nociua expellere , \\\hline
2.1.17 & por el esforçamiento dela calentraa del uientre de la madremas pueden guardar las ceraturas \textbf{ e fazer las mas fuertes } ¶La segunda razon & magis possunt conseruare suos foetus , \textbf{ et eos perfectiores faciunt . } Secunda via ad inuestigandum hoc idem , \\\hline
2.2.4 & que de los fijos alos padres \textbf{ e es mas fuerte e mas afincado¶ } La segunda razon & incipiunt eos diligere . Diuturnior est ergo amor parentum ad filios , \textbf{ quam econuerso : } quare fortior et vehementior . Secunda via ad inuestigandum hoc idem , sumitur ex certitudine prolis . \\\hline
2.2.15 & e los de nuruega de vannar los sus fijos en los trios muy frios \textbf{ por que los fagan muy fuertes } Enpero deuemos entender & quod apud aliquas Barbaras nationes consuetudo est in fluminibus frigidis balneare filios , \textbf{ ut eos fortiores reddant . Attendendum est tamen , } quod cum dicimus pueros paruos assuescendos esse ad hoc vel ad illud , \\\hline
2.2.15 & Et pues que assi es \textbf{ por que los moços sean mas fuertes } e mas rezios deuen les defender & secundum Philosophum septimo Politicorum , facit ad robur corporis . \textbf{ Ut ergo pueri robustiores fiant , sunt a ploratu illo cohibendi . } Cum distinguimus aetates filiorum per septennia , \\\hline
2.2.16 & segunt el departimiento delas ꝑssonas . \textbf{ Ca algunos son mas fuertes ent cuerpo entdeçimo año } que otros en el seyto deçimo . & secundum diuersitatem personarum . \textbf{ Nam aliqui sunt robustiores corpore in duodecim annis , } quam alii in sedecim . Ideo \\\hline
2.2.16 & e de grado en grado amas altos trabaios \textbf{ e amas fuertes mouimientos . } Ca el trebeio dela pella & debent gradatim assuescere ad ulteriores labores , \textbf{ et ad fortiora exercitia . } Ludus enim pilae \\\hline
2.2.16 & e los mouimientos del cuerpo \textbf{ deuen ser mas altos e mas fuertes } que en el primero . & secundum Philosophum videntur esse debita exercitia in iuuenibus . In secundo tamen septennio sunt ulteriora exercitia assumenda quam in primo . \textbf{ Ad hoc tamen in tali aetate , } eo \\\hline
2.2.16 & non deuen tomar obras de caualłia \textbf{ nin otras obras fuertes . } Et por ende el pho enł viij̊ libro delas politicas dize & eo \textbf{ quod nimis sit tenera , non sunt assumenda opera militaria nec opera ardua . } Unde Philosophus 8 Polit’ ait , \\\hline
2.2.16 & quanto al cuerpo del vi̊ año fasta xuiij . \textbf{ Et que deuen tomar del xiiij año adelante trebeios mas fuertes que enl primero setenario . } finca de demostrar & ut iuuenes sint bene dispositi quantum ad corpus , \textbf{ a septimo usque ad quartumdecimum annum debent fortiores labores assumere , | quam in primo septennio . } Restat videre , \\\hline
2.2.17 & Ca todos los trabaios que tomaron en los xiiiij años passados deuen ser liuianos e ligeros . \textbf{ Et destonçe adelante deuen tomar trabaios mas fuertes } Et esto dize elpho enł viij̊ libro delas politicas & et quasi debiles : \textbf{ ex tunc autem assumendi sunt labores fortiores . } Nam et Philosophus 8 Poli’ \\\hline
2.2.17 & que fasta los . xiiij años los moços deuen ser acostunbrados a trabaios ligeros \textbf{ mas dende adelante se deuen acostunbrar a trabaios mas fuertes . } Et en tanto que segunt el philosofo desde los . xiiij̊ años & quod usque ad quartumdecimum annum pueri assuescendi sunt ad labores leues : \textbf{ sed deinde debent assumere labores fortes . Adeo enim } secundum ipsum a quartodecimo anno assuescendi sunt pueri ad labores fortes , \\\hline
2.2.17 & Et en tanto que segunt el philosofo desde los . xiiij̊ años \textbf{ se deuen acostunbrar los mocos a trabaios fuertes . } Assi commo al vso dela lucha & secundum ipsum a quartodecimo anno assuescendi sunt pueri ad labores fortes , \textbf{ ut ad exercitationem luctatiuam , } vel ad aliquam aliam exercitationem similem exercitationi bellicae ; \\\hline
2.2.17 & quando lo ha tal qual demanda el su ofiçio \textbf{ la qual cosa non puede ser sin fuerte trabaio de su cuerpo . } ¶ Et pues que assi es commo todos aquellos que quieren beuir uida çiuil & quale requirit officium militare . \textbf{ Quod sine forti exercitatione corporis esse non potest . } Cum ergo omnes volentes viuere vita politica , \\\hline
2.2.17 & ¶ Et pues que assi es commo todos aquellos que quieren beuir uida çiuil \textbf{ conuiene les de sofrir alguas uegadas fuertes trabaios } por defendemiento dela tierra . & oporteat \textbf{ aliquando sustinere fortes labores pro defensione reipublicae : } omnes volentes viuere tali vita a quartodecimo anno ultra sic debent assuefieri ad aliqua officia robusta , \\\hline
2.2.17 & Et por ende desde los xiiij̊ annos adelante \textbf{ deuen se acostunbrar a fuertes trabaios } assi que si viniere tienpo & aliquando sustinere fortes labores pro defensione reipublicae : \textbf{ omnes volentes viuere tali vita a quartodecimo anno ultra sic debent assuefieri ad aliqua officia robusta , } quod si aduenerit tempus \\\hline
2.2.18 & e de non quedar de se vsar de trabaiar enłbso delas armas \textbf{ por que el mouimiento conuenible del cuerpo faze el cuerpo mas fuerte } e mas rezio & nec omnino inexercitatos esse circa armorum usum . \textbf{ Exercitatio enim corporalis debita reddit corpus robustius , } ut facilius duriciem armorum sustinere possit . \\\hline
2.3.14 & es de dar serudunbre legal de ley puesta por los omes \textbf{ segunt la qual los flacos e los vençidos deuen seruir alos fuertes e alos beçedores . } Ca decho eslegal & et quasi positiuam , \textbf{ secundum quam debiles | et victi seruirent victoribus } et potentibus . Est enim iustum legale , \\\hline
2.3.14 & assi commo \textbf{ por que son fermosos e fuertes enl cuerpo } much mas dereches de sentir esto en los bienes del alma & quia excedunt in bonis corporis , \textbf{ ut quia sunt pulchri et fortes ; } multo iustius est hic diffiniri in anima , \\\hline
2.3.14 & que los bienes del alma e los biens dedem̊ \textbf{ por que la ley diesse iuyzio fuerte } e de cosa çierta . & ø \\\hline
2.3.15 & que las partes \textbf{ que son mas çeranas dela fuerte } mas abonden & ø \\\hline
3.1.7 & que biuen de rapina \textbf{ mas fuert s̃ paresçen las fenbras } que los mallos calicuydaremos & et in auibus viuentibus \textbf{ ex rapina ferociores esse videntur foeminae quam mares : } nam si consideramus aues ipsas viuentes ex raptu , \\\hline
3.1.12 & e sean rezios e esforcados de coraçon \textbf{ e fuertes e ualientes de cuerpo } por que es neçessaria cautella e sabiduria en las batallas & secundum tria quae requiruntur ad bellum . Homines enim bellatores decet esse mente cautos \textbf{ et prouidos : } corde viriles et animosos : \\\hline
3.1.12 & e ayan de dar grandes colpes \textbf{ conuieneles de auer fuertes honbros e fuertes rennes } para sofrir la pesadura delas armas & Nam cum bellantes oporteat diu sustinere armorum pondera , et dare magnos ictus , \textbf{ expedit eos habere magnos humeros et renes ad sustinendum armorum grauedinem , } et habere fortia brachia ad faciendum percussiones fortes : \\\hline
3.1.12 & para sofrir la pesadura delas armas \textbf{ e conuiene les de auer fuertes braços } para fazer fuertes colpes . & expedit eos habere magnos humeros et renes ad sustinendum armorum grauedinem , \textbf{ et habere fortia brachia ad faciendum percussiones fortes : } mulieres igitur \\\hline
3.1.12 & e conuiene les de auer fuertes braços \textbf{ para fazer fuertes colpes . } Et por que las mugers esto non pueden auer & expedit eos habere magnos humeros et renes ad sustinendum armorum grauedinem , \textbf{ et habere fortia brachia ad faciendum percussiones fortes : } mulieres igitur \\\hline
3.1.19 & ca toda passion e toda ferida \textbf{ quanto mas fuerte es } tanto mas destruyel dela sustançia . & quia talia ad mortem ordinantur , \textbf{ eo quod omnis passio magis facta abiiciat a substantia . } Quarto intromisit se de distinctione iudicantium . \\\hline
3.2.3 & ca quanto la uirtud es mas ayuntada \textbf{ e vna en ssi tanto es mas fuerte en ssi que si fuesse esꝑzida } assi commo se declara en el libro de los prinçipios & Nam quanto virtus est magis unita , \textbf{ fortior est seipsa dispersa , } ut declarari habet in libro de Causis , \\\hline
3.2.3 & si todo el pode rio çiuil \textbf{ que es en muchs prinçipantes fuesse ayuntado en vn prinçipante e vn sennor mas fuerte seria . } Et aquel prinçipe & Quare si tota ciuilis potentia , \textbf{ quae est in pluribus principantibus , congregaretur in uno Principe , efficacior esset ; } et ille principans propter abundantiorem potentiam melius posset politiam gubernare . \\\hline
3.2.8 & en el capitulo dela fortaleza \textbf{ entre aquellos son los omes muy fuertes } entre los quales los fuertes omes son muy honrrados . & Nam ( ut dicitur 3 Ethicorum capitulo de fortitudine ) \textbf{ apud illos sunt fortes , } apud quos maxime honorantur : \\\hline
3.2.8 & entre aquellos son los omes muy fuertes \textbf{ entre los quales los fuertes omes son muy honrrados . } Bien assi entre aquellos son los sabios & apud illos sunt fortes , \textbf{ apud quos maxime honorantur : } apud illos sunt sapientes et boni , \\\hline
3.2.36 & La primera que sean bien fechores e liberales e francos ¶ \textbf{ La segunda que sean fuertes e de grant coraçon . la terçera que sean eguales e derechureros¶ } La primera se prueua assi . & et liberales . Secundo fortes \textbf{ et magnanimi . Tertio aequales et iusti . Primum autem sic patet . Nam vulgus non percipit } nisi sensibilia bona , \\\hline
3.2.36 & e aueres la segunda cosa \textbf{ para que los Reyes sean ama dos del pueblo es que deuon ser fuertes e de grandes coraçones } esponiendo se assi & et honorat beneficos in pecunia , et liberales . \textbf{ Secundo ut Reges amentur in populo , | debent esse fortes et magnanimi , } ponentes \\\hline
3.2.36 & e por el defendemiento del regno . \textbf{ Ca el pueblo muchama los fuertes } e alos de grandes coraçones & et defensione regni . \textbf{ Nam populus valde diligit fortes } et magnanimos , \\\hline
3.2.36 & que amamos a aquellos que nos pueden fazer bien saluandonos e librado nos . \textbf{ Et por ende amamos los fuertes } e los de grandes coraçones ¶ & qui possunt nobis benefacere nos saluando \textbf{ et liberando , ideo diligimus fortes } et cordatos . Tertio , ut Reges diligantur a populo , \\\hline
3.3.1 & Et otrossi por el agrauiamiento de las perssonas flacas podemos dezir \textbf{ que assi commo el fuerte pertenesçe prinçipalmente de saber bien en obras de batallas . } Et de si a esse mismo pertenesçe & et ex oppressione debilium personarum , dicere possumus quod sicut ad fortem \textbf{ principaliter spectat bene se habere in opere bellico , } ex consequenti vero spectat \\\hline
3.3.2 & Ca en todas las partes son sabidores \textbf{ e algunos fuertes de coraçon . } Enpero en la mayor parte & ut tam prudentia quam animositate participent . Aduertendum tamen circa talia , documenta accipienda esse ut in pluribus . Nam in omnibus partibus sunt aliqui industres , \textbf{ et aliqui animosi : } ut plurimum tamen soli propinqui animositate deficiunt , \\\hline
3.3.2 & por que non pueden sin grant osadia acometer los puercos monteses \textbf{ e les otros fuertes venados . } Et por ende tales son de grant coraçon & etiam aprorum admittendi sunt ad huiusmodi opera : \textbf{ quia non sine magna audacia contingit aliquos inuadere apros . } Sunt ergo tales animosi \\\hline
3.3.2 & Por ende los que non temen los periglos de los puercos monteses \textbf{ e de las otras bestias fuertes . } señal es que non temerien lasbatallas de los enemigos . & quam pugnare cum hoste . \textbf{ Nam non timentes aprorum pericula , } signum est eos non timere hostium bella . \\\hline
3.3.3 & que los mançebos fuessen usados e acostubrados del . xiiij° . \textbf{ año adelante a fuertes trabaios } assi conmo a los trabaios de la caualleria & ø \\\hline
3.3.3 & e escoger varones vsados \textbf{ e lidiadores escogidos e fuertes . } Visto en qual hedat son de acostunbrar a obras de batalla & ut Rex aut Princeps debeat committere bellum , \textbf{ viros exercitatos et bellatores strenuos debet assumere . } Viso in qua aetate assuescendi sunt \\\hline
3.3.3 & que los temerosos e de flacos coraçones . \textbf{ Otrossi los omnes fuertes de cuerpo } e duros de mienbros & quam timidos . Rursus , \textbf{ homines fortes et duros corpore , } quia potentiores sunt viribus , sunt magis eligendi ad opus bellicum . Amplius cum videamus aliqua animalia bellicosa , aliqua vero timida : homines similiores animalibus bellicosis , \\\hline
3.3.3 & e los muslos bien rezios e los laçertos \textbf{ que son cuerdas de los muslos bien fuertes e bien rezios } Ca segunt el philosofo en el . viij° . libro de las . & sunt durities carnis , \textbf{ et compactio neruorum , masculorum , et lacertorum . } Nam secundum Philos’ \\\hline
3.3.3 & e los neruios espessos e firmes \textbf{ e los laçertos fuertes e rezios } estos tales son fuertes de cuerpo & duri carne habentes compactos neruos , et lacertos , sunt virosi \textbf{ et fortiores corpore , sunt aptiores ad pugnam . Signa vero conformantia nos animalibus bellicosis , sunt magnitudo extremitatum , et latitudo pectoris . Videmus enim leones animalium fortissimos habere magna brachia , } et latum pectus . Quando igitur in homine videmus , \\\hline
3.3.3 & e los laçertos fuertes e rezios \textbf{ estos tales son fuertes de cuerpo } e menos sotiles de coraçon & duri carne habentes compactos neruos , et lacertos , sunt virosi \textbf{ et fortiores corpore , sunt aptiores ad pugnam . Signa vero conformantia nos animalibus bellicosis , sunt magnitudo extremitatum , et latitudo pectoris . Videmus enim leones animalium fortissimos habere magna brachia , } et latum pectus . Quando igitur in homine videmus , \\\hline
3.3.3 & Ca veemos que los leones \textbf{ que son mas fuertes } que todas las otras animalas & erectus ceruice , \textbf{ durus in carne , } compactus in neruis et musculis , \\\hline
3.3.4 & Ca commo toda la hueste sea puesta a periglos de muerte en la batalla \textbf{ nunca ninguno es fuerte de coraçon } nin buen lidiador & Nam cum tota operatio bellica exposita sit periculis mortis , \textbf{ nunquam quis est fortis animo } et bonus bellator , \\\hline
3.3.4 & que sin miedo en los periglos de la muerte . \textbf{ Ca pertenesçe al fuerte e al buen lidiador } assi commo dize el philosofo en el terçero libro de las Ethicas & sit impauidus circa pericula mortis . Spectat enim ad fortem \textbf{ et ad bonum bellatorem , } ut innuit Philosophus 3 Ethic’ \\\hline
3.3.4 & Ca assi commo dize el philosofo en el tercero libro de las Ethicas \textbf{ entre aquellos son los varones muy fuertes entre los quales los fuertes son muy honrrados . } Mas entre todas las cosas & ut dicitur 3 Ethic’ \textbf{ apud illos sunt viri fortissimi , | apud quos honorantur fortes . } Inter caetera autem quae reddunt hominem bellicosum , \\\hline
3.3.8 & ally son de catar \textbf{ e de escoger mas fuertes guarniçiones } e son de fazer mas anchas carcauas . & aut ibi debet \textbf{ per modicum tempus existere , } non oportet tantas munitiones expetere . Modum autem , \\\hline
3.3.8 & e alta de siete . \textbf{ Mas si la fuerça de los enemigos paresciere mas fuerte } conuiene de fazer las carcauas mas anchas & fossa debet esse lata pedes nouem , alta septem . \textbf{ Sed si aduersariorum vis acrior imminet , } contingit fossam ampliorem et altiorem facere ita , \\\hline
3.3.9 & La segunda quales son mas osados . \textbf{ La terçera quales son mas fuertes en sofrir los daños e las neçessidades de la batalla } Lo quarto quales son mas rezios e mas duros en el cuerpo . & ex qua parte sunt plures bellatores . Secundo , \textbf{ qui sunt magis exercitati . Tertio , | qui sunt fortiores in sustinendo necessitates , et incommoda . Quarto , } qui sunt robustiores , \\\hline
3.3.9 & e mas arteros para lidiar . \textbf{ Lo sexto quales son mas osados e mas fuertes de coracon . Estonçe el cabdiello de la hueste mesurado } e en viso segunt & qui sunt industriores , \textbf{ et sagaciores mente . Sexto , | qui sunt audaciores , et viriliores corde . } Et tunc dux sobrius , \\\hline
3.3.10 & aquella parte es la meior para lidiar . \textbf{ s sienpre la uirtud ayuntada e ordenada es mas fuerte } que quando esta desparzida e desordenada . & est pars potior ad bellandum . \textbf{ Semper virtus unita fortior est seipsa dispersa } et confusa . \\\hline
3.3.10 & assi es con grant sabiduria es de escoger el alferez \textbf{ assi que sea fuerte de cuerpo } e firme de coraçon & Cum magna igitur diligentia est vexillifer eligendus , \textbf{ ut sit corpore fortis , } animo constans , \\\hline
3.3.10 & si quisieren ser buenos lidiadores \textbf{ deuan ser fuertes } e rezios en los sus cuerpos & Quare cum pedites , \textbf{ si debent boni bellatores existere debeant esse fortes viribus , } proceri statura , scientes proiicere hastas \\\hline
3.3.10 & que al que deue ser ante puesto a los peones en la batalla \textbf{ deue ser fuerte en el cuerpo . } grande en su estado & qui in pugna pedicibus praeponitur \textbf{ esse fortis viribus , } procer statura , \\\hline
3.3.10 & mucho mas deue ser prouado en las armas ligero de cuerpo \textbf{ e fuerte en los mienbros } aquel que deue ser ante puesto a los caualleros . & et procer corpore , \textbf{ et fortis viribus } qui est equitibus praeponendus : \\\hline
3.3.11 & e las qualidades de las carreras \textbf{ e los fuertes passos de los caminos } e los departimientos de las carreras & et qualitates viarum , \textbf{ compendia et diuerticula , } et montes , \\\hline
3.3.11 & que ayan cauallos muy ligeros \textbf{ e muy fuertes los quales vayan en la delantera e en la çaguera e en las costaneras } que demuestren e descubren los assechos e las çeladas & habentes equos veloces et fortes ; \textbf{ qui ante et a tergo , | et a dextra et a leua percurrant , } illustrantes et discooperientes insidias , \\\hline
3.3.12 & sin el cuento de los lidiadores \textbf{ que fazen el az son de guardar algunos buenos et fuertes lidiadores fuera del az } que puedan acorrer a aquella parte & etiam aduertendum , \textbf{ quod in qualibet acie praeter numerum pugnatorum constituentium aciem , reseruandi sunt aliqui strenui bellatores extra ipsam aciem qui possint } ad illam partem succurrere \\\hline
3.3.12 & asi commo demanda la manera de la batalla . \textbf{ Lo segundo que los mas fuertes lidiadores sean puestos en aquellas partes de la az } en las quales mas ayna se puede ronper & ut requirit bellum committendum . Secundo , \textbf{ ut probiores bellatores in illis partibus aciei apponantur , } in quibus magis potest confundi et perforari acies . \\\hline
3.3.14 & Et pues que assi es todas aquellas cosas \textbf{ que fazen los enemigos ser fuertes } para lidiar con sus enemigos & nobis est nociuum : \textbf{ et econuerso . Quaecunque igitur reddunt hostes fortiores } ad resistendum bellantibus , \\\hline
3.3.14 & Mas quanto partenesçe a lo presente podemos contar siete cosas . \textbf{ por las quales los enemigos son mas fuertes contra sus enemigos . } Lo primero es si fueren las azes ordenadas & possumus septem enumerare , \textbf{ per quae hostes sunt fortiores contra impugnantes . } Primum est , \\\hline
3.3.14 & Ca commo la uertud ayuntada \textbf{ assi commo dicho es dessuso sea mas fuerte } que quando esta esparzida & Nam cum ipsa virtus unita , \textbf{ ( ut etiam supra tangebatur ) fortior sit se ipsa dispersa : } si hostes sint bene uniti \\\hline
3.3.14 & en el az commo deuen \textbf{ si los acometieren sus enemigos mas fuertes seran de vençerLo . } segundo que faze los enemigos mas fuertes & si inuadantur , \textbf{ difficilius euincuntur . } Secundum quod reddit hostes fortiores ad resistendum , \\\hline
3.3.14 & si los acometieren sus enemigos mas fuertes seran de vençerLo . \textbf{ segundo que faze los enemigos mas fuertes } para lidiar es el logar & difficilius euincuntur . \textbf{ Secundum quod reddit hostes fortiores ad resistendum , } est locus . \\\hline
3.3.14 & Assi el logar conuenible \textbf{ e bueno fazelos mas fuertes } para se defender . & reddit eos debiliores ad bellandum : \textbf{ sic locus aptus facit eos potentiores ad resistendum . Tertium , } est ipsum tempus . \\\hline
3.3.14 & e en qual manera los lidiadores deuen acometer sus enemigos . \textbf{ Ca commo en las siete maneras contadas sean los enemigos mas fuertes } quando son las maneras contrarias & quomodo et qualiter bellantes suos hostes inuadere debeant . \textbf{ Nam cum septem modis enumeratis hostes fortiores existant ; } cum modo opposito se habent , \\\hline
3.3.15 & prinçipalmente enbia su uirtud a la parte derecha . \textbf{ Assi que la parte derecha en las animalias es mas fuerte en mouer } e mas apareiada a mouimiento . & quod est in animali principium motus , principalius influit in partem dextram ita , \textbf{ quod pars dextra in animalibus fortior est in mouendo , } et aptior ad motum : \\\hline
3.3.15 & mas reziamente \textbf{ e faz mas fuerte colpe . } Enpero maguer que podamos folgar tan bien sobre la parte derecha & quo vibrato vehementius mouet aerem , \textbf{ et fortius ferit . Licet enim tam } secundum partem dextram quam secundum sinistram possumus quiescere \\\hline
3.3.15 & nin les semeiare bueno de acometer la batalla . \textbf{ Et esto por que los enemigos son mas fuertes que ellos } e non pueden estar contra ellos & nec videatur bonam pugnam committere , \textbf{ eo quod hostes sint fortiores , } et non possumus illis resistere . \\\hline
3.3.17 & a las mayores fortalezas \textbf{ e a los mas fuertes adarues del castielloo de la çibdat çercada . } Et por cueuas deuen venir & sed procedendum est \textbf{ ad maiores munitiones et ad maiora moenia castri , vel ciuitatis obsessae , } et per similes vias subterraneas est similiter faciendum circa ea , \\\hline
3.3.18 & m muchas uegadas contesçe \textbf{ que algunas fortalezas çercadas son fundadas sobre pennas muy fuertes } o son cercadas de agua & Contingit autem pluries , \textbf{ munitiones aliquas obsessas | super lapides fortissimos esse constructas , } vel esse aquis circumdatas , \\\hline
3.3.19 & Ca por razon del fierro \textbf{ que ponen y . ha muy fuerte } et muy . dura fruente para ferir & ideo appellatur aries , \textbf{ quia ratione ferri ibi appositi durissimam habet frontem ad percutiendum . } Huiusmodi autem trabs funibus , \\\hline
3.3.19 & e a manera de carnero se tira atras . \textbf{ Et despues da muy fuerte colpe en los muros de la fortaleza çercada } assi que los ronpen et los quebrantan . & et ad modum arietis se subtrahit : \textbf{ et postea fortiter muros munitionis obsessae percutit } et disrumpit . Cum enim per huiusmodi trabem sic ferratam multis ictibus percussus est murus ita , \\\hline
3.3.19 & e fazese este artificio \textbf{ quando las tablas gruessas e fuertes son bien iuntadas e dobladas } o se fazen dos tablados & Fit autem hoc , \textbf{ cum tabulae grossae et fortes optime conligantur , | et duplicantur , } siue fit duplex tabulatum , \\\hline
3.3.20 & en tal façion de los castiellos e de las fortalezas \textbf{ por los quales las fortalezas son mas fuertes } e mas graues de tomar . & ne faciliter impugnentur . Sunt autem quinque in huiusmodi aedificatione consideranda : \textbf{ per quae munitiones fortiores existunt , } et difficiliores ad capiendum . \\\hline
3.3.20 & e mas graues de tomar . \textbf{ Lo primero son las fortalezas mas fuertes } e son mas graues para las conbatir & et difficiliores ad capiendum . \textbf{ Primo quidem fortificantur munitiones , } et sunt difficiliores ad bellandum ex natura loci . Secundo ex angularitate murorum . Tertio ex terratis . \\\hline
3.3.20 & por la natura del logar las çibdades \textbf{ e las fortalezas son mas fuertes } si son assentadas en pennas o en logares & Ex natura quidem loci , \textbf{ urbs et munitiones fortiores existunt } si editae sint in praeruptis rupibus , \\\hline
3.3.20 & assi que en tal logar sean fundados \textbf{ que por en aquel assentamiento sean mas fuertes . } Et si non han uagar de fundar las fortalezas de nueuo & ut in tali loco aedificentur , \textbf{ quod ex ipso situ fortiores existant . } Vel si non vacat munitiones de nouo aedificare , \\\hline
3.3.20 & o si ouieren poderio es de fazer tal fortaleza \textbf{ por que la natura del logarsea mas fuerte } e mas graue de conbatir . & si adsit facultas quaerenda est munitio talis , \textbf{ quae ex ipsa natura loci fortior existat , } et difficilior ad impugnandum . \\\hline
3.3.20 & Lo segundo las çibdades \textbf{ e las fortalezas son mas fuertes } para non ser vençidas & et difficilior ad impugnandum . \textbf{ Secundo urbes et munitiones sunt difficiliores } ad impugnandum ex angularitate murorum . \\\hline
3.3.20 & por que se pueda la fortaleza meior defender . \textbf{ Lo tercero que faze la fortaleza mas fuerte } para se non poder entrar & ut munitio faciliter defendi possit . \textbf{ Tertium , quod reddit munitionem difficiliorem ad capiendum , } dicuntur esse terrata , \\\hline
3.3.20 & que sean bien feridas e bien tapiadas \textbf{ e fazen la fortaleza mas fuerte . } Por la qual cosa mucho cunple fazer tales muros & etiam turres ex terra facere , \textbf{ si bene condensetur : } propter quod non est inconueniens construere huiusmodi muros ex terra depressata ; \\\hline
3.3.20 & en ssi todas las piedras del engeñio sin grand daño . \textbf{ Lo quarto que faze las fortalezas mas fuertes son torres e menas e cadahalsos . } Ca sienpre son de fazer en los muros torres e cadahalsos & quia tunc quasi absque laesione suscipiet lapides emissos a machinis . \textbf{ Quartum autem quod facit munitiones fortiores sunt turres , | et propugnacula . } Nam in ipsis muris construendae sunt turres , \\\hline
3.3.20 & si contesçiesse que los enemigos pusiessen fuego a las puertas . \textbf{ Lo quinto que faze las fortalezas mas fuertes e peores } para las entrar es quando las carcauas son muy anchas & si contingeret ipsum ab obsidentibus esse appositum . \textbf{ Quintum quod facit munitiones } magis inacessibiles , \\\hline
3.3.20 & las quales carcauas deuen ser lleñas de agua \textbf{ si sse podiere fazer Et pues que assi es en estas maneras sobredichas son las fortalezas mas fuertes } e peores de tomar . & et fortiores : \textbf{ est latitudo , } et profunditas fossarum : \\\hline
3.3.22 & Et esta saeta enuiada \textbf{ por muy fuerte ballesta al . } engeñio muchas vegadas le quema . & quem ignem cum stupa conuolutum bellatores antiqui Incendiarium vocauerunt . \textbf{ Huiusmodi autem sagitta per ballistam fortem emissa usque ad machinam , } multotiens succendit ipsam . \\\hline
3.3.22 & e por saetas \textbf{ que lançen fuego fuerte } e por otros engeñios que lançen piedras contra los de fuera & et per homines de nocte latenter emissos , \textbf{ et per sagittam deferentem ignem fortem , } et per machinas alias emittentes lapides , \\\hline
3.3.22 & por la dureza de la cabeça es llamado carnero \textbf{ et contra este carnero se puede fazer vn fierro coruo dentado de dientes muy fuertes e muy agudos } e atado con fuertes cuerdas & Dicebatur enim , trabem ferratam percutientem muros munitionis obsessae propter duritiem capitis vocari Arietem . \textbf{ Contra hoc autem constituitur quoddam ferrum curuum dentatum dentibus fortissimis , | et acutis , } et ligatum funibus , \\\hline
3.3.22 & et contra este carnero se puede fazer vn fierro coruo dentado de dientes muy fuertes e muy agudos \textbf{ e atado con fuertes cuerdas } con el qual prenden & et acutis , \textbf{ et ligatum funibus , } cum quo capitur caput arietis , \\\hline
3.3.23 & e vençer a sus enemigos . \textbf{ Lo primero es fuego fuerte } que ellos llaman & Possumus tamen , \textbf{ quantum ad praesens , } decem enumerare , \\\hline
3.3.23 & Et lançandolos assi en las naues quebrantan se los cantaros \textbf{ e aquel fuego fuerte ençiende } e quema la naue . & ø \\\hline
3.3.23 & por que de muchas partes se pueda quemar la naue . \textbf{ Et entonçe deuen acometer muy fuerte batalla contra los enemigos } por que se non puedan acorrer & ut ex multis partibus possit nauis succendi ; \textbf{ et cum proiiciuntur talia , | tunc est contra nautas committendum durum bellum , } ne possint currere ad extinguendum ignem . Secundo ad committendum marinum bellum multum valent insidiae . \\\hline
3.3.23 & nin ha poder de lidiar . lo . vij° . \textbf{ suelen avn los marineros auer coruos de fierro muy fuertes } e quando veen & et quodammodo inutilior ad pugnandum . \textbf{ Septimo consueuerunt e iam nautae habere uncos ferreos fortes , } ut cum vident se esse plures hostibus , \\\hline

\end{tabular}
