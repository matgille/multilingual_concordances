\begin{tabular}{|p{1cm}|p{6.5cm}|p{6.5cm}|}

\hline
1.1.1 & quod modum procedendi in hac scientia oportet \textbf{ esse figuralem et grossum . } Prima via sumitur ex parte materiae , & e en esta sçiençia conujene \textbf{ que sea figural e gruesa , } ¶ la primera rrazon se toma de parte dela materia \\\hline
1.1.6 & quare si constat \textbf{ eos habere mores seniles , } et vigere Prudentia , & por la qual cosa si fuer çierto \textbf{ qualos mançebos han costunbres de me nos } e han sabiduria e entendimiento \\\hline
1.1.11 & nec in pulchritudine corporis , sed animae . \textbf{ Non igitur quis credat se esse felicem , } si habeat aequatos humores , & nin en fortaleza del cuerpo mas del alma . \textbf{ Et pues que assi es non crea ninguno | que es bien auentra ado } si ouiere los humores egualados \\\hline
1.2.7 & quod polleat prudentia , et intellectu . \textbf{ Quot , et quae oporteat habere Regem , } si & ø \\\hline
1.2.15 & et tactus magis directe \textbf{ et immediate videntur ordinari ad conseruationem nostram : } ut delectabilia gustus & son mas derechamente \textbf{ e mas ayuntadamente ordenades alanr̃auida | e alanr̃a conseruaçion } assi commo las cosas \\\hline
1.2.19 & et minus non videantur \textbf{ diuersificare speciem , } et naturam rerum , & Mas commo en cada cosa \textbf{ mas e menos non fagan departimiento en la naturaleza } e en la semeiança delas cosas \\\hline
1.2.27 & et deficere , \textbf{ oportet ibi dare virtutem aliquam , } per quam dirigamur ad bene agendum , & e tal sesçer conuiene de dar y . \textbf{ alguna uirtud por la qual seamos enderesçados } abien obrar \\\hline
1.2.33 & et tales dicuntur \textbf{ habere virtutes purgatorias . Aliqui vero sunt } quodammodo iam assecuti similitudinem illam : & e tales son dichos auer uirtudes pgatorias . \textbf{ Mas otros algunos son que en algua manera han ya conssigo esta semeiança diuinal } e tales son dichos auer uirtudes de pgado coraçon . \\\hline
1.3.6 & in seipso contrahitur , \textbf{ et redditur immobilis . Quare si indecens est caput regni siue Regem esse immobilem et contractum , } indecens est ipsum timere timore immoderato . & et pierde el mouimiento . \textbf{ Et por ende si es cosa desconuenible | que la cabeca del regno o el Rey } sea tal que se non mueua \\\hline
1.4.1 & sunt digniora quam alia . \textbf{ Rursus decet eos esse magnanimos : } quia ( ut dicebatur & e alos prinçipes de ser magranimos \textbf{ e de grand coraçon } Ca assi commo es dicho dessuso \\\hline
1.4.6 & dispicientes alios , \textbf{ et credentes se esse super eos , } eo quod videant illos indigere bonis eorum . & despreçiando alos otros \textbf{ e cuydando que son mayores que ellos } por que veen \\\hline
2.1.4 & oportuit \textbf{ dare communitatem ciuitatis . } Communitas ergo ciuitatis esse videtur & conuiene de dar comunidat ala çibdat \textbf{ sobre la comunidat deluarrio . } Et por ende \\\hline
2.1.7 & ostendere \textbf{ qualis amicitia sit viri ad uxorem , } probat amicitiam illam esse secundum naturam : & quariendo mostrar \textbf{ qual es el amistança del uaron } a la muger prueua \\\hline
2.1.10 & Detestabile est ergo unum virum plures habere uxores : \textbf{ sed detestabilius est unam uxorem plures habere viros , } quia per hoc magis impeditur certitudo filiorum . & que vn ome aya muchas mugers . \textbf{ Et mucho mas de denostares | que vna muger aya muchos maridos } ca por esto se enbargaria mas la çertidunbre de los fijos . \\\hline
2.1.16 & ne possit bene speculari , \textbf{ et ne possit libere exequi actiones suas . } Nascentes ergo ex tali coniugio & por que non pueda bien entender \textbf{ e que non pueda faze sus obras libremente . } ¶ Et pues que assi es los que nasçen de tal casamiento \\\hline
2.1.24 & nisi per diuturna tempora sint experti , \textbf{ eas esse discretas , prudentes , et stabiles , } et non esse secretorum propalatiuas . & saluo a aquellas de que han prouado de luengot \textbf{ pon que son sabias e entendidas e estables en vn proponimiento } e que non son descobrideras delos secretos \\\hline
2.2.7 & et ab incolis illius terrae semper cognoscitur \textbf{ ipsum fuisse aduenam , } et non fuisse in illis partibus oriundus . & Mas luego son conosçidos de los moradores de aqual la tierra \textbf{ que son auenedizos } e que non nasçieron en aquella tierra . \\\hline
2.2.15 & unde Philosophus septimo Politi’ ait , \textbf{ quod mox expedit pueris paruis consuescere ad frigora . } Assuescere enim pueros ad frigora utile est ad duo . & Onde el philosofo en el septimo libro delas politicas \textbf{ dizeque luego conuiene alos mocos pequanos } de acostunbrar los alos frios \\\hline
2.3.4 & In ordine autem Uniuersi , \textbf{ prout requiri aedificium construendum , } sunt tria consideranda , & mas en la arden del mundo \textbf{ segunt que demanda la morada } que es de fazer son de penssar tres cosas \\\hline
2.3.12 & qui tantae sapientiae secularis praedicabatur , \textbf{ habuisse massaritias multas . } Non obstante enim quod terrae fertilissimae dominabatur , & que era de tan grand sabiduria del sieglo que auie greyes \textbf{ maguer que fuesse señor de tierra muy abondosa } en la qual auya muchͣs uiandas e de grand mercado . \\\hline
2.3.17 & Nam non omnes decet \textbf{ habere aequalia indumenta . } In tantis enim domibus & por que non conuiene que todos sean uestidos \textbf{ de eguales uestiduras caenta } grandescasas non solamente son legos mas avn aycłigos \\\hline
3.1.1 & gratia alicuius boni , \textbf{ oportet ciuitatem ipsam constitutam esse propter aliquod bonum . } Probat autem Philosophus primo Polit’ duplici via , & commo toda comunidat sea por graçia de algun bien . \textbf{ Conuiene que la çibdat sea establesçida por algun bien | Ca pruena el pho } enl primero libro delas politicas \\\hline
3.1.5 & quod semper oporteat ciuitatem \textbf{ ex propriis possessionibus habere omnia quae requiruntur ad vitam : } sed sufficit ciuitatem sic esse sitam , quod per mercationes , & para aquellas cosas \textbf{ que son menester ala uida | mas cunple } que assi sea la çibdat establesçida \\\hline
3.1.9 & non oporteret ciues \textbf{ omnes pueros reputare filios proprios . } Immo quia puerorum aliqui essent & assi comunes non conuernia \textbf{ que los çibdadanos cuydassen | que todos los moços fuessen sus fijos propreos } por que alguons de los moços son semeiantes \\\hline
3.1.12 & Hominis ergo est \textbf{ secundum debitam oeconomiam } et secundum debitam dispensationem ordinare domum et ciuitatem . & Et pues que assi es los omes \textbf{ a quien parte nesçe } de ordenar la casa \\\hline
3.1.17 & esse liberales et temperatos : \textbf{ non ergo bene dictum est quod ad bonum regimen ciuitatis sufficit ciues habere possessiones aequatas , } nisi aliquid determinetur & que los çibdadanos sean liberales e francos \textbf{ e por ende non es bien dicho | que a buen gouernamiento dela çibdat } cunple de ser las possessiones egualadas \\\hline
3.2.5 & omnino esse melius \textbf{ et dignius dominationem regiam et principatum ire per electionem } quam per haereditatem . & e mas digna cosa \textbf{ qua el senñorio real | e el prinçipado venga por elecçio } que non por heredamiento \\\hline
3.2.7 & et concordiam adinuicem : \textbf{ rursus nolunt eos esse magnanimos et virtuosos : } nec etiam volunt ipsos esse sapientes et disciplinatos . & que los çibdadanos ayan paz nin concordia entre ssi . \textbf{ Otrossi non quiere | que ellos se que de grandes coraçones e uirtuosos } e avn non quiere \\\hline
3.2.10 & volunt habere exploratores multos , \textbf{ ut si viderent aliquos ex populo machinari aliquid contra eos , } possint obuiare illis . & por que en muchͣs cosas le aguauian quieren auer muchs assechadores \textbf{ por que si vieren | que alguon ssele una tan contra ellos } que los puedan contradezer ante \\\hline
3.2.15 & quod expendunt , et quomodo possunt \textbf{ reddere rationem sui victus : } nam qui huiusmodi rationem non potest reddere , & et commo pueden dar razon de su uida \textbf{ e de comm̃ se mantienen } ca aquel que non puede dar razon desto señal \\\hline
3.2.20 & Nam in qualibet ciuitate oporteret \textbf{ esse aliquod praetorium ordinarium } ad quod causae reducantur & ca en cada vna çibdat conuiene \textbf{ que aya vna alcalłia otdinaria } ala qual deuen venir todos los pleitos \\\hline
3.2.25 & ut conuenimus cum animalibus aliis : \textbf{ sic dicitur esse ius naturale . } Ideo in Instituta , & siguiere la nuestra natura en quanto auemos conueniençia con las otras aian lias \textbf{ assi es dich derech natural . } Et por ende en la instituta del derecho natural \\\hline
3.2.29 & dirigere legem positiuam , \textbf{ et esse supra iustitiam legalem , } et non obseruare legem , & que enderesçe la ley positiua \textbf{ e que sea sobre la iustiçia legal } e qua non guarde la ley positiua \\\hline
3.2.33 & ex quibus sumi possunt quatuor viae , \textbf{ ostendentes meliorem esse politiam , } vel melius esse regnum et ciuitatem , & delas quales se pueden tomar quatro razonnes \textbf{ que muestran que meior es la poliçia } o meior es el regno o la çibdat \\\hline
3.3.1 & ad dignitatem militarem , \textbf{ nisi constet ipsum diligere bonum regni et commune , } et nisi spes habeatur & para dignidat de caualleria \textbf{ si non fueren çiertos | que el ama el bien del regno e el bien comun } e si non ouieren esperança \\\hline
3.3.6 & Et cum viderit magister bellorum \textbf{ aliquem non tenere ordinem debitum in acie , } ipsum increpet et corrigat : & Et quando vieren los caudiellos maestros de las batallas \textbf{ que alguno non guarda orden en la az } deuenle denostar e castigar \\\hline
3.3.16 & Contingit etiam aliquando aliquos \textbf{ inuadere aliquas munitiones eorum ; } propter quod eos oportet & Et algunas vezes çerca villas o castiellos o fortalezas . \textbf{ Et avn algunas vezes contesçe que algunos otros çercan sus villas o sus castiellos . } Por la qual cosa les conuiene de vsar de batalla defenssiua para se defender . \\\hline
3.3.22 & restat videre quomodo obsessi debeant \textbf{ obuiare impugnationi factae per lapidarias machinas . } Contra has autem quadrupliciter subuenitur . & Visto en qual manera auemos de contrallar a la batalla fecha \textbf{ por los engenios que lançan las piedras . } Et podemos dar contra los engeñios quatro maneras de acorro . \\\hline

\end{tabular}
