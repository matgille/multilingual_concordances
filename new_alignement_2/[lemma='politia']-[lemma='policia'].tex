\begin{tabular}{|p{1cm}|p{6.5cm}|p{6.5cm}|}

\hline
2.3.13 & Sed hoc accidit \textbf{ ex peruersitate politiae : } nam sicut in homine pestilente , & Mas esto contesçe \textbf{ por desordenança dela poliçia e dela çibdat . } Ca assi commo en el omne pestilençial e malo \\\hline
2.3.19 & Nam sicut decet ciues \textbf{ ut debitam politiam seruent } esse iustos legales , & ca assi conmo conuiene alos çibdadanos de ser iustos e legales \textbf{ para guardar su poliçia conueniblemente } assi conuiene alos sermient \\\hline
3.1.16 & statuens quomodo posset \textbf{ optime politia ordinari . } Dicebat autem , & que se entremi tio del ordenamiento dela çibdat \textbf{ establesçiendo en qual manera se podria ordenas muy bien la poliçia e la çibdat } ca dizia \\\hline
3.1.20 & Hippodami ergo opinionem recitauimus , \textbf{ quia in sua politia multas bonas sententias promulgauit : } aliqua tamen incongrue statuit . & e por ende contamos la opinion de ipodomio \textbf{ por que el en la su poliçia manifesto muchͣs bueanssmans . } Empo algunas cosas establesçio non conuenible mente . \\\hline
3.2.2 & Nam regnum aristocratia , \textbf{ et politia sunt principatus boni : } tyrannides , oligarchia , et democratia sunt mali . & que quiere dezer sennorio de buenos \textbf{ e la poliçia | que quiere dezer pueblo bien } enssenoreante son bueons prinçipados . \\\hline
3.2.2 & et dicit ipsum esse Politiam . \textbf{ Politia enim quasi idem est , } quod ordinatio ciuitatis quantum & e diz el poliçia \textbf{ por que poliçia es } assi commo ordenamiento bueno de çibdat \\\hline
3.2.2 & qui dominantur omnibus aliis . \textbf{ Politia enim consistit } maxime in ordine summi principatus , & enssennorea a todos los otros \textbf{ ca la poliçia esta mayormente en el ordenamiento del grant prinçipado } que es en la çibdat \\\hline
3.2.2 & Omnis ergo ordinatio , \textbf{ ciuitatis Politia dici potest . } Principatus tamen populi si rectus sit , & Pues que assi es todo ordenamiento de çibdat \textbf{ puede ser dicħ poliçia . } Enpero el prinçipado del pueblo si derecho es \\\hline
3.2.7 & quod sicut Regnum est optima \textbf{ et dignissima politia , } sic tyrannis est pessima : & Et esta razon tanne el philosofo çerca el comienço del quarto libro delas politicas . \textbf{ do dize que assi conmo el regno es muy buena et muy digna poliçia . } assi la tirania es muy mala \\\hline
3.2.7 & ( ut ibi dicitur ) \textbf{ quia tyrannis plurimum distat a politia , } idest a communi bono . & assi commo y dize el pho \textbf{ por quela tirama much se arriedra dela poliçia e del bien comun . } la segunda manera para prouar esto mismo se toma . \\\hline
3.2.15 & Tangit autem Philosophus 5 Polit’ decem \textbf{ quae politiam saluant , } et quae oportet facere Regem ad hoc & anne el pho en el quànto libro delas politicas diez cosas \textbf{ que saluna la poliçia | e el gouernamiento del regno } e dela çibdat \\\hline
3.2.15 & bene uti ciuibus \textbf{ non solum praeseruat politiam rectam , } sed etiam principatus ex hoc durabilior redditur , & pol . \textbf{ bien vsar de los çibdadanos non solamente guarda la poliçia | e el gouernamiento derecho . } Mas avn por esta razon el prinçipado \\\hline
3.2.15 & nam corruptiones longe secundum rem , \textbf{ prope autem secundum timorem politiam saluant : } ciues enim magis sunt subiecti Principi & que es puesto enlos omans guardan \textbf{ e saluna lo poliçia | e el gouernamiento dela çibdat } ca los çibdadanos son mas subiectos al prinçipe \\\hline
3.2.15 & ne repente constituatur aliquis in maximo principatu . \textbf{ Septimum saluans regnum et politiam , } est Regem siue principantem habere dilectionem & que adesora non sea ninguno puesto en muy grant senorio . \textbf{ La vi jncosa que salua el regno | e la poliçia es } que el Rey \\\hline

\end{tabular}
