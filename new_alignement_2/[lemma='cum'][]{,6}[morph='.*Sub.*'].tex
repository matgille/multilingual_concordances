\begin{tabular}{|p{1cm}|p{6.5cm}|p{6.5cm}|}

\hline
1.1.2 &  \textbf{ Cum ergo non requiratur tanta industria ad regendum seipsum , }  &  \textbf{ que commo non sea demandada | tanta sabiduria ya gouerna mj̊ }  \\\hline
1.1.2 &  \textbf{ Nam cum finis sit operationum nostrarum principium , }  &  \textbf{ Lo quarto de parte delas costunbres ¶ | Ca commo la fin sea comjenço delas nr̃as obras }  \\\hline
1.1.3 &  \textbf{ cum nunquam quis plene erudiatur , }  &  \textbf{ njnguno conplidamente puenda ser enssennado | sy non fueᷤ begniuolo }  \\\hline
1.1.3 &  \textbf{ Cum ergo in hoc libro intendatur , }  &  \textbf{ Pues que asy es commo en este libro ente damos demostrͣ }  \\\hline
1.1.5 &  \textbf{ Cum ergo nunquam contingat recte agere , }  &  \textbf{ pues que asy es com̃ nunca pueda omne bien | e derechamente obrar asy }  \\\hline
1.1.6 &  \textbf{ Cum enim corpus ordinetur ad animam , }  &  \textbf{ Ca el | cuerpoes ordenado al alma }  \\\hline
1.1.6 &  \textbf{ Cum enim corpus ordinetur ad animam , }  &  \textbf{ mas quedtalma . ¶ pues que asi es commo el cuerpo sea ordenado al alma }  \\\hline
1.1.7 &  \textbf{ qui cum nimis esset auidus auri }  &  \textbf{ el qual ero muy codiçioso de auerors }  \\\hline
1.1.7 &  \textbf{ cum tamen fame periret . }  &  \textbf{ Enpero muria de fanbre }  \\\hline
1.1.7 &  \textbf{ ex institutione Hominum , tum quia cum sint corporalia , }  &  \textbf{ por ordenamiento e estableçemiento | delons omes }  \\\hline
1.1.7 &  \textbf{ Nam cum felicitas sit bonum optimum , }  &  \textbf{ ¶ Por que commo la feliçidat | e la bien andança sea muy gñdvien }  \\\hline
1.1.7 &  \textbf{ Cum ergo anima sit potior corpore , }  &  \textbf{ siguese que commo el alma sea meior | que el cuerpo la feliçidat }  \\\hline
1.1.7 &  \textbf{ Immo ( cum ille sit Magnanimus , }  &  \textbf{ Et la razon es esta | que aquel es magnamimo }  \\\hline
1.1.7 &  \textbf{ Cum ergo finis maxime diligatur , }  &  \textbf{ deua ser muy desseada | a aquel que pone las un feliçidat }  \\\hline
1.1.7 &  \textbf{ cum non intendat principaliter bonum publicum , sed priuatum . }  &  \textbf{ asi que poniendo el prinçipe la su feliçidat | e la su bien andança en las riquezas corporales }  \\\hline
1.1.8 &  \textbf{ cum sit reuerentia exhibita }  &  \textbf{ por que es reuerençia fecha }  \\\hline
1.1.8 &  \textbf{ cum sufficiat ad hoc }  &  \textbf{ conmoabaste acanda vno }  \\\hline
1.1.8 &  \textbf{ nam cum finis maxime diligatur , }  &  \textbf{ Ca commo cada vn omne mucho ame la su fin }  \\\hline
1.1.9 &  \textbf{ cum ad diuersas partes diuulgare possit : }  &  \textbf{ e se estiende a muchas partes }  \\\hline
1.1.9 &  \textbf{ cum scientia sua falli non possit . }  &  \textbf{ Ca lascian de dios non puede resçebir enganno . | Et ahun dezimos mas adelante }  \\\hline
1.1.10 &  \textbf{ Cum igitur violenta non diu durent , }  &  \textbf{ non puede mucho durar¶ | pues que assi es que las cosas forçadas non pueden mucho durar }  \\\hline
1.1.10 &  \textbf{ cum sit violentum , }  &  \textbf{ que sienpre ha de durar ¶ }  \\\hline
1.1.10 &  \textbf{ Cum ergo Principatus se extendant adinuicem , }  &  \textbf{ pues que assi es commo los prinçipados | e los sennorios se estiendan }  \\\hline
1.1.10 &  \textbf{ cum non sit liberorum , }  &  \textbf{ Et por que el señorio por fuerça e por poderio çiuil commo non sea delons libres }  \\\hline
1.1.10 &  \textbf{ Nam cum felicitas sit finis omnium operatorum , }  &  \textbf{ e la bien andança | sea fin de todas las nuestras obras }  \\\hline
1.1.10 &  \textbf{ Nam , cum ut plurimum studuerit , }  &  \textbf{ Ca commo en la mayor parte non aya estudiando }  \\\hline
1.1.11 &  \textbf{ nam , cum ipse sit caput Regni , }  &  \textbf{ Ca commo el Rey sea cabesça de su Regno }  \\\hline
1.1.13 &  \textbf{ cum semper amor sit ad similes , et conformes , }  &  \textbf{ Et commo el amor sienpre sean los semeiables e acordables con el . }  \\\hline
1.1.13 &  \textbf{ cum possint transgredi , }  &  \textbf{ conmolos podiessen }  \\\hline
1.1.13 &  \textbf{ Cum ergo magnae virtuti debeatur magna merces , }  &  \textbf{ Et pues que assi es commo grant uirtud deua auer grant merçed }  \\\hline
1.1.13 &  \textbf{ Cum ergo bonum gentis sit diuinius , }  &  \textbf{ pues que assi es commo el bien comun | e el bien dela gente sea mas diuinal }  \\\hline
1.2.1 &  \textbf{ cum ergo natura sit determinata ad unum , }  &  \textbf{ ¶ Et pues que assi es commo la natura sea determimada a vna cosa }  \\\hline
1.2.2 &  \textbf{ Nam cum animalia sint supra inanimata , }  &  \textbf{ ca commo las aina las sean sobre las cosas | que non han alma }  \\\hline
1.2.2 &  \textbf{ Nam cum bonum secundum se dicat prosequendum , }  &  \textbf{ Ca commo el bien | por si diga tal cosa }  \\\hline
1.2.3 &  \textbf{ Nam cum subiectum virtutis sit , }  &  \textbf{ Ca commo el subiecto delas uirtudes sea o el entendimiento o la uoluntad o el appetito senssitiuo . }  \\\hline
1.2.5 &  \textbf{ Rursus cum contingat operari recte et non recte , }  &  \textbf{ Otrosi commo contesca de obrar derechamente | e non derechamente }  \\\hline
1.2.5 &  \textbf{ Cum ergo passiones quaedam impellant nos ad malum , }  &  \textbf{ Et pues que assi es commo algunas delas passiones nos mueuen a mal }  \\\hline
1.2.5 &  \textbf{ cum sit directiua omnium aliarum , }  &  \textbf{ por que es endereçadora e regladora de todas las otras ¶ }  \\\hline
1.2.5 &  \textbf{ cum fortitudo magis ordinetur ad bonum gentis , }  &  \textbf{ por que la fortaleza es mas ordenada al bien dela gente }  \\\hline
1.2.6 &  \textbf{ Cum enim Prudentia sit circa agibilia , }  &  \textbf{ Ca commo la pradençia aya de ser en las obras . }  \\\hline
1.2.12 &  \textbf{ cum aliquo casu leges obseruari non debeant , }  &  \textbf{ Ca algun caso ay en que se non deuen guardar las leyes }  \\\hline
1.2.12 &  \textbf{ cum omnia per regulam regulentur . }  &  \textbf{ que por la regla se reglan | e se egualan todas las cosas . }  \\\hline
1.2.12 &  \textbf{ Cum enim deceat regulam esse rectam et aequalem , }  &  \textbf{ Ca commo conuenga ala regla de ser derecha }  \\\hline
1.2.13 &  \textbf{ sed cum aggressi fuerint , }  &  \textbf{ mas quando la han acometida estan }  \\\hline
1.2.13 &  \textbf{ quia cum aegritudo sit aliquid in nobis existens , }  &  \textbf{ por que la enfermedat es alguna cosa | que esta en nos }  \\\hline
1.2.13 &  \textbf{ Cum ergo difficilius sit durare , et sustinere pericula illa }  &  \textbf{ Et pues que assi es commo sea mas | guaue cosa de endurar }  \\\hline
1.2.13 &  \textbf{ Cum ergo naturaliter tristia fugiamus , }  &  \textbf{ Et pues que assi es commo nos natural mente fuyamos dela tristeza }  \\\hline
1.2.14 &  \textbf{ ut cum aliquis non ut vitet opprobria , }  &  \textbf{ quando alguno non | por esquiuar denuestos }  \\\hline
1.2.14 &  \textbf{ et cum toto suo exercitu transfretaret , }  &  \textbf{ e con todas sus naues passase la mar }  \\\hline
1.2.15 &  \textbf{ qui , cum esset pultiuorax , orauit , }  &  \textbf{ quel feziese la garganta }  \\\hline
1.2.16 &  \textbf{ cum hoc sit magis voluntarium , }  &  \textbf{ destenpranca por que esto es mas de uoluntad }  \\\hline
1.2.16 &  \textbf{ cum eam sine periculo possit acquirere : }  &  \textbf{ tenpranca | commo puede ganar la tenpranca sin ningun periglo . }  \\\hline
1.2.16 &  \textbf{ qui cum esset totus muliebris , }  &  \textbf{ apalo que por que era todo mugeril | e dado a mugers }  \\\hline
1.2.16 &  \textbf{ quod , cum quidam Dux exercitus diu ei seruiuisset , }  &  \textbf{ Et acaesçio que vn prinçipe mucho su priuado que grant t p̃o le auia seruido e fiel mente . }  \\\hline
1.2.17 &  \textbf{ quia cum liberales maxime amentur , }  &  \textbf{ por que los liberales | e los francos son mas amados sobre todos los otros . }  \\\hline
1.2.18 &  \textbf{ Quare cum natura humana modicis contenta sit , }  &  \textbf{ por la qual razon commo la naturaleza de los omes se tenga | por pagada de pocas cosas }  \\\hline
1.2.18 &  \textbf{ Quare cum prodigus non sit amator pecuniae , }  &  \textbf{ por la qual razon commo el gastador non sea amador de los }  \\\hline
1.2.18 &  \textbf{ quis cum sit prodigus , }  &  \textbf{ qual quier gastador liberal e franco ¶ }  \\\hline
1.2.18 &  \textbf{ Cum ergo tanto deceat fontem habere os largius , }  &  \textbf{ ¶pues que assi es conmo tanto conuenga ala fuente auer la boca | mas ancha }  \\\hline
1.2.19 &  \textbf{ Cum ergo liberalitas non respiciat sumptus secundum se , }  &  \textbf{ ¶Pues que assi es commo la libalidat | non cate alas espenssas }  \\\hline
1.2.20 &  \textbf{ quia cum paruificus nihil faciat , }  &  \textbf{ que quando el parufico faze alguna cosa }  \\\hline
1.2.20 &  \textbf{ Cum ergo Rex sit caput regni , }  &  \textbf{ Et pues que assi es commo el Rey sea }  \\\hline
1.2.21 &  \textbf{ Cum enim ille sit magnificus , }  &  \textbf{ Ca commo aquel sea magnifico }  \\\hline
1.2.22 &  \textbf{ cum non multum reputemus ipsa . }  &  \textbf{ por que los non tenemos en mucho }  \\\hline
1.2.23 &  \textbf{ Nam cum talia inter exteriora bona computentur , }  &  \textbf{ Ca porque tales cosas commo estas son contadas | entroͤ los bienes de fuera }  \\\hline
1.2.23 &  \textbf{ cum sint regula aliorum , }  &  \textbf{ por que son regla de los otros }  \\\hline
1.2.24 &  \textbf{ Nam cum honor inter exteriora bona sit bonum excellens , }  &  \textbf{ entre los bienes de fuera | sea mas alto e meior bien }  \\\hline
1.2.25 &  \textbf{ cum absque humilitate virtutes haberi non possint . }  &  \textbf{ si non ouiessen humildat }  \\\hline
1.2.25 &  \textbf{ Cum ergo retrahere et impellere sint quodammodo opposita , }  &  \textbf{ Et por ende commo tirar nos | de aquello que la razon manda }  \\\hline
1.2.26 &  \textbf{ nam cum magnanimi sit tendere in magnum , }  &  \textbf{ Ca commo almagranimo pertenesca de yr | e entender en cosas grandes la magranimidat }  \\\hline
1.2.26 &  \textbf{ Quare cum distincta sit virtus haec ab illa , }  &  \textbf{ por la qual cosa commo esta uirtud | que es dicha humildança sea apartada dela magnanimidat }  \\\hline
1.2.26 &  \textbf{ quod cum virtus magis sit retrahens quam impellens , }  &  \textbf{ mas nos trahe e tira | que nos allega e esfuerca . }  \\\hline
1.2.27 &  \textbf{ Nam cum ira peruertat iudicium rationis , }  &  \textbf{ Ca por que la yr a | tristorna el iuyzio dela razon }  \\\hline
1.2.27 &  \textbf{ cum in eis maxime vigere debeat ratio et intellectus . Sicut enim videmus }  &  \textbf{ por que en ellos mayormente deue seer apoderada la razon e el entendemiento | que en otros ningunos }  \\\hline
1.2.27 &  \textbf{ cum hoc faciat mansuetudo , }  &  \textbf{ Et commo esto faga la manssedunbre }  \\\hline
1.2.28 &  \textbf{ Quare cum recta ratio dictet , }  &  \textbf{ Por la qual cosa commo la razon derecha }  \\\hline
1.2.29 &  \textbf{ Quare cum mendacium sit semper fugiendum , }  &  \textbf{ Por la qual cosa commo la mentira | por si misma sea mala deuemos foyr della }  \\\hline
1.2.29 &  \textbf{ cum tamen illis careant : }  &  \textbf{ commo quier que en ellos nen sean }  \\\hline
1.2.30 &  \textbf{ Quare cum in talibus contingat peccare , }  &  \textbf{ Por la qual cosa sientales cosas | contesçe de pecar }  \\\hline
1.2.30 &  \textbf{ Quare cum iocus immoderatus , vel inhonestus distrahat nos a bonis operibus , }  &  \textbf{ nin honesto nos parta | e nos tire delas buenas obras }  \\\hline
1.2.31 &  \textbf{ Cum ergo omnino manifestum sit , }  &  \textbf{ Et pues que assi es commo en todo en todo es manifiesto e prouado }  \\\hline
1.2.31 &  \textbf{ Sed cum non sit perfecta via }  &  \textbf{ que son ordenadas a aquella fin . | Mas commo non puede seer camino }  \\\hline
1.2.31 &  \textbf{ cum non possint se excusare }  &  \textbf{ e alos prinçipes | commo ellos non se puedan escusar }  \\\hline
1.2.32 &  \textbf{ cum quaedam praegnans esset , }  &  \textbf{ que commo fuesse vna mugier prenada }  \\\hline
1.2.34 &  \textbf{ Quare cum cabulia consilietur , }  &  \textbf{ Por la qual cosa commo esta uirtud | que es dicha eubolia conseie }  \\\hline
1.2.34 &  \textbf{ Sed cum continentia potior sit , }  &  \textbf{ Mas commo la continençia sea meior }  \\\hline
1.3.1 &  \textbf{ Sed cum hoc sciri non possit , }  &  \textbf{ Mas por que esto non se puede saber }  \\\hline
1.3.1 &  \textbf{ Sed cum sit quaedam virtus }  &  \textbf{ Mas por que ha de ser alguna uirtud entre la sanna e la mansedunbre }  \\\hline
1.3.2 &  \textbf{ cum sumantur respectu boni , }  &  \textbf{ Ca por que son tomadas }  \\\hline
1.3.2 &  \textbf{ Nam cum aliquid prius sit futurum , }  &  \textbf{ Ca commo algunan cosa primero sea futura de uenir }  \\\hline
1.3.3 &  \textbf{ Erit fortis ; quia cum bonum cumune proponat bono priuato , }  &  \textbf{ Otrosi sera fuerte por que ante pone el bien comunal bien propreo }  \\\hline
1.3.3 &  \textbf{ qui cum esset tyrannus , }  &  \textbf{ que commo fuesse tyrano }  \\\hline
1.3.3 &  \textbf{ Quare cum de ratione odii sit exterminare , }  &  \textbf{ por la qual cosa commo de razon dela mal querençia sea matar }  \\\hline
1.3.5 &  \textbf{ cum ergo humilitas moderet spem , }  &  \textbf{ Et por ende por que la humildat tienpra la esperança }  \\\hline
1.3.5 &  \textbf{ Nam cum Reges et Principes sint latores legum , }  &  \textbf{ e alos prinçipes . | Ca conmolos Rayes sean fazedores e conponedores delas leyes . }  \\\hline
1.3.5 &  \textbf{ cum talia sint bona excellentia et ardua , }  &  \textbf{ por que tales bienes son bienes mas sobrepiunates | e mas altos que los otros }  \\\hline
1.3.5 &  \textbf{ Cum ergo regium officium requirat hominem prudentem }  &  \textbf{ Et pues que assi es conmo el ofiçio de los Reyes | demande omne sabio }  \\\hline
1.3.7 &  \textbf{ Nam cum homo in communi non iniurietur nobis , }  &  \textbf{ Ca commo el omne en comun non faga iniuria nin tuerto a nos }  \\\hline
1.3.7 &  \textbf{ nam cum odium sit mali }  &  \textbf{ Ca commo la mal querençia sea algun mal segunt si }  \\\hline
1.3.7 &  \textbf{ Nam cum ira satietur , }  &  \textbf{ Porque commo la sanna se pueda fartar }  \\\hline
1.3.7 &  \textbf{ cum sit quid insatiabile . }  &  \textbf{ que se non farta . }  \\\hline
1.3.7 &  \textbf{ Cum ergo conditiones odii sint multo peiores , }  &  \textbf{ Et pues que assi es commo las condiconnes dela mal querençia | sean mucho peores }  \\\hline
1.3.7 &  \textbf{ quare cum per iram accendatur sanguis circa cor , }  &  \textbf{ Por la qual cosa commo por la saña se ençienda la sangre cerca el coraçon tornasse el cuerpo destenprado }  \\\hline
1.3.8 &  \textbf{ Nam cum loquela non possit negari , }  &  \textbf{ Ca assy commo la fabla non puede ser negada sinon por la fabla . }  \\\hline
1.3.8 &  \textbf{ cum ergo alia conueniant bestiis , alia hominibus : }  &  \textbf{ Por ende commo algunas cosas conuengan alas bestias | e algunos alos omes }  \\\hline
1.3.8 &  \textbf{ Quare cum in seipsis pacem non habeant , }  &  \textbf{ Por la qual cosa commo ellos non ayan paz en si mismos non gozan de ssi mismos . }  \\\hline
1.3.8 &  \textbf{ Nam cum de dolore amicorum sit dolendum , }  &  \textbf{ Ca quando nos dolemos del dolor de los amigos dolemos nos nos }  \\\hline
1.3.9 &  \textbf{ Cum ergo aliquid maxime sit bonum , }  &  \textbf{ Et por ende commo alguna cosa | entonçe sea dicha muy buean }  \\\hline
1.3.9 &  \textbf{ Sed cum ex passionibus diuersificari habeant opera nostra , }  &  \textbf{ Mas commo las nr̃as obras ayan de ser departidas | por estas passiones . }  \\\hline
1.4.1 &  \textbf{ Cum ergo memoria sit respectu praeteritorum , }  &  \textbf{ Et pues que assi es commo la memoria sea en conparaçion del tienpo passado | por que es recordaçion delas cosas }  \\\hline
1.4.1 &  \textbf{ cum sint liberales , et cum sint animosi et bonae spei , }  &  \textbf{ Et pues que assi es commo los mancebos }  \\\hline
1.4.1 &  \textbf{ Nam cum iuuenes sint percalidi , }  &  \textbf{ Ca por que los mançebos son muy calientes }  \\\hline
1.4.2 &  \textbf{ Nam cum iuuenes sint percalidi , }  &  \textbf{ La primera razon es que por que los mancebos son muy calient s̃ }  \\\hline
1.4.2 &  \textbf{ Cum ergo naturale sit , }  &  \textbf{ Et pues que assi es commo natural cosa sea }  \\\hline
1.4.2 &  \textbf{ Nam cum multos habeant adulatores , }  &  \textbf{ Ca commo ellos ayan muchos lisongeros }  \\\hline
1.4.2 &  \textbf{ quia cum alia sint moderanda per mensuram , }  &  \textbf{ de una ser tenp̃das | por me lura }  \\\hline
1.4.3 &  \textbf{ Cum ergo timidi efficiantur frigidi , }  &  \textbf{ Et pues que assi es commo los temerosos sean esfriados . }  \\\hline
1.4.3 &  \textbf{ cum senes adinuicem congregantur , }  &  \textbf{ Ca nos veemos que quando los uieios se ayuntan en vno }  \\\hline
1.4.3 &  \textbf{ cum sit timor inhonorationis , }  &  \textbf{ Ca por que la uerguença es temor de desonera non pertenesçe alos uieios }  \\\hline
1.4.4 &  \textbf{ Cum enim nulla sit actio animae , }  &  \textbf{ Ca commo non sea ninguna obra del alma en el cuerpo }  \\\hline
1.4.5 &  \textbf{ Cum ergo semper sit dare initium , }  &  \textbf{ Et pues que assi es comm sienpre ayamos de dar comienço }  \\\hline
1.4.5 &  \textbf{ quare cum nobilitas semper inclinet animum nobilium }  &  \textbf{ Por la qual razon commo la nobleza | sienpre incline el coraçon de los nobles }  \\\hline
1.4.5 &  \textbf{ cum magna diligentia nutriantur , }  &  \textbf{ Ca por que los nobles son cados con grand acuçia }  \\\hline
1.4.5 &  \textbf{ et cum magna cura proprium corpus custodiant : }  &  \textbf{ e con grand cura | e con grand guarda delos sus cuerpos }  \\\hline
1.4.5 &  \textbf{ Cum ergo molles carne aptos mente dicamus , }  &  \textbf{ Et pues que assi es conmolos bien conplissionados | que son muelles en las carnes sean mas engennosos e sotiles en las almas }  \\\hline
1.4.5 &  \textbf{ cum non possint naturaliter dominari , }  &  \textbf{ e los prinçipes non puedan naturalmente ensseñorear }  \\\hline
1.4.6 &  \textbf{ cum ex hoc quis excellere videatur , }  &  \textbf{ Et commo por tal razon commo esta alguno parezca de ser mas alto }  \\\hline
1.4.7 &  \textbf{ quare cum multos nobiles videamus esse impotentes , }  &  \textbf{ por la qual cosa commo nos veamos muchos ser nobles | qua non son poderosos }  \\\hline
2.1.1 &  \textbf{ quare cum viuere sit homini naturale , }  &  \textbf{ Por la qual cosa commo el beuir sean | atuer tal cosa al omne }  \\\hline
2.1.1 &  \textbf{ cum enim homo sit nobilioris complexionis }  &  \textbf{ por que el omne es de mas nobł conplission }  \\\hline
2.1.1 &  \textbf{ cum homo solitarius non sufficiat sibi }  &  \textbf{ commo el omne | que biue solo non abaste assi mismo }  \\\hline
2.1.1 &  \textbf{ Mulier autem cum parit , nescit qualiter se debeat habere in partu , }  &  \textbf{ Mas la mug̃r quan do pare non labe | en qual manera se deua auer en el parto }  \\\hline
2.1.2 &  \textbf{ cum omnis alia communitas communitatem illam praesupponat . }  &  \textbf{ por que todas las comuni dades ençierran en ssi | e ante ponen esta comunidat dela casa ¶ Et }  \\\hline
2.1.3 &  \textbf{ cum adepto fine cesset operatio , }  &  \textbf{ quando ouiessemos ganada la | finçessarie la obra }  \\\hline
2.1.3 &  \textbf{ cum ipsemet dicat ciuitatem procedere ex multiplicatione vici , }  &  \textbf{ e por tienpo commo el mismo | diga }  \\\hline
2.1.3 &  \textbf{ Nam cum natura non praesupponat artem , }  &  \textbf{ Ca commo la natura non presupone | nin antepone arte }  \\\hline
2.1.4 &  \textbf{ cum non sit proprie communitas nec societas ad seipsum , }  &  \textbf{ Et commo non sea propreamente comunidat | nin conpannia de vno }  \\\hline
2.1.5 &  \textbf{ Nam cum generata non possint conseruari in esse }  &  \textbf{ Ca commo las cosas engendradas | non pueden ser conseruadas }  \\\hline
2.1.5 &  \textbf{ nisi ( cum pergit ) dirigatur ab aliquo , }  &  \textbf{ si quando anda non fue regua ado }  \\\hline
2.1.6 &  \textbf{ Quare cum communitas patris ad filium sumat originem }  &  \textbf{ commo la comunidat del padre al fijo tome nasçençia e comienço de aquello que el padre e la madre }  \\\hline
2.1.6 &  \textbf{ quare cum proprium actiuum generationis sit masculus , }  &  \textbf{ Por la qual razon commo el propre o fazedor en la generaçion sea el mas lo . }  \\\hline
2.1.6 &  \textbf{ Nam cum in domo perfecta sint tria regimina , }  &  \textbf{ Ca commo en la casa acabada sean tres gouernamientos . }  \\\hline
2.1.7 &  \textbf{ Cum ergo domus sit prior vico , ciuitate , et regno : }  &  \textbf{ Et pues que assi es commo la casa sea primero | que el uarrio }  \\\hline
2.1.7 &  \textbf{ cum prima communitas ipsius domus sit coniunctio viri et uxoris , }  &  \textbf{ commo la primera comunidat dela casa sea | ayuntamientode uaron }  \\\hline
2.1.7 &  \textbf{ quare cum homo et omnia animalia naturaliter inclinentur , }  &  \textbf{ e natra al apetito | por la qual cosa commo todas las ainalias naturalmente sean inclinadas }  \\\hline
2.1.8 &  \textbf{ Cum enim inter virum et uxorem sit amicitia naturalis , }  &  \textbf{ Ca commo entre el uaron e la muger sea amistança natural }  \\\hline
2.1.8 &  \textbf{ Sed cum omnis amor vim quandam unitiuam dicat , }  &  \textbf{ Mas coͣtra odo amor aya alguna fuerça | para ayuncar los omes }  \\\hline
2.1.9 &  \textbf{ Sed cum excellens amor non possit esse ad plures , }  &  \textbf{ Mas commo el grand amor non pueda ser departido amuchͣs partes }  \\\hline
2.1.9 &  \textbf{ Nam cum coniugium sit quid naturale : }  &  \textbf{ Ca commo el matermonio sea cosa natural }  \\\hline
2.1.9 &  \textbf{ Sed cum dictum sit , }  &  \textbf{ Mas commo dicho es }  \\\hline
2.1.10 &  \textbf{ Nam cum quilibet moleste ferat , }  &  \textbf{ Ca commo qual si quier sufra }  \\\hline
2.1.11 &  \textbf{ Nam cum ex naturali ordine debeamus parentibus debitam subiectionem , }  &  \textbf{ Ca commo por la orden natural deuamos auer | subiectiuo al padre e ala madre }  \\\hline
2.1.11 &  \textbf{ ut quis cum matre contraheret . }  &  \textbf{ que ninguno cassase con su madre }  \\\hline
2.1.12 &  \textbf{ Cum ergo debite et congrue nobili societur : }  &  \textbf{ el noble deua ser aconpannado }  \\\hline
2.1.13 &  \textbf{ quia cum aliqua persona ociosa existat , }  &  \textbf{ Ca quando alguna persona esta de uagar mas ligeramente es inclinada a aquellas cosas }  \\\hline
2.1.14 &  \textbf{ cum ciuitas sit pars uniuersi , }  &  \textbf{ commo la çibdat }  \\\hline
2.1.14 &  \textbf{ cum sint adulti : }  &  \textbf{ quando fueren criados }  \\\hline
2.1.15 &  \textbf{ cum uxor naturaliter sit ordinata ad generandum , }  &  \textbf{ commo la muger sea | ordenadanatraalmente ala generaçion de los fijos }  \\\hline
2.1.15 &  \textbf{ Sed cum carens rationis usu sit naturaliter seruus , }  &  \textbf{ Mas commo aquel que es priuado de vso de razon e de entendemiento sea naturalmente sieruo }  \\\hline
2.1.16 &  \textbf{ quia cum anima sequatur complexiones corporis }  &  \textbf{ ca el alma sigue las conplissiones del cuerpo . }  \\\hline
2.1.18 &  \textbf{ Nam cum mulieres sint naturaliter adeo timidae , }  &  \textbf{ Ca commo las muger ssean naturalmente temerosas | en tanto que semeia }  \\\hline
2.1.19 &  \textbf{ qui cum essent impeditae linguae , }  &  \textbf{ assi los quales commo ouiessen las lenguas enbargadas }  \\\hline
2.1.20 &  \textbf{ Immo cum ostensum sit supra uxorem }  &  \textbf{ Mas por que fue mostrado de suso que la mugni non se deue auer al marido }  \\\hline
2.1.20 &  \textbf{ qualiter cum eis debeant conuersari . }  &  \textbf{ en qual manera deuen beuir conellas }  \\\hline
2.1.21 &  \textbf{ Nam cum vir suam uxorem regere debeat , }  &  \textbf{ Ca quando el marido gouierna e castiga a su muger }  \\\hline
2.1.21 &  \textbf{ quia credit quod in cum plures aspiciant , }  &  \textbf{ mas por que cree que muchos catan ael }  \\\hline
2.1.22 &  \textbf{ quare cum una cura impediat aliam , }  &  \textbf{ Por la qual cosa commo el vn cuydado enbargue el otro . }  \\\hline
2.1.23 &  \textbf{ Natura enim cum moueatur ab intelligentiis , et a Deo , }  &  \textbf{ por que la natura toda es mouida delos angeles | e de dios }  \\\hline
2.1.23 &  \textbf{ esse perfectum quam vir . Quare cum anima sequatur complexionem corporis , }  &  \textbf{ Por la qual cosa | commo el alma sigua ala conplission del cuerpo . }  \\\hline
2.1.24 &  \textbf{ statim cum aliqua persona eis applaudet , }  &  \textbf{ luego que algunas ꝑssonas les comiençan a lisongar }  \\\hline
2.1.24 &  \textbf{ cum ostensum sit , }  &  \textbf{ commo seaya prouado }  \\\hline
2.1.24 &  \textbf{ cum etiam declaratum sit , }  &  \textbf{ Et avn commo sea declarado }  \\\hline
2.1.24 &  \textbf{ et quomodo cum eis debeant conuersari , }  &  \textbf{ e como de una beuir con ellas . }  \\\hline
2.2.1 &  \textbf{ quare cum inter patrem et filium sit amor naturalis , }  &  \textbf{ por la qual cosa commo entre el padre | e el fijo sea amor natraal }  \\\hline
2.2.3 &  \textbf{ Quare cum regimen filiorum sit ex arbitrio , }  &  \textbf{ commo las muger suarones Por la qual razon como el gouernamiento de los fijos }  \\\hline
2.2.3 &  \textbf{ Quare cum quilibet suam perfectionem diligat , }  &  \textbf{ Et commo quier que cada vno ame su perfecçion . }  \\\hline
2.2.4 &  \textbf{ Quare cum amor quandam unionem importet , }  &  \textbf{ por la qual razon commo el amor faga algun ayuntamiento los fijos }  \\\hline
2.2.4 &  \textbf{ Sed cum filii afficiantur ad parentes , }  &  \textbf{ Mas commo los fijos sean inclinados alos padres . }  \\\hline
2.2.5 &  \textbf{ Cum ergo magis simus assuefacti ad ea , }  &  \textbf{ Et pues que assi es commo mas somos usados a aquellas cosas }  \\\hline
2.2.6 &  \textbf{ Quare cum rationis sit concupiscentias refraenare et lasciuias , }  &  \textbf{ e del entendimiento | es de refrenar los desseos e las locanias . }  \\\hline
2.2.7 &  \textbf{ cum ponuntur in aliquo dominio tyrannizent , }  &  \textbf{ quando son puestos en algun sennorio non tiraniçen | nin sean tirannos }  \\\hline
2.2.10 &  \textbf{ et ad lasciuiam proni . Quare cum semper sit adhibenda cautela }  &  \textbf{ e son inclinados a orgullos e aloçania . | Por la qual cosa commo sienpre deuemos dar }  \\\hline
2.2.10 &  \textbf{ cum maturitate eleuent , }  &  \textbf{ assi que alçen las palpebras de los oios con grand madureza }  \\\hline
2.2.11 &  \textbf{ Quare cum turpis modus sumendi cibum signum sit cuiusdam gulositatis , }  &  \textbf{ Por la qual cosa commo la manera torpe de resçebir la vianda | sea señal de golosina }  \\\hline
2.2.11 &  \textbf{ ut cum ad debitam aetatem peruenerint , }  &  \textbf{ por que quando venieren a hedat }  \\\hline
2.2.12 &  \textbf{ quare cum semper sit adhibenda cautela , }  &  \textbf{ alli deue omne poner mayor remedio }  \\\hline
2.2.13 &  \textbf{ Nam mens humana nescit ociosa esse : cum ergo quis vacat ocio , }  &  \textbf{ Ca la uoluntad del omne non sabe ser ocçiosa | nin estar de vagar }  \\\hline
2.2.13 &  \textbf{ Iuuenes , maxime cum ad aliam aetatem venerint , }  &  \textbf{ Et por que mucho conuiene alos mançebos | quando vinieren a aquella hedat }  \\\hline
2.2.14 &  \textbf{ cum quibus sociis debeant conuersari . }  &  \textbf{ e en qual conpannia deuen beuir . }  \\\hline
2.2.21 &  \textbf{ cum loquacitate stare non possit , }  &  \textbf{ tomadesto }  \\\hline
2.3.8 &  \textbf{ cum diuitiae maxime videantur hoc efficere , }  &  \textbf{ e por que las riquezas prinçipalmente fazen esto }  \\\hline
2.3.8 &  \textbf{ Cum ergo diuitiae et possessiones ordinentur ad nutrimentum }  &  \textbf{ ¶ Et pues que assi es las riquezas | e las possessiones sean ordenadas }  \\\hline
2.3.9 &  \textbf{ cum sint magni ponderis , }  &  \textbf{ por que son de grand peso }  \\\hline
2.3.11 &  \textbf{ cum denarius sit quid artificiale , }  &  \textbf{ e como los diueros sean cosas artifiçiales }  \\\hline
2.3.11 &  \textbf{ Cum ergo usus ipsius domus sit domum inhabitare , }  &  \textbf{ Et por ende commo el uso dela casa sea morar en la casa }  \\\hline
2.3.12 &  \textbf{ Ipse enim cum esset pauper , }  &  \textbf{ Este mille sio commo fuesse muy pobre }  \\\hline
2.3.12 &  \textbf{ cum semper in egestate viueret . }  &  \textbf{ e aquel aprouechaua suph̃ia pues siengͤ biue en pobreza e en mengua . }  \\\hline
2.3.13 &  \textbf{ cum societas hominum sit naturalis , }  &  \textbf{ que en ssennore e a todas aquellas muchͣs . }  \\\hline
2.3.13 &  \textbf{ cum ergo videamus aliquos homines respectu aliorum }  &  \textbf{ Et pues que assi es commo nos ueamos | que algs omes en conparaçion de los otros }  \\\hline
2.3.14 &  \textbf{ Nam cum legum latores sint homines , }  &  \textbf{ Ca commo los establesçedores sołas leyes sean omes }  \\\hline
2.3.17 &  \textbf{ Cum enim deceat Regem esse magnificum , }  &  \textbf{ Ca commo conuenga alos Reyes | e alos prinçipes ser magnificos }  \\\hline
2.3.18 &  \textbf{ quis cum aliis conuersetur , }  &  \textbf{ algbiuiere et morare con los otros alegremente }  \\\hline
2.3.19 &  \textbf{ cum ipsis sit conuersandum . }  &  \textbf{ que es en qual manera han de beuir los sennores con sus ofiçiales }  \\\hline
3.1.1 &  \textbf{ cum ciuitas sit opus humanum , }  &  \textbf{ o commo la çibdat | sea obra de los omes }  \\\hline
3.1.4 &  \textbf{ Quare cum communitas ciuilis has communitates comprehendat , }  &  \textbf{ por la qual cosa commo la comuidat çiuil | o la çibdat conprehenda estas dos comuindades }  \\\hline
3.1.7 &  \textbf{ Nam cum sit maxima unitas , }  &  \textbf{ Ca commo sea muy grant vnidat }  \\\hline
3.1.8 &  \textbf{ Nam cum ciuitas sit ordo ciuium }  &  \textbf{ ca commo la çibdat sea orden }  \\\hline
3.1.8 &  \textbf{ Quare cum hoc diuersitatem requirat , }  &  \textbf{ por ende commo estas cosas demanden departimiento }  \\\hline
3.1.9 &  \textbf{ Quare cum in regimine ciuitatis primo sit politia ordinanda , }  &  \textbf{ por la qual cosa commo en el gouernamiento dela çibdat | primeramente se ha de ordenar la poliçia }  \\\hline
3.1.11 &  \textbf{ sed cum esse non possit , }  &  \textbf{ mas commo non pueda ser }  \\\hline
3.1.11 &  \textbf{ Cum ergo custodes ciuitatis nobiliores sint agricolis , }  &  \textbf{ Et pues que assi es commo las guardas | e los defendedores dela çibdat sean mas nobles que los labradores }  \\\hline
3.1.12 &  \textbf{ nam cum humanum sit timere mortem , }  &  \textbf{ ca commo todos los omes | teman la muerte los esforçados }  \\\hline
3.1.13 &  \textbf{ Quare cum deceat regia maiestatem }  &  \textbf{ por la qual razon commo venga ala real magestad }  \\\hline
3.1.14 &  \textbf{ Quare cum patefactum sit in praecedentibus , }  &  \textbf{ por la quel cosa commo sea manifiesto | por las cosas dichͣs de suso }  \\\hline
3.1.14 &  \textbf{ cum adesset oportunitas : }  &  \textbf{ quando fuesse me este }  \\\hline
3.1.15 &  \textbf{ cum Plato eius discipulus fuisset . }  &  \textbf{ por que platon fue su disçipulo . }  \\\hline
3.1.19 &  \textbf{ cum non possint defendere iura sua . }  &  \textbf{ por que non pueden defender su derecho }  \\\hline
3.2.4 &  \textbf{ cum ipse pluries dicat in eisdem politicis , }  &  \textbf{ ca el dize muchͣs uezeᷤ | en esse mismo libro delas politicas }  \\\hline
3.2.4 &  \textbf{ cum nunquam plures recte dominari possint , }  &  \textbf{ ca nunca pueden much | senssennorear }  \\\hline
3.2.5 &  \textbf{ quare cum omne voluntarium sit minus onerosum et difficile , }  &  \textbf{ ca commo toda cosa uoluntaria | sea de menor carga e menos graue }  \\\hline
3.2.7 &  \textbf{ cum unus sit dominans , }  &  \textbf{ enssennoreare commo vno sea el señor }  \\\hline
3.2.9 &  \textbf{ quod cum quidam Rex partem sui regni dimisisset , }  &  \textbf{ que commo vn Rey dexasse vna parte de su regno . }  \\\hline
3.2.10 &  \textbf{ nam cum intendat bonum ipsorum ciuium et subditorum , }  &  \textbf{ Ca commo el entienda enl bien de los çibdadanos natural }  \\\hline
3.2.10 &  \textbf{ Cum enim tyranni sciant se non diligi a populo , }  &  \textbf{ Ca commo los tyranos sepan | que non lon amados del pueblo . }  \\\hline
3.2.12 &  \textbf{ qui cum a fratre suo cotidie increparetur , }  &  \textbf{ que cada dia era denostado de vn su hͣrmano }  \\\hline
3.2.12 &  \textbf{ Et cum frater eius timore horribili inuaderetur , }  &  \textbf{ Et estonçe commo aquel su hͣrmano tomasse grant espanto }  \\\hline
3.2.12 &  \textbf{ cum videat se esse populis odiosum . }  &  \textbf{ quando bee | que es aborresçido delos pueblos }  \\\hline
3.2.13 &  \textbf{ cum deuiare a recto regimine sit tyrannizare , }  &  \textbf{ e arredrarse los rreyes del | gouernemjento derecho sea tiranizar }  \\\hline
3.2.13 &  \textbf{ Nam cum non quaerant bonum commune , }  &  \textbf{ ca por qua non qͥeren el bien comun }  \\\hline
3.2.14 &  \textbf{ cum tot modis dissoluatur tyrannicus principatus . }  &  \textbf{ que non tiraniz en commo en tantas maneras | segunt dicho es se aya de destroyr el prinçipado tiranico . }  \\\hline
3.2.16 &  \textbf{ Sed , cum dicat Philosophus }  &  \textbf{ Mas commo el pho diga en el segundo libro delas ethicas }  \\\hline
3.2.17 &  \textbf{ ut cum aliis conferamus quid agendum , }  &  \textbf{ que con los otros ayamos acuerdo }  \\\hline
3.2.17 &  \textbf{ Quare cum plures plura experti sint , }  &  \textbf{ por la quel cosa commo muchs mas cosas ayan prouadas }  \\\hline
3.2.18 &  \textbf{ cum tales mentiri nolint , }  &  \textbf{ e los bueons non quieren mentir }  \\\hline
3.2.19 &  \textbf{ Sed utrum cum extraneis debeamus habere pacem }  &  \textbf{ Mas si deuemos auer paz con los estrannos o guerra pue de ser cosa dubdosa }  \\\hline
3.2.20 &  \textbf{ Sed cum iudicium fiat per leges , }  &  \textbf{ mas commo el iuyzio se deua fazer }  \\\hline
3.2.23 &  \textbf{ Nam cum natura humana de se sit debilis , }  &  \textbf{ ca commo la natura humanal }  \\\hline
3.2.23 &  \textbf{ cum bestiae hoc agant : }  &  \textbf{ a aquellos que se les homillan | e esto prouamos }  \\\hline
3.2.27 &  \textbf{ quare cum bonum commune principaliter intendatur a tota communitate }  &  \textbf{ Por la qual cosa commo el bien comun sea entendido | prinçipalmente de toda la comunidat }  \\\hline
3.2.27 &  \textbf{ Sed cum alia sit lex naturalis , }  &  \textbf{ Mas commo otra sea la ley natural e otra la positiua en vna manera se deue publicar la vna }  \\\hline
3.2.29 &  \textbf{ Nam Rex cum sit homo }  &  \textbf{ Ca el Rey por que es omne }  \\\hline
3.2.30 &  \textbf{ cum procedant ex interiori appetitu . }  &  \textbf{ quando uieñe del apetito | et ple desseo del coraçon . }  \\\hline
3.2.30 &  \textbf{ Quare cum in humanis iudiciis cadere possit dubieras et error , }  &  \textbf{ Por la qual cosa commo en los iuyzios humanales pueda caer dubda e yerro . | cosa muy aprouechable }  \\\hline
3.2.34 &  \textbf{ si cum magna diligentia obediat regibus , et principibus , }  &  \textbf{ si obedesçiere alos Reyes | e alos prinçipes con grand acuçia . }  \\\hline
3.2.34 &  \textbf{ cum virtus faciat habentem bonum ; }  &  \textbf{ Et la uirtud faze al que la ha buenon }  \\\hline
3.2.34 &  \textbf{ Cum enim bestiae sint naturae seruilis : }  &  \textbf{ Ca commo las bestias sean de natura seruil }  \\\hline
3.2.34 &  \textbf{ cum ex hoc consurgat tantum bonum , }  &  \textbf{ Commo desto se leunate tan grant bien }  \\\hline
3.3.2 &  \textbf{ Nam cum naturaliter habeant modicum sanguinis , }  &  \textbf{ Ca por que naturalmente han poca sangre }  \\\hline
3.3.2 &  \textbf{ cum assueti sint ad occisionem animalium , }  &  \textbf{ por que son acostunbrados a matar las animalias }  \\\hline
3.3.4 &  \textbf{ Nam cum tota operatio bellica exposita sit periculis mortis , }  &  \textbf{ Ca commo toda la hueste sea puesta | a periglos de muerte en la batalla }  \\\hline
3.3.4 &  \textbf{ Sed cum omnis bellica operatio contineatur sub militari , }  &  \textbf{ Mas conmo todas las obras de la batalla sean contenidas so la caualleria }  \\\hline
3.3.4 &  \textbf{ Quare cum maxime contingat bellantes vincere , }  &  \textbf{ Por la qual cosa commo mayormente contezca a los lidiadores vençer }  \\\hline
3.3.5 &  \textbf{ Quare cum communiter nobiles homines industriores sint rusticis , }  &  \textbf{ Por la qual cosa commo los nobles omnes sean mas sotiles comunalmente }  \\\hline
3.3.6 &  \textbf{ Et cum viderit magister bellorum }  &  \textbf{ Et quando vieren los caudiellos maestros de las batallas }  \\\hline
3.3.7 &  \textbf{ se cum hostibus possint coniungere , }  &  \textbf{ mas puesto que los lidiadores se puedan ayuntar con los enemigos }  \\\hline
3.3.7 &  \textbf{ qui cum pro populo Romano certare deberet , }  &  \textbf{ por el pueblo de roma non cuydaua vençer en otra manera a los enemigos }  \\\hline
3.3.7 &  \textbf{ quem non primo cum funda percuterent . }  &  \textbf{ fasta que ferien con la fonda en logar çierto . }  \\\hline
3.3.8 &  \textbf{ ut cum castrametari voluerit , }  &  \textbf{ por que quando quisiere la hueste folgar en algun logar parezca }  \\\hline
3.3.10 &  \textbf{ et cum debeant esse vigilantes , agiles , sobrii , }  &  \textbf{ e avn que ayan los oios bien espiertos | e que sean ligeros e mesurados en beuer e gerrdados de vino }  \\\hline
3.3.11 &  \textbf{ Itaque cum pericula visa minus noceant , }  &  \textbf{ Et por ende por que los periglos que son ante vistos menos enpeesçen . }  \\\hline
3.3.14 &  \textbf{ Nam cum septem modis enumeratis hostes fortiores existant ; }  &  \textbf{ Ca commo en las siete maneras contadas | sean los enemigos mas fuertes }  \\\hline
3.3.16 &  \textbf{ Quare cum sint quatuor genera pugnarum , }  &  \textbf{ Por la qual cosa commo sean quatro maneras da batallas }  \\\hline
3.3.17 &  \textbf{ Nam cum contingat obsessiones }  &  \textbf{ Ca commo contezca }  \\\hline
3.3.17 &  \textbf{ ( cum fuerint occupati obsidentes somno , }  &  \textbf{ quando los que çercan durmieren o comieren o estudieren de vagar o fueren derramados }  \\\hline
3.3.19 &  \textbf{ si vero visus protendatur magis basse , cum tabula sit existente ad pedes , }  &  \textbf{ e fuere mas baxa aluengesse | mas con la tabla en tal manera }  \\\hline
3.3.20 &  \textbf{ quia cum lapis eiectus a machina perueniret ad huiusmodi murum , }  &  \textbf{ Ca quando la piedra del engennio firiere en el muro de tierra . }  \\\hline
3.3.21 &  \textbf{ et cum temperamento dispensetur . }  &  \textbf{ si non fueren partidas con tenpramiento et escassamente . }  \\\hline
3.3.21 &  \textbf{ quod cum Romanis neruorum copia defecisset , }  &  \textbf{ que quando a los romanos fallesçieron los neruios }  \\\hline
3.3.22 &  \textbf{ quod cum perceperint , }  &  \textbf{ Et quando esto entendieren }  \\\hline
3.3.23 &  \textbf{ quia cum pugnatores marini quasi fixi stent in naui , }  &  \textbf{ por que los lidiadores de la mar esten firmes }  \\\hline
3.3.23 &  \textbf{ ut cum ipso percuti possit tam nauis , }  &  \textbf{ ca pueden ferir tan bien en la }  \\\hline
3.3.23 &  \textbf{ cum per ipsa coeperit abundare aqua , }  &  \textbf{ quando entrare mucha agua dentro en la }  \\\hline

\end{tabular}
