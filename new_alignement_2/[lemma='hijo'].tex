\begin{tabular}{|p{1cm}|p{6.5cm}|p{6.5cm}|}

\hline
1.1.8 & Et desto auemos \textbf{ enxienplo en vn fijo de vn prinçipe Romano } que dizien torcato & Exemplum huiusmodi habemus \textbf{ de filio cuiusdam Romani Principis nomine Torquati , } qui nimii honoris auidus , \\\hline
1.1.8 & porque los otros tomasen exienplo dello \textbf{ e que non fuesen cobdiçon sos de honrra mato a su fijo presunptuoso } e soƀuio commo quier & ne alii ex hoc exemplum assumerent , \textbf{ et ne essent nimii honoris auidi , | filium sic praesumptuosum occidit , } non obstante quod dictus filius \\\hline
1.1.8 & e soƀuio commo quier \textbf{ que aquel su fijo ouiese auido uictoria de los sus enemigos ¶ } Pues que assi es el prinçipe & non obstante quod dictus filius \textbf{ victoriam obtinuerat ab hoste . } Ne ergo Princeps se praecipitet , et ne nimis praesumat , \\\hline
1.1.11 & e por guarda del su linage \textbf{ o por engendrar fijos . } Ca por fallesçimiento de los fijos & propter conseruationem speciei , \textbf{ siue propter procreationem prolis : } nam ex defectu filiorum multa regna \\\hline
1.1.11 & muchos regnos ouieron grand departimiento e grandes escandalos . \textbf{ Ca muriendo los Reys sin fijos legitimos } muchos se le una tan para seer señores & passa sunt diuisionem , et scandala : \textbf{ decedentibus enim Principibus absque liberis , } plures insurgunt , \\\hline
1.2.20 & que son ayuntadas a el assi commo es la muger \textbf{ e los fijos auiendo moradas honrradas } e faziendo bodas conuenibles e honrradas & ut erga uxorem et filios , \textbf{ habendo habitationes honorabiles , } faciendo nuptias decentes , \\\hline
1.2.32 & Et otra cosa que es peor desta \textbf{ que presta un a los sus fiios en los conbites a sus vezinos que los comiessen . } Ca quando alguno quaria conbidar a & et ( quod peius est ) \textbf{ praestabant sibi filios inconuiuiis . } Cum enim qui alios conuiuare volebat , \\\hline
1.2.32 & Ca quando alguno quaria conbidar a \textbf{ otrossi el su fiio non era en casa tomaua prestado el fijo de otro su vezino } e aprestaual para fazer el conbit & Cum enim qui alios conuiuare volebat , \textbf{ si filius suus domi non erat , | a vicino suo mutuabat filium , } et ipsum parabat in conuiuium , \\\hline
1.2.32 & por la qual cosa non le paresçia \textbf{ que era fijo de padre } e de varon meral & quod erat valde bonus : \textbf{ propter quod non videbatur existere puer , } viri moralis , sed Dei . \\\hline
1.2.32 & e bien acostunbrado \textbf{ mas que era fijo de dios } Mas aquella uirtud por la qual alguno es dich̃o bueno & ø \\\hline
1.4.1 & Et pues que assi es commo las muger \textbf{ ssienpre amonestan asus fijos a honestad } e a honrrata honestades & Cum ergo matres semper moneant \textbf{ suos filios ad honesta ; } quia honestum idem est \\\hline
1.4.5 & que sienpre la fechura quiera semeiar a su fazedor \textbf{ por que los fijos son fechuras de los padron natural cosa es que los fuos semeien alos paradres . } Et por ende los nobles teniendo mientes & cum filii sint \textbf{ quidam effectus parentum , | naturale est filios imitari parentes . } Nobiles ergo aduertentes \\\hline
1.4.5 & otrascondiconnes son eguales \textbf{ sienpre los fijos son mas nobles que los padres . } Ca assi commo dich̃es la nobleza & Nam caeteris paribus \textbf{ semper filii sunt nobiliores parentibus . } Nam ( ut dictum est ) nobilitas idem est \\\hline
1.4.5 & e sienpra son mas \textbf{ antiguadas las riquezas en los fijos que en los padres . } Por la qual razon commo la nobleza & et semper sunt magis antiquatae diuitiae \textbf{ in filiis quam in parentibus : } quare cum nobilitas semper inclinet animum nobilium \\\hline
1.4.5 & que los nobles han de ser magnificos \textbf{ e ahun mas los fijos que los padres } por que en alguna manera son mas nobles que elos . & sequitur nobiles esse magnificos , \textbf{ etiam magis quam parentes : } quia quodammodo sunt nobiliores illis . \\\hline
2.1.2 & Ca primeramente fue fecha vna casa \textbf{ e despues cresçiendo los fijos e las fijas } e por que non podieron por muchedunbre dellos morar todos en vna cala & quia primo facta fuit una aliqua domus : \textbf{ sed crescentibus filiis et filiabus , } et non valentibus praemultitudine habitare in domo illa , \\\hline
2.1.2 & que el uarrio es vezindat de casas \textbf{ las quales llaman algunos nietos e fijos e fijos de fijos . } Et pues que assi es natural fazimiento & quod vicus est vicinia domorum , \textbf{ quos vocant quidam collectaneos | et pueros puerorum . } Naturalis ergo origo vici , \\\hline
2.1.2 & las quales se fizieron \textbf{ de muchedunbre de mietos e de fijos . } Ca assi commo dicho es de suso cresçiendo los nietos e los fijos de los fijos & quas construxit multitudo collectaneorum , \textbf{ et puerorum , siue filiorum . | Nam , ut tangebatur , } crescentibus collectaneis \\\hline
2.1.2 & de muchedunbre de mietos e de fijos . \textbf{ Ca assi commo dicho es de suso cresçiendo los nietos e los fijos de los fijos } por que non podien todos morar en vna casa & Nam , ut tangebatur , \textbf{ crescentibus collectaneis | idest nepotibus , et filiis , et filiorum filiis : } et non valentibus habitare in una domo , \\\hline
2.1.6 & y la terçera comunidat \textbf{ que es de padre e de fijo . } Ca ueemos en las cosas naturales & oportet ibi dare communitatem tertiam , \textbf{ scilicet patris et filii . } Videmus enim in naturalibus rebus \\\hline
2.1.6 & por la qual cosa \textbf{ commo la comunidat del padre al fijo tome nasçençia e comienço de aquello que el padre e la madre } engendran su semeiança esta tal comunidat non es dicha de razon dela primera casa & nisi sit iam perfectus . \textbf{ Quare cum communitas patris ad filium sumat originem | ex eo quod parentes sibi simile produxerunt : } huiusmodi communitas non dicitur \\\hline
2.1.6 & que ala casa acabada parte nesçe la terçera cemuidat \textbf{ que es del padre e del fijo . } Mas que ala perfectiuo dela & quod ad domum perfectam requiritur communitas tertia , \textbf{ videlicet patris et filii . } Quod autem ad perfectionem domus requiratur haec tertia communitas , \\\hline
2.1.6 & que es del padͤ \textbf{ e del fijo pertenescan a conplimiento dela casa podemos lo prouar . } ¶Lo primero de parte dela generaçion & quae est patris et filii , \textbf{ primo possumus probare } ex parte generationis et fructificationis naturalis . \\\hline
2.1.6 & ay ayuntamiento de omne e de muger \textbf{ e non sea y generaçion de fijo o de fija } siguese que es nengua de parte del uaron & coniunctio maris et foeminae ; \textbf{ et tamen non est procreatio prolis , } vel mas est imperfectum actiuum , \\\hline
2.1.6 & es dicha ser menguada \textbf{ en que non ay engendraçion de fijos . } ¶ La segunda razon para prouar esto mesmo se toma de acabamiento de comunindat natural duradera por sienpre . & tota illa domus dicitur imperfecta , \textbf{ ubi non est pollulatio filiorum . } Secunda via ad inuestigandum hoc idem , \\\hline
2.1.6 & mas en alguna manera la uida del omne es duradera para sienpre \textbf{ por suçession e generaçion de los fiios el padre } engendrando el fuo el fijo otro fijo & sed quodammodo perpetuatur humana vita \textbf{ per successionem filiorum : } domus ubi est carentia prolis , \\\hline
2.1.6 & por suçession e generaçion de los fiios el padre \textbf{ engendrando el fuo el fijo otro fijo } e assi biue por sienpre & sed quodammodo perpetuatur humana vita \textbf{ per successionem filiorum : } domus ubi est carentia prolis , \\\hline
2.1.6 & en vna manera es natural \textbf{ assi commos por generaçion de fijos aquella casa sea continuadamente morada } en otra manera es & uno modo est naturalis , \textbf{ ut si per creationem filiorum domus | ista continue habitetur . } Alio modo est quasi casualis , \\\hline
2.1.6 & si non quando por generaçion \textbf{ que es obra de natura fuere enlla cresçençia e multiplicaçion de fiios . } mas si la casa duradera & nisi per generationem , \textbf{ quae est opus naturae , } et nisi sit ibi excrescentia filiorum . \\\hline
2.1.6 & de parte dela bien andança \textbf{ ca los fijos e el poderio çiuil e las otras } cosastales commo quier que non sean essençiales ala bien andança . & ex parte ipsius felicitatis . \textbf{ Nam filii , et ciuilis potentia , | et cetera talia , } licet non sint essentialia felicitati : \\\hline
2.1.6 & e non es noble \textbf{ e es seneriego e sin fijos en ninguna manera non puede ser bien andante . } Ca tal commo este en toda manera es dicho mal andante & et sine filiis , \textbf{ omnino felix esse non potest ; } dicitur enim talis non esse omnino felix , \\\hline
2.1.6 & Mas en la comunidat del padre \textbf{ e del fijo el padre deua sienpre mandar } e el fij̉o ser obediente . & in communitate vero patris et filii , \textbf{ pater debet esse imperans , } et filius obtemperans ; \\\hline
2.1.6 & ¶ \textbf{ La quarta el fijo . } ¶ & ibi vir , \textbf{ secunda uxor , tertia pater , quarta filius , } quinta dominus , sexta seruus . \\\hline
2.1.7 & la primera razon se toma de parte dela conpannia humanal¶ \textbf{ La segunda de parte dela generaçion de los fijos ¶ } La terçera de parte de obras & ex parte societatis humanae . \textbf{ Secunda ex parte procreationis prolis . } Tertia ex parte operum . \\\hline
2.1.7 & aianl coniuigable \textbf{ de parte dela generaçion de los fijos . } Ca aquella & Secundo homo est naturaliter animal coniugale \textbf{ ex parte procreationis prolis . } Nam illud maxime videtur naturale , \\\hline
2.1.8 & La segunda razon para prouar esto \textbf{ mesmo se toma de parte dela generaçion de los fijos . } Ca commo quier que el casamiento sea mannero & Secunda via ad inuestigandum \textbf{ hoc idem sumitur | ex parte prolis . } Nam licet coniugium , \\\hline
2.1.8 & Empero si fuere y el bien de los fijos \textbf{ por que los fijos es vn bien comun } en el qual se ay unta el marido e la muger . & attamen si adsit ibi bonum prolis , \textbf{ quia proles est | quoddam commune bonum } in quo coniungitur vir et uxor , \\\hline
2.1.8 & por que es bien comunal dellos \textbf{ assi los fijos ayuntan } e tienen los padres e las madres & quoddam commune bonum ipsorum : \textbf{ sic filii coniungunt } et continent ipsos parentes , \\\hline
2.1.8 & que alos otros \textbf{ ca non auer cuydado de los fijos del Rey } mas puede fazer danno a todo el regno & prae omnibus aliis \textbf{ debent diligentiorem habere curam . | Incuria enim regiae prolis } plus potest \\\hline
2.1.8 & quanto mayor es de tomar la cura \textbf{ e el acuçia de los fijos de los Reyes } que de los fiios de los otros tanto & quanto maior cura et diligentia adhibenda est \textbf{ circa prolem regiam } quam circa alias , \\\hline
2.1.8 & e el acuçia de los fijos de los Reyes \textbf{ que de los fiios de los otros tanto } mas conuiene alos Reyes & circa prolem regiam \textbf{ quam circa alias , } tanto magis decet Reges , et Principes , \\\hline
2.1.9 & La segunda departe dela mugr¶ \textbf{ la terçera de parte de los fijos ¶ } La primera se prueua assi . & Secunda ex parte ipsius uxoris . \textbf{ Tertia est ex parte prolis . } Prima via sic patet . \\\hline
2.1.9 & ¶ La terçera razon para prouar esto mesmo se toma de parte dela \textbf{ cerazon delos fijos . } Ca commo el matermonio sea cosa natural & Tertia via ad inuestigandum hoc idem , \textbf{ sumitur ex nutritione filiorum . } Nam cum coniugium sit quid naturale : \\\hline
2.1.9 & que en algunas ainalias vna sola fenbra \textbf{ abasta para cança de muchos fijos . } assi commo paresçe en los canes e en las gallinas & sola foemella sufficit \textbf{ ad nutritionem filiorum , } ut patet in canibus , in gallinis , \\\hline
2.1.9 & Por la qual cola commo la conuiction \textbf{ e el ayuntamiento del mallo e dela fenbra sea ordenado en todas las ainalias abien delos fiios . } Et en aquellas aianlias & et in pluribus aliis animalibus . \textbf{ Quare cum coniunctio maris , et foeminae in omnibus animalibus ordinetur } ad bonum prolis , \\\hline
2.1.9 & en las quales vna fenbra sola non abasta \textbf{ para dar conuenible nudͣmiento alos fijos vn } mas lo non se ayunta & non sufficit \textbf{ ad praestandum filiis debitum nutrimentum , } unus masculus non adhaeret \\\hline
2.1.9 & mientra dura el tienpo del parto \textbf{ por el creamiento de los fijos . } assi conmo paresçe en las palomas & et quam diu durat tempus partus , \textbf{ propter nutritionem filiorum } ( ut patet in columbis \\\hline
2.1.9 & e alas uegadas el otro \textbf{ e dela carga de los fijos parten sufren las fenbras } e parte los mas los . & quae alternatim oua fouent ) \textbf{ et onerum filiorum partem portat foemella , } et partem masculus . \\\hline
2.1.9 & abastarian \textbf{ para dar conuenible nudermiento alos fijos . } Empero por que non es iudgar la cosa natural & quia facultatibus abundant sufficerent \textbf{ ad praestandum filiis debitum nutrimentum , } quia tamen naturale non est iudicandum illud quod est in paucioribus , \\\hline
2.1.9 & que tan bien el mas o commo la fenbra \textbf{ sufran las cargas de los fijos . } Mas commo en las otras aianlias & quid naturale , \textbf{ ut tam mas quam foemina supportent onera filiorum . } Sed cum in aliis animalibus , \\\hline
2.1.9 & sufren las carguas de los fijos \textbf{ e mientra los fijos han mester ayuda del padre } e dela madre sea cosa natural & in quibus tam mas quam foemina supportant onera filiorum , \textbf{ quam diu filii indigent parentibus , } naturale sit \\\hline
2.1.9 & que en los omes le acola natural \textbf{ que mientra que los fijos han menester ayuda del padre } e dela madre en todo esse tienpo vn & sequitur in hominibus esse quid naturale , \textbf{ ut quam diu filii indigent parentibus , } tam diu unus masculus uni foeminae \\\hline
2.1.9 & mas lo se deue ayuntar a vna fenbra por matrimoino . \textbf{ Mas commo los fijos mientra biuen ayan } menester ayuda del padre & per coniugium copuletur . \textbf{ Sed filii quam diu viuunt } indigent ope parentum , \\\hline
2.1.9 & commo las aues \textbf{ que sufren las cargas de los fijos auegadas } commo sean las aues & sicut in auibus alternatim supportantibus \textbf{ onera filiorum } se habent masculus et foemina tempore partus . \\\hline
2.1.10 & e aguarda dela orden natural \textbf{ e apaz conueinble mas avn es ordenado a generacion de los fijos . } ¶ Lo quarto & ad conseruationem ordinis naturalis , \textbf{ et ad debitam pacem , | sed etiam ordinatur ad procreationem filiorum . } Quarto , sicut ordinatur coniugium \\\hline
2.1.10 & nin seria y conuenible generaçion de los fijos \textbf{ nin les seria dado alos fijos conuenible nudermiento } Et que por esto se tire la orden natural & non erit debita ibi procreatio filiorum , \textbf{ non tribuetur filiis debitum nutrimentum . } Quod autem ex hoc tollatur naturalis ordo , \\\hline
2.1.10 & son mas maneras que las otras mugers . \textbf{ Et por ende departe dela generaçion de lons fijos es } cosadesconuenible en qua vna fenbra aya muchos maridos & quam alias mulieres . \textbf{ Igitur ex parte procreationis filiorum omnino indecens est } unam foeminam plures habere uiros . \\\hline
2.1.10 & por que creen firmemente \textbf{ que ellos son sus fiios . } Et pues que assi es & Nam ex hoc parentes solicitantur circa pueros , \textbf{ quia firmiter credunt eos esse eorum filios : } quicquid ergo impedit certitudinem filiorum , \\\hline
2.1.10 & Et pues que assi es \textbf{ qual si quier cosa que enbarga la çertidunbre de los fijos enbarga alos padres } que non los prouean diligentemente en la hedat & quia firmiter credunt eos esse eorum filios : \textbf{ quicquid ergo impedit certitudinem filiorum , } impedit ne patres diligenter \\\hline
2.1.10 & Mas si vna fenbra casare con muchos uarones \textbf{ los padres non podrian ser çiertos de sus fijos . } Et por ende non aurian tan grand cuydado & Sed si una foemina pluribus nubat viris , \textbf{ patres de suis filiis certi esse non poterunt , } quare non adhibebunt illam diligentiam \\\hline
2.1.10 & que vna muger aya muchos maridos \textbf{ ca por esto se enbargaria mas la çertidunbre de los fijos . } Por la qual cosa sy conuiene a todos los çibdadanos & sed detestabilius est unam uxorem plures habere viros , \textbf{ quia per hoc magis impeditur certitudo filiorum . } Quare si decet omnes ciues certos esse de suis filiis , \\\hline
2.1.10 & en quanto es mas periglo mayor \textbf{ de non auer cuydado de los fijos de los Reyes } et de los prinçipes & in quantum incuria \textbf{ circa eorum filios periculosior est , } quam incuria aliorum . \\\hline
2.1.10 & et de los prinçipes \textbf{ que de los fijos delons otros . } ¶ & circa eorum filios periculosior est , \textbf{ quam incuria aliorum . } Crederet forte aliquis \\\hline
2.1.11 & Et cosa desconueinente \textbf{ si e que la madre fuesse subiecta al fijo . } ¶ Et pues que assi es non conuiene alas fijas de casar con su padre nin alos fijos con su madre & inconueniens esset \textbf{ sic matrem filio esse subiectam . } Non licet ergo filiis contrahere cum parentibus \\\hline
2.1.11 & si e que la madre fuesse subiecta al fijo . \textbf{ ¶ Et pues que assi es non conuiene alas fijas de casar con su padre nin alos fijos con su madre } por la grand reuerençia & sic matrem filio esse subiectam . \textbf{ Non licet ergo filiis contrahere cum parentibus } propter mutuam reuerentiam , \\\hline
2.1.11 & Ca por el casamiento \textbf{ non solamente viene el bien de los fijos . } Mas ahun escusase el destenp̃miento & Per coniugium enim \textbf{ non solum producitur bonum prolis , } sed etiam vitatur intemperantiae malum : \\\hline
2.1.13 & La qual fialdat guardando escusan la fornicaçion \textbf{ e avn el bien dela generaçion de los fijos . } Mas paresçe parte nesçer derechamente al casamiento & quam seruando fornicationem vitant ) \textbf{ et bonum prolis magis directe pertinere videntur ad coniugium , } quam ea quae in praecedenti capitulo diximus . \\\hline
2.1.13 & e para guardar la fialdat del casamiento \textbf{ e para engendrar conueniblemente los fijos deuen ser demandadas enla muger . } Ca beemos que la grandeza del cuerpo faze al bien dela generaçion de los fijos . & ad fidem coniugum conseruandam , \textbf{ et ad prolem debite producendam , } in coniuge quaeri debent . Videmus autem quod magnitudo corporis facit ad bonum prolis . \\\hline
2.1.13 & e para engendrar conueniblemente los fijos deuen ser demandadas enla muger . \textbf{ Ca beemos que la grandeza del cuerpo faze al bien dela generaçion de los fijos . } Por que los fijos enla quantidat del cuerpo & et ad prolem debite producendam , \textbf{ in coniuge quaeri debent . Videmus autem quod magnitudo corporis facit ad bonum prolis . } Nam filii in quantitate corporis \\\hline
2.1.13 & Ca beemos que la grandeza del cuerpo faze al bien dela generaçion de los fijos . \textbf{ Por que los fijos enla quantidat del cuerpo } en la mayor parte sallen ala madre & in coniuge quaeri debent . Videmus autem quod magnitudo corporis facit ad bonum prolis . \textbf{ Nam filii in quantitate corporis } ut plurimum matrizant , \\\hline
2.1.13 & por el bien dela generaçion de los fijos \textbf{ e por que los fijos dellos } resplandezcan por grandeza de cuepo de demandar en las sus mugers grandeza de cuerpo . & Decet omnes ciues propter bonum prolis , \textbf{ ut filii polleant magnitudine corporali , } quaerere in suis uxoribus magnitudinem corporis : \\\hline
2.1.13 & e alos prinçipes \textbf{ quanto ellos deuen auer mayor cuydado de sus fijos propreos } por que dellos cuelga el bien comun & tanto tamen magis hoc decet Reges et Principes , \textbf{ quanto ipsi circa proprios filios , | eo quod ex eis dependeat } bonum commune et salus regni , \\\hline
2.1.13 & Lo segundo entre los bienes del cuerpo es de demandar en la muger fermosura \textbf{ e apostura por que esto fazal bien de la generaçion dellos fijos . } Ca assi commo en la mayor parte de los grandes nasçen quandes . & in uxore formositas et pulchritudo : \textbf{ nam et hoc facit ad bonum prolis . } Nam sicut ut plurimum ex magnis nascuntur magni : \\\hline
2.1.13 & e alos prinçipes de ser cuydadosos que resplandez cau \textbf{ por fiios grandes e fermosos . } Conuiene a ellos de demandar en las sus mugieres grandeza e fermosura corporal . & et maxime Reges et Principes solicitari , \textbf{ ut polleant filiis pulchris et magnis ; | decet eos } in suis uxoribus quaerere magnitudinem , \\\hline
2.1.14 & que en otra manera son las mugers de gouernar \textbf{ e en otra los fijos ¶ } La primera razon se toma de parte dela manera de gouernar ¶ & quod alio regimine regendae sunt coniuges , \textbf{ et alio filii . } Prima via sumitur \\\hline
2.1.14 & lons quales el vno es paternal \textbf{ que es del padre al fijo . } Et el otro mater moianl & secundum Philosophum in Polit’ \textbf{ assimilantur } duo regimina domus , paternale , et coniugale . \\\hline
2.1.14 & e escoge para si el uaron \textbf{ mas los fijos non sonudgados } assi con el padre a cosas eguales & et eligit sibi virum . \textbf{ Filii autem non sic iudicantur } ad paria eum patre , \\\hline
2.1.14 & pues que assi es el aruernamiento matermonial soes \textbf{ assi natil commo el del padre al fijo . } Ca en ninguna manera los fijos non escogen assi su padre ¶ & Coniugale ergo regimen non est sic naturale , \textbf{ ut paternum : } quia filii nullo modo eligunt sibi patrem . \\\hline
2.1.14 & assi natil commo el del padre al fijo . \textbf{ Ca en ninguna manera los fijos non escogen assi su padre ¶ } Visto en qual manera se departe el & ut paternum : \textbf{ quia filii nullo modo eligunt sibi patrem . } Viso , quomodo differt regimen coniugale \\\hline
2.1.14 & aque non ordena ala mus . \textbf{ Ca los fijos son de enssennar } alas obras de caualleria & quam uxorem . \textbf{ Nam filii instruendi sunt ad opera militaria , } vel ciuilia , \\\hline
2.1.15 & El otro espaternal en el qual el padre \textbf{ enssennorea alos fijos . } Et el otro es enssenoreador & paternale , \textbf{ quo pater praeest filiis ; } et dominatiuum , \\\hline
2.1.16 & non deuamos vsar del casamiento ¶ \textbf{ La primera razon se toma de parte del dannamiento delos fijos ¶ } La segunda dela destenprança delas muger s¶ & quod in aetate nimis iuuenili non est utendum coniugio . \textbf{ Prima ratio sumitur | ex electione filiorum . } Secunda , ex intemperantia mulierum . \\\hline
2.1.16 & assi commo praeua el philosofo en las politicas \textbf{ dende sallen los fijos dannados quanto al cuerpo . } Ca en la mayor parte son muy flacos de cuerpo & ( ut probat Philosophus in Polit’ ) \textbf{ laeduntur inde filii quantum ad corpus , } quia ut plurimum sunt \\\hline
2.1.16 & por ayuntamiento dela hedat muy de moços \textbf{ seleunata danno alos fijos . } En essa misma manera & ex coniunctione nimis iuuenili , \textbf{ consurgit laesio filiorum : } sic ex tali coniunctione laeduntur filii \\\hline
2.1.16 & En essa misma manera \textbf{ por tal ayuntamiento sallen los fijos menguados } non solamente quanto al cuerpo mas avn quanto al alma & consurgit laesio filiorum : \textbf{ sic ex tali coniunctione laeduntur filii } non solum quantum ad corpus , \\\hline
2.1.16 & quanto ellos deuen ser mas cuydadosos \textbf{ por que los fijos dellos sean fermosos } e de grandes cuerpos & quanto ipsi plus debent esse soliciti , \textbf{ ut eorum filii sint formosi } et elegantes corpore , \\\hline
2.1.16 & Ca si en la hedat de grand moçedat las mugers se ayuntaren a sus maridos \textbf{ non solamente los fijos resçiben ende danno } mas avn ellas mismas se fazen destenpradas e orguollosas & Nam si in aetate valde iuuenili uxores suis viris copulentur , \textbf{ non solum filii inde laeduntur , } sed etiam ipsae uxores efficiuntur \\\hline
2.1.17 & que los omes non deue dar obra al casamiento \textbf{ en la he perdat de grand mançebia demanda en quet pon deuen dar mas obra ala generaçion delos fijos . } Et dizen que esto otorgan tan bien los naturales & non esse dandam operam coniugio in aetate nimis iuuenili : \textbf{ inquirit quo tempore | magis insistendum est procreationi filiorum , } et ait , quod tam a naturalibus , \\\hline
2.1.17 & Et pues que assi es conuiene a todos los çibdadanos vsar mas del casamiento \textbf{ en elt pon que es meior la generaçion de los fijos . } Enpero esto tanto mas conuiene alos Reyes & uti magis coniugio tempore , \textbf{ quo sit melior procreatio filiorum : } tanto tamen hoc magis decet Reges et Principes , \\\hline
2.1.17 & e alos prinçipes \textbf{ quanto mas les conuiene aellos de auer los fijos grandes e esforcados de cuerpo } euedes saber & tanto tamen hoc magis decet Reges et Principes , \textbf{ quanto decet eos elegantiores habere filios . } Mulierum autem mores \\\hline
2.1.19 & mas avn \textbf{ por la generaçion de los fijos . } Ca si las mugieres non guardassen castidat de ligero el fuero & non solum propter fidem seruandam suis viris , \textbf{ sed etiam propter procreandam prolem . } Nam si coniux castitatem non seruat , \\\hline
2.1.19 & e de los prinçipes \textbf{ quanto de los fijos non legitimos dellas } podria nasçer mayor contienda & coniuges Regum et Principum , \textbf{ quanto ex earum illegitima prole } potest \\\hline
2.1.19 & que van e demuestran desonestad . \textbf{ Ca non abasta que el fijo ageno non he de la h̃edat } de aquel que non es su padre . & quae videntur inhonestatem protendere : \textbf{ non enim sufficit } ut alius filius non succedat in haereditatem , \\\hline
2.1.19 & Mas conuiene que el \textbf{ padresea çierto de su fijo . } Et pues que assi es por que las señales desonestas & sed requiritur \textbf{ ut pater sit certus de sua prole . } Cum ergo signa inhonesta , \\\hline
2.2.1 & fincanos de dezir dela segunda parte \textbf{ en la qual ¶ diremos del gouernamiento del padre alos fijos . } Ende non abasta al padre dela casa saber gouernar a su muger & restat exequi de secunda , \textbf{ in qua agetur de regimine paternali : } non enim sufficit patrifamilias , \\\hline
2.2.1 & Ende non abasta al padre dela casa saber gouernar a su muger \textbf{ si non sopiere gouernar conueniblemente asus fijos . } Et pues que assi es & scire coniugem regere , \textbf{ nisi nouerit filios debite gubernare . } Sciendum igitur , \\\hline
2.2.1 & e la conpannia del uaron e dela muger e del sennor e del sieruo part enescan ala casa primera \textbf{ Mas la comunindat del padre e del fijo parte nesçan ala casa ya acabada en su ser } por que la casa primera es ante que la & ad domum primam : \textbf{ communitas vero patris , | et filii pertineat } ad domum iam inesse perfectam : \\\hline
2.2.1 & Et por ende nos propusie mos \textbf{ de determinar primeramente del gouernamiento de los fiios que de los sieruos } assi commo de aquello de que deuemos auer mayor cuydado . & ideo decreuimus prius determinare \textbf{ de regimine filiali | quam de regimine seruili , } tanquam de eo circa quod esse debet amplior cura . \\\hline
2.2.1 & primeramente queremos mostrar \textbf{ que conuiene a todos los padres de ser muy cuydadosos de los fiios . } Ca penssada la acuçia & primo ostendere uolumus , \textbf{ quod decet omnes patres | circa proprios filios esse solicitos . } Nam cognita solicitudine , \\\hline
2.2.1 & que los padres son comienço \textbf{ e razon de los fijos . Et los fijos han su ser } por los padres ¶ & ex eo quod patres sunt causa filiorum , \textbf{ et filii habent ab eis esse . } Secunda , ex eo quod sunt superiores et praestantiores illis . \\\hline
2.2.1 & La segunda por aquello que los padres son mayores \textbf{ e mas entendidos que los fijos ¶ } Mas la terçera & et filii habent ab eis esse . \textbf{ Secunda , ex eo quod sunt superiores et praestantiores illis . } Tertia vero , \\\hline
2.2.1 & e razon de los fijos \textbf{ e los fijos naturalmente han el ser de los padres . Conuiene alos padres de auer cuydado de los fijos } e ser cuydadosos dellos & et filii naturaliter \textbf{ a patribus esse habent , | decet patres habere curam filiorum , } et solicitari erga eos , \\\hline
2.2.1 & ¶ Et pues que assi es por que los padres enssenore a naturalmente \textbf{ alos fijos deuen sor muy } cuydadosos çerca el gouernamiento dellos ¶ & et prouidentiam totius Uniuersi . \textbf{ Patres ergo eo ipso quod naturaliter praesunt filiis , } debent circa eorum regimen esse soliciti . \\\hline
2.2.1 & por la qual cosa commo entre el padre \textbf{ e el fijo sea amor natraal } assi commo se praeua en el viij delas . ethicas . & quilibet enim solicitatur circa dilectum : \textbf{ quare cum inter patrem et filium sit amor naturalis , } ut probatur 8 Ethicorum , \\\hline
2.2.1 & Conuiene que los padres por amor natural \textbf{ que han alos fijos sean cuy dadosos della } aguer que todos los padres de una & decet patres ex ipso amore naturali , \textbf{ quem habent ad filios , | solicitari circa eos . } Licet omnes patres deceat solicitari \\\hline
2.2.2 & ¶ La primera razon se toma del entendimiento delos padres ¶ \textbf{ La legunda dela bondat delos fijos ¶ } La terçera del prouecho del regno¶ & ex eorum intelligentia . \textbf{ Secunda , ex bonitate filiorum . } Tertia vero , ex utilitate regni . \\\hline
2.2.2 & que en los otros ¶ \textbf{ La segunda razon para propuar esto mismo se toma dela bondat de los fijos . } Ca conuiene alos fijos de los Reyes & mentis industria et prudentia regitiua . \textbf{ Secunda via ad inuestigandum hoc idem , | sumitur ex bonitate filiorum . } Decet enim filios Regum et Principum \\\hline
2.2.2 & La segunda razon para propuar esto mismo se toma dela bondat de los fijos . \textbf{ Ca conuiene alos fijos de los Reyes } e de los prinçipes & sumitur ex bonitate filiorum . \textbf{ Decet enim filios Regum et Principum } maiori bonitate pollere quam alios : \\\hline
2.2.2 & del exienplo de beuir . \textbf{ Et pues que assi es los fijos de los Reyes } maguer non sean Reyes . & ut ceteri ex ipso possint sumere viuendi exemplum . \textbf{ Filii ergo regum licet } non omnes sint reges , \\\hline
2.2.2 & mucho les conuiene de ser sabios e buenos . \textbf{ Mas commo los fijos bengan a mayor bondat e a mayor sabiduria } si los padres ouieron cuydado dellos mas & maxime decet eos esse prudentes et bonos . \textbf{ Et cum filii perueniunt | ad maiorem bonitatem et prudentiam , } si patres circa eos sint soliciti , \\\hline
2.2.2 & e alos prinçipes de ser acuçiosos a sus fijos \textbf{ quanto los sus fijos deuen auer mayor sabidina } e mayor bondat ¶ & quanto filii eorum pollere debent \textbf{ maiori prudentia et ampliori bonitate . } Tertia via ad hoc ostendendum sumitur \\\hline
2.2.2 & que ayan sabiduria e bondat \textbf{ quanto mayor prouecho seleunata al regno dela bodat de los fijos delos Reyes } que deuen auer el prinçipado & ut polleant prudentia et bonitate ; \textbf{ quanto maior utilitas consurgit ipsi regno | ex bonitate filiorum Regum , } qui debent habere principatum et dominium in regno ; \\\hline
2.2.3 & Conuienne de uer onde toma comienço el gouernamiento paternal . \textbf{ Et por qual gduernamiento son de gouernar los fijos ¶ } pues que assi es conuiene de saber & unde sumit originem regimen paternum , \textbf{ et quo regimine regendi sunt filii . } Sciendum ergo triplex esse regimen . \\\hline
2.2.3 & Ca el padre \textbf{ enssennorea alos fijos por aluedrio } e non por & assimilatur regimini regali . \textbf{ Nam filiis praeest pater ex arbitrio , } non secundum conuentiones et pacta . \\\hline
2.2.3 & de esceger su gouernador . \textbf{ Mas non es en poderio de los fijos de escoger } assi mismos padres & nisi sit in potestate subiecti eligere sibi rectorem : \textbf{ non est autem in potestate filiorum eligere sibi patrem , } si ex naturali origine filii procederent a parentibus . \\\hline
2.2.3 & mas por natra al nasçençia \textbf{ los fijos desçenden de los padres . } Ca los fiios non escogen & non est autem in potestate filiorum eligere sibi patrem , \textbf{ si ex naturali origine filii procederent a parentibus . } Non enim sic filii eligunt sibi patres , \\\hline
2.2.3 & los fijos desçenden de los padres . \textbf{ Ca los fiios non escogen } assis padres & si ex naturali origine filii procederent a parentibus . \textbf{ Non enim sic filii eligunt sibi patres , } ut uxores viros . \\\hline
2.2.3 & commo las muger suarones Por la qual razon como el gouernamiento de los fijos \textbf{ sea por aluedrio e sea por el bien de los fijos . } Este gouernamiento non es semeiante al gouernamiento çiuil mas al Real . & Quare cum regimen filiorum sit ex arbitrio , \textbf{ et sit propter bonum ipsorum filiorum ; } huiusmodi regimen non assimilatur regimini politico , sed regali . \\\hline
2.2.3 & que el uaron deue \textbf{ ensennorear ala mug̃ e alos fiios . } assi conmo a franços e alibres . & Unde et Philosophus 1 Politicorum ait , \textbf{ virum praeesse mulieri , } et natis tanquam liberis . \\\hline
2.2.3 & Ca si el padre deue \textbf{ enssennorear alos fiios realmente } e por el bien dollos commo amar a alguno sea esso mismo & Nam si pater debet \textbf{ praeesse filiis regaliter } et propter bonum ipsorum : \\\hline
2.2.3 & coma cenienço de amor e desto paresçe manifestamente \textbf{ que non deuen ser couernados les fiios de aquel gouernamiento } de que deuer ser gouernados los sieruos . & Ex hoc ergo manifeste ostenditur , \textbf{ quod non eodem regimine debent regi filii , } quo regendi sunt serui . \\\hline
2.2.3 & Por la qual cosa si el gonernamiento del padre desto tora a comienco \textbf{ por que el fijo naturalmente es vria semerança que desçende del cadre . } Canmo segunr nacsta & ex hoc sumit originem , \textbf{ quia filius naturaliter est | quaedam similitudo procedens a patre : } cum secundum naturam ad huiusmodi similia fit dilectio , \\\hline
2.2.3 & enssennorear alos fiios \textbf{ por el bien de los fijos . } Et por ende non son de gouernar los fuos & patet quod filiis debet \textbf{ praeesse pater propter bonum filiorum . } Non ergo regendi sunt filii eodem regimine , \\\hline
2.2.4 & quanto es el amor de los padres alos fijos \textbf{ e de los fijos alos padres } por que nos conosca mos & Videndum est igitur quantus sit amor patrum ad filios , \textbf{ et filiorum ad patres , } ut nobis innotescat , \\\hline
2.2.4 & que los padres aman mas alos fijos \textbf{ que los fijos alos padres } ¶La primera razon se toma del alongamiento del tp̃o¶ & Sciendum ergo per Philosophum 8 Ethic’ triplici ratione probare , \textbf{ parentes plus diligere filios quam econtra . } Prima via sumitur \\\hline
2.2.4 & ¶La primera razon se toma del alongamiento del tp̃o¶ \textbf{ La segunda dela cercidunbre de los fijos ¶ } La terçera del ayuntamiento de los padres alos fijos ¶ & ex diuturnitate temporis . \textbf{ Secunda , ex certitudine prolis . } Tertia , ex unione parentum ad filios . \\\hline
2.2.4 & La segunda dela cercidunbre de los fijos ¶ \textbf{ La terçera del ayuntamiento de los padres alos fijos ¶ } La primera razon se puede prouar assi . & Secunda , ex certitudine prolis . \textbf{ Tertia , ex unione parentum ad filios . } Prima via sic patet . \\\hline
2.2.4 & Ca quanto el amor mas dura tanto se faze mayor . \textbf{ Mas el amor de los padres alos fijos es de } mayortpon que de los fijos alos padres . & tanto vehementior efficitur : \textbf{ amor autem parentum ad filios diuturnior est , } quam filiorum ad parentes . \\\hline
2.2.4 & Mas el amor de los padres alos fijos es de \textbf{ mayortpon que de los fijos alos padres . } Ca luego que los fijos nasçen los aman los padres . & amor autem parentum ad filios diuturnior est , \textbf{ quam filiorum ad parentes . } Nam statim cum filii nascuntur , \\\hline
2.2.4 & mayortpon que de los fijos alos padres . \textbf{ Ca luego que los fijos nasçen los aman los padres . } Enpero los fijos & quam filiorum ad parentes . \textbf{ Nam statim cum filii nascuntur , | parentes diligunt eos : } non tamen filii statim incipiunt amare parentes , \\\hline
2.2.4 & amorde los padres alos fiios \textbf{ que de los fijos alos padres } e es mas fuerte e mas afincado¶ & Diuturnior est ergo amor parentum ad filios , \textbf{ quam econuerso : } quare fortior et vehementior . \\\hline
2.2.4 & Ca los padres mas ciertos son de los fiios \textbf{ que los fijos de los padres . Ca los fijos non pueden ser } çiertos quales fueron sus padres & de sua prole \textbf{ quam proles de suis parentibus : | proles enim certificari non potest } qui fuerint parentes eius , \\\hline
2.2.4 & Et por ende mas pueden ser cercificados los padres de los fuos \textbf{ que los fijos de los padres . } Por la qual cosa aman mucħ & quam econuerso , \textbf{ propter quod et vehementius diligunt ipsam , } quam econuerso . \\\hline
2.2.4 & mas los padres alos fijos \textbf{ que los fiios alos padres } Ca si entre los padres e los fijos es amor natural & propter quod et vehementius diligunt ipsam , \textbf{ quam econuerso . } Nam si inter parentes et filios est amor naturalis , \\\hline
2.2.4 & que los fiios alos padres \textbf{ Ca si entre los padres e los fijos es amor natural } tanto este amor es mas guande & quam econuerso . \textbf{ Nam si inter parentes et filios est amor naturalis , } tanto huiusmodi amor est validior , \\\hline
2.2.4 & Et por aquesta razon se puede mostrar \textbf{ que las madres aman mas los fiios que los padres } por que son mas çiertas & Ex hac autem ratione ostendi potest , \textbf{ quod et matres plus diligunt filios , | quam patres : } quia de illis certiores existunt . \\\hline
2.2.4 & ¶ \textbf{ La tercera razon que peua esto mismo se toma del ayuntamiento de los padres alos fijos . } Ca los fijos son mas ayuntados & quia de illis certiores existunt . \textbf{ Tertia via probans hoc idem , | sumitur ex unione parentum ad filios . } Nam filii sunt magis propinqui \\\hline
2.2.4 & La tercera razon que peua esto mismo se toma del ayuntamiento de los padres alos fijos . \textbf{ Ca los fijos son mas ayuntados } e mas cercanos alos padres & sumitur ex unione parentum ad filios . \textbf{ Nam filii sunt magis propinqui } et magis uniti parentibus , \\\hline
2.2.4 & e mas cercanos alos padres \textbf{ que los padres alos fijos . } por la qual razon commo el amor faga algun ayuntamiento los fijos & et magis uniti parentibus , \textbf{ quam econuerso . } Quare cum amor quandam unionem importet , \\\hline
2.2.4 & e mas cercanos alos padres mas son amados dellos \textbf{ que non los padres de los fiios . } Ca assi ymagina el philosofo & filii tanquam magis uniti et magis propinqui parentibus , \textbf{ magis diliguntur ab ipsis , | quam econuerso . } Sic enim imaginatur Philosophus , \\\hline
2.2.4 & Ca assi ymagina el philosofo \textbf{ que los fijos son } assi como vn aꝑte de los padres . & Sic enim imaginatur Philosophus , \textbf{ quod filii sunt } quasi quaedam pars parentum : \\\hline
2.2.4 & assi como vn aꝑte de los padres . \textbf{ Ca el fijo es vna part entaiada del padre . } Mas la parte mas es ayuntada al su todo & quasi quaedam pars parentum : \textbf{ Nam filii est quaedam pars | a parentibus abscisa . } Pars autem magis unitur toti , \\\hline
2.2.4 & e que segunt en algrian manera non pertenesca a ellos . \textbf{ Empero los fijos non se mueuen } assi con tan grant feruor al amor deles padres . & et quod secundum aliquem modum non pertineat ad illos . \textbf{ Filii tamen non sic vehementer mouentur } ad dilectionem parentum : \\\hline
2.2.4 & assi alos fijos \textbf{ commo lo de los fijos alos padres } donde se sigue & quia quod est parentum , \textbf{ non sic pertinet ad filios , } et per consequens non est sic unitum eis : \\\hline
2.2.4 & que non es ayuntado aellos \textbf{ assi commo aquello que es de los fijos parte nesçe alos padres } Mas destas razones sobredichͣs & et per consequens non est sic unitum eis : \textbf{ sed quod est filiorum , | pertinet ad parentes . } Ex his autem viis Philosophi \\\hline
2.2.4 & conplidamente \textbf{ que parte nesçe alos padres de ser muy cuydadosos del gouernamiento de sus fijos . } Ca cada vno deue ser cuy dados o de aquellas cosas & sufficienter arguere possumus , \textbf{ quod ad parentes spectat solicitari | circa regimen filiorum , } quia quilibet solicitus esse debet \\\hline
2.2.4 & que ama con grant amor . \textbf{ Mas alos fiios parte nesçe de obedesçer alos padres } por que cada vno deue obedesçer a aquel & circa ea quae vehementi amore diligit . \textbf{ Ad filios vero pertinet | obedire parentibus : } quia quilibet illis obedire debet , \\\hline
2.2.4 & e mas les pertenesca de querer el bien de los fiios \textbf{ que los fijos de los padres . } Enpero non es cosa desconuenible & magis afficiantur circa filios , \textbf{ et magis intense velint bonum filiorum quam econuerso : } non est tamen inconueniens quantum \\\hline
2.2.4 & Enpero non es cosa desconuenible \textbf{ quanto ha algun bien de los fijos . } Mas amar alos padres & non est tamen inconueniens quantum \textbf{ ad aliquod bonum filios } magis diligere quam econuerso . \\\hline
2.2.4 & Mas amar alos padres \textbf{ que los padres alos fijos . } Ca assi commo dize el philosofo en el viij̊ . & ad aliquod bonum filios \textbf{ magis diligere quam econuerso . } Nam ut dicitur 8 Ethicorum parentes \\\hline
2.2.4 & pues que assi es el amor de \textbf{ e los padres alos fijos nasçe dela cosa } que es faze dora ala cosa que es fecha . & quia sunt ab illis . \textbf{ Amor ergo parentum circa filios procedit } a causa ad effectum , \\\hline
2.2.4 & Et desçende delo mas alto alo mas baxo \textbf{ Mas el amor de los fijos alos padres va } assi commo dela obra al obrador & et a superiori ad inferius : \textbf{ sed amor filiorum ad eos procedit } ab effectu ad causam , \\\hline
2.2.4 & e mantener le en ella . \textbf{ ¶ Et pues que assi es los padres allegan para los fiios . } Mas los fijos non allegan para los padres . & et conseruentur in esse . \textbf{ Parentes ergo congregant pro filiis , } non autem econuerso . \\\hline
2.2.4 & ¶ Et pues que assi es los padres allegan para los fiios . \textbf{ Mas los fijos non allegan para los padres . } Mas los fijos quando pueden furtan & Parentes ergo congregant pro filiis , \textbf{ non autem econuerso . | Immo filii , } cum possunt , furantur , \\\hline
2.2.4 & Mas los fijos non allegan para los padres . \textbf{ Mas los fijos quando pueden furtan } e cobdician los bienes de los padres & Immo filii , \textbf{ cum possunt , furantur , } et rapiunt bona parentum . \\\hline
2.2.4 & si los padres aman mas los fijos \textbf{ que los fijos alos padres } commo amara alguno sea essa misma cosa & Si ergo quaeratur , \textbf{ utrum parentes magis diligant } quam econuerso ; \\\hline
2.2.4 & Assi los padres mas aman alos fijos \textbf{ que los fiios alos padres } por que ayuntan algo para ellos & ad sufficientiam vitae : \textbf{ sic parentes magis diligunt filios , } quam econuerso ; \\\hline
2.2.4 & por que ayuntan algo para ellos \textbf{ e non los fiios para los padres . } Mas si fablaremos del bien & quia congregant pro eis , \textbf{ non ipsi pro illis . } Sed si loqueris de bono , \\\hline
2.2.4 & En esta manera los fues mas aman alos padres \textbf{ que los padres alos fijos . } Ca los padres mas sufren los denuestos & sic filii magis diligunt parentes , \textbf{ quam econuerso . } Nam potius parentes sustinent vituperia \\\hline
2.2.4 & Porque natra al cosa es \textbf{ que avn los fiios non pue den oyr } nin sofrir los deuuestos que fazen los otros a sus padres . & quam econuerso . \textbf{ Naturale est enim quod etiam usque ad auditum } contumelias parentum sustinere non possint . \\\hline
2.2.4 & nin ta grand pesar de los denuestos delos fujes \textbf{ commo los fijos de los padres . } ¶ Visto qual es el amor de los radres alos fijos . & Nam sic indignantur parentes de contumelia filiorum , \textbf{ quam econuerso . } Viso , qualis est amor \\\hline
2.2.4 & commo los fijos de los padres . \textbf{ ¶ Visto qual es el amor de los radres alos fijos . } Ca los padres son inclia a dos a amar los fijos & quam econuerso . \textbf{ Viso , qualis est amor | inter patrem et filium , } quia parentes afficiuntur ad filios \\\hline
2.2.4 & por el amor \textbf{ que han los padres alos fijos los deuen gouernar } Mas commo los fijos sean inclinados alos padres . & ex amore quem habent patres \textbf{ ad filios debent eos regere et gubernare . } Sed cum filii afficiantur ad parentes , \\\hline
2.2.4 & que han los padres alos fijos los deuen gouernar \textbf{ Mas commo los fijos sean inclinados alos padres . } assi commo aquellos que quieren auer en honrra e en reuerençia . & ad filios debent eos regere et gubernare . \textbf{ Sed cum filii afficiantur ad parentes , } tanquam ad eos , \\\hline
2.2.4 & e auer reuerençia a otro sea en alguna manera ser subiecto a el . \textbf{ Por ende assi commo por el amor que han los padres alos fijos los deuen gouernar ben } assy & quodammodo subiici illi ; \textbf{ sicut ex amore quem habent patres | ad filios debent eos regere et gubernare , } sic ex dilectione \\\hline
2.2.4 & mas aman los padres \textbf{ que los padres alos fijos . } Enpero sienpre dezimos & quod est honor et reuerentia , \textbf{ filii plus diligunt patres , } quam econuerso . \\\hline
2.2.4 & que los padres mas aman alos fijnos \textbf{ que los fijos alos padres } por que los padres son muy mas cuydadosos & Simpliciter tamen parentes plus dicuntur diligere filios , \textbf{ quam filii ipsos : } quia plus solicitantur , \\\hline
2.2.4 & et mas continuadamente cuydan del prouecho de los fiios \textbf{ que los fijos dela honrra } de la reuerençia de los padres . & et magis assidue cogitant de utilitate filiorum , \textbf{ quam filii de honore } et reuerentia ipsorum . \\\hline
2.2.5 & que tienen el padre e la madre . \textbf{ Ca si e las otras leyes los padres son acuçiosos de enssennar sus fijos en aquellas cosas } que son de su fe & quam parentes tenent . \textbf{ Si enim in aliis legibus parentes statim sunt soliciti erudire proprios filios } in iis quae sunt fidei suae , \\\hline
2.2.5 & que es vn dios poderoso todo criador de todas las cosas \textbf{ e que es padre e fijo e spunsanto } e que adam primero padre peco & quod unus est Deus omnipotens creator omnium , \textbf{ qui est pater et filius et spiritus sanctus . } Quod Adam primo parente nostro peccante , \\\hline
2.2.5 & e que el humanal linage fue ensuziado por el pecadodt \textbf{ Et por ende el fijo de dios } por que nos redimiesse tomo carne en la bien auentraada santamͣ & et humano genere per peccatum eius infecto , \textbf{ Dei filius , } ut nos redimeret , \\\hline
2.2.5 & por que nos redimiesse tomo carne en la bien auentraada santamͣ \textbf{ e nasçio della . Et que esse mismo fijo de dios padesçio } e fue muerto e soterrado & assumpsit carnem in beata Virgine , \textbf{ et est natus ex ipsa . | Quod ipse Dei filius } propter peccata nostra fuit passus , mortuus , et sepultus . \\\hline
2.2.6 & e los prinçipes ler mas acuçiosos \textbf{ por que los lo fijos sean acabados tan bien en el alma commo en el cuerro . } Et pues que assi es si ellos son acuçiosos en los h̃edamientos & et Principes magis solicitari debent \textbf{ ut proprii filii sint perfecti in anima , | quam in corpore . } Si ergo solicitantur \\\hline
2.2.6 & e en los aueres del mundo \textbf{ por que puedan fazer asus fijos Ricos } e acorrer los & et circa numismata , \textbf{ ut possint subuenire filiis } quantum ad indigentiam corporalem : \\\hline
2.2.6 & e alos prinçipes \textbf{ quanto la bondat de sus fijos es mas prouechosa al regno . } Et quanto dela maliçia dellos vernie mayor periglo a todo el regno & Tanto tamen hoc magis decet Reges et Principes , \textbf{ quanto bonitas filiorum est utilior ipsi regno , } et quanto ex eorum malitia potest \\\hline
2.2.7 & si non fueren acuçiosos \textbf{ en el gouernamiento de sus fijos asp } que en su moçedat & omnino reprehensibiles existunt , \textbf{ si non sic solicitantur erga regimen filiorum , } ut etiam ab ipsa infantia \\\hline
2.2.7 & si quisieren \textbf{ que los sus fijos departidamente } e derechamente fablen las palabras delas letras & et maxime Reges , et Principes , \textbf{ si volunt suos filios distincte } et recte loqui literales sermones , \\\hline
2.2.7 & e por entendimiento tanto \textbf{ mas conuiene alos fijos de los Reyes } luego en su moçedat de trabaiar se en las letris & et magis viget prudentia et intellectu : \textbf{ tanto magis decet filios Regum , } et Principum \\\hline
2.2.7 & Et por esta razd̃ serian tir annos e robadores del pueblo . \textbf{ Et pues que assi es por que los fijos de los Reyes } e de los prinçipes & et populi depraedator . \textbf{ Ne ergo filii Regum , } et Principum \\\hline
2.2.8 & Et estas sçiençias todas llaman libales \textbf{ por que los fijos de los liberales e de los francos } e delos nobles se ponian a aprender las . & Has autem omnes liberales vocant , \textbf{ eo quod filii liberorum , } et nobilium ponebantur ad illas . \\\hline
2.2.8 & e delos nobles se ponian a aprender las . \textbf{ Ca primeramente los fijos de los nobłs aprindian la guamatica } por que la guamatica & et nobilium ponebantur ad illas . \textbf{ Addiscebant enim primo filii nobilium grammaticam . } Nam grammatica secundum Alpharabium inuenta est , \\\hline
2.2.8 & Por ende la g̃matica es contada entre las sçiençias libales \textbf{ por que los fijos de los liberales } e delos nobles eran enssennados en ella ¶ & inter liberales scientias est computanda , \textbf{ quia filii liberorum et nobilium instruebantur in illa . } Secunda liberalis scientia \\\hline
2.2.8 & para argumentar gruessa mente . \textbf{ Et esta es neçessaria alos fijos de los nobles e de los libres } e mayormente alos fijos de los Reyes e de los prinçipes . & grossum et figuralem . \textbf{ Haec autem necessaria est filiis liberorum et nobilium , } et maxime Regum , et Principum : \\\hline
2.2.8 & Et esta es neçessaria alos fijos de los nobles e de los libres \textbf{ e mayormente alos fijos de los Reyes e de los prinçipes . } Ca a estos buenos parte nesçe de beuir entre las gentes & Haec autem necessaria est filiis liberorum et nobilium , \textbf{ et maxime Regum , et Principum : } quia horum est conuersari \\\hline
2.2.8 & delectaconnes conuenibles e sin danno . \textbf{ Et mayormente esto conuiene alos fijos delos rreys } e de los nobles & quae sunt licitae et innocuae . \textbf{ Maxime autem hoc decens est filiis liberorum et nobilium , } qui non vacantes moechanicis artibus , \\\hline
2.2.8 & por las quales se podrie mostrar \textbf{ que conuienea los fijos de los nobles } de aprender la musica & per quas ostendi posset , \textbf{ quod filios nobilium decet } addiscere musicam . \\\hline
2.2.8 & ala qual sçiençia \textbf{ por auentura los fijos de los nobles eran puestos } por quela musica non se podia saber acabadamente sin ella & docens proportiones numerorum , \textbf{ ad quam forte filii liberorum ideo tradebantur , } quia sine ea musica sciri non potest . \\\hline
2.2.8 & Et aesta eran puestos \textbf{ por auentura los fijos delos nobles } por que sin ella la astrologia & quae docet cognoscere mensuras et quantitates rerum . \textbf{ Ad hanc autem filii nobilium , } ideo forte tradebantur , \\\hline
2.2.8 & astrolozia çerca la qual en antigo tienpo \textbf{ por auentura trabaiaun a los fijos de los nobles } por que los gentiles e tan muy acuçiosos & dicitur esse astronomia , \textbf{ circa quam forte antiquitus filii nobilium ideo insudabant , } quia gentiles circa iudicia astrorum \\\hline
2.2.8 & Et por que esto se sabie por el astrolozia . \textbf{ Por ende los fijos de los libres } e delos nobles & et quia hoc per astronomiam cognoscitur , \textbf{ ideo filii liberorum } et nobilium volebant \\\hline
2.2.8 & mucho son aprouechables \textbf{ e neçessarias alos fijos de los bueons omes e de los nobles } mas assi commo paresçra en & valde sunt utiles \textbf{ et necessariae filiis liberorum et nobilium . } Immo ( ut in prosequendo patebit ) filii nobilium , \\\hline
2.2.8 & mas assi commo paresçra en \textbf{ signiendo esta materia los fijos de los nobles } e mayormente los fijos de los Reyes & et necessariae filiis liberorum et nobilium . \textbf{ Immo ( ut in prosequendo patebit ) filii nobilium , } et maxime filii Regum et Principum , \\\hline
2.2.8 & signiendo esta materia los fijos de los nobles \textbf{ e mayormente los fijos de los Reyes } e de los prinçipes & Immo ( ut in prosequendo patebit ) filii nobilium , \textbf{ et maxime filii Regum et Principum , } si velint politice viuere , \\\hline
2.2.8 & assi departidas de ligero puede paresçer cerça \textbf{ quales sçiençias deuen trabaiar los fijos de los nobles } e mayormente delos Reyes e de los prinçipes . & de leui patere potest , \textbf{ circa quas scientias filii nobilium , } et maxime Regum , \\\hline
2.2.8 & sin sabiduria de los otros . \textbf{ Et pues que assi es los fijos de los nobłs } maguera que entiendan ser caualleros & et legere absque aliorum scitu . \textbf{ Filii ergo nobilium } quantumcunque intendant esse milites , \\\hline
2.2.9 & commo delas falladas . \textbf{ Ca puesto que los fijas delos nobles } e mayormente de los Reyes & tam de inuentis quam de intellectis . \textbf{ Nam et dato quod filii nobilium , } et maxime Regum , \\\hline
2.2.9 & Por ende aquellas cosas pocas \textbf{ que cobdiçian saber los fijos de los nobles mas ligeramente e mas claramente e mas derechamente las pueden entender del sabio } que del nesçio & quae scire cupiunt , \textbf{ facilius , clarius , | et rectius intelligent a sciente , } quam ab inscio . \\\hline
2.2.12 & de dize ocho a nons e el vaton de veynte e dos \textbf{ porque en tal hedat se engendran los fijos ma acabados } segunt que dize el philosofo . & in masculo sex et triginta : \textbf{ in tali enim aetate } ( secundum ipsum ) \\\hline
2.2.15 & mostraremos qual cuydado deua ser \textbf{ tomado çerca los fijos . } Et pues que assi es¶ & ostendemus qualis cura \textbf{ circa filios sit gerenda . } Primo enim declarabimus \\\hline
2.2.15 & Et pues que assi es¶ \textbf{ Lo primero mostraremos qual cuydado auemos de tomar de los fijos fasta los siete años } ¶ lo segundo qual cuydado deuemos tomas dellos del septimo anero fasta el catorzeno & Primo enim declarabimus \textbf{ qualis cura habenda sit | de filiis usque ad septem annos . } Secundo qualis a septimo usque ad decimumquartum annum . \\\hline
2.2.15 & assi commo son los alimanes \textbf{ e los de nuruega de vannar los sus fijos en los trios muy frios } por que los fagan muy fuertes & quod apud aliquas Barbaras nationes consuetudo est \textbf{ in fluminibus frigidis balneare filios , } ut eos fortiores reddant . \\\hline
2.2.16 & Mas en este tp̃o que es del septimo año fasta el año xiiij̊ . \textbf{ son de penssar tres cosas enl gouernamiento de los fijos . } Ca el omne enł primero departimiento & quod est a septimo usque ad decimumquartum annum , tria sunt consideranda \textbf{ circa regimen filiorum . } Nam homo prima diuisione diuiditur in animam , et corpus . \\\hline
2.2.16 & deuen ser penssados \textbf{ enl gouernamiento de los fijos . } Et porque sea escusada la ꝑeza en los moços & A principio quidem natiuitatis , \textbf{ ut vitetur inertia puerorum , } assuescendi sunt pueri \\\hline
2.2.17 & quales ya fiziemos mençion . \textbf{ ni emos dessuso que trs cosas deuemos entender c̃ca los fijos . } Conuiene a saber en qual manera ayamos el cuerpo bien apareiado . & de quibus fecimus mentionem . \textbf{ Dicebatur supra circa filios tria intendenda esse , } videlicet quomodo habeant bene dispositum corpus , \\\hline
2.2.17 & mayormente se pue de fazer \textbf{ si vsaren los fijos a ex̉çiçios } e amouimientos conuenibles ¶ & quod maxime fieri contingit , \textbf{ si ad debita exercitia assuescant . } Viso , quomodo a quartodecimo anno ultra solicitari debent patres erga filios , \\\hline
2.2.17 & razono breues \textbf{ por que conuiene alos fijos de ser lubiectos } e obedientes a lus padres e alos vieios . & tres breues rationes , \textbf{ quare decet filios esse subiectos , } et obedire senioribus , et patribus . \\\hline
2.2.18 & non se deuen vsar egualmente . \textbf{ Ca los fijos delos Reyes } e de los prinçipes atales & sunt equaliter exercitandi : \textbf{ nam filii Regum et Principum } ad huiusmodi labores corporales \\\hline
2.2.18 & Enpero por que mas conuiene de ser sabios \textbf{ que lidiadores alos fijos de los Reyes } e de los prinçipes & eos esse magis prudentes quam bellatores , \textbf{ filii Regum et Principum } et maxime primogeniti \\\hline
2.2.19 & non sola mente nazcanfuos \textbf{ mas pueden nasçer fijos e fijas spues } que dixiemos qual cuydado deuen auer los padres & non solum oriantur filii et mares , \textbf{ sed oriri possunt filiae et foeminae : } postquam diximus \\\hline
2.2.19 & que dixiemos qual cuydado deuen auer los padres \textbf{ cerca los fiios fincanos de dez qual cuydado deuen auer los padres çerca las fijas } mas esto ha menester muy pequano tractado & qualis cura gerenda est circa filios , \textbf{ restat dicere , | qualis gerenda sit circa filias . } Sed hoc breui tractatu indiget : \\\hline
2.2.19 & ca assi commo conuiene alas madres \textbf{ de ser continentes e castas e guardadas e mesuradas en essa misma manera conuiene alas fijas de ser tales } Et pues que assi es estas cosas & Nam sicut decet coniuges esse continentes , \textbf{ pudicas , abstinentes , et sobrias : | sic decet et filias . } Haec ergo , et multa alia , \\\hline
2.3.1 & en qual manera conuiene a los maridos de gouernar a sus mugers . \textbf{ Et en qual maneta los padres deuen regir e gouernar a sus fijos . } finca de dezer dela terçera ꝑte & uiros suas coniuges regere , \textbf{ et qualiter patres suos filios gubernare . } Restat exequi de parte tertia , \\\hline
2.3.6 & que las mugeres \textbf{ e los fijos fuessen comunes } e que vsassen todos dellos & ipsas foeminas , \textbf{ et filios voluit esse communes . } Dicebat enim Socrates \\\hline
2.3.6 & Et avn en essa misma manera todos los omes amarien a todos los moços \textbf{ assi commo a sus fijos propraos } por que el padre non & Tunc enim omnes viri diligerent omnes foeminas tanquam proprias , \textbf{ sic etiam omnes homines diligerent omnes pueros tanquam filios proprios , } eo quod nesciret pater \\\hline
2.3.6 & e amarie a todos los mocos \textbf{ assi commo si fuessen sus fijos propreos } e esta tuela opinion de socrates e de platon & sed reputaret quemlibet proprium filium , \textbf{ et diligeret omnes pueros tanquam natos proprios : } sic ergo Socrates et Plato senserunt . \\\hline
2.3.6 & entre los que han cosas comunes algunas . \textbf{ caueemos que los hͣrmaros fijos de vn padre entre los quales seg̃tel philosofo enł . viij̊ } libro delas ethicas es amistança natural & inter participantes aliquid commune : \textbf{ videmus enim ipsos fratres ex eodem patre natos , } inter quos secundum Philosophum 8 Ethicorum est amicitia naturalis , \\\hline
2.3.15 & Conuiene que los prinçipes se ayan çerca ellos \textbf{ assi commo cerca de fijos . } Et conuiene les alos prinçipes delos gouernar non & quos virtus et amor boni inclinat ad seruiendum , \textbf{ decet principantes se habere quasi ad filios , } et decet eos regere non regimine seruili , \\\hline
3.1.4 & que acaban la casa son cosa natural . \textbf{ ca la casa se faze de comuidat de omne e de su muger e de sennor e de sieruo e de padre e de fijos . Et cada vna destas comuidades es cosa segunt natura bien } assi avn la comunidat del uarrio es cosa natural & sunt quid naturale : \textbf{ constat enim domus | ex communitate viri et uxoris , domini , et serui , patris et filii , quarum quaelibet est secundum naturam . } Sic etiam communitas vici est \\\hline
3.1.4 & mas avn por que la generaçion del uarro ha de ser cosa natal \textbf{ ca fazesse el uarion a tal monte de acresçentamiento de fijos e de metos e de parientes e de vezinos } assi commo dize el philosofo & fit enim vicus naturaliter \textbf{ ex crescentia filiorum collectaneorum , | et nepotum , } ut vult Philosophus 1 Polit’ \\\hline
3.1.6 & ¶La primera manera es aquella dela qual en el segundo libro feziemos mençion desuso do dixiemos \textbf{ que por las cresçençias de los fijos e de los nietos e de los bisnietos e de los parientes } La casa podria cresçet enuarrio & in secundo libro fecimus mentionem , \textbf{ ubi diximus quod propter excrescentiam filiorum collectaneorum } et nepotum domus potest in vicum , \\\hline
3.1.7 & e assi se signiria \textbf{ que ouiessen los fijos comunes . } ca por el uso comun delas fenbras los padres non ian ètificados de sus fijos & ut quod quilibet accederet ad quamlibet , \textbf{ et per consequens haberent communes filios : } nam quia propter communem usum foeminarum patres \\\hline
3.1.7 & mas cuydarian de cada vno moço \textbf{ que era su fijo propreo } por que paresçia a socrates e a platon & reputarent \textbf{ quemlibet puerorum esse filium proprium . } Videbatur enim Socrati et Platoni totam dissensionem ciuium consurgere \\\hline
3.1.7 & mas fuessen comunes \textbf{ los fijos seria comunes } e por que es muy grant amor de los padres alos fijos . & sed communibus , \textbf{ essent communes filii : } et quia patrum ad filios est maxima dilectio , \\\hline
3.1.7 & los fijos seria comunes \textbf{ e por que es muy grant amor de los padres alos fijos . } Por ende en aquella çibdat seria muy grant amor & essent communes filii : \textbf{ et quia patrum ad filios est maxima dilectio , } in ciuitate illa esset maximus amor , \\\hline
3.1.7 & Ca commo sea muy grant vnidat \textbf{ e grant ayuntamiento de los padres alos fijos los mas antiguos } cuydarian & Nam cum sit maxima unitas , \textbf{ et maxima coniunctio patrum ad filios , | antiquiores reputarent se } habere maximam unitatem \\\hline
3.1.9 & por que cuydarian los çibdadanos \textbf{ que todos los moços eran sus fijos propreos } e por ende en la çibdat seria muy grant amor Et pues que assi es nos podemos mostrar & cessarent litigia , \textbf{ quia crederent ciues omnes pueros esse filios suos , } et sic esset in ciuitate maximus amor . \\\hline
3.1.9 & Lo segundo por quelos çibdadanos non cuydarian \textbf{ que todos los moços fuessen sus fijos propreos } ¶ & Secundo , quia ciues non reputarent \textbf{ omnes pueros proprios filios . } Tertio , quia inter eos non esset magnus amor . \\\hline
3.1.9 & por la qual quarie socrates \textbf{ que los fijos fuessen comunes } esto non podria ser & communitatem tamen foeminarum , \textbf{ propter quam uolebat Socrates filios esse communes , } non est possibile obseruare absque litigiis . \\\hline
3.1.9 & que los çibdadanos cuydassen \textbf{ que todos los moços fuessen sus fijos propreos } por que alguons de los moços son semeiantes & non oporteret ciues \textbf{ omnes pueros reputare filios proprios . } Immo quia puerorum aliqui essent \\\hline
3.1.9 & e cada vno de los çibdadanos apropriaria \textbf{ assi por fijo a aquel que viesse } que lo semeiaua . & quilibet ciuis appropriaret sibi in filium , \textbf{ quem videret sibi esse similem . } Unde et Philosophus narrat 2 Politicor’ \\\hline
3.1.9 & la mayor maguera que las muger ssean comunes \textbf{ enpero los çibdadanos parten los fijos entre ssi segunt su semeiança } ca cada vno a proprea & uxores sunt communes : \textbf{ filios tamen ciues diuidunt secundum similitudinem , } quilibet enim appropriabat sibi in filios , \\\hline
3.1.9 & ca cada vno a proprea \textbf{ assi por fijos a aquellos que vee } que mas le semeian & quilibet enim appropriabat sibi in filios , \textbf{ quos videbat ei esse similes . } Nam et in aliis animalibus hoc contingit , \\\hline
3.1.9 & que todos los çibdadanos creyessen \textbf{ que todos los moços serian sus fijos propreos } La terçera razon paresçe & Videlicet , ut ciues omnes pueros crederent \textbf{ esse proprios filios . } Tertia via sic patet . \\\hline
3.1.9 & e el pho llama primos alos \textbf{ que son fijos de dos hͣmanos . } Et otro ama a otro & ( appellat autem Philosophus fratrueles \textbf{ qui sunt ex duobus fratribus nati ) } alius vero diligit \\\hline
3.1.10 & El segundo es abiltamiento de los malos omes . \textbf{ El terçero es non auer cuydado delos fijos ¶ } El quarto es destenpremiento en las cosas dela lururia . & Tertium , iniuria filiorum . \textbf{ Quartum , intemperantia venereorum . } Quintum , abusio parentum . \\\hline
3.1.10 & e cada vnos sean çiertos de sus parientes \textbf{ por que por esta non sabiduria los fiios non ayan de fazer miurias } nin tuertosa sus padres e a sus parientes & et quoslibet certificari de eorum consanguineis , \textbf{ ne propter ignorantiam filii } in proprios parentes et consanguineos \\\hline
3.1.10 & que ordeno socrates \textbf{ tire la cercidunbre de los fijos e el conosçimiento del parentesco } non es de ella bartal comunidat & quam Socrates ordinauit , \textbf{ tollat certitudinem filiorum | et notitiam consanguineitatis , } non est commendanda : \\\hline
3.1.10 & assi ca por esta tal comiundat delas mugers \textbf{ e delos fijos siguese abiltamiento de los nobles omes } e enxalçamiento de los labradores & Nam supposita communitate uxorum et filiorum \textbf{ sequeretur vilificatio nobilium , } et exaltatio agricolarum , \\\hline
3.1.10 & poues que assi es non puede ser de razon \textbf{ que las mugers et los fijos sean comunes } si non fuere puesta igual cura & et personarum vilium . \textbf{ Nam esse non potest uxores et filios communes , } nisi aequalis cura geratur de filiis nobilium , \\\hline
3.1.10 & si non fuere puesta igual cura \textbf{ de los fijos de los nobłs̃ } e de los mayores dela çibdat & Nam esse non potest uxores et filios communes , \textbf{ nisi aequalis cura geratur de filiis nobilium , } et custodum ciuitatis , \\\hline
3.1.10 & e de los mayores dela çibdat \textbf{ e essa misma sea puesta de los fijos de los labradores } e de las uiles perssonas & et custodum ciuitatis , \textbf{ quae geritur de filiis agricolarum } et personarum vilium . \\\hline
3.1.10 & e de las uiles perssonas \textbf{ por la qual cosa los fijos de los nobles seria abatidos } e los fijos de los uiles serian exalçados & et personarum vilium . \textbf{ Quare filii nobilium deprimuntur , } et vilium exaltabantur . \\\hline
3.1.10 & por la qual cosa los fijos de los nobles seria abatidos \textbf{ e los fijos de los uiles serian exalçados } e assi non seria amistança entre los çibdadanos & Quare filii nobilium deprimuntur , \textbf{ et vilium exaltabantur . } Non ergo erit amicitia inter ciues , \\\hline
3.1.10 & e por ende querer \textbf{ que los fijos de los çibdadanos sean comunes } e sea auido ygual cuydado & secundum proportionem seruitii . \textbf{ Velle ergo filios ciuium communes esse , } et aequalem curam geri \\\hline
3.1.10 & e sea auido ygual cuydado \textbf{ delos fijos de los nobles } e de los fijos de aquellos & et aequalem curam geri \textbf{ de filiis nobilium et ignobilium , est vilificare nobiles , } et exaltare ignobiles , \\\hline
3.1.10 & delos fijos de los nobles \textbf{ e de los fijos de aquellos } que non son nobles . & et aequalem curam geri \textbf{ de filiis nobilium et ignobilium , est vilificare nobiles , } et exaltare ignobiles , \\\hline
3.1.10 & assi ca conmo de la comunidat sobredichͣ delas mugiets \textbf{ e de los fijos se sigua iniuria et tuerto de los fijos } assi commo dize el philosofo en el viij libro delas ethicas non puede ser amistança acabada de vno a muchos & Nam supposita praedicta communitate , \textbf{ sequeretur in curia filiorum } nam ut dicit Philosophus 8 Ethic’ \\\hline
3.1.10 & que mucho plasa muchos non es cosa ligera \textbf{ por la qual cosa si cada vno de los çibdadanos cuydasse que cada vno de los moços era su fijo propreo } por que el amor del se partia en tanta muchedunbre de fijos & non est facile . \textbf{ Quare si quilibet ciuis crederet } quemlibet puerorum esse proprium filium , quia partiretur eius amor in tantam multitudinem , \\\hline
3.1.10 & si non fuesse loco en ninguna manera non podria sospechͣr \textbf{ que todos los moços fuessen sus fijos . } Et pues que assi es si a todos amassen & nullo modo suspicari posset \textbf{ omnes pueros esse suos filios . } Si ergo omnes diligerent tanquam filios , \\\hline
3.1.10 & por razon de dos o tres moços \textbf{ los quales cuydaria que eran suᷤ fijos propreos } e por que aquellos nonl serian conosçidos çiertamente & hoc esset ratione duorum vel trium puerorum , \textbf{ quos crederent esse proprios filios : } et quia illi non essent eis certitudinaliter noti , \\\hline
3.1.10 & ca por dos o por tres o por pocos mocos querera mar grant muchedunbre de moços \textbf{ assi conmo a fijos propreos } esto es poner poco de miel en muchͣ agua . & propter duos vel tres \textbf{ vel propter paucos pueros velle magnam multitudinem diligere puerorum tanquam proprios filios , } hoc est ponere parum de melle in multa aqua . \\\hline
3.1.10 & non se puede auer cuydado conuenible \textbf{ nin ç̉ança qual conuiene delos fijos . } El quarto mal se puede & non habeatur debita cura , \textbf{ nec diligentia debita erga filios . } Quartum malum sic ostendi potest \\\hline
3.1.10 & ca quando los padres e las madres non ouiessen conosçimiento de sus propios fijos \textbf{ e de sus fijas de ligero auria } y mal uso delas madres & Nam non habentibus parentibus cognitionem \textbf{ de propriis filiis et filiabus , } de leui fieret abusus parentum et consanguineorum , \\\hline
3.1.10 & e de las parientas \textbf{ assi que los fijos baratarian e iazdrian con sus madres } e los padres con sus fijas . & de leui fieret abusus parentum et consanguineorum , \textbf{ ut quod filii cognoscerent matres , } et patres filias . \\\hline
3.1.10 & assi que los fijos baratarian e iazdrian con sus madres \textbf{ e los padres con sus fijas . } Empero socrates quariendo escusar este mal dix̉o & ut quod filii cognoscerent matres , \textbf{ et patres filias . } Socrates volens hoc inconueniens vitare , \\\hline
3.1.10 & que al prinçipe dela çibdat pertenesçia de auer cuydado e acuçia \textbf{ por que los fijos non yoguiessen con sus madres } nin los padres con sus fuas . & habere curam et diligentiam , \textbf{ ne filii coirent cum matribus , } et patres cum filiabus . \\\hline
3.1.10 & Mas assi commo el pho dize esto non abasta \textbf{ ca si las fuas e los fijos fuessen comunes } e non les fuessen mostrados sus padres proprios & hoc non sufficit , \textbf{ quia si filii et filiae communes erant , } et non iudicabantur eis proprii parentes , \\\hline
3.1.10 & opimon de socrates dela comunidat \textbf{ que puso delas mugeres e de los fijos . } Et pues que assi es conuiene alos Reyes & reprehensibilis erat opinio Socratis \textbf{ de communitate uxorum et filiorum . } Decet ergo Reges et Principes \\\hline
3.1.11 & que ninguna cosa non fuesse prop̃a en la çibdat \textbf{ nin possessiones nin fructos nin fiios nin muger } o ca dizia & ut superius tangebatur , \textbf{ quod si nihil esset proprium in ciuitate nec possessiones nec fructus ne filii nec uxores , } quod nulla dissensio oriretur in ciuitate . \\\hline
3.1.14 & que non conuiene ala çibdat \textbf{ que las possessiones nin los fijos nin las mugieres sean comunes } assi commo ordenaua socrates & non expedire ciuitati possessiones , \textbf{ uxores , et filios esse communes , } ut Socrates statuebat ; \\\hline
3.1.15 & que todos los çibdadanos deuian auer las possessionos e las mugers \textbf{ e los fijos comunes } la qual cosa assi entendida & ciuitatem sic esse regendam et gubernandam , \textbf{ ut ciuibus communes essent uxores , et filii , et possessiones . } Quod si intelligitur \\\hline
3.1.15 & assy commo assi mesmo . \textbf{ En essa misma manera deue amar los fijos e las mugers } e las possessiones de los otros çibdadanos & sicut seipsum : \textbf{ sic debent diligere uxores , filios , } et possessiones aliorum , \\\hline
3.1.15 & assi commo si fuessen suyas . \textbf{ Mas en las mugers e en los fijos deue ser guardada comunidat } non solamente quanto al amor & ac si essent suae . \textbf{ In uxoribus autem ex filiis debet | reseruari communitas quantum ad amorem : } sed in possessionibus non solum debet \\\hline
3.1.17 & si non fuere establesçido \textbf{ que ayan fijos yguales } por que en vano se pone la ley & nisi statuatur \textbf{ quod habeant aequales filios : } frustra enim ponitur lex \\\hline
3.1.17 & si non poniendo \textbf{ que tantos fijos aya el vn çibdadano commo el otro } Mas por que algunos casamientos son en toda manera manentos & nisi totidem filii procedant ab uno ciuium , \textbf{ quot procedunt ab alio : } sed quia aliqua connubia sunt omnino sterilia , \\\hline
3.1.17 & que de los pobres \textbf{ e los fijos de les ricos serian mas pobres } que los fijos delos pobres . & quam pauperum : \textbf{ et filii diuitum pauperiores erunt , } quam filii pauperum . \\\hline
3.1.17 & e los fijos de les ricos serian mas pobres \textbf{ que los fijos delos pobres . } Et por esta razon en dos maneras & et filii diuitum pauperiores erunt , \textbf{ quam filii pauperum . } Ex hoc autem dupliciter contingent iniuriae , \\\hline
3.1.17 & assi commo el philosofo muestra llanamente en el segundo libro de la rectoriça \textbf{ e por ende farien tuertos los fijos de los pobres } alos fiios de los ricos & Philosop’ 2 Rhet’ . \textbf{ Iniuriabuntur ergo aliis . } Filii enim pauperum inflati , \\\hline
3.1.17 & e por ende farien tuertos los fijos de los pobres \textbf{ alos fiios de los ricos } ca los fuos de los pobres inchados & Iniuriabuntur ergo aliis . \textbf{ Filii enim pauperum inflati , } eo quod videant se \\\hline
3.1.17 & non solamente de parte de los fuos de los pobres \textbf{ mas avn de parte de los fijos delos ricos } ca assi commo dize el philosofo & ex parte filiorum pauperum , \textbf{ sed etiam ex parte filiorum diuitum : } quia ut dicitur 2 Polit’ \\\hline
3.1.17 & en el segundo libro delas politicas meesteres ala pazer dela çibdat \textbf{ que los fijos de los ricos } non sean sobuios & opus est \textbf{ ad pacem ciuitatis filios diuitum } non esse insolentes . \\\hline
3.1.18 & ante tanto deuen ser mas acuçiosos en esto \textbf{ que en aquello quanto las mugers e las fijas son cosas mas cercanas } que las sustançias de fuera ¶ La terçera razon se toma de parte delas rstezas las quales los omes fuyen por su poder . & circa hoc quam circa illud , \textbf{ quanto uxores et filiae sunt | quid magis coniunctum } quam exteriores substantiae . \\\hline
3.1.19 & Mas quanto alos lidiadores \textbf{ ordeno que los fijos de aquellos } que moriessen en la batalla fecha & Quantum ad bellatores ordinauit , \textbf{ quod filii eorum , } qui morerentur in bello facto \\\hline
3.2.5 & non solamente en su uida \textbf{ mas avn por heredat en sus fijos . } mas terna que el bien del regno es su bien propreo & debere se principari super regnum non solum ad vitam , \textbf{ sed etiam per haereditatem in propriis filiis , } magis reputabit bonum regni \\\hline
3.2.5 & por que toda la esperança del padre esta enlos fijos \textbf{ e con muy grant ardor los padres se mueuen al amor delos fijos . } por ende el Rey en toda cosa & Immo quia tota spes patris requiescit in filiis , \textbf{ et nimio ardore mouentur | patres erga dilectionem filiorum : } ideo omni cura \\\hline
3.2.5 & que los padres pueden dar \textbf{ assi el gouierno del regno asus fijos propreos . } ca dize que esto es assi commo por uirtud de dios & videlicet quod sic patres possint \textbf{ tradere regimen regni propriis filiis . } Innuit enim hoc esse \\\hline
3.2.5 & e non dura una prolongadamente \textbf{ assi que los fijos en esta manera ouiessen la heredat de los padres } o podemos dezir sinplemente & et non diu durabant , \textbf{ ut filii hoc modo succederent | in haereditatem paternam . } Vel simpliciter dicitur hoc esse diuinum , \\\hline
3.2.5 & çerca las cołas diuinales pocas uezes \textbf{ contesçe que los fiios regnen en pos de los padres . } Et si regnan los fijos & bene se habeant erga diuina , \textbf{ raro contingit filios regnare post patres : } et si regnant filii , \\\hline
3.2.5 & o nunca regna \textbf{ non los fijos de los fijos . } pues que assi es departe del Rey & et si regnant filii , \textbf{ vix aut nunquam regnabunt filii filiorum . } Ex parte ergo regis \\\hline
3.2.5 & e regua con crueldat . \textbf{ Mas si el gouernamiento real ui merepot heredat los fijos de los Reyes } non se enssoberuesçen & et inflati corde et inerudite regnant . \textbf{ Sed si regale regimen | per haereditatem vadat , } filii ex hoc non inflantur \\\hline
3.2.5 & ha por aguisado de obedesçer alos fijos \textbf{ e alos fijos de sus fijos } assi commo a cosanatal & obediuit patribus , filiis , et filiorum filiis , \textbf{ quasi naturaliter inclinantur } ut voluntarie obediant : \\\hline
3.2.5 & por suçession \textbf{ e si algun fallesçemiento ouiere en el fijo del Rey } a quien parte nesçe & filium succedere in regimine patris : \textbf{ et si aliquis defectus esset in filio regio , } ad quem deberet regia cura peruenire , \\\hline
3.2.5 & e non ouieren muy grant cuydado \textbf{ que los sus fijos de su ninnez sean bie doctrinados } e bien acostunbrados & si non nimia cura solicitentur \textbf{ quod filii ab ipsa infantia et disciplina } et bonis moribus sint imbuti ; \\\hline
3.2.5 & e non sabe ninguno \textbf{ a qual de los fijos pertenesçra el regno . } Por ende por que el bien comun non sea puesto a peligro de todos los fios deuen auer los padres grant cuydado & quid diuina prouidentia statuit , \textbf{ et nescitur cui filiorum succedat regnum . } Ideo ne periclitetur bonum commune , \\\hline
3.2.8 & e por ende por que en ssi mismos non pueden durar \textbf{ dessean naturalmente de durar en sus fijos . } si quier sean naturales & inde est ergo quod quia in seipsis durare non possunt , \textbf{ naturaliter appetunt perpetuari } in suis filiis siue sint naturales siue adoptiui . \\\hline
3.2.8 & Si los Reyes e los prinçipes non ouieren cuydado \textbf{ en qual manera los fijos ayan la hͣedat de los padres } e los postrimeros de los primeros & si Reges et Principes non solicitentur \textbf{ qualiter posteriores succedant } in haereditatem priorum : \\\hline
3.2.25 & Mas si aquellas reglas se tomaren en quanto el omne naturalmente \textbf{ dessea fazer fiios e carlos } assi podria ser de derecho natural & ex eo quod homo naturaliter \textbf{ appetit filios producere et educare : } sic esse poterunt de iure naturali , \\\hline
3.2.25 & mas de derech natural \textbf{ que dessear de engendrar fijos e criar los . } Et pues que assi es esta sera la orden entre estos de ti xu rechos & est plus de iure naturali , \textbf{ quam appetere procreare filios , | et nutrire prolem . } Erit igitur hic ordo , \\\hline
3.2.27 & en el primero libro delas politicas \textbf{ dizia omero que cada vno podia fazer leyes a sus fijos e a so mugers . } Caquaria ommo que las leyes e los mandamientos & quod unusquisque statuit \textbf{ legis pueris et uxoribus : | volebat enim Homerus } quod monitiones et praecepta , \\\hline
3.2.32 & que los nobles e los altos \textbf{ e los mas fijos dalgo sean mas buenos } e mas uirtuosos & decet nobiles et ingenuos \textbf{ esse magis bonos et virtuosos } quam ciues alios : \\\hline
3.2.32 & Et por ende el regno es dicho ser muchedunbre \textbf{ en la qual son muchs nobles e altos e fiios dalgo non } por que ellos se ayan en qual quier manera . Mas porque buian segunt uirtud & Inde est igitur quod regnum dicitur esse multitudo , \textbf{ in qua sunt multi nobiles et ingenui : } non quocunque modo se habentes , \\\hline
3.2.35 & Et generalmente todos los sus parientes . \textbf{ Et la muger e los fijos e los sus subditos . } Et por ende parte nesçe alos moradores del regno sinon & videlicet parentes et uniuersaliter omnes cognati , \textbf{ uxor filii , et subditi . } Spectat itaque ad habitatores regni \\\hline
3.2.35 & e generalmente a todos los moradores del regno \textbf{ que enssennen a sus fijos en sudlxxi ninnes } que amen al Rey & et uniuersaliter omnes habitatores regni \textbf{ ab ipsa infantia | prouocare filios } ad dilectionem Regis : \\\hline
3.2.36 & e el iusto non perdona a ninguno por iustiçia . \textbf{ Ca nin por fijo nin por amigo } nin por otro ninguon non deue dexar de fazer iustiçia & quia nec pro patre , \textbf{ nec pro filio , | nec pro amico , } nec pro aliquo alio \\\hline
3.2.36 & a quien dan las penas \textbf{ quando non perdonan a fijos nin a amigos . } nin a ninguons otros & timentur Reges et Principes , \textbf{ quando nec amicis , } nec aliis parcunt , \\\hline

\end{tabular}
