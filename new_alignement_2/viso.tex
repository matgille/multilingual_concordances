\begin{tabular}{|p{1cm}|p{6.5cm}|p{6.5cm}|}

\hline
2.3.3 & quilibet enim de populo \textbf{ hoc viso opinatur } principem esse tantum , quod quasi impossibile sit ipsum inuadere : & por que cada vno del pueblo \textbf{ quando esto vee pienssa en su coraço } que el prinçipe es tan grande \\\hline
2.3.19 & et velle se de quibuscumque inimicis intrommittere , \textbf{ nullatenus decet ipsos . Hoc viso restat } ostendere tertium , & de quales quier ofiçiales \textbf{ nin de sus ofiçios | ca esto ꝑtenesçe alos menores . } ¶ Esto iusto finça de demostrar lo terçero \\\hline
3.1.11 & Considerata ergo infirmitate hominum , \textbf{ et diligenter viso , } quod communiter populus & Et pues que assi es penssada la enfermedat de los omes \textbf{ e iusto acuçiosamente } que el pueblo comunalmente se desuia de carrera derecha \\\hline
3.1.16 & faciliter fieri poterat : \textbf{ quia viso numero ciuium } et computata multitudine camporum , & quando la çibdat en el comienço se establesçia \textbf{ ca iusto el cuento de los çibdadanos } e contado la muchedunbre de los canpos de ligero \\\hline
3.2.24 & determinans qua poena sint talia punienda . \textbf{ Hoc viso quantum } ad praesens spectat & tales cosas castigadas o condep̃nadas . \textbf{ ¶ Esto uisto quanto pertenesçe alo presente podemos mostrar dos departimientos } e dos diferençias \\\hline

\end{tabular}
