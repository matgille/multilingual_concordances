\begin{tabular}{|p{1cm}|p{6.5cm}|p{6.5cm}|}

\hline
1.1.1 & asi commo dize el philosofo \textbf{ en el septimo libr̊ dela metafisica , } ¶E por ende si del gouernamjento delos prinçipes & et non magis neque minus , \textbf{ ut vult Philosophus 7 Metaph’ . } Si ergo de regimine Principum , \\\hline
1.1.1 & asi commo dize el philosopho \textbf{ en el terçero libro de la rethorica } que quanto mayor es el pueblo tendo menores & sed pauci sunt vigentes acumine intellectus , \textbf{ propter quod dicitur 3 Rhetoricorum , } quod quanto maior est populus , \\\hline
1.1.2 & a asi commo dize el philosopho \textbf{ en el primer libro delos posteriores toda doctrina } e toda disçiplina desçiende & esse grossum et figuralem . \textbf{ Cum omnis doctrina } et omnis disciplina ex praeexistenti fiat cognitione , \\\hline
1.1.2 & mesmos¶ Onde dize el philosopho \textbf{ en el octauo libro delas ethicas } que las cosas amigables e buenas & ex iis quae sunt ad nos ipsos . \textbf{ Unde 9 Ethic’ scribitur , } quod amicabilia quae sunt ad amicos , \\\hline
1.1.2 & pues que asy es en el primo libro \textbf{ en el qual tractaremos del gouerna mjeto del omne . } En sy mesmo son quatro cosas de declarar e de demostrͣ & secundo scire suam familiam gubernare , \textbf{ tertio scire regere regnum , et ciuitatem . In primo autem libro in quo agetur de regimine sui , } sunt quatuor declaranda . \\\hline
1.1.2 & aquello que dize el philosofo \textbf{ en el segundo libro delas ethicas } que prouechosa cosa es en la scian moral & quae diuersificant mores et actiones . \textbf{ Inde est ergo , quod vult Philosophus 2 Ethic’ } proficuum esse morali negocio , \\\hline
1.1.2 & Ca segunt que dize el philosofo \textbf{ en el segundo libro delas ethicas señal dela } disponiconno delascina engendoͣda & Rursus quia \textbf{ ( ut dicitur 2 Ethicorum ) signum generati habitus , } est delectationem , \\\hline
1.1.2 & disponiconno delascina engendoͣda \textbf{ en el alma es auer } en la obra deleta çion o tristeza & ( ut dicitur 2 Ethicorum ) signum generati habitus , \textbf{ est delectationem , } et tristitiam fieri in opere , \\\hline
1.1.2 & Ca asy commo dize el philosofo \textbf{ en el terçero libro delas ethicas } quales cada vno & praestituimus nobis alios et alios fines . \textbf{ Nam ( ut dicitur 3 Ethic’ ) } qualis unusquisque est , \\\hline
1.1.3 & asy commo dize el philosofo tan bien \textbf{ en el libro de los granpes morals } commo en las ethicre¶ Ca algunos bienes son deletables ¶ & ut patet per Philosophum \textbf{ tam in Magnis moralibus , } quam etiam in Ethicis , \\\hline
1.1.3 & Ca segunt el philosofo \textbf{ en el nono delans ethicas } El omne mas es entendemiento e rrazon & homo enim \textbf{ ( secundum Philosophum 9 Ethi’ ) } maxime est intellectus et ratio : \\\hline
1.1.4 & Ca segunt que dize el philosofo \textbf{ en el primero libro delas ethicas } los tales & nam tales \textbf{ ( ut dicitur 1 Ethicorum ) } sunt vitam pecudum eligentes . \\\hline
1.1.4 & de dentro del ala aprouechado \textbf{ en el amor de dios } Mas esta deuoçion de dentro del alma & intenditur interna deuotio , \textbf{ et diuinus amor . } Hanc autem internam deuotionem \\\hline
1.1.5 & Ca segunt que dize el philosofo \textbf{ en el segundo libro delos fisicos } la fin non dize solamente cosa postrimera & sed contrarium finis : \textbf{ nam finis ut dicitur 2 Physic’ } non solum dicit \\\hline
1.1.5 & por la qual cosa dize el philosofo \textbf{ en el primero libro delas ethicas } que la fin e el bien es vna cosa Et el que tuelle o enbarga la fin tuelle & ø \\\hline
1.1.5 & Ca asy lo dize el philosofo \textbf{ en el terçero libro de las ethicas } que njguon non es bien auer turado & non laudamur nec vituperamur : \textbf{ unde in eodem 3 dicitur , } quod nullus est beatus nisi volens . \\\hline
1.1.5 & Ca asy lo praeua el philosofo \textbf{ en el segundo libro delas ethicas } que conuiene que las obras & in medietate consistens , \textbf{ ut dicitur 2 Ethic . ) } oportet operationes , \\\hline
1.1.5 & Onde dize el philosofo \textbf{ en el segundo libro delas ethicas } que non cunple solamente fazer buenas obras & et ex habitu efficit opus illud . \textbf{ Unde Philosophus 2 Ethic’ vult , } quod non sufficit agere bona , \\\hline
1.1.6 & ¶onde dize el philosofo \textbf{ en el primero libro delas ethicas } quariendo mostrar & in quo consistit suum perfectum bonum . \textbf{ Unde Philosophus 1 Ethicorum } describens felicitatem , \\\hline
1.1.6 & asi commo muestra el philosofo \textbf{ en el primero libro delas ethicas ¶ } pues que asi es & consistit perfectum bonum humanum , \textbf{ ut declarari habet 1 Ethic’ . } Cum igitur voluptates , \\\hline
1.1.6 & Ca asi lo dize el philosofo \textbf{ en el terçero libro delas ethicas } que si las tales delecta connes carnales fueren grandes e afincadas & rationem obnubilant , \textbf{ iuxta illud 3 Ethicorum . } Si tales delectationes magnae , \\\hline
1.1.6 & Ca asi commo dize el philosofo \textbf{ en el quinto libro delas politicas el prinçipado } e el señorio deue responder ala grandeza e ala dignidat de la persona & et totaliter diuinum . \textbf{ Nam ( ut dicitur 5 Politicorum principatus debet } respondere magnitudini , et dignitati , \\\hline
1.1.6 & Ca dize el philosofo \textbf{ en el libro delas topicas } que njnguon non deue escoger alos mançebos & eos esse Prudentes , \textbf{ indignum est enim Puerum principari . } Sed ( ut dicitur 1 Ethic’ ) \\\hline
1.1.6 & Mas asi commo el dize \textbf{ en el primero libro delas ethicas non ay departimiento entre moço e mançebo de hedat . } Et entre moço e mançebo en costunbres . & differt autem nihil iuuenis \textbf{ secundum aetatem , } aut secundum morem . \\\hline
1.1.7 & Ca puede contesçer asy commo dizeel philosofo \textbf{ en el primero libro delas politicas contado vna buena fabla } que alguno podia ser rico de mucho oro & Potest enim contingere \textbf{ ( ut dicitur 1 Politicorum ) } quod quis diues pecunia , fame moriatur . \\\hline
1.1.7 & e deua ser puesta por nos \textbf{ en el mayor bien que nos podemos dessear . } siguese que commo el alma sea meior & Nam cum felicitas sit bonum optimum , \textbf{ in optimo nostro quaeri debet . } Cum ergo anima sit potior corpore , \\\hline
1.1.7 & assi commo dize el philosofo \textbf{ en el quarto libro delas ethicas } en el capitulo dela maganimidat & ut vult Philosophus \textbf{ 4 Ethicorum cap’ de Magnanimitate ) } quia in opinione auari , \\\hline
1.1.7 & en el quarto libro delas ethicas \textbf{ en el capitulo dela maganimidat } que quiere dezir grandeza de coraçon & ut vult Philosophus \textbf{ 4 Ethicorum cap’ de Magnanimitate ) } quia in opinione auari , \\\hline
1.1.8 & quel faze propnamente es \textbf{ en el que se indina } e non en aquel a quien se inclina & constat inclinationem illam proprie esse in inclinante , \textbf{ non in eo cui inclinatio exhibetur . } Accidens enim proprie est in suo subiecto , \\\hline
1.1.9 & ¶Onde dize el philosofo \textbf{ en el primero libro dela Retorica } que las partes dela honrra son estas & per quaecunque signa exteriora : \textbf{ unde dicitur 1 Rhetoric’ } quod partes honoris sunt sacrificia , \\\hline
1.1.9 & honrra¶por que dize el philosofo \textbf{ en el quinto libro delas ethicas } que gualardon alguno deuemos dar alos prinçipes . & in hoc suam felicitatem ponere debeat , \textbf{ dicente Philosopho 5 Ethic’ } quod merces quaedam danda est Principibus , \\\hline
1.1.9 & asi commo dize el philosofo \textbf{ en el deçimo dela methafisica ¶ } Otrosi el conosçimiento de dios es tal en qua non puede caer enganno . & in tribus differt a notitia nostra : \textbf{ nam notitia Dei causat res , } notitia nostra causatur a rebus , ut vult Commen’ 12 Met’ . \\\hline
1.1.10 & Et por ende dize el philosofo \textbf{ en el se partimo libro delas } polticas que el prinçipado & non est optimus , neque dignus . \textbf{ Ideo 7 Politicorum dicitur , } quod principatus liberorum , \\\hline
1.1.10 & asi commo dize el philosofo \textbf{ en el septimo libro delas politicas } es mayor bien que la fortaleza ¶ & ut dicit Philosophus \textbf{ 7 Politic’ est maius bonum , } quam sit fortitudo . \\\hline
1.1.10 & por la qual cosa dize el philosofo \textbf{ en el septimo libro delas politicas } deno stando alos griegos & et incurret nocumentum \textbf{ secundum animam . Propter quod Philosophus 7 Politicorum vituperans Lacedaemones , } ponentes felicitatem \\\hline
1.1.11 & Ca assi commo dize el philosofo \textbf{ en el terçero libro dela } seth̃s fablando delas cosas & et neruis . \textbf{ Nam ( ut vult Philosop’ 3 De eligendis ) } sanitas est debita adaequatio humorum . \\\hline
1.1.11 & Et esto paresçe por el philosofo \textbf{ en el septimo libro delas politicas } que dize & Quod autem in bonis interioribus sit proprie felicitas , \textbf{ patet per Philosophum 7 Politicorum dicentem , } quod testis est nobis Deus , \\\hline
1.1.12 & Ca assi lo quiere el philosofo \textbf{ en el primero libro delas ethicas } ¶ & non videntur discerni felices a miseris , \textbf{ ut innuit Philosophus 1 Ethicorum . } In actu ergo , \\\hline
1.1.12 & que dize el philosofo \textbf{ en el primero libro delas ethicas la feliçidat } e la bien andança es obra del alma segunt uirtud acabada Et por que la uertud acabada & Felicitas ergo \textbf{ ( ut dicitur 1 Ethicor’ ) } est operatio animae \\\hline
1.1.13 & Et por esso dize aristotiles \textbf{ en el quinto libro delas ethicas } que el prinçipado & multas transgressiones efficerent . \textbf{ Propter quod Ethic’ 5 scribitur , } quod principatus virum ostendit . \\\hline
1.2.1 & Ca alguno dellos es propio al omne \textbf{ en el qual non partiçipa con las bestias . } assi commo es el appetito & nam quidam appetitus est in homine , \textbf{ in quo non communicat cum brutis , } ut est appetitus sequens intellectum : \\\hline
1.2.1 & segund razon assi commo dize el philosofo \textbf{ en el segundo libro delas ethicas . } Mas los poderios segund que son naturales non obedescena la razon . & secundum rationem , \textbf{ ut probari habet 2 Ethic’ . } Potentiae autem naturales \\\hline
1.2.2 & Et mostrado en qual manera han de seer \textbf{ en el entendimiento e en el appetito . } Pues que assi es auemos de departir destas uirtudes & quomodo distinguuntur virtutes , \textbf{ et quomodo in appetitu et intellectu existunt . } Distinguendum est igitur de virtutibus , \\\hline
1.2.2 & Et por ende dize el philosofo \textbf{ en el seyto libro delas ethicas . ca non puede ser el omne pradente e sabio } e non ser bueno . & Inde est ergo quod dicitur 6 Ethic’ \textbf{ quod impossibile est prudentem } esse non existentem bonum . \\\hline
1.2.2 & assi commo ya dixiemos de suso Mas solamente estas uirtudes ha de seer \textbf{ en el entendimiento e en el appetito . } Mas en nos ay dos apetitos . & non habent esse virtutes , \textbf{ sed solum esse habent in intellectu , et appetitu . } In nobis autem duplex est appetitus , intellectiuus , et sensitiuus . \\\hline
1.2.3 & las quales pone el philosofo \textbf{ en el segundo libro de las ethicas . } Mas el cuento destas uirtudes puede se assi tomar . & quas tangit Philosophos \textbf{ circa finem 2 Ethicor’ . } Possunt alio modo sic accipi hae virtutes . \\\hline
1.2.3 & Ca el temor e la osadia han de seer \textbf{ en el appetito en ssannador . } Mas si estas passiones selenantan del mal & Si vero huiusmodi passiones oriuntur \textbf{ ex malo praesenti , } quod est nobis iam illatum , \\\hline
1.2.3 & Enpero assi commo dize el philosofo \textbf{ en el quarto libro delas ethicas } por que non auemos nonbre propio & attamen \textbf{ ( ut dicitur 4 Ethicorum ) quia non habemus proprium nomen respectu huiusmodi virtutis , } nominatur nomine passionis deficientis , \\\hline
1.2.3 & assi commo dize el philosofo \textbf{ en el segundo libro delas ethicas . } pueden se tomar tres uirtudes . & ut communicamus cum aliis , \textbf{ sic ( ut dicitur secundo Ethicorum ) sumuntur tres virtutes . } Nam cum aliis communicamus in verbis , et operibus : \\\hline
1.2.3 & e en estos ha de seer el amor de honrra¶ Ahun assi los bienes que son . \textbf{ en el omne en conparaçion de los otros } entrs maneras se pueden entender . & Sic etiam bona in ordine \textbf{ ad alium tripliciter possunt considerari : } vel ut deseruiunt nobis ad manifestationem , \\\hline
1.2.4 & e mas que iusta dela qual fabla el philosofo \textbf{ en el septimo libro delas ethicas . } Ca assi commo algunos omes son bestiales & siue virtus heroica et superiusta , \textbf{ de qua determinatur 7 Eth’ . } Nam sicut aliqui homines sunt sicut bestiae , \\\hline
1.2.5 & e resçebidores de passiones \textbf{ en el alma derechamente } e non derecha mente . & per quam dirigimur in operanda ipsa agibilia . \textbf{ Amplius quia contingit nos passionari recte et non recte , } oportet dare virtutes aliquas , \\\hline
1.2.5 & o en la uoluntad o en el appe tito enssannador . \textbf{ en el appetito desseador ¶ } Et pues que assi es commo en el entendimiento pratico la mas prinçipal uirtud son la pradençia . & esse in quatuor potentiis animae , \textbf{ videlicet , in intellectu , in voluntate , in irascibili , in concupiscibili : } cum ergo in intellectu practico \\\hline
1.2.6 & ala qual uirtud llama el philosofo \textbf{ en el sexto libro delas ethicas . } Eubullia que quiere dezer uirtud para bien coseiar . & et bene confiliemur , \textbf{ quam Philosophus Ethic’ 6 appellat eubuliam , } idest bene consiliatiua . \\\hline
1.2.6 & Onde dize el philosofo \textbf{ en el sexto libro delas ethicas } que assi commo . & quod est praeceptiua inuentorum et iudicatorum . \textbf{ Unde Ethic’ 6 dicitur , } quod sicut eubuliae est inuenire , \\\hline
1.2.6 & Por la qual cosa dize el philosofo \textbf{ en el sexto libro delas ethicas } que el maestro & et praesupponit rectitudinem appetitus : \textbf{ propter quod scribitur Ethic’ 6 } quod artifex voluntarie peccans , \\\hline
1.2.7 & ¶ Onde dize el philosofo \textbf{ en el sexto libro delas ethicas } que aquellos tenemos por sabios & sit per prudentiam . \textbf{ Unde dicitur Ethic’ 6 quod illos extimamus esse prudentes , } qui sibi et aliis possunt \\\hline
1.2.7 & Ca assi commo dize el philosofo \textbf{ en el primero libro delas politicas } la hmuger ha poco de sabiduria & Sic etiam viri dominantur foeminis , \textbf{ quia ( ut declarari habet 1 Politic’ ) } foemina habet consilium inualidum . \\\hline
1.2.8 & Ca assi commo dize el philosofo \textbf{ en el segundo libro de la retorica en las obras } que acaesçe n o pueden acaesçer & quid euenire debeat in futurum . \textbf{ Nam ( ut scribitur secundo Rhetoricorum ) } in contingentibus agibilibus , \\\hline
1.2.9 & segunt el gouernamiento del tp̃o passado \textbf{ en el qual el su regno meior } e mas en paz fue gouernado . & Nam semper debet suum regimen conformare regimini retroacto , \textbf{ sub quo regnum tutius , } et melius regebatur . \\\hline
1.2.9 & que dize el philosofo \textbf{ en el sexto libro delas ethicas } que no puede ser & Bene igitur dictum est , \textbf{ quod scribitur Ethicorum 6 } quod impossibile est prudentem esse , \\\hline
1.2.10 & lphilosofo \textbf{ en el quinto libro delas ethicas } departe la iustiçia en dos maneras ¶ & quae sunt fugienda . \textbf{ Philosophus in 5 Ethicorum distinguit } duplicem Iustitiam , \\\hline
1.2.10 & Mas assi commo dize el philosofo \textbf{ en el primero libro dela grand ph̃ia moral . } La ley manda fazer las obras de todas las uirtudes . & quia adimplet praecepta legis . \textbf{ Sed ( ut dicitur primo Magnorum Moralium ) } lex praecipit actus omnium virtutum . \\\hline
1.2.10 & Et en essa misma manera ahun dize el philosofo \textbf{ en el quinto libro delas ethicas } que la ley manda non del enparar elaz en la fazienda & quod Iustitia legalis est perfecta virtus . \textbf{ Sic etiam Ethicorum 5 scribitur , } quod lex praecipit non derelinquere aciem , \\\hline
1.2.11 & La segunda se toma de parte del regno \textbf{ en el qual tal iustiçia deue ser } guardada¶ & Secunda , ex parte regni , \textbf{ in quo debet | talis Iustitia obseruari . } Prima via sic patet : \\\hline
1.2.11 & lo que dixo el philosofo \textbf{ en el primero libro dela grant ph̃ia moral } en el capitulo dela iustiçia & Bene ergo dictum est , \textbf{ quod dicitur 1 Magnorum moralium , } cap’ de iustitia , \\\hline
1.2.11 & en el primero libro dela grant ph̃ia moral \textbf{ en el capitulo dela iustiçia } que cosa iusta es vna cosa bien ygualada e bien ordenada & quod dicitur 1 Magnorum moralium , \textbf{ cap’ de iustitia , } quod iustum est \\\hline
1.2.12 & assi commo dize el philosofo \textbf{ en el quinto libro delas ethicas } qual quier iuez & ut haberi potest \textbf{ ex 5 Ethicor’ ipse iudex , } et multo magis ipse Rex , \\\hline
1.2.12 & Et por ende dize el philosofo \textbf{ en el quinto libro de las ethicas que paresçe que la iustiçia es muy mayor } e mas respladeciente entre todas . & Unde 5 Ethic’ dicitur , \textbf{ quod praeclarissima virtutum videtur esse Iustitia : } et neque Hesperus , \\\hline
1.2.12 & Et por ende dize el philosofo \textbf{ en el primers libro dela methaphisica } que señal manifiesta es de omne sabio & et quando scientia sua ad alios se extendit . \textbf{ Ideo scribitur 1 Metaphys’ } quod signum omnino scientis , \\\hline
1.2.12 & Et por ende dize el philosofo \textbf{ en el quinto libro delas ethicas } que el prinçipado e la dignidat & Ideo dicitur 5 \textbf{ Ethicorum principatus virum ostendit . } Propter quod et prouerbialiter dicitur : \\\hline
1.2.12 & mas es manifestada la perfecçion de bondat \textbf{ en el prinçipe } por la iustiçia & magis manifestatur perfectio bonitatis , \textbf{ quam in aliis : } sic ex Iustitia \\\hline
1.2.12 & Ca assi commo dize el philosofo \textbf{ en el quinto libro delas ethicas } que assi commo meior es el omne & quae ex iniustitia consurgit . \textbf{ Nam ( ut dicitur 5 Ethicor’ ) } sicut melior est \\\hline
1.2.13 & Por la qual cosa dize el philosofo \textbf{ en el primero libro dela grant phia moral } que si alguno fuere tan sin temor & quod non est Fortitudinis , sed insaniae . \textbf{ Propter quod 1 Magnorum moralium scribitur , } quod si quis valde impauidus existat , \\\hline
1.2.13 & por la qual cosa dize aristotiles \textbf{ en el tercero libro delas ethicas } que prinçipalmente es dicho fuerte & circa pericula alia . \textbf{ Propter quod dicitur 3 Ethicorum , } quod principaliter dicetur \\\hline
1.2.13 & Onde el philosofo en el terçero libro de las ethicas \textbf{ en el capitulo dela fortaleza } dize & Unde Philosophus 3 Ethicorum cap’ de Fortitudine ait , \textbf{ quod Fortitudo est } circa timores , et audacias . \\\hline
1.2.14 & Ca assi commo dize vegeçio \textbf{ en el libro del fecho dela caualleria } ninguno non duda de fazer & quae videntur esse terribilia . \textbf{ Nam ( ut dicit Vegetius in libro De re militari ) , } Nullus attentare dubitat , \\\hline
1.2.15 & Onde dize el philosofo \textbf{ en el terçero libro delas ethicas } que esta uirtud de tenprança ha de seer çerca & Est huiusmodi virtus \textbf{ ( ut vult Philosophus 3 Ethic’ ) } circa delectabilia illa , \\\hline
1.2.15 & Ca dize el philosofo . \textbf{ en el terçero libro delas ethicas que vn omne que aua nonbre philosemo } que era muy goloso de puchas & Credendum est enim in talibus iudicio gulosorum . \textbf{ Recitat autem Philosophus Ethicorum 3 de quodam , } nomine Phyloxenus , \\\hline
1.2.15 & Ca segunt el philosofo \textbf{ en el segundo delas ethicas deuemos fazer lo que fezieron los bieios de troya } contra & secundum Philosophum Ethicorum 2 hoc pati , \textbf{ quod senes Troiae patiebantur , } ad Helenam dicentes : \\\hline
1.2.16 & conpara el apetito desseador \textbf{ en el qual ha de ser el pecado dela } destenpranca al moço . & Unde Philosophus 3 Ethicorum vim concupiscibilem , \textbf{ secundum quam habet } esse vitium intemperantiae assimilat puero : \\\hline
1.2.17 & Mas assi commo dize el philosofo \textbf{ en el quarto libro delas ethicas } çerca los bienes prouechosos & Circa autem bona utilia \textbf{ ( ut determinari habet 4 Ethic’ ) } est duplex virtus : \\\hline
1.2.17 & Et por esso el philosofo \textbf{ en el quarto libro delas ethicas llama a estos tales non liberales } que quiere dezir non francos & Usurpans enim bona utilia , \textbf{ et non accipiens ea sicut debet , } nimis videtur auidus pecuniae . Propter quod Philosophus 4 Ethic’ usurarios , lenones , \\\hline
1.2.17 & segunt dize el philosofo \textbf{ en el quarto libro delas ethicas . } por la qual cosa non es la franqueza prinçipalmente en & quam illiberales , \textbf{ ut vult Philosophus 4 Ethic’ . } Quare non est liberalitas principaliter \\\hline
1.2.18 & egunt que dize el philosofo \textbf{ en el quarto libs de las ethicas . } La franqueza non esta en mucho dar & quam deficere . \textbf{ Vult Philosophus 4 Ethicorum liberalitatem } non esse in multitudine datorum , \\\hline
1.2.18 & Assi conmo praeua el philosofo \textbf{ en el segundo libro de los fisicos . } Pues que assi es nin en el gouernamiento de los omes & In naturalibus autem nihil est ociosum , \textbf{ ut probant physica dicta . } Ergo nec in regimine vitae humanae \\\hline
1.2.19 & assi commo dize el philosofo \textbf{ en el primero libro delas politicas . } por ende semeiaria a alguons & et naturam rerum , \textbf{ ut vult Philosophus 1 Politicorum , } non videtur sufficienter \\\hline
1.2.19 & Et por ende dize el philosofo . \textbf{ en el quarto libro delas ethicas } que el magnifico deue fazer honrradas espenssas en aquellas cosas & constituendo ( si facultates tribuant ) templa magnifica , sacrificia honorabilia , praeparationes dignas . \textbf{ Ideo dicitur 4 Ethicorum , } quod honorabiles sumptus , \\\hline
1.2.19 & assi commo dize aristotiles \textbf{ en el quarto libro delas ethicas de apareiar } muy conueniblemente las sus moradas & Decet enim magnificum \textbf{ ( ut dicitur 4 Ethicor’ ) } habitationem praeparare decenter , \\\hline
1.2.21 & Et por ende dize el philosofo . \textbf{ en el quarto libro delas etris } que la grant auariçia de tomar cuenta faze al omne de poca fazienda ¶ & non est magnificus , sed paruificus . \textbf{ Ideo dicitur 4 Ethic’ } quod diligentia ratiocinii est paruifica . \\\hline
1.2.22 & Mas el philosofo \textbf{ en el quarto libro delas ethicas dize } que la magranimidat & circa quae habet esse . Videtur autem Philosophus \textbf{ 4 Ethicor’ velle , } magnanimitatem esse circa honores , \\\hline
1.2.23 & assi commo dize el philosofo \textbf{ en el quarto libro delas ethicas¶ } La segunda propiedat que parte nesçe al magnanimo es & non parcat vitae , \textbf{ ut Philosophus ait 4 Ethic’ . Secundo competit magnanimo se habere bene circa retributiones . } Magnanimus enim parum appreciatur exteriora bona , \\\hline
1.2.24 & assi commo dize el philosofo \textbf{ en el quarto libro delas ethicas } es vn honrramiento de todas las uirtudes . & Magnanimitas ergo \textbf{ ( ut dicitur quarto Ethicorum ) } est quidam ornatus omnium virtutum . \\\hline
1.2.24 & Et por ende dize el philosofo \textbf{ en el quarto libro delas ethicas } que non pertenesce almagranimo foyr & opera singularum virtutum . \textbf{ Ideo dicitur quarto Ethicorum , } quod non congruit \\\hline
1.2.25 & Mas el magnanimo assi commo dize el philosofo \textbf{ en el quarto libro delas ethicas despreçia alos otros . } Et pues que assi es deuedes saber & magnanimus vero \textbf{ ( ut dicitur 4 Ethicorum ) | alios despicit . } Sciendum igitur \\\hline
1.2.28 & assi commo se puede prouar \textbf{ en el primero libro delas politicas . } Conuiene cerca las palauras & Si enim homo est naturaliter animal sociale , \textbf{ ut probari habet 1 Politicorum , } oportet circa uerba , \\\hline
1.2.29 & assi commo dize aristotiles \textbf{ en el quarto libro delas ethicas } que aquellos que non son uerdaderos & Quare cum mendacium sit semper fugiendum , \textbf{ ut dicitur 4 Ethicorum , } qui non sunt veraces nec aperti , \\\hline
1.2.29 & Et por ende el philosofo \textbf{ en el quarto libro delas ethicas en el capitulo dela uerdat } dize que declinar alo menos & et intendit despectiones moderare . \textbf{ Ideo Philosophus 4 Ethicorum cap’ } de veritate ait , \\\hline
1.2.30 & Et por ende dize el philosofo \textbf{ en el quarto libro delas ethicas } que paresce que la folgura & quia est quodammodo necessarius in vita . \textbf{ Ideo dicitur quarto Ethic’ } quod videtur requies \\\hline
1.2.30 & de los quales dize el philosofo \textbf{ en el quarto libro delas ethicas } que algunos mas se esfuerçan de & omnino risum facere , \textbf{ de quibus 4 Ethicorum dicitur , } quod magis conantur risum facere , \\\hline
1.2.30 & segunt que dize el philosofo \textbf{ en el quarto libro delas ethicas } deuen ser liberales e honestos . & quibus uti debemus , \textbf{ secundum Philosophum 4 Ethicorum , } debent esse liberales et honesti , \\\hline
1.2.31 & Et por ende dize el philosofo \textbf{ en el sexto libro delas ethicas } que las uirtudes morales rectifican & in finem illum . \textbf{ Ideo dicitur 6 Ethic’ } quod virtutes morales rectificant finem , \\\hline
1.2.32 & con grant estudio las palauras del philosofo \textbf{ en el sesto libro delas ethicas } uenmos departir quatro quados de malos & Si verba Philosophi \textbf{ in 7 Ethicorum diligentius considerentur , } distinguere possumus quatuor gradus malorum , \\\hline
1.2.32 & Ca segunt el philosofo \textbf{ en el libro dela generaçion muelle es dicho aquel } que da logar a otro & quod ipsum nomen designat . \textbf{ Nam secundum Philosophum in Meteora , } molle est illud , \\\hline
1.2.32 & Ca segunt que dize el philosofo \textbf{ en el sexto libro } delas ethicas el delicamiento es vna molleza . & et sunt nimis delicati . \textbf{ Nam , ut in 7 Ethicor’ scribitur , } delicia quaedam mollicies est . \\\hline
1.2.32 & Et por ende el philosofo \textbf{ en el septimo libro delas ethicas dize } que la continençia es mas de escoger e de loar que la persseuerança . & nisi contra passiones se teneat . \textbf{ Ideo 7 Ethicorum dicitur , } quod continentia elegibilior est , \\\hline
1.2.32 & Mas conuiene aellos de ser \textbf{ en el mas alto grado de los buenos } por que aquel que dessea de prinçipar & et quod non sint nec molles , nec incontinentes , nec intemperati , nec bestiales , \textbf{ sed oportet eos esse in summo gradu bonorum : } qui enim aliis dominari , \\\hline
1.2.34 & por que assi commo es dich̃o \textbf{ en el segundo libro delas ethicas a las } uirtudespertenesçe que por ellas primeramente sea el omne sabio & a ratione virtutis . \textbf{ Nam ( ut dicitur 2 Ethicorum ) } ad virtutem pertinet primo scire , \\\hline
1.3.1 & assi commo dize el philosofo \textbf{ en el quarto libro delas ethicas . } Ca la manssedunbre nonbra propriamente passion contraria ala saña . & sed hoc est propter vocabulorum penuriam , \textbf{ ut ait Philosophus 4 Ethic’ . } Mansuetudo enim proprie nominare videtur \\\hline
1.3.1 & assi commo fablamos aqui de passion \textbf{ en el appetiuo senssitiuo del seso . } Mas el appetito senssitiuo del seso assi commo mas largamente dixiemos de suso & ( ut hic de passione loquimur ) \textbf{ in appetitu sensitiuo . } Sensitiuus autem appetitus \\\hline
1.3.3 & or que las passiones fazen departimiento \textbf{ en el nuestro gouernamiento } e en lanr̃a uida & Passiones autem \textbf{ quia diuersificant regnum et vitam nostram , } ideo necessarium est ostendere \\\hline
1.3.3 & assi commo dize el philosofo \textbf{ en el primero libro delas ethicas . } Et otrosi por que el bien comun es ençerrado el bien propio de cada vno & diuinius quam singulare , \textbf{ ut dicitur 1 Ethic’ } et quia in communi bono \\\hline
1.3.3 & Ca segunt el philosofo \textbf{ en el quarto libro delas ethicas } la magnificençia & Erit magnificus ; \textbf{ quia secundum Philosophum 4 Ethic’ magnificentia potissime habet } esse circa diuina , et communia . \\\hline
1.3.4 & e en qual manera el Rey se deue auer a su regno \textbf{ en el terçero libro lo mostraremos mas conplidamente } uando determinamos dela ança orden delas passiones del alma dixiemos & et quomodo Rex se debeat habere ad ipsum regnum , \textbf{ in tertio libro diffusius ostendetur . } Cum determinauimus de ordine passionum animae , \\\hline
1.3.5 & assi commo dize el philosofo \textbf{ en el tercero libro delas ethicas } e las cosas passadas non se pueden mudar & quia nullus consiliatur de impossibilibus , \textbf{ ut vult Philosophus 3 Ethicorum , praeterita , } quae immutabilia sunt , \\\hline
1.3.6 & egunt que dize el philosofo \textbf{ en el terçero libro delas ethicas çerca las costunbres ¶ } Los penssamientos singulares de los omes son mas prouechosos que los generales . & Secundum Philosophum \textbf{ in 4 Ethicorum circa mores , } singulares considerationes magis proficiunt . \\\hline
1.3.6 & assi commo dize el philosofo \textbf{ en el primero libro de los fisicos . } por ende en este primero libro & Verum quia sunt nobis nota confusa magis , \textbf{ ut dicitur 1 Physicorum , } deo in hoc primo de moribus Regum oportet \\\hline
1.3.6 & Enpero non conuiene en todo en todo de desçender \textbf{ en el libro terçero alas cosas particulares } por que aquellos que son prouados en las cortes & magis particulariter tractabimus facta regni : \textbf{ non tamen etiam in illo libro expediet penitus } usque ad particularia descendere : \\\hline
1.3.6 & assi commo dize el philosofo \textbf{ en el primero libͤ de los grandes morales . } Et pues que assi es conuiene deuer & ut dicitur 1 Magnorum moralium , \textbf{ non est fortis , sed satuus . } Oportet ergo videre \\\hline
1.3.6 & que alguna cosa non se leunate \textbf{ en el regno } que pueda dannar e menguar el estado del regno . & Moderato enim timore omnes principantes timere debent , \textbf{ ne aliquid insurgat in regno , } quod eius bonum statum deprauare possit . \\\hline
1.3.6 & Ca assi commo es dicho en el segundo libro delas ethicas \textbf{ en el capitulo del temor } que el temor nos faze auer consseio . & Prima via sic patet : \textbf{ nam , ut dicitur 2 Rhetoric’ cap’ de timore , } Timor consiliatiuos facit , \\\hline
1.3.6 & assi commo fue dicho de suso \textbf{ en el capitulo dela fortaleza de ligero puede paresçer } en qual manera se deuen auer los Reyes ala osadia . & ut dictum fuit supra Capitulo de fortitudine : \textbf{ de facili videri potest , } quomodo se habere debeant \\\hline
1.3.7 & assi commo dize el philosofo \textbf{ en el segundo libro dela rectorica . } amares essa misma cosa & amare autem \textbf{ ( ut dicitur 2 Rhetor’ ) } est idem quod velle alicui bonum secundum se . \\\hline
1.3.7 & entre ellas las quales pone el philosofo \textbf{ en el segundo libro de la rectorica ¶ } Li premera diferençia es & sumuntur octo differentiae , \textbf{ quas assignat Philos’ 2’ Rhetor’ . } Prima differentia est , \\\hline
1.3.7 & segunt el philosofo \textbf{ en el septimo libro delas ethicas es assemeiada alos canes } o es assemeiada a los sieruos ligeros . & secundum Philosoph’ \textbf{ 7 Ethicor’ assimilatur canibus , } vel assimilatur seruis velocibus . \\\hline
1.3.8 & assi commo paresçe por el philosofo \textbf{ en el quarto libro delas ethicas } En essa misma manera el que pone que toda delectaçiones de esquiuar e de foyr pone que alguna delectaciones de segnir . & concedit loquelam \textbf{ ( ut patet per Philos 4 Metaphy’ ) } sic ponens omnem delectationem esse fugiendam , \\\hline
1.3.8 & Ca assi commo dize el philosofo \textbf{ en el nono delas ethicas los malos } et los uiciosos non gozan de ssi mismos & quia ( ut vult Phil’ 9 Ethicor’ ) \textbf{ Mali et vitiosi seipsis non gaudent , } non enim inueniunt \\\hline
1.3.10 & ¶ Mas sin todas estas passiones paresce que el philosofo \textbf{ en el segundo libro de la } rectorica cuenta otras seys passiones . & Sed praeter omnes has passiones Philosop’ \textbf{ 2 Rhetor’ } sex alias passiones enumerare videtur , \\\hline
1.3.10 & e declarado assi por el philosofo \textbf{ en el segundo libro dela rectorica } diziendo que es tristeza destos bienes tales & diffinitur \textbf{ a Philosopho 2 Rheto’ } quod est tristitia \\\hline
1.3.10 & Ca assi commo es di ch̃o \textbf{ en el segundo libro de la rectorica la miscderia non es otra cosa } si non alguna tristeza sobre mala paresçiente & Nam ( ut dicitur 2 Rhetor’ ) \textbf{ misericordia nihil aliud est , } quam tristitia quaedam \\\hline
1.3.10 & Ca segunt el philosofo \textbf{ en el segundo libro dela rectorica . Nemessis o indignaçiones auer tristeza de aquel } que ha algun bien & Nam ( secundum Philosophum 2 Rhetoricorum ) \textbf{ nemesis vel indignatio , | est tristari de eo } qui indigne videtur bene prosperari . \\\hline
1.3.11 & por la qual cosa es dicho \textbf{ en el quarto libro delas ethicas } que nos loamos los moços uergonosos & unde verecundari possint ; \textbf{ propter quod dicitur 4 Ethicorum , } quod laudamus iuuenes verecundos ; \\\hline
1.4.1 & que tanne el philosofo de los mançebos \textbf{ en el segundo de la rectorica tanne seys costunbres de loar } e seys de denostar ¶ & Inter alia quidem quae tangit Philosophus \textbf{ de iuuenibus 2 Rhetoricorum , | tangit sex mores laudabiles , } et sex vituperabiles . \\\hline
1.4.1 & assi commo praeua el philosofo \textbf{ en el segundo libro dela Rectorica . } Ca los mançebos son animosos e de buean esꝑança . & quod triplici ratione contingit , \textbf{ ut probat Philosophus 2 Rhetoricorum . } Sunt enim iuuenes animosi , \\\hline
1.4.1 & Et por ende dize aristotiles \textbf{ en el segundo libro de la rectorica } que los moços mesuran & sic credunt esse in aliis . \textbf{ Ideo dicitur secundo Rhetoricorum , } quod pueri sua innocentia alios mensurant . \\\hline
1.4.1 & Ca assi commo es dicho dessuso \textbf{ en el capitulo dela magranimidat } mucho & quia ( ut dicebatur \textbf{ in quodam capitulo de magnanimitate ) } maxime magnanimitas competit Regibus et Principibus , \\\hline
1.4.1 & Ca assi commo dize el philosofo \textbf{ en el segundo libro de la rectorica } losomes las mas cosas fazen e obran mal & eos esse miseratiuos . \textbf{ Nam ( ut ait Philosophus 2 Rhetoricorum ) homines } ut plurimum mala faciunt , \\\hline
1.4.2 & las quales pone el philosofo \textbf{ en el segundo libro de la rectorica¶ } Lo primero son los mançebos & sic enumerare possumus sex vituperabiles : \textbf{ quas etiam tangit Philosophus 2 Rhetoricorum . } Primo enim iuuenes sunt passionum insecutiui . \\\hline
1.4.2 & Et por ende dize el philosofo \textbf{ en el segundo libro de la rectorica } que los mançebos cobdician muy agudamente & sic ipsi habent voluntates et concupiscentias valde vertibiles . \textbf{ Ideo dicitur 2 Rhetoricorum , } quod iuuenes acute concupiscunt , \\\hline
1.4.2 & Onde dize el philosofo \textbf{ en el segundo libro de la rectorica } que aman mucho & sed omnia faciunt valde . \textbf{ Unde dicitur 2 Rhet’ } quod amant valde , \\\hline
1.4.3 & Por ende dize el philosofo \textbf{ en el segundo libro de la rectorica } que por que los uieios visquieron muchos años & sed credunt omnes alios esse deceptores . \textbf{ Ideo dicitur 2 Rhetoricorum , } quod senes multis annis vixerunt , \\\hline
1.4.3 & otrosa la peor parte . Et por ende dize el philosofo \textbf{ en el segundo libro de la rectorica } que por que los uieios uisquieron much s̃ annos non puede ser & et in peiorem partem referunt eorum opera . \textbf{ Unde dicitur 2 Rhetoricorum ; } quod quia senes vixerunt multis annis , \\\hline
1.4.3 & Ca segunt que dize el philosofo \textbf{ en el segundo libro de la rectorica el en triamiento dela sangre } e de los humores apareia la carrera para temer . & Sunt etiam timidi : \textbf{ quia ( ut dicitur 2 Rhetoricorum . ) } Infrigidatio praeparat viam formidini . \\\hline
1.4.3 & Et assi lo dize el philosofo \textbf{ en el segundo libro dela Rectorica } Ca por que la uerguença es temor de desonera non pertenesçe alos uieios & ut vult Philosophus 2 Rhetoricorum . \textbf{ Verecundia ergo , } cum sit timor inhonorationis , \\\hline
1.4.4 & assi commo dize el ꝑh̃o \textbf{ en el segundo libro de la rectorica¶ } Lo segundo los uieios son misericordiosos & ideo senes ratione frigiditatis concupiscentias habent remissas , \textbf{ ut vult Philosophus 2 Rhetoricorum . } Secundo senes sunt miseratiui : \\\hline
1.4.4 & Por ende dize el philosofo \textbf{ en el segundo libro dela rectorica } que los que son en el estado medianero & sed habent se medio modo . \textbf{ Ideo dicitur 2 Rhetoricorum , } quod qui sunt in statu , nec sunt omnibus credentes , \\\hline
1.4.4 & Mas assi commo diz el philosofo \textbf{ en el segundo sibro de la rectorica } son esforçados contenprànça & nec sunt formidolosi , \textbf{ et inuiriles ut senes : } sed sunt viriles cum temperantia , \\\hline
1.4.5 & assi common dize el philosofo \textbf{ en el segundo libro delan } rectorica si antigua miente desçendieron de aquel linage muchos granados omes et muchos nobles . & ( ut vult Philos’ 2 Rhet’ ) \textbf{ si ab antiquo ex illo genere } processerunt \\\hline
1.4.5 & assi commo dize el philosofo \textbf{ en el segundo del alma contesçe } alos nobles de auer el alma mas apareiada e de ser ellos mas enssennados e mas engennosos que los otros & Cum ergo molles carne aptos mente dicamus , \textbf{ ut vult Philos’ 2 de Anima : } contingit nobiles habere mentem aptam , \\\hline
1.4.6 & Ca assi commo dize el philosofo \textbf{ en el segundo libro dela rectorica } grant abiuamiento han para ser tales . & et despectores aliorum . \textbf{ Nam ( ut innuit Philosoph’ 2 Rhetor’ ) } maximum incitamentum habent , \\\hline
1.4.6 & Ca cuenta el philosofo \textbf{ en el segundo libro de la rectorica } que fue demandado a vna muger qual cosa era meior ser rico o ser sabio . & eo quod videant illos indigere bonis eorum . \textbf{ Recitat enim Philosophus 2 Rhetoricorum , } quod cum quaesitum fuisset \\\hline
1.4.6 & assi commo dize el philosofo \textbf{ en el segundo libro dela } rectorica deue ser a propreada a fado e ordenamiento de dios & acquisitio diuitiarum , \textbf{ ut dicitur 2 Rhetoricor’ } attribuenda est fato , \\\hline
1.4.6 & e esto es lo que dize el philosofo \textbf{ en el segundo libro de la rectorica } que vna buena costunbre se sigue & circa diuina bene se habere debent . \textbf{ Hoc est ergo quod dicitur 2 Rhetorico . } Unus optimus mos assequuntur diuites : \\\hline
1.4.7 & assi conmo dize el philosofo \textbf{ en el segundo libro de la rectorica } en toda manera han meiores costunbres que los ricos . & Potentes autem \textbf{ ( ut vult Philosophus 2 Rhetor’ ) } omnino habent meliores mores , \\\hline
1.4.7 & Et por ende dize el philosofo \textbf{ en el segundo libro dela rectorica } que si los poderosos fazen tuerto a & quam diuites . \textbf{ Ideo scribitur 2 Rhetor’ } si potentes iniuriantur , \\\hline
1.4.7 & Et por ende dize el philosofo \textbf{ en el segundo libro dela rectorica } que aquel que del otro dia aca es enrriqueçido non es sabidor en fecho de las riquezas & quae sunt bona fortunae . \textbf{ Ideo dicitur 2 Rhetor’ } quod nuper ditatum esse , \\\hline
2.1.1 & La qual segunt el philosofo \textbf{ en el terçero libro del alma } es organo e instrumento sobre todos los instrumentos . & sed dedit ei manum , \textbf{ quae secundum Philosophum 3 de Anima , } est organum organorum . \\\hline
2.1.1 & Et por ende el philosofo \textbf{ en el primero libro delas politicas entre las otras razones } que tanne & Unde et Philosophus 1 Politicorum \textbf{ inter alias rationes , } quas tangit , \\\hline
2.1.1 & Et por ende dize el philosofo \textbf{ en el primero libro delas politicas el que toma e escoge de beuir uida sola } e apartada non es parte dela çibdat & Ideo dicitur primo Politicorum \textbf{ quod eligens solitariam vitam , } non est pars ciuitatis : \\\hline
2.1.2 & Ca segunt dize el philosofo \textbf{ en el primero libro de las politicas } la comunindat dela casa & sed ciuilis : \textbf{ quia secundum Philosophum 1 Politicorum , } Communitas domus , \\\hline
2.1.2 & non parte nesçe a este libro \textbf{ en el qual tractamos del gouernamiento dela casa } mas pertenesçe al terçero libro & non spectat ad hunc librum , \textbf{ ubi agitur de regimine domus ; } sed ad tertium , \\\hline
2.1.2 & mas pertenesçe al terçero libro \textbf{ en el qual diremos del gouernamiento dela çibdat . paresçe que auemos trispassado los terminos desta arte determinando en el capitulo passado algunas cosas } que pertenesçen ala comunidat dela çibdat . & sed ad tertium , \textbf{ ubi agitur de regimine ciuitatis : | videmur transgressi fuisse limites huius artis , } determinando in praecedenti capitulo aliqua pertinentia ad communitatem ciuitatis . \\\hline
2.1.2 & Ca assi commo paresçe por el philosofo \textbf{ en el primero libro delas politicas } ca uatural nasçemiento & Naturalis enim origo ciuitatis \textbf{ ut patet per Philosophum 1 Politicorum , } hoc modo existit , \\\hline
2.1.4 & que assi comm̃ departe el philosofo el philosofo \textbf{ en el primero libro delas politicas } que algunas delas obras de los omes & quomodo domus sit communitas constituta in omnem diem . \textbf{ Ad cuius euidentiam aduertendum , } quod humanorum operum , \\\hline
2.1.4 & Et por ende dize el philosofo \textbf{ en el primero libro delas politicas } que assi commo la comunidat dela casa es establesçida & reperiatur in alia . \textbf{ Propter quod Philosophus 1 Politicorum ait , } quod sicut communitas domus \\\hline
2.1.4 & e assi commo dize el philosofo \textbf{ en el primero libro delas politicas } non solamente la casa es vna comiundat & immo ( ut infra patebit , \textbf{ et ut vult Philosophus 1 Polit’ ) } non solum domus est communitas quaedam , \\\hline
2.1.5 & Mas aqueles propraamente sieruo segunt dize el philosofo \textbf{ en el primero libro delas politicas } que fallesçe en el entondemiento & ille vero est proprie seruus \textbf{ ( ut patet per Philosophus 1 Politic’ ) } qui deficiens intellectu , \\\hline
2.1.5 & assi commo el philosofo dize \textbf{ en el segundo libro del alma } que los que son blandos de carnes & et deficiant corporalibus viribus \textbf{ ( iuxta illud Philosophi in 2 de Anima , } Molles carne aptos mente esse dicimus ) \\\hline
2.1.7 & en este segundo libro \textbf{ en el qual tractaremos del gouernamiento dela casa } segunt que enla casa han de ser tres gouernamientos . & tria esse determinanda in hoc secundo libro , \textbf{ in quo agitur de regimine domus , } secundum quod in ipsa domo tria contingit esse regimina ; \\\hline
2.1.7 & Ca segunt dize el philosofo \textbf{ en el primero libro delas politicas } en la comunidat dela casa & primo agendum est de regimine coniugali : \textbf{ quia secundum Philosophum 1 Politic’ } in communitate domestica , \\\hline
2.1.7 & Et esta razon tanne el philosofo \textbf{ en el octauo libro delas ethas } do dize & Hanc autem rationem tangit Philosophus \textbf{ 8 Ethic’ dicens : } Homo enim natura \\\hline
2.1.7 & Ca assi commo dize el philosofo \textbf{ en el octauo libro delas ethicas } luego man amano son departidas las obras del uaron e dela & Tertia ratio sumitur ex parte operum : \textbf{ quia ( ut dicitur 8 Ethicorum ) } confestim enim \\\hline
2.1.8 & Et esta razon tanne vałio el grande \textbf{ en el segundo libro de los fechos rememorables } en el capitulo delas constituçiones antiguas & Hanc autem rationem videtur \textbf{ tangere Valerius Maximus in libro de factis memorabilibus , } capitulo de institutis antiquis , \\\hline
2.1.8 & en el segundo libro de los fechos rememorables \textbf{ en el capitulo delas constituçiones antiguas } o dize & tangere Valerius Maximus in libro de factis memorabilibus , \textbf{ capitulo de institutis antiquis , } ubi ait , \\\hline
2.1.8 & por que los fijos es vn bien comun \textbf{ en el qual se ay unta el marido e la muger . } Conuiene por razon de los fiios & quoddam commune bonum \textbf{ in quo coniungitur vir et uxor , } ratione ipsius prolis decet \\\hline
2.1.9 & e las otras ainalias \textbf{ en el tienpo del parto . } Mas commo dicho es & onera filiorum \textbf{ se habent masculus et foemina tempore partus . } Sed cum dictum sit , \\\hline
2.1.12 & deuen querer \textbf{ en el muchedunbre de amigos } Et en quanto es ordenado el casamiento & sed prout ordinatur ad esse pacificum , \textbf{ quaerenda est multitudo amicorum : } prout vero ordinatur \\\hline
2.1.12 & deuen querer \textbf{ en el muchedunbre de riquezas . } Ca prouado es de suso & ad sufficientiam vitae , \textbf{ quaerenda est pluralitas diuitiarum . } Probabatur enim supra , \\\hline
2.1.13 & assi commo dize el philosofo \textbf{ en el septimo delas politicas . } non sabe ser uagarosa . & Nam mens humana \textbf{ ( ut innuit Philosophus 7 Politicorum ) } nescit ociosa esse ; \\\hline
2.1.14 & Et por ende es dicho tal gouernamiento politico e çiuil por que es semeiado a aquel gouernamiento \textbf{ en el qual los çibdadanos llamando a su señor muestran le los pleitos } e las con diconnes & quia assimilatur illi regimini , \textbf{ quo ciues vocantes dominum , | ostendunt ei pacta } et conuentiones \\\hline
2.1.15 & Et por ende dize el philosofo \textbf{ en el primero libro delas politicas } que na falmente se departe la muger del sieruo . & non erit ordinata ad seruiendum . \textbf{ Ideo dicitur 1 Politicorum , } quod naturaliter distinguuntur \\\hline
2.1.15 & Assi conmo dize el philosofo \textbf{ en el primero libro delas politicas do dize que entre los barbaros la fenbra } e el sieruo han vna orden & ut recitat Philosophus 1 Politicorum dicens , \textbf{ quod inter Barbaros foemina et seruus eundem habent ordinem . } Utebantur enim illi coniugibus tanquam seruis . \\\hline
2.1.15 & Onde dize el philosofo \textbf{ en el sexto delas politicas } que los pobres & si uxor et seruus habeant eundem ordinem . \textbf{ Unde dicitur 6 Politicorum , } quod pauperes , \\\hline
2.1.16 & segunt la doctrina del philosofo \textbf{ en el segundo libro delas ethicas } en tales cosas & quia cum negocium morale circa particularia consistat \textbf{ ( secundum doctrinam Philosophi 2 Ethicorum ) } in talibus particulares sermones plus proficiunt . \\\hline
2.1.18 & ¶ Et pues que assi es en el primero libto \textbf{ en el qual tractamos delas costunbres } en general & In primo ergo libro \textbf{ ubi uniuersaliter tractabamus de moribus , } non curauimus speciale capitulum facere de moribus mulierum : \\\hline
2.1.20 & assi commo dize el philosofo \textbf{ en el terçer libro delas ethicas } que el desseo dela cobdiçia carnal es tal que se non farta . & quia vis generatiua est nimis corrupta , \textbf{ et ( ut vult Philos’ 3 Ethic’ ) } insatiabilis est concupiscentiae appetitus : \\\hline
2.1.20 & assi commo dize el philosofo \textbf{ en el primero libro delas ethicas } por la qual cosa se tira la opim̃o de algunos omes bestiales & et ut venerea operetur , \textbf{ vult Philosophus 7 Ethicorum . } Propter quod eliditur \\\hline
2.1.21 & mas desconueinbles esto muestra el philosofo \textbf{ en el primero libro de las rectorica } do dize fablando de los laçedemonios & nec appetentibus licita sed illicita , \textbf{ ostendit Philosophus 1 Rhet’ } qui loquens de Lacedaemoniis , \\\hline
2.1.21 & las quales tanne andronico peri patetico \textbf{ en el libro que fizo delas uirtudes . } Et estas tres uirtudes son humisdat tenprança sinpleza . & quam tangit Andromicus Peripateticus in libello \textbf{ quem fecit de virtute . } Huiusmodi triplex virtus , \\\hline
2.1.21 & por uana eglesia podria pecar \textbf{ en el conponimiento del cuerpo } si non fuese tenprada . & non ornans se propter vanam gloriam , \textbf{ posset delinquere in ornatum , } si non esset moderata , \\\hline
2.1.21 & pueden las muger specar \textbf{ en el fallesçimiento del su conponimiento } ¶ Lo primero si esto se fiziere & ( ut communiter ponitur ) \textbf{ contingit delinquere circa defectum . } Primo si hoc fiat \\\hline
2.1.22 & Et por esso dize el philosofo \textbf{ en el segundo dela } rectorica que cobdiçia es de aquello que el ome non ha . & quod semper prohibitio auget concupiscentiam . \textbf{ Nam , ut dicitur 2 Rhetoricorum } concupiscentia est eius quod abest . \\\hline
2.1.23 & assi conmo dize el philosofo \textbf{ en el primero delas politicas es flaco . } Ca assi comm̃el moço ha consseio menguado por que & Consilium mulierum , \textbf{ ut dicitur 1 Politicorum est inualidum : } nam sicut puer habet consilium imperfectum , \\\hline
2.1.23 & Ca assi conmo dize el philosofo \textbf{ en el libro delas aian lias las ainalias menores e mas flacas . } mas ayna vienen a su conplimienta . & Nam , ut vult Philosophus in de Animalibus , \textbf{ omnia minora et debiliora } citius veniunt ad suum complementum . \\\hline
2.2.2 & Ca segunt que dize el philosofo \textbf{ en el libro delas aian lias entre las aiałias } la que es mas entendida & Nam secundum Philosophum in de Animalibus , \textbf{ quod est ex animalibus intelligentius , } magis habet solicitudinem circa filios : \\\hline
2.2.5 & assi commo dize el philosofo \textbf{ en el libro dela memoria e dela reminisçençia } Mas la razon desto es demostrada en la rectorica & ut dicitur in libro \textbf{ De memoria et reminiscentia : } ratio autem huius assignatur in Rhet’ \\\hline
2.2.7 & si non fueren acuçiosos \textbf{ en el gouernamiento de sus fijos asp } que en su moçedat & omnino reprehensibiles existunt , \textbf{ si non sic solicitantur erga regimen filiorum , } ut etiam ab ipsa infantia \\\hline
2.2.8 & quarian ser sabios \textbf{ en el arte dela astrologia . } Et pues que assi es assi sentieron los antigos delas artes libales & et nobilium volebant \textbf{ in astronomicis esse instructi . } Sic ergo antiqui de liberalibus senserunt , \\\hline
2.2.8 & ssobredichos de la qual sçiençia el philosofo \textbf{ en el primero libro dela methaphisica dize } que ninguna sçiençia non es mas digna que ella la qual cosa se deue entender destas sçiençias & et etiam quam aliqua praedictarum , \textbf{ de qua Philosophus ait primo Meta’ } quia nulla est dignior ipsa \\\hline
2.2.9 & Por que seg̃t que dize el philosofo \textbf{ en el primero libro delas politicas } mayor cuydado deuemos auer sienpre delas cosas & possessionibus , et rebus inanimatis : \textbf{ quia secundum Philosophum 1 Politicorum , } semper de animatis amplior cura habenda \\\hline
2.2.10 & por que asi commo es dicho de suso \textbf{ en el primer libro } los moços alos mançebos son segnidores de passiones & Loquuntur enim de leui lasciua , \textbf{ quia ( ut superius in primo libro dicebatur ) } iuuenes sunt insecutores passionum , \\\hline
2.2.11 & Ca assi commo da a entender el philosofo \textbf{ en el terçero libro delas ethicas } pequana delectaçiones & sumentes cibum auide . \textbf{ Nam ( ut innuit Philosophus 3 Ethicorum ) } modica delectatio est , \\\hline
2.2.11 & de aquel que la toma o de desordenaçion del alma \textbf{ en el resçibimiento del maniar } non solamente deuemos esquiuar la golosina & vel inordinationis mentis \textbf{ in sumptione cibi , } non solum cauendus est ardor et nimietas , \\\hline
2.2.11 & ¶ Lo quinto pecan los omes \textbf{ en el resçibimiento delas viandas } si demandat en viandas muy delicadas & obseruanda est in sumptione cibi . \textbf{ Quinto peccatur circa sumptionem cibi , } si quaerantur cibaria nimis lauta et delicata \\\hline
2.2.12 & muestra lo el philosofo \textbf{ en el septimo libro delas politicas } o dize & In qua autem aetate debeant uti coniugio , \textbf{ ostendit Philosophus 7 Poli’ dicens , } quod in muliere requiritur aetas decem et octo annorum , \\\hline
2.2.13 & assi commo prueua el philosofo \textbf{ en el viij̊ libro delas politicas } es neçessario enla vida & ut probat Philosophus \textbf{ 8 Poli’ } est necessarius in vita quod \\\hline
2.2.13 & que era propuesto \textbf{ en el comienço del capitulo . } Conuienea saber en qual manera se de una auer los moços & restat exequi de tertio , \textbf{ quod proponebatur in principio capituli , } videlicet qualiter se habere debeant \\\hline
2.2.15 & segunt dize el philosofo \textbf{ en el septimo libro delas politicas } faze a fortaleza del cuerpo . & Detinere autem spiritum et anhelitum \textbf{ secundum Philosophum septimo Politicorum , } facit ad robur corporis . \\\hline
2.2.21 & assi commo dize el philosofo \textbf{ en el ij̊ libro de la rectorica } quanto alguna cosa & eius quod abest , \textbf{ ut vult Philosophus 2 Retor’ } quanto aliquid \\\hline
2.3.3 & ca segunt el philosofo \textbf{ en el quarto libro delas ethicas } enł capitulo dela magnifiçençia & Prima via sic patet : \textbf{ nam secundum Philosophum 4 Ethicorum capitulo de Magnificentia , } maxime gloriosos et nobiles decet esse magnificos : \\\hline
2.3.4 & e por ende tanne paladio \textbf{ en el libro de la agnicultura seys cosas } que dize & ob infectionem aquae infirmitatem contrahant . \textbf{ Tangit autem Palladius in libro De agricultura sex } quae ait esse consideranda \\\hline
2.3.5 & mas segunt el philosofo \textbf{ en el primero libro delas politicas } la possession de las cosas es neçessaria en el & quae sunt necessaria in vita politica ; \textbf{ sed secundum Philosophum primo Polit’ } necessaria est rerum possessio in gubernatione domus , \\\hline
2.3.7 & es aquello que es triuio el philosofo \textbf{ en el primero libro delas politicas } que la natura dio anos tales cosas & Propter quod bene dictum est \textbf{ quod scribitur 1 Polit’ } quod natura dedit nobis talia , \\\hline
2.3.9 & assi commo da a entender el philosofo \textbf{ en el prim̃o libro delas politicas } buuendo en su sinpliçidat & sed etiam denariorum ad denarios . Antiquitus enim homines \textbf{ ( ut satis innuit Philosophus primo Politicorum ) } in simplicitate viuentes \\\hline
2.3.10 & segunt el pho \textbf{ en el primero libro delas politicas } e el dinero es comienço e fin por que esta e arte comiencaen erl dinero & Sed in ea \textbf{ ( secundum Philosophum primo Politicorum ) } denarius est elementum et terminus , \\\hline
2.3.11 & bien dicho es lo que dize el ph̃co \textbf{ en el primero libro delas politicas } que la usuraes de denostar & bene dictum est \textbf{ quod dicitur 1 Poli’ usuram esse } quid detestabile et contra naturam . \\\hline
2.3.12 & que auien de coger todos los labradores de aquel regno \textbf{ en el ano que auie de uenir . } Et por estarazon demando dineros prestados & et ab omnibus incolis regionis illius emit tantum oleum , \textbf{ quod recollecturi erant in anno futuro . } Mutuata ergo pecunia , et data atra bona pro futuro oleo : \\\hline
2.3.13 & la qual cosa praeua el philosofo \textbf{ en el primero libro delas politicas por quatro razones . } ¶ La primera razon se toma dela semeiança & et quod naturaliter expedit aliquibus aliis esse subiectos : \textbf{ quod probat Philosophus primo Polit’ quadruplici via , } sumpta ex quadruplici similitudine . \\\hline
2.3.13 & ssennorea ala feribra dela qual dize el philosofo \textbf{ en el primero libro delas politicas } que ha conseio muy flaco & naturaliter dominari foeminae , \textbf{ de qua dicitur primo Politicorum } quod habet consilium inualidum : \\\hline
2.3.16 & Et la razon desto pone el philosofo \textbf{ en el segundo libro delas politicas } do dize que alguas vezes peor siruen los muchs seruidores que los pocos . & Ratio autem assignatur 2 Polit’ \textbf{ ubi dicitur , } quod aliquando deterius seruiunt \\\hline
2.3.16 & Et esta regla es muy neçessaria \textbf{ en el gouernamiento delas casas de los Reyes } En las quales por la grandeza de los offiçios conuiene de & Est autem hoc documentum maxime necessarium \textbf{ in gubernatione domorum regalium , } ubi propter magnitudinem officiorum oportet \\\hline
2.3.16 & que son dichͣs \textbf{ en el quarto libro delas politicas . } Ca deuemoos assi ymaginar que comm̃o se ha la guand çibdat ala pequeña . & Ratio autem huius haberi potest \textbf{ ex iis quae dicuntur 4 Polit’ . } Debemus enim sic imaginari \\\hline
2.3.17 & assi commo prueua el philosofo \textbf{ en el sesto libro delas politicas¶ } Lo segundo çerca las uestidas es de penssar la semeiança de los siruientes . & decet eos magnifica facere , \textbf{ ut probat Philosophus 7 Poli’ . } Secundo circa vestitum consideranda est uniformitas ministrantium . \\\hline
2.3.19 & la qual cosa muestra el philosofo en alguna manera \textbf{ en el quarto libro delas ethicas do dize que conuiene alos magnanimos } e de altos coraçones de se auer & Quod aliquomodo tradit Philosophus \textbf{ 4 Ethi’ ubi vult , | quod ad humiles decet } magnanimos se habere moderate , \\\hline
2.3.19 & que son dichͣs \textbf{ en el quinto libro delas politicas } do dize que la persona del prinçipe & sumi potest \textbf{ ex iis quae dicuntur 5 Polit’ } ubi dicitur , \\\hline
2.3.19 & que algua familiaridat es de loar \textbf{ en el çibdadano } e en el cauallero & Est tamen aduertendum , \textbf{ quod aliqua familiaritas esset laudabilis in ciue vel milite , } quae non esset laudabilis in Rege : \\\hline
2.3.20 & por la qual cosa segunt el philosofo \textbf{ en el terçero libro del alma } conmola lengua sea dada & Quare cum secundum Philosophum \textbf{ in tertio de Anima , } lingua congruat in duo opera naturae , \\\hline
3.1.1 & assi ca assi commo dize el philosofo \textbf{ en el primero libro delas ethicas } que toda obra e toda electiuo dessea de auer algun bien . & Prima via sic patet . \textbf{ Quia ( ut dicitur primo Ethicorum ) } omnis actus et electio bonum \\\hline
3.1.1 & e esto es lo que dize el philosofo \textbf{ en el primero libro delas politicas } que si nos dizimos & maxime ordinatur ad bonum . \textbf{ Hoc est ergo quod dicitur primo Polit’ } quod si communitatem omnem gratia alicuius boni dicimus constitutam , \\\hline
3.1.2 & lo que dize el pho \textbf{ en el primero libro delas politicas } que la comunidat & Bene ergo dictum est , \textbf{ quod scribitur primo Politicorum } quod communitas , \\\hline
3.1.2 & Et por ende dize el pho \textbf{ en el primero libro delas politicas } que fue fecha la çibdat & secundum leges et virtuose . \textbf{ Ideo dicitur primo Politicorum } quod facta \\\hline
3.1.2 & aquello que dize el pho \textbf{ en el primero libro delas politicas } que el primo & bene dictum est \textbf{ quod scribitur primo Politicorum , } quod \\\hline
3.1.3 & assi commo dize el philosofo \textbf{ en el primero libro delas politicas } do dize que maldicho es el & Hi autem sunt illi , \textbf{ quos ( ut recitat Philosophus 1 Poli’ ) } maledicebat Homerus , dicens , \\\hline
3.1.3 & por la qual cosa dize el philosofo \textbf{ en el primero libro delas politicas } que los que non pueden beuir en conpanna con los otros & eligens altiorem vitam : \textbf{ propter quod scribitur primo Poli’ } quod non potens aliis communicare , \\\hline
3.1.4 & ca assi commo prueua el pho \textbf{ en el primero libro delas politicas } lo que es fin dela generaçion & illarum communitatum finis et complementum . \textbf{ Nam ut arguit Philosophus primo Politicorum } quod est finis generationis naturalium , \\\hline
3.1.4 & assi commo dize el philosofo \textbf{ en el primero libro delas politicas } por la qual cosa & et nepotum , \textbf{ ut vult Philosophus 1 Polit’ } propter quod si tale crementum est naturale , \\\hline
3.1.4 & assi capuado fue \textbf{ en el comienço del segundo libro delas politicas de parte dela palabra } que el omne es naturalmente & Probatur enim in principio secundi libri , \textbf{ ex parte sermonis } hominem esse naturaliter animal sociale , \\\hline
3.1.5 & assi ca segunt que dize el philosofo \textbf{ en el primero libro delas politicas } que la comunidat acabada & Prima via sic patet . \textbf{ Nam cum ait Philosophus primo Polit’ } quod communitas perfecta , \\\hline
3.1.6 & por el philosofo \textbf{ en el segundo libro dela methaphisica } e cerca la fin de los elencos & Verum quia ut patet \textbf{ per Philosop’ 2 Meta’ } et circa finem Elenchorum , \\\hline
3.1.9 & ca assi commo dixiemos de suso \textbf{ en el segundo libro esta es uida de o ensacabados non auer prỏo } e por ende los acabados son pocos¶ & Nam ( ut supra in secundo libro tetigimus ) \textbf{ haec est uita perfectorum : | non habere proprium : } perfecti autem sunt pauci . \\\hline
3.1.9 & assi commo dize el pho \textbf{ en el segundo libro delas politicas } por que vno ama a otro & Nam plus est modo de dilectione in ciuitate , \textbf{ ut Philosophus innuit 2 Politicor’ , } quia unus diligit alium tanquam filium , \\\hline
3.1.11 & assi commo dize el philosofo \textbf{ en el segundo libro delas politicas } en tres maneras se puede entender & Esse res communes , \textbf{ ut ait Philosophus 2 Politic’ , } tripliciter potest intelligi . \\\hline
3.1.11 & mas assi commo dize el philosofo \textbf{ en el segundo libro delas politicas } en los fechos particulares & inter quos tanta communitas obseruatur . \textbf{ Sed ut dicitur secundo Politicorum } in actibus particularibus oportet \\\hline
3.1.11 & que por la mayor parte han contiendas e uaraias por la qual cosa dize el philosofo \textbf{ en el segundo libro de las politicas } que de los siruientes & ostenditur ut plurimum homines habere lites et iurgia \textbf{ propter quod Philosophus ait 2 Polit’ } quod ab ipsis famulis , \\\hline
3.1.11 & assi commo cuenta el philosofo \textbf{ en el segundo libro delas politicas } tanta era la franqueza & Unde et apud Lacedaemones , \textbf{ ut recitatur secundo Politi’ } tanta erat liberalitas , \\\hline
3.1.12 & por que segunt el philosofo \textbf{ en el segundo libro delas politicas } alas bestias & insufficiens est : \textbf{ quia secundum Philosophum 2 Poli’ } bestiis nihil attinet oeconomice , \\\hline
3.1.13 & assi ca assi commo dize el philosofo \textbf{ en el quinto libro delas ethicas el prinçipado } e el senñorio muestra qual es el uaron & Prima via sic patet . \textbf{ Nam ut dicitur 5 Ethic’ } principatus virum ostendit \\\hline
3.1.13 & ca assi commo dize el philosofo \textbf{ en el tercero libro delas ethicas } el conseio non es dela fin & non est consiliari de pace et de concordia ciuium . \textbf{ Nam ut dicitur 3 Ethic’ } consilium non est de fine , \\\hline
3.1.13 & assi commo dize elpho \textbf{ en el segundo libro delas politicas } ca si menospreçiando alos vnos sienpre los otros fueren puestos en los ofiçios & et pacificum statum ciuium . \textbf{ Nam si spretis aliis semper iidem in magistratibus et praeposituris praeficiantur , } alii videntes se esse despectos \\\hline
3.1.13 & pone el pho \textbf{ en el segundo libro delas politicas } quando & Hanc autem tertiam rationem improbantem ordinationem Socraticam \textbf{ tangit Philosophus 2 Poli’ } cum ait . \\\hline
3.1.14 & por la qual cosa el philosofo \textbf{ en el libro delas politicas } reprehende a socrates deste & ø \\\hline
3.1.14 & Lo terçero erraua socrates \textbf{ en el ordenamiento dela çibdat } por que establesçia cuento determinado de los lidiadores & Tertio delinquebat Socrates \textbf{ in ordinando ciuitatem , } in constituendo determinatum numerum bellatorum . \\\hline
3.1.14 & ca segunt dize el philosofo \textbf{ en el segundo libro delas politicas } el que quiere poner leyes o fazer ordenaçion alguna en la çibdat a tres & in constituendo determinatum numerum bellatorum . \textbf{ Nam secundum Philosophum secundo Politicorum , } volens ponere leges \\\hline
3.1.16 & assi commo cuenta el philosofo \textbf{ en el segundo libro delas politicas } que se entremi tio del ordenamiento dela çibdat & ut narrat Philosophus \textbf{ 2 Politicorum intromisit se de ordine ciuitatis , } statuens quomodo posset \\\hline
3.1.17 & i fueren penssados los dichos del philosofo \textbf{ en el segundo libro delas politicas } quanto pertenesçe alo presente & volens eos aequatas possessiones habere . \textbf{ Si considerentur dicta Philosophi 2 Politicorum , } quantum ad praesens spectat , \\\hline
3.1.17 & ca assi commo dize el philosofo \textbf{ en el segundo libro delas politicas meesteres ala pazer dela çibdat } que los fijos de los ricos & quia ut dicitur 2 Polit’ \textbf{ opus est } ad pacem ciuitatis filios diuitum \\\hline
3.1.18 & assi commo cuenta el philosofo \textbf{ en el segundo libro delas politicas era ley } que para que las suertes antiguas fuessen guardadas & Sic etiam apud Locros , \textbf{ ut recitat Philosophus 2 Politic’ lex erat , } quod ad hoc ut antiquae sortes seruarentur illesae , \\\hline
3.1.18 & assi commo prueua el philosofo muy llanamente \textbf{ en el segundo libro delas politicas . } Mas desto diremos adelante mas conplidamente & circa reprimendas concupiscentias quam circa alia , \textbf{ ut plane probat Philosophus 2 Polit’ . } Sed de hoc inferius diffusius dicemus . \\\hline
3.1.18 & que es entre la ethica e la rectorica e la politica \textbf{ en el qual si los dichs fueren penssados superfiçialmeᷤte } e sin sotileza paresçe & De differentia Ethicae Rhetoricae et Politicae , \textbf{ ubi dicta superficialiter considerata contradicere videntur } his quae nunc diximus . Sed illa controuersia infra tolletur . \\\hline
3.1.19 & assi commo dize el philosofo \textbf{ en el segundo libro delas politicas } deuiese de vieios sabios escogidos & Volebat quidem principale praetorium , \textbf{ ut narrat Philosophus 2 Politicorum , } debere esse ex senibus electis : \\\hline
3.1.20 & Et segunt la suia del philosofo \textbf{ en el segundo libro dela mecha phisica deuemos dar grans aquellos } que se desuian de la uerdat & et secundum sententiam Philosophi 2 Metaphysicae \textbf{ debemus gratias reddere eis } qui a veritate deuiant , \\\hline
3.2.1 & e quantas son de penssar \textbf{ en el gouernamiento del regno e dela çibdat } Mas el philosofo en el terçero libro delas politicas & quae et quot consideranda sunt \textbf{ in tali regimine . } Videtur autem Philos’ 3 Polit’ tangere , \\\hline
3.2.1 & por el qual es de gouernar la çibdat \textbf{ en el tp̃o dela guerra ¶ } La primera razon se toma de aquellas cosas & in regimine \textbf{ quo regenda est ciuitas tempore pacis . } Prima via sumitur \\\hline
3.2.3 & assi commo se declara \textbf{ en el libro de los prinçipios } en las proposiconnes de proclo . & fortior est seipsa dispersa , \textbf{ ut declarari habet in libro de Causis , } et in propositionibus Procli . \\\hline
3.2.7 & Et esta razon tanne avn el philosofo \textbf{ en el quarto libro delas politicas } do dize que la tirania es muy mal prinçipado & Hanc autem rationem tangit Philosophus \textbf{ in eodem 4 Politicorum ubi ait , } tyrannidem esse pessimum principatum , \\\hline
3.2.7 & e esta razon tanne el philosofo \textbf{ en el quinto libro delas politicas } do dize que la tirnia es la postrimera obligarçia & multa mala efficere . \textbf{ Hanc autem rationem tangit Philosophus quinto Politicorum ubi ait , } tyrannidem esse oligarchiam \\\hline
3.2.8 & que es de dar \textbf{ en el arte del gouernamiento de los Reyes } Ca el arte semeia mucha la natura . & deriuari debet regimen , \textbf{ quod trahendum est in arte de regimine regum ; } est enim ars imitatrix naturae . \\\hline
3.2.8 & Ca assi commo dize el philosofo \textbf{ en el terçero libro delas ethicas } en el capitulo dela fortaleza & bene apta remunerare . \textbf{ Nam ( ut dicitur 3 Ethicorum capitulo de fortitudine ) } apud illos sunt fortes , \\\hline
3.2.8 & en el terçero libro delas ethicas \textbf{ en el capitulo dela fortaleza } entre aquellos son los omes muy fuertes & bene apta remunerare . \textbf{ Nam ( ut dicitur 3 Ethicorum capitulo de fortitudine ) } apud illos sunt fortes , \\\hline
3.2.9 & e las donaçiones \textbf{ en el bien comun } e en el bien del regno . & et oblationes ordinare \textbf{ in bonum commune regni , } sed etiam bona communia \\\hline
3.2.9 & Mas avn assi commo dize el philosofo \textbf{ en el terçero libro delas politicas deue enduziras Ꝯmugres propraas } por que sean familiares e bien querençiosas alas mugers & per quos bonus status regni conseruari potest , \textbf{ sed etiam ut ait Philosophus in Polit’ inducere debent uxores proprias } ut sint familiares et beniuolae uxoribus praedictorum : \\\hline
3.2.9 & ca assi conmo dize el pho \textbf{ en el quinto libro delas politicas } nunca es menospreçiado el mesurado . & ne a subditis habeatur in contemptu : \textbf{ nam ut dicitur 5 Polit’ } non contemnitur \\\hline
3.2.9 & assi commo dize el philosofo \textbf{ en el quinto libro delas politicas . } muchos de los tiranos non solamente non son tenpdos mas quieren paresçertenprados & Immo ( quod peius est ) \textbf{ ut narrat Philosophus 5 Politic’ } multi tyrannorum \\\hline
3.2.9 & Ca assi commo dize el philosofo \textbf{ en el tercero libro delas politicas } mas durable es regnar sobre pocos que sobre muchos . & ø \\\hline
3.2.9 & Ca cuenta el philosofo \textbf{ en el quanto libro delas politicas } que commo vn Rey dexasse vna parte de su regno . & per usurpationem et iniustitiam . \textbf{ Recitat autem Philosophus 5 Polit’ } quod cum quidam Rex partem sui regni dimisisset , \\\hline
3.2.10 & e el buen estado del regno \textbf{ en el qual el entiende } prinçipalmente ha de ser meiorado ¶ & et bonus status regni , \textbf{ quem principaliter intendit , } meliorari habet . \\\hline
3.2.12 & qual quier cosa de maldat es \textbf{ en el mal sennorio de los ricos } e en el mal señorio del pueblo todo es ayuntado enla tirania & quicquid peruersitatis est \textbf{ aliquo principatu diuitum , } et in peruerso dominio populi , \\\hline
3.2.14 & E por ende dize el pho \textbf{ en el terçero libro delas politicas } que la tirama quanto mayor es tanto menos dura & corrumpitur et durare non potest . \textbf{ Ideo dicitur in Politi’ } quod tyrannis quanto intensior , \\\hline
3.2.15 & Ca assi commo dize el philosofo \textbf{ en el primero libro delas } pol . & et non iniuriando eis . \textbf{ Nam ut innuit Philosophus in Poli’ } bene uti ciuibus \\\hline
3.2.15 & assi commo dize el philosofo \textbf{ en el segundo libre de la rectorica } que alli es grant salud & Timor autem consiliatiuos facit , \textbf{ ut dicitur 2 Rhet’ } ibi enim est magna salus , \\\hline
3.2.15 & assi commo dize el philosofo \textbf{ en el libro delas grandes costunbres . } la iustiçia guarda las cortesias & est habere ciuilem potentiam . \textbf{ Nam ( ut dicitur in Magnis moralibus ) } iustitia urbanitates conseruat . \\\hline
3.2.15 & ca assi commo dize el philosofo \textbf{ en el quinto libro delas politicas } mayor uirtud es menester & est esse regem bonum et virtuosum . \textbf{ Nam ut dicitur 5 Politicorum , } maior virtus requiritur \\\hline
3.2.15 & que estonçe entendra much \textbf{ en el bien comun del regno ¶ } La xͣ cosa que salua la poliçia es & si Rex sit bonus et virtuosus , \textbf{ quia intendet bono regni et communi . } Decimum , est Regem non ignorare qualis sit illa politia \\\hline
3.2.16 & lo que dize el philosofo \textbf{ en el terçero libro delas ethicas } que delas securas e delas luuias & ø \\\hline
3.2.16 & que por nuestros fech̃o non se pueden mudar . Et por ende dize el philosofo \textbf{ en el terçero libro delas ethicas } que en qual quier manera & quia non cadunt sub consilio nostro opera illorum hominum , quae propter nostra facta mutari non possunt . \textbf{ Ideo dicitur in Ethic’ } quod qualiter utique Scythae optime conuersentur , \\\hline
3.2.16 & Ca assi commo dize el philosofo \textbf{ en el terçero libro delas ethicas } cada vne de los omes toma conseio de aquellas obras & de iis quae sunt operabilia per nos : \textbf{ nam ut dicitur 3 Ethicorum } singuli autem hominum \\\hline
3.2.17 & Ca assi commo dize el philosofo \textbf{ en el sesto libro delas ethicas } aquel que demanda conseio & Est autem omne consilium quaedam quaestio , \textbf{ quia ( ut dicitur 6 Ethicorum ) } consilians siue bene siue male consiliatur , \\\hline
3.2.17 & Ca assi commo dize el philosofo \textbf{ en el tercero libro delas ethres } mague que todo conseio se aquestiuo & non tamen econuerso : \textbf{ nam , ut dicitur 3 Ethicorum , } licet omne consilium sit quaestio , \\\hline
3.2.17 & assi conmo dize el philosofo \textbf{ en el primero libro dela metaphisica } mas aprouecha el que ha la prueua dela cosa & Nam in talibus , \textbf{ ut dicitur primo Meta’ } plus proficit expertus , quam artifex . \\\hline
3.2.17 & alos quel guardananca era guaruido de grant fialdat . \textbf{ en el qual conssistorio } quando ellos entra una tirando dessi el amor propreo & silentique salubritate munitum : \textbf{ cuius limen intrantes abiecta priuata dilectione } ita dilectionem publicam inducebant , \\\hline
3.2.17 & En essa misma manera abn segunt dize el pho \textbf{ en el terçero libro de la rectorica } que vn poeta & Sic etiam ut recitat Philosophus \textbf{ 2 Rhetor’ } quidam poeta nomine Alexander videns \\\hline
3.2.17 & Pues que assi es lo que dize el philosofo \textbf{ en el sexto libro delas ethicas . que tomamos conseio en mucht pon . } mas obramos en poco tienpo & Bene ergo dictum est \textbf{ quod scribitur 6 Ethicorum | quod consiliemur multo tempore , } operamur autem prompte : \\\hline
3.2.18 & e es bien de creer \textbf{ en el paresçer de los omes } todas aquellas cosas & ø \\\hline
3.2.18 & assi commo dize el philosofo \textbf{ en el quato libro delas etihͣses por si mala cosa } e es de denostar ¶ & ut dicitur 4 Ethic’ \textbf{ per se est malum et detestabile . } Secundo consiliarii debent esse \\\hline
3.2.19 & conssei penssaremos en los diioschos del philosofo \textbf{ en el primero libro de delas } quales los omes han de dar & et amici , et sapientes . \textbf{ Si consideretur dicta Philosophi 1 Rhet’ } quinque sunt \\\hline
3.2.19 & poconuenible \textbf{ en el qual podamos tomar } conueinblemente uenganças dellos ¶ & nisi occurrat opportunitas temporis , \textbf{ in quo ex eis congrue possimus vindictas assumere . } Quintum circa quod sunt consilia adhibenda , \\\hline
3.2.19 & Ca assi commo dize el philosofo \textbf{ en el primero libro de la rectorica } enlas leyes es la salut dela çibdat . & est Lator legum : \textbf{ nam ut dicitur primo Rhetoricorum } in legibus est salus ciuitatis : \\\hline
3.2.20 & por quatro razones delas quales las tres tanne el philosofo \textbf{ en el primero libro de la rectorica } e la quat catanne en el sexto libro delas politicas & Quod quadruplici via inuestigare possumus , \textbf{ quarum tres tanguntur 1 Rhet’ } quarta vero tangitur 1 Polit’ . \\\hline
3.2.20 & Et estas tres cosas tanne elpho \textbf{ en el primero libro de la rectorica } diziendo & et quam paucissima arbitrio iudicum committere . \textbf{ Has autem tres rationes tangit Philosophus 1 Rhetoricorum dicens } quod maxime quidem contingit \\\hline
3.2.23 & Las quales dies cosas con que tanne el philosofo \textbf{ en el primero libro } uiene que tenga el iuez sienpre mientes & Quantum ad praesens spectat decem numerare possumus , \textbf{ quae videtur tangere Philos’ 1 Rhet’ } ad quae decet \\\hline
3.2.23 & Et por ende dize el pho \textbf{ en el primero libro de la rectorica } que el que iudga non deue tener & debet ad clementiam declinare ideo \textbf{ dicitur 1 Rhet’ } quod iudicans debet \\\hline
3.2.23 & Et por ende dize el pho \textbf{ en el primero libro de la rectonca } que el uiez non deue catar ala parte mas al todo & quae prius fecit : \textbf{ ideo dicitur 1 Rhetor’ } quod iudicans non debet respicere ad partem , \\\hline
3.2.23 & Et por ende dize el pho \textbf{ en el primero libro de la rectorica } que el ues deue catar & ut ad hanc particulam temporis in qua deliquit . \textbf{ Ideo dicitur 1 Rhetor’ } quod iudex non debet \\\hline
3.2.23 & Et por ende dize el pho \textbf{ en el primero sibro de la rectorica } que auemos de perdonar alas obras humanaleᷤ & est cum illo magis misericorditer agendum . \textbf{ Ideo dicitur 1 Rhet’ } quod indulgendum est humanis \\\hline
3.2.23 & Et por ende dize el philosofo \textbf{ en el primero libro dela } rectorica & est cum eo mitius agendum . \textbf{ Ideo dicitur 1 Rhet’ } quod Iudex epiikis idest superiustus debet indulgere humanis , \\\hline
3.2.23 & Et por ende dize el philosofo \textbf{ en el segundo libro de la rectorica } que contra los que son homillosos deue quedar la sanna & et prosternentes se coram eis . \textbf{ Ideo dicitur 2 Rhet’ } quod autem ad humiliantes cesset ira \\\hline
3.2.24 & assi commo dize el philosofo \textbf{ en el quinto libro delas ethicas ha en todo logar vn poderio . } Mas el & quia naturale ut traditur 5 Ethicor’ \textbf{ ubique eandem habet potentiam ; } sed positiuum ex principio antequam sic statutum , \\\hline
3.2.24 & lo que dize el philosofo \textbf{ en el primero libro del pari armenas } que las palabras e las smones son a uoluntad . & non tamen omnes proferunt idem idioma . \textbf{ Inde est etiam quod Philosophus 1 Perihermenias , } voces et sermones dicit esse ad placitum , \\\hline
3.2.24 & assi commo cuenta el pho \textbf{ en el primero libro de la } rectorica al derecho natural ayre o fuego & Inde est ergo quod Empedocles , \textbf{ ut recitat Philosoph’ 1 Rhet’ appellat } ius naturale aetherem siue ignem , \\\hline
3.2.25 & que la natura mostro a todas las aianlas \textbf{ en el qual todas las aiałias participanes } mas comun quel derecho delas gentes . & quod omnia animalia docuit \textbf{ et in quo omnia animalia communicant , } est communius quam ius gentium , \\\hline
3.2.26 & Et por ende dize el philosofo \textbf{ en el quarto libro delas politicas } que non conuiene de apropar las comunidades & Ideo dicitur 4 Politicorum \textbf{ quod non oportet } adaptare politias legibus , \\\hline
3.2.26 & las quales assi commo dize al philosofo \textbf{ en el x̊ libro delas ethicas han uirtud } e poderio de costrennir los malos . & propter tales statuere leges , \textbf{ quae ( ut dicitur 10 Ethicorum ) } coactiuam habent potentiam . \\\hline
3.2.27 & assi commo cuenta el philosofo \textbf{ en el primero libro delas politicas } dizia omero que cada vno podia fazer leyes a sus fijos e a so mugers . & dicebat Homerus \textbf{ quod unusquisque statuit } legis pueris et uxoribus : \\\hline
3.2.27 & Ca segunt dize el philosofo \textbf{ en el quarto libro delas politicas . } çerca las leyes deuen ser tomados dos cuydados . & et promulgatas custodire et obseruaret : \textbf{ quia secundum Philosophum 4 Politicorum } circa leges duplex cura esse debet : \\\hline
3.2.29 & corronper el rey que la ley ¶ la primera razon paresçe assi . Ca assi commo dize el philosofo \textbf{ en el quinto libro delas ethicas } el prinçipe deue ser guardador de derecho o de derecha ley . & Prima via sic patet . \textbf{ Nam ut dicitur 5 Ethicorum Princeps | debet } esse custos iusti , \\\hline
3.2.29 & Et esto es lo que dize el philosofo \textbf{ en el tercero delas politicas } que mas de escoger es & quam Regi optimo Rege . \textbf{ Hoc est ergo quod ait Philosophus 3 Politicorum , } quod eligibilius est principari lege , \\\hline
3.2.29 & lo que dize el philosofo \textbf{ en el tercero delas politicas } que en el derecho gouernamiento & Propterea bene dictum est \textbf{ quod innuit Philosophus tertio Politicorum } quod in recto regimine principari \\\hline
3.2.29 & assi conmo dize el philosofo \textbf{ en el quinto delas ethicas non se pueden mesurar } por regla derecha & ut vult Philosophus 5 Ethic’ \textbf{ non possunt | mensurari regula inflexibili , } ut puta ferrea : \\\hline
3.2.30 & assi commo paresçe por el pho \textbf{ en el segundo libro delas politicas } do disputa contra socrates . & Immo ut patet \textbf{ per Philosophum 2 Politicorum } cum disputat contra Socratem , \\\hline
3.2.31 & assi commo dize elpho \textbf{ en el segundo libro delas politicas . } es acostunbrar sea non obedesçer alas leyes & Nam assuescere inducere nouas leges \textbf{ ( ut innuit Philosophus 2 Pol’ ) } est assuescere non obedire legibus . \\\hline
3.2.31 & e alos reyes muestra lo el philosofo \textbf{ en el primero libro delaL rectorica } do dize que mas enpeesçe acostunbrar se los omes & non obedire Regibus et legibus , \textbf{ ostendit Philosophus 1 Rhetor’ } qui ait , magis nocere , \\\hline
3.2.32 & Bien diches lo que dize el philosofo \textbf{ en el terçero libro delas politicas . } que la morada del lo gar & accipienda est eius notitia , \textbf{ benedictum est quod dicitur 3 Polit’ } quod unitas loci , communicatio connubiorum , compugnationis gratia , commutatio rerum , \\\hline
3.2.33 & Ca assy commo dize el philosofo \textbf{ en el quarto libro delas ethicas . } los muy pobres muy poco aman alos prinçipes & vel inter commorantes in eadem ciuitate . \textbf{ Nam ut dicitur 4 Politic’ } isti videlicet pauperes minime amant Principes , \\\hline
3.2.34 & Ca assi commo dize el philosofo \textbf{ en el primero libro de la rectorica enlas leyes es salud dela çibdat . } Et maguer algunos cuyden que guardan las leyes & Credunt enim aliqui , \textbf{ quod obseruare leges , } et obedire regi , \\\hline
3.2.34 & si fueren penssadas las palabras del philosofo \textbf{ en el quato libro delas politicas . } El qual philosofo con para el Rey al regno & si considerentur verba Philosophi \textbf{ 4 Polit’ qui comparat Regem ad regnum , } sicut animam ad corpus . \\\hline
3.2.34 & Et dende nasçe aquello que dize el philosofo \textbf{ en el primero libro de la rectorica } que non enpeesçe tanto pecar & aut in ciuitate . \textbf{ Inde est ergo quod dicitur primo Rhet’ } quod non tantum nocet peccare \\\hline
3.2.35 & Et por ende dize el philosofo \textbf{ en el segundo libro de la rectorica } que nos nos enssannamos contra aquellos que nos solian honrrar & merito prouocatur ad iram . \textbf{ Ideo dicitur 2 Rhet’ } quod irascimur iis qui tenentur , \\\hline
3.2.35 & Et esto es lo que dize el philosofo \textbf{ en el segundo libro de la rectorica } en el capitulo dela sanna & forefacit in regem \textbf{ quotiescunque alicui existenti } in regno efficitur iniuria aliqua . Hoc est ergo quod dicitur 2 Rhet’ cap’ de ira , \\\hline
3.2.35 & en el segundo libro de la rectorica \textbf{ en el capitulo dela sanna } que nos nos enssanamos & forefacit in regem \textbf{ quotiescunque alicui existenti } in regno efficitur iniuria aliqua . Hoc est ergo quod dicitur 2 Rhet’ cap’ de ira , \\\hline
3.2.35 & segunt dize el philosofo \textbf{ en el quinto libro delas positicas } non enssennar los mocos auertudes & Pessimum est enim \textbf{ ( ut dicitur 7 Pol’ ) } non instruere pueros ad virtutem , \\\hline
3.2.36 & Et por ende dize el philosofo \textbf{ en el segundo libro de la rectorica } en el capitulo del amor & amat , et reueretur . \textbf{ Ideo dicitur 2 Rheto’ } cap’ de amore \\\hline
3.2.36 & en el segundo libro de la rectorica \textbf{ en el capitulo del amor } que el pueblo ama & Ideo dicitur 2 Rheto’ \textbf{ cap’ de amore } quod populus amat , \\\hline
3.2.36 & Et por ende dize el pho \textbf{ en el segundo libro de la rectorica } que nos amamos alos bien fechores & credit enim per tales salutem consequi . \textbf{ Ideo dicitur 2 Rhetor’ } quod quia diligimus beneficos in salutem , \\\hline
3.2.36 & Et por ende el philosofo \textbf{ en el segundo libro de la rectorica dize } que much amamos los derechureros . & si viderit ipsum non obseruare iustitiam : \textbf{ Ideo dicitur 2 Rhet’ } quod iustos maxime diligimus . \\\hline
3.2.36 & assi commo dize el philosofo \textbf{ en el segundo libro de la rectorica . } es por las penas dela iustiçia & Potissime autem timentur potentes \textbf{ ( ut patet in 2 Rhet’ ) } propter punitiones , \\\hline
3.2.36 & Et por ende dize el philosofo \textbf{ en el segundo libro de la rectorica } que los omes temen a aquellos de que son sabidores & inexquisitas crudelitates exerceant . \textbf{ Ideo dicitur 2 Rhetor’ } quod homines timent eos , \\\hline
3.2.36 & Ante assi conmo dize el philosofo \textbf{ en el septimo libro delas roliticas conuiene alos Reyes e alos prinçipeᷤ } por que sean mas temidos & Imo , ut vult Philos’ 7 Polit’ \textbf{ ut decet Reges } magis timeantur , \\\hline
3.2.36 & por ende dize el philosofo \textbf{ en el segundo libro dela } rectorica & quin puniantur . \textbf{ Ideo dicitur 2 Rhet’ } quod latitiui magis timentur , \\\hline
3.3.1 & assi commo prueua Uegeçio \textbf{ en el libro } do tracta del Fecho de la . & ut patet per Vegetium \textbf{ in De re militari , } plus confert ad obtinendam victoriam , \\\hline
3.3.1 & assi commo dize el philosofo \textbf{ en el sesto libro de las Ethicas } por que sabe bien consseiar & Nam prudens ex hoc aliquis dicitur , \textbf{ ut patet ex septimo Ethicorum : } quia scit bene consiliari , \\\hline
3.3.1 & lo que dize Uegeçio \textbf{ en el primer libro del Fecho de la caualleria } que non conuiene a ninguno saber mas cosas nin meiores & Propter quod bene dictum est \textbf{ quod ait Vegetius in primo libro de re militari , } quod neque quemquam magis decet \\\hline
3.3.2 & la razon desto muestra vegeçio \textbf{ en el primer libro del Fecho de la caualleria } en el capitulo segundo do dize que las nasçiones e las gentes & Ratio autem huius assignatur \textbf{ a Vegetio primo libro } De re militari capitulo secundo , \\\hline
3.3.2 & en el primer libro del Fecho de la caualleria \textbf{ en el capitulo segundo do dize que las nasçiones e las gentes } que son muy cercanas al sol & a Vegetio primo libro \textbf{ De re militari capitulo secundo , | ubi dicitur . } Nationes quae vicinae sunt soli , \\\hline
3.3.2 & e llanamente lo dize el philosofo \textbf{ en el vij libro de las . } que las gentes que son muy çercanas & Experimento enim videmus , \textbf{ et plane hoc vult Philosophus 7 Politicorum } quod gentes nimis propinquae soli abundant sagacitate et industria , \\\hline
3.3.3 & Ca assi conmo el dize \textbf{ en el segundo del Alma } los que han las carnes muelles & omnino requirunt modum contrarium . \textbf{ Nam ut scribitur in 2 de Anima , } molles carne aptos mente dicimus . \\\hline
3.3.4 & assi commo dize el philosofo \textbf{ en el terçero libro de las Ethicas } de non auer cuydado & et ad bonum bellatorem , \textbf{ ut innuit Philosophus 3 Ethic’ } non curare in bello bene mori . \\\hline
3.3.4 & Ca assi commo dize el philosofo \textbf{ en el primero libro de las Ethicas } la fin de la caualleria es victoria e vençer . & et ad feriendum alios . \textbf{ Nam ut dicitur Ethic’ primo , } Finis militaris , est victoria . \\\hline
3.3.4 & Ca assi commo dize el philosofo \textbf{ en el tercero libro de las Ethicas } entre aquellos son los varones muy fuertes entre los quales los fuertes son muy honrrados . & et erubescere turpem fugam . \textbf{ Nam , ut dicitur 3 Ethic’ } apud illos sunt viri fortissimi , \\\hline
3.3.5 & a que se deuen vsar los lidiadores \textbf{ en el capitulo } que se sigue se mostrara . & ad quae debeant exercitari bellantes ; \textbf{ in sequenti capitulo ostendetur . } Recitat Vegetius in libro \\\hline
3.3.8 & quales cosas son de penssar \textbf{ en el fazimiento de los castiellos . } Por que en el fazimiento de las carcauas & quae sunt attendenda \textbf{ in constructione castrorum . } In faciendis enim fossis , \\\hline
3.3.8 & en qual manera de guarnimiento es de catar \textbf{ en el fazer de los castiellos . } Ca si la hueste mucho ouiere de morar & quis munitionis modus attendendus sit \textbf{ in constructione castrorum . } Nam si exercitus diu \\\hline
3.3.12 & Et por ende estas tres cosas son de guardar \textbf{ en el ordenamiento de las azes . } Lo primero que el az sea bien ordenada & Haec igitur tria obseruanda sunt \textbf{ in constitutione acierum . } Primo , ut acies bene ordinetur \\\hline
3.3.14 & et bien ordenados e bien ordenados \textbf{ en el az commo deuen } si los acometieren sus enemigos & si hostes sint bene uniti \textbf{ et debite in acie ordinati , } si inuadantur , \\\hline
3.3.15 & que non taiando \textbf{ en el qual mostramos a los caualleros } e a los peones acometer la batalla . & percutiendum esse punctim non caesim : \textbf{ in quo docuimus milites , } et etiam pedites . \\\hline
3.3.16 & meior es de fazer la çerca \textbf{ en el tienpo del estiuo } por que entonçe mas se dessecan las aguas & Nam si per sitim sunt munitiones obtinendae , \textbf{ melius est facere obsessionem tempore aestiuo , } eo quod tunc magis desiccantur aquae , \\\hline
3.3.16 & Pues que assi es o las cercas son de fazer \textbf{ en el tienpo del estiuo o łi por muchos tienpos han de durar las cercas } deuen se començar en el tienpo del estuo & Vel igitur obsessiones fiendae sunt tempore aestiuo , \textbf{ vel si per multa tempora obsessiones durare debent , } saltem inchoandae sunt tempore aestiuo , \\\hline
3.3.17 & en la çibdato \textbf{ en el castiello çercado } e assi podran ganar aquellas fortalezas . & et per aditum factum ex muris cadentibus reliqui obsidentes ingrediantur castrum , \textbf{ vel ciuitatem obsessam : } et sic poterunt obtinere illam . \\\hline
3.3.21 & Ca segunt el philosofo \textbf{ en el libro de los 
                     Metaurores 
                   toda agua salada } que passa por los foradillos menudos de la çera toda se torna dulçe . & totum in dulce conuertitur . \textbf{ Deferendum est etiam ad munitionem obsidendem } in magna copia acerum , \\\hline

\end{tabular}
