\begin{tabular}{|p{1cm}|p{6.5cm}|p{6.5cm}|}

\hline
1.2.1 & Virtutes autem quaedam sunt quidam ornatus , \textbf{ et quaedam perfectiones animae . Oportet ergo prius ostendere , } quot sunt potentiae animae , & Ca las uirtudes son vnos hornamentos e conponimientos e hunas perfectiones \textbf{ que fazen acabada el alma . | Pues que assi es conuiene } mostrͣ primero quantos son los poderios del alma \\\hline
1.2.1 & et appetitus intellectiuus , et sensitiuus , \textbf{ oportet esse in talibus virtutes morales . } Quomodo distinguuntur virtutes : & que es la uoluntad Et el appetito sen setiuo \textbf{ que sigue al seso conuiene } por fuerça \\\hline
1.2.2 & virtutem esse aliquid secundum rationem : \textbf{ oportet ergo esse rationalem potentiam , } in qua potest esse virtus . & segunt razon \textbf{ e por ende conuiene | que sea poderio razonable } aquel en que esta la uirtud ¶ \\\hline
1.2.3 & de quibus omnibus quid sunt , \textbf{ et quomodo decet eas Reges habere , } et quas partes habent , & con estas dichas diez \textbf{ son doze las uirtudes morales delas quales todas que cosas son e en qual manera . | Conuiene a los Reyes } e alos prinçipes delos auer \\\hline
1.2.3 & nisi circa ea quae sunt in potestate nostra , \textbf{ in quibus decet nos ponere medietatem , } vel aequalitatem , siue rectitudinem : & que son en nuestro poder . \textbf{ En las quales cosas nos conuiene } de poner meatado ygualdat o derechura . \\\hline
1.2.6 & et agibilia sint singularia , \textbf{ oportet prudentiam esse circa particularia , } applicando uniuersales regulas & en las cosas singulares . \textbf{ Conuiene que la pradençia sea cerca las cosas singulares | e particulares } allegando las reglas generales alos negoçios singulares \\\hline
1.2.7 & quod polleat prudentia , et intellectu . \textbf{ Quot , et quae oporteat habere Regem , } si & ø \\\hline
1.2.8 & ut possit eam melius in debitum finem dirigere . \textbf{ Ultimo oportet ipsum esse cautum . } Nam sicut in speculabilibus falsa aliquando admiscentur veris , & e traher los ala fin que deue¶ \textbf{ Lo postrimero conuiene al Rey | que sea muy aꝑçebido . } Ca assi commo en las sçiençias especulatuias algunas cosas falsas \\\hline
1.2.8 & sed apparent bona . \textbf{ Oportet igitur Regem esse cautum , } respuendo apparenter bona , & maguera que lo non sean . \textbf{ commo quier que paresçan buenas Et pues que assi es conuiene al Rey sea aꝑcebido } para desechar e despreçiar aquellas cosas \\\hline
1.2.9 & quae superius diximus , \textbf{ oportet ipsos esse bonos , } et non habere voluntatem deprauatam : & que dixiemos de suso \textbf{ conuieneles | que sean buenos } e que non ayan uoluntad mala nin desordenada \\\hline
1.2.13 & et non recte , \textbf{ oportet dare virtutem aliquam } circa timores , et audacias . & e en las osadias \textbf{ por la qual sea el omne reglado en ellos . | por que contesce que algunos remen algunas cosas } que han de temer e alas uegadas temen alguas cosas \\\hline
1.2.20 & distribuere bona regni , \textbf{ maxime decet ipsum esse magnificum . Nam quia est caput regni , } et gerit in hoc Dei vestigium , & e a el pertenesca de partir los bienes del regno mucho le conuiene a el de ser magnifico . \textbf{ Ca porque es cabeça del regno } e ha en esto semeiança de dios \\\hline
1.2.21 & Sed , ut ibidem dicitur , \textbf{ tales oportet esse nobiles et gloriosos . } Quare quanto est nobilior aliis , & por que non puede cada vno fazer grandes espenssas \textbf{ Mas assi commo alli dize el philosofo tales son los nobles e los głiosos . } por la quel cosa en quanto el Rey es mas noble \\\hline
1.2.23 & quae sunt multa . \textbf{ Quarto decet esse apertos , } ut esse veridicos ; & que son muchos alos otros ¶ \textbf{ Lo quarto conuiene alos Reyes } de seer manifiestos e claros e seer uerdaderos \\\hline
1.2.27 & de virtutibus respicientibus exteriora bona , \textbf{ et ostendimus quomodo Reges et Principes decet ornari virtutibus illis , } restat dicere de mansuetudine , & que catan alos bienes de fuera . \textbf{ Et mostramos en qual mana conuiene alos Reyes e alos | prinçipesser honrrados destas uirtudes . } fincanos agora de dezir dela \\\hline
1.2.27 & et deficere , \textbf{ oportet ibi dare virtutem aliquam , } per quam dirigamur ad bene agendum , & e tal sesçer conuiene de dar y . \textbf{ alguna uirtud por la qual seamos enderesçados } abien obrar \\\hline
1.2.27 & virtuosus non esset . \textbf{ Tanto ergo magis decet Reges et Principes moueri } ad punitionem faciendam , & non seria uirtuoso . \textbf{ Et pues que assi es tanto | mas conuiene alos Reyes } e alos prinçipes de se mouer a dar penas . \\\hline
1.2.30 & et econuerso . \textbf{ Si igitur decet homines reprimere } superfluitates ludorum , & que es sobeia a otro . \textbf{ Et por ende si conuiene aton dos los omes } de repremir las sobeianias de los iuegos mucho \\\hline
1.2.31 & in hanc sententiam conuenerunt , \textbf{ quod oportet virtutes connexas esse . } Dixerunt enim & commo los p̃h̃osacuerdan en esta sentençia \textbf{ que conuiene que todas las uirtudes sean ayinntadas la vna con la vna con la otra . } Ca dixieron que aquel que ha vna uirtud \\\hline
1.3.5 & et debeant prouidere bona futura possibilia ipsi regno : \textbf{ decet eos esse bene sperantes per magnanimitatem , } quia habent omnia & que han de venir e los bienes que pueden acahesçer a su regno . \textbf{ Por ende conuiene a ellos de serbine esparautes | por la magnanimidat } que han en ssi \\\hline
1.4.1 & quin committant aliqua turpia , \textbf{ de quibus decet eos uerecundari : } Reges tamen et Principes , & que non acometan algunas cosas torpes \textbf{ delas quales les conuiene | que tomne uerguença . } Enpero esto non es de loar en los uieios nin en los Reyes . \\\hline
1.4.1 & Reges tamen et Principes , \textbf{ quos decet esse quasi semideos , } non solum quod turpia committant , & por que los Reyes e los prinçipes alos quales conuiene de ser \textbf{ assi commo medios dioses } non solamente non les conuiene de fazer cosas torpes \\\hline
1.4.1 & quod debeant uerecundari . \textbf{ Non ergo decet eos uerecundari , } nisi ex suppositione : & que de una auer uerguença saluo \textbf{ por alguna razon } assi commo si contesçiesse \\\hline
1.4.1 & sunt digniora quam alia . \textbf{ Rursus decet eos esse magnanimos : } quia ( ut dicebatur & e alos prinçipes de ser magranimos \textbf{ e de grand coraçon } Ca assi commo es dicho dessuso \\\hline
1.4.7 & aliquos malos mores : \textbf{ quia non oportet omnes esse tales , } sed sufficit reperiri illud in pluribus : & li dellos contamos algunas malas costunbres \textbf{ ca non conuiene que todos seantales . } Mas abasta que aquellas costunbres sean falladas en muchos por que non \\\hline
2.1.2 & ad per se sufficientiam vitae , \textbf{ oportet communitatem domus necessariam esse . } Reges ergo et Principes , & e por si vale a conplimiento dela uida . \textbf{ Conuiene que la comunidat dela casa sea mas neçessaria } Et pues que assi es los Reyes e los prinçipes \\\hline
2.1.3 & et typo ostendere , \textbf{ quod decet homines habere habitationes decentes } secundum suam possibilem facultatem ; & por que ael parte nesçe generalmente demostrar \textbf{ por figera e por exienplo que conuiene alos omes de auer conueibles moradas } segunt el su poder e la su riqueza . \\\hline
2.1.5 & et serui ad conseruationem . Quare si generatio et conseruatio est quid naturale , \textbf{ oportet domum quid naturale esse . } Amplius , quia generatio et conseruatio & conseruaçique por la qual cosa si la generaçion e la conseruaçion es cosa natural \textbf{ conuiene | que la casa sea cosa natural ¶ } Otrossi por que la generaçion e la conseruaçion non pueden ser apartadas la vna dela otra \\\hline
2.1.6 & si domus debet esse perfecta , \textbf{ oportet ibi dare communitatem tertiam , } scilicet patris et filii . & Emposi la casa fuere acabada conuiene de dar \textbf{ y la terçera comunidat } que es de padre e de fijo . \\\hline
2.1.6 & potest sibi simile producere , \textbf{ sed oportet prius ipsam esse perfectam . } statim enim , & luego que es fecha fazer otra semeiante \textbf{ assi mas conuiene que ella primeramente sea acabada } enssi \\\hline
2.1.6 & nec statim potest sibi simile producere , \textbf{ sed oportet prius ipsum esse perfectum : } producere ergo sibi similem , & luego otro su semeinante \textbf{ mas conuiene que primeramente el sea acabado . } Et pues que assi es engendrar su semeiante non pertenesçe a cosa natural tomada en qual quier manera mas pertenesçe a cosa natural en quanto ella es acabada . \\\hline
2.1.6 & Patet ergo quod ad hoc quod domus habeat esse perfectum , \textbf{ oportet ibi esse tres communitates : } unam viri et uxoris , aliam domini et serui , & Et por ende paresçe que para que la casa sea acabada \textbf{ que conuiene que sean enlla tres comuundades . } ¶ La vna del uaron e dela muger ¶ \\\hline
2.1.6 & Nam cum in domo perfecta sint tria regimina , \textbf{ oportet hunc librum tres habere partes ; } in quarum prima tractetur primo de regimine coniugali : & Ca commo en la casa acabada sean tres gouernamientos . \textbf{ Ca conuiene que este libro sea partido en tres partes . } ¶ En la primera delas quales tractaremos del gouernamiento mater moianl . \\\hline
2.1.7 & et maxime Reges \textbf{ et Principes deceat sumere . } Deinde ostendemus , & quales si quier çibdadanos \textbf{ e mayormente los Reyes e los prinçipes . } Despues mostraremosen qual manera los uarones deuen gouernar sus mugers \\\hline
2.1.7 & et uniuersaliter omnem venereorum usum illicitum , \textbf{ tanto magis decet fugere Reges et Principes , } quanto decet eos meliores et virtuosiores esse . & La qual fornicaçion e general mente todo vso de luxuria non conueinble tanto \textbf{ mas conuiene alos Reyes } e alos prinçipes delo esquiuar quanto mas conuiene aellos de ser meiores \\\hline
2.1.8 & Probant autem Philosophi , \textbf{ quod decet coniugia indiuisibilia esse . } Ad quod ostendendum adducere possumus duas vias , & os philosofos prue una que los con̊uiene \textbf{ que los casamientos sean sin departimiento ninguno } e que non le puedan partir . \\\hline
2.1.10 & ut si quis subiicitur Proposito et Regi , \textbf{ oportet Propositum illum ad Regem ordinari , } et esse sub ipso repugnat & assi commo si algun çibdadano es subiecto al preuoste e al Rey . \textbf{ Conuiene que el | prinoste sea ordenado al Rey e sea so el . } Et por ende contradize ala orden natural \\\hline
2.1.10 & simul viris pluribus detestabilius esse debet . \textbf{ Decet ergo coniuges omnium ciuium uno viro esse contentas : } multo magis tamen hoc decet & que vna muger case en vno con mugons varones . \textbf{ ¶ Et pues que assi es conuiene | quelas mugers de todos los çibdadanos sean pagadas de vn uaron . } Enpero mucho mas conuiene esto alas mugers de los Reyes \\\hline
2.1.13 & quam intemperantia coniugum aliorum . \textbf{ Decet ergo coniuges temperatas esse . } Decet eas etiam amare operositatem : & dellos puede fazer mayor danno e enpeçemiento que la destenprança delas mugers de los otros . \textbf{ Et pues que al sy es conuiene | que las muger ssean tenpradas . } Et avn les conuiene aellas de amar fazer buenas obras . \\\hline
2.1.14 & Patet ergo , \textbf{ quomodo decet omnes ciues alio regimine praeesse uxoribus , } et alio filiis : & Et pues que assi es paresçe \textbf{ en qual manera conuiene a todos los çibdadanos | que enssennore en } por otro gouernamiento alos mugers \\\hline
2.1.15 & et quicquid natura praeparatur , \textbf{ oportet ordinatissimum esse : } quia ille naturam dirigit , & por la natura \textbf{ conuiene que sea muy ordenado . } Ca aquel gnia la natura de que viene todo ordenamiento \\\hline
2.1.19 & Nam quicunque vult aliquid bene regere , \textbf{ oportet ipsum speciales habere cautelas ad ea , } circa quae videt ipsum magis deficere . & conuiene \textbf{ que el aya algunas cautelas espeçiales | para aquellas cosas } en las quales vee \\\hline
2.1.19 & in loquela dirigere , \textbf{ oporteret eos instruere , } ut specialem pugnam & que qualiesse enderesçar los tartamudos en la fabla conuenir le ha \textbf{ que los ensseñasse que tomassen espeçial batalla } e espeçial esfuerço cerca aquellas palauras \\\hline
2.1.19 & vel dissensio oriri . \textbf{ Secundo decet eas esse pudicas } et honestas . & e mayor discordia \textbf{ que lo segundo couiene a el de los otros . } las de ser linpias e honestas \\\hline
2.1.21 & circa ornatum corporis ; \textbf{ quare decet viros cognoscere } quis mulierum ornatus sit licitus , & mayormente pecan en el conponimiento de los cuerpos . \textbf{ Por la qual cosa conuiene alos uarones } de saber qual conponimiento es conuenible alas mugieres \\\hline
2.1.23 & Quia ergo sic est , \textbf{ oportet foeminas deficere a ratione , } et habere consilium inualidum . & que las mugers \textbf{ que fallezcan de vso de razon e que ayan el conseio flaço . } Ca quando el cuerpo es meior conplissionado tanto \\\hline
2.2.2 & eo quod principantis sit alios regere et gubernare : \textbf{ tanto ergo magis decet Reges et Principes solicitari } circa proprios filios , & por que alos prinçipes parte nesçe de gouernar e de garalo sots . \textbf{ pues que assi es tanto mas conuiene alos Reyes } e alos prinçipes de ser acuçiosos de sus fijos \\\hline
2.2.6 & Nam cum aliquis est pronus ad aliquid , \textbf{ oportet ipsum multum assuescere in contrarium , } ne inclinetur ad illud : & Ca quando alguno es inclinado a alguacosa . \textbf{ Conuiene que el vse mucho en el contrario } por que non sea inclinado a aquella cosa . \\\hline
2.2.6 & a lasciuiis retrahantur . \textbf{ Decet ergo omnes ciues solicitari erga filios , } ut ab ipsa infantia instruentur ad bonos mores . & e por bueons castigos sean tirados delas loçanias . \textbf{ Et pues que assi es | conuieneque todos los çibdadanos ayan grand cuydado de sus fijos } assi que luego en su moçedat \\\hline
2.2.9 & quam doctor : \textbf{ decet igitur ipsum esse inuentiuum . } Secundo decet ipsum esse intelligentem et perspicacem . & este tal mas es rezador que doctor ¶ \textbf{ Et pues que assi es conuiene al maestro | que non tan solamente sea fallador delas cosas } mas que sea entendido e sotil . Ca assi commo ninguon non puede abastar \\\hline
2.2.9 & et intelligens aliorum dicta . \textbf{ Tertio oportet ipsum esse iudicatiuum : } nam perfectio scientiae potissime & e que entienda los dichͣs de los otros . \textbf{ ¶ Lo terçero conuiene que sea iudgador } e que aya razon para iudgar . \\\hline
2.2.9 & Quantum vero ad prudentiam agibilium , \textbf{ decet ipsum esse memorem , } prouidum , cautum , et circumspectum . & que son de fazer \textbf{ conuiene le que el doctor sea menbrado e prouado e sabio e acatado . } Mas quanto ala uida deue ser honesto e bueno . \\\hline
2.2.14 & est virtus organica siue corporalis . \textbf{ Quare oportet talem appetitum sumere modum , } et mensuram ex ipso corpore . & mas el appetito de los sesos es uirtud organica o corporal . \textbf{ Por la qual cosa conuiene } que tal desseo tome manera e mesura del cuerpo . \\\hline
2.2.19 & qualis cura gerenda sit circa filias . \textbf{ Nam sicut decet coniuges esse continentes , } pudicas , abstinentes , et sobrias : & çerca delas fijas \textbf{ ca assi commo conuiene alas madres } de ser continentes e castas e guardadas e mesuradas en essa misma manera conuiene alas fijas de ser tales \\\hline
2.2.20 & infra declarandum esse , \textbf{ circa quae opera deceat foeminas esse intentas . } Ostenso , & casamien toca y dixiemos que adelante serie de declarar cerca quales obras conuenia \textbf{ que las mugers fuesen acuçiosas . } ostrado que non conuiene alas moças de andar uagarosas a quande e allende \\\hline
2.2.21 & cautos proferre sermones , \textbf{ decet eas non esse loquaces : } sed oportet ipsas esse debite taciturnitas , & tomadesto \textbf{ que las mugrͣ̃s non sean prestas avaraias e apeleas } ca commo las muger se mayormente las mocas \\\hline
2.2.21 & decet eas non esse loquaces : \textbf{ sed oportet ipsas esse debite taciturnitas , } ut possint omnem sermonem prolatum diligenter excutere . & que las mugrͣ̃s non sean prestas avaraias e apeleas \textbf{ ca commo las muger se mayormente las mocas } fallezcan de vso de razon \\\hline
2.3.2 & et mensuram ex illis : \textbf{ oportet organa domus ordinata esse , } et organa inferiora , & Et resçiben dellas manera de seruiçio e mesura . \textbf{ Conuiene avn que los estrumentos dela casa sean ordenados } e que los instrumentos mas baxos \\\hline
2.3.3 & nam secundum Philosophum 4 Ethicorum capitulo de Magnificentia , \textbf{ maxime gloriosos et nobiles decet esse magnificos : } Reges ergo et Principes , & en el quarto libro delas ethicas \textbf{ enł capitulo dela magnifiçençia | que much mas conuiene alos Reyes } e alos prinçipes \\\hline
2.3.3 & In domibus ergo Regum et Principum \textbf{ oportet multos abundare ministros , } ut ergo non solum personas Regis et Principis , & e de los prinçipes conuiene \textbf{ que ayan muchos ofiçiales | e much ssiruient s̃ Et pues que assi es } por que non solamente la persona del Rey o del prinçipe mas avn \\\hline
2.3.3 & in aedificiis constructis , \textbf{ oportet ipsa esse magnifica . } Viso , qualia debent esse aedificia , & que ellos fazen a \textbf{ conuiene | que ellos sean muy grandes e muy costosas } ¶ Visto quales deuen ser las moradas \\\hline
2.3.5 & ut in prima parte huius secundi libri diffusius probabatur : \textbf{ oportet aliquomodo naturalia esse } quae sunt necessaria in vita politica ; & mas conplidamente en la primera parte deste segundo libro . \textbf{ conuiene en algunan manera } que las cosas naturales sean neçessarias enla uida politica . \\\hline
2.3.9 & Ut ergo sciamus quomodo huiusmodi commutationes \textbf{ oportuit introduci , } sciendum quod si non esset & Et pues que assi es por que sepamos en qual manera conuiene \textbf{ que estas tales muda connes fuessen puestas en la } tiecra deuedes saber \\\hline
2.3.9 & quibus non abundant frigidae et econuerso . \textbf{ Propter quod non solum oportet communicare } et conuersari & Et por el cotrario en alguas abonda friura \textbf{ que non abonda ca lentura } Por la qual cosa non solamente conuiene alos omes morar e conuerssar los vnos con los otros \\\hline
2.3.9 & et rerum ad numismata , \textbf{ oportuit inuenire commutationem numismatum ad numismata . } Patet ergo quot sunt commutationes , & e delas cosas alos \textbf{ diueros otra mudaçiones | que es de monedas alas monedas . } Et pues que assi es paresçe \\\hline
2.3.15 & quos virtus et amor boni inclinat ad seruiendum , \textbf{ decet principantes se habere quasi ad filios , } et decet eos regere non regimine seruili , & e el amor de bien los inclina asuir . \textbf{ Conuiene que los prinçipes se ayan çerca ellos | assi commo cerca de fijos . } Et conuiene les alos prinçipes delos gouernar non \\\hline
2.3.16 & si debet esse ordinata , \textbf{ oportet reduci in unum aliquem , } a quo ordinetur . & En essa misma manera cada vna muchedunbre si bien ordenada es \textbf{ conuiene que sea aduchͣa vn ordenador } de quien ella sea ordenada . \\\hline
2.3.19 & magnanimos decet operari pauca et magna , \textbf{ ut decet ipsos solicitari } circa ea quae directe spectant & conuiene les de obrar pocas cosas \textbf{ e grandes ca les conuiene } de ser acuçiosos \\\hline
2.3.20 & et ne intemperati appareant , \textbf{ decet in mensis vitare sermonum multitudinem , } decet etiam hoc ipsos ministrantes , & e por que non parezcan destenprados \textbf{ assi commo dicho es . } Avn esto mismo conuiene alos seruientes por que la orden e la manera del seruir \\\hline
3.1.1 & gratia alicuius boni , \textbf{ oportet ciuitatem ipsam constitutam esse propter aliquod bonum . } Probat autem Philosophus primo Polit’ duplici via , & commo toda comunidat sea por graçia de algun bien . \textbf{ Conuiene que la çibdat sea establesçida por algun bien | Ca pruena el pho } enl primero libro delas politicas \\\hline
3.1.4 & ut natura non deficiat in necessariis , \textbf{ oportet quid naturale esse } quicquid secundum se deseruit & conuiene \textbf{ que sea cosa natural todo aquello } que sirue a conplimiento de uida \\\hline
3.1.4 & quam communitates illae , \textbf{ oportet eam esse secundum naturam . } Secunda uia ad inuestigandum hoc idem , & que estas dos comuidades \textbf{ por ende conuiene | que la çibdat lea comuidat natraal ¶ } La segunda razon para prouar \\\hline
3.1.6 & in inueniendo artem aliquam , \textbf{ sed oportet ad hoc iuuari } per auxilium praecedentium & niguno non abasta assi mismo en fallar algunan arte \textbf{ mas conuiene que sea ayuda de } por ayuda de los que passaronante \\\hline
3.1.8 & Quia ergo diuersis indigemus ad vitam , \textbf{ oportet in ciuitate diuersitatem esse . } Tertia via declarans & por que nos auemos me estermuchͣs cosas departidas para abastamiento dela uida \textbf{ conuiene que enla çibdat sea algun departimiento . } La tercera razon que declara e manifiesta las razones \\\hline
3.1.8 & nisi sit ibi diuersitas officiorum . \textbf{ Decet ergo hoc Reges , et Principes cognoscere , } quod nunquam quis bene nouit regere ciuitatem , & e de los ofiçiales . \textbf{ Et pues que assi es conuiene alos Reyes | e alos prinçipes de sabesto } por que munca ninguno sopo bien gouernar çibdat \\\hline
3.1.11 & si haberent haereditatem communem , \textbf{ et si oporteret eos valde ad inuicem conuersari . } Tertia via sumitur , & los quales non quarrien ser subiectos el vno al otrosi ouiessen la heredat en comun \textbf{ e ouiessen de beuir plongadamente en vno . } ¶ La terçera razon se toma \\\hline
3.1.14 & et onerosius et quasi omnino importabile esset sustentare sic quinque milia : \textbf{ oporteret enim ciuitatem illam habere possessiones quasi ad votum , } ut posset ex communibus sumptibus & ca conuerne \textbf{ que aquella çibdat ouiesse tantas possessiones | quantas quisiesse a ssu uoluntad } por que pudiesse de las rentas comunes abondar atanta muchedunbre \\\hline
3.1.14 & Propter quod Philosophus 2 Politicorum reprehendens Socratem de huiusmodi ordine ciuitatis , \textbf{ ait , quod oportet ciuitatem illam sic institutam esse in regione Babilonica , } ubi forte propter magnitudinem desertorum & que aquella çibdat assi establesçida deuie ser \textbf{ tamannera | commo babilonia en la qual } por auentura \\\hline
3.1.17 & possumus triplici via venari , \textbf{ quod non oportet possessiones aequatas esse , } ut Phaleas statuebat . & por tres razones podemos prouar \textbf{ que non conuiene | que las possessiones sean egualadas } en aquella manera \\\hline
3.2.5 & per haereditatem transferatur ad posteros , \textbf{ oportet eam transferre in filios , } quia secundum lineam consanguinitatis filii parentibus maxime sunt coniuncti : & por hedamiento conuiene alos pueblos \textbf{ que tomne alos fijos } ca segunt el linage del patente \\\hline
3.2.5 & quia ( ut ait Philosophus in Politiis ) \textbf{ decet iuniores senioribus obedire . } Immo quia patres plus communiter primogenitos diligunt ; & conuiene \textbf{ que los mas mançebos obedescan alos mas uieios e avn } por que los padres comunalmente \\\hline
3.2.6 & cum calefit et rarefit , \textbf{ oportet raritatem et calorem perfectius reperiri } in igne iam generato & por que la materia estonçe es puesta e tornada en fuego \textbf{ quando es muy escalentada | e muy enraleçida conuiene que la raledat e la calentura mas acabadamente sea fallada en el fuego } despues que fuere engendrado e ençendido . \\\hline
3.2.15 & et epiikis idest super iustus : \textbf{ decet enim talem esse quasi semideum , } ut sicut alios dignitate et potentia excellit , & ca conuiene \textbf{ que el tal que sea | assi commo dios } assi que commo lieua auna taia de los otros en dignidat e en poderio \\\hline
3.2.16 & utrum ciues inter se pacem debeant habere , \textbf{ et utrum regnum oporteat esse in bono statu : } sed haec accipit tanquam certa et nota , & si los çibdadanos deuen auer entre ssi paz . \textbf{ Et si conuiene | que el regno se en buen estado . } mas esto sopone \\\hline
3.2.18 & Sed ad hoc quod aliquis sit bene creditiuus , \textbf{ non oportet ipsum esse existenter talem , } sed sufficit quod videatur & mas para que alguno sea bien de creer \textbf{ non conuiene | que el sea tal fechmas cunple } que parezca tal cael o en iudga las cosas que paresçen de fuera por las cosas que vee \\\hline
3.2.19 & quia absque iustitia nequeunt regna subsistere . \textbf{ Decet autem scire Regem } quot sunt genera dominorum , & nin mucho durar sin iustiçia . \textbf{ Et por ende conuiene al Rey } de saber quantas son las . \\\hline
3.2.22 & ab alia vero recedit per odium , \textbf{ oportet ipsum iudicare inique : } quia tunc iudicium non procedit & por abortençia o por mal querençia . \textbf{ conuiene que el uiez judgue mal e desigual mente . } Ca entonçe el uuzio non salle de zelo de iustiçia \\\hline
3.2.26 & et hoc sequi volumus , \textbf{ oportet hoc agere . } Tales ergo debent esse leges , & e esto queremos alcançar \textbf{ conuiene que fagamos estas cosas . } Et pues que assi estales deuen ser las leyes \\\hline
3.2.26 & et bonum priuatum ordinetur ad ipsum , \textbf{ oportet tales leges fieri } non quales requirit bonum priuatum , & porque el bien propra o es ordenado al bien comun . \textbf{ Conuiene que las leyes tales sean non } quales demanda el bien propre \\\hline
3.2.26 & tales debet eis leges imponere . \textbf{ Viso quales leges Reges et Principes deceat ponere , } quia condendae sunt leges , & tales leyes les deue poner . \textbf{ Voisto quales leyes los Reyes } e los prinçipes deuen poner Ca deuen poner buenas e aprouechables \\\hline
3.2.27 & ad hoc quod lex habeat vim obligandi , \textbf{ oportet eam promulgatam esse . } Sed cum alia sit lex naturalis , & Poque la ley aya uirtud e fuerça de obligar \textbf{ conuiene que sea publicada e pregonada . } Mas commo otra sea la ley natural e otra la positiua en vna manera se deue publicar la vna \\\hline
3.2.29 & quae sit applicabilis humanis actibus . \textbf{ Oportet igitur aliquando legem plicare ad partem unam , } et agere mitius cum delinquente , & e allegar alas obras delos omes . \textbf{ Et por ende conuiene quela ley que se ençorue } e se allegue algunas vezes ala vna parte e que obre mas manssamente con el que peca \\\hline
3.2.29 & quam lex dictat : \textbf{ aliquando etiam oportet eam plicare ad partem oppositam , } et rigidius punire peccantem , & quela ley demanda o que la ley nidga . \textbf{ Et algunas vezes conuiene que la regla se encorue | ala parte contraria } e que mas reziamente de pena \\\hline
3.2.30 & attingere punctalem formam viuendi , \textbf{ ideo oportet aliqua peccata dissimulare } et non punire lege humana , & comunalmente non puede alcançar forma de beuir en punto . \textbf{ Por ende conuiene que | dessemeie alguons pecados } e que les de passada \\\hline
3.2.32 & propter quod regem ipsum tanquam omnibus excellentiorem \textbf{ decet esse optimum , } et quasi semideum . & assi commo aquel que sobrepula todos los otros en dignidat \textbf{ e en pero de rio sea muy bueno } e sea assi commo medio dios . \\\hline
3.2.32 & in ciuitate et regno , \textbf{ oportet esse talem , } quod viuat bene et virtuose . & que es en el regno e enla çibdat . \textbf{ conuiene que sea atal que biuna bien e uirtuosamente . } Et por ende assi conmo dize el philosofo en el terçero libro delas politicas \\\hline
3.2.33 & ex personis mediis . \textbf{ Decet ergo Reges et Principes adhibere cautelas , } ut in regno suo abundent multae personae mediae ; & establesçidas de perssonas medianeras . \textbf{ Et pues que assi es conuiene | que los reyes e los prinçipes ayan cautelas e sabidurias . } por que en el su regno sean muchͣs perssonas medianeras \\\hline
3.2.36 & quiescant male agere : \textbf{ oportuit ergo aliquos inducere ad bonum , } et retrahere a malo timore poenae . & Por la qual cosa conuiene \textbf{ que alguon s | enduxiessemosa bien } e arredrassemos del mal \\\hline
3.3.1 & per quam quis scit regere domum et familiam , \textbf{ oportet esse aliam a prudentia , } qua quis nouit seipsum regere . & por la qual cada vno sabe gouernar la casa e la conpaña . \textbf{ Conuiene que sea otra e departida de la sabiduria } por la qual cada vno sabe gouernar a ssi mismo . \\\hline
3.3.1 & et gubernare ciues . \textbf{ Omnes autem tres prudentias decet habere Regem , } videlicet particularem , oeconomicam et regnatiuam . & e en quanto ha de poner leyes e gouernar los çibdadanos . \textbf{ Et todas estas tres sabidurias | conuiene que aya el Rey . } Conuiene a saber . \\\hline
3.3.1 & Hanc autem prudentiam videlicet militarem , \textbf{ maxime decet habere Regem . } Nam licet executio bellorum , et remouere impedimenta ipsius communis boni , & Et esta sabiduria de caualleria \textbf{ mas pertenesçe al rey que a otro ninguno . | Ca commo quier que pertenezca a los caualleros } la essecuçion de las batallas \\\hline
3.3.4 & Rex aut Princeps eligere . \textbf{ Primo enim oportet pugnatiuos homines posse } sustinere magnitudinem ponderis . & e quales deuen escoger el Rey o el principe para la batalla \textbf{ Lo primero conuiene } que los omnes lidiadores puedan sofrir grandes pesos . \\\hline
3.3.8 & ut ultra quam debeat , \textbf{ oporteat exercitum constringi et constipari . } Quarto si oporteat in loco illo exercitum moram contrahere , & ø \\\hline
3.3.9 & erga necessitates corporis . \textbf{ Nam existentes in exercitu oportet multa incommoda tolerare : } quare si sint ibi aliqui molles , & penssada la sufrençia en las neçessidades del cuerpo . \textbf{ ca los que estan en las huestes | conuiene que sufran muchos males . } por la qual cosa si fueren y algunos muelles e mugerilles \\\hline
3.3.14 & difficilius se defendere poterunt : \textbf{ quia oportet eos sparsim incedere . Quare sicut locus ineptus defensioni , } si in eo hostes inueniantur , & e con mayor trabaio . \textbf{ Ca conuieneles que anden esparzidos . | Por la qual cosa } assi commo el logar malo \\\hline
3.3.16 & per quam pergit aqua ad obsessos , \textbf{ oportebit ipsos pati aquarum penuriam . } Rursus , aliquando munitiones sunt altae , & por do viene el agua a los cercados han de auer los cercados \textbf{ por fuerça mengua de agua . } Otrossi algunas uegadas las fortalezas son altas \\\hline
3.3.18 & sicut praedicta tria genera machinarum : \textbf{ tamen non oportet tantum tempus apponere } ad proportionandum huiusmodi machinam , & commo los tres engeñios sobredichos . \textbf{ Enpero non es menester tanto tienpo } para armar este engeñio commo en los otros tres sobredichos . \\\hline

\end{tabular}
