\begin{tabular}{|p{1cm}|p{6.5cm}|p{6.5cm}|}

\hline
1.2.19 & si \textbf{ ouiereconplidamente las riquezas fazer espenssas conuenibles en toda la comunidat } por que los bienes comunes lon en alguna manera diuinales & ( si adsit facultas ) \textbf{ facere decentes sumptus | circa totam communitatem . } Nam ipsa bona communia \\\hline
1.2.19 & resplandesçe mas fermosa . \textbf{ mente en toda la comunidat } ¶ & valde debiliter repraesentatur \textbf{ in una persona singulari : sed in tota communitate pulchrius elucescit diuinum bonum . } Tertio magnificus se decenter habere debet \\\hline
1.2.20 & Mas por el que es persona publica e comuna \textbf{ la qual toda la comunidat dela gente } e todo el regno es ordenado & Quia vero est persona publica , \textbf{ ad quam ordinatur tota communitas , } et totum regnum , \\\hline
1.3.5 & en bien alto e grande e guaue de fazer De mas desto \textbf{ quanto mayor es la comunidat } tantomas cosas le pueden auenir e contesçer . & sed etiam decet eos tendere in bonum arduum . \textbf{ Amplius quanto maior est communitas , } tanto plura possunt ei contingere , \\\hline
2.1.1 & determinaremos del gouernamiento de la casa \textbf{ mas commo la conpanna o la casa sea vna comunidat } e sea comunidat natraal & In hoc ergo secundo libro determinabitur de regimine domus . \textbf{ Sed cum familia domus | sit communitas quaedam , } et sit communitas naturalis : \\\hline
2.1.1 & mas commo la conpanna o la casa sea vna comunidat \textbf{ e sea comunidat natraal } si queremos fablar dela casa & sit communitas quaedam , \textbf{ et sit communitas naturalis : } si de domo determinare volumus , \\\hline
2.1.1 & Et pues que assi es neçessaria fue \textbf{ e es la comunidat dela casa } e las otras comuindades . & ut sit animal sociale ; \textbf{ necessaria ergo fuit communitas domus , } et communitates aliae , \\\hline
2.1.1 & e las otras comuindades . \textbf{ assi commo son comunidades de varrio et de çibdat e de regno } para esto & et communitates aliae , \textbf{ cuiusmodi sunt communitas ciuitatis , et regni , } ad hoc quod homines perfecte sibi in vita sufficiant . \\\hline
2.1.1 & siguese que beuir en conpannia \textbf{ e en comunidat es en alguna manera natural alos omes } ¶ & Sed si haec sunt necessaria ad conseruandam hominis naturalem vitam , \textbf{ viuere in communitate et in societate est quodammodo homini naturale . } Tertia via ad inuestigandum hoc idem , \\\hline
2.1.2 & que nos non sabiamos los terminos destas sçiençias \textbf{ por que aquellas conpania o aquella comunidat } por la qual abastamos a nos en la uianda e en el uestido & dicens nos scientiarum limites ignorare : \textbf{ quia societas , siue communitas illa , } per quam nobis sufficimus in victu et vestitu , \\\hline
2.1.2 & e en las otras cosas neçessarias ala uida non paresçe \textbf{ que sea comunidat çiuil de casa mas paresce } que sea comunidat çiuil e de çibdat . & et in aliis necessariis ad vitam , \textbf{ non videtur esse communitas domestica , } sed ciuilis : \\\hline
2.1.2 & por la qual cosa sin \textbf{ determinarde la comunidat de los çibdadanos } non parte nesçe a este libro & ut in tertio libro plenius ostendetur . \textbf{ Quare si determinare de communitate ciuium } non spectat ad hunc librum , \\\hline
2.1.2 & en el qual diremos del gouernamiento dela çibdat . paresçe que auemos trispassado los terminos desta arte determinando en el capitulo passado algunas cosas \textbf{ que pertenesçen ala comunidat dela çibdat . } Mas si con estudio e con acuçia penssaremos & videmur transgressi fuisse limites huius artis , \textbf{ determinando in praecedenti capitulo aliqua pertinentia ad communitatem ciuitatis . } Sed si diligenter aspicimus , \\\hline
2.1.2 & Mas si con estudio e con acuçia penssaremos \textbf{ en qual manera la comunidat dela casa se ha a todas las } otrascomuidades & Sed si diligenter aspicimus , \textbf{ quomodo communitas domestica se habet ad communitates alias : } cum quaelibet communitas \\\hline
2.1.2 & commo cada vna delas otras comundades \textbf{ ençierre en ssi la comunidat dela casa } por que non puede ser çibdat nin varrio & cum quaelibet communitas \textbf{ includat communitatem domesticam , } nec possit esse ciuitas , \\\hline
2.1.2 & humanal siguese \textbf{ que la comunidat dela casa es neçessaria a esta uida . } Et pues que assi es en el capitulo sobredicho auemos determinado dela conpannia humanal & si communitas aliqua est necessaria in humana vita : \textbf{ sequitur communitatem domus | ad huiusmodi vitam necessariam esse . } In praecedenti ergo capitulo determinauimus de societate humana , \\\hline
2.1.2 & por que por esta razon se muestra manifiestamente \textbf{ que la comunidat dela casa es neçessaria } por que todas las comuni dades ençierran en ssi & quia per hoc manifeste ostenditur \textbf{ necessariam esse communitatem domesticam : } cum omnis alia communitas communitatem illam praesupponat . \\\hline
2.1.2 & por que todas las comuni dades ençierran en ssi \textbf{ e ante ponen esta comunidat dela casa ¶ Et } pues que assy es deuedes saber & necessariam esse communitatem domesticam : \textbf{ cum omnis alia communitas communitatem illam praesupponat . } Aduertendum ergo quod \\\hline
2.1.2 & que pueden fazer çibdat e regno . \textbf{ Et por ende la comunidat dela casa } sea alas trͣs comunidades & ex consequenti constituere possunt ciuitatem , et regnum . \textbf{ Hoc ergo modo communitas domus se habet } ad communitates alias : \\\hline
2.1.2 & Et por ende la comunidat dela casa \textbf{ sea alas trͣs comunidades } en tal manera que todas las otras comunidades ençierren & Hoc ergo modo communitas domus se habet \textbf{ ad communitates alias : } quia omnes aliae ipsam praesupponunt : \\\hline
2.1.2 & o el regno \textbf{ sienpre la comunidat dela casa se ha alas otras comuindades } en esta manera & et qualitercunque constituatur vicus , ciuitas , siue regnum , \textbf{ semper sic se domus habet | ad communitates alias , } quod est pars omnium aliarum , \\\hline
2.1.2 & e la ponen ante si¶ visto \textbf{ en qual manera la comunidat dela casa se ha alas otras comuidades de ligero puede parescer } en qual manera esta comunidat & et aliquo modo eam omnes aliae praesupponunt . \textbf{ Viso , quomodo communitas domus se habeat | ad communitates alias : } de leui patet , \\\hline
2.1.2 & en qual manera la comunidat dela casa se ha alas otras comuidades de ligero puede parescer \textbf{ en qual manera esta comunidat } es neçessaria ala uida humanal . & de leui patet , \textbf{ quomodo huiusmodi communitas sit necessaria in humana vita . } Nam si omnes communitates aliae domum praesupponunt : \\\hline
2.1.2 & es neçessaria ala uida humanal . \textbf{ Ca si todas las otras comunidades } antepo nen la comuidat dela & quomodo huiusmodi communitas sit necessaria in humana vita . \textbf{ Nam si omnes communitates aliae domum praesupponunt : } si aliqua communitas est necessaria \\\hline
2.1.2 & e por si vale a conplimiento dela uida . \textbf{ Conuiene que la comunidat dela casa sea mas neçessaria } Et pues que assi es los Reyes e los prinçipes & ad per se sufficientiam vitae , \textbf{ oportet communitatem domus necessariam esse . } Reges ergo et Principes , \\\hline
2.1.3 & e dela fechura dela çibdat \textbf{ en quanto estas cosas son ordenadas ala comunidat } e ala polliçia e ordenamiento de los çibdadanos . & et de fabrica ciuitatis , \textbf{ ut ordinantur ad communitatem , } et ad politiam ciuium . \\\hline
2.1.3 & Et pues que assi es nos entendemos de determinar dela casa \textbf{ que es comunidat delas perssonas dela casa } en qual manera ella es la comunidat primera . & Intendimus ergo ostendere de domo , \textbf{ quae est communitas personarum domesticarum , } quomodo sit communitas prima . \\\hline
2.1.3 & que es comunidat delas perssonas dela casa \textbf{ en qual manera ella es la comunidat primera . } Et por ende deuemos notar & quae est communitas personarum domesticarum , \textbf{ quomodo sit communitas prima . } Notandum ergo , \\\hline
2.1.3 & Et pues que assi es departidas las cosas primeras en esta manera de ligero puede paresçer \textbf{ en qual manera la comunidat dela casa se ha } ala comunindat dela çibdat & de leui patet , \textbf{ quomodo communitas domus se habet } ad communitatem ciuitatis , \\\hline
2.1.3 & ala comunindat dela çibdat \textbf{ e alas otras comunidades . } Ca en la obra dela execucion la casa es primero que el uarrio & ad communitatem ciuitatis , \textbf{ et ad communitates alias . } Nam in executione et opere domus \\\hline
2.1.3 & e en la entençion \textbf{ laso trisco munindades son primero que la comunidat dela } casa¶Otrossi la comunidat dela casa es primero & sed in uoluntate \textbf{ et in intentione communitates illae } praecedunt communitatem domesticam . \\\hline
2.1.3 & laso trisco munindades son primero que la comunidat dela \textbf{ casa¶Otrossi la comunidat dela casa es primero } que las otras comunidades & et in intentione communitates illae \textbf{ praecedunt communitatem domesticam . } Rursus uia generationis \\\hline
2.1.3 & casa¶Otrossi la comunidat dela casa es primero \textbf{ que las otras comunidades } en manera de generaçion & et in intentione communitates illae \textbf{ praecedunt communitatem domesticam . } Rursus uia generationis \\\hline
2.1.3 & mas las otras comunindades son primero \textbf{ que la comunidat delan casa } en manera de perfectiuo & Rursus uia generationis \textbf{ et temporis domestica communitas praecedit communitates alias : } sed in uia perfectionis \\\hline
2.1.3 & ot de conplimiento \textbf{ por que paresçe que la comunidat dela casa se ha en dos maneras } alas otras comunidades . & Videtur communitas domus \textbf{ ad communitates alias dupliciter se habere . } Primo , quia huiusmodi communitas \\\hline
2.1.3 & por que paresçe que la comunidat dela casa se ha en dos maneras \textbf{ alas otras comunidades . } ¶ La primera es por que esta comunidat en conparaçion delas otras es mas menguada & Videtur communitas domus \textbf{ ad communitates alias dupliciter se habere . } Primo , quia huiusmodi communitas \\\hline
2.1.3 & alas otras comunidades . \textbf{ ¶ La primera es por que esta comunidat en conparaçion delas otras es mas menguada } e las otras son mas conplidas que ella & ad communitates alias dupliciter se habere . \textbf{ Primo , quia huiusmodi communitas | respectu aliarum est imperfecta : } omnes uero aliae sunt perfectiores ipsa ; \\\hline
2.1.3 & e las otras son mas conplidas que ella \textbf{ por que todas las otras comunidades ençierran en ssi la comunidat dela casa } e ennaden alguna cosa sobre ella . & omnes uero aliae sunt perfectiores ipsa ; \textbf{ cum enim omnis alia communitas | includat communitatem domus , } et addat aliquid supra ipsam ; \\\hline
2.1.3 & e ennaden alguna cosa sobre ella . \textbf{ Et por ende todas las otras comunidades son mas conplidas que ella . } Et pueᷤ que assi es la comuidat dela casa se ha & et addat aliquid supra ipsam ; \textbf{ omnes aliae communitates sunt perfectiores ea . } Habet ergo se communitas domus \\\hline
2.1.3 & assi commo la parte al su todo ¶ \textbf{ Otrossi la comunidat dela casa se ha alas otras comunidades } assi commo las çosas que son ordenadas & et sicut pars ad totum . \textbf{ Rursus huiusmodi communitas | se habet ad alias , } sicut quod est ad finem se habet ad ipsum finem . \\\hline
2.1.3 & e la çibdat para fazer el regno . \textbf{ Et pues que assi es la comunidat del uarrio } es fin dela comunidat dela casa & ciuitas propter regnum . \textbf{ Communitas ergo vici est finis communitatis domus , } communitas ciuitatis communitatis vici : \\\hline
2.1.3 & Et pues que assi es la comunidat del uarrio \textbf{ es fin dela comunidat dela casa } e la comunidat dela çibdat & ciuitas propter regnum . \textbf{ Communitas ergo vici est finis communitatis domus , } communitas ciuitatis communitatis vici : \\\hline
2.1.3 & es fin dela comunidat dela casa \textbf{ e la comunidat dela çibdat } es fin dela comunidat del uarrio . & Communitas ergo vici est finis communitatis domus , \textbf{ communitas ciuitatis communitatis vici : } sed communitas regni est finis omnium praedictorum . \\\hline
2.1.3 & e la comunidat dela çibdat \textbf{ es fin dela comunidat del uarrio . } Mas la comunidat del regno es fin de todas las otras comuindades sobredichas . & Communitas ergo vici est finis communitatis domus , \textbf{ communitas ciuitatis communitatis vici : } sed communitas regni est finis omnium praedictorum . \\\hline
2.1.3 & es fin dela comunidat del uarrio . \textbf{ Mas la comunidat del regno es fin de todas las otras comuindades sobredichas . } Por la qual cosa conmola cosa non acabada sea primero & communitas ciuitatis communitatis vici : \textbf{ sed communitas regni est finis omnium praedictorum . } Quare cum imperfectum \\\hline
2.1.3 & e ala casa dize \textbf{ quela comunidat dela çibdat es la primera . } Et esto non se deue entender que es primera por generaçion & ad vicum et domum , \textbf{ ait , quod prima communitas est communitas ciuitatis ; } quod non est intelligendum \\\hline
2.1.3 & por manera de perfecçion e de cunplimiento . \textbf{ Otrossi commo la comunidat dela casa } non solamente se aya alas otras comuindades & et complementi . \textbf{ Amplius cum communitas domus ad communitates alias } non solum se habeat \\\hline
2.1.3 & mas conplidamente en el terçero libro \textbf{ Visto en qual manera la comunidat dela casa es primero en alguna manera que las otras comuni dades de ligero puede paresçer } en alguna manera esta comunidat dela cała es natural & ut in tertio libro plenius ostendetur . \textbf{ Viso , quomodo communitas domus | aliquo modo est prior , } quam communitates aliae : \\\hline
2.1.3 & Visto en qual manera la comunidat dela casa es primero en alguna manera que las otras comuni dades de ligero puede paresçer \textbf{ en alguna manera esta comunidat dela cała es natural } Ca commo la natura non presupone & quam communitates aliae : \textbf{ de leui videri potest , | quomodo sit huiusmodi communitas naturalis . } Nam cum natura non praesupponat artem , \\\hline
2.1.3 & ai al comiuncable e aconpanable \textbf{ commo todas las comunidades presupongan } e ante pongan la comunidat dela casa & naturaliter animal communicatiuum et sociale , \textbf{ cum omnis communitas } praesupponat communitatem domus , \\\hline
2.1.3 & commo todas las comunidades presupongan \textbf{ e ante pongan la comunidat dela casa } conuiene quala comunidat dela casa o la casa sea cosa natural . & cum omnis communitas \textbf{ praesupponat communitatem domus , } oportet communitatem domesticam siue domum \\\hline
2.1.3 & e ante pongan la comunidat dela casa \textbf{ conuiene quala comunidat dela casa o la casa sea cosa natural . } Et por ende conuiene alos Reyes e alos prinçipes & praesupponat communitatem domus , \textbf{ oportet communitatem domesticam siue domum | quid naturale esse . } Reges ergo et Principes decet \\\hline
2.1.3 & e que conoscan que cosa \textbf{ e qual es la comunidat dela casa } ca es comunidat en alguna manera natural & ut sciant domum propriam gubernare , \textbf{ et ut cognoscant quae et qualis est communitas domus } ut se habet ad regnum et ciuitatem , \\\hline
2.1.3 & e qual es la comunidat dela casa \textbf{ ca es comunidat en alguna manera natural } Et en algunan manera esta comunidat se ha al regno & ut sciant domum propriam gubernare , \textbf{ et ut cognoscant quae et qualis est communitas domus } ut se habet ad regnum et ciuitatem , \\\hline
2.1.4 & e ya es declarado en alguna manera en el capitulo \textbf{ sobredicho qual es la comunidat dela casa . } Ca ya mostrado es & Est autem ex praecedenti capitulo aliqualiter declaratum , \textbf{ qualis sit communitas domus : } cum ostensum sit \\\hline
2.1.4 & que el omne es naturalmente ainalia domestica e de casa \textbf{ e quela comunidat dela casa es en alguna manera natural . } Empero por que esto non auemos & quod homo est naturaliter animal domesticum , \textbf{ et quod communitas domus est quodammodo naturalis . } Attamen quia per hoc non sufficienter habetur \\\hline
2.1.4 & conplidamente qual es esta comuidat dela \textbf{ casapor ende entendemos de dezir algunas cosas dela comunidat dela casa . } Pues que assi es deuedes saber & qualis sit huiusmodi communitas , \textbf{ ideo intendimus aliqua dicere de communitate domestica . } Sciendum ergo , \\\hline
2.1.4 & diziendo \textbf{ que la casa es comunidat } segunt natura construyda e fecha para cada dia & Philosophum 1 Politicorum sic describere communitatem domus : \textbf{ videlicet , quod domus est communitas secundum naturam , } constituta quidem in omnem diem . \\\hline
2.1.4 & sobredich̃o Et alguna cola finca adelante de declarar . \textbf{ Ca que la casa sea comunidat } segunt natura e natural de suso es prouado gruesamente e figuaalmente & et aliquid restat ulterius declarandum . \textbf{ Nam quod domus sic communitas secundum naturam , } superius grosse et figuraliter probabatur , \\\hline
2.1.4 & Pues que assi es finça de declarar en la difiniçion sobredichͣ \textbf{ en qual manera la casa sea comunidat establesçida para cada dia . } Et para auer esto deuedes saber & in descriptione praedicta , \textbf{ quomodo domus sit communitas constituta in omnem diem . } Ad cuius euidentiam aduertendum , \\\hline
2.1.4 & han mester cada dia conprar e vender . \textbf{ pues que assi es la comunidat dela casa fue fecha para aquellas cosas } que auemos mester de cada dia . & vel venditione continue egeant . \textbf{ Communitas ergo domus facta fuit propter ea , } quibus quotidie indigemus . \\\hline
2.1.4 & todas las cosas neçessarias para la uida \textbf{ non cunplie la comunidat de vna casa } mas conuiene de dar comunidat de varrio . & Verum quia in una domo non reperiuntur omnia necessaria ad vitam , \textbf{ non sufficiebat communitas domestica , } sed oportuit dare communitatem vici , \\\hline
2.1.4 & non cunplie la comunidat de vna casa \textbf{ mas conuiene de dar comunidat de varrio . } Por que commo el uarrio sea fech̃ de muchas casas & non sufficiebat communitas domestica , \textbf{ sed oportuit dare communitatem vici , } ita quod cum vicus constet \\\hline
2.1.4 & en el primero libro delas politicas \textbf{ que assi commo la comunidat dela casa es establesçida } para las obras de cada dia & Propter quod Philosophus 1 Politicorum ait , \textbf{ quod sicut communitas domus | constituta est } in omnem diem , \\\hline
2.1.4 & todas las cosas neçessarias ala uida \textbf{ conuiene de dar comunidat ala çibdat } sobre la comunidat deluarrio . & omnia necessaria ad vitam , \textbf{ praeter communitatem vici | oportuit } dare communitatem ciuitatis . \\\hline
2.1.4 & conuiene de dar comunidat ala çibdat \textbf{ sobre la comunidat deluarrio . } Et por ende & oportuit \textbf{ dare communitatem ciuitatis . } Communitas ergo ciuitatis esse videtur \\\hline
2.1.4 & ordende todas estas cosas \textbf{ que la casa es comunidat } segunt natura establesçida para cada dia . & Erit ergo hic ordo , \textbf{ quod domus est communitas } secundum naturam constituta in omnem diem . \\\hline
2.1.4 & segunt natura establesçida para cada dia . \textbf{ Mas el uatrio es comunidat estableçida } para las obras & secundum naturam constituta in omnem diem . \textbf{ Vicus autem est communitas constituta } in opera non diurnalia . \\\hline
2.1.4 & que non son mester de cada dia . \textbf{ Et la çibdat es comunidat establesçida } para conplimiento delas cosas & in opera non diurnalia . \textbf{ Ciuitas vero est communitas constituta } ad sufficientiam in vita tota . \\\hline
2.1.4 & que son menester para toda la uida \textbf{ Mas el regno es comunidat establesçida } non solamente para conplir las menguas dela uida & ad sufficientiam in vita tota . \textbf{ Sed regnum est communitas constituta } non solum ad supplendum indigentias vitae , \\\hline
2.1.4 & e otros muchos castiellos \textbf{ Et por ende paresçe qual es la comunidat dela casa . } Ca es comunidat natural & adiunctae aliae plurimae ciuitates et castra . \textbf{ Patet ergo qualis sit communitas domus , } quia est communitas naturalis constituita \\\hline
2.1.4 & Et por ende paresçe qual es la comunidat dela casa . \textbf{ Ca es comunidat natural } e establesçida para las obras cotidianas & Patet ergo qualis sit communitas domus , \textbf{ quia est communitas naturalis constituita } propter opera diurnalia et quotidiana . \\\hline
2.1.4 & Ca assi commo paresçe por las cosas \textbf{ sobredichͣs la casa es vna comunidat } e vna conpannia de muchͣs ꝑssonas . & ( ut patet ex habitis ) \textbf{ sit communitas quaedam et societas personarum : } cum non sit proprie communitas nec societas ad seipsum , \\\hline
2.1.4 & e vna conpannia de muchͣs ꝑssonas . \textbf{ Et commo non sea propreamente comunidat } nin conpannia de vno & sit communitas quaedam et societas personarum : \textbf{ cum non sit proprie communitas nec societas ad seipsum , } si in domo communitatem saluare volumus , \\\hline
2.1.4 & non solamente la casa es vna comiundat \textbf{ mas en la casa conuiene de dar muchͣs comunidades } la qual cosa non puede ser sin muchͣs perssonas . & non solum domus est communitas quaedam , \textbf{ sed in domo oportet | dare plures communitates : } quod sine pluralitate personarum \\\hline
2.1.4 & por las cosas ya dichͣs en la uida humanal \textbf{ non solamente es menester la comunidat dela casa } mas ahun la comunidat del uarrio e dela çibdat e del regno & in vita humana \textbf{ non solum est | expediens communitas domus , } sed et vici , ciuitatis , et regni . \\\hline
2.1.4 & que por las que ya dichos son \textbf{ es menester la comunidat dela çibdat } e del regno esto se mostrara mas conplidamente en el terçero libro . & quam propter iam dictas , \textbf{ sit expediens communitas ciuitatis , et regni , } in tertio libro plenius ostendetur . \\\hline
2.1.5 & Conuiene a saber \textbf{ De comunidat de uaron et de muger . } Et de comunidat de sennor e de sieruo . & quod ex duabus communitatibus , \textbf{ videlicet , ex communitate viri et uxoris , domini et serui , } constat domus prima . \\\hline
2.1.5 & De comunidat de uaron et de muger . \textbf{ Et de comunidat de sennor e de sieruo . } Ca dize que la primera casa deue ser establesçida destas dos comuindades & quod ex duabus communitatibus , \textbf{ videlicet , ex communitate viri et uxoris , domini et serui , } constat domus prima . \\\hline
2.1.5 & fazen ser la casa cosa natural . \textbf{ Ca la comunidat del uaron e dela mugnies ordenada ala generacion . } Mas la comunidat del sennor e del sieruo es ordenada ala & Hoc ergo modo hae duae communitates faciunt domum esse quid naturale : \textbf{ quia communitas viri et uxoris ordinatur ad generationem , } communitas vero domini \\\hline
2.1.5 & Ca la comunidat del uaron e dela mugnies ordenada ala generacion . \textbf{ Mas la comunidat del sennor e del sieruo es ordenada ala } conseruaçique por la qual cosa si la generaçion e la conseruaçion es cosa natural & quia communitas viri et uxoris ordinatur ad generationem , \textbf{ communitas vero domini } et serui ad conseruationem . Quare si generatio et conseruatio est quid naturale , \\\hline
2.1.5 & e dela fenbra . \textbf{ Mas que la comunidat del sennor e del sieruo } sea por salud e por conseruaçion dela casa & instituta est societas maris et foeminae . \textbf{ Sed quod communitas domini et serui sit } propter salutem et conseruationem , \\\hline
2.1.5 & en essa misma manera es \textbf{ y . menester la comunidat del sennor } e del sieruo & communitas viri et uxoris propter generationem , \textbf{ sic requiritur ibi communitas domini } et serui propter salutem \\\hline
2.1.5 & por que sin ellas la primera casa non puede estar conueniblemente . \textbf{ Mas si la comunidat del sennor e del sieruo } en otra manera es & non potest existere . \textbf{ Utrum autem communitas domini } et serui naturaliter instituta sit \\\hline
2.1.5 & Ca por auentra a paresçrie alguno que commo la casa primera \textbf{ sea establesçida de dos comunidades } e cada vna delas comunidades & quod cum domus prima constet \textbf{ ex duabus communitatibus , } et quaelibet communitas requirat \\\hline
2.1.5 & sea establesçida de dos comunidades \textbf{ e cada vna delas comunidades } aya menester dos perssonas o dos linages de perssonas & ex duabus communitatibus , \textbf{ et quaelibet communitas requirat } duas personas uel duo genera personarum , \\\hline
2.1.6 & Emposi la casa fuere acabada conuiene de dar \textbf{ y la terçera comunidat } que es de padre e de fijo . & si domus debet esse perfecta , \textbf{ oportet ibi dare communitatem tertiam , } scilicet patris et filii . \\\hline
2.1.6 & son las primeras obras dela natura . \textbf{ por ende con razon la comunidat del uaron } e dela muger que es para la generaçion & sunt prima opera naturae ; \textbf{ merito ergo communitas viri et uxoris , } quae est propter generationem ; \\\hline
2.1.6 & ca sin ellas non puede ser la primera casa conuenible mente . \textbf{ Mas la terçera comunidat } que es de padre e de fijo & quia sine eis domus congrue esse non potest : \textbf{ sed tertiam etiam communitatem , } quae est patris et filii , \\\hline
2.1.6 & por la qual cosa \textbf{ commo la comunidat del padre al fijo tome nasçençia e comienço de aquello que el padre e la madre } engendran su semeiança esta tal comunidat non es dicha de razon dela primera casa & nisi sit iam perfectus . \textbf{ Quare cum communitas patris ad filium sumat originem | ex eo quod parentes sibi simile produxerunt : } huiusmodi communitas non dicitur \\\hline
2.1.6 & commo la comunidat del padre al fijo tome nasçençia e comienço de aquello que el padre e la madre \textbf{ engendran su semeiança esta tal comunidat non es dicha de razon dela primera casa } mas es de razon dela casa ya acabada & ex eo quod parentes sibi simile produxerunt : \textbf{ huiusmodi communitas non dicitur | esse de ratione domus primae , } sed est de ratione domus perfectae ; \\\hline
2.1.6 & Mas que ala perfectiuo dela \textbf{ casafaga menester esta terçera comunidat } pademos lo prouar & videlicet patris et filii . \textbf{ Quod autem ad perfectionem domus requiratur haec tertia communitas , } triplici via venari possumus . \\\hline
2.1.6 & que non son bien auentraadas . \textbf{ Pues que assi es que la terçera comunidat } que es del padͤ & Sic et feliciora sunt perfectiora infelicibus . \textbf{ Ergo quod ad perfectionem domus requiratur tertia communitas , } quae est patris et filii , \\\hline
2.1.6 & o por mengua de amos \textbf{ Mas commo el maslo e la fenbra e el marido e la muger sea la primera parte dela casar la primera comunidat } que es menester para la uida dela casa & vir et uxor sit \textbf{ prima pars domus | et prima communitas , } quae requiritur in vita domestica : \\\hline
2.1.6 & e la fenbra obediente . \textbf{ Mas en la comunidat del padre } e del fijo el padre deua sienpre mandar & et foemina obsequens : \textbf{ in communitate vero patris et filii , } pater debet esse imperans , \\\hline
2.1.6 & e el fij̉o ser obediente . \textbf{ Et en la comunidat del señor } e del sieruo el señor deue mandar & et filius obtemperans ; \textbf{ in communitate quidem domini et serui , } dominus debet esse praecipiens , \\\hline
2.1.6 & ¶ Pues que assi es paresçe \textbf{ quantas son las comunidades } en la casa ̀ acabada & et dominus seruorum . \textbf{ Patet ergo quot communitates sunt in domo perfecta , } et quot regimina , \\\hline
2.1.7 & en el primero libro delas politicas \textbf{ en la comunidat dela casa } primeramente conuiene de ayuntar el uaron con la mugni & quia secundum Philosophum 1 Politic’ \textbf{ in communitate domestica , } primum oportet \\\hline
2.1.7 & e puede engendrar semeiable dessi ¶ \textbf{ Et pues que assi es la comunidat del uaton e dela muger } que es para la generaçion es la primera parte dela casa & et potest sibi simile producere . \textbf{ Communitas ergo maris et foeminae , } quae est propter generationem , \\\hline
2.1.7 & que quiere dezir ꝑtiçipante con otro \textbf{ Mas la comunidat en la uida humanal } assi commo dicho es dessuso & hominem esse naturaliter animal sociale et communicatiuum . \textbf{ Communitas autem in vita humana } ( ut supra tangebatur ) \\\hline
2.1.7 & ¶ Et la otra de regno . \textbf{ Mas todas estas comunidades } ante ponen la comunidat dela casa . & quaedam ciuitatis , quaedam regni . \textbf{ Omnes autem hae communitates } praesupponunt communitatem domesticam . \\\hline
2.1.7 & Mas todas estas comunidades \textbf{ ante ponen la comunidat dela casa . } Et pues que assi es commo la casa sea primero & Omnes autem hae communitates \textbf{ praesupponunt communitatem domesticam . } Cum ergo domus sit prior vico , ciuitate , et regno : \\\hline
2.1.7 & que çiuil e de çibdat . \textbf{ Et la comunidat dela casa mas paresçe } que es natural al omne & magis est animal de mesticum , \textbf{ quam ciuile : et communitas domus } magis videtur esse naturalis ipsi homini , \\\hline
2.1.7 & que es natural al omne \textbf{ que la comunidat del uarrio } nin dela çibdat nin del regno . & magis videtur esse naturalis ipsi homini , \textbf{ quam communitas vici , ciuitatis , et regni . } Nam si considerentur ea , \\\hline
2.1.7 & e al mantenemiento del humanal linage \textbf{ assi commo es ordenada la comunidat dela casa } Ca si las comunidades dichas del uarrio & propter conseruationem speciei , \textbf{ sicut communitas domus : } nam si praedictae communitates \\\hline
2.1.7 & assi commo es ordenada la comunidat dela casa \textbf{ Ca si las comunidades dichas del uarrio } e dela çibdat son ordenadas al mantenemiento & sicut communitas domus : \textbf{ nam si praedictae communitates } ordinantur \\\hline
2.1.7 & que de çibdat \textbf{ commo la primera comunidat dela casa sea } ayuntamientode uaron & quam politicum : \textbf{ cum prima communitas ipsius domus sit coniunctio viri et uxoris , } sequitur ex parte ipsius communitatis humanae , \\\hline
2.1.7 & e que mas es ayuntable \textbf{ por comunidat coniugable e de matermoino } que por comunidat de barrio & quod homo magis sit animal coniugale \textbf{ quod politicum ; et quod magis sit communicatiuum communitate coniugali , } quam communitate vici ciuitatis , et regni : \\\hline
2.1.7 & por comunidat coniugable e de matermoino \textbf{ que por comunidat de barrio } nin de çibdat nin de regno . & quod politicum ; et quod magis sit communicatiuum communitate coniugali , \textbf{ quam communitate vici ciuitatis , et regni : } quia domus , \\\hline
2.2.1 & Et pues que assi es \textbf{ deuedessaber que commo la comunidat } e la conpannia del uaron e dela muger e del sennor e del sieruo part enescan ala casa primera & Sciendum igitur , \textbf{ quod cum communitas viri et uxoris , } et domini et serui pertineant \\\hline
2.3.9 & que si non fuesse sinon la comiundat dela casa \textbf{ que es la comunidat primera } non seria ninguna muda conn neçessaria & nisi communitas domus \textbf{ quae est communitas prima , } nulla commutatio esset necessaria . Nam in domo dominatur paterfamilias , \\\hline
2.3.9 & enł primero libro delas politicas \textbf{ que en la primera comunidat } que es comuidat dela casa es cosa prouada & quod dicitur 1 Politicorum \textbf{ quod in prima communitate quae est domus , } manifestum est nullum esse opus ipsius commutationis igitur \\\hline
2.3.9 & sin la comiundat dela casa \textbf{ assi commo es comunidat del uarrio o dela çibdat } o de todo el regno o dela prouiçia . & a communitate domus : \textbf{ ut communitas vici , vel ciuitatis , } vel totius regni , et prouinciae , \\\hline
3.1.1 & or que toda çibdat conuiene que sea alguna comunindat \textbf{ commo toda comunidat sea por graçia de algun bien . } Conuiene que la çibdat sea establesçida por algun bien & esse communitatem quandam , \textbf{ cum omnis communitas fit | gratia alicuius boni , } oportet ciuitatem ipsam constitutam esse propter aliquod bonum . \\\hline
3.1.1 & vna inclinaçion de natura es en todos los orans \textbf{ atal comunidat } qual es la comuidat dela çibdat . & natura quidem impetus in omnibus inest \textbf{ ad talem communitatem , } qualis est communitas ciuitatis . \\\hline
3.1.1 & que es mas prinçipal \textbf{ e esta tal es la comunidat dela çibdat } la qual en conparacion dela comunidat dela casa e del barrio es mas prinçipal & ad ipsum communitas principalissima : \textbf{ huius autem est communitas ciuitatis , } quae respectu communitatis domus , \\\hline
3.1.1 & e esta tal es la comunidat dela çibdat \textbf{ la qual en conparacion dela comunidat dela casa e del barrio es mas prinçipal } por la qual razon & huius autem est communitas ciuitatis , \textbf{ quae respectu communitatis domus , | et vici principalissima existit . } Quare si communitas domestica ordinatur ad bonum \\\hline
3.1.1 & assi commo es prouado de suso \textbf{ mas conplidamente en el segundo libro la comunidat del barno } que es mas prinçipal & et etiam ad multa bona , \textbf{ ut supra in secundo libro diffusius probabatur : | communitas vici , } quae est principalior communitate domestica , \\\hline
3.1.1 & mas sera ordenada a bien \textbf{ e avn la comunidat dela çibdat } que es much mas prinçipal & multo magis ordinatur ad bonum : \textbf{ et ad hoc communitas ciuitatis , } quae est principalissima communitas respectu vici , \\\hline
3.1.1 & que es much mas prinçipal \textbf{ que la comunidat del barrio } nin dela casa much mas es ordenada abien que todas las otras & et ad hoc communitas ciuitatis , \textbf{ quae est principalissima communitas respectu vici , } et domus , \\\hline
3.1.1 & que si nos dizimos \textbf{ que toda comunidat es establesçida } por grande algun bien & Hoc est ergo quod dicitur primo Polit’ \textbf{ quod si communitatem omnem gratia alicuius boni dicimus constitutam , } maxime autem principalissimam omnium , \\\hline
3.1.1 & por grande algun bien \textbf{ e esta es comunidat politica } que es llamada & et omnes alias circumplectens , potissime gratia boni constitutam esse contingit : \textbf{ haec autem est communitas politica , } quae communi nomine vocatur ciuitas . \\\hline
3.1.1 & mas prinçipal \textbf{ mas a vn otra comunidat ay mas prinçipal } que ella la qual es comunidat del regno & sed respectu communitatis domus , et vicio . \textbf{ Est autem alia communitas principalior ea , } cuiusmodi est communitas regni , \\\hline
3.1.1 & mas a vn otra comunidat ay mas prinçipal \textbf{ que ella la qual es comunidat del regno } dela qual diremos en su logar & Est autem alia communitas principalior ea , \textbf{ cuiusmodi est communitas regni , } de qua suo loco dicetur : \\\hline
3.1.1 & ca mostraremos \textbf{ que la comunidat del regno es prouechosa en la uida humanal } e es mas prinçipal & de qua suo loco dicetur : \textbf{ ostendemus enim communitatem regni | utilem esse in vita humana , } et esse principaliorem communitate ciuitatis . \\\hline
3.1.1 & e es mas prinçipal \textbf{ que la comunidat dela çibdat ca paresçe } que assi se ha la comunidat del regno & utilem esse in vita humana , \textbf{ et esse principaliorem communitate ciuitatis . } Videtur enim suo modo communitas regni \\\hline
3.1.1 & que la comunidat dela çibdat ca paresçe \textbf{ que assi se ha la comunidat del regno } ala comunidat dela çibdat & et esse principaliorem communitate ciuitatis . \textbf{ Videtur enim suo modo communitas regni | se habere } ad communitatem ciuitatis , \\\hline
3.1.1 & que assi se ha la comunidat del regno \textbf{ ala comunidat dela çibdat } commo la comunidat dela çibdat se ha & se habere \textbf{ ad communitatem ciuitatis , } sicut haec communitas se habet ad domum , et vicum . \\\hline
3.1.1 & ala comunidat dela çibdat \textbf{ commo la comunidat dela çibdat se ha } ala comiundat dela casa e del uarrio . & ad communitatem ciuitatis , \textbf{ sicut haec communitas se habet ad domum , et vicum . } Nam ciuitas sicut complectitur domum , et vicum ; \\\hline
3.1.1 & que estas dichos dos comuindades . \textbf{ bien assi la comunidat del regno ençierra en ssi la comunidat de la çibdat } e es mucho mas acabada et conplida en la uida humanal & quam communitates praedictae : \textbf{ sic communitas regni | circumplectitur communitatem ciuitatis , } et est multo perfectior \\\hline
3.1.1 & e es mucho mas acabada et conplida en la uida humanal \textbf{ que la comunidat dela çibdat } on a basta de dezer & et magis sufficiens in vita , \textbf{ quam communitas illa . } Non sufficit dicere ciuitatem constitutam \\\hline
3.1.2 & porque por ella alcançan los omes ser acabados \textbf{ e beuir en comunidat politica e de çibdat } ca sin ella la uida del omne non puede ser & quia per eam homines consequuntur \textbf{ omnia tria praedicta bona . | Nam ipsum viuere consequuntur homines ex communitate politica : } quia sine ea vita hominis \\\hline
3.1.2 & en el primero libro delas politicas \textbf{ que la comunidat } que es cibdat es cosa & quod scribitur primo Politicorum \textbf{ quod communitas , } quae est ciuitas constans \\\hline
3.1.2 & que es fechͣ de much suarrios \textbf{ et tales comunidat acabada } ca esto se sigͤel dezir & ex pluribus vicis , \textbf{ est communitas perfecta : } quia iam \\\hline
3.1.2 & ca esto se sigͤel dezir \textbf{ que tal comunidat es la que ha termino } por si e todo conplimiento e abastamiento de uida & ( ut consequens est dicere ) \textbf{ huiusmodi communitas est | habens terminum omnis } per se sufficientiae vitae . \\\hline
3.1.2 & nin durar \textbf{ otdenaron la comunidat politica } que era fechͣ & constituta ciuitas stare non posset , \textbf{ ordinauerunt communitatem politicam , } quae facta erat ad viuere , \\\hline
3.1.4 & ¶ La primera razon se tomadesto \textbf{ que esta comunindat ençierra en si la comunidat dela casa } e la comunidat del uartio ¶ & Prima uia sumitur \textbf{ ex eo quod communitas complectitur domum et uicum . } Secunda ex eo quod est illarum finis et complementum . \\\hline
3.1.4 & que esta comunindat ençierra en si la comunidat dela casa \textbf{ e la comunidat del uartio ¶ } La segunda razon desto & Prima uia sumitur \textbf{ ex eo quod communitas complectitur domum et uicum . } Secunda ex eo quod est illarum finis et complementum . \\\hline
3.1.4 & que sirue a conplimiento de uida \textbf{ e por ende la comunidat dela casa } e avn del uartio son cosas naturales & ad sufficientiam uitae . \textbf{ Communitas ergo domestica } et etiam uici naturalia sunt , \\\hline
3.1.4 & que la çibdat es fin \textbf{ et conplimiento de las dichͣs dos comunidades } ca assi commo prueua el pho & sumitur ex eo quod ciuitas est \textbf{ illarum communitatum finis et complementum . } Nam ut arguit Philosophus primo Politicorum \\\hline
3.1.4 & ca la casa se faze de comuidat de omne e de su muger e de sennor e de sieruo e de padre e de fijos . Et cada vna destas comuidades es cosa segunt natura bien \textbf{ assi avn la comunidat del uarrio es cosa natural } non solamente por que sirue a conplimiento dela uida & ex communitate viri et uxoris , domini , et serui , patris et filii , quarum quaelibet est secundum naturam . \textbf{ Sic etiam communitas vici est | quid naturale , } non solum quia deseruit \\\hline
3.1.4 & por ende \textbf{ si la comunidat dela casa es ordenada a alcançar lo que es delectable } e para foyr & et quid iniustum . \textbf{ Si ergo communitas domestica ordinatur ad prosequendum conferens , } et ad fugiendum nociuum : \\\hline
3.1.4 & lo que es enpeçible . \textbf{ Et la comunidat dela çibdat sobre esto es ordenada a segnir } lo que es iusto e a foyr & et ad fugiendum nociuum : \textbf{ communitas vero ciuitatis | ultra hoc ordinatur } ad prosequendum iustum , \\\hline
3.1.4 & conuiene \textbf{ que la comunidat dela casa } e la comunidat dela çibdat sean cosas naturales & et ad fugiendum iniustum , \textbf{ oportet communitatem domesticam } et ciuilem esse quid naturale . \\\hline
3.1.4 & e la comunidat dela çibdat sean cosas naturales \textbf{ ca si la natura dio al omne palabra natural aquella comunidat } que es ordenada a aquellas cosas & et ciuilem esse quid naturale . \textbf{ Nam si natura dedit homini sermonem , | naturalis est illa communitas } quae ordinatur ad illa , \\\hline
3.1.4 & non han de ser propreamente \textbf{ en la comuidat dela casa mas enla comunidat dela çibdat } ca en la çibdat do los çibdadanos han sus possessiones propias & esse in communitate domestica , \textbf{ sed in communitate ciuili . | In ciuitate enim , } ubi ciues habent possessiones proprias et distinctas , \\\hline
3.1.4 & por las quales se pueden manteñ en la uida e esto contesçe mayormente segunt que dize el pho \textbf{ por la comunidat dela çibdat } por que deue contener en ssi todas aquellas cosas & ( secundum Philosophum ) \textbf{ per communitatem ciuilem , } eo quod ciuitas debeat \\\hline
3.1.5 & e podemos mostrar por tres razones \textbf{ que sin la comunidat dela çibdat } cosa aprouechosa fue ala uida humanal & Possumus autem triplici via ostendere , \textbf{ quod praeter communitatem ciuitatis , } utile est humanae vitae \\\hline
3.1.5 & cosa aprouechosa fue ala uida humanal \textbf{ de establesçer comunidat de regno ¶ } La primera razon se toma de parte del conplimiento dela uida & utile est humanae vitae \textbf{ statuere communitatem regni . } Prima via sumitur \\\hline
3.1.5 & en el primero libro delas politicas \textbf{ que la comunidat acabada } que es çibdat se faga de muchsuarios & Nam cum ait Philosophus primo Polit’ \textbf{ quod communitas perfecta , } quae est ciuitas constans \\\hline
3.1.9 & mas de quanto valie . \textbf{ Et pues que assi es la comunidat delas possessiones } que ordenaua socrates & plus ualere quam ualeat . \textbf{ Communitas ergo possessiones } quam ordinabat Socrates \\\hline
3.1.9 & e por ende llama una la la derechura . \textbf{ Et pue tal que assi es puesta comunidat delas mugers } non se signiria aquel bien & propter quod vocata est Iusta . \textbf{ Non ergo supposita communitate uxorum } esset bonum illud \\\hline
3.1.9 & por que sepan ordenar la çibdat \textbf{ assi commo conuiene ala comunidat de los çibdadanos } lpho prueua en el primero libro delas politicas & ut sciant sic ciuitatem ordinare , \textbf{ ut expedit communitati ciuium . } Philosophus 2 Polit’ probat multa mala sequi in ciuitate , \\\hline
3.1.10 & El terçero mal se declara \textbf{ assi ca conmo de la comunidat sobredichͣ delas mugiets } e de los fijos se sigua iniuria et tuerto de los fijos & Tertium malum sic declaratur . \textbf{ Nam supposita praedicta communitate , } sequeretur in curia filiorum \\\hline
3.1.10 & e alos prinçipes de ordenar assi la çibdat \textbf{ por que defendia la comunidat delas fenbras e delas mugeres casadas } e los padres sean çiertos de sus propreos fijos & sic ordinare ciuitatem , \textbf{ ut prohibita communitate foeminarum et uxorum certificentur parentes de propriis filiis . } Esse res communes , \\\hline
3.1.11 & que la çibdat es es si ordenada \textbf{ do es tanta comunidat } de los çibdadanos tienen la çibdat & sic ordinatam esse \textbf{ ubi est tanta communitas ciuium , } reputat eam felicem esse , \\\hline
3.1.11 & e que biuen sin contienda \textbf{ entre los quales se guarda tan grant comunidat . } mas assi commo dize el philosofo & et absque litigio viuere , \textbf{ inter quos tanta communitas obseruatur . } Sed ut dicitur secundo Politicorum \\\hline
3.1.11 & que todos los çibdadanos \textbf{ por la comunidat delas fenbras } e delas mugers creyessen & quod omnes ciues \textbf{ propter communitatem mulierum et uxorum crederent } se esse consanguinitate coniunctos ; \\\hline
3.1.15 & assi commo si fuessen suyas . \textbf{ Mas en las mugers e en los fijos deue ser guardada comunidat } non solamente quanto al amor & ac si essent suae . \textbf{ In uxoribus autem ex filiis debet | reseruari communitas quantum ad amorem : } sed in possessionibus non solum debet \\\hline
3.1.15 & desocͣtes \textbf{ quanto ala comunidat de los çibdadanos } En essa misma manera podemos saluar el su dicho & Saluauimus igitur dictum Socraticum \textbf{ quantum ad communitatem ciuium : } sic etiam saluare possumus dictum eius quantum ad unitatem ciuitatis . \\\hline
3.1.15 & Et pues que assi es \textbf{ assi es poniendo la entençio de socrates dela comunidat delas cosas } e dela vnidat de los çibdadanos & quando ciues se amando et diligendo maxime unirentur . \textbf{ Sic ergo exposita mente Socratis | de communitate rerum } et de unitate ciuium , \\\hline
3.2.27 & Por la qual cosa commo el bien comun sea entendido \textbf{ prinçipalmente de toda la comunidat } assi commo de todo el pueblo o del prinçipe & secundum quas intendimus in bonum illud : \textbf{ quare cum bonum commune principaliter intendatur a tota communitate } ut a toto populo , vel a principante , \\\hline

\end{tabular}
