\begin{tabular}{|p{1cm}|p{6.5cm}|p{6.5cm}|}

\hline
1.1.5 & et agonibus \textbf{ non coronantur fortissimi , } sed agonizantes : & o es aquellas batallas \textbf{ non son coronados los muy fuertes } mas los bien lidiantes \\\hline
1.1.5 & sed agonizantes : \textbf{ qui enim fortissimi sunt , } potentes agonixare , & mas los bien lidiantes \textbf{ ca los que son muy fuertes pueden lidiar . } Enpero si non lidiaren de fech̃o non les es deuida corona ¶ \\\hline
1.1.10 & ( secundum eundem Vegetium ) \textbf{ hoc esse debet principalissimum in intentione Principis , } quod abundet in ciuili potentia , & e todas las gentes . \textbf{ ¶ Por la qual cosa segunt este philosofo vegeçio esto deue ser la cosa mas prinçipal | en la entençion del prinçipe } que abonde en poderio ciuil que es poderio de çibdat e de regno \\\hline
1.2.2 & et difficilis : \textbf{ nam arduitas , et difficultas potissime sunt repugnantia , et prohibentia , } ne possimus consequi bonum , & Ca la fortaleza e la graueza son cosas \textbf{ que nos enbargan mucho | e nos retienen } por que non podamos seguir el bien \\\hline
1.2.12 & Unde 5 Ethic’ dicitur , \textbf{ quod praeclarissima virtutum videtur esse Iustitia : } et neque Hesperus , & Et por ende dize el philosofo \textbf{ en el quinto libro de las ethicas que paresçe que la iustiçia es muy mayor | e mas respladeciente entre todas . } las uirtudes . \\\hline
1.2.12 & Si ergo decet Reges et Principes \textbf{ habere clarissimas virtutes } ex parte ipsius Iustitiae , & Et pues que assi es si conuiene alos Reyes \textbf{ e alos prinçipes de auer | muy claras } uirtudes paresçe de parte dela iustiçia \\\hline
1.2.12 & ex parte ipsius Iustitiae , \textbf{ quae est quaedam clarissima virtus , probari potest , } quod decet eos obseruare Iustitiam . & uirtudes paresçe de parte dela iustiçia \textbf{ que es muy clara uirtud | que se puede prouar } que conuiene alos Reyes \\\hline
1.2.14 & unde ait Philosophus , \textbf{ quod secundum hanc Fortitudinem fortissimi videntur esse } apud illas gentes , & Onde dize el philosofo \textbf{ que segunt esta manera de fortaleza | aquellos son dichos muy fuertes } que quieren gauar honrra entre aquellas gentes . \\\hline
1.2.16 & maxime prouocant alios contra se . \textbf{ Et quia potissime timendum est Regibus et Principibus , } ne furor populi incitetur contra eos , & Por ende mucho mueuen alos pueblos en sanna contra si . \textbf{ Et por que mucho han de temer los Reyes | e los prinçipes } que el pueblo non se leuate en sanna contra ellos \\\hline
1.2.18 & quia per eam homines communicant sua bona , \textbf{ per quam communicationem ab aliis potissime diliguntur : } nam liberales sunt potissime amabiles . & por que por ellas los omes parten los sus bienes \textbf{ por la qual participaçion se departen estri̊madamente de los otros . } Ca los liberales son estrimadamente amables e de amar \\\hline
1.2.18 & per quam communicationem ab aliis potissime diliguntur : \textbf{ nam liberales sunt potissime amabiles . } Quare si maxime decet Reges et Principes , & por la qual participaçion se departen estri̊madamente de los otros . \textbf{ Ca los liberales son estrimadamente amables e de amar } por la qual razon si mucho conuiene alos Reyes \\\hline
1.2.19 & ut circa personas honore dignas . \textbf{ Nam in hoc potissime apparet magnificentia , } quando quis magna facit & que son dignas de honrra . \textbf{ Ca en esto paresçe mayormente la magnificençia } quando alguno faze grandes bienes \\\hline
1.2.21 & opus optimum , \textbf{ et decentissimum , } quam qualiter , & en qual manera faga obra muy buena \textbf{ e muy conuenible que entender en qual manera } e quanta despenssa fara en aquella obra . \\\hline
1.2.27 & Quidam autem sunt irascibiles , \textbf{ omnem iniuriam dissimulantes . } Quorum neutrum bonum est . & Et otros son que en la sanna encubren mucho \textbf{ e asconden la miuria delas quales dos cosas ninguna non es buena } por que enssannar se de qual se quier cosa \\\hline
1.2.31 & secundum suam facultatem sunt vere , \textbf{ et perfecte liberales propinquissimum est , } ut sint magnifici : & Empero si los pobres segunt su poder son uerdaderamente \textbf{ e acabadamente liberales muy cercanos son para ser magnificos } por que si abondassen en los bienes de fuera \\\hline
1.3.3 & Erit magnificus ; \textbf{ quia secundum Philosophum 4 Ethic’ magnificentia potissime habet } esse circa diuina , et communia . & et ahun sera magnifico . \textbf{ Ca segunt el philosofo | en el quarto libro delas ethicas } la magnificençia \\\hline
1.3.3 & ex amore sumit originem ; \textbf{ potissimum ergo in intentione cuiuslibet } esse debet quid amandum . & e nasçençia del amor \textbf{ Et pues que assi es la prinçipal entençion de cada vno deue ser } que cosa ha de amar \\\hline
1.3.10 & pollet bono illo . \textbf{ Et potissime est inuidia circa similes , } ut figuli inuident figulis , & a quien ha inuidia pesale de aquel bien . \textbf{ Et mayormente la inuidia es | entre los que se semeian } assi conmolos olleros \\\hline
1.4.2 & ergo quia aliorum debent esse regula et mensura , \textbf{ potissime eos decet moderatos esse . } Enumeratis moribus iuuenum , & Et pues que assi es que los Reyes deuen ser regla \textbf{ e mesura de todos los otros mucho les conuiene aellos de ser mas mesurados que los otros . } ontadas las costunbres de los mançebos \\\hline
2.1.1 & esse sociale animal , \textbf{ potissime innititur huic rationi , videlicet , } quod quis sermo est ad alterum & por las quales praeua el omne es naturalmente aial aconpannable \textbf{ prinçipalmente se firma en esta razon la qual es } que por que la fabla es a otro \\\hline
2.1.13 & tamen cum tradenda est aliqua nuptui , \textbf{ potissime inquirendum est , } utrum polleat temperantia , & segunt la manera que les conuiene . \textbf{ Enpero quando la fenbra es de dar a algun marido mayormente deuemos tener } mientessi resplandesçe por tenprança \\\hline
2.1.15 & et dirigitur \textbf{ a sapientissimos artifice } ut a Deo , nihil agit superfluum , & por que es mouida \textbf{ e gada de maestro muy sabio } assi commo de dios \\\hline
2.1.15 & et quicquid natura praeparatur , \textbf{ oportet ordinatissimum esse : } quia ille naturam dirigit , & por la natura \textbf{ conuiene que sea muy ordenado . } Ca aquel gnia la natura de que viene todo ordenamiento \\\hline
2.1.16 & magis vitanda sunt in Princibus et Regibus quam in aliis , \textbf{ potissime non decet } eos uti coniugio in nimia iuuentute . & e en los prinçipes \textbf{ que en los otros | Et por ende mayormente conuiene aellos } de non vsar de casamiento en grand moçedat . \\\hline
2.1.21 & ut plurimum appetant videri pulchrae , \textbf{ potissime delinquunt } circa ornatum corporis ; & desseen desseruistas fermosas \textbf{ mayormente pecan en el conponimiento de los cuerpos . } Por la qual cosa conuiene alos uarones \\\hline
2.2.13 & extendunt pedes et crura , \textbf{ vel mouent nimis spissim brachia , } vel erigunt humeros , & e las prinas o mueuen los braços \textbf{ mucho a menudo o leuna tan los onbros } o fazen aquellas cosas \\\hline
2.2.16 & non tamen habent perfectum rationis usum , \textbf{ potissime videtur esse curandum } circa ipsos & Enpero non han vso acabado de razon e de entendimiento . \textbf{ Por ende paresçe que mayormente deuemos auer cuydado çerca los mocos } por que ayan ordenada uoluntad . \\\hline
2.2.17 & Nam si volunt viuere vita politica et militari , \textbf{ potissime studere debent } in moralibus scientiis : & e de çibdat e de caualłia \textbf{ deuen estudiar mayormente enlas sciençias morales } por que por ellas podran saber \\\hline
2.2.19 & ne prosequatur illicita maximum fraenum foeminarum , \textbf{ et potissime puellarum , } ne prorumpant in turpia , & Et pues que assi es muy grant freno delas fenbras \textbf{ e mayormente delas moças es la uerguença } por que non puedan sallir a fazer cosas torpes . \\\hline
2.2.19 & et multo magis nobiles , \textbf{ et potissime Reges et Principes } tanto maiorem curam circa proprias filias adhibere debent , & e mucho mas alos nobles \textbf{ e mayoͬmente alos Reyes } e alos prinçipes de auer tanto mayor cuydado çerca las sus fiias propraas \\\hline
2.2.21 & in locutionem incautam , \textbf{ haec videtur esse potissima , } ut nullum sermonem proferat , & para fablar la cosa \textbf{ que non ha penssada esta paresçe muy grande } que ninguno non diga \\\hline
2.2.21 & nam cum foeminae , \textbf{ et potissime puellae deficiant a rationis usu , } nisi sint modo debito taciturnae , & en manera que les conuiene \textbf{ e si non examinaten con grand cordura las palabras } que han de dezer \\\hline
2.2.21 & ne in verba litigiosa prorumpant . \textbf{ A verbis autem litigiosis potissime foeminae sibi cauere debent ; } quia postquam litigare incipiunt & de fablar palabras de contienda \textbf{ ca mayormente las mugers se deuen guardar de palabras de baraia e de pelea } por que despues que comiençan a pelear non saben \\\hline
2.3.3 & qui debent esse nobiles et praeclari , \textbf{ potissime decet esse magnificos . } Alii enim moderatas possessiones habentes , & que anings de los otros nobles \textbf{ ca ellos conuiene de ser nobles prinçipalmente | e magnificos en todas sus cosas . } ca si los otros nobles \\\hline
2.3.5 & eo enim ipso quod homo respectu corporalium \textbf{ et sensibilium est creaturarum dignissima , } habet naturale dominium super ipsa : & mas digna de todas las ceraturas \textbf{ en | conparaconn de las cosas corporales e sensibles } por ende han señorio natural sobrellas . \\\hline
2.3.12 & habuisse massaritias multas . \textbf{ Non obstante enim quod terrae fertilissimae dominabatur , } ubi victualia modici precii existebant ; nihilominus quasi tamen semper ex propriis alimenta carnium volebat assumere , & maguer que fuesse señor de tierra muy abondosa \textbf{ en la qual auya muchͣs uiandas e de grand mercado . } Enpero con todo esto quaria tomar gouierno de carnes \\\hline
3.1.1 & maxime tamen ordinatur \textbf{ ad ipsum communitas principalissima : } huius autem est communitas ciuitatis , & enpero mayormente es ordenada aquel bien la comunidat \textbf{ que es mas prinçipal } e esta tal es la comunidat dela çibdat \\\hline
3.1.1 & quae respectu communitatis domus , \textbf{ et vici principalissima existit . } Quare si communitas domestica ordinatur ad bonum & e esta tal es la comunidat dela çibdat \textbf{ la qual en conparacion dela comunidat dela casa e del barrio es mas prinçipal } por la qual razon \\\hline
3.1.1 & et ad hoc communitas ciuitatis , \textbf{ quae est principalissima communitas respectu vici , } et domus , & e avn la comunidat dela çibdat \textbf{ que es much mas prinçipal | que la comunidat del barrio } nin dela casa much mas es ordenada abien que todas las otras \\\hline
3.1.1 & quod si communitatem omnem gratia alicuius boni dicimus constitutam , \textbf{ maxime autem principalissimam omnium , } et eam quae est omnium maxime principalis , & que toda comunidat es establesçida \textbf{ por grande algun bien | mayormente aquella que es mas prinçipal de todas } e aquella que es sobre todas mas prinçipal . \\\hline
3.1.1 & et eam quae est omnium maxime principalis , \textbf{ et omnes alias circumplectens , potissime gratia boni constitutam esse contingit : } haec autem est communitas politica , & Et toda cosa que encieira en ssi las o triscosas \textbf{ couiene que sea establesçida prinçipalmente | por grande algun bien } e esta es comunidat politica \\\hline
3.1.2 & et aliud virtuose viuere . \textbf{ Nam esse latissimum est , } ut dicitur in libro de Causis : & e otra cosa es beuir uirtuosamente \textbf{ por que el seres cosa muy general | e muy ancha } assi commo es dicho en el libro de causis \\\hline
3.1.7 & arguitur esse summe bonus . \textbf{ Videtur ergo ciuitas esse potissime bona , } si sit potissime una ; & que dios es muy bueno \textbf{ e por ende paresçe que la çibdat es muy | buenasi fuere muy vna . } Et pues que assi es quanto mas se allega a vnidat \\\hline
3.1.7 & Videtur ergo ciuitas esse potissime bona , \textbf{ si sit potissime una ; } igitur quanto plus ad unitatem procedit , & buenasi fuere muy vna . \textbf{ Et pues que assi es quanto mas se allega a vnidat } tanto \\\hline
3.1.15 & ut nobiles : \textbf{ hi videlicet nobiles potissime debent defendere patriam , } et eorum maxime est vacare & assi commo los nobles \textbf{ Por ende conuiene que estos nobles de una | prinçipalmente defender la tierra entre los otros } e a ellos parte nesçe mayormente de entender çerca la sabiduria delas armas . \\\hline
3.1.16 & ex his quae videbat in politiis aliis , \textbf{ statuit potissime curandum esse de possessionibus ciuium , } volens eos aequatas possessiones habere . & que veya en las otrå sçibdades . \textbf{ Et por ende establesçio | que much deuia ser tomado grant cuydado delas possessiones de los çibdadanos } e quaria que los çibdadanos ouiessen las possessiones eguales \\\hline
3.2.4 & cum ipse pluries dicat in eisdem politicis , \textbf{ regnum esse dignissimum principatum : } inter principatus enim rectos , & en esse mismo libro delas politicas \textbf{ que el regno es prinçipado muy digno } por que entre los prinçipados derechs el prinçipado de vno \\\hline
3.2.4 & Censendum est igitur , \textbf{ regnum esse dignissimum principatum , } et secundum rectum dominium melius est dominari unum , & Et pues̃ que assi es deuemos otorgar \textbf{ que el regno es prinçipado muy digno } e segut derech \\\hline
3.2.7 & Secunda , ex eo quod est maxime innaturale . \textbf{ Tertia , ex eo quod est efficacissimum ad nocendum . } Quarta , ex eo quod impedire habet & muchodes natural ¶ \textbf{ La terçera se toma | por razon que tal prinçipado es muy afincado por enpesçer . } la quarta por razon que tal prinçipado ha de enbargar \\\hline
3.2.7 & quod sicut Regnum est optima \textbf{ et dignissima politia , } sic tyrannis est pessima : & Et esta razon tanne el philosofo çerca el comienço del quarto libro delas politicas . \textbf{ do dize que assi conmo el regno es muy buena et muy digna poliçia . } assi la tirania es muy mala \\\hline
3.2.7 & ex eo quod talis principatus \textbf{ est efficacissimus ad nocendum . } Nam sicut principatus Regis & ¶La terçera razon se toma \textbf{ por que tal prinçipado es muy afincado para enpeesçer . } casi commo el prinçipado del Rey \\\hline
3.2.7 & eo quod sit maxime unitus , \textbf{ est efficacissimus ad proficiendum : } sic tyrannis efficacissima ad nocendum . & casi commo el prinçipado del Rey \textbf{ por que es muy vno es muy afincado para aprouechar . } Assi la tirania es muy afincada para enpees çer \\\hline
3.2.7 & est efficacissimus ad proficiendum : \textbf{ sic tyrannis efficacissima ad nocendum . } Monarchia enim quia ibi dominatur unus , & por que es muy vno es muy afincado para aprouechar . \textbf{ Assi la tirania es muy afincada para enpees çer } ca el senñorio de vno \\\hline
3.2.8 & et Rex regum , \textbf{ a quo rectissime regitur } uniuersa tota natura : & e Rey de los Reyes \textbf{ por el qual es gouernada muy derechamente toda la natura del mundo . } Por ende del gouernamiento \\\hline
3.2.12 & et ligari : \textbf{ et supra caput eius acutissimum gladium } pendentem tenuissimo filo apponi fecit : & e fizola tar \textbf{ e fizol colgar una espada sobre su cabesça muy aguda de vn filo muy delgado } e fizo \\\hline
3.2.12 & et supra caput eius acutissimum gladium \textbf{ pendentem tenuissimo filo apponi fecit : } circa ipsum quosdam homines cum ballistis , sagittis appositis , & e fizola tar \textbf{ e fizol colgar una espada sobre su cabesça muy aguda de vn filo muy delgado } e fizo \\\hline
3.2.15 & et diuturnam experientiam non accepit . \textbf{ Ideo potissime obseruandum est , } ne repente constituatur aliquis in maximo principatu . & e luenga prueua \textbf{ e por ende muches de guardar } que adesora non sea ninguno puesto en muy grant senorio . \\\hline
3.2.15 & quam in duce exercitus . \textbf{ Nam in duce exercitus requiritur potissime experientia , } sed custos ciuitatis & que en el caudiello dela fazienda \textbf{ Ca en el caudiello dela fazienda es mucho menester espiençia e prueua . } mas el guardador de la çibdat \\\hline
3.2.19 & ut eligens optimum modum principandi , \textbf{ ferat leges iustissimas , } secundum quas saluari habet principatus ille . & por que escogiendo la meior manera de prinçipar o de \textbf{ enssennorear | ponga leyes muy derechos . } segunt las quales aquel prinçipado se ha de saluar . \\\hline
3.2.20 & per legem omnia determinari , \textbf{ et quam paucissima arbitrio iudicum committere . } Secunda via sic ostenditur , & por leyes \textbf{ e muy pocas sean acomnedadas al aluedrio de los uiezes } ¶ \\\hline
3.2.20 & omnia lege determinare , \textbf{ et quam paucissima arbitrio iudicum committere . } Has autem tres rationes tangit Philosophus 1 Rhetoricorum dicens & por las leyes \textbf{ e muy pocas cosas sean dexadas en poder | e en aluedrio de los iuezes . } Et estas tres cosas tanne elpho \\\hline
3.2.20 & quaecunque possibile est determinare : \textbf{ et quam paucissima committere iudicantibus . } Primum quidem quia facilius est habere & quanto pueden ser todas las cosas \textbf{ et que muy pocas cosas sean dexadas alos iuezes ¶ } Lo primero por que mas ligeramente pueden auer los omes vn sabio o pocos que muchos . \\\hline
3.2.28 & et quae mala sunt prohibenda , \textbf{ et quae dissimulanda , } et permittenda . & Et quales malas son de vedar e poner so pena \textbf{ et quales son de desseneiar e de sofrir } o quales son de conssentir \\\hline
3.2.30 & Nam legibus humanis aliquando \textbf{ dissimulantur minora mala } ut vitentur maiora : & humanal non defiende todas las cobdiçias de dentro a as avn non defiende todos los pecados de fuera del coraçon . \textbf{ Ca en las leyes humanales algunas uezes se consienten los menores males } por que se escusen los mayores . \\\hline
3.2.32 & ad quod ordinantur : \textbf{ ut potissime scimus } quid sit domus , & aque son ordenadas . \textbf{ assi commo prinçipalmente sabemos } que cosa es casa \\\hline
3.2.32 & aliquo modo ciuitas constituta , \textbf{ potissime tamen constituta est } propter viuere eligibiliter et victuose . & maguera que por todas las cosas sobredichas sea fecho la çibdat en alguna manera . \textbf{ Enpero prinçipalmente fue ella establesçida } por que biuiessen los omes \\\hline
3.2.36 & quod ut Reges et Principes communiter amentur a populo , \textbf{ tria potissime in se habere debent . } Primo quidem esse debent benefici , & comunalmente sean amados del pueblo \textbf{ deuen auer en ssi tres cosas prinçipalmente . } La primera que sean bien fechores e liberales e francos \\\hline
3.2.36 & ut timeantur ab eis . \textbf{ Potissime autem timentur potentes } ( ut patet in 2 Rhet’ ) & Mas la cosa del mundo \textbf{ por que mas son temidos los poderosos } assi commo dize el philosofo \\\hline
3.3.3 & et latitudo pectoris . \textbf{ Videmus enim leones animalium fortissimos habere magna brachia , } et latum pectus . & son grandeza de los mienbros e anchura de los pechos . \textbf{ Ca veemos que los leones | que son mas fuertes que todas las otras animalas } por que han grandes braços e anchos pechos . \\\hline
3.3.4 & Nam , ut dicitur 3 Ethic’ \textbf{ apud illos sunt viri fortissimi , } apud quos honorantur fortes . & en el tercero libro de las Ethicas \textbf{ entre aquellos son los varones muy fuertes entre los quales los fuertes son muy honrrados . } Mas entre todas las cosas \\\hline
3.3.8 & habere formam quadrilateram oblongam . \textbf{ Attamen quia figura circularis est capacissima , } est elegibilius facere munitiones & deuen ser quadradas e luengas . \textbf{ Enpero por que la forma redonda conprehende | mas que las otras } por ende es \\\hline
3.3.11 & quae requiruntur ad bellum . \textbf{ Mors est quid terribilissimum , } et finis omnium terribilium , & que son menester para la batalla . \textbf{ l lA muerte es cosa muy espantable } e fin de todas cosas \\\hline
3.3.11 & et in qualibet acie \textbf{ habere aliquos equites fidelissimos et strenuissimos , } habentes equos veloces et fortes ; & que deue el señor de la hueste en cada conpaña \textbf{ e en cada vna az auer vnos caualleros muy fieles | e muy estremados } que ayan cauallos muy ligeros \\\hline
3.3.11 & Itaque cum pericula visa minus noceant , \textbf{ per velocissimos equites sunt detegendae insidiae , } ne exercitus circa aliquam partem ex improuiso patiatur molestias . & Et por ende por que los periglos que son ante vistos menos enpeesçen . \textbf{ por caualleros muy ligeros son de descobrir las çeladas } por que la hueste non aya de resçebir a desora en alguna parte algunos daños . \\\hline
3.3.13 & aduersarium sauciat antequam videat . \textbf{ Unde hoc genere percutiendi potissime usi sunt Romani . } Deridebant enim Romani milites , & o a su enemigo mata ante que lo vea . \textbf{ Et por esso los romanos vsaron prinçipalmente de esta materia de ferir . } Ca los romanos escarnesçien de todos los caualleros \\\hline
3.3.13 & Inter cetera enim in bellis est \textbf{ hoc potissime attendendum : } ut pugnantes absque nimia fatigatione sui possint & que son las batalla \textbf{ mayormente es de penssar esto | que los lidiadores } sin grand canssamiento de sus mienbros puedan ferir mucho a sus enemigos e a sus contrarios . \\\hline
3.3.18 & munitiones aliquas obsessas \textbf{ super lapides fortissimos esse constructas , } vel esse aquis circumdatas , & e assi podran ganar aquellas fortalezas . \textbf{ m muchas uegadas contesçe que algunas fortalezas çercadas son fundadas sobre pennas muy fuertes } o son cercadas de agua \\\hline
3.3.18 & vel esse aquis circumdatas , \textbf{ vel habere profundissimas foueas , } vel aliquo alio modo esse munitas : & o son cercadas de agua \textbf{ o han carcauas muy fondas } o son \\\hline
3.3.19 & ideo appellatur aries , \textbf{ quia ratione ferri ibi appositi durissimam habet } frontem ad percutiendum . & Ca por razon del fierro \textbf{ que ponen y . | ha muy fuerte et muy . } dura fruente para ferir \\\hline
3.3.21 & ( ut ait Vegetius ) \textbf{ illae pudicissimae foeminae } cum maritis conuiuere deformato capite , & Ca dize vegeçio \textbf{ que mas quisieron aquellas buenas mugeres muy castas beuir con sus maridos trasquiladas } que non yr con sus enemigos con cabellos . \\\hline
3.3.22 & Contra hoc autem constituitur \textbf{ quoddam ferrum curuum dentatum dentibus fortissimis , } et acutis , et ligatum funibus , & et contra este \textbf{ carnero se puede fazer vn fierro | coruo dentado de dientes muy fuertes e muy agudos } e atado con fuertes cuerdas \\\hline

\end{tabular}
