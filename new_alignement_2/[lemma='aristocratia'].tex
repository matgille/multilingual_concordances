\begin{tabular}{|p{1cm}|p{6.5cm}|p{6.5cm}|}

\hline
3.2.2 & quorum tres sunt boni , et tres sunt mali . \textbf{ Nam regnum aristocratia , } et politia sunt principatus boni : tyrannides , & de los quales tro son buenos e los trsson malos . \textbf{ ca el regno e la aristo carçia } que quiere dezer sennorio de buenos e la poliçia que quiere dezer pueblo bien enssenoreante son bueons prinçipados . \\\hline
3.2.2 & tunc illi pauci vel sunt virtuosi et boni , et intendunt commune bonum ; \textbf{ et tunc talis principatus dicitur Aristocratia , } quod idem est quod principatus bonorum et virtuosorum . Inde autem venit ut maiores in populo , & aquellos pocos o son uirtuosos o buenos que entienden al bien comun \textbf{ e tal prinçipado es dicħa ristrocaçia } que quiere dezer prinçipado de buenos omes e uir̉tuosos e dende vienen \\\hline
3.2.12 & quia boni et virtuosi , est rectus principatus , \textbf{ et vocatur aristocratia siue principatus bonorum . Si vero dominentur non quia boni , } sed quia diuites , est peruersus et vocatur oligarchia . Sed si dominatur totus populus et intendat bonum omnium tam insignium quam aliorum , est principatus rectus , et vocatur regimen populi . & e es llamado anstrocraçia que quiere dezer señorio de buenos . \textbf{ Mas si enssennorear en pocos | non por que son buenos } mas por que son ricos es llamado obligarçia que quiere dezer señorio tuerto . \\\hline

\end{tabular}
