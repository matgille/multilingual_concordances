\begin{tabular}{|p{1cm}|p{6.5cm}|p{6.5cm}|}

\hline
1.1.2 & sic in operabilibus perfectam industriam praecedit astutia imperfecta . \textbf{ Cum ergo non requiratur tanta industria ad regendum seipsum , } quanta requiritur in gubernatione familiae : & ante que la sabiduria conplida donde se sigue \textbf{ que commo non sea demandada | tanta sabiduria ya gouerna mj̊ } asi mismo qunata es demandada \\\hline
1.1.2 & videlicet , ex finibus , habitibus , passionibus , et moribus . \textbf{ Nam cum finis sit operationum nostrarum principium , } secundum quod quis sibi alium et alium finem praestituit , & Lo terçero de parte delas pasiones ¶ \textbf{ Lo quarto de parte delas costunbres ¶ | Ca commo la fin sea comjenço delas nr̃as obras } segunt que cada vno ordena asy mjsmo a \\\hline
1.1.3 & opus istud suscepimus gratia eruditionis Principum : \textbf{ cum nunquam quis plene erudiatur , } nisi sit beniuolus , docilis , et attentus : & et commo nunca el prinçipen ̃j otro \textbf{ njnguno conplidamente puenda ser enssennado | sy non fueᷤ begniuolo } e uolenteroso para a prinder doçible e engennoso \\\hline
1.1.3 & et includunt bonitatem utilium bonorum . \textbf{ Cum ergo in hoc libro intendatur , } quomodo maiestas regia fiat virtuosa , & e ençierran en sy bondat de tondos los bienes prouechosos ¶ \textbf{ Pues que asy es commo en este libro ente damos demostrͣ } commo la magesad Real aya de ser uertuosa \\\hline
1.1.5 & consequitur suam perfectionem et felicitatem . \textbf{ Cum ergo nunquam contingat recte agere , } ut requirit consecutio finis , & e su bien andança acabada¶ \textbf{ pues que asy es com̃ nunca pueda omne bien | e derechamente obrar asy } commo demanda la fin que ha de seguir . \\\hline
1.1.6 & non in bonis corporis . \textbf{ Cum enim corpus ordinetur ad animam , } sicut materia ad formam , & e non en los bienes del cuerpo . \textbf{ Ca el | cuerpoes ordenado al alma } bien commo la materia ala forma . \\\hline
1.1.6 & quam animae . \textbf{ Cum enim corpus ordinetur ad animam , } et non econuerso : & son bienes del cuerpo \textbf{ mas quedtalma . ¶ pues que asi es commo el cuerpo sea ordenado al alma } e non el alma al cuerpo \\\hline
1.1.7 & tum quia sunt diuitiae \textbf{ ex institutione Hominum , tum quia cum sint corporalia , } ipsi indigentiae corporali & Lo segundo que por que estas Riquezas son riquezas \textbf{ por ordenamiento e estableçemiento | delons omes } e non en otra gnisa¶ \\\hline
1.1.7 & de leui patet . \textbf{ Nam cum felicitas sit bonum optimum , } in optimo nostro quaeri debet . & esto ligeramente lo podemos prouar . \textbf{ ¶ Por que commo la feliçidat | e la bien andança sea muy gñdvien } e deua ser puesta por nos \\\hline
1.1.7 & in optimo nostro quaeri debet . \textbf{ Cum ergo anima sit potior corpore , } felicitas non est ponenda & en el mayor bien que nos podemos dessear . \textbf{ siguese que commo el alma sea meior | que el cuerpo la feliçidat } e la bien andança non es de poner en tales riquezas \\\hline
1.1.7 & nihil magnum attentabit . \textbf{ Immo ( cum ille sit Magnanimus , } cui nihil corporale est magnum , & Ca temiendo deꝑder los des e las riquezas nunca acometra grandes cosas \textbf{ Et la razon es esta | que aquel es magnamimo } e de grant coraçon \\\hline
1.1.7 & ei aliquod bonum proprium . \textbf{ Cum ergo finis maxime diligatur , } ponens suam felicitatem in numismate , & e la bien andança de los omes \textbf{ deua ser muy desseada | a aquel que pone las un feliçidat } e la su bien andança en las riquezas \\\hline
1.1.7 & sed Tyrannus , \textbf{ cum non intendat principaliter bonum publicum , sed priuatum . } Tertio hoc posito sequitur & ¶Lo terçero se declara \textbf{ asi que poniendo el prinçipe la su feliçidat | e la su bien andança en las riquezas corporales } dende se sigue \\\hline
1.1.8 & Honor ergo habet rationem boni extrinseci , \textbf{ cum sit reuerentia exhibita } per quaedam exteriora signa . & que esta de fuera \textbf{ por que es reuerençia fecha } por señales mostradas de fuera¶ \\\hline
1.1.8 & Nam si Princeps suam felicitatem in honoribus ponat , \textbf{ cum sufficiat ad hoc } quod quis honoretur , & si pusiere la su bien andança en honrras \textbf{ conmoabaste acanda vno } para que sea honrrado \\\hline
1.1.8 & quia ex hoc efficietur periclitator Populi , et praesumptuosus : \textbf{ nam cum finis maxime diligatur , } si Princeps suam felicitatem in honoribus ponat , & e pornia los pueblos en peligro¶ \textbf{ Ca commo cada vn omne mucho ame la su fin } en que pone la su bien andança \\\hline
1.1.9 & esse magnae latitudinis , \textbf{ cum ad diuersas partes diuulgare possit : } et magnae diuturnitatis , & por que ella es de grant anchura \textbf{ e se estiende a muchas partes } Et otrosi por que es de grant durança det pon \\\hline
1.1.9 & Rursus circa bonitatem nostram notitia Dei non fallit , \textbf{ cum scientia sua falli non possit . } Amplius bonitatem , & conosçimientode dios no puede ser engannado en lanr̃a bondat . \textbf{ Ca lascian de dios non puede resçebir enganno . | Et ahun dezimos mas adelante } que dios mas claramente vee \\\hline
1.1.10 & Violentia autem perpetuitatem nescit . \textbf{ Cum igitur violenta non diu durent , } talis principatus diu durare non potest . & Mas ninguna cosa que es por fuerça \textbf{ non puede mucho durar¶ | pues que assi es que las cosas forçadas non pueden mucho durar } tal prinçipado non puede mucho durar \\\hline
1.1.10 & quia tale dominium \textbf{ cum sit violentum , } et contra naturam , & Mas es de poner en aquello \textbf{ que sienpre ha de durar ¶ } La segunda razon se declara \\\hline
1.1.10 & per coactionem , et violentiam . \textbf{ Cum ergo Principatus se extendant adinuicem , } secundum eos quibus aliquis principatus , & por costrinimiento e por fuerça¶ \textbf{ pues que assi es commo los prinçipados | e los sennorios se estiendan } segunt aquellos a quien \\\hline
1.1.10 & et ciuilem potentiam , \textbf{ cum non sit liberorum , } sed seruorum , & e mas digno que ensseñorear a los sieruos . \textbf{ Et por que el señorio por fuerça e por poderio çiuil commo non sea delons libres } mas de los sieruos non puede ser muy bueno nin digno¶ \\\hline
1.1.10 & ut plurimum nocumentum . \textbf{ Nam cum felicitas sit finis omnium operatorum , } quilibet totam vitam suam , & Ca commo la feliçidat \textbf{ e la bien andança | sea fin de todas las nuestras obras } ¶ Cada vno porna toda su uida \\\hline
1.1.11 & siue propter conseruationem propriae personae : \textbf{ nam , cum ipse sit caput Regni , } ex defectu eius posset & para conseruaçion e guarda dela su persona . \textbf{ Ca commo el Rey sea cabesça de su Regno } por fallesçimiento dela su persona \\\hline
1.1.13 & nisi ex amore , \textbf{ cum semper amor sit ad similes , et conformes , } oportet esse similem , & por amor qual ha . \textbf{ Et commo el amor sienpre sean los semeiables e acordables con el . } Conuiene que aquel que es para de ser \\\hline
1.1.13 & si non transgrediantur , \textbf{ cum possint transgredi , } maioris meriti esse videntur . & si non trispassaren los mandamientos de dios \textbf{ conmolos podiessen } trispassar son de mayor meresçimiento . \\\hline
1.1.13 & sed etiam totum regnum . \textbf{ Cum ergo magnae virtuti debeatur magna merces , } magnum erit meritum & mas ahun a todo el regno . \textbf{ Et pues que assi es commo grant uirtud deua auer grant merçed } e gm̃t gualardon gerad sera el meresçimiento e el gualardon de los Reyes . \\\hline
1.1.13 & si dirigat personam aliquam singularem . \textbf{ Cum ergo bonum gentis sit diuinius , } quam bonum aliquod singulare , & si ama alguna ꝑson a singular . \textbf{ pues que assi es commo el bien comun | e el bien dela gente sea mas diuinal } que nigun bien singular paresçe quela materia \\\hline
1.2.1 & per virtuosos ad agendum bene : \textbf{ cum ergo natura sit determinata ad unum , } et potentiae naturales sufficienter determinentur ad agendum , & e uirtudes se apareian a bien obrar . \textbf{ ¶ Et pues que assi es commo la natura sea determimada a vna cosa } e los poderios naturales sean determinandos conplidamente para obrar \\\hline
1.2.2 & Appetitus autem sensitiuus duplex est . \textbf{ Nam cum animalia sint supra inanimata , } si natura elementis & Mas el appetito senssitiuo es en dos maneras \textbf{ ca commo las aina las sean sobre las cosas | que non han alma } si la natura de los helementos dio a todas las cosas \\\hline
1.2.2 & et malum inquantum habent rationem difficilis , et ardui . \textbf{ Nam cum bonum secundum se dicat prosequendum , } malum vero quid fugiendum : & en quanto han razon de cosa guaue e fuerte . \textbf{ Ca commo el bien | por si diga tal cosa } que deue el omne seguir \\\hline
1.2.3 & Numerus autem earum sic potest accipi . \textbf{ Nam cum subiectum virtutis sit , } vel intellectus , vel voluntas , & assi se puede tomar . \textbf{ Ca commo el subiecto delas uirtudes sea o el entendimiento o la uoluntad o el appetito senssitiuo . } toda uirtud moral o es en el entendimiento \\\hline
1.2.5 & per quam de ipsis agibilibus rectas rationes faciamus . \textbf{ Rursus cum contingat operari recte et non recte , } sic ut est dare virtutem , & que fazemos fagamos razones derechas ¶ \textbf{ Otrosi commo contesca de obrar derechamente | e non derechamente } assi commo auemos a dar uirtud . \\\hline
1.2.5 & per quas modificentur in ipsis passionibus . \textbf{ Cum ergo passiones quaedam impellant nos ad malum , } ut passiones concupiscibiles , & por las quales seamos tenprados e reglados en aquellas passiones ¶ \textbf{ Et pues que assi es commo algunas delas passiones nos mueuen a mal } assi commo las passiones dela cobdiçia \\\hline
1.2.5 & principalior omnibus aliis , \textbf{ cum sit directiua omnium aliarum , } et iustitia sit principalior & ¶Otrosi por que la prudençia es mas prinçipal que todas las otras \textbf{ por que es endereçadora e regladora de todas las otras ¶ } Et la iustiçia en pos ella es mas prinçipal que la fortaleza e la tenperança . \\\hline
1.2.5 & et quia fortitudo est principalior quam temperantia , \textbf{ cum fortitudo magis ordinetur ad bonum gentis , } et ad bonum commune quam ipsa temperantia : & en que prinçipalmente se trabaia lanr̃auida¶ Et otrosi por que semeiablemente la fortaleza es mas prinçipal que la tenꝑança . \textbf{ por que la fortaleza es mas ordenada al bien dela gente } e al bien comun que la tenperança . \\\hline
1.2.6 & circa quam versatur . \textbf{ Cum enim Prudentia sit circa agibilia , } et agibilia sint singularia , & ala materia en que obra . \textbf{ Ca commo la pradençia aya de ser en las obras . } Et las obras ayan de ser \\\hline
1.2.12 & ut possit ipsas leges dirigere : \textbf{ cum aliquo casu leges obseruari non debeant , } ut infra patebit . & e de tan grant egualdat por que pueda enderesçar e egualar las leyes . \textbf{ Ca algun caso ay en que se non deuen guardar las leyes } assi commo paresçra adelante . \\\hline
1.2.12 & nihil regulatum erit : \textbf{ cum omnia per regulam regulentur . } Sic si Reges sint iniusti , & Ca çierta cosa es \textbf{ que por la regla se reglan | e se egualan todas las cosas . } Et en essa misma gnisa \\\hline
1.2.12 & ut sint Reges . \textbf{ Cum enim deceat regulam esse rectam et aequalem , } Rex quia est quaedam animata lex , & enpero non son dignos de seer Reyes . \textbf{ Ca commo conuenga ala regla de ser derecha } e egual e el Rey sea vna ley animada e vna regla . \\\hline
1.2.13 & Non enim sic per fugam vitare possumus aegritudines : \textbf{ quia cum aegritudo sit aliquid in nobis existens , } per fugam eam vitare non possumus . & assi por foyr escapar las enfermedades \textbf{ por que la enfermedat es alguna cosa | que esta en nos } e por foyr non la podemos escusar . \\\hline
1.2.13 & sicut pericula belli . \textbf{ Cum ergo difficilius sit durare , et sustinere pericula illa } quae per fugam vitare possumus , & assi escusar commo los periglos delas batallas . \textbf{ Et pues que assi es commo sea mas | guaue cosa de endurar } e de sufrir aquellos periglos que podemos escusar \\\hline
1.2.13 & quilibet fugit , sicut naturaliter sequitur delectabilia . \textbf{ Cum ergo naturaliter tristia fugiamus , } difficile est reprimere timores , & assi commo naturalmente sigue las cosas delectables . \textbf{ Et pues que assi es commo nos natural mente fuyamos dela tristeza } graue cosa es de repmir los temores \\\hline
1.2.14 & quam prima : \textbf{ ut cum aliquis non ut vitet opprobria , } vel ut consequatur honores : & que la primera alsi commo \textbf{ quando alguno non | por esquiuar denuestos } o por gauar honrras \\\hline
1.2.16 & quam per intemperantiam : \textbf{ cum hoc sit magis voluntarium , } quam sit illud . & que non por \textbf{ destenpranca por que esto es mas de uoluntad } que aquello otro ¶ \\\hline
1.2.16 & Valde est ergo increpandus carens tempesantia , \textbf{ cum eam sine periculo possit acquirere : } non autem adeo increpandus est & que non ha \textbf{ tenpranca | commo puede ganar la tenpranca sin ningun periglo . } Mas non es tanto de denostas el \\\hline
1.2.17 & Quinto hoc idem patet : \textbf{ quia cum liberales maxime amentur , } circa illud maxime consistit liberalitas , & Lo quinto se puede prouar esso mismo \textbf{ por que los liberales | e los francos son mas amados sobre todos los otros . } Et por ende la franqueza esta prinçipalmente en aquello \\\hline
1.2.18 & aliquid ociosum esse debet . \textbf{ Quare cum natura humana modicis contenta sit , } quia uni personae modica sufficiunt : si una aliqua persona multitudine diuitiarum superabundat , & non deue ser ninguna cosa ociosa nin baldia . \textbf{ por la qual razon commo la naturaleza de los omes se tenga | por pagada de pocas cosas } por que a cada vna delas personas pocas cosas le abastan \\\hline
1.2.18 & et cuius gratia debet . \textbf{ Quare cum prodigus non sit amator pecuniae , } sicut nec liberalis , & nin por la razon que deue . \textbf{ por la qual razon commo el gastador non sea amador de los } desbien commo el libal non lo es de ligero se puede fazer \\\hline
1.2.18 & de leui \textbf{ quis cum sit prodigus , } fieri poterit liberalis . & desbien commo el libal non lo es de ligero se puede fazer \textbf{ qual quier gastador liberal e franco ¶ } pues que assi es si es conueinble al Rey \\\hline
1.2.18 & quae continet . \textbf{ Cum ergo tanto deceat fontem habere os largius , } quanto ex eo plures participare debent : & Ca ha . manera daua so ancho e largo e da conplidamente lo que tiene \textbf{ ¶pues que assi es conmo tanto conuenga ala fuente auer la boca | mas ancha } quanto della deuen \\\hline
1.2.19 & et proportionati diuitibus . \textbf{ Cum ergo liberalitas non respiciat sumptus secundum se , } sed ut proportionantur facultatibus , & e conuenibles alos ricos \textbf{ ¶Pues que assi es commo la libalidat | non cate alas espenssas } segunt \\\hline
1.2.20 & Sexta est , \textbf{ quia cum paruificus nihil faciat , } videtur tamen ei quod semper agat maiora , & La sexta propiendat es \textbf{ que quando el parufico faze alguna cosa } aparesçe ael \\\hline
1.2.20 & et circa seipsum . \textbf{ Cum ergo Rex sit caput regni , } et sit persona honorabilis , reuerenda , et publica , & e erca si mismo \textbf{ Et pues que assi es commo el Rey sea } ca besça del regno e leaꝑlona honrrada \\\hline
1.2.21 & et quidam ornatus omnium virtutum . \textbf{ Cum enim ille sit magnificus , } qui in magnis operibus facit decentes sumptus : & assi commo la magnanimidat es vna ꝑfectiuo e vn honrramiento de todas las uirtudes \textbf{ Ca commo aquel sea magnifico } que faze conuenibles espenssas enlas grandes obras \\\hline
1.2.22 & decenter tolerabimus , \textbf{ cum non multum reputemus ipsa . } Recitat autem Philosophus & tenporales sofrir la hemos conueniblemente \textbf{ por que los non tenemos en mucho } e dedes saber \\\hline
1.2.23 & nec quod alii vituperentur . \textbf{ Nam cum talia inter exteriora bona computentur , } ipse non multum curat de eis , & ni de seer denostado . \textbf{ Ca porque tales cosas commo estas son contadas | entroͤ los bienes de fuera } non faze grant fuerça dellas . \\\hline
1.2.23 & ut esse veridicos ; \textbf{ cum sint regula aliorum , } quae obliquari , & de seer manifiestos e claros e seer uerdaderos \textbf{ por que son regla de los otros } La qual regla non se deue torcer nin falssar \\\hline
1.2.24 & et ea faciet excellenter . \textbf{ Nam cum honor inter exteriora bona sit bonum excellens , } faciens opera virtutum , & Ca commo la honrra \textbf{ entre los bienes de fuera | sea mas alto e meior bien } el que faze obras de uirtudes \\\hline
1.2.25 & quia docuissemus eos esse sine virtutibus , \textbf{ cum absque humilitate virtutes haberi non possint . } Ad plenam igitur & que serian sin uirtudes \textbf{ si non ouiessen humildat } por que sin la humildat las otras uirtudes non se pueden auer Et pues que assi es para auer conplida declaraçion de la uerdat \\\hline
1.2.25 & quam retrahat nos ab illis . \textbf{ Cum ergo retrahere et impellere sint quodammodo opposita , } et formaliter differant , & nin nos arriedra dellas . \textbf{ Et por ende commo tirar nos | de aquello que la razon manda } e allegarnos a esso \\\hline
1.2.26 & Non tamen aequae principaliter operatur utrunque : \textbf{ nam cum magnanimi sit tendere in magnum , } magnanimitas magis est & e el entendimiento muestran . Enpero estas dos cosas non las obra egualmente nin prinçipalmente \textbf{ Ca commo almagranimo pertenesca de yr | e entender en cosas grandes la magranimidat } mas es uirtud \\\hline
1.2.26 & ut supra diffusius dicebatur . \textbf{ Quare cum distincta sit virtus haec ab illa , } videndum est & assi commo dicho es de ssuso mas conplidamente \textbf{ por la qual cosa commo esta uirtud | que es dicha humildança sea apartada dela magnanimidat } conuienenos de veer \\\hline
1.2.26 & Est enim hoc notabiliter attendendum , \textbf{ quod cum virtus magis sit retrahens quam impellens , } principaliter opponitur superabundantiae , & que quando la uirtud . \textbf{ mas nos trahe e tira | que nos allega e esfuerca . } Estonçe prinçipalmente \\\hline
1.2.27 & ostendere non est difficile . \textbf{ Nam cum ira peruertat iudicium rationis , } non decet Reges et Principes esse iracundos , & esto non es cosa guaue mas ligera . \textbf{ Ca por que la yr a | tristorna el iuyzio dela razon } e del entendimiento non conuiene alos Reyes \\\hline
1.2.27 & non decet Reges et Principes esse iracundos , \textbf{ cum in eis maxime vigere debeat ratio et intellectus . Sicut enim videmus } quod lingua infecta per coleram , & et alos prinçipes de seer sannudos \textbf{ por que en ellos mayormente deue seer apoderada la razon e el entendemiento | que en otros ningunos } Ca assi commo veemos \\\hline
1.2.27 & ut fiant punitiones et vindictae , \textbf{ cum hoc faciat mansuetudo , } decet eos mansuetos esse . & para fazer uenganças e dar penas . \textbf{ Et commo esto faga la manssedunbre } conuiene a ellos de ser manssos \\\hline
1.2.28 & per quam debite conseruetur . \textbf{ Quare cum recta ratio dictet , } quod secundum diuersitatem personarum & por la qual conueniblemente sepa conuerssar e beuir con los otros . \textbf{ Por la qual cosa commo la razon derecha } e el entendemiento muestre \\\hline
1.2.29 & a veritate recedit ratione defectus . \textbf{ Quare cum mendacium sit semper fugiendum , } ut dicitur 4 Ethicorum , & en razon de fallesçimiento . \textbf{ Por la qual cosa commo la mentira | por si misma sea mala deuemos foyr della } assi commo dize aristotiles \\\hline
1.2.29 & Concedunt enim de se aliqui magnas bonitates , \textbf{ cum tamen illis careant : } et promittunt amicis & Ca otorgan alguons de ssi mismos grandes bondades \textbf{ commo quier que en ellos nen sean } e prometen alos amigos conosçidos grandes bienes e grandes ayudas . \\\hline
1.2.30 & et laboribus nostris . \textbf{ Quare cum in talibus contingat peccare , } et bene facere , & e non baldias nin en vano . \textbf{ Por la qual cosa sientales cosas | contesçe de pecar } e de bien fazer \\\hline
1.2.30 & quorum curam habere debemus . \textbf{ Quare cum iocus immoderatus , vel inhonestus distrahat nos a bonis operibus , } et a debitis curis : & por la qual cosa commo el iuego non te prado \textbf{ nin honesto nos parta | e nos tire delas buenas obras } e de los cuydados conuenibles \\\hline
1.2.31 & omnibus caret . \textbf{ Cum ergo omnino manifestum sit , } quod decet Reges & non ha ningunan de todas las otras \textbf{ Et pues que assi es commo en todo en todo es manifiesto e prouado } que conuiene alos Reyes \\\hline
1.2.31 & Prudentia vero rectificat viam . \textbf{ Sed cum non sit perfecta via } nisi ordinetur in bonum finem et terminum , & e enderesça el camino a aquellas cosas \textbf{ que son ordenadas a aquella fin . | Mas commo non puede seer camino } e carrera acabada \\\hline
1.2.31 & quia perfecte una virtus sine aliis haberi non potest . Immo expedit Regibus et Principibus , \textbf{ cum non possint se excusare } per defectum exteriorum bonorum , & mas ante conuiene alos Reyes \textbf{ e alos prinçipes | commo ellos non se puedan escusar } por mengua de los bienes \\\hline
1.2.34 & et immobiliter operari . \textbf{ Quare cum cabulia consilietur , } synesis iudicet , & Et lo terçero que firmemente e si mouemiento obre . \textbf{ Por la qual cosa commo esta uirtud | que es dicha eubolia conseie } Et esta otra que llaman sinesis \\\hline
1.2.34 & quam virtus . \textbf{ Sed cum continentia potior sit , } quam perseuerantia & ala uirtud que uirtud . \textbf{ Mas commo la continençia sea meior } e mayor que la perseuerança tomando la perseueraça \\\hline
1.3.1 & et quos motus animi Reges et Principes debeant imitari . \textbf{ Sed cum hoc sciri non possit , } nisi prius sciuerimus & deuen seguir los Reyes e los prinçipes . \textbf{ Mas por que esto non se puede saber } si non sopieremos primero \\\hline
1.3.1 & passionem oppositam irae . \textbf{ Sed cum sit quaedam virtus } inter iram et mansuetudinem , & Ca la manssedunbre nonbra propriamente passion contraria ala saña . \textbf{ Mas por que ha de ser alguna uirtud entre la sanna e la mansedunbre } la qual uirtud non podemos nonbrar \\\hline
1.3.2 & Nam spes , et desperatio , \textbf{ cum sumantur respectu boni , } praecedunt timorem , et audaciam , iram , et mansuetudinem , & Maen el tercero logar son de poner la esperançar la desesꝑança . \textbf{ Ca por que son tomadas } por razon de bien son puestas primero que el temor e la oladia \\\hline
1.3.2 & et mansuetudinem . \textbf{ Nam cum aliquid prius sit futurum , } quam praesens : & que sasanna e la manssedunbre . \textbf{ Ca commo algunan cosa primero sea futura de uenir } ante que sea presente . \\\hline
1.3.3 & esse circa diuina , et communia . \textbf{ Erit fortis ; quia cum bonum cumune proponat bono priuato , } non dubitabit etiam personam exponere , & prinçipalmente ha de ser çerca los bienes diuinales e comunes . \textbf{ Otrosi sera fuerte por que ante pone el bien comunal bien propreo } e avn non dubdara de poner la persona a muerte siuiere \\\hline
1.3.3 & et uniuersaliter omnia vitia . \textbf{ Quare cum de ratione odii sit exterminare , } et nunquam satiari nisi exterminet , & Et generalmente todos males e todos pecados \textbf{ por la qual cosa commo de razon dela mal querençia sea matar } e nunca se fartar \\\hline
1.3.5 & potissime competere debent Regibus et Principibus . \textbf{ Nam cum Reges et Principes sint latores legum , } quia secundum Philosophum proprie spectat & e nesçera los Reyes \textbf{ e alos prinçipes . | Ca conmolos Rayes sean fazedores e conponedores delas leyes . } Ca segunt el philosofo propriamente pertenesçe alos Reyes \\\hline
1.3.5 & debet esse bonum diuinum et commune , \textbf{ cum talia sint bona excellentia et ardua , } non solum spectat ad Reges & assi commo mostramos de suso deue ser el bien diuinal e comunal \textbf{ por que tales bienes son bienes mas sobrepiunates | e mas altos que los otros } Por ende non solamente parte nesçe alos Reyes \\\hline
1.3.5 & quae non valent perficere . \textbf{ Cum ergo regium officium requirat hominem prudentem } et non passionatum immoderata passione , & ¶ \textbf{ Et pues que assi es conmo el ofiçio de los Reyes | demande omne sabio } e non passionado \\\hline
1.3.7 & Sed odienti quidem nihil differt : \textbf{ nam cum odium sit mali } secundum se et absolute , & Mas por çierto el que quiere mal a otro non cura desto . \textbf{ Ca commo la mal querençia sea algun mal segunt si } e sueltamente abasta el mal quariente \\\hline
1.3.7 & non autem odio . \textbf{ Nam cum ira satietur , } si multa mala inferantur alteri , & Mas la mal querençia non \textbf{ Porque commo la sanna se pueda fartar } si muchos males fueren fechos al otro \\\hline
1.3.7 & ei sed odium pro nullo miserebitur , \textbf{ cum sit quid insatiabile . } Octaua differentia est : & Mas la mal querençia de ninguno non se apiada por que es cosa \textbf{ que se non farta . } ¶ La octaua diferençia es \\\hline
1.3.7 & sed vult eum interimi et non esse . \textbf{ Cum ergo conditiones odii sint multo peiores , } quam conditiones irae , & e non sea . \textbf{ Et pues que assi es commo las condiconnes dela mal querençia | sean mucho peores } que las condiconnes dela saña . Mas nos deuemos guardar dela mal querençia \\\hline
1.3.7 & impedimur ab usu rationis , \textbf{ quare cum per iram accendatur sanguis circa cor , } corpus redditur intemperatum , & Ca el cuerpo non estando en tenpramiento conuenible somos enbargados en el vso dela razon . \textbf{ Por la qual cosa commo por la saña se ençienda la sangre cerca el coraçon tornasse el cuerpo destenprado } e non podemos conueniblemente vsar de la razon . \\\hline
1.3.8 & ponit aliquam delectationem esse prosequendam . \textbf{ Nam cum loquela non possit negari , } nisi per loquelam , & En essa misma manera el que pone que toda delectaçiones de esquiuar e de foyr pone que alguna delectaciones de segnir . \textbf{ Ca assy commo la fabla non puede ser negada sinon por la fabla . } Ca nengando la fabla fabla el omne en fablando otorga e pone la fabla . \\\hline
1.3.8 & ex coniunctione conuenientis cum conuenienti : \textbf{ cum ergo alia conueniant bestiis , alia hominibus : } aliquae delectationes sunt conuenientes bestiis , & Otrossi por que la delectacion se faze por ayuntamiento dela cosa conuenible con cosa conuenible \textbf{ Por ende commo algunas cosas conuengan alas bestias | e algunos alos omes } algunans delectaçiones seran conuenientes alas bestias \\\hline
1.3.8 & et aliud passione agunt . \textbf{ Quare cum in seipsis pacem non habeant , } de seipsis non gaudent . & e lo al fazen despues por la obra \textbf{ Por la qual cosa commo ellos non ayan paz en si mismos non gozan de ssi mismos . } Et por ende grand remedio es anos \\\hline
1.3.8 & nec videtur usque quaque vera . \textbf{ Nam cum de dolore amicorum sit dolendum , } cum nos dolemus , & que en todas maneras sea uerdadera \textbf{ Ca quando nos dolemos del dolor de los amigos dolemos nos nos } e veemos los amigos doler . \\\hline
1.3.9 & ut habet rationem ardui . \textbf{ Cum ergo aliquid maxime sit bonum , } cum iam est adeptum , & Mas el appetito enssannador va a bien en quanto aquel bien ha razon de ser alto e grande . \textbf{ Et por ende commo alguna cosa | entonçe sea dicha muy buean } quando ya es ganada \\\hline
1.3.9 & spes et timor sunt principales passiones respectu irascibilis . \textbf{ Sed cum ex passionibus diuersificari habeant opera nostra , } decet nos diligenter intendere , & en conparacion del appetito enssannador . \textbf{ Mas commo las nr̃as obras ayan de ser departidas | por estas passiones . } Cconuiene a nos de acuçiosamente entender \\\hline
1.4.1 & et secundum cursum naturalem debent multum viuere in futuro . \textbf{ Cum ergo memoria sit respectu praeteritorum , } et spes respectu futurorum : & que ha de uenir . \textbf{ Et pues que assi es commo la memoria sea en conparaçion del tienpo passado | por que es recordaçion delas cosas } que passaron e la esperança es en conparacion \\\hline
1.4.1 & Iuuenes ergo , \textbf{ cum sint liberales , et cum sint animosi et bonae spei , } non habent unde retrahantur & e entremetesse de fazer grandes cosas . \textbf{ Et pues que assi es commo los mancebos } non ayan ninguna cosa \\\hline
1.4.1 & Posset etiam ad hoc specialis ratio assignari . \textbf{ Nam cum iuuenes sint percalidi , } et calidi sit superferri : & Et avn podemos aesto adozir otra razon espeçial . \textbf{ Ca por que los mançebos son muy calientes } e la calentura quiere sienpre sobir arriba e sobrepuiar . \\\hline
1.4.2 & quod duplici de causa contingit . \textbf{ Nam cum iuuenes sint percalidi , } et corpore calefacto fiat venereorum appetitus , & la qual cosa les contesçe por dos razones ¶ \textbf{ La primera razon es que por que los mancebos son muy calient s̃ } e el cuerpo escalentado faze appetito \\\hline
1.4.2 & sed sua innocentia alios mensurant . \textbf{ Cum ergo naturale sit , } quod quis de facili credat ei , & e por la su sinpleza mesuran alos otros . \textbf{ Et pues que assi es commo natural cosa sea } que qual quier omne de ligero cree a aquel que cuyda que es bueno . \\\hline
1.4.2 & eos esse nimis creditiuos . \textbf{ Nam cum multos habeant adulatores , } et plurimi sint in eorum auribus susurrantes , & e alos prinçipes de çreer de ligero . \textbf{ Ca commo ellos ayan muchos lisongeros } e muchos les estenruyendo alas oreias \\\hline
1.4.2 & in actionibus suis : \textbf{ quia cum alia sint moderanda per mensuram , } maxime decet mensuram & Ca commo todas las sus obras \textbf{ de una ser tenp̃das | por me lura } e por regla mucho \\\hline
1.4.3 & naturaliter passionatur passione illa . \textbf{ Cum ergo timidi efficiantur frigidi , } quicunque est naturaliter frigidus , & natraalmente es passionado de aquella passion \textbf{ Et pues que assi es commo los temerosos sean esfriados . } Et qual si quier que naturalmente es frio naturalmente es temeroso \\\hline
1.4.3 & Videmus autem quod \textbf{ cum senes adinuicem congregantur , } semper recitant res gestas , & en que se delecta . \textbf{ Ca nos veemos que quando los uieios se ayuntan en vno } sienpre cuentan las cosas passadas \\\hline
1.4.3 & Verecundia ergo , \textbf{ cum sit timor inhonorationis , } non competit senibus ; & en el segundo libro dela Rectorica \textbf{ Ca por que la uerguença es temor de desonera non pertenesçe alos uieios } por que may orcuidado han del prouecho \\\hline
1.4.4 & quam se extendant ad alia . \textbf{ Cum enim nulla sit actio animae , } in qua non utatur & que non se estiendan alas otras cosas que non han . \textbf{ Ca commo non sea ninguna obra del alma en el cuerpo } en la qual non vse el alma \\\hline
1.4.5 & si ab antiquo affluebat diuitiis . \textbf{ Cum ergo semper sit dare initium , } in quo genitores alicuius ditari inceperunt : & si de antigo tienpo abondo en riquezas . \textbf{ Et pues que assi es comm sienpre ayamos de dar comienço } en que los padres de alguons comne caron de se enrriqueçer \\\hline
1.4.5 & in filiis quam in parentibus : \textbf{ quare cum nobilitas semper inclinet animum nobilium } ut faciant magna , & antiguadas las riquezas en los fijos que en los padres . \textbf{ Por la qual razon commo la nobleza | sienpre incline el coraçon de los nobles } para fazer grandes cosas siguese \\\hline
1.4.5 & Cum enim nobiles \textbf{ cum magna diligentia nutriantur , } et cum magna cura proprium corpus custodiant : & ¶la primera razon le prueua alsi . \textbf{ Ca por que los nobles son cados con grand acuçia } e con grand cura \\\hline
1.4.5 & cum magna diligentia nutriantur , \textbf{ et cum magna cura proprium corpus custodiant : } rationabile est , & Ca por que los nobles son cados con grand acuçia \textbf{ e con grand cura | e con grand guarda delos sus cuerpos } con razon es \\\hline
1.4.5 & et bene complexionatum . \textbf{ Cum ergo molles carne aptos mente dicamus , } ut vult Philos’ 2 de Anima : & que ellos ayan los cuerpos bien ordenados e bien conplissionados . \textbf{ Et pues que assi es conmolos bien conplissionados | que son muelles en las carnes sean mas engennosos e sotiles en las almas } assi commo dize el philosofo \\\hline
1.4.5 & Reges ergo et Principes , \textbf{ cum non possint naturaliter dominari , } nisi sint boni et virtuosi , & Et pues que assi es commo los Reyes \textbf{ e los prinçipes non puedan naturalmente ensseñorear } si non fueren bueons \\\hline
1.4.6 & et volunt videri esse excellentes : \textbf{ cum ex hoc quis excellere videatur , } si potest aliis contumelias inferre : & mas altos que todos los otros . \textbf{ Et commo por tal razon commo esta alguno parezca de ser mas alto } si puede fazer tuertos alos otros \\\hline
1.4.7 & et habet multos sub suo dominio ; \textbf{ quare cum multos nobiles videamus esse impotentes , } et non posse principari , & e ha muchos so su sennorio . \textbf{ por la qual cosa commo nos veamos muchos ser nobles | qua non son poderosos } e non pueden ser prinçipes \\\hline
2.1.1 & esse desinerent ; \textbf{ quare cum viuere sit homini naturale , } omnia illa , & si ansi luego que son fechͣs començassen afallesçer . \textbf{ Por la qual cosa commo el beuir sean | atuer tal cosa al omne } todas aquellas cosas \\\hline
2.1.1 & Homini autem non sufficienter prouidet natura in vestitu : \textbf{ cum enim homo sit nobilioris complexionis } quam animalia alia , & Mas la natura non prouee al omne tan conplidamente en uestidura \textbf{ por que el omne es de mas nobł conplission } que las otras aianlias e mas ayna resçibedano \\\hline
2.1.1 & Quare si naturale est homini desiderare conseruationem vitae , \textbf{ cum homo solitarius non sufficiat sibi } ad habendum congruum victum et vestitum , & Par la qual cosa si natural cosa es al omne de dessear conseruaçion e guarda de su uida \textbf{ commo el omne | que biue solo non abaste assi mismo } para auer uianda conueinble nin uestidura \\\hline
2.1.1 & si nunquam vidisset canes alias peperisse . \textbf{ Mulier autem cum parit , nescit qualiter se debeat habere in partu , } nisi per obstetrices sit sufficienter edocta . & avn que nunca uiesse o trisperras parir . \textbf{ Mas la mug̃r quan do pare non labe | en qual manera se deua auer en el parto } si non fuere enssennada conuenibłmente por las parteras . \\\hline
2.1.2 & necessariam esse communitatem domesticam : \textbf{ cum omnis alia communitas communitatem illam praesupponat . } Aduertendum ergo quod & que la comunidat dela casa es neçessaria \textbf{ por que todas las comuni dades ençierran en ssi | e ante ponen esta comunidat dela casa ¶ Et } pues que assy es deuedes saber \\\hline
2.1.3 & aliquid primo operatum , \textbf{ cum adepto fine cesset operatio , } nunquam operaremur ea , & por que si la fin fuesse primeramente alguna cosa obrada \textbf{ quando ouiessemos ganada la | finçessarie la obra } e nunca obrariemos nada de aquellas cosas \\\hline
2.1.3 & de prioritate generationis vel temporis , \textbf{ cum ipsemet dicat ciuitatem procedere ex multiplicatione vici , } sicut et vicus procedit ex multiplicatione domorum , & Et esto non se deue entender que es primera por generaçion \textbf{ e por tienpo commo el mismo | diga } que la çibdat se faze de muchos uarrios \\\hline
2.1.3 & quomodo sit huiusmodi communitas naturalis . \textbf{ Nam cum natura non praesupponat artem , } sed ars naturam : & en alguna manera esta comunidat dela cała es natural \textbf{ Ca commo la natura non presupone | nin antepone arte } mas el arte presunpone \\\hline
2.1.4 & sit communitas quaedam et societas personarum : \textbf{ cum non sit proprie communitas nec societas ad seipsum , } si in domo communitatem saluare volumus , & e vna conpannia de muchͣs ꝑssonas . \textbf{ Et commo non sea propreamente comunidat | nin conpannia de vno } assi commo si queremos saluar la comuidat dela casa conuiene que ella sea establesçida de muchͣs perssonas \\\hline
2.1.5 & quia haec illam praesupponit . \textbf{ Nam cum generata non possint conseruari in esse } nisi prius per generationem acceperint esse , & e antepone la generaçion . \textbf{ Ca commo las cosas engendradas | non pueden ser conseruadas } nin guardadas en su ser si primeramente non resçibieren el su ser por generaçion . \\\hline
2.1.5 & Sicut enim caecus corporaliter , \textbf{ nisi ( cum pergit ) dirigatur ab aliquo , } de leui obuiat alicui offensiuo : & que si alguno es çiego corporalmente \textbf{ si quando anda non fue regua ado } por alguno otro de ligero \\\hline
2.1.6 & nisi sit iam perfectus . \textbf{ Quare cum communitas patris ad filium sumat originem } ex eo quod parentes sibi simile produxerunt : & por la qual cosa \textbf{ commo la comunidat del padre al fijo tome nasçençia e comienço de aquello que el padre e la madre } engendran su semeiança esta tal comunidat non es dicha de razon dela primera casa \\\hline
2.1.6 & non producat sibi simile ; \textbf{ quare cum proprium actiuum generationis sit masculus , } et proprium susceptiuum sit foemina : & e non puede fazer su semeiante . \textbf{ Por la qual razon commo el propre o fazedor en la generaçion sea el mas lo . } Et la propre a materia para \\\hline
2.1.6 & in quo tractatur de regimine domus . \textbf{ Nam cum in domo perfecta sint tria regimina , } oportet hunc librum tres habere partes ; & eł gouertiamiento dela casa \textbf{ Ca commo en la casa acabada sean tres gouernamientos . } Ca conuiene que este libro sea partido en tres partes . \\\hline
2.1.7 & praesupponunt communitatem domesticam . \textbf{ Cum ergo domus sit prior vico , ciuitate , et regno : } homo naturaliter & ante ponen la comunidat dela casa . \textbf{ Et pues que assi es commo la casa sea primero | que el uarrio } e que la çibdat \\\hline
2.1.7 & quam politicum : \textbf{ cum prima communitas ipsius domus sit coniunctio viri et uxoris , } sequitur ex parte ipsius communitatis humanae , & que de çibdat \textbf{ commo la primera comunidat dela casa sea | ayuntamientode uaron } e de muger siguese de parte desta conñçion humanal \\\hline
2.1.7 & ad quod homo habet naturalem impetum : \textbf{ quare cum homo et omnia animalia naturaliter inclinentur , } ut velint producere sibi simile , quia in hominibus hoc debite sit per coniugium , & ala qual el omne ha natural inclinaçion \textbf{ e natra al apetito | por la qual cosa commo todas las ainalias naturalmente sean inclinadas } para querer engendrar otro semeiable \\\hline
2.1.8 & et econuerso . \textbf{ Cum enim inter virum et uxorem sit amicitia naturalis , } ut probatur 8 Ethicorum , & e esso mismo la muger al uaron . \textbf{ Ca commo entre el uaron e la muger sea amistança natural } assi commo se prueua en el viij̊ libro delas ethicas \\\hline
2.1.8 & augmentatur eorum amicitia naturalis . \textbf{ Sed cum omnis amor vim quandam unitiuam dicat , } augmentato amore propter prolem genitam , & acresçientase entre ellos amorio natural \textbf{ Mas coͣtra odo amor aya alguna fuerça | para ayuncar los omes } el actes çentamiento del amor \\\hline
2.1.9 & est amicitia excellens et naturalis . \textbf{ Sed cum excellens amor non possit esse ad plures , } ut vult Philosophus 9 Ethicor’ , & entre ellos es amistança muy grande e muy natural . \textbf{ Mas commo el grand amor non pueda ser departido amuchͣs partes } assi conmo dize el philosofo en elix̊ . \\\hline
2.1.9 & sumitur ex nutritione filiorum . \textbf{ Nam cum coniugium sit quid naturale : } quomodo debito modo fieri debeat & cerazon delos fijos . \textbf{ Ca commo el matermonio sea cosa natural } en qual manera se deua fazer \\\hline
2.1.9 & se habent masculus et foemina tempore partus . \textbf{ Sed cum dictum sit , } quod toto tempore partus in huiusmodi auibus masculus & en el tienpo del parto . \textbf{ Mas commo dicho es } que en todo el tp̃o del parto \\\hline
2.1.10 & ad quam coniugium ordinatur . \textbf{ Nam cum quilibet moleste ferat , } si in usu suae rei delectabilis impeditur ; & ala qual es ordenado el casamiento . \textbf{ Ca commo qual si quier sufra } guauemente si le enbargaren del vso de aquella cosa \\\hline
2.1.11 & Prima via sic patet . \textbf{ Nam cum ex naturali ordine debeamus parentibus debitam subiectionem , } et consanguineis debitam reuerentiam , & La primera razon se declara assi . \textbf{ Ca commo por la orden natural deuamos auer | subiectiuo al padre e ala madre } e reuerençia conueible alos parientes \\\hline
2.1.12 & in quandam societatem debitam et naturalem . \textbf{ Cum ergo debite et congrue nobili societur : } Reges et Principes , & e conueniblemente \textbf{ el noble deua ser aconpannado } ala noble los Reyes \\\hline
2.1.13 & Decet eas etiam amare operositatem : \textbf{ quia cum aliqua persona ociosa existat , } leuius inclinatur & Et avn les conuiene aellas de amar fazer buenas obras . \textbf{ Ca quando alguna persona esta de uagar mas ligeramente es inclinada a aquellas cosas } que la razon iueda \\\hline
2.1.14 & quod debet esse in uno homine : \textbf{ cum ciuitas sit pars uniuersi , } regimen totius ciuitatis multo magis reseruabitur in una domo . & que deua ser en vn omne \textbf{ commo la çibdat } se aparte de todo el mundo el gouernamiento de teda la çibdat mucho mas es fallada e nonacasa . \\\hline
2.1.14 & quibus vacare debeant \textbf{ cum sint adulti : } ad quae non sunt instruendae uxores , & e alas obras çiuiles alas quales deuen entender \textbf{ quando fueren criados } e mayores alas quales cosas non son de enssennar las mugers \\\hline
2.1.15 & quando unum ordinatur ad unum officium : \textbf{ cum uxor naturaliter sit ordinata ad generandum , } non erit ordinata ad seruiendum . & çquano cada cosa es ordenada a vn ofiçio \textbf{ commo la muger sea | ordenadanatraalmente ala generaçion de los fijos } non deue ser ordenada a seruiçio . \\\hline
2.1.15 & nisi careat usu rationis et intellectus . \textbf{ Sed cum carens rationis usu sit naturaliter seruus , } quia nescit seipsum dirigere , & si non fuese priuado de vso de razon e de entendimiento . \textbf{ Mas commo aquel que es priuado de vso de razon e de entendemiento sea naturalmente sieruo } por que non sabe gniar assi mismo \\\hline
2.1.16 & sed etiam quantum ad animam : \textbf{ quia cum anima sequatur complexiones corporis } ( nam cum aliquis non est bene proportionatus in corpore , & non solamente quanto al cuerpo mas avn quanto al alma \textbf{ ca el alma sigue las conplissiones del cuerpo . } Ca quando alguno non es bien conplissionado en el cuerpo \\\hline
2.1.18 & ex timiditate cordis . \textbf{ Nam cum mulieres sint naturaliter adeo timidae , } quod quasi omnia expauescunt ; & por el temor del coraçon . \textbf{ Ca commo las muger ssean naturalmente temerosas | en tanto que semeia } que de todas las cosas se espantan . \\\hline
2.1.20 & redundat in persona ipsius viri . \textbf{ Immo cum ostensum sit supra uxorem } non se habere & tornase en la perssona del marido . \textbf{ Mas por que fue mostrado de suso que la mugni non se deue auer al marido } assi commo sierua mas assi commo conpanera . \\\hline
2.1.20 & restat ostendere , \textbf{ qualiter cum eis debeant conuersari . } Tunc autem viri ad uxorem est conuersatio congrua , & honrradamente finca de demostrar \textbf{ en qual manera deuen beuir conellas } mas estonçe es dicha la conuersaçion \\\hline
2.1.21 & suas coniuges debite se habere . \textbf{ Nam cum vir suam uxorem regere debeat , } eam dirigendo ad actiones honestas , & de se auer conueniblemente en el conponimiento e honrramiento de sus cuerpos . \textbf{ Ca quando el marido gouierna e castiga a su muger } deue la castigar a obras honestas \\\hline
2.1.21 & quod infirmior magis gloriatur , \textbf{ quia credit quod in cum plures aspiciant , } et sperat se plures eleemosynas accepturum : & contesçe que el mas enfermose eglesia \textbf{ mas por que cree que muchos catan ael } e es para que resçibra mas helemosinas que los otros . \\\hline
2.1.22 & et per consequens semper sunt in anxietate cordis : \textbf{ quare cum una cura impediat aliam , } oportet sic zelantes & que sienpreson en grand angostura de su çoraçon . \textbf{ Por la qual cosa commo el vn cuydado enbargue el otro . } Conuiene alos tales çelosos de ser enbargados enlos cuydados \\\hline
2.1.23 & elegibilius esset consilium muliebre quam virile . \textbf{ Natura enim cum moueatur ab intelligentiis , et a Deo , } in quo est suprema prudentia ; & e la razon es esta \textbf{ por que la natura toda es mouida delos angeles | e de dios } en que es conplimiento de sabiduria . Et por ende conuiene que aquellas cosas \\\hline
2.1.23 & ad corpus cuius habet \textbf{ esse perfectum quam vir . Quare cum anima sequatur complexionem corporis , } sicut ipsum corpus muliebre & que el uaron . \textbf{ Por la qual cosa | commo el alma sigua ala conplission del cuerpo . } A assi commo el cuerpo dela muzer \\\hline
2.1.24 & ut plurimum sunt molles et ductibiles , \textbf{ statim cum aliqua persona eis applaudet , } et ridet in facie earum , & e mas de ligero son mouibles \textbf{ luego que algunas ꝑssonas les comiençan a lisongar } e a Reyr en su faz dellas \\\hline
2.1.24 & His visis , \textbf{ cum ostensum sit , } quomodo communitas viri et uxoris sit naturalis , & ¶ vistas estas cosas \textbf{ commo seaya prouado } que la conpannia del uaron \\\hline
2.1.24 & ad alias communitates domus : \textbf{ cum etiam declaratum sit , } quomodo Reges et Principes , & se aya alas otras conpannias . \textbf{ Et avn commo sea declarado } en qual manera los Reyes e los prinçipes \\\hline
2.1.24 & et uniuersaliter omnes ciues se habere debeant ad suas coniuges , \textbf{ et quomodo cum eis debeant conuersari , } imponatur finis primae parti huius secundi Libri , & e generalmente todos los çibdad a uos se de una auer alus mugers \textbf{ e como de una beuir con ellas . } Pongamos fin e acabamiento a esta primera parte deste segundo libro \\\hline
2.2.1 & quilibet enim solicitatur circa dilectum : \textbf{ quare cum inter patrem et filium sit amor naturalis , } ut probatur 8 Ethicorum , & Ca cada vno ha cuydado de su amor . \textbf{ por la qual cosa commo entre el padre | e el fijo sea amor natraal } assi commo se praeua en el viij delas . ethicas . \\\hline
2.2.3 & ut uxores viros . \textbf{ Quare cum regimen filiorum sit ex arbitrio , } et sit propter bonum ipsorum filiorum ; & assis padres \textbf{ commo las muger suarones Por la qual razon como el gouernamiento de los fijos } sea por aluedrio e sea por el bien de los fijos . \\\hline
2.2.3 & quando potest sibi simile generare . \textbf{ Quare cum quilibet suam perfectionem diligat , } naturaliter pater diligit filium , & quando ꝑuede engendrar su semeiante . \textbf{ Et commo quier que cada vno ame su perfecçion . } Emperona traalmente el padre ama el fijo \\\hline
2.2.4 & quam econuerso . \textbf{ Quare cum amor quandam unionem importet , } filii tanquam magis uniti et magis propinqui parentibus , & que los padres alos fijos . \textbf{ por la qual razon commo el amor faga algun ayuntamiento los fijos } assi conmomas ayuntados \\\hline
2.2.4 & ad filios debent eos regere et gubernare . \textbf{ Sed cum filii afficiantur ad parentes , } tanquam ad eos , & que han los padres alos fijos los deuen gouernar \textbf{ Mas commo los fijos sean inclinados alos padres . } assi commo aquellos que quieren auer en honrra e en reuerençia . \\\hline
2.2.5 & et tanto feruentius adhaeremus illi . \textbf{ Cum ergo magis simus assuefacti ad ea , } circa quae in ipsa infantia insudamus , & e con mayor acuçianos llegamos a ella . \textbf{ Et pues que assi es commo mas somos usados a aquellas cosas } en que trabaiamos enla moçedat \\\hline
2.2.6 & retrahantur a lasciuiis . \textbf{ Quare cum rationis sit concupiscentias refraenare et lasciuias , } quanto aliquis magis a ratione deficit , & por que de la razon \textbf{ e del entendimiento | es de refrenar los desseos e las locanias . } Et por ende quanto alguon mas fallesçe en razon \\\hline
2.2.7 & et Principum \textbf{ cum ponuntur in aliquo dominio tyrannizent , } decet ipsos etiam ab ipsa infantia insudare literis , & e de los prinçipes \textbf{ quando son puestos en algun sennorio non tiraniçen | nin sean tirannos } Conuiene les avn de trabaiar \\\hline
2.2.10 & iuuenes sunt insecutores passionum , \textbf{ et ad lasciuiam proni . Quare cum semper sit adhibenda cautela } ubi periculum imminet , & los moços alos mançebos son segnidores de passiones \textbf{ e son inclinados a orgullos e aloçania . | Por la qual cosa commo sienpre deuemos dar } algunan cautellado paresçe el peligro . \\\hline
2.2.10 & ut instruantur quod palpebras oculorum \textbf{ cum maturitate eleuent , } ut non habeant oculos vagabundos . & quanto ala manera de ver \textbf{ assi que alçen las palpebras de los oios con grand madureza } e que non echen los oios \\\hline
2.2.11 & si contingat ex inordinatione animae . \textbf{ Quare cum turpis modus sumendi cibum signum sit cuiusdam gulositatis , } vel inordinationis mentis & por desordenamiento del alma . \textbf{ Por la qual cosa commo la manera torpe de resçebir la vianda | sea señal de golosina } de aquel que la toma o de desordenaçion del alma \\\hline
2.2.12 & maxime est prona ad intemperantiam , \textbf{ quare cum semper sit adhibenda cautela , } ubi maius periculum imminet , & Ca do mayor es el periglo \textbf{ alli deue omne poner mayor remedio } Et por ende en la hedat de los moços deuemos guardar \\\hline
2.2.13 & Prima via sic patet . \textbf{ Nam mens humana nescit ociosa esse : cum ergo quis vacat ocio , } et non intendit aliquibus delectationibus licitis , & La primera razon paresçe assi . \textbf{ Ca la uoluntad del omne non sabe ser ocçiosa | nin estar de vagar } Et pues que assi es quando alguon se da adagar \\\hline
2.2.14 & nisi sciuerint \textbf{ cum quibus sociis debeant conuersari . } Quanto autem ad praesens spectat , & si non sopieren con quales conpanneros \textbf{ e en qual conpannia deuen beuir . } Mas quanto pertenesçe alo presente quatro cosas paresçe \\\hline
2.2.21 & Cum ergo diligens examinatio \textbf{ cum loquacitate stare non possit , } ut foeminae etiam a puellari aetate discant & para prouar esto se \textbf{ tomadesto } que las mugrͣ̃s non sean prestas avaraias e apeleas \\\hline
2.3.8 & ut viuant secundum corporis voluptatem ; \textbf{ cum diuitiae maxime videantur hoc efficere , } ut per eas quilibet consequi possit & que biuna segunt los delectes del cuerpo \textbf{ e por que las riquezas prinçipalmente fazen esto } assi que por ellas cada vno cuyda \\\hline
2.3.8 & quantum sufficiat ad nutrimentum animalis generati . \textbf{ Cum ergo diuitiae et possessiones ordinentur ad nutrimentum } et ad sufficientiam vitae , & para el nudermiento dela aian lia que es engendrada . \textbf{ ¶ Et pues que assi es las riquezas | e las possessiones sean ordenadas } para nudermiento e abastamiento dela uida \\\hline
2.3.9 & et talia quibus indigemus ad vitam , \textbf{ cum sint magni ponderis , } commode ad partes longinquas portari non possunt . & que auemos menester para la uida \textbf{ por que son de grand peso } non las poderemos leuar conueniblemente a luengas tierras . \\\hline
2.3.11 & et quia hoc est contra naturam artificialium , \textbf{ cum denarius sit quid artificiale , } bene dictum est & e por que esto es contra natura delas cosas artifiçiales \textbf{ e como los diueros sean cosas artifiçiales } bien dicho es lo que dize el ph̃co \\\hline
2.3.11 & id est , rapina usus . \textbf{ Cum ergo usus ipsius domus sit domum inhabitare , } non domum alienare ; & tomassemos serie y usura que quieredezer robo de uso . \textbf{ Et por ende commo el uso dela casa sea morar en la casa } et non enagenar la casa \\\hline
2.3.13 & nisi ibi naturaliter aliud sit praedominans : \textbf{ cum societas hominum sit naturalis , } quia homo est naturaliter animal sociale , & fuereynatraalmente alguna cosa \textbf{ que en ssennore e a todas aquellas muchͣs . } Et commo la conpannia de los omes sean a falpor el ome es aianlia \\\hline
2.3.13 & quod habet consilium inualidum : \textbf{ cum ergo videamus aliquos homines respectu aliorum } plus deficere & que ha conseio muy flaco \textbf{ Et pues que assi es commo nos ueamos | que algs omes en conparaçion de los otros } pueden \\\hline
2.3.14 & si considerentur legum conditores . \textbf{ Nam cum legum latores sint homines , } quibus magis sunt nota bona corporis et exteriora , & si pararemos mientes alos establesçedores delas leyes . \textbf{ Ca commo los establesçedores sołas leyes sean omes } alos quales mas son conosçidos los bienes del cuerpo \\\hline
2.3.17 & et congruentia temporum . \textbf{ Cum enim deceat Regem esse magnificum , } ut supra in primo libro diffusius probabatur , & La conueniençia de los tiepos . \textbf{ Ca commo conuenga alos Reyes | e alos prinçipes ser magnificos } assi commo es prouado mas conplidamente en el primero libro \\\hline
2.3.18 & Sic etiam si hylariter et affabiliter \textbf{ quis cum aliis conuersetur , } si hoc agit & si \textbf{ algbiuiere et morare con los otros alegremente } e amigablemente si esto faze \\\hline
2.3.19 & videlicet qualiter \textbf{ cum ipsis sit conuersandum . } Quod aliquomodo tradit Philosophus & ¶ Esto iusto finça de demostrar lo terçero \textbf{ que es en qual manera han de beuir los sennores con sus ofiçiales } la qual cosa muestra el philosofo en alguna manera \\\hline
3.1.1 & Si ergo omnes homines ordinant sua opera in id quod videtur bonum , \textbf{ cum ciuitas sit opus humanum , } ex parte hominum constituentium ciuitatem oportet & a aquello que paresçe bien \textbf{ o commo la çibdat | sea obra de los omes } de parte de los omes \\\hline
3.1.4 & quia deseruiunt ad sufficientiam uitae humanae . \textbf{ Quare cum communitas ciuilis has communitates comprehendat , } et perfectius deseruiat & por que siruen al conplimiento dela uida humanal \textbf{ por la qual cosa commo la comuidat çiuil | o la çibdat conprehenda estas dos comuindades } e mas acabadamente sirua al conplimiento dela uida humanal \\\hline
3.1.7 & et maxima coniunctio in ciuitate . \textbf{ Nam cum sit maxima unitas , } et maxima coniunctio patrum ad filios , & e muy grant ayuntamiento enla çibdat . \textbf{ Ca commo sea muy grant vnidat } e grant ayuntamiento de los padres alos fijos los mas antiguos \\\hline
3.1.8 & ad principantes et subiectos . \textbf{ Nam cum ciuitas sit ordo ciuium } ad aliquem principantem vel dominantem , & por conparaicion de los prinçipes e de los subditos \textbf{ ca commo la çibdat sea orden } delos çibdadanos a algun prinçipe o algun sennor \\\hline
3.1.8 & et aliqui subiecti . \textbf{ Quare cum hoc diuersitatem requirat , } oportet in ciuitate & e alguons que fuessen subditos \textbf{ por ende commo estas cosas demanden departimiento } conuiene de dar en la çibdat algun departimiento . \\\hline
3.1.9 & ne circa ipsa contingat error . \textbf{ Quare cum in regimine ciuitatis primo sit politia ordinanda , } diu inuestigandum est , & escodrinnar por que çerca ellos non contezca yerro . \textbf{ por la qual cosa commo en el gouernamiento dela çibdat | primeramente se ha de ordenar la poliçia } muy luengamente es de buscar \\\hline
3.1.11 & tanto magis ad inuicem conuersantur : \textbf{ sed cum esse non possit , } aliquos valde & mas han de beuir en vno \textbf{ mas commo non pueda ser } que alguons \\\hline
3.1.11 & quae et Socrates concedebat . \textbf{ Cum ergo custodes ciuitatis nobiliores sint agricolis , } tanquam meliores et nobiliores estimarent se plus esse accepturos & e esto otorgauas ocrates . \textbf{ Et pues que assi es commo las guardas | e los defendedores dela çibdat sean mas nobles que los labradores } assi commo meiores \\\hline
3.1.12 & quam eos in societate habere \textbf{ nam cum humanum sit timere mortem , } viriles etiam et animosi trepidant & que auerlos en su conpannia \textbf{ ca commo todos los omes | teman la muerte los esforçados } e de grandes coraçones temen \\\hline
3.1.13 & vel ad aliquem magistratum assumitur . \textbf{ Quare cum deceat regia maiestatem } et uniuersaliter omnem ciuem , & commo se conosçe despues que esle un atada en alguna dignidat o en algun maestradgo o en algun poderio \textbf{ por la qual razon commo venga ala real magestad } e generalmente a qual quier que ha de dar \\\hline
3.1.14 & quid circa huiusmodi regimen sit censendum . \textbf{ Quare cum patefactum sit in praecedentibus , } non expedire ciuitati possessiones , & que auemos de iudgar en este gouernamiento \textbf{ por la quel cosa commo sea manifiesto | por las cosas dichͣs de suso } que non conuiene ala çibdat \\\hline
3.1.19 & eo quod talibus alii de facili iniuriantur , \textbf{ cum non possint defendere iura sua . } Multa bona consequimur & por que tales perssonas los otros de ligero les fazen tuerto \textbf{ por que non pueden defender su derecho } uchos bienes se nos siguen delas opiniones de los phos antigos \\\hline
3.2.4 & dum tamen utrunque sit rectum , \textbf{ cum ipse pluries dicat in eisdem politicis , } regnum esse dignissimum principatum : & Puesto que amos los ssennorios sean derechs \textbf{ ca el dize muchͣs uezeᷤ | en esse mismo libro delas politicas } que el regno es prinçipado muy digno \\\hline
3.2.4 & quam dominari unum ; \textbf{ cum nunquam plures recte dominari possint , } nisi inquantum tenent locum unius , & que si enssennoreas se vno . \textbf{ ca nunca pueden much | senssennorear } derechͣmente \\\hline
3.2.5 & ut voluntarie obediant : \textbf{ quare cum omne voluntarium sit minus onerosum et difficile , } ut libentius et facilius obediat populus mandatis regis , & assi commo a cosanatal \textbf{ ca commo toda cosa uoluntaria | sea de menor carga e menos graue } por que mas de grado \\\hline
3.2.7 & Sed si tyrannus dominetur , \textbf{ cum unus sit dominans , } et non intendat nisi bonum proprium ; & Mas si el tirano \textbf{ enssennoreare commo vno sea el señor } e non entienda \\\hline
3.2.10 & et de se confidere ; \textbf{ nam cum intendat bonum ipsorum ciuium et subditorum , } naturale est & e que fien vnos de otros . \textbf{ Ca commo el entienda enl bien de los çibdadanos natural } cosaes que sea amado dellos . \\\hline
3.2.10 & quicquid a ciuibus agitur . \textbf{ Cum enim tyranni sciant se non diligi a populo , } eo quod in multis offendant ipsum , & delo que fazen los çibdadanos . \textbf{ Ca commo los tyranos sepan | que non lon amados del pueblo . } por que en muchͣs cosas le aguauian quieren auer muchs assechadores \\\hline
3.2.12 & Priuatur ergo tyrannus a maxima delectatione , \textbf{ cum videat se esse populis odiosum . } Viso tyrannidem cauendam esse , & e por ende el tirano es pri uado de grant delectaçion \textbf{ quando bee | que es aborresçido delos pueblos } Disto que la tirauja es de esquiuar e de aborresçer \\\hline
3.2.13 & ut recte et debite gubernent populum sibi commissum : \textbf{ cum deuiare a recto regimine sit tyrannizare , } et iniuriari subditis , & e commo desinarse \textbf{ e arredrarse los rreyes del | gouernemjento derecho sea tiranizar } e fazer tuerto alos subditos \\\hline
3.2.13 & per quae se contemptibiles reddunt . \textbf{ Nam cum non quaerant bonum commune , } sed delectationes corporis , & por que se fazen despreçiados de los pueblos \textbf{ ca por qua non qͥeren el bien comun } mas quieren delectaçiones del su cuerpo \\\hline
3.2.14 & Debent ergo cauere Reges et Principes ne tyranizent , \textbf{ cum tot modis dissoluatur tyrannicus principatus . } Tertio dissoluitur tyrannis & e los prinçipes \textbf{ que non tiraniz en commo en tantas maneras | segunt dicho es se aya de destroyr el prinçipado tiranico . } Lo terçero se desfaze la tirani a non solamente \\\hline
3.2.16 & quae tractanda sunt circa ipsum . \textbf{ Sed , cum dicat Philosophus } 3 Ethic’ & quales cosas son de trattrar çerca el . \textbf{ Mas commo el pho diga en el segundo libro delas ethicas } que por çierto alguno tomara consseio non de aquellas cosas \\\hline
3.2.17 & in consiliis hunc habere modum : \textbf{ ut cum aliis conferamus quid agendum , } quod ex duobus patet . & ca de grant sabiduria es en los consseios tener esta manera \textbf{ que con los otros ayamos acuerdo } delo que auemos de fazer la qual cosa paresçe por dos cosas . \\\hline
3.2.17 & plus proficit expertus , quam artifex . \textbf{ Quare cum plures plura experti sint , } quam unus solus : & que el que ha el arte della \textbf{ por la quel cosa commo muchs mas cosas ayan prouadas } que vno solo conuiene de llamar otros \\\hline
3.2.18 & nam quia dicens creditur esse bonus , \textbf{ cum tales mentiri nolint , } de facili creditur eorum dictis . & que es bueno \textbf{ e los bueons non quieren mentir } de ligero creen los omes asodichos . \\\hline
3.2.19 & et in hoc non est quaestio nec consilium . \textbf{ Sed utrum cum extraneis debeamus habere pacem } vel bella dubitabile esse potest , & questiuo nin consseio . \textbf{ Mas si deuemos auer paz con los estrannos o guerra pue de ser cosa dubdosa } e çerca esto pueden ser tomados consseios çerca la qual \\\hline
3.2.20 & et alia quae circa istam occurrunt materiam . \textbf{ Sed cum iudicium fiat per leges , } aut per arbitrium , & que pueden acaesçer çerca desta materia \textbf{ mas commo el iuyzio se deua fazer } por las leyes o por aluedrio o por amas estas cosas . \\\hline
3.2.23 & pro clementia delinquentis . \textbf{ Nam cum natura humana de se sit debilis , } et mutabilis , et prona ad malum , & por la piadat del que peca . \textbf{ ca commo la natura humanal } dessi sea flaca e mouible \\\hline
3.2.23 & omnino enim contra rationem est humilitati non parcere , \textbf{ cum bestiae hoc agant : } canes quidem non offendunt humiliantes se , & ca las bestias perdonan \textbf{ a aquellos que se les homillan | e esto prouamos } por que los canes non fazen mal a aquellos que se les homillan \\\hline
3.2.27 & secundum quas intendimus in bonum illud : \textbf{ quare cum bonum commune principaliter intendatur a tota communitate } ut a toto populo , vel a principante , & por las quales leyes ymosa aquel bien . \textbf{ Por la qual cosa commo el bien comun sea entendido | prinçipalmente de toda la comunidat } assi commo de todo el pueblo o del prinçipe \\\hline
3.2.27 & oportet eam promulgatam esse . \textbf{ Sed cum alia sit lex naturalis , } alia positiua : & conuiene que sea publicada e pregonada . \textbf{ Mas commo otra sea la ley natural e otra la positiua en vna manera se deue publicar la vna } e en otra manera la otra . \\\hline
3.2.29 & quam legem . \textbf{ Nam Rex cum sit homo } non dicit intellectum bonum tantum , & corconper el rey que la ley . \textbf{ Ca el Rey por que es omne } non dize entendemiento tan solamente mas dize entendimiento con cobdiçia . \\\hline
3.2.30 & et ad vitium si sint mali , \textbf{ cum procedant ex interiori appetitu . } Sed si consideretur lex humana & e a pecado si son malas . \textbf{ quando uieñe del apetito | et ple desseo del coraçon . } Mas si fuere penssa para la ley \\\hline
3.2.30 & quod secundum aliorum iudicium est iniustum . \textbf{ Quare cum in humanis iudiciis cadere possit dubieras et error , } expediens fuit lex euangelica & la qual segunt el iuyzio de los otros non es derechͣ . \textbf{ Por la qual cosa commo en los iuyzios humanales pueda caer dubda e yerro . | cosa muy aprouechable } e muy neçessaria fue la ley e una gelical e diuinal \\\hline
3.2.34 & ( quantum ad praesens spectat ) tria , \textbf{ si cum magna diligentia obediat regibus , et principibus , } et obseruet leges regias . & uanto alo presente parte nesçe el pueblo alcança tres bienes \textbf{ si obedesçiere alos Reyes | e alos prinçipes con grand acuçia . } Et si guardare las leyes de los Reyes \\\hline
3.2.34 & inducere alios ad virtutem , \textbf{ cum virtus faciat habentem bonum ; } et opus bonum , & por que la su entençion es enduzir los otros a uirtud . \textbf{ Et la uirtud faze al que la ha buenon } Esta buean obra conuiene que sea en el gouernamiento derech \\\hline
3.2.34 & esse seruitutem . \textbf{ Cum enim bestiae sint naturae seruilis : } quanto quis magis accedit & e obedesçer alos Reyes es seruidunbre . \textbf{ Ca commo las bestias sean de natura seruil } quanto alguno mas se allega ala natura bestial \\\hline
3.2.34 & et obseruationem legum : \textbf{ cum ex hoc consurgat tantum bonum , } quantum bona est pax & e guardar las leyes . \textbf{ Commo desto se leunate tan grant bien } quanto es la paz e el assessiego de los que son en el regno \\\hline
3.3.2 & quia naturaliter metuunt vulnera . \textbf{ Nam cum naturaliter habeant modicum sanguinis , } naturaliter timent sanguinis amissionem : & naturalmente han miedo de las feridas . \textbf{ Ca por que naturalmente han poca sangre } naturalmente temen de perder la sangre . \\\hline
3.3.2 & quia non horrent sanguinis effusionem , \textbf{ cum assueti sint ad occisionem animalium , } et ad effundendum sanguinem Venatores & por que non aborresçen el derramamiento de la sangre \textbf{ por que son acostunbrados a matar las animalias } e a esparzer la sangre ellas . \\\hline
3.3.4 & quasi non appretiari corporalem vitam . \textbf{ Nam cum tota operatio bellica exposita sit periculis mortis , } nunquam quis est fortis animo & por la iustiçia e por el bien comun . \textbf{ Ca commo toda la hueste sea puesta | a periglos de muerte en la batalla } nunca ninguno es fuerte de coraçon \\\hline
3.3.4 & Finis militaris , est victoria . \textbf{ Sed cum omnis bellica operatio contineatur sub militari , } ut supra ostendebatur , & la fin de la caualleria es victoria e vençer . \textbf{ Mas conmo todas las obras de la batalla sean contenidas so la caualleria } assi commo es mostrado desuso \\\hline
3.3.4 & victoria esse finis . \textbf{ Quare cum maxime contingat bellantes vincere , } si bene sciant se protegere & la uictoria es dicha fin de todas las obras de la batalla . \textbf{ Por la qual cosa commo mayormente contezca a los lidiadores vençer } si bien se sopieren cobrir \\\hline
3.3.5 & quam corporis fortitudo . \textbf{ Quare cum communiter nobiles homines industriores sint rusticis , } sequitur hos meliores esse pugnantes . & que la fortaleza del pueblo . \textbf{ Por la qual cosa commo los nobles omnes sean mas sotiles comunalmente } e mas arteros que los aldeanos \\\hline
3.3.7 & immo dato quod pugnantes \textbf{ se cum hostibus possint coniungere , } antequam coniungantur & prouechosa cosa es lançar las saetas \textbf{ mas puesto que los lidiadores se puedan ayuntar con los enemigos } ante que se apunte con ellos \\\hline
3.3.10 & portare scutum ad se protegendum : \textbf{ et cum debeant esse vigilantes , agiles , sobrii , } habentes armorum experientiam : & para encobrirse meior \textbf{ e avn que ayan los oios bien espiertos | e que sean ligeros e mesurados en beuer e gerrdados de vino } e avn que ayan vso de las armas . \\\hline
3.3.11 & quod et hoc posset , ad aures hostium peruenire \textbf{ Itaque cum pericula visa minus noceant , } per velocissimos equites sunt detegendae insidiae , & que avn este podria venir en las oreias de los enemigos . \textbf{ Et por ende por que los periglos que son ante vistos menos enpeesçen . } por caualleros muy ligeros son de descobrir las çeladas \\\hline
3.3.14 & suos hostes inuadere debeant . \textbf{ Nam cum septem modis enumeratis hostes fortiores existant ; } cum modo opposito se habent , & e en qual manera los lidiadores deuen acometer sus enemigos . \textbf{ Ca commo en las siete maneras contadas | sean los enemigos mas fuertes } quando son las maneras contrarias \\\hline
3.3.16 & nauales dicuntur . \textbf{ Quare cum sint quatuor genera pugnarum , } postquam diximus de campestri , & son dichas batallas nauales e de naues . \textbf{ Por la qual cosa commo sean quatro maneras da batallas } despues que dixiemos de la batalla de la tierra fincanos de dezir de las otras tres . \\\hline
3.3.17 & ab obsessis molestari poterunt . \textbf{ Nam cum contingat obsessiones } per multa aliquando durare tempora , & de los que estan çercados \textbf{ Ca commo contezca } que las çercas puedan durar algunas vezes \\\hline
3.3.19 & trahat se magis prope aedificium illud , \textbf{ si vero visus protendatur magis basse , cum tabula sit existente ad pedes , } et sic iacens in terra , & e si la vista del oio descendiere \textbf{ e fuere mas baxa aluengesse | mas con la tabla en tal manera } que catando \\\hline
3.3.21 & nisi parce , \textbf{ et cum temperamento dispensetur . } Tertio est in talibus attendendum , & Ca non aprouecha nada traer muchas viandas \textbf{ si non fueren partidas con tenpramiento et escassamente . } Lo terçero es de proueer en tales cosas \\\hline
3.3.23 & homines melius esse armatos , quam in terrestri : \textbf{ quia cum pugnatores marini quasi fixi stent in naui , } et quasi modicum se moueant , & meior armados que en la de la tierra \textbf{ por que los lidiadores de la mar esten firmes } e mueuen se muy poco . \\\hline
3.3.23 & quia hoc facto maior habetur commoditas , \textbf{ ut cum ipso percuti possit tam nauis , } quam etiam existentes in ipsa . & ca esto echo siguiesse meior prouecho del \textbf{ ca pueden ferir tan bien en la } naue commo en los que estan en ella \\\hline

\end{tabular}
