\begin{tabular}{|p{1cm}|p{6.5cm}|p{6.5cm}|}

\hline
1.2.13 & nisi imaginetur sibi periculum imminere : \textbf{ nec omnis dicitur audax , } nisi aggrediatur aliquod terribile , et periculosum . & si non quando emagina algunan cosa en que puede auer peligro . \textbf{ nin ninguno non es dicho osado } si non quando acomete alguna cosa espantable e peligrosa . \\\hline
1.2.14 & ut cum aliquis ignorans fortitudinem aduersarii , bellatur . \textbf{ Ut puta si habitantes in septentrione sunt fortes , et audaces , } in meridiano vero sunt debiles , & nin conosçiendo la fortaleza del su contrario . \textbf{ Enxient lo desto . | assi commo si algunos morasen en setenturon } e fuesen fuertes e osados . \\\hline
1.2.15 & et si volumus esse fortes , \textbf{ debemus magis esse audaces , } quam timidi : & Et si nos quisieremos fazer nos fuertes \textbf{ mas auemos aser osados e temerosos . } assi en essa misma manera la tenpranca mas conuiene con el non sentimiento \\\hline
1.3.6 & Oportet ergo videre \textbf{ quo modo eos esse deceat timidos , et audaces . } Timor autem si moderatus sit , & Et pues que assi es conuiene deuer \textbf{ en qual manera conuiene alos Reyes de sertemosos | e de ser osados } por que el temor si fuere tenprado es conuenible alos Reyes e alos prinçipes . \\\hline
1.4.4 & ubi est timendum ; \textbf{ et audaces , } ubi est audendum . & Et por ende son temerosos do lo han de ser \textbf{ e osados do lo han de ser . } ¶ En essa misma manera ahun por que non son del todo \\\hline
2.3.20 & eloquia multiplicare , \textbf{ et homines abundantes vino quasi calefacti et audaces , } libenter in verba prorumpunt , & por que paresçe que el vino acresçienta las fablas \textbf{ e los omes que han beuido vino | assi commo escalentados } fazense osados \\\hline
3.1.7 & nam si consideramus aues ipsas viuentes ex raptu , \textbf{ maiores corpore et audaciores corde } et praestantiores viribus sunt foeminae quam masculi : & que los mallos calicuydaremos en las aues \textbf{ que biuen de rapina mayores son de cuerpo | e mas osadas de coraçon } e mas apareiadas en fuerça son las tenbras \\\hline
3.2.13 & quod nimis fugans timidum , \textbf{ vi compellit esse audacem . } Sic etiam et alia animalia & que quien muncho faze foyr al temeroso \textbf{ por fuerça lo costrange desee oscido en essa misma manera } avn en las otras ainalias \\\hline
3.3.2 & quod in qualibet arte sint \textbf{ aliqui bellicosi et audaces ; } aliqui vero timidi et pusillanimes . & Enpero puede contesçer \textbf{ que en cada vna destas artes son algunos buenos lidiadores e atreuidos } e ay otros temerosos e de flacos coraçones . \\\hline
3.3.3 & ex quibus signis cognosci habeant homines bellicosi . \textbf{ Sciendum igitur viros audaces et cordatos } utiliores esse ad bellum , & finca de ver por quales señales se han de conosçer los buenos lidiadores . \textbf{ Et para esto conuiene de saber | que los omnes osados eatreuidos } e de grandes coraçones son mas prouechosos para la batalla \\\hline
3.3.5 & ( ut ait Philosophus 3 Ethic’ ) \textbf{ quod Hectorem fecit audacem . } Dicebat enim Hector , & que dize el philosofo en el tercero libro de las . \textbf{ Lo que fizo a ector atreuido en las armas e buen lidiador . } Ca dizia ector \\\hline
3.3.9 & tanto citius victoriam obtinent . Sexto , attendenda est virilitas et audacia mentis , \textbf{ quia audaciores et magis cordati } ut plurimum in pugna victoriam obtinent . & et la osadia del coraçon . \textbf{ Ca los mas osados | e de mayores coraçons } por la mayor parte alcançan en la batalla la victoria . \\\hline
3.3.9 & et sagaciores mente . \textbf{ Sexto , qui sunt audaciores , } et viriliores corde . & e mas arteros para lidiar . \textbf{ Lo sexto quales son mas osados } e mas fuertes de coracon . \\\hline
3.3.12 & Tertio , ut extra quamlibet aciem \textbf{ reseruentur aliqui milites strenui et audaces , } qui possint succurrere ad partem illam , & lo tercero que fuera de cada vna de las azes \textbf{ sean guardados algunos estremados caualleros } e osados que puedan acorrer a aquella parte \\\hline
3.3.15 & quod non pateat aliquis aditus fugiendi : \textbf{ quia desperantes quasi necessitate compulsi efficiuntur audaces , } videntes enim se necessario moriendos , & finque algun logar para foyr . \textbf{ Ca estonçe con desesperamiento | assi commo costreñidos por fuerça fazen se mas osados } veyendo que non les finca si non la muerte . \\\hline

\end{tabular}
