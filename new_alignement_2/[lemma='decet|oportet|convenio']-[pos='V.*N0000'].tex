\begin{tabular}{|p{1cm}|p{6.5cm}|p{6.5cm}|}

\hline
1.1.1 & secundum Philosophum est figuralis et grossus : \textbf{ oportet enim in talibus typo } et figuraliter pertransire , & segund el philosofo es figural e gruessa \textbf{ Ca conuiene enlas tales cosas vsar de figuras de enxenplos } Ca los fechos morales \\\hline
1.1.1 & Possumus autem triplici via venari , \textbf{ quod modum procedendi in hac scientia oportet } esse figuralem et grossum . & onde por tres cosas podemos mostrar \textbf{ que La manera que deuemos tener en esta arte | e en esta sçiençia conujene } que sea figural e gruesa , \\\hline
1.1.1 & Immo quia ( secundum Philosophum in Politicis ) \textbf{ quae oportet dominum scire praecipere , } haec oportet subditum scire facere : & Ca Segund dize el philosopho enlas politicas \textbf{ que aquellas cosas | que conujene al Senonr } de saber mandar essas mesmas \\\hline
1.1.1 & quae oportet dominum scire praecipere , \textbf{ haec oportet subditum scire facere : } si per hunc librum instruuntur Principes , & que conujene al Senonr \textbf{ de saber mandar essas mesmas } conujene al subdito \\\hline
1.1.1 & suis subditis imperare , \textbf{ oportet doctrinam hanc extendere usque ad populum , } ut sciat qualiter debeat & E si por qual manera Deuen mandar a los sus Subditos \textbf{ conujene esta doctrina | e esta sçiençia estender la fasta el pueblo } por que Sepa commo ha de obedesçer a sus prinçipes \\\hline
1.1.1 & ( ut tactum est ) nisi per rationes superficiales et sensibiles : \textbf{ oportet modum procedendi in hoc opere , } esse grossum et figuralem . & Conuie ne \textbf{ que la manera que deuemos tener enesta obra sea gruesa e figural e exenplar } a asi commo dize el philosopho \\\hline
1.1.2 & quanta in gubernatione ciuitatis et regni : \textbf{ ordine naturali decet regiam maiestatem } primo scire se ipsum regere , & para gouerna mj̊ de çibdado den rregno \textbf{ Conuiene Segund orden natural ala rreal magestad | primeramente que el Ruy sepa gouernar asy mesmo ¶ } Lo segundo que sepa gouernar su conpanna¶ \\\hline
1.1.2 & volens tractare de regimine sui , \textbf{ oportet ipsum notitiam tradere de omnibus his } quae diuersificant mores et actiones . & el que quiere tractar del gouernaiento \textbf{ e fablarde sy mesmo conujene de tractar | e de dar conosçimiento de todas aquellas cosas } que departen las obras e los fechos de los omes \\\hline
1.1.4 & Hanc autem internam deuotionem \textbf{ tanto magis decet } habere reges et principes , & Mas esta deuoçion de dentro del alma \textbf{ tantomas conviene dela auer los rreys } e los prinçipes \\\hline
1.1.5 & non debetur eis corona . \textbf{ Oportet igitur actu bene agere , } ut per opera nostra mereamur & Enpero si non lidiaren de fech̃o non les es deuida corona ¶ \textbf{ pues que asi es conviene bien fazer de fecho } por que por las nr̃as obras merescamos de auer buena fino buena ventura \\\hline
1.1.5 & sed ut bene agamus , \textbf{ oportet nobis } praestituere finem bonum et debitum : & pues que asi es para \textbf{ que nos bien obremos conuiene nos de estableçer alguna buean fin e conuenible } que todas las nuestras obras toman nasçençia dela fin \\\hline
1.1.5 & magis utique adipiscentur \textbf{ id quod oportet . } Tertio praecognitio finis & mas a cierto tiran aella asy los que conosçen ante la fin \textbf{ mas a çierto se ordenan a bien obrar ¶ } Lo terçero conesçer ante la fin \\\hline
1.1.5 & Patet ergo , \textbf{ quod maxime decet regiam maiestatem } cognoscere suam felicitatem , & pues que asy co asaz paresçe \textbf{ que muy mas conuiene al Reio al prinçipe conosçer la su fin } e la su bien andança \\\hline
1.1.6 & Dictum est enim \textbf{ quod decet Principem esse supra Hominem , } et totaliter diuinum . & Ca dicho n auemos ya desuso \textbf{ que conuiene al Rei | e al prinçipe sier } mas que omne e ser del todo diuinal . \\\hline
1.1.6 & sed qui dormiens , \textbf{ decet Regiam maiestatem } tales delectationes immoderatas fugere , & Mas quando es dormidor \textbf{ e enbriago es de menospreçiar | por ende } ¶Conuienea la real magestad de searedrar de tales delectaçonnes desmesuradas e carnales \\\hline
1.1.6 & ne contemptibilis uideatur . \textbf{ Tertio decet } Principem talia detestari , ne principari efficiatur indignus , & por que non sea menospreçiado de su pueblo¶ \textbf{ La terçera razon por que el prinçipe non deue poner su bien andança en las delecta çonnes corporales es esta } e conuiene al \\\hline
1.1.6 & debet immoderatas voluptates despicere . \textbf{ Quod non decet regiam maiestatem , } Philosophus 1 Politicor’ distinguit & que non es digno de ser prinçipe \textbf{ deue menospreçiar las delecta connes desmesuradas e carnales } philosofo en el primero libro delas politicas \\\hline
1.1.7 & et Magnanimitas sunt maxima bona , \textbf{ et maxime decet regiam maiestatem } esse ornatam talibus virtutibus , & e la grandeza de coraçon son muy grandesbienes \textbf{ los quales bienes deue auer la Real magestad | mucho conuiene al Rei } e ala Real magestad de ser conpuesta e ennobleçida de tales uirtudes \\\hline
1.1.8 & dicit esse fictos , et superficiales . \textbf{ Si ergo maxime decet } Regem esse bonum existentem , & que estos son infintos e superfiçiales . \textbf{ Et pues que assi es si mucho conuiene al Rey de ser bueno uerdaderamente } e non solamente de paresçer bueno . \\\hline
1.1.8 & ne sit iniustus et inaequalis : \textbf{ decet enim Principem } sua bona distribuere & que sea iniusto nin desegual ¶ \textbf{ Mas conuiene le partir los sus bienes } alos sus vassallos \\\hline
1.1.9 & quod exterius bona praetendat . \textbf{ Quare cum Regem deceat } esse totum diuinum , & e muestre algua bondat de fuera \textbf{ por la qual cosa commo al Rey conuenga ser todo diuinal e semeiante a dios } si non es cosa conuenible \\\hline
1.1.9 & honor tamen eos consequitur , \textbf{ et decet eos acceptare honorem sibi exhibitum , } non habentibus Hominibus aliquid maius , & enpero la honrra les parte nesçe a ellos . \textbf{ Et conuiene les alos Reys de resçebir la honrra | que les fazenn los omes } por que los omes non les pueden dar mayor cosa que honrra \\\hline
1.1.11 & et exercere operationes virtutum : \textbf{ decet enim Regem esse magnificum , } beneficiare personas dignas : & e fazer obras de uertudes . \textbf{ E conuiene al Rey de seer magnifico e largo } por que pueda bien fazera las personas dignas \\\hline
1.1.12 & et maxime a materia separatus . \textbf{ Secundo decet Principem } suam felicitatem & e muy alongado de toda materia¶ \textbf{ La segunda razon por que el rey ha de poner la su bien andaça } en dios solo es esta . \\\hline
1.1.12 & et suum praemium expectare ab ipso . \textbf{ Tertio hoc decet Regem , ex eo , } quod est multitudinis rector : & gualardon e merçed \textbf{ ¶La terçera razon por que el Rey ha de poner su bien andança en dios } es por que es gouernador de mucho \\\hline
1.1.12 & ponere suam felicitatem , \textbf{ oportet ipsum huiusmodi felicitatem ponere } in actu illius virtutis , & e la su bien andança en dios . \textbf{ Conuiene le dela poner en la obra de aquella uirtud } por la qual masyna se puede ayuntar con dios . \\\hline
1.1.13 & cum semper amor sit ad similes , et conformes , \textbf{ oportet esse similem , } et conformem Deo , & Et commo el amor sienpre sean los semeiables e acordables con el . \textbf{ Conuiene que aquel que es para de ser } gualardonado de dios \\\hline
1.1.13 & quia si bono communi non vacarent , \textbf{ non oportet eos esse maioris meriti } ex hoc quod transgredi possent , & Ca si non trabaiassen en el bien comun \textbf{ non les conuernia de ser de mayor meresçimiento } por que ellos pueden passar los mandamientos \\\hline
1.2.1 & suam felicitatem debeant ponere , \textbf{ quia non decet } eos suum finem ponere in diuitiis , & e la su bien andança . \textbf{ Et que non los conuiene poner la su fin en riquezas } nin en poderio çiuil \\\hline
1.2.3 & quod cuilibet tribuatur \textbf{ quod decet , } vel quod ei detur & Ca faze que a cada vno sea dado lo qual e conuiene \textbf{ o lo quel deuen dar } o lo que es suyo . \\\hline
1.2.5 & ratiocinari recte et non recte , \textbf{ oportet dare virtutem aliquam , } quae sit recta ratio , & Ca commo contesca de razonar derechamente \textbf{ e non derechamente conuiene de dar alguna uirtud } que sea razon derecha . \\\hline
1.2.5 & Amplius quia contingit nos passionari recte et non recte , \textbf{ oportet dare virtutes aliquas , } per quas modificentur in ipsis passionibus . & e non derecha mente . \textbf{ Conuiene nos de dar uirtudes algunas } por las quales seamos tenprados e reglados en aquellas passiones ¶ \\\hline
1.2.5 & ut passiones irascibiles : \textbf{ circa passiones oportet } dare virtutem aliquam , & assi commo son las passiones dela saña . \textbf{ Conuiene dar alguna uirtud en las passiones } por la qual las passiones non nos pueden mouer \\\hline
1.2.5 & ad id quod ratio vetat : \textbf{ et oportet dare virtutem aliam , } ne passiones retrahant nos ab eo , & nin inclinar a aquelo que uieda la razon ¶ \textbf{ Et otrosi nos conuiene de dar otra uirtud } por la qual las passiones non nos pueden arredrar \\\hline
1.2.7 & restat ostendere , \textbf{ quod decet Reges , } et Principes esse prudentes . & finca de demostrar \textbf{ que conuiene alos Reyes e alos prinçipes de ser prudentes e sabios . } Mas quanto pertenesçe alo presente tres cosas son a \\\hline
1.2.7 & Triplici ergo via inuestigare possumus , \textbf{ quod decet Regem esse prudentem . Primo , quia sine prudentia non est } Rex & Et pues que assi es podemos prouar en tres maneras \textbf{ que conuiene al rey de seer sabio ¶ | La primera es que sin la pradençia non puede seer Rey segunt } uerdatmas solamente lo serie segunt el nonbre \\\hline
1.2.7 & secundum rei veritatem , \textbf{ et ut decet licet forte } sit & niguno non puede ser rey segunt uirdat . \textbf{ Et assi commo conuiene delo ser } maguera que por auentura sea Rey segunt el nonbre . \\\hline
1.2.7 & non solum nomine sed re , \textbf{ decet ipsum habere prudentiam . } Secundo hoc decet eum , & non solamente segunt el nonbre \textbf{ mas segunt el fech̃o | conuiene le de auer sabiduria . } La segunda manera por que conuiene al Rey de ser sabio \\\hline
1.2.7 & decet ipsum habere prudentiam . \textbf{ Secundo hoc decet eum , } ne de facili in tyrannum conuertatur . & conuiene le de auer sabiduria . \textbf{ La segunda manera por que conuiene al Rey de ser sabio } es por que non se torne de ligero en tirano . \\\hline
1.2.7 & qualitercunque possit pecuniam extorquere . \textbf{ Tertio decet Reges , } et Principes habere prudentiam , & si non commo podra sacardes e algo del su pueblo . \textbf{ La terçera manera por que conuiene al Rey de auer sabiduria es } por que sin ella non puede ser señor \\\hline
1.2.8 & si debeat aliquis esse perfecte prudens , \textbf{ oportet ipsum habere omnia } quae concurrunt ad prudentiam , & si alguno ouiere aser sabio conplida mente . \textbf{ Conuienel e de auer todas aquellas cosas } que son necessarias ala sabiduria . \\\hline
1.2.8 & aut Princeps debeat esse prudens , \textbf{ oportet ipsum esse memorem , } prouidum , intelligentem , rationabilem , & La septima experiençia e prueua¶ \textbf{ La viii jncauçion q quiere dezir escogimiento delo meior | e foyr delo peor . } ¶ \\\hline
1.2.8 & ad quae dirigit , \textbf{ oportet Regem esse memorem , et prouidum : } propter modum & aque guiasa pradençia . \textbf{ Conuiene al Rey de ser acordable e prouisor ¶ } Et por razon dela manera segunt laquel guia . \\\hline
1.2.8 & secundum quem dirigit , \textbf{ oportet ipsum esse intelligentem , } et rationabilem : & Et por razon dela manera segunt laquel guia . \textbf{ Conuiene al Rey de ser entendido e razonable ¶ } por razon de la su propia persona \\\hline
1.2.8 & ratione propriae personae \textbf{ quae alios est dirigens , oportet quod sit solers , et docilis : } ratione vero gentis quam dirigit , & por razon de la su propia persona \textbf{ que ha de guiar los otros . | Conuiene le de ser sotil e doctrinable ¶ } Mas por razon dela gente \\\hline
1.2.8 & quia nulli agenti hoc est possibile , \textbf{ sed decet Regem habere praeteritorum memoriam , } ut possit ex praeteritis cognoscere , & Ca esto ninguno non lo pie de fazer . \textbf{ Mas conuiene al Rey de auer memoria delans cosas passadas | por que pue da } por las cosas passadas conosçer e tomar \\\hline
1.2.8 & ut plurimum futura sunt praeteritis similia . \textbf{ Secundo decet ipsum habere prouidentiam futurorum : } quia homines prouidentes futura bona , & que son passadas \textbf{ ¶lo segundo conuiene al Rey de auer | prouisionde las cosas } que han de venir . \\\hline
1.2.8 & ad quae dirigit , \textbf{ oportet Regem esse memorem , } et prouidum : & aque ha de guiar el Rey su pueblo \textbf{ le conuiene de ser acordable e prouisor . } Assi por razon de la manera \\\hline
1.2.8 & sic ratione modi per quem dirigit , \textbf{ oportet ipsum esse intelligentem , et rationalem . } Sed ratione propriae personae & por la qual ha de guiar el pueblo \textbf{ le conuiene de ser entendido e razonable . } Mas por razon dela su persona propia \\\hline
1.2.8 & quae bona sunt regno utilia excogitando , \textbf{ oportet ipsum esse docilem , } aliorum consiliis acquiescendo . & que son aprouechables a su regno \textbf{ ahun conuiene le de ser doctrinable resçebiendo e tomando coseio de bueons } quel han bien de conseiar . \\\hline
1.2.8 & ipsum fugere commouentem . \textbf{ Non enim decet Regem } in omnibus sequi caput suum , & aquel que bien le conseia \textbf{ por la qual cosa non le conuiene al Rey de seguir en todas cosas su cabeça } nin atener se sienpre al su engennio propio . \\\hline
1.2.8 & nec inniti semper solertiae propriae : \textbf{ sed oportet ipsum esse docilem , } ut sit habilis ad capescendam doctrinam aliorum , & nin atener se sienpre al su engennio propio . \textbf{ Mas conuiene le de ser doctrinable | por que sea ido neo } para tomar doctrina de los otros tom̃ado conseio de buenos \\\hline
1.2.8 & quae est alios dirigens , \textbf{ oportet Regem esse solertem , et docilem . } Sed ratione gentis quam dirigit , & que es gouernandor de los otros . \textbf{ Conuiene al rey de ser engennioso e doctrinable . } Mas por razon dela gente \\\hline
1.2.8 & Sed ratione gentis quam dirigit , \textbf{ oportet ipsum esse expertum , } et cautum . & e del pueblo \textbf{ a que ha de gouernar . | Conuiene al Rey } que sea j muy prouado et muy aꝑçebido \\\hline
1.2.10 & ut quod velit habere plus de iis , \textbf{ quam eum deceat : } ex hoc infertur nocumentum aliis ciuibus : & assi conmosi quisiere auer mas de aquellos bienes \textbf{ de quanto le conuiene auer } por esta razon viene danno alos otros çibdadanos \\\hline
1.2.12 & ut sint Reges . \textbf{ Cum enim deceat regulam esse rectam et aequalem , } Rex quia est quaedam animata lex , & enpero non son dignos de seer Reyes . \textbf{ Ca commo conuenga ala regla de ser derecha } e egual e el Rey sea vna ley animada e vna regla . \\\hline
1.2.12 & quae est quaedam clarissima virtus , probari potest , \textbf{ quod decet eos obseruare Iustitiam . } Tertio hoc probari potest & que se puede prouar \textbf{ que conuiene alos Reyes | de guardar la iustiçia . } lo terçero esso mismo se puede prouar \\\hline
1.2.12 & in principatu aliquorum , \textbf{ quia oportet , } quod bonitas sua ad alios se extendat , & Mas quando es puesto en algun prinçipado o en algun sennorio \textbf{ por que la su bondat se ha de estender a otros } estonçe meior paresçe quales si es bueno o malo \\\hline
1.2.12 & quam ex aliis virtutibus moralibus . \textbf{ Decet ergo Reges et Principes esse iustos , } tum quia debent esse regula agendorum , & que a ninguno de los otros . \textbf{ Et pues que assi es conuiene alos Reyes | e alos prinçipes de ser iustolo vno } por que deue ser regla de todas las cosas \\\hline
1.2.12 & tum etiam quia ex ea manifestatur perfectio bonitatis . \textbf{ Quarto hoc decet Reges , } et Principes ex magnitudine malitiae , & por que en la iustiçia es manifestada la perfeccion dela su bondat \textbf{ ¶La quarta manera por que podemos prouar esso mismo | que conienea los Reyes } e alos prinçipes de guardar iustiçia \\\hline
1.2.13 & et bene agere , \textbf{ oportet dare virtutem aliquam , } per quam regulentur in agendo . & e pecar en obrando . \textbf{ Conuiene de dar e de ponetur algua uirtud } por la qual seamos reglados en las obras \\\hline
1.2.15 & Nam si volumus nutrimentales delectationes reprimere , \textbf{ oportet nos temperari a potu , et cibo . } Temperando nos a potu , & delectaçonnes nutermentales \textbf{ con que se cera el cuerpo . | Conuiene nos que seamos tenprados en comer e en beuer } ca tenprando nos en el beuer seremos mesurados \\\hline
1.2.17 & de his quatuor virtutibus , \textbf{ et ostendimus quomodo Reges et Principes illis virtutibus decet } esse ornatos . & por ende despues que dixiemos destas quatro uirtudes \textbf{ e mostramos commo los Reyes | e los prinçipes deuen ser conpuestos e honrrados } dellas fincanos \\\hline
1.2.17 & contra rectam regulam rationis , \textbf{ oportet dare virtutem aliquam mediam } inter auaritiam , et prodigalitatem : & escontra regla derecha de razon e de entendimiento . \textbf{ Conuiene de dar alguna uirtud medianera } entre la auariçia e el gastamiento . \\\hline
1.2.17 & lucratur enim ab amicis , \textbf{ quibus oportet bene facere . } Est igitur liberalitas & nin libal por que gana de los \textbf{ amigosa los quales le conuenia bien fazer . } Et por ende la franqueza es en non tomar \\\hline
1.2.18 & quod minora faciunt , \textbf{ quam deceat . } Ex hoc autem apparere potest & que menos dan de quanto les conuiene ᷤ dar \textbf{ Et menos fazen de quanto les conuiene de fazer . } Et desto puede bien paresçer \\\hline
1.2.18 & quae continet . \textbf{ Cum ergo tanto deceat fontem habere os largius , } quanto ex eo plures participare debent : & Ca ha . manera daua so ancho e largo e da conplidamente lo que tiene \textbf{ ¶pues que assi es conmo tanto conuenga ala fuente auer la boca | mas ancha } quanto della deuen \\\hline
1.2.18 & quanto ex eo plures participare debent : \textbf{ tanto decet Regem largiorem esse , } quanto influentia eius ad plures extendenda est , & mas omes tomar e sacar . \textbf{ tanto conuiene al Rey de ser mas largo } quanto la su magnifiçençia \\\hline
1.2.18 & qui sunt in Regno , \textbf{ maxime decet eos liberales esse . } Spectat autem ad liberalem primo & los que son en el su regno \textbf{ mucho les conuiene de ser liberales e francos } Mas par tenesce al libal e alstan ço de catar tres cosas ¶ \\\hline
1.2.18 & Secundo debet respicere quibus det , \textbf{ ut non det quibus non oportet . } Tertio videndum est cuius gratia det , & Lo segundo deue catar aqui lo da \textbf{ por que non de | aquien non deue dar ¶ } Lo terçero deue veer \\\hline
1.2.18 & quam si expendat \textbf{ ubi non oportet . } Deuiant autem a liberalitate Reges , & que si espendiere \textbf{ do non le conuiene espender } Mas los Reyes e los prinçipes de suranse \\\hline
1.2.18 & vel aliis , \textbf{ quibus non oportet dare : } quia magis deceret & e a otros semeiantes \textbf{ a quien non conuiene de dar . } por que aquestos tales mas les conuiene de ser pobres \\\hline
1.2.18 & quibus non oportet dare : \textbf{ quia magis deceret } eos esse pauperes , & a quien non conuiene de dar . \textbf{ por que aquestos tales mas les conuiene de ser pobres } que non ser ricos . \\\hline
1.2.18 & Sic etiam dant \textbf{ cuius gratia non oportet . } Non enim dant boni gratia , & que non ser ricos . \textbf{ Et en essa misma manera dan non por la razon que deuen dar . } Ca non lo dan por razon de bien \\\hline
1.2.18 & vel propter aliquam aliam causam . \textbf{ Decet igitur Reges esse liberales : } et ut liberales sint , & e alos prinçipes de ser liƀͣales . \textbf{ Et para ser libales conuiene les de bien fazer alos buenos } e por razon devien non \\\hline
1.2.19 & quam quomodo faciat eas sophysticas et apparentes . \textbf{ Sic etiam decet magnificum , nuptias , et militias , } et talia quae raro occurrunt , & nin tan firmes en si . \textbf{ En essa misma manera conuiene al magnifico de fazer muy | honrradamente las sus bodas e las sus cauallerias } e aquellas cosas \\\hline
1.2.20 & Regem esse paruificum . \textbf{ Quod autem deceat } ipsum esse magnificum , & en todas maneras es de denostar \textbf{ que el Rey sea periufico mas que conuengaal Rey de ser magnifico } e de fazer grandes espenssas \\\hline
1.2.20 & distribuere bona regni , \textbf{ omnino decet eum magnifice se habere erga personas dignas , } quibus digne competunt illa bona . & prinçipalmente partir los bienes del regno \textbf{ en todas maneras le conuiene ael de se auer grande | e honrradamente a aquellas personas } que son dignas \\\hline
1.2.21 & tangit sex proprietates magnifici , \textbf{ quas habere decet Reges et Principes . } Prima proprietas est , & euedes saber que el philosofo enl quarto libro delas ethicas capitulo dela magnificençia pone seys \textbf{ propiedadesdel magnifico las quales conuiene alos Reyes e alos prinçipes auer ¶ } La primera es que el magnifico es semeiante al sabio \\\hline
1.2.21 & et quomodo dona illa sint magna et decentia , \textbf{ quam quot et quanta numismata oporteat } ipsum consumere propter huiusmodi opera . & mas deue entender en qual manera aquel tenplo o aquella eglesia sera marauillosa e muy fermosa \textbf{ e en qual manera aquellos dones sean grandes e conuenibles que entender e cuydar quantos des } e quanto auer le conuiene ael de despender en estas obras ¶ \\\hline
1.2.21 & et diuitiis , \textbf{ tanto magis decet } eos ampliores retributiones facere , & e en riquezas \textbf{ tantomas les conuiene aellos de fazer mayores particonnes e mayores dones } e mas espender delectable ment en sin detenimiento \\\hline
1.2.21 & quam quomodo parcant nummis et expensis . \textbf{ Oportet etiam eos esse excellenter liberales , } et semper facere magnifica opera . & para las non fazer \textbf{ Otrosi avn conuiene alos Reyes | e alos prinçipes de ser liberales muy altamente } e de fazer sienpre obras muy grandes e magnificas . \\\hline
1.2.21 & Quare quanto est nobilior aliis , \textbf{ tanto decet ipsum pollere magnificentia , } et habere proprietates magnifici . & que los otros en tanto \textbf{ mas le conuiene ael de resplandesçer } por magnificençia e auer propiedades de magnifico \\\hline
1.2.23 & talia autem non multotiens occurrunt , \textbf{ ideo decet magnanimum esse paucorum operatiuum . } Quarto decet magnanimum esse apertum , & Et tales cosas commo estas non contesçen muchas vezes . \textbf{ Et por ende conuiene al magnanimo ser de pocas obras ¶ } La quarta propiedat es que conuiene al maguanimo ser magnifiesto \\\hline
1.2.23 & ideo decet magnanimum esse paucorum operatiuum . \textbf{ Quarto decet magnanimum esse apertum , } ut sit veridicus , & Et por ende conuiene al magnanimo ser de pocas obras ¶ \textbf{ La quarta propiedat es que conuiene al maguanimo ser magnifiesto } assi que sea uerdadero \\\hline
1.2.23 & si bene intelligantur , \textbf{ habere decet Reges , et Principes . } Decet enim Reges se non exponere & si bien le entendieren \textbf{ conuiene alos Reyes | e alos prinçipes delas auer } Por que conuiene alos Reyes non se poner \\\hline
1.2.23 & habere decet Reges , et Principes . \textbf{ Decet enim Reges se non exponere } pro quibuscunque periculis , & e alos prinçipes delas auer \textbf{ Por que conuiene alos Reyes non se poner } a quales se quier periglos \\\hline
1.2.23 & et in retributionibus debent alios superare . \textbf{ Tertio decet eos esse paucorum operatiuos : } negocia enim ardua pauca sunt respectu aliorum . & e enpartimientos de gualardones ¶ \textbf{ Lo terçero çonuiene alos Reyes de seer de pocas obras } A por que los negoçios muy altos son pocos \\\hline
1.2.23 & quantumcumque modica expedire per seipsos , \textbf{ nec decet eos omnium esse operatiuos ; } sed ut possit liberius intendere expeditioni negociorum magnorum & mayormente los que son pequa nons \textbf{ nin conuiene aellos de seer obradores de todas las cosas } mas por que puedan mas liberalmente entender a desenbargar los grandes negoçios que son pocos deuena comne dar los otros negoçios \\\hline
1.2.23 & et falsificari non debet . \textbf{ Decet etiam eos esse manifestos oditores , et amatores , } ut manifeste odiant vitia , & La qual regla non se deue torcer nin falssar \textbf{ Et ahun conuiene les de seer manifiestos aborresçedores e amadores } por que manifiesta miente \\\hline
1.2.23 & ubi agetur de regimine Regni , \textbf{ non decet eos esse plangitiuos , } vel deprecatiuos pro exterioribus bonis . & o determinaremos del gouernamiento del regno \textbf{ que non conuiene alos Reyes de seer lloradores nin rogadores } por los bienes de fuera . \\\hline
1.2.23 & Omnes ergo assignatae proprietates competere debent Regibus et Principibus . \textbf{ Quare decet eos esse magnanimos . } Amatores honorum aliquando vituperantur , & deuen part enesçer alos Reyes \textbf{ e alos prinçipes . | por la qual cosa les conuienea ellos de seer magnanimos . ¶ } euedes saber que los amadores de las honrras \\\hline
1.2.24 & et Principes esse magnificos , et liberales : \textbf{ sic decet eos esse magnanimos , } et honoris amatiuos . Reges enim et Principes decet honores diligere modo quo dictum est ; & e alos prinçipes de seer magnificos e liberales \textbf{ assi en essa misma manera les conuiene de seer magranimos } e amado res de honrra . \\\hline
1.2.24 & sic decet eos esse magnanimos , \textbf{ et honoris amatiuos . Reges enim et Principes decet honores diligere modo quo dictum est ; } videlicet , ut diligant et cupiant facere opera , & e amado res de honrra . \textbf{ Conuiene alos Reyes | e alos prinçipes amar las honrras } en la manera que dich̃ones de suso . \\\hline
1.2.24 & bene dictum est , \textbf{ quod decet eos esse magnanimos , } et honoris amatiuos . & Bien dicho es \textbf{ que a ellos pertenesce seer magnanimos } e amadores de honrra \\\hline
1.2.25 & si unum et idem aliter et aliter acceptum nos retrahit et impellit , \textbf{ oportebit circa illud dare duas virtutes , } unam impellentem , & e nos allega a aquello que la razon manda o uieda . \textbf{ Conuiene de dar en aquella cosa dos uirtudes ¶ La vna que nos allegue . } Et la otra qua nos arriedre dello . \\\hline
1.2.26 & circa moderationem deiectionis : \textbf{ restat videre quod decet Reges } et Principes esse humiles , & e despues desto cerca la tenprança del despreçiamiento e del abaxamiento . \textbf{ finca de ver | que conuiene alos Reyes } e alos prinçipes ser humildosos \\\hline
1.2.26 & et Principes esse magnanimos , \textbf{ decet eos esse humiles . } Debent enim Reges & e alos prinçipes de seer magranimos \textbf{ conuiene les de ser humildosos . } Por que conuiene alos Reyes \\\hline
1.2.26 & et ut videantur excellere . \textbf{ Secundo decet eos esse humiles ratione operum fiendorum . } Nam superbus quaerens suam excellentiam ultra quam debeat , & ca de honrra¶ \textbf{ Lo segundo conuiene alos Reyes de ser humildosos | por razon delas obras } que han de fazer . \\\hline
1.2.26 & ad ea quae proprias vires excellunt . \textbf{ Ideo decet homines esse humiles , } ut considerato proprio defectu & de quanto puede el su poder . \textbf{ Et por ende conuiene alos omes de ser humildosos } por que cuydando el en su fallescimiento propio \\\hline
1.2.27 & contingit superabundare et deficere : \textbf{ oportet ibi dare virtutem } aliquam reprimentem superabundantias , & e uenganças del contesçe de sobrepiuar e de fallesçer . \textbf{ Conuiene de dar y alguna uirtud } que reprima las sobrepuianças \\\hline
1.2.27 & mansuetudo nominat temperamentum irae . \textbf{ Quod autem deceat Reges et Principes esse mansuetos , } ostendere non est difficile . & La manssedunbre nonbra tenpramiento de sana . \textbf{ mas mostrar que conuiene alos Reyes | e alos prinçipes de ser manssos } esto non es cosa guaue mas ligera . \\\hline
1.2.27 & Nam cum ira peruertat iudicium rationis , \textbf{ non decet Reges et Principes esse iracundos , } cum in eis maxime vigere debeat ratio et intellectus . Sicut enim videmus & tristorna el iuyzio dela razon \textbf{ e del entendimiento non conuiene alos Reyes | et alos prinçipes de seer sannudos } por que en ellos mayormente deue seer apoderada la razon e el entendemiento \\\hline
1.2.27 & cum hoc faciat mansuetudo , \textbf{ decet eos mansuetos esse . } Ut postulat praesens negocium , & Et commo esto faga la manssedunbre \textbf{ conuiene a ellos de ser manssos } egund que demanda este presente negoçio suficientemente dixiemos de las uirtudes \\\hline
1.2.28 & circa quam contingit abundare et deficere , \textbf{ oportet dare uirtutem } aliquam reprimentem superabundantias , & cerca la qual contesçe de sobrepuiar e de fallesçer . \textbf{ Conuiene de dar uirtud alguna } que reprima las sobrepuianças \\\hline
1.2.28 & dando cautelas Regum et Principum , \textbf{ ait , quod decet Reges et Principes } apparere personas reuerendas , & e alos prinçipes dize \textbf{ que conuiene alos Reyes | e alos prinçipes de paresçer } perssonas reuerendas a quien deuen fazer reuerençia \\\hline
1.2.29 & idest irrisores , et despectores . \textbf{ Oportet ergo dare aliquam virtutem mediam , } per quam moderentur diminuta , & que quiere dezir escarnidores e despreçiadores dessi mismos . \textbf{ Et pues que assi es conuiene de dar alguna uirtud medianera } por la qual sean tenpradas las cosas menguadas \\\hline
1.2.29 & de leui patet \textbf{ quod decet Reges } et Principes esse veraces . & e cerca quales cosas ha de seer de sigero paresçe que conuiene alos Reyes \textbf{ e alos prinçipes de seer uerdaderos } por que aquellos que se inclinan notablemente \\\hline
1.2.29 & sed gratiosos et amabiles : \textbf{ decet eos non esse iactatores vel derisores , } sed apertos et veraces , & nin guaues mas guatiosos e amables . \textbf{ Conuiene a ellos de non seer alabadores de ssi mismos } nin escarnidores dessi mas manifiestos \\\hline
1.2.29 & vel promittendo aliis maiora quam faciant . \textbf{ Immo tanto magis decet Reges et Principes cauere iactantiam , } quanto plures habent incitantes ipsos ad iactantiam , & nin prometiendo alos otros mayores cosas que faran . \textbf{ Mas por tanto conuiene alos Reyes | e alos prinçipes de escusar } e de foyr el alabança \\\hline
1.2.30 & superfluitates ludorum , \textbf{ magis hoc decet Reges et Principes : } immo oportet Reges , & de repremir las sobeianias de los iuegos mucho \textbf{ mas esto conuiene alos Reyes e alos prinçipes en tanto vsar tenpradamente delas delecta connes delos iuegos } que si esto feziesen algunas ottas personas comunes paresçeria \\\hline
1.2.30 & magis hoc decet Reges et Principes : \textbf{ immo oportet Reges , } et Principes adeo moderate & de repremir las sobeianias de los iuegos mucho \textbf{ mas esto conuiene alos Reyes e alos prinçipes en tanto vsar tenpradamente delas delecta connes delos iuegos } que si esto feziesen algunas ottas personas comunes paresçeria \\\hline
1.2.31 & quia ( ut plane patet ) \textbf{ oportet eos esse prudentes et iustos : } omnino manifestum esse debet , & Ca assi commo \textbf{ parescellanamente conuiene a ellos de seer pradentes e sabios e iustos . } Et por ende en todo en todo deue ser manifiesto \\\hline
1.2.31 & Virtutes autem naturales et imperfectae , \textbf{ non oportet esse connexas . } Videmus enim aliquos naturaliter habere & e non conplidas \textbf{ non conuiene de ser conexas | nin ayuntadas vna a otra } por que veemos alguons naturalmente auer alguna n industria \\\hline
1.2.31 & sed complete et perfecte nullatenus fieri potest . \textbf{ Quare sic decet Reges , } et Principes esse quasi semideos , & sin todas las otras . \textbf{ Por la qual cosa si conuiene alos Reyes e alos prinçipes de ser } assi commo medios dioses \\\hline
1.2.31 & et habere virtutes perfectas : \textbf{ decet eos habere omnes virtutes , } quia perfecte una virtus sine aliis haberi non potest . Immo expedit Regibus et Principibus , & e auer las uirtudesacabadas . \textbf{ Conuiene a ellos de auer todas las uirtudes } por que acabadamente vna uirtud \\\hline
1.2.32 & et quod non sint nec molles , nec incontinentes , nec intemperati , nec bestiales , \textbf{ sed oportet eos esse in summo gradu bonorum : } qui enim aliis dominari , & nin destenprados nin bestiales . \textbf{ Mas conuiene aellos de ser | en el mas alto grado de los buenos } por que aquel que dessea de prinçipar \\\hline
1.2.32 & In hoc ergo gradu debent esse Reges et Principes . \textbf{ Et si in tale gradu esse decet bonos et perfectos Principes seculares } quales esse debeant & e alos prinçipes de ser bueons . \textbf{ Et si en tal grado de buenos | conuiene alos prinçipes seglares de ser buenos } e acabados \\\hline
1.2.33 & continentes , et temperatos : \textbf{ sed decet eos quodammodo esse diuinos . } Virtutes ergo competentes eis , & e continentes \textbf{ e tenprados mas conuienel es de ser | en algunan manera diuinales } Et por ende las uirtudes que parten esten aellos \\\hline
1.2.33 & et Principes alios excellere debent , \textbf{ tanto ardentius decet eos diuinam gratiam postulare . } In hoc ergo eliditur Philosophorum elatio , & e los prinçipes deuen sobrepular los otros en bondat tanto mas cobdiçiosamente \textbf{ e con mayor desseo deuen demandar la gera de dios . | ¶ Et pues que assi es por esto se puede tirar } e quebrantar el orgullo \\\hline
1.2.34 & bonas dispositiones mentis habere , \textbf{ decet eos haec genera dispositionum cognoscere . } TERTIA PARS Primi Libri de regimine Principum : & por que los Reyes e los prinçipes puedan auer estas bueans disposiconnes del alma \textbf{ conuiene aellos de conosçer estos linages } e estas maneras destas lueans disposiciones . \\\hline
1.3.1 & in quo Reges et Principes suum finem ponere debeant , \textbf{ et quomodo oportet } eos virtuosos esse . & e los prinçipes poner su fin e su bien andança . \textbf{ Et otrosi mostrado es en commo les conuiene de ser uirtuosos } ¶ Agora finca de dezir dela tercera parte deste libro \\\hline
1.3.3 & quomodo nos habere debeamus ad illas . \textbf{ Oportebat ergo enumerare omnes passiones , } ut sciremus numerum passionum , & en qual manera nos deuemos auer a aquellas passiones \textbf{ Et por ende conuena de contar tondas las passiones } por que sopiessemos el cuento dellas delas \\\hline
1.3.3 & de quibus determinare debemus . \textbf{ Oportebat etiam ostendere ordinem earum , } ut sciremus quo ordine determinaremus de illis . & quales auemos de determinar e de dezir . \textbf{ ¶ Otrosi conuenia avn demostrar la orden dellas } por que sopiessemos \\\hline
1.3.3 & ad Rempublicam fecit Romam esse principantem et monarcham . \textbf{ Hoc ergo modo quoslibet homines decet esse amatiuos , } ut primo et principaliter diligant & e auer sennorio en todo el mundo . \textbf{ Pues que assi es que esto conuiene a todos los omes de ser amadores } assi que primero e prinçipalmente amen el bien diuinal \\\hline
1.3.3 & si considerentur virtutes , \textbf{ quibus decet Reges esse ornatos . } Sicut enim detestabilius est & ¶Lo segundo esto mesmo se praeua assi si pensaremos las uirtudes \textbf{ por las quales conuiene alos reyes de ser honrrados . } Ca assi commo es \\\hline
1.3.3 & Considerando ergo virtutes , \textbf{ quibus decet Reges } et Principes esse ornatos , & Et pues que assi espenssando las uirtudes \textbf{ por las quales deuen ser los Reyes honrrados } prinçipalmente deuen ellos amar el bien diuinal \\\hline
1.3.4 & intenditur sanitas corporis : \textbf{ naturaliter decet Reges et Principes } intendere & prinçipalmente es entendida la salud del cuerpo natural . \textbf{ Conuiene alos Reyes e alos prinçipes entender e amar } prinçipalmente el bien del regno e el bien comun . \\\hline
1.3.5 & Dicebatur enim supra , \textbf{ quod eos esse decet humiles et magnanimos : } cum ergo humilitas moderet spem , & Ca dixiemos de suso \textbf{ que conuenia alos Reyes de ser humildosos e de ser mag̃nimos . } Et por ende por que la humildat tienpra la esperança \\\hline
1.3.5 & Possumus autem quadrupliciter ostendere , \textbf{ quod deces Reges et Principes decenter se habere circa spem , } et sperare speranda , & Mas nos podemos mostrar en quatro maneras \textbf{ que conuiene alos Reyes | e alos prinçipes de se auer } conueniblemente cerca la esperança \\\hline
1.3.5 & propter quae arguere possumus , \textbf{ quod decet Reges et Principes esse bene sperantes . } Spes enim primo est de bono : & por las quales podemos mostrar \textbf{ que conuiene alos Reyes e alos prinçipes de ser bien esperantes ¶ } Lo primero que la espança es de algun bien \\\hline
1.3.5 & et Principes tendere in bonum , \textbf{ sed etiam decet eos tendere in bonum arduum . } Amplius quanto maior est communitas , & e alos prinçipes de entender en el bien \textbf{ Mas avn les conuiene de entender | en bien alto e grande e guaue de fazer De mas desto } quanto mayor es la comunidat \\\hline
1.3.5 & in non sperando non speranda . \textbf{ Decet enim eos } cum magna diligentia inuestigare , & que non deuen esparar \textbf{ Ca conuiene alos Reyes de cuydar } e escodinar con grand diligençia \\\hline
1.3.5 & Possumus autem duplici via inuestigare , \textbf{ quod decet Reges et Principes aliquid aggredi ultra vires , } et sperare ultra quam sit sperandum . & por dos maneras \textbf{ que non conuiene alos Reyes | e alos prinçipes de acometer ninguna cosa } mas que la fuerca suya \\\hline
1.3.6 & ut dicitur 1 Physicorum , \textbf{ deo in hoc primo de moribus Regum oportet } pertransire uniuersaliter typo : & en el primero libro de los fisicos . \textbf{ por ende en este primero libro | conuiene de tractar delas costunbres de lons Reyes } uniuersalmente \\\hline
1.3.6 & non est fortis , sed satuus . \textbf{ Oportet ergo videre } quo modo eos esse deceat timidos , et audaces . & en el primero libͤ de los grandes morales . \textbf{ Et pues que assi es conuiene deuer } en qual manera conuiene alos Reyes de sertemosos \\\hline
1.3.6 & Oportet ergo videre \textbf{ quo modo eos esse deceat timidos , et audaces . } Timor autem si moderatus sit , & Et pues que assi es conuiene deuer \textbf{ en qual manera conuiene alos Reyes de sertemosos | e de ser osados } por que el temor si fuere tenprado es conuenible alos Reyes e alos prinçipes . \\\hline
1.3.6 & Ostensum est ergo , \textbf{ quod decet Reges , } et Principes moderatum habere timorem . & por las quales queremos foyr de aquel temor \textbf{ Et por ende mostrado es que los Reyes e los prinçipes deuen auer temor tenprado . } Enpero temer destenpradamente en ninguna manera \\\hline
1.3.6 & quare si hoc est indecens , \textbf{ non decet } Regem immoderato timore timere . & si esto es cosa desconuenible \textbf{ non conuiene alos Reyes de temer } e por temor deste prado \\\hline
1.3.6 & quomodo se habere debeant \textbf{ ad audacias decet } enim eos non habere audaciam immoderatam , & en el capitulo dela fortaleza de ligero puede paresçer \textbf{ en qual manera se deuen auer los Reyes ala osadia . } Ca conuiene aellos de non auer osadia destenprada \\\hline
1.3.7 & uniuersaliter omnes fures : \textbf{ non tamen oportet , } quod tristitia committetur huiusmodi odium . & Ca nos podemos natanlmente querer mal a todos los ladrones . \textbf{ pero non conuiene de temer quetsteza se aconpanne a esta mal querençia . } ¶ La septima diferençia es \\\hline
1.3.7 & a Regibus , et Princibus , \textbf{ quia eos maxime decet sequi imperium rationis . } Cauenda est ergo ira inordinata , & e alos prinçipes \textbf{ por que mucho mas conuiene aellos | de segnir el iuyzio dela razon e del entendimiento . } Et pues que assi es paresçe \\\hline
1.3.7 & circa iram et mansuetudinem , \textbf{ tanto magis decet Reges et Principes } quanto magis decet & si assi se deuen auer los omes çerca dela sanna e dela manssedunbre . \textbf{ tanto mas conuiene a los Reyes | e alos prinçipes de se auer } assi commo dicho es \\\hline
1.3.9 & Sed cum ex passionibus diuersificari habeant opera nostra , \textbf{ decet nos diligenter intendere , } in quibus delectemur , et tristemur , & por estas passiones . \textbf{ Cconuiene a nos de acuçiosamente entender } en quales cosas nos deuemos delec tar \\\hline
1.3.10 & diuersificare habent omnes operationes nostras , \textbf{ decet nos omnes eas cognoscere ; } et tanto magis hoc decet Reges et Principes , & Et pues que assi es si todas estas passiones han de partir \textbf{ todasnr̃as obras conuiene a nos delas cognosçer todas . } Et tanto mas esta conuiene alos Reyes \\\hline
1.3.11 & imitari debent . \textbf{ Decet enim eos esse gratiosos , et misericordes . } Ipsi enim maxime esse debent & en quanto son passiones de loar \textbf{ Por que conuiene a ellos de ser guaçiosos e mis cordiosos . } Ca ellos deuen ser muy conuenibles partidores de los bienes \\\hline
1.3.11 & sunt Regibus et Principibus . \textbf{ Non decet enim Reges verecundos esse , } quia non decet & nin en toda manera son de auer alos Reyes e alos prinçipes . \textbf{ Ca non les conuiene alos Reyes | e alos prinçipes de ser uergonçosos } por que non les conuiene a ellos de obrar tales obras \\\hline
1.3.11 & Non decet enim Reges verecundos esse , \textbf{ quia non decet } eos talia operari & e alos prinçipes de ser uergonçosos \textbf{ por que non les conuiene a ellos de obrar tales obras } donde puedan resçebir uirguença \\\hline
1.3.11 & quia non existimamus \textbf{ ipsum oportere operari , } in quibus est verecundia . & Ca non cuydamos \textbf{ que conuenga aellos de obrar ninguna cosa } en que caya uerguença . \\\hline
1.3.11 & eius autem non est praua operari . \textbf{ Quare si decet Reges esse studiosos , } et esse senes moribus , & por que la uerguença es delas cosas malas . Mas al estudioso non conuiene obrar ningunas cosas malas . \textbf{ Por la qual cosa si conuiene alos Reyes | e alos prinçipes de ser estudiosos } e de ser uieios en constunbres \\\hline
1.3.11 & et esse senes moribus , \textbf{ non decet ipsos verecundari , } nisi ex suppositione : & e de ser uieios en constunbres \textbf{ non les conuiene a ellos de ser uer gonçosos } si non por alguna condiçion . \\\hline
1.4.1 & adaptare possumus Regibus et Principibus : \textbf{ quia decet eos esse liberales , } bonae spei , & e alos prinçipes \textbf{ por que conuiene aellos de ser liberales } e de buena esperança . \\\hline
1.4.1 & videlicet , esse verecundos , \textbf{ non decet simpliciter competere Regibus et Principibus . Decet enim Reges et Principes esse liberales : } quia contra naturam agerent , & Esta non conuiene nin pertenesçe \textbf{ por si alos Reyes . | Ca conuiene alos Reyes de ser liberales } por que farien contra natura \\\hline
1.4.1 & maxime magnanimitas competit Regibus et Principibus , \textbf{ quia eos maxime magna decet operari , } et in ardua tendere . & e alos prinçipes la magernimidat \textbf{ que alos otros | por que mucho conuiene a ellos de obrar grandes cosas } e entender cerca las cosas altas \\\hline
1.4.2 & ergo quia aliorum debent esse regula et mensura , \textbf{ potissime eos decet moderatos esse . } Enumeratis moribus iuuenum , & Et pues que assi es que los Reyes deuen ser regla \textbf{ e mesura de todos los otros mucho les conuiene aellos de ser mas mesurados que los otros . } ontadas las costunbres de los mançebos \\\hline
1.4.3 & ut pueri : \textbf{ non tamen decet } eos esse incredulos , & assi conmo los moços . \textbf{ Enpero non les conuiene aellos de ser incredulos } e non creyentes del todo . \\\hline
1.4.3 & secundum dictamen et ordinem rationis . \textbf{ Secundo non decet } eos esse suspitiosos , & e segunt iuzio de entendemiento ¶ \textbf{ Lo segundo non conuiene alos Reyes de ser sospechosos } assi que to das las cosas retuercan ala peor parte \\\hline
1.4.3 & et incurrerent maliuolentiam subditorum . \textbf{ Tertio non decet eos esse timidos et pusillanimes , } immo fortes et magnanimos : & e caerien en malquerençia de los sus subditos ¶ \textbf{ Lo terçero non conuiene a ellos de ser tem̃osos e de flacos coraçones } mas conuiene les de ser fuertes e de grandes coraçones \\\hline
1.4.3 & sint magna et ardua , \textbf{ oportet eos esse fortes et magnanimos . } Quarto detestabile est & e los prinçipes se deuen trabaiar son grandes e altos \textbf{ por ende conuiene les aellos de ser fuertes | e de grandes coraçones } ¶ \\\hline
1.4.3 & magnifica faciendo . \textbf{ Quinto oportet eos esse bonae spei : } quia si in omnibus se crederent deficere , & e granados fazie do grandes cosas ¶ \textbf{ Lo quinto con uiene a ellos de ser de buena elperança . } Ca si en todas las cosas creyessen \\\hline
1.4.3 & et periclitaretur regnum . \textbf{ Sexto non decet } eos esse inuerecundos & e assi el regno serie en periglo ¶ \textbf{ Lo sexto non conuiene aellos de ser desuergonçados } en aquella manera \\\hline
1.4.3 & quam circa opera honore digna . \textbf{ Non decet tamen eos verecundari : } quia indecens est ipsos operari turpia , & que son dignas de honrra . \textbf{ Ca por esto son denostados los uieios | e assi non les conuiene aellos de auer uerguenna } por que non les conuiene de obrar cosas torꝑes \\\hline
1.4.4 & ( secundum quod huiusmodi sunt ) \textbf{ eos habere decet . } Nam cum Reges , et Principes magis debeant & que son de loar en los vieios \textbf{ en quanto talos costunbres son de loar . } Ca commo los Reyes \\\hline
1.4.4 & viuere ratione quam passione , \textbf{ decet eos habere concupiscentias temperatas : } quia ( ut supra dicebatur ) & por razon que por passion dela carne \textbf{ conuiene les aellos | de auer las cobdiçias tenpdas . } Ca assy commo es dicho desuso las cobdiçias \\\hline
1.4.4 & rationem percutiunt . \textbf{ Decet etiam eos esse miseratiuos , } non propter defectum , & e confonden el entendemiento . \textbf{ Otrosy les conuiene de ser misconiosos non por fallesçimiento nin por llaqueza de } coraçon quales en los vieios . \\\hline
1.4.5 & Unde Philos’ 4 Eth’ ait , \textbf{ quod magnanimos et magnificos decet } esse nobiles et gloriosos . & Onde el philosofo dize en el quarto libro de la rectorica \textbf{ que conuiene de ser los nobles magranimos | e de grandes coraçones e magnificos } e de grandes fechos e głiosos e much̃ honrrados \\\hline
1.4.5 & subtiliter inuestigantes \textbf{ quid decet eos facere , } ne opera eorum , & e escodrinnadores sotilmente de todo aquello \textbf{ que les conuiene de fazer } por que las sus obras \\\hline
1.4.5 & cum de virtute tractauimus , \textbf{ quomodo decet Reges , } et Principes esse magnanimos , quomodo magnificos , & quando tractamos delas uirtudes \textbf{ en qual manera conuiene alos Reyes de ser magranimos } e en qual mauera de ser magnificos \\\hline
1.4.5 & nisi sint boni et virtuosi , \textbf{ decet eos sequi bonos mores nobilium , } ut sint magnanimi et magnifici , prudentes et affabiles : & e uirtuosos conuiene les aellos \textbf{ de segnir las bueans costunbres de los nobles } por que sean de grand coraçon e de grand fazienda \\\hline
1.4.7 & Potentes vero et principantes , \textbf{ quia oportet eos intendere exterioribus curis , } retrahuntur , & Mas los poderosos et los prinçipes \textbf{ por que les conuiene de entender | e auer cuydados de muchͣs cosas retrahen se } e tiran se \\\hline
1.4.7 & videlicet , diuitiis , nobilitate , potentia : \textbf{ decet eos esse eruditos , et temperatos . } Nam per nobilitatem & que son rriquezas nobleza e poderio . \textbf{ Conuiene les de ser enssennados e tenprados } ca por la nobleza \\\hline
1.4.7 & Rursus quia pollent principatu et potentia , \textbf{ et oportet eos diuersis curis intendere , } retrahuntur a venereis , & e por su poderio \textbf{ conuiene les de auer grandes cuydados } e de se tyrar de obras lux̉iosas \\\hline
2.1.4 & non sufficiebat communitas domestica , \textbf{ sed oportuit dare communitatem vici , } ita quod cum vicus constet & non cunplie la comunidat de vna casa \textbf{ mas conuiene de dar comunidat de varrio . } Por que commo el uarrio sea fech̃ de muchas casas \\\hline
2.1.4 & praeter communitatem vici \textbf{ oportuit } dare communitatem ciuitatis . & todas las cosas neçessarias ala uida \textbf{ conuiene de dar comunidat ala çibdat } sobre la comunidat deluarrio . \\\hline
2.1.4 & propter opera diurnalia et quotidiana . \textbf{ Quod autem oporteat domum } ex pluribus constare personis , videre non est difficile . & e de cada dia \textbf{ mas non es cosa fuerte de veer } que la casa sea establesçida de muchas perssonas . \\\hline
2.1.4 & si in domo communitatem saluare volumus , \textbf{ oportet eam } ex pluribus constare personis ; & nin conpannia de vno \textbf{ assi commo si queremos saluar la comuidat dela casa conuiene que ella sea establesçida de muchͣs perssonas } mas assi commo adelanţe paresçra \\\hline
2.1.4 & non solum domus est communitas quaedam , \textbf{ sed in domo oportet } dare plures communitates : & non solamente la casa es vna comiundat \textbf{ mas en la casa conuiene de dar muchͣs comunidades } la qual cosa non puede ser sin muchͣs perssonas . \\\hline
2.1.5 & vel habent aliquid aliud loco bonis . \textbf{ Decet autem omnes ciues cognoscere partes , } ex quibus componitur domus : & o alguna otra cosa en logar de bueye . \textbf{ Et conuiene a todos los çibdadanos de conosçer } e saber las partes de que se conpone la casa \\\hline
2.1.6 & Ex his autem patere potest , \textbf{ quod oportet } in domo perfecta esse tria regimina . & La terçera del padre e del fij̉o . \textbf{ Et destas cosas puede paresçer } que en la casa acabada \\\hline
2.1.7 & in communitate domestica , \textbf{ primum oportet } congregare marem , et foeminam . & en la comunidat dela casa \textbf{ primeramente conuiene de ayuntar el uaron con la mugni } e esta orden es muy con razon . \\\hline
2.1.7 & tanto magis decet fugere Reges et Principes , \textbf{ quanto decet eos meliores et virtuosiores esse . } His visis , & mas conuiene alos Reyes \textbf{ e alos prinçipes delo esquiuar quanto mas conuiene aellos de ser meiores | e mas uirtuosos que los otros . } ¶ Estas cosas dichͣs \\\hline
2.1.8 & et inrepudiatum . \textbf{ Decet ergo omnes ciues coniungi } suis uxoribus indiuisibiliter absque repudiatione , & por la qual el casamiento non deue ser partido nin repoyado . \textbf{ Et pues que assy es conuiene a todos los çibdadanos de se ayuntar | asus muger } ssin departimiento ninguno \\\hline
2.1.9 & una sola uxore esse contentos . \textbf{ Et tanto magis hoc decet Reges et Principes , } quanto decet eos meliores esse aliis , & cada vno de vna sola mugier . \textbf{ Et tanto esto mas pertenesçe a los Reyes | e alos prinçipes de segnir mas orden natural } quanto mas conuiene aellos de ser meiores \\\hline
2.1.9 & Et tanto magis hoc decet Reges et Principes , \textbf{ quanto decet eos meliores esse aliis , } et magis sequi ordinem naturalem . & e alos prinçipes de segnir mas orden natural \textbf{ quanto mas conuiene aellos de ser meiores | que todos los otros . } Et pues que assi es paresçe \\\hline
2.1.10 & coniuges Regum et Principum , \textbf{ quia in eorum coniugio magis quam in alio decet } naturalem ordinem conseruare . & por que en el casamiento \textbf{ dellos conuiene de guardar la orden natural mas que en otro ninguno . } ¶ Lo segundo esso mismo pue de ser mostrada \\\hline
2.1.10 & lis et discordia oriretur . \textbf{ Quare decet coniuges omnium ciuium uno uiro esse contentas : } tanto tamen hoc magis decet & e amistança nasçria contienda e discordia . \textbf{ por la qual cosa conuiene alas mugieres de todos los çibdadanos de ser pagadas de vn solo uaron . } Empero tanto o mas conuiene esto \\\hline
2.1.10 & quia per hoc magis impeditur certitudo filiorum . \textbf{ Quare si decet omnes ciues certos esse de suis filiis , } ut eis diligenter prouideant in haereditate et in nutrimento : & ca por esto se enbargaria mas la çertidunbre de los fijos . \textbf{ Por la qual cosa sy conuiene a todos los çibdadanos | ser çier tos de lus fios } por que los puedan proueer \\\hline
2.1.10 & ut eis diligenter prouideant in haereditate et in nutrimento : \textbf{ decet coniuges omnium ciuium uno viro esse contentas ; } tanto tamen hoc magis decet & con grand acuçia en las hedades e en el nudrimiento . \textbf{ Conuiene alas mugers de todos los çibdadanos | de ser pagadas de vn solo uaron . } Et esto tanto conuiene mas alas mugers de los Reyes \\\hline
2.1.11 & non sunt connubia contrahenda . \textbf{ Decet ergo omnes ciues } non contrahere coniugia & que son muy cercanas por parentesto . \textbf{ Et pues que assi es conuiene a todos los çibdadanos de non fazer matermonios } con quales quier perssonas . \\\hline
2.1.11 & quanto magis eos obseruare \textbf{ decet ordinem naturalem . } Secunda via ad inuestigandum hoc idem , & Enpero tanto mas esto conuiene alos Reyes e alos prinçipes \textbf{ quanto mas conuiene a ellos de guardar la orden natural } ¶La segunda razon para prouar esto mesmo se toma del bien \\\hline
2.1.11 & tanta multiplicaretur dilectio , \textbf{ quod oporteret eos nimium vacare venereis . } Decet ergo omnes ciues non inire connubia & tanto se acrescentarie el amor entre ellos \textbf{ que les conuernie de enteder | e de darse mucho alas obras de lux̉ia . } Et pues que assi es conuiene a todos los çibdadanos \\\hline
2.1.11 & quod oporteret eos nimium vacare venereis . \textbf{ Decet ergo omnes ciues non inire connubia } cum personis nimia consanguinitate coniunctis ; & e de darse mucho alas obras de lux̉ia . \textbf{ Et pues que assi es conuiene a todos los çibdadanos | de non fazer casamientos entre perssonas } que son muy ayuntadas en parentesço \\\hline
2.1.12 & sed pluralitas diuitiarum est intendenda quasi ex consequenti . \textbf{ Decet enim eos talem uxorem acceptare , } quae sit nobilis genere , & assi commo a cosa que se sigue . \textbf{ Conuiene a ellos de tomar tal muger } que sea noble por linage \\\hline
2.1.12 & quae deseruiunt ad sufficientiam vitae : \textbf{ decet eos in suis coniugibus } principalius quaerere , & que siruen a abastamiento dela uida . \textbf{ Conuiene aellos de demandar en las sus mugers } mas prinçipalmente que ellas sean nobles de linage \\\hline
2.1.13 & Decet ergo coniuges temperatas esse . \textbf{ Decet eas etiam amare operositatem : } quia cum aliqua persona ociosa existat , & que las muger ssean tenpradas . \textbf{ Et avn les conuiene aellas de amar fazer buenas obras . } Ca quando alguna persona esta de uagar mas ligeramente es inclinada a aquellas cosas \\\hline
2.1.15 & inter regimen coniugale , et seruile . \textbf{ Quare si decet ciues esse industres , } et cognoscere modum & e entre el del señor e del sieruo . \textbf{ Et por ende si conuiene alos çibdadanos de ser sabidores } e conosçer la manerar la orden natural \\\hline
2.1.16 & et quomodo utendum sit eo . \textbf{ Oportet ergo magis in particulari descendere , } qualiter omnes ciues & e en qual manera de una vsar del . \textbf{ Et pues que assi es conuiene de desçender | mas en particular } mostrando en qual manera todos los çibdadanos et mayormente los Reyes \\\hline
2.1.17 & danda opera coniugali copulae . \textbf{ Decet ergo omnes ciues } uti magis coniugio tempore , & mas es de dar obra al ayuntamiento del casamiento \textbf{ Et pues que assi es conuiene a todos los çibdadanos vsar mas del casamiento } en elt pon que es meior la generaçion de los fijos . \\\hline
2.1.17 & tanto tamen hoc magis decet Reges et Principes , \textbf{ quanto decet eos elegantiores habere filios . } Mulierum autem mores & e alos prinçipes \textbf{ quanto mas les conuiene aellos de auer los fijos grandes e esforcados de cuerpo } euedes saber \\\hline
2.1.19 & quando sunt castae , honestae , abstinentes , et sobriae . \textbf{ Decet enim coniuges esse castas } non solum propter fidem seruandam suis viris , & quando son castas e honestas e abstinents e mesuradas . \textbf{ Ca conuiene alas mugieres de ser castas } non solamente \\\hline
2.1.19 & et cauere sibi ab operibus illicitis : \textbf{ sed oportet eas esse pudicas et honestas , } ut sibi caueant a signis et a verbis , & e se guarden de malas obras . \textbf{ mas conuiene a ellas de ser linpias e honestas } assi que se guarden de señales \\\hline
2.1.19 & expedit coniuges pudicas esse . \textbf{ Tertio oportet eas esse abstinentes , } ut caueant sibi a superfluitate cibi . & conuiene que las mugers sean linpias e honestas e guardadas en sus palauras ¶ \textbf{ Lo terçero conuiene a ellas de ser abstinentes } e que se guarden de demasia dela uianda . \\\hline
2.1.19 & Nam cibi superfluitas ad incontinentiam inclinat . \textbf{ Quarto decet eas esse sobrias , } ut caueant sibi a superfluitate potus . & por que la demasia dela uianda inclina mucho ala lux̉ia ¶ \textbf{ Lo quarto conuiene aellas de ser mesuradas } por que se guarden de sobrepuiança de vino \\\hline
2.1.19 & et ciuili potentia , \textbf{ decet inquirere matronas } aliquas boni testimonii & e en poderio çiuil \textbf{ conuiene les de bulcar buenas mugers } e antiguas de buen testimoino prouadas \\\hline
2.1.19 & et ad maiorem amorem viros inducunt : \textbf{ decet ergo eas esse taciturnas . } Sic etiam decet esse stabiles : & e then a sus maridos a mayor amor . \textbf{ Et por ende les conuiene de ser calladas } e en essa misma manera avn les conuiene de ser estables e firmes \\\hline
2.1.19 & decet ergo eas esse taciturnas . \textbf{ Sic etiam decet esse stabiles : } quia quanto uxor est magis constans , & Et por ende les conuiene de ser calladas \textbf{ e en essa misma manera avn les conuiene de ser estables e firmes } que quanto la mug̃res mas firme e mas estable \\\hline
2.1.19 & vel per cautelas alias adhibendo . \textbf{ Quare decet omnes ciues } sic suas coniuges regere : & non o fallando otras cautellas para esto . \textbf{ por la qual cosa conuiene a todos los çibdadanos de gouernar a sus mugers assi . } Et esto tanto mas conuiene alos Reyes \\\hline
2.1.20 & diligenter consideranda , \textbf{ in quibus viros circa proprias coniuges decet } debite se habere . & en que deuemos cuydar con grand acuçia \textbf{ en las quales conuiene alos uarones de se auer } con \\\hline
2.1.20 & Tertio debent cum eis debite conuersari . \textbf{ Decet enim eos suis coniugibus } moderate et discrete uti : & ¶ Lo primero se praeua \textbf{ assi que conuien e alos uarones de vsar con sus muger } stenpdamente e sabia mente . \\\hline
2.1.20 & et semper intemperantior redditur . \textbf{ Decet ergo omnes ciues } uti moderate coniugali copula , & e sienpra se faz mas destenprado . \textbf{ Pues que assi es conuiene a todos los çibdadanos de vsar } tenpradamente e mesuradamente del ayuntamiento matermoinal . \\\hline
2.1.20 & quibus est orationibus vacandum , \textbf{ decet a talibus abstinere : } sic etiam temporibus , & en que deuen estar en oraçion \textbf{ conuiene les de se arredrar de tales obras . } Et avn assi en los trons \\\hline
2.1.20 & insurgere nocumentum proli , \textbf{ decet abstinere a tali copula : } est ergo obseruandum tempus debitum . & conuiene les de guardar se \textbf{ de allegar se mucho alas mugers . Et pues que assi es conuiene les alos casados } de guardar tienpo conuenible \\\hline
2.1.20 & non solum amicitia delectabilis , sed honesta . Viso , \textbf{ quomodo decet viros suis uxoribus moderate uti et discrete : } restat videre , & mas avn amistança honesta . \textbf{ ¶ Visto en qual manera conuiene alos uarons de vlar labiamente } e tenpradamente de sus muger sfinca de ver \\\hline
2.1.21 & et uniuersaliter omnes ciues scire , \textbf{ quomodo circa ornatum corporis deceat } suas coniuges debite se habere . & onuiene alos Reyes e alos prinçipes \textbf{ e generalmente a todos los çibdadanos saber en qual manera } couiene alas sus mugers \\\hline
2.1.21 & sunt licita et honesta . \textbf{ Decet enim viros } secundum suum statum , & si se fizieren commo deuen e commo cunple . \textbf{ Ca conuiene alos maridos de proueer conueniblemente a sus } mugerssegunt sus estados e en vestiduras conuenibles \\\hline
2.1.21 & non superflua vestimenta quaerunt . \textbf{ Decet enim uxorem militis } magis esse ornatam vestibus , & nin demandan uestiduras sobeias . \textbf{ Ca conuiene ala muger del cauallero de ser mas honrrada de uestiduras } que ala muger del çibdadano sinple . \\\hline
2.1.21 & appeteret ornamenta . \textbf{ Tertio decet foeminas } circa ornatum corporis esse simplices , & mas que demanda el su estado . \textbf{ ¶ Lo terçero conuiene alas mugers de ser } sinples enel conponimiento de su cuerpo \\\hline
2.1.22 & quare cum una cura impediat aliam , \textbf{ oportet sic zelantes } retrahi a debitis curis , & Por la qual cosa commo el vn cuydado enbargue el otro . \textbf{ Conuiene alos tales çelosos de ser enbargados enlos cuydados } conueibles \\\hline
2.1.22 & ut plurimum oriuntur lites et iurgia . \textbf{ Non ergo decet uiros } de suis coniugibus & uezes varaias e contiendas . \textbf{ ¶ Et pues que assi es non conuiene alos maridos ser muy çelosos de sus mugrs } nin avn les conuiene \\\hline
2.1.22 & et debitam diligentiam adhibere . \textbf{ Sic enim decet uirum quemlibet } erga suam coniugem ornatum habere zelum , & e deue auer acuçia conuenible de su casa . \textbf{ Ca assi conuiene a cada vn marido de auer çelo ordenado de su mugni } por que sea entre ellos amistança natural delectable e honesta \\\hline
2.1.23 & ut quia illud est citius in suo complemento , \textbf{ sic oporteret repentino operari , } forte elegibilius esset huiusmodi consilium . & por que el consseio de la muger es mas ayna el su conplimiento que deluats . \textbf{ por que si acaesçiesse de obrar alguna cosa adesora } por auentra a seria \\\hline
2.1.24 & describenda sunt opera , \textbf{ quae decet ipsas coniuges exercere . } Sed quia de eis infra dicetur , & assy commo dicho es de \textbf{ suso deuemos mostrar } quales obras conuiene que vsen las mugers \\\hline
2.2.1 & primo ostendere uolumus , \textbf{ quod decet omnes patres } circa proprios filios esse solicitos . & primeramente queremos mostrar \textbf{ que conuiene a todos los padres de ser muy cuydadosos de los fiios . } Ca penssada la acuçia \\\hline
2.2.1 & Possumus autem triplici via venari , \textbf{ quod decet huiusmodi solicitudinem habere parentes . } Prima via sumitur & Et nos podemos mostrar por tres razones \textbf{ que conuienen a todos los padres | de auer grand cuydado de sus fijos } ¶ \\\hline
2.2.1 & a patribus esse habent , \textbf{ decet patres habere curam filiorum , } et solicitari erga eos , & e razon de los fijos \textbf{ e los fijos naturalmente han el ser de los padres . Conuiene alos padres de auer cuydado de los fijos } e ser cuydadosos dellos \\\hline
2.2.2 & solicitari circa eos . \textbf{ Licet omnes patres deceat solicitari } circa proprios filios , & que han alos fijos sean cuy dadosos della \textbf{ aguer que todos los padres de una } auer cuydado de sus fijos \\\hline
2.2.2 & ut patet per rationes superius assignatas : \textbf{ maxime tamen decet Reges , } et Principes talem solicitudinem gerere . & por las razones ya dichos . \textbf{ Mayormente conuiene alos Reyes e alos prinçipeᷤ de auer tal cuydado . } Et esto podemos prouar por tres razones . \\\hline
2.2.2 & et in aliquo dominio , \textbf{ in quo oportet eos alios gubernare ; } maxime decet eos esse prudentes et bonos . & e en algun sennorio \textbf{ en quel conuiene gouernar los otros } mucho les conuiene de ser sabios e buenos . \\\hline
2.2.2 & in quo oportet eos alios gubernare ; \textbf{ maxime decet eos esse prudentes et bonos . } Et cum filii perueniunt & en quel conuiene gouernar los otros \textbf{ mucho les conuiene de ser sabios e buenos . } Mas commo los fijos bengan a mayor bondat e a mayor sabiduria \\\hline
2.2.6 & Si ergo tantam pronitatem habemus ad malum , \textbf{ et oportet nos sic } per diuturna tempora & por que nos podamos guardar mas ligeramente delas delecta con nes non conueinbles . \textbf{ Et pues que assi es si nos auemos tanta inclinaçion a mal conuiene nos de acostunbrar nos } por luengost pons al contrario e al bien \\\hline
2.2.7 & in regno maius periculum imminere . \textbf{ Licet deceret omnes homines cognoscere literas ; } ut per eas prudentiores effecti , & Et quanto dela maliçia dellos vernie mayor periglo a todo el regno \textbf{ omo quier que conuenga a todos los omes de saber letras } por que por ellas puedan ser mas sabios \\\hline
2.2.7 & quae est ex scientia acquirenda . \textbf{ Decet enim volentes literas discere , } literales sermones scire distincte proferre . & que omne aprende ¶ \textbf{ Lo primero paresce assi . | Ca conuiene que los que quieren aprinder sçiençia de letris } que aprendan pronunçiar departidamente las palabras delas letras ¶ \\\hline
2.2.7 & literales sermones scire distincte proferre . \textbf{ Decet etiam eos esse attentos , } et studiosos circa ea , & que aprendan pronunçiar departidamente las palabras delas letras ¶ \textbf{ Et avn conuiene les de ser acuçiosos } e estudiosos çerca aquellas cosas \\\hline
2.2.7 & insudare literalibus disciplinis , \textbf{ quanto decet eos intelligentiores et prudentiores esse , } ut possint naturaliter dominari . & e en las sçiençias liberales \textbf{ quanto mas les conuiene de ser mas entendudos | e mas sabios que los otros } por que puedan \\\hline
2.2.7 & cum ponuntur in aliquo dominio tyrannizent , \textbf{ decet ipsos etiam ab ipsa infantia insudare literis , } ut vigere possint prudentia et intellectu . & nin sean tirannos \textbf{ Conuiene les avn de trabaiar | entp̃o dela moçedat en sciençias e ensseñaça de deletris } por que se pueda enoblesçer \\\hline
2.2.8 & et per debitas rationes manifestemus propositum . \textbf{ Oportuit ergo inuenire aliquam scientiam docentem modum , } quo formanda sunt argumenta , et rationes . & i anifestamos nr̃a uoluntad e nr̃a entençion . \textbf{ Et por ende conuiene de fallar algua sçiençia | que nos mostrasse } en qual manera son de enformar los argumentos e las razones . \\\hline
2.2.8 & et Principum debeant insudare . \textbf{ Nam cum oporteat } eos esse quasi semideos , & e mayormente delos Reyes e de los prinçipes . \textbf{ Ca commo les conuenga a ellos de ser } assi commo medios dioses e de entender conueinblemente \\\hline
2.2.8 & subtiliter perscrutari scientias : \textbf{ maxime igitur decet } ipsos bene se habere circa diuina , & escodrinnar sotilmente las sçiençias \textbf{ mucho les conuiene aellos de se auer bien cerca las cosas diuinales } e ser enssennados e firmes en la fe \\\hline
2.2.8 & inquantum deseruiunt morali negocio . \textbf{ Decet igitur eos scire grammaticam , } ut intelligant idioma literale : & en quanto siruen ala ph̃ia moral . \textbf{ Et pues que assi es conuiene les a ellos de saber la guamatica } por que entiendan el lenguage delas letras \\\hline
2.2.8 & secundum Philosophum in politicis \textbf{ eos scire decet , } inquantum deseruit ad bonos mores . & Otrossi segunt el philosofo en las politicas conuiene les \textbf{ alguacosado saber dela musica } en quanto ella sirue alas bueans costunbres . \\\hline
2.2.9 & decet igitur ipsum esse inuentiuum . \textbf{ Secundo decet ipsum esse intelligentem et perspicacem . } Nam sicut nullus bene et perfecte & que non tan solamente sea fallador delas cosas \textbf{ mas que sea entendido e sotil . Ca assi commo ninguon non puede abastar } asi en la uida bien \\\hline
2.2.9 & quomodo obliquata essent . \textbf{ Decet igitur aliorum directorem memorem esse praeteritorum . } Secundo decet & Et pues que assi es conuiene \textbf{ que el que ha degniar los otros | que sea acordado delas cosas passadas } ¶ \\\hline
2.2.9 & per quae in posterum facilius obliquari possint . \textbf{ Tertio decet } ipsum esse cautum , & por que ponaga melezmas a aquellas cosas \textbf{ por que podrien los omes de ligero errar ¶ } Lo terçero le conuiene al maestro \\\hline
2.2.9 & Quare sicut doctor est in speculabilibus , \textbf{ sic decet esse diligentem et cautum , } ut proponat suis auditoribus vera & Por la qual cosa \textbf{ assi commo conuiene al doctor e al maestro en las sçiençias especulatiuas de ser acuçioso e sabio } en manera que proponga a sus disçipulos cosas uerdaderas \\\hline
2.2.9 & Quantum autem ad vitam , \textbf{ decet ipsum esse honestum , et bonum . } Si ergo Reges et Principes & conuiene le que el doctor sea menbrado e prouado e sabio e acatado . \textbf{ Mas quanto ala uida deue ser honesto e bueno . } ¶ Et pues que assi es si los Reyes e los prinçipes \\\hline
2.2.10 & circa locutionem , visionem , et auditum : \textbf{ non enim decet } pueros qualitercunque loqui , & e cerca la vista e cerca el oyr . \textbf{ Ca non conuiene alos moços de fablar qual quier manera } nin de oyr a quales si quier cosas . \\\hline
2.2.10 & pueros qualitercunque loqui , \textbf{ nec decet eos qualiacunque videre , } vel qualiacunque audire ; & Ca non conuiene alos moços de fablar qual quier manera \textbf{ nin de oyr a quales si quier cosas . } Mas deuen y tomar alguna manera çerca la fabla . \\\hline
2.2.10 & Quantum ad visibilia quidem , \textbf{ quia sicut non decet } eos turpia sequi : & ¶ La primera quanto alas cosas uisibles \textbf{ que assi commo non les conuiene de fablar cosas torpes } Et la razon desto pone el philosofo en łvij̊ libro delas ethicas \\\hline
2.2.10 & et indecens audire turpia : \textbf{ sic decet eos audire viros bonos et honestos , } et cohibendi sunt & e desconuenible de oyr cosas torpes \textbf{ assi les | conuienea ellos de oyr a bueons omes e honestos } e son de refrenar \\\hline
2.2.11 & in morali negocio sermones uniuersaliores minus proficiunt , \textbf{ oportet specialiter tradere , } quomodo iuuenes sunt & que las palauras generales en esta sçiençia moral menos aprouechan que las spanles \textbf{ por ende conuiene nos de dezir } speçialmente en qual manera los moços deuen ser enssennados e enformados en bueans costunbres . \\\hline
2.2.12 & ex sumptione cibi : \textbf{ sed etiam decet eos esse sobrios , } ut non efficiantur ebrii & non solamente que se non fagan gollosos por el comͣ \textbf{ mas avn les conuiene de ser mesurados } que non se fagan beodos \\\hline
2.2.12 & sit contra rationis dictamen , \textbf{ quia decet patrem sic solicitari erga filios , } ut sint virtuosi , & sean contra ordenamiento de \textbf{ razon conuiene al padre de ser acuçioso çerca de sus tijos } por que non puedan ser malos nin viçiosos . \\\hline
2.2.13 & antequam consequatur illum , \textbf{ ideo oportet interponere aliquos ludos , } et aliquas delectationes , & Et por que algunas vegadas establesce assi fin en que trabaia luengamente ante que alcançe aquella fin \textbf{ por ende conuienele de entroponer alguons trebeios } e algunas delectaconnes \\\hline
2.2.16 & Restat videre , \textbf{ quomodo oporteat } eos ordinari ad virtutes , & Et que deuen tomar del xiiij año adelante trebeios mas fuertes que enl primero setenario . \textbf{ finca de demostrar en qual manera conuiene alos moços de ser dispuestos e ordenados alas uirtudes } porque ayan bien dispuesta e bien ordenada la uoluntad e el entendimiento . \\\hline
2.2.17 & sed possunt instrui in illis scientiis , \textbf{ ad quas sciendas oportet } recurrere ad intellectum rerum , & Mas avn pueden ser enssennados en aquellas sçiençias \textbf{ alas quales nos deuemos acoger } para saber el entendimiento delas cosas \\\hline
2.2.17 & Cum ergo omnes volentes viuere vita politica , \textbf{ oporteat aliquando sustinere fortes labores } pro defensione reipublicae : & que quieren beuir uida çiuil \textbf{ conuiene les de sofrir alguas uegadas fuertes trabaios } por defendemiento dela tierra . \\\hline
2.2.17 & Sufficiat autem ad praesens scire , \textbf{ quod decet patres sic solicitari erga regimen filiorum , } ut habeant sic bene dispositum corpus , & mas quanto alo presente abonda de saber \textbf{ que conuiene alos padres ser assi | acuçiossos çerca el gouernamiento delos fijos } por que ayan el cuerpo bien ordenado \\\hline
2.2.17 & tres breues rationes , \textbf{ quare decet filios esse subiectos , } et obedire senioribus , et patribus . & razono breues \textbf{ por que conuiene alos fijos de ser lubiectos } e obedientes a lus padres e alos vieios . \\\hline
2.2.18 & et adhuc primogeniti \textbf{ qui regere debent decet } minores labores assumere . & merenos son doma e devlar que los otros . \textbf{ Et avn el primero gento que deue regnar } conuiene de tomar menores trabaios \\\hline
2.2.18 & nec pro alio casu audeant arma assumere ; \textbf{ attamen quia decet } eos esse magis prudentes quam bellatores , & nin por otra uentra a qual si quier non osen tomar armas . \textbf{ Enpero por que mas conuiene de ser sabios } que lidiadores alos fijos de los Reyes \\\hline
2.2.19 & pudicas , abstinentes , et sobrias : \textbf{ sic decet et filias . } Haec ergo , et multa alia , & ca assi commo conuiene alas madres \textbf{ de ser continentes e castas e guardadas e mesuradas en essa misma manera conuiene alas fijas de ser tales } Et pues que assi es estas cosas \\\hline
2.2.20 & nisi in aliquibus delectemur : \textbf{ decet nos assumere } aliqua opera licita et honesta , & si non nos delectaremos en alguas cosas \textbf{ conuiene de tomar alguas obras conuenibles e honestas } çerca las quales entendamos \\\hline
2.2.20 & debent operositatem amare : \textbf{ et decet eas exercitati } circa aliqua opera licita et honesta . & azendosassienpre en algua cosa conueinble \textbf{ e conuiene les de vsar en algunas obras } que sean conueinbles e honestas . \\\hline
2.2.20 & aliquibus delectationibus licitis , \textbf{ decet eas intentas esse } circa aliqua opera licita et honesta & por que las muger sspue dan resçebir recreaçion \textbf{ en alguas delecta connes conuenbłs conuiene a ellas de ser } cuy dadosas çerca las obras \\\hline
2.2.20 & circa qualia opera solicitari debent : \textbf{ oportet in talibus differenter loqui } secundum diuersitatem personarum . & Mas si alguno demandare \textbf{ de que se deuen trabaiar las mugers conuiene de fablar en tales cosas departidamente } segunt el departimiento delas perssonas \\\hline
2.2.21 & Ostenso , \textbf{ quod non decet puellas esse vagabundas , } nec decet eas viuere otiose : & que las mugers fuesen acuçiosas . \textbf{ ostrado que non conuiene alas moças de andar uagarosas a quande e allende } nin les conuiene de beuir ociosas \\\hline
2.2.21 & restat ut nunc tertio ostendamus , \textbf{ quod decet eas taciturnas esse , } quod triplici via venari possumus . & finca que agora lo terçero mostremos \textbf{ que deuen ser callantias | e non parleras } la qual cosa podemos mostrar \\\hline
2.2.21 & pertinentes ad lites , et ad iurgia . \textbf{ Decet ergo ipsas } per debitam taciturnitatem adeo examinare dicenda , & que pertenesçen a peleas e abaraias \textbf{ e por ende les conuiene aellas de ser callanţias en manera conuenible } e en tanto examinar las cosas \\\hline
2.2.21 & propter quod iudicentur litigiosae , et discolae : \textbf{ quare decet ipsas esse taciturnas , } ne in verba litigiosa prorumpant . & e por desacordadas \textbf{ por la qual razon les | conuiene a ellas de ser callantias } e que non se entremetan \\\hline
2.3.1 & quia ostensum est , \textbf{ qualiter decet } uiros suas coniuges regere , & Ca es mostrai ser do \textbf{ en qual manera conuiene a los maridos de gouernar a sus mugers . } Et en qual maneta los padres deuen regir e gouernar a sus fijos . \\\hline
2.3.1 & et uniuersaliter omnes ciues habere debeant : \textbf{ quomodo deceat } ipsos se habere circa possessiones , & e generalmente todos los çibdadanos . \textbf{ Et en qual manera se deuan auer çerca las possesiones } e çerca las riquezas e los dineros \\\hline
2.3.1 & Volens ergo tradere notitiam de arte fabrili , \textbf{ oportet ipsum determinare de martello , et incude , } et aliis instrumentis fabrilibus ; & Et por ende los que quieren dar conosçimiento dela arte del ferrero \textbf{ conuiene les de determinar del martiello e dela yunque } e de los otros estrumentos del ferrero . \\\hline
2.3.3 & qui debent esse nobiles et praeclari , \textbf{ potissime decet esse magnificos . } Alii enim moderatas possessiones habentes , & que anings de los otros nobles \textbf{ ca ellos conuiene de ser nobles prinçipalmente | e magnificos en todas sus cosas . } ca si los otros nobles \\\hline
2.3.3 & quantum ad industriam operis , \textbf{ decet habere habitationes mirabiles . } Alii vero ciues tales habitationes & e alos prinçipes \textbf{ quanto ala maestera dela obra pertenesçe auer moradas matauillosas } e por ende los otros çibdadanos deuen auer tales moradas \\\hline
2.3.7 & si uolunt domus proprias debite gubernare , \textbf{ decet eos scire } quot sunt uitae , & las sus casas propreas \textbf{ conuiene les de saber quantas son las uidas } e quantas son las maneras de beuir \\\hline
2.3.8 & et Principibus quam in aliis , \textbf{ quanto decet habere ordinatiorem uoluptatem , } et meliorem aestimationem finis : & que en los otros \textbf{ quanto mas conuiene aellos de auer mayor ordenamiento dela uoluntad } e meior estimacion dela finca \\\hline
2.3.9 & quod habetur in toto regno , \textbf{ oportuit introduci commutationem rerum ad denarios , } et econuerso . & que hades es en todo el regno \textbf{ conuiene de poner m̃udaçion delas cosas alos dineros } e de los diueros alas cosas \\\hline
2.3.9 & et prouinciarum , \textbf{ oportuit introduci } non solum commutationem rerum ad res , & e de departidas prouinçias \textbf{ conuiene de poner non sola mente mudaçion delas cosas alas cosas } o delas cosas alos dineros \\\hline
2.3.9 & commode ad partes longinquas portari non possunt . \textbf{ Oportuit ergo inuenire aliquid } quod esset portabile , & non las poderemos leuar conueniblemente a luengas tierras . \textbf{ Et pues que assi es conuiene de fablar alguna cosa } que se podiesse leuar \\\hline
2.3.9 & ut volentes habere tantum vini , \textbf{ oportebat dare tantum ponderis argenti , vel auri , } vel etiam alterius metalli , & assi que los que quirien auer tunerto de vino \textbf{ conuimeles a dar tanto de peso de plata o de oro o avn de otro metal } assi commo plazia de establesçer en aquel tienpo alos pueblos e alos Reyes . \\\hline
2.3.10 & tamen et principibus \textbf{ quod decet esse quasi semideos , } exercere non congruit . & commo quier que le conlientan alos mercadores e algunos otros . \textbf{ Enpero alos Reyes e alos prinçipes los quales deuen ser medios dioses } non los conuiene de usar dellas \\\hline
2.3.10 & quae est oeconomica et quasi naturalis , decet . \textbf{ Decet enim ipsos abundare } in possessionibus et in redditibus , ex quorum fructu pro defensione regni et aliis necessariis possunt abundare pecunia . & e assi commo natural \textbf{ ca conuiene alos Rey | e de abondar en possessiones } e en rentas del fact̃o \\\hline
2.3.11 & nisi concedatur eius substantia : \textbf{ cum ad talem usum oporteat } ipsam substantiam alienare . & si non se otorgare la sustançia \textbf{ dellos commo atal uso pertenezca de enagenar la sustançia . } Et pues que assi es por que el açidente desçende dela sustançia del subieto \\\hline
2.3.12 & ex fructibus earum pecuniam acquirit . \textbf{ Decet enim ( secundum Philosophum ) } oeconomicum et dispensatorem domus & dellas resçibe muchs dineros . \textbf{ Ca conuiene segunt el philosofo al mayordomo e al despenssero dela casa de ser prouado } e sabio \\\hline
2.3.12 & pro suae voluntatis arbitrio : volentem ergo pecuniam acquirere , \textbf{ oportet haec } et similia particularia gesta , & Et por ende el que quiere gana rriqueza \textbf{ conuiene le de tener enla memoria estos fechs particulares e otros semeiantes } por los quales algunos ganaron grandes algos \\\hline
2.3.12 & sed etiam in possessionibus mobilibus . \textbf{ Decet enim ipsos pollere multitudine bestiarum , } et etiam auium , & mas avn en las posessiones muebles . \textbf{ Ca conuiene a ellos de resplandesçer | por muchedunbre de bestias e avn de aues . } por las quales pueden satisfazer ala mengua dela uida . \\\hline
2.3.13 & ut si plures voces efficiunt aliquam harmoniam , \textbf{ oportet ibi dare aliquam vocem praedominantem , } secundum quam tota harmonia diiudicatur . & Assi commo si muchas uozes fiziess en alguna armonia o concordança de canto . \textbf{ Conuerna de dar y alguna bos | que enssennoreasse sobre las otras } segunt la qual serie iudgada toda aquella concordança delas uozes delas otras avn en essa misma manera \\\hline
2.3.14 & Sicut praeter ius naturale \textbf{ propter commune bonum oportuit } dare leges aliquas positiuas , & ssi commo sin el derecho natraal \textbf{ por el bien comun | connino de dar } e de fazer alg̃s leyes pointiuas \\\hline
2.3.15 & hoc debet esse ex consequenti . \textbf{ Oportuit autem dare ministrationem conductam et dilectiuam } praeter ministrationem naturalem & tenporal esto deue ser despues de aquel bien que entiende . \textbf{ Mas conuiene de dar a ministraçion de alquiler e de amor sin la ministt̃ion natural et segunt ley . } Ca por que en nos es el appetito corrupto \\\hline
2.3.15 & decet principantes se habere quasi ad filios , \textbf{ et decet eos regere non regimine seruili , } sed magis quasi paternali et regali . & assi commo cerca de fijos . \textbf{ Et conuiene les alos prinçipes delos gouernar non } por gouernamiento seruil \\\hline
2.3.17 & et congruentia temporum . \textbf{ Cum enim deceat Regem esse magnificum , } ut supra in primo libro diffusius probabatur , & La conueniençia de los tiepos . \textbf{ Ca commo conuenga alos Reyes | e alos prinçipes ser magnificos } assi commo es prouado mas conplidamente en el primero libro \\\hline
2.3.17 & ut supra in primo libro diffusius probabatur , \textbf{ decet ipsum erga suos ministros decenter se habere in apparatu debito , } et in debitis indumentis . & assi commo es prouado mas conplidamente en el primero libro \textbf{ conuiene les de auer sus siruientes apareiados | conueniblemente en el parescer de fuera } e en uestiduras conuenibles \\\hline
2.3.17 & et ne a populis condemnantur , \textbf{ decet eos magnifica facere , } ut probat Philosophus 7 Poli’ . & Enpero por que los Reyes e los prinçipes sean guardados en su estado granado \textbf{ e por qua non sean despreçiados de los pueblos conuieneles de fazer grandes } assi commo prueua el philosofo \\\hline
2.3.17 & et quidam inferiores : \textbf{ propter quod decet } eos aliter & algsson mayores e algs menores \textbf{ e por ende conuiene a ellos de ser honrrados de uestiduras en departidas maneras } Ca assi commo ueemos en la orden de todo el mundo \\\hline
2.3.18 & eos esse prudentiores aliis , \textbf{ et secundum quem decet } eos esse meliores aliis ; & de ser ellos mas labios que los otros . \textbf{ Et segunt el su estado conuienel es de ser meiores } que los \\\hline
2.3.18 & sed quia volunt retinere mores curiae et nobilium , \textbf{ quos decet datiuos esse ; } propter quod tales curiales dici debent . & e de los no nobles omes alos \textbf{ que les conuienne de ser dadores } e cobidadores \\\hline
2.3.18 & de leui patet \textbf{ quod decet ministros Regum et Principum curiales esse . } Nam si decet Reges et Principes & Visto que cosa es la curialidat et la cortesia de ligero puede paresçer \textbf{ que los seruientes de los Reyes | e delos prinçipes deuen ser curiales } e cortesesca \\\hline
2.3.18 & habere mores nobiles et curiales , ministros , \textbf{ quos in bonis decet suos dominos imitari , } oportet curiales esse . & e de ser curiales e nobles \textbf{ assi conuiene alos seruientes dellos | los que quieren semeiar a sus sennors } de ser buenos e mesurados e corteses . \\\hline
2.3.18 & quos in bonis decet suos dominos imitari , \textbf{ oportet curiales esse . } Ostensum est , & los que quieren semeiar a sus sennors \textbf{ de ser buenos e mesurados e corteses . | ¶ } ostrado es quales deuen ser los seruientes de los Reyes \\\hline
2.3.19 & quia debent habere nobiles et curiales . \textbf{ Nam sicut decet ciues } ut debitam politiam seruent & ca los deuen auer nobles e curiales e corteses \textbf{ ca assi conmo conuiene alos çibdadanos de ser iustos e legales } para guardar su poliçia conueniblemente \\\hline
2.3.19 & ut seruent decentiam curiae \textbf{ et honoris statum curiales esse quare si scimus quales oportet esse ministros , } restat ostendere qualiter Reges et Principes & por que guarden el estado e la honrra dela corte \textbf{ conueiblemente | por ende si sabemos quales deuen ser los seruientes } fincanos de demostrar \\\hline
2.3.19 & et aperienda sunt secreta . \textbf{ Quinto et ultimo oportet cognoscere , } qualiter sunt beneficiandi , & e en qual manera les son de descobrar las poridades ¶ \textbf{ Lo quinto | e lo postrimero conuiene de saber } en qual manera los señores les han de fazer bien \\\hline
2.3.19 & esse \textbf{ magnanimos decet operari pauca et magna , } ut decet ipsos solicitari & ¶ Et pues que assi es alos Reyes \textbf{ e alos prinçipes alos quales couiene de auer altos coraçones | conuiene les de obrar pocas cosas } e grandes ca les conuiene \\\hline
2.3.19 & Reges ergo et Principes , \textbf{ quos decet esse magnanimos } ad proprios ministros , & e los prinçipes \textbf{ alos quales conuiene de ser magn animos | deuen se mostrar } tonprados a los sus seruientes propreos \\\hline
2.3.20 & Possumus autem duplici via ostendere , \textbf{ quod non decet } in mensis Regum et Principum & Mas podemos mostrar por dos razones \textbf{ que non conuiene de fablar mucho en las mesas de los Reyes } nin de los prinçipes \\\hline
2.3.20 & Reges ergo et Principes , \textbf{ quos decet maxime temperatos esse , } et obseruare ordinem naturalem & Et pues que assi es los Reyes \textbf{ e los prinçipeᷤ alos quales conuiene ser muy tenprados } e guardar la orden natural en toda \\\hline
2.3.20 & decet in mensis vitare sermonum multitudinem , \textbf{ decet etiam hoc ipsos ministrantes , } ne negligatur , & assi commo dicho es . \textbf{ Avn esto mismo conuiene alos seruientes por que la orden e la manera del seruir } non sea despreçiada nin enbargada . \\\hline
3.1.6 & restat uidere \textbf{ in quot partes oportet } hunc tertium librum diuidere , & de establesçimiento de çibdat e de Regno . \textbf{ finca de ver en quantas partes conuiene } de partir este terçero libro \\\hline
3.1.6 & tradentium notitiam aliquam de arte illa : \textbf{ ideo oportet } propter sufficientiam artis regiminis ciuitatis et regni citare & que dieron algun conosçimiento \textbf{ et algun entendimiento de aquella arte por ende conuiene para auer arte conplida de gouernamiento dela çibdat } e del regno de dezer e de contar \\\hline
3.1.8 & Maximam unitatem et aequalitatem \textbf{ non oportet } quaerere in omnibus rebus . & quales si acaesçiere logar nos podremos dellas fazer mençion . \textbf{ on conuiene de demandar en todas las cosas } grant egualdat cosas fuessen \\\hline
3.1.8 & secundum suum statum sit maxime perfectum , \textbf{ oportet ibi dare diuersa secundum speciem . } Nam quia tota bonitas uniuersi non potest & e por que el mundo segunt su estado sea muy acabado \textbf{ conuietie de dar en el | departidas speçias } e departidas semeianças \\\hline
3.1.8 & reseruari in una specie , \textbf{ oportet ibi dare species diuersas ; } ut in pluribus speciebus entium reseruetur maior perfectio , & nin en vna semeiança \textbf{ conuiene de dar | y deꝑ tidas espeçies } e departidas semeianças \\\hline
3.1.8 & esse perfectum , \textbf{ oportet dare diuersitatem aliquam , nec oportet ibi esse } omnimodam conformitatem et aequalitatem , & para que aya ser acabada \textbf{ conuiene de dar ay algun departimiento | nin conuiene de ser } y en toda manera confirmada egualdat \\\hline
3.1.8 & ut ambulatione , tactu , visione , \textbf{ et auditus ideo oportet } ibi dare diuersa membra exercentia diuersos actus : & assi commo de andar e de tanner e de oyr e deuer . \textbf{ por ende conuiene de dar . | y departidos mienbros } que fagan estas obras departidas . \\\hline
3.1.8 & et aliis huiusmodi ; \textbf{ oportet in ciuitate } dare diuersitatem aliqua , & auemos mester casas e uestid̃as e viandas e otras cosas tales \textbf{ por ende conuiene de dar algun departimiento en la çibdat por que en ella sean falladas todas las cosas } que cunplen ala uida . \\\hline
3.1.8 & ad aliquem principantem vel dominantem , \textbf{ ut cum in ciuitate oporteat } dare aliquos magistratus , & delos çibdadanos a algun prinçipe o algun sennor \textbf{ e commo en la çibdat conuenga de dar alguons ofiçioso } alguons maestradgos o algunas alcaldias \\\hline
3.1.8 & Quare cum hoc diuersitatem requirat , \textbf{ oportet in ciuitate } dare diuersitatem aliquam . & por ende commo estas cosas demanden departimiento \textbf{ conuiene de dar en la çibdat algun departimiento . } La quanta razon se toma \\\hline
3.1.8 & requiruntur diuersa , \textbf{ ideo oportet ciuitatem } habere aliquam diuersitatem in se , & para abastamiento deuida son meester muchͣs cosas departidas \textbf{ por ende conuiene enla çibdat de auer en ssi algun departimiento } e de auer departidos uarrios \\\hline
3.1.8 & et nisi cognoscat \textbf{ quod oportet in ea diuersitatem esse . } Sermo in principiis debet esse longus , & si non sopiere en qual manera es establesçida la çibdat \textbf{ e si non sopiere en qual manera conuiene de auer en ella departimiento de ofiçios e de ofiçiales } l sermon en los comienços deueser luengo \\\hline
3.1.9 & diu inuestigandum est , \textbf{ qualiter ciuitatem oportet esse unam , } et quam diuersitatem habere debet , & muy luengamente es de buscar \textbf{ e de escodrinnar | en qual manera la çibdat conuiene de ser vna } e qual departimiento deue auer enlła \\\hline
3.1.9 & et quam diuersitatem habere debet , \textbf{ et quomodo ciues decet } se habere ad inuicem , & e qual departimiento deue auer enlła \textbf{ e en qual manera conuiene alos çibdadanos de se auer los vnos con los otros } e en quales cosas deuen partiçipar \\\hline
3.1.11 & ex parte ipsorum communicantium in haereditate communi , \textbf{ eo quod oporteat eos valde ad inuicem conuersari , } ostenditur ut plurimum homines habere lites et iurgia & los que han la heredat en comun paresçe \textbf{ por que han de beuir en vno } que por la mayor parte han contiendas e uaraias por la qual cosa dize el philosofo \\\hline
3.1.11 & et indignamur erga illos , \textbf{ quia oportet nos habere } ad illos multa colloquia , & e nos enssannamos contra ellos muchͣs uezes \textbf{ por que nos conuiene de fablar muchͣs uezes con ellos } e de beuir conellos \\\hline
3.1.12 & ne igitur reddantur bellantes pusillanimes , \textbf{ quos constat esse timidos oportet } ab exercitu expelli . & Et por ende por que los lidiadores non se enflaquezcan en las batallas \textbf{ conuiene de echar dela batalla } e dela fazienda alos de flaco \\\hline
3.1.14 & ab artificibus et ab aliis ciuibus , \textbf{ quod ciues alii pro defensione patriae bellare non oporteat } melius est ergo dicere in ciuitate & que sienpre los çibdadanos \textbf{ non les conuenga de lidiar | por defendimiento de su tierra } quando fuere meester \\\hline
3.1.14 & Nam variatis conditionibus vicinorum , \textbf{ oportet aliter et aliter determinare } de conditionibus bellantium . & Ca departidas las condiçonnes de los uezinos \textbf{ departidamente se deue tomar el cuento de los lidiadores } por la qual cosa la arte e la sçiençia non pueden ser \\\hline
3.1.15 & Posset enim casus contingere , \textbf{ quod et mulieres bellare oporteret . } Multotiens autem circa partes Italiae hoc contigit , & por que podria contesçer algun caso \textbf{ que las mugers deuian batallar } ca muchͣs uegadas contesçio esto \\\hline
3.1.17 & sumitur ex parte virtutum \textbf{ quas decet habere ciues : } decet enim ipsos & non es conuenible se toma de parte delas uirtudes \textbf{ que deuen auer los çibdadanos } por que conuiene \\\hline
3.1.18 & et uniuersaliter eos quorum est leges ferre , \textbf{ decet aliquas leges statuere } circa possessiones ciuium . & e generalmente a todos aquellos \textbf{ a quien parte nesçe de poner leyes de establesçer algunas leyes } cerca las possessiones de los çibdadanos \\\hline
3.1.18 & quod suum est possideat : \textbf{ sed multa ordinare decet } circa possessiones reprimendas , & assi que cada vno aya lo que suyo es \textbf{ Mas conuiene les de ordenar muchͣs cosas } para repremir los desseos desordenados \\\hline
3.1.20 & ut si iudices discordarent , \textbf{ oporteret collationem habere ad inuicem , } quorum sententia tenenda esset . & assi que si los uiezes descordassen en publico \textbf{ conuernia de auer fablas los vnos con los otros de quales miezes serie de tener la sentençia¶ } Lo terçero fallesçie el dicho philosofo \\\hline
3.2.1 & spectat ad consilium : \textbf{ decet enim princeps } adeo sapientes consiliarios habere , & por sabiduria \textbf{ ca pertenesçe alos prinçipes de auer conseieros tan sabios } por que pueidan fallar aquellas cosas \\\hline
3.2.1 & Quare si considerentur quae requiruntur ad hoc quod tempore pacis per leges bene gubernetur ciuitas , \textbf{ oportet in huiusmodi regimine } de praedictis quatuor considerationem facere . & entp̃o dela paz \textbf{ por las leyes conuiene de fazer tractado destas quatro cosas sobredichͣs en este gouernamiento ¶ } La segunda razon para prouar \\\hline
3.2.5 & ex qua praeficiendus est dominus , \textbf{ sed etiam oportet determinare personam . } Nam sicut oriuntur dissentiones et lites , & linage donde ha de ser tomado el sennor . \textbf{ Mas avn conuiene de determinar la perssona . } Ca assi commo nasçen discordias \\\hline
3.2.6 & in ipsa monarchia perfectius reperiri . \textbf{ Decet enim ipsum regem volentem recte regere } ( quantum ad praesens spectat ) & que ante ca conuiene \textbf{ que el Rey | que quiere bien gouernar su regno } quanto parte nesçe alo presente \\\hline
3.2.8 & organice deseruiunt res exteriores . \textbf{ Decet ergo Reges et Principes sic regere ciuitates et regna , } ut sibi subiecti abundent rebus exterioribus & assi commo son las riquezas e los algos . \textbf{ Et por ende conuiene alos Reyes | e alos prinçipes de gouernar } assi las çibdades e los regnos \\\hline
3.2.9 & et Principem non ostendere se nimis terribilem et seuerum , \textbf{ nec decet se nimis familiarem exhibere , } sed apparere debet persona grauius et reuerenda , & ¶ Lo terçeto conuiene al Rey et al prinçipe de non mostrarsse muy espantable nin muy cruel . \textbf{ nin le conuiene otrosi de se fazer muy familiar alos omnes } Mas deue paresçer perssona pesada \\\hline
3.2.9 & vituperatur autem auaritia et gulositas . \textbf{ Septimo decet verum Regem ornare } et munire ciuitates & e la destenprança e la auariçia¶ \textbf{ Lo septimo conuiene al uerdadero Rey de conponer } e guarnesçer las çibdades e los castiellos que son en el su regno \\\hline
3.2.9 & non honorant , sed perimunt . \textbf{ Nono decet verum Regem per usurpationem et iniustitiam } non dilatare suum dominium . & mas matan los e destierran los \textbf{ ¶ Loye conuiene al Rey uerdadero de non enssanchar su regno } por tomar lo ageno \\\hline
3.2.10 & sed ad hostes , et ad extraneos . \textbf{ Utrum autem deceat Reges } habere exploratores in regno propter aliam causam , & Mas tales assechadores pone contra los enemigos e contra los estrannos . \textbf{ Mas si conuiene al Rey auer assechadores en el regno } por otra razon que por la que dichͣes . \\\hline
3.2.10 & quibus indigent , \textbf{ ut non vacet eis aliquid machinari contra ipsos , nec oporteat ipsos habere aliquam custodiam propter illos . } Verus autem Rex & en que han de de beuir de cada dia \textbf{ por que no les uague de fazer ayuntamiento contra ellos | nin los tiranos non ayan menester ninguna guarda } por temor dellos . \\\hline
3.2.12 & quantam veri reges : \textbf{ tum quia oportet } eos multa expendere superuacue , tum etiam quia veris regibus plus donatur & titannas quantas han los uerdaderos Reyes ¶ \textbf{ Lo vno por que les conuiene a ellos responder muchͣs cosas superfluas . } lo otro por que alos uerdaderos Reyes \\\hline
3.2.13 & sunt paucissimi numero , \textbf{ supponi oportet } eos nihil curare , & segund que dize el philosofo \textbf{ ca conujene de dar a entender } que estos tales non han cuydado de saluar su vida ¶ \\\hline
3.2.14 & deficeret esse bonus et virtuosus . \textbf{ Decet ergo regiam maiestatem } summo studio cauere tyrannidem , & e uirtuoso se pusiesse contra otro bueno e uirtuoso dexaria de ser bueno e uirtuoso . \textbf{ Et pues que assi es conuiene ala Real magestad de escusar con grant estudio } e con grant acuçia la tirama \\\hline
3.2.15 & quae politiam saluant , \textbf{ et quae oportet facere Regem ad hoc } ut se in suo principatu praeseruet . & e el gouernamiento del regno \textbf{ e dela çibdat | las quals conuiene al Rey de fazer } para que se pueda man tener en lu prinçipado e en lu lennorio ¶ \\\hline
3.2.15 & et saluant . \textbf{ Decet ergo Regem frequenter meditari et habere memoriam praeteritorum } quae contigerunt in regno , & e qual cosa lo salua . \textbf{ Et pues que assi es conuiene al Rey de penssar mucha menudo | e muchͣs uezes delas cosas que passaron . } Et conuiene le de auer memoria de los fecho passados \\\hline
3.2.16 & manifestauimus item quod sit Regis officium , \textbf{ et quae oporteat ipsum facere } ut recte regat populum sibi commissum : & Reyr \textbf{ quales cosas le conuiene de fazer } para que derechamente gouierne el pueblo qual es acomendado . \\\hline
3.2.17 & quam unus solus : \textbf{ decet ad huiusmodi negocia alios aduocare , } ut per eorum consilium possit & por la quel cosa commo muchs mas cosas ayan prouadas \textbf{ que vno solo conuiene de llamar otros } para los negoçios . por que por el conseio dellos pueda ser escogida la meior carrera \\\hline
3.2.17 & operamur autem prompte : \textbf{ et quod oportet consiliari tarde , } sed facere consiliata velociter . & mas obramos en poco tienpo \textbf{ e luego e que conuiene de touiar conseio prolongadamente } mas conuiene de fazerl cosas conseiadas mucho ayna . \\\hline
3.2.18 & quales consiliarios habere \textbf{ deceat regiam maiestatem , } et quae et quot sunt in consiliis requirenda : & por la qual cosa si queremos saber \textbf{ quales consseieros deue auer la real magestad e quales e quantas cosas son menester en los consseios } conuiene de saber \\\hline
3.2.18 & debet habere apparenter , \textbf{ oportet quod bonus consiliator habeat existenter : } satis apparet quales consiliatores deceat & deue auer en el \textbf{ e paresçer todas aquellas cosas | que ha todo buen conseiero en ssi de fecho . } Et por ende assaz parelçe quales conseieros deue auer el rey \\\hline
3.2.18 & oportet quod bonus consiliator habeat existenter : \textbf{ satis apparet quales consiliatores deceat } quaerere regiam maiestatem ; & que ha todo buen conseiero en ssi de fecho . \textbf{ Et por ende assaz parelçe quales conseieros deue auer el rey } ca deue tomar tales \\\hline
3.2.19 & et de pace et bello , et de legislatore : \textbf{ circa haec ergo quinque oportet } consiliatores esse instructos . & Lo quanto commo se han de poner las leyes \textbf{ e commo se han de guardar . | Et en estas çinco cosas pueden ser enformados } e enssenados los conseieros e los sabidores dellas . \\\hline
3.2.19 & et prouentus regni , \textbf{ quos oportet peruenire ad regem , } qui et quanti sunt : & Et conuiene que sepan las rentas del regno \textbf{ las que han de venir al Rey quales e quantas son } por que si alguͣ cosa es superflua \\\hline
3.2.21 & obligare habent iudicem , \textbf{ quem esse oportet } quasi regulam in iudicando . & La primera seqma par aquello que tales palabras han de to terçeres desegualar eliez \textbf{ el qual conuiene de ser } assi commo regla derecha en \\\hline
3.2.21 & Nam cum lis fit de aliquo facto vel de aliquare , \textbf{ nihil oportet dici in iudicio } nisi pertinens ad rem vel ad negocium , & o de alguna cosa \textbf{ non se deue dezir ninguna cosa en iuizio } si non lo que pertenesçe a aquel fecho \\\hline
3.2.22 & Possumus autem quatuor enumerare , \textbf{ quae oportet habere iudices , } ut vera iudicia proferant , & demos contar quatro cosas \textbf{ que conuiene de auer alos iuezes } para que den uerdaderos iuisios \\\hline
3.2.22 & et quales discussores causarum quaerere \textbf{ deceat regiam maiestatem : } nam decet eos tales quaerere & Et de aqui paresçe quales iuezes \textbf{ e quales examinadores de los pleitos deue tomar el Rey . } Ca deue tales tomar \\\hline
3.2.22 & deceat regiam maiestatem : \textbf{ nam decet eos tales quaerere } qui sint humiles , & e quales examinadores de los pleitos deue tomar el Rey . \textbf{ Ca deue tales tomar } que sean omildosos \\\hline
3.2.23 & quibus congruit ampliori bonitate pollere . \textbf{ Decet itaque eos esse clementes et benignos , } non quia iustitiam deserant , & por mayor bondat . \textbf{ Et pues que assi es conuiene a ellos de ser piadosos e benignos non } por que dexen la iustiçiaca sin ella la paz del regno \\\hline
3.2.24 & Ratio autem , \textbf{ quare iuri naturali oportuit } superaddere positiuum , & de obligar alos omes . \textbf{ Mas la razon por que al derech natural conuinio anneder derecho positiuo es esta } por que muchas cosas son derechas naturalmente \\\hline
3.2.24 & ne a memoria recederet , \textbf{ oportuit ipsum scribi } in aliqua exteriori substantia . & por que non se pierda dela memoria de los omes \textbf{ conuiene de ser escerpto en algun libro . } Enpero cada vno destos dos derechs tan bien el natural commo el positiuo se puede escͥuir en algun libro \\\hline
3.2.26 & Ideo dicitur 4 Politicorum \textbf{ quod non oportet } adaptare politias legibus , & Et por ende dize el philosofo \textbf{ en el quarto libro delas politicas | que non conuiene de apropar las comunidades } delas çibdades alas leyes . \\\hline
3.2.26 & sed leges politiae , \textbf{ quas leges oportet diuersas esse } secundum diuersitatem politiarum . & Mas las leyes alas comunidades \textbf{ de las çibdades las quales leyes conuiene de ser departidas } segunt el departimiento delas comunidades . \\\hline
3.2.29 & Quare si nomen regis a regendo sumptum est , \textbf{ et decet Regem regere alios , } et esse regulam aliorum , & sy el nonbre del Rey es tomado de gouernamiento . \textbf{ Conuiene al rey de gouernar los otros } e de ser regla de los otros . \\\hline
3.2.30 & ut in prosequendo patebit : \textbf{ oportuit igitur dare legem euangelicam et diuinam , } secundum quam prohiberentur & assi commo paresçra adelante . \textbf{ Et por ende conuiene de dar ley diuinal } e e un agłical segunt la qual fuessen vedados los pecados todos . \\\hline
3.2.31 & Immo magnam efficaciam habent ex diuturnitate et assuefactione . \textbf{ Decet ergo reges et principes obseruare bonas consuetudines principatus et regni , } et non innouare patrias leges , & Et por ende conuiene alos Reyes \textbf{ e alos prinçipes | de guardar las bueans costunbres del prinçipado e del regno } e non renouar las leyes dela tierra saluo \\\hline
3.2.36 & ut Reges diligantur a populo , \textbf{ decet eos esse iustos , et aequales . } Nam maxime prouocatur populus ad odium Regis , & La terçera cosa para que los Reyes sean amados del pueblo \textbf{ es quales conuiene de ser derechureros e eguales . } Ca el pueblo mayormente se le una taria a mal querençia del Rey \\\hline
3.3.1 & Nam et si quis solitariam vitam duceret , \textbf{ adhuc oporteret } ipsum habere aliqualem prudentiam & e morasse solo avn conuenir le \textbf{ ya de auer alguna sabiduria } por la qual se sopiesse gouernar . \\\hline
3.3.2 & in quibus regionibus meliores sunt bellatores , \textbf{ oportet attendere circa praedicta duo . } In partibus igitur nimis propinquis soli , & o en quales tierras son meiores lidiadores . \textbf{ Conuiene de tener mientes en estas dos cosas sobredichas . } Et pues que asy es en las partes \\\hline
3.3.3 & esse videtur armorum industria . \textbf{ Nam siue equitem siue peditem oportet esse bellantem , } quasi fortuito videtur & nin ligera arte auer sabiduria de las armas . \textbf{ Ca si quier sea cauallero si quier peon el que ha de lidiar paresçe } que por uentura alcaça uictoria \\\hline
3.3.4 & et per consequens bene bellare non potest . \textbf{ Septimo decet eos habere aptitudinem , } et industriam ad protegendum se , & que non podra bien lidiar . \textbf{ Lo vij° . conuiene a los lidiadores de auer disposiçion e sabiduria para cobrirse e defender se } e para ferir a los otros . \\\hline
3.3.4 & infra patebit . \textbf{ Octauo decet bellatores verecundari , } et erubescere turpem fugam . & e otras cosas que se ayuntan a estas adelante paresçra Lo . viij° \textbf{ que pertenesçe a los lidiadores | es de auer uerguença } e de guardar se \\\hline
3.3.5 & quid sit de quaesito tenendum , \textbf{ oportet aduertere , } quod secundum diuersitatem pugnarum & Et pues que assi es para saber la uerdat \textbf{ que auemos de tener } desta question de tener mientes \\\hline
3.3.8 & debet celeriter castra construere . \textbf{ Oportet autem semper construendis castris , } et faciendis fossis aliquos magistros praestitui , & e fazer muy apriessa . \textbf{ Mas conuiene de poner algunos maestros | para costruyr los castiellos } e fazer las carcauas \\\hline
3.3.8 & et unicuique iniungant \textbf{ quod ipsum oporteat facere . } Ostenso utile esse castra construere , & que acuçien los negligentes \textbf{ e manden a cada vno qual cosa deua fazer . } Mostrado que prouechosa cosa es de fazer los castiellos . \\\hline
3.3.8 & oporteat exercitum constringi et constipari . \textbf{ Quarto si oporteat in loco illo exercitum moram contrahere , } et adsit possibilitas est eligenda & nin avn sea tomado tan pequeno espaçio por que la hueste este a mayor estrechura que deue . \textbf{ Lo quarto si conueniere que aquella hueste aya de fazer | en aquel logar alguna tardança } e fuere cosa que se puede fazer \\\hline
3.3.8 & quia si multum timeretur de impetu hostium , \textbf{ oporteret foueas facere multorum angulorum , } eo quod illa est magis defensioni apta , & por que si temen mucho del cometemiento de los enemigos \textbf{ coñuiene de fazer carcauas de muchos rencones } por que aquella figura es mas conuenible \\\hline
3.3.8 & per modicum tempus existere , \textbf{ non oportet tantas munitiones expetere . } Modum autem , & e son de fazer mas anchas carcauas . \textbf{ mas solamente quieren y estar vna noche o por poco tienpo non conuiene de fazer tantas guarniçiones . } Mas la manera e la quantidat de las carcauas pone la vegeçio \\\hline
3.3.10 & non sufficiunt ad dirigendum bellantes , \textbf{ sed oportet dare euidentia signa ; } ut quilibet solo intuitu sciat & para guiar los lidiadores . \textbf{ Mas conuiene de dar otras seña les manifiestas . por que cada vno viendo aquellas señales } se sepa tener ordenadamente en su az \\\hline
3.3.10 & habentes armorum experientiam : \textbf{ oportet omnia haec peramplius } et perfectius reperiri in eo , & e avn que ayan vso de las armas . \textbf{ Et si todas estas cosas deuen ser falladas en los buenos lidiadores | mucho } mas conuiene \\\hline
3.3.12 & si debeat publica pugna committi , \textbf{ et quibus cautelis abundare decet bellorum ducem } ne suus exercitus laedatur & si se deue la batalla acometer publicamente . \textbf{ Et quales cautelas ha de auer el señor de la batalla } por que la su hueste non sea dañada en el camino . \\\hline
3.3.13 & quia quanto illi annuli magis sunt compacti , \textbf{ tanto oportet plures ex eis frangere } ut vulnera noceant . & por que quanto aquellos aniellos mas son ayuntados . \textbf{ tanto conuiene de cortar mas dellos } para que los colpes enpeescan . \\\hline
3.3.13 & ad cor vel ad membra vitalia , \textbf{ oporteret magnam plagam facere } et multa ossa incidere : & ante que el colpe veniesse al coraçon o a los mienbros de vida \textbf{ conuernie de fazer muy grant llaga } e de cortar muchos huessos . \\\hline
3.3.13 & In percutiendo autem caesim , \textbf{ quia oportet fieri magnum brachiorum motum prius quam infligatur plaga , } aduersarius ex longinquo potest prouidere vulnus , & Mas en feriendo cortando . \textbf{ por que conuiene de fazer grand mouimiento de los braços | ante que se de el colpe el enemigo } o el contrario de \\\hline
3.3.14 & qualiter debeant dimicare : \textbf{ propter quod oportebit eos fugam eligere . } Quarto dux exercitus sic se temperare debet : & en qual manera deuen lidiar . \textbf{ Por la qual cosa les conuiene de foyr . } Lo quarto el señor de la hueste se deue tenprar \\\hline
3.3.16 & inuadere aliquas munitiones eorum ; \textbf{ propter quod eos oportet } uti pugna defensiua . & Et avn algunas vezes contesçe que algunos otros çercan sus villas o sus castiellos . \textbf{ Por la qual cosa les conuiene de vsar de batalla defenssiua para se defender . } Otrossi contesçe que en el prinçipado \\\hline
3.3.16 & quis debeat se habere , \textbf{ non oportet circa alia bellorum genera diutius immorari . } Primo tamen dicemus de bello obsessiuo . & e en toda batalla en qual manera cada vno se deue auer en ellas . \textbf{ non conuiene çerca las otras maneras de las batallas | de de tener nos mas luengamente . } Enpero diremos de la batalla osse ssiua \\\hline
3.3.16 & Contingit enim aliquando obsessos carere aqua : \textbf{ ideo vel oportet eos siti perire , } vel munitiones reddere . & Ca contesçe algunas vegadas que los cercados non han agua . \textbf{ e por ende o les conuiene de peresçer o de morir } de sedo de dar las fortalezas . \\\hline
3.3.18 & vel ciuitatis obsessae , \textbf{ oportet talibus uti argumentis } ut habeatur intentum . & fasta las menas del castiello o de la çibdat cercada . \textbf{ Conuiene de vsar de tales armadijas o de tales armamientos } por que puedan ganar el logar \\\hline
3.3.22 & qua suffossa , et castro demerso in ipsam propter magnitudinem ponderis , \textbf{ oportet castrum iterum construi , } eo quod non possit & la qual tierra cauada \textbf{ conuiene de apoyar bien el castiello o la çerca } por que se non funda \\\hline

\end{tabular}
