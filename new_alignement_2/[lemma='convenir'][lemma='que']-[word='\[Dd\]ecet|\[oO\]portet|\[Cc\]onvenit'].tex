\begin{tabular}{|p{1cm}|p{6.5cm}|p{6.5cm}|}

\hline
1.1.5 & en el segundo libro delas ethicas \textbf{ que conuiene que las obras } por las quales nos alcançamos la fin & ut dicitur 2 Ethic . ) \textbf{ oportet operationes , } per quas finem consequimur , \\\hline
1.1.6 & Pues si la bien auentraança es bien acabado e conplido . \textbf{ Conuiene que sea bien segunt el en tedimiento } e segunt Razon & si felicitas ponitur \textbf{ esse perfectum bonum , oportet quod sit bonum } secundum intellectum , et rationem : \\\hline
1.1.6 & e segunt Razon \textbf{ O conuiene que sea tal bien qual iudga la razon derecha } que nos auemos de segnir & secundum intellectum , et rationem : \textbf{ vel ( quod idem est ) | oportet quod sit tale bonum , } quale recta ratio prosequendi iudicet . \\\hline
1.1.8 & si conplida mente quiere demostrar le que significa \textbf{ conuiene que sea conosçida cosa e magnifiesta . } Mas las cosas que son de dentro del alma & si plene manifestare vult ipsum signatum , \textbf{ oportet quod sit | quid notum et manifestum : } intrinseca autem non sunt nobis nota , \\\hline
1.1.12 & e de gouernar prinçipalmente e acabadamente . \textbf{ Conuiene que qual se quier } prinçipeo Rey & et perfecte solus Deus , \textbf{ oportet quod quicunque principatur , } siue regnat , \\\hline
1.1.13 & Et commo el amor sienpre sean los semeiables e acordables con el . \textbf{ Conuiene que aquel que es para de ser } gualardonado de dios & cum semper amor sit ad similes , et conformes , \textbf{ oportet esse similem , } et conformem Deo , \\\hline
1.2.2 & Et el otro es para cobdiçiar \textbf{ ¶ Conuiene que toda uirtud moral } o sea en el entendimiento o en la uoluntad o en el appetito enssannador & irascibilis scilicet , et concupiscibilis : \textbf{ oportet omnem virtutem moralem , } vel esse in intellectu , \\\hline
1.2.6 & en las cosas singulares . \textbf{ Conuiene que la pradençia sea cerca las cosas singulares } e particulares & et agibilia sint singularia , \textbf{ oportet prudentiam esse circa particularia , } applicando uniuersales regulas \\\hline
1.2.8 & e la su conpanna a alguons bienes . \textbf{ Conuiene que aya memoria de las cosas passadas . } Et que aya prouision delas cosas passadas & gentem aliquam ad bonum dirigere , \textbf{ oportet quod habeat memoriam praeteritorum , } et prouidentiam futurorum . \\\hline
1.2.8 & Ca la manera por que el Rey guia el su pueblo \textbf{ Conuiene que sea manera de omne . } Ca el Rey omne es & quo Rex suum populum dirigit , \textbf{ oportet quod sit humanus , } quia Rex ipse homo est . \\\hline
1.2.31 & commo los p̃h̃osacuerdan en esta sentençia \textbf{ que conuiene que todas las uirtudes sean ayinntadas la vna con la vna con la otra . } Ca dixieron que aquel que ha vna uirtud & in hanc sententiam conuenerunt , \textbf{ quod oportet virtutes connexas esse . } Dixerunt enim \\\hline
1.3.9 & terca la qual es el dolor e la tristeza . \textbf{ Conuiene que la delectaçion e la tristeza } sean prinçipales passiones & circa quod est dolor et tristitia \textbf{ oportet delectationem et tristitiam } esse principales passiones respectu concupiscibilis . \\\hline
1.4.7 & li dellos contamos algunas malas costunbres \textbf{ ca non conuiene que todos seantales . } Mas abasta que aquellas costunbres sean falladas en muchos por que non & aliquos malos mores : \textbf{ quia non oportet omnes esse tales , } sed sufficit reperiri illud in pluribus : \\\hline
2.1.2 & e por si vale a conplimiento dela uida . \textbf{ Conuiene que la comunidat dela casa sea mas neçessaria } Et pues que assi es los Reyes e los prinçipes & ad per se sufficientiam vitae , \textbf{ oportet communitatem domus necessariam esse . } Reges ergo et Principes , \\\hline
2.1.3 & e antepuesta non puede ser propreamente cosa artifiçial \textbf{ mas conuiene que aquello en quanto es tal sea cosa natural } Por la qual cosa si el omne es naturalmente & non proprie quid artificiale erit , \textbf{ sed oportet illud | ( secundum quod huiusmodi est ) naturale esse . } Quare si homo est \\\hline
2.1.4 & nin conpannia de vno \textbf{ assi commo si queremos saluar la comuidat dela casa conuiene que ella sea establesçida de muchͣs perssonas } mas assi commo adelanţe paresçra & cum non sit proprie communitas nec societas ad seipsum , \textbf{ si in domo communitatem saluare volumus , | oportet eam } ex pluribus constare personis ; \\\hline
2.1.6 & luego que es fecha fazer otra semeiante \textbf{ assi mas conuiene que ella primeramente sea acabada } enssi & potest sibi simile producere , \textbf{ sed oportet prius ipsam esse perfectam . } statim enim , \\\hline
2.1.6 & luego otro su semeinante \textbf{ mas conuiene que primeramente el sea acabado . } Et pues que assi es engendrar su semeiante non pertenesçe a cosa natural tomada en qual quier manera mas pertenesçe a cosa natural en quanto ella es acabada . & nec statim potest sibi simile producere , \textbf{ sed oportet prius ipsum esse perfectum : | producere ergo sibi similem , } non est de ratione rei naturalis \\\hline
2.1.6 & Ca quando el ome es primero \textbf{ conuiene que sea engendrado . } Et la natura luego es acuçiosa de su salud . & Cum enim primo homo est , \textbf{ oportet quod sit genitus : } et natura statim est solicita de salute eius ; \\\hline
2.1.6 & Ca para ser la cosa acabada \textbf{ non conuiene que faga sienpre } e engendre su semeiante & Ad hoc enim quod aliquid sit perfectum , \textbf{ non oportet | quod sibi simile producat , } sed quod possit sibi simile producere : \\\hline
2.1.6 & Et por ende paresçe que para que la casa sea acabada \textbf{ que conuiene que sean enlla tres comuundades . } ¶ La vna del uaron e dela muger ¶ & Patet ergo quod ad hoc quod domus habeat esse perfectum , \textbf{ oportet ibi esse tres communitates : } unam viri et uxoris , aliam domini et serui , \\\hline
2.1.6 & e tres gouernamientos de ligero puede paresçer \textbf{ que conuiene que sean y . } quatro linages de perssonas & de leui patere potest , \textbf{ quod ibi oportet } esse quatuor genera personarum . \\\hline
2.1.6 & Ca commo en la casa acabada sean tres gouernamientos . \textbf{ Ca conuiene que este libro sea partido en tres partes . } ¶ En la primera delas quales tractaremos del gouernamiento mater moianl . & Nam cum in domo perfecta sint tria regimina , \textbf{ oportet hunc librum tres habere partes ; } in quarum prima tractetur primo de regimine coniugali : \\\hline
2.1.8 & que sea segunt natura \textbf{ e para que entre el uaron e la muger sea amistança natural conuiene que guarden vno a otro fe e lealtad } assi que non se puedan partir vno de otro . & ad hoc quod coniugium sit secundum naturam , \textbf{ et ad hoc quod inter uxorem et virum sit amicitia naturalis , | oportet quod sibi inuicem seruent fidem , } ita quod ab inuicem non discedant . \\\hline
2.1.9 & e segunt ordenna traal . \textbf{ Conuiene que todos los çibdadanos sean pagados } cada vno de vna sola mugier . & et ordinem naturalem , \textbf{ decet omnes ciues } una sola uxore esse contentos . \\\hline
2.1.10 & assi commo si algun çibdadano es subiecto al preuoste e al Rey . \textbf{ Conuiene que el } prinoste sea ordenado al Rey e sea so el . & ut si quis subiicitur Proposito et Regi , \textbf{ oportet Propositum illum ad Regem ordinari , } et esse sub ipso repugnat \\\hline
2.1.10 & ala generacion de los fijos \textbf{ conuiene que las mugers de todos los çibdadanos } por que non sea enbargado el & ad bonum prolis , \textbf{ decet coniuges omnium ciuium , } ne impediatur earum foecunditas , \\\hline
2.1.15 & por la natura \textbf{ conuiene que sea muy ordenado . } Ca aquel gnia la natura de que viene todo ordenamiento & et quicquid natura praeparatur , \textbf{ oportet ordinatissimum esse : } quia ille naturam dirigit , \\\hline
2.2.1 & assi commo se praeua en el viij delas . ethicas . \textbf{ Conuiene que los padres por amor natural } que han alos fijos sean cuy dadosos della & ut probatur 8 Ethicorum , \textbf{ decet patres ex ipso amore naturali , } quem habent ad filios , \\\hline
2.2.6 & Ca quando alguno es inclinado a alguacosa . \textbf{ Conuiene que el vse mucho en el contrario } por que non sea inclinado a aquella cosa . & Nam cum aliquis est pronus ad aliquid , \textbf{ oportet ipsum multum assuescere in contrarium , } ne inclinetur ad illud : \\\hline
2.2.7 & Lo primero paresce assi . \textbf{ Ca conuiene que los que quieren aprinder sçiençia de letris } que aprendan pronunçiar departidamente las palabras delas letras ¶ & quae est ex scientia acquirenda . \textbf{ Decet enim volentes literas discere , } literales sermones scire distincte proferre . \\\hline
2.2.9 & e que entienda los dichͣs de los otros . \textbf{ ¶ Lo terçero conuiene que sea iudgador } e que aya razon para iudgar . & et intelligens aliorum dicta . \textbf{ Tertio oportet ipsum esse iudicatiuum : } nam perfectio scientiae potissime \\\hline
2.2.11 & Ca si la vianda se ouiere bien a cozer \textbf{ conuiene que sea bien proporçionada ala calentura natural } Por la qual cosa si en tan grand quantia se & Si enim cibus digeri debeat , \textbf{ oportet | ipsum esse proportionatum calori naturali . } Quare si in tanta quantitate sumatur , \\\hline
2.3.14 & la primera razon paresçe assi . \textbf{ Ca conuiene que el sennor } segunt & Prima congruitas sic patet : \textbf{ oportet enim dominans } ( ut dicitur in Politic’ ) \\\hline
2.3.15 & e el amor de bien los inclina asuir . \textbf{ Conuiene que los prinçipes se ayan çerca ellos } assi commo cerca de fijos . & quos virtus et amor boni inclinat ad seruiendum , \textbf{ decet principantes se habere quasi ad filios , } et decet eos regere non regimine seruili , \\\hline
2.3.16 & En essa misma manera cada vna muchedunbre si bien ordenada es \textbf{ conuiene que sea aduchͣa vn ordenador } de quien ella sea ordenada . & si debet esse ordinata , \textbf{ oportet reduci in unum aliquem , } a quo ordinetur . \\\hline
2.3.17 & Lo terçero çerca la prouision delas uestidas es de penssar la condiçion delas personas \textbf{ por que non conuiene que todos sean uestidos } de eguales uestiduras caenta & consideranda est conditio personarum . \textbf{ Nam non omnes decet } habere aequalia indumenta . \\\hline
3.1.1 & commo toda comunidat sea por graçia de algun bien . \textbf{ Conuiene que la çibdat sea establesçida por algun bien } Ca pruena el pho & gratia alicuius boni , \textbf{ oportet ciuitatem ipsam constitutam esse propter aliquod bonum . } Probat autem Philosophus primo Polit’ duplici via , \\\hline
3.1.6 & niguno non abasta assi mismo en fallar algunan arte \textbf{ mas conuiene que sea ayuda de } por ayuda de los que passaronante & in inueniendo artem aliquam , \textbf{ sed oportet ad hoc iuuari } per auxilium praecedentium \\\hline
3.1.8 & por que nos auemos me estermuchͣs cosas departidas para abastamiento dela uida \textbf{ conuiene que enla çibdat sea algun departimiento . } La tercera razon que declara e manifiesta las razones & Quia ergo diuersis indigemus ad vitam , \textbf{ oportet in ciuitate diuersitatem esse . } Tertia via declarans \\\hline
3.1.12 & que son meester para la batalla \textbf{ ca los omes lidiadores conuiene que sean cuerdos } por entendimiento e sabios & secundum tria quae requiruntur ad bellum . \textbf{ Homines enim bellatores decet } esse mente cautos et prouidos : \\\hline
3.2.6 & quando es muy escalentada \textbf{ e muy enraleçida conuiene que la raledat e la calentura mas acabadamente sea fallada en el fuego } despues que fuere engendrado e ençendido . & cum calefit et rarefit , \textbf{ oportet raritatem et calorem perfectius reperiri } in igne iam generato \\\hline
3.2.8 & e la fuente delas esc̀ yturas . \textbf{ conuiene que de ally tome todo el pueblo algun enssennamiento } e de prinda alguna sabiduria . & et fons scripturarum , \textbf{ oportet quod inde totus populus } aliquam eruditionem accipiat : \\\hline
3.2.16 & ca el nuestro consseio non es dela fu . \textbf{ por que conuiene que en el conseio sorongamos la fin } e que non tomemos consseio della & sed de his quae sunt ad finem : \textbf{ oportet enim in consilio | praesupponere finaliter intentum , } et non consiliari de ipso , \\\hline
3.2.18 & para fazer tales cosas \textbf{ Morende para que alguno faga fe delas cosas de que fabla o conuiene que sea sabio } o que sea tenido por sabio . & existimantur talia facere : \textbf{ ideo ad hoc quod aliquis ex rebus | de quibus loquitur fidem faciat , vel oportet quod sit prudens } vel quod credatur esse prudens . \\\hline
3.2.18 & a cuyos dichos creen los omes \textbf{ e es dada feo conuiene que sea bueono } que sea amigo & vel omnis ille cuius dictis creditur \textbf{ et adhibetur fides , | vel oportet quod sit bonus , } vel quod amicus , \\\hline
3.2.19 & e a buen estado del Rey e del pueblo \textbf{ Et pues que assi es conuiene que los } consseierossepan las entradas e las sallidas del regno & et ad bonum statum eius : \textbf{ decet ergo consiliarios } scire introitus \\\hline
3.2.22 & por abortençia o por mal querençia . \textbf{ conuiene que el uiez judgue mal e desigual mente . } Ca entonçe el uuzio non salle de zelo de iustiçia & ab alia vero recedit per odium , \textbf{ oportet ipsum iudicare inique : } quia tunc iudicium non procedit \\\hline
3.2.23 & mayoraspeza e mayor dureza de quanta deue . \textbf{ Conuiene que por el entendimiento piadoso sea atenprada la guaueza dela pena } e esto es lo que dize el pho & sunt amplioris seueritatis contentiua , \textbf{ decet ut per pium intellectum moderetur supplicii magnitudo , } hoc est ergo quod dicitur 1 Rhetor’ \\\hline
3.2.26 & en quanto es conparada ala ley de natura \textbf{ conuiene que sea derechͣ . } Et en quanto es conparada al bien comun & ad legem naturae , \textbf{ oportet quod sit iusta : } ut comparatur ad bonum commune , \\\hline
3.2.26 & el qual pueblo deueser reglado por aquella ley . \textbf{ conuiene que sea conuenible } e que conuenga con el uso & et debet regulari per huiusmodi legem , \textbf{ oportet quod sit competens } et compossibilis consuetudini patriae et tempori : \\\hline
3.2.26 & Ca segunt que estas tales cosas se departen \textbf{ conuiene que enlas leyes sea algun departimiento . } ues que assi es . & quia secundum quod talia diuersificantur , \textbf{ oportet in ipsis legibus } aliquam diuersitatem existere . \\\hline
3.2.26 & ues que assi es . \textbf{ Lo primero conuiene que la ley humanal o positiua sea derecha } en quanto es conparada ala razon natural o ala ley de natura . & aliquam diuersitatem existere . \textbf{ Primo igitur oportet legem humanam | siue positiuam esse iustam } ut comparatur ad rationem naturalem \\\hline
3.2.26 & en la qual es entendido el bien propreo . \textbf{ Ca conuiene que enlas leyes } si derechͣs fueren & in qua intenditur priuatum bonum ; \textbf{ oportet enim in legibus } ( si rectae sint ) \\\hline
3.2.26 & e esto queremos alcançar \textbf{ conuiene que fagamos estas cosas . } Et pues que assi estales deuen ser las leyes & et hoc sequi volumus , \textbf{ oportet hoc agere . } Tales ergo debent esse leges , \\\hline
3.2.26 & porque el bien propra o es ordenado al bien comun . \textbf{ Conuiene que las leyes tales sean non } quales demanda el bien propre & et bonum priuatum ordinetur ad ipsum , \textbf{ oportet tales leges fieri } non quales requirit bonum priuatum , \\\hline
3.2.27 & Poque la ley aya uirtud e fuerça de obligar \textbf{ conuiene que sea publicada e pregonada . } Mas commo otra sea la ley natural e otra la positiua en vna manera se deue publicar la vna & ad hoc quod lex habeat vim obligandi , \textbf{ oportet eam promulgatam esse . } Sed cum alia sit lex naturalis , \\\hline
3.2.29 & generalmente aquello que non es general mente \textbf{ Por que conuiene que las leyes humanales } commo quier que sean examinadas de fallesçer en algun caso . & quod non est uniuersaliter : \textbf{ oportet enim humanas leges } quantumcunque sint exquisitae \\\hline
3.2.29 & assi commo por regla de fierro . \textbf{ Mas conuiene que se reglen con regla de plommo } que se pueda en coruar & ut puta ferrea : \textbf{ sed oportet | quod mensurentur regula plumbea , } quae sit applicabilis humanis actibus . \\\hline
3.2.29 & quela ley demanda o que la ley nidga . \textbf{ Et algunas vezes conuiene que la regla se encorue } ala parte contraria & quam lex dictat : \textbf{ aliquando etiam oportet eam plicare ad partem oppositam , } et rigidius punire peccantem , \\\hline
3.2.30 & comunalmente non puede alcançar forma de beuir en punto . \textbf{ Por ende conuiene que } dessemeie alguons pecados & attingere punctalem formam viuendi , \textbf{ ideo oportet aliqua peccata dissimulare } et non punire lege humana , \\\hline
3.2.30 & por la qual somos ordenados a aquel bien . \textbf{ Et por ende conuiene que los Reyes e los prinçipes } alos quales parte nesçe ser & per quam ordinamur ad illud bonum . \textbf{ Decet ergo reges et principes , } quos competit esse quasi semideos , \\\hline
3.2.32 & que es en el regno e enla çibdat . \textbf{ conuiene que sea atal que biuna bien e uirtuosamente . } Et por ende assi conmo dize el philosofo en el terçero libro delas politicas & in ciuitate et regno , \textbf{ oportet esse talem , | quod viuat bene et virtuose . } Inde est ergo quod ait Philosop’ 3 Politicorum quod magis est ciuis abundans \\\hline
3.2.33 & uenta el philosofo en el quarto libro delas politicas \textbf{ que conuiene que sean tres partes dela çibdat . } Ca alguons son muy ricos . & Quarto Politicorum ait Philosophus , \textbf{ quod tres oportet | esse partes ciuitatis . } Nam alii quidem sunt opulenti valde , \\\hline
3.2.34 & Et la uirtud faze al que la ha buenon \textbf{ Esta buean obra conuiene que sea en el gouernamiento derech } que el buen çibdada no sea buen omne . & cum virtus faciat habentem bonum ; \textbf{ et opus bonum , | oportet in recto regimine , } quod bonus ciuis sit bonus homo ; \\\hline
3.3.1 & por la qual cada vno sabe gouernar la casa e la conpaña . \textbf{ Conuiene que sea otra e departida de la sabiduria } por la qual cada vno sabe gouernar a ssi mismo . & per quam quis scit regere domum et familiam , \textbf{ oportet esse aliam a prudentia , } qua quis nouit seipsum regere . \\\hline
3.3.1 & Et todas estas tres sabidurias \textbf{ conuiene que aya el Rey . } Conuiene a saber . & et gubernare ciues . \textbf{ Omnes autem tres prudentias decet habere Regem , } videlicet particularem , oeconomicam et regnatiuam . \\\hline
3.3.9 & ca los que estan en las huestes \textbf{ conuiene que sufran muchos males . } por la qual cosa si fueren y algunos muelles e mugerilles & erga necessitates corporis . \textbf{ Nam existentes in exercitu oportet multa incommoda tolerare : } quare si sint ibi aliqui molles , \\\hline
3.3.23 & en la batalla de la naue . \textbf{ Ante conuiene que en esta batalla de la } naue sean los omnes . & sic et haec requiruntur in bello nauali . \textbf{ Immo in huiusmodi pugna oportet } homines melius esse armatos , quam in terrestri : \\\hline

\end{tabular}
