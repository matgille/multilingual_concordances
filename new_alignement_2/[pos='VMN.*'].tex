\begin{tabular}{|p{1cm}|p{6.5cm}|p{6.5cm}|}

\hline
1.1.1 & ¶E por ende si del gouernamjento delos prinçipes \textbf{ o delos Reyes entendemos dar arte } e sçiençia conuiene & Si ergo de regimine Principum , \textbf{ siue Regum intendimus artem , } et notitiam tradere , \\\hline
1.1.1 & E primero veamos qual es la manera \textbf{ que deuemos gunardar en esta arte } E segund esto deuemos saber & primo videndum est , \textbf{ quis sit modus procedendi in hac arte . } Sciendum ergo , \\\hline
1.1.1 & que deuemos gunardar en esta arte \textbf{ E segund esto deuemos saber } que en toda la moral philosophia la manera de fablar & quis sit modus procedendi in hac arte . \textbf{ Sciendum ergo , } quod in toto morali negotio modus procedendi \\\hline
1.1.1 & E segund esto deuemos saber \textbf{ que en toda la moral philosophia la manera de fablar } segund el philosofo es figural e gruessa & Sciendum ergo , \textbf{ quod in toto morali negotio modus procedendi } secundum Philosophum est figuralis \\\hline
1.1.1 & segund el philosofo es figural e gruessa \textbf{ Ca conuiene enlas tales cosas vsar de figuras de enxenplos } Ca los fechos morales & et grossus : \textbf{ oportet enim in talibus typo et figuraliter pertransire , } quia gesta moralia complete sub narratione non cadunt . Possumus autem triplici via venari , \\\hline
1.1.1 & so rrecontamjento onde \textbf{ por tres cosas podemos mostrar } que La manera que deuemos tener en esta arte & ø \\\hline
1.1.1 & por tres cosas podemos mostrar \textbf{ que La manera que deuemos tener en esta arte } e en esta sçiençia conujene que sea figural e gruesa , & quia gesta moralia complete sub narratione non cadunt . Possumus autem triplici via venari , \textbf{ quod modum procedendi in hac scientia oportet esse figuralem et grossum . } Prima via sumitur ex parte materiae , \\\hline
1.1.1 & e le moral ph̃ia \textbf{ asi commo dicho es non puede auer estrudiñamjento sotil } Ca es delos negoçios delos fechon s singulares delos omes & quia materia moralis \textbf{ ( ut dictum est ) | non patitur perscrutationem subtilem , } sed est de negociis singularibus : quae \\\hline
1.1.1 & enel segundo libro delas ethicas \textbf{ non pueden aver certidunbre de Razon } por la mudaçion & ( ut declarari habet 2 Ethicorum ) propter sui variabilitatem , \textbf{ magnam incertitudinem habent . } Quia ergo sic est , \\\hline
1.1.1 & e se mudan de cadal dia \textbf{ Asi demuestran que deuemos en ellas tener maneras de figuras e de exenplos } E esta rrazon tañe el philosopho en eL segundo libro delas ethicas & quae sunt materia huius operis , \textbf{ ostendunt incedendum esse figuraliter et typo . } Hanc autem rationem videtur tangere Philosophus 1 Ethicorum , \\\hline
1.1.1 & Onde dize el philosofo mas adelante \textbf{ que de omne sabio es en tanto demandar çertidunbre de cada cosa } en quanto la naturaleza dessa mismͣ cosa lo demanda & quod disciplinati est , \textbf{ intantum certitudinem inquirere | secundum unumquodque genus , } inquantum natura rei recipit . Videtur enim natura \\\hline
1.1.1 & e Delas obras delos omnes son superfiçiales e gruessas \textbf{ donde se sigue quel geometrico non ha de amonestar } mas de demostrar & ut ait Coment’ 2 Met’ rationes vero morales sunt superficiales \textbf{ et grossae . Geometrae igitur est non persuadere , sed demonstrare : } Rhetoris vero , et Politici , \\\hline
1.1.1 & donde se sigue quel geometrico non ha de amonestar \textbf{ mas de demostrar } E el Rethorico e moral non ha de Demostraͬ & ut ait Coment’ 2 Met’ rationes vero morales sunt superficiales \textbf{ et grossae . Geometrae igitur est non persuadere , sed demonstrare : } Rhetoris vero , et Politici , \\\hline
1.1.1 & mas de demostrar \textbf{ E el Rethorico e moral non ha de Demostraͬ } mas de amonestar & et grossae . Geometrae igitur est non persuadere , sed demonstrare : \textbf{ Rhetoris vero , et Politici , | non est demonstrare , } sed persuadere . \\\hline
1.1.1 & E el Rethorico e moral non ha de Demostraͬ \textbf{ mas de amonestar } Por la qual rrazon dize a philosofo enel primero libro delas ethicas & non est demonstrare , \textbf{ sed persuadere . } Propter quod 1 Ethicorum scribitur , \\\hline
1.1.1 & que semeie ante e egual pecado es \textbf{ quel mathematico tiente de amonestar } E el rretorico quiera & Propter quod 1 Ethicorum scribitur , \textbf{ quod per peccatum est , mathematicum persuadentem acceptare , } et rhetoricum demonstrationes expetere . \\\hline
1.1.1 & non por grande contenplaçion \textbf{ e de saber } mas por que obremos bien & non contemplationis gratia , \textbf{ neque ut sciamus , } sed ut boni fiamus . Finis ergo intentus in hac scientia , \\\hline
1.1.1 & que se entiende en esta sçiençia non es conosçimjento \textbf{ mas obra njn es por graçia de buscar verdad delas cosas } mas por saber la bondad Dellas & non est sui negocii cognitio , sed opus : \textbf{ nec est veritas , } sed bonum . \\\hline
1.1.1 & mas obra njn es por graçia de buscar verdad delas cosas \textbf{ mas por saber la bondad Dellas } E pues que asi es commo las Razones sotiles & nec est veritas , \textbf{ sed bonum . } Cum ergo rationes subtiles \\\hline
1.1.1 & que enlas sçiençias speculatiuas \textbf{ enlas quales se ha De alunbrar prinçipalmente el entendimjento } ¶ Auemos de yr por Demostraçiones e sotilmente Mas enlas sçiençias morales & in scientiis speculatiuis , \textbf{ ubi principaliter quaeritur illuminatio intellectus , } procedendum est demonstratiue \\\hline
1.1.1 & enlas quales se ha De alunbrar prinçipalmente el entendimjento \textbf{ ¶ Auemos de yr por Demostraçiones e sotilmente Mas enlas sçiençias morales } enlas quales deuemos buscar derechura de bondad & ubi principaliter quaeritur illuminatio intellectus , \textbf{ procedendum est demonstratiue | et subtiliter in negocio morali , } ubi quaeritur rectitudo voluntatis , \\\hline
1.1.1 & ¶ Auemos de yr por Demostraçiones e sotilmente Mas enlas sçiençias morales \textbf{ enlas quales deuemos buscar derechura de bondad } por que seamos buenos deuemos yr por Amonestaçiones e por figuras ¶ & et subtiliter in negocio morali , \textbf{ ubi quaeritur rectitudo voluntatis , | et } ut boni fiamus , \\\hline
1.1.1 & enlas quales deuemos buscar derechura de bondad \textbf{ por que seamos buenos deuemos yr por Amonestaçiones e por figuras ¶ } Onde dize el philosopho enel primero libro delas ethicas & et \textbf{ ut boni fiamus , | procedendum est persuasiue et figuraliter . } Unde 1 Ethicorum scribitur , morale negocium amabile de talibus et ex talibus dicentes , \\\hline
1.1.1 & que conteçe munchans vezes \textbf{ e non sienpre ouieremos demostrar la v̉dad de ella gruessamente } e por figuras ¶ la terçera rrazon se toma de parte del oydor & et de iis quae sunt \textbf{ ut frequentius , grossae et figuraliter veritatem ostendere . Tertia via sumitur ex parte } auditoris , qui erudiendus in hac arte . \\\hline
1.1.1 & Ca maguerque el titulo deste libro sea del enseñamjento delos prinçipes \textbf{ enpero todo el pueblo se ha de enseñar por este libro } Ca commo quier que cada vno non pueda ser rrey & Nam licet intitulatus sit hic liber de eruditione Principum , \textbf{ totus tamen populus erudiendus est per ipsum . } Quamuis enim non quilibet possit esse Rex vel Princeps : \\\hline
1.1.1 & Ca commo quier que cada vno non pueda ser rrey \textbf{ nin prinçipe empero cada vno deue studiar } quanto pudiere & Quamuis enim non quilibet possit esse Rex vel Princeps : \textbf{ quilibet tamen summopere studere debet , } ut talis sit , \\\hline
1.1.1 & quanto pudiere \textbf{ que se tal que sea digno para gouernar } e prinçipaͬ la qual cosa non puede ser sinon supier & ut talis sit , \textbf{ quod dignus sit regere et principari , } quod esse non potest , \\\hline
1.1.1 & que se tal que sea digno para gouernar \textbf{ e prinçipaͬ la qual cosa non puede ser sinon supier } e non guardare todas aquellas cosas & quod dignus sit regere et principari , \textbf{ quod esse non potest , | nisi sciantur , } et obseruentur , \\\hline
1.1.1 & e non guardare todas aquellas cosas \textbf{ que sean de dezjr en este libro , } ¶ E pues que asi es todo el pueblo deue seer Oydor & et obseruentur , \textbf{ quae in hoc opere sunt dicenda , } totus ergo populus auditor quodammodo est huius artis , \\\hline
1.1.1 & asi commo demuestra el philosofo enel primero libro dela recthorica¶ \textbf{ E pues que asi es commo todo el pueblo pueda entender las cosas sotiles deuemosyr en este libro } por enxenplos E gruesamente & ut ostendit in 1 Rhetoricorum . \textbf{ Cum igitur totus populus subtilia comprehendere non possit , } incedendum est in morali negocio figuraliter et grosse . \\\hline
1.1.1 & Ca Segund dize el philosopho enlas politicas \textbf{ que aquellas cosas que conujene al Senonr de saber mandar essas mesmas conujene al subdito De saber fazer ¶ } E si por qual manera Deuen mandar a los sus Subditos & Immo quia ( secundum Philosophum in Politicis ) \textbf{ quae oportet dominum scire praecipere , | haec oportet subditum scire facere : } si per hunc librum instruuntur Principes , \\\hline
1.1.1 & que aquellas cosas que conujene al Senonr de saber mandar essas mesmas conujene al subdito De saber fazer ¶ \textbf{ E si por qual manera Deuen mandar a los sus Subditos } conujene esta doctrina & si per hunc librum instruuntur Principes , \textbf{ quomodo debeant se habere , | et qualiter debeant suis subditis imperare , } oportet doctrinam hanc extendere usque ad populum , \\\hline
1.1.1 & conujene esta doctrina \textbf{ e esta sçiençia estender la fasta el pueblo } por que Sepa commo ha de obedesçer a sus prinçipes & et qualiter debeant suis subditis imperare , \textbf{ oportet doctrinam hanc extendere usque ad populum , } ut sciat qualiter debeat suis Principibus obedire . \\\hline
1.1.1 & e esta sçiençia estender la fasta el pueblo \textbf{ por que Sepa commo ha de obedesçer a sus prinçipes } E por que esto non puede ser asi commo Dicho nes & oportet doctrinam hanc extendere usque ad populum , \textbf{ ut sciat qualiter debeat suis Principibus obedire . } Et quia hoc fieri non \\\hline
1.1.1 & E por que esto non puede ser asi commo Dicho nes \textbf{ Si non por Razones superfiçiales e sensibles Conuie ne que la manera que deuemos tener enesta obra sea gruesa e figural e exenplar } a asi commo dize el philosopho en el primer libro & ( \textbf{ ut tactum est ) nisi per rationes superficiales | et sensibiles : } oportet modum procedendi in hoc opere , \\\hline
1.1.2 & asi commo el conosçimiento del entendimjento nasçe del conosçimjento Delos sesos \textbf{ Por ende buena cosa es de Recontar la orden de las cosas } que se han de dezir & ut dicitur 1 Posteriorum , \textbf{ bene se habet narrare ordinem dicendorum , } ut de ipsis quaedam praecognitio habeatur . \\\hline
1.1.2 & Por ende buena cosa es de Recontar la orden de las cosas \textbf{ que se han de dezir } por que dellas podamos tomar algun conosçimjento & bene se habet narrare ordinem dicendorum , \textbf{ ut de ipsis quaedam praecognitio habeatur . } Hac enim praecognitione praehabita , \\\hline
1.1.2 & que se han de dezir \textbf{ por que dellas podamos tomar algun conosçimjento } E este conoscimjento auido el entendimjento delas cosas & bene se habet narrare ordinem dicendorum , \textbf{ ut de ipsis quaedam praecognitio habeatur . } Hac enim praecognitione praehabita , \\\hline
1.1.2 & E este conoscimjento auido el entendimjento delas cosas \textbf{ que se aqui han de dezir } mas ligeramente se podra auer & ut de ipsis quaedam praecognitio habeatur . \textbf{ Hac enim praecognitione praehabita , } intellectus dicendorum facilius capietur . \\\hline
1.1.2 & que se aqui han de dezir \textbf{ mas ligeramente se podra auer } Pues que asi es conuiene de saber & Hac enim praecognitione praehabita , \textbf{ intellectus dicendorum facilius capietur . } Sciendum ergo , \\\hline
1.1.2 & mas ligeramente se podra auer \textbf{ Pues que asi es conuiene de saber } que todo este libro entendemos partir en tres libros particulares & intellectus dicendorum facilius capietur . \textbf{ Sciendum ergo , } quod hunc totalem librum intendimus in tres partiales libros diuidere . \\\hline
1.1.2 & Pues que asi es conuiene de saber \textbf{ que todo este libro entendemos partir en tres libros particulares } ¶ & Sciendum ergo , \textbf{ quod hunc totalem librum intendimus in tres partiales libros diuidere . } In quorum primo ostendetur , \\\hline
1.1.2 & que es el Rey \textbf{ e enpos ella cada vno delos omnes ha de gouernar asi mismo ¶ } E enel segundo mostraremos & In quorum primo ostendetur , \textbf{ quomodo quilibet homo seipsum regere debeat . In secundo vero manifestabitur , } quomodo debeat suam familiam gubernare . \\\hline
1.1.2 & commo deue el Rey \textbf{ e Cada vno delos otros gouernar su conpaña } ¶E enel terçero declaremos & quomodo quilibet homo seipsum regere debeat . In secundo vero manifestabitur , \textbf{ quomodo debeat suam familiam gubernare . } In tertio autem declarabitur , \\\hline
1.1.2 & ¶E enel terçero declaremos \textbf{ Commo deue el rrey enseñorear gouernar la çibdad } e EL rreyno¶ & In tertio autem declarabitur , \textbf{ quomodo praeesse debeat ciuitati , } et regno . Primo ergo libro deseruiet Ethica siue Monastica . Secundo Oeconomica . Tertio Politica . \\\hline
1.1.2 & a quell manyconomica \textbf{ que quiere dezjr gouernamjento de conpanans de casa ¶ al terçero libro sirue la politica } que es sçiençia de gouernamjento delas çibdades e del rreyno ¶ & rationalis , et naturalis . \textbf{ Rationalis quidem , } quoniam ea , \\\hline
1.1.2 & que es sçiençia de gouernamjento delas çibdades e del rreyno ¶ \textbf{ E deuedes saber } que esta orden es de rrazon & Rationalis quidem , \textbf{ quoniam ea , } quae sunt ad alterum , \\\hline
1.1.2 & que asi se ha a todo su amjgo a quien ama commo asi mjsmo pues que asi es aquello \textbf{ que es dicho dela amistança ha verdad dela sabiduria de gouernar } Ca aquel que quiere ser sabidor & qui sic se habet ad amicum cui amicatur , ut ad seipsum . \textbf{ Quod ergo dictum est de amicabilitate , | veritatem habet de ipsa industria regiatiua , } qui enim industris esse vult \\\hline
1.1.2 & Ca aquel que quiere ser sabidor \textbf{ para gouernar los otros deue ser sabidor } para gouernar asi mismo & qui enim industris esse vult \textbf{ ut alios regat , | debet industris esse } ut seipsum gubernet : \\\hline
1.1.2 & para gouernar los otros deue ser sabidor \textbf{ para gouernar asi mismo } Por la qual cosa de rrazon es & debet industris esse \textbf{ ut seipsum gubernet : } quare rationabile est , \\\hline
1.1.2 & e despues alo mas conplido ¶ \textbf{ E asy dandose a aprender } e a especulaçion continuadamente aprouecha en La sçiençia & et cognitionem imperfectam , \textbf{ et postea habeat eam perfectiorem : } et sic dando se speculationi , continue in scientiam perficitur , donec secundum modum sibi possibilem habeat perfectam notitiam . \\\hline
1.1.2 & fasta que segun la su manera \textbf{ e quanto el puede aya de venir asçia conplida¶ } Pues que asi es aquello & ø \\\hline
1.1.2 & qunata es demandada \textbf{ para gouernar la conpaña } njn sea menester tanta sabiduria & ø \\\hline
1.1.2 & Segund orden natural ala rreal magestad primeramente \textbf{ que el Ruy sepa gouernar asy mesmo ¶ } Lo segundo que sepa gouernar su conpanna¶ & quanta in gubernatione ciuitatis et regni : ordine naturali decet regiam maiestatem \textbf{ primo | scire se ipsum regere , } secundo scire suam familiam gubernare , tertio \\\hline
1.1.2 & que el Ruy sepa gouernar asy mesmo ¶ \textbf{ Lo segundo que sepa gouernar su conpanna¶ } Lo terçero & scire se ipsum regere , \textbf{ secundo scire suam familiam gubernare , tertio } scire regere regnum , \\\hline
1.1.2 & Lo terçero \textbf{ que sepa gouernả su rregno e sus çibdades ¶ pues que asy es en el primo libro } en el qual tractaremos del gouerna mjeto del omne . & scire regere regnum , \textbf{ et ciuitatem . In primo autem libro in quo agetur de regimine sui , } sunt quatuor declaranda . \\\hline
1.1.2 & en el qual tractaremos del gouerna mjeto del omne . \textbf{ En sy mesmo son quatro cosas de declarar e de demostrͣ } Ca primamente demostrͣemos & et ciuitatem . In primo autem libro in quo agetur de regimine sui , \textbf{ sunt quatuor declaranda . } Nam Primo ostendetur in quo regia maiestas debeat suum finem , \\\hline
1.1.2 & pon su fin e su bienandança¶ \textbf{ Lo segundo demostrͣemos quales uertudes deue auer el Rey } e el gouernador ¶ & et suam felicitatem ponere . \textbf{ Secundo quas virtutes debeat habere , } tertio quas passiones debeat sequi . \\\hline
1.1.2 & e el gouernador ¶ \textbf{ Lo terçero demostrͣemos quales passiones deue segnir } o quales non ¶ & Secundo quas virtutes debeat habere , \textbf{ tertio quas passiones debeat sequi . } Quarto quos mores debeat imitari . \\\hline
1.1.2 & o quales non ¶ \textbf{ Lo quato quales costunbres deue auer } e quales non & tertio quas passiones debeat sequi . \textbf{ Quarto quos mores debeat imitari . } Nam cum bene vivere , \\\hline
1.1.2 & e quales non \textbf{ Ca commo bien beuir e bien gouernar } asy mesmo non pue da ser & Quarto quos mores debeat imitari . \textbf{ Nam cum bene vivere , | et bene regere seipsum , } esse non possit , \\\hline
1.1.2 & por orden de Razon \textbf{ el que quiere tractar del gouernaiento } e fablarde sy mesmo conujene de tractar e de dar conosçimiento & et bonis operibus regulatis ordine rationis : \textbf{ volens tractare de regimine sui , } oportet ipsum notitiam tradere de omnibus his quae diuersificant mores \\\hline
1.1.2 & el que quiere tractar del gouernaiento \textbf{ e fablarde sy mesmo conujene de tractar e de dar conosçimiento } de todas aquellas cosas & volens tractare de regimine sui , \textbf{ oportet ipsum notitiam tradere de omnibus his quae diuersificant mores } et actiones . Inde est ergo , quod vult Philosophus 2 Ethic’ proficuum esse morali negocio , \\\hline
1.1.2 & que prouechosa cosa es en la scian moral \textbf{ e en la sçiençia de costunbres escod̀nar } aquellas cosas & ø \\\hline
1.1.2 & que son cerca delas obras \textbf{ commo las deue omne fazer } Mas las nr̃as obras & ø \\\hline
1.1.2 & quanto alo prisente parte nesçe de quatro gujsas \textbf{ e de quatro maneras las veemos nasçer } e departir ¶ primeramente de parte delas fines & quomodo faciendum sit eas . Operationes autem nostrae ex quatuor \textbf{ ( quantum ad praesens spectat ) videntur oriri , et diuersificari ; } videlicet , \\\hline
1.1.2 & e de quatro maneras las veemos nasçer \textbf{ e departir ¶ primeramente de parte delas fines } que entieden¶ & quomodo faciendum sit eas . Operationes autem nostrae ex quatuor \textbf{ ( quantum ad praesens spectat ) videntur oriri , et diuersificari ; } videlicet , \\\hline
1.1.2 & pues que asy es para saber \textbf{ que deuemos obrar muy aprouechable cosa es de saber } que fin deuemos entender & et alia operatur . \textbf{ Ad sciendum ergo quae operari debemus , maxime proficuum esse videtur , } quem finem nobis praestituere debeamus . Rursus \\\hline
1.1.2 & que deuemos obrar muy aprouechable cosa es de saber \textbf{ que fin deuemos entender } e en que deuemos poner nr̃afin¶ & Ad sciendum ergo quae operari debemus , maxime proficuum esse videtur , \textbf{ quem finem nobis praestituere debeamus . Rursus } quia \\\hline
1.1.2 & que fin deuemos entender \textbf{ e en que deuemos poner nr̃afin¶ } Otrosy lo segundo se departen las obras & Ad sciendum ergo quae operari debemus , maxime proficuum esse videtur , \textbf{ quem finem nobis praestituere debeamus . Rursus } quia \\\hline
1.1.2 & Ca segunt que dize el philosofo en el segundo libro delas ethicas señal dela disponiconno delascina engendoͣda en el alma \textbf{ es auer en la obra deleta çion o tristeza } por que segunt que auemos departidas dispoçones & ut dicitur 2 Ethicorum ) signum generati habitus , \textbf{ est delectationem , | et tristitiam fieri in opere , } secundum quod alios et alios habitus habemus , \\\hline
1.1.2 & en las obras departidas delecta connes ¶ \textbf{ Lo terçero deuedes saber } que non solamente son departidos las obras & et aliis actibus delectamur . \textbf{ Tertio diuersificantur actiones , } et opera non solum ex finibus vel ex habitibus , \\\hline
1.1.2 & e fuyen dela batalla \textbf{ mas los que han esperanço de vençer acometen los enemjgos } e entra en łlos rreziamente & turpiter agunt , et dimittunt aciem , \textbf{ et fugiunt de bello : sperantes autem se vincere , inuadunt bellantes , } et aggrediuntur hostes . \\\hline
1.1.2 & que dicho es dela esperança \textbf{ e del temor esso mesmo se deue entender } de todas las otras parassiones & Quod ergo dictum est de spe , et timore , \textbf{ intelligendum est de aliis passionibus : } singulae enim affectiones \\\hline
1.1.2 & que estas quatro cosas dichas han alguna conparaçion entre sy \textbf{ por que departidas costunbres han de nasçer departidos pasiones } Ce de departidas pasiones se leuna tan deꝑrtidas disposiçonnes e sçiençias nasçen departidos fines aque son ordenadas & et creditiui . Videntur autem haec quatuor habere aliquam analogiam adinuecem . \textbf{ Nam ex aliis , et aliis motibus , | habent esse aliae , } et aliae passiones : \\\hline
1.1.2 & asy commo aquel que es destenprado en los desseos dela carne paresçe \textbf{ que toda su bien andança es usar de delectaçiones carnałs bien asy ahun los que han otras disposiconnes departidas son inclinados por ellos a escoger otros fines concordables } alas sus dipo inconnes & et magna felicitas , \textbf{ sit uti venereis voluptatibus : | sicut etiam habentes alios habitus inclinantur , } ut finem sibi praestituant conformem suo habitui . \\\hline
1.1.2 & que dichͣ s son \textbf{ ¶Conujene a saber } dela fin o dela bienandança de los prinçipes & videlicet , de fine , \textbf{ siue de felicitate Principum , } de eorum virtutibus , \\\hline
1.1.2 & por que la fin en conparaçion delas obras \textbf{ que se ha de fazer } es & Primo tamen dicemus de ipso fine siue de ipsa felicitate : \textbf{ quia finis respectu agendorum , est principalius principium , } quam aliquod aliorum . \\\hline
1.1.3 & or que asy commo dicho es esta obra tomamos \textbf{ e comneçamos a fazer porgera de ensseñar los pnçipes } et commo nunca el prinçipen ̃j otro njnguno conplidamente puenda ser enssennado & Quoniam ( ut dictum est ) \textbf{ opus istud suscepimus gratia eruditionis Principum : } cum nunquam quis plene erudiatur , \\\hline
1.1.3 & sy non fueᷤ begniuolo e uolenteroso \textbf{ para a prinder doçible } e engennoso para preguntar actento & cum nunquam quis plene erudiatur , \textbf{ nisi sit beniuolus , docilis , et attentus : } postquam in primo capitulo reddidimus regiam maiestatem beniuolam , \\\hline
1.1.3 & para a prinder doçible \textbf{ e engennoso para preguntar actento } e acuçioso para rretener e tomar & cum nunquam quis plene erudiatur , \textbf{ nisi sit beniuolus , docilis , et attentus : } postquam in primo capitulo reddidimus regiam maiestatem beniuolam , \\\hline
1.1.3 & e engennoso para preguntar actento \textbf{ e acuçioso para rretener e tomar } Et pues que ya en el primero capitulo fizimos la magestad Real begniuola & nisi sit beniuolus , docilis , et attentus : \textbf{ postquam in primo capitulo reddidimus regiam maiestatem beniuolam , } ostendendo , quae dicenda sunt , \\\hline
1.1.3 & e uolunterosa mostradol aquellas cosas \textbf{ que son de dezer en este libro } que nos prometiemos de tractar ligniamente ¶ & postquam in primo capitulo reddidimus regiam maiestatem beniuolam , \textbf{ ostendendo , quae dicenda sunt , } nos esse faciliter tractaturos : \\\hline
1.1.3 & que son de dezer en este libro \textbf{ que nos prometiemos de tractar ligniamente ¶ } et en el segundo capitulo fiziemos essa mjsma magestad doçible & ostendendo , quae dicenda sunt , \textbf{ nos esse faciliter tractaturos : } et in secundo reddidimus eam docilem , \\\hline
1.1.3 & et en el segundo capitulo fiziemos essa mjsma magestad doçible \textbf{ e engannosa contando la orden de las cosas que aqui auemos de dezir ¶finca que en este terçero capitulo fagamos la Real magestad atenta } e acuçiosa declarandol & narrando ordinem dicendorum : \textbf{ restat | ut in hoc capitulo tertio reddamus eam attentam , } declarando quanta sit utilitas in dicendis . \\\hline
1.1.3 & quanto es el prouecho delas cosas \textbf{ que aqui auemos a dezir . } mas por que comunalmente aborresçen los omes los sermones escd̀innadores & declarando quanta sit utilitas in dicendis . \textbf{ Nam } quia communiter homines odiunt sermonem perscrutatum , \\\hline
1.1.3 & mas por la orden de lans cosas \textbf{ que son de dezer le faremos doçible e engeñoso } por que cada vno mayormente es fechͣo ensennable e engennoso & In huiusmodi ergo arte ex facilitate tradendi redditur auditor beniuolus : \textbf{ sed ex ordine dicendorum redditur docilis ; } nam quis maxime efficitur docilis idest habilis ad capiendum doctrinam , \\\hline
1.1.3 & por que cada vno mayormente es fechͣo ensennable e engennoso \textbf{ para aprender la doctrina } sy las cosas quelan de dezir & sed ex ordine dicendorum redditur docilis ; \textbf{ nam quis maxime efficitur docilis idest habilis ad capiendum doctrinam , } si ei dicenda quadam serie , \\\hline
1.1.3 & para aprender la doctrina \textbf{ sy las cosas quelan de dezir } sy las pusiern ordenadamente & nam quis maxime efficitur docilis idest habilis ad capiendum doctrinam , \textbf{ si ei dicenda quadam serie , } et ordine proponantur , \\\hline
1.1.3 & e por buena orden \textbf{ Mas por el prouecho delas cosas que son de dezer es fecho el oydor atento } e acuçioso para aprender & et ordine proponantur , \textbf{ ex utilitate autem dicendorum redditur auditor attentus , } nam quilibet attente audit , \\\hline
1.1.3 & Mas por el prouecho delas cosas que son de dezer es fecho el oydor atento \textbf{ e acuçioso para aprender } por que cada vno acuçiosamente oye & et ordine proponantur , \textbf{ ex utilitate autem dicendorum redditur auditor attentus , } nam quilibet attente audit , \\\hline
1.1.3 & ¶ pues que asy es delas cosas \textbf{ que aqui son de dezir } sy los Reys o los prinçipes derechamente las sopieren & si \textbf{ recte cognoscantur , } et debite obseruentur , \\\hline
1.1.3 & commo deuen \textbf{ segnir se les an quatro cosas } en quanto ꝑtenesçe aeste presente arte & consequitur maiestas regia \textbf{ ( quantum ad praesens spectat ( quatuor , } quae quilibet maxime amare , \\\hline
1.1.3 & en quanto ꝑtenesçe aeste presente arte \textbf{ Las quales cosas mucho deue cada vno amar } e mucho dessear¶ & ( quantum ad praesens spectat ( quatuor , \textbf{ quae quilibet maxime amare , } et desiderare debet . \\\hline
1.1.3 & e los poderios del alma \textbf{ Ca en estos bienes puede auer los malos parte } tan bien commo los buenos ¶ & huiusmodi sunt industria mentis , ingenium naturale , potentiae animae : \textbf{ his enim bonis etiam ipsi mali participant . } Bona vero maxima , \\\hline
1.1.3 & Mas los muy gran dos bienes son los que son ascondidos e ençerrados \textbf{ En los quales los malos non pueden auer parte } Et estos son las uertudes delas quales njnguon non puede mal vsar & sunt bona interiora , \textbf{ quae mali participare non possunt : } cuiusmodi sunt virtutes , \\\hline
1.1.3 & En los quales los malos non pueden auer parte \textbf{ Et estos son las uertudes delas quales njnguon non puede mal vsar } segunt dizen los scons e los philosofos ¶ & quae mali participare non possunt : \textbf{ cuiusmodi sunt virtutes , | quibus ( secundum sanctos , } et etiam secundum Philosophos ) \\\hline
1.1.3 & Ca dize el philosofo ennl primero libro dela rretorica \textbf{ que las uirtu dessolas son aquells de quien njnguno o puede mal vsar ¶ } En la segunda manera se departen los bienes & dicitur enim 1 Rhetor’ , \textbf{ quod solae virtutes sunt , } quibus non contingit male uti . Alio modo distinguuntur bona , \\\hline
1.1.3 & e commo conuiene alos rreys \textbf{ que aquellos que han de gouernar a y los de enduzir e traher a honetad e uirtud ¶ } El qua Muy grande es el prouecho en las cosas & et quomodo eos , \textbf{ quos habet regere , inducat ad honestatem et virtutem : } maxima est utilitas in dicendis , \\\hline
1.1.3 & El qua Muy grande es el prouecho en las cosas \textbf{ que se aqui han de dezir } Ca estas cosas guardadas podran auer los bienes honestos & maxima est utilitas in dicendis , \textbf{ quia , eis obseruatis , } habebuntur bona maxima , \\\hline
1.1.3 & que se aqui han de dezir \textbf{ Ca estas cosas guardadas podran auer los bienes honestos } e los muy grandes bienes ¶ & quia , eis obseruatis , \textbf{ habebuntur bona maxima , } et honesta . Secundo est maxima utilitas in dicendis , \\\hline
1.1.3 & ca non solamente delas cosas \textbf{ que se aqui han de dezir } ganara el omneo el oydor los muy gerades bienes & et honesta . Secundo est maxima utilitas in dicendis , \textbf{ quia ex eis non solum quis lucrabitur maxima bona , } sed etiam lucrabitur seipsum . Est enim morale negocium \\\hline
1.1.3 & Et el malo ha mengua de sy mesmo \textbf{ Ca asy lo deuemos ymaginar } que asy conmo es la çibdat & bonus autem vir seipsum habet , \textbf{ malus autem seipso caret . Sic enim imaginari debemus , } quia sicut in ciuitate , \\\hline
1.1.3 & por sy mesmo peren \textbf{ asy es asy commo el rrey non puede auer el Regno } njn el caudiello non puede auer la çibdat & qui est rationalis per essentiam : \textbf{ sicut ergo rex non dicitur habere regnum , } nec dux dicitur habere ciuitatem , \\\hline
1.1.3 & asy es asy commo el rrey non puede auer el Regno \textbf{ njn el caudiello non puede auer la çibdat } sy en el rregno o en la çibdat ouiere discordia & sicut ergo rex non dicitur habere regnum , \textbf{ nec dux dicitur habere ciuitatem , } si in regno vel ciuitate sunt aliqui , \\\hline
1.1.3 & o sy los moradores dela çibdat o del rregno non obedesçieren al rrey o al cabdiello \textbf{ asy qual quier omne singular non puede auer asy mesmo sy el apetito discordare dela Razon e del entendemjento } Et sy los sesos conosçedores & qui non obediant regi , vel duci : \textbf{ sic homo aliquis singularis dicitur non habere seipsum , | si appetitus dissentiat rationi , } et si rationale per participationem non obediat rationali per essentiam : \\\hline
1.1.3 & asy mesmo es digno \textbf{ que sea fecho gouernador e senonr de los otros } Ca el que ha sabiduria & dignus est , \textbf{ ut efficiatur rector , | et dominus aliorum . } Nam vigens prudentia , \\\hline
1.1.3 & prinçipe \textbf{ njn de gouernar a ninguno } et puesto que enssennore & ø \\\hline
1.1.3 & Et pues que asy es tanto es el prouech̃o en estas cosas \textbf{ que aqui se han de dezer en este libro } que ellas guardadas ganaremos los bien es muy grandes & propter quod conformantur primo Principio , et in seipsis ipsum Deum habere dicuntur . \textbf{ Tanta igitur est utilitas in dicendis , quod , eis obseruatis , lucrabimur maxima bona , } nos ipsos , et etiam alios : \\\hline
1.1.3 & Mas por que estas cosas \textbf{ de que auemos de dar doctrina e sabiduria non se pueden auer } njn guardar & et per consequens felicitatem aeternam . \textbf{ Verum quia ea , quorum trademus notitiam , } absque diuina gratia obseruari non possunt , decet quemlibet hominem , et maxime regiam maiestatem \\\hline
1.1.3 & de que auemos de dar doctrina e sabiduria non se pueden auer \textbf{ njn guardar } syn la gera de dios conviene de cada vn omne & et per consequens felicitatem aeternam . \textbf{ Verum quia ea , quorum trademus notitiam , } absque diuina gratia obseruari non possunt , decet quemlibet hominem , et maxime regiam maiestatem \\\hline
1.1.3 & tanto ha mas mester la gera de dios \textbf{ por que pueda usar de obras de uirtudes } e por que pueda enduzir e traher los sus subienctos a uirtudes . & tanto magis indiget diuina gratia , \textbf{ ut possit virtutum opera exercere , } et sibi subditos valere inducere ad virtutem . \\\hline
1.1.3 & por que pueda usar de obras de uirtudes \textbf{ e por que pueda enduzir e traher los sus subienctos a uirtudes . } puestos ya vnos preanbulos neçesarios al proposito & ut possit virtutum opera exercere , \textbf{ et sibi subditos valere inducere ad virtutem . | Quot sunt modi viuendi , } et quomodo in \\\hline
1.1.4 & feziemos la Real magestad begniuola et uolunterosa \textbf{ para oyr e a prinder } por la liger eza dela manera de tractar e de fablar & quia respectu sequentis operis \textbf{ ex facilitate modi tradendi reddidimus regiam maiestatem beniuolam , } ex ordine dicendorum reddidimus eam docilem , \\\hline
1.1.4 & para oyr e a prinder \textbf{ por la liger eza dela manera de tractar e de fablar } Et ahun diemos la & quia respectu sequentis operis \textbf{ ex facilitate modi tradendi reddidimus regiam maiestatem beniuolam , } ex ordine dicendorum reddidimus eam docilem , \\\hline
1.1.4 & e fiziemos la ensennable e engennosa \textbf{ para preguntar } por la orden delas cosas & ø \\\hline
1.1.4 & por la orden delas cosas \textbf{ que auemos de dezir } Et aun fiziemos la actenta e acuçiosa & ex facilitate modi tradendi reddidimus regiam maiestatem beniuolam , \textbf{ ex ordine dicendorum reddidimus eam docilem , } ex utilitate reperta in eis reddidimus ipsam attentam , \\\hline
1.1.4 & Et aun fiziemos la actenta e acuçiosa \textbf{ para tomar e rretener } por el prouecho & ex ordine dicendorum reddidimus eam docilem , \textbf{ ex utilitate reperta in eis reddidimus ipsam attentam , } restat dicere seriatim quae in hoc opere sunt dicenda . \\\hline
1.1.4 & que se falla enestas cosas \textbf{ que auemos de dezer fincanos de dezer ordenadamente } que cosas son de dezir en esta obra e en este libro ¶ Mas por que la fin es comjenço & ex utilitate reperta in eis reddidimus ipsam attentam , \textbf{ restat dicere seriatim quae in hoc opere sunt dicenda . } Verum quia finis est principium agibilium principalius , \\\hline
1.1.4 & que auemos de dezer fincanos de dezer ordenadamente \textbf{ que cosas son de dezir en esta obra e en este libro ¶ Mas por que la fin es comjenço } mas prinçipal de todas las obras & restat dicere seriatim quae in hoc opere sunt dicenda . \textbf{ Verum quia finis est principium agibilium principalius , } quam aliquod aliorum , \\\hline
1.1.4 & asy commo dicho es de suso \textbf{ por ende auemos a comneçar en la fin } e en la bien andança de todos los bienes obrantes . & ut dicebatur supra , \textbf{ ideo a fine et felicitate inchoandum est . } Cum ergo \\\hline
1.1.4 & e orden en asy mesmos a departidas fines \textbf{ por ende conujene de contar las maneras de beujr } e mostrͣemos commo enllas es de poner la bien andança & secundum diuersos modos viuendi diuersi diuersimode sibi finem praestituant , \textbf{ narrandi sunt modi viuendi , } et ostendendum est , \\\hline
1.1.4 & por ende conujene de contar las maneras de beujr \textbf{ e mostrͣemos commo enllas es de poner la bien andança } que es la fin de los buenos ¶ & narrandi sunt modi viuendi , \textbf{ et ostendendum est , } quomodo in eis felicitas est ponenda . Distinxerunt autem Philosophi ( ut patet ex 1 Ethic’ ) triplicem vitam , \\\hline
1.1.4 & por el primero libro delas ethicas ¶ \textbf{ Conuien de saber ujda delectosa et plazentera ¶ vida politica e çiuil¶ } Et uida contenplatina e acabada¶ & quomodo in eis felicitas est ponenda . Distinxerunt autem Philosophi ( ut patet ex 1 Ethic’ ) triplicem vitam , \textbf{ videlicet , voluptuosam , | politicam , } et contemplatiuam . \\\hline
1.1.4 & et es sobre todas las bestias \textbf{ ¶pues que asy es en tres maneras podemos pensar } e fablar del omne ¶ & quibus est inferior . \textbf{ Tripliciter igitur poterit considerari homo : } Primo ut communicat cum brutis : \\\hline
1.1.4 & ¶pues que asy es en tres maneras podemos pensar \textbf{ e fablar del omne ¶ } primeramente & quibus est inferior . \textbf{ Tripliciter igitur poterit considerari homo : } Primo ut communicat cum brutis : \\\hline
1.1.4 & que es Razon derecha de todas las cosas \textbf{ que ha de obrar e de fazer ¶ } Mas es dicho bien auenturado en vida contenplatina & habendo in se prudentiam , \textbf{ quae est recta ratio agibilium : } dicatur felix contemplatiue , \\\hline
1.1.4 & pues que asy es llaman al omne acabado en las obras \textbf{ que ha de fazer bien auenturado çiudadanamente } Mas al omne acabado enlas sçanses & et aliquid melius homine . Perfectum igitur in agibilibus , \textbf{ vocabant felicem politice : } sed perfectum in speculabilibus vocabant felicem contemplatiue , et appellabant ipsum , non hominem , \\\hline
1.1.4 & mas mejor que omne \textbf{ Ca fazer e partiçipar enlas obras } con los otros omes conviene al omne & sed homine meliorem . \textbf{ Nam agere et communicare in actionibus cum aliis , competit homini ut homo est : } sed speculari et cognoscere veritatem , \\\hline
1.1.4 & en quanto es omne . \textbf{ Mas estudiar e conosçer la uerdat conviene al omne en quanto es en el entedimjento especłatino e escodrinador } que es alg̃cos e diujnal . & Nam agere et communicare in actionibus cum aliis , competit homini ut homo est : \textbf{ sed speculari et cognoscere veritatem , | competit ei ut est in eo intellectus speculatiuus , } qui est aliquid diuinum , \\\hline
1.1.4 & pues que asy es tanto es el d partimjento entre el sabio en las obras \textbf{ que ha de fazer } Et el acabado ente las sçiençias especulatians quanta es entre el que biue vida humanal e vida politica & et viuit vita contemplatiua . \textbf{ Tanta est ergo differentia inter prudentem in agibilibus , } et perfectum in speculabilibus , \\\hline
1.1.4 & Mas las que son dados alas sçiençias especulatinas \textbf{ que son en entender } e non en obrar & et angelica . Dediti enim operabilibus , propter diuersitatem negociorum emergentium , turbantur erga plurima , et ut plurimum isti sentiunt passiones carnis : \textbf{ dediti vero speculabilibus , } quodammodo ab his passionibus sunt abstracti . \\\hline
1.1.4 & que son en entender \textbf{ e non en obrar } estos en alguna manera son apartados destas pasiones de lancarne ¶ & ø \\\hline
1.1.4 & que son los omes mucho entendidos \textbf{ con mucho de honrrar . } Ca son assentandos sobre omne & et viri speculatiui , \textbf{ sunt maxime honorandi , } quia sunt supra hominem collocati . \\\hline
1.1.4 & que son falladas en ellas . \textbf{ Enpero non pudieron alcançar la uerdat conplidamente } e segunt manera acabada . & et de felicitatibus repertis in ipsis , Philosophi distinxerunt : \textbf{ non tamen ad plenum , et per omnem modum potuerunt attingere veritatem . Nam licet vere dixerunt } quod in vita voluptuosa non est quaerenda felicitas , \\\hline
1.1.4 & Ca mager que dissiese nudat \textbf{ que en la vida seliçonsa non es de poner bien andança } asy como adelante lo mostraremos mas claramente ¶ & non tamen ad plenum , et per omnem modum potuerunt attingere veritatem . Nam licet vere dixerunt \textbf{ quod in vita voluptuosa non est quaerenda felicitas , } ut infra clarius ostendetur : \\\hline
1.1.4 & la qual los theologos llaman vida actiua \textbf{ que quiere dezir vida de bien obrar } Et dela vida contenplatian e intellectual non sintieron la uerdat conplidamente & de vita tamen politica , \textbf{ quam Theologi vocant vitam actiuam , } et de vita contemplatiua non usquequaque vera senserunt : \\\hline
1.1.4 & ca creyeron que qual quier ome naturalmente syn otra ayuda . \textbf{ njnguna de gera pudie esquiuar todo pecado } e beuir acabadamente & quod ex puris naturalibus absque alio auxilio gratiae posset \textbf{ quis omnia peccata euitare , } et perfecte viuere \\\hline
1.1.4 & e beuir acabadamente \textbf{ segunt vida actiua e de obrar } Et segunt vida contenplatian & et perfecte viuere \textbf{ secundum vitam actiuam , } vel contemplatiuam . \\\hline
1.1.4 & Et avn pusieron los philosofos \textbf{ que la vida contenplatiua estapuramente en el entender } la qual cosa es falsa & et Rectoribus aliorum . Posuerunt \textbf{ etiam vitam contemplatiuam esse in pura speculatione . } Quod est falsum . \\\hline
1.1.4 & que conujene ala rreal magestad \textbf{ e a todo rrey saber } e conosçer estas maneras de beuir & ne sit homine peior : \textbf{ nam tales } ( ut dicitur 1 Ethicorum ) sunt vitam pecudum eligentes . Vitam autem actiuam \\\hline
1.1.4 & e a todo rrey saber \textbf{ e conosçer estas maneras de beuir } Et conviene le de foyr & nam tales \textbf{ ( ut dicitur 1 Ethicorum ) sunt vitam pecudum eligentes . Vitam autem actiuam } et contemplatiuam in se habere debet , \\\hline
1.1.4 & e conosçer estas maneras de beuir \textbf{ Et conviene le de foyr } e de arredrarse dela vida delectosa e carnal & ( ut dicitur 1 Ethicorum ) sunt vitam pecudum eligentes . Vitam autem actiuam \textbf{ et contemplatiuam in se habere debet , } ut per vitam actiuam vacet aliis , \\\hline
1.1.4 & Et conviene le de foyr \textbf{ e de arredrarse dela vida delectosa e carnal } por qua non sea peor que omne . & et contemplatiuam in se habere debet , \textbf{ ut per vitam actiuam vacet aliis , } magnifica faciendo , \\\hline
1.1.4 & los tales que biuen vida deliconsa e carnal son tales commo los que escogen uido de bestias . Mas cada vn rrey e cada vn prinçipe es dicho avn en sy vida actiua \textbf{ e de obrar ¶ } Et contenplatian e de entender . & recte regendo . Per vitam contemplatiuam vacet \textbf{ sibi per internam deuotionem et Dei dilectionem , } in Dei amore proficiendo , \\\hline
1.1.4 & e de obrar ¶ \textbf{ Et contenplatian e de entender . } por que por la ujda actiua entienden las obras faziendo cosas granadas e honrra das & recte regendo . Per vitam contemplatiuam vacet \textbf{ sibi per internam deuotionem et Dei dilectionem , } in Dei amore proficiendo , \\\hline
1.1.4 & et gouernando derechamente los sus . subditos ¶ \textbf{ Et por la vida contenplatiua entienda aprouechar asy mesmo con deuoçion } de dentro del ala aprouechado & ut Philosophi sentiebant . \textbf{ Unde si in speculatione diuinorum vita contemplatiua consistit , } hoc est , \\\hline
1.1.4 & Mas esta deuoçion de dentro del alma \textbf{ tantomas conviene dela auer los rreys e los prinçipes } por quanto han de dar mayor cuenta ant̃la siella del primer alcałł & et diuinus amor . Hanc autem internam deuotionem \textbf{ tanto magis decet habere reges et principes , } quanto apud tribunal summi Iudicis reddituri sunt de pluribus rationem . \\\hline
1.1.4 & tantomas conviene dela auer los rreys e los prinçipes \textbf{ por quanto han de dar mayor cuenta ant̃la siella del primer alcałł } que es dios Et quanto de mayores conpannas han de dar Razon e cuenta & tanto magis decet habere reges et principes , \textbf{ quanto apud tribunal summi Iudicis reddituri sunt de pluribus rationem . } Quod maxime expedit regiae maiestati \\\hline
1.1.4 & por quanto han de dar mayor cuenta ant̃la siella del primer alcałł \textbf{ que es dios Et quanto de mayores conpannas han de dar Razon e cuenta } as conuiene de notar e de saber acuçiosamente & quanto apud tribunal summi Iudicis reddituri sunt de pluribus rationem . \textbf{ Quod maxime expedit regiae maiestati } Est autem diligenter notandum , \\\hline
1.1.5 & que es dios Et quanto de mayores conpannas han de dar Razon e cuenta \textbf{ as conuiene de notar e de saber acuçiosamente } que asy commo la materia & Quod maxime expedit regiae maiestati \textbf{ Est autem diligenter notandum , } quod sicut materia per debitas transmutationes \\\hline
1.1.5 & que asy commo la materia \textbf{ por sus conueientes e ordenadas t̃ns muta connes viene a rresçebir su forma } e su perfecçion & quod sicut materia per debitas transmutationes \textbf{ consequitur } suam perfectionem \\\hline
1.1.5 & asi el omne por derechas \textbf{ e conueni entes obras viene a auer su perfecçion } e su bien andança acabada¶ & et formam , \textbf{ sic homo per rectas et debitas operationes consequitur suam perfectionem et felicitatem . } Cum ergo nunquam contingat recte agere , \\\hline
1.1.5 & pues que asy es com̃ nunca pueda omne bien \textbf{ e derechamente obrar } asy commo demanda la fin & sic homo per rectas et debitas operationes consequitur suam perfectionem et felicitatem . \textbf{ Cum ergo nunquam contingat recte agere , } ut requirit consecutio finis , \\\hline
1.1.5 & asy commo demanda la fin \textbf{ que ha de seguir . } sy la su fin non sopiere . Conviene a todo omne & Cum ergo nunquam contingat recte agere , \textbf{ ut requirit consecutio finis , } ignorato ipso fine , \\\hline
1.1.5 & sy la su fin non sopiere . Conviene a todo omne \textbf{ que quiera alcançar } e auer su fin & ignorato ipso fine , \textbf{ expedit volenti consequi suum finem , } vel suam felicitatem , \\\hline
1.1.5 & que quiera alcançar \textbf{ e auer su fin } e la su bien andança & ignorato ipso fine , \textbf{ expedit volenti consequi suum finem , } vel suam felicitatem , \\\hline
1.1.5 & e la su bien andança \textbf{ de auer ante algun conosçimjento dela su fin } e dela su bien andança . & vel suam felicitatem , \textbf{ habere praecognitionem ipsius finis . } Possumus autem dicere quod ( ut ad praesens spectat ) \\\hline
1.1.5 & e dela su bien andança . \textbf{ Mas podemosdezer } quanto pertenesçe alo presente & habere praecognitionem ipsius finis . \textbf{ Possumus autem dicere quod ( ut ad praesens spectat ) } duplici via venari possumus , \\\hline
1.1.5 & que en dos maneras \textbf{ e por todas carreras pondemos auer e cobrar prouar } que conujene al rrey en toda manera de conosçer la su fin ¶ & Possumus autem dicere quod ( ut ad praesens spectat ) \textbf{ duplici via venari possumus , } quod expedit regi suum finem cognoscere . Prima est , \\\hline
1.1.5 & e por todas carreras pondemos auer e cobrar prouar \textbf{ que conujene al rrey en toda manera de conosçer la su fin ¶ } La primera rrazon es en quanto el rrey & duplici via venari possumus , \textbf{ quod expedit regi suum finem cognoscere . Prima est , } inquantum per sua opera cooperatur , \\\hline
1.1.5 & Ca sy non feziese bien \textbf{ mas mal non podria alcançar buena fin } mas avria el contrario & Si enim non ageret bene \textbf{ sed male , | non consequeretur finem , } sed contrarium finis : \\\hline
1.1.5 & Mas non solamente los que mal fazen non alcançan buena fin \textbf{ mas avn los que pueden bien fazer } sy non fazen bien & sed contrarium finis . Immo non solum male agentes non consequuntur finem , \textbf{ sed potentes bene agere , } nisi bene agant , \\\hline
1.1.5 & que en olimpiedes \textbf{ que quiere dezer en aquellas faziendas } o es aquellas batallas non son coronados los muy fuertes & quod in Olimpidiadibus , \textbf{ idest in illis bellis et agonibus non coronantur fortissimi , } sed agonizantes : \\\hline
1.1.5 & mas los bien lidiantes \textbf{ ca los que son muy fuertes pueden lidiar . } Enpero si non lidiaren de fech̃o & sed agonizantes : \textbf{ qui enim fortissimi sunt , potentes agonixare , attamen si non actu agonizant , } non debetur eis corona . Oportet igitur \\\hline
1.1.5 & non les es deuida corona ¶ \textbf{ pues que asi es conviene bien fazer de fecho } por que por las nr̃as obras merescamos de auer buena fino buena ventura & non debetur eis corona . Oportet igitur \textbf{ actu bene agere , } ut per opera nostra mereamur consequi finem , \\\hline
1.1.5 & pues que asi es conviene bien fazer de fecho \textbf{ por que por las nr̃as obras merescamos de auer buena fino buena ventura } segunt deuemos los omes fas̉ & actu bene agere , \textbf{ ut per opera nostra mereamur consequi finem , | vel felicitatem . } Secundo requiritur \\\hline
1.1.5 & Ca aqual que faze bien acaso e auentura \textbf{ por esto non es de alabar } njn por esto non le es deujda buean fin & Nam qui casu vel fortuitu bene agit , \textbf{ ex hoc non est laudandus , } nec debetur ei ex hoc finis vel felicitas : \\\hline
1.1.5 & en el terçero libro de las ethicas \textbf{ que njguon non es bien auer turado } si non obra de voluntad & Ex inuoluntariis autem ( ut patet per Philosophum 3 Ethic’ ) non laudamur nec vituperamur : \textbf{ unde in eodem 3 dicitur , quod nullus est beatus nisi volens . Sed quae non agimus ex electione , non agimus volentes : } ergo per se loquendo \\\hline
1.1.5 & ca la uirtud co sabidiria \textbf{ que muestra a omne escoger . } Et esta sabiduria esta en medio delas buenas obras & Immo cum ex operibus virtuosis felicitatem consequamur \textbf{ ( quia virtus est habitus electiuus in medietate consistens , } ut dicitur 2 Ethic . ) oportet operationes , \\\hline
1.1.5 & Onde dize el philosofo en el segundo libro delas ethicas \textbf{ que non cunple solamente fazer buenas obras } mas fazerlas bien njn cunple de obrar obras iustas & Unde Philosophus 2 Ethic’ vult , \textbf{ quod non sufficit agere bona , sed bene : nec sufficit operari iusta , } sed iuste . Contigit enim aliquos prauos facere aliqua de genere bonorum , \\\hline
1.1.5 & que non cunple solamente fazer buenas obras \textbf{ mas fazerlas bien njn cunple de obrar obras iustas } mas fazer las iustamente e con iustiçia . & Unde Philosophus 2 Ethic’ vult , \textbf{ quod non sufficit agere bona , sed bene : nec sufficit operari iusta , } sed iuste . Contigit enim aliquos prauos facere aliqua de genere bonorum , \\\hline
1.1.5 & mas fazerlas bien njn cunple de obrar obras iustas \textbf{ mas fazer las iustamente e con iustiçia . } por que contesçe & ø \\\hline
1.1.5 & que nos que conoscamos primero alanr̃a fin \textbf{ por que la podamos alcançar } Ca asy lo deuemos ymaginar & Expedit ergo \textbf{ ( ut finem consequamur ) finem praecognoscere . Sic enim imaginari debemus , } quod sicut est in causis efficientibus , \\\hline
1.1.5 & por que la podamos alcançar \textbf{ Ca asy lo deuemos ymaginar } que asy commo es en los mouedores & Expedit ergo \textbf{ ( ut finem consequamur ) finem praecognoscere . Sic enim imaginari debemus , } quod sicut est in causis efficientibus , \\\hline
1.1.5 & que mouiese lanr̃a uoluntad \textbf{ ninguon otro bien non la podria mouer } por que qual quier cosa & ut bonum ultimatum quod voluntatem moueret , \textbf{ nullum aliud bonum voluntatem mouere posset ; } quicquid enim vult voluntas , \\\hline
1.1.5 & pues que asy es non conosçiendo alguna cosa \textbf{ so Razon de fin nos non podemos bien obrar . } Ca esto puesto queda toda obra humanal ¶ & Non apprehenso ergo aliquo sub ratione finis , \textbf{ non contingit nos bene agere : } quia hoc posito cessat \\\hline
1.1.5 & pues que asi es \textbf{ para que nos bien obremos conuiene nos de estableçer alguna buean fin e conuenible } que todas las nuestras obras toman nasçençia dela fin & sed \textbf{ ut bene agamus , oportet nobis praestituere finem bonum et debitum : } quia ex fine opera nostra speciem summunt : \\\hline
1.1.5 & donde se sigue \textbf{ que antes auemos de conosçer la fin } por que segunt ella podamos obrar bien & ut bene agamus . \textbf{ Secundo talis praecognitio requiritur , } ut ex electione agamus . \\\hline
1.1.5 & que antes auemos de conosçer la fin \textbf{ por que segunt ella podamos obrar bien } Ca asy comon los que lançan la saeta & Secundo talis praecognitio requiritur , \textbf{ ut ex electione agamus . } Nam sicut sagittantes non videntes signum , \\\hline
1.1.5 & mas es por auentra a¶ \textbf{ Onde el philosofo quariendo mostrar en el primero libro delas ethicas } que es neçesario de connosçer & sed à fortuna . \textbf{ Unde Philosophus 1 Ethicor’ volens ostendere necessariam esse praecognitionem finis , ait , } quod cognitio finis \\\hline
1.1.5 & Onde el philosofo quariendo mostrar en el primero libro delas ethicas \textbf{ que es neçesario de connosçer } ante la fin dize & sed à fortuna . \textbf{ Unde Philosophus 1 Ethicor’ volens ostendere necessariam esse praecognitionem finis , ait , } quod cognitio finis \\\hline
1.1.5 & ante la fin dize \textbf{ que para lanr̃auida grant acresçentamiento faze connosçer ante la fin } Ca por esta Razon alcançaremos ante la fin & quod cognitio finis \textbf{ ad vitam nostram magnum habet incrementum : } consequemur enim ex hoc magis ipsum finem ; \\\hline
1.1.5 & mas a cierto tiran aella asy los que conosçen ante la fin \textbf{ mas a çierto se ordenan a bien obrar } ¶ & quemadmodum sagittatores signum habentes , \textbf{ magis utique adipiscentur id quod oportet . } Tertio praecognitio finis non solum facit nos agere bene , \\\hline
1.1.5 & Lo terçero conesçer ante la fin \textbf{ non solamente nos faze bien obrar } e obrar de uoluntad & magis utique adipiscentur id quod oportet . \textbf{ Tertio praecognitio finis non solum facit nos agere bene , } et ex electione , \\\hline
1.1.5 & non solamente nos faze bien obrar \textbf{ e obrar de uoluntad } mas ahun faze nos obrar delectosamente & Tertio praecognitio finis non solum facit nos agere bene , \textbf{ et ex electione , } sed etiam delectabiliter . \\\hline
1.1.5 & e obrar de uoluntad \textbf{ mas ahun faze nos obrar delectosamente } Ca pensada la bien andança & et ex electione , \textbf{ sed etiam delectabiliter . | Nam grauia efficiuntur delectabilia et dulcia , } considerata beatitudine , \\\hline
1.1.5 & e la buena uentraa \textbf{ que omne ha por las buean s obras las obras guaues e fuertes de fazer se fazen muy delectables e plazenteras ¶ } pues que asi e sacanda hun omne conuiene ante connosçerlo su fin & et felicitate , \textbf{ quam ex ipsis consequimur . Cuilibet ergo homini , } ut agat bene , ex electione , et delectabiliter , \\\hline
1.1.5 & que omne ha por las buean s obras las obras guaues e fuertes de fazer se fazen muy delectables e plazenteras ¶ \textbf{ pues que asi e sacanda hun omne conuiene ante connosçerlo su fin } e la su bien andança . & et felicitate , \textbf{ quam ex ipsis consequimur . Cuilibet ergo homini , } ut agat bene , ex electione , et delectabiliter , \\\hline
1.1.5 & e la su bien andança . \textbf{ por que pueda obrar bien } e de uoluntad e delectosamente . & quam ex ipsis consequimur . Cuilibet ergo homini , \textbf{ ut agat bene , ex electione , et delectabiliter , } expedit suam felicitatem praecognoscere : \\\hline
1.1.5 & mas conuiene al Rey o al prinçipe . \textbf{ por que entondas sus obras deue entender al bien dela gente } e al bien comun & sed maxime hoc expedit regiae maiestati , \textbf{ quia in operibus suis debet intendere bonum gentis } et commune , \\\hline
1.1.5 & que muy mas conuiene al Reio al prinçipe \textbf{ conosçer la su fin } e la su bien andança & quod maxime decet regiam maiestatem \textbf{ cognoscere suam felicitatem , } ut opera communia , \\\hline
1.1.5 & que a otro ninguno . \textbf{ por que pueda fazer buenas obras e comunes } que son en alguna manera obras diuina les . & cognoscere suam felicitatem , \textbf{ ut opera communia , } quae sunt quodammodo diuina , \\\hline
1.1.5 & donde se sigue \textbf{ que mas conuiene al saetero de conosçer lasseñal } que non ala saeta & quia a sagittante in signum dirigitur , \textbf{ sicut igitur magis expedit sagittatorem signum percipere quam sagitta , } eo quod sit sagittae director : sic magis expedit regiam maiestatem felicitatem , \\\hline
1.1.5 & Et asi se sigue \textbf{ que mas conuiene al Rei o al prinçipe de conosçer } la su bien andança & ø \\\hline
1.1.6 & que son mas sensibles \textbf{ asi commo son las delecta connes del tanner e del gostar } llaman tan solamente plazenterias e delecta connes . & solas delectationes maxime sensibiles , \textbf{ cuiusmodi sunt delectationes , | tactus , et gustus , } absolute appellamus voluptates , \\\hline
1.1.6 & que las delectaçonnes sensibles de los sesos ¶ \textbf{ Mas por tres Razones podemos nos prouar } que en estas delecta çonnes sensibles de los sesos . & quam voluptates sensibiles . In huiusmodi autem voluptatibus sensibilibus non esse felicitatem ponendam , \textbf{ triplici via venari possumus . } Quantum enim ad praesens spectat , felicitas tria importare videtur . Felicitas enim dicit perfectum , \\\hline
1.1.6 & que en estas delecta çonnes sensibles de los sesos . \textbf{ non es de poner la feliçidat e la bien andança . Ca quanto parte nesçe alo prèsente la bien andança ençierra en si tres cosas ¶ } La primera es & triplici via venari possumus . \textbf{ Quantum enim ad praesens spectat , felicitas tria importare videtur . Felicitas enim dicit perfectum , } et per se sufficiens bonum . Nam tunc dicimus aliquem esse felicem , \\\hline
1.1.6 & ¶onde dize el philosofo en el primero libro delas ethicas \textbf{ quariendo mostrar } que cosa es la fe liçidat & Unde Philosophus 1 Ethicorum describens felicitatem , \textbf{ ait , Felicitas est finis operatorum , } et perfectum , \\\hline
1.1.6 & e es bien acabado e bien sufiçiente \textbf{ por si para fazer al omne acabado ¶ } Et desto primero se sigue lo segundo & et perfectum , \textbf{ et per se sufficiens bonum . } Ex hoc autem primo sequitur secundum , \\\hline
1.1.6 & e segunt Razon O conuiene que sea tal bien qual iudga la razon derecha \textbf{ que nos auemos de segnir ¶ } Et de estas dos cosas se sigue la terçera & oportet \textbf{ quod sit tale bonum , | quale recta ratio prosequendi iudicet . } Ex his autem sequitur tertium , \\\hline
1.1.6 & nin segunt Razon assaz paresçe \textbf{ que en las tales delectaçiones non auemos nos de poner lanr̃a feliçidat nin lanr̃a bien andança } Mas que estas plazenterias & et secundum rationem , \textbf{ constat in talibus non esse felicitatem ponendam . Quod autem huiusmodi voluptates , } non sint bonum perfectum , et sufficiens , de leui patet . \\\hline
1.1.6 & nin conplido ligeramente paresçe . Ca lo que es bien conplido farta el apetito del omes \textbf{ Mas estas delectaçonnes non pueden fartar el apetito } asi commo cada vno prueba en si mesmo . & Nam quod sic bonum est , satiat appetitum : \textbf{ huiusmodi autem voluptates appetitum satiare non possunt , } ut in seipso quilibet experiri potest . \\\hline
1.1.6 & asi commo cada vno prueba en si mesmo . \textbf{ ¶ Onde el philosofo en el terçero libro delas ethicas fablando de tales delectaçonnes dize que el apetito delectable de los sesos non se puede fartar delas delectaçones ¶ } La segunda Razon es esta & ut in seipso quilibet experiri potest . \textbf{ unde Philosophus 3 Ethi’ loquens de talibus delectationibus ait , | quod insatiabilis est delectabilis appetitus . } Secundo in talibus non est ponenda felicitas , \\\hline
1.1.6 & La segunda Razon es esta \textbf{ que en estas delecta con nes non es de poner feliçidat nin bien andança } Ca non son bienes segunt razon . & quia non sunt bona \textbf{ secundum rationem , | sed magis sunt bona } secundum sensum , \\\hline
1.1.6 & e afincadas çiegan la razon e el entendimiento \textbf{ e nol dexan entender lo que . } cunple ¶ & Si tales delectationes magnae , \textbf{ et vehementes sint , } cognitionem idest rationem percutiunt . Tertio non est ponenda felicitas in voluptatibus sensibilibus , \\\hline
1.1.6 & La terçera razon \textbf{ por que non es de poner la feliçidat o la bien andança } en las delecta connes & et vehementes sint , \textbf{ cognitionem idest rationem percutiunt . Tertio non est ponenda felicitas in voluptatibus sensibilibus , } quia talia magis sunt bona corporis , \\\hline
1.1.6 & al qual bien son ordenados todos los otros biens del omne . \textbf{ Este bien final non es de poner en las plazenterias sensibles } que son bienes del cuerpo & ad quod bonum alia ordinantur , \textbf{ huiusmodi bonum non est ponendum in voluptatibus sensibilibus , } quae sunt bona corporis , \\\hline
1.1.6 & que son bienes del cuerpo \textbf{ mas es de poner en las obras de uertudes } que son bienes del alma . & quae sunt bona corporis , \textbf{ sed magis in operibus virtutum , } quae sunt bona animae . Possumus enim dicere , \\\hline
1.1.6 & que son bienes del alma . \textbf{ por la qual razon podemos dezer } que maguera que alguas delecta connes sean conuenibles e honestas & sed magis in operibus virtutum , \textbf{ quae sunt bona animae . Possumus enim dicere , } quod licet sint delectationes aliquae licitae , et honestae , \\\hline
1.1.6 & por si mesma maguera \textbf{ que se pueda conseguir ala feliçidat e ala bien andança . } Mas declarar esto non parte nesçe a esta arte presente . & nulla tamen delectatio est essentialiter ipsa felicitas , licet possit esse aliquid felicitatem consequens , \textbf{ sed hoc declarare non est praesentis negocii . } Forte tamen de hoc aliquid infra dicetur . \\\hline
1.1.6 & que se pueda conseguir ala feliçidat e ala bien andança . \textbf{ Mas declarar esto non parte nesçe a esta arte presente . } Enpero que por auentra a adelante diremos alguna cosa desto . & nulla tamen delectatio est essentialiter ipsa felicitas , licet possit esse aliquid felicitatem consequens , \textbf{ sed hoc declarare non est praesentis negocii . } Forte tamen de hoc aliquid infra dicetur . \\\hline
1.1.6 & Enpero que por auentra a adelante diremos alguna cosa desto . \textbf{ Mas quanto a esto presente cunple de saber } qua non conuiene a ningun omne & Forte tamen de hoc aliquid infra dicetur . \textbf{ Ad praesens autem scire sufficiat , } quod non decet aliquem hominem suam felicitatem ponere in delectationibus sensibilibus . \\\hline
1.1.6 & qua non conuiene a ningun omne \textbf{ de poner la feliçidat suia } e la su bien andança enlas delectaçiones sensibles & Ad praesens autem scire sufficiat , \textbf{ quod non decet aliquem hominem suam felicitatem ponere in delectationibus sensibilibus . } Est ergo detestabile cuilibet Homini ponere suam felicitatem in voluptatibus . \\\hline
1.1.6 & que toda su bien andança pone en delecta connes dela carne \textbf{ mas mucho mas es de denostar el Rey } qua ninguno otrosi en estas delecta con nes pone sir bien andança . & Est ergo detestabile cuilibet Homini ponere suam felicitatem in voluptatibus . \textbf{ Sed maxime hoc est detestabile Regiae maiestati : } quod \\\hline
1.1.6 & mas mucho mas es de denostar el Rey \textbf{ qua ninguno otrosi en estas delecta con nes pone sir bien andança . } La qual cosa podemos ahun prouar & Sed maxime hoc est detestabile Regiae maiestati : \textbf{ quod } etiam triplici via venari possumus . \\\hline
1.1.6 & qua ninguno otrosi en estas delecta con nes pone sir bien andança . \textbf{ La qual cosa podemos ahun prouar } por tres razones & quod \textbf{ etiam triplici via venari possumus . } Primo enim talia sunt detestabilia Regi , \\\hline
1.1.6 & Ca asi commo dize el philosofo en el quinto libro delas politicas el prinçipado \textbf{ e el señorio deue responder ala grandeza e ala dignidat de la persona } asi que cada vno & ( ut dicitur 5 Politicorum principatus debet respondere magnitudini , \textbf{ et dignitati , } ut quanto quis maior Princeps existit , \\\hline
1.1.6 & asi que cada vno \textbf{ quanto es mayor prinçipe tanto mas deue sobrepuiar los otros } en dignidat deuida e en grandeza de bondat ¶ & ut quanto quis maior Princeps existit , \textbf{ tanto alios magis excellere debet in dignitate vitae , } et magnitudine bonitatis . \\\hline
1.1.6 & pues que asi es el que esta en tan alto grado \textbf{ non deue escoger uida de bestia . } Ca por tal uida seria muy abaxado e desonuestra do¶ & In tanto ergo gradu existens , \textbf{ indignum est , } ut vitam pecudum eligat , \\\hline
1.1.6 & La segunda Razon \textbf{ por que el Rei non deuͤ poner su bien andança } en las delecta connes carnales & indignum est , \textbf{ ut vitam pecudum eligat , } quia per eam valde deprimitur . Secundo hoc est detestabile Regi , quia seipsum contemptibilem reddit . \\\hline
1.1.6 & que mucho conuiene alos prinçipes ser mesurados e tenprados en las delecta con nes corpora les . \textbf{ Et da entender y el philosofo } que los que se dan alas delecta connes corporales son semeiantes & esse moderatos in delectationibus corporibus . \textbf{ Innuit enim ipse ibidem quod dantes se delectationibus talibus , } sunt similes dormientibus , et ebriis : \\\hline
1.1.6 & e los enbriagos \textbf{ que non pueden usar de razon nin de entendimiento¶ } pues que asi es commo seg̃t ese mismo philosofo & sicut dormientes , \textbf{ et ebrii uti ratione non possunt . } Cum ergo \\\hline
1.1.6 & Mas quando es dormidor \textbf{ e enbriago es de menospreçiar } por ende ¶Conuienea la real magestad de searedrar de tales delectaçonnes desmesuradas e carnales & sed qui ebrius , \textbf{ nec qui uigil , } sed qui dormiens , decet Regiam maiestatem tales delectationes immoderatas fugere , \\\hline
1.1.6 & e enbriago es de menospreçiar \textbf{ por ende ¶Conuienea la real magestad de searedrar de tales delectaçonnes desmesuradas e carnales } por que non sea menospreçiado de su pueblo¶ & nec qui uigil , \textbf{ sed qui dormiens , decet Regiam maiestatem tales delectationes immoderatas fugere , } ne contemptibilis uideatur . Tertio decet Principem talia detestari , ne principari efficiatur indignus , \\\hline
1.1.6 & La terçera razon \textbf{ por que el prinçipe non deue poner su bien andança en las delecta çonnes corporales es esta } e conuiene al prinçipede dexar e posponer las tales delectaçiones & sed qui dormiens , decet Regiam maiestatem tales delectationes immoderatas fugere , \textbf{ ne contemptibilis uideatur . Tertio decet Principem talia detestari , ne principari efficiatur indignus , } nam nullus eligit Iuuenes in Duces , \\\hline
1.1.6 & por que el prinçipe non deue poner su bien andança en las delecta çonnes corporales es esta \textbf{ e conuiene al prinçipede dexar e posponer las tales delectaçiones } ca si las non dexase non seria digno de ser prinçipe . & ne contemptibilis uideatur . Tertio decet Principem talia detestari , ne principari efficiatur indignus , \textbf{ nam nullus eligit Iuuenes in Duces , } eo quod non constet eos esse Prudentes , \\\hline
1.1.6 & Ca dize el philosofo en el libro delas topicas \textbf{ que njnguon non deue escoger alos mançebos por prinçipes } ca non es digno el moço de ser prinçipe & nam nullus eligit Iuuenes in Duces , \textbf{ eo quod non constet eos esse Prudentes , } indignum est enim Puerum principari . \\\hline
1.1.6 & ca non es digno el moço de ser prinçipe \textbf{ nin deue gouernar puebło . } Mas asi commo el dize & eo quod non constet eos esse Prudentes , \textbf{ indignum est enim Puerum principari . } Sed ( ut dicitur 1 Ethic’ ) \\\hline
1.1.6 & non es digno de ser prinçipe ¶ \textbf{ pues que asi es muy acuçiosamente deuemos notar } que asy commo non ay departimento entre moço en hedat & quia est Puer moribus , \textbf{ indignus est principari . Est ergo diligenter notandum , } quod sicut non differt esse Puerum aetate , et moribus : sic non refert esse Senem moribus et aetate , propter quod sicut si sit Senex tempore , \\\hline
1.1.6 & e han sabiduria e entendimiento \textbf{ para gouernar } dignos son de ser prinçipes e gouernadores¶ & quare si constat eos habere mores seniles , et vigere Prudentia , \textbf{ digni sunt principari . } Sit ergo conclusio capituli , quod , \\\hline
1.1.6 & e por quanon sea uidgado \textbf{ que non es digno de ser prinçipe deue menospreçiar las delecta connes desmesuradas e carnales } philosofo en el primero libro delas politicas & et ne \textbf{ indignus sit principari , debet immoderatas voluptates despicere . | Quod non decet regiam maiestatem , } Philosophus 1 Politicor’ distinguit duo genera diuitiarum \\\hline
1.1.7 & para el beuir \textbf{ e para el uestir } son contadas entre las riquezas natraales ¶ & et uniuersaliter quaecunque immediate ad victum , \textbf{ et vestitum deseruiunt , } inter naturales diuitias computantur . Artificiales autem diuitiae sunt , \\\hline
1.1.7 & enpero son riquezas artifiçiales \textbf{ Mas en njngunas destas riquezas non es de poner la bien andança } Ca el philosofo tanne tres razones & ø \\\hline
1.1.7 & enl primero libro delas politicas \textbf{ por las quales nos pondemos prouar } que la feliçidat e la bien andança non es de poner en les riquezas artifiçiales ¶ & In neutris autem diuitiis est ponenda felicitas . Tangit enim 1 Politicor’ tria , \textbf{ propter quae venari possumus , } in artificialibus diuitiis felicitatem non esse ponendam . Primo , \\\hline
1.1.7 & por las quales nos pondemos prouar \textbf{ que la feliçidat e la bien andança non es de poner en les riquezas artifiçiales ¶ } La primera razon es por que las riquezas artifiçiales son orderandas alas riquezas natraales & propter quae venari possumus , \textbf{ in artificialibus diuitiis felicitatem non esse ponendam . Primo , } quia artificiales diuitiae ad naturales ordinantur . Secundo , \\\hline
1.1.7 & Ca dichones de suso \textbf{ que la feliçidat e la bien andança es de poner en aquel bien } aque todos los otros bienes son ordenados & secundum seipsas indigentiae non satisfaciunt . Dicebatur enim supra , felicitatem \textbf{ esse illud bonum , } ad quod alia bona ordinantur , \\\hline
1.1.7 & por la qual razon las riquezas artisiçiales \textbf{ por razon que son ordenadas alas riquezas natra a les non pueden auer razon nin manera e bien andanças la segunda razon } por que non auemos de estableçer & propter quod diuitiae artificiales \textbf{ eo ipso quod ordinantur ad diuitias naturales , } rationem felicitatis habere non possunt . \\\hline
1.1.7 & por razon que son ordenadas alas riquezas natra a les non pueden auer razon nin manera e bien andanças la segunda razon \textbf{ por que non auemos de estableçer } nin de poner lanr̃a feliçidat ni lanr̃abine andança en las riquezas & propter quod diuitiae artificiales \textbf{ eo ipso quod ordinantur ad diuitias naturales , } rationem felicitatis habere non possunt . \\\hline
1.1.7 & por que non auemos de estableçer \textbf{ nin de poner lanr̃a feliçidat ni lanr̃abine andança en las riquezas } artifiçiales & eo ipso quod ordinantur ad diuitias naturales , \textbf{ rationem felicitatis habere non possunt . | Secundo in talibus non est ponenda felicitas , } quia non habent \\\hline
1.1.7 & aquellos que usan dellas ¶ La terçera Razon \textbf{ por que en las riquezas artifiçiales non es de poner la feliçidat e la bien andança es esta } Ca el oro e la plata & et dispositio utentium eis . \textbf{ Tertio in talibus non est ponenda felicitas , } quia aurum , \\\hline
1.1.7 & nin les cunplen . \textbf{ Ca puede contesçer asy commo dizeel philosofo } en el primero libro delas politicas & ø \\\hline
1.1.7 & que alguno podia ser rico de mucho oro o de muchos dineros \textbf{ e morir de fanbre . } Asi commo cuentay de un omne a que dezian meda el qual ero muy codiçioso de auerors & quod quis diues pecunia , fame moriatur . Recitatur enim ibi de quodam , cuius nomen erat Mida , \textbf{ qui cum nimis esset auidus auri } ( ut fabulose dicitur ) \\\hline
1.1.7 & asi que quanto el tannia todo se le tornaua oro \textbf{ Et por que el seso del tanner es en todas las partes del cuerpo } e ahun en la boca non podia tanner ninguna cosa & fieret aurum . \textbf{ Cum ergo tactus reseruetur in singulis partibus corporis , } etiam ore nihil tangere poterat , \\\hline
1.1.7 & Et por que el seso del tanner es en todas las partes del cuerpo \textbf{ e ahun en la boca non podia tanner ninguna cosa } que tonda non se le tornauaoro & Cum ergo tactus reseruetur in singulis partibus corporis , \textbf{ etiam ore nihil tangere poterat , } quin conuertetur in aurum . \\\hline
1.1.7 & Et pues que asi es concludendo todo lo \textbf{ que dichones podemos dezir } que non es de poner la feliçidat & et vere satisfacerent indigentiae corporali . \textbf{ Tum ergo } quia numismata sunt diuitiae in ordine ad aliud , \\\hline
1.1.7 & que dichones podemos dezir \textbf{ que non es de poner la feliçidat } e la bien andança en las riquezas artifiçiales & Tum ergo \textbf{ quia numismata sunt diuitiae in ordine ad aliud , } tum quia sunt diuitiae ex institutione Hominum , tum quia cum sint corporalia , \\\hline
1.1.7 & Lo primero por que los riquezas e las monedas artifiçiales son riquezas ordenandas a otra cosa . \textbf{ conuiene saber a los riquezas natraales } ¶ & tum quia sunt diuitiae ex institutione Hominum , tum quia cum sint corporalia , \textbf{ ipsi indigentiae corporali per se non sufficiunt , } in eis non est ponenda felicitas . \\\hline
1.1.7 & Lo terçero commo ellas sean riquezas corporales \textbf{ non pueden abondar nin conplir } por si alas menguas corporales . & Quod autem in naturalibus diuitiis , \textbf{ cuiusmodi sunt cibus , et potus , } et ea quae per se indigentiae corporali satisfaciunt , \\\hline
1.1.7 & por si alas menguas corporales . \textbf{ Et por ende non es de poner la bien andança enellas . } Otrosy que nos non auemos de poner la nuestra feliçidat & cuiusmodi sunt cibus , et potus , \textbf{ et ea quae per se indigentiae corporali satisfaciunt , } non sit ponenda felicitas \\\hline
1.1.7 & Et por ende non es de poner la bien andança enellas . \textbf{ Otrosy que nos non auemos de poner la nuestra feliçidat } e lanr̃a bien andança en las riquezas natraales & et ea quae per se indigentiae corporali satisfaciunt , \textbf{ non sit ponenda felicitas } de leui patet . \\\hline
1.1.7 & e lanr̃a bien andança en las riquezas natraales \textbf{ las quales son aquellas que parte nesçen al comer e al beuer . } Et ahun aquellas que abondan e cunplen & ø \\\hline
1.1.7 & Et ahun aquellas que abondan e cunplen \textbf{ por si las menguas corporales esto ligeramente lo podemos prouar . } ¶ Por que commo la feliçidat & non sit ponenda felicitas \textbf{ de leui patet . } Nam cum felicitas sit bonum optimum , \\\hline
1.1.7 & en el mayor bien \textbf{ que nos podemos dessear . } siguese que commo el alma sea meior & Nam cum felicitas sit bonum optimum , \textbf{ in optimo nostro quaeri debet . } Cum ergo anima sit potior corpore , felicitas non est ponenda in talibus diuitiis , \\\hline
1.1.7 & que el cuerpo la feliçidat \textbf{ e la bien andança non es de poner en tales riquezas } que son bienes del cuerpo¶ & in optimo nostro quaeri debet . \textbf{ Cum ergo anima sit potior corpore , felicitas non est ponenda in talibus diuitiis , } quae sunt \\\hline
1.1.7 & que son bienes del cuerpo¶ \textbf{ Mas asi conmodeximos de ssuso es de poner en las obras de uirtudes } que son bienes del alma & bona corporis , \textbf{ sed } ( ut superius dicebatur ) ponenda est in actibus virtutum , \\\hline
1.1.7 & que son bienes del alma \textbf{ ¶pres que assi es mucho es de denostar todo } en que pone su feliçidat & sed \textbf{ ( ut superius dicebatur ) ponenda est in actibus virtutum , } quae sunt bona animae . Cuilibet ergo Homini detestabile est ponere suam felicitatem in diuitiis , \\\hline
1.1.7 & e su bien andança en las riquezas corporales . \textbf{ Mas mayor mente es de denostar la Real magestad } e el Rey o el prinçipe & quae sunt bona animae . Cuilibet ergo Homini detestabile est ponere suam felicitatem in diuitiis , \textbf{ sed maxime detestabile est regiae maiestati . } Nam si Rex aut Princeps ponat suam felicitatem in diuitiis , \\\hline
1.1.7 & El segundo es que se faze \textbf{ por ende tirano quiere dezer le unadoro . } apremiador del pueblo & tria maxima mala inde consequuntur . Primo , \textbf{ quia amittit maxima bona . Secundo , } quia efficitur Tyrannus . Tertio , \\\hline
1.1.7 & assi commo su bien andança \textbf{ e teme delas partir o delas esꝑzer } nunca pue de ser magnifico nin granado & ut finem , \textbf{ et timen pecuniam elargiri , } nunquam potest esse magnificus , \\\hline
1.1.7 & nunca pue de ser magnifico nin granado \textbf{ el qual magnifico ha de fazer grandes espensas } para ser granado & nunquam potest esse magnificus , \textbf{ cuius est facere magnos sumptus : } nec etiam potest esse Magnanimus , \\\hline
1.1.7 & nin de grant coraçon . \textbf{ Ca temiendo deꝑder los des } e las riquezas nunca acometra grandes cosas & nec etiam potest esse Magnanimus , \textbf{ quia metuens pecuniam perdere , } nihil magnum attentabit . Immo \\\hline
1.1.7 & en el capitulo dela maganimidat \textbf{ que quiere dezir grandeza de coraçon } Ca en la opinion del auariento & ut vult Philosophus 4 Ethicorum cap’ \textbf{ de Magnanimitate ) } quia in opinione auari , \\\hline
1.1.7 & e la grandeza de coraçon son muy grandesbienes \textbf{ los quales bienes deue auer la Real magestad } mucho conuiene al Rei & Si ergo Magnificentia , \textbf{ et Magnanimitas sunt maxima bona , } et maxime decet regiam maiestatem \\\hline
1.1.7 & commo estas ¶ \textbf{ Et mucho seria de denostar } si la su bien andança pusiese en estas riquezas corporales ¶ & esse ornatam talibus virtutibus , \textbf{ detestabile est ei suam felicitatem in talibus ponere . Secundo detestabile est Regi , } vel Principi suam felicitatem ponere in diuitiis , \\\hline
1.1.7 & Lo segundo se declara \textbf{ assi mucho es de denostar el Rei o el prinçipe } que pone su bien andança en las riquezas corporales . & detestabile est ei suam felicitatem in talibus ponere . Secundo detestabile est Regi , \textbf{ vel Principi suam felicitatem ponere in diuitiis , } quia hoc facto Tyrannus efficitur . Est enim differentia \\\hline
1.1.7 & e la su bien andança en las riquezas e en los aueres \textbf{ prinçipalmente entiende de thesaurizar } e fazer thesoro & ponens suam felicitatem in numismate , \textbf{ principaliter intendit reseruare sibi , } et congregare pecuniam . Non ergo est Rex , \\\hline
1.1.7 & prinçipalmente entiende de thesaurizar \textbf{ e fazer thesoro } e llegar muchos dineros & ponens suam felicitatem in numismate , \textbf{ principaliter intendit reseruare sibi , } et congregare pecuniam . Non ergo est Rex , \\\hline
1.1.7 & e fazer thesoro \textbf{ e llegar muchos dineros } Et por ende se sigue & principaliter intendit reseruare sibi , \textbf{ et congregare pecuniam . Non ergo est Rex , } sed Tyrannus , \\\hline
1.1.7 & deue ser tan afincadamente amada e desseanda \textbf{ que cada huno deue estudiar en toda manera } e por qual quier carrera que pudiere & Nam finis adeo intense diligitur , \textbf{ quod quilibet studet omni via , omni modo , } quo potest , \\\hline
1.1.7 & e por qual quier carrera que pudiere \textbf{ por que pueda alcançar aquella fin } e aquel bien & quod quilibet studet omni via , omni modo , \textbf{ quo potest , } consequi finem suum . Ponens igitur suam felicitatem in diuitiis , \\\hline
1.1.7 & solamente \textbf{ que el pueda allegar riquezas e dineros . } Ca aquel que bien entiende & si depraedetur populum et Rem publicam , \textbf{ dum tamen possit pecuniam congregare , } qui enim bene intelligit , \\\hline
1.1.7 & Ca aquel que bien entiende \textbf{ que quiere dezir } e quanto lieua este nonbre fin & qui enim bene intelligit , \textbf{ quid importatur nomine finis , } non potest eum latere quemlibet , omni via qua potest , \\\hline
1.1.7 & e quanto lieua este nonbre fin \textbf{ e bien andança non se le puede esconder } por ninguna manera & quid importatur nomine finis , \textbf{ non potest eum latere quemlibet , omni via qua potest , } velle \\\hline
1.1.7 & por ninguna manera \textbf{ que non pueda querer seguir la su fin } ante se trabaia dela alcançar & non potest eum latere quemlibet , omni via qua potest , \textbf{ velle } consequi suum finem . Est igitur Rex Tyrannus , \\\hline
1.1.7 & que non pueda querer seguir la su fin \textbf{ ante se trabaia dela alcançar } quanto puede ¶ & non potest eum latere quemlibet , omni via qua potest , \textbf{ velle } consequi suum finem . Est igitur Rex Tyrannus , \\\hline
1.1.7 & e sea tyrano ¶ \textbf{ Bien asi es cosa contra razon de poner el Rey la su feliçidat } e la su bien andança en las riquezas corporales . & Regem admittere maxima bona , \textbf{ esse Tyrannum , et depraedatorem detestabile quoque est suam felicitatem in diuitiis ponere . } Forte multi viuentes vita politica credunt felicitatem ponendam esse in honoribus , \\\hline
1.1.8 & ora uentra a muchos biuen uida politica \textbf{ e creen que es de poner la feliçidat } e la bien andança enlas honrras . & esse Tyrannum , et depraedatorem detestabile quoque est suam felicitatem in diuitiis ponere . \textbf{ Forte multi viuentes vita politica credunt felicitatem ponendam esse in honoribus , } eo quod ut plurimum Ciues maxime honorari desiderant . \\\hline
1.1.8 & por que por la mayor ꝑͣte todos les çibdadanos dessean honrra et de ser honrrados . \textbf{ Mas en las honrras son tres cosas de cuydar } por las quales podemos en tres maneras mostrar & eo quod ut plurimum Ciues maxime honorari desiderant . \textbf{ Sunt autem in honoribus tria attendenda , } per quae triplici via venari possumus , \\\hline
1.1.8 & Mas en las honrras son tres cosas de cuydar \textbf{ por las quales podemos en tres maneras mostrar } que non deuemos poner lanr̃a bien andança en las honrras & Sunt autem in honoribus tria attendenda , \textbf{ per quae triplici via venari possumus , } felicitatem in eis ponendam non esse . \\\hline
1.1.8 & por las quales podemos en tres maneras mostrar \textbf{ que non deuemos poner lanr̃a bien andança en las honrras } ¶ & per quae triplici via venari possumus , \textbf{ felicitatem in eis ponendam non esse . } Primo enim honor est bonum \\\hline
1.1.8 & que es ordenado a uirtud . \textbf{ la bien andança non es de poner en la honrra } mas es de poner enlas uirtudes o en las obras dellas ¶ & quare si honor est bonum ordinatum ad virtutem , \textbf{ in honore non est ponenda felicitas , } sed potius in ipsis virtutibus , vel in actibus earum . \\\hline
1.1.8 & la bien andança non es de poner en la honrra \textbf{ mas es de poner enlas uirtudes o en las obras dellas ¶ } Lo segundo se praeua & in honore non est ponenda felicitas , \textbf{ sed potius in ipsis virtutibus , vel in actibus earum . } Secundo non debite ponitur felicitas in honoribus , \\\hline
1.1.8 & Lo segundo se praeua \textbf{ assi que la bien andança non se deue poner en las honrras } por que la honrra es bien de fuera del cuerpo & sed potius in ipsis virtutibus , vel in actibus earum . \textbf{ Secundo non debite ponitur felicitas in honoribus , } quia honor non est bonum intrinsecum , \\\hline
1.1.8 & Mas la señal o el testimoino \textbf{ si conplida mente quiere demostrar le que significa conuiene } que sea conosçida cosa e magnifiesta . & vel testimonium , \textbf{ si plene manifestare vult ipsum signatum , } oportet quod sit quid notum et manifestum : \\\hline
1.1.8 & e mani fiestas anos . \textbf{ Ca non podemos conosçer } lo que cada vno pienssa en su coraçon & sed extrinseca : \textbf{ non enim cognoscimus quae quis in corde cogitat , } sed quae exterius repraesentat . Reuerentia ergo , \\\hline
1.1.8 & que es dada en testimo non de uertud \textbf{ e ha de magnifestar la uirtud de aquel a quien la fazen non cunple } que la uertud sea penssada en el coraçon & quae est honor , \textbf{ si debet manifestare virtutem eius } cui exhibetur , non sufficit , quod si cogitata in corde , \\\hline
1.1.8 & Por la qual razon \textbf{ si la bien andança non es de poner } en los bienes de fuera & cum sit reuerentia exhibita per quaedam exteriora signa . \textbf{ Quare si felicitas non est ponenda in bonis extrinsecis , } quae sunt bona minora ; \\\hline
1.1.8 & que son bienes pequanos . \textbf{ mas es de poner en los bienes del alma } que son bienes mayores . & quae sunt bona minora ; \textbf{ sed in intrinsecis , } quae sunt bona maiora , \\\hline
1.1.8 & que son bienes mayores . \textbf{ Siguese que la bien andança non se deue poner en las honrras } que son bienes de fuera ¶ & quae sunt bona maiora , \textbf{ in honoribus felicitas poni non debet . } Tertio hoc idem patet \\\hline
1.1.8 & por la qual cosa si la bien andança fuere puesta en las honrras \textbf{ conteçer a vn yerro muy desaguisado } que la bien andança & ut plane vult Philosophus 1 Ethicorum quare si in honoribus ponatur felicitas , \textbf{ continget illa detestanda peruersitas , quod felicitas magis } sit in alio , quam in eo , \\\hline
1.1.8 & que es mas en el bien andante \textbf{ que non en otro ninguno non es de poner en las honrras ¶ } Pues que assi es muy sin Razon es & vel in ipso beatificato , \textbf{ in honoribus non est ponenda felicitas . Indecens est ergo cuilibet homini ponere suam felicitatem in honoribus , } ut credat se esse felicem , \\\hline
1.1.8 & ponga la su bien andança en las honrras . \textbf{ Et ahun esto podemos prouar } por tres razones . & Maxime tamen hoc est indecens regiae maiestati : \textbf{ quod etiam triplici via venari potest . } Si enim Rex suam felicitatem in honoribus ponat , sequentur ipsum tria mala : erit enim superficialiter bonus erit praesumptuosus , et erit iniustus , et inaequale . \\\hline
1.1.8 & de ser bueno uerdaderamente \textbf{ e non solamente de paresçer bueno . } muy mas desconueinble cosaes ael que otro ninguon de poner su bienandança en las honrras & Si ergo maxime decet Regem esse bonum existentem , \textbf{ maxime indecens est ipsum ponere felicitatem in honoribus , } ne sit fictus , et superficialis . Secundo indecens est Regi , \\\hline
1.1.8 & e non solamente de paresçer bueno . \textbf{ muy mas desconueinble cosaes ael que otro ninguon de poner su bienandança en las honrras } por que non paresca infinto e superfiçial ¶ & Si ergo maxime decet Regem esse bonum existentem , \textbf{ maxime indecens est ipsum ponere felicitatem in honoribus , } ne sit fictus , et superficialis . Secundo indecens est Regi , \\\hline
1.1.8 & Lo segundo se muestra \textbf{ assi que muy desconueible cosa es al Rey poner su bien andança en las honrras } Ca por esso seria prisuptuoso e sob̃uio & ø \\\hline
1.1.8 & si el prinçipe pusiere la su bien andança enlas honrras \textbf{ por que pueda delo que feziere honrra alcançar presumira de poner los pueblos a todo peligro } por que pueda alcançar aquella honrra¶ & si Princeps suam felicitatem in honoribus ponat , \textbf{ ut possit honorem consequi , | praesumet suam gentem exponere omni periculo . Exemplum huiusmodi habemus } de filio cuiusdam Romani Principis nomine Torquati , \\\hline
1.1.8 & por que pueda delo que feziere honrra alcançar presumira de poner los pueblos a todo peligro \textbf{ por que pueda alcançar aquella honrra¶ } Et desto auemos enxienplo en vn fijo de vn prinçipe Romano & praesumet suam gentem exponere omni periculo . Exemplum huiusmodi habemus \textbf{ de filio cuiusdam Romani Principis nomine Torquati , } qui nimii honoris auidus , \\\hline
1.1.8 & contra el imperio de su padre . \textbf{ Et por que pudiese ganar honrra pusose a muchos peligros de batallas . } Onde este torcato prinçipe romano & contra Imperium Patris , \textbf{ ut honorem sequeretur , | se exposuit periculis bellicis : } unde Torquatus Romanus Princeps , \\\hline
1.1.8 & por que non caya en peligro non presuma \textbf{ nin se ensoƀuezca mucho nol conuiene de poner su bien andança en las honrras ¶ } Lo terçero se demuestra assi & Ne ergo Princeps se praecipitet , et ne nimis praesumat , \textbf{ non expedit ei suam felicitatem in honoribus ponere . } Tertio hoc est indicens ei , \\\hline
1.1.8 & que sea iniusto nin desegual ¶ \textbf{ Mas conuiene le partir los sus bienes alos sus vassallos } segunt las dignidades delas personas ¶ & ne sit iniustus et inaequalis : \textbf{ decet enim Principem sua bona distribuere } secundum dignitatem personarum , \\\hline
1.1.8 & e muy desseada \textbf{ non fara fuerça de dar gualardon alas personas } segunt sus dignidades & quia finem summo ardore diligit , \textbf{ non curabit remunerare personas } secundum propriam dignitatem , \\\hline
1.1.8 & Et sera malo al pueblo \textbf{ qual es acomnedado de gouernar } Ca non fara fuerça de poner el pueblo a grandes peligros presuptuosamente e arrebatadamente . & erit malus in suis rebus , \textbf{ quia eas non distribuet aequaliter } secundum personarum dignitatem . \\\hline
1.1.8 & qual es acomnedado de gouernar \textbf{ Ca non fara fuerça de poner el pueblo a grandes peligros presuptuosamente e arrebatadamente . } Et sera malo en partir sus aueres . & quia eas non distribuet aequaliter \textbf{ secundum personarum dignitatem . } Quod non decet regiam maiestatem , \\\hline
1.1.8 & Ca non fara fuerça de poner el pueblo a grandes peligros presuptuosamente e arrebatadamente . \textbf{ Et sera malo en partir sus aueres . } Ca non los partir & quia eas non distribuet aequaliter \textbf{ secundum personarum dignitatem . } Quod non decet regiam maiestatem , \\\hline
1.1.8 & Et sera malo en partir sus aueres . \textbf{ Ca non los partir } a egualmente segt̃ las dignidades de las personas & secundum personarum dignitatem . \textbf{ Quod non decet regiam maiestatem , } suam ponere felicitatem in gloria , \\\hline
1.1.9 & a egualmente segt̃ las dignidades de las personas \textbf{ t deuedes saber } que entre estas dos cosas eglesia e fama ha grand diferençia & vel in \textbf{ Differunt autem gloria , } et fama ab honore , et laude : \\\hline
1.1.9 & que parte nesçena reuereçia ¶ \textbf{ Mas la eglesia e la fama suelen los omes tomar por vna cosa Ca la eglesia es vn claro conosçe ineto dela persona } por que es vna claridat magnifiesta & et caetera talia exteriora signa pertinentia ad reuerentiam . \textbf{ Gloria quidem et fama pro eodem accipi consueuit : | nam gloria est quaedam clara notitia de aliquo : } est enim gloria claritas quaedam : \\\hline
1.1.9 & Ca la fama es vn claro conosçimiento con loor . \textbf{ Enpero si quisieremos fazer depart ineto entre la fama e la eglesia diremos que la fama nasçe de la eglesia } Pues que assi es esta es la orden destas cosas & Si tamen vellemus aliquo modo \textbf{ distinguere | inter gloriam , et famam : diceremus quod fama oritur ex gloria : } erit ergo hic ordo , \\\hline
1.1.9 & que la feliçidat \textbf{ e la bien andança es de poner } e en fama e en eglesia & et famam large , \textbf{ et pro eodem , } posset forte alicui videri felicitatem ponendam esse in fama et gloria , \\\hline
1.1.9 & por que dura por mucho stp̃os \textbf{ e non se puede desfazer . Et ahun paresçe } que los prinçipes deuen poner mayormente la su feliçidat & et magnae diuturnitatis , \textbf{ cum per multa tempora contingat ipsam indelebilem esse . } Videtur ergo quod maxime Princeps in hoc suam felicitatem ponere debeat , \\\hline
1.1.9 & e non se puede desfazer . Et ahun paresçe \textbf{ que los prinçipes deuen poner mayormente la su feliçidat } e la su bien andança en la eglesia e en la & cum per multa tempora contingat ipsam indelebilem esse . \textbf{ Videtur ergo quod maxime Princeps in hoc suam felicitatem ponere debeat , } dicente Philosopho 5 Ethic’ \\\hline
1.1.9 & honrra¶por que dize el philosofo en el quinto libro delas ethicas \textbf{ que gualardon alguno deuemos dar alos prinçipes . } Et este gualardon es honrra e eglesia de los quales & dicente Philosopho 5 Ethic’ \textbf{ quod merces quaedam danda est Principibus , } haec autem est honor et gloria , \\\hline
1.1.9 & Mas que esta opinion non sea uerdadera \textbf{ que la feliçidat e la bien andança es de poner } en la fama e en la honra & ø \\\hline
1.1.9 & en la fama e en la honra \textbf{ podemos lo prouar } por tres razones & Sed hoc stare non potest ; \textbf{ quod triplici via venari possumus . } Quantum enim ad praesens spectat , \\\hline
1.1.9 & mas los malos e los desauentraados son auidos en fama e en eglesia ¶ \textbf{ Lo terçero en la fama non es de poner la feliçidat nin la bienandança por que mas paresçe en las sennales de fuera } que en la bondat de dentro . & et infelices sunt in gloria et in fama . \textbf{ Tertio in fama non est ponenda felicitas , | quia magis innititur exterioribus signis , } quam interiori bonitati : \\\hline
1.1.9 & e semeiante a dios \textbf{ si non es cosa conuenible de poner la feliçidat } e la bien andança & et quasi semideum , \textbf{ inconueniens est felicitatem ponere in eo , } quod est signum bonitatis , \\\hline
1.1.9 & que en la bondat . \textbf{ Et non es cosa conuenible de poner la bien andança } en aquello que puede auer los malos & quam sit bonitas , \textbf{ et quod ipse praui participare possunt , } et magis innittitur exterioribus signis , quam interiori bonitati : \\\hline
1.1.9 & Et non es cosa conuenible de poner la bien andança \textbf{ en aquello que puede auer los malos } njn en aquello que paresçe en las señales de fuera & quam sit bonitas , \textbf{ et quod ipse praui participare possunt , } et magis innittitur exterioribus signis , quam interiori bonitati : \\\hline
1.1.9 & njn en aquello que paresçe en las señales de fuera \textbf{ mas es de poner en las bondades de dentro . } Et non es cosa conuenible & et quod ipse praui participare possunt , \textbf{ et magis innittitur exterioribus signis , quam interiori bonitati : } inconueniens enim est , \\\hline
1.1.9 & o si es głioso en los pueblos ¶ \textbf{ Et pues que assi es el rey non deue creer } que es bien auenturado & vel si sit in populis gloriosus . \textbf{ Non igitur debet Rex se credere esse beatum , } si sit in gloria apud homines : \\\hline
1.1.9 & en el deçimo dela methafisica ¶ \textbf{ Otrosi el conosçimiento de dios es tal en qua non puede caer enganno . } Et el nuestro conosçimiento es tal en que muchas vezes puede caer enganno ¶ & notitia nostra causatur a rebus , ut vult Commen’ 12 Met’ . \textbf{ Rursus notitia Dei est infallibilis , } notitia nostra pluries fallitur . \\\hline
1.1.9 & Otrosi el conosçimiento de dios es tal en qua non puede caer enganno . \textbf{ Et el nuestro conosçimiento es tal en que muchas vezes puede caer enganno ¶ } La terçera diferençia es entre el conosçimiento de dios & Rursus notitia Dei est infallibilis , \textbf{ notitia nostra pluries fallitur . } Amplius notitia Dei est de ipsis intimis nostris , \\\hline
1.1.9 & ¶Otro si el conosçimientode dios no puede ser engannado en lanr̃a bondat . \textbf{ Ca lascian de dios non puede resçebir enganno . } Et ahun dezimos mas adelante & Rursus circa bonitatem nostram notitia \textbf{ Dei non fallit , | cum scientia sua falli non possit . Amplius bonitatem , } et malitiam nostram intimam Deus clare videt , \\\hline
1.1.9 & que es ante el . \textbf{ Enpo en ninguna manera non es de poner la bienandança de lons omes } en la fama de los omes & et de gloria apud ipsum Deum , \textbf{ nequaquam tamen in fama , } et in gloria hominum felicitas est ponenda . \\\hline
1.1.9 & que la fama et lagłia era cosa durable \textbf{ e que se estendie muchon podemos dezir } que es el contrarioça es muy pequana e muy estrecha . & Quod vero dicebatur , \textbf{ quod est diuturna et lata . } Dici potest , \\\hline
1.1.9 & non passo el monte de caucaso ¶ \textbf{ Et pues que assi es commo fama nin eglesia de vir omne } non se puede estender & quia est modica et arta : \textbf{ nam } secundum Boetium fama Romani populi nunquam transiuit Caucasum montem . Quomodo ergo fama unius hominis per uniuersam terram se extendet ? Sed , cum tota terra sit \\\hline
1.1.9 & Et pues que assi es commo fama nin eglesia de vir omne \textbf{ non se puede estender } por todo el mundo . & quia est modica et arta : \textbf{ nam } secundum Boetium fama Romani populi nunquam transiuit Caucasum montem . Quomodo ergo fama unius hominis per uniuersam terram se extendet ? Sed , cum tota terra sit \\\hline
1.1.9 & en aquella deue ser puesta la bien andança de los omes \textbf{ que sienpre ha de durar } e nunca ha de fallesçer ¶ & in eo debet felicitatem ponere , \textbf{ quod immortaliter et perpetuo durare possit . } Quod vero dicebatur mercedem tribuendam esse Regibus , \\\hline
1.1.9 & que sienpre ha de durar \textbf{ e nunca ha de fallesçer ¶ } Mas aquello que algunos dizian & in eo debet felicitatem ponere , \textbf{ quod immortaliter et perpetuo durare possit . } Quod vero dicebatur mercedem tribuendam esse Regibus , \\\hline
1.1.9 & que el gualardon \textbf{ que los Reis deuian auer era en eglesia e en honrra segunt el philosofo dezie . } El testo del philosofo non se deue & Quod vero dicebatur mercedem tribuendam esse Regibus , \textbf{ et hunc esse honorem } et gloriam . Non est intelligendus textus Philosophi , \\\hline
1.1.9 & El testo del philosofo non se deue \textbf{ assi entender que los Reys prinçipalmente } por su meresçimiento deuen demandar & et gloriam . Non est intelligendus textus Philosophi , \textbf{ quod Reges principaliter pro suo merito quaerere debeant gloriam , } et famam Hominum , \\\hline
1.1.9 & assi entender que los Reys prinçipalmente \textbf{ por su meresçimiento deuen demandar } e quere reglesia e fama de los omes . & et gloriam . Non est intelligendus textus Philosophi , \textbf{ quod Reges principaliter pro suo merito quaerere debeant gloriam , } et famam Hominum , \\\hline
1.1.9 & por que conuiene alos Reys alos prinçipes de ser manificos e grandes e magnanimos e de grandes coraçones . \textbf{ Ca commo quier los de altos coraçones entienden prinçipalmente de tomar honrra } mas de alcançar algun bien & et magnanimos . \textbf{ Magnanimi autem licet non intendant principaliter honorem , } sed bonum : \\\hline
1.1.9 & Ca commo quier los de altos coraçones entienden prinçipalmente de tomar honrra \textbf{ mas de alcançar algun bien } enpero la honrra les parte nesçe a ellos . Et conuiene les alos Reys de resçebir la honrra & Magnanimi autem licet non intendant principaliter honorem , \textbf{ sed bonum : } honor tamen eos consequitur , \\\hline
1.1.9 & mas de alcançar algun bien \textbf{ enpero la honrra les parte nesçe a ellos . Et conuiene les alos Reys de resçebir la honrra } que les fazenn los omes & sed bonum : \textbf{ honor tamen eos consequitur , } et decet eos acceptare honorem sibi exhibitum , \\\hline
1.1.9 & que les fazenn los omes \textbf{ por que los omes non les pueden dar mayor cosa que honrra } que non han meior cosa & honor tamen eos consequitur , \textbf{ et decet eos acceptare honorem sibi exhibitum , } non habentibus Hominibus aliquid maius , \\\hline
1.1.9 & que non han meior cosa \textbf{ que les puedan dar ¶ } assi commo dize el philosofo & non habentibus Hominibus aliquid maius , \textbf{ quod eis tribuant , } ut dicitur 4 Ethic’ , \\\hline
1.1.9 & assi commo ally dize el philosofo ¶ \textbf{ pues que assi es en dos maneras se puede entender } que . . omne resçiba honrra e gualardon egual e digno al su mesçimiento ¶ & non quod honor fit condigna retributio eis , \textbf{ ut ibidem dicitur . Acceptare ergo honorem tanquam retributionem condignam , potest intelligi dupliciter , } vel ratione ipsius honoris in se , vel ut procedit \\\hline
1.1.9 & La primera manera es pensando \textbf{ que cosa es la honrra en si¶ } La otra manera es teniendo mientes al talante de aquellos que la fazen . & ut ibidem dicitur . Acceptare ergo honorem tanquam retributionem condignam , potest intelligi dupliciter , \textbf{ vel ratione ipsius honoris in se , vel ut procedit } ex affectione dantium . \\\hline
1.1.9 & nin mas conuenible \textbf{ que puedan dar a su prinçipe . } Et en esta manera deuen resçebir la honrra que les fazen & maius quod retribuant , \textbf{ congruit Principi hoc modo honorem exhibitum acceptare . Propter quod ( ut plane patet ) } in huiusmodi retributione acceptatur affectio dantium , \\\hline
1.1.9 & que puedan dar a su prinçipe . \textbf{ Et en esta manera deuen resçebir la honrra que les fazen } Por la qual cosa & maius quod retribuant , \textbf{ congruit Principi hoc modo honorem exhibitum acceptare . Propter quod ( ut plane patet ) } in huiusmodi retributione acceptatur affectio dantium , \\\hline
1.1.9 & Por la qual cosa \textbf{ assi commo llanamente paresçe en tal glardones de catar la uoluntad de aquellos } que fazen la honrra & congruit Principi hoc modo honorem exhibitum acceptare . Propter quod ( ut plane patet ) \textbf{ in huiusmodi retributione acceptatur affectio dantium , } et non proprie honor datus . \\\hline
1.1.10 & que fizo dela caualleria \textbf{ que sobre todas las cosas es de alabar la maestria e la sabiduria delas batallas . } Et esta es vna cosa segunt & Vegetius in libro De re militari , \textbf{ super omnia commendare videtur bellorum industriam . } Hoc enim ( secundum ipsum ) \\\hline
1.1.10 & Et sobre todas las cosas estudiaron \textbf{ commo pudiessen subiugar todas las naçiones } e todas las gentes . & et summo opere studuerunt , \textbf{ quomodo possent sibi subiicere nationes . } Propter quod ( secundum eundem Vegetium ) hoc esse debet principalissimum in intentione Principis , \\\hline
1.1.10 & que abonde en poderio ciuil que es poderio de çibdat e de regno \textbf{ e que por este poder puereda subiugar } assi las naçiones e las gentes ¶ & Propter quod ( secundum eundem Vegetium ) hoc esse debet principalissimum in intentione Principis , \textbf{ quod abundet in ciuili potentia , } et quod per eam sibi subiiciat nationes \\\hline
1.1.10 & por çinco razones \textbf{ que la feliçidat e la bien andança de los Reyes e de los prinçipes non se deue poner } en el poderio çiuil¶ La primera razon se tomadaquello & probat enim Philosophus in 7 Pol’ quinque rationibus felicitatem non esse ponendam in ciuili potentia . \textbf{ Prima via sumitur , } ex eo quod talis principatus non multum durat . Secunda , \\\hline
1.1.10 & La primera razon se puede \textbf{ assi declarar } Ca querer subiugar las naconnes e las gentes & Prima via sic patet . \textbf{ Nam per ciuilem potentiam velle sibi subiicere nationes , } hoc est , \\\hline
1.1.10 & assi declarar \textbf{ Ca querer subiugar las naconnes e las gentes } por poderio çiuil esto esquerer enssen onrear por fuerça e non pornatraa & Prima via sic patet . \textbf{ Nam per ciuilem potentiam velle sibi subiicere nationes , } hoc est , \\\hline
1.1.10 & Ca querer subiugar las naconnes e las gentes \textbf{ por poderio çiuil esto esquerer enssen onrear por fuerça e non pornatraa } Mas ninguna cosa & Nam per ciuilem potentiam velle sibi subiicere nationes , \textbf{ hoc est , } velle dominari per violentiam . Violentia autem perpetuitatem nescit . \\\hline
1.1.10 & pues que assi es \textbf{ que las cosas forçadas non pueden mucho durar } tal prinçipado non puede mucho durar & velle dominari per violentiam . Violentia autem perpetuitatem nescit . \textbf{ Cum igitur violenta non diu durent , } talis principatus diu durare non potest . \\\hline
1.1.10 & que las cosas forçadas non pueden mucho durar \textbf{ tal prinçipado non puede mucho durar } ca es por fuerça & Cum igitur violenta non diu durent , \textbf{ talis principatus diu durare non potest . } Immo sicut ignis , \\\hline
1.1.10 & si el pueblo liberalmente \textbf{ e de su uoluntad quisiere segnir los manerdamientos del prinçipe } e non por fuerça ¶ & si populus libere , \textbf{ et voluntarie praecepta Principis exequatur . } Non ergo Rex debet se credere esse felicem , \\\hline
1.1.10 & e non por fuerça ¶ \textbf{ Pues que assi es non deue creer el Rey } nin el prinçipe & et voluntarie praecepta Principis exequatur . \textbf{ Non ergo Rex debet se credere esse felicem , } si per violentiam , et per ciuilem potentiam dominetur : \\\hline
1.1.10 & Ca tal señorio commo sea por fuerça \textbf{ e cotran atraa non puede mucho durar ¶ } Et por ende la feliçidat & cum sit violentum , \textbf{ et contra naturam , } diu durare non potest : felicitas enim non est ponenda in aliquo transitorio , \\\hline
1.1.10 & Et por ende la feliçidat \textbf{ e la bien andaça non es de poner } en ninguna cosa passadera nin tenporal . & et contra naturam , \textbf{ diu durare non potest : felicitas enim non est ponenda in aliquo transitorio , } sed magis in aliquo sempiterno . \\\hline
1.1.10 & en ninguna cosa passadera nin tenporal . \textbf{ Mas es de poner en aquello } que sienpre ha de durar ¶ & diu durare non potest : felicitas enim non est ponenda in aliquo transitorio , \textbf{ sed magis in aliquo sempiterno . } Secundo in ciuili potentia non est ponenda felicitas , \\\hline
1.1.10 & Mas es de poner en aquello \textbf{ que sienpre ha de durar ¶ } La segunda razon se declara & diu durare non potest : felicitas enim non est ponenda in aliquo transitorio , \textbf{ sed magis in aliquo sempiterno . } Secundo in ciuili potentia non est ponenda felicitas , \\\hline
1.1.10 & La segunda razon se declara \textbf{ assi que la feliçidat e la bien andança non es de poner en poderio çiuilca tal poderio puede ser en alguonssin bondat de uida . } Ca la feliçidat e la bien andança & Secundo in ciuili potentia non est ponenda felicitas , \textbf{ quia hoc potest inesse alicui absque bonitate vitae , felicitas enim , } ut supra dicebatur , \\\hline
1.1.10 & Et por ende el philosofo dize en el se partimo libro delas politicas \textbf{ que cosa de escarnio es cuydar } que alguno puede seer bien auentraado & Unde Philosophus 7 Politicorum ait , \textbf{ quod ridiculum est } aliquem putare esse felicem , \\\hline
1.1.10 & si despreçia el bien beuir . \textbf{ Et pues que assi es non es de poner la feliçidat } e la bien andança en poderio çiuil & aliquem putare esse felicem , \textbf{ si abiiciat bene viuere . Non ergo in ciuili potentia est ponenda felicitas , } quae sine bonitate vitae inesse potest . \\\hline
1.1.10 & la terçera razon muestra \textbf{ que la feliçidat et la bien andança non es de poner en este poderio çiuil . } Por que este sennorio non es muy bueno nin muy digno . & quae sine bonitate vitae inesse potest . \textbf{ Tertio in huiusmodi potentia non est ponenda felicitas , } quia huiusmodi Principatus non est optimus , \\\hline
1.1.10 & muy bueno e muy digno . \textbf{ Mas ensennorear } por poderio çiuiles ensseñorear alos sieruos & Si enim felicitas in aliquo Principatu poni debet , \textbf{ ponenda est in Principatu optimo , et digno . Principari autem per ciuilem potentiam , est principari seruis , } non liberis : \\\hline
1.1.10 & Mas ensennorear \textbf{ por poderio çiuiles ensseñorear alos sieruos } e non alos libres & Si enim felicitas in aliquo Principatu poni debet , \textbf{ ponenda est in Principatu optimo , et digno . Principari autem per ciuilem potentiam , est principari seruis , } non liberis : \\\hline
1.1.10 & que los sieruos . \textbf{ en tanto ensseñorear alos libres } e alos francos es meior & ø \\\hline
1.1.10 & e mas digno \textbf{ que ensseñorear a los sieruos . } Et por que el señorio por fuerça e por poderio çiuil commo non sea delons libres & est melius et dignius , \textbf{ quam principari seruis . } Principatus ergo per coactionem , et ciuilem potentiam , \\\hline
1.1.10 & que es con uirtud es muy meior \textbf{ que ensennorear despotice } que quiere dezir enssennorear . seruilmente e sobre los sieruos ¶ & est melior , \textbf{ quam principari despotice , } idest dominaliter . \\\hline
1.1.10 & que ensennorear despotice \textbf{ que quiere dezir enssennorear . seruilmente e sobre los sieruos ¶ } La quarta razon es & quam principari despotice , \textbf{ idest dominaliter . } Quarto non est ponenda felicitas in ciuili potentia : \\\hline
1.1.10 & La quarta razon es \textbf{ que la feliçidat e la bien andança non se deue poner en poderio çiuil . } Por que si el prinçipe o el Rey crea & idest dominaliter . \textbf{ Quarto non est ponenda felicitas in ciuili potentia : } quia si Princeps se crederet esse felicem , \\\hline
1.1.10 & Et aquellas cosas \textbf{ por que pueda subiugar } assi las naçiones e los pueblos . & si abundet in ciuili potentia , non ordinabit ciues , nisi ad exercitum armorum , \textbf{ et ad ea , per quae sibi possit subiicere nationes . } Inducet ergo ciues non ad virtutem iustitiae , \\\hline
1.1.10 & assi las naçiones e los pueblos . \textbf{ Et por ende non induzir a los çibdadanos a uirtud de iustiçia } mas a uertud de fortaleza . & et ad ea , per quae sibi possit subiicere nationes . \textbf{ Inducet ergo ciues non ad virtutem iustitiae , } sed ad virtutem fortitudinis . Iustitia autem , \\\hline
1.1.10 & que el bien singular et personal . \textbf{ Non conuiene alos prinçipes poner su bien andança en el poderio çiuil¶ } La quinta razon & quam aliquod bonum singulare , \textbf{ non decet principem | suam felicitatem ponere in ciuili potentia . } Quinto hoc non decet ipsum , \\\hline
1.1.10 & La quinta razon \textbf{ por que non conuiene al prinçipe poner la su bien andança en el poderio çiuiles } por que este sennorio faze grant danno en las mas cosas . & suam felicitatem ponere in ciuili potentia . \textbf{ Quinto hoc non decet ipsum , } quia huiusmodi principatus infert \\\hline
1.1.10 & e su bien andança en poderio çiuil \textbf{ e en subiugar las naçions e las gentes e los pueblos . } Por auentra a auer se ha muy bien & ponens ergo suam felicitatem in ciuili potentia , \textbf{ et in subiiciendo sibi nationes , } forte bene se habebit tempore belli : \\\hline
1.1.10 & mas sera uiçioso e pecador . \textbf{ Et contesçer le ha muy grant daño segunt su alma . } por la qual cosa dize el philosofo en el septimo libro delas politicas & nesciet viuere , \textbf{ sed fiet vitiosus , et incurret nocumentum } secundum animam . Propter quod Philosophus 7 Politicorum vituperans Lacedaemones , \\\hline
1.1.10 & por las quales cosas ya dichas \textbf{ si non es cosa conuenible de poner la bien andança } en alguna cosa & fieri vitiosus . \textbf{ Quare si inconueniens est ponere felicitatem in aliquo non diuturno , } et in eo quod potest \\\hline
1.1.10 & Et por que cuyde que es bien auentraado \textbf{ quando pudiere subiugar } assi las naçiones . et las gentes . & et quod credat se esse felicem , \textbf{ si possit sibi subiicere nationes multas . | Quod non deceat Regiam maiestatem } suam felicitatem ponere in robore corporali , \\\hline
1.1.11 & por las cosas que son dichas \textbf{ quanones de poner la bien andança en los bienes corporales } Empero tres son los bienes corporales & Satis per habita manifestum est , \textbf{ quod in bonis corporalibus non est ponenda felicitas . } Sunt \\\hline
1.1.11 & que ponen su bien andança en tales cosas \textbf{ Mas deuedes saber } que los bienes del cuerpo & sanitatem , pulchritudinem , et robur . \textbf{ Immo nonnulli in talibus suam felicitatem ponunt . Videtur enim omnino esse contrarium de bonis corporis , } et animae . \\\hline
1.1.11 & de qual quier qua non se diesse a ellas \textbf{ e preçiar le yan poco ¶ } Et pues que assi es que los bienes corporales & Quod si bene eam cognoscerent , \textbf{ non solum non increparent dantes se scientiis et virtutibus , } sed etiam deriderent quicunque talibus non vacarent . Bona ergo corporalia , \\\hline
1.1.11 & que creen que ellos son de tan manna a una taia \textbf{ que si los podiessen auer } que serien por ellos bien andantes ¶ & quia non habita reputantur maiora quam sint , multi carentes corporalibus bonis , adeo affectant ea , \textbf{ et credunt ipsa esse tantae excellentiae , } ut si haberent illa , reputarent se esse felices . Ideo dicitur 1 Ethic’ \\\hline
1.1.11 & Mas que la feliçidat e la bien andança \textbf{ non es de poner en fortaleza } nin en sanidat nin en fermosura & turpes in pulchritudine , debiles in robore . \textbf{ Sed quod non sit felicitas in robore , } nec in sanitate , \\\hline
1.1.11 & nin en sanidat nin en fermosura \textbf{ podemos lo prouar } por tres razones ¶ & nec in pulchritudine , \textbf{ triplici via venari possumus . Primo , } quia talia bona sunt corporalia . Secundo , \\\hline
1.1.11 & La segunda por que son bienes de fuera \textbf{ e non de dentro del alma¶ } La terçera es & quia talia bona sunt corporalia . Secundo , \textbf{ quia quodammodo exteriora . Tertio , } quia sunt de facili mutabilia . Corporalia enim sunt , \\\hline
1.1.11 & La terçera es \textbf{ por que de ligero se pueden mudar e perder¶ } La primera razon paresçe & quia quodammodo exteriora . Tertio , \textbf{ quia sunt de facili mutabilia . Corporalia enim sunt , } quia habent esse in corporalibus , \\\hline
1.1.11 & fablando delas cosas \textbf{ que omne ha de escoger } que la salut es egualamiento conuenible de los humores . & Nam ( ut vult Philosop’ \textbf{ 3 De eligendis ) } sanitas est debita adaequatio humorum . Pulchritudo est debita commensuratio membrorum . \\\hline
1.1.11 & e que se tienen de parte del alma . \textbf{ Et por ende non es de poner la bien andança } en sanidat & quae se tenent ex parte animae : \textbf{ ponenda est ergo felicitas } non in sanitate , \\\hline
1.1.11 & e la bien andança conplida \textbf{ non se puede auer en esta uida . } Et pues que assi es & credat se esse felicem . Dicimus autem \textbf{ ( ut exigit suus status ) } quia plena felicitas \\\hline
1.1.11 & Et pues que assi es \textbf{ si la bien andança non se deue poner } en la sanidat & quia plena felicitas \textbf{ in hac vita haberi non potest . In sanitate ergo , } et fortitudine , \\\hline
1.1.11 & Mas en los bienes solos de dentro del alma \textbf{ es propiamente de poner la feliçidat e la bien andaça¶ } Et esto paresçe por el philosofo en el septimo libro delas politicas & felicitas poni non debet . \textbf{ Quod autem in bonis interioribus sit proprie felicitas , } patet per Philosophum 7 Politicorum dicentem , \\\hline
1.1.11 & que dios es a nos testigo \textbf{ que la feliçidat e la bien andança es de poner en los bienes de dentro del alma . } Et esto testigua dios & quod testis est nobis Deus , \textbf{ quod felicitas in bonis interioribus est ponenda . Testificatur enim hoc Deus per seipsum , } ut idem ibidem innuit : \\\hline
1.1.11 & por si mesmo \textbf{ assi commo da a enteder el philosofo en aquel logar } do dize que dios non es bien auentraado & quod felicitas in bonis interioribus est ponenda . Testificatur enim hoc Deus per seipsum , \textbf{ ut idem ibidem innuit : } nam Deus non est beatus per aliqua exteriora bona , \\\hline
1.1.11 & La terçera razon es \textbf{ que en tales bienes non es de poner la feliçidat } nin la bien andança & nisi per bona , \textbf{ quae sunt in seipsa . Tertio in talibus bonis non est ponenda felicitas , } quia sunt valde mutabilia . Aequatio enim humorum , \\\hline
1.1.11 & que se mueuen \textbf{ e se puerden perder de ligero . } Ca la egualança de los humores & quia sunt valde mutabilia . Aequatio enim humorum , \textbf{ cum subsunt motui supercoelestium corporum , } variationi aeris , mutationi ciborum , \\\hline
1.1.11 & commo sea subiecta \textbf{ al mouimientode los cuere pos çelestiales } e al moumiento del ayre & variationi aeris , mutationi ciborum , \textbf{ de facili variationem recipit . } Sed amissa sanitate , amittitur robur corporis , \\\hline
1.1.11 & Et pues que assi es non conuiene al rey \textbf{ nin a ningun omnen poner la su feliçidat } e la su bien andança en tales cosas & Non decet ergo Regem , \textbf{ nec aliquem hominem in talibus suam felicitatem ponere , quae sunt corporalia , } et quodammodum exteriora , \\\hline
1.1.11 & que son de fuera \textbf{ e que ligeramente se pueden mudar e perder¶ } Pues que assi es para entender todas las cosas sobredichas & et quodammodum exteriora , \textbf{ et de facili mutabilia . Dicamus ergo ad intelligentiam omnium dictorum , } quod non decet Principem felicitatem ponere in voluptatibus corporalibus , nec in diuitiis , \\\hline
1.1.11 & e que ligeramente se pueden mudar e perder¶ \textbf{ Pues que assi es para entender todas las cosas sobredichas } digamos & et quodammodum exteriora , \textbf{ et de facili mutabilia . Dicamus ergo ad intelligentiam omnium dictorum , } quod non decet Principem felicitatem ponere in voluptatibus corporalibus , nec in diuitiis , \\\hline
1.1.11 & que non conuiene al Rey \textbf{ nin al prinçipe poner su feliçidat } e su bien andança & et de facili mutabilia . Dicamus ergo ad intelligentiam omnium dictorum , \textbf{ quod non decet Principem felicitatem ponere in voluptatibus corporalibus , nec in diuitiis , } nec in honoribus , nec in fama , \\\hline
1.1.11 & ¶ \textbf{ Enpero de tondas estas colas deue husar } assi commo de instrumentos & nec in pulchritudine : \textbf{ omnibus tamen istis debet uti , } ut sunt organa ad felicitatem . \\\hline
1.1.11 & assi commo de instrumentos \textbf{ para alcançar la feliçidat e la bien andança . } Ca deue husar de viandas & omnibus tamen istis debet uti , \textbf{ ut sunt organa ad felicitatem . } Debet enim uti cibis , \\\hline
1.1.11 & para alcançar la feliçidat e la bien andança . \textbf{ Ca deue husar de viandas } en las quales es delectaçion corporal & ut sunt organa ad felicitatem . \textbf{ Debet enim uti cibis , } in quibus est delectatio corporalis , \\\hline
1.1.11 & Ca commo el Rey sea cabesça de su Regno \textbf{ por fallesçimiento dela su persona podria uenir mucha mengua en la gente } Otrosi deue husar dela obra del matrimoino & cum ipse sit caput Regni , \textbf{ ex defectu eius posset consurgere malum gentis . } Debet uti actu matrimoniali propter conseruationem speciei , \\\hline
1.1.11 & por fallesçimiento dela su persona podria uenir mucha mengua en la gente \textbf{ Otrosi deue husar dela obra del matrimoino } por conseruaçion & ex defectu eius posset consurgere malum gentis . \textbf{ Debet uti actu matrimoniali propter conseruationem speciei , } siue propter procreationem prolis : \\\hline
1.1.11 & e por guarda del su linage \textbf{ o por engendrar fijos . } Ca por fallesçimiento de los fijos muchos regnos ouieron grand departimiento & Debet uti actu matrimoniali propter conseruationem speciei , \textbf{ siue propter procreationem prolis : } nam ex defectu filiorum multa regna passa sunt diuisionem , et scandala : \\\hline
1.1.11 & e fazen grand discordia en el pueblo ¶ \textbf{ Otrossi deuen los prinçipes auer riquezas sufiçientes } por que puedan defender los regnos & et faciunt dissensionem in Populo . \textbf{ Debet enim Princeps possidere sufficientes diuitias , } ut possit regnum defendere , \\\hline
1.1.11 & Otrossi deuen los prinçipes auer riquezas sufiçientes \textbf{ por que puedan defender los regnos } e fazer obras de uertudes . & Debet enim Princeps possidere sufficientes diuitias , \textbf{ ut possit regnum defendere , } et exercere operationes virtutum : \\\hline
1.1.11 & por que puedan defender los regnos \textbf{ e fazer obras de uertudes . } E conuiene al Rey de seer magnifico e largo & ut possit regnum defendere , \textbf{ et exercere operationes virtutum : } decet enim Regem esse magnificum , \\\hline
1.1.11 & por que pueda bien fazera las personas dignas \textbf{ la qual cosa non se puede fazer sin riquezas ¶ } Ahun en essa misma manera es el Rey digno de honrra & beneficiare personas dignas : \textbf{ quod sine diuitiis fieri non potest . Sic etiam , } ne vilipendatur maiestas regia , \\\hline
1.1.11 & por que non sea menospreçiada la Real magestad . \textbf{ Et por ende le conuiene de auer poderio çeuil . } Ca por el menospreçiamientodel prinçipe muchͣs vezes contesçe que alguons fazen & ne vilipendatur maiestas regia , \textbf{ est Rex dignus honore , et expedit ei habere ciuilem potentiam : } nam propter paruipensionem Principis , \\\hline
1.1.11 & la qual cosa non conuiene al Regno . \textbf{ Et ahun en essa misma guła deue auer cuydado el prinçipe de auer buena fama . } Ca por esso se enduzen los sus subditos & etiam \textbf{ debet esse curae ipsi Principi de debita fama , quia propter hoc inducuntur subditi ad virtutem . } Nam ( ut probatum est ) \\\hline
1.1.11 & por bueno los subditos toman manera \textbf{ para fazer bien . } Ahun en essa misma manera la sanidat e la fermosura & subditi suscipiunt materiam benefaciendi . Sic etiam , \textbf{ sanitas , pulchritudo , } et fortitudo competunt Principi , \\\hline
1.1.11 & mas por que son instru mentos \textbf{ para ganar la feliçidat e la bien andança . } Et pues que assi es estas cosas tales son de amar & non quod in eis sit proprie felicitas , \textbf{ sed quia possunt esse organa ad felicitatem . } Talia ergo diligenda sunt , \\\hline
1.1.11 & para ganar la feliçidat e la bien andança . \textbf{ Et pues que assi es estas cosas tales son de amar } en quanto son instrumentos para ganar la feliçidat & sed quia possunt esse organa ad felicitatem . \textbf{ Talia ergo diligenda sunt , } ut sunt organa ad felicitatem , \\\hline
1.1.11 & Et pues que assi es estas cosas tales son de amar \textbf{ en quanto son instrumentos para ganar la feliçidat } e la bien andança . & Talia ergo diligenda sunt , \textbf{ ut sunt organa ad felicitatem , } et ut faciunt ad quandam claritatem felicitatis . \\\hline
1.1.11 & e dela bien andança ¶ \textbf{ por la qual cosa destas cosas commo son de amar } aqui passamos lo breuemente & de his , \textbf{ quomodo diligenda sunt , hic breuiter pertransiuimus , } quia de eis inferius sumus aliqua tractaturi . \\\hline
1.1.12 & por que adelante lo tractaremos mas conplida mente . \textbf{ euedes saber que el philosofo puso dos feliçidades e dos bien andanças } la vna es politica e actiua & quia de eis inferius sumus aliqua tractaturi . \textbf{ Duas autem felicitates Philosophus posuit , } unam politicam , \\\hline
1.1.12 & que es en el entedemiento \textbf{ Ca dize que la feliçidat e la bien andança non se deue poner } en las fuerças corporales & aliam contemplatiuam . \textbf{ Voluit autem felicitatem non esse ponendam in viribus , } siue in potentiis animae , \\\hline
1.1.12 & Ca assi lo quiere el philosofo en el primero libro delas ethicas ¶ \textbf{ Et pues que assi es deue se poner la bienandança en las obras del alma } e non en las obras de pecados & non videntur discerni felices a miseris , \textbf{ ut innuit Philosophus 1 Ethicorum . In actu ergo , siue in operatione animae est ponenda felicitas : non in operatione vitii , sed in operatione virtutis : non virtutis cuiuslibet , sed virtutis perfectae . Felicitas ergo } ( ut dicitur 1 Ethicor’ ) \\\hline
1.1.12 & Et pues que assi es qual si quier omne \textbf{ que sepa bien gouernar los otros } segunt pradençia & secundum ipsum , \textbf{ quicunque scit alios bene regulare } secundum Prudentiam , \\\hline
1.1.12 & que es derecha sabiduria de todas las obras \textbf{ que han de fazer } es bien auentado en la uidapolitica . & secundum Prudentiam , \textbf{ est felix politice : } qui vero scit bene speculari \\\hline
1.1.12 & es bien auentado en la uidapolitica . \textbf{ Et qual quier que sepa bien entender } segund methaphisica & est felix politice : \textbf{ qui vero scit bene speculari } secundum Metaphysicam , \\\hline
1.1.12 & Et dela otra determino en el decimo dela metha phisica \textbf{ Mas de commo esto se ha de entender } o si es uerdadera la opinion del philosofo & ut de felicitate politica : \textbf{ de alia vero , scilicet contemplatiua determinat in 10 . Utrum autem sit vera illa positio Philosophi , } vel non , \\\hline
1.1.12 & o si es uerdadera la opinion del philosofo \textbf{ o non non la auemos aqui de determinar . } Enpero por que paresca lo primero & de alia vero , scilicet contemplatiua determinat in 10 . Utrum autem sit vera illa positio Philosophi , \textbf{ vel non , | non est praesentis speculationis . } Tamen , \\\hline
1.1.12 & en qual manera conuenga ala Real magestad \textbf{ de poner la primera feliçidat en las obras de pradençia . ¶ Et la segunda commo le conuiene de poner er la su bien andança solamente en dios . } Esto pondemos prouar por tres razones ¶ & ponere \textbf{ suam felicitatem in actu prudentiae , | sciendum quod decet Regem maxime suam felicitatem ponere in ipso Deo , } quod triplici via venari possumus . \\\hline
1.1.12 & de poner la primera feliçidat en las obras de pradençia . ¶ Et la segunda commo le conuiene de poner er la su bien andança solamente en dios . \textbf{ Esto pondemos prouar por tres razones ¶ } La primera es esta¶ & sciendum quod decet Regem maxime suam felicitatem ponere in ipso Deo , \textbf{ quod triplici via venari possumus . } Rex enim est homo , \\\hline
1.1.12 & en quanto es omne \textbf{ e ha razon e entendimiento de poner la su bien andança en bien muy comun e muy entelligible } Et este tal bienes dios & et rationem participat , \textbf{ ponere suam felicitatem in bono maxime uniuersali , | et maxime intelligibili : } hoc autem est ipse Deus , \\\hline
1.1.12 & La segunda razon \textbf{ por que el rey ha de poner la su bien andaça } en dios solo es esta . & quia est maxime simplex , \textbf{ et maxime a materia separatus . Secundo decet Principem suam felicitatem ponere in ipso Deo , } non solum quia homo est , \\\hline
1.1.12 & e non lo ha conplidamente es instrumento de aquel que la ha naturalmente e conplidamente . \textbf{ Et pues que assi es commo dios solo aya poderio de regnar } e de gouernar prinçipalmente e acabadamente . & quod habet illud essentialiter et perfecte , \textbf{ quia ergo vim regitiuam , } et potentiam regendi habet principaliter , \\\hline
1.1.12 & Et pues que assi es commo dios solo aya poderio de regnar \textbf{ e de gouernar prinçipalmente e acabadamente . } Conuiene que qual se quier prinçipeo Rey & quia ergo vim regitiuam , \textbf{ et potentiam regendi habet principaliter , | et perfecte solus Deus , } oportet quod quicunque principatur , \\\hline
1.1.12 & Conuiene que qual se quier prinçipeo Rey \textbf{ que ha de gouernar sea instrumento de dios } e que sea su ofiçial & siue regnat , \textbf{ sit diuinum organum , } siue sit minister Dei . \\\hline
1.1.12 & e que sea su ofiçial \textbf{ para fazer las sus obras . } Por la qual cosa & sit diuinum organum , \textbf{ siue sit minister Dei . } Quare si minister , suam mercedem , \\\hline
1.1.12 & si los ofiçiales \textbf{ e los seruientes del señor deuen poner la su merçed } e el su gualardon en el su señor & Quare si minister , suam mercedem , \textbf{ et suum praemium } debet ponere in suo Domino , et debet eam expectare ab ipso , \\\hline
1.1.12 & e el su gualardon en el su señor \textbf{ e deuen la esparar del . } Conuiene al Rey & et suum praemium \textbf{ debet ponere in suo Domino , et debet eam expectare ab ipso , } decet Regem , \\\hline
1.1.12 & Conuiene al Rey \textbf{ que es ofiçial de dios poner la su bien andança en dios que es prinçipal señor } e del solo deue esperar gualardon e merçed & decet Regem , \textbf{ qui est Dei minister , suam felicitatem ponere in ipso Deo , } et suum praemium expectare ab ipso . Tertio hoc decet Regem , ex eo , \\\hline
1.1.12 & que es ofiçial de dios poner la su bien andança en dios que es prinçipal señor \textbf{ e del solo deue esperar gualardon e merçed } ¶La terçera razon & qui est Dei minister , suam felicitatem ponere in ipso Deo , \textbf{ et suum praemium expectare ab ipso . Tertio hoc decet Regem , ex eo , } quod est multitudinis rector : \\\hline
1.1.12 & ¶La terçera razon \textbf{ por que el Rey ha de poner su bien andança } en dios es & ø \\\hline
1.1.12 & por que es gouernador de mucho Ca \textbf{ el que gouienna a muchos deue tener mientesal bien comun de todos . } Et por ende deue poner la su feliçidat & quod est multitudinis rector : \textbf{ nam regens multitudinem debet intendere commune bonum . } In eo ergo debet suam felicitatem ponere , \\\hline
1.1.12 & el que gouienna a muchos deue tener mientesal bien comun de todos . \textbf{ Et por ende deue poner la su feliçidat } e la su bien andança & nam regens multitudinem debet intendere commune bonum . \textbf{ In eo ergo debet suam felicitatem ponere , } quod est maxime , et commune bonum . \\\hline
1.1.12 & al bien comun \textbf{ deue pener la su feliçidat } e la su bien andança en dios & et tum quia intendit bonum commune , \textbf{ debet suam felicitatem ponere in Deo , } cui seruit , \\\hline
1.1.12 & e la su bien andança en dios \textbf{ a quien deue seruir . } Ca el es bien muy enteligible & debet suam felicitatem ponere in Deo , \textbf{ cui seruit , } qui est bonum maxime intelligibile , et maxime uniuersale , \\\hline
1.1.12 & e muy unun sal et muy comun ¶ \textbf{ Et pues que el Rey deue poner la su feliçidat } e la su bien andança en dios . & et commune . \textbf{ Si ergo Rex debet in Deo ponere suam felicitatem , oportet ipsum huiusmodi felicitatem ponere in actu illius virtutis , } per quem maxime Deo coniungimur : \\\hline
1.1.12 & e la su bien andança en dios . \textbf{ Conuiene le dela poner en la obra de aquella uirtud } por la qual masyna se puede ayuntar con dios . & ø \\\hline
1.1.12 & Conuiene le dela poner en la obra de aquella uirtud \textbf{ por la qual masyna se puede ayuntar con dios . } Et esta es obra de caridat e de amor de dios & Si ergo Rex debet in Deo ponere suam felicitatem , oportet ipsum huiusmodi felicitatem ponere in actu illius virtutis , \textbf{ per quem maxime Deo coniungimur : } huiusmodi autem est actus dilectionis , \\\hline
1.1.12 & Et esta es obra de caridat e de amor de dios \textbf{ Ca el amor e la caridat ha muy grant fuerça para nos ayuntar con dios ¶ } Et por ende dionisio & siue charitatis . \textbf{ Nam amor , | et dilectio maxime vim unitiuam , } et coniunctiuam habent . \\\hline
1.1.12 & commo aquel commo humanal o angelical o diuinal \textbf{ ha muy grant uirtud de ayuntar } al que ama & siue humanum , \textbf{ siue naturalem , } siue animalem , \\\hline
1.1.12 & Pues que assi es en el amor de dios \textbf{ es de poner la feliçidat en la bien andança } e por que la praeua del amor & siue animalem , \textbf{ unitiuam quandam dicimus esse virtutem . In amore ergo diuino est ponenda felicitas . } Sed cum probatio dilectionis sit exhibitio operis , \\\hline
1.1.12 & aquello que el su amigoquiere¶ Si el prinçipe es bien auenturado \textbf{ amando a dios deue creer } que es bien auenturado & si Princeps est felix diligendo Deum , \textbf{ debet credere se esse felicem operando quae Deus vult . } Maxime autem Deus requirit a Regibus et Principibus , \\\hline
1.1.12 & Et pues que assi es los Reyes \textbf{ et los prinçipes deuen poner la su feliçidat } e la su bien andança & Regibus ergo , \textbf{ et Principibus ponenda est felicitas in actu prudentiae , } non simpliciter , \\\hline
1.1.12 & sin ningun medio enlła \textbf{ deuemos poner la feliçidat e la bien andança } mas que en las obras dela pradençia & per quam immediatius coniungimur ipsi Deo , \textbf{ magis est ponenda felicitas , } quam in actu prudentiae : \\\hline
1.1.12 & mas que en las obras dela pradençia \textbf{ Como quier que en estas obras dela perdençia es de poner en alguna manera la feliçidat et la bien andança } segunt dicho es & quam in actu prudentiae : \textbf{ licet in huiusmodi actu modo } quo dictum est , \\\hline
1.1.13 & segunt dicho es \textbf{ or çinco razones podemos prouar } quant grant es el gualardon de los reyes & aliqualiter felicitas sit ponenda . \textbf{ Magnum autem esse praemium Regis , } et magnam eius esse felicitatem , \\\hline
1.1.13 & e las un bien andança \textbf{ que han de auer } si gouernare el su pueblo & et magnam eius esse felicitatem , \textbf{ si per prudentiam , } et legem recte regat populum sibi commissum , \\\hline
1.1.13 & quanto ꝑ tenesçe alo presente \textbf{ a çinço cosas se puede conparar } ¶L primero a dios el que ha a dar el gualardon¶ & Nam merces Regis ( ut ad praesens spectat ) \textbf{ ad quinque comparari videtur ; | videlicet , } ad Deum , \\\hline
1.1.13 & a çinço cosas se puede conparar \textbf{ ¶L primero a dios el que ha a dar el gualardon¶ } Lo segundo al Rey & videlicet , \textbf{ ad Deum , | a quo redditur : } ad Regem , \\\hline
1.1.13 & Lo segundo al Rey \textbf{ que lo ha de resçebir ¶ } Lo terçero ala obra dela qual se leunata el gualadon¶ & ad Regem , \textbf{ cui tribuitur : } ad actum , \\\hline
1.1.13 & que ninguno de los sus subditos . \textbf{ ante por esso mesmo que el es Rey deue estudiar } por que gouierne su regno & ut sit Deo conformior , \textbf{ quam eius subditi . Immo eo ipso quod Rex studet per legem , et prouidentiam suum regnum regere , } quomodo Deus totum uniuersum regit \\\hline
1.1.13 & que es en ellas . \textbf{ Ca los que pueden passar los mandamientos } e non los passan mas de loarso & nam omnis actus ex ipsa difficultate operis quandam bonitatem assumit : \textbf{ potentes enim transgredi , } si non transgrediantur , \\\hline
1.1.13 & Ca muchos esta non tal estado \textbf{ que podrian mal fazer } e guardan se delo fazer & laudabiliores fiunt : \textbf{ multi enim non existentes in statu quo possint mala facere , } praeseruant se a malo : \\\hline
1.1.13 & que podrian mal fazer \textbf{ e guardan se delo fazer } Enpero si a mayor estado fuese leunata dos aurian razon & multi enim non existentes in statu quo possint mala facere , \textbf{ praeseruant se a malo : } quod si tamen ad statum dignitatis assumerentur , \\\hline
1.1.13 & Enpero si a mayor estado fuese leunata dos aurian razon \textbf{ para fazer muchͣs males . } Et por esso dize aristotiles en el quinto libro delas ethicas & quod si tamen ad statum dignitatis assumerentur , \textbf{ multas transgressiones efficerent . Propter quod Ethic’ 5 scribitur , } quod principatus virum ostendit . Tunc enim apparet qualis homo sit , cum in principatu existens , in quo potest bene et male facere , \\\hline
1.1.13 & quando es puesto en señorio \textbf{ en que pueda fazer bien e mal . } Et aquella hora entiendan los omes & ø \\\hline
1.1.13 & assi mismo \textbf{ a quien es de dar paresçe } que deue auer vna grandeza & ad ipsum Regem , \textbf{ cui reddenda est , } quandam magnitudinem habere videtur : \\\hline
1.1.13 & a quien es de dar paresçe \textbf{ que deue auer vna grandeza } e vna aun ataia sobre el gualardon de los otros & cui reddenda est , \textbf{ quandam magnitudinem habere videtur : } Reges enim vacantes communi bono , \\\hline
1.1.13 & si non trispassaren los mandamientos de dios \textbf{ conmolos podiessen trispassar son de mayor meresçimiento . } Et dezimos que los reys trabaian en el bien comun . & si non transgrediantur , \textbf{ cum possint transgredi , | maioris meriti esse videntur . Dicimus autem } ( vacantes communi bono ) \\\hline
1.1.13 & non les conuernia de ser de mayor meresçimiento \textbf{ por que ellos pueden passar los mandamientos } e non los passan & quia si bono communi non vacarent , \textbf{ non oportet eos esse maioris meriti ex hoc quod transgredi possent , } quia non considerato communi bono , \\\hline
1.1.13 & por el bien comun menguarian \textbf{ enlo que han de fazer } e non acresçentarian en su meresçimiento . & ø \\\hline
1.1.13 & e de los prinçipes \textbf{ es esta penssando en las obras por las quales han de auer su gualardon bueno o mal . } Ca por tanto es dicha la obra uiçiosa e mala & magnum est meritum Principis , \textbf{ considerato actu , | per quem tale meritum habet esse . } Nam ex hoc actus est vitiosus , \\\hline
1.1.13 & e contra orden de natura e de razon \textbf{ Et por tanto es dicha la obrar buena e uertuosa } en quanto es segunt natura & et contra rationis ordinem . \textbf{ Ex hoc autem est bonus , | et virtuosus , } inquantum est \\\hline
1.1.13 & luego se pone en toda su fuerça \textbf{ por defender la cabeça } por que todo el cuerpo non peresça¶ & et totum corpus , brachium quod est pars corporis , \textbf{ statim exponit se totaliter pro capite , } ne totum corpus pereat . \\\hline
1.1.13 & pues que assi es los reyes \textbf{ si bien gouernar en las gentes } que les son acomne dadas & ne totum corpus pereat . \textbf{ Reges ergo si bene regant gentem sibi commissam , } ex operibus eorum consequenter mercedem magnam : \\\hline
1.1.13 & Ca menor uirtud cunple atondo omne \textbf{ para gouernar } assi mesmo & per qua quis meretur huiusmodi meritum : \textbf{ nam minor virtus requiritur ad regendum seipsum , } quam ad regendum familiam , \\\hline
1.1.13 & assi mesmo \textbf{ que para gouernar a su conpanna . } Et menorꝑan gouernar a su conpanna & nam minor virtus requiritur ad regendum seipsum , \textbf{ quam ad regendum familiam , } et quam ad regendum ciuitatem : \\\hline
1.1.13 & que para gouernar a su conpanna . \textbf{ Et menorꝑan gouernar a su conpanna } que para gouernar vna çibdat o vn regno ¶ & quam ad regendum familiam , \textbf{ et quam ad regendum ciuitatem : } magna ergo debet esse virtus Regis , \\\hline
1.1.13 & Et menorꝑan gouernar a su conpanna \textbf{ que para gouernar vna çibdat o vn regno ¶ } pues que assi es grande deue ser la uirtud del Rey & quam ad regendum familiam , \textbf{ et quam ad regendum ciuitatem : } magna ergo debet esse virtus Regis , \\\hline
1.1.13 & pues que assi es grande deue ser la uirtud del Rey \textbf{ a quien parte nesçe de gouernar } non solamente assi mesmo & magna ergo debet esse virtus Regis , \textbf{ ad quem spectat regere non solum se , et suam familiam , sed etiam totum regnum . } Cum ergo magnae virtuti debeatur magna merces , \\\hline
1.1.13 & mas ahun a todo el regno . \textbf{ Et pues que assi es commo grant uirtud deua auer grant merçed } e gm̃t gualardon gerad sera el meresçimiento & ad quem spectat regere non solum se , et suam familiam , sed etiam totum regnum . \textbf{ Cum ergo magnae virtuti debeatur magna merces , } magnum erit meritum bene regentium regnum suum . Quinto , \\\hline
1.1.13 & e el gualardon de los Reyes . \textbf{ quando bien gouernar en sus regnos ¶ } La quinta razonn & Cum ergo magnae virtuti debeatur magna merces , \textbf{ magnum erit meritum bene regentium regnum suum . Quinto , } magnum erit praemium ipsorum Regum , \\\hline
1.1.13 & en la materia \textbf{ en que deuen obrar . } Ca çida vno es es alabado & si consideretur materia , \textbf{ circa quam operatur : } laudatur enim aliquis , \\\hline
1.2.1 & Et mostramos \textbf{ en que deuen poner los Reyes } e los prinçipes la su feliçidat & in quo agitur de regimine sui , \textbf{ ostendentes in quo Reges } et Principes \\\hline
1.2.1 & e la su bien andança . \textbf{ Et que non los conuiene poner la su fin en riquezas } nin en poderio çiuil & suam felicitatem debeant ponere , \textbf{ quia non decet eos suum finem ponere in diuitiis , } nec in ciuili potentia , \\\hline
1.2.1 & nin en nuguaso tris cosas corporales nin tenporales \textbf{ Mas assi commo prouamos conplidamente de suso deuen husar de todas estas cosas } assi commo de instrumentos & nec in aliquibus talibus , \textbf{ sed omnibus his } ( \\\hline
1.2.1 & assi commo de instrumentos \textbf{ para ganar la feliçidat e la bien andança . } Mas la su bien andança deuen poner en obras de pradençia e de sabiduria & sed omnibus his \textbf{ ( } ut supra plenius probabatur ) debent \\\hline
1.2.1 & para ganar la feliçidat e la bien andança . \textbf{ Mas la su bien andança deuen poner en obras de pradençia e de sabiduria } segund que tales obras son regladas & sed omnibus his \textbf{ ( } ut supra plenius probabatur ) debent \\\hline
1.2.1 & e la su feliçidat bien andança \textbf{ qual deuen auer } e qual parte nesçe a su estado & Suam autem felicitatem ponere debent in actu prudentiae , \textbf{ prout talis actus est imperatus a charitate : } nam tunc Reges habent felicitatem suo statui debitam , \\\hline
1.2.1 & e la bien andança de los Reyes \textbf{ e de los prinçipes es de poner en solo dios . } Ca deuen ellos ordenar la su uida & et iuste regant . \textbf{ Principaliter ergo Regum felicitas ponenda est in ipso Deo , } et ex cognitione et dilectione eius studium suum , \\\hline
1.2.1 & e de los prinçipes es de poner en solo dios . \textbf{ Ca deuen ellos ordenar la su uida } e el su estado & et iuste regant . \textbf{ Principaliter ergo Regum felicitas ponenda est in ipso Deo , } et ex cognitione et dilectione eius studium suum , \\\hline
1.2.1 & e demostrado en qual cosahan los Reyes \textbf{ e los prinçipes de poner su fin } segund la orden ya dicha & His ergo itaque peractis , \textbf{ et ostenso in quo Reges ponere debeant suum finem } secundum ordinem superius praetaxatum , \\\hline
1.2.1 & segund la orden ya dicha \textbf{ e assignada finca nos de demostrar en quales uirtudes deuen resplandesçer los Reyes e los prinçipes . } Ca las uirtudes son vnos hornamentos e conponimientos e hunas perfectiones & secundum ordinem superius praetaxatum , \textbf{ restat ostendere quibus virtutibus Reges pollere debeant . Virtutes autem quaedam sunt quidam ornatus , } et quaedam perfectiones animae . Oportet ergo prius ostendere , \\\hline
1.2.1 & Et despues desto mostremos \textbf{ en qual manera se depart̃ las uirtudes . } Et despues desto mostr̉emos quantas son & Consequenter autem manifestabitur , \textbf{ quomodo virtutes sunt distinguendae . Postea vero ostendemus , } quot sunt numero huiusmodi virtutes , \\\hline
1.2.1 & Et en qual manera conuiene alos Reyes \textbf{ e alos prinçipes de auer estas uirtudes ¶ } pues que assi es los poderios del alma se pueden & et quomodo decet Reges , \textbf{ et Principes tales virtutes habere . Potentiae autem animae sic distingui possunt , } quia potentiae animae quaedam sunt naturales , quaedam cognitiuae sensitiuae , quaedam appetitiuae , \\\hline
1.2.1 & pues que assi es los poderios del alma se pueden \textbf{ assi departir } ca alguons destos poderios del alma son naturales & ø \\\hline
1.2.1 & que cresçen en la tr̃ra . \textbf{ assi commo son poderio de cerar e de acresçentar e de engendrar } e los quales pertenesçen tan bien alos arboles e alas plantas commo anos ¶ & et plantis , \textbf{ ut potentia nutritiua , augmentatiua , generatiua , } et talia quae etiam ipsis arboribus competunt . Potentiae vero cognitiuae sensitiuae , sunt visus , \\\hline
1.2.1 & e el odoramento \textbf{ e el tannimiento o el veer } e el oyr e el gostar & ø \\\hline
1.2.1 & e el tannimiento o el veer \textbf{ e el oyr e el gostar } e el oler e el tanner & ø \\\hline
1.2.1 & e el oyr e el gostar \textbf{ e el oler e el tanner } Et estos tales son comunes & gustus , auditus , \textbf{ et talia , } in quibus communicamus cum brutis . Appetitiuae vero distinguuntur : \\\hline
1.2.1 & Et el appetito que sigue al entendimiento es llamando uoluntad . \textbf{ Segund essa manera de fablar las bestias han senssualidat et appetito sensitiuo . } Mas non han uoluntad nin appetito & ut appetitus sequens sensum . Appetitus autem sequens sensum potest nominari sensualitas : sequens intellectum nominatur voluntas : \textbf{ secundum quem modum loquendi bruta habent sensualitatem , et appetitum sensitiuum , } sed non habent voluntatem , et intellectum . \\\hline
1.2.1 & Et pues que assi es las uirtudes \textbf{ de que aqui auemos de fablar } que son vnas uestiduras muy loables & Virtutes ergo , \textbf{ de quibus loqui intendimus , } quae sunt quidam habitus laudabiles , \\\hline
1.2.1 & Mas que en los poderios naturales \textbf{ non pueden ser las uirtudes podemos lo prouar } por tres razons ¶ & vel in omnibus his , \textbf{ vel in aliquibus horum . In potentiis autem naturalibus esse non possunt , } quod tripliciter patet . Primo , \\\hline
1.2.1 & por que las disposiconnes e las uirtudes son \textbf{ para se determinar los poderios } para bien obrar o mal assi commo contesçe & et virtutes , \textbf{ qui habitus sunt ad determinandum potentias , } ut bene \\\hline
1.2.1 & para se determinar los poderios \textbf{ para bien obrar o mal assi commo contesçe } que por las malas disposiconnes se determinan los poderios del alma & qui habitus sunt ad determinandum potentias , \textbf{ ut bene | vel male agant , } ut per habitus vitiosos determinatur potentia ad agendum male , \\\hline
1.2.1 & que por las malas disposiconnes se determinan los poderios del alma \textbf{ para mal fazer . } Et por las buenas disposiconnes e uirtudes se apareian a bien obrar . & vel male agant , \textbf{ ut per habitus vitiosos determinatur potentia ad agendum male , } per virtuosos ad agendum bene : \\\hline
1.2.1 & para mal fazer . \textbf{ Et por las buenas disposiconnes e uirtudes se apareian a bien obrar . } ¶ Et pues que assi es commo la natura sea determimada a vna cosa & ut per habitus vitiosos determinatur potentia ad agendum male , \textbf{ per virtuosos ad agendum bene : } cum ergo natura sit determinata ad unum , \\\hline
1.2.1 & e los poderios naturales sean determinandos conplidamente \textbf{ para obrar } segund su natura & cum ergo natura sit determinata ad unum , \textbf{ et potentiae naturales sufficienter determinentur ad agendum , } ex natura sua \\\hline
1.2.1 & que las uirtudes non son en los poderios naturales \textbf{ se puede prouar } que las uirtudes non son en el conosçimiento senssitiuo & per quas probatum est virtutes non esse in potentiis naturalibus , \textbf{ probari potest eas non esse in cognitione sensitiua , } siue in potentiis sensitiuis . \\\hline
1.2.1 & assi commo alguon \textbf{ por mucho comer } o por mucho beuer ouiesse tal enfermadat & nisi forte hoc esset per accidens , \textbf{ ut si quis ex superflua comestione , } vel ex nimia potatione incurrisset ophthalmiam oculorum , \\\hline
1.2.1 & por mucho comer \textbf{ o por mucho beuer ouiesse tal enfermadat } que llaman otalmia & nisi forte hoc esset per accidens , \textbf{ ut si quis ex superflua comestione , } vel ex nimia potatione incurrisset ophthalmiam oculorum , \\\hline
1.2.1 & por que perdiesse la uista delos oios \textbf{ que non pudiesse bien veer } o perdiesse la uirtud del estomago & vel ex nimia potatione incurrisset ophthalmiam oculorum , \textbf{ ne bene videret , } vel debilitatem stomachi , \\\hline
1.2.1 & o perdiesse la uirtud del estomago \textbf{ qua non pudiesse bien moler } la uianda seria denostado & vel debilitatem stomachi , \textbf{ ne bene digereret : } increparetur ille , \\\hline
1.2.1 & e beuio mucho . \textbf{ Ca en su poderio era de vsar } tenpradamente del comer e del beuer ¶ La segunda razon es & vel ex superfluo cibo , \textbf{ nam erat in potestate sua , } ut posset uti potu , et cibo moderate . \\\hline
1.2.1 & Ca en su poderio era de vsar \textbf{ tenpradamente del comer e del beuer ¶ La segunda razon es } por que las uirtudes morales non deuen ser puestas enlos poderios senssibles . & nam erat in potestate sua , \textbf{ ut posset uti potu , et cibo moderate . } Secundo in sensibus non est ponenda virtus moralis , \\\hline
1.2.1 & Ca non es en el poderio del omne \textbf{ de veer } mas claramente nin menos . & non enim est in potestate hominis videre clarius , \textbf{ vel minus clare : } quare si virtus moralis est aliquid \\\hline
1.2.1 & por su naturaleza . \textbf{ Ca assi commo el fuego tanto escalienta quato puede escalentar } por que es determimado en la su obra & sic et sensus . \textbf{ Nam sicut ignis tantum calefacit , | quantum potest calefacere , } propter quod , \\\hline
1.2.1 & por que es determimado en la su obra \textbf{ por esso nies de loar } nin de denostar & quia determinatus est in actione sua , \textbf{ nec laudatur , } nec vituperatur , \\\hline
1.2.1 & por esso nies de loar \textbf{ nin de denostar } por que escalienta nin es en el uirtud moral . & nec laudatur , \textbf{ nec vituperatur , } nec est in eo virtus moralis : \\\hline
1.2.1 & por que escalienta nin es en el uirtud moral . \textbf{ Bien assi por que el poderio de moler la uianda } tanto muele quanto puede . & nec est in eo virtus moralis : \textbf{ sic quia potentia digestiua tantum digerit } quantum potest digerere , et oculus tantum videt \\\hline
1.2.1 & tanto muele quanto puede . \textbf{ Et el oio tanto vequa no puede ueer } en ellos & sic quia potentia digestiua tantum digerit \textbf{ quantum potest digerere , et oculus tantum videt } quantum potest videre , \\\hline
1.2.1 & en ellos \textbf{ non son de poner las uirtudesmorales . } Ca si por la uirtud moral el poderio es determinado para obrar . & quantum potest videre , \textbf{ in eis virtus moralis | esse non potest . } Nam per virtutem moralem determinatur potentia ad agendum : \\\hline
1.2.1 & non son de poner las uirtudesmorales . \textbf{ Ca si por la uirtud moral el poderio es determinado para obrar . } Et estos poderios naturales e senssibles son determinados & esse non potest . \textbf{ Nam per virtutem moralem determinatur potentia ad agendum : } haec autem sufficienter determinantur ad actiones proprias per naturam : \\\hline
1.2.2 & Et mostrado en qual manera han de seer en el entendimiento e en el appetito . \textbf{ Pues que assi es auemos de departir destas uirtudes } et destos poderios & restat ostendere , \textbf{ quomodo distinguuntur virtutes , | et quomodo in appetitu et intellectu existunt . Distinguendum est igitur de virtutibus , } et de huiusmodi potentiis : \\\hline
1.2.2 & Et estas tales uirtudes son iustiçia e tenꝑança e fortaleza e mansedunbre \textbf{ e las o tristałs delas quales auemos de fablar } en este libͤespeçialmente de cada vna dellas ¶ & et caetera talia , \textbf{ de quibus sumus singulariter tractaturi . } Virtutes autem mediae inter intellectuales et morales , \\\hline
1.2.2 & que es regla derecha \textbf{ para bien obrar . } Et las otras uirtudes que se ayuntan a esta . & sunt virtutes existentes in intellectu practico , ut Prudentia , \textbf{ et aliae virtutes sibi annexae . Prudentia autem , } secundum Commentatorem super libris Ethicorum , \\\hline
1.2.2 & para bien obrar . \textbf{ Et las otras uirtudes que se ayuntan a esta . } Ca la pradençia segunt que dize el comnetador & sunt virtutes existentes in intellectu practico , ut Prudentia , \textbf{ et aliae virtutes sibi annexae . Prudentia autem , } secundum Commentatorem super libris Ethicorum , \\\hline
1.2.2 & e entre las inellectuales . \textbf{ Enpero puede se contar con las uirtudes morales } por que la pradençia non es sinon en los buenos omes . & media est inter virtutes morales , et intellectuales ; computari tamen potest \textbf{ cum virtutibus moralibus : } nam Prudentia non est nisi in hominibus bonis , \\\hline
1.2.2 & que son en el entendimiento especulatino . \textbf{ Aqui solamente auemos de fablar dela pradençia } e de las uirtudes morales & ut dimissis scientiis speculatiuis , \textbf{ de prudentia , } et de virtutibus moralibus est tractandum . Suscepimus enim ( ut pluries diximus ) \\\hline
1.2.2 & aquel en que esta la uirtud ¶ \textbf{ Mas deuedes saber } que poderio razonable es en dos maneras & ø \\\hline
1.2.2 & por eennçia Enpero partiçipa con la razon \textbf{ por que es apareiado e inclinado de obedesçer al entendimiento } e ala razon & dicitur tamen participare rationem , \textbf{ quia est aptus natus rationi obedire . Potentiae autem naturales , et sensus , } nec sunt rationales per essentiam , \\\hline
1.2.2 & por que dessean natural . \textbf{ mente descender ayuso ¶ } Bien assi las cosas que lan conosçimiento en quanto han en si mismas aquella forma & et quidam appetitus naturalis , \textbf{ ut naturaliter desiderent esse deorsum : } sic habentia cognitionem prout habent in seipsis formam apprehensam , sequitur quaedam inclinatio , \\\hline
1.2.2 & por los sus contrarios \textbf{ que non podiesse sobir } por su lyuiandat a su lugar & ne ergo ignis per quaecunque contraria agentia impediretur , \textbf{ ne per leuitatem in proprio loco quiesceret , } dedit ei natura potentiam calefactiuam , \\\hline
1.2.2 & por su lyuiandat a su lugar \textbf{ do auia de folgar } dio le natura poderio de escalentar & ne per leuitatem in proprio loco quiesceret , \textbf{ dedit ei natura potentiam calefactiuam , } ut per eam resisteret , \\\hline
1.2.2 & do auia de folgar \textbf{ dio le natura poderio de escalentar } con que pudiesse uençer & ne per leuitatem in proprio loco quiesceret , \textbf{ dedit ei natura potentiam calefactiuam , } ut per eam resisteret , \\\hline
1.2.2 & dio le natura poderio de escalentar \textbf{ con que pudiesse uençer } e corronper a sus contrarios . & dedit ei natura potentiam calefactiuam , \textbf{ ut per eam resisteret , } et ageret in contraria corruptiua . \\\hline
1.2.2 & con que pudiesse uençer \textbf{ e corronper a sus contrarios . } Et pues que assi es si la natura . dio alas cosas & ut per eam resisteret , \textbf{ et ageret in contraria corruptiua . } Si ergo igni , \\\hline
1.2.2 & que non han alma estos dos poderios . \textbf{ vno por que pudiessen yr a ssu lugar } e asu folgura . & ø \\\hline
1.2.2 & e asu folgura . \textbf{ Et otro por que pudiessen uençer } e corronper a sus contrarios mucho mas la natura dio estos dos po de rios alas aian lias & Si ergo igni , \textbf{ et rebus inanimatis natura dedit duplicem potentiam , unam per quam adipiscuntur propriam quietem , et aliam per quam agunt in prohibentia et contraria : } multo magis hoc dedit animalibus . \\\hline
1.2.2 & Et otro por que pudiessen uençer \textbf{ e corronper a sus contrarios mucho mas la natura dio estos dos po de rios alas aian lias } que son mas acabadas ¶ & Si ergo igni , \textbf{ et rebus inanimatis natura dedit duplicem potentiam , unam per quam adipiscuntur propriam quietem , et aliam per quam agunt in prohibentia et contraria : } multo magis hoc dedit animalibus . \\\hline
1.2.2 & e uan contra todas aquellas cosas \textbf{ que les quieren enbargar e contrallar } assi commo es el appetito enssannador & et alium per quem resistunt , \textbf{ et aggrediuntur prohibitiua , } ut contraria , \\\hline
1.2.2 & por que no sean corrunpidos de su propia natura \textbf{ e ayan de desanparar los logares propios } e la su propia folgura Vien & ne corrumpantur a propria natura , \textbf{ et deserant propria loca , } et propriam quietem . \\\hline
1.2.2 & por el qual se enssanan acometen les sus contrarios \textbf{ que los podan enbargar de sus propias delectaçiones ¶ v̉bigera . } assi commo el leon & Per irascibilem vero aggrediuntur contraria , \textbf{ quae possent ea ab huiusmodi delectationibus prohibere . } Ut Leo per concupiscibilem pergit , \\\hline
1.2.2 & Et por el appetito enssannador acomete todas la so trisaina \textbf{ las quel quieren enbargar la su uianda } qual es ael conueinble . & ut prosequatur cibum tanquam quid delectabile : \textbf{ per irascibilem vero aggreditur animalia illa volentia ipsum prohibere in cibo adepto . Propter quod bene dictum est , } quod concupiscibilis respicit bonum , \\\hline
1.2.2 & por si diga tal cosa \textbf{ que deue el omne seguir } Et el mal tal cosa & Nam cum bonum \textbf{ secundum se dicat prosequendum , } malum vero \\\hline
1.2.2 & Et el mal tal cosa \textbf{ de que deue omne foyr siguese } que el appetito desseador & malum vero \textbf{ quid fugiendum : } concupiscibilis , \\\hline
1.2.2 & Mas el appetito enssañador \textbf{ segund el qual ha de cometer } e defenderse de los sue contrarios va alos bienes e alos males senssibles & et mala sensibilia , non \textbf{ secundum se considerata , } sed ut habent rationem ardui , et difficilis : \\\hline
1.2.2 & segund el qual ha de cometer \textbf{ e defenderse de los sue contrarios va alos bienes e alos males senssibles } que siente non tornando los bienes e los males segt̃ & secundum se considerata , \textbf{ sed ut habent rationem ardui , et difficilis : } nam arduitas , \\\hline
1.2.2 & e nos retienen \textbf{ por que non podamos seguir el bien } e esquiuar el mal . & et difficultas potissime sunt repugnantia , et prohibentia , \textbf{ ne possimus consequi bonum , } et vitare malum . Imperfecto ergo egisset natura , \\\hline
1.2.2 & por que non podamos seguir el bien \textbf{ e esquiuar el mal . } pues que assi es menguadamiente lo fiziera la natura & ne possimus consequi bonum , \textbf{ et vitare malum . Imperfecto ergo egisset natura , } si dedisset animalibus concupiscibilem , \\\hline
1.2.2 & Pues que assi es en quanto las . aian las \textbf{ segund su manera conuenible se pueden delectar conplidamente } por el appetito desseador fueles dado appetito enssañador & ut igitur animalia , \textbf{ secundum modum sibi conuenientem , | prout bene possint delectari per concupiscibilem , } data est eis irascibilis , \\\hline
1.2.2 & por el appetito desseador fueles dado appetito enssañador \textbf{ por el qual pueden acometer e arredrar tondas las cosas } que enbargan la su delectaçion . & data est eis irascibilis , \textbf{ per quam resistant , } et aggrediantur impedientia delectationem illam . Duplex est ergo appetitus sensitiuus , irascibilis , \\\hline
1.2.2 & Ca commo el entendimiento sea mas general que el seso . \textbf{ Mas generalmente cata aquello en que ha de obrar } que el seso . & Nam \textbf{ cum intellectus uniuersaliori modo respiciat suum obiectum quam sensus , } appetitus intellectiuus , \\\hline
1.2.2 & Enpero el appetito del entendimiento seyendo vno \textbf{ e esse mismo ua a todo bien que se puede entender ¶ } pues que assi es non es otro & appetitus tamen intellectiuus unus , \textbf{ et idem existens fertur in omne bonum intelligibile . } Non ergo est alius appetitus intellectiuus , \\\hline
1.2.2 & segund el qual segnimos los bienes delectables por el entendimiento . \textbf{ Et acometemos los bienes fuertes de alcançar } assi commo es otro & Non ergo est alius appetitus intellectiuus , \textbf{ secundum quem prosequimur bona delectabilia per intellectum , et aggredimur bona ardua : sicut est alius appetitus sensitiuus , } secundum quem prosequimur \\\hline
1.2.2 & e este es departido . \textbf{ Ca el vno es para enssannar } Et el otro es para cobdiçiar ¶ & et hoc duplex , \textbf{ irascibilis scilicet , } et concupiscibilis : \\\hline
1.2.2 & Ca el vno es para enssannar \textbf{ Et el otro es para cobdiçiar ¶ } Conuiene que toda uirtud moral & irascibilis scilicet , \textbf{ et concupiscibilis : } oportet omnem virtutem moralem , \\\hline
1.2.2 & en quanto la pradençia es dicha vna uirtud moral \textbf{ assi podemos dezir } que segund estos quatro poderios del alma & prout ipsa prudentia dicitur quaedam virtus moralis , \textbf{ dicere possumus } quod \\\hline
1.2.2 & que segund estos quatro poderios del alma \textbf{ en los quales ha de seer la uirtud podemos tomar } e entender quatro uirtudes cardinales e generales delas quales ¶ & dicere possumus \textbf{ quod } secundum has quatuor potentias animae in quibus habet esse virtus , sumptae sunt quatuor Virtutes Cardinales ; \\\hline
1.2.2 & en los quales ha de seer la uirtud podemos tomar \textbf{ e entender quatro uirtudes cardinales e generales delas quales ¶ } La vna es la pradençia e la otra la iustiçia & quod \textbf{ secundum has quatuor potentias animae in quibus habet esse virtus , sumptae sunt quatuor Virtutes Cardinales ; } videlicet , Prudentia , Iustitia , \\\hline
1.2.3 & La sexta es magnifiçençia \textbf{ que es uirtud para fazer grandes cosas ¶ } La septima es manssedunbre ¶ & Magnanimitatem , Largitatem , Magnificentiam , \textbf{ Mansuetudinem , } Veritatem , Affabilitatem , \\\hline
1.2.3 & que es uirtud \textbf{ que faze omne desçender a seer buen conpannon de todos ¶ } pues que assi es contando la iustiçia e la pradençia con estas dichas diez son & et Eutrapeliam , \textbf{ quam bene vertibilitatem , } vel societatem appellare possumus . Igitur computata Iustitia , et Prudentia duodecim sunt virtutes morales ; \\\hline
1.2.3 & Conuiene a los Reyes \textbf{ e alos prinçipes delos auer } Et quales partidas han o quales uirtudes son ayuntadas con estas . & et quomodo decet eas Reges habere , \textbf{ et quas partes habent , vel virtutes annexas , } singulariter est dicendum . Numerus autem earum sic potest accipi . \\\hline
1.2.3 & Mas el cuento dellas \textbf{ assi se puede tomar . } Ca commo el subiecto delas uirtudes sea o el entendimiento o la uoluntad o el appetito senssitiuo . & Nam cum subiectum virtutis sit , vel intellectus , vel voluntas , \textbf{ vel appetitus sensitiuus : } omnis virtus moralis , \\\hline
1.2.3 & las quales pone el philosofo en el segundo libro de las ethicas . \textbf{ Mas el cuento destas uirtudes puede se assi tomar . } Ca las uirtudes son alguas meatades & quas enumerauimus , \textbf{ quas tangit Philosophos circa finem 2 Ethicor’ . Possunt alio modo sic accipi hae virtutes . } Nam virtutes sunt medietates quaedam , \\\hline
1.2.3 & que son en nuestro poder . \textbf{ En las quales cosas nos conuiene de poner meatado ygualdat o derechura . } Et estos son tres conuiene de saber quales . & nisi circa ea quae sunt in potestate nostra , \textbf{ in quibus decet nos ponere medietatem , | vel aequalitatem , siue rectitudinem : } huiusmodi autem , tria sunt , scilicet , \\\hline
1.2.3 & En las quales cosas nos conuiene de poner meatado ygualdat o derechura . \textbf{ Et estos son tres conuiene de saber quales . } Ca son razones derechas & vel aequalitatem , siue rectitudinem : \textbf{ huiusmodi autem , tria sunt , scilicet , } rationes , passiones , et operationes exteriores : \\\hline
1.2.3 & e son obras que fazemos de fuera Et en estas nos conuiene \textbf{ e poner meatad e egualdat . } Ca por las uirtudes deuemos auer las razones derechas & per virtutes enim debemus \textbf{ habere rationes rectas , } passiones moderatas , \\\hline
1.2.3 & e poner meatad e egualdat . \textbf{ Ca por las uirtudes deuemos auer las razones derechas } e las pasiones ordenadas e tenprados . & per virtutes enim debemus \textbf{ habere rationes rectas , } passiones moderatas , \\\hline
1.2.3 & Mas en quanto son en nos las passions regladas \textbf{ e orde nadas podemos tomar } e entender las otras diez uirtudes morales & ø \\\hline
1.2.3 & e orde nadas podemos tomar \textbf{ e entender las otras diez uirtudes morales } delas quales fezimos mençion en comienço deste capitulo ¶ & et aequatas habet esse Iustitia , \textbf{ sed prout sunt in nobis passiones moderatae accipiuntur illae decem virtutes morales , de quibus in principio huius capituli fecimus mentionem . } Iustitia ergo , \\\hline
1.2.3 & e conuiene \textbf{ o lo quel deuen dar } o lo que es suyo . & quod cuilibet tribuatur quod decet , \textbf{ vel quod ei detur quod suum est . } Si autem moderat , et rectificat passiones , \\\hline
1.2.3 & que son en nos dentro en el alma¶ \textbf{ Mas en la terçera manera se puede tomar esta misma diuision } e departimiento de las uirtudes . & passiones in nobis rectificantur , \textbf{ et moderantur . Tertio autem , | haec eadem diuisio sumi potest . } Nam omnis virtus moralis tendit in bonum rationis . \\\hline
1.2.3 & delas quales el cuento \textbf{ e la sufiçiençia se puede tomar en esta manera . Ca dicho es ya } que aquellas uirtudes han de seer çerca delas passiones & et sic sumuntur illae decem virtutes morales superius numeratae . \textbf{ Quarum numerus et sufficientia sic potest accipi . Dictum est enim virtutes illas esse circa passiones : } quas per se habent moderare , \\\hline
1.2.3 & que aquellas uirtudes han de seer çerca delas passiones \textbf{ las quales passiones han de mesurar } e de ygualar & Quarum numerus et sufficientia sic potest accipi . Dictum est enim virtutes illas esse circa passiones : \textbf{ quas per se habent moderare , } et adaequare . \\\hline
1.2.3 & las quales passiones han de mesurar \textbf{ e de ygualar } por si ¶ & quas per se habent moderare , \textbf{ et adaequare . } Prout ergo ex aliis , \\\hline
1.2.3 & tan departidas passiones en el alma \textbf{ e en el coraçon pueden se tomar departidas uirtudes . } En quanto en aquellas passiones & et aliis obiectis surgunt aliae , \textbf{ et aliae passiones , accipi poterant aliae , et aliae virtutes , } prout in illis passionibus \\\hline
1.2.3 & En quanto en aquellas passiones \textbf{ segunt la orden de razon se pueden fallar departidas me atadespues } que assi es & prout in illis passionibus \textbf{ secundum ordinem rationis aliud , } et aliud medium reperitur . \\\hline
1.2.3 & que assi es \textbf{ por que nos conuenga de passar por figurança } e superfiçialmente digamos & et aliud medium reperitur . \textbf{ Ut ergo liceat typo , et superficialiter pertransire , } dicamus quod passiones , \\\hline
1.2.3 & assi se leuantan en nos passiones de temor e de osadia . \textbf{ temoͬ quando fuymos del mal futur . } Osadia quando acometemos algun mal futuro ¶ & et audacia : \textbf{ timor | cum ab eo refugimus , } audacia cum illud aggredimur . \\\hline
1.2.3 & O sele una tan \textbf{ para dar pena a alguno . } Et entonçe es sanna & quod est nobis iam illatum , \textbf{ vel insurgimus ad puniendum , } et tunc est ira , \\\hline
1.2.3 & qua nos fazen dessea pena a otre en vengança de aquel mal . \textbf{ Mas si fallesçemos de dar pena a otre } o de vengaͬ algunan cosa & vel propter malum illatum appetit paenam in vindictam . \textbf{ Si autem ab hac punitione , | vel ab hac vindicta deficimus , } sic est mitiditas : \\\hline
1.2.3 & Assi commo la fortaleza es çerca delas passiones quan asçen del mal futuro \textbf{ que ha de venir . } Et la manssedunbre es çerca delas passiones & quae oriuntur ex malo , \textbf{ ut fortitudo est circa passiones ortas ex malo futuro : } mansuetudo circa passiones ortas ex malo praesenti . \\\hline
1.2.3 & qua nasçendel bien . \textbf{ Entonçe conuiene de departir del bien . } Ca ay vn bien del omne en si . & quae oriuntur ex bono , \textbf{ tunc distinguendum est de bono ; } quia quoddam est bonum hominis in se , quoddam vero est bonum eius , \\\hline
1.2.3 & Ca ay vn bien delectable \textbf{ asi commo bie de comer o de beuer } e de o tristales cosas ¶ & quia quoddam est delectabile , \textbf{ ut cibi , et potus , } et talia : \\\hline
1.2.3 & que ha de seer çerca delas espessas comunales \textbf{ o aquel bien prouechoso es e alto e guaue de alcançar } Et assi es manificençia & quae est circa mediocres sumptus . \textbf{ Vel est illud bonum arduum , } et sic est Magnificentia , \\\hline
1.2.3 & Et daqui paresçe \textbf{ que por que los bienes delectables non pueden auer } assi manera de guaueza & Ex quo patet , \textbf{ quod quia bona delectabilia non sic possunt habere rationem ardui sicut utilia , } et honesta , \\\hline
1.2.3 & assi commo dize el philosofo en el segundo libro delas ethicas . \textbf{ pueden se tomar tres uirtudes . } Ca con los otros omes partiçipamos en palauras e en obras . & quae oriuntur ex bonis , \textbf{ ut communicamus cum aliis , sic ( ut dicitur secundo Ethicorum ) sumuntur tres virtutes . } Nam cum aliis communicamus in verbis , \\\hline
1.2.3 & La otra es eutropolia \textbf{ que quiere dezir buena conuerssaçion } o buena manera de beuir . & opera autem , et verba , \textbf{ ut communicamus } cum aliis deseruiunt nobis ad veritatem , vitam , et ludum . \\\hline
1.2.3 & Mas entropolia \textbf{ que quiere dezir buena conpanma } o buena manera de beuir en conpanna es & non est discolus , sed est affabilis , et curialis . Eutrapelia vero siue bona versio , \textbf{ est , } quando aliquis sic se habet in ludis , \\\hline
1.2.3 & o buena manera de beuir en conpanna es \textbf{ quando alguno se sabe bien auer en los trebeios } assi que non sea garçon o aluardan & est , \textbf{ quando aliquis sic se habet in ludis , } ut non sit histrio , \\\hline
1.2.3 & assi que non sea garçon o aluardan \textbf{ que de todo quier trobar } nin sea monte & ut non sit histrio , \textbf{ quod de omnibus velit ludere : } nec sit agrestis , \\\hline
1.2.3 & nin sea monte \textbf{ sino que non quiera de ninguna cosa fablar nin trebeiar . } Mas sea buen conpanon & nec sit agrestis , \textbf{ quod de nullo velit ludere : } sed sit Eutrapelus \\\hline
1.2.3 & Et la su uirtud es eutropolia \textbf{ que ̀ere dezir buena conpanma . } Mas estas tres uirtudes ya dichas & et bene se vertens , \textbf{ ut se habeat circa ludos prout expedit . } Omnes autem hae tres virtutes , \\\hline
1.2.3 & que commo sean quatro poderios del alma \textbf{ que pueden resçebir las uirtudes } de los quales nos fablamos son doze estas uirtudes & quia non sunt circa aliquid arduum , sunt in concupiscibili . Patet ergo quod cum quatuor potentiae animae sint susceptibiles virtutum \textbf{ de quibus loquimur , } duodecim sunt huiusmodi virtutes , \\\hline
1.2.3 & que es razon derecha \textbf{ para bien obrar . } Et en la uoluntad es la iustiçia & quatuor in irascibili , \textbf{ et sex in concupiscibili . } In intellectu est prudentia . In voluntate iustitia . \\\hline
1.2.3 & Et en la uoluntad es la iustiçia \textbf{ para dar a cada vno lo suyo . } Et en el appetito enssannadores la fortaleza & et sex in concupiscibili . \textbf{ In intellectu est prudentia . In voluntate iustitia . } In irascibili est Fortitudo , \\\hline
1.2.3 & Et en el appetito enssannadores la fortaleza \textbf{ para acometer grandes cosas . } Et la massedunbre & In irascibili est Fortitudo , \textbf{ Mansuetudo , } Magnanimitas , \\\hline
1.2.3 & para seer mansso e mesurado . \textbf{ Et magnanimidat por auer alto coraçon } Et la magnificençia & Mansuetudo , \textbf{ Magnanimitas , } et Magnificentia , \\\hline
1.2.3 & Et la magnificençia \textbf{ para fazer grandezas e grandes cosas . Las quales uirtudes se pueden tomar assi . } Ca la fortaleza e la manssedunbre son çerca delas passiones del coraçon & et Magnificentia , \textbf{ quae sic accipiuntur : } quia Fortitudo , \\\hline
1.2.3 & que nasçen de los males futuros \textbf{ que han de uenir . } ¶ La manssedunbre cerca delas passiones & quae sic accipiuntur : \textbf{ quia Fortitudo , } et Mansuetudo sunt circa passiones ortas ex malis , \\\hline
1.2.3 & que son grandeza e alteza de coraçon \textbf{ son çerca de los bienes guaues e fuertes de alcançar . } Enpero de departidas maneras . & ut Fortitudo est circa passiones ortas ex malis futuris , Mansuetudo circa passiones ortas ex malis praesentibus . Magnificentia vero , \textbf{ et Magnanimitas sunt circa bona ardua , aliter et aliter : } quia Magnificentia est circa magna bona utilia , \\\hline
1.2.3 & Ca la magnifiçençia es çerca de los bienes grandes e prouechosos \textbf{ assi commo en fazer grandes espenssas . } En la qual cosa se muestra omne por magnifico e granado . & quia Magnificentia est circa magna bona utilia , \textbf{ ut circa magnos sumptus : } Magnanimitas vero circa magna bona honesta , \\\hline
1.2.3 & assi commo çerca de gran deshonrras . \textbf{ Ca el que quiere alcançar grandes honrras es magnanimo et de alto coraçon ¶ } Mas en el appetito cobdiçiador son seys uirtudes & Magnanimitas vero circa magna bona honesta , \textbf{ ut circa magnos honores . In concupiscibili autem sunt sex virtutes , } videlicet , Temperantia , Liberalitas , \\\hline
1.2.3 & que es buen a conpania . \textbf{ Las quales seys uirtudes se pueden assi tomar . } Ca las tres dellas & et Eutrapelia : \textbf{ quae sic accipiuntur , } quia tres harum , \\\hline
1.2.3 & en el omne en conparaçion \textbf{ de los otros entrs maneras se pueden entender . } O en quanto siruenanos & quaedam honesta , \textbf{ circa quae est honoris amatiua . Sic etiam bona in ordine ad alium tripliciter possunt considerari : } vel ut deseruiunt nobis ad manifestationem , \\\hline
1.2.3 & O en quanto siruenanos \textbf{ para manifestar } e para demostrar lo que queremos . & circa quae est honoris amatiua . Sic etiam bona in ordine ad alium tripliciter possunt considerari : \textbf{ vel ut deseruiunt nobis ad manifestationem , } et sic est veritas : \\\hline
1.2.3 & para manifestar \textbf{ e para demostrar lo que queremos . } e assi es la uerdat & vel ut deseruiunt nobis ad manifestationem , \textbf{ et sic est veritas : } vel ad vitam , \\\hline
1.2.3 & Et assi es afabilidat o familiaridat \textbf{ que es en bien fablar e en bien beuir . } O los bienes nos siruen & vel ad vitam , \textbf{ et sic est affabilitas : } vel ad ludum , \\\hline
1.2.3 & O los bienes nos siruen \textbf{ para nos mostrar } por de buen solas & et sic est affabilitas : \textbf{ vel ad ludum , } et sic est Eutrapelia . Ostensum est ergo , \\\hline
1.2.4 & assi commo la fortaleza e la manssedunbre et la magnifiçençia \textbf{ que es uirtud para fazer grandescosas . } Et la magnanimidat & et Magnanimitas , \textbf{ et sex in concupiscibili , } ut Temperantia , Liberalitas , \\\hline
1.2.4 & que ya dixiemos . \textbf{ propusiemos de mostrar } que fablando delans buenas disposiconnes del alma & Ne ergo aliquis crederet non esse aliquas alias bonas dispositiones praeter uirtutes enumeratas : decreuimus ostendere , \textbf{ quod bonarum dispositionum } ( loquendo de bonis dispositionibus , de quibus locuti sunt Philosophi , \\\hline
1.2.4 & delas quales fablaron los philosofos . \textbf{ Ca delas otras non entendemos aqui fablar . } dezimos que delas buenas disposiconnes algunas son uirtudes & ( loquendo de bonis dispositionibus , de quibus locuti sunt Philosophi , \textbf{ quia de aliis ad praesens non intendimus tractatum constituere ) quaedam sunt uirtutes , } quaedam ancillantes uirtuti , \\\hline
1.2.4 & Et algunas sobre las uirtudes \textbf{ Ca largamente tomando la uirtud todas estas buenas disposiconnes se pueden llamar uirtudes . } Enpero algunas destas buenas disposiconnes siruen alas uirtudes . & Large enim accipiendo uirtutem , \textbf{ omnes huiusmodi bonae dispositiones , | uirtutes appellari possunt . } Nihilominus tamen quaedam bonae dispositiones ancillantur uirtuti , \\\hline
1.2.4 & Enpero algunas destas buenas disposiconnes siruen alas uirtudes . \textbf{ assi commo bien consseiar } e bien iudgar delo consseiado siruen ala pradençia e ala sabiduria ¶ & Nihilominus tamen quaedam bonae dispositiones ancillantur uirtuti , \textbf{ ut bene consiliari , } et bene iudicare de consiliatis , \\\hline
1.2.4 & assi commo bien consseiar \textbf{ e bien iudgar delo consseiado siruen ala pradençia e ala sabiduria ¶ } Pues que assi es aquel & ut bene consiliari , \textbf{ et bene iudicare de consiliatis , | ancillantur Prudentiae : } qui ergo bene dispositus est ad hoc , \\\hline
1.2.4 & qui es bien despuesto \textbf{ para bien consseiar } e bie iudgar es apto e ydoneo para seer sabio . & qui ergo bene dispositus est ad hoc , \textbf{ aptus est } ut sit prudens . Quaedam uero bonae dispositiones non sunt completa virtus , \\\hline
1.2.4 & para bien consseiar \textbf{ e bie iudgar es apto e ydoneo para seer sabio . } Mas algunas delas buenas disposiconnes & qui ergo bene dispositus est ad hoc , \textbf{ aptus est } ut sit prudens . Quaedam uero bonae dispositiones non sunt completa virtus , \\\hline
1.2.4 & mas son disposiconnes para uirtudes . \textbf{ Ca aquel es dicho perseuerar } que ahun que sea tentado non cahe . & sed dispositio ad virtutem . \textbf{ Nam ille perseuerare dicitur , } qui non cadit , \\\hline
1.2.4 & Mas esta uirtud diuinal \textbf{ que es en alguna manera mas que uirtud deuen auer mayormente los Reyes e los prinçipes . } Ca assi commo dicho es deuen seer diuinales & Huiusmodi autem uirtutem diuinam , \textbf{ quae est quodammodo super virtus , maxime habere debent Reges | et Principes , } qui \\\hline
1.2.4 & en qual manera los Reyes e los prinçipes . \textbf{ han de auer uirtudes e seer uirtuosos . } Et ahun determinaremos delas otras disposiconnes & ostendentes , \textbf{ quomodo Reges et Principes debent habere uirtutes . Determinabimus } etiam de adminiculantibus uirtuti , \\\hline
1.2.5 & Ca es costunbrado entre los santos \textbf{ e ahun enre los philosofos de fazer departimiento entre las uirtudes . } Por que alguas son prinçipales e cardinales . & nec sunt aeque principales . Consueuit enim apud Sanctos , \textbf{ et etiam apud Philosophos distingui inter virtutes : } quia quaedam sunt Cardinales et principales , \\\hline
1.2.5 & Mas que estas quatro uirtudes sean prinçipales \textbf{ e cardinales podemos lo mostrar } por tres razones ¶ & et Temperantia . Annexae autem et non principales sunt aliae octo , \textbf{ de quibus supra fecimus mentionem . Has autem quatuor virtutes esse Cardinales et principales , triplici via inuestigare possumus . Prima via sumitur ex parte materiae , circa quam versantur . Secunda ex parte subiecti , } in quo existunt . \\\hline
1.2.5 & La primera razon se toma de parte dela materia \textbf{ en la qual han de obrar } ¶La segunda de parte del subiecto & de quibus supra fecimus mentionem . Has autem quatuor virtutes esse Cardinales et principales , triplici via inuestigare possumus . Prima via sumitur ex parte materiae , circa quam versantur . Secunda ex parte subiecti , \textbf{ in quo existunt . } Tertia ex parte conditionum , \\\hline
1.2.5 & que son menester para las uirtudes . \textbf{ ¶ La primera razon se puede declarar assi . } Ca todas las uirtudes o son & quae ad virtutes requiruntur . \textbf{ Prima via sic patet : } nam omnis virtus , vel est circa rationes , \\\hline
1.2.5 & Ca todas las uirtudes o son \textbf{ para mesurar las razones } o para tenprar las passiones & Prima via sic patet : \textbf{ nam omnis virtus , vel est circa rationes , } vel circa passiones , \\\hline
1.2.5 & para mesurar las razones \textbf{ o para tenprar las passiones } o para reglar las obras ¶ & nam omnis virtus , vel est circa rationes , \textbf{ vel circa passiones , } vel circa operationes . \\\hline
1.2.5 & o para tenprar las passiones \textbf{ o para reglar las obras ¶ } Ca commo contesca de razonar derechamente & vel circa passiones , \textbf{ vel circa operationes . } Cum enim contingat ratiocinari recte et non recte , \\\hline
1.2.5 & o para reglar las obras ¶ \textbf{ Ca commo contesca de razonar derechamente } e non derechamente conuiene de dar alguna uirtud & vel circa operationes . \textbf{ Cum enim contingat ratiocinari recte et non recte , } oportet dare virtutem aliquam , \\\hline
1.2.5 & Ca commo contesca de razonar derechamente \textbf{ e non derechamente conuiene de dar alguna uirtud } que sea razon derecha . & Cum enim contingat ratiocinari recte et non recte , \textbf{ oportet dare virtutem aliquam , } quae sit recta ratio , \\\hline
1.2.5 & que fazemos fagamos razones derechas ¶ \textbf{ Otrosi commo contesca de obrar derechamente } e non derechamente & per quam de ipsis agibilibus rectas rationes faciamus . \textbf{ Rursus cum contingat operari recte et non recte , } sic ut est dare virtutem , \\\hline
1.2.5 & e non derechamente \textbf{ assi commo auemos a dar uirtud . } Por la qual cosa somos endereçados & Rursus cum contingat operari recte et non recte , \textbf{ sic ut est dare virtutem , } per quam dirigimur in ratiocinando de agibilibus : \\\hline
1.2.5 & en razonando delans obras \textbf{ en essa mis ma guas a auemos de dar uirtud } por la qual seamos endereçados en obrando las nr̃as obras & per quam dirigimur in ratiocinando de agibilibus : \textbf{ sic est dare virtutem , } per quam dirigimur in operanda ipsa agibilia . \\\hline
1.2.5 & e non derecha mente . \textbf{ Conuiene nos de dar uirtudes algunas } por las quales seamos tenprados e reglados en aquellas passiones ¶ & et non recte , \textbf{ oportet dare virtutes aliquas , } per quas modificentur in ipsis passionibus . \\\hline
1.2.5 & por las quales somos muy prestos e inclinados \textbf{ para fazer aquells males que cobdiçiamos } Et algunas passiones son & ut passiones concupiscibiles , \textbf{ quia proni sumus } ad agendum illa : quaedam vero retrahunt nos a bono , \\\hline
1.2.5 & assi commo son las passiones dela saña . \textbf{ Conuiene dar alguna uirtud en las passiones } por la qual las passiones non nos pueden mouer & ut passiones irascibiles : \textbf{ circa passiones oportet dare virtutem aliquam , } ne passiones nos impellant ad id quod ratio vetat : \\\hline
1.2.5 & Conuiene dar alguna uirtud en las passiones \textbf{ por la qual las passiones non nos pueden mouer } nin inclinar a aquelo que uieda la razon ¶ & circa passiones oportet dare virtutem aliquam , \textbf{ ne passiones nos impellant ad id quod ratio vetat : } et oportet dare virtutem aliam , \\\hline
1.2.5 & por la qual las passiones non nos pueden mouer \textbf{ nin inclinar a aquelo que uieda la razon ¶ } Et otrosi nos conuiene de dar otra uirtud & circa passiones oportet dare virtutem aliquam , \textbf{ ne passiones nos impellant ad id quod ratio vetat : } et oportet dare virtutem aliam , \\\hline
1.2.5 & nin inclinar a aquelo que uieda la razon ¶ \textbf{ Et otrosi nos conuiene de dar otra uirtud } por la qual las passiones non nos pueden arredrar nin tirar de aquello que manda la razon & ne passiones nos impellant ad id quod ratio vetat : \textbf{ et oportet dare virtutem aliam , } ne passiones retrahant nos ab eo , quod ratio dictat . \\\hline
1.2.5 & Et otrosi nos conuiene de dar otra uirtud \textbf{ por la qual las passiones non nos pueden arredrar nin tirar de aquello que manda la razon } Et pues que assi es toda uirtud o endereça las razones & et oportet dare virtutem aliam , \textbf{ ne passiones retrahant nos ab eo , quod ratio dictat . } Omnis ergo virtus , vel rectificat rationes , \\\hline
1.2.5 & o tienpra las passiones \textbf{ que nos non puedan mouer } nin inclinar a aquello que uieda la razon & vel modificat passiones , \textbf{ ne nos impellant ad id quod ratio vetat : } vel modificat eas , \\\hline
1.2.5 & que nos non puedan mouer \textbf{ nin inclinar a aquello que uieda la razon } o tienpra las passiones & vel modificat passiones , \textbf{ ne nos impellant ad id quod ratio vetat : } vel modificat eas , \\\hline
1.2.5 & Et la fortaleza prinçipalmente tienpra las passiones \textbf{ por que nos non puedan arredrar } daquello que la razon manda¶ & Fortitudo principaliter modificat eas \textbf{ ne nos retrahant } ab eo quod ratio dictat : \\\hline
1.2.5 & Ca son cerca tal materia cerca la qual prinçipal mente se trabaia toda la uida humanal . \textbf{ La segunda razon para mostrar que estas uirtudes quatro son cardinales } e prinçipales se puede tomar de parte . & et principales ; \textbf{ quia sunt circa materiam illam , circa quam principaliter versatur humana vita . Secunda via ad inuestigandum has esse virtutes cardinales et principales , } sumi potest ex parte subiecti , \\\hline
1.2.5 & La segunda razon para mostrar que estas uirtudes quatro son cardinales \textbf{ e prinçipales se puede tomar de parte . } del subiecto & quia sunt circa materiam illam , circa quam principaliter versatur humana vita . Secunda via ad inuestigandum has esse virtutes cardinales et principales , \textbf{ sumi potest ex parte subiecti , } in quo existunt . Dictum est enim supra , \\\hline
1.2.5 & de que aqui fablamos son en los poderios del alma . \textbf{ Conuiene a saber o en el entendimiento . } o en la uoluntad o en el appe tito enssannador . & esse in quatuor potentiis animae , \textbf{ videlicet , | in intellectu , } in voluntate , \\\hline
1.2.5 & Et en el apetito cobdiçian dor la mas prinçipal sea virtud la tenprança ¶ \textbf{ por ende estas quatro uirtudesconuiene a saber . Orundençia . ustiçia¶ fortaleza . } Et t̃ pranço son dichas prinçipales e cardinales en conparacion delas otros ¶ & in irascibili vero principalior sit Fortitudo , \textbf{ et in concupiscibili Temperantia : ideo haec quatuor virtutes , scilicet Prudentia , Iustitia , Fortitudo , et Temperantia , principales et cardinales esse dicuntur . } Tertia via sumitur \\\hline
1.2.5 & Et ahun en essa misma manera la magnifiçençia \textbf{ que es uirtud para fazer grandes cosas ha alguna prinçipalidat } por la grandeza delas cosas espenssas & ø \\\hline
1.2.5 & la qual cosa \textbf{ en qual manera ha de seer mostrar } lo hemos mas claramente adelante & et cardinales esse dicuntur : \textbf{ quod quomodo sit , } in prosequendo de eis singulariter plenius ostendetur . \\\hline
1.2.5 & e caddinales en conparaçonn delas otras . \textbf{ primeramente auemos dellas de dezir } que delas otras & et cardinales respectu aliarum , \textbf{ primo de his quatuor est dicendum . } Rursus \\\hline
1.2.5 & por ende esta es la orden \textbf{ que deuemos tener . } Ca primeramente diremos & quam ipsa temperantia : \textbf{ ideo hic ordo est tenendus . } Primo dicemus quid est prudentia , \\\hline
1.2.6 & la pradençia e la sabiduria \textbf{ de que primeramente auemos de fablaͬ puede se conparar a muchas cosas e departidas . } ¶ Et en quanto se conpara a departidas cosas ha departidas declara con nes . & Prudentia autem , \textbf{ de qua primo tractare intendimus , | ad diuersa comparari habet , } et prout ad diuersa comparatur , \\\hline
1.2.6 & Ca en quanto parte nesçe alo presente . \textbf{ la pradençia e la sabiduria a çinco cosas se puede conparar } ¶ & Quantum enim ad praesens spectat , \textbf{ prudentiam ad quinque comparari possumus ; } videlicet , \\\hline
1.2.6 & ¶ \textbf{ Lo primero se pue de conparar alas uirtudes morales } las quales ha de endereçar & videlicet , \textbf{ ad virtutes morales , } quarum est directiua : \\\hline
1.2.6 & Lo primero se pue de conparar alas uirtudes morales \textbf{ las quales ha de endereçar } ¶L segundo alas uirtu des intellectuales & ad virtutes morales , \textbf{ quarum est directiua : } ad virtutes intellectuales , \\\hline
1.2.6 & Lo terçero ala materia \textbf{ en la qual ha de obrar } ¶L quarto alascina ¶ & ad materiam , \textbf{ circa quam versatur : } ad scientiam , et ad artem , \\\hline
1.2.6 & assi commo la tenprança inclina al omne a mesura e a ser mesurado . \textbf{ Et a contradezir alas cosas } que son de luxuria & nam virtutes morales de se inclinant in finem sibi conuenientem , \textbf{ ut Temperantia inclinat in sobrietatem , } et in detestationem venereorum : \\\hline
1.2.6 & o a fin conuenible delas otras uirtudes morales \textbf{ si non sopiere en qual manera el puede alcançar tal fin . } e esto ha de saber & vel in finem aliarum virtutum moralium , \textbf{ nisi sciamus , | quomodo possumus consequi talem finem , } quod fit per prudentiam . Per virtutes ergo morales praestituimus nobis debitos fines : \\\hline
1.2.6 & si non sopiere en qual manera el puede alcançar tal fin . \textbf{ e esto ha de saber } por la pradençia ¶ Et pues que assi es por las uirtudes morales & ø \\\hline
1.2.6 & Mas por la pradençia somos reglados \textbf{ en qual manera pondemos alcançar aquellas fines . } Mas la pradençia toma aquellas fines delas uirtudes morałs & sed per prudentiam \textbf{ ( cum habemus ) dirigimur in fines illos . } Accipiat autem prudentia fines illos a virtutibus moralibus : \\\hline
1.2.6 & Et por aquellas cosas que son ordenadas a aquellos fines la pradençia regla derechamente a omne \textbf{ para alcançar aquellos fines . } Por la qual cosa la pradençia & et per ea quae sunt ad finem , \textbf{ recte dirigit in fines illos . } Quare prudentia , \\\hline
1.2.6 & en quanto es conparanda alas uirtudes morales \textbf{ puede se declarar } assi diziendo . & Quare prudentia , \textbf{ ut comparatur ad virtutes morales , sic diffiniri potest , } quod est perfectio intellectus , \\\hline
1.2.6 & ¶ \textbf{ lo segundo la pradençia se puede conparar alas uirtudes intellectuales } en conparaçon delas quales es señora e mandadora . & siue quod est bona qualitas mentis , directiua in finem virtutum moralium . \textbf{ Secundo potest comparari Prudentia ad virtutes intellectuales , } respectu quorum est praeceptiua : \\\hline
1.2.6 & Por que en quanto parte nesçe alo presente \textbf{ si quisieremos bien obrar en aquellas cosas } que fazemos tres cosas deuemos auer ¶ & prout enim ad praesens spectat , \textbf{ si circa agibilia bene negociari volumus , } tria habere debemus . Primo debemus diuersas vias inuenire . \\\hline
1.2.6 & si quisieremos bien obrar en aquellas cosas \textbf{ que fazemos tres cosas deuemos auer ¶ } Lo primero deuemos buscar muchas carreras e departidas ¶ & si circa agibilia bene negociari volumus , \textbf{ tria habere debemus . Primo debemus diuersas vias inuenire . } Secundo de inuentis debemus iudicare . \\\hline
1.2.6 & que fazemos tres cosas deuemos auer ¶ \textbf{ Lo primero deuemos buscar muchas carreras e departidas ¶ } Lo segundo deuemos bien iudgar delas carreras falladas & si circa agibilia bene negociari volumus , \textbf{ tria habere debemus . Primo debemus diuersas vias inuenire . } Secundo de inuentis debemus iudicare . \\\hline
1.2.6 & Lo primero deuemos buscar muchas carreras e departidas ¶ \textbf{ Lo segundo deuemos bien iudgar delas carreras falladas } ¶ lo terçero deuemos mandar & tria habere debemus . Primo debemus diuersas vias inuenire . \textbf{ Secundo de inuentis debemus iudicare . } Tertio debemus praecipere \\\hline
1.2.6 & Lo segundo deuemos bien iudgar delas carreras falladas \textbf{ ¶ lo terçero deuemos mandar } que se fagan las obras & Secundo de inuentis debemus iudicare . \textbf{ Tertio debemus praecipere } ut fiant opera \\\hline
1.2.6 & segunt las carreras falladas e iudgadas . Verbigera . ¶ \textbf{ Si nos quesieremos tomar algun castiello } segunt manera de guerra . Pmeramente deuemos buscar carreras & et iudicata : \textbf{ ut si vellemus } secundum opera bellica castrum aliquod capere . \\\hline
1.2.6 & Si nos quesieremos tomar algun castiello \textbf{ segunt manera de guerra . Pmeramente deuemos buscar carreras } e escodrinnar maneras e sotilezas & ut si vellemus \textbf{ secundum opera bellica castrum aliquod capere . | Primo inueniendae essent viae , } et cogitandi essent modi , \\\hline
1.2.6 & segunt manera de guerra . Pmeramente deuemos buscar carreras \textbf{ e escodrinnar maneras e sotilezas } por que podamos tomar aquel castiello ¶ & Primo inueniendae essent viae , \textbf{ et cogitandi essent modi , | per } quos castrum istud capi posset . Secundo iudicandum esset de viis inuentis , utrum bonae essent \\\hline
1.2.6 & e escodrinnar maneras e sotilezas \textbf{ por que podamos tomar aquel castiello ¶ } Lo segundo deuemos iudgar de aquellas carreras e maneras & per \textbf{ quos castrum istud capi posset . Secundo iudicandum esset de viis inuentis , utrum bonae essent } ad propositum prosequendum . \\\hline
1.2.6 & por que podamos tomar aquel castiello ¶ \textbf{ Lo segundo deuemos iudgar de aquellas carreras e maneras } que fablamos & per \textbf{ quos castrum istud capi posset . Secundo iudicandum esset de viis inuentis , utrum bonae essent } ad propositum prosequendum . \\\hline
1.2.6 & que fablamos \textbf{ si son buenas para alcançar aquello que queremos . } ¶ lo terçero deuemos mandar & quos castrum istud capi posset . Secundo iudicandum esset de viis inuentis , utrum bonae essent \textbf{ ad propositum prosequendum . } Tertio praecipiendum esset \\\hline
1.2.6 & si son buenas para alcançar aquello que queremos . \textbf{ ¶ lo terçero deuemos mandar } que se fagan las obras & ad propositum prosequendum . \textbf{ Tertio praecipiendum esset } ut fierent opera \\\hline
1.2.6 & por la qual bien busquemos \textbf{ lo que auemos de buscar } e bien nos aconseiemos & In intellectu ergo nostro debent esse tres virtutes . \textbf{ Una per quam bene inueniamus } et bene confiliemur , \\\hline
1.2.6 & en el sexto libro delas ethicas . \textbf{ Eubullia que quiere dezer uirtud para bien coseiar . } la otra es & quam Philosophus Ethic’ 6 appellat eubuliam , \textbf{ idest bene consiliatiua . } Alia vero per quam bene iudicamus de inuentis , \\\hline
1.2.6 & la qual llama el philosofo sinesis . \textbf{ que quiere dezir uirtud de bien iudgar ¶ } la terçera es uirtud & quam Philosophus appellat synesin , \textbf{ idest bene iudicatiuam . Tertia , } per quam praecipiamus \\\hline
1.2.6 & Et pues que assi es la pradençia mas derechamente regla las obras \textbf{ que se han de fazer } por que las manda luego fazer & et iudicant , ista praecipit ut fiat . Prudentia ergo immediatus dirigit in opera fienda , \textbf{ eo quod praecipiat illa fieri , } quam faciat virtus inuentiua et iudicatiua . \\\hline
1.2.6 & que se han de fazer \textbf{ por que las manda luego fazer } que la uirtud buscadora e falladora . & et iudicant , ista praecipit ut fiat . Prudentia ergo immediatus dirigit in opera fienda , \textbf{ eo quod praecipiat illa fieri , } quam faciat virtus inuentiua et iudicatiua . \\\hline
1.2.6 & que son uirtudes intellectuales \textbf{ puedese asi difinir } e declarar diziendo . & quae sunt virtutes intellectuales , \textbf{ sic diffiniri potest , } quod est praeceptiua inuentorum et iudicatorum . \\\hline
1.2.6 & puedese asi difinir \textbf{ e declarar diziendo . } que la pradençia es mandadora delas cosas falladas e iudgadas ¶ & quae sunt virtutes intellectuales , \textbf{ sic diffiniri potest , } quod est praeceptiua inuentorum et iudicatorum . \\\hline
1.2.6 & que assi commo . \textbf{ eubullia es uirtud para fallar . } Et sinesis es uirtud para iudgar & 6 dicitur , \textbf{ quod sicut eubuliae est inuenire , } et synesis est iudicare : \\\hline
1.2.6 & eubullia es uirtud para fallar . \textbf{ Et sinesis es uirtud para iudgar } assi la pradençia es uirtud para mandar . & quod sicut eubuliae est inuenire , \textbf{ et synesis est iudicare : } sic prudentiae est praecipere . Tertio , Prudentia comparari potest ad materiam , \\\hline
1.2.6 & Et sinesis es uirtud para iudgar \textbf{ assi la pradençia es uirtud para mandar . } Lo terçero la pradençia se pue de conparar ala materia & et synesis est iudicare : \textbf{ sic prudentiae est praecipere . Tertio , Prudentia comparari potest ad materiam , } circa quam versatur . \\\hline
1.2.6 & assi la pradençia es uirtud para mandar . \textbf{ Lo terçero la pradençia se pue de conparar ala materia } en que obra . & et synesis est iudicare : \textbf{ sic prudentiae est praecipere . Tertio , Prudentia comparari potest ad materiam , } circa quam versatur . \\\hline
1.2.6 & pues que assi es conparando la pradençia ala materia \textbf{ en que obra puede se assi difinir } e declarar diziendo que la pradençia es uirtud & 6 declarari habet . Comparatiua ergo prudentia ad materiam circa quam versatur , \textbf{ sic describi potest , } quod prudentia est virtus \\\hline
1.2.6 & en que obra puede se assi difinir \textbf{ e declarar diziendo que la pradençia es uirtud } que iudga alos negoçios particulares & sic describi potest , \textbf{ quod prudentia est virtus } secundum uniuersales maximas particularia facta concernens . \\\hline
1.2.6 & Et otras cosas tales \textbf{ por las quales nos podemos reglar e ordenar en todas las cosas } que auemos de fazer . & debitae consuetudines , \textbf{ et alia per quae regulari possumus in agendis . } Quarto comparari habet prudentia \\\hline
1.2.6 & por las quales nos podemos reglar e ordenar en todas las cosas \textbf{ que auemos de fazer . } lo quarto puede se conparar la pradençia & debitae consuetudines , \textbf{ et alia per quae regulari possumus in agendis . } Quarto comparari habet prudentia \\\hline
1.2.6 & que auemos de fazer . \textbf{ lo quarto puede se conparar la pradençia } ala sçiençia & et alia per quae regulari possumus in agendis . \textbf{ Quarto comparari habet prudentia } ad ipsam scientiam , a qua distinguitur : \\\hline
1.2.6 & que han sustançia non ꝑ mudable \textbf{ que seño puede mudar . } Mas la pradençia es delas obras de los omes & nam scientia proprie , \textbf{ est de rebus necessariis , } iuxta Boetium primo Arithmeticae , \\\hline
1.2.6 & Mas la pradençia es delas obras de los omes \textbf{ e delas cosas que pue den contesçer } que son en nuestro senñorio de las fazer & iuxta Boetium primo Arithmeticae , \textbf{ Scientia est eorum quae immutabilem substantiam sortiuntur : } Prudentia autem est actuum humanorum , \\\hline
1.2.6 & e delas cosas que pue den contesçer \textbf{ que son en nuestro senñorio de las fazer } o non las fazer . & iuxta Boetium primo Arithmeticae , \textbf{ Scientia est eorum quae immutabilem substantiam sortiuntur : } Prudentia autem est actuum humanorum , \\\hline
1.2.6 & que son en nuestro senñorio de las fazer \textbf{ o non las fazer . } Et aquella sabiduria es dicha pradençia & ø \\\hline
1.2.6 & pue dese \textbf{ assi difinir e declarar . } que la prudençia es dicha razon delas cosas & Prudentia autem est actuum humanorum , \textbf{ et rerum contingentium , } quae sunt in potestate nostra . \\\hline
1.2.6 & que la prudençia es dicha razon delas cosas \textbf{ que pue den contesçer } e son en nuestro poderio ¶lo quanto puede se conparar la pradençia & et rerum contingentium , \textbf{ quae sunt in potestate nostra . } Quinto comparari potest prudentia ad artem , \\\hline
1.2.6 & que pue den contesçer \textbf{ e son en nuestro poderio ¶lo quanto puede se conparar la pradençia } ala arte & quae sunt in potestate nostra . \textbf{ Quinto comparari potest prudentia ad artem , } a qua etiam distingui habet . \\\hline
1.2.6 & porque el arte es en conpara con delas cosas \textbf{ que se pueden fazer } e non requiere reglamiento derecho dela uoluntad . & a qua etiam distingui habet . \textbf{ Nam ars est respectu factibilium , } et non praesupponit rectitudinem voluntatis : \\\hline
1.2.6 & mas la pradençia es en conparaçion delas cosas \textbf{ que se han de fazer } e de obranr . & et non praesupponit rectitudinem voluntatis : \textbf{ Prudentia vero est respectu agibilium , } et praesupponit rectitudinem appetitus : \\\hline
1.2.6 & que se han de fazer \textbf{ e de obranr . } Et requiere e demanda reglamiento derecho dela uoluntad . & Prudentia vero est respectu agibilium , \textbf{ et praesupponit rectitudinem appetitus : } propter quod scribitur Ethic’ \\\hline
1.2.6 & Et pues que assi es en las obras mecanicas \textbf{ que . siruen al arte meior es pecar de uoluntad } que sin uoluntad . & ø \\\hline
1.2.6 & que es uoluntad reglada por razon . \textbf{ Peor es pecar de uoluntad } que sin uoluntad . & circa quae versatur prudentia , \textbf{ peius est peccare voluntarie , } quam inuoluntarie . \\\hline
1.2.6 & assi commo dicho es puede se \textbf{ assi difinir } e declarar diziendo . & Prout ergo prudentia comparatur ad artem a qua distinguitur , \textbf{ sic diffiniri potest , } quam est recta ratio agibilium , \\\hline
1.2.6 & assi difinir \textbf{ e declarar diziendo . } que la pradençia es razon derecha de todas las obras & sic diffiniri potest , \textbf{ quam est recta ratio agibilium , } praesupponens rectitudinem voluntatis . \\\hline
1.2.6 & que la pradençia es razon derecha de todas las obras \textbf{ que auemos de fazer } que requiere e demanda reglamiento de uoluntad ¶ & quam est recta ratio agibilium , \textbf{ praesupponens rectitudinem voluntatis . } Ex omnibus ergo his , \\\hline
1.2.6 & que requiere e demanda reglamiento de uoluntad ¶ \textbf{ Et pues que assi es de todas estas cosas sobredichas podemos tomar vna difiniçonn } o declaraçion comun dela pradençia . & praesupponens rectitudinem voluntatis . \textbf{ Ex omnibus ergo his , } de ipsa prudentia unam communem descriptionem formare possumus , dicendo , \\\hline
1.2.6 & que cata sienpre las obras particulares \textbf{ que pue den contesçer segunt las reglas uniuerssales } e que demanda endereçamiento e reglamiento de uoluntad & secundum uniuersales maximas , \textbf{ particularia contingentia agibilia concernens , } praesupponens rectitudinem voluntatis . \\\hline
1.2.7 & a la qual nos inclinan las uirtudes morales . \textbf{ finca de demostrar } que conuiene alos Reyes & in quem inclinant virtutes morales : \textbf{ restat ostendere , } quod decet Reges , \\\hline
1.2.7 & Mas quanto pertenesçe alo presente tres cosas son \textbf{ a que muchon deue tener mientes el Rey } ¶la primera parte & Sunt autem \textbf{ ( quantum ad praesens spectat ) tria quae maxime Rex attendere debet . } Primo enim spectat ad ipsum summe intendere , \\\hline
1.2.7 & ¶la primera parte \textbf{ nesçe al Rey de tener . mucho mientes } que sea Rey en uerdat & ( quantum ad praesens spectat ) tria quae maxime Rex attendere debet . \textbf{ Primo enim spectat ad ipsum summe intendere , } ut sit Rex \\\hline
1.2.7 & e non tan solamente por nonbre ¶ \textbf{ Lo segundo deue estudiar el Rey } que el su prinçipadgo & ut sit Rex \textbf{ secundum rei veritatem , } et non nomine tantum . Secundo studere debet , \\\hline
1.2.7 & que es señorio malo e desigual \textbf{ ¶Lo terçero deue estudiar } que en sennor ee natural . mente . & et non nomine tantum . Secundo studere debet , \textbf{ ne suus principatus in tyrannidem conuertatur . Tertio studere debet , } ut naturaliter dominetur . Triplici ergo via \\\hline
1.2.7 & ¶ Et pues \textbf{ que assi es podemos prouar en tres maneras } que conuiene al rey de seer sabio ¶ & ne suus principatus in tyrannidem conuertatur . Tertio studere debet , \textbf{ ut naturaliter dominetur . Triplici ergo via } inuestigare possumus , quod decet Regem esse prudentem . Primo , quia sine prudentia non est Rex \\\hline
1.2.7 & Ca el nonbre del Rey es tomado de gouernamiento . \textbf{ Mas gouernar alos otros } e guiar los en su fin conuenible . & nomen enim regum a regendo sumptum est : \textbf{ regere autem alios , } et dirigere ipsos in finem debitum , sit per prudentiam . \\\hline
1.2.7 & Mas gouernar alos otros \textbf{ e guiar los en su fin conuenible . } Esto ha de ser por la pradençia ¶ & regere autem alios , \textbf{ et dirigere ipsos in finem debitum , sit per prudentiam . } Unde dicitur Ethic’ \\\hline
1.2.7 & que aquellos tenemos por sabios \textbf{ que assi e alos otros pueden catar } e pueer buenas cosas . & 6 quod illos extimamus esse prudentes , \textbf{ qui sibi et aliis possunt bona speculari } et prouidere . \\\hline
1.2.7 & que assi e alos otros pueden catar \textbf{ e pueer buenas cosas . } Pues que assi es la pradençia & qui sibi et aliis possunt bona speculari \textbf{ et prouidere . } Prudentia ergo est quidam oculus , \\\hline
1.2.7 & Et el que non ha este oio \textbf{ non puede conplidamente ueer el bien } nin la su fin conuenible & Qui ergo hoc oculo caret , \textbf{ non sufficienter videre potest ipsum bonum , } nec ipsum debitum finem , \\\hline
1.2.7 & nin la su fin conuenible \textbf{ ala qual es de guiar el pueblo . } Ca bien assi commo el que lança la saet a non puede conplidamente & nec ipsum debitum finem , \textbf{ in quem est populus dirigendus . } Sicut ergo sagittator non posset sufficienter sagittare , \\\hline
1.2.7 & Ca bien assi commo el que lança la saet a non puede conplidamente \textbf{ nin derechamente endereçar la saeta ala señal . } si non viere la señal & Sicut ergo sagittator non posset sufficienter sagittare , \textbf{ siue sagittam in signum dirigere , } nisi ipsum signum videret : \\\hline
1.2.7 & a que lança \textbf{ assi el Rey non puede gouernar el pueblo } nin guiar le ala su fin conuenible & nisi ipsum signum videret : \textbf{ sic nec Rex potest populum dirigere siue regere , } et ipsum in debitum finem dirigere , \\\hline
1.2.7 & assi el Rey non puede gouernar el pueblo \textbf{ nin guiar le ala su fin conuenible } si non viere o sopiere & sic nec Rex potest populum dirigere siue regere , \textbf{ et ipsum in debitum finem dirigere , } nisi ipsum finem per prudentiam speculetur . \\\hline
1.2.7 & por la pradençia aquella fina \textbf{ que ha de guiar el pueblo . } ¶ & ø \\\hline
1.2.7 & Pues que assi es \textbf{ si sin la pradençia ninguno non puede conplidamente gouernar la gente } nin guiar a su fin conuenible siguese & nisi ipsum finem per prudentiam speculetur . \textbf{ Si ergo sine prudentia nullus potest sufficienter gentem regere , } et ipsam dirigere in finem bonum et debitum , \\\hline
1.2.7 & si sin la pradençia ninguno non puede conplidamente gouernar la gente \textbf{ nin guiar a su fin conuenible siguese } que sin la prudençia niguno non puede ser rey segunt uirdat . & Si ergo sine prudentia nullus potest sufficienter gentem regere , \textbf{ et ipsam dirigere in finem bonum et debitum , } sine ea nullus erit Rex \\\hline
1.2.7 & por que el sea Rey non solamente segunt el nonbre \textbf{ mas segunt el fech̃o conuiene le de auer sabiduria . } La segunda manera & sit Rex non solum nomine sed re , \textbf{ decet ipsum habere prudentiam . } Secundo hoc decet eum , \\\hline
1.2.7 & ala qual nos inclinan las uirtudes morales . \textbf{ Ca de omne sabio es proueer buenas cosas . } assi e alos otros & quod per prudentiam dirigimur in bonum finem , \textbf{ in quem inclinant virtutes morales . Est enim prudentis , prouidere bona sibi et aliis , } et dirigere se et alios in optimum finem . \\\hline
1.2.7 & assi e alos otros \textbf{ e de guiar assi e alos otros a buena fin ¶ } pues si alguno non ouiere sabiduria & in quem inclinant virtutes morales . Est enim prudentis , prouidere bona sibi et aliis , \textbf{ et dirigere se et alios in optimum finem . } Si ergo aliquis prudentia careat , per quam dirigimur in optima bona \\\hline
1.2.7 & La terçera manera \textbf{ por que conuiene al Rey de auer sabiduria es } por que sin ella non puede ser señor & non curabit qualitercunque possit pecuniam extorquere . Tertio decet Reges , \textbf{ et Principes habere prudentiam , } quia sine ea non possunt naturaliter dominari . \\\hline
1.2.7 & por que sin ella non puede ser señor \textbf{ nin enssennorear natural mente . } Ca assi commo dize el philosofo en el primero libro delas politicas & et Principes habere prudentiam , \textbf{ quia sine ea non possunt naturaliter dominari . } Nam ( ut declarari habet 1 Polit’ ) \\\hline
1.2.7 & por que es menguado de entendimiento \textbf{ e non sabe gouernar a ssi mismo . } Et por ende es dicho alguno naturalmente señor & quia deficit intellectu , \textbf{ et nescit seipsum regere . } Ex hoc autem naturaliter est Dominus , \\\hline
1.2.7 & por que es conplido de entendimiento e de sabiduria . \textbf{ Et sabe gouernar assi } e guiar alos otros a buena fin . & quia viget intellectu et prudentia , \textbf{ et nouit se } et alios in debitum finem dirigere . Hanc enim veritatem non solum approbant physica dicta , \\\hline
1.2.7 & Et sabe gouernar assi \textbf{ e guiar alos otros a buena fin . } Ca esta uirtud de sabiduria non sola mente la alaban los dichos de los philosofos & et nouit se \textbf{ et alios in debitum finem dirigere . Hanc enim veritatem non solum approbant physica dicta , } sed etiam confirmant singula regimina naturalia . Videmus enim naturaliter homines dominari bestiis , \\\hline
1.2.7 & que sean mas sabias que los omes \textbf{ esto contesçer alamente } e pocas vezes . & quod si reperiantur mulieres aliquae prudentiores viris , \textbf{ hoc est | ut raro , } et in paucioribus ut plurimum . \\\hline
1.2.7 & Et do quier que ay sabiduria ha naturalmente sennorio . \textbf{ Por ende sienpre el prinçipe deue auer sabiduria . } por la qual sera sennor & et illud natureliter dominatur , semper principans pollet prudentia , a qua deficit qui naturaliter seruus existit . \textbf{ Ut igitur Rex naturaliter dominetur oportet } quod polleat prudentia , et intellectu . Quot , \\\hline
1.2.8 & que florescaen sabiduria e en entendimiento \textbf{ or que nunca conplidamente se pueda auer el todo } si non se ouieren todas las sus partes & si \textbf{ Quoniam nunquam perfecte habetur aliquod totum , } nisi habeantur partes eius : \\\hline
1.2.8 & si alguno ouiere aser sabio conplida mente . \textbf{ Conuienel e de auer todas aquellas cosas } que son necessarias ala sabiduria . & nisi habeantur partes eius : \textbf{ si debeat aliquis esse perfecte prudens , } oportet ipsum habere omnia quae concurrunt ad prudentiam , \\\hline
1.2.8 & que son necessarias ala sabiduria . \textbf{ Et connuiene le de auer todas las partidas de la sabiduria . } Mas suele le sennalar & si debeat aliquis esse perfecte prudens , \textbf{ oportet ipsum habere omnia quae concurrunt ad prudentiam , } et omnes partes eius . Consueuerunt autem assignari octo partes prudentiae , \\\hline
1.2.8 & Et connuiene le de auer todas las partidas de la sabiduria . \textbf{ Mas suele le sennalar } e departir ocho partes dela praderçia e dela sabiduria¶ & oportet ipsum habere omnia quae concurrunt ad prudentiam , \textbf{ et omnes partes eius . Consueuerunt autem assignari octo partes prudentiae , } videlicet , memoria , prouidentia , \\\hline
1.2.8 & Mas suele le sennalar \textbf{ e departir ocho partes dela praderçia e dela sabiduria¶ } La primera es memoria ¶ & oportet ipsum habere omnia quae concurrunt ad prudentiam , \textbf{ et omnes partes eius . Consueuerunt autem assignari octo partes prudentiae , } videlicet , memoria , prouidentia , \\\hline
1.2.8 & La septima experiençia e prueua¶ \textbf{ La viii jncauçion q quiere dezir escogimiento delo meior } e foyr delo peor . & videlicet , memoria , prouidentia , \textbf{ intellectus , ratio , solertia , docilitas , experientia , et cautio . } Quare si Rex , \\\hline
1.2.8 & La viii jncauçion q quiere dezir escogimiento delo meior \textbf{ e foyr delo peor . } ¶ por la qual razon & intellectus , ratio , solertia , docilitas , experientia , et cautio . \textbf{ Quare si Rex , } aut Princeps debeat esse prudens , \\\hline
1.2.8 & Et enssennable ¶ Espierto ¶ Et prouado ¶ \textbf{ Cauto que quiere dezir conosçedor } e escogendor delo meior . ¶ & ø \\\hline
1.2.8 & que son dichas part s̃ dela sabiduria \textbf{ assi se pueden tomar . } Ca assi commo paresçe & ø \\\hline
1.2.8 & por esso es alguon dicho sabio \textbf{ porque es suficiente para enderesçar assi e alos otros e de guiar assi e alos otros a alguons bienes } o a algunas buenas fines ¶ & ex hoc aliquis dicitur esse prudens , \textbf{ quia est sufficiens dirigere se , | et alios in aliqua bona , } siue in aliquos bonos fines . \\\hline
1.2.8 & o a algunas buenas fines ¶ \textbf{ Pues que assi es quatro cosas nos conuieney de penssar . } Conuiene de saber los bienes & siue in aliquos bonos fines . \textbf{ Quatuor ergo est ibi considerare , } videlicet , bona , \\\hline
1.2.8 & Pues que assi es quatro cosas nos conuieney de penssar . \textbf{ Conuiene de saber los bienes } aque guia la pradençia¶ & Quatuor ergo est ibi considerare , \textbf{ videlicet , bona , } ad quae dirigit : \\\hline
1.2.8 & por razon de la su propia persona \textbf{ que ha de guiar los otros . } Conuiene le de ser sotil e doctrinable ¶ & secundum quem dirigit , oportet ipsum esse intelligentem , et rationabilem : ratione propriae personae \textbf{ quae alios est dirigens , oportet quod sit solers , et docilis : } ratione vero gentis quam dirigit , \\\hline
1.2.8 & Conuiene le de ser sotil e doctrinable ¶ \textbf{ Mas por razon dela gente a quien ha de gouernar } Conuiene le de ser prouado & quae alios est dirigens , oportet quod sit solers , et docilis : \textbf{ ratione vero gentis quam dirigit , } congruit quod sit expertus et cautus . \\\hline
1.2.8 & e cauto ete sogedor de bien ¶ \textbf{ Ca si el Rey ha a guiar la su gente } e la su conpanna a alguons bienes . & congruit quod sit expertus et cautus . \textbf{ Si enim Rex debet gentem aliquam ad bonum dirigere , } oportet quod habeat memoriam praeteritorum , \\\hline
1.2.8 & Et que aya prouision delas cosas passadas \textbf{ e que han de venir } Ca deue el Rey auer memoria & oportet quod habeat memoriam praeteritorum , \textbf{ et prouidentiam futurorum . } Debet enim habere praeteritorum memoriam , \\\hline
1.2.8 & e que han de venir \textbf{ Ca deue el Rey auer memoria } e remenbrança delas cosas passadas & et prouidentiam futurorum . \textbf{ Debet enim habere praeteritorum memoriam , } non quod possit praeterita immutare , \\\hline
1.2.8 & e remenbrança delas cosas passadas \textbf{ non por que las pueda mudar . } Ca esto ninguno non lo pie de fazer . & Debet enim habere praeteritorum memoriam , \textbf{ non quod possit praeterita immutare , } quia nulli agenti hoc est possibile , \\\hline
1.2.8 & non por que las pueda mudar . \textbf{ Ca esto ninguno non lo pie de fazer . } Mas conuiene al Rey de auer memoria delans cosas passadas & non quod possit praeterita immutare , \textbf{ quia nulli agenti hoc est possibile , } sed decet Regem habere praeteritorum memoriam , \\\hline
1.2.8 & Ca esto ninguno non lo pie de fazer . \textbf{ Mas conuiene al Rey de auer memoria delans cosas passadas } por que pue da & quia nulli agenti hoc est possibile , \textbf{ sed decet Regem habere praeteritorum memoriam , } ut possit ex praeteritis cognoscere , \\\hline
1.2.8 & por que pue da \textbf{ por las cosas passadas conosçer } e tomar e ꝑcebimiento delas cosas que han de venir ¶ & sed decet Regem habere praeteritorum memoriam , \textbf{ ut possit ex praeteritis cognoscere , } quid euenire debeat in futurum . \\\hline
1.2.8 & por las cosas passadas conosçer \textbf{ e tomar e ꝑcebimiento delas cosas que han de venir ¶ } Ca assi commo dize el philosofo en el segundo libro de la retorica & ut possit ex praeteritis cognoscere , \textbf{ quid euenire debeat in futurum . } Nam ( ut scribitur secundo Rhetoricorum ) in contingentibus agibilibus , \\\hline
1.2.8 & en las obras \textbf{ que acaesçe n o pueden acaesçer } por la mayor partida las cosas que han de venir & ø \\\hline
1.2.8 & que acaesçe n o pueden acaesçer \textbf{ por la mayor partida las cosas que han de venir } son semeiantes alos cosas que son passadas & Nam ( ut scribitur secundo Rhetoricorum ) in contingentibus agibilibus , \textbf{ ut plurimum futura sunt praeteritis similia . Secundo decet ipsum habere prouidentiam futurorum : } quia homines prouidentes futura bona , \\\hline
1.2.8 & son semeiantes alos cosas que son passadas \textbf{ ¶lo segundo conuiene al Rey de auer prouisionde las cosas que han de venir . } Ca los omes & ut plurimum futura sunt praeteritis similia . Secundo decet ipsum habere prouidentiam futurorum : \textbf{ quia homines prouidentes futura bona , } excogitant vias , \\\hline
1.2.8 & Ca los omes \textbf{ que bien proueen de los bienes que han de venir pienssan carreras e maneras por las quales puedan ligeramente alcançar aquellos bienes ¶ } pues que assi es & quia homines prouidentes futura bona , \textbf{ excogitant vias , | per quas faciliter illa adipisci valeant . } Ergo ratione bonorum , \\\hline
1.2.8 & pues que assi es \textbf{ por razon de aquellos bienesa que el Rey deue guiar su gente e su conpanna . } Conuiene le & Ergo ratione bonorum , \textbf{ ad quae Rex gentem suam dirigere debet , } expedit ut habeat prouidentiam futurorum , \\\hline
1.2.8 & Conuiene le \textbf{ que aya prouision delas cosas que han de venir } por que mas ligeramente pueda alcançar aquellos bienes que han de venir . & ad quae Rex gentem suam dirigere debet , \textbf{ expedit ut habeat prouidentiam futurorum , } ut facilius illa futura bona adipisci valeat , \\\hline
1.2.8 & que aya prouision delas cosas que han de venir \textbf{ por que mas ligeramente pueda alcançar aquellos bienes que han de venir . } Et conuiene le & expedit ut habeat prouidentiam futurorum , \textbf{ ut facilius illa futura bona adipisci valeat , } et ut habeat memoriam praeteritorum , \\\hline
1.2.8 & que aya memoria delas cosas passadas \textbf{ por que delas cosas passadas sepa lo que ha de fazer } en lo que ha de venir . & et ut habeat memoriam praeteritorum , \textbf{ ut ex actis praeteritis sciat quid agere debeat in futurum . Ratione vero modi per quem dirigit , oportet quod habeat intellectum et rationem , } siue oportet quod sit intelligens \\\hline
1.2.8 & por que delas cosas passadas sepa lo que ha de fazer \textbf{ en lo que ha de venir . } Mas por razon dela manera & et ut habeat memoriam praeteritorum , \textbf{ ut ex actis praeteritis sciat quid agere debeat in futurum . Ratione vero modi per quem dirigit , oportet quod habeat intellectum et rationem , } siue oportet quod sit intelligens \\\hline
1.2.8 & Mas por razon dela manera \textbf{ por la qual deue guiar el Rey . } Conuiene le & ut ex actis praeteritis sciat quid agere debeat in futurum . Ratione vero modi per quem dirigit , oportet quod habeat intellectum et rationem , \textbf{ siue oportet quod sit intelligens } et rationale . Modus enim , \\\hline
1.2.8 & Ca assi commo se fazen razones demostratiuas \textbf{ para demostrar e conosçer } que cosa es la uerdat bien & Nam sicut fiunt rationes , \textbf{ ut demonstretur , } quid sit verum cognoscendum : sic fiunt rationes , \\\hline
1.2.8 & que cosa es la uerdat bien \textbf{ assi se fazen razones praticas para enduzir alos omes } qual es aquel bien que han de seguir ¶ & quid sit verum cognoscendum : sic fiunt rationes , \textbf{ ut persuadeatur } quid sit bonum prosequendum . Ratione igitur huiusmodi cognoscendi , \\\hline
1.2.8 & assi se fazen razones praticas para enduzir alos omes \textbf{ qual es aquel bien que han de seguir ¶ } Et pues que assi es por razon desta manera de conos çer & ut persuadeatur \textbf{ quid sit bonum prosequendum . Ratione igitur huiusmodi cognoscendi , } qui est inditus hominibus , \\\hline
1.2.8 & qual es aquel bien que han de seguir ¶ \textbf{ Et pues que assi es por razon desta manera de conos çer } que es enxerida naturalmente alos omes . & ut persuadeatur \textbf{ quid sit bonum prosequendum . Ratione igitur huiusmodi cognoscendi , } qui est inditus hominibus , \\\hline
1.2.8 & que es enxerida naturalmente alos omes . \textbf{ El que quiere alos otros guiar } conuiene le que sea entendido & quid sit bonum prosequendum . Ratione igitur huiusmodi cognoscendi , \textbf{ qui est inditus hominibus , } volens alios dirigere , oportet quod sit intelligens , cognoscendo principia , et praemissa , et rationalis , \\\hline
1.2.8 & e daquellas razones las conclusiones e las razones \textbf{ que quiere ençerrar ¶ } Et otrosi conuiene al Rey & ratiocinando , et eliciendo ex illis praemissis cunclusiones intentas . Vel oportet quod sit intelligens , \textbf{ sciendo leges , } et consuetudines bonas , \\\hline
1.2.8 & que pueden ser prinçipios \textbf{ e reglas para lo que ha de fazer ¶ } Et otrosi conuiene al Rey & ø \\\hline
1.2.8 & por aquellas reglas e por aquellos prinçipios \textbf{ que es aquello que le conuiene de fazer . } pues que assi es & speculando ex illis regulis \textbf{ quid agere congruit . } Sicut ergo ratione bonorum ad quae dirigit , \\\hline
1.2.8 & por razon de los bienes \textbf{ aque ha de guiar el Rey } su pueblo le conuiene de ser acordable e prouisor . & Sicut ergo ratione bonorum ad quae dirigit , \textbf{ oportet Regem esse memorem , } et prouidum : \\\hline
1.2.8 & Assi por razon de la manera \textbf{ por la qual ha de guiar el pueblo le conuiene de ser entendido e razonable . } Mas por razon dela su persona propia & sic ratione modi per quem dirigit , \textbf{ oportet ipsum esse intelligentem , | et rationalem . } Sed ratione propriae personae quae est alios dirigens , \\\hline
1.2.8 & Mas por razon dela su persona propia \textbf{ que es tal que ha de gouernar los otros . } Conuiene le de sor sotil e doctrinable . & et rationalem . \textbf{ Sed ratione propriae personae quae est alios dirigens , } oportet quod sit solers , et docilis . \\\hline
1.2.8 & Ca aquel que esta en tanta alteza de dignidat \textbf{ que es puesto para gouernar tanta gente e tanto pueblo . } Conuiene le que sea engennoso e sotil & Nam qui in tanto culmine est positus , \textbf{ ut tantam gentem regere habeat , oportet quod sit industris , et solers , } ut sciat ex se inuenire \\\hline
1.2.8 & Conuiene le que sea engennoso e sotil \textbf{ por que sepa por si buscar e fallar aquellos bienes } que conuiene a su pueblo e asu gente ¶ & ut tantam gentem regere habeat , oportet quod sit industris , et solers , \textbf{ ut sciat ex se inuenire } bona gentis sibi commissae . \\\hline
1.2.8 & que conuiene a su pueblo e asu gente ¶ \textbf{ Mas porque ningun omne non puede conplidamente penssar aquellas cosas } que son aprouechables a todo el regno . & bona gentis sibi commissae . \textbf{ Verum quia nullus homo sufficit ad excogitandum omnia quae possunt esse utilia toti regno , } cum hoc quod Regem expedit esse solertem ex se , \\\hline
1.2.8 & ahun conuiene le de ser doctrinable resçebiendo e tomando coseio de bueons \textbf{ quel han bien de conseiar . } Ca podemos dezir del Rey & oportet ipsum esse docilem , \textbf{ aliorum consiliis acquiescendo . Possumus enim dicere de Rege , } quod dicitur de Magnanimo 4 Ethicorum , \\\hline
1.2.8 & quel han bien de conseiar . \textbf{ Ca podemos dezir del Rey } aquello que dize el philosofo del magnanimo & oportet ipsum esse docilem , \textbf{ aliorum consiliis acquiescendo . Possumus enim dicere de Rege , } quod dicitur de Magnanimo 4 Ethicorum , \\\hline
1.2.8 & por la qual cosa non le conuiene al Rey \textbf{ de seguir en todas cosas su cabeça } nin atener se sienpre al su engennio propio . & quod non decet ipsum fugere commouentem . \textbf{ Non enim decet Regem in omnibus sequi caput suum , } nec inniti semper solertiae propriae : \\\hline
1.2.8 & de seguir en todas cosas su cabeça \textbf{ nin atener se sienpre al su engennio propio . } Mas conuiene le de ser doctrinable & Non enim decet Regem in omnibus sequi caput suum , \textbf{ nec inniti semper solertiae propriae : } sed oportet ipsum esse docilem , \\\hline
1.2.8 & por que sea ido neo \textbf{ para tomar doctrina de los otros tom̃ado conseio de buenos } assi de Ricos omes commo de uieios commo de sabios commo de los otras & sed oportet ipsum esse docilem , \textbf{ ut sit habilis ad capescendam doctrinam aliorum , | acquiescendo doctrinis , } et consiliis baronum , \\\hline
1.2.8 & e del pueblo \textbf{ a que ha de gouernar . } Conuiene al Rey & Sed ratione gentis quam dirigit , \textbf{ oportet ipsum esse expertum , } et cautum . \\\hline
1.2.8 & que sea j muy prouado et muy aꝑçebido \textbf{ para conosçer el bien e el mal } por que la praeua es delas cosas particulares & oportet ipsum esse expertum , \textbf{ et cautum . } Experientia enim est rerum particularium . \\\hline
1.2.8 & e por que las cosas particulares son muchas e muy departidas \textbf{ e en quanto alguon ha de gouernar departidas gentes son le mester departidas cosas . } por ende conuiene al Rey e al prinçipe en conparacion de su gente & et alia particularia , \textbf{ et prout aliquis negociatur circa aliam , } et aliam gentem , sunt alia , \\\hline
1.2.8 & de ser my prouado conosçiendo las condiconnes particulares de su gente e de su pueblo \textbf{ por que pue da meior guiar e gouernar su pueblo e su gente } e traher los ala fin & et aliam gentem , sunt alia , \textbf{ et alia exquirenda . Oportet igitur Principem respectu gentis cui praeest , esse expertum , cognoscendo particulares conditiones gentis sibi commissae , } ut possit eam melius in debitum finem dirigere . \\\hline
1.2.8 & por que pue da meior guiar e gouernar su pueblo e su gente \textbf{ e traher los ala fin } que deue¶ & et alia exquirenda . Oportet igitur Principem respectu gentis cui praeest , esse expertum , cognoscendo particulares conditiones gentis sibi commissae , \textbf{ ut possit eam melius in debitum finem dirigere . } Ultimo oportet ipsum esse cautum . \\\hline
1.2.8 & assi en las sçiençias praticas \textbf{ que son para obrar muchas vezes algunas malas cosas se mezclan con las buenas . } Por la qual cosa cuydan los omes & sed apparent vera : \textbf{ sic in agibilibus mala multotiens admiscentur bonis , } propter quod creduntur bona , \\\hline
1.2.8 & que assi es conuiene al Rey sea aꝑcebido \textbf{ para desechar } e despreçiar aquellas cosas & sed non sunt bona , \textbf{ sed apparent bona . Oportet igitur Regem esse cautum , } respuendo apparenter bona , \\\hline
1.2.8 & para desechar \textbf{ e despreçiar aquellas cosas } que paresçen bueans & sed apparent bona . Oportet igitur Regem esse cautum , \textbf{ respuendo apparenter bona , } et eligendo bona simpliciter , \\\hline
1.2.8 & e non lo son \textbf{ ¶Et otrosi para escoger las buenas } que son dessi buenas & respuendo apparenter bona , \textbf{ et eligendo bona simpliciter , } ad quae debet dirigere gentem sibi commissam . Quomodo Reges , et Principes possunt \\\hline
1.2.8 & que son dessi buenas \textbf{ alas quales deue el Rey guiar } e enderesçar su pueblo e su gente ¶ & et eligendo bona simpliciter , \textbf{ ad quae debet dirigere gentem sibi commissam . Quomodo Reges , et Principes possunt } Anima in sedendo , \\\hline
1.2.8 & alas quales deue el Rey guiar \textbf{ e enderesçar su pueblo e su gente ¶ } ssi commo dize el philosofo en el septimo libro de los fisicos . & et eligendo bona simpliciter , \textbf{ ad quae debet dirigere gentem sibi commissam . Quomodo Reges , et Principes possunt } Anima in sedendo , \\\hline
1.2.9 & si quieren ser sabios \textbf{ non se deuen dar auanidades } mas deuen espender la mayor parte de su uida & si desiderant esse prudentes , \textbf{ non debent vanitatibus intendere : } sed maiorem partem vitae suae debent expendere in cogitando quae possunt esse regno proficua . \\\hline
1.2.9 & non se deuen dar auanidades \textbf{ mas deuen espender la mayor parte de su uida } en cuydar quales son las cosas & non debent vanitatibus intendere : \textbf{ sed maiorem partem vitae suae debent expendere in cogitando quae possunt esse regno proficua . } Quod non sic intelligendum est , \\\hline
1.2.9 & mas deuen espender la mayor parte de su uida \textbf{ en cuydar quales son las cosas } que mas aprouechosas son a su regno . & non debent vanitatibus intendere : \textbf{ sed maiorem partem vitae suae debent expendere in cogitando quae possunt esse regno proficua . } Quod non sic intelligendum est , \\\hline
1.2.9 & assi que ellos non de una auer alguas vezes algunos solazes corporales e honestos . \textbf{ Mas deuen usar dellos tenpradamente } en tal manera que non sean enbargados en el gouernamiento del regno ¶ & ut nullas recreationes corporales habere debeant , \textbf{ sed debent eis adeo moderate uti , } ut non impediantur in regimine regni sui . \\\hline
1.2.9 & Pues que assi es los Reyes \textbf{ e los prinçipes se pueden fazer sabios } por quatro maneras ¶ & ut non impediantur in regimine regni sui . \textbf{ Seipsos ergo poterunt prudentes facere , } ut naturaliter regnum regant : \\\hline
1.2.9 & para que ellos gouiernen el regno \textbf{ naturalmente deuen cuydar primero en los tpons passados } en los quales meior se gouerno el regno . & ut naturaliter regnum regant : \textbf{ excogitando primo tempora retroacta , } sub quibus temporibus regnum melius regebatur , propter quod habeant memoriam praeteritorum , ex quibus scire poterunt , \\\hline
1.2.9 & en los quales meior se gouerno el regno . \textbf{ Et para esto deuen leer las coronicas } e los fechos antigos de los buenos Reyes & ø \\\hline
1.2.9 & por que ayan memoria de los fechos que passaron \textbf{ Et por las cosas que passaron pueden saber } con mon han de fazer & excogitando primo tempora retroacta , \textbf{ sub quibus temporibus regnum melius regebatur , propter quod habeant memoriam praeteritorum , ex quibus scire poterunt , } quid agendum sit in futurum . \\\hline
1.2.9 & Et por las cosas que passaron pueden saber \textbf{ con mon han de fazer } en lo que ha de venir . & sub quibus temporibus regnum melius regebatur , propter quod habeant memoriam praeteritorum , ex quibus scire poterunt , \textbf{ quid agendum sit in futurum . } Nam semper debet suum regimen conformare regimini retroacto , \\\hline
1.2.9 & con mon han de fazer \textbf{ en lo que ha de venir . } Ca sienpre deue el Rey conformar e ordenar el su gouernamiento & sub quibus temporibus regnum melius regebatur , propter quod habeant memoriam praeteritorum , ex quibus scire poterunt , \textbf{ quid agendum sit in futurum . } Nam semper debet suum regimen conformare regimini retroacto , \\\hline
1.2.9 & en lo que ha de venir . \textbf{ Ca sienpre deue el Rey conformar e ordenar el su gouernamiento } segunt el gouernamiento del tp̃o passado & quid agendum sit in futurum . \textbf{ Nam semper debet suum regimen conformare regimini retroacto , } sub quo regnum tutius , \\\hline
1.2.9 & son mas sabios en todas aquellas cosas \textbf{ que han de fazer ¶ } La segunda manera es esta & sic Reges et Principes cogitando acta suorum praedecessorum , \textbf{ fiunt magis prudentes in agibilibus . } Secundo debent diligenter intueri futura bona , \\\hline
1.2.9 & La segunda manera es esta \textbf{ que deuen los reyes muy acuçiosamente catar las bueans cosas } e los bueons fechos & fiunt magis prudentes in agibilibus . \textbf{ Secundo debent diligenter intueri futura bona , } quae possunt esse proficua regno : \\\hline
1.2.9 & e los bueons fechos \textbf{ que son de venir que pueden ser prouechosos al su regno . } Et otrosi deuen penssar en los malos fecho & Secundo debent diligenter intueri futura bona , \textbf{ quae possunt esse proficua regno : } et mala , quae possunt esse nociua . \\\hline
1.2.9 & que son de venir que pueden ser prouechosos al su regno . \textbf{ Et otrosi deuen penssar en los malos fecho } que pue den ser dannosos al su regno & quae possunt esse proficua regno : \textbf{ et mala , quae possunt esse nociua . } Nam ex hoc habebunt prouidentiam futurorum , \\\hline
1.2.9 & por que por esta manera aur̃a sabiduria delas cosas \textbf{ que han de venir } por que los males puedan meior esquiuar . & et mala , quae possunt esse nociua . \textbf{ Nam ex hoc habebunt prouidentiam futurorum , } ut possint mala expeditius vitare , \\\hline
1.2.9 & que han de venir \textbf{ por que los males puedan meior esquiuar . } Et los bienes mas ligeramente alcançar¶ & Nam ex hoc habebunt prouidentiam futurorum , \textbf{ ut possint mala expeditius vitare , } et bona facilius adipisci . Tertio debent saepe recogitare bonas consuetudines , \\\hline
1.2.9 & Et los bienes mas ligeramente alcançar¶ \textbf{ La terçera manera es que los Reyes et los prinçipes deuen penssar muchas uezes } e traer a su memoria las buenas costun bres & ut possint mala expeditius vitare , \textbf{ et bona facilius adipisci . Tertio debent saepe recogitare bonas consuetudines , } et bonas leges : \\\hline
1.2.9 & La terçera manera es que los Reyes et los prinçipes deuen penssar muchas uezes \textbf{ e traer a su memoria las buenas costun bres } e las buenas leyes . & ut possint mala expeditius vitare , \textbf{ et bona facilius adipisci . Tertio debent saepe recogitare bonas consuetudines , } et bonas leges : \\\hline
1.2.9 & e las buenas leyes son principalmente comienços \textbf{ e razones para bien obrar et para bien gouernar . Ca el entendimiento de los prinçipes } e delas leyes & nam talia sunt maxime principia agibilium , \textbf{ sicut intellectus principiorum est . } Tanto ergo Rex magis intelligens est circa agibilia , \\\hline
1.2.9 & e delas leyes \textbf{ e delas costunbres les faze auer manera } para bien gouernar . & ø \\\hline
1.2.9 & e delas costunbres les faze auer manera \textbf{ para bien gouernar . } Et por ende tanto deue el Rey ser & ø \\\hline
1.2.9 & mas acuçioso cerca \textbf{ lo que deue fazer } quanto mas buean s leyes e bueans costunbres tiene en su memoria & ø \\\hline
1.2.9 & quanto mas buean s leyes e bueans costunbres tiene en su memoria \textbf{ para gouernar } por las quales puede saber & et bonas consuetudines in mente habet : \textbf{ ex quibus scire potest , } quid in quolibet negotio sit agendum . \\\hline
1.2.9 & para gouernar \textbf{ por las quales puede saber } que ha de fazer en cada negoçio¶ & et bonas consuetudines in mente habet : \textbf{ ex quibus scire potest , } quid in quolibet negotio sit agendum . \\\hline
1.2.9 & por las quales puede saber \textbf{ que ha de fazer en cada negoçio¶ } La quarta manera es que muchas e muchas uezes deue cuydar & ex quibus scire potest , \textbf{ quid in quolibet negotio sit agendum . } Quarto saepe saepius excogitare debet , \\\hline
1.2.9 & que ha de fazer en cada negoçio¶ \textbf{ La quarta manera es que muchas e muchas uezes deue cuydar } en qual manera ahun & quid in quolibet negotio sit agendum . \textbf{ Quarto saepe saepius excogitare debet , } quomodo per huiusmodi bonas leges , et consuetudines debite regnum regat , eliciendo ex eis debitas conclusiones agibilium . \\\hline
1.2.9 & en qual manera ahun \textbf{ por estas leys buenas e buenas costunbres puede bien gouernar su regno } tomando delas razones conuenibles conclusiones & Quarto saepe saepius excogitare debet , \textbf{ quomodo per huiusmodi bonas leges , et consuetudines debite regnum regat , eliciendo ex eis debitas conclusiones agibilium . } Non enim sufficit esse intelligentem , \\\hline
1.2.9 & para todas las cosas \textbf{ que ha de fazer . } Ca non abasta seer entendido & quomodo per huiusmodi bonas leges , et consuetudines debite regnum regat , eliciendo ex eis debitas conclusiones agibilium . \textbf{ Non enim sufficit esse intelligentem , } habendo cognitionem legum , \\\hline
1.2.9 & sabiendo las leyes e las costunbres \textbf{ que son comiencos para obrar en las cosas } que ha de fazer & habendo cognitionem legum , \textbf{ et consuetudinem , quae sunt principia agibilium : } nisi quis sit rationalis , \\\hline
1.2.9 & que son comiencos para obrar en las cosas \textbf{ que ha de fazer } si el non fuere razonable & et consuetudinem , quae sunt principia agibilium : \textbf{ nisi quis sit rationalis , } ex illis legibus , \\\hline
1.2.9 & e de aquellas costunbres razones e conclusiones conuenibles \textbf{ para lo que ha de fazer } la qual cosa commo se ha de fazer & ex illis legibus , \textbf{ et consuetudinibus debitas conclusiones agibilium eliciendo . } Quod quomodo fieri debeat , \\\hline
1.2.9 & para lo que ha de fazer \textbf{ la qual cosa commo se ha de fazer } mostrar & et consuetudinibus debitas conclusiones agibilium eliciendo . \textbf{ Quod quomodo fieri debeat , } in tertio Libro , \\\hline
1.2.9 & la qual cosa commo se ha de fazer \textbf{ mostrar } lo hemos mas conplidamente en el terçero libro & Quod quomodo fieri debeat , \textbf{ in tertio Libro , } ubi agetur de regimine regni , \\\hline
1.2.9 & delas quales fablamos \textbf{ ya en el capitulo soƀ dicho podran fazer assi mismos sabios } Mas por que la malicia es corronpadera dela razon & de quibus in praecedenti capitulo fecimus mentionem , \textbf{ poterunt seipsos prudentes facere . } Verum quia malitia est corruptiua principii . \\\hline
1.2.9 & Mas por que la malicia es corronpadera dela razon \textbf{ e del comienco para obrar . } Ca assi commo aquel que ha el gosto corronpido mal iudga delos sabores . & poterunt seipsos prudentes facere . \textbf{ Verum quia malitia est corruptiua principii . } Sicut enim quis habens corruptum gustum , \\\hline
1.2.9 & por maliçia es ciego en el entendimiento e en la razon \textbf{ por que iudge mal en lo que ha de fazer } Ca alas vezes iudgaque ha de fazer aquello que deuia escusar & ut male iudicet de agibilibus : \textbf{ iudicat enim esse agendum quod est fugiendum , } et e conuerso . \\\hline
1.2.9 & por que iudge mal en lo que ha de fazer \textbf{ Ca alas vezes iudgaque ha de fazer aquello que deuia escusar } e alas vezes el contrario¶ & ut male iudicet de agibilibus : \textbf{ iudicat enim esse agendum quod est fugiendum , } et e conuerso . \\\hline
1.2.9 & que deuen ser acordables prouisores engennosos e doctrinables \textbf{ e auer las otras cosas } que dixiemos de suso conuieneles & et Principes volunt esse prudentes , \textbf{ cum hoc quod debent esse memores , prouidi , solertes , et dociles , et alia , } quae superius diximus , oportet ipsos esse bonos , et non habere voluntatem deprauatam : \\\hline
1.2.9 & Et que non yerren en el iuyzio Judgando \textbf{ que han de fazer } aquello que deuien escusar & et iudicent esse agenda , \textbf{ quae sunt fugienda . } Philosophus in 5 Ethicorum distinguit duplicem Iustitiam , legalem , et aequalem . Legalis enim Iustitia est quid generale , \\\hline
1.2.9 & que han de fazer \textbf{ aquello que deuien escusar } lphilosofo en el quinto libro delas ethicas & et iudicent esse agenda , \textbf{ quae sunt fugienda . } Philosophus in 5 Ethicorum distinguit duplicem Iustitiam , legalem , et aequalem . Legalis enim Iustitia est quid generale , \\\hline
1.2.10 & en el primero libro dela grand ph̃ia moral . \textbf{ La ley manda fazer las obras de todas las uirtudes . } Ca manda la ley obrar obras fuertes e obras tenpradas . & Sed ( ut dicitur primo Magnorum Moralium ) \textbf{ lex praecipit actus omnium virtutum . } Praecipit enim lex operari fortia et temperata , \\\hline
1.2.10 & La ley manda fazer las obras de todas las uirtudes . \textbf{ Ca manda la ley obrar obras fuertes e obras tenpradas . } Et generalmente todas las obras & lex praecipit actus omnium virtutum . \textbf{ Praecipit enim lex operari fortia et temperata , } et uniuersaliter omnia quae dicuntur \\\hline
1.2.10 & que la ley manda \textbf{ non del enparar elaz en la fazienda } nin foyr dela fazienda & quod Iustitia legalis est perfecta virtus . Sic etiam Ethicorum 5 scribitur , \textbf{ quod lex praecipit non derelinquere aciem , } neque fugere , \\\hline
1.2.10 & non del enparar elaz en la fazienda \textbf{ nin foyr dela fazienda } nin echar las armas & quod lex praecipit non derelinquere aciem , \textbf{ neque fugere , } neque obiicere arma , \\\hline
1.2.10 & nin foyr dela fazienda \textbf{ nin echar las armas } dessi altp̃o del mester & neque fugere , \textbf{ neque obiicere arma , } quod spectat ad fortitudinem . \\\hline
1.2.10 & que non faga luxuria \textbf{ la qual cosa pertenesçe ala tenperança . Et otrosi manda non ferir nin contender nin fazer tuerto a otro } que son obras de manssedunbre . & Et praecipit non moechari , \textbf{ quod pertinet ad temperantiam . | Et non percutere , } neque contendere , \\\hline
1.2.10 & que son obras de manssedunbre . \textbf{ Et por ende la ley generalmente manda fazer } e conplir todas las uirtudes & neque contendere , \textbf{ quae sunt opera mansuetudinis . Lex igitur uniuersaliter iubet omnem virtutem implere , } et malitiam fugere . \\\hline
1.2.10 & Et por ende la ley generalmente manda fazer \textbf{ e conplir todas las uirtudes } e esq̉uarton dos los males & neque contendere , \textbf{ quae sunt opera mansuetudinis . Lex igitur uniuersaliter iubet omnem virtutem implere , } et malitiam fugere . \\\hline
1.2.10 & e aquello que es suyo . \textbf{ Mas en otra manera se puede tomar la diferençia destas dos iustiçias . } Ca assi commo dize el philosofo en el quinto libro delas ethicas . & quod est aequum , \textbf{ idest quod sibi debetur . Differentia autem harum Iustitiarum sic potest accipi . } Nam \\\hline
1.2.10 & e ordenan toda manera de bondat Et por ende seer el omne iusto segunt la ley \textbf{ e conplir la iustiçia legales } segnir todo bien & secundum legem , \textbf{ et implere legalem Iustitiam , est sequi omne bonum , } et fugere omne vitium , \\\hline
1.2.10 & e conplir la iustiçia legales \textbf{ segnir todo bien } e esquiuar todo mal . & secundum legem , \textbf{ et implere legalem Iustitiam , est sequi omne bonum , } et fugere omne vitium , \\\hline
1.2.10 & segnir todo bien \textbf{ e esquiuar todo mal . } Et es auer en alguna manera toda uirtud & et implere legalem Iustitiam , est sequi omne bonum , \textbf{ et fugere omne vitium , } et habere quodammodo omnem virtutem , propter quod legalis Iustitia dicta est quodammodo omnis virtus , quia exercet opera omnium virtutum . \\\hline
1.2.10 & e esquiuar todo mal . \textbf{ Et es auer en alguna manera toda uirtud } ¶Et por ende la iustiçia legales en alguno manera toda uirtud & et implere legalem Iustitiam , est sequi omne bonum , \textbf{ et fugere omne vitium , } et habere quodammodo omnem virtutem , propter quod legalis Iustitia dicta est quodammodo omnis virtus , quia exercet opera omnium virtutum . \\\hline
1.2.10 & ¶Et por ende la iustiçia legales en alguno manera toda uirtud \textbf{ por que manda fazer las obras de todas las uirtudes . } Enpero conuiene de saber & et fugere omne vitium , \textbf{ et habere quodammodo omnem virtutem , propter quod legalis Iustitia dicta est quodammodo omnis virtus , quia exercet opera omnium virtutum . } Non est autem simpliciter legalis \\\hline
1.2.10 & por que manda fazer las obras de todas las uirtudes . \textbf{ Enpero conuiene de saber } que la iustiçia legal non es dicha toda uirtud & ø \\\hline
1.2.10 & non en quanto se deleyte en ellas \textbf{ mas en quanto las manda fazer la ley e el quiere conplir la ley } es dicho iusto legal . & non quia delectatur in eis , \textbf{ sed quia ea lex praecipit , | et vult implere legem , } iustus legalis est . \\\hline
1.2.10 & por si \textbf{ e en quanto es iusto legal deleytase en conplir la ley . } Mas si el iusto legal se deleyta & per se , \textbf{ et secundum quod homo , } delectatur in impletione legis . Si autem delectatur in operibus singularium virtutum , \\\hline
1.2.10 & e son dadas alas çibdades subiec tasal mandamiento del prinçipe \textbf{ al qual ꝑtenesçe confirmar las leyes¶ } Pues que assi es ser el ome acabado en orden alas leyes & subditis imperio Principis , \textbf{ cuius est leges facere . Perfici ergo in ordine ad leges , } est perfici in ordine ad Principem , \\\hline
1.2.10 & al qual parte nesçe \textbf{ commo dicho es confirmar la ley } o en orden atonda la çibdat & est perfici in ordine ad Principem , \textbf{ cuius est legem ferre , } vel in ordine ad totam Ciuitatem , \\\hline
1.2.10 & aque que las ha \textbf{ segunt si¶ } Pues que assi es paresçe commo el iusto se gales & et aliae virtutes huiusmodi perficiant habentem \textbf{ secundum se . Patet ergo , } quomodo Iustitia legalis est quodammodo omnis virtus , et quod non determinat sibi specialem Iustitiam , sed agit opera specialium virtutum . \\\hline
1.2.10 & Mas si en estos bien es de fuera alguno fuere malo \textbf{ assi conmosi quisiere auer mas de aquellos bienes de quanto le conuiene auer } por esta razon viene danno alos otros çibdadanos & Sed si in bonis exterioribus aliquis malus sit , \textbf{ ut quod velit habere plus de iis , | quam eum deceat : } ex hoc infertur nocumentum aliis ciuibus : \\\hline
1.2.10 & Et por ende se sigue \textbf{ que esta iustiçia egual manda dar a cada vno su derecho } ca el derechon esta en vna ygualdat & ut quod unusquisque in huiusmodi exterioribus bonis habeat quod aequum est . Inde est ergo quod haec Iustitia dicitur unicuique suum tribuere , \textbf{ quia ius in quadam aequalitate consistit : } haec autem unicuique tribuit \\\hline
1.2.10 & ca el derechon esta en vna ygualdat \textbf{ Mas esta iustiçia egual manda a cada vno dar } lo que es e suyo & quia ius in quadam aequalitate consistit : \textbf{ haec autem unicuique tribuit | quod iustum } vel aequum est . \\\hline
1.2.10 & e lo que es igual \textbf{ Et assi es dicha dar a cada vno } lo que es suyo . & vel aequum est . \textbf{ Sic etiam dicitur unicuique tribuere quod suum est : } quia aequum est , \\\hline
1.2.10 & si esta iustiçia sp̃al es dicha igual por que entiende a egualdat . \textbf{ Como los çibdadanos pue dan estos biens de fuera partiçipar en dos maneras desigualmente } siguese que dos maneras ay desta iustiçia particular . & Si igitur haec Iustitia specialis aequalis dicitur , \textbf{ et aequalitati intendit : | cum bona exteriora dupliciter ciues inaequaliter participare possint , } dupliciter erit huiusmodi particularis Iustitia . Accidit autem aliquos participare bona inaequaliter in commutationibus , \\\hline
1.2.10 & La vna es conmutatiua \textbf{ para mudar vna cosa en otra ¶ } La otra es distributiua & Erit igitur dupliciter specialis Iustitia , \textbf{ commutatiua , et distributiua . } Omnis enim Iustitia , \\\hline
1.2.10 & La otra es distributiua \textbf{ para partir los bienes e los galardones pues que assi es toda iustiçi } si quier se alegal & commutatiua , et distributiua . \textbf{ Omnis enim Iustitia , } siue sit legalis , \\\hline
1.2.10 & que la iustiçia legal . \textbf{ Mas por auentura adelante aura logar de fablar desto } mas quanto alo presente cunple & quam legalis . \textbf{ Sed de hoc forte alibi erit locus . } Ad praesens autem in tantum dictum est , \\\hline
1.2.10 & que es dicha ygual \textbf{ por que ha de ygualar las cosas } e esta se departe en dos maneras . & et quaedam specialis , \textbf{ quae dicitur aequalis . } Et hoc dupliciter , \\\hline
1.2.11 & Mas que sin la iustiçia general non pueden los regnos \textbf{ durar } podemos lo prouar en dos maneras & Quod quidem absque Iustitia generali regna durare non possint , \textbf{ duplici via inuestigare possumus . } Prima sumitur \\\hline
1.2.11 & durar \textbf{ podemos lo prouar en dos maneras } ¶la primera se toma de parte dessa misma iustiçia general¶ & Quod quidem absque Iustitia generali regna durare non possint , \textbf{ duplici via inuestigare possumus . } Prima sumitur \\\hline
1.2.11 & assi se declara . Ca assi como es dicho en el capitulo sobredicho la iustiçia legales en alguna meranera toda uirtud \textbf{ Ca auer esta iustiçia es conplir la ley ¶ } Pues que assi es si la ley manda & nam \textbf{ ( ut in praecedenti capitulo dicebatur ) Legalis Iustitia est quodammodo omnis virtus . Habere enim huiusmodi Iustitiam , est implere legem . } Si ergo lex iubet omne bonum , et prohibet omne malum : \\\hline
1.2.11 & e uieda todo mal \textbf{ cunplir la les es seer omne uertuoso acabadamente . } Et por ende dize el philosofo & implere legem , \textbf{ est esse perfecte virtuosum . } Ideo primo Magnorum moralium , \\\hline
1.2.11 & e si fuere mal entero \textbf{ non se puede sos rir ¶ } pues que assi es . & si integrum sit , \textbf{ importabile fit . Importabilia igitur esset illud regnum , et durare non posset illa ciuitas , } cuius ciues integre essent mali , \\\hline
1.2.11 & si los çibdadanos fuessen enteramente malos . \textbf{ en ninguna cosa non quisi es en cunplir la ley } ni quisiesen tomar ninguna parte dela ley nin dela iustiçia . & cuius ciues integre essent mali , \textbf{ et in nullo vellent implere legem , } nec vellent in aliquo participare legalem Iustitiam . \\\hline
1.2.11 & en ninguna cosa non quisi es en cunplir la ley \textbf{ ni quisiesen tomar ninguna parte dela ley nin dela iustiçia . } el regno no los podrie sos rir & et in nullo vellent implere legem , \textbf{ nec vellent in aliquo participare legalem Iustitiam . } Ex parte igitur ipsius legalis Iustitiae , \\\hline
1.2.11 & ni quisiesen tomar ninguna parte dela ley nin dela iustiçia . \textbf{ el regno no los podrie sos rir } ni la su çibdat non podrie mucho durar . & et in nullo vellent implere legem , \textbf{ nec vellent in aliquo participare legalem Iustitiam . } Ex parte igitur ipsius legalis Iustitiae , \\\hline
1.2.11 & el regno no los podrie sos rir \textbf{ ni la su çibdat non podrie mucho durar . } Et por ende de parte dela iustiçia legal . & ø \\\hline
1.2.11 & Et el su contrario es coplida \textbf{ maliçia se puede prouar } que sin lan iustiçia legal non pueden los regnos estar ni durar ¶ & cuius oppositum est perfecta malitia , \textbf{ probari potest } quod absque legali Iustitia non valent regna subsistere . \\\hline
1.2.11 & maliçia se puede prouar \textbf{ que sin lan iustiçia legal non pueden los regnos estar ni durar ¶ } La segunda manera & probari potest \textbf{ quod absque legali Iustitia non valent regna subsistere . } Secunda via ad inuestigandum hoc idem sumitur ex parte ipsius regni . Regnum enim \\\hline
1.2.11 & La segunda manera \textbf{ para prouar esto mismo se toma de parte del regno . } Ca el regno & quod absque legali Iustitia non valent regna subsistere . \textbf{ Secunda via ad inuestigandum hoc idem sumitur ex parte ipsius regni . Regnum enim } et omnis politia est quidam ordo , \\\hline
1.2.11 & que sin la iustiçia legal non pueden los regnos estar \textbf{ nin durar . } Mas ahun que sin la iustiçia sp̃al & Facile est ergo ostendere , \textbf{ quod absque legali Iustitia regna durare non possunt . } Sed quod absque Iustitia speciali , \\\hline
1.2.11 & Mas ahun que sin la iustiçia sp̃al \textbf{ que se parte en iustiçias comutatiua e distributiua non pueden estar nin durar esto } assi se puede declarar . & Sed quod absque Iustitia speciali , \textbf{ quae diuiditur in Iustitiam commutatiuam , | et distributiuam , } non subsistat regnum , \\\hline
1.2.11 & que se parte en iustiçias comutatiua e distributiua non pueden estar nin durar esto \textbf{ assi se puede declarar . } Ca cada vno de los regnos & non subsistat regnum , \textbf{ sic declarari potest . Quodlibet enim regnum , } et quaelibet congregatio assimilatur cuidam corpori naturali . Sicut enim videmus corpus animalis constare ex diuersis membris connexis , \\\hline
1.2.11 & e cada vna delas comunidades es conpuesta de de pattidas personas ayuntadas e ordenadas a vna cosa . \textbf{ Et por ende pues nos conuiene de fablar en semeiança de los mienbros } dezimos que en cada vno de los mienbros de los cuerpos . & et ordinatis ad unum aliquid . \textbf{ Ut ergo liceat figuraliter loqui , } in membris eiusdem corporis est quodammodo duplex Iustitia , \\\hline
1.2.11 & e non es sinon de departidas personas . \textbf{ Enpero tomando la en semeiança podemos dezir } que la iustiçia es de vno assi mismo & et non est nisi diuersarum personarum . \textbf{ Metaphorice tamen , } et per quandam similitudinem , est Iustitia eiusdem ad se ipsum , et in membris eiusdem corporis possumus aliquo modo contemplari Iustitiam . Unius enim , \\\hline
1.2.11 & que la iustiçia es de vno assi mismo \textbf{ e en mienbros de vn cuerpo podemos entender la iustiçia en alguno manera . } Ca los mienbros de vn cuerpo mismo han ordenamiento entre si mismos & Metaphorice tamen , \textbf{ et per quandam similitudinem , est Iustitia eiusdem ad se ipsum , et in membris eiusdem corporis possumus aliquo modo contemplari Iustitiam . Unius enim , } et eiusdem corporis membra habent ordinem ad se inuicem ; \\\hline
1.2.11 & enla qual abonda el otro . \textbf{ Et por ende por que cada vno pudiesse proueer } a la su mengua & et deficit in alio , \textbf{ in quo ille abundat . Ideo ut quilibet suae indigentiae prouideret , } inuenta fuit commutatiua Iustitia . Contingit enim aliquem abundare in pecunia , in qua alter deficit : \\\hline
1.2.11 & sue fallada la iustiçia mudadora \textbf{ para dar vna cosa por otra . } Ca contesçe que alguno ha abondamiento dedes & ø \\\hline
1.2.11 & Et por ende fue fallada la iustiçia mundadora \textbf{ por que por ella pudiesen los omes traher a egualdat este abondamieto } e esta mengua dando vna cosa por otra . & et deficere in frumento , \textbf{ in quo ille abundat . Per commutatiuam ergo iustitiam huiusmodi superabundantia , } et defectus reducitur ad aequalitatem : \\\hline
1.2.11 & e para los otros mienbros \textbf{ Enpero fallesçe en poderio de andar . } Ca non puede el oio andar & deficit tamen a potentia gressiua , \textbf{ quia non potest pergere . Pes autem abundat ingressiua potentia : } deficit tamen in visione , \\\hline
1.2.11 & Enpero fallesçe en poderio de andar . \textbf{ Ca non puede el oio andar } mas el pie ha poderio de andar & deficit tamen a potentia gressiua , \textbf{ quia non potest pergere . Pes autem abundat ingressiua potentia : } deficit tamen in visione , \\\hline
1.2.11 & Ca non puede el oio andar \textbf{ mas el pie ha poderio de andar } e fallesçe en veer . & quia non potest pergere . Pes autem abundat ingressiua potentia : \textbf{ deficit tamen in visione , } quia nihil videt . \\\hline
1.2.11 & mas el pie ha poderio de andar \textbf{ e fallesçe en veer . } ca non vee ninguna cosa . & quia non potest pergere . Pes autem abundat ingressiua potentia : \textbf{ deficit tamen in visione , } quia nihil videt . \\\hline
1.2.11 & pues que assi es el oio \textbf{ por el veer } en que abonda acorre ala mengua del pie . & Oculus igitur \textbf{ secundum visionem in qua abundat , } subuenit indigentiae pedis , \\\hline
1.2.11 & por que non de en la piedra \textbf{ Mas el pie por el poderio que ha de andar acorre ala mengua del oio } que non puede andar & quia dirigit ipsum . \textbf{ Pes autem per potentiam gressiuam qua pollet , } subuenit indigentiae oculi , \\\hline
1.2.11 & Mas el pie por el poderio que ha de andar acorre ala mengua del oio \textbf{ que non puede andar } por quel trahe sobre si . & Pes autem per potentiam gressiuam qua pollet , \textbf{ subuenit indigentiae oculi , } quia portat eum . \\\hline
1.2.11 & Et tomasse del otro aquello \textbf{ de que ha mengua la çibdat non podria estar ni dirrar ¶ } Pues que assi es & et acciperet illud in quo deficit , \textbf{ ciuitas durare non posset . } Sicut ergo in membris eiusdem corporis prout habent ordinem ad se inuicem , \\\hline
1.2.11 & por sabiduria natural \textbf{ sin la qual el cuerpo nal non podria durar . } Dien assi en quanto los çibdadanos de vna çibdat o de vn regno han ordenamiento entre si mismos & est in eis quaedam commutatiua Iustitia , \textbf{ sine qua corpus naturale durare non posset : } sic prout ciues eiusdem ciuitatis , \\\hline
1.2.11 & assi commo asu Rey o a su cabdiello \textbf{ que les deue dar e partir las honrras e los bienes } segunt sus uirtudes e sus dignidades ¶ & vel ad ducem , \textbf{ qui } secundum eorum virtutem , \\\hline
1.2.11 & maguer que cada vna desigualdat le enflaquezca \textbf{ e le faga enfermar . } Bien assi cada vna mengua de iustiçia non corronpe del todo el regno e la comunidat . & tamen quaelibet inaequalitas aegrotat , \textbf{ et infirmat ipsum : sic non quaelibet Iniustitia corrumpit totaliter regnum , } et politiam , \\\hline
1.2.11 & Empero en qual quier manera que se tome la iustiçia \textbf{ si ella non puede durar el regno } nin la çibdat ¶ & Quocunque tamen modo sumatur Iustitia , \textbf{ sine ea ciuitas vel regnum durare non potest . } Bene ergo dictum est , \\\hline
1.2.11 & Et si la iustiçia es tanto bien del rey \textbf{ mucho afincadamente deue el rey estudiar } por que en los sus regnos sea guardada la iustiçia & Si igitur Iustitia est tantum bonum Regis , et regni , \textbf{ summo opere Rex studere debet , } ut in suo Regno , \\\hline
1.2.12 & sin la qual las çibdades \textbf{ e los regnos non pueden durar } Empero por que el coraçon del noble ome sienprees cobdiçioso de oyr nueuas razones . & cum sine ea ciuitates , \textbf{ et regna durare non possint . } Tamen quia est animus hominis generosus , \\\hline
1.2.12 & e los regnos non pueden durar \textbf{ Empero por que el coraçon del noble ome sienprees cobdiçioso de oyr nueuas razones . } Por ende aduremos agora nueuas maneras e razones & et regna durare non possint . \textbf{ Tamen quia est animus hominis generosus , | et semper auidus nouas rationes audire : } adducemus nouos modos , \\\hline
1.2.12 & Por ende aduremos agora nueuas maneras e razones \textbf{ por las quales podremos mostrar } que mucho conuiene alos reyes & adducemus nouos modos , \textbf{ quibus ostendi poterit , } quod maxime decet Reges , \\\hline
1.2.12 & e alos prinçipes \textbf{ de ser iustos et de guardar iustiçia . } Mas esta uerdat podemos prouar en quatro maneras & quod maxime decet Reges , \textbf{ et Principes esse iustos . Possumus autem hanc veritatem quadruplici via venari , } secundum quatuor quae tanguntur de Iustitia in 5 Ethicorum . Prima via sumitur ex parte personae regiae . Secunda ex parte ipsius Iustitiae . \\\hline
1.2.12 & de ser iustos et de guardar iustiçia . \textbf{ Mas esta uerdat podemos prouar en quatro maneras } segunt quatro cosas & quod maxime decet Reges , \textbf{ et Principes esse iustos . Possumus autem hanc veritatem quadruplici via venari , } secundum quatuor quae tanguntur de Iustitia in 5 Ethicorum . Prima via sumitur ex parte personae regiae . Secunda ex parte ipsius Iustitiae . \\\hline
1.2.12 & ¶ \textbf{ La primera manera se puede assi declarar . } Ca si la ley es regla de todas las obras & quae ex iniustitia consurgit . \textbf{ Prima via sic patet . } Nam si lex est regula agendorum , \\\hline
1.2.12 & Ca si la ley es regla de todas las obras \textbf{ que auemos de fazer } assi commo dize el philosofo en el quinto libro delas ethicas & Nam si lex est regula agendorum , \textbf{ ut haberi potest ex 5 Ethicor’ ipse iudex , } et multo magis ipse Rex , \\\hline
1.2.12 & qual quier iuez \textbf{ e muchon mas el Rey a quien parte nesçe de poner las leyes deue seer vna regla } en aquellas cosas & ut haberi potest ex 5 Ethicor’ ipse iudex , \textbf{ et multo magis ipse Rex , } cuius est leges ferre , \\\hline
1.2.12 & en aquellas cosas \textbf{ que se deuen fazer . } Ca el rey o el prinçipe es vna ley & et multo magis ipse Rex , \textbf{ cuius est leges ferre , } debet esse quaedam regula in agendis . Est enim Rex siue Princeps quaedam lex , \\\hline
1.2.12 & que non ha alma . \textbf{ tanto el Rey o el prinçipe deue sobrepuiar la ley . } Ca deue el prinçipe o el Rey ser de tan grant iustiçia & Princeps vero est quaedam animata lex . Quantum ergo animatum inanimatum superat , \textbf{ tantum Rex siue Princeps debet superare legem . } Debet \\\hline
1.2.12 & Ca deue el prinçipe o el Rey ser de tan grant iustiçia \textbf{ e de tan grant egualdat por que pueda enderesçar } e egualar las leyes . & etiam Rex esse tantae Iustitiae , \textbf{ et tantae aequitatis , } ut possit ipsas leges dirigere : \\\hline
1.2.12 & e de tan grant egualdat por que pueda enderesçar \textbf{ e egualar las leyes . } Ca algun caso ay & et tantae aequitatis , \textbf{ ut possit ipsas leges dirigere : } cum aliquo casu leges obseruari non debeant , \\\hline
1.2.12 & Ca algun caso ay \textbf{ en que se non deuen guardar las leyes } assi commo paresçra adelante . & ut possit ipsas leges dirigere : \textbf{ cum aliquo casu leges obseruari non debeant , } ut infra patebit . \\\hline
1.2.12 & assi commo paresçra adelante . \textbf{ Pues que assi es dubdar } si el Rey deue ser egual & ut infra patebit . \textbf{ Dubitare ergo , } utrum Rex debeat esse aequalis et iustus , \\\hline
1.2.12 & si el Rey deue ser egual \textbf{ e iusto es dubdar } si la regla deue seer egual e derecha & utrum Rex debeat esse aequalis et iustus , \textbf{ est dubitare , } utrum ipsa regula debeat esse regulata . \\\hline
1.2.12 & nin derechureros desordenan e desigualan el regno \textbf{ por que en el non se pueda guardar iustiçia . } Et por ende mucho se deue guardar & Sic si Reges sint iniusti , \textbf{ disponunt regnum , } ut in eo non obseruetur Iustitia . Maxime ergo studere debent , \\\hline
1.2.12 & por que en el non se pueda guardar iustiçia . \textbf{ Et por ende mucho se deue guardar } e mucho deuen estudiar los reyes & disponunt regnum , \textbf{ ut in eo non obseruetur Iustitia . Maxime ergo studere debent , } ne sint iniusti , et inaequales : \\\hline
1.2.12 & Et por ende mucho se deue guardar \textbf{ e mucho deuen estudiar los reyes } que non sean iniustos e desiguales & disponunt regnum , \textbf{ ut in eo non obseruetur Iustitia . Maxime ergo studere debent , } ne sint iniusti , et inaequales : \\\hline
1.2.12 & e el Rey sea vna ley animada \textbf{ e vna regla . animada de todo lo que le ha de fazer . Paresçe de parte dela persona del Rey } que conuiene mucho al Rey de guardar la iustiçia¶ & quia est quaedam animata lex , \textbf{ et quaedam animata regula agendorum , } ex parte ipsius personae regiae maxime decet ipsum seruare Iustitiam . \\\hline
1.2.12 & e vna regla . animada de todo lo que le ha de fazer . Paresçe de parte dela persona del Rey \textbf{ que conuiene mucho al Rey de guardar la iustiçia¶ } La segunda manera & et quaedam animata regula agendorum , \textbf{ ex parte ipsius personae regiae maxime decet ipsum seruare Iustitiam . } Secundo possumus \\\hline
1.2.12 & La segunda manera \textbf{ por que podemos prouar } que conuiene al Rey de ser iusto & ex parte ipsius personae regiae maxime decet ipsum seruare Iustitiam . \textbf{ Secundo possumus } inuestigare hoc idem ex parte ipsius Iustitiae . \\\hline
1.2.12 & que conuiene al Rey de ser iusto \textbf{ et de guardar la iustiçia se puede tomar de parte dela iustiçia . } Et esta se declara assi . & Secundo possumus \textbf{ inuestigare hoc idem ex parte ipsius Iustitiae . } Nam Iustitia est quoddam magnum bonum , \\\hline
1.2.12 & por nonbre comunal \textbf{ que quiere dezir cosa clara e cosa apuesta . } Et esta estrella algunas vezes nasçe ante del sol & ø \\\hline
1.2.12 & e por esso le llaman uespero . \textbf{ Et por ende la entençion del philosofo es dezir } que esta estrella & Aliquando vero sequitur ipsum , \textbf{ et tunc apparet in sero , et dicitur Hesperus . Est ergo intentio Philosophi dicere , quod Venus , } quae est tam pulcherrima stella , \\\hline
1.2.12 & si conuiene alos Reyes \textbf{ e alos prinçipes de auer muy claras uirtudes } paresçe de parte dela iustiçia & Si \textbf{ ergo decet Reges et Principes habere clarissimas virtutes ex parte ipsius Iustitiae , quae est quaedam clarissima virtus , } probari potest , \\\hline
1.2.12 & que es muy clara uirtud \textbf{ que se puede prouar } que conuiene alos Reyes de guardar la iustiçia . & ergo decet Reges et Principes habere clarissimas virtutes ex parte ipsius Iustitiae , quae est quaedam clarissima virtus , \textbf{ probari potest , } quod decet eos obseruare Iustitiam . \\\hline
1.2.12 & que se puede prouar \textbf{ que conuiene alos Reyes de guardar la iustiçia . } lo terçero esso mismo se puede prouar & probari potest , \textbf{ quod decet eos obseruare Iustitiam . } Tertio hoc probari potest ex ipsa perfectione bonitatis , \\\hline
1.2.12 & que conuiene alos Reyes de guardar la iustiçia . \textbf{ lo terçero esso mismo se puede prouar } de parte dela perfecçion dela bondat & quod decet eos obseruare Iustitiam . \textbf{ Tertio hoc probari potest ex ipsa perfectione bonitatis , } quae ex Iustitia innotescit . \\\hline
1.2.12 & que cada vna cosa es acabada \textbf{ enssi quando puede fazer otra tal commo si . } Et quando la su obra se estiende alos otros & sed si sit in Regibus et Principibus ostendit eos esse perfecte bonos . Sic enim videmus in aliis rebus quod unumquodque perfectum est , \textbf{ cum potest sibi simile producere , } et cum actio sua ad alios se extendit : \\\hline
1.2.12 & Ca estonçe es dicha alguna cosa conplidamente caliente \textbf{ quando puede calentar alas otras cosas . } Et quando la su obra & ut tunc aliquid est perfecte calidum , \textbf{ quando potest alia calefacere , } et quando actio sua ad alia se extendit . \\\hline
1.2.12 & e la su calentura se estiende alos otros . Et esso mismo estonçe es dicho el ome conplidamente sabio \textbf{ quando puede ensseñar los otros } e quando la su sçiençia se estiende alos otros . & Et tunc est aliquis perfecte sciens , \textbf{ quando potest alios docere , } et quando scientia sua ad alios se extendit . Ideo scribitur 1 Metaphys’ quod signum omnino scientis , \\\hline
1.2.12 & que señal manifiesta es de omne sabio \textbf{ quando puede enssennar los otros . } Et por ende fablando & et quando scientia sua ad alios se extendit . Ideo scribitur 1 Metaphys’ quod signum omnino scientis , \textbf{ est posse docere . } Ergo a simili , \\\hline
1.2.12 & Et por ende fablando \textbf{ por semeiança podemos dezir } que estonce es dicho el omne conplidamente bueon & Ergo a simili , \textbf{ tunc est aliquis perfecte bonus , } quando bonitas sua usque \\\hline
1.2.12 & Et por ende la bondat acabada de los omes non es conosçida \textbf{ fasta que son puestos en alguna dignidat o en algun prinçipado de sennorio . Ca quando algun omne non ha de gouernar } si non assi mismo non paresçe bien quales & quod non cognoscitur perfecta bonitas hominum , \textbf{ nisi constituantur in aliquo principatu . Nam quandiu aliquis non habet regere nisi seipsum , } non plene apparet qualis sit , \\\hline
1.2.12 & Mas quando es puesto en algun prinçipado o en algun sennorio \textbf{ por que la su bondat se ha de estender a otros estonçe meior paresçe } quales si es bueno o malo & Sed quando constituitur in principatu aliquorum , \textbf{ quia oportet , | quod bonitas sua ad alios se extendat , } tunc melius apparet qualis sit , \\\hline
1.2.12 & que todas las otras uirtudes \textbf{ por que las ha de guiar la iustiçia } es la mas acabada . & quae aliis virtutibus perfectior est , \textbf{ quia est earum directiua , } omnes aliae virtutes morales , \\\hline
1.2.12 & Et por ende todas las otras uirtudes morałs̃ que acaban el omne \textbf{ en ssi deuen auer la iustiçia } assi commo reina e sennora & quae perficiunt hominem in se , \textbf{ se videntur habere ad Iustitiam , } quae perficit hominem in ordine ad alterum , \\\hline
1.2.12 & assi conmo los subditos \textbf{ que en alguna manera solamente han de gouernar assi mismos . } han se a su prinçipe & sicut subditi , \textbf{ qui quodammodo solum habent regere seipsos , } se habent \\\hline
1.2.12 & mas al Rey e al prinçipe de ser iusto \textbf{ e de auer iustiçia } que a ninguno de los otros . & quam ex aliis virtutibus moralibus . \textbf{ Decet ergo Reges et Principes esse iustos , } tum \\\hline
1.2.12 & por que deue ser regla de todas las cosas \textbf{ que se deuen fazer en el regno . } lo otro por quela iustiçia es muy clara uirtud . & ø \\\hline
1.2.12 & ¶La quarta manera \textbf{ por que podemos prouar esso mismo } que conienea los Reyes & tum quia Iustitia est praeclara virtus , tum etiam quia ex ea manifestatur perfectio bonitatis . \textbf{ Quarto hoc decet Reges , } et Principes ex magnitudine malitiae , \\\hline
1.2.12 & que conienea los Reyes \textbf{ e alos prinçipes de guardar } iustiçia & Quarto hoc decet Reges , \textbf{ et Principes ex magnitudine malitiae , } quae ex iniustitia consurgit . \\\hline
1.2.12 & tanto mas acuçiosos deuen ser los Reyes e los prinçipes \textbf{ e mas afincadamente deuen estudiar } para guardar la iustiçia & tanto peior existit . \textbf{ Tanto igitur summopere studere debent Reges , et Principes ut seruent Iustitiam , et iniustitiam vitent : } quanto ex eorum Iustitia potest \\\hline
1.2.12 & e mas afincadamente deuen estudiar \textbf{ para guardar la iustiçia } e para escusar la mi ustiçia e el mal quanto por la mengua dela su iustiçia se puede seguir mayor mal & Tanto igitur summopere studere debent Reges , et Principes ut seruent Iustitiam , et iniustitiam vitent : \textbf{ quanto ex eorum Iustitia potest } consequi maius malum , \\\hline
1.2.12 & para guardar la iustiçia \textbf{ e para escusar la mi ustiçia e el mal quanto por la mengua dela su iustiçia se puede seguir mayor mal } Et puede venir mayor deño a muchos . & quanto ex eorum Iustitia potest \textbf{ consequi maius malum , } et potest inferri pluribus nocumentum . \\\hline
1.2.12 & e para escusar la mi ustiçia e el mal quanto por la mengua dela su iustiçia se puede seguir mayor mal \textbf{ Et puede venir mayor deño a muchos . } Mas avn conuiene mas de declarar commo los Reyes & consequi maius malum , \textbf{ et potest inferri pluribus nocumentum . } Esset autem ulterius declarandum , \\\hline
1.2.12 & Et puede venir mayor deño a muchos . \textbf{ Mas avn conuiene mas de declarar commo los Reyes } e los prinçipes pueden ganar et auer la iustiçia & et potest inferri pluribus nocumentum . \textbf{ Esset autem ulterius declarandum , | quomodo Reges , } et Principes possunt Iustitiam acquirere : \\\hline
1.2.12 & Mas avn conuiene mas de declarar commo los Reyes \textbf{ e los prinçipes pueden ganar et auer la iustiçia } e de commo deuen guardar la iustiçia en los sus regnos . & quomodo Reges , \textbf{ et Principes possunt Iustitiam acquirere : } et quomodo debeant Iustitiam obseruare . \\\hline
1.2.12 & e los prinçipes pueden ganar et auer la iustiçia \textbf{ e de commo deuen guardar la iustiçia en los sus regnos . } Mas esto en el tercero libro aura logar & et Principes possunt Iustitiam acquirere : \textbf{ et quomodo debeant Iustitiam obseruare . } Sed in tertio libro , \\\hline
1.2.12 & do sera mas conplidamente declarado \textbf{ e do diremos commo se deue gouernar el pegno derechamente con iustiçia } t deuedes saber & ubi agetur \textbf{ quomodo Regnum iuste Regi debeat , | plenius ostendetur . } Quia circa quodcunque contingit peccare , \\\hline
1.2.13 & e do diremos commo se deue gouernar el pegno derechamente con iustiçia \textbf{ t deuedes saber } que quando en alguas cosas podemos bien obrar & plenius ostendetur . \textbf{ Quia circa quodcunque contingit peccare , } et bene agere , \\\hline
1.2.13 & t deuedes saber \textbf{ que quando en alguas cosas podemos bien obrar } e pecar en obrando . & plenius ostendetur . \textbf{ Quia circa quodcunque contingit peccare , } et bene agere , \\\hline
1.2.13 & que quando en alguas cosas podemos bien obrar \textbf{ e pecar en obrando . } Conuiene de dar & Quia circa quodcunque contingit peccare , \textbf{ et bene agere , } oportet dare virtutem aliquam , \\\hline
1.2.13 & e pecar en obrando . \textbf{ Conuiene de dar } e de ponetur algua uirtud & et bene agere , \textbf{ oportet dare virtutem aliquam , } per quam regulentur in agendo . \\\hline
1.2.13 & Conuiene de dar \textbf{ e de ponetur algua uirtud } por la qual seamos reglados en las obras & et bene agere , \textbf{ oportet dare virtutem aliquam , } per quam regulentur in agendo . \\\hline
1.2.13 & por la qual seamos reglados en las obras \textbf{ que auemos de fazer ¶ } pues que assi es commo los omes alguas vezes puedan & oportet dare virtutem aliquam , \textbf{ per quam regulentur in agendo . } Cum igitur circa timores , \\\hline
1.2.13 & pues que assi es commo los omes alguas vezes puedan \textbf{ e les contezca de se auer derechamente } e non derechamente en los temores e en las osadias . & per quam regulentur in agendo . \textbf{ Cum igitur circa timores , } et audacias contingat aliquem se habere recte , \\\hline
1.2.13 & e non derechamente en los temores e en las osadias . \textbf{ Conuiene de dar alguna uirtud medianera en los temores e en las osadias } por la qual sea el omne reglado en ellos . & Cum igitur circa timores , \textbf{ et audacias contingat aliquem se habere recte , } et non recte , \\\hline
1.2.13 & que algunos remen algunas cosas \textbf{ que han de temer e alas uegadas temen alguas cosas } que non han de temer . & et non recte , \textbf{ oportet dare virtutem aliquam circa timores , } et audacias . Accidit enim aliquos timere timenda , \\\hline
1.2.13 & que han de temer e alas uegadas temen alguas cosas \textbf{ que non han de temer . } Assi commo alguons & oportet dare virtutem aliquam circa timores , \textbf{ et audacias . Accidit enim aliquos timere timenda , } et non timenda . Sunt enim aliqui adeo pauidi , \\\hline
1.2.13 & mas es loco e landio . \textbf{ Et pues que assi es al fuerte pertenesçe temer las cosas } que ha de temer & non est Fortis , \textbf{ sed insanus . Spectat igitur ad fortem timere timenda , } et audere audenda . \\\hline
1.2.13 & Et pues que assi es al fuerte pertenesçe temer las cosas \textbf{ que ha de temer } e de ser osado alas cosas & ø \\\hline
1.2.13 & por que por ellos non sea el omne rerenido \textbf{ nin enbargado de fazer } e acometer aquellas cosas & et moderans audacias . Reprimit enim Fortitudo timores , \textbf{ ne per eos quis retrahatur ab eo , } quod ratio dictat . Moderat autem audacias , \\\hline
1.2.13 & nin enbargado de fazer \textbf{ e acometer aquellas cosas } que la razon e el entendemiento mando . & ø \\\hline
1.2.13 & ¶VVisto que cosa es la fortaleza finca nos deuer çerca quales cosas ha de ser la fortaleza \textbf{ Pues que assi es comuene de saber } que temer & quod ratio vetat . Viso \textbf{ quid est Fortitudo , } restat videre circa quae esse habeat talis virtus . Sciendum igitur , quod timere , et audere , proprie respiciunt pericula . Nullus autem timet , \\\hline
1.2.13 & Pues que assi es comuene de saber \textbf{ que temer } e auer osadia propiamente catan & ø \\\hline
1.2.13 & que temer \textbf{ e auer osadia propiamente catan } e han de ser enlos peligros . & quid est Fortitudo , \textbf{ restat videre circa quae esse habeat talis virtus . Sciendum igitur , quod timere , et audere , proprie respiciunt pericula . Nullus autem timet , } nisi imaginetur sibi periculum imminere : \\\hline
1.2.13 & si non quando emagina algunan cosa \textbf{ en que puede auer peligro . } nin ninguno non es dicho osado & restat videre circa quae esse habeat talis virtus . Sciendum igitur , quod timere , et audere , proprie respiciunt pericula . Nullus autem timet , \textbf{ nisi imaginetur sibi periculum imminere : } nec omnis dicitur audax , \\\hline
1.2.13 & si non quando acomete alguna cosa espantable e peligrosa . \textbf{ Mas deuedes saber } que los peligros so en dos maneras . & nec omnis dicitur audax , \textbf{ nisi aggrediatur aliquod terribile , } et periculosum . Periculorum autem quaedam sunt bellica , \\\hline
1.2.13 & e en los otros negoçios \textbf{ en los quales pueden conteçer periglos . } Otrosi en los periglos delas faziendas los omes sean en departidas maneras . & et in aegritudinibus , \textbf{ et in aliis circa quae conuenit esse pericula . } Rursus in periculis bellorum homines diuersimode se habent . \\\hline
1.2.13 & que los otros periglos . \textbf{ Et ahun por que en los periglos delas batallas mas fuerte cosa es de repremer los temores } que de restenar las osadias . & ( quia difficiliora , \textbf{ et terribiliora sunt pericula bellica , } quam alia : \\\hline
1.2.13 & Et ahun por que en los periglos delas batallas mas fuerte cosa es de repremer los temores \textbf{ que de restenar las osadias . } Otrosi por que en auiendo osadia non es tan fuerte & ( quia difficiliora , \textbf{ et terribiliora sunt pericula bellica , } quam alia : \\\hline
1.2.13 & Otrosi por que en auiendo osadia non es tan fuerte \textbf{ nin tan graue cosa acometer la batalla } e la pellea commo sofrir & et terribiliora sunt pericula bellica , \textbf{ quam alia : } et etiam quia in periculis bellicis difficilius est reprimere timores , \\\hline
1.2.13 & nin tan graue cosa acometer la batalla \textbf{ e la pellea commo sofrir } e estar contra los lidiadors & quam alia : \textbf{ et etiam quia in periculis bellicis difficilius est reprimere timores , } quam moderare audacias : \\\hline
1.2.13 & Et mas prinçipalmente esta esta uirtud \textbf{ en repremir los temores } que acaesçen en los periglos delas faziendas & rursus \textbf{ quia in audendo non tam difficile est aggredi pugnam , } sicut tolerare , et sustinere pugnantes . ) Fortitudo , \\\hline
1.2.13 & e delas batallas \textbf{ que en restenar las osadias } que son en ellas . & quia in audendo non tam difficile est aggredi pugnam , \textbf{ sicut tolerare , et sustinere pugnantes . ) Fortitudo , } quae est virtus circa pericula , principalius est in reprimendo timores contingentes circa pericula talia , quam in moderando audacias circa ipsa . Sic etiam principalius est huiusmodi virtus in sustinendo pugnantes , \\\hline
1.2.13 & Mas que los periglos delas batallas sean mas fuertes \textbf{ e mas graues de sofrir } que los otros periglos . & quae est virtus circa pericula , principalius est in reprimendo timores contingentes circa pericula talia , quam in moderando audacias circa ipsa . Sic etiam principalius est huiusmodi virtus in sustinendo pugnantes , \textbf{ quam in aggrediendo eos . Quod autem pericula bellica sint difficiliora ad sustinendum , } quam pericula alia : \\\hline
1.2.13 & que los otros periglos . \textbf{ esto por tres maneras lo podemos prouar . } ¶ La primera es esta & quam pericula alia : \textbf{ triplici via ostendi potest . } Primo , \\\hline
1.2.13 & assi commo sentimos del tannimiento dela espada e del cuchiello . \textbf{ pues que assi es los periglos delas batallas son mas guaues de softir } que los otros lo vno & sicut ex tactu gladii . \textbf{ Pericula ergo bellica tum } quia manifesta sunt , tum etiam , \\\hline
1.2.13 & por que ymaginamos \textbf{ que por foyr podemos ligeramente escapar dellos . } Ca nos non podemos & sed quia imaginamur \textbf{ quod per fugam ea de facili vitare non possumus . } Non enim sic per fugam vitare possumus aegritudines : \\\hline
1.2.13 & Ca nos non podemos \textbf{ assi por foyr escapar las enfermedades } por que la enfermedat es alguna cosa & quod per fugam ea de facili vitare non possumus . \textbf{ Non enim sic per fugam vitare possumus aegritudines : } quia cum aegritudo sit aliquid in nobis existens , per fugam eam vitare non possumus . pericula etiam maris non sic per fugam vitari possunt , \\\hline
1.2.13 & que esta en nos \textbf{ e por foyr non la podemos escusar . } Otrosi los periglos dela mar non los podemos por foyr & Non enim sic per fugam vitare possumus aegritudines : \textbf{ quia cum aegritudo sit aliquid in nobis existens , per fugam eam vitare non possumus . pericula etiam maris non sic per fugam vitari possunt , } sicut pericula belli . \\\hline
1.2.13 & e por foyr non la podemos escusar . \textbf{ Otrosi los periglos dela mar non los podemos por foyr } assi escusar & Non enim sic per fugam vitare possumus aegritudines : \textbf{ quia cum aegritudo sit aliquid in nobis existens , per fugam eam vitare non possumus . pericula etiam maris non sic per fugam vitari possunt , } sicut pericula belli . \\\hline
1.2.13 & Otrosi los periglos dela mar non los podemos por foyr \textbf{ assi escusar } commo los periglos delas batallas . & quia cum aegritudo sit aliquid in nobis existens , per fugam eam vitare non possumus . pericula etiam maris non sic per fugam vitari possunt , \textbf{ sicut pericula belli . } Cum ergo difficilius sit durare , et sustinere pericula illa quae per fugam vitare possumus , \\\hline
1.2.13 & Et pues que assi es commo sea \textbf{ mas guaue cosa de endurar } e de sufrir aquellos periglos & sicut pericula belli . \textbf{ Cum ergo difficilius sit durare , et sustinere pericula illa quae per fugam vitare possumus , } quam ad quae sustinenda necessitamur , \\\hline
1.2.13 & mas guaue cosa de endurar \textbf{ e de sufrir aquellos periglos } que podemos escusar por foyr & sicut pericula belli . \textbf{ Cum ergo difficilius sit durare , et sustinere pericula illa quae per fugam vitare possumus , } quam ad quae sustinenda necessitamur , \\\hline
1.2.13 & e de sufrir aquellos periglos \textbf{ que podemos escusar por foyr } que aquellos que por fuerça avemos de sofrir & Cum ergo difficilius sit durare , et sustinere pericula illa quae per fugam vitare possumus , \textbf{ quam ad quae sustinenda necessitamur , } eo quod per fugam ea vitare non possumus ; \\\hline
1.2.13 & que podemos escusar por foyr \textbf{ que aquellos que por fuerça avemos de sofrir } por que por tal foyr non los podemos escusar . & Cum ergo difficilius sit durare , et sustinere pericula illa quae per fugam vitare possumus , \textbf{ quam ad quae sustinenda necessitamur , } eo quod per fugam ea vitare non possumus ; \\\hline
1.2.13 & que aquellos que por fuerça avemos de sofrir \textbf{ por que por tal foyr non los podemos escusar . } Mas guaue cosa es de sofrir los periglos delas batallas & quam ad quae sustinenda necessitamur , \textbf{ eo quod per fugam ea vitare non possumus ; } difficilius sustinentur pericula bellica , \\\hline
1.2.13 & por que por tal foyr non los podemos escusar . \textbf{ Mas guaue cosa es de sofrir los periglos delas batallas } que los otros ¶lo terçero & eo quod per fugam ea vitare non possumus ; \textbf{ difficilius sustinentur pericula bellica , } quam alia . \\\hline
1.2.13 & por la qual cosa mas ymaginamos la muerte forçada . \textbf{ Ca morir porla enfermedat } o por alguna otra manera non paresçe & et per incissuram corporis , \textbf{ per quam maxime apprehendimus violentam mortem . Mori enim per aegritudinem , } vel per aliquem alium modum , \\\hline
1.2.13 & que non es temeroso . \textbf{ Et por ende al fuerte parte nesçe non temer } qual si quier periglo & ø \\\hline
1.2.13 & que la razon o el entendimiento iudga sinplemente \textbf{ que non son de temer . } Enpero mas prinçipalmente es la fortaleza cerca los periglos delas batallas & et in aegritudinibus intimidus est , \textbf{ qui est fortis . Amplius licet Fortitudo sit circa pericula bellica , } reprimendo timores , \\\hline
1.2.13 & Enpero mas prinçipalmente es la fortaleza cerca los periglos delas batallas \textbf{ por que son mas graues de sofrir } que los orros & qui est fortis . Amplius licet Fortitudo sit circa pericula bellica , \textbf{ reprimendo timores , } et moderando audacias : \\\hline
1.2.13 & Et pues que assi es commo nos natural mente fuyamos dela tristeza \textbf{ graue cosa es de repmir los temores } por los quales fuyamos dela tristeza . & Cum ergo naturaliter tristia fugiamus , \textbf{ difficile est reprimere timores , } per quos tristia fugimus . \\\hline
1.2.13 & por los quales fuyamos dela tristeza . \textbf{ Mas non es assi graue de refrenar las osadias } por las quales uenimos a tristeza . & per quos tristia fugimus . \textbf{ Non autem sic difficile est moderare audacias , } per quas tristia aggredimur . \\\hline
1.2.13 & assi commo todos dizen comunal \textbf{ mente esto podemos prouar } por tres razones ¶ & ut communiter ponitur , \textbf{ triplici via venari potest . Primo , } quia aggrediendum , est fortioris : sustinere autem , debilioris est . Aggrediens autem comparatur ad alios , sicut ad debiliores \\\hline
1.2.13 & por tres razones ¶ \textbf{ La primera por que acometer parte nesçe al mas fuerte . } Mas sufrir parte nesçe al mas fiaco & triplici via venari potest . Primo , \textbf{ quia aggrediendum , est fortioris : sustinere autem , debilioris est . Aggrediens autem comparatur ad alios , sicut ad debiliores } sed sustinens , \\\hline
1.2.13 & La primera por que acometer parte nesçe al mas fuerte . \textbf{ Mas sufrir parte nesçe al mas fiaco } por que el que acomete es conparado a aquellos & triplici via venari potest . Primo , \textbf{ quia aggrediendum , est fortioris : sustinere autem , debilioris est . Aggrediens autem comparatur ad alios , sicut ad debiliores } sed sustinens , \\\hline
1.2.13 & assi commo amas fuertes . \textbf{ Et por ende mas guaue cosa es de esforçarse contra los mas fuertes } que contra los mas flacos . & sicut ad fortiores : \textbf{ Difficilius est autem inniti contra fortiores , } quam contra debiliores . \\\hline
1.2.13 & que contra los mas flacos . \textbf{ Et por çierto mas guauecosa es de sefrir lo batalla } que de acometer los lidiadores ¶ & quam contra debiliores . \textbf{ Propter quod difficilius est sustinere pugnam , } quam aggredi pugnantes . Secundo est difficilius , \\\hline
1.2.13 & Et por çierto mas guauecosa es de sefrir lo batalla \textbf{ que de acometer los lidiadores ¶ } Lo segundo esto es mas guaue & Propter quod difficilius est sustinere pugnam , \textbf{ quam aggredi pugnantes . Secundo est difficilius , } quia aggrediens imaginatur malum ut futurum : \\\hline
1.2.13 & por que aquel que acomete ymagina el mal \textbf{ assi commo cosa que ha de venir . } Mas el que sufre ha el mal ante los oios & quam aggredi pugnantes . Secundo est difficilius , \textbf{ quia aggrediens imaginatur malum ut futurum : } sed sustinens habet malum prae oculis , \\\hline
1.2.13 & e assi conmo presente . \textbf{ Et por ende mas guaue cosa es de esforçar se el omne } e auer se fuertemente contra los males presentes & et ut praesens . \textbf{ Difficilius autem est inniti , } et habere se fortiter contra mala praesentia , \\\hline
1.2.13 & Et por ende mas guaue cosa es de esforçar se el omne \textbf{ e auer se fuertemente contra los males presentes } que contra los males que han de venir & Difficilius autem est inniti , \textbf{ et habere se fortiter contra mala praesentia , } quam contra mala futura . \\\hline
1.2.13 & e auer se fuertemente contra los males presentes \textbf{ que contra los males que han de venir } ¶ lo terçero esto es mas guaue cosa & et habere se fortiter contra mala praesentia , \textbf{ quam contra mala futura . } Tertio hoc est difficilius , \\\hline
1.2.13 & ¶ lo terçero esto es mas guaue cosa \textbf{ por que acometer puede se fazer adesora } mas sofrir requiere mas luengotron . & Tertio hoc est difficilius , \textbf{ quia aggredi potest fieri subito : } sed sustinere requirit diuturnitatem , et tempus . Difficilius est autem habere se fortiter , et constanter in sustinendo bella , \\\hline
1.2.13 & por que acometer puede se fazer adesora \textbf{ mas sofrir requiere mas luengotron . } Et por ende mas guaue cosa es auerse ome fuertemente & quia aggredi potest fieri subito : \textbf{ sed sustinere requirit diuturnitatem , et tempus . Difficilius est autem habere se fortiter , et constanter in sustinendo bella , } quod requirit durabilitatem et tempus , \\\hline
1.2.13 & mas sofrir requiere mas luengotron . \textbf{ Et por ende mas guaue cosa es auerse ome fuertemente } e firmemente en sufriendo las batallas & quia aggredi potest fieri subito : \textbf{ sed sustinere requirit diuturnitatem , et tempus . Difficilius est autem habere se fortiter , et constanter in sustinendo bella , } quod requirit durabilitatem et tempus , \\\hline
1.2.13 & Et adelante dize \textbf{ que la fortaleza es en sofrir las cosas tristes . } Et por ende ya declarado es cerca quales cosas ha de seer la fortaleza . & quam circa audacias moderando ipsas \textbf{ ( et subdit ) Fortitudinem esse in sustinendo tristia . Declaratum est igitur , } circa quae habet esse Fortitudo : \\\hline
1.2.13 & Et por ende ya declarado es cerca quales cosas ha de seer la fortaleza . \textbf{ Pues que assi es fincanos de declarar } en qual manera podemos fazer anos mismos fuertes & circa quae habet esse Fortitudo : \textbf{ restat ergo declarandum , } quomodo possumus facere nos ipsos fortes . Notandum ergo , \\\hline
1.2.13 & Pues que assi es fincanos de declarar \textbf{ en qual manera podemos fazer anos mismos fuertes } Pues que assi es deuen dos notar & restat ergo declarandum , \textbf{ quomodo possumus facere nos ipsos fortes . Notandum ergo , } quod licet virtus opponatur duabus malitiis , \\\hline
1.2.13 & en qual manera podemos fazer anos mismos fuertes \textbf{ Pues que assi es deuen dos notar } e entender & restat ergo declarandum , \textbf{ quomodo possumus facere nos ipsos fortes . Notandum ergo , } quod licet virtus opponatur duabus malitiis , \\\hline
1.2.13 & Pues que assi es deuen dos notar \textbf{ e entender } que commo quier que la uirtud sea contraria . & quomodo possumus facere nos ipsos fortes . Notandum ergo , \textbf{ quod licet virtus opponatur duabus malitiis , } quarum una superabundat , \\\hline
1.2.13 & Et al temor que fallesçe \textbf{ e mengua en acometer . Enpero sienpre la uirtud } mas es contraria a vna de aquellas maliçias & quae deficit : \textbf{ semper tamen virtus plus opponitur uni malitiae , } quam alii . \\\hline
1.2.13 & assi commo \textbf{ mas guaue cosa es de repremir los temores } que refrenar las osadias . & quam alii . \textbf{ Sed quia difficilius est reprimere timores , } quam moderare audacias : \\\hline
1.2.13 & mas guaue cosa es de repremir los temores \textbf{ que refrenar las osadias . } la fortaleza mas es & Sed quia difficilius est reprimere timores , \textbf{ quam moderare audacias : } Fortitudo magis insistit \\\hline
1.2.13 & la fortaleza mas es \textbf{ e mas esta en repremir los temores } que en reftenar las osadias & Fortitudo magis insistit \textbf{ ut reprimat timores , } quam ut moderet audacias : \\\hline
1.2.13 & e mas esta en repremir los temores \textbf{ que en reftenar las osadias } Et mas contradize el temor ala fortaleza & ut reprimat timores , \textbf{ quam ut moderet audacias : } et plus repugnat , et contradicit timor fortitudini , quam faciat audacia . \\\hline
1.2.13 & ¶Pues que assi es \textbf{ porque non podemos en punto alcançar el medio entre la osadia e el temor . } por ende auemos de inclinar nos mas ala osadia & et plus repugnat , et contradicit timor fortitudini , quam faciat audacia . \textbf{ Quia igitur non possumus punctualiter attingere medium } inter audaciam , \\\hline
1.2.13 & porque non podemos en punto alcançar el medio entre la osadia e el temor . \textbf{ por ende auemos de inclinar nos mas ala osadia } por que menos contradize ala fortaleza & Quia igitur non possumus punctualiter attingere medium \textbf{ inter audaciam , } et timorem : declinandum est ad audaciam , quae minus repugnat Fortitudini , \\\hline
1.2.13 & que el temor \textbf{ Et por ende si quisieremos fazer fuertes a nos mismos conuiene de inclinar nos ante ala osadia } que al temor ¶ & inter audaciam , \textbf{ et timorem : declinandum est ad audaciam , quae minus repugnat Fortitudini , } quam timor , \\\hline
1.2.13 & es cerca los otros periglos \textbf{ que pueden conteçer . Et ahun es en refrenando las osadias } e en acometiendo los lidiadores . & ø \\\hline
1.2.13 & ¶ Lo terçero ya declaramos \textbf{ en qual manera podemos fazer a nos mismos fuertes . } Ca mayormente nos podemos fazer fuertes & est circa pericula alia , \textbf{ et in moderando audacias , et in aggrediendo pugnantes . Tertio declaratum fuit , quomodo possumus facere nos ipsos fortes : } quia maxime hoc faciemus , declinando magis ad audaciam , \\\hline
1.2.13 & en qual manera podemos fazer a nos mismos fuertes . \textbf{ Ca mayormente nos podemos fazer fuertes } si mas nos inclinaremos ala osadia & et in moderando audacias , et in aggrediendo pugnantes . Tertio declaratum fuit , quomodo possumus facere nos ipsos fortes : \textbf{ quia maxime hoc faciemus , declinando magis ad audaciam , } quae non tantum repugnat fortitudini , \\\hline
1.2.14 & commo el temor \textbf{ euedes saber } que el philosofo en el terçer libdelas ethicas en el capitulo dela fortaleza de parte siere espeçias o . siete manas de fortaleza ¶ & sicut timor . \textbf{ Distinguit Philosophus 3 Ethicorum cap’ de fortitudine septem species , } seu septem maneries Fortitudinis . Prima , \\\hline
1.2.14 & quando alguno temiendo uerguença \textbf{ e quariendo ganar honrra acomete alguna cosa fuerte e espantable . } Onde dize el philosofo & quando aliquis timens verecundiam , \textbf{ et volens honorem adipisci , | aggreditur aliquod terribile , } unde ait Philosophus , \\\hline
1.2.14 & que segunt esta manera de fortaleza \textbf{ aquellos son dichos muy fuertes que quieren gauar honrra entre aquellas gentes . } Entre las quales los temerosos son desonrrados & secundum hanc Fortitudinem fortissimi videntur esse \textbf{ apud illas gentes , } apud quas timidi inhonorati sunt , fortes vero honorantur . Hoc modo \\\hline
1.2.14 & delan otra parte auia manera \textbf{ para le dezir muchos denuestos . } Et por ende temiendo qual denostaria su contrario era fuerte . & ø \\\hline
1.2.14 & do non son conosçidos acometen alguas torpedades \textbf{ las quales non quarrian acometer nin tentar entre los sus çibdadanos en ningunan manera } e entre los sus conosçientes ¶ & et quaerit honores . Videmus enim aliquos , \textbf{ quum sunt in ignotis partibus , committere aliqua turpia , quae inter ciues et notos nullatenus attentarent . } Secunda Fortitudo dicitur seruilis , \\\hline
1.2.14 & alsi commo \textbf{ quando alguno non por esquiuar denuestos } o por gauar honrras & quam prima : \textbf{ ut cum aliquis non ut vitet opprobria , } vel ut consequatur honores : \\\hline
1.2.14 & quando alguno non por esquiuar denuestos \textbf{ o por gauar honrras } mas por temor de pena o enduzido & ut cum aliquis non ut vitet opprobria , \textbf{ vel ut consequatur honores : } sed timore poenae , \\\hline
1.2.14 & que en tal manera fuese atormentado \textbf{ que non pudiese fazerfuyr los canes . } Et a esta manera de fortaleza enduzen los pueblos los caudiellos dela hueste & taliter aptaretur \textbf{ quod non esset sufficiens fugare canes . } Ad hanc Fortitudinem inducunt populum Duces exercitus , statuentes poenam fugientibus , \\\hline
1.2.14 & e faziendo cauas e cercas . \textbf{ por que non puedan foyr las conpannas } mas que esten costrenidas para lidiar & faciendo foueas , \textbf{ ne possit exercitus fugere , sed quadam necessitate bellare cogatur . } Hoc autem modo quidam Dux dicitur exercitum suum coegisse ad Fortitudinem . Nam , \\\hline
1.2.14 & por que non puedan foyr las conpannas \textbf{ mas que esten costrenidas para lidiar } por alguna necesidat . & faciendo foueas , \textbf{ ne possit exercitus fugere , sed quadam necessitate bellare cogatur . } Hoc autem modo quidam Dux dicitur exercitum suum coegisse ad Fortitudinem . Nam , \\\hline
1.2.14 & e con todas sus naues passase la mar \textbf{ por que ninguno de sus conpannas non ouiese manera de fuyr quebranto todas las naues . } Et assi fizo lidiar e vençer a los suyos . & et cum toto suo exercitu transfretaret , \textbf{ ne aliquis de suo exercito haberet materiam fugiemdi , | omnes naues confregit . } Tertia fortitudo dicitur militaris , \\\hline
1.2.14 & por que ninguno de sus conpannas non ouiese manera de fuyr quebranto todas las naues . \textbf{ Et assi fizo lidiar e vençer a los suyos . } ¶ La terçera fortaleza es dicha caualleril o de caualleros & omnes naues confregit . \textbf{ Tertia fortitudo dicitur militaris , } et haec est fortitudo experientiae . Milites enim propter experientiam \\\hline
1.2.14 & en el libro del fecho dela caualleria \textbf{ ninguno non duda de fazer } e de acometer & Nam ( ut dicit Vegetius in libro De re militari ) , \textbf{ Nullus attentare dubitat , } quod se bene didicisse confidit . Videmus enim aliquos audientes solum strepitum armorum fugiunt , nescientes discernere \\\hline
1.2.14 & ninguno non duda de fazer \textbf{ e de acometer } aquello & Nam ( ut dicit Vegetius in libro De re militari ) , \textbf{ Nullus attentare dubitat , } quod se bene didicisse confidit . Videmus enim aliquos audientes solum strepitum armorum fugiunt , nescientes discernere \\\hline
1.2.14 & e fiade ssi que lo ha bien aprendido . \textbf{ Ca veemos que algunos quando oyen solamente vn rroydo delas armas fuyen non sabiendo departir } nin conosçer quales son los periglos delas batallas & Nullus attentare dubitat , \textbf{ quod se bene didicisse confidit . Videmus enim aliquos audientes solum strepitum armorum fugiunt , nescientes discernere } quae sunt periculosa in bellis , \\\hline
1.2.14 & Ca veemos que algunos quando oyen solamente vn rroydo delas armas fuyen non sabiendo departir \textbf{ nin conosçer quales son los periglos delas batallas } e quales non . & quod se bene didicisse confidit . Videmus enim aliquos audientes solum strepitum armorum fugiunt , nescientes discernere \textbf{ quae sunt periculosa in bellis , } et quae non . \\\hline
1.2.14 & que la su praeua \textbf{ que han en las armas tornan se a fuyr¶ } La quarta fortaleza es rauiosa et de grant saña . & quod excedat eorum experientiam , \textbf{ in fugam conuertuntur . } Quarto Fortitudo dicitur furiosa . \\\hline
1.2.14 & Ca quando esta sin temor \textbf{ por razon dela sana passada la sanna en comienca de temer } Et assi non sufre la batalla mas encomienca luego de fuyr¶ La quinta fortaleza es acostunbrada e ganada por costunbre . & quia qui fit increpidus propter furorem , \textbf{ tranquillato furore , | incipit timere , } et non sustinet bellum , \\\hline
1.2.14 & por razon dela sana passada la sanna en comienca de temer \textbf{ Et assi non sufre la batalla mas encomienca luego de fuyr¶ La quinta fortaleza es acostunbrada e ganada por costunbre . } Ca alguons & incipit timere , \textbf{ et non sustinet bellum , | sed arripit fugam . Quinta Fortitudo est consuetudinalis . Aliqui enim , } quia in pluribus bellis fuerunt , \\\hline
1.2.14 & e toman alguna esꝑança de victoria \textbf{ e toman osadia para lidiar } Pues que assi es tales semeian fuertes & quandam spem victoriae , \textbf{ et accipiunt quandam audaciam bellandi . Tales ergo fortes esse videntur , } quia aggrediuntur pugnam , sperantes de victoria , \\\hline
1.2.14 & et sufrieren algun mal contra su esperança \textbf{ luego comiençan a fuyr ¶ } La sexta fortaleza es testial & quia si inueniant resistentiam , \textbf{ et contra sperata malum aliquod patiantur , fugiunt . } Sexta fortitudo dicitur esse bestialis , \\\hline
1.2.14 & e de bestia assi commo \textbf{ quando alguno comiença de lidiar } non sabiendo & ø \\\hline
1.2.14 & que aya auido de batalla \textbf{ o por non saber el poder de sus enemigos acomete la batalla . } Mas acomete la & ut cum aliquis non coactione , vel furore , vel propter experientiam , \textbf{ vel propter ignorantiam aggreditur bellum : } sed propter bonum , \\\hline
1.2.14 & Et pues que assi es los Reyes \textbf{ e los prinçipes deuen saber estas maneras de fortaleza } que son dichas & et ex electione . \textbf{ Reges ergo et Principes licet has maneries fortitudinum scire debeant , } ut cognoscant qualiter populus suus fortis est , \\\hline
1.2.14 & por que sepan en qual manera han de ser fuertes . \textbf{ Et en commo pueden lidiar con sus enemigos . } Enpero ellos deuen ser fuertes de fortaleza uirtuosa & ut cognoscant qualiter populus suus fortis est , \textbf{ et quomodo possunt cum aduersariis bellare : } ipsi tamen debent esse fortes fortitudine virtuosa , \\\hline
1.2.14 & si non quando ouieren razon derecha \textbf{ para auer batalla . } Et si non vieren & nisi habeant iusta bella , \textbf{ et nisi videant magnum bonum patriae } vel regni , \\\hline
1.2.14 & Et si non vieren \textbf{ que se puede seguir muy grant bien ala tierra o al regno } por aquella batalla que acometen . & et nisi videant magnum bonum patriae \textbf{ vel regni , } quod ex tali bello consequi possit . \\\hline
1.2.15 & por aquella batalla que acometen . \textbf{ euedes saber } que la tenprança entre las quatro uirtudes cardinales tiene el postrimer grado . & quod ex tali bello consequi possit . \textbf{ Temperantia inter virtutes cardinales ultimum gradum tenet . } Prudentia enim et Iustitia principaliores sunt virtutibus moralibus ; \\\hline
1.2.15 & Mas entre todas las cosas \textbf{ que nos pueden tirar e enbargar } que non siguamos el bien de tazon & Inter caetera autem , \textbf{ quae possunt nos retrahere a bono rationis , } sunt timores belli , \\\hline
1.2.15 & Et dela fortaleza que es mas prinçipal que la tenperança . \textbf{ finca . nos de dezir dela tenperança } que entre las uirtudes prinçipales tiene el postrimero grado . & quae est principalior Temperantia . Restat dicere de ipsa Temperantia , quae \textbf{ inter virtutes principales } ultimum gradum tenet . Sciendum ergo , \\\hline
1.2.15 & que entre las uirtudes prinçipales tiene el postrimero grado . \textbf{ Et pues que assi es deuedes saber } que assi commo la fortaleza es medianera entre los temores e las osadias . & inter virtutes principales \textbf{ ultimum gradum tenet . Sciendum ergo , } quod sicut Fortitudo media est inter timores , \\\hline
1.2.15 & que teme \textbf{ lo que deue temer } e es osado & sed qui timet timenda , \textbf{ et audet audenda . } Sic Temperantia , \\\hline
1.2.15 & Et montanes a aquel que fuye de todas las delecta conn escorporales . \textbf{ Et pues que assi es el non sentirse es foyr delas delecta con nessenssibles e corporales } mas que la razon manda & Vocamus autem insensibilem et agrestem , \textbf{ qui omnes delectationes corporales fugit . Insensibilitas ergo est ultra , } quam ratio dictet , \\\hline
1.2.15 & Mas aquel que fuye e escusa las cosas \textbf{ que ha de escusar . } Et sigue las cosas que ha de segnir aquel es uirtuoso e tenprado . & Qui vero fugit fugienda , \textbf{ et prosequitur prosequenda : virtuosus , et temperatus est . } contingit autem peccare non solum delectationes sensibiles prosequendo , \\\hline
1.2.15 & que ha de escusar . \textbf{ Et sigue las cosas que ha de segnir aquel es uirtuoso e tenprado . } Ca contesçe alos omes de pecar & Qui vero fugit fugienda , \textbf{ et prosequitur prosequenda : virtuosus , et temperatus est . } contingit autem peccare non solum delectationes sensibiles prosequendo , \\\hline
1.2.15 & Et sigue las cosas que ha de segnir aquel es uirtuoso e tenprado . \textbf{ Ca contesçe alos omes de pecar } non solamente signiendo las delecta connes corporales & et prosequitur prosequenda : virtuosus , et temperatus est . \textbf{ contingit autem peccare non solum delectationes sensibiles prosequendo , } sed etiam eas fugiendo . \\\hline
1.2.15 & mas ahun fuyendo dellas . \textbf{ Ca aquel que del todo faze abstinençia del comer e del beuer } e delas otras delecta connes corporales & sed etiam eas fugiendo . \textbf{ Nam qui adeo abstineret a cibo et potu , } et a licitis delectationibus , \\\hline
1.2.15 & e refrena las delecta conn escorpora \textbf{ les conuiene nos de saber } çerca quałs delecta connes corporales & sic temperantia est reprimens delectationes sensibiles , et moderans insensibilitates , \textbf{ inter quas habet esse . } Si ergo Temperantia reprimit delectationes sensibiles , videndum est circa quas delectationes sensibiles habet esse . \\\hline
1.2.15 & ha de seer la tenprança . \textbf{ Et deuedes saber } que alguas delas delectaconnes corporales son fuertes & Si ergo Temperantia reprimit delectationes sensibiles , videndum est circa quas delectationes sensibiles habet esse . \textbf{ Delectationum autem sensibilium quaedam sunt fortes , } quaedam sunt debiles , \\\hline
1.2.15 & Ca maguera sean çinco los sesos \textbf{ e nos contezça de tomar delectaconn } en las cosas senssibles & quaedam minus . Licet enim sint quinque sensus , \textbf{ et circa sensibilia omnium sensuum contingat nos delectari : } tamen fortiores delectationes sunt \\\hline
1.2.15 & Empero las mas fuertes delecta connes son en el gusto e en el tannimiento \textbf{ que del veer e del oyr e del oler la qual cosa podemos prouar } por dos maneras o por dos razones & secundum gustum et tactum , \textbf{ quam secundum visum , } auditum , \\\hline
1.2.15 & que en los otros . \textbf{ Ca podemos veer e oyr e oler cosas que estan arredradas de nos . } mas non podemos gostar nin tanner & sed non possumus gustare , \textbf{ et tangere , } nisi nobis coniuncta . \\\hline
1.2.15 & Ca podemos veer e oyr e oler cosas que estan arredradas de nos . \textbf{ mas non podemos gostar nin tanner } si non las cosas que son ayuntadas anos . & et tangere , \textbf{ nisi nobis coniuncta . } Ideo in huiusmodi sensibilibus arditius , \\\hline
1.2.15 & La segunda razon \textbf{ por que podemos prouar esso mismo es } por que las cosas senssibles del gusto e del tannimiento & et feruentius delectamur . \textbf{ Secundo hoc idem patet : } quia sensibilia gustus , \\\hline
1.2.15 & e cerca cosa guaue \textbf{ prinçipalmente es de poner la tenpranca çerca aquellas delectaçonnes } delas quales nos arredramos con mayor g̃ueza & et \textbf{ ut saluaretur species . Quare si virtus est circa bonum et delectabile , ponenda est principaliter Temperantia circa delectationes illas , } a quibus est difficilius abstinere . \\\hline
1.2.15 & por razon que los otros sesos \textbf{ por algun accidente resçiben delecta connes del gostar e del tanner . } Onde dize el philosofo & hoc est per accidens : \textbf{ quia aliis sensus per accidens percipiunt delectabilia gustus , } et tactus . Est huiusmodi virtus \\\hline
1.2.15 & en las quales partiçipan todas las aialias . \textbf{ Mas laso trisa inalias se delectan engostar } e en el taner por si . & circa delectabilia illa , \textbf{ in quibus reliqua animalia communicant . } Reliqua autem animalia in gustu , \\\hline
1.2.15 & Mas laso trisa inalias se delectan engostar \textbf{ e en el taner por si . } Et en los otros sesos se delectan & in quibus reliqua animalia communicant . \textbf{ Reliqua autem animalia in gustu , } et tactu delectantur per se : in aliis vero sensibus delectantur per accidens . Propter quod in eodem libro dicitur , \\\hline
1.2.15 & con la vision del çierto \textbf{ si non en quanto por el veer } e por el oyr conosce & nec visione cerui , \textbf{ nisi inquantum per visum , } et auditum , \\\hline
1.2.15 & si non en quanto por el veer \textbf{ e por el oyr conosce } que estan cerca el bue e el çieruo & nisi inquantum per visum , \textbf{ et auditum , } cognoscit eos prope esse , \\\hline
1.2.15 & e se fazen \textbf{ por el seso del gostar son ordenadas aguarda dela persona . } Mas las delecta conns del matermonio & Nam delectationes nutrimentales quae fiunt per gustum , ordinantur ad conseruationem propriae personae : \textbf{ sed delectationes matrimoniales quae fiunt per tactum , ordinantur ad procreationem , } filiorum , \\\hline
1.2.15 & si contra la mas guaue cosa deuemos \textbf{ mas lidiarmas prinçipalmente deuemos trabaiar } que por la tenpranca refrenemos las delectaçiones carnales & et ad conseruationem speciei . Si ergo contra difficilius magis bellandum est , \textbf{ principalius insistendum est , } ut per Temperantiam refraenemus delectationes venereas \\\hline
1.2.15 & quando el comer \textbf{ e el beuer llega ala garganta } que quando llega ala lengua . & ubi non est gustus , \textbf{ sed tactus . } Et \\\hline
1.2.15 & que quando llega ala lengua . \textbf{ Et estas cosas deuemos creer } al iuizio de los golosos . & sed tactus . \textbf{ Et } ( ut manifeste experimur ) magis delectamur , cum cibus aut potus attingit guttur , quam cum coniungitur linguae . Credendum est enim in talibus iudicio gulosorum . Recitat autem Philosophus Ethicorum 3 de quodam , \\\hline
1.2.15 & mas luenga que garganta de grulla \textbf{ por que podiese tomar mayor delecta conn en ellas . } mas non rogo & orauit , \textbf{ ut guttur eius longius quam gruis fieret . } Non enim orauit , \\\hline
1.2.15 & Mas por algun accidente es cerca las cosas delectables de los otros sesos . \textbf{ Et assi podemos tomar de ligero las maneras dela tenpranca . } Ca si parte nesçe ala tenpranca & et ex consequenti circa gustum : per accidens autem est circa delectabilia aliorum sensuum . Species autem temperantiae de leui sumere possumus . \textbf{ Nam si spectat ad Temperantiam reprimere delectationes nutrimentales , } et venereas : \\\hline
1.2.15 & Ca si parte nesçe ala tenpranca \textbf{ de repremirlas delecta connes matermentales con que seca el cuerpo nos refrenaremos estas delectaconnes nutermentales } que can el cuerpo sy fuermos mesurados e abstinentes en el comer e en el beuer . & et venereas : \textbf{ quia delectationes nutrimentales reprimimus , } si fuerimus sobrii , \\\hline
1.2.15 & de repremirlas delecta connes matermentales con que seca el cuerpo nos refrenaremos estas delectaconnes nutermentales \textbf{ que can el cuerpo sy fuermos mesurados e abstinentes en el comer e en el beuer . } Mas las delecta connes carnales abaxaremos & quia delectationes nutrimentales reprimimus , \textbf{ si fuerimus sobrii , | et abstinentes : } venereas vero , \\\hline
1.2.15 & o quatro son las maneras della . \textbf{ Conuiene a saber ¶ Mesura ¶astinençia ¶ Castidat ¶ Et linpieza . } Ca si quisieremos repremir las delectaçonnes nutermentales & siue quatuor erunt species ipsius ; \textbf{ videlicet , | sobrietas , abstinentia , castitas , et pudicitia . } Nam si volumus nutrimentales delectationes reprimere , \\\hline
1.2.15 & Conuiene a saber ¶ Mesura ¶astinençia ¶ Castidat ¶ Et linpieza . \textbf{ Ca si quisieremos repremir las delectaçonnes nutermentales } con que se cera el cuerpo . & sobrietas , abstinentia , castitas , et pudicitia . \textbf{ Nam si volumus nutrimentales delectationes reprimere , } oportet nos temperari a potu , et cibo . Temperando nos a potu , sumus sobrii : \\\hline
1.2.15 & con que se cera el cuerpo . \textbf{ Conuiene nos que seamos tenprados en comer e en beuer } ca tenprando nos en el beuer seremos mesurados & Nam si volumus nutrimentales delectationes reprimere , \textbf{ oportet nos temperari a potu , et cibo . Temperando nos a potu , sumus sobrii : } sicut \\\hline
1.2.15 & Conuiene nos que seamos tenprados en comer e en beuer \textbf{ ca tenprando nos en el beuer seremos mesurados } assi commo aquellos que sobrepuian en el beuer son beodos & oportet nos temperari a potu , et cibo . Temperando nos a potu , sumus sobrii : \textbf{ sicut } qui excedunt in potando , sunt ebrii . Temperando vero nos a cibo , sumus abstinentes . Abstinentia ergo , \\\hline
1.2.15 & ca tenprando nos en el beuer seremos mesurados \textbf{ assi commo aquellos que sobrepuian en el beuer son beodos } Mas nos tenprando nos en el comer lo mos astinentes . & sicut \textbf{ qui excedunt in potando , sunt ebrii . Temperando vero nos a cibo , sumus abstinentes . Abstinentia ergo , } et sobrietas deseruiunt in reprimendo delectationes nutrimentales . \\\hline
1.2.15 & assi commo aquellos que sobrepuian en el beuer son beodos \textbf{ Mas nos tenprando nos en el comer lo mos astinentes . } Et pues que assi es la astinençia et la mesura en el beuer & sicut \textbf{ qui excedunt in potando , sunt ebrii . Temperando vero nos a cibo , sumus abstinentes . Abstinentia ergo , } et sobrietas deseruiunt in reprimendo delectationes nutrimentales . \\\hline
1.2.15 & Mas nos tenprando nos en el comer lo mos astinentes . \textbf{ Et pues que assi es la astinençia et la mesura en el beuer } siruen en abayar las delectaçonnes nutermentales & qui excedunt in potando , sunt ebrii . Temperando vero nos a cibo , sumus abstinentes . Abstinentia ergo , \textbf{ et sobrietas deseruiunt in reprimendo delectationes nutrimentales . } Sed castitas , \\\hline
1.2.15 & Et pues que assi es la astinençia et la mesura en el beuer \textbf{ siruen en abayar las delectaçonnes nutermentales } en que se el omne & qui excedunt in potando , sunt ebrii . Temperando vero nos a cibo , sumus abstinentes . Abstinentia ergo , \textbf{ et sobrietas deseruiunt in reprimendo delectationes nutrimentales . } Sed castitas , \\\hline
1.2.15 & Et por ende conuiene aquel que uerdaderamente es tenprado \textbf{ non vsar de delecta conn escarnales . } nin de delecta connes del gusto . & et pudicitia venereas delectationes refraenant . \textbf{ Oportet enim vere temperatum non exercere opera venerea , } neque gestus . \\\hline
1.2.15 & que dichas son de ligero paresçe \textbf{ commo nos mismos nos podemos fazer tenprados . } Ca la tenpranca & His visis de leui patet , \textbf{ quomodo nosipsos facere possumus temperatos . } Nam Temperantia , \\\hline
1.2.15 & assi commo en manera contraria la vna dela otra . \textbf{ Ca la fortaleza es en acometer las cosas espantables } mas la tenprança esen tener se & et Fortitudo \textbf{ quasi e contrario se habent . Fortitudo enim est in aggrediendo terribilia : } sed Temperantia in retrahendo se ab his , quae sunt delectabilia . \\\hline
1.2.15 & Ca la fortaleza es en acometer las cosas espantables \textbf{ mas la tenprança esen tener se } e alongar se de aquellas cosas & quasi e contrario se habent . Fortitudo enim est in aggrediendo terribilia : \textbf{ sed Temperantia in retrahendo se ab his , quae sunt delectabilia . } Sicut ergo Fortitudo magis conuenit cum audacia ; \\\hline
1.2.15 & mas la tenprança esen tener se \textbf{ e alongar se de aquellas cosas } que son delectables . & quasi e contrario se habent . Fortitudo enim est in aggrediendo terribilia : \textbf{ sed Temperantia in retrahendo se ab his , quae sunt delectabilia . } Sicut ergo Fortitudo magis conuenit cum audacia ; \\\hline
1.2.15 & assi conmo la fortaleza mas conuiene con la osadi \textbf{ Et si nos quisieremos fazer nos fuertes } mas auemos aser osados e temerosos . & Sicut ergo Fortitudo magis conuenit cum audacia ; \textbf{ et si volumus esse fortes , } debemus magis esse audaces , \\\hline
1.2.15 & que con la senssiblidat de los sesos . \textbf{ Et por ende si nos quisieremos fazer } a nos mismos tenprados deuemos declinara aquella parte & cum insensibilitate . \textbf{ Si ergo volumus nosipsos facere temperatos , } ad illam partem declinandum est , \\\hline
1.2.15 & a nos mismos tenprados deuemos declinara aquella parte \textbf{ por que nos podemos guardar delas delectaçonnes de los sesos . } Et por ende meior es fuyr a algunas delecta conns et viçios & ad illam partem declinandum est , \textbf{ ut a delectationibus sensibilibus caueamus . Melius est enim aliquas delectationes etiam licitas vitare , } quam aliquas illicitas insequi . Declarata igitur sunt illa quatuor , \\\hline
1.2.15 & por que nos podemos guardar delas delectaçonnes de los sesos . \textbf{ Et por ende meior es fuyr a algunas delecta conns et viçios } avn que sean conuenibles & ad illam partem declinandum est , \textbf{ ut a delectationibus sensibilibus caueamus . Melius est enim aliquas delectationes etiam licitas vitare , } quam aliquas illicitas insequi . Declarata igitur sunt illa quatuor , \\\hline
1.2.15 & e tienpra los non sentimientos \textbf{ ¶L segundo auemos declarado cerca quales cosas ha descer la tenpranca . } Ca prinçipalmente es çerca el tannimiento & quia est virtus reprimens delectationes sensibiles , \textbf{ et moderans insensibilitates . Secundo vero declarabatur circa quae habet esse : } quia principaliter est circa tactum , \\\hline
1.2.15 & que refrenan las delectaconnes delas viandas \textbf{ que comemos assi commo es mesura en beuer . } Et astinençia que es refrenanca en el comer . & duae moderantes delectationes nutrimentales , \textbf{ ut sobrietas , } et abstinentia , \\\hline
1.2.15 & ¶Lo quarto declaramos \textbf{ en qual manera podemos fazer a nos mismos tenprados . } Ca esto podemos fazer mayormente & Quarto vero declaratum fuit , \textbf{ quomodo possumus nosipsos facere temperatos : } quia hoc maxime faciemus a delectactionibus abstinendo . Debemus enim \\\hline
1.2.15 & en qual manera podemos fazer a nos mismos tenprados . \textbf{ Ca esto podemos fazer mayormente } si nos guardaremos e arredraremos de las cosas delectables de los sesos . & quomodo possumus nosipsos facere temperatos : \textbf{ quia hoc maxime faciemus a delectactionibus abstinendo . Debemus enim } secundum Philosophum Ethicorum \\\hline
1.2.15 & si nos guardaremos e arredraremos de las cosas delectables de los sesos . \textbf{ Ca segunt el philosofo en el segundo delas ethicas deuemos fazer } lo que fezieron los bieios de troya contra elena & quia hoc maxime faciemus a delectactionibus abstinendo . Debemus enim \textbf{ secundum Philosophum Ethicorum } 2 hoc pati , quod senes Troiae patiebantur , ad Helenam dicentes : \\\hline
1.2.15 & que era vna fermosa muger \textbf{ e dixieron echemos la de nos que quieredezir tanto commo non la catemos nin la veamos . } rueua el philosofo en el terçero libro delas ethicas & 2 hoc pati , quod senes Troiae patiebantur , ad Helenam dicentes : \textbf{ Abiiciamus eam , id est , | non respiciamus in ipsam . } Probat Philosophus 3 Ethic’ quatuor rationibus , \\\hline
1.2.16 & mas de voluntad \textbf{ tanto es mas de denostar } Otrosi & Nam quanto magis aliquis voluntarie peccat , \textbf{ tanto magis est increpandus . } Rursus quanto aliquis facilius potest benefacere , \\\hline
1.2.16 & Otrosi \textbf{ quando alguno mas ligeramente puede fazer bien } e non lo faze & tanto magis est increpandus . \textbf{ Rursus quanto aliquis facilius potest benefacere , } si non benefaciat , \\\hline
1.2.16 & e non lo faze \textbf{ mas es de denostar e de reprehender . } Et por ende el que no es tenprado & si non benefaciat , \textbf{ magis est detestandus , et reprehensibilis . Intemperatus ergo magis est detestandus , } et reprehensibilis : \\\hline
1.2.16 & Et por ende el que no es tenprado \textbf{ es mas de denostar e de reprehender } que el temeroso¶ & magis est detestandus , et reprehensibilis . Intemperatus ergo magis est detestandus , \textbf{ et reprehensibilis : } tum quia magis voluntarie peccat , \\\hline
1.2.16 & ¶Lo otro \textbf{ por que mas ligeramente puede bien fazer } e ganar tenprança que fortaleza . & tum quia magis voluntarie peccat , \textbf{ tum etiam quia facilius est ei facere bonum , } et acquirere temperantiam , \\\hline
1.2.16 & por que mas ligeramente puede bien fazer \textbf{ e ganar tenprança que fortaleza . } Mas que el destenprado pequemas de voluntad & tum etiam quia facilius est ei facere bonum , \textbf{ et acquirere temperantiam , } quam sit acquirere fortitudinem . \\\hline
1.2.16 & Mas que el destenprado pequemas de voluntad \textbf{ que el temeroso puede se demostrar } por dos razons¶ & et acquirere temperantiam , \textbf{ quam sit acquirere fortitudinem . } Quod autem magis voluntarie peccet intemperatus quam timidius dupliciter ostendi potest . Primo , \\\hline
1.2.16 & por dos razons¶ \textbf{ La primera es por que segnir plazenterias desfenpradas es cosa delectable } Mas fuyr e temer es cosatste . & quam sit acquirere fortitudinem . \textbf{ Quod autem magis voluntarie peccet intemperatus quam timidius dupliciter ostendi potest . Primo , } quia insequi voluntates intemperatas , est delectabile : fugere autem et timere , est tristabile . \\\hline
1.2.16 & La primera es por que segnir plazenterias desfenpradas es cosa delectable \textbf{ Mas fuyr e temer es cosatste . } Et mas de uoluntad faze & Quod autem magis voluntarie peccet intemperatus quam timidius dupliciter ostendi potest . Primo , \textbf{ quia insequi voluntates intemperatas , est delectabile : fugere autem et timere , est tristabile . } Magis quis voluntarie agit quod facit cum delectatione , \\\hline
1.2.16 & que el peca con temor ¶ \textbf{ Lo segundo esto mesmo se puede prouar } por otra razon . & quam qui peccat ex timore . \textbf{ Secundo hoc idem patet : } quia ( ut ait Philosophus ) timor obstupefacit , \\\hline
1.2.16 & e faze al omne \textbf{ que se non puede mouer } e que finque espantado la qual cola non fazela delectaçion & quia ( ut ait Philosophus ) timor obstupefacit , \textbf{ et reddit naturam immobilem , } et attonitam : \\\hline
1.2.16 & por uoluntad ñcon delibramiento . \textbf{ Et por ende mas de foyr } e de escusares de pecar & est quasi extra se , \textbf{ nec voluntarie et deliberate agit quod agit . } Tolerabilius est igitur peccare per timorem , \\\hline
1.2.16 & Et por ende mas de foyr \textbf{ e de escusares de pecar } por temor o por miedo & nec voluntarie et deliberate agit quod agit . \textbf{ Tolerabilius est igitur peccare per timorem , } quam per intemperantiam : cum hoc sit magis voluntarium , \\\hline
1.2.16 & que el destenprado en dos maneras es \textbf{ mas de denostar } que el temeroso & Patet ergo intemperatum dupliciter \textbf{ esse magis increpandum , } quam timidum : \\\hline
1.2.16 & porque mas de uoluntad faze mal que el temeroso Otrosi \textbf{ por que mas ligeramente puede bien fazer } e ganar tenpranca & quam timidum : \textbf{ quia magis voluntarie male agit . } Sic etiam dupliciter potest ostendi ipsum esse magis increpandum : \\\hline
1.2.16 & por que mas ligeramente puede bien fazer \textbf{ e ganar tenpranca } que el temeroso fortaleza . & ø \\\hline
1.2.16 & que el temeroso fortaleza . \textbf{ Et assi en dos maneras se puede prouar } e mostrar que el destenprado es mas de denostar & quia magis voluntarie male agit . \textbf{ Sic etiam dupliciter potest ostendi ipsum esse magis increpandum : } quia facilius potest benefacere . \\\hline
1.2.16 & Et assi en dos maneras se puede prouar \textbf{ e mostrar que el destenprado es mas de denostar } por que mas ligeramente puede fazer bien . & quia magis voluntarie male agit . \textbf{ Sic etiam dupliciter potest ostendi ipsum esse magis increpandum : } quia facilius potest benefacere . \\\hline
1.2.16 & e mostrar que el destenprado es mas de denostar \textbf{ por que mas ligeramente puede fazer bien . } Ca mas ligeramente puede ganar la tenprança & Sic etiam dupliciter potest ostendi ipsum esse magis increpandum : \textbf{ quia facilius potest benefacere . } Facilius enim potest acquiri temperantia , \\\hline
1.2.16 & por que mas ligeramente puede fazer bien . \textbf{ Ca mas ligeramente puede ganar la tenprança } que el temeroso la fortaleza & quia facilius potest benefacere . \textbf{ Facilius enim potest acquiri temperantia , } quam fortitudo . \\\hline
1.2.16 & e tirando nos delas delecta connes senssibles . \textbf{ Mas la fortaleza podemos ganar } acometiendo las cosas muy espantables & et retrahendo nos a delectationibus sensibilibus : \textbf{ fortitudinem vero acquirere possumus , } aggrediendo terribilia , \\\hline
1.2.16 & e prounado la batalla \textbf{ mas tirar se el omne delas delecta connes senssibles puede se fazer sin todo periglo } mas acometer las cosas espantables & et experiendo pugnam : \textbf{ retrahi autem a delectationibus , potest fieri sine omni periculo : sed aggredi terribilia , } et experiri bellum , sine periculo non potest . \\\hline
1.2.16 & mas tirar se el omne delas delecta connes senssibles puede se fazer sin todo periglo \textbf{ mas acometer las cosas espantables } e puar las batallas non se puede fazer sin periglo . & et experiendo pugnam : \textbf{ retrahi autem a delectationibus , potest fieri sine omni periculo : sed aggredi terribilia , } et experiri bellum , sine periculo non potest . \\\hline
1.2.16 & mas acometer las cosas espantables \textbf{ e puar las batallas non se puede fazer sin periglo . } Et pues que assi es mucho es de denostar & retrahi autem a delectationibus , potest fieri sine omni periculo : sed aggredi terribilia , \textbf{ et experiri bellum , sine periculo non potest . } Valde est ergo increpandus carens tempesantia , \\\hline
1.2.16 & e puar las batallas non se puede fazer sin periglo . \textbf{ Et pues que assi es mucho es de denostar } el que non ha tenpranca & et experiri bellum , sine periculo non potest . \textbf{ Valde est ergo increpandus carens tempesantia , } cum eam sine periculo possit acquirere : \\\hline
1.2.16 & el que non ha tenpranca \textbf{ commo puede ganar la tenpranca sin ningun periglo . } Mas non es tanto de denostas & Valde est ergo increpandus carens tempesantia , \textbf{ cum eam sine periculo possit acquirere : } non autem adeo increpandus est carens fortitudine , \\\hline
1.2.16 & e se gana con mayor periglo . \textbf{ Otrosi mas ligeramente se puede ganar la tenpranca } que la fortaleza & et cum maiori periculo acquiritur . \textbf{ Rursus facilius acquiri potest temperantia , } quam fortitudo : \\\hline
1.2.16 & non nos podemos \textbf{ assi acostunbrar alasobras dela fortaleza } commo ala sobrde tenpranca & Sed forte toto tempore vitae hominis , non occurrit ei unum iustum bellum . \textbf{ Non ergo sic possumus assuefieri ad opera fortitudinis , } sicut ad opera temperantiae : \\\hline
1.2.16 & por la qual cosa \textbf{ mas de denostar son los omes } por non ser tenprados & sicut ad opera temperantiae : \textbf{ quare exprobrabilius est nos esse intemperatos , } quam non esse fortes . \\\hline
1.2.16 & por ello . \textbf{ Et es mas de denostar } si fuer deste prado e segnidor de passiones . & et non esse constantem animo est exprobrabile , \textbf{ patet quod est exprobrabilius ipsum esse intemperatum , } et insecutorem passionum . Possumus \\\hline
1.2.16 & si fuer deste prado e segnidor de passiones . \textbf{ Enpero podemos aduzir nueuas razones } para prouar & et insecutorem passionum . Possumus \textbf{ tamen nouas rationes adducere , ostendentes , } quam detestabile sit , Regem intemperatum esse . Tangit enim Philosophus 3 Ethic’ \\\hline
1.2.16 & Enpero podemos aduzir nueuas razones \textbf{ para prouar } que mucho es de denostar el Rey & et insecutorem passionum . Possumus \textbf{ tamen nouas rationes adducere , ostendentes , } quam detestabile sit , Regem intemperatum esse . Tangit enim Philosophus 3 Ethic’ \\\hline
1.2.16 & para prouar \textbf{ que mucho es de denostar el Rey } sinon fuer tenprado . & ø \\\hline
1.2.16 & ca tres cosas \textbf{ delas quales podemos tomar tres razones } para prouar & quam detestabile sit , Regem intemperatum esse . Tangit enim Philosophus 3 Ethic’ \textbf{ de ipsa intemperantia tria , } ex quibus tres rationes sumi possunt , quod maxime decet Reges et Principes temperatos esse . Est enim intemperantia \\\hline
1.2.16 & delas quales podemos tomar tres razones \textbf{ para prouar } que mucho conuiene alos Reyes et alos prinçipes de ser tenprados . & de ipsa intemperantia tria , \textbf{ ex quibus tres rationes sumi possunt , quod maxime decet Reges et Principes temperatos esse . Est enim intemperantia } ( ut ibidem tangitur ) \\\hline
1.2.16 & segunt las quales cosas es cosa comun anos \textbf{ e alas bestias de nos delectar . } Et por ende en el terçero libro delas ethicas dize el philosofo & secundum quae , \textbf{ delectari commune est nobis , } et brutis . Ideo 3 Ethic’ dicitur , \\\hline
1.2.16 & si non es cosa conuenible al rey \textbf{ que ha de enssennorear alos otros } de ser bestial e sieruo & Si ergo indecens est Regem , \textbf{ cuius est aliis dominari , } esse bestialem et seruilem : \\\hline
1.2.16 & en el qual ha de ser el pecado dela destenpranca al moço . \textbf{ Ca assi commo el moço se deue gouernar } por su ayo o por su maestro & secundum quam habet esse vitium intemperantiae assimilat puero : \textbf{ quia sicut puer debet regi per paedagogum , } sic vis concupiscibilis est regenda , et regulanda per rationem . \\\hline
1.2.16 & assi el apetito cobdiciador \textbf{ e desseador se deue gouernar e reglar } por la razon e por el entendimiento . & quia sicut puer debet regi per paedagogum , \textbf{ sic vis concupiscibilis est regenda , et regulanda per rationem . } Si ergo indecens est Regem esse puerum moribus , \\\hline
1.2.16 & Et por ende si non es cosa conueniente de ser el Rey moço en costunbres \textbf{ e de non segnir razon } e en entendimiento mas passiones e delectaçiones & Si ergo indecens est Regem esse puerum moribus , \textbf{ et non sequi rationem , sed passionem : } indecens est ipsum esse intemperatum . \\\hline
1.2.16 & ca este pecado de destenpranca es muy torpe \textbf{ e muy de menospreçiar . } Ca los omes destenprados & Tertio est hoc indecens Regi : \textbf{ quia huiusmodi vitium maxime est turpe , et contemptibile . Homines enim intemperati quia appetunt vilia et turpia , } ideo , \\\hline
1.2.16 & por la qual cosa si pertenesçe ala persona del Rey \textbf{ de se mostrar muy reuerenda } e muy digna de honrra . & sed passionem . \textbf{ Quare si decet personam regiam ostendere se reuerendam } et honore dignam , \\\hline
1.2.16 & non salia fuera dela camara \textbf{ a auer fabla con los Ricos omes } e cauałłos de su regno . & non exibat extra , \textbf{ ut haberet colloquia cum baronibus regni sui ; } sed omnes collocutiones eius erant in cameris ad mulieres : \\\hline
1.2.16 & e por escͥptos enbiaua todas sus razones alos ricos omes e alos prinçipe ᷤ en que les mandaua \textbf{ lo que auian de fazer } Et acaesçio & sed omnes collocutiones eius erant in cameris ad mulieres : \textbf{ et per literas mittebat Baronibus } et Ducibus , \\\hline
1.2.16 & que grant t p̃o le auia seruido e fiel mente . \textbf{ Et el Rey que tiendo fazer plazer } a aquel prinçipe & quid vellet eos facere . Accidit autem , \textbf{ quod , } cum quidam Dux exercitus diu ei seruiuisset , et fideliter , Rex ille volens complacere illi Duci , \\\hline
1.2.16 & e delas sus costunbres \textbf{ e quaso yr contra el } para lo matar . & et indignatus de turpitudine eius , \textbf{ voluit eum inuadere . Rex autem timens , } fugit : \\\hline
1.2.16 & e quaso yr contra el \textbf{ para lo matar . } Et el Rey temiendo lo fuyo . & et indignatus de turpitudine eius , \textbf{ voluit eum inuadere . Rex autem timens , } fugit : \\\hline
1.2.16 & Et por que creya \textbf{ que non podia foyr delas manos de aquel duque } ençerrosse en vna casa & fugit : \textbf{ et quia credebat se non posse fugere manus illius Ducis , } clausit se in quadam domo , \\\hline
1.2.16 & Por ende mucho mueuen alos pueblos en sanna contra si . \textbf{ Et por que mucho han de temer los Reyes e los prinçipes } que el pueblo non se leuate en sanna contra ellos & maxime prouocant alios contra se . \textbf{ Et quia potissime timendum est Regibus et Principibus , } ne furor populi \\\hline
1.2.17 & fincanos \textbf{ de dezir delas otras och̃o uirtudes . } Onde conuiene saber & et ostendimus quomodo Reges et Principes illis virtutibus decet esse ornatos . \textbf{ Reliquum est pertransire ad virtutes alias . } Virtutes autem aliae vel respiciunt exteriora bona , \\\hline
1.2.17 & de dezir delas otras och̃o uirtudes . \textbf{ Onde conuiene saber } que esta s ochon uirtudes o catan alos bienes tenporales de fuera o alos males tenporales de fuera . & ø \\\hline
1.2.17 & en quanto son ordenados a otra cola . \textbf{ Et deuedes laber que los bienes tenporales de fuera o son aprouechosos } assi conmo los dineros o las riquezas & secundum se . Postea determinabimus de virtutibus respicientibus exteriora bona in ordine ad aliud . \textbf{ Bona autem exteriora vel sunt utilia , } ut pecunia , \\\hline
1.2.17 & e todas aquellas cosas \textbf{ que se pueden conparar e mesurar } por dineros & et diuitiae , \textbf{ et omnia quae numismate mensurari possunt : } vel sunt honesta , \\\hline
1.2.17 & que cata alas grandes despenssas . \textbf{ Et esto en qual manera se deue entender adelante lo mostrͣemos¶ } pues que assi es en faziendo espenssas contesçe alas vezes de fallesçer & magnificentia vero dicitur respicere magnos sumptus ; \textbf{ quod quomodo sit intelligendum , } in prosequendo patebit . \\\hline
1.2.17 & Et esto en qual manera se deue entender adelante lo mostrͣemos¶ \textbf{ pues que assi es en faziendo espenssas contesçe alas vezes de fallesçer } e esto faze la auariçia . & quod quomodo sit intelligendum , \textbf{ in prosequendo patebit . | Si igitur in faciendo sumptus conuenit deficere , } quod facit auaritia : \\\hline
1.2.17 & e esto faze la auariçia . \textbf{ Et contesçe alas vezes de sobrepuiar e dar . } mas que conuiene & quod facit auaritia : \textbf{ et superabundare , } quod facit prodigalitas , \\\hline
1.2.17 & Et por que cada vna destas cosas escontra regla derecha de razon e de entendimiento . \textbf{ Conuiene de dar alguna uirtud medianera entre la auariçia e el gastamiento . } Et esta uirtud es libalidat e franqueza . & quia utrunque est contra rectam regulam rationis , \textbf{ oportet dare virtutem aliquam mediam inter auaritiam , | et prodigalitatem : } huiusmodi autem virtus est liberalitas . Patet ergo \\\hline
1.2.17 & que repreme las auariçias \textbf{ e tienpra los gastamientos . Et esta uirtud esta en bien vsar del auer e de los dineros . } Mas para bien usar del auer & ideo est virtus reprimens auaritias , \textbf{ et moderans prodigalitates . | Consistit autem haec virtus in recto usu pecuniae . } Ad rectum autem usum pecuniae tria requiruntur . \\\hline
1.2.17 & e tienpra los gastamientos . Et esta uirtud esta en bien vsar del auer e de los dineros . \textbf{ Mas para bien usar del auer } e de lons dineros tres cosasson menester & Consistit autem haec virtus in recto usu pecuniae . \textbf{ Ad rectum autem usum pecuniae tria requiruntur . } Primo quod non accipiat eam unde non debet . Secundo quod accipiat unde debet . \\\hline
1.2.17 & ¶La primera es que non tome el auer nin los dineros \textbf{ donde non los deue tomar } ¶La segunda que tomne el auero los didos & Ad rectum autem usum pecuniae tria requiruntur . \textbf{ Primo quod non accipiat eam unde non debet . Secundo quod accipiat unde debet . } Tertio quod expendat \\\hline
1.2.17 & ¶La segunda que tomne el auero los didos \textbf{ donde los deuen tomar } ¶ & ø \\\hline
1.2.17 & si mas por quelos orden \textbf{ e para fazer espenssas quales deue fazer . } Enpero para que pueda fazer espenssas quales deue fazer & secundum se , \textbf{ sed ut eam ordinet ad debitos sumptus : } tamen \\\hline
1.2.17 & e para fazer espenssas quales deue fazer . \textbf{ Enpero para que pueda fazer espenssas quales deue fazer } non deue las suᷤ propias rentas esparzer & sed ut eam ordinet ad debitos sumptus : \textbf{ tamen | ut possit debitos sumptus facere , } non debet proprios redditus inaniter dispergere . \\\hline
1.2.17 & Enpero para que pueda fazer espenssas quales deue fazer \textbf{ non deue las suᷤ propias rentas esparzer } nin espender vanamente & ut possit debitos sumptus facere , \textbf{ non debet proprios redditus inaniter dispergere . } Ergo non usurpare redditus alienos , \\\hline
1.2.17 & non deue las suᷤ propias rentas esparzer \textbf{ nin espender vanamente } nin deue tomar las rentas agenas & ut possit debitos sumptus facere , \textbf{ non debet proprios redditus inaniter dispergere . } Ergo non usurpare redditus alienos , \\\hline
1.2.17 & nin espender vanamente \textbf{ nin deue tomar las rentas agenas } por fuerca & non debet proprios redditus inaniter dispergere . \textbf{ Ergo non usurpare redditus alienos , } habere debitam curam de propriis , \\\hline
1.2.17 & e delas sus rentas propias \textbf{ e fazer dellas sus espenssas } quales conuiene ¶ & habere debitam curam de propriis , \textbf{ et ex eis debitos sumptus facere : } sunt illa tria circa quae videtur esse liberalitas . \\\hline
1.2.17 & mas non ha de ser çerca estas tres cosas egualmente e prinçipal mente . \textbf{ Ca primeramente en espender } e en fazer espenssas & et primo . \textbf{ Nam circa expendere et circa debitos sumptus facere , } est liberalitas principaliter , \\\hline
1.2.17 & Ca primeramente en espender \textbf{ e en fazer espenssas } quales deuees la franqueza prinçipalmente e primero . & et primo . \textbf{ Nam circa expendere et circa debitos sumptus facere , } est liberalitas principaliter , \\\hline
1.2.17 & quales deuees la franqueza prinçipalmente e primero . \textbf{ Mas despues desto es en guardar las tentas propias . } Et despues es en non tomar nin forcar los bienes agenos . & est liberalitas principaliter , \textbf{ et primo . Circa autem proprios redditus custodire , } et circa non accipere alienos , \\\hline
1.2.17 & Mas despues desto es en guardar las tentas propias . \textbf{ Et despues es en non tomar nin forcar los bienes agenos . } Ca aquel que vsurpa & et primo . Circa autem proprios redditus custodire , \textbf{ et circa non accipere alienos , } est ex consequenti . Usurpans enim bona utilia , \\\hline
1.2.17 & este paresçe \textbf{ que es muy cobdicioso de auer ¶ } Et por esso el philosofo en el quarto libro delas ethicas & et non accipiens ea \textbf{ sicut debet , nimis videtur auidus pecuniae . Propter quod Philosophus 4 Ethic’ usurarios , } lenones , \\\hline
1.2.17 & llama a estos tales non liberales \textbf{ que quiere dezir non francos } assi commo son los logreros e los garçons & sicut debet , nimis videtur auidus pecuniae . Propter quod Philosophus 4 Ethic’ usurarios , \textbf{ lenones , } idest viuentes de meretricio , \\\hline
1.2.17 & por que gana de los amigosa \textbf{ los quales le conuenia bien fazer . } Et por ende la franqueza es en non tomar on de non deue tomar . & lucratur enim ab amicis , \textbf{ quibus oportet bene facere . } Est igitur liberalitas \\\hline
1.2.17 & los quales le conuenia bien fazer . \textbf{ Et por ende la franqueza es en non tomar on de non deue tomar . } Enꝑnon es cerca esto prinçipal mente . & quibus oportet bene facere . \textbf{ Est igitur liberalitas | circa non accipere unde non debet : } non tamen est circa haec principaliter . Nam accipientes unde non debent , \\\hline
1.2.17 & que non i liberales \textbf{ que quieredezir non francos } segunt dize el philosofo en el quarto libro delas ethicas . & ø \\\hline
1.2.17 & por la qual cosa non es la franqueza prinçipalmente en tomaronde non deue \textbf{ nin en tomar onde deue } mas es prinçipalmente en espender commo deue Ca & ut vult Philosophus 4 Ethic’ . Quare non est liberalitas principaliter circa non accipere unde non debet , \textbf{ nec circa accipere unde debet : } sed est principaliter circa expendere quomodo debet . \\\hline
1.2.17 & nin en tomar onde deue \textbf{ mas es prinçipalmente en espender commo deue Ca } por qual si quier cosa & nec circa accipere unde debet : \textbf{ sed est principaliter circa expendere quomodo debet . } Nam propter quod unumquodque tale , \\\hline
1.2.17 & esto por tanto lo faze \textbf{ por que pueda fazer espessas } quales deuede sus rentas prop̃as & hoc ideo facit , \textbf{ ut possit } ex propriis redditibus debitos sumptus facere , et expendere \\\hline
1.2.17 & quales deuede sus rentas prop̃as \textbf{ e por que pueda espender } assi commo deue . & ut possit \textbf{ ex propriis redditibus debitos sumptus facere , et expendere } sicut debet : \\\hline
1.2.17 & Et por ende de razon paresçe \textbf{ que mas prinçipalmente es la franqueza en espender e en fazer bien alos otros . } Et despues desto es en guardar las sus rentas propreas & sicut debet : \textbf{ merito liberalitas principalius est in expendendo | et in benefaciendo aliis ; } ex consequenti autem est circa custodire proprios redditus , \\\hline
1.2.17 & que mas prinçipalmente es la franqueza en espender e en fazer bien alos otros . \textbf{ Et despues desto es en guardar las sus rentas propreas } e non vsurpar nin tomar las agenas . & et in benefaciendo aliis ; \textbf{ ex consequenti autem est circa custodire proprios redditus , } et circa non usurpare alienos . \\\hline
1.2.17 & Et despues desto es en guardar las sus rentas propreas \textbf{ e non vsurpar nin tomar las agenas . } Ca el philosofo praeua en el quarto libro delas ethicas & ex consequenti autem est circa custodire proprios redditus , \textbf{ et circa non usurpare alienos . } Probat enim Philosophus 4 Ethicorum quinque rationibus , liberalitatem magis esse circa expendere \\\hline
1.2.17 & que la fraquanza es \textbf{ mas en espender } e en bien fazer alos otros & Probat enim Philosophus 4 Ethicorum quinque rationibus , liberalitatem magis esse circa expendere \textbf{ et circa beneficiare alios , } quam circa proprios redditus custodire . Liberalitas enim \\\hline
1.2.17 & mas en espender \textbf{ e en bien fazer alos otros } que en guardar lo suyo mismo & Probat enim Philosophus 4 Ethicorum quinque rationibus , liberalitatem magis esse circa expendere \textbf{ et circa beneficiare alios , } quam circa proprios redditus custodire . Liberalitas enim \\\hline
1.2.17 & e en bien fazer alos otros \textbf{ que en guardar lo suyo mismo } o las sus rentas propias & et circa beneficiare alios , \textbf{ quam circa proprios redditus custodire . Liberalitas enim } ( \\\hline
1.2.17 & assi commo dicho es ha de seer \textbf{ en uso conueinble de espender del auer . } Ca husar del auer es en espender lo & ut dictum est ) \textbf{ esse debet circa debitum usum pecuniae . | Uti autem pecunia , } est expendere eam et tribuere eam aliis . \\\hline
1.2.17 & en uso conueinble de espender del auer . \textbf{ Ca husar del auer es en espender lo } e partir lo alos otros & Uti autem pecunia , \textbf{ est expendere eam et tribuere eam aliis . } Custodire autem proprios redditus , \\\hline
1.2.17 & Ca husar del auer es en espender lo \textbf{ e partir lo alos otros } mas guardar el omne lo suyo non es husar del auerante es mas ganar lo e allegar lo . & Uti autem pecunia , \textbf{ est expendere eam et tribuere eam aliis . } Custodire autem proprios redditus , \\\hline
1.2.17 & e partir lo alos otros \textbf{ mas guardar el omne lo suyo non es husar del auerante es mas ganar lo e allegar lo . } por la qual cosa paresçe & est expendere eam et tribuere eam aliis . \textbf{ Custodire autem proprios redditus , | non est uti pecunia , } sed magis est acquirere \\\hline
1.2.17 & que la franqueza es \textbf{ mas en espender } e partir el auer alos otros & et generare ipsam . \textbf{ Propter quod patet liberalitatem esse magis circa expendere } et circa tribuere pecuniam aliis , \\\hline
1.2.17 & mas en espender \textbf{ e partir el auer alos otros } que en guardar las rentas propias ¶ & Propter quod patet liberalitatem esse magis circa expendere \textbf{ et circa tribuere pecuniam aliis , } quam circa proprios redditus custodire . \\\hline
1.2.17 & e partir el auer alos otros \textbf{ que en guardar las rentas propias ¶ } Lo segundo esto mismo se praeua & et circa tribuere pecuniam aliis , \textbf{ quam circa proprios redditus custodire . } Secundo hoc idem patet , \\\hline
1.2.17 & Lo segundo esto mismo se praeua \textbf{ assi por que ala uirtud mas prinçipal parte nesçe de fazer mayor bien . } Et mayor bien es en bien fazer & Secundo hoc idem patet , \textbf{ quia ad virtutem principalius spectat facere maius bonum . Maius autem bonum est benefacere , } quam bene pati , \\\hline
1.2.17 & assi por que ala uirtud mas prinçipal parte nesçe de fazer mayor bien . \textbf{ Et mayor bien es en bien fazer } que enbien sofrir . & Secundo hoc idem patet , \textbf{ quia ad virtutem principalius spectat facere maius bonum . Maius autem bonum est benefacere , } quam bene pati , \\\hline
1.2.17 & Et mayor bien es en bien fazer \textbf{ que enbien sofrir . } Et mayor bien es en bien obrar & quia ad virtutem principalius spectat facere maius bonum . Maius autem bonum est benefacere , \textbf{ quam bene pati , } et bene operari , \\\hline
1.2.17 & que enbien sofrir . \textbf{ Et mayor bien es en bien obrar } que en non obrar cosas torpes . & quam bene pati , \textbf{ et bene operari , } quam turpia non operari . \\\hline
1.2.17 & Et mayor bien es en bien obrar \textbf{ que en non obrar cosas torpes . } Et aquel que despiende commo deue & et bene operari , \textbf{ quam turpia non operari . } Qui autem debite expendit \\\hline
1.2.17 & este non obra mal \textbf{ mas guardasse de mal obrar . } Et pues que assi es & Non usurpans autem aliena , \textbf{ non operatur turpia . } Si ergo melius est benefacere , \\\hline
1.2.17 & Et pues que assi es \textbf{ si meior cosa es bien fazer } que non mal fazer & non operatur turpia . \textbf{ Si ergo melius est benefacere , } quam non malefacere , \\\hline
1.2.17 & si meior cosa es bien fazer \textbf{ que non mal fazer } o que bien sofrir & Si ergo melius est benefacere , \textbf{ quam non malefacere , } vel quam bene pati : \\\hline
1.2.17 & que non mal fazer \textbf{ o que bien sofrir } meior cosa es bien espender & quam non malefacere , \textbf{ vel quam bene pati : } melius est bene expendere , \\\hline
1.2.17 & o que bien sofrir \textbf{ meior cosa es bien espender } que non tomar las cosas agenas & vel quam bene pati : \textbf{ melius est bene expendere , } quam aliena non surripere vel quam propria custodire . \\\hline
1.2.17 & meior cosa es bien espender \textbf{ que non tomar las cosas agenas } o que guardar las cosas propias ¶ & melius est bene expendere , \textbf{ quam aliena non surripere vel quam propria custodire . } Tertio hoc idem patet : \\\hline
1.2.17 & que non tomar las cosas agenas \textbf{ o que guardar las cosas propias ¶ } Lo terçero esto mismo se puede prouar & melius est bene expendere , \textbf{ quam aliena non surripere vel quam propria custodire . } Tertio hoc idem patet : \\\hline
1.2.17 & o que guardar las cosas propias ¶ \textbf{ Lo terçero esto mismo se puede prouar } por que en aquello esta & quam aliena non surripere vel quam propria custodire . \textbf{ Tertio hoc idem patet : } quia circa illud magis consistit virtus , circa quod consurgit maior laus . Maior autem laus consurgit in bene expendendo , \\\hline
1.2.17 & Et pues que assi es la franqueza \textbf{ mas prinçipalmente esta en despender commo deue } e en bien fazer alos otros & vel in non usurpando aliena . \textbf{ Liberalitas ergo principalius consistit in debite expendendo , } et benefaciendo aliis . \\\hline
1.2.17 & mas prinçipalmente esta en despender commo deue \textbf{ e en bien fazer alos otros } que en guardar lo suyo & Liberalitas ergo principalius consistit in debite expendendo , \textbf{ et benefaciendo aliis . } Quarto hoc idem patet : \\\hline
1.2.17 & e en bien fazer alos otros \textbf{ que en guardar lo suyo } e non tomar lo ageno ¶ & ø \\\hline
1.2.17 & que en guardar lo suyo \textbf{ e non tomar lo ageno ¶ } Lo quarto se puede prouar esso mismo . & ø \\\hline
1.2.17 & e non tomar lo ageno ¶ \textbf{ Lo quarto se puede prouar esso mismo . } ca la uirtud mas prinçipalmente es cerca lo mas guaue . & et benefaciendo aliis . \textbf{ Quarto hoc idem patet : } quia virtus principalius est circa difficilius . \\\hline
1.2.17 & ca la uirtud mas prinçipalmente es cerca lo mas guaue . \textbf{ Et mas guaue es dar los sus bienes e fazer bien alos otros } que guardar las sus rentas propias & quia virtus principalius est circa difficilius . \textbf{ Difficilius autem est aliis dona tribuere , quam proprios redditus custodire , } vel quam aliena non surripere . Nam custodire propria \\\hline
1.2.17 & Et mas guaue es dar los sus bienes e fazer bien alos otros \textbf{ que guardar las sus rentas propias } o que non tomar los bienes agenos . & quia virtus principalius est circa difficilius . \textbf{ Difficilius autem est aliis dona tribuere , quam proprios redditus custodire , } vel quam aliena non surripere . Nam custodire propria \\\hline
1.2.17 & que guardar las sus rentas propias \textbf{ o que non tomar los bienes agenos . } Ca guardar omne lo suyo propio non es cosa fuerte por si . & Difficilius autem est aliis dona tribuere , quam proprios redditus custodire , \textbf{ vel quam aliena non surripere . Nam custodire propria } secundum se non est difficile : \\\hline
1.2.17 & o que non tomar los bienes agenos . \textbf{ Ca guardar omne lo suyo propio non es cosa fuerte por si . } Ca cada hun omne es naturalmente inclinado a amar asi mismo & vel quam aliena non surripere . Nam custodire propria \textbf{ secundum se non est difficile : } quia unusquisque naturaliter inclinatur \\\hline
1.2.17 & Ca guardar omne lo suyo propio non es cosa fuerte por si . \textbf{ Ca cada hun omne es naturalmente inclinado a amar asi mismo } e aguardar los sus biens propos & secundum se non est difficile : \textbf{ quia unusquisque naturaliter inclinatur | ut se diligat , } et ut sua bona custodiat . Dare autem propria bona , \\\hline
1.2.17 & Ca cada hun omne es naturalmente inclinado a amar asi mismo \textbf{ e aguardar los sus biens propos } Mas dar los sus biens propios ha alguna guaueza por si . & ut se diligat , \textbf{ et ut sua bona custodiat . Dare autem propria bona , } secundum se difficultatem habet : \\\hline
1.2.17 & e aguardar los sus biens propos \textbf{ Mas dar los sus biens propios ha alguna guaueza por si . } Ca los bienes propios son cosaque parte nesçen a nos mismos & ut se diligat , \textbf{ et ut sua bona custodiat . Dare autem propria bona , } secundum se difficultatem habet : \\\hline
1.2.17 & nin tan ayunta dos \textbf{ a nos bien commo es cosa guauedar lo propio } a que auemos mayor amor & non est adeo difficile non usurpare aliena , \textbf{ ad quae non ita afficimur , } quia non sunt nobis coniuncta : \\\hline
1.2.17 & e assi commo ayuntado connusco . \textbf{ siguese que en bien espender e commo deue omne } e en dar e partir los bienes alos otros . & sicut est difficile dare propria , \textbf{ ad quae magis afficimur , } quia nobis quodammodo coniunguntur . Quare circa debite expendere , et circa aliis bona tribuere , principalius consistet liberalitas , tanquam circa magis difficile . \\\hline
1.2.17 & siguese que en bien espender e commo deue omne \textbf{ e en dar e partir los bienes alos otros . } Esta mas prinçipalmente la franqueza & sicut est difficile dare propria , \textbf{ ad quae magis afficimur , } quia nobis quodammodo coniunguntur . Quare circa debite expendere , et circa aliis bona tribuere , principalius consistet liberalitas , tanquam circa magis difficile . \\\hline
1.2.17 & assi commo en cosa mas guaue¶ \textbf{ Lo quinto se puede prouar esso mismo } por que los liberales e los francos son mas amados & quia nobis quodammodo coniunguntur . Quare circa debite expendere , et circa aliis bona tribuere , principalius consistet liberalitas , tanquam circa magis difficile . \textbf{ Quinto hoc idem patet : } quia cum liberales maxime amentur , \\\hline
1.2.17 & mas non es mas amado el omne \textbf{ por non tomar lo ageno } nin por guardar bien lo suyo . & Non autem maxime amatur aliquis , \textbf{ si aliena bona non surripiat , } vel si proprios redditus custodiat . \\\hline
1.2.17 & por non tomar lo ageno \textbf{ nin por guardar bien lo suyo . } Mas es muy amado & si aliena bona non surripiat , \textbf{ vel si proprios redditus custodiat . } Sed maxime diligitur \\\hline
1.2.17 & e en las dona connes conuenibles \textbf{ e despues desto ha de ser en guardar las sus rentas e en non tomar las agenas . } Et assi de ligero paresçe & et circa debitas rationes ; \textbf{ ex consequenti autem est in custodiendo redditus proprios , | et non in usurpando alienos : } de leui patet , \\\hline
1.2.17 & Et assi de ligero paresçe \textbf{ en qual manera podemos fazer a nos mismos liberales e francos . } Ca bien conmo la fortaleza & de leui patet , \textbf{ quomodo nos possumus facere liberales . } Nam sicut quia fortitudo plus opponitur timori quam audaciae , \\\hline
1.2.17 & Et por ende para ser libales e francos \textbf{ mas deuemos declinar al gastamiento en del pender } que non ala auariçia . & ita quod potius plus audeamus , \textbf{ quam minus . Sic quia liberalitas magis opponitur auaritiae quam prodigalitati , declinandum est magis ad prodigalitatem quam ad auaritiam , } et magis debemus superabundare in dando , quam deficere . \\\hline
1.2.17 & que non ala auariçia . \textbf{ Et mas deuemos sobrepuiar en dando } que fallesçer en reci & quam minus . Sic quia liberalitas magis opponitur auaritiae quam prodigalitati , declinandum est magis ad prodigalitatem quam ad auaritiam , \textbf{ et magis debemus superabundare in dando , quam deficere . } Vult Philosophus 4 Ethicorum liberalitatem non esse in multitudine datorum , \\\hline
1.2.17 & Et mas deuemos sobrepuiar en dando \textbf{ que fallesçer en reci } egunt que dize el philosofo & quam minus . Sic quia liberalitas magis opponitur auaritiae quam prodigalitati , declinandum est magis ad prodigalitatem quam ad auaritiam , \textbf{ et magis debemus superabundare in dando , quam deficere . } Vult Philosophus 4 Ethicorum liberalitatem non esse in multitudine datorum , \\\hline
1.2.18 & en el quarto libs de las ethicas . \textbf{ La franqueza non esta en mucho dar } nin en muchedunbre de donadios & ø \\\hline
1.2.18 & e en voluntad del \textbf{ que ha de dar } e en querer dar & idest in facultate et voluntate dantis . \textbf{ Nam } ( ut ibidem scribitur ) \\\hline
1.2.18 & que ha de dar \textbf{ e en querer dar } si touiese de que . & idest in facultate et voluntate dantis . \textbf{ Nam } ( ut ibidem scribitur ) \\\hline
1.2.18 & que las sus donacones \textbf{ e las sus espenssas non pueden sobrepuiar } ala muchedunbre delas sus possesiones . & ø \\\hline
1.2.18 & non solamente non pueden ser gastadores dando \textbf{ mas apenas pueden alcançar a que sean francos dando e espendiendo . } Ca sienpre deuen penssar & non solum non possunt esse prodigi , \textbf{ sed vix possunt attingere | ut sint liberales . } Semper ergo cogitare debent , \\\hline
1.2.18 & mas apenas pueden alcançar a que sean francos dando e espendiendo . \textbf{ Ca sienpre deuen penssar } que menos dan de quanto les conuiene ᷤ dar & ut sint liberales . \textbf{ Semper ergo cogitare debent , } quod minora faciunt , \\\hline
1.2.18 & Ca sienpre deuen penssar \textbf{ que menos dan de quanto les conuiene ᷤ dar } Et menos fazen de quanto les conuiene de fazer . & Semper ergo cogitare debent , \textbf{ quod minora faciunt , } quam deceat . \\\hline
1.2.18 & que menos dan de quanto les conuiene ᷤ dar \textbf{ Et menos fazen de quanto les conuiene de fazer . } Et desto puede bien paresçer & Semper ergo cogitare debent , \textbf{ quod minora faciunt , } quam deceat . \\\hline
1.2.18 & Et menos fazen de quanto les conuiene de fazer . \textbf{ Et desto puede bien paresçer } quanto son de denostar los Reyes & quod minora faciunt , \textbf{ quam deceat . } Ex hoc autem apparere potest quod indecens sit eos esse auaros . \\\hline
1.2.18 & Et desto puede bien paresçer \textbf{ quanto son de denostar los Reyes } e los prinçipessi fueten auarientos & quod minora faciunt , \textbf{ quam deceat . } Ex hoc autem apparere potest quod indecens sit eos esse auaros . \\\hline
1.2.18 & que el gastamiento . \textbf{ por la qual razon muy de denostar son los Reyes e los prinçipes } si fueren auarientos . & quod auaritia peior est prodigalitate , \textbf{ propter quod omnino detestabile est Reges } et Principes esse auaros : \\\hline
1.2.18 & ¶La primera razon por que es esta \textbf{ Ca meior cosa es enfermar el omne de enfermedat } de que puede guaresçer & quod si possent esse prodigi , melius esset eos esse prodigos quam auaros . Primo enim , \textbf{ quia melius est infirmari } morbo curabili , quam incurabili . Prodigalitas autem morbus est curabilis , vel ab aetate , vel ab egestate . \\\hline
1.2.18 & Ca meior cosa es enfermar el omne de enfermedat \textbf{ de que puede guaresçer } que non enfermar de enfermedat & quod si possent esse prodigi , melius esset eos esse prodigos quam auaros . Primo enim , \textbf{ quia melius est infirmari } morbo curabili , quam incurabili . Prodigalitas autem morbus est curabilis , vel ab aetate , vel ab egestate . \\\hline
1.2.18 & de que puede guaresçer \textbf{ que non enfermar de enfermedat } de que non pueda guaresçer . & quia melius est infirmari \textbf{ morbo curabili , quam incurabili . Prodigalitas autem morbus est curabilis , vel ab aetate , vel ab egestate . } Nam qui prodigi sunt in iuuentute \\\hline
1.2.18 & que non enfermar de enfermedat \textbf{ de que non pueda guaresçer . } Et el gastamiento es tal enfermedat & quia melius est infirmari \textbf{ morbo curabili , quam incurabili . Prodigalitas autem morbus est curabilis , vel ab aetate , vel ab egestate . } Nam qui prodigi sunt in iuuentute \\\hline
1.2.18 & e non gastadores . \textbf{ Et otrosi desta dicha enferme dat pueden guaresçer por mengua . } ca aquellos que son gastadores muchas vezes son menguados & Curari \textbf{ etiam potest ab egestate : } nam qui prodigi sunt , \\\hline
1.2.18 & el gastamiento es enfermedat \textbf{ que puede gresçer en estas dichas dos maneras } o por hedat o por mengua . & secundum Philosophum prodigalitas moribus curabilis , \textbf{ vel ab aetate , } vel ab egestate . \\\hline
1.2.18 & Mas la auariçia es enfermedat \textbf{ de que non puede omne guaresçer . } Ca quanto la auariçia es mas raygada en cada vno & vel ab egestate . \textbf{ Sed auaritia est moribus incurabilis : } quia quanto quis procedit in auaritia , \\\hline
1.2.18 & que es el Rey de quien desçende \textbf{ e deue descender todo el gouernamiento del regno } non le conuiene de enfermar en las sus costunbres & Si ergo caput regni , \textbf{ a quo totum regnum dirigi debet , } indecens est aegrotare \\\hline
1.2.18 & e deue descender todo el gouernamiento del regno \textbf{ non le conuiene de enfermar en las sus costunbres } de tal enfermedat & a quo totum regnum dirigi debet , \textbf{ indecens est aegrotare } secundum mores morbo incurabili , \\\hline
1.2.18 & de tal enfermedat \textbf{ de que non pueda guaresçer } por ende mucho de denostar & indecens est aegrotare \textbf{ secundum mores morbo incurabili , } omnino detestabile est Regem esse auarum , \\\hline
1.2.18 & de que non pueda guaresçer \textbf{ por ende mucho de denostar } es el Rey & indecens est aegrotare \textbf{ secundum mores morbo incurabili , } omnino detestabile est Regem esse auarum , \\\hline
1.2.18 & commo el libal non lo es \textbf{ de ligero se puede fazer } qual quier gastador liberal e franco ¶ & de leui quis cum sit prodigus , \textbf{ fieri poterit liberalis . } Si ergo omnino decens est Regem esse virtuosum , \\\hline
1.2.18 & en toda manera de ser uirtuoso \textbf{ tanto mas de denostar es el Rey } si fuer auariento & Si ergo omnino decens est Regem esse virtuosum , \textbf{ tanto detestabilius est ipsum esse auarum , } quam prodigum : \\\hline
1.2.18 & Et el gastadora muchos aprouecha dando . \textbf{ Et por ende muy de depostar es el Rey } si fuer auariento ¶ visto que los Reyes non pueden ser gastadores & quia etiam sibiipsi nequam est : prodigus autem multis prodest . Omnino ergo detestabile est , Regem esse auarum . \textbf{ Viso quod quasi impossibile est Reges esse prodigos , } et quod omnino detestabile est eos esse auaros : \\\hline
1.2.18 & si fuer auariento ¶ visto que los Reyes non pueden ser gastadores \textbf{ e que muchon son de denostar } si fueren auarientos fincanos de demostrar & Viso quod quasi impossibile est Reges esse prodigos , \textbf{ et quod omnino detestabile est eos esse auaros : } restat ostendere , \\\hline
1.2.18 & e que muchon son de denostar \textbf{ si fueren auarientos fincanos de demostrar } que conuiene alos Reyes de ser largos liberales e dadores . & et quod omnino detestabile est eos esse auaros : \textbf{ restat ostendere , } quod deceat eos esse largos , \\\hline
1.2.18 & ¶pues que assi es conmo \textbf{ tanto conuenga ala fuente auer la boca mas ancha } quanto della deuen mas omes tomar e sacar . & quod in eis est . Abundans ergo in sumptibus , et dationibus , dicitur largus : \textbf{ quia ad modum largi vasis abunde emittit quae continet . Cum ergo tanto deceat fontem habere os largius , } quanto ex eo plures participare debent : \\\hline
1.2.18 & tanto conuenga ala fuente auer la boca mas ancha \textbf{ quanto della deuen mas omes tomar e sacar . } tanto conuiene al Rey de ser mas largo & quia ad modum largi vasis abunde emittit quae continet . Cum ergo tanto deceat fontem habere os largius , \textbf{ quanto ex eo plures participare debent : } tanto decet Regem largiorem esse , \\\hline
1.2.18 & quanto la su magnifiçençia \textbf{ e la su largueza amas se ha de estender } que la largueza de los otros ¶ & tanto decet Regem largiorem esse , \textbf{ quanto influentia eius ad plures extendenda est , } quam influentia aliorum . \\\hline
1.2.18 & Ca los liberales son estrimadamente amables \textbf{ e de amar } por la qual razon & per quam communicationem ab aliis potissime diliguntur : \textbf{ nam liberales sunt potissime amabiles . } Quare si maxime decet Reges et Principes , \\\hline
1.2.18 & los que son en el su regno mucho les conuiene de ser liberales e francos \textbf{ Mas par tenesce al libal e alstan ço de catar tres cosas ¶ } La primera deue catar la quantidat delo que da & maxime decet eos liberales esse . \textbf{ Spectat autem ad liberalem primo respicere quantitatem dati , } ut non det minus , \\\hline
1.2.18 & Mas par tenesce al libal e alstan ço de catar tres cosas ¶ \textbf{ La primera deue catar la quantidat delo que da } por que non de menos o mas delo & maxime decet eos liberales esse . \textbf{ Spectat autem ad liberalem primo respicere quantitatem dati , } ut non det minus , \\\hline
1.2.18 & por que non de menos o mas delo \textbf{ que deue dar ¶ } Lo segundo deue catar aqui lo da & vel plus , \textbf{ quam debeat . } Secundo debet respicere quibus det , \\\hline
1.2.18 & que deue dar ¶ \textbf{ Lo segundo deue catar aqui lo da } por que non de aquien non deue dar ¶ & quam debeat . \textbf{ Secundo debet respicere quibus det , } ut non det quibus non oportet . \\\hline
1.2.18 & Lo segundo deue catar aqui lo da \textbf{ por que non de aquien non deue dar ¶ } Lo terçero deue veer & Secundo debet respicere quibus det , \textbf{ ut non det quibus non oportet . } Tertio videndum est \\\hline
1.2.18 & por que non de aquien non deue dar ¶ \textbf{ Lo terçero deue veer } por qual razon lo da & ut non det quibus non oportet . \textbf{ Tertio videndum est } cuius gratia det , \\\hline
1.2.18 & e non por otra razon ninguna \textbf{ Mas los Reyes e los prinçipes apenas pueden desuiar se dela liberalidat en dando mucho } por que la grandeza delas espenssas apenas puede sobrepuiar & ut det boni gratia , \textbf{ non propter aliquam aliam causam . Reges enim et Principes vix possunt deuiare a liberalitate in dando plus , } quia magnitudo expensarum vix potest excedere multitudinem reddituum . Imo si contingat liberalem dare plus quam deceat , \\\hline
1.2.18 & Mas los Reyes e los prinçipes apenas pueden desuiar se dela liberalidat en dando mucho \textbf{ por que la grandeza delas espenssas apenas puede sobrepuiar } ala muchedunbre de las sus rentas . Por ende si contesçe algunas uegadas al liberal de dar & non propter aliquam aliam causam . Reges enim et Principes vix possunt deuiare a liberalitate in dando plus , \textbf{ quia magnitudo expensarum vix potest excedere multitudinem reddituum . Imo si contingat liberalem dare plus quam deceat , } ut vult Philosophus , \\\hline
1.2.18 & por que la grandeza delas espenssas apenas puede sobrepuiar \textbf{ ala muchedunbre de las sus rentas . Por ende si contesçe algunas uegadas al liberal de dar } mas de quanto deue legunt dize el philosofo tomara tristeza tenpradamente & non propter aliquam aliam causam . Reges enim et Principes vix possunt deuiare a liberalitate in dando plus , \textbf{ quia magnitudo expensarum vix potest excedere multitudinem reddituum . Imo si contingat liberalem dare plus quam deceat , } ut vult Philosophus , \\\hline
1.2.18 & o si non espendiere do deua \textbf{ que si espendiere do non le conuiene espender } Mas los Reyes e los prinçipes de suranse & ubi oportet , \textbf{ quam si expendat | ubi non oportet . } Deuiant autem a liberalitate Reges , \\\hline
1.2.18 & e a otros semeiantes \textbf{ a quien non conuiene de dar . } por que aquestos tales mas les conuiene de ser pobres & vel aliis , \textbf{ quibus non oportet dare : } quia magis deceret eos esse pauperes , quam diuites . Sic etiam dant \\\hline
1.2.18 & non por la razon \textbf{ que deuen dar . } Ca non lo dan por razon de bien & cuius gratia non oportet . Non enim dant boni gratia , \textbf{ sed magis dant ut laudentur , } et propter inanem gloriam \\\hline
1.2.18 & e alos prinçipes de ser liƀͣales . \textbf{ Et para ser libales conuiene les de bien fazer alos buenos } e por razon devien & et ut liberales sint , \textbf{ oportet eos beneficiare bonos , } et boni gratia . \\\hline
1.2.19 & la qual laman magnifiçençia \textbf{ que quiere dezir grandeza en despender . } Mas commo en cada cosa mas e menos non fagan departimiento en la naturaleza & quae respicit sumptus magnos , \textbf{ quam magnificentiam nominant . } Sed cum magis , \\\hline
1.2.19 & Et la liberalidat çerca las medianas e mesuradas ¶ \textbf{ Et por ende deuedes saber } que de razon de uirtudes & illa vero circa mediocres . \textbf{ Sciendum ergo quod de ratione virtutis est , } quod sit circa bonum , \\\hline
1.2.19 & que se a çerca bien grande \textbf{ e guaue de fazer . } por la qual cosa commo en las mayores espenssas sea fallada . & Sciendum ergo quod de ratione virtutis est , \textbf{ quod sit circa bonum , } et difficile . quare cum in maioribus sumptibus reperiatur specialis ratio bonitatis et difficultatis , \\\hline
1.2.19 & que en las espenssas medianas e mesuradas . \textbf{ Conuiene de dezir } que la magnifiçençia & quae non reperitur in mediocribus sumptibus , \textbf{ dici potest magnanimitatem } quae est circa magnos sumptus , \\\hline
1.2.19 & que es çerca delas medianas mesuradas espenssas . \textbf{ E podemos dezir de otra guisa } que las espenssas se pueden conparar a dos cosas . & quae est circa mediocres . \textbf{ Vel possumus dicere , } quod sumptus ad duo comparari possunt . \\\hline
1.2.19 & E podemos dezir de otra guisa \textbf{ que las espenssas se pueden conparar a dos cosas . } Conuiene de saber & Vel possumus dicere , \textbf{ quod sumptus ad duo comparari possunt . } Videlicet , \\\hline
1.2.19 & que las espenssas se pueden conparar a dos cosas . \textbf{ Conuiene de saber } que se pueden conparar a las riquezas de aquel & quod sumptus ad duo comparari possunt . \textbf{ Videlicet , } ad facultates eius qui facit sumptus , \\\hline
1.2.19 & Conuiene de saber \textbf{ que se pueden conparar a las riquezas de aquel } que faze las espenssas . & Videlicet , \textbf{ ad facultates eius qui facit sumptus , } et ad opera in quae sumptus illi expenduntur . \\\hline
1.2.19 & que faze las espenssas . \textbf{ Et pueden se conparar alas obras } en que se espienden e se fazen las espenssas ¶ & ad facultates eius qui facit sumptus , \textbf{ et ad opera in quae sumptus illi expenduntur . } Erit ergo circa pecuniam duplex virtus . \\\hline
1.2.19 & e las passiones alas sus riquezas . \textbf{ Mas la magnifiçençia e el magnifico deue proporcionar } e conparar las espenssas alas obras . & et sumptus facultatibus . \textbf{ Sed magnificus vult proportionare sumptus operibus : } unde et nomen accepit . \\\hline
1.2.19 & Mas la magnifiçençia e el magnifico deue proporcionar \textbf{ e conparar las espenssas alas obras . } Ende el magnifico nonbre toma de aquello que faze es nonbrada la magnificençia & Sed magnificus vult proportionare sumptus operibus : \textbf{ unde et nomen accepit . } Ab ipsa enim factione , \\\hline
1.2.19 & que ome aya \textbf{ si quier muchas si quier medianas guaue cosa es bien husar dellas } e bien egualar las espenssas alas riquezas . & siue multas , \textbf{ siue mediocres : | difficile est bene } uti illis et bene aequare sumptus facultatibus . Ideo liberalitas se extendit ad mediocres sumptus . Sumptus enim , \\\hline
1.2.19 & si quier muchas si quier medianas guaue cosa es bien husar dellas \textbf{ e bien egualar las espenssas alas riquezas . } Por ende la liłalidat se estiende alas espenssas mesuradas & difficile est bene \textbf{ uti illis et bene aequare sumptus facultatibus . Ideo liberalitas se extendit ad mediocres sumptus . Sumptus enim , } vel possunt considerari \\\hline
1.2.19 & Por ende la liłalidat se estiende alas espenssas mesuradas \textbf{ ca las espenssas o se pueden penssar } segunt si o se pueden penssar & uti illis et bene aequare sumptus facultatibus . Ideo liberalitas se extendit ad mediocres sumptus . Sumptus enim , \textbf{ vel possunt considerari } secundum se , \\\hline
1.2.19 & ca las espenssas o se pueden penssar \textbf{ segunt si o se pueden penssar } en conpara conn de las riquezas & vel possunt considerari \textbf{ secundum se , } vel ut proportionantur facultatibus . \\\hline
1.2.19 & Ca ahun que las riquezas sean medianas \textbf{ guaue cosa es dese auer bien los omes en ellas . } Mas la magnificençia & se extendit ad quascunque facultates : \textbf{ quia etiam in mediocribus facultatibus difficile est bene se habere in illis . Magnificentia vero , } respiciens sumptus \\\hline
1.2.19 & nin tiene oio quales sean las obras . \textbf{ Ca non es guaue cosa de fazer conuenibles espenssas } en quales se quier obras . & non respicit quaecunque opera : \textbf{ quia non est difficile facere decentes sumptus in quibuscunque operibus , } sed in magnis . Ideo magnificentia a magnis operibus sumpsit nomen . In magnis autem operibus contingit aliquos deficere , \\\hline
1.2.19 & Et por ende la magnificençia toma nonbre de grandes fechos e de grandesobras . \textbf{ Mas deuedes saber } que enlas grandes obras contesçe a algimos de fallesçer . & ø \\\hline
1.2.19 & Mas deuedes saber \textbf{ que enlas grandes obras contesçe a algimos de fallesçer . } Ca non entienden commo han de fazer las grandes obras & sed in magnis . Ideo magnificentia a magnis operibus sumpsit nomen . In magnis autem operibus contingit aliquos deficere , \textbf{ quia non intendunt quomodo magna opera faciant , } sed quomodo parum expendant . \\\hline
1.2.19 & que enlas grandes obras contesçe a algimos de fallesçer . \textbf{ Ca non entienden commo han de fazer las grandes obras } mas tienen mientes a poco espender . & sed in magnis . Ideo magnificentia a magnis operibus sumpsit nomen . In magnis autem operibus contingit aliquos deficere , \textbf{ quia non intendunt quomodo magna opera faciant , } sed quomodo parum expendant . \\\hline
1.2.19 & Ca non entienden commo han de fazer las grandes obras \textbf{ mas tienen mientes a poco espender . } Et estos son llamados pariuficos & quia non intendunt quomodo magna opera faciant , \textbf{ sed quomodo parum expendant . } Et tales vocantur paruifici . \\\hline
1.2.19 & Et estos son llamados pariuficos \textbf{ que quiere dezer omes de poca fazienda } e que espienden poco . & ø \\\hline
1.2.19 & a estos tales llama los han asos \textbf{ que quiere dezir fuegos e fornos } por que estos tałs & Philosophus vero vocat eos chaunos idest ignes \textbf{ et fornaces , } quia tales sicut fornax omnia consumunt . Quidam vero in magnis operibus faciunt decentes sumptus : \\\hline
1.2.19 & assi la magnificençia faze espenssas conuenibles alas grandes obras ¶ \textbf{ visto que cosa es la magnificençia fincanos de veer } en quales cosas ha de ser . & sic magnificentia est faciens sumptus decentes magnis operibus . \textbf{ Viso | quid est magnificentia : } restat videre circa quae habet esse . \\\hline
1.2.19 & en quales cosas ha de ser . \textbf{ Onde conuiene saber } que el omne & quid est magnificentia : \textbf{ restat videre circa quae habet esse . } Homo enim ( quantum ad praesens ) \\\hline
1.2.19 & que el omne \textbf{ quanto pertenesçe alo presente puede se conparar a quatro cosas¶ } Lo primero a dios & Homo enim ( quantum ad praesens ) \textbf{ ad quatuor potest comparari : | videlicet , } ad ipsum Deum , \\\hline
1.2.19 & Lo quarto assi mismo . \textbf{ Ca el magnefico se deue auer conueniblemente çerca estas quatro cosas . } Mas enpero non deue entender egualmente nin prinçipalmente cerca estas quatro cosas . & Nam principaliter et primo , \textbf{ homo debet esse magnificus circa diuina , constituendo | ( si facultates tribuant ) } templa magnifica , \\\hline
1.2.19 & Ca el magnefico se deue auer conueniblemente çerca estas quatro cosas . \textbf{ Mas enpero non deue entender egualmente nin prinçipalmente cerca estas quatro cosas . } Ca primero e prinçipalmente deue seer el omne magnifico çerca las cosas dauinołs & ( si facultates tribuant ) \textbf{ templa magnifica , } sacrificia honorabilia , \\\hline
1.2.19 & Ca primero e prinçipalmente deue seer el omne magnifico çerca las cosas dauinołs \textbf{ Et si cunplieren las sus riquezas deue fazer grandes eglesias e sac̀fiçios honrrados et apareiamientos dignos } assi commo uestimentas e calices e otros apareiamientostales . & templa magnifica , \textbf{ sacrificia honorabilia , } praeparationes dignas . Ideo dicitur 4 Ethicorum , \\\hline
1.2.19 & en el quarto libro delas ethicas \textbf{ que el magnifico deue fazer honrradas espenssas en aquellas cosas } que parte nesçen a dios¶ & praeparationes dignas . Ideo dicitur 4 Ethicorum , \textbf{ quod honorabiles sumptus , | quos debet facere magnificus , } sunt circa Deum . \\\hline
1.2.19 & Lo segundo pertenesçe al magnifico \textbf{ si ouiereconplidamente las riquezas fazer espenssas conuenibles en toda la comunidat } por que los bienes comunes lon en alguna manera diuinales & Secundo spectat ad magnificum \textbf{ ( si adsit facultas ) | facere decentes sumptus circa totam communitatem . } Nam ipsa bona communia quodammodo diuina sunt . \\\hline
1.2.19 & mente en toda la comunidat ¶ \textbf{ Lo tercero el magnifico se deue auer conueniblemente çerca algunas personas espeçiales } assi commo cerca aquellas personas & Bonum enim diuinum valde debiliter repraesentatur in una persona singulari : sed in tota communitate pulchrius elucescit diuinum bonum . \textbf{ Tertio magnificus se decenter habere debet | circa aliquas personas speciales , } ut circa personas honore dignas . \\\hline
1.2.19 & a aquellos que son mas dignos ¶ \textbf{ Lo quarto el magnifico se deue auer conueinblemente cerca la su perssona propia } por que deue cada vno granadamenᷤte se auer cerca las grandes obras en conparaçion dela su perssona & qui sunt magis digni . \textbf{ Quarto debet se decenter habere magnificus circa personam propriam : } debet enim quis magnifice se habere circa magna opera respectu personae propriae . \\\hline
1.2.19 & Lo quarto el magnifico se deue auer conueinblemente cerca la su perssona propia \textbf{ por que deue cada vno granadamenᷤte se auer cerca las grandes obras en conparaçion dela su perssona } mas las grandes obras pueden ser dichas & Quarto debet se decenter habere magnificus circa personam propriam : \textbf{ debet enim quis magnifice se habere circa magna opera respectu personae propriae . } Magna autem opera possunt dici illa , \\\hline
1.2.19 & en el quarto libro delas ethicas \textbf{ de apareiar muy conueniblemente las sus moradas } e mas deue entender en qual manera deue fazer muuy marauillosas & et militiae . Decet enim magnificum \textbf{ ( ut dicitur 4 Ethicor’ ) habitationem praeparare decenter , } et magis debet intendere quomodo facere debeat admirabiles , \\\hline
1.2.19 & de apareiar muy conueniblemente las sus moradas \textbf{ e mas deue entender en qual manera deue fazer muuy marauillosas } e muy durables las sus casas & ( ut dicitur 4 Ethicor’ ) habitationem praeparare decenter , \textbf{ et magis debet intendere quomodo facere debeat admirabiles , } et diuturnas domus , \\\hline
1.2.19 & nin tan firmes en si . \textbf{ En essa misma manera conuiene al magnifico de fazer muy honrradamente } las sus bodas e las sus cauallerias & et apparentes . \textbf{ Sic etiam decet magnificum , nuptias , } et militias , \\\hline
1.2.19 & e cerca quales cosas ha de ser ligeramente paresçe \textbf{ commo podemos fazer a nos mismos magnificos . } Ca assi commo la liberalidat & de leui patet , \textbf{ quomodo possumus nosipsos magnificos facere . } Nam sicut liberalitas plus contrariatur auaritiae , \\\hline
1.2.19 & Et declinado al gastamiento \textbf{ podemos a nos mismos fazer libales e francos . } Assi la magnificençia es mas contraria ala pariuuficençia & ø \\\hline
1.2.19 & de que declinando \textbf{ mas al gastar e al destroyr } que ala parui ficençia & ( si facultas adsit ) faciemus magnificos , \textbf{ declinando ad consumptionem , } ut etiam in magnis operibus potius superabundent sumptus , \\\hline
1.2.19 & que non en mengunado \textbf{ e fallesciendo de despender . } l philosofo en el quarto libro delas ethicas & ut etiam in magnis operibus potius superabundent sumptus , \textbf{ quam deficiant . } Tangit autem Philosophus 4 Ethicor’ sex proprietates ipsius paruifici : \\\hline
1.2.20 & pone seys propiedades del parufico \textbf{ que es omne menguado en despender } las quales si fueren en los Reyes o en los prinçipes mucho menguaran ensi & Tangit autem Philosophus 4 Ethicor’ sex proprietates ipsius paruifici : \textbf{ quae si inessent regibus , } maxime derogarent regiae maiestati . \\\hline
1.2.20 & que el paruifico fallesce en todas las cosas \textbf{ que ha de fazer } Et esto se praeua & Prima proprietas est , \textbf{ quia circa omnia deficit , } quod ex ipso nomine patet . \\\hline
1.2.20 & La segunda propiendat es \textbf{ que si contezta que el paruifico aya de despender grandes espenssas } pierde muy grant & Secunda proprietas est , \textbf{ quia si contingat paruificum , | et magna expendere pro paruo , } perdit magnum bonum . Unde et prouerbialiter dicitur , Viles homines nuptias , \\\hline
1.2.20 & esto non es \textbf{ si non por que quiere bazer pequanas el penssas } Et assi todo el conbite se pierde & Dum enim circa magnum conuiuium non faciunt decentes sumptus , volentes parcere modicae expensae , totum conuiuium indecens redditur . Tertia proprietas est , \textbf{ quod quaecunque facit paruificus , } semper facit tardans . \\\hline
1.2.20 & Ca paresçe leal paruifico \textbf{ que tirar el auer de ssi estaiarle los mienbros de su cuerpo . } Por ende & quod remouere a se pecuniam , \textbf{ sit abscindere membra a proprio corpore . } Ideo sicut dato \\\hline
1.2.20 & assi commo si fuesse menester \textbf{ que alguno ouiesse de taiar su mienbro fuyria } e tardaria quanto podiesse & Ideo sicut dato \textbf{ quod necesse sit aliquem mutilari , | et inscindi mutilationem illam , } et inscissuram tardat , \\\hline
1.2.20 & Bien alłi puesto que el paruifico \textbf{ e al escasso sea dado de fazer grandes espenssas } sienpre tarda e fuye & et subterfugit quantum potest : \textbf{ sic dato quod paruificum oporteat expensas facere , } tamen illos sumptus tardat , \\\hline
1.2.20 & bien \textbf{ assi en el apartamento del auer } e de los desque se faze & ideo sicut in incisione corporis propter diuisionem continui resultat ibi tristitia \textbf{ et dolor : sic in separatione pecuniae , quae fit per expensas , } quia est ibi quasi quaedam diuisio continui , \\\hline
1.2.20 & e continuados con ella . \textbf{ Por ende non puede espender el auer } e non puede partir de ssi los ds e el auer sin tristeza e sin dolor & quasi sibi incorporatam et continuatam , \textbf{ non potest eam expendere , } et ipsam a se remouere sine tristitia , \\\hline
1.2.20 & Por ende non puede espender el auer \textbf{ e non puede partir de ssi los ds e el auer sin tristeza e sin dolor } assi conmo el mienbro non se podria partir del cuerpo lindolor . & non potest eam expendere , \textbf{ et ipsam a se remouere sine tristitia , } et dolore . Participat enim in hoc paruificus cum auaro : \\\hline
1.2.20 & e non puede partir de ssi los ds e el auer sin tristeza e sin dolor \textbf{ assi conmo el mienbro non se podria partir del cuerpo lindolor . } Et en esto el partu fico partiçipa con el auariento & non potest eam expendere , \textbf{ et ipsam a se remouere sine tristitia , } et dolore . Participat enim in hoc paruificus cum auaro : \\\hline
1.2.20 & si non el auer ¶ \textbf{ Pues que assi es commo abes pueda el parufico dar vna cosa muy pequana dela cosa } que mucho ama . & et non appreciatur ea . \textbf{ Cum igitur vix possit aliquis dare ita modicum de caro , } dato etiam quod multum recipiat de vili , \\\hline
1.2.20 & puesto avn que mucho resciba dela cosa vil nol parezca a el que da \textbf{ mas que deue dar . } assi en essa misma manera non puede el parufico fazer despenssas tan pequanas & quin videatur ei plus dare , \textbf{ quam debeat : } non potest paruificus ita modicum sumptum facere erga quodcunque opus , \\\hline
1.2.20 & mas que deue dar . \textbf{ assi en essa misma manera non puede el parufico fazer despenssas tan pequanas } en qual si quier obra que faga & quam debeat : \textbf{ non potest paruificus ita modicum sumptum facere erga quodcunque opus , } quin semper videatur ei quod agat maiora , \\\hline
1.2.20 & que deua . \textbf{ Por la qual cosa si cosa muy denostada es en la real magestad fallesçer } e menguar en todas las cosas & quam debeat . \textbf{ Quare si detestabile est regiam maiestatem } circa omnia deficere , \\\hline
1.2.20 & Por la qual cosa si cosa muy denostada es en la real magestad fallesçer \textbf{ e menguar en todas las cosas } e perder muy grandes bienes & Quare si detestabile est regiam maiestatem \textbf{ circa omnia deficere , } magna bona pro modico perdere , \\\hline
1.2.20 & e menguar en todas las cosas \textbf{ e perder muy grandes bienes } por muy pequana cosa . & circa omnia deficere , \textbf{ magna bona pro modico perdere , } semper tardare , \\\hline
1.2.20 & por muy pequana cosa . \textbf{ Et tardar sienpre en las cosas } que ha de fazer & magna bona pro modico perdere , \textbf{ semper tardare , } et nihil prompte facere , \\\hline
1.2.20 & Et tardar sienpre en las cosas \textbf{ que ha de fazer } e non fazer ninguna cosa aꝑçebidamente & magna bona pro modico perdere , \textbf{ semper tardare , } et nihil prompte facere , \\\hline
1.2.20 & que ha de fazer \textbf{ e non fazer ninguna cosa aꝑçebidamente } e de uoluntad & semper tardare , \textbf{ et nihil prompte facere , } et nunquam intendere quomodo faciat magna opera virtutum , \\\hline
1.2.20 & e de uoluntad \textbf{ e nunca entender } en commo faga grandes obras de uirtud & et nihil prompte facere , \textbf{ et nunquam intendere quomodo faciat magna opera virtutum , } sed quomodo modicum expendat , \\\hline
1.2.20 & en commo faga grandes obras de uirtud \textbf{ mas penssar } sienpre en commo espienda poco & et nunquam intendere quomodo faciat magna opera virtutum , \textbf{ sed quomodo modicum expendat , } semper facere \\\hline
1.2.20 & sienpre en commo espienda poco \textbf{ e fazer sienpre espenssa con tristeza e con dolor . } Et quando non faze ningunan cosa cree el & semper facere \textbf{ sumptum cum tristitia et dolore ; et cum nihil facit , } credere se magna operari , \\\hline
1.2.20 & Et por que todas estas cosas ponen grand mengua en la Real magestad \textbf{ en todas maneras es de denostar } que el Rey sea periufico & quia omnia haec valde derogant regiae maiestati , \textbf{ omnino detestabile est Regem esse paruificum . } Quod autem deceat ipsum esse magnificum , \\\hline
1.2.20 & mas que conuengaal Rey de ser magnifico \textbf{ e de fazer grandes espenssas conplidamente es prouado } por lo que dicho es de suso & Quod autem deceat ipsum esse magnificum , \textbf{ sufficienter probant superiora dicta : } in quibus ostendimus circa quae magnificentia habet esse . Dicebatur enim magnificentiam principaliter esse circa opera diuina , \\\hline
1.2.20 & e de grand reuerençia e persona publica \textbf{ e a el pertenesca de partir los bienes del regno mucho le conuiene a el de ser magnifico . } Ca porque es cabeça del regno & et sit persona honorabilis , reuerenda , et publica , \textbf{ et ad ipsum pertineat distribuere bona regni , maxime decet ipsum esse magnificum . Nam quia est caput regni , } et gerit in hoc Dei vestigium , \\\hline
1.2.20 & que es cabesça e prinçipe de todo el mundo . \textbf{ mucho parte nesçe a el de se auer granadamente } e honrradamente cerca las eglesias saguadas & et Princeps uniuersi , \textbf{ maxime spectat ad ipsum magnifice se habere circa templa sacra , } et erga praeparationes diuinorum . \\\hline
1.2.20 & e todo el regno es ordenado \textbf{ mucho parte nesçe a el de se auer granadamente } e honrradamente çerca los bien es comuns e cerca todas aquellas cosas que pertenesçen a todo el regno guardando las & et totum regnum , \textbf{ maxime spectat ad eum magnifice se habere circa bona communia , } et circa ea quae respiciunt regnum totum . Rursus quia ad ipsum maxime spectat distribuere bona regni , \\\hline
1.2.20 & e proprouechando las mucho ¶ \textbf{ Otrosi por que a el parte nesce prinçipalmente partir los bienes del regno } en todas maneras le conuiene ael de se auer grande e honrradamente & maxime spectat ad eum magnifice se habere circa bona communia , \textbf{ et circa ea quae respiciunt regnum totum . Rursus quia ad ipsum maxime spectat distribuere bona regni , } omnino decet eum magnifice se habere erga personas dignas , \\\hline
1.2.20 & Otrosi por que a el parte nesce prinçipalmente partir los bienes del regno \textbf{ en todas maneras le conuiene ael de se auer grande e honrradamente } a aquellas personas & et circa ea quae respiciunt regnum totum . Rursus quia ad ipsum maxime spectat distribuere bona regni , \textbf{ omnino decet eum magnifice se habere erga personas dignas , } quibus digne competunt illa bona . Amplius quia ( ut dictum est ) regia persona debet esse reuerenda et honore digna , \\\hline
1.2.20 & a aquellas personas \textbf{ que son dignas e meresçen de auer aquellos bienes ¶ } Otrosi por que assi commo dicho es . La persona del Rey deue ser de grand reuerençia & omnino decet eum magnifice se habere erga personas dignas , \textbf{ quibus digne competunt illa bona . Amplius quia ( ut dictum est ) regia persona debet esse reuerenda et honore digna , } spectat ad Regem magnifice se habere erga personam propriam , \\\hline
1.2.20 & e digna de grand honrra parte nesçe mucho al Rey \textbf{ de se auer granadamente } e honrradamente çerca dela su persona propia & quibus digne competunt illa bona . Amplius quia ( ut dictum est ) regia persona debet esse reuerenda et honore digna , \textbf{ spectat ad Regem magnifice se habere erga personam propriam , } et erga personas sibi coniunctas , \\\hline
1.2.21 & e mucho honrradas . \textbf{ euedes saber } que el philosofo enl quarto libro delas ethicas capitulo dela magnificençia & exercendo militias admirabiles . \textbf{ Philosophus 4 Ethicorum capitulo De magnificentia , } tangit sex proprietates magnifici , \\\hline
1.2.21 & pone seys propiedadesdel magnifico \textbf{ las quales conuiene alos Reyes e alos prinçipes auer ¶ } La primera es que el magnifico es semeiante al sabio & tangit sex proprietates magnifici , \textbf{ quas habere decet Reges et Principes . Prima proprietas est , } quia magnificus assimilatur scienti . \\\hline
1.2.21 & Ca dixiemos de suso \textbf{ que conuenia al magnifico de fazer conuenientes espenssas en las grandes obras . } Mas conosçer en qual es grandes obras & quia magnificus assimilatur scienti . \textbf{ Dicebatur enim spectare ad magnificum in magnis operibus facere decentes sumptus . } Cognoscere autem quibus magnis operibus \\\hline
1.2.21 & que conuenia al magnifico de fazer conuenientes espenssas en las grandes obras . \textbf{ Mas conosçer en qual es grandes obras } quales espenssas son conuenibles . & Dicebatur enim spectare ad magnificum in magnis operibus facere decentes sumptus . \textbf{ Cognoscere autem quibus magnis operibus } qui sumptus sint conuenientes , \\\hline
1.2.21 & si non en aquel que ouiere grand sabaduria e grand entendimiento¶ \textbf{ La segunda propiedat del magnifico es fazer grandes espenssas } non por que se muestre nin por vanagłia & et intellectu . \textbf{ Secunda proprietas magnifici , est facere magnos sumptus , } non ut ostendat seipsum , \\\hline
1.2.21 & mas por razon de algun bien \textbf{ ca esto es comun a cada vna delas uirtudes obrar } non por honrra o por vanagłoia de los omes & non ut ostendat seipsum , \textbf{ sed boni gratia . Est enim hoc commune cuilibet virtuti , agere non propter fauorem , } vel propter gloriam hominum , \\\hline
1.2.21 & Ca commo en las grandes obras \textbf{ sienpre deua tener omne nayentes a buena fin } assi commo quando alguno granadamente sea en la honrra de dios & Nam cum in operibus magnificis , \textbf{ ut cum aliquis magnifice se habet erga cultum diuinum , } et erga rempublicam , \\\hline
1.2.21 & Enpero guaue cosa es en tales cosas \textbf{ non demandar loor delas gentes } Et por quela uirtud es cerca bien & et maxime quis ab hominibus laudatur , \textbf{ difficile est in talibus non quaerere humanam laudem . } Et quia virtus est circa bonum \\\hline
1.2.21 & en las sus muy grandes obras \textbf{ e en las sus parti connsenteder finalmente el bien } e non honrra nin vana eglesia de los omes & et difficile , \textbf{ ideo maxime spectat ad magnificum in suis magnificis operibus , et distributionibus intendere finaliter bonum , } et non fauorem , \\\hline
1.2.21 & ¶ \textbf{ la terçera propiedat del magnifico es fazer espenssas e dardones delectablemente et sin detenemiento . } Ca aquel que primeramente & et gloriam hominum . \textbf{ Tertia proprietas magnifici , est facere sumptus et distributiones delectabiliter } et prompte . Nam qui prius quam sumptus faciat , \\\hline
1.2.21 & Ca aquel que primeramente \textbf{ ante que faga las despenssas quiere tomar cuenta dellas } e quiere contar todas las despenssas & et prompte . Nam qui prius quam sumptus faciat , \textbf{ diu ratiocinatur , } et omnes minutas expensas inquirit , \\\hline
1.2.21 & ante que faga las despenssas quiere tomar cuenta dellas \textbf{ e quiere contar todas las despenssas } por menudo & diu ratiocinatur , \textbf{ et omnes minutas expensas inquirit , } et non delectabiliter , \\\hline
1.2.21 & por menudo \textbf{ e non da delectablemente sin detenimiento aquello que ha de dar } non es dicho magnifico & et omnes minutas expensas inquirit , \textbf{ et non delectabiliter , | et prompte largitur quae largiti debet , } non est magnificus , \\\hline
1.2.21 & mas pariufico \textbf{ que quiere dezir de pequena fazienda . } Et por ende dize el philosofo . & non est magnificus , \textbf{ sed paruificus . } Ideo dicitur 4 Ethic’ \\\hline
1.2.21 & en el quarto libro delas etris \textbf{ que la grant auariçia de tomar cuenta faze al omne de poca fazienda ¶ } La quarta propiedat es & Ideo dicitur 4 Ethic’ \textbf{ quod diligentia ratiocinii est paruifica . Quarta , } est magis intendere qualiter faciat opus optimum , \\\hline
1.2.21 & La quarta propiedat es \textbf{ que el magnifico deue mas entender } en qual manera faga obra muy buena & quod diligentia ratiocinii est paruifica . Quarta , \textbf{ est magis intendere qualiter faciat opus optimum , } et decentissimum , \\\hline
1.2.21 & e muy conuenible \textbf{ que entender en qual manera e quanta despenssa fara en aquella obra . } assi commo si el magnifico ouiese de fazer algun tenplo & et decentissimum , \textbf{ quam qualiter , | et quantum sumptum faciat ad opus illud : } ut si debet magnificus aliquod templum construere in honorem diuinum , \\\hline
1.2.21 & que entender en qual manera e quanta despenssa fara en aquella obra . \textbf{ assi commo si el magnifico ouiese de fazer algun tenplo } es alguna eglesia en honrra de dios & et quantum sumptum faciat ad opus illud : \textbf{ ut si debet magnificus aliquod templum construere in honorem diuinum , } vel aliqua dona conferre personis dignis , \\\hline
1.2.21 & es alguna eglesia en honrra de dios \textbf{ o ouiesse de dar e de partir alguons dones a personas dignis } mas deue entender & ut si debet magnificus aliquod templum construere in honorem diuinum , \textbf{ vel aliqua dona conferre personis dignis , } magis intendet quomodo templum illud sit admirabile \\\hline
1.2.21 & o ouiesse de dar e de partir alguons dones a personas dignis \textbf{ mas deue entender } en qual manera aquel tenplo o aquella eglesia sera marauillosa e muy fermosa & ut si debet magnificus aliquod templum construere in honorem diuinum , \textbf{ vel aliqua dona conferre personis dignis , } magis intendet quomodo templum illud sit admirabile \\\hline
1.2.21 & e conuenibles \textbf{ que entender e cuydar quantos des } e quanto auer le conuiene ael de despender en estas obras ¶ & et quomodo dona illa sint magna et decentia , \textbf{ quam quot et quanta numismata oporteat } ipsum consumere propter huiusmodi opera . \\\hline
1.2.21 & que entender e cuydar quantos des \textbf{ e quanto auer le conuiene ael de despender en estas obras ¶ } La quinta propiedates & quam quot et quanta numismata oporteat \textbf{ ipsum consumere propter huiusmodi opera . } Quinta proprietas est , \\\hline
1.2.21 & si faz conuenibles espenssas \textbf{ faz omne ser liberal fazer muy grandes } e muy conuenibles espenssas & qui in magnis operibus facit decentes sumptus : \textbf{ si facere decentes sumptus est esse liberalem , } facere maximos decentes sumptus , \\\hline
1.2.21 & por muy pequana cosa pierden lo mucho . \textbf{ Et pues que assi es el magnifico aqui parte nesçe de non auer cuydado de contar } lo que despiende de despenssa & pro modico multum perdunt . Magnificus ergo , \textbf{ cui non est curae de ratiocinio ab aequali sumptu , } facit opus magis magnificum , \\\hline
1.2.21 & mas en alguna obra que el libal \textbf{ por que sienpre se esfuerça el auariento de tirar algunas despenssas conuenibles } avn que sean pequeñas & quam liberalem : \textbf{ tamen quia semper nititur auarus aliquas decentes expensas subtrahere , } etiam \\\hline
1.2.21 & e en los prinçipes \textbf{ non es guaue delo veer . } por que ellos mucho mas deuen ser sabios & quam paruificus , et auarus . Has autem proprietates debere inesse Regibus et Principibus , \textbf{ videre non est difficile : } ipsi enim maxime debent esse scientes , \\\hline
1.2.21 & e conosçedores quales despenssas a quales obras conuienen . \textbf{ Et aellos otrosi mucho mas pertenesçe de fazer grandes donaconnes } e lobre puiantes de espenssas & qui sumptus quibus operibus deceant . \textbf{ Ad eos autem maxime spectat facere magnas largitiones , } et excellentes sumptus boni gratia \\\hline
1.2.21 & por razon e por fin de bien . \textbf{ por que aquel que deue enderesçar } e gouernar todo el regno en bien mucho & et excellentes sumptus boni gratia \textbf{ Nam maxime } debet bonum intendere , \\\hline
1.2.21 & por que aquel que deue enderesçar \textbf{ e gouernar todo el regno en bien mucho } mas deue entender a bien & et excellentes sumptus boni gratia \textbf{ Nam maxime } debet bonum intendere , \\\hline
1.2.21 & e gouernar todo el regno en bien mucho \textbf{ mas deue entender a bien } que otro ninguno & Nam maxime \textbf{ debet bonum intendere , } qui regnum \\\hline
1.2.21 & que otro ninguno \textbf{ e avn essa misma manera pertenesçe a ellos fazer despenssas muy delectablemente e sin detenimiento . } Ca assi commo dicho es de suso & qui regnum \textbf{ debet in bonum dirigere . Sic etiam ad eos spectat delectabiliter , | et prompte sumptus facere . } Nam \\\hline
1.2.21 & e en riquezas \textbf{ tantomas les conuiene aellos de fazer mayores particonnes e mayores dones } e mas espender delectable ment en sin detenimiento & et diuitiis , \textbf{ tanto magis decet eos ampliores retributiones facere , } et magis delectabiliter , \\\hline
1.2.21 & tantomas les conuiene aellos de fazer mayores particonnes e mayores dones \textbf{ e mas espender delectable ment en sin detenimiento } Et otrosi avn conuiene alos Reyes & tanto magis decet eos ampliores retributiones facere , \textbf{ et magis delectabiliter , | et prompte expendere . Decet } etiam Reges , \\\hline
1.2.21 & Et otrosi avn conuiene alos Reyes \textbf{ e alos prinçipes de entender e cuydar } mas en qual manera deuen fazer obras muy grandes de uirtudesque cuydar & etiam Reges , \textbf{ et Principes magis intendere , } quomodo faciant excellentia opera virtutum , \\\hline
1.2.21 & e alos prinçipes de entender e cuydar \textbf{ mas en qual manera deuen fazer obras muy grandes de uirtudesque cuydar } en qual manera guarden los des & et Principes magis intendere , \textbf{ quomodo faciant excellentia opera virtutum , } quam quomodo parcant nummis et expensis . Oportet etiam eos esse excellenter liberales , \\\hline
1.2.21 & e perdonen alas espenssas \textbf{ para las non fazer } Otrosi avn conuiene alos Reyes & quam quomodo parcant nummis et expensis . Oportet etiam eos esse excellenter liberales , \textbf{ et semper facere magnifica opera . Omnes igitur proprietates magnifici per amplius , et perfectius decet ipsos Reges habere . Unde et Philos’ 4 Ethic’ vult , } quod non quilibet possit esse magnificus : \\\hline
1.2.21 & e alos prinçipes de ser liberales muy altamente \textbf{ e de fazer sienpre obras muy grandes e magnificas . } Et pues que assi es todas las propiedades del magnifico conuiene auer a los Reyes & quod non quilibet possit esse magnificus : \textbf{ quia non quilibet potest facere magnos sumptus . } Sed , \\\hline
1.2.21 & e de fazer sienpre obras muy grandes e magnificas . \textbf{ Et pues que assi es todas las propiedades del magnifico conuiene auer a los Reyes } e ales prinçipes & quia non quilibet potest facere magnos sumptus . \textbf{ Sed , } ut ibidem dicitur , \\\hline
1.2.21 & que non pue de cada vno ser magnifico \textbf{ por que non puede cada vno fazer grandes espenssas } Mas assi commo alli dize el philosofo tales son los nobles e los głiosos . & tales oportet esse nobiles \textbf{ et gloriosos . } Quare quanto est nobilior aliis , \\\hline
1.2.21 & que los otros \textbf{ en tanto mas le conuiene ael de resplandesçer } por magnificençia e auer propiedades de magnifico & Quare quanto est nobilior aliis , \textbf{ tanto decet ipsum pollere magnificentia , } et habere proprietates magnifici . \\\hline
1.2.21 & en tanto mas le conuiene ael de resplandesçer \textbf{ por magnificençia e auer propiedades de magnifico } e de ome muy guanade & tanto decet ipsum pollere magnificentia , \textbf{ et habere proprietates magnifici . } Bonorum exteriorum \\\hline
1.2.22 & Et otra es que cata las honrras medianas \textbf{ que comunalmente se puede dezir uirtud amadora de honrra } Mas en tres maneras se puede cada vno auer en las grandes honrras & et alia quae respicit mediocres , \textbf{ quae communi nomine dici potest honoris amatiua . In magnis autem honoribus tripliciter quis se habere potest . } Nam quidam in talibus deficiunt , \\\hline
1.2.22 & que comunalmente se puede dezir uirtud amadora de honrra \textbf{ Mas en tres maneras se puede cada vno auer en las grandes honrras } Ca alguon sson & et alia quae respicit mediocres , \textbf{ quae communi nomine dici potest honoris amatiua . In magnis autem honoribus tripliciter quis se habere potest . } Nam quidam in talibus deficiunt , \\\hline
1.2.22 & Et estos son dichos magnanimos \textbf{ que quiere dezir omes de grand coraçon ca nos ueemos algunos } que dessi son aptos e apareiados & ut magnanimi . \textbf{ Videmus enim aliquos de se aptos ad magna , } potentes magna et ardua exercere : \\\hline
1.2.22 & que dessi son aptos e apareiados \textbf{ para fazer grandes cosas . } Et poderosos para vsar de cosas grandes e altas . & Videmus enim aliquos de se aptos ad magna , \textbf{ potentes magna et ardua exercere : } quadam tamen pusillanimitate ducti , \\\hline
1.2.22 & para fazer grandes cosas . \textbf{ Et poderosos para vsar de cosas grandes e altas . } Enpero por fiaqueza de coraçon retrahen se destas cosas guanadas & Videmus enim aliquos de se aptos ad magna , \textbf{ potentes magna et ardua exercere : } quadam tamen pusillanimitate ducti , \\\hline
1.2.22 & Enpero por fiaqueza de coraçon retrahen se destas cosas guanadas \textbf{ que poderan muy bien fazer . } Et por ende estos fallesçen en tales cosas & quadam tamen pusillanimitate ducti , \textbf{ retrahunt se ab huiusmodi magnis . } Tales ergo in talibus deficiunt . \\\hline
1.2.22 & que fazen el contrario \textbf{ que sobrepuian en poner se adelante } para fazer algunas cosas & Sed quidam econtrario superabundant , \textbf{ ingerentes se ad aliqua , } quae digne complere non possunt , \\\hline
1.2.22 & que sobrepuian en poner se adelante \textbf{ para fazer algunas cosas } que digna mientre non las pueden conplir . & Sed quidam econtrario superabundant , \textbf{ ingerentes se ad aliqua , } quae digne complere non possunt , \\\hline
1.2.22 & para fazer algunas cosas \textbf{ que digna mientre non las pueden conplir . } Et estos tałs ̃ llama el philosofo & ingerentes se ad aliqua , \textbf{ quae digne complere non possunt , } quos Philosophus vocat chaunos , \\\hline
1.2.22 & Et estos tałs ̃ llama el philosofo \textbf{ caymos que quiere dezir fumosos e ventosos } mas nos podemos los llamar prasunptuosos & ø \\\hline
1.2.22 & caymos que quiere dezir fumosos e ventosos \textbf{ mas nos podemos los llamar prasunptuosos } Pues que assi es el magnanimo es medianero entre el pusillanimo & ø \\\hline
1.2.22 & que es aquel que comete aquellas cosas \textbf{ que non puede bien conplir . } Et por ende el magria nimonen se retraye & idest fimosos \textbf{ et ventosos . Nos autem eos praesumptuosos vocare possumus . Magnanimus vero medius est inter pusillanimum , et praesumptuosum . Non enim se retrahit ab arduis operibus , } quae potest digne agere , \\\hline
1.2.22 & nin fuye delas obras altas \textbf{ que puede bien fazer } assi commo el pusillanimo & et ventosos . Nos autem eos praesumptuosos vocare possumus . Magnanimus vero medius est inter pusillanimum , et praesumptuosum . Non enim se retrahit ab arduis operibus , \textbf{ quae potest digne agere , } ut pusillanimus , \\\hline
1.2.22 & nin se mete mas adelante \textbf{ en aquello que non puede conplir } assi commo faze el presuptuoso . & nec se ingerit ad ea , \textbf{ quae digne complere non potest , } ut praesumptuosus . \\\hline
1.2.22 & Por ende es dicha uirtud que repreme las auariçias \textbf{ e tienpra los gastamientos en espender . } En essa misma manera la magnanimidat & ideo est virtus quaedam reprimens auaritias , \textbf{ et moderans prodigalitates : } sic magnanimitas , \\\hline
1.2.22 & e tienpra las presunpçiones \textbf{ que son sobrepuiamientos en cometer las grandes cosas } que non pueden acabar . & quid est magnanimitas , \textbf{ de leui patet , circa } quae habet esse . Videtur autem Philosophus 4 Ethicor’ velle , \\\hline
1.2.22 & que son sobrepuiamientos en cometer las grandes cosas \textbf{ que non pueden acabar . } ¶ Visto que cosa es la uirtud & de leui patet , circa \textbf{ quae habet esse . Videtur autem Philosophus 4 Ethicor’ velle , } magnanimitatem esse circa honores , \\\hline
1.2.22 & Por que los omes las mas vezes ordenan estas cosas \textbf{ para alcançar } por ellas honrra . & quia ut plurimum homines haec ordinant , \textbf{ ut consequantur honores . } Sicut ergo dicebatur de fortitudine , \\\hline
1.2.22 & Ca el magnanimo conosçe \textbf{ que conueniblemente se deue auer en qual si quier estado . } Mas al pusill animo & non deiicitur . \textbf{ In quolibet enim statu nouit magnanimus se decenter habere . } Ad pusillanimem enim pertinet nescire fortunas ferre . \\\hline
1.2.22 & Mas al pusill animo \textbf{ e de flaco coraçon pertenesçe non saber sofrir buenas uenturas . } Por ende dize andronico el sabio philosofo & In quolibet enim statu nouit magnanimus se decenter habere . \textbf{ Ad pusillanimem enim pertinet nescire fortunas ferre . } Ideo Andron’ Perip’ ait : \\\hline
1.2.22 & que las obras dela pusilla nimidat son tales \textbf{ que non pueden sofrir honrras } nin desonrras & Opera pusillanimitatis esse , \textbf{ quae neque honorem , } neque inhonorationem , \\\hline
1.2.22 & ¶Mostrado que cosa es la magnan midat \textbf{ e cerca quales cosas ha de ser sinca de demostrar } en qual manera podemos a nos mismos fazer magnanimos . & et parum bene fortunatum extolli . Ostenso \textbf{ quid est magnanimitas , et circa quae habet esse . Restat ostendere , quomodo possumus nosipsos magnanimos facere . } Inter caetera autem , \\\hline
1.2.22 & e cerca quales cosas ha de ser sinca de demostrar \textbf{ en qual manera podemos a nos mismos fazer magnanimos . } Mas entre todas las otras cosas & et parum bene fortunatum extolli . Ostenso \textbf{ quid est magnanimitas , et circa quae habet esse . Restat ostendere , quomodo possumus nosipsos magnanimos facere . } Inter caetera autem , \\\hline
1.2.22 & que inclinan anos a magnanimidat \textbf{ es poco preçiar } todos los bienes de fuera & quae nos ad magnanimitatem trahunt , \textbf{ est parua pretiari exteriora bona , } quaecunque sint illa , \\\hline
1.2.22 & Ca dicho es de suso \textbf{ que el pusillanimo non sabe sofrir buenas uenfas } mas de muy pequana buena uentura se leunata en vana gloria & siue quaecunque alia huiusmodi bona . \textbf{ Dictum est enim pusillanimem nescire fortunas ferre , } sed ex modico fortunio extolli , \\\hline
1.2.22 & mas el magranimo non taze ali . \textbf{ Ca assi commo dicho es de suso el magnanimo sabe sosrir buenas venturas e sabe conueniblemente se auer en todo estado } Mas la razon & sed ( ut dicebatur ) nouit fortunas ferre , \textbf{ et in quolibet statu scit se decenter habere . Causa autem , } quare quis nescit fortunas ferre , est , \\\hline
1.2.22 & Mas la razon \textbf{ por que algunos non sabe sofrir buenas uenturas } es por que preçia mucho los bienes de fuera . & et in quolibet statu scit se decenter habere . Causa autem , \textbf{ quare quis nescit fortunas ferre , est , } quia nimis appretiatur exteriora bona . Ideo quando aliquid \\\hline
1.2.22 & en todas manerasseremos magnanimos \textbf{ e sabremos sofrir buenas u entraas . } Ca si fuermos acreçentados en riquezas & omnino erimus magnanimi , \textbf{ et sciemus fortunas ferre . } Nam si augebimur in diuitiis , \\\hline
1.2.22 & en estos bienes tenporales \textbf{ sofrir la hemos conueniblemente } por que los non tenemos en mucho & Et si contingat nos infortunari circa ea , decenter tolerabimus , \textbf{ cum non multum reputemus ipsa . } Recitat autem Philosophus 4 Ethicor’ multas proprietates magnanimi : \\\hline
1.2.23 & por que los non tenemos en mucho \textbf{ e dedes saber } que el philosofo pone en el quarto libro delas ethicas muchas propiedades del magnanimo & cum non multum reputemus ipsa . \textbf{ Recitat autem Philosophus 4 Ethicor’ multas proprietates magnanimi : } inter quas possumus numerare sex , \\\hline
1.2.23 & que el philosofo pone en el quarto libro delas ethicas muchas propiedades del magnanimo \textbf{ delas quales podemos tomar las seys } las quales deuen auer los prinçipes e los Reyes ¶ & Recitat autem Philosophus 4 Ethicor’ multas proprietates magnanimi : \textbf{ inter quas possumus numerare sex , } quas Reges , \\\hline
1.2.23 & delas quales podemos tomar las seys \textbf{ las quales deuen auer los prinçipes e los Reyes ¶ } La primera propriedat del magnanimo es que se deue bien auer çerca los periglos & inter quas possumus numerare sex , \textbf{ quas Reges , | et Principes habere debent . } Prima proprietas magnanimi , \\\hline
1.2.23 & las quales deuen auer los prinçipes e los Reyes ¶ \textbf{ La primera propriedat del magnanimo es que se deue bien auer çerca los periglos } Mas auer se bien çerca los periglos & et Principes habere debent . \textbf{ Prima proprietas magnanimi , | est bene se habere circa pericula . } Bene autem se habere circa ea , \\\hline
1.2.23 & La primera propriedat del magnanimo es que se deue bien auer çerca los periglos \textbf{ Mas auer se bien çerca los periglos } es non ser amador de los periglos & est bene se habere circa pericula . \textbf{ Bene autem se habere circa ea , } est non esse amatorem periculorum , \\\hline
1.2.23 & es non ser amador de los periglos \textbf{ nin esponer su cuerpo a periglos } por pequanans cosas & est non esse amatorem periculorum , \textbf{ neque se exponere pro paruis periculis , } sed pro magnis , \\\hline
1.2.23 & mas por las grandes \textbf{ e por tales delas quales se puede leunatar } e muy grand prouecho & sed pro magnis , \textbf{ ut pro iis ex quibus potest consurgere magna utilitas . } Cum autem sic se periculis exponit , \\\hline
1.2.23 & que parte nesçe al magnanimo \textbf{ es auer se bien çerca las particones delos dones } dando a cada vno gualardon commo lo meresçe . & Magnanimus enim parum appreciatur exteriora bona , \textbf{ et multum appreciatur opera virtutum . } Propter quod , \\\hline
1.2.23 & e dador de los galardones \textbf{ es fazer obras de uirtudes . } Ca assi commo dize el philosofo en el quarto libro delas ethicas pertenesce mucho almagnanimo ser mucho partidor e dador de gualardones ¶ & quia esse plurimum retributiuum , \textbf{ est agere opera virtutum , } conuenit magnanimo esse plurimum retributiuum , \\\hline
1.2.23 & assi commo cerca aquellas \textbf{ de que se pueden le unatar grandeshonrras . } Et tales cosas commo estas non contesçen muchas vezes . & ut circa ea , \textbf{ ex quibus consurgere possunt magni honores ; } talia autem non multotiens occurrunt , \\\hline
1.2.23 & si bien le entendieren conuiene alos Reyes \textbf{ e alos prinçipes delas auer } Por que conuiene alos Reyes non se poner & quam quod precibus comparatur . Has autem proprietates narratas a Philosopho , \textbf{ si bene intelligantur , } habere decet Reges , et Principes . Decet enim Reges se non exponere pro quibuscunque periculis , \\\hline
1.2.23 & e alos prinçipes delas auer \textbf{ Por que conuiene alos Reyes non se poner } a quales se quier periglos & si bene intelligantur , \textbf{ habere decet Reges , et Principes . Decet enim Reges se non exponere pro quibuscunque periculis , } nisi pro negociis arduis , \\\hline
1.2.23 & Mas si acaesçiere algun caso tan alto \textbf{ por que al Rey conuenga de poner su gente } o avn assi mismo aperigłsᷤen & vel pro tuitione regni . Si vero occurrat casus adeo arduus , \textbf{ quod Rex gentem suam , } vel etiam seipsum debeat exponere periculis : \\\hline
1.2.23 & que por el bien de dios \textbf{ e por el bien comun sea apareiado de esponer su uida } e ponersea la muerte & et magna nimus , \textbf{ ut pro bono diuino et communi , paratus sit vitam exponere . } Secundo decet Reges , \\\hline
1.2.23 & que los otros tanto \textbf{ mas deuen sobrepuiar los otros } en obras de uirtudes & quam alii , \textbf{ et quanto plus affluunt diuitiis quam alii : } tanto in operibus virtutum , \\\hline
1.2.23 & Ca non conuiene alos Reyes \textbf{ nin alos prinçipes desenbargar } por si mismos todos los negoçios & negocia enim ardua pauca sunt respectu aliorum . \textbf{ Non autem decet Reges et Principes omnia negocia quantumcumque modica expedire per seipsos , } nec decet eos omnium esse operatiuos ; \\\hline
1.2.23 & nin conuiene aellos de seer obradores de todas las cosas \textbf{ mas por que puedan mas liberalmente entender a desenbargar los grandes negoçios que son pocos deuena comne dar los otros negoçios pequa nons } que son muchos alos otros ¶ & nec decet eos omnium esse operatiuos ; \textbf{ sed ut possit liberius intendere expeditioni negociorum magnorum quae sunt pauca , } debet alia minora negocia aliis committere , quae sunt multa . \\\hline
1.2.23 & por que son regla de los otros \textbf{ La qual regla non se deue torcer nin falssar } Et ahun conuiene les de seer manifiestos aborresçedores e amadores & cum sint regula aliorum , \textbf{ quae obliquari , | et falsificari non debet . } Decet etiam eos esse manifestos oditores , \\\hline
1.2.23 & Lo quinto conuiene alos Reyes e alos prinçipes \textbf{ de non auer cuydado } que sean alabados de los omes . & ut sit iustus , \textbf{ et virtuosus . Quinto decet Reges , et Principes non curare , } ut laudentur ab hominibus . \\\hline
1.2.23 & Mas si creyeren atales lisongeros conuienea ellos \textbf{ de non obrar segunt las leyes } et segunt razon & contingit eos non agere \textbf{ secundum legem et rationem , } sed secundum passionem et voluntatem . \\\hline
1.2.23 & et segunt razon \textbf{ mas obrar segunt pasion } e segunt la uoluntad . & secundum legem et rationem , \textbf{ sed secundum passionem et voluntatem . } Tanto igitur Reges et Principes , \\\hline
1.2.23 & quanto mas han de lisongeros \textbf{ los quales loando los se esfuercana los tris tornar } e alos destroyr ¶ & quanto plures habent adulatores , \textbf{ qui eos laudando conantur ipsos peruertere . | Sexto } quia Reges , \\\hline
1.2.23 & los quales loando los se esfuercana los tris tornar \textbf{ e alos destroyr ¶ } Lo sexto es que los Reyes e los prinçipes deuen ser conplidos & Sexto \textbf{ quia Reges , } et Principes debent sibi esse sufficientes in exterioribus bonis , \\\hline
1.2.23 & por los bienes de fuera . \textbf{ Et pues que assi es todas las dichas propiedades deuen part enesçer alos Reyes e alos prinçipes . } por la qual cosa les conuienea ellos de seer magnanimos . ¶ & non decet eos esse plangitiuos , \textbf{ vel deprecatiuos pro exterioribus bonis . Omnes ergo assignatae proprietates competere debent Regibus et Principibus . Quare decet eos esse magnanimos . } Amatores honorum aliquando vituperantur , \\\hline
1.2.24 & por la qual cosa les conuienea ellos de seer magnanimos . ¶ \textbf{ euedes saber que los amadores de las honrras algunas vezes son denostados } e algunas vezes son loados & vel deprecatiuos pro exterioribus bonis . Omnes ergo assignatae proprietates competere debent Regibus et Principibus . Quare decet eos esse magnanimos . \textbf{ Amatores honorum aliquando vituperantur , } aliquando vero laudantur , \\\hline
1.2.24 & Et por ende lo amos aquellos que non curan de su honrra . \textbf{ ¶ Pues que assi es auer cuydado ome de su propia honrra } en vna manera es de loar & et rursus quia vituperamus ambitiosos laudamus non curantes \textbf{ de honore suo . | Curare igitur de proprio honore , } uno modo est laudabile , \\\hline
1.2.24 & ¶ Pues que assi es auer cuydado ome de su propia honrra \textbf{ en vna manera es de loar } e en otra manera es de denostar . & Curare igitur de proprio honore , \textbf{ uno modo est laudabile , } et alio vituperabile . \\\hline
1.2.24 & en vna manera es de loar \textbf{ e en otra manera es de denostar . } Ca non auer cuydado dela honrra & uno modo est laudabile , \textbf{ et alio vituperabile . } Nam non curare de honore , \\\hline
1.2.24 & e en otra manera es de denostar . \textbf{ Ca non auer cuydado dela honrra } por que non quiere fazer obras dignas de honrra esto es de denostar & et alio vituperabile . \textbf{ Nam non curare de honore , } quia non vult agere opera honore digna , \\\hline
1.2.24 & Ca non auer cuydado dela honrra \textbf{ por que non quiere fazer obras dignas de honrra esto es de denostar } mas auer cuydado de honrra & Nam non curare de honore , \textbf{ quia non vult agere opera honore digna , } vituperabile est . Nobis igitur \\\hline
1.2.24 & por que non quiere fazer obras dignas de honrra esto es de denostar \textbf{ mas auer cuydado de honrra } en quanto quiere fazer obras dignas de honrra & quia non vult agere opera honore digna , \textbf{ vituperabile est . Nobis igitur } debet esse curae de honore , \\\hline
1.2.24 & mas auer cuydado de honrra \textbf{ en quanto quiere fazer obras dignas de honrra } esto es de loar . & vituperabile est . Nobis igitur \textbf{ debet esse curae de honore , } non quod simus ambitiosi , \\\hline
1.2.24 & en quanto quiere fazer obras dignas de honrra \textbf{ esto es de loar . } Et por ende nos deuemos auer cuydado de honrra & vituperabile est . Nobis igitur \textbf{ debet esse curae de honore , } non quod simus ambitiosi , \\\hline
1.2.24 & mas por que fagos obras dignas de honrra . \textbf{ Mas las obras dignas de honrra se puede entender en dos maneras . } O en quanto son proporçionadas anos . & nec quod finem nostrum ponamus in honoribus , \textbf{ sed quod agamus opera honore digna . Opera autem honore digna dupliciter considerari possunt . } Vel ut sunt proportionata nobis . \\\hline
1.2.24 & E en quanto son dignas de grant honrra . \textbf{ Onde assi cerca las despenssas son dos uirtudes conuiene saber La liƀalidat Et franqza } que cata alas despenssas & Vel ut sunt digna magno honore . \textbf{ Sicut enim circa sumptus sunt duae virtutes , | videlicet , } liberalitas , \\\hline
1.2.24 & Por que todas las obras delas uirtudes son dignas de honrra \textbf{ por la qual cosa el magnanimo obrar } a obras de cada vna delas uirtu des singulares . & sic magnanimitas est quidam ornatus omnium virtutum . \textbf{ Nam opera omnium virtutum sunt honore digna . } Quare magnanimus operabitur opera singularum virtutum . Ideo dicitur quarto Ethicorum , \\\hline
1.2.24 & Et por ende dize el philosofo en el quarto libro delas ethicas \textbf{ que non pertenesce almagranimo foyr de aquel qual bien conseia . } Ca esto es obra de pradençia & Quare magnanimus operabitur opera singularum virtutum . Ideo dicitur quarto Ethicorum , \textbf{ quod non congruit magnanimo fugere commouentem , } quod est actus prudentiae , \\\hline
1.2.24 & Ca esto es obra de pradençia \textbf{ nin parte nesçe de fazer cosas non iustas } o que non conuienen ala iustiçia & quod est actus prudentiae , \textbf{ nec facere iniusta , quod pertinet ad iniustitiam . } Operatur ergo magnanimus opera aliarum virtutum , \\\hline
1.2.24 & Et pues que assi es el magranimo obrara las obras delas otras uirtudes \textbf{ e fazer las ha mas altamente . } Ca commo la honrra entre los bienes de fuera sea mas alto & Operatur ergo magnanimus opera aliarum virtutum , \textbf{ et ea faciet excellenter . } Nam cum honor inter exteriora bona sit bonum excellens , \\\hline
1.2.24 & Conuiene alos Reyes \textbf{ e alos prinçipes amar las honrras } en la manera que dich̃ones de suso . & sic decet eos esse magnanimos , et honoris amatiuos . Reges enim et Principes decet honores \textbf{ diligere modo } quo dictum est ; \\\hline
1.2.24 & en la manera que dich̃ones de suso . \textbf{ Conuiene saber } que amen e cobdicien fazer lobras & videlicet , \textbf{ ut diligant et cupiant facere opera , } quae sint honore digna . \\\hline
1.2.24 & Conuiene saber \textbf{ que amen e cobdicien fazer lobras } que sean dignas de honrra . & videlicet , \textbf{ ut diligant et cupiant facere opera , } quae sint honore digna . \\\hline
1.2.25 & mas llamale tenprado . \textbf{ Et por ende auer algun tenpramiente en las honrras es esso mismo } que auer humildat . & sed temperatum . \textbf{ Cum igitur habere temperantiam in honoribus , | sit idem , } quod habere humilitatem : \\\hline
1.2.25 & Et por ende auer algun tenpramiente en las honrras es esso mismo \textbf{ que auer humildat . } Et aquella uirtud o razon de tenprança trahe consigo vna humildat de coraçon & sit idem , \textbf{ quod habere humilitatem : | virtus illa , } quae moderationem animi , \\\hline
1.2.25 & en el quarto libro delas ethicas despreçia alos otros . \textbf{ Et pues que assi es deuedes saber } que si todo magranimo non fuesse humildoso & ( ut dicitur 4 Ethicorum ) alios despicit . \textbf{ Sciendum igitur quod nisi omnis magnanimus esset humilis , } non dedissemus principibus congruum documentum , \\\hline
1.2.25 & si non ouiessen humildat \textbf{ por que sin la humildat las otras uirtudes non se pueden auer Et } pues que assi es para auer conplida declaraçion de la uerdat conuiene de demostrar & quia docuissemus eos esse sine virtutibus , \textbf{ cum absque humilitate virtutes haberi non possint . } Ad plenam igitur declarationem veritatis ostendendum est , \\\hline
1.2.25 & por que sin la humildat las otras uirtudes non se pueden auer Et \textbf{ pues que assi es para auer conplida declaraçion de la uerdat conuiene de demostrar } que sin la humildat ninguno non puede ser magnanimo . & cum absque humilitate virtutes haberi non possint . \textbf{ Ad plenam igitur declarationem veritatis ostendendum est , } quod sine humilitate nullus potest esse magnanimus . Magnanimitas autem , et humilitas , et huiusmodi virtutes morales , \\\hline
1.2.25 & por que tienpran las nr̃as passiones \textbf{ en manera que non podamos desuiarnos del bien de razon . } Mas que las passiones nos fagan desuiar del bien de razon & quia moderant passiones nostras , \textbf{ ne deuiemus a bono rationis . Sed quod passiones deuiare nos faciant } a bono \\\hline
1.2.25 & en manera que non podamos desuiarnos del bien de razon . \textbf{ Mas que las passiones nos fagan desuiar del bien de razon } en dos maneras puede contesçer & quia moderant passiones nostras , \textbf{ ne deuiemus a bono rationis . Sed quod passiones deuiare nos faciant } a bono \\\hline
1.2.25 & Mas que las passiones nos fagan desuiar del bien de razon \textbf{ en dos maneras puede contesçer } segunt que dicho es de suso ¶ primeramente & ne deuiemus a bono rationis . Sed quod passiones deuiare nos faciant \textbf{ a bono | secundum rationem } ( ut supra dicebatur ) dupliciter potest contingere . Primo , \\\hline
1.2.25 & que nos tire de aquello mas auemos menester uirtud \textbf{ que nos ayude a conplir aquello } asi commo paresçe en la fortaleza & non indigemus virtute retrahente , \textbf{ sed magis impellente , } ut patet in fortitudine , \\\hline
1.2.25 & nin nos arriedra dellas . \textbf{ Et por ende commo tirar nos } de aquello que la razon manda & quam retrahat nos ab illis . \textbf{ Cum ergo retrahere et impellere sint quodammodo opposita , } et formaliter differant , \\\hline
1.2.25 & de aquello que la razon manda \textbf{ e allegarnos } a esso mesmo sean dos cosas contrarias & Cum ergo retrahere et impellere sint quodammodo opposita , \textbf{ et formaliter differant , } aeque principaliter \\\hline
1.2.25 & e nos allega a aquello que la razon manda o uieda . \textbf{ Conuiene de dar en aquella cosa dos uirtudes ¶ } La vna que nos allegue . & quod si unum \textbf{ et idem aliter } et aliter acceptum nos retrahit et impellit , \\\hline
1.2.25 & Et pues que assi es cerca las grandeshonrras \textbf{ e cerca los grandes bienes contesçe de pecar en dos maneras ¶ } Lo primero & ut tendamus in ipsum . Ergo circa magnos honores , \textbf{ et circa magna bona dupliciter contingit peccare . Primo , } si ultra quam ratio dictet prosequamur ea inquantum bona sunt . Secundo , \\\hline
1.2.25 & si mas que la razon e el entendimientomanda nos tiremos de aquellas cosas \textbf{ por que son muy altas e muy graues de alcançar } ¶Lo segundo si desmesuradamente & et circa magna bona dupliciter contingit peccare . Primo , \textbf{ si ultra quam ratio dictet prosequamur ea inquantum bona sunt . Secundo , } si infra quam ratio dictet retrahamus nos ab illis , \\\hline
1.2.25 & e mas de quanto manda la razon fueremos \textbf{ e quisieremos alcançar aquellas cosas } por razon dela bondat & si ultra quam ratio dictet prosequamur ea inquantum bona sunt . Secundo , \textbf{ si infra quam ratio dictet retrahamus nos ab illis , } eo quod ardua et difficilia sint . \\\hline
1.2.25 & La vna que nos allegue adelante \textbf{ por que non nos podamos tirar atras } por la graueza que es en alcançar . & Quare circa talia duplici virtute indigemus . Una impellente nos , \textbf{ ne retrahamur propter difficultatem , } et haec est magnanimitas : \\\hline
1.2.25 & por que non nos podamos tirar atras \textbf{ por la graueza que es en alcançar . } Et esta uirtud es magnammidat ¶ & Quare circa talia duplici virtute indigemus . Una impellente nos , \textbf{ ne retrahamur propter difficultatem , } et haec est magnanimitas : \\\hline
1.2.25 & que aella sea ayuntada la humildat \textbf{ por que non pueda passar allende de quanto la razon manda } por razon dela bondat . & oportet quod ei sit annexa humilitas , \textbf{ ne ultra quam ratio dictet tendat in ea ratione bonitatis , } aliter enim esset vitiosus . Tendere igitur in magnos honores , et ad magna bona , esse potest a magnanimitate , \\\hline
1.2.25 & Ca en otra manera el magnani mo seria vicioso e pecaria . \textbf{ Et pues que assi es yr a grandes honrras } e a grandes biens puede ser por dos uirtudes . & ne ultra quam ratio dictet tendat in ea ratione bonitatis , \textbf{ aliter enim esset vitiosus . Tendere igitur in magnos honores , et ad magna bona , esse potest a magnanimitate , } et ab humilitate : \\\hline
1.2.25 & que mas despreçia las obras e las palabras de los omes \textbf{ que quiera partir se } por ellas delas obras uirtuosas & et sic impellitur a virtute , \textbf{ ut potius despiciat opera et verba hominum , quam velit propter ea desistere ab operibus virtuosis . } Humilitas autem e contrario principaliter retrahit : \\\hline
1.2.25 & por razon dela guaueza \textbf{ que non pueda alcançar obras dignas de grand sonrra . } Mas la humildat prinçipalmente tienpra la esꝑanca & ne aliquis ratione difficultatis desperet , \textbf{ ne tendat in opera magno honore digna . Humilitas vero principaliter moderat ipsam spem , } ne aliquis nimis sperans \\\hline
1.2.25 & por que aquel es dicho humildoso \textbf{ que tienpra la esperança de ganar grandes honrras } e va a ellas medianeramente . & quia ille dicitur humilis , \textbf{ qui moderans spem ipsam ad adipiscendum honores magnos , } mediocriter tendit in honores illos . \\\hline
1.2.25 & Mas si la humildat es essa misma cosa sinplemente \textbf{ que amar las honrras medianeras . } O si es essa misma cosa sinplemente con aquella uirtud & Utrum autem humilitas \textbf{ sit idem simpliciter quod diligere mediocres honores , } vel utrum sit idem simpliciter cum virtute illa quam Philosophus distinguens a magnanimitate appellat eam honoris amatiuam ? \\\hline
1.2.25 & en que fablamos . \textbf{ Mas si a nos contesçiese de conponer algunas cosas mas altas en la scian moral . } mostrariamos que la uirtud de que fabla el philosofo non es en toda manera vna cosa misma con la humildat & Non est praesentis speculationis . \textbf{ Sed si in Moralibus contingeret nos ulteriora componere , | ostendemus virtutem } de qua loquitur Philosophus , \\\hline
1.2.26 & e deꝑtimiento dela magranimidat . \textbf{ Ca commo quier que vna e essa misma uirtud es aquella quereꝑme las suꝑ habundançias } e tienpra los fallesçimientos . & humilitatem a magnanimitate differre . \textbf{ Nam licet una et eadem virtus reprimet superabundantias , } et moderet defectus : \\\hline
1.2.26 & e tienpra los fallesçimientos . \textbf{ Empero nunca estas dos cosas egualmente e prinçipalmente pueden parte nesçer a vna uirtud } e por ende pertenesçe ala magranimidat repremir la desparaçion & et moderet defectus : \textbf{ nunquam tamen aeque principaliter haec duo eidem virtuti competere possunt . } Spectat igitur ad magnanimitatem reprimere desperationem , \\\hline
1.2.26 & Empero nunca estas dos cosas egualmente e prinçipalmente pueden parte nesçer a vna uirtud \textbf{ e por ende pertenesçe ala magranimidat repremir la desparaçion } por que non desesꝑemos de los bienes muy altos . & nunquam tamen aeque principaliter haec duo eidem virtuti competere possunt . \textbf{ Spectat igitur ad magnanimitatem reprimere desperationem , } ne desperemus \\\hline
1.2.26 & por que non desesꝑemos de los bienes muy altos . \textbf{ Et otrosi pertenesçe a ella de tenprar la esperança } por que non podamos yr a aquellas cosas & de bonis arduis , \textbf{ et spectat ad ipsam moderare spem , } ne ultra quam ratio dictet tendamus in illa . Non tamen aequae principaliter operatur utrunque : \\\hline
1.2.26 & Et otrosi pertenesçe a ella de tenprar la esperança \textbf{ por que non podamos yr a aquellas cosas } mas que la razon e el entendimiento muestran . Enpero estas dos cosas non las obra egualmente nin prinçipalmente & de bonis arduis , \textbf{ et spectat ad ipsam moderare spem , } ne ultra quam ratio dictet tendamus in illa . Non tamen aequae principaliter operatur utrunque : \\\hline
1.2.26 & mas que la razon e el entendimiento muestran . Enpero estas dos cosas non las obra egualmente nin prinçipalmente \textbf{ Ca commo almagranimo pertenesca de yr } e entender en cosas grandes la magranimidat & et spectat ad ipsam moderare spem , \textbf{ ne ultra quam ratio dictet tendamus in illa . Non tamen aequae principaliter operatur utrunque : } nam cum magnanimi sit tendere in magnum , magnanimitas magis est virtus impellens in magna , \\\hline
1.2.26 & Ca commo almagranimo pertenesca de yr \textbf{ e entender en cosas grandes la magranimidat } mas es uirtud quanos allega & ne ultra quam ratio dictet tendamus in illa . Non tamen aequae principaliter operatur utrunque : \textbf{ nam cum magnanimi sit tendere in magnum , magnanimitas magis est virtus impellens in magna , } quam retrahens nos ab illis . Principalius ergo magnanimitas reprimit desperationem , \\\hline
1.2.26 & por que non podamos rethernos \textbf{ e tirarnos de las cosas altas . } Et despues desto tienpra la esperanca & quam retrahens nos ab illis . Principalius ergo magnanimitas reprimit desperationem , \textbf{ ne retrahamur ab arduis : } et ex consequenti moderat spem , \\\hline
1.2.26 & Et despues desto tienpra la esperanca \textbf{ por que non podamos yr a aquellas cosas } mas que la razon e el entendimiento manda . & ne retrahamur ab arduis : \textbf{ et ex consequenti moderat spem , } ne praeter rationem tendamus in illa . Humilitas autem e contrario se habet , \\\hline
1.2.26 & que es dicha humildança sea apartada dela magnanimidat \textbf{ conuienenos de veer } que cosa es esta uirtud la qual llamamos . humildança . & Quare cum distincta sit virtus haec ab illa , \textbf{ videndum est } quid sit huiusmodi virtus , \\\hline
1.2.26 & que cosa es esta uirtud la qual llamamos . humildança . \textbf{ Et pues que assi es conuiene saber } que assi commo cerca la magnanimidat puede contesçer de sobrepuiar & quam humilitatem vocamus . \textbf{ Sciendum igitur quod sicut circa magnanimitatem conuenit abundare } et deficere : \\\hline
1.2.26 & Et pues que assi es conuiene saber \textbf{ que assi commo cerca la magnanimidat puede contesçer de sobrepuiar } e defallesçer en essa misma manera puede contesçer cerca la humildat . & quam humilitatem vocamus . \textbf{ Sciendum igitur quod sicut circa magnanimitatem conuenit abundare } et deficere : \\\hline
1.2.26 & que assi commo cerca la magnanimidat puede contesçer de sobrepuiar \textbf{ e defallesçer en essa misma manera puede contesçer cerca la humildat . } Ca assi commo paresce delas cosas ya dichas & Sciendum igitur quod sicut circa magnanimitatem conuenit abundare \textbf{ et deficere : | sic et circa humilitatem esse habet . } Nam ( ut patet ex dictis ) magnanimitas est virtus \\\hline
1.2.26 & assi commo la magnanimidat es median era entre la presup̃çion e la pusillanimidat . \textbf{ ¶ Visto que cosa es la humisdat ligeramente puede paresçer çerca quales cosas ha de seer . } Ca el humildoso entiende repremir las soƀͣiuas & Viso \textbf{ quid est humilitas , | de leui videri potest circa quae habet esse . } Intendit enim humilis reprimere superbias , \\\hline
1.2.26 & ¶ Visto que cosa es la humisdat ligeramente puede paresçer çerca quales cosas ha de seer . \textbf{ Ca el humildoso entiende repremir las soƀͣiuas } e tenprar los despreçiamientos e los decaemientos . & de leui videri potest circa quae habet esse . \textbf{ Intendit enim humilis reprimere superbias , } et moderare deiectiones . \\\hline
1.2.26 & Ca el humildoso entiende repremir las soƀͣiuas \textbf{ e tenprar los despreçiamientos e los decaemientos . } Et pues que assi es la humildat sera çerca las soƀͣuias & Intendit enim humilis reprimere superbias , \textbf{ et moderare deiectiones . } Erit ergo humilitas circa superbias , \\\hline
1.2.26 & egualmente nin prinçipalmente \textbf{ por que la humildat prinçipalmente entiende repremir las soƀͣmas . } mas despues desto entiende tenprar los despreçiamientos & non tamen est circa haec aeque principaliter . \textbf{ Nam humilitas principaliter intendit reprimere superbias , } ex consequenti vero moderare deiectiones . Est enim hoc notabiliter attendendum , \\\hline
1.2.26 & por que la humildat prinçipalmente entiende repremir las soƀͣmas . \textbf{ mas despues desto entiende tenprar los despreçiamientos } Onde en esto deuemos notablemente entender & Nam humilitas principaliter intendit reprimere superbias , \textbf{ ex consequenti vero moderare deiectiones . Est enim hoc notabiliter attendendum , } quod cum virtus magis sit retrahens quam impellens , \\\hline
1.2.26 & mas despues desto entiende tenprar los despreçiamientos \textbf{ Onde en esto deuemos notablemente entender } e ser aꝑçebidos & Nam humilitas principaliter intendit reprimere superbias , \textbf{ ex consequenti vero moderare deiectiones . Est enim hoc notabiliter attendendum , } quod cum virtus magis sit retrahens quam impellens , \\\hline
1.2.26 & mas despues desto es contraria al fallescemiento e al menospreçiamiento \textbf{ Ca en quariendo omne obrar obras } que son dignas de grant honrra & ex consequenti vero contrariatur deiectioni . \textbf{ Inquirendo enim opera honore digna , } non solum contingit peccare per superbiam , \\\hline
1.2.26 & que son dignas de grant honrra \textbf{ non solamente pueden pecar } por sobrepuiamiento & Inquirendo enim opera honore digna , \textbf{ non solum contingit peccare per superbiam , } sed etiam per deiectionem . \\\hline
1.2.26 & por sobrepuiamiento \textbf{ mas ahun puede pecar por fallesçimiento . } Et si alguno & non solum contingit peccare per superbiam , \textbf{ sed etiam per deiectionem . | Nam } si quis ultra quam suus status requirat , \\\hline
1.2.26 & e alabadores de ssi \textbf{ que quiere dezir alabadores } que se alaban & iactatores et superbos appellat : \textbf{ quia ultra quam eorum status requireret , } vilius induebantur : \\\hline
1.2.26 & e despues desto cerca la tenprança del despreçiamiento e del abaxamiento . \textbf{ finca de ver } que conuiene alos Reyes & ex consequenti vero circa moderationem deiectionis : \textbf{ restat videre quod decet Reges } et Principes esse humiles , \\\hline
1.2.26 & e alos prinçipes ser humildosos \textbf{ la qual cosa podemos prouar en dos maneras ¶ } La primera se tome de parte dela magnanimidat ¶ & et Principes esse humiles , \textbf{ quod duplici via probare possumus . Prima sumitur ex parte magnanimitatis . } Secunda vero \\\hline
1.2.26 & La segunda de parte delas obras \textbf{ que se deuen fazer } Ca assi commo dicho es de suso ninguno non puede serudaderamente magnanimo & quod duplici via probare possumus . Prima sumitur ex parte magnanimitatis . \textbf{ Secunda vero } ex parte operum fiendorum . Dicebatur enim supra , \\\hline
1.2.26 & Por que conuiene alos Reyes et alos prinçipes \textbf{ assi de madar las obras dignas de honrra } que non sean mas que la razon & decet eos esse humiles . \textbf{ Debent enim Reges sic quaerere opera honore digna , } non ultra quam ratio dictet , \\\hline
1.2.26 & Et otrosi que non pongan la su bien andança en sobrepuiança de honrra lo que fazen los sobuios \textbf{ por que deuen fazer los Reyes bueans obras } e dignas de honrra & quod tamen suam felicitatem non ponant in excellentia et honore , \textbf{ quod faciunt superbi . Debent enim agere bona opera } et honore digna boni gratia , \\\hline
1.2.26 & Lo segundo conuiene alos Reyes de ser humildosos \textbf{ por razon delas obras que han de fazer . } Ca el sobra uio demandado & non ut ostendant se , \textbf{ et ut videantur excellere . Secundo decet eos esse humiles ratione operum fiendorum . } Nam superbus quaerens suam excellentiam ultra quam debeat , \\\hline
1.2.26 & e alos otros a grandes periglos \textbf{ por que non pueden conplir las cosas que comiençan } Et por ende entanto conuiene alos Re yes & se et alios exponunt periculis , \textbf{ non valentes adimplere quod inchoant . } Tanto ergo Reges et Principes debent a se superbiam remouere , \\\hline
1.2.26 & Et por ende entanto conuiene alos Re yes \textbf{ e alos prinçipes de tirar de si la sobrauia en } quanto mas contraria es dessi & non valentes adimplere quod inchoant . \textbf{ Tanto ergo Reges et Principes debent a se superbiam remouere , } quanto peius est communia bona periculis exponere . Superbus enim Dominus , \\\hline
1.2.26 & quanto mas contraria es dessi \textbf{ por non poner a periglo los sus bienes } e las sus gentes . & Tanto ergo Reges et Principes debent a se superbiam remouere , \textbf{ quanto peius est communia bona periculis exponere . Superbus enim Dominus , } ut plurimum periclitator efficitur populorum . \\\hline
1.2.27 & e alos prinçipesser honrrados destas uirtudes . \textbf{ fincanos agora de dezir dela mansedunbre } que es uirtud & ornari virtutibus illis , \textbf{ restat dicere de mansuetudine , } quae respicit exteriora mala . \\\hline
1.2.27 & qualquier cosa \textbf{ en que podemos pecar } e bien obrar & quod , \textbf{ quia circa quaecunque contingit peccare , } et bene agere , \\\hline
1.2.27 & en que podemos pecar \textbf{ e bien obrar } o çerca qual quier cosa & quia circa quaecunque contingit peccare , \textbf{ et bene agere , } vel contingit abundare , et deficere , \\\hline
1.2.27 & o çerca qual quier cosa \textbf{ en que podemos sobre puiar } e tal sesçer conuiene de dar & et bene agere , \textbf{ vel contingit abundare , et deficere , } oportet ibi dare virtutem aliquam , \\\hline
1.2.27 & en que podemos sobre puiar \textbf{ e tal sesçer conuiene de dar } y . alguna uirtud & vel contingit abundare , et deficere , \textbf{ oportet ibi dare virtutem aliquam , } per quam dirigamur ad bene agendum , \\\hline
1.2.27 & y . alguna uirtud \textbf{ por la qual seamos enderesçados abien obrar } que reprema las menguas & oportet ibi dare virtutem aliquam , \textbf{ per quam dirigamur ad bene agendum , } reprimentem defectus , \\\hline
1.2.27 & e asconden la miuria delas quales dos cosas ninguna non es buena \textbf{ por que enssannar se } de qual se quier cosa & Irasci enim cuilibet , \textbf{ et semper , } et de quolibet vindictam exposcere , \\\hline
1.2.27 & de qual se quier cosa \textbf{ e sobre qual cosa dessear uengança } la qual cosa fazen los sannudos & et semper , \textbf{ et de quolibet vindictam exposcere , } quod faciunt iracundi , \\\hline
1.2.27 & la qual cosa fazen los sannudos \textbf{ esto es de denostar . } Otrosi nunca se enssannar el omne & quod faciunt iracundi , \textbf{ vituperabile est . Rursum , } nunquam irasci , \\\hline
1.2.27 & esto es de denostar . \textbf{ Otrosi nunca se enssannar el omne } e en ninguna manera non querer penna a otre & vituperabile est . Rursum , \textbf{ nunquam irasci , } et in nullo modo velle aliquem puniri , \\\hline
1.2.27 & Otrosi nunca se enssannar el omne \textbf{ e en ninguna manera non querer penna a otre } nin vengan & nunquam irasci , \textbf{ et in nullo modo velle aliquem puniri , } non est laudabile , \\\hline
1.2.27 & nin vengan \textbf{ ca assi esto non es de loar } por que esto es obrar fuera de orden de razon e de entendimiento & et in nullo modo velle aliquem puniri , \textbf{ non est laudabile , } quia est agere praeter ordinem rationis . \\\hline
1.2.27 & ca assi esto non es de loar \textbf{ por que esto es obrar fuera de orden de razon e de entendimiento } por que la razon demanda & non est laudabile , \textbf{ quia est agere praeter ordinem rationis . } Ratio enim dictat punitiones aliquas esse faciendas , \\\hline
1.2.27 & e que cada vno se enssanne \textbf{ quando deue e commo deue . Por que enssannarse el omne en el logar } que deue e en el tp̃o & et quomodo . \textbf{ Irasci enim loco et tempore , } est virtutis opus . \\\hline
1.2.27 & e en desseando penas al su contrario \textbf{ e uenganças del contesçe de sobrepiuar e de fallesçer . } Conuiene de dar & et in appetendo punitiones \textbf{ et vindictas , | contingit superabundare et deficere : } oportet ibi dare virtutem aliquam reprimentem superabundantias , \\\hline
1.2.27 & e uenganças del contesçe de sobrepiuar e de fallesçer . \textbf{ Conuiene de dar } y alguna uirtud & ø \\\hline
1.2.27 & por la qual desseamos ser vengados . \textbf{ Et entre tanto nunca auer sanna } por la qual del todo perdonamos los males & ø \\\hline
1.2.27 & e tienpralas non sannas \textbf{ que es nunca se enssannar ¶ } Visto que cosa es la mansedunbre ligeramente paresçe & Viso \textbf{ quid est mansuetudo , } de leui patet circa quae habet esse . Est enim \\\hline
1.2.27 & e erca el contrario dela saña \textbf{ que es non se ensannar en ninguna manera . } Enpero mas prinçipalmente es çerca las sanas & ø \\\hline
1.2.27 & por que la mansedunbre primero \textbf{ e prinçipalmente entiende repremir las sañas } mas despues desto entiende tenprar las passiones contrarias dela sana & ut circa irascibilitatem , principalius tamen est circa iras . Mansuetudo enim principaliter et primo intendit reprimere iras , \textbf{ ex consequenti autem intendit moderare passiones oppositas irae . } Nam naturale est nobis \\\hline
1.2.27 & e prinçipalmente entiende repremir las sañas \textbf{ mas despues desto entiende tenprar las passiones contrarias dela sana } que es nunca se enssanar & ut circa irascibilitatem , principalius tamen est circa iras . Mansuetudo enim principaliter et primo intendit reprimere iras , \textbf{ ex consequenti autem intendit moderare passiones oppositas irae . } Nam naturale est nobis \\\hline
1.2.27 & mas despues desto entiende tenprar las passiones contrarias dela sana \textbf{ que es nunca se enssanar } por lo que ha razon de se ensannar . Ca natural cosa es anos & ex consequenti autem intendit moderare passiones oppositas irae . \textbf{ Nam naturale est nobis } ut ex malis nobis illatis appetamus punitionem , \\\hline
1.2.27 & que es nunca se enssanar \textbf{ por lo que ha razon de se ensannar . Ca natural cosa es anos } que por los males e por las jniurias & ex consequenti autem intendit moderare passiones oppositas irae . \textbf{ Nam naturale est nobis } ut ex malis nobis illatis appetamus punitionem , \\\hline
1.2.27 & Et non solamente naturalmente somos inclinados \textbf{ para querer ser vengados } e dar pena a aquellos que nos fazen alguons males . & non solum naturaliter inclinamur , \textbf{ ut velimus puniri inferentes nobis aliqua mala , } sed etiam quodammodo naturale est nobis appetere punitionem ultra condignum . \\\hline
1.2.27 & para querer ser vengados \textbf{ e dar pena a aquellos que nos fazen alguons males . } Mas avn en alguna manera natural cosa esa nos de dessear & non solum naturaliter inclinamur , \textbf{ ut velimus puniri inferentes nobis aliqua mala , } sed etiam quodammodo naturale est nobis appetere punitionem ultra condignum . \\\hline
1.2.27 & e dar pena a aquellos que nos fazen alguons males . \textbf{ Mas avn en alguna manera natural cosa esa nos de dessear } de ser vengados dellos & ut velimus puniri inferentes nobis aliqua mala , \textbf{ sed etiam quodammodo naturale est nobis appetere punitionem ultra condignum . } Nam quia malum nobis illatum videtur nobis maius esse , \\\hline
1.2.27 & por el mal que nos fazen . \textbf{ Et por que muy guaue cosa es de repremir las sannas } e de non dessear uengança delas iniurias & iniuriatores nostros plus puniri volumus , \textbf{ quam puniendi sint . Difficile est ergo valde reprimere iras , } et non appetere punitiones iniuriarum ultra quam dictet ratio . Plures ergo peccant in appetendo plus : \\\hline
1.2.27 & Et por que muy guaue cosa es de repremir las sannas \textbf{ e de non dessear uengança delas iniurias } mas que la razon e el entendimiento muestra & quam puniendi sint . Difficile est ergo valde reprimere iras , \textbf{ et non appetere punitiones iniuriarum ultra quam dictet ratio . Plures ergo peccant in appetendo plus : } pauci vero delinquunt in appetendo minus . \\\hline
1.2.27 & La uirtud que es mansedunbre prinçipalmente es contraria ala sana \textbf{ e entiende de repremir las sanas . } Mas despues desto es contraria a nunca se enssannar & et difficile , mansuetudo principaliter opponitur irae , \textbf{ et intendit iras reprimere : } ex consequenti autem opponitur irascibilitati , \\\hline
1.2.27 & e entiende de repremir las sanas . \textbf{ Mas despues desto es contraria a nunca se enssannar } por lo que deue . & et intendit iras reprimere : \textbf{ ex consequenti autem opponitur irascibilitati , } et intendit eam moderare , \\\hline
1.2.27 & por lo que deue . \textbf{ Et entiende tenprar esto } assi commo el nonbre dela manssedunbre muestra . & ex consequenti autem opponitur irascibilitati , \textbf{ et intendit eam moderare , } quod ipsum nomen designat : \\\hline
1.2.27 & La manssedunbre nonbra tenpramiento de sana . \textbf{ mas mostrar } que conuiene alos Reyes e alos prinçipes & si enim vim nominis consideremus , \textbf{ mansuetudo nominat temperamentum irae . } Quod autem deceat Reges et Principes esse mansuetos , \\\hline
1.2.27 & esto non es cosa guaue mas ligera . \textbf{ Ca por que la yr a tristorna el iuyzio dela razon } e del entendimiento & ostendere non est difficile . \textbf{ Nam cum ira peruertat iudicium rationis , } non decet Reges et Principes esse iracundos , \\\hline
1.2.27 & Et avn en essa misma manera non es cosa conuenible al Rey \textbf{ de nunca se enssanar } e de nunca se mouer a dar pena . & et obliquatur . Sic etiam , \textbf{ si nullo modo esset irascibilis , } et nullo modo commoueretur ad \\\hline
1.2.27 & de nunca se enssanar \textbf{ e de nunca se mouer a dar pena . } por que si las penas non se diessen & si nullo modo esset irascibilis , \textbf{ et nullo modo commoueretur ad } punitionem faciendam , indecens esset : quia si punitiones non fierent in regno , \\\hline
1.2.27 & Et la paliçia \textbf{ que es ordenança dela çibdat non podria durar ¶ } Pues que assi es ninguno non deue dessear uengança e pena por rancor & homines fierent iniuriatores aliorum , \textbf{ et politia durare non posset . } Nullus igitur debet irasci per odium , \\\hline
1.2.27 & que es ordenança dela çibdat non podria durar ¶ \textbf{ Pues que assi es ninguno non deue dessear uengança e pena por rancor } mas por amor e por zelo de bien se deue enssannar . & et politia durare non posset . \textbf{ Nullus igitur debet irasci per odium , | nec debet punitiones appetere propter odium , } sed propter amorem \\\hline
1.2.27 & Pues que assi es ninguno non deue dessear uengança e pena por rancor \textbf{ mas por amor e por zelo de bien se deue enssannar . } Et son de dessear las uenganças e las penas & nec debet punitiones appetere propter odium , \textbf{ sed propter amorem | et zelum est irascendum , } et sunt punitiones appetendae . Appetenda est enim punitio propter amorem et zelum iustitiae , \\\hline
1.2.27 & mas por amor e por zelo de bien se deue enssannar . \textbf{ Et son de dessear las uenganças e las penas } por amor et por zelo de iustiçia & et zelum est irascendum , \textbf{ et sunt punitiones appetendae . Appetenda est enim punitio propter amorem et zelum iustitiae , } vel propter amorem Reipublicae \\\hline
1.2.27 & o por amor de la comunidat \textbf{ por que sin ella la comunidat non podrie durar . } por la qual cosa si el Rey o el prinçipe o otro & vel propter amorem Reipublicae \textbf{ quia sine ea Respublica durare non posset . } Quare si quis in tantum esset mitis , \\\hline
1.2.27 & que mas quasi esse que pesçiesse la iustiçia \textbf{ que demandar la vengança non seria uirtuoso . } Et pues que assi es tanto & quod potius vellet iustitiam perire , \textbf{ quam iustitiam exposcere , virtuosus non esset . } Tanto ergo magis decet Reges et Principes moueri ad punitionem faciendam , quanto magis spectat ad ipsos esse custodes iustitiae , \\\hline
1.2.27 & Et pues que assi es tanto \textbf{ mas conuiene alos Reyes e alos prinçipes de se mouer a dar penas . } e fazer uengancas & quam iustitiam exposcere , virtuosus non esset . \textbf{ Tanto ergo magis decet Reges et Principes moueri ad punitionem faciendam , quanto magis spectat ad ipsos esse custodes iustitiae , } et conseruatores Reipublicae . \\\hline
1.2.27 & mas conuiene alos Reyes e alos prinçipes de se mouer a dar penas . \textbf{ e fazer uengancas } quanto mas pertenesçe a ellos & quam iustitiam exposcere , virtuosus non esset . \textbf{ Tanto ergo magis decet Reges et Principes moueri ad punitionem faciendam , quanto magis spectat ad ipsos esse custodes iustitiae , } et conseruatores Reipublicae . \\\hline
1.2.27 & si alos Reyes non conuiene de seer sannudos \textbf{ e deuen se mouer } segunt orden de razon e de entendimiento & Quare si Reges non debent esse iracundi , \textbf{ et debent moueri } secundum ordinem rationis , \\\hline
1.2.27 & segunt orden de razon e de entendimiento \textbf{ para fazer uenganças e dar penas . } Et commo esto faga la manssedunbre & secundum ordinem rationis , \textbf{ ut fiant punitiones et vindictae , } cum hoc faciat mansuetudo , \\\hline
1.2.28 & e delas que catan alos małs̃ de fuera . \textbf{ fincanos de dezir de las uirtudes } que catan alos bienes de fuera & et de respicientibus exteriora mala . \textbf{ Restat dicere } de virtutibus respicientibus bona exteriora , \\\hline
1.2.28 & en las quales partiçipamos con los otros siruen a nos en tres cosas . \textbf{ Conuiene saber . a amistanca ¶ } Et a uerdat ¶ & ø \\\hline
1.2.28 & assi commo deuemos somos amigables e afabiles \textbf{ que quiere dezir amigos bien fablantes . } Pues que assi es non es otra cosa amistanȩ & et recipiendo ipsos \textbf{ ut debemus , sumus amicabiles , } et affabiles . Nihil est ergo aliud amicabilitas , \\\hline
1.2.28 & Pues que assi es non es otra cosa amistanȩ \textbf{ o bien fablança dela qual entendemos aqui determinar } si non derechamente beuir con todos & et affabiles . Nihil est ergo aliud amicabilitas , \textbf{ siue affabilitas , | de qua hic determinare intendimus , } nisi recte conuersari cum hominibus , \\\hline
1.2.28 & si non derechamente beuir con todos \textbf{ e ordenar las nr̃as palauras } e las nuestras obras a buena conuerssaçion e conuenible . & nisi recte conuersari cum hominibus , \textbf{ et ordinare opera , } et verba nostra ad debitam conuersationem . Secundo , verba , \\\hline
1.2.28 & por que por ellas somos iudgados quales somos . \textbf{ Mas la uerdat dela qual aqui entendemos determinar } non es otra cosa si non que el omne non sea alabador dessi mismo & quia per ea iudicamur quales sumus . \textbf{ Veritas enim , | de qua hic determinare intendimus , } non est aliud nisi quod homo non sit iactator , \\\hline
1.2.28 & en las quales partiçipamos con los otros han de ser tres uirtudes \textbf{ conuiene saber ¶amistança } que en otro nonbre puede ser dicha afabilidat & habet esse triplex virtus , \textbf{ videlicet , amicabilitas , } quae alio nomine affabilitas dici potest : \\\hline
1.2.28 & que es uirtud \textbf{ para bien fablar . } Otrosi es la uerdat & quae alio nomine affabilitas dici potest : \textbf{ veritas , } quae apertio nuncupatur : et debita iocunditas , \\\hline
1.2.28 & Et otrosi alegera conuenible la qual el philosofo llama heutropeliam \textbf{ que quiere dezir buena conpanina . } Et pues que assi es & quae apertio nuncupatur : et debita iocunditas , \textbf{ quam eutrapeliam vocat . Communicando igitur cum aliis , } si bene conuersari volumus , \\\hline
1.2.28 & Et pues que assi es \textbf{ si quisieremos bien couerssar partiçipando } con los otros & quam eutrapeliam vocat . Communicando igitur cum aliis , \textbf{ si bene conuersari volumus , } debemus esse debite iocundi , veraces , \\\hline
1.2.28 & deuemos seer alegres conueniblemente \textbf{ e uerdaderos e amigables delas quales cosas todas auemos aqui de dozir } mas primero diremos dela amistan & debemus esse debite iocundi , veraces , \textbf{ et amicabiles , de quibus omnibus est dicendum . } Sed primo de amicabilitate . \\\hline
1.2.28 & Ca estos en tanto se muestran conpanneros \textbf{ que non quieren fazer pesar } nin tristeza a ninguno & Hi enim adeo se ostendunt communicabiles et sociales , \textbf{ ut nullum contristari velint ; } sed omnia dicta et facta aliorum laudant . Aliqui vero econtrario , \\\hline
1.2.28 & por que ninguno non le deue \textbf{ en tanto mostrar conpanero alos otros } por que sea visto plazentero e falaguero . & et agrestes , non valentes cum aliis conuersari . Uterque autem a recta ratione deficiunt , \textbf{ quia nec quis se debet tantum aliis ostendere socialem , } ut videatur placidus , \\\hline
1.2.28 & por que sea visto plazentero e falaguero . \textbf{ Otrossi non se deue tirar } en tanto dela conpana & ut videatur placidus , \textbf{ et blanditor : nec se debet tantum a societate subtrahere , } ut uideatur discolus , \\\hline
1.2.28 & mengunate en la conuerssa conn de los omes \textbf{ cerca la qual contesçe de sobrepuiar e de fallesçer . } Conuiene de dar uirtud alguna & in conuersatione hominum , \textbf{ circa quam contingit abundare et deficere , } oportet dare uirtutem aliquam reprimentem superabundantias , \\\hline
1.2.28 & cerca la qual contesçe de sobrepuiar e de fallesçer . \textbf{ Conuiene de dar uirtud alguna } que reprima las sobrepuianças & circa quam contingit abundare et deficere , \textbf{ oportet dare uirtutem aliquam reprimentem superabundantias , } et moderantem defectus . \\\hline
1.2.28 & en quanto es uirtud \textbf{ si non auer se el omne medianeramente } e bien en la conuerssa conn de los omes & ut est uirtus , \textbf{ quam medio modo se habere in conuersatione hominum : } ut non superabundemus in huiusmodi conuersatione quod faciunt adulatores : \\\hline
1.2.28 & assi commo el philosofo dize en el segundo e en el quarto delas ethicas \textbf{ muy de ligero puede paresçer } çerca quales cosas ha de ser la amistança . & et 4 Ethicorum , \textbf{ de leui apparere potest circa quae habet esse : } quia est circa opera , et uerba , \\\hline
1.2.28 & por que se aconpanna a otro \textbf{ assi commo se puede prouar en el primero libro delas politicas . } Conuiene cerca las palauras & ø \\\hline
1.2.28 & en las quales el omne partiçipa con los otros \textbf{ de dar alguna uirtud } por la qual conueniblemente sepa conuerssar & in quibus communicat cum aliis , \textbf{ dare uirtutem aliquam , } per quam debite conseruetur . \\\hline
1.2.28 & de dar alguna uirtud \textbf{ por la qual conueniblemente sepa conuerssar } e beuir con los otros . Por la qual cosa commo la razon derecha & dare uirtutem aliquam , \textbf{ per quam debite conseruetur . } Quare cum recta ratio dictet , \\\hline
1.2.28 & por la qual conueniblemente sepa conuerssar \textbf{ e beuir con los otros . Por la qual cosa commo la razon derecha } e el entendemiento muestre & per quam debite conseruetur . \textbf{ Quare cum recta ratio dictet , } quod \\\hline
1.2.28 & e el entendemiento muestre \textbf{ que segt el departimiento delons omes ha omne de conuerssar departidamente con ellos } commo quier que todos los omes & Quare cum recta ratio dictet , \textbf{ quod } secundum diuersitatem personarum diuersimode sit conuersandum : licet omnes homines uolentes viuere politice debeant esse amicabiles \\\hline
1.2.28 & por amistad e atabiles \textbf{ por bien fablar } Empero non deuen todos en vna manera seramigables e bien fablantes . & secundum diuersitatem personarum diuersimode sit conuersandum : licet omnes homines uolentes viuere politice debeant esse amicabiles \textbf{ et affabiles , } non tamen omnes eodem modo amicabiles debent esse . \\\hline
1.2.28 & e la dignidat Real non sea abiltada nin menospreçiada \textbf{ mas cuerdamente se deuen auer que los otro . } Onde el philosofo enel quarto libro delas politicas & et ne dignitas regia vilescat , \textbf{ maturius se habere debent , | quam alii . } Unde Philosophus 5 Politicorum dando cautelas Regum et Principum , ait , \\\hline
1.2.28 & que conuiene alos Reyes \textbf{ e alos prinçipes de paresçer perssonas reuerendas } a quien deuen fazer reuerençia & Unde Philosophus 5 Politicorum dando cautelas Regum et Principum , ait , \textbf{ quod decet Reges et Principes apparere personas reuerendas , } ne contemptibiles habeantur . Sicut enim in sumptione et in aliis recta ratio dictat , \\\hline
1.2.28 & e alos prinçipes de paresçer perssonas reuerendas \textbf{ a quien deuen fazer reuerençia } por qua non sean auidos en despreçiamiento . & ø \\\hline
1.2.29 & ssi commo çerca la conuerssaçion de los omes \textbf{ en la uida contesçede sobrepuiar } e de fallesçer & sed syluestris . \textbf{ Sicut circa conuersationem in vita contingit superabundare , et deficere : } sic circa veritatem contingit superabundare , \\\hline
1.2.29 & en la uida contesçede sobrepuiar \textbf{ e de fallesçer } assi como es mostrado dessuso & sed syluestris . \textbf{ Sicut circa conuersationem in vita contingit superabundare , et deficere : } sic circa veritatem contingit superabundare , \\\hline
1.2.29 & assi como es mostrado dessuso \textbf{ assi en essa misma manera cerca la uerdat contesçe de sobrepuiar } e defallesçer & Sicut circa conuersationem in vita contingit superabundare , et deficere : \textbf{ sic circa veritatem contingit superabundare , } et deficere . \\\hline
1.2.29 & assi en essa misma manera cerca la uerdat contesçe de sobrepuiar \textbf{ e defallesçer } ca la uerdat esta en vn egualamiento . & sic circa veritatem contingit superabundare , \textbf{ et deficere . } Veritas enim in quadam adaequatione consistit . \\\hline
1.2.29 & Por la qual cosa commo la mentira \textbf{ por si misma sea mala deuemos foyr della } assi commo dize aristotiles & a veritate recedit ratione defectus . \textbf{ Quare cum mendacium sit semper fugiendum , } ut dicitur 4 Ethicorum , \\\hline
1.2.29 & nin se muestranabiercamente \textbf{ tales quales son estos son mucho de reprehender } por que vn mentires non ser el omne manifiesto & qui non sunt veraces nec aperti , \textbf{ nec ostendunt se tales , quales sunt , reprehensibiles existunt . } Quid enim aliud est mentiri , \\\hline
1.2.29 & por que vn mentires non ser el omne manifiesto \textbf{ nin se mostrar tal qual es . } Et pues que assi es desta uirtud & Quid enim aliud est mentiri , \textbf{ nisi non esse apertum , et non ostendere se talem , qualis est . } Ab hac ergo veritate , \\\hline
1.2.29 & Et estos llama el philosofo yrones \textbf{ que quiere dezir escarnidores } e despreçiadores dessi mismos . & quos Philosophus vocat irones , \textbf{ idest irrisores , et despectores . } Oportet ergo dare aliquam virtutem mediam , per quam moderentur diminuta , \\\hline
1.2.29 & e despreçiadores dessi mismos . \textbf{ Et pues que assi es conuiene de dar alguna uirtud medianera } por la qual sean tenpradas las cosas menguadas & idest irrisores , et despectores . \textbf{ Oportet ergo dare aliquam virtutem mediam , per quam moderentur diminuta , } et reprimentur superflua : \\\hline
1.2.29 & e repreme los alabamientos ¶visto \textbf{ que cosa es la uerdat finca de veer } traca quales cosas ha de seer . & et reprimens iactantias . \textbf{ Viso quid est veritas , } restat videre circa quae habet esse . \\\hline
1.2.29 & traca quales cosas ha de seer . \textbf{ Et pues que assi es conuiene saber } que maguera firmar cada vno de ser & restat videre circa quae habet esse . \textbf{ Sciendum ergo quod licet affirmare in se esse quod non est , } vel negare quod est , \\\hline
1.2.29 & Et pues que assi es conuiene saber \textbf{ que maguera firmar cada vno de ser } en ssi aquello que non es o negar aquello que es & restat videre circa quae habet esse . \textbf{ Sciendum ergo quod licet affirmare in se esse quod non est , } vel negare quod est , \\\hline
1.2.29 & que maguera firmar cada vno de ser \textbf{ en ssi aquello que non es o negar aquello que es } en ssi sea mentira & Sciendum ergo quod licet affirmare in se esse quod non est , \textbf{ vel negare quod est , } sit mentiri : \\\hline
1.2.29 & en ssi sea mentira \textbf{ Empero non dezir todo aquello } que es en el puede ser sin mentira . & sit mentiri : \textbf{ tamen non dicere totum quod est , } absque mendacio fieri potest . \\\hline
1.2.29 & Pues que assi es el que quiere ser uerdadero \textbf{ non deue dezir } nin segnit de ssi bondat & absque mendacio fieri potest . \textbf{ Volens igitur esse verax , } non debet de se fingere habere bonitatem quam non habet , nec debet concedere in se esse malitiam , \\\hline
1.2.29 & Ca commo quier \textbf{ que nin guon non deua mentir } enpero non es de dezir cada vna uerdat sienpre & quam sibi inesse cognoscit : \textbf{ quia licet nullus mentiri debeat , } non tamen semper \\\hline
1.2.29 & que nin guon non deua mentir \textbf{ enpero non es de dezir cada vna uerdat sienpre } e en todo logar & quia licet nullus mentiri debeat , \textbf{ non tamen semper } et ubique \\\hline
1.2.29 & segunt que la razon e el entendemiento muestra \textbf{ podemos callar la uerdat . } Et pues que assi es la uerdat es uirtud & secundum dictamen \textbf{ rationis possumus veritatem tacere . } Est igitur veritas circa repressionem iactantiarum , \\\hline
1.2.29 & Et pues que assi es la uerdat es uirtud \textbf{ para repremir los alabamientos } e para tonprar los despreçiamientos . & rationis possumus veritatem tacere . \textbf{ Est igitur veritas circa repressionem iactantiarum , } et circa moderationem despectionum : \\\hline
1.2.29 & para repremir los alabamientos \textbf{ e para tonprar los despreçiamientos . } Enpero non es cerca estas cosas & Est igitur veritas circa repressionem iactantiarum , \textbf{ et circa moderationem despectionum : } non tamen est circa haec aeque principaliter . \\\hline
1.2.29 & mas primero e prinçipalmente la uerdat es contraria al alabamiento \textbf{ e entiende repremir los alabamientos . Mas despues desto es contraria al despreçiamiento } por que entiende tenprar los despreçiamientos . & Sed principalius veritas opponitur iactantiae , \textbf{ et intendit iactantias reprimere . | Ex consequenti autem opponitur despectioni , } et intendit despectiones moderare . Ideo Philosophus 4 Ethicorum cap’ \\\hline
1.2.29 & e entiende repremir los alabamientos . Mas despues desto es contraria al despreçiamiento \textbf{ por que entiende tenprar los despreçiamientos . } Et por ende el philosofo en el quarto libro delas ethicas & Ex consequenti autem opponitur despectioni , \textbf{ et intendit despectiones moderare . Ideo Philosophus 4 Ethicorum cap’ } de veritate ait , \\\hline
1.2.29 & en el capitulo dela uerdat dize \textbf{ que declinar alo menos } e dezir dessi menores cosas que sean es obra de sabio . & de veritate ait , \textbf{ quod declinare ad minus , } et dicere de se minora quam sint , est opus prudentis . Spectat igitur ad veracem nullo modo dicere de se maiora , \\\hline
1.2.29 & que declinar alo menos \textbf{ e dezir dessi menores cosas que sean es obra de sabio . } Pues que assi es parte nesce al uerdadero & quod declinare ad minus , \textbf{ et dicere de se minora quam sint , est opus prudentis . Spectat igitur ad veracem nullo modo dicere de se maiora , } quam sint , \\\hline
1.2.29 & Pues que assi es parte nesce al uerdadero \textbf{ non dezir } dessi mayores cosas & ø \\\hline
1.2.29 & e non manifiestos nin uerdaderos . \textbf{ Et por ende en ninguna manera en tales cosas non deuemos sobrepuiar } en lo mas & et veraces , \textbf{ nullo modo in talibus est excedendum in plus , } sed magis declinandum in minus : \\\hline
1.2.29 & en lo mas \textbf{ mas deuemos nos inclinar alo menos } si se pudiere fazer sin mentira . & nullo modo in talibus est excedendum in plus , \textbf{ sed magis declinandum in minus : } dum tamen sine mendacio fiat , et non notabiliter recedat a medio : \\\hline
1.2.29 & mas deuemos nos inclinar alo menos \textbf{ si se pudiere fazer sin mentira . } Et non deue el omne partir se notablemente del medio e de la egualdat & sed magis declinandum in minus : \textbf{ dum tamen sine mendacio fiat , et non notabiliter recedat a medio : } quia si notabiliter a medio recederet , non in hoc appareret verax , sed derisor . \\\hline
1.2.29 & si se pudiere fazer sin mentira . \textbf{ Et non deue el omne partir se notablemente del medio e de la egualdat } por que si notablemente se alongasse dela medianera & sed magis declinandum in minus : \textbf{ dum tamen sine mendacio fiat , et non notabiliter recedat a medio : } quia si notabiliter a medio recederet , non in hoc appareret verax , sed derisor . \\\hline
1.2.29 & que los otros fuesen çiertos \textbf{ que pudiessen lidiar contra çiento } si aquel non otorgasse & Ut si aliquis adeo esset fortis et strenuus , \textbf{ quod constaret aliis quod contra centum bellare posset : } si ille non plus de se concederet , \\\hline
1.2.29 & si aquel non otorgasse \textbf{ mas dessi si non que podria lidiar contra vn flaco todos los otros conoscrian en ssi mismos } que este fablaria dessi por escarnio . & si ille non plus de se concederet , \textbf{ nisi quod contra unicum et debilem bellaret , | omnes intra se cognoscerent } quod derisorie loqueretur . \\\hline
1.2.29 & Et por ende este menospreciador \textbf{ de ssi es denetar de yerro e de escarnesçimiento } quando alguno notablemente otorga de ssi mismo algunas cosas & quod derisorie loqueretur . \textbf{ Inde igitur sumptum est nomen ironiae et derisionis , } quando aliquis notabiliter de se minora concedit quam sint . \\\hline
1.2.29 & quando alguno notablemente otorga de ssi mismo algunas cosas \textbf{ que son en el Et pues que assi es ala uerdat parte nesçe de tenprar estos tales escarnesçimientos } e de repremir los alabamientos . & quando aliquis notabiliter de se minora concedit quam sint . \textbf{ Ad veritatem ergo spectat moderare huiusmodi derisiones , } et reprimere iactantias . Principalius tamen spectat ad ipsum iactantias reprimere , quam derisiones moderare : \\\hline
1.2.29 & que son en el Et pues que assi es ala uerdat parte nesçe de tenprar estos tales escarnesçimientos \textbf{ e de repremir los alabamientos . } Enpero mas prinçipalmente parte nesçe ala uerdat de repremir los alabamientos que tenprar los escarnesçimientos & Ad veritatem ergo spectat moderare huiusmodi derisiones , \textbf{ et reprimere iactantias . Principalius tamen spectat ad ipsum iactantias reprimere , quam derisiones moderare : } quia \\\hline
1.2.29 & e de repremir los alabamientos . \textbf{ Enpero mas prinçipalmente parte nesçe ala uerdat de repremir los alabamientos que tenprar los escarnesçimientos } por que assi commo dicho es & Ad veritatem ergo spectat moderare huiusmodi derisiones , \textbf{ et reprimere iactantias . Principalius tamen spectat ad ipsum iactantias reprimere , quam derisiones moderare : } quia \\\hline
1.2.29 & por que assi commo dicho es \textbf{ mas deuemos de elinar en tales cosas alo menos . . } deziendo cada vno dessi menores cosas que sean en el & ( \textbf{ ut dictum est ) in talibus magis est declinandum in minus dicendo de se minora } quam sint , \\\hline
1.2.29 & por dos razones \textbf{ que en tales cosas sienp̊ͤ deuemos declinar alo menos ¶ } Lo primera se toma de parte de ssi mismo¶ & Possumus autem duplicem causam assignare \textbf{ quare in talibus semper est declinandum in minus . } Prima sumitur ex parte sui . \\\hline
1.2.29 & La segunda de parte de los otros ¶ \textbf{ La primera por que cada vno naturalmente es Inclinado a querer su bien propio } en tal manera que sienpre creade ssi mismo mas de aquello que es . & Secunda ex parte aliorum . \textbf{ Quilibet enim ita naturaliter afficitur | ad proprium bonum , } ut ipsum semper plus credat esse quam sit . \\\hline
1.2.29 & cuydando que valen mas de quantovalen . \textbf{ Et por ende en contando cada vno los sus propreos bienes deue se sienpre inclinar alo menos . } Ca deuemos cuydar & plus credentes se plus valere , quam valeant . In narrando ergo propria bona , \textbf{ semper declinandum est in minus : } quia aestimare debemus \\\hline
1.2.29 & Et por ende en contando cada vno los sus propreos bienes deue se sienpre inclinar alo menos . \textbf{ Ca deuemos cuydar } que nos & semper declinandum est in minus : \textbf{ quia aestimare debemus } quod affecti ad propria bona , videntur nobis illa esse maiora , \\\hline
1.2.29 & quando dize \textbf{ que pertenesçe al sabio de declinar alo menos . } Ca muy grand pradençia & cum ait , \textbf{ quod prudentis est declinare in minus . } Nam magnae prudentiae est , \\\hline
1.2.29 & Ca muy grand pradençia \textbf{ e grant sabiduria es conosçer } assi mismo . omne e saber & Nam magnae prudentiae est , \textbf{ cognoscere seipsum , } et sciri quod propria bona semper aestimantur maiora quam sint . \\\hline
1.2.29 & e grant sabiduria es conosçer \textbf{ assi mismo . omne e saber } que los sus bienes propreos sienpreles son vistos mayores que son ¶ & cognoscere seipsum , \textbf{ et sciri quod propria bona semper aestimantur maiora quam sint . } Secunda ratio sumitur ex parte aliorum . \\\hline
1.2.29 & Por ende avn de parte de los otros \textbf{ conuiene en tales cosas declinar a lo menos } por que los omes non sean alos otros pesados e guaues . & ideo \textbf{ etiam ex parte aliorum decet in talibus declinare in minus , } ne homines sint aliis onerosi . Hanc autem rationem tangit Philosophus in eodem 4 Ethicorum dicens , \\\hline
1.2.29 & Mas por tanto conuiene alos Reyes \textbf{ e alos prinçipes de escusar } e de foyr el alabança & vel promittendo aliis maiora quam faciant . \textbf{ Immo tanto magis decet Reges et Principes cauere iactantiam , } quanto plures habent incitantes ipsos ad iactantiam , \\\hline
1.2.29 & e alos prinçipes de escusar \textbf{ e de foyr el alabança } en quantos han muchos mas & ø \\\hline
1.2.29 & que los endugan a alabança . \textbf{ e en quanto mas de muchas cosas se podrian alabar } que otros . & quanto plures habent incitantes ipsos ad iactantiam , \textbf{ et de pluribus se iactare possent . } Credunt enim aliqui , omnia verba vel facta deseruientia iocis esse ociosa , \\\hline
1.2.30 & assi commo los sesos corporales . \textbf{ assi commo el veer } e el oyr t̃baian en sintiendo las cosas senssibles . & Ideo dicitur quarto Ethic’ quod videtur requies et ludus esse aliquid necessarium in vita . Sicut ergo sensus corporales , \textbf{ ut visus , et auditus , } quia laborant in sentiendo , natura ordinauit somnum propter eorum requiem , \\\hline
1.2.30 & assi commo el veer \textbf{ e el oyr t̃baian en sintiendo las cosas senssibles . } Et por ende la natura ordeno el su enno & ut visus , et auditus , \textbf{ quia laborant in sentiendo , natura ordinauit somnum propter eorum requiem , } et est necessarius somnus in vita . \\\hline
1.2.30 & e non baldias nin en vano . \textbf{ Por la qual cosa sientales cosas contesçe de pecar } e de bien fazer & et laboribus nostris . \textbf{ Quare cum in talibus contingat peccare , } et bene facere , \\\hline
1.2.30 & Por la qual cosa sientales cosas contesçe de pecar \textbf{ e de bien fazer } conuiene erca tales iuegos & Quare cum in talibus contingat peccare , \textbf{ et bene facere , } oportet circa ipsos iocos dare virtutem aliquam , \\\hline
1.2.30 & conuiene erca tales iuegos \textbf{ e cerca tales delecta connes deuiegos dar alguna uirtud . } por la qual conueniblemente nos ayamos alos iuegos e alos trabaios . & et bene facere , \textbf{ oportet circa ipsos iocos dare virtutem aliquam , } per quam debite nos habeamus ad ludos . \\\hline
1.2.30 & por la qual conueniblemente nos ayamos alos iuegos e alos trabaios . \textbf{ Et por que en estos iuegos algunos sobre punan desseando auer riso dellos } de los quales dize el philosofo en el quarto libro delas ethicas & per quam debite nos habeamus ad ludos . \textbf{ In ipsis enim iocis quidam superabundant desiderantes omnino risum facere , } de quibus 4 Ethicorum dicitur , \\\hline
1.2.30 & que algunos mas se esfuerçan de fazerriso \textbf{ que de dezir cosas fermosas . } Et estos tales son los iuiglares & de quibus 4 Ethicorum dicitur , \textbf{ quod magis conantur risum facere , quam decora dicere . } Huiusmodi autem sunt histriones , \\\hline
1.2.30 & Ca assi commo aquellas aues non cura una \textbf{ en qual manera podian tomar de aquella prea alguna cosa } en essa misma manera & praedam sacrificatam in templis Gentilium . \textbf{ Sicut enim aues illae non curabant qualitercunque possent aliquid } de illa praeda capere : \\\hline
1.2.30 & en essa misma manera \textbf{ los que quieren fazer de todo en todo riso } e enduzir alos otros & de illa praeda capere : \textbf{ sic volentes | omnino facere risum , } et prouocare alios ad cachinnum , \\\hline
1.2.30 & los que quieren fazer de todo en todo riso \textbf{ e enduzir alos otros } a escarnio non curan & omnino facere risum , \textbf{ et prouocare alios ad cachinnum , } non curant qualitercunque possint capere dicta , \\\hline
1.2.30 & a escarnio non curan \textbf{ en qual se quier manera puedan tomar los dichos o los fechos de los otros } e conuertir los entrebeio e en escarno . Et pues que assi es estos . & et prouocare alios ad cachinnum , \textbf{ non curant qualitercunque possint capere dicta , | vel facta aliorum , } et conuertere ea in ludum , \\\hline
1.2.30 & en qual se quier manera puedan tomar los dichos o los fechos de los otros \textbf{ e conuertir los entrebeio e en escarno . Et pues que assi es estos . } tales sobrepuian en los trebeios . & vel facta aliorum , \textbf{ et conuertere ea in ludum , } et cachinnum . \\\hline
1.2.30 & Ca uirtudes prinçipalmente çerca aquellas cosas \textbf{ que son mas guaues de fazer } mas repremir las superfluydadesde los iegos & non est tamen circa haec aeque principaliter , \textbf{ quia virtus semper est principalius circa difficilius . Reprimere autem superfluitates ludorum est difficilius , quam moderare defectus . Habet enim ipse ludus quandam delectationem annexam , propter quam magis inclinamur , } ut sequamur delectationes iocosas , \\\hline
1.2.30 & que son mas guaues de fazer \textbf{ mas repremir las superfluydadesde los iegos } es muy mas guaue cosa & non est tamen circa haec aeque principaliter , \textbf{ quia virtus semper est principalius circa difficilius . Reprimere autem superfluitates ludorum est difficilius , quam moderare defectus . Habet enim ipse ludus quandam delectationem annexam , propter quam magis inclinamur , } ut sequamur delectationes iocosas , \\\hline
1.2.30 & es muy mas guaue cosa \textbf{ que tenprar los fallesçimientos dellos . } Por que el iuego ha vna tentaçion ayuntada & quia virtus semper est principalius circa difficilius . Reprimere autem superfluitates ludorum est difficilius , quam moderare defectus . Habet enim ipse ludus quandam delectationem annexam , propter quam magis inclinamur , \textbf{ ut sequamur delectationes iocosas , } quam ut fugiamus illas . \\\hline
1.2.30 & Et despues desto ha de ser entenprando los fallescemientos . \textbf{ Et pues que assi es finca deuer } en qual manera conuiene alos Reyes & ex consequenti circa moderationem defectuum . \textbf{ Restat ergo videre , } quomodo Reges , \\\hline
1.2.30 & e alos prinçipes de ser alegres e iugadores . \textbf{ Ca assi commo puede paresçer delas cosas } que dichas son & et Principes decet esse iocundos . \textbf{ Ut enim ex habitis patere potest , } semper extrema vituperantur ; \\\hline
1.2.30 & que dichas son \textbf{ sienpre los estremos delas obras son de denostar } assi commo son los sobrepuiamentos de los iuegos & Ut enim ex habitis patere potest , \textbf{ semper extrema vituperantur ; } medium autem laudatur . \\\hline
1.2.30 & assi commo son los sobrepuiamentos de los iuegos \textbf{ mas tener el medio es de loar } mas este medio & semper extrema vituperantur ; \textbf{ medium autem laudatur . } Huiusmodi autem medium , \\\hline
1.2.30 & que repreme las passiones e los mouemientos \textbf{ non es de tomar } commo esta en la cosa & quod est in virtutibus reprimentibus passiones et motus , \textbf{ non est accipiendum } secundum rem , \\\hline
1.2.30 & commo esta en la cosa \textbf{ mas es de tomar commo paresçe a nos . } Ca assi commo es dicho en los capitulos de suso & secundum rem , \textbf{ sed quo ad nos . } Nam \\\hline
1.2.30 & Et por ende si conuiene aton dos los omes \textbf{ de repremir las sobeianias de los iuegos } mucho mas esto conuiene alos Reyes e alos prinçipes en tanto vsar tenpradamente delas delecta connes delos iuegos & et econuerso . \textbf{ Si igitur decet homines reprimere superfluitates ludorum , } magis hoc decet Reges et Principes : \\\hline
1.2.30 & de repremir las sobeianias de los iuegos \textbf{ mucho mas esto conuiene alos Reyes e alos prinçipes en tanto vsar tenpradamente delas delecta connes delos iuegos } que si esto feziesen algunas ottas personas comunes paresçeria & Si igitur decet homines reprimere superfluitates ludorum , \textbf{ magis hoc decet Reges et Principes : | immo oportet Reges , } et Principes adeo moderate uti iocosis delectationibus , \\\hline
1.2.30 & que serian montesinos e siluestres . \textbf{ Et pues que assi es non deuemos nos entender } que los Reyes e los prinçipes se deuen ti rar de todos solazis & viderentur esse durae et agrestes . \textbf{ Non ergo intelligendum est Reges , } et Principes non debere recreare aliquibus solaciis , \\\hline
1.2.30 & Et pues que assi es non deuemos nos entender \textbf{ que los Reyes e los prinçipes se deuen ti rar de todos solazis } nin de todos iuegos & Non ergo intelligendum est Reges , \textbf{ et Principes non debere recreare aliquibus solaciis , } vel aliquibus iocis : \\\hline
1.2.30 & Mas qua los trebeios e los uiegos \textbf{ de que deuemos iugar } segunt que dize el philosofo en el quarto libro delas ethicas & sed quia ludi , \textbf{ quibus uti debemus , } secundum Philosophum 4 Ethicorum , debent esse liberales et honesti , illis debent Reges , et Principes adeo moderate uti , \\\hline
1.2.30 & Et por ende los Reyes e los prinçipes \textbf{ en tanto deuen vsar tenpradamente destos uiegos } por que sienpre parescan maduros e entendudos & secundum Philosophum 4 Ethicorum , debent esse liberales et honesti , illis debent Reges , et Principes adeo moderate uti , \textbf{ ut semper videantur maturi , et nullo modo appareant pueriles . } Ab ipsa enim infantia \\\hline
1.2.30 & que aya en el alguna moçedat ayuntada Et pues que assi es en tanto conuiene alos Reyes \textbf{ e alos prinçipes de vsar tenpradamente delas delecta connes de los iuegos } en quanto mas de denostar es aellos de paresçer moços . & aliquam puerilitatem videtur habere annexam . \textbf{ Tanto igitur decet Reges et Principes moderate uti delectationibus ludorum , } quanto detestabilius est eos esse pueriles . \\\hline
1.2.30 & e alos prinçipes de vsar tenpradamente delas delecta connes de los iuegos \textbf{ en quanto mas de denostar es aellos de paresçer moços . } Otrossi assi commo paresçe & Tanto igitur decet Reges et Principes moderate uti delectationibus ludorum , \textbf{ quanto detestabilius est eos esse pueriles . } Amplius \\\hline
1.2.30 & en tanto mayor pecado es alos Reyes \textbf{ e alos prinçipes de vsar destenp̃damente } e desonestamente de las delecta connes de los iuegos & et a debitis curis : \textbf{ tanto detestabilius est Reges , et Principes immoderate , } vel inhoneste uti delectationibus ludorum , \\\hline
1.2.30 & en quanto el bien comun dela gente \textbf{ que pertenesçe de cuydar al prinçipe es mas alto et muy mayor } que auer cuydado de algun bien particular . & quanto cura boni communis , \textbf{ quae spectat ad Principem excellentior est , } quam sit cura alicuius particularis boni . \\\hline
1.2.30 & que pertenesçe de cuydar al prinçipe es mas alto et muy mayor \textbf{ que auer cuydado de algun bien particular . } Et pues que assi es non deuemos defender en ninguna manera las palauras o los fechos de iuego & quae spectat ad Principem excellentior est , \textbf{ quam sit cura alicuius particularis boni . } Non ergo omnino prohibenda sunt verba , \\\hline
1.2.30 & que auer cuydado de algun bien particular . \textbf{ Et pues que assi es non deuemos defender en ninguna manera las palauras o los fechos de iuego } si fueren honestos o tenprados . & quam sit cura alicuius particularis boni . \textbf{ Non ergo omnino prohibenda sunt verba , | vel facta iocosa , } si sint honesta et moderata : \\\hline
1.2.31 & que conuiene alos Reyes \textbf{ e alos prinçipes de auer alguas uirtudes . } Ca assi commo parescellanamente conuiene a ellos & Cum ergo omnino manifestum sit , \textbf{ quod decet Reges et Principes aliquas virtutes habere , } quia ( ut plane patet ) oportet eos esse prudentes et iustos : \\\hline
1.2.31 & Et por ende en todo en todo deue ser manifiesto \textbf{ que deue auer todas las uirtudes } por que la pradençia e la iustiçia sin las otras uirtudes non se pueden auer ante dezunos & omnino manifestum esse debet , \textbf{ quod omnes virtutes habere debent : } quia prudentia et iustitia sine virtutibus aliis haberi non possunt . \\\hline
1.2.31 & que deue auer todas las uirtudes \textbf{ por que la pradençia e la iustiçia sin las otras uirtudes non se pueden auer ante dezunos } que nunca es vna uirtud & quod omnes virtutes habere debent : \textbf{ quia prudentia et iustitia sine virtutibus aliis haberi non possunt . } Immo nunquam est aliqua una virtus , \\\hline
1.2.31 & empero que non pueden seer magnificos \textbf{ por que non pue den fazer grandes cosas } por que non han nin pueden fazer grandes espenssas . & qui tamen non possunt esse magnifici : \textbf{ quia nequeunt | magna facere , } eo quod non habeant magnos sumptus . \\\hline
1.2.31 & por que non pue den fazer grandes cosas \textbf{ por que non han nin pueden fazer grandes espenssas . } Et pues que assi es deuedes saber & magna facere , \textbf{ eo quod non habeant magnos sumptus . } Sciendum igitur , Philosophum circa finem 6 Ethicor’ \\\hline
1.2.31 & por que non han nin pueden fazer grandes espenssas . \textbf{ Et pues que assi es deuedes saber } quel philosofo çerca la fin del sexto libro delas ethicas prueua manifiestamente & eo quod non habeant magnos sumptus . \textbf{ Sciendum igitur , Philosophum circa finem 6 Ethicor’ } manifeste probare virtutes connexas esse . \\\hline
1.2.31 & e ayuntadas vna con otra . \textbf{ Mas por que pueda soluer estas razones } e estas abusiones sobredichas dize & manifeste probare virtutes connexas esse . \textbf{ Sed ut soluat huiusmodi obiectiones , } ait , \\\hline
1.2.31 & e estas abusiones sobredichas dize \textbf{ que las uirtudes pueden se tomar en dos maneras . } O en quanto son naturales & ait , \textbf{ quod virtutes dupliciter considerari possunt : } vel ut sunt naturales , \\\hline
1.2.31 & nin ayuntadas vna a otra \textbf{ por que veemos alguons naturalmente auer alguna n industria } e alguna sotileza de entendimiento . & non oportet esse connexas . \textbf{ Videmus enim aliquos naturaliter habere quandam industriam , } et quandam subtilitatem mentis : \\\hline
1.2.31 & los quales non son castos . \textbf{ Mas veemos a otros fazer el contrario desto } por que han algpradençia natural empero non son liberales Et pues que assi es en esta manera las uirtudes & qui non sunt casti : \textbf{ aliqui vero econtrario habent | quandam naturalem pudicitiam , } non tamen liberales existunt . Sic ergo virtutes non connectuntur . \\\hline
1.2.31 & luego farian grandes cosas . \textbf{ Et pues que assi es deuemos declarar } para conplido entendimiento de las cosas dichͣs & quia si bonis exterioribus abundarent , statim magnifica facerent . \textbf{ Declarandum est ergo ad plenam intelligentiam dictorum , } quod nulla virtus potest haberi perfecte , \\\hline
1.2.31 & si non fueren auidas todas las otras uirtudes . \textbf{ Et por ende deuedes notar e saber } que para esto nos son mester las uirtudes & nisi omnes virtutes aliae habeantur . \textbf{ Notandum ergo , } ad hoc nos indigere virtutibus , \\\hline
1.2.31 & Et por que derechamente e conueniblemente uayamos . a ellos . \textbf{ Et pues que assi es en dos maneras podemos pecar en tales cosas ¶ } Lo primero si establesçieremos & et debite tendamus in fines illos . \textbf{ Dupliciter ergo in talibus peccare contingit . Primo , } si proponamus nobis malum finem , \\\hline
1.2.31 & assi commo los auarientos establesçen \textbf{ assy en logar de fin estudiar en auariçia et en cobdiçia . } Et los destenprados en delectaçiones carnales establesçen & ut vitiosi faciunt . Auari enim proponunt sibi , \textbf{ ut finem , studere auaritiae . Intemperati vero , venerea , } et sic de aliis . Secundo in talibus peccare contingit , \\\hline
1.2.31 & assi en logar de fin obras de luxͣia . \textbf{ Et en essa misma manera deuemos entender en todos los otros que pecan¶ } Lo segundo contesçe de pecar en tales cosas & ut finem , studere auaritiae . Intemperati vero , venerea , \textbf{ et sic de aliis . Secundo in talibus peccare contingit , } si non debite tendamus in bonum finem . Volunt enim aliqui esse distributores bonorum , \\\hline
1.2.31 & Et en essa misma manera deuemos entender en todos los otros que pecan¶ \textbf{ Lo segundo contesçe de pecar en tales cosas } si non entendieremos conueniblemente en buena fin . & ut finem , studere auaritiae . Intemperati vero , venerea , \textbf{ et sic de aliis . Secundo in talibus peccare contingit , } si non debite tendamus in bonum finem . Volunt enim aliqui esse distributores bonorum , \\\hline
1.2.31 & donde o de qual parte tomen \textbf{ para fazer estas obras } tanto que puedan dar algunos dones alos otros . & attamen non curant undecunque accipiant , \textbf{ dum possint aliis dona aliqua elargiri : } et aliquando per furtum , \\\hline
1.2.31 & para fazer estas obras \textbf{ tanto que puedan dar algunos dones alos otros . } Et tales commo estos algunas vezes toman por furto & attamen non curant undecunque accipiant , \textbf{ dum possint aliis dona aliqua elargiri : } et aliquando per furtum , \\\hline
1.2.31 & e algunas uegadas por robo \textbf{ por que puedan fazer obras de largueza e de franqueza . } Et pues que assi es estos c̃mo & aliquando per manifestam oppressionem alios depraedantur , \textbf{ ut exerceant opera largitatis . } Hi ergo licet proponant sibi bonum finem , \\\hline
1.2.31 & mas pecan en la carrera \textbf{ e en la manera de ganar aquella fin } que demandan & sed peccant in via , \textbf{ et in modo adipiscendi finem intentum . Propter quod tales , } et si aliquo modo sunt liberales , non \\\hline
1.2.31 & por que pertenesçe ala uirtud acabada \textbf{ non solamente establesçer fin conuenible } mas ahun deuen yr conueniblemente a aquella fin . & ø \\\hline
1.2.31 & non solamente establesçer fin conuenible \textbf{ mas ahun deuen yr conueniblemente a aquella fin . } Et pues que assi es auemos meester las uirtudes morales & tamen perfecte liberales dici debent : quia ad perfectam virtutem spectat non solum proponere bonum finem , \textbf{ sed etiam debite tendere in illum finem . } Indigemus ergo virtutibus moralibus , \\\hline
1.2.31 & Et el liberal e el franco propone \textbf{ assi de despender . } Et el auariento de guardar . & ø \\\hline
1.2.31 & assi de despender . \textbf{ Et el auariento de guardar . } Por la qual cosa las uirtudes morales & ut intemperatus proponit sibi , \textbf{ ut finem venerea ; temperatus , casta ; liberalis , expendere ; auarus conseruare . Quare virtutes morales perficientes appetitum , } rectificant finem : \\\hline
1.2.31 & por que conueniblemente podamos yra aquella fin . \textbf{ Por la qual cosa deuemos nocaͬ } e saber que las uirtudes morales nos enderesçanala fin & et eligimus nobis rectam viam , \textbf{ ut debite tendamus in finem illum . Ideo dicitur 6 Ethic’ } quod virtutes morales rectificant finem , \\\hline
1.2.31 & Por la qual cosa deuemos nocaͬ \textbf{ e saber que las uirtudes morales nos enderesçanala fin } mas la pradençia e la sabiduria espeçialmente no os enderesça a aquellas cosas & ut debite tendamus in finem illum . Ideo dicitur 6 Ethic’ \textbf{ quod virtutes morales rectificant finem , } Prudentia vero facit operari \\\hline
1.2.31 & e enderesçan al omne ala fin . \textbf{ Mas la pradençia e la sabiduria faze obrar al omne derechͣmente aquellas cosas que son ordenadas a aquella fin . } Et pues que assi es fablando delas uirtudes dezimos & Loquendo ergo principaliter \textbf{ et primo , } virtus moralis rectificat terminum : \\\hline
1.2.31 & por buena e derecha carrera nunca nos dabadamente \textbf{ podemos auer ningunan uirtud moral } si aella non fuere ayuntada la pradençia e la sabiduria Et por ende entiende el philosofo fazer & nisi tendamus in ipsum per bonam viam , \textbf{ nunquam perfecte habebimus aliquam virtutem moralem , } nisi sit coniuncta prudentiae . Talem ergo rationem intendit \\\hline
1.2.31 & podemos auer ningunan uirtud moral \textbf{ si aella non fuere ayuntada la pradençia e la sabiduria Et por ende entiende el philosofo fazer } tal razon çerca la fin del sexto libro delas ethicas & nunquam perfecte habebimus aliquam virtutem moralem , \textbf{ nisi sit coniuncta prudentiae . Talem ergo rationem intendit } facere Philosophus circa finem 6 Ethicor’ . \\\hline
1.2.31 & e faga la su obra buena . \textbf{ Por ende commo havien escoger } e a buena obra fazer & et opus suum bonum reddat : \textbf{ cum ad bene eligere , et ad bonum opus , } sufficiat proponere bonum finem , \\\hline
1.2.31 & Por ende commo havien escoger \textbf{ e a buena obra fazer } non abasta de entender buena fin & et opus suum bonum reddat : \textbf{ cum ad bene eligere , et ad bonum opus , } sufficiat proponere bonum finem , \\\hline
1.2.31 & e a buena obra fazer \textbf{ non abasta de entender buena fin } si non fuere a aquella fin & cum ad bene eligere , et ad bonum opus , \textbf{ sufficiat proponere bonum finem , } nisi per bonam viam eatur in finem illum , \\\hline
1.2.31 & sin la pradençia e sabiduria . \textbf{ por la qual derechamente entendemos nos yr a aquella fin . } En essa misma manera avn la pradençia non puede ser sin uirtud moral . mas deuedes sabra & per quam nobis proponimus bonum finem , \textbf{ non potest esse sine prudentia per quam recte tendimus in finem illum . } Sic et prudentia esse non potest sine virtute morali . \\\hline
1.2.31 & En essa misma manera avn la pradençia non puede ser sin uirtud moral . mas deuedes sabra \textbf{ e notar } que ay departimiento entre la pradençia e la industria moral & Sic et prudentia esse non potest sine virtute morali . \textbf{ Differt enim prudentia , } et industria , \\\hline
1.2.31 & e de moticos \textbf{ si sopieren cuydar las carreras e los caminos } por los quales pueden alcançar las cosas delectables & et denidici , \textbf{ si sciant excogitare vias , } per quas consequantur venerea et turpia , quae sibi proponunt ut fines . Prudens tamen nullus dicitur , \\\hline
1.2.31 & si sopieren cuydar las carreras e los caminos \textbf{ por los quales pueden alcançar las cosas delectables } segunt la carne & ø \\\hline
1.2.31 & por las quals proponemos e ordenamos a nos abuea fin \textbf{ mas alguno diria o podria dezir } que la uirtud moral non podria ser sin la pradençia et sabiduria & per quas nobis proponimus bonum finem . \textbf{ Sed dicet aliquis virtutem moralem non posse esse sine prudentia , } nec econuerso , \\\hline
1.2.31 & que aquel que ha vna uirtud moral aya todas las uirtudes morales \textbf{ por que puede alguno auer acabadamente la tenpnca } e auer la pradençia e la sabiduria & non tamen oportet quod habens perfecte unam virtutem moralem , \textbf{ habeat omnes virtutes morales . Potest enim quis habere perfecte temperantiam , } et habere prudentiam , \\\hline
1.2.31 & por que puede alguno auer acabadamente la tenpnca \textbf{ e auer la pradençia e la sabiduria } en quanto sirue ala tenprança . & habeat omnes virtutes morales . Potest enim quis habere perfecte temperantiam , \textbf{ et habere prudentiam , } ut deseruit temperantiae : \\\hline
1.2.31 & e si non ouiere las otras uirtudes \textbf{ nin puede ninguno acabadamente auer alguna uirtud } si non ouiere todas las uirtudes . & et nisi habeat virtutes alias , \textbf{ non potest autem aliquis habere aliquam virtutem , } nisi habeat omnes virtutes . \\\hline
1.2.31 & Enpero por que fuyesse las feridas o los taiamientos de los mienbros \textbf{ o la muerte escogeria de obrar las cosas de luxia . } En essa misma manera avn si alguno fuesse auariento & ut fugeret verberationem , vel mutilationem , \textbf{ vel mortem , | eligeret venerea operari . } Sic etiam si esset auarus , \\\hline
1.2.31 & En essa misma manera avn si alguno fuesse auariento \textbf{ por que pusiesse la su fin en auer riquezas e dineros } commo quier que por auentra asi non plogeres en ael las cosas de luxuria . & Sic etiam si esset auarus , \textbf{ quia finem suum poneret in habendo pecuniam , } licet forte secundum se ei displicerent venerea : \\\hline
1.2.31 & commo quier que por auentra asi non plogeres en ael las cosas de luxuria . \textbf{ Enpero si pudiesse ganar el auer e los dineros } la qual cosa entendie & licet forte secundum se ei displicerent venerea : \textbf{ tamen si posset lucrari pecuniam , } quam intenderet \\\hline
1.2.31 & la qual cosa entendie \textbf{ assi commo su fin non aurie cuydado de fazer lux̉ia } en tal que pudiesse ganar algo . & quam intenderet \textbf{ ut finem , | non curaret moechari . Incomplete ergo , } et imperfecte potest haberi una virtus sine aliis : \\\hline
1.2.31 & assi commo su fin non aurie cuydado de fazer lux̉ia \textbf{ en tal que pudiesse ganar algo . } Et pues que assi es puede se auer vna atud & ø \\\hline
1.2.31 & en tal que pudiesse ganar algo . \textbf{ Et pues que assi es puede se auer vna atud } sin las otras & non curaret moechari . Incomplete ergo , \textbf{ et imperfecte potest haberi una virtus sine aliis : } sed complete et perfecte nullatenus fieri potest . \\\hline
1.2.31 & nin acabadamente mas conplida \textbf{ e acabadamente non se puede auer vna sin todas las otras . } Por la qual cosa si conuiene alos Reyes e alos prinçipes de ser & et imperfecte potest haberi una virtus sine aliis : \textbf{ sed complete et perfecte nullatenus fieri potest . } Quare sic decet Reges , et Principes esse quasi semideos , \\\hline
1.2.31 & assi commo medios dioses \textbf{ e auer las uirtudesacabadas . } Conuiene a ellos de auer todas las uirtudes & Quare sic decet Reges , et Principes esse quasi semideos , \textbf{ et habere virtutes perfectas : } decet eos habere omnes virtutes , \\\hline
1.2.31 & e auer las uirtudesacabadas . \textbf{ Conuiene a ellos de auer todas las uirtudes } por que acabadamente vna uirtud sin las otras non puede ser auida & et habere virtutes perfectas : \textbf{ decet eos habere omnes virtutes , } quia perfecte una virtus sine aliis haberi non potest . Immo expedit Regibus et Principibus , \\\hline
1.2.31 & e alos prinçipes \textbf{ commo ellos non se puedan escusar } por mengua de los bienes & quia perfecte una virtus sine aliis haberi non potest . Immo expedit Regibus et Principibus , \textbf{ cum non possint se excusare } per defectum exteriorum bonorum , \\\hline
1.2.31 & nin de los algos tenporales \textbf{ de auer todas las uirtudes } non solamente delas auer en disposi conn çercana . & per defectum exteriorum bonorum , \textbf{ habere omnes virtutes , } non solum in propinqua dispositione , \\\hline
1.2.31 & de auer todas las uirtudes \textbf{ non solamente delas auer en disposi conn çercana . } Mas avn sinplemente & habere omnes virtutes , \textbf{ non solum in propinqua dispositione , } sed etiam simpliciter , \\\hline
1.2.31 & Mas avn sinplemente \textbf{ e acabadamente las deuen auer . } Ca pue de algun pobre ser acabado en uirtudes & sed etiam simpliciter , \textbf{ et secundum se . Potest enim pauper aliquis esse perfectus , } si non operetur magnifica , \\\hline
1.2.32 & en el sesto libro delas ethicas \textbf{ uenmos departir quatro quados de malos } e quitro quados de buenos & Si verba Philosophi in 7 Ethicorum diligentius considerentur , \textbf{ distinguere possumus quatuor gradus malorum , } et quatuor bonorum . \\\hline
1.2.32 & e otros son incontinentes \textbf{ que non se pueden contener . } Et alguons son destenpdos . & quidam sunt molles , \textbf{ quidam incontinentes , } quidam intemperati , \\\hline
1.2.32 & segunt el philosofo \textbf{ que non quieren trabaiar } e son muy delicados . & quod de facile cedit . Dicuntur ergo molles secundum Philosophum , \textbf{ qui nolunt laborare , } et sunt nimis delicati . \\\hline
1.2.32 & Et por ende estos tales \textbf{ non quariendo sofrir ninguna cosa guaue } luego que padelçen o son passionados por alguna passion & delicia quaedam mollicies est . \textbf{ Tales ergo nihil difficile sustinere volentes , } statim cum passionantur , \\\hline
1.2.32 & e puestos los non continentes \textbf{ que se non pueden contener . } mas estos incontinentes han departimiento & ø \\\hline
1.2.32 & que son dichos muelles \textbf{ Ca los muelles non pueden sofrir ninguna tentaçion } nin ninguna cosa guaue & Hi autem differunt a primis . \textbf{ Nam molles nullam pugnam , } nec aliquid difficile sustinere possunt . Incontinentes vero pugnam sustinent , \\\hline
1.2.32 & assi commo lo muestrael uocablo mismo . \textbf{ Ca contener se este nerse contra alguna cosa . } Et por enerde el non contener se es acometer algua batalla & ø \\\hline
1.2.32 & Ca contener se este nerse contra alguna cosa . \textbf{ Et por enerde el non contener se es acometer algua batalla } e en aquella batalla non se poder tener & sed in sustinendo deficiunt . Continere enim , \textbf{ ut ipsum nomen designat , hoc est , contra aliud se tenere . Incontinere ergo est aggredi pugnam , et in pugna non posse se tenere , } sed deficere . \\\hline
1.2.32 & Et por enerde el non contener se es acometer algua batalla \textbf{ e en aquella batalla non se poder tener } mas fallesçer en el ła¶ & sed in sustinendo deficiunt . Continere enim , \textbf{ ut ipsum nomen designat , hoc est , contra aliud se tenere . Incontinere ergo est aggredi pugnam , et in pugna non posse se tenere , } sed deficere . \\\hline
1.2.32 & e en aquella batalla non se poder tener \textbf{ mas fallesçer en el ła¶ } En el terçero guado de malos son los destenprados . & ut ipsum nomen designat , hoc est , contra aliud se tenere . Incontinere ergo est aggredi pugnam , et in pugna non posse se tenere , \textbf{ sed deficere . } In tertio gradu sunt intemperati . Intemperatus enim tunc dicitur aliquis , \\\hline
1.2.32 & e es uençido por las tentaçiones \textbf{ mas non le es cosa delectable de mal fazer } quando cae . & et per tentationes deiicitur , \textbf{ sed delectabile est ei malefacere . Incontinentes ergo } et molles non delectantur in malefacere : \\\hline
1.2.32 & Et pues que assi es los non continentes \textbf{ e los muelles non se delectan en mal fazer } mas escogen conueniblemente & sed delectabile est ei malefacere . Incontinentes ergo \textbf{ et molles non delectantur in malefacere : } immo aliud eligunt , \\\hline
1.2.32 & mas escogen conueniblemente \textbf{ e obran o coriiençan a obrar conueniblemente } por que ellos estando fuera delas passiones & immo aliud eligunt , \textbf{ et aliud agunt existentes enim extra passiones tam incontinentes , } quam molles , \\\hline
1.2.32 & por las quales son tentados tan bien los non continentes \textbf{ commo los muelles escogen bien fazer } e ponen assi muy buean sleyes & quam molles , \textbf{ eligunt benefacere , } et proponunt optimas leges . Passionati vero cadunt , \\\hline
1.2.32 & assemeia los alos paraliticos \textbf{ los quales quieran yr ala diestra parte . } Enpero por la dissoluçion & ø \\\hline
1.2.32 & Enpero por la dissoluçion \textbf{ del que non puede bien gouernar el cuerpo } van ala simestro En essa misma manera los muelles & et nullam de illis legibus obseruant . Tales autem Philosophus assimilat paralyticis , \textbf{ qui eligentes ire in dextram , propter dissolutionem corporis , et non valentes corpus regere , } vadunt in sinistram . \\\hline
1.2.32 & van ala simestro En essa misma manera los muelles \textbf{ e los non continentes proponen de bien fazer } e escogen de yr ala diestra . & vadunt in sinistram . \textbf{ Sic molles et incontinentes proponunt benefacere , } et eligunt ire in dextram : \\\hline
1.2.32 & e los non continentes proponen de bien fazer \textbf{ e escogen de yr ala diestra . } Empero por que han los poderios del alma dessoluidos & Sic molles et incontinentes proponunt benefacere , \textbf{ et eligunt ire in dextram : } tamen quia habent potentias animae dissolutas , \\\hline
1.2.32 & que comne las carnes delos omes e beune la sangre de los omes \textbf{ Et tales cosas commo estas non pueden uenir } si non dela bestialidat . & ø \\\hline
1.2.32 & que commo fuesse vna mugier prenada \textbf{ e non pudiesse parir desto conçibio tan grant dolor } que se torno en bestialidat & multas huiusmodi bestialitates . Dicit enim quod cum quaedam praegnans esset , \textbf{ et non potuisset parere , | ex hoc tantum dolorem concepit , } et sic in bestialitatem conuersa est , \\\hline
1.2.32 & e fuessen abiertas \textbf{ e porque ella non podia parir } quaria & ut omnes praegnantes resideret ; \textbf{ quia ergo ipsa parere non poterat , } volebat \\\hline
1.2.32 & que ninguon dende adelante non pariesse ¶ \textbf{ Er essa misma manera } assi commo dize el philosofo & ut nulla de caetero parturiret . \textbf{ Sic etiam ( ut ait ) circa quandam Insulam maris } ut circa Pontum existebant \\\hline
1.2.32 & que los comiessen . \textbf{ Ca quando alguno quaria conbidar a } otrossi el su fiio non era en casa tomaua prestado el fijo de otro su vezino & praestabant sibi filios inconuiuiis . \textbf{ Cum enim | qui alios conuiuare volebat , } si filius suus domi non erat , \\\hline
1.2.32 & e aprestaual \textbf{ para fazer el conbit et prometial } que quando quisiesse fazer conbit & a vicino suo mutuabat filium , \textbf{ et ipsum parabat in conuiuium , spondens } quod quando vellet conuiuium facere , ei suum filium tribueret . Sic \\\hline
1.2.32 & para fazer el conbit et prometial \textbf{ que quando quisiesse fazer conbit } que el qual daria su fijo & et ipsum parabat in conuiuium , spondens \textbf{ quod quando vellet conuiuium facere , ei suum filium tribueret . Sic } etiam multae de Phalaride bestialitates narrantur . Quaedam \\\hline
1.2.32 & ¶ dicho de los quatroguados delons malos \textbf{ finca de dezir delos quatro linages de los buenos } Ca assi conmo algunos muelles son malos & qui ultra modum hominum male agunt . Dicto de quatuor generibus malorum , \textbf{ restat dicere de quatuor generibus bonorum . } Nam sicut quidam mali sunt molles , \\\hline
1.2.32 & Mas enel segundo grado de los buenos son puestos los continentes \textbf{ por que contener se el omne } e tener se de fazer males & quam perseuerare . \textbf{ Nam si quis etiam non tentatus , } vel modicum tentatus non ruat , \\\hline
1.2.32 & por que contener se el omne \textbf{ e tener se de fazer males } mas queꝑ seuerar . & quam perseuerare . \textbf{ Nam si quis etiam non tentatus , } vel modicum tentatus non ruat , \\\hline
1.2.32 & e tener se de fazer males \textbf{ mas queꝑ seuerar . } Ca quando alguno tentado poco o muy tentado non cae & Nam si quis etiam non tentatus , \textbf{ vel modicum tentatus non ruat , } dicitur perseuerare : \\\hline
1.2.32 & Ca quando alguno tentado poco o muy tentado non cae \textbf{ este es dicho ꝑse uerar . } Mas non es dicho contener se saluos & vel modicum tentatus non ruat , \textbf{ dicitur perseuerare : } sed non dicitur continere , \\\hline
1.2.32 & este es dicho ꝑse uerar . \textbf{ Mas non es dicho contener se saluos } i fuere fuertemente passionado & dicitur perseuerare : \textbf{ sed non dicitur continere , } nisi fortiter passionatus passiones illas vincat . \\\hline
1.2.32 & Et por ende el philosofo en el septimo libro delas ethicas dize \textbf{ que la continençia es mas de escoger e de loar } que la persseuerança . & Ideo 7 Ethicorum dicitur , \textbf{ quod continentia elegibilior est , } quam perseuerantia . In tertio autem gradu bonorum sunt temperati . Illi enim temperati esse dicuntur , \\\hline
1.2.32 & que non sienten aquella batalla nin aquella tentaçion \textbf{ mas es los a ellos cosa delectable de bien fazer ¶ } Et pues que assi es & quod quasi pugnam non sentiunt , \textbf{ et delectabile est eis benefacere . } Sicut ergo perseuerantes opponuntur mollibus , \\\hline
1.2.32 & en essa misma man era los tenprados son contrarios alos destenprados . \textbf{ Ca assi commo es cosa delectable alos destenprados de mal fazer } assi es cosa delectable a los tenprados de bien obrar ¶ & sic temperati opponuntur intemperatis . \textbf{ Nam sicut delectabile est intemperatis mala facere , } sic delectabile est temperatis bona operari . In quarto \\\hline
1.2.32 & Ca assi commo es cosa delectable alos destenprados de mal fazer \textbf{ assi es cosa delectable a los tenprados de bien obrar ¶ } En el quartoguado & Nam sicut delectabile est intemperatis mala facere , \textbf{ sic delectabile est temperatis bona operari . In quarto } et in supremo gradu bonorum , sunt homines diuini . \\\hline
1.2.32 & es llamada del philosofo eroyca \textbf{ que quiere dezir prinçipante e sennor ante } por que es señora delas otras uirtudes ¶ & a Philosopho heroica idest principans , \textbf{ et dominatiuat . } Ex hoc ergo manifeste patet , \\\hline
1.2.32 & que los Reyes e los prinçipes \textbf{ si derechamente deuen ensseñorear } non les abasta a ellos de foyr todos los grados de los malos & et Principes si debens \textbf{ recte dominari , } non sufficit eos fugere omnes gradus malorum , \\\hline
1.2.32 & si derechamente deuen ensseñorear \textbf{ non les abasta a ellos de foyr todos los grados de los malos } assi que non sean muelles & recte dominari , \textbf{ non sufficit eos fugere omnes gradus malorum , } et quod non sint nec molles , \\\hline
1.2.32 & Mas conuiene aellos de ser en el mas alto grado de los buenos \textbf{ por que aquel que dessea de prinçipar e enssenorear alos otros . } Conuienele que aya aquella uirtud & nec intemperati , nec bestiales , sed oportet eos esse in summo gradu bonorum : \textbf{ qui enim aliis dominari , | et principari desiderant , } oportet quod habeant virtutem illam , \\\hline
1.2.33 & que van en semeiança diunal \textbf{ e tales son dichos auer uirtudes pgatorias . } Mas otros algunos son & Nam aliqui sunt tendentes , \textbf{ et euntes in diuina similitudine : } et tales dicuntur habere virtutes purgatorias . Aliqui vero sunt quodammodo iam assecuti similitudinem illam : \\\hline
1.2.33 & que en algua manera han ya conssigo esta semeiança diuinal \textbf{ e tales son dichos auer uirtudes de pgado coraçon . } Mas commo quier que estos digan cosas uerdaderas & et tales dicuntur habere virtutes purgatorias . Aliqui vero sunt quodammodo iam assecuti similitudinem illam : \textbf{ et tales habere dicuntur virtutes purgati animi . } Sed hi licet \\\hline
1.2.33 & ¶Et pues que assi es siguiendo el camino \textbf{ et la carrera de los philosofos podemos dezir } que assi commo contesçe de dar guados de ptidos de bueons & quam ponebant , \textbf{ dicebant esse acquisitam . Sectando ergo Philosophorum viam , } dicere possumus , \\\hline
1.2.33 & et la carrera de los philosofos podemos dezir \textbf{ que assi commo contesçe de dar guados de ptidos de bueons } assi conuiene de dar den parti dos linages de uirtudes . & dicebant esse acquisitam . Sectando ergo Philosophorum viam , \textbf{ dicere possumus , | quod sicut est dare diuersos gradus bonorum , sic est dare diuersa virtutum genera , } ita quod \\\hline
1.2.33 & que assi commo contesçe de dar guados de ptidos de bueons \textbf{ assi conuiene de dar den parti dos linages de uirtudes . } Conuiene a saber que segunt que cada vno es mas altamente bueno & quod sicut est dare diuersos gradus bonorum , sic est dare diuersa virtutum genera , \textbf{ ita quod } secundum quod aliquis est excellentior bonus , excellentiorem etiam gradum virtutum habet . \\\hline
1.2.33 & assi conuiene de dar den parti dos linages de uirtudes . \textbf{ Conuiene a saber que segunt que cada vno es mas altamente bueno } ha mas alto grado de uirtudes . & ita quod \textbf{ secundum quod aliquis est excellentior bonus , excellentiorem etiam gradum virtutum habet . } Sicut ergo Philosophus innuit quatuor genera bonorum , \\\hline
1.2.33 & e algunos tenprados e algunos diuinales . \textbf{ Et en essa misma manera podemos departir quatro ordenes de uirtudes } assi que a cada vn linage de los bueons demos su orden de uirtudes . & aliquos continentes , \textbf{ aliquos temperatos , } aliquos vero diuinos , \\\hline
1.2.33 & Et que estos linages destas uirtudesse \textbf{ de una reduzir } alos dichos linages de los buenos esto paresçe & et purgatoriae politicas . \textbf{ Quod autem haec genera virtutum adaptari debeant praedictis generibus bonorum , } patet per Plotinum dicentem , \\\hline
1.2.33 & do dize \textbf{ que las primeras uirtudes conuiene saber . } Las politicas amollesçen & patet per Plotinum dicentem , \textbf{ quod primae virtutes , } scilicet politicae , molliunt idest ad medium reducunt . Secundae , \\\hline
1.2.33 & Las politicas amollesçen \textbf{ e desponen el coraçon a bien fazer } e reduzen lo a medio¶ & ø \\\hline
1.2.33 & e reduzen lo a medio¶ \textbf{ Et las uirtudes segundas conuiene saber las pgatorias tiran } e fazen oluidar las passiones & quod primae virtutes , \textbf{ scilicet politicae , molliunt idest ad medium reducunt . Secundae , } scilicet purgatoriae , \\\hline
1.2.33 & Et las uirtudes segundas conuiene saber las pgatorias tiran \textbf{ e fazen oluidar las passiones } de que el alma padesçe ¶ & scilicet politicae , molliunt idest ad medium reducunt . Secundae , \textbf{ scilicet purgatoriae , } auferunt . Tertiae , \\\hline
1.2.33 & que son del coraçon pragado \textbf{ fazen oluidar del todo las passiones . ¶ } Mas en las quartas uirtades & quae sunt purgati animi , \textbf{ obliuiscuntur . } Sed in quartis , scilicet exemplaribus , \\\hline
1.2.33 & que son exenplares \textbf{ serie muy mala cosa deñobrar ninguna cosa torpe en ellas } por la qual cosa bien dicho es & Sed in quartis , scilicet exemplaribus , \textbf{ nefas est turpe aliquod nominari . } Quare bene dictum est , \\\hline
1.2.33 & Et pues que assi es las uirtudes politicas \textbf{ que amollesçen e ordenan el coraçon abien fazer } e lo reduzen a medio parte nesçen alos perseuerantes & Virtutes ergo politicae quae molliunt , \textbf{ idest disponunt animum ad benefaciendum , } et reducunt ipsum ad medium , \\\hline
1.2.33 & e lidian contra ellas . \textbf{ Mas la manera para vençer estas passiones es tirar se } e partir se dellas . & ø \\\hline
1.2.33 & Mas la manera para vençer estas passiones es tirar se \textbf{ e partir se dellas . } Et por ende es dich̃o & et abstinet se a delectationibus sensibilibus . Continentes enim , sunt euntes in passionibus , et pugnant contra ipsas . Modus autem vincendi eas est auferre , \textbf{ et remouere se de eis . } Ideo continentibus dicuntur competere virtutes purgatoriae , \\\hline
1.2.33 & que estas uirtudes pragatorias pertenesçen alos continentes \textbf{ por que el oficio dellas es tyrar } e partir nos delas passiones & Ideo continentibus dicuntur competere virtutes purgatoriae , \textbf{ quorum officium est auferre , } et abstinere nos a passionibus immoderatis . Virtutes vero purgati animi competunt temperatis : \\\hline
1.2.33 & por que el oficio dellas es tyrar \textbf{ e partir nos delas passiones } e delas tentaçiones destenpradas & quorum officium est auferre , \textbf{ et abstinere nos a passionibus immoderatis . Virtutes vero purgati animi competunt temperatis : } nam istae non auferunt , \\\hline
1.2.33 & que non tiran las passiones \textbf{ mas fazen las oluidar . } Ca el tenprado & et abstinere nos a passionibus immoderatis . Virtutes vero purgati animi competunt temperatis : \textbf{ nam istae non auferunt , } sed obliuiscuntur . Temperatus enim habet sic appetitum castigatum , \\\hline
1.2.33 & mas oluida las passiones destepradas . \textbf{ Et en algunan manera le es delectable de bien fater . } Et por ende con grant razon es dicho el tal auer las uirtudes del coraçon pragado & sed obliuiscitur passionum immoderatarum , \textbf{ et quodammodo delectabile est ei benefacere . } Merito ergo talis dicitur habere virtutes purgati animi facientes ipsum obliuisci passiones illas crebras , \\\hline
1.2.33 & Et en algunan manera le es delectable de bien fater . \textbf{ Et por ende con grant razon es dicho el tal auer las uirtudes del coraçon pragado } las quales le fazen escaeçer e oluidar las passiones & et quodammodo delectabile est ei benefacere . \textbf{ Merito ergo talis dicitur habere virtutes purgati animi facientes ipsum obliuisci passiones illas crebras , } quia \\\hline
1.2.33 & Et por ende con grant razon es dicho el tal auer las uirtudes del coraçon pragado \textbf{ las quales le fazen escaeçer e oluidar las passiones } e las delectaçiones desordenadas & et quodammodo delectabile est ei benefacere . \textbf{ Merito ergo talis dicitur habere virtutes purgati animi facientes ipsum obliuisci passiones illas crebras , } quia \\\hline
1.2.33 & nin de tales tentaçiones . \textbf{ ¶ Mas las quartas uirtudes conuiene a saber . } las ezenplares pueden part & et castigatum , \textbf{ ut non sit ei curae de talibus passionibus . Quartae vero virtutes , } videlicet , exemplares adaptari possunt hominibus diuinis : \\\hline
1.2.33 & las ezenplares pueden part \textbf{ e nesçer alos omes diuinales } por que tales en tanto deuen ser acabados & ut non sit ei curae de talibus passionibus . Quartae vero virtutes , \textbf{ videlicet , exemplares adaptari possunt hominibus diuinis : } tales enim perfecti adeo debent esse , \\\hline
1.2.33 & es lo primero \textbf{ que propusiemos en este capitulo conuiene saber } que son departidos quados de uirtudes . & et exemplar . Declaratum est ergo primum , \textbf{ quod in principio capituli proponebatur , } videlicet diuersos esse gradus virtutum . Declarare vero \\\hline
1.2.33 & que son departidos quados de uirtudes . \textbf{ Mas lo segundo declarar e demostrar } que estas tales uirtudes deuen parte nesçer alos Reyes e alos prinçipes & videlicet diuersos esse gradus virtutum . Declarare vero \textbf{ secundum , | et ostendere } cuiusmodi virtutes Regibus , \\\hline
1.2.33 & Mas lo segundo declarar e demostrar \textbf{ que estas tales uirtudes deuen parte nesçer alos Reyes e alos prinçipes } si fueren entendidas las cosas & et ostendere \textbf{ cuiusmodi virtutes Regibus , | et Principibus competere debeant , } si intelligantur praehabita , \\\hline
1.2.33 & e la grant perfeçion del prinçipe e del señor . \textbf{ por la qual cosanon solamente es mucho de denostar } e de renphender alos Reyes & videns vitam et perfectionem principantis . \textbf{ Quare apud Reges et Principes non solum detestabile esse } debet eos turpia operari , \\\hline
1.2.33 & por la qual cosanon solamente es mucho de denostar \textbf{ e de renphender alos Reyes } e alos prinçipes de cbrar cosas torpes & videns vitam et perfectionem principantis . \textbf{ Quare apud Reges et Principes non solum detestabile esse } debet eos turpia operari , \\\hline
1.2.33 & e de renphender alos Reyes \textbf{ e alos prinçipes de cbrar cosas torpes } mas avn delas oyr no obra ¶ Et pues que assi es mucho ꝑ tenesçen aellos las uirtudes exenplares & Quare apud Reges et Principes non solum detestabile esse \textbf{ debet eos turpia operari , } sed etiam nominare , \\\hline
1.2.33 & e alos prinçipes de cbrar cosas torpes \textbf{ mas avn delas oyr no obra ¶ Et pues que assi es mucho ꝑ tenesçen aellos las uirtudes exenplares } segunt las quales uirtudes & debet eos turpia operari , \textbf{ sed etiam nominare , | et audire turpia nefas esse debet . } Bene ergo eis competunt exemplares virtutes , \\\hline
1.2.33 & segunt las quales uirtudes \textbf{ assi commo dize plotino aquel philosofo grand pecado es de nonbrar cosas torpes . } Et commo ninguno non pueda ser de tan grand bondat & Bene ergo eis competunt exemplares virtutes , \textbf{ quibus secundum Plotinum nefas est turpia nominari : } sed cum tantae bonitatis nullus esse possit absque Dei gratia , et eius auxilio : \\\hline
1.2.33 & tanto mas cobdiçiosamente \textbf{ e con mayor desseo deuen demandar la gera de dios . } ¶ Et pues que & tanto ardentius decet \textbf{ eos diuinam gratiam postulare . } In hoc ergo eliditur Philosophorum elatio , \\\hline
1.2.33 & ¶ Et pues que \textbf{ assi es por esto se puede tirar } e quebrantar el orgullo & ø \\\hline
1.2.33 & assi es por esto se puede tirar \textbf{ e quebrantar el orgullo } e la soƀͣuia de los philosofos & eos diuinam gratiam postulare . \textbf{ In hoc ergo eliditur Philosophorum elatio , } violentium quod ex puris naturalibus possemus \\\hline
1.2.33 & que dixieron \textbf{ que por prinçipios puros naturales podriemos escusar todos los males } e podriemos ganar bondat acabada . & In hoc ergo eliditur Philosophorum elatio , \textbf{ violentium quod ex puris naturalibus possemus | omnia mala vitare , } et perfectam bonitatem acquirere . \\\hline
1.2.33 & que por prinçipios puros naturales podriemos escusar todos los males \textbf{ e podriemos ganar bondat acabada . } ssi commo es dicho dsuso commo quier que largamente tomando las uirtudes & omnia mala vitare , \textbf{ et perfectam bonitatem acquirere . } Dicebatur enim supra , \\\hline
1.2.34 & por las quales todas estas cosas an de ser mostradas . \textbf{ Mas iustas las cosas dichas de suso non es cosa guaue demostrar } en qual manera estas cosas se han assi . & per quae haec omnia innotescunt . \textbf{ Sed visis praehabitis , } ostendere quomodo haec sic se habent , non est difficile . \\\hline
1.2.34 & en qual manera estas cosas se han assi . \textbf{ Ca enbolia que es uirtud para conseiar } e sin esis & Sed visis praehabitis , \textbf{ ostendere quomodo haec sic se habent , non est difficile . } Nam eubulia , \\\hline
1.2.34 & que desponen al omne \textbf{ para yr ala uirtud } assi commo es la perseuerança e la continençia . & sed quaedam sunt disponentes ad virtutem , \textbf{ ut perseuerantia , } et continentia . \\\hline
1.2.34 & por que al uirtuoso es cosa delectable \textbf{ de bien obrar } e de bien fazer . & Continentia enim non proprie est virtus , \textbf{ quia virtuoso delectabile est benefacere : } continens enim licet non sequatur passionem , \\\hline
1.2.34 & de bien obrar \textbf{ e de bien fazer . } Et el continente commo quier que non sigua las passiones & quia virtuoso delectabile est benefacere : \textbf{ continens enim licet non sequatur passionem , } sed rationem . \\\hline
1.2.34 & por razon dela lid \textbf{ que siente non es a el cosa delectable de bien fazer Et pues } que assi es en quanto alguno es continente & nisi \textbf{ quia contra passiones se tenet , ratione pugnae quam sentit , non est ei delectabile benefacere . Quandiu ergo aliquis est continens , et quandiu habet passiones fortes , } non habet perfectum usum rationis et virtutis : \\\hline
1.2.34 & que se sigue ala uirtud . \textbf{ Mas declarar en qual manera la perseuerança es disposicion ala uirtud . } Et en qual manera es condicion segniente ala uirtud & sed vel est dispositio ad virtutem , \textbf{ vel est quaedam conditio sequens virtutem . Declarare ergo quomodo perseuerantia est dispositio ad virtutem , } et quomodo est conditio sequens virtutem , non est praesentis speculationis . \\\hline
1.2.34 & non es deste presente negoçio \textbf{ nin parte nesçe anos de tractar dello } mas quanto pertenesçe a lo presente abasta de saber & et quomodo est conditio sequens virtutem , non est praesentis speculationis . \textbf{ Sufficit autem ad praesens scire , } quod loquendo de perseuerantia \\\hline
1.2.34 & nin parte nesçe anos de tractar dello \textbf{ mas quanto pertenesçe a lo presente abasta de saber } que fablando dela perseuerança & et quomodo est conditio sequens virtutem , non est praesentis speculationis . \textbf{ Sufficit autem ad praesens scire , } quod loquendo de perseuerantia \\\hline
1.2.34 & mas la uirtud heroica \textbf{ la qual podemos llamar diuinal } mas es sobre uirtud & est quaedam dispositio ad virtutem . Virtus autem heroica , \textbf{ quam diuinam vocare possumus , } magis est supra virtutem , \\\hline
1.2.34 & que disponen al omne \textbf{ para yr a uirtud } e algunas son & quaedam annexae virtutibus , \textbf{ quaedam disponentes ad virtutes , } quaedam vero sunt supra virtutes . \\\hline
1.2.34 & ¶ Et pues que asy es \textbf{ por que los Reyes e los prinçipes puedan auer estas bueans disposiconnes del alma } conuiene aellos de conosçer estos linages & quaedam vero sunt supra virtutes . \textbf{ Ut ergo Reges , et Principes possint bonas dispositiones mentis habere , } decet eos haec genera dispositionum cognoscere . \\\hline
1.2.34 & por que los Reyes e los prinçipes puedan auer estas bueans disposiconnes del alma \textbf{ conuiene aellos de conosçer estos linages } e estas maneras destas lueans disposiciones . & Ut ergo Reges , et Principes possint bonas dispositiones mentis habere , \textbf{ decet eos haec genera dispositionum cognoscere . } TERTIA PARS Primi Libri de regimine Principum : \\\hline
1.3.1 & Ca mostrado es de ssuso \textbf{ enque deuen los Reyes e los prinçipes poner su fin e su bien andança . } Et otrosi mostrado es en commo les conuiene de ser uirtuosos ¶ & Expeditis duabus partibus huius operis , \textbf{ quia ostensum est in quo Reges et Principes suum finem ponere debeant , } et quomodo oportet eos virtuosos esse . \\\hline
1.3.1 & Et otrosi mostrado es en commo les conuiene de ser uirtuosos ¶ \textbf{ Agora finca de dezir dela tercera parte deste libro } e mostrando quales passiones & et quomodo oportet eos virtuosos esse . \textbf{ Restat exequi de tertia parte huius primi libri , } ostendendo quas passiones , \\\hline
1.3.1 & e quales mouimientos de coraçon \textbf{ deuen seguir los Reyes e los prinçipes . } Mas por que esto non se puede saber & ostendendo quas passiones , \textbf{ et quos motus animi Reges et Principes debeant imitari . } Sed cum hoc sciri non possit , \\\hline
1.3.1 & deuen seguir los Reyes e los prinçipes . \textbf{ Mas por que esto non se puede saber } si non sopieremos primero & et quos motus animi Reges et Principes debeant imitari . \textbf{ Sed cum hoc sciri non possit , } nisi prius sciuerimus \\\hline
1.3.1 & e quales dellas son mas prinçipales \textbf{ e quales de alabar } e quales de denostar ¶ & et quae illarum sunt magis principales , \textbf{ et quae sunt laudabiles , } et quae vituperabiles : \\\hline
1.3.1 & e quales de alabar \textbf{ e quales de denostar ¶ } por ende primeramente tractaremos de estas passiones & et quae sunt laudabiles , \textbf{ et quae vituperabiles : } ideo de his primo tractabimus . Accipiendo autem numerum passionum , \\\hline
1.3.1 & que eran doze uirtudes \textbf{ assi podemos dezinr } que las passiones son & ø \\\hline
1.3.1 & que las passiones son \textbf{ doze conuiene saber amor } e mal querençia e desseo . & ø \\\hline
1.3.1 & e mal querençia e desseo . \textbf{ e aborrençia er delectacion . } e tristeza e esperança e desesperança e temor e osadia . & sicut dicebamus esse duodecim virtutes , sic dicere possumus quod sunt duodecim passiones : \textbf{ videlicet , amor , odium , desiderium , abominatio , delectatio , tristitia , spes , desperatio , timor , audacia , ira , et mansuetudo . Computabatur enim supra mansuetudo inter virtutes : } sed hoc est propter vocabulorum penuriam , \\\hline
1.3.1 & Mas por que ha de ser alguna uirtud entre la sanna e la mansedunbre \textbf{ la qual uirtud non podemos nonbrar } por su nonbre propreo nonbramos la por nonbre de mansedunbre & Sed cum sit quaedam virtus \textbf{ inter iram et mansuetudinem , } quia virtutem illam proprio nomine nominare nescimus , nominamus eam nomine mansuetudinis , \\\hline
1.3.1 & e quoco ala uirtud e ala passion contraria dela sana . \textbf{ Mas si alguno quisi esse trabaiar } de enponer o fallar & eo quod illa virtus plus communicat cum mansuetudine , quam cum ira . Erit mansuetudo aequiuocum ad virtutem , et ad passionem oppositam irae . \textbf{ Si quis autem laborare vellet , } cuilibet posset inuenire nomen proprium . \\\hline
1.3.1 & Mas si alguno quisi esse trabaiar \textbf{ de enponer o fallar } non bͤapio a cada vna cosa podia lo fazer & Si quis autem laborare vellet , \textbf{ cuilibet posset inuenire nomen proprium . } Sed cum constat de re , \\\hline
1.3.1 & de enponer o fallar \textbf{ non bͤapio a cada vna cosa podia lo fazer } mas quando nos somos çiertos dela cosa non deuemos auer cuydado delas palauras . & Si quis autem laborare vellet , \textbf{ cuilibet posset inuenire nomen proprium . } Sed cum constat de re , \\\hline
1.3.1 & non bͤapio a cada vna cosa podia lo fazer \textbf{ mas quando nos somos çiertos dela cosa non deuemos auer cuydado delas palauras . } Et pues que assi es contadas las passiones & cuilibet posset inuenire nomen proprium . \textbf{ Sed cum constat de re , } de verbis minime est curandum . Enumeratis ergo passionibus , \\\hline
1.3.1 & Et pues que assi es contadas las passiones \textbf{ puede se tomar el cuento dellas assi . } Ca las passiones propiamente non han de ser & de verbis minime est curandum . Enumeratis ergo passionibus , \textbf{ sic potest accipi earum numerus , } quia passiones proprie esse non habent nisi in appetitu sensitiuo . \\\hline
1.3.1 & Mas el appetito senssitiuo del seso assi commo mas largamente dixiemos de suso partese en apetito iraçibile \textbf{ que quiere dezir enssannador } e concupiçible que quiere dezir desseador . & in appetitu sensitiuo . Sensitiuus autem appetitus \textbf{ ( ut supra diffusius diximus ) diuiditur in irascibilem , } et concupiscibilem . \\\hline
1.3.1 & que quiere dezir enssannador \textbf{ e concupiçible que quiere dezir desseador . } Et por ende las sobredichas passiones & ( ut supra diffusius diximus ) diuiditur in irascibilem , \textbf{ et concupiscibilem . } Praedictae ergo passiones sic distinguuntur , \\\hline
1.3.1 & assi se departen . \textbf{ Ca las primeras seys conuiene saber } ¶ & Praedictae ergo passiones sic distinguuntur , \textbf{ quia primae sex videlicet , } amor , \\\hline
1.3.1 & por el appetito desseador \textbf{ assi se puede tomar } por que toda passion e todo mouimiento del alma & quia omnis passio \textbf{ et omnis motus animae pertinens } ad concupiscibilem , \\\hline
1.3.1 & Ca el bien conosçido primera mente nos plaze ¶ \textbf{ Lo segundo ymos a ello¶ } Lo terçero ganado aquel bien folgamos en el ¶ & Nam bonum apprehensum \textbf{ primo nobis placet , secundo tendimus in ipsum : tertio adepto quietamur in eo . } Prout ergo nobis bonum aliquod placet , \\\hline
1.3.1 & fincanos \textbf{ deuer en qual manera se han de tomar las passiones del appetito enssannador } Mas estas passiones han diferençia & Differunt autem hae passiones ab illis , \textbf{ quia passiones concupiscibiles respiciunt bonum } et malum absolute sumptum : \\\hline
1.3.1 & Ca las passiones del apetito cobdiçiador catan al bien o al mal \textbf{ en quanto es guaue de alcançar } Et pues que assi es estas tales passiones & sed passiones irascibiles respiciunt bonum \textbf{ vel malum in eo quod arduum . } Huiusmodi ergo passiones vel sumuntur respectu boni , \\\hline
1.3.1 & si non yra algun bien alto \textbf{ e gue de alcançar } Pot que cerca los bienes tomados sueltamente & ø \\\hline
1.3.1 & en razon de algun mal \textbf{ ta el mal puede ler tomado en dos maneras } O en quanto es futuro & Aliae vero passiones irascibiles sumuntur respectu mali . \textbf{ Nam malum dupliciter potest considerari , } vel ut futurum , \\\hline
1.3.1 & O en quanto es futuro \textbf{ e ha de uenir . } E en quanto es presente si fuere tomado & vel ut futurum , \textbf{ vel ut praesens . } Si consideretur \\\hline
1.3.1 & Mas si el mal fuere presente esto es avn en dos maneras . \textbf{ Ca en quanto por el nos leunatamos a fazer uengança } assi es saña . & hoc \textbf{ etiam est dupliciter : | quia vel ex hoc consurgimus ad faciendam vindictam , } et sic est ira : \\\hline
1.3.1 & ¶ Et pues que assi es conmo los nuestros mouimientos del alma \textbf{ et las nr̃as afectiones e passiones non se puedan departir } en mas maneras & Cum ergo non possint pluribus modis variari nostri motus \textbf{ et nostrae affectiones , } in uniuerso duodecim erunt passiones : \\\hline
1.3.1 & de las quales tractamos conplidamente en la rectorica . \textbf{ Mas aqui abastanos de tractar dellas superfiçialmente } orque niguno non puede bien gor̉inar assi mismo & De quibus omnibus in Rhetoricis diffusius diximus ; \textbf{ haec autem sufficiant superficialiter pertransire . } Quia nullus bene seipsum regere potest , \\\hline
1.3.2 & Mas aqui abastanos de tractar dellas superfiçialmente \textbf{ orque niguno non puede bien gor̉inar assi mismo } si non sopiere quals passiones son de fuyr & haec autem sufficiant superficialiter pertransire . \textbf{ Quia nullus bene seipsum regere potest , } nisi sciat quae passiones sunt fugiendae , \\\hline
1.3.2 & orque niguno non puede bien gor̉inar assi mismo \textbf{ si non sopiere quals passiones son de fuyr } e quales son de leguir & Quia nullus bene seipsum regere potest , \textbf{ nisi sciat quae passiones sunt fugiendae , } et quae prosequendae : \\\hline
1.3.2 & si non sopiere quals passiones son de fuyr \textbf{ e quales son de leguir } por ende en este libro primero & nisi sciat quae passiones sunt fugiendae , \textbf{ et quae prosequendae : } et quia in hoc primo libro determinare intendimus de regimine sui , \\\hline
1.3.2 & por ende en este libro primero \textbf{ en que entendemos de determinar del gouernamiento del omne en si mismo . } en quanto es omne conuiene de ueer & et quae prosequendae : \textbf{ et quia in hoc primo libro determinare intendimus de regimine sui , } videndum est quot sunt passiones , \\\hline
1.3.2 & en que entendemos de determinar del gouernamiento del omne en si mismo . \textbf{ en quanto es omne conuiene de ueer } quantas son las passiones & et quia in hoc primo libro determinare intendimus de regimine sui , \textbf{ videndum est quot sunt passiones , } et quem ordinem habent adinuicem , \\\hline
1.3.2 & e conosçida conoscremos \textbf{ qual sson de loar e quales son de denostar . } Et quales son de seguir & qua inspecta cognoscere possumus quae sunt laudabiles , \textbf{ et quae sunt vituperabiles , } et quae sunt sequendae , \\\hline
1.3.2 & qual sson de loar e quales son de denostar . \textbf{ Et quales son de seguir } e quales de fuyr ¶ & et quae sunt vituperabiles , \textbf{ et quae sunt sequendae , } et quae sunt fugiendae . Viso ergo quot sunt passiones , et quomodo accipitur earum numerus : \\\hline
1.3.2 & Et quales son de seguir \textbf{ e quales de fuyr ¶ } Et pues que assi es visto & et quae sunt vituperabiles , \textbf{ et quae sunt sequendae , } et quae sunt fugiendae . Viso ergo quot sunt passiones , et quomodo accipitur earum numerus : \\\hline
1.3.2 & e en qual manera se toma el cuento dellas . \textbf{ Conuiene de veer } que orden han entres si vnas aotras & et quae sunt fugiendae . Viso ergo quot sunt passiones , et quomodo accipitur earum numerus : \textbf{ videndum est } quem ordinem habeant ad se inuicem . \\\hline
1.3.2 & que orden han entres si vnas aotras \textbf{ mas la orden dellas puede se tomar en dos maneras . } O singu larmente cada vna por lli . & quem ordinem habeant ad se inuicem . \textbf{ Ordo autem earum dupliciter potest accipi : } vel singulariter , \\\hline
1.3.2 & pues que assi estomando esta orden tal segunt conbinaçion \textbf{ e ayuntamiento podemos dezir } que las primeras passiones son amor e mal querençia . & secundum combinationem , \textbf{ dicere possumus primas passiones esse , } amor , et odium . In secundo vero gradu sunt desiderium , et abominatio . In tertio vero , spes , \\\hline
1.3.2 & Ca quando amamos alguna cosa \textbf{ luego desseamos auer aquella cosa . } O si la ouieremos desseamos la de guardar en auiendo la . & cum enim amamus aliquid , \textbf{ uel desideramus ipsum habere , } uel si ipsum habemus , \\\hline
1.3.2 & luego desseamos auer aquella cosa . \textbf{ O si la ouieremos desseamos la de guardar en auiendo la . } Mas la aborrençia sin ningun medio se ayunta ala mal querençia & uel desideramus ipsum habere , \textbf{ uel si ipsum habemus , } desideramus conseruari in habendo ipsum . Abominatio uero immediate innititur odio : \\\hline
1.3.2 & en essa misma manera el desseo e la aborrençia son las segundas \textbf{ Maen el tercero logar son de poner la esperançar la desesꝑança . } Ca por que son tomadas & sic desiderium , \textbf{ et abominatio sunt passiones secundae . In tertio autem loco ponendae sunt spes , et desperatio . Nam spes , } et desperatio , cum sumantur respectu boni , praecedunt timorem , et audaciam , iram , \\\hline
1.3.2 & que sasanna e la manssedunbre . \textbf{ Ca commo algunan cosa primero sea futura de uenir } ante que sea presente . & et mansuetudinem . \textbf{ Nam cum aliquid prius sit futurum , } quam praesens : \\\hline
1.3.2 & que son tomadas por razon de mal \textbf{ que a de venir son primeras que la sanna e la manssedunbre } que son tomadas & quae sumuntur respectu mali futuri , \textbf{ praecedunt iram , | et mansuetudinem , } quae sumuntur respectu mali praesentis . \\\hline
1.3.2 & e apartadamente cada vna \textbf{ por si deuen se assi ordenar dizienda } que el amor es primero que la mal quetençia ¶ & si sumantur binae et binae . Acceptae uero singulariter , \textbf{ sic ordinari debent : } quia amor est prior odio : \\\hline
1.3.2 & Mas si fuy del mal esto es despues \textbf{ en quanto fuyr de mal a razon de bien ¶ } Et pues que assi es el amor es en todo en todo el primero mouimiento & hoc est ex consequenti , \textbf{ inquantum fugere malum , habet rationem boni . } Amor ergo est omnino primus motus , \\\hline
1.3.2 & que todas las otraspassiones . \textbf{ Et por que yr al bien nos ayunta } mas al bien que fallesçer del bien & praecedit alias passiones : \textbf{ et quia tendere in bonum est magis coniungi bono } quam deficere ab ipso , \\\hline
1.3.2 & Et por que yr al bien nos ayunta \textbf{ mas al bien que fallesçer del bien } por ende la esꝑança & praecedit alias passiones : \textbf{ et quia tendere in bonum est magis coniungi bono } quam deficere ab ipso , \\\hline
1.3.2 & por si \textbf{ e primeramente va ayuntar se al bien . } por ende la passion que va al bien es primera & Nam sicut quia appetitus per se \textbf{ et primo intendit coniungi bono , } ideo passio quae tendit in bonum est prior passione \\\hline
1.3.2 & que la passion que fallesce del bien En essa misma guisa \textbf{ por que fuyr del mal ha razon de bien } por ende el temor & quae deficit ab ipso : \textbf{ sic quia refugere malum habet rationem boni , } ideo timor \\\hline
1.3.2 & por que la mansedunbre propriamente non es pasion \textbf{ mas es fallesçer de passion . } Por ende en el linage delas passiones & Nam mansuetudo proprie non est passio , \textbf{ sed magis est deficere a passione , } ideo in genere passionum ira mansuetudinem praecedit . \\\hline
1.3.2 & que la saña \textbf{ esto non lo auemos de escrudinar aqui ¶ } Otrosi la delectaçion & secundum aliquem alium modum mansuetudo praecedat iram , \textbf{ inuestigare non est praesentis negocii . Delectatio autem , } quae est respectu boni , \\\hline
1.3.2 & en alguna manera se nos demuestra la nataleza dellas superfiçialmente \textbf{ e en figua a la qual naturaleza conosçida poremos conosçer } en qual manera auemos de segiuir & ex his quae dicta sunt aliquo modo figuraliter et typo innotescit nobis natura ipsarum : \textbf{ qua cognita , } cognoscere possumus quomodo sint prosequendae , \\\hline
1.3.2 & e en figua a la qual naturaleza conosçida poremos conosçer \textbf{ en qual manera auemos de segiuir } e de esquiuar las pasiones sobredichas ¶ & qua cognita , \textbf{ cognoscere possumus quomodo sint prosequendae , } et quomodo vitandae passiones praedictae . \\\hline
1.3.2 & en qual manera auemos de segiuir \textbf{ e de esquiuar las pasiones sobredichas ¶ } La qual cosa sobre tanto & cognoscere possumus quomodo sint prosequendae , \textbf{ et quomodo vitandae passiones praedictae . } Quod scire tanto magis decet Reges et Principes , \\\hline
1.3.2 & mas parte nesçe alos Reyes e alos prinçipes \textbf{ en quanto por las passiones dellos mayor mal puede venir } o mayer bien & Quod scire tanto magis decet Reges et Principes , \textbf{ quanto per passiones ipsorum maius valet induci malum , } et potest bonum excellentius impediri . \\\hline
1.3.2 & en quanto por las passiones dellos mayor mal puede venir \textbf{ o mayer bien } e mas granado se puede enbargar . & quanto per passiones ipsorum maius valet induci malum , \textbf{ et potest bonum excellentius impediri . } De hoc tamen infra diffusius tractabitur . \\\hline
1.3.2 & o mayer bien \textbf{ e mas granado se puede enbargar . } Empero desto mas conꝑlidamente tractaremos adelante & quanto per passiones ipsorum maius valet induci malum , \textbf{ et potest bonum excellentius impediri . } De hoc tamen infra diffusius tractabitur . \\\hline
1.3.3 & e en lanr̃a uida \textbf{ por ende escoła neçesaria de mostrar } en qual manera nos deuemos auer a aquellas passiones & Passiones autem quia diuersificant regnum et vitam nostram , \textbf{ ideo necessarium est ostendere } quomodo nos habere debeamus ad illas . \\\hline
1.3.3 & por ende escoła neçesaria de mostrar \textbf{ en qual manera nos deuemos auer a aquellas passiones } Et por ende conuena de contar tondas las passiones & ideo necessarium est ostendere \textbf{ quomodo nos habere debeamus ad illas . } Oportebat ergo enumerare omnes passiones , \\\hline
1.3.3 & en qual manera nos deuemos auer a aquellas passiones \textbf{ Et por ende conuena de contar tondas las passiones } por que sopiessemos el cuento dellas & quomodo nos habere debeamus ad illas . \textbf{ Oportebat ergo enumerare omnes passiones , } ut sciremus numerum passionum , \\\hline
1.3.3 & por que sopiessemos el cuento dellas \textbf{ delas quales auemos de determinar } e de dezir . & ut sciremus numerum passionum , \textbf{ de quibus determinare debemus . } Oportebat \\\hline
1.3.3 & delas quales auemos de determinar \textbf{ e de dezir . } ¶ Otrosi conuenia avn demostrar la orden dellas & ut sciremus numerum passionum , \textbf{ de quibus determinare debemus . } Oportebat \\\hline
1.3.3 & e de dezir . \textbf{ ¶ Otrosi conuenia avn demostrar la orden dellas } por que sopiessemos & de quibus determinare debemus . \textbf{ Oportebat | etiam ostendere ordinem earum , } ut sciremus quo ordine determinaremus de illis . \\\hline
1.3.3 & Por la qual cosa commo el amor e la mal querençia sean las primeras passiones \textbf{ primero deuemos ver } en qual manera conuiene alos Reyes & et odium sint passiones primae , \textbf{ prius videndum est , } quomodo deceat Reges et Principes esse amatiuos , \\\hline
1.3.3 & e en el apetito intellectiuo \textbf{ que es la uoluntad podemos dezir } que la razon del amor es sienpre algun bien . & sed etiam ut reperitur in appetitu sensitiuo et intellectiuo : \textbf{ dicere possumus quod semper obiectum amoris est bonum . } Ubi ergo reperitur \\\hline
1.3.3 & e perdido dios \textbf{ donde el quisiese lo podria refazer } por la qual cosa ningun omne & et si annihilatum esset bonum nostrum , \textbf{ Deus unde vellet , posset illud reficere . } Quare cum nullus homo sine diuino auxilio possit seipsum bonum facere , \\\hline
1.3.3 & sin ayuda de dios non pue da \textbf{ assi mismo fazer bueno o guardar } assymismo en bondat . La razon natural muestra & Quare cum nullus homo sine diuino auxilio possit seipsum bonum facere , \textbf{ vel se in bonitate conseruare , dictat } naturalis ratio \\\hline
1.3.3 & assymismo en bondat . La razon natural muestra \textbf{ que mas deue amar el omne el bien diuinal } que assi mismo . & naturalis ratio \textbf{ ut magit diligat Deum quam seipsum : } quia bonum uniuscuiusque principaliter est a Deo , \\\hline
1.3.3 & Et otrosi por que el bien comun es ençerrado el bien propio de cada vno \textbf{ sienpredeuemos ante poner el bien comun } e dela comunidat al bien propio e personal de cada vno . & ut dicitur 1 Ethic’ \textbf{ et quia in communi bono includitur } bonum priuatum , \\\hline
1.3.3 & esto era \textbf{ por que los çibdadanos non duda una de se poner ala muerte } por el bien comun de todos . & hoc erat , \textbf{ quia ciues pro Republica non dubitabant se morti exponere . } Dilectatio enim quam habebant Romani \\\hline
1.3.3 & e publicofizo a Roma ser sennora \textbf{ e auer sennorio en todo el mundo . } Pues que assi es & ad Rempublicam fecit Romam esse principantem \textbf{ et monarcham . } Hoc ergo modo \\\hline
1.3.3 & Et commo quier que esto conuiene a todos los omes Empero mucho mas conuiene alos Reyes e a los prinçipes \textbf{ la qual cosa podemos declarar } por tres maneras & tamen hoc decet Reges \textbf{ et Principes , quod triplici via declarare possumus . } Regi enim dignitas \\\hline
1.3.3 & quanto parte nesçe alo presente puede ser conparada a tres cosas . \textbf{ Conuiene saber ala tirania . } del tirano & ( quantum ad praesens ) \textbf{ ad tria comparari potest scilicet ad tyrannidem , } cui contrariatur : \\\hline
1.3.3 & e alos males e pecados \textbf{ de que deue fuyr la real magestad } ¶la primera razon se praeua assi . & et ad vitia , \textbf{ quae debet fugere . Prima via sic patet : } nam \\\hline
1.3.3 & Mas si los Reyes e los tyranos se han en manera contraria \textbf{ por que la manera del amor del tirano es ante poner } e preçiar mas el bien propio & et tyrannides : \textbf{ cum modus amoris tyrannici sit } ut bonum priuatum praeponat bono communi , \\\hline
1.3.3 & por que la manera del amor del tirano es ante poner \textbf{ e preçiar mas el bien propio } que el bien comun . & cum modus amoris tyrannici sit \textbf{ ut bonum priuatum praeponat bono communi , } modus amoris regis esse debet \\\hline
1.3.3 & e ꝑson a publica e comun . \textbf{ espeçialmente ꝑtenesçe alos Reyes e alos prinçipes de ante poner e preçiar } mas el bien diuianl & et communis , \textbf{ speciali modo spectat ad Reges et Principes bonum diuinum } et commune praeponere cuilibet priuato bono . Secundo hoc idem patet , \\\hline
1.3.3 & por las quales conuiene alos reyes de ser honrrados . \textbf{ Ca assi commo es cosamas de denostar en el maestro } que non aya sçiençia & si considerentur virtutes , quibus decet Reges esse ornatos . \textbf{ Sicut enim detestabilius est in magistro carere scientia } quam in discipulo , \\\hline
1.3.3 & por que el ma estro es en estado \textbf{ en que deue dar sçiençia alos otros . } En essa misma manera es mas de denostar el Rey en fallesçer en . & quam in discipulo , \textbf{ quia magister est in statu in quo ipso debet scientiam aliis tradere : } sic detestabilius est in Rege carere virtutibus , \\\hline
1.3.3 & en que deue dar sçiençia alos otros . \textbf{ En essa misma manera es mas de denostar el Rey en fallesçer en . } las uirtudes que los subditos & quia magister est in statu in quo ipso debet scientiam aliis tradere : \textbf{ sic detestabilius est in Rege carere virtutibus , } quam in subditis , \\\hline
1.3.3 & que pueden los Reyes \textbf{ e los prinçipes aduzir a uirtudes es que amen prinçipalmente el bien diuianl e el bien comun } por que si el Rey prinçipallmente entendiere & quae inducere possent alios ad virtutes , \textbf{ est , | ut bonum diuinum et commune principaliter diligant . } Nam si Rex principaliter bonum commune intendat , \\\hline
1.3.3 & por que aya memoria delas cosas passadas \textbf{ e prouidençia delas cosas que han de venir } por que sea prouado e aꝑçebido & studebit \textbf{ ut habeat memoriam praeteritorum , et prouidentiam futurorum , } ut sit expertus , cautus , \\\hline
1.3.3 & que son meester ala pradençia e ala sabiduria . \textbf{ por las quales pue da meior gouernar su pueblo . } Mas si ante pusiere & et ut habeat omnia quae ad prudentiam requiruntur , \textbf{ per quam possit melius suum populum regere . } Immo si bonum commune praeponat bono priuato , \\\hline
1.3.3 & quanto mayor pradençia e mayor sabiduria es meester \textbf{ para guardar el bien comun } que el bien propreo . & ut prudentia polleat , \textbf{ quanto maior prudentia requiritur ad custodiendum bonum commune , } quam proprium . \\\hline
1.3.3 & Otrosi sera fuerte por que ante pone el bien comunal bien propreo \textbf{ e avn non dubdara de poner la persona a muerte } siuiere que sea cosa & et communia . Erit fortis ; quia cum bonum cumune proponat bono priuato , \textbf{ non dubitabit | etiam personam exponere , } si viderit quod expediat regno . \\\hline
1.3.3 & e avn sera tenprado \textbf{ por que si la entencion suya prinçipal fuere en trabaiar en el bien del regno } despreçiar a las delectaçiones destenpradas de los sesos & Erit temperatus ; \textbf{ quia si intentio sua principaliter versetur circa bonum regni , } spernet delectationes sensibiles immoderatas , \\\hline
1.3.3 & por que si la entencion suya prinçipal fuere en trabaiar en el bien del regno \textbf{ despreçiar a las delectaçiones destenpradas de los sesos } por que por ellas non se pueda enbargar la cura conuenible del regno . & quia si intentio sua principaliter versetur circa bonum regni , \textbf{ spernet delectationes sensibiles immoderatas , } ne per eas impediatur debita cura regni . \\\hline
1.3.3 & despreçiar a las delectaçiones destenpradas de los sesos \textbf{ por que por ellas non se pueda enbargar la cura conuenible del regno . } Et pues que assi es & spernet delectationes sensibiles immoderatas , \textbf{ ne per eas impediatur debita cura regni . } Ut ergo sit ad unum dicere , \\\hline
1.3.3 & Et pues que assi es \textbf{ por que lo podamos todo traer en vna sentençia el amor del bien diuinal } e del comunnos & Ut ergo sit ad unum dicere , \textbf{ amor diuini boni } et communis inductiuus est ad virtutes singulas . Considerando ergo virtutes , \\\hline
1.3.3 & por las quales deuen ser los Reyes honrrados \textbf{ prinçipalmente deuen ellos amar el bien diuinal } e el bien comun¶ & et Principes esse ornatos , \textbf{ et principaliter debent diligere bonum diuinum et commune . Tertio hoc idem patet , } si considerentur vitia , quae debent fugere . \\\hline
1.3.3 & si penssaremos los males e los pecados \textbf{ que deuen los Reyes foyr } por que assi conmo el amor diuinal & et principaliter debent diligere bonum diuinum et commune . Tertio hoc idem patet , \textbf{ si considerentur vitia , quae debent fugere . } Nam sicuti amor diuinus \\\hline
1.3.3 & Et tales commo estos son los tiranos \textbf{ que quieren conplir su uoluntad proprea } e demandan grandia singular de su persona & tales enim sunt tyranni , \textbf{ volentes explere voluptatem propriam , } et quaerentes excellentiam singularem : \\\hline
1.3.3 & visto en qual manera los Reyes e los prinçipes se de una auer al amor \textbf{ Ca nal e comun de ligero puede paresçer } en qual manera se de una auer los Reyes ala mal querençia . & quia principaliter debent amare bonum diuinum et commune : \textbf{ de facili patere potest , } quomodo se habere debeant ad odium . \\\hline
1.3.3 & Et pues que assi es la prinçipal entençion de cada vno deue ser \textbf{ que cosa ha de amar } Et por ende mostrado & omnis passio siue omnis motus animi ex amore sumit originem ; \textbf{ potissimum ergo in intentione cuiuslibet esse debet } quid amandum . Ostenso ergo quomodo Reges et Principes quodam speciali modo \\\hline
1.3.3 & que los Reyes et los prinçipes \textbf{ por alguna manera especial sobre todos los otros deuen amar el bien diuinal } e el bien comunal & quid amandum . Ostenso ergo quomodo Reges et Principes quodam speciali modo \textbf{ prae aliis debent diligere bonum diuinum et commune , } et quodam speciali modo \\\hline
1.3.3 & e el bien comunal \textbf{ en alguna manera espeçial sobre todos los otros deuen aborresçer todas aquellas cosas } que son contrarias al bien diuinal e comunal . & prae aliis debent diligere bonum diuinum et commune , \textbf{ et quodam speciali modo | prae alios odire debent } quae contrariantur bono diuino \\\hline
1.3.3 & Et generalmente todos males e todos pecados \textbf{ por la qual cosa commo de razon dela mal querençia sea matar } e nunca se fartar & et uniuersaliter omnia vitia . \textbf{ Quare cum de ratione odii sit exterminare , } et nunquam satiari nisi exterminet , \\\hline
1.3.3 & por la qual cosa commo de razon dela mal querençia sea matar \textbf{ e nunca se fartar } si non de matar assi commo dize el philosofo en el segundo libro dela rectorica . & Quare cum de ratione odii sit exterminare , \textbf{ et nunquam satiari nisi exterminet , } ut dicitur 2 Rhetoricorum , \\\hline
1.3.3 & e nunca se fartar \textbf{ si non de matar assi commo dize el philosofo en el segundo libro dela rectorica . } Conuiene alos Reyes e alos prinçipes & et nunquam satiari nisi exterminet , \textbf{ ut dicitur 2 Rhetoricorum , } decet Reges et Principes amare Iustitiam , \\\hline
1.3.3 & Conuiene alos Reyes e alos prinçipes \textbf{ assi saber amar iustiçia } e aborresçer todos los pecados & ut dicitur 2 Rhetoricorum , \textbf{ decet Reges et Principes amare Iustitiam , } et odit vitia , \\\hline
1.3.3 & assi saber amar iustiçia \textbf{ e aborresçer todos los pecados } que non se farten & decet Reges et Principes amare Iustitiam , \textbf{ et odit vitia , } ut non satientur , \\\hline
1.3.3 & Ca los omes \textbf{ por si non son de destroyr } njn de aborresçer & nisi ea extirpent \textbf{ et exterminent . Per se enim homines non sunt exterminandi , } et odiendi : \\\hline
1.3.3 & por si non son de destroyr \textbf{ njn de aborresçer } mas por que las malas obras de petado son de destroyr e de aborresçer & et exterminent . Per se enim homines non sunt exterminandi , \textbf{ et odiendi : } sed \\\hline
1.3.3 & njn de aborresçer \textbf{ mas por que las malas obras de petado son de destroyr e de aborresçer } si por auentra a non pueden en otra manera destroyr los males & et odiendi : \textbf{ sed | quia vitia sunt extirpanda et odienda } si non possunt aliter vitia extirpari , nec potest aliter durare commune bonum , \\\hline
1.3.3 & mas por que las malas obras de petado son de destroyr e de aborresçer \textbf{ si por auentra a non pueden en otra manera destroyr los males } nin puede en otra manera durar el bien comun & quia vitia sunt extirpanda et odienda \textbf{ si non possunt aliter vitia extirpari , nec potest aliter durare commune bonum , } nisi exterminando maleficos homines , \\\hline
1.3.3 & si por auentra a non pueden en otra manera destroyr los males \textbf{ nin puede en otra manera durar el bien comun } si non destruiendo e matando los omes malos & quia vitia sunt extirpanda et odienda \textbf{ si non possunt aliter vitia extirpari , nec potest aliter durare commune bonum , } nisi exterminando maleficos homines , \\\hline
1.3.3 & que fazen mal . \textbf{ Por ende tales son de destroyr e de matar } por que non peres prinçipalmente & nisi exterminando maleficos homines , \textbf{ extirpandi sunt tales , } ne pereat commune bonum . \\\hline
1.3.3 & por que non peres prinçipalmente \textbf{ deuen amar el bien } diuica el bien comun . & extirpandi sunt tales , \textbf{ ne pereat commune bonum . } Amare ergo commune bonum , \\\hline
1.3.3 & diuica el bien comun . \textbf{ Et pues que assi es amar el bien comun } e querer mal a las colas malfechas & ne pereat commune bonum . \textbf{ Amare ergo commune bonum , } et odire malefica , \\\hline
1.3.3 & Et pues que assi es amar el bien comun \textbf{ e querer mal a las colas malfechas } que son contrarias al bien comun & Amare ergo commune bonum , \textbf{ et odire malefica , } quae ei contrariantur : \\\hline
1.3.4 & icho del amor e dela mal querençia \textbf{ que son las primeras passiones finca de dezir } en qual manera los Reyes et los pnçipes se deue auer al desseo e ala aborrençia & et odio , \textbf{ quae sunt passiones primae : | dicere restat , } quomodo Reges et Principes se habere debeant ad desiderium , \\\hline
1.3.4 & que son las primeras passiones finca de dezir \textbf{ en qual manera los Reyes et los pnçipes se deue auer al desseo e ala aborrençia } que son las passiones segundas & dicere restat , \textbf{ quomodo Reges et Principes se habere debeant ad desiderium , | et abominationem , } quae sunt passiones secundae . \\\hline
1.3.4 & que son las passiones segundas \textbf{ Et deuedes saber } que ha diferençia e departimiento entre el desseo e el amor & et abominationem , \textbf{ quae sunt passiones secundae . } Differt autem desiderium ab amore : \\\hline
1.3.4 & qual es a el conueniente e proportionado . \textbf{ Mas en los cuerpos pesados e liuianos deuemos pensar tres cosas ¶ } Lo primero la forma del cuerpo pesado o liuiano & sic per amorem tendit quis in bonum sibi proportionatum \textbf{ et conueniens . In grauibus ergo } et leuibus est tria considerare . Primo formam grauis vel leuis , per quam conformatur loco sursum vel deorsum . \\\hline
1.3.4 & por la qual cosa es conformado al su logar de yuso o de suso ¶ \textbf{ Lo segundo es hy de penssar el mouimiento } por el qual van a aquel lugar ¶ & et leuibus est tria considerare . Primo formam grauis vel leuis , per quam conformatur loco sursum vel deorsum . \textbf{ Secundo est ibi considerare motum , } per quem tendunt in talem locum . \\\hline
1.3.4 & En essa misma manera en las obras morales \textbf{ assi commo todos dizen comunalmente deuemos penssar estas tres cosas . } Ca quando aprendemos algun bien primeramente & Sic et in gestis moralibus , \textbf{ ut communiter ponitur , | est tria considerare . } Nam cum bonum aliquod apprehendimus , \\\hline
1.3.4 & nos mueue ael \textbf{ e la delectaçion nos faze folgar en el . } Et esto que dicho es del bien en conparaçion del amor & desiderium nos mouet : \textbf{ et delectatio nos quietat . } Et quod dictum est de bono respectu amoris , desiderii , et delectationis , \\\hline
1.3.4 & assi commo dessuso dixiemos . \textbf{ assi de uemos entender del mal en conparaçion de la mal querençia } e dela aborrençia e dela tristeza & ut supra tangebatur , \textbf{ intelligendum est de malo respectu odii , } abominationis , \\\hline
1.3.4 & quier que non sea vna cosa con el amor \textbf{ enpero deue resçebir medida e manera del amor . } En essa misma manera commo & et si contingat ipsum adipisci , \textbf{ dolemus et tristamur . Desiderium ergo licet non sit idem quod amor , mensuram tamen et modum debet suscipere ex amore . Sic } et abominatio licet non sit idem quod odium , \\\hline
1.3.4 & que la mal querençia . \textbf{ Empero deue resçebir medida e mesura dela mal querençia . } Ca assi commo en las cosas naturales beemos & et abominatio licet non sit idem quod odium , \textbf{ tamen ex odio debet mensuram suscipere . } Sic enim in naturalibus videmus , \\\hline
1.3.4 & tanto mas ligeramente se mueuen ayusoo ala tr̃ra . \textbf{ Por la qual cosa si queremos ver } en qual manera los Reyes e los prinçipes & tanto velocius mouentur ad medium . \textbf{ Quare si videre volumus , } quomodo Reges et Principes debeant aliqua desiderare et abominari , \\\hline
1.3.4 & en qual manera los Reyes e los prinçipes \textbf{ de una alguna cosa dessear } e aborresçer deuemos ver & Quare si videre volumus , \textbf{ quomodo Reges et Principes debeant aliqua desiderare et abominari , } videndum est quomodo debeant amare \\\hline
1.3.4 & de una alguna cosa dessear \textbf{ e aborresçer deuemos ver } en qual manera de una amar et mal querer . & quomodo Reges et Principes debeant aliqua desiderare et abominari , \textbf{ videndum est quomodo debeant amare } et odire . Videmus autem in singulis artibus , \\\hline
1.3.4 & e aborresçer deuemos ver \textbf{ en qual manera de una amar et mal querer . } Ca nos veemos en cada vna delas obras & quomodo Reges et Principes debeant aliqua desiderare et abominari , \textbf{ videndum est quomodo debeant amare } et odire . Videmus autem in singulis artibus , \\\hline
1.3.4 & Mas aquellas cosas que van ala fin son desseadas segunt la manera e la mesura de aquella fin \textbf{ assi commo el fisico entiende enduzir } quanto puede sanidat en el enfermo la mayor e meior que pudiere & secundum modum et mensuram ipsius finis : \textbf{ ut medicus intendit inducere sanitatem , } quantum potest , \\\hline
1.3.4 & Et por ende el fisico non entiende \textbf{ quanto mayor melezina o sanguaa puede dar } ca luego matarie al entermo . & potionem autem et phlebotomiam non intendit \textbf{ quanto maiorem potest , } quia tunc exterminaret infirmum : \\\hline
1.3.4 & ca luego matarie al entermo . \textbf{ mas entiende le de dar estas cosas } segunt medida e manera de salud . & quia tunc exterminaret infirmum : \textbf{ sed intendit | secundum modum } et mensuram sanitatis . \\\hline
1.3.4 & segunt medida e manera de salud . \textbf{ Et pues que assi es commo en la arte del regnar } e de enssennorear prinçipalmente & et mensuram sanitatis . \textbf{ Cum ergo in arte regnandi et principandi principaliter } et finaliter intendatur salus Regni et Principatus , \\\hline
1.3.4 & Et pues que assi es commo en la arte del regnar \textbf{ e de enssennorear prinçipalmente } e finalmente se entiende la salud del Regno e del prinçipado & et mensuram sanitatis . \textbf{ Cum ergo in arte regnandi et principandi principaliter } et finaliter intendatur salus Regni et Principatus , \\\hline
1.3.4 & Conuiene alos Reyes \textbf{ e alos prinçipes entender e amar prinçipalmente } el bien del regno & sicut in arte medicandi principaliter intenditur sanitas corporis : naturaliter decet Reges \textbf{ et Principes intendere } et amare bonum regni et commune . \\\hline
1.3.4 & Por la qual cosa \textbf{ si el desseo deue tomar mesura del amor } Conuiene que los Reyes e los prinçipes desse en prinçipalmente el buen estado del regno & Quare \textbf{ si desiderium debet mensuram sumere ex amore , } principaliter Reges et Principes debent desiderare bonum statum regni : \\\hline
1.3.4 & e por si et essençialmente nasçe el buen estado del regno . \textbf{ Mas las otras cosas deuen los reyes dessear } en quanto han orden a estas ¶ & a quibus per se et essentialiter dependet bonus status regni . \textbf{ Alia autem desiderare debent , } ut habent ordinem ad istas diuitias ergo , \\\hline
1.3.4 & e todos los otros dales bienes \textbf{ commo estos deuen los Reyes dessear en tanto en } quanto por ellos pueden apremiar los malos & et ciuilem potentiam , \textbf{ et caetera talia bona in tantum desiderare debent , } inquantum per ea possunt cohercere malos , punire iniusta , et facere talia , \\\hline
1.3.4 & commo estos deuen los Reyes dessear en tanto en \textbf{ quanto por ellos pueden apremiar los malos } e dar las penas & et ciuilem potentiam , \textbf{ et caetera talia bona in tantum desiderare debent , } inquantum per ea possunt cohercere malos , punire iniusta , et facere talia , \\\hline
1.3.4 & quanto por ellos pueden apremiar los malos \textbf{ e dar las penas } por las cosas desiguales e malas . & ø \\\hline
1.3.4 & por las cosas desiguales e malas . \textbf{ Et fazer o tris cosas tales delas quales nasçe } e cuelga la salud del regno ¶ & et caetera talia bona in tantum desiderare debent , \textbf{ inquantum per ea possunt cohercere malos , punire iniusta , et facere talia , } a quibus regni salus dependere videtur . Viso , quae , \\\hline
1.3.4 & e cuelga la salud del regno ¶ \textbf{ Visto quales cosas deuen los Reyes dessear } e en qual manera & inquantum per ea possunt cohercere malos , punire iniusta , et facere talia , \textbf{ a quibus regni salus dependere videtur . Viso , quae , } et quomodo Reges et Principes desiderare debent , \\\hline
1.3.4 & e en qual manera \textbf{ et quales cosas son las que deuen amar . } Ca deuen primeramente & a quibus regni salus dependere videtur . Viso , quae , \textbf{ et quomodo Reges et Principes desiderare debent , } quia sicut debent amare primo et principaliter bonum diuinum et commune , \\\hline
1.3.4 & Ca deuen primeramente \textbf{ e prinçipalmente amar el bien diuinal } e el bien comun delas gentes . & et quomodo Reges et Principes desiderare debent , \textbf{ quia sicut debent amare primo et principaliter bonum diuinum et commune , } alia autem sunt desideranda \\\hline
1.3.4 & e el bien comun delas gentes . \textbf{ Mas las otras cosas deuen amar } en quanto son ordenadas a estas . & quia sicut debent amare primo et principaliter bonum diuinum et commune , \textbf{ alia autem sunt desideranda } ut ordinantur ad ista : \\\hline
1.3.4 & Et essa misma manera primeramente \textbf{ e prinçipalmente deuen dessear el bien diuinal } e comunal . & sic desiderare debent primo \textbf{ et principaliter bonum diuinum et commune , } alia autem sunt desideranda ut ordinantur ad ista : \\\hline
1.3.4 & e comunal . \textbf{ mas las o tris cosas deuen dessear } en quanto son ordenadas a estas ¶ & et principaliter bonum diuinum et commune , \textbf{ alia autem sunt desideranda ut ordinantur ad ista : } de leui patere potest quae \\\hline
1.3.4 & en quanto son ordenadas a estas ¶ \textbf{ Et pues que assi es de ligero puede paresçer } quales cosas deuen aborresçer los prinçipes & alia autem sunt desideranda ut ordinantur ad ista : \textbf{ de leui patere potest quae } et qualia abominari debent , \\\hline
1.3.4 & Et pues que assi es de ligero puede paresçer \textbf{ quales cosas deuen aborresçer los prinçipes } e en qual manera Ca prinçipalmente son de aborresçer & de leui patere potest quae \textbf{ et qualia abominari debent , } quia principaliter sunt abominanda quae expressus contradicunt bono diuino \\\hline
1.3.4 & quales cosas deuen aborresçer los prinçipes \textbf{ e en qual manera Ca prinçipalmente son de aborresçer } a qual las cosas que espiessamente contradizen el bien diuinal e comun . & de leui patere potest quae \textbf{ et qualia abominari debent , } quia principaliter sunt abominanda quae expressus contradicunt bono diuino \\\hline
1.3.4 & a qual las cosas que espiessamente contradizen el bien diuinal e comun . \textbf{ Mas las otras son de aborresçer del pues destas mas esta tal aborrençia } e desseo tanto & et communi , \textbf{ alia autem abominanda sunt ex consequenti . Huiusmodi autem abominationem } et desiderium \\\hline
1.3.4 & mas conuiene a ellos \textbf{ de auer cuydado del regno e del bien comun . } Mas quales cosas son aquellas que guardan el regno en buen estado & tanto magis decet Reges et Principes , quanto magis eos decet habere curam de bono regni \textbf{ et communi . } Quae sunt autem illa quae regnum conseruant in bono statu , \\\hline
1.3.4 & Mas quales cosas son aquellas que guardan el regno en buen estado \textbf{ e en qual manera el Rey se deue auer a su regno } en el terçero libro lo mostraremos mas conplidamente & Quae sunt autem illa quae regnum conseruant in bono statu , \textbf{ et quomodo Rex se debeat habere ad ipsum regnum , } in tertio libro diffusius ostendetur . \\\hline
1.3.5 & Et en pos estas en el tercero \textbf{ lo guar eran de assentar la esperança e la desesꝑança . } Et pues que assi os finca de veer & post has autem tertio loco collocandae erant spes , \textbf{ et desperatio . } Restat vero videre quomodo Reges \\\hline
1.3.5 & lo guar eran de assentar la esperança e la desesꝑança . \textbf{ Et pues que assi os finca de veer } en qual manera los Reyes e los prinçipes se deuen auer cerca la esperança & et desperatio . \textbf{ Restat vero videre quomodo Reges } et Principes se habere debeant circa spem \\\hline
1.3.5 & Et pues que assi os finca de veer \textbf{ en qual manera los Reyes e los prinçipes se deuen auer cerca la esperança } e cerca la desesperança & Restat vero videre quomodo Reges \textbf{ et Principes se habere debeant circa spem } et desperationem . \\\hline
1.3.5 & Mas si fuesen bien entendidos los dichos de suso \textbf{ non seria guaue de saber } en qual manera los Reyes e los prinçipes se deuen auer ala esperança e ala desesꝑança . & Sed si considerentur dicta praehabita , \textbf{ non est difficile scire , } quomodo Reges et Principes \\\hline
1.3.5 & non seria guaue de saber \textbf{ en qual manera los Reyes e los prinçipes se deuen auer ala esperança e ala desesꝑança . } Ca dixiemos de suso & non est difficile scire , \textbf{ quomodo Reges et Principes | ad hoc se habere debeant . Dicebatur enim supra , } quod eos esse decet humiles \\\hline
1.3.5 & ca los humildosos conosçiendo su propre o fallesçemiento \textbf{ non esperan mas de aquello que deuen esparar . } Mas la magnanimidat repreeme la desesperança & cum ergo humilitas moderet spem , \textbf{ quia humiles cognoscentes defectum proprium , non sperant ultra quam debeant : magnanimitas vero reprimat desperationem , } quia magnanimi propter difficultatem operis non desperant : \\\hline
1.3.5 & e magranimos esperan las cosas \textbf{ que deuen espar } por la magninimidat & si Reges et Principes fuerint humiles \textbf{ et magnanimi , } superabunt speranda propter magnanimitatem , \\\hline
1.3.5 & que ha en ellos \textbf{ e non esperan aquellas cosas que non deuen esparar } por la humildat que ha en ellos . & superabunt speranda propter magnanimitatem , \textbf{ et non sperabunt non speranda propter humilitatem . Possumus autem quadrupliciter ostendere , } quod deces Reges et Principes decenter se habere circa spem , \\\hline
1.3.5 & por la humildat que ha en ellos . \textbf{ Mas nos podemos mostrar en quatro maneras } que conuiene alos Reyes e alos prinçipes & et non sperabunt non speranda propter humilitatem . Possumus autem quadrupliciter ostendere , \textbf{ quod deces Reges et Principes decenter se habere circa spem , } et sperare speranda , \\\hline
1.3.5 & que conuiene alos Reyes e alos prinçipes \textbf{ de se auer conueniblemente cerca la esperança } e esparar las cosas & quod deces Reges et Principes decenter se habere circa spem , \textbf{ et sperare speranda , } et aggredi aggredienda . \\\hline
1.3.5 & de se auer conueniblemente cerca la esperança \textbf{ e esparar las cosas } que son es paradas & quod deces Reges et Principes decenter se habere circa spem , \textbf{ et sperare speranda , } et aggredi aggredienda . \\\hline
1.3.5 & que son es paradas \textbf{ e acometer las cosas que son acometidas . } Ca si los Reyes non esparasen ninguna cola & et sperare speranda , \textbf{ et aggredi aggredienda . } Nam si Reges nihil sperarent , \\\hline
1.3.5 & assi commo dizen todos \textbf{ comunalmente son quatro cosas de penssar } por las quales podemos mostrar que conuiene alos Reyes & et non debite pertractarent negocia regni . Sunt autem in spe , \textbf{ ut communiter ponitur , quatuor consideranda , } propter quae arguere possumus , quod decet Reges et Principes esse bene sperantes . \\\hline
1.3.5 & comunalmente son quatro cosas de penssar \textbf{ por las quales podemos mostrar que conuiene alos Reyes } e alos prinçipes de ser bien esperantes ¶ & ut communiter ponitur , quatuor consideranda , \textbf{ propter quae arguere possumus , quod decet Reges et Principes esse bene sperantes . } Spes enim primo est de bono : \\\hline
1.3.5 & si non cerca de bien alto \textbf{ e guaue de alcançar . } Ca ninguno non es & nisi circa bonum arduum ; \textbf{ nullus enim sperare dicitur , } nisi sibi videatur esse bonum arduum , \\\hline
1.3.5 & para si non bien alto \textbf{ e guaue de alcançar } ¶lo terçero la esperança ha de ser cerca el bien futuro que ha de uenir . & nisi sibi videatur esse bonum arduum , \textbf{ et difficile . } Tertio spes habet esse circa bonum futurum : \\\hline
1.3.5 & e guaue de alcançar \textbf{ ¶lo terçero la esperança ha de ser cerca el bien futuro que ha de uenir . } Por que de los bienes presentes & et difficile . \textbf{ Tertio spes habet esse circa bonum futurum : } de praesentibus enim bonis non est spes , \\\hline
1.3.5 & ¶ \textbf{ Et estas quatro cosas conuiene saber El bien . } Et el bien alto e guaue Et el futuro que ha de ser . & quia circa impossibile nullus sperat , sed desperat . \textbf{ Haec autem quatuor , | videlicet , } bonum , arduum , \\\hline
1.3.5 & Ca segunt el philosofo propriamente pertenesçe alos Reyes \textbf{ e alos prinçipes de poner las leyes . } Et parte nesçe a ellos de es par algun bien . & Nam cum Reges et Principes sint latores legum , \textbf{ quia secundum Philosophum proprie spectat ad Reges et Principes leges ponere , } spectat ad eos sperare bonum . \\\hline
1.3.5 & Por ende non solamente parte nesçe alos Reyes \textbf{ e alos prinçipes de entender en el bien } Mas avn les conuiene de entender en bien alto e grande e guaue de fazer De mas desto & cum talia sint bona excellentia et ardua , \textbf{ non solum spectat ad Reges et Principes tendere in bonum , } sed etiam decet eos tendere in bonum arduum . \\\hline
1.3.5 & e alos prinçipes de entender en el bien \textbf{ Mas avn les conuiene de entender en bien alto e grande e guaue de fazer De mas desto } quanto mayor es la comunidat & non solum spectat ad Reges et Principes tendere in bonum , \textbf{ sed etiam decet eos tendere in bonum arduum . } Amplius quanto maior est communitas , \\\hline
1.3.5 & quanto mayor es la comunidat \textbf{ tantomas cosas le pueden auenir e contesçer . } Por ende ha menester de seer de mayor prouidençia & Amplius quanto maior est communitas , \textbf{ tanto plura possunt ei contingere , } ideo maiori indiget prouidentia \\\hline
1.3.5 & e el consseio non sean sinon de las cosas \textbf{ que han de venir . } Et non delas passadas & cum ergo prouidentia \textbf{ et consilium non sit } nisi de rebus futuris , \\\hline
1.3.5 & assi commo dize el philosofo en el tercero libro delas ethicas \textbf{ e las cosas passadas non se pueden mudar } assi commo dize el philosofo en el septimo libro delas ethicas & ut vult Philosophus 3 Ethicorum , praeterita , \textbf{ quae immutabilia sunt , } ut haberi potest ex 6 Ethic’ non cadunt sub consilio , \\\hline
1.3.5 & Et pues que assi es conuiene alos Reyes \textbf{ e alos prinçipes de penssar los bienes } non solamente en quanto son altos e grandes & ut haberi potest ex 6 Ethic’ non cadunt sub consilio , \textbf{ nec etiam cadunt sub prouidentia . Decet ergo Reges et Principes considerare bona non solum ut sunt ardua , } sed ut sunt futura . Congruit \\\hline
1.3.5 & non solamente en quanto son altos e grandes \textbf{ mas avn les conuiene de los penssar } en quanto son bienes que han de venir & nec etiam cadunt sub prouidentia . Decet ergo Reges et Principes considerare bona non solum ut sunt ardua , \textbf{ sed ut sunt futura . Congruit } etiam eos considerare talia , \\\hline
1.3.5 & mas avn les conuiene de los penssar \textbf{ en quanto son bienes que han de venir } Et avn les conuiene de penssar tales bienes & nec etiam cadunt sub prouidentia . Decet ergo Reges et Principes considerare bona non solum ut sunt ardua , \textbf{ sed ut sunt futura . Congruit } etiam eos considerare talia , \\\hline
1.3.5 & en quanto son bienes que han de venir \textbf{ Et avn les conuiene de penssar tales bienes } en quanto pueden ser . & sed ut sunt futura . Congruit \textbf{ etiam eos considerare talia , } ut possibilia . Nam pauperes , impotentes et ignobiles \\\hline
1.3.5 & e las riquezas e la nobleza non les siruen aellos \textbf{ por que puedan alcançar tales bienes . } Mas los Reyes e los prinçipes alos quales sirue la nobleza de linage & et subtrahunt se ab aliquibus bonis arduis , \textbf{ videntur mereri indulgentiam , } quia ciuilis potentia , diuitiae , \\\hline
1.3.5 & Mas los Reyes e los prinçipes alos quales sirue la nobleza de linage \textbf{ e el poderio çiuil e abondança de riquezas non se pueden escusar } que non sean de flacos coraçones & videntur mereri indulgentiam , \textbf{ quia ciuilis potentia , diuitiae , } et nobilitas non adminiculantur eis , \\\hline
1.3.5 & si non creyeren \textbf{ que ellos pueden alcançar tan grandes bienes } e tan dignos de grand honrra . & et nobilitas non adminiculantur eis , \textbf{ ut possint prosequi talia bona : } Reges autem et Principes , \\\hline
1.3.5 & de una entender alos bienes altos e grandes \textbf{ e de una proueer los biens } que han de venir e los bienes & potentia ciuilis , \textbf{ abundantia diuitiarum , } inexcusabiles esse videntur , \\\hline
1.3.5 & e de una proueer los biens \textbf{ que han de venir e los bienes } que pueden acahesçer a su regno . & abundantia diuitiarum , \textbf{ inexcusabiles esse videntur , } si sint pusillanimes , \\\hline
1.3.5 & que han de venir e los bienes \textbf{ que pueden acahesçer a su regno . } Por ende conuiene a ellos de serbine esparautes & abundantia diuitiarum , \textbf{ inexcusabiles esse videntur , } si sint pusillanimes , \\\hline
1.3.5 & que han en ssi \textbf{ e por que han todas aquellas cosas queꝑ tenesçen ala espança conuenible ¶ visto conmolos Reyes } e los prinçipes se deuen bien auer en esparar & si sint pusillanimes , \textbf{ et non credant eis esse possibile prosequi bona ardua et magno honore digna . } Quare cum Reges et Principes tendere debeant in bona ardua , \\\hline
1.3.5 & e por que han todas aquellas cosas queꝑ tenesçen ala espança conuenible ¶ visto conmolos Reyes \textbf{ e los prinçipes se deuen bien auer en esparar } las cosas & et non credant eis esse possibile prosequi bona ardua et magno honore digna . \textbf{ Quare cum Reges et Principes tendere debeant in bona ardua , } et debeant prouidere bona futura possibilia ipsi regno : decet eos esse bene sperantes per magnanimitatem , \\\hline
1.3.5 & las cosas \textbf{ que son de esparfinca de veer } en qual manera se deuen bien auer & Quare cum Reges et Principes tendere debeant in bona ardua , \textbf{ et debeant prouidere bona futura possibilia ipsi regno : decet eos esse bene sperantes per magnanimitatem , } quia habent omnia quae ad spem debitam requiruntur . Viso , \\\hline
1.3.5 & que son de esparfinca de veer \textbf{ en qual manera se deuen bien auer } e non esparar las cosas & Quare cum Reges et Principes tendere debeant in bona ardua , \textbf{ et debeant prouidere bona futura possibilia ipsi regno : decet eos esse bene sperantes per magnanimitatem , } quia habent omnia quae ad spem debitam requiruntur . Viso , \\\hline
1.3.5 & en qual manera se deuen bien auer \textbf{ e non esparar las cosas } que non deuen esparar & et debeant prouidere bona futura possibilia ipsi regno : decet eos esse bene sperantes per magnanimitatem , \textbf{ quia habent omnia quae ad spem debitam requiruntur . Viso , } quomodo decet Reges et Principes bene se habere in sperando speranda , \\\hline
1.3.5 & e non esparar las cosas \textbf{ que non deuen esparar } Ca conuiene alos Reyes de cuydar & et debeant prouidere bona futura possibilia ipsi regno : decet eos esse bene sperantes per magnanimitatem , \textbf{ quia habent omnia quae ad spem debitam requiruntur . Viso , } quomodo decet Reges et Principes bene se habere in sperando speranda , \\\hline
1.3.5 & que non deuen esparar \textbf{ Ca conuiene alos Reyes de cuydar } e escodinar con grand diligençia & quia habent omnia quae ad spem debitam requiruntur . Viso , \textbf{ quomodo decet Reges et Principes bene se habere in sperando speranda , | restat videre quomodo se habere debeant in non sperando non speranda . } Decet enim eos cum magna diligentia inuestigare , \\\hline
1.3.5 & Ca conuiene alos Reyes de cuydar \textbf{ e escodinar con grand diligençia } que cosa es aquello que deuen esparar & restat videre quomodo se habere debeant in non sperando non speranda . \textbf{ Decet enim eos cum magna diligentia inuestigare , } quid sperent , \\\hline
1.3.5 & e escodinar con grand diligençia \textbf{ que cosa es aquello que deuen esparar } e que cosa es aquello que deuen acome ter . Ca assi commo & Decet enim eos cum magna diligentia inuestigare , \textbf{ quid sperent , } et quid aggrediantur . \\\hline
1.3.5 & que cosa es aquello que deuen esparar \textbf{ e que cosa es aquello que deuen acome ter . Ca assi commo } por la magranimidat deuen ser apareiados & quid sperent , \textbf{ et quid aggrediantur . } Nam sicut per magnanimitatem debent esse prompti , \\\hline
1.3.5 & por la magranimidat deuen ser apareiados \textbf{ para acometer las cosas altas e guaues } e esparar las cosas & Nam sicut per magnanimitatem debent esse prompti , \textbf{ ut aggrediantur ardua , et sperent speranda : } sic per humilitatem debent esse moderati , \\\hline
1.3.5 & para acometer las cosas altas e guaues \textbf{ e esparar las cosas } que son de esparar . & Nam sicut per magnanimitatem debent esse prompti , \textbf{ ut aggrediantur ardua , et sperent speranda : } sic per humilitatem debent esse moderati , \\\hline
1.3.5 & e esparar las cosas \textbf{ que son de esparar . } assi por la humildat deuen ser tenprados & ut aggrediantur ardua , et sperent speranda : \textbf{ sic per humilitatem debent esse moderati , } ut non aggrediantur aliquid ultra vires proprias , \\\hline
1.3.5 & e que non es ꝑen las cosas \textbf{ que non deuen esparar . } Mas nos podemos prouar e mostrar & ø \\\hline
1.3.5 & que non deuen esparar . \textbf{ Mas nos podemos prouar e mostrar } por dos maneras & ut non aggrediantur aliquid ultra vires proprias , \textbf{ et ut non sperent non speranda . Possumus autem duplici via inuestigare , } quod decet Reges et Principes aliquid aggredi ultra vires , \\\hline
1.3.5 & que non conuiene alos Reyes e alos prinçipes \textbf{ de acometer ninguna cosa } mas que la fuerca suya & et ut non sperent non speranda . Possumus autem duplici via inuestigare , \textbf{ quod decet Reges et Principes aliquid aggredi ultra vires , } et sperare ultra quam sit sperandum . \\\hline
1.3.5 & Otrossi non deuen esparaquellas cosas \textbf{ que non deuen esparar ¶ La primera razon se toma de parte del ofiçio de los } Reyeᷤ¶ la segunda de parte dela gente del pueblo quales acomnedado . & et sperare ultra quam sit sperandum . \textbf{ Prima via sumitur ex parte officii regis . Secunda , } ex parte gentis sibi commissae . \\\hline
1.3.5 & Por que es parmas \textbf{ que deue omne esparar } e acometer alguna obra & ex parte gentis sibi commissae . \textbf{ Sperare enim ultra quam sit sperandum , } et aggredi opus ultra vires suas , \\\hline
1.3.5 & que deue omne esparar \textbf{ e acometer alguna obra } mas que la su fuerca demanda paresçe & Sperare enim ultra quam sit sperandum , \textbf{ et aggredi opus ultra vires suas , } videtur ex imprudentia procedere , \\\hline
1.3.5 & por que los que non son prouados enlos fechͣs \textbf{ non pueden saber las cosas altas e grandes abiertamente } por que non sopieron la guaueza & iuuenes bene sperant , \textbf{ quod contingit ex ignorantia : inexperti enim non possunt } cognoscere arduitatem operis . Contingit \\\hline
1.3.5 & mucho la calentura natural \textbf{ vienen ayna atentar algunas cosas } que non pueden acabar & nam quia nimis abundat in eis calor , \textbf{ prorumpunt } ut attentent aliqua quae consumate non possunt . Sic etiam dicere possemus , \\\hline
1.3.5 & vienen ayna atentar algunas cosas \textbf{ que non pueden acabar } nin alcançar ¶ & nam quia nimis abundat in eis calor , \textbf{ prorumpunt } ut attentent aliqua quae consumate non possunt . Sic etiam dicere possemus , \\\hline
1.3.5 & que non pueden acabar \textbf{ nin alcançar ¶ } En elsa misma manera avn podemos dezir & prorumpunt \textbf{ ut attentent aliqua quae consumate non possunt . Sic etiam dicere possemus , } quod ebriosi plus sperant quam debent : \\\hline
1.3.5 & nin alcançar ¶ \textbf{ En elsa misma manera avn podemos dezir } que los beddos mas esperan & prorumpunt \textbf{ ut attentent aliqua quae consumate non possunt . Sic etiam dicere possemus , } quod ebriosi plus sperant quam debent : \\\hline
1.3.5 & que los beddos mas esperan \textbf{ que deuen esparar } por que quando son escalentados por el vino & ut attentent aliqua quae consumate non possunt . Sic etiam dicere possemus , \textbf{ quod ebriosi plus sperant quam debent : } quia calefacti ex vino et ebrietate amittentes rationis usum , \\\hline
1.3.5 & e tientan algunas cosas \textbf{ que non pueden acabar ¶ } Et pues que assi es conmo el ofiçio de los Reyes demande omne sabio & quia calefacti ex vino et ebrietate amittentes rationis usum , \textbf{ attentant aliqua quae non valent perficere . } Cum ergo regium officium requirat hominem prudentem \\\hline
1.3.5 & e alos prin çipes \textbf{ de non acometer ninguna cosa } mas que la su fuerça demanda . & decet Reges et Principes non aggredi aliquid ultra vires , \textbf{ et non sperare aliqua non speranda . } Secundo hoc decet eos ex parte populi sibi commissi , \\\hline
1.3.5 & mas que la su fuerça demanda . \textbf{ Otrossi les conuiene de non esparar alguas cosas } que non son de esparar ¶ & ø \\\hline
1.3.5 & Otrossi les conuiene de non esparar alguas cosas \textbf{ que non son de esparar ¶ } Lo segundo esto les conuiene alos Reyes & ø \\\hline
1.3.5 & que la su fuerça demanda¶ \textbf{ Et por ende si cosa desconuenible es poner toda la gente } e todo el regno a peligros deuen los Reyes & qui aggreditur aliquid \textbf{ ultra vires . Si ergo inconueniens est totam gentem } et totum regnum periculis exponere , \\\hline
1.3.5 & e los prinçipes con grand diligençia \textbf{ e con conseio grande e prolongado cuydar } qual cosa de una acometer & diuturno consilio \textbf{ et magna diligentia excogitare debent Reges et Principes } quid aggrediantur , \\\hline
1.3.5 & e con conseio grande e prolongado cuydar \textbf{ qual cosa de una acometer } por que non acometan cosa mas alta & et magna diligentia excogitare debent Reges et Principes \textbf{ quid aggrediantur , } ne assumant arduum aliquod ultra vires , \\\hline
1.3.5 & e porque non es ꝑen alguna cosa \textbf{ que non deuen esparar . } egunt que dize el philosofo & ne assumant arduum aliquod ultra vires , \textbf{ et ne sperent aliquid non sperandum . } Secundum Philosophum in 4 Ethicorum circa mores , \\\hline
1.3.6 & assi commo dize el philosofo en el primero libro de los fisicos . \textbf{ por ende en este primero libro conuiene de tractar delas costunbres de lons Reyes } uniuersalmente e generalmente e en semeiança . & ut dicitur 1 Physicorum , \textbf{ deo in hoc primo de moribus Regum oportet pertransire uniuersaliter typo : } quia in secundo , \\\hline
1.3.6 & mas alas cercunstançias particulares de cada vno . \textbf{ Empero conuiene de tractar primero estas cosas uniuersales et generales } por que el conosçimiento dellas faze mucho al conosçimiento delas cosas & et maxime in tertio plus descendemus ad particulares circumstantias . \textbf{ Expedit | tamen haec uniuersalia pertransire , } quia horum cognitio faciet \\\hline
1.3.6 & enlos fecho del regno . \textbf{ Enpero non conuiene en todo en todo de desçender en el libro terçero alas cosas particulares } por que aquellos que son prouados en las cortes & ad cognitionem sequentium . In tertio ergo libro magis particulariter tractabimus facta regni : \textbf{ non tamen etiam in illo libro expediet penitus usque ad particularia descendere : } quia experti in curiis , et maxime Reges \\\hline
1.3.6 & e en semeiança enformamos los Reyes e los prinçipes \textbf{ en qual manera se deuen auer çerca la escanca } e cerca la desesperanca & et Principes , \textbf{ quomodo se habere debeant circa spem et desperationem quae respiciunt futurum bonum ; } sic \\\hline
1.3.6 & que catan al bien \textbf{ que ha de venir . } En essa misma manera seg̃t essa misma sciençia los podemos ensseñar & quomodo se habere debeant circa spem et desperationem quae respiciunt futurum bonum ; \textbf{ sic } secundum eandem methodum eos instruere possumus , \\\hline
1.3.6 & que ha de venir . \textbf{ En essa misma manera seg̃t essa misma sciençia los podemos ensseñar } en qual manera se de una auer çerca la osadia & sic \textbf{ secundum eandem methodum eos instruere possumus , } quomodo se habere debeant contra timorem , \\\hline
1.3.6 & que catan al mal \textbf{ que ha de uenir } Mas por auenta parescrie a alguno & quomodo se habere debeant contra timorem , \textbf{ et audaciam , } quae respiciunt futurum malum . Videtur autem forte aliquibus Reges , \\\hline
1.3.6 & commo estas paresçen ser contrarias ala Real magestad . \textbf{ Ca los Reyes e los prinçipes suelen auer muchos } que los mueuen e los enduzen & quae respiciunt futurum malum . Videtur autem forte aliquibus Reges , \textbf{ et Principes in nullo debere esse timidos , quia talia regiae maiestati derogare dicuntur . Multos autem sic incitantes Reges habere consueuerunt , } persuadentes eis \\\hline
1.3.6 & assi commo dize el philosofo en el primero libͤ de los grandes morales . \textbf{ Et pues que assi es conuiene deuer en qual manera conuiene alos Reyes } de sertemosos e de ser osados & non est fortis , \textbf{ sed satuus . Oportet ergo videre quo modo eos esse deceat timidos , } et audaces . Timor autem si moderatus sit , expediens est Regibus et Principibus . \\\hline
1.3.6 & es conuenible alos Reyes e alos prinçipes . \textbf{ Ca por temor tenprado todos los prinçipes deuen temer } que alguna cosa non se leunate en el regno & et audaces . Timor autem si moderatus sit , expediens est Regibus et Principibus . \textbf{ Moderato enim timore omnes principantes timere debent , } ne aliquid insurgat in regno , \\\hline
1.3.6 & que alguna cosa non se leunate en el regno \textbf{ que pueda dannar e menguar el estado del regno . } Mas quanto pertenesce a lo presente nos podemos declarar en dos maneras & ne aliquid insurgat in regno , \textbf{ quod eius bonum statum deprauare possit . | Possumus autem } ( quantum ad praesens ) \\\hline
1.3.6 & que pueda dannar e menguar el estado del regno . \textbf{ Mas quanto pertenesce a lo presente nos podemos declarar en dos maneras } que el temortenprado es neçessario alos Reyes e alos prinçipes & Possumus autem \textbf{ ( quantum ad praesens ) | duplici via inuestigare , } quod moderatus timor necessarius sit Regibus . \\\hline
1.3.6 & ¶La primera se toma de parte del conseio \textbf{ que es de tomar ¶ } La segunda se to ma de parte dela obra & Prima via sumitur \textbf{ ex parte consilii habendi . } Secunda ex parte operis fiendi . \\\hline
1.3.6 & La segunda se to ma de parte dela obra \textbf{ que es de fazer } la primera manera se puede assi mostrar . & ex parte consilii habendi . \textbf{ Secunda ex parte operis fiendi . } Prima via sic patet : \\\hline
1.3.6 & que es de fazer \textbf{ la primera manera se puede assi mostrar . } Ca assi commo es dicho en el segundo libro delas ethicas & Secunda ex parte operis fiendi . \textbf{ Prima via sic patet : | nam , } ut dicitur 2 Rhetoric’ cap’ de timore , \\\hline
1.3.6 & en el capitulo del temor \textbf{ que el temor nos faze auer consseio . } Ca por que alguno teme alguna cosa & ut dicitur 2 Rhetoric’ cap’ de timore , \textbf{ Timor consiliatiuos facit , } ex eo , \\\hline
1.3.6 & Ca por que alguno teme alguna cosa \textbf{ por ende ha conseio en qual manera pueda foyr de aquel mal } de que dubda e teme . & quod aliquis ei timet , consiliatur , \textbf{ quomodo possit effugere malum , } quod dubitat . \\\hline
1.3.6 & conuiene alos Reyes \textbf{ e alos prinçipes de auer algun temor tenprado } sienpre tomado conseio sobrello ¶ & Cum ergo unum totum regnum absque magno consilio debite gubernari non possit , \textbf{ expedit Principibus , et Regibus ut consiliatiui reddantur , habere aliquem moderatum timorem . Secundo hoc idem inuestigare possumus } ex parte operis fiendi . \\\hline
1.3.6 & sienpre tomado conseio sobrello ¶ \textbf{ Lo segundo podemos esso mismo mostrar de parte dela obra } que es de fazer . & expedit Principibus , et Regibus ut consiliatiui reddantur , habere aliquem moderatum timorem . Secundo hoc idem inuestigare possumus \textbf{ ex parte operis fiendi . } Nam non sufficit solicitari circa consilia et iudicare \\\hline
1.3.6 & Lo segundo podemos esso mismo mostrar de parte dela obra \textbf{ que es de fazer . } Ca non abasta de ser acuciosos çerca los conseios & ex parte operis fiendi . \textbf{ Nam non sufficit solicitari circa consilia et iudicare } de consiliatis , \\\hline
1.3.6 & Ca non abasta de ser acuciosos çerca los conseios \textbf{ e iuidgar delas cosas conseiadas } si non pusieremosen obra conueniblemente & Nam non sufficit solicitari circa consilia et iudicare \textbf{ de consiliatis , } nisi consiliata , \\\hline
1.3.6 & mas acuciosamente fazemos las obras \textbf{ por las quales queremos foyr de aquel temor } Et por ende mostrado es & diligentius agimus opera , \textbf{ per quae fugere credimus timorem illum . Ostensum est ergo , } quod decet Reges , et Principes moderatum habere timorem . \\\hline
1.3.6 & Et por ende mostrado es \textbf{ que los Reyes e los prinçipes deuen auer temor tenprado . } Enpero temer destenpradamente & per quae fugere credimus timorem illum . Ostensum est ergo , \textbf{ quod decet Reges , et Principes moderatum habere timorem . } Attamen immoderate timere nullo modo decet eos . Immoderatus enim timor quatuor habere videtur , \\\hline
1.3.6 & que los Reyes e los prinçipes deuen auer temor tenprado . \textbf{ Enpero temer destenpradamente } en ninguna manera non conuiene alos Reyes . & per quae fugere credimus timorem illum . Ostensum est ergo , \textbf{ quod decet Reges , et Principes moderatum habere timorem . } Attamen immoderate timere nullo modo decet eos . Immoderatus enim timor quatuor habere videtur , \\\hline
1.3.6 & al ome ser encogido \textbf{ e non semauer a las cosas ¶ } Lo segundo faze e omne que non puede auer consseio & primo reddit hominem immobilem , et contractum . \textbf{ Secundo facit ipsum inconsiliatiuum . } Tertio facit eum tremulentum . \\\hline
1.3.6 & e non semauer a las cosas ¶ \textbf{ Lo segundo faze e omne que non puede auer consseio } ¶lotraçero faz al omne tremuliento & primo reddit hominem immobilem , et contractum . \textbf{ Secundo facit ipsum inconsiliatiuum . } Tertio facit eum tremulentum . \\\hline
1.3.6 & Lo quarto faz al omne \textbf{ que non puede obrar } por que quando alguno teme la calentura natural & Tertio facit eum tremulentum . \textbf{ Quarto facit eum inoperatiuum . } Cum enim quis timet , calor ad interiora progreditur ; modum enim , \\\hline
1.3.6 & en todas las cosas \textbf{ podemos lo ueer en la calentura del cuerpo natural . } Ca quando algunos omes & quem videmus in hominibus , \textbf{ aspicere possumus in calore corporis naturalis . } Cum enim homines existentes in campis timent , \\\hline
1.3.6 & e sea encogido \textbf{ cosa muy desconuenjble es al Rey de temer } e de auer temor destenprado e sin razon¶ & sic cum quis timet , calor existens in exterioribus membris , statim confugit ad interiora ; propter quod homo confugit , in seipso contrahitur , et redditur immobilis . Quare si indecens est caput regni siue Regem esse immobilem et contractum , \textbf{ indecens est ipsum timere timore immoderato . } Secundo hoc est indecens , \\\hline
1.3.6 & cosa muy desconuenjble es al Rey de temer \textbf{ e de auer temor destenprado e sin razon¶ } Lo segundo esto es cosa desconuenible & sic cum quis timet , calor existens in exterioribus membris , statim confugit ad interiora ; propter quod homo confugit , in seipso contrahitur , et redditur immobilis . Quare si indecens est caput regni siue Regem esse immobilem et contractum , \textbf{ indecens est ipsum timere timore immoderato . } Secundo hoc est indecens , \\\hline
1.3.6 & e es puesto fuera de entendimiento \textbf{ e non sabe que faze . por la qual cosa non se acuerda de auer conseio . } Et sil fuere dado consseio & et ignorat \textbf{ quid faciat , | propter quod non recordatur consiliari : } et si consilium ei detur , \\\hline
1.3.6 & e que el Rey sea sin conseio . \textbf{ Cosa desconuenible es mucho al regno de temer } por temor destenprado et sin razon ¶ & ut facta regni sine consilio gerantur , \textbf{ et } ut Rex sit inconsiliatiuus ; indecens est ipsum timere immoderato timore . Tertio immoderatus timor reddit hominem tremulentum . \\\hline
1.3.6 & Et por ende quando los neruios son enfriados \textbf{ e non pueden sofrir los mienbros del cuerpo acahesçeles } e viene les luego el tremer ¶ & exteriora membra frigida manent . Nerui ergo fiunt frigefacti , \textbf{ et non valentes sustinere membra , } quare accidit ei tremor . \\\hline
1.3.6 & e non pueden sofrir los mienbros del cuerpo acahesçeles \textbf{ e viene les luego el tremer ¶ } Et pues que assi es si cosa desconuenible es al Rey de ser tremuliento & et non valentes sustinere membra , \textbf{ quare accidit ei tremor . } Ergo si inconueniens est Regem esse tremulentum , \\\hline
1.3.6 & Et pues que assi es si cosa desconuenible es al Rey de ser tremuliento \textbf{ la qual cosa deue seruaron costante e firme desconuenible cosa es ael de temer } por temor destenpdo e sin razon¶ & Ergo si inconueniens est Regem esse tremulentum , \textbf{ qui debet esse virilis | et constans , } inconueniens est ipsum timere \\\hline
1.3.6 & e esta atomeçido \textbf{ e non se puede mouer } e non sabe que se faza ¶ Et pues & immoderatum tremens \textbf{ et obstupefactus immobilitatur , } et nescit \\\hline
1.3.6 & que no nobre \textbf{ e non pueda mandar } por el temor destenprado e sin razon & si Rex sit inoperatiuus , \textbf{ et imperare non valeat propter immoderatum timorem , } toti regno praeiudicium gignitur : \\\hline
1.3.6 & si esto es cosa desconuenible \textbf{ non conuiene alos Reyes de temer } e por temor deste prado e sin razon . & quare si hoc est indecens , \textbf{ non decet Regem immoderato timore timere . } Viso quomodo Reges se habere debeant ad timorem , \\\hline
1.3.6 & e por temor deste prado e sin razon . \textbf{ ¶ visto en qual manera los Reyes se deuen auer al temor } por que cosa mas guaue es de repremir el temor que tenprar la osadia & non decet Regem immoderato timore timere . \textbf{ Viso quomodo Reges se habere debeant ad timorem , } quia difficilius est reprimere timorem , \\\hline
1.3.6 & ¶ visto en qual manera los Reyes se deuen auer al temor \textbf{ por que cosa mas guaue es de repremir el temor que tenprar la osadia } assi commo fue dicho de suso en el capitulo dela fortaleza & Viso quomodo Reges se habere debeant ad timorem , \textbf{ quia difficilius est reprimere timorem , | quam moderare audaciam , } ut dictum fuit supra Capitulo de fortitudine : \\\hline
1.3.6 & assi commo fue dicho de suso en el capitulo dela fortaleza \textbf{ de ligero puede paresçer } en qual manera se deuen auer los Reyes ala osadia . & ut dictum fuit supra Capitulo de fortitudine : \textbf{ de facili videri potest , } quomodo se habere debeant ad audacias decet enim eos non habere audaciam immoderatam , sed moderatam . \\\hline
1.3.6 & de ligero puede paresçer \textbf{ en qual manera se deuen auer los Reyes ala osadia . } Ca conuiene aellos de non auer osadia destenprada & de facili videri potest , \textbf{ quomodo se habere debeant ad audacias decet enim eos non habere audaciam immoderatam , sed moderatam . } Nam si audacia sit immoderata , homo plus praesumit quam debeat , \\\hline
1.3.6 & en qual manera se deuen auer los Reyes ala osadia . \textbf{ Ca conuiene aellos de non auer osadia destenprada } e sin razon mas tenprada & de facili videri potest , \textbf{ quomodo se habere debeant ad audacias decet enim eos non habere audaciam immoderatam , sed moderatam . } Nam si audacia sit immoderata , homo plus praesumit quam debeat , \\\hline
1.3.6 & en toda manera seria cosa desconuenible \textbf{ por que estonçe non acometria ninguna cosa¶ Et pues que assi es auersse tenpradamente al temor } e ala osadia es cosa en todo en todo conuenible alos Reyes e alos prinçipes . & Si vero quis nullam audaciam habeat , omnino est indecens : \textbf{ quia tunc nihil aggreditur . | Moderate ergo se habere ad timorem , } et ad audaciam Regibus et Principibus omnino congruit . \\\hline
1.3.7 & e grand vezindat con la mal querençia . \textbf{ primeronte que mostremos en qual manera los reyes e los prinçipes se deuen auer cerca la sanna } e çerca la mansedunbre deuemos & Prius quam ostendamus , \textbf{ quomodo Reges , | et Principes se habere debeant circa iram , et mansuetudinem . } Videndum est quomodo ira differat ab odio : \\\hline
1.3.7 & que querera alguno algun bien \textbf{ segunt si En essa misma manera querer mala alguna cosa esquerer } que luenga algun mal siplemente e suelta mente . & ut dicitur 2 Rhetor’ ) est idem quod velle alicui bonum \textbf{ secundum se . Sic odire aliquem est velle malum ei simpliciter , } et absolute . Ira autem non sit : \\\hline
1.3.7 & si mas es desseo de mal en conparaçion ala uengança . \textbf{ Et por ende se puede assi difinir } e declarar que la saña es appetito de pena ordenada ala uengança & sed in ordine ad vindictam . \textbf{ Potest enim diffiniri ira , } quod est appetitus poenae in vindictam . \\\hline
1.3.7 & Et por ende se puede assi difinir \textbf{ e declarar que la saña es appetito de pena ordenada ala uengança } Et desta diferençia prinçipalmente la san na e la mal querençia le toman ocho disterençias & Potest enim diffiniri ira , \textbf{ quod est appetitus poenae in vindictam . } Ex hac autem differentia principali inter iram et odium , \\\hline
1.3.7 & que parte nesçen assi mismo . \textbf{ Mas aborresçer } que es queter mala alguno & vel in pertinentibus ad ipsum . \textbf{ Sed odire , } quod est velle malum \\\hline
1.3.7 & Mas aborresçer \textbf{ que es queter mala alguno } segunt si puede ser de cosas & Sed odire , \textbf{ quod est velle malum } secundum se , \\\hline
1.3.7 & Por que luego quando sabemos que alguno es ladron \textbf{ podemos le mal querer } si quiera aya fecho mal a nos o a otros¶ & cum scimus aliquem esse malum , \textbf{ ut cum scimus aliquem esse furem , possumus ipsum odire , siue fore fecerit in nos , } siue in alios . Secunda differentia est , \\\hline
1.3.7 & mas la mal querençia puede ser en comun \textbf{ por que alguno puede mal queter todo ladron o todo retrahendor de mal comunal mente . } Mas enssannar se non puede & quia ira semper est in singulari : \textbf{ odire potest esse in communi . Odire autem potest aliquis communiter omnem furem , } et detractorem : \\\hline
1.3.7 & por que alguno puede mal queter todo ladron o todo retrahendor de mal comunal mente . \textbf{ Mas enssannar se non puede } si non a alguno en espeçial . & et detractorem : \textbf{ sed irasci non potest nisi alicui speciali . } Nam cum homo in communi non iniurietur nobis , \\\hline
1.3.7 & por algun omne espeçial . \textbf{ Et por ende podemos querer mal generalmente alos ladrones } enpero non nos enssannamos & sed semper committatur iniuria per aliquem hominem specialem : \textbf{ licet odire possumus fures uniuersaliter ; } non tamen irascimur , \\\hline
1.3.7 & aquel que nos mal queremos \textbf{ non podtia tanto auer de mal } que nos non quisiesemos & nam si odium est appetitus mali simpliciter , \textbf{ ille quem odimus , non posset tantum habere de malo , quin vellemus quod haberet plus . } Sed ira quae est appetitus poenae , \\\hline
1.3.7 & mas Mas la saña \textbf{ que es appetito de prinar non sinplemente } mas segunt que es ordenada ala uengança es alguna cosa & Sed ira quae est appetitus poenae , \textbf{ non simpliciter , } sed ut ordinatur ad vindictam , \\\hline
1.3.7 & e tuelga el sanudo ¶ \textbf{ La quarta diferençia es que el sanudo dessea contristar } e de fazer triste & quod videatur irato ultionem decentem factam esse , satiatur ira , et quiescit iratus . \textbf{ Quarta differentia est , } quia iratus appetit contristare : \\\hline
1.3.7 & La quarta diferençia es que el sanudo dessea contristar \textbf{ e de fazer triste } a a qual contra quien a saña . & Quarta differentia est , \textbf{ quia iratus appetit contristare : } sed odiens appeti nocere . \\\hline
1.3.7 & a a qual contra quien a saña . \textbf{ Mas el que quiere mal a alguno tenssea dele enpeçer . } Ca el sannudo quiere dar dolor e tsteza & quia iratus appetit contristare : \textbf{ sed odiens appeti nocere . } Vult enim iratus inferre dolorem , et tristitiam : \\\hline
1.3.7 & Mas el que quiere mal a alguno tenssea dele enpeçer . \textbf{ Ca el sannudo quiere dar dolor e tsteza } mas el mal quariente quiere fazer danno e enpeçemiento¶ & sed odiens appeti nocere . \textbf{ Vult enim iratus inferre dolorem , et tristitiam : } sed odiens vult inferre damnum , \\\hline
1.3.7 & Ca el sannudo quiere dar dolor e tsteza \textbf{ mas el mal quariente quiere fazer danno e enpeçemiento¶ } La quinta diferençia es & Vult enim iratus inferre dolorem , et tristitiam : \textbf{ sed odiens vult inferre damnum , } et nocumentum . Quinta differentia est , \\\hline
1.3.7 & en \textbf{ qual si quier manera que aquel mal le pueda al otro contesçer ¶ } La sexta diferençia es & quod alter patiatur malum , \textbf{ qualitercunque ei accidat malum illud . Sexta differentia est , } quia ira semper est cum tristitia : \\\hline
1.3.7 & por que la mal querençia puede ser contra alguna cosa en comun . \textbf{ Ca nos podemos natanlmente querer mal a todos los ladrones . } pero non conuiene de temer & quasi continue est in tristia . Sed odium sine tristitia esse potest : \textbf{ nam odium esse valet ad aliquid in communi . Odire enim possumus uniuersaliter omnes fures : } non tamen oportet , \\\hline
1.3.7 & Ca nos podemos natanlmente querer mal a todos los ladrones . \textbf{ pero non conuiene de temer } quetsteza se aconpanne a esta mal querençia . & nam odium esse valet ad aliquid in communi . Odire enim possumus uniuersaliter omnes fures : \textbf{ non tamen oportet , } quod tristitia committetur huiusmodi odium . \\\hline
1.3.7 & Mas la mal querençia non \textbf{ Porque commo la sanna se pueda fartar } si muchos males fueren fechos al otro & non autem odio . \textbf{ Nam cum ira satietur , } si multa mala inferantur alteri , \\\hline
1.3.7 & sean mucho peores \textbf{ que las condiconnes dela saña . Mas nos deuemos guardar dela mal querençia } que dela sanna ante segunt que dize sat̃ agostin la saña passar se en mal querençia & Cum ergo conditiones odii sint multo peiores , \textbf{ quam conditiones irae , | magis cauendum est odium quam ira . } Immo iram transire in odium \\\hline
1.3.7 & que las condiconnes dela saña . Mas nos deuemos guardar dela mal querençia \textbf{ que dela sanna ante segunt que dize sat̃ agostin la saña passar se en mal querençia } esto es de vna paia fazer ugalagar ¶ & magis cauendum est odium quam ira . \textbf{ Immo iram transire in odium | secundum Augustinum , } hoc est , \\\hline
1.3.7 & que dela sanna ante segunt que dize sat̃ agostin la saña passar se en mal querençia \textbf{ esto es de vna paia fazer ugalagar ¶ } Et pues que assi es la mal querençia es de esquiuar en toda manera alos Reyes e alos principes & secundum Augustinum , \textbf{ hoc est , | trabem facere de festuca . } Est ergo huiusmodi odium cauendum a quolibet . \\\hline
1.3.7 & esto es de vna paia fazer ugalagar ¶ \textbf{ Et pues que assi es la mal querençia es de esquiuar en toda manera alos Reyes e alos principes } por que podrien fazer mucho danno & trabem facere de festuca . \textbf{ Est ergo huiusmodi odium cauendum a quolibet . | Magis tamen cauendum est Regibus , et Principibus : } quia inferre possunt pluribus nocumentum . Sic igitur sentiendum est de odio et ira : \\\hline
1.3.7 & Et pues que assi es la mal querençia es de esquiuar en toda manera alos Reyes e alos principes \textbf{ por que podrien fazer mucho danno } e mucho mal a muchos . & ø \\\hline
1.3.7 & por que paresca en qual manera los Reyes e los prinçipesse de una auer çerca la sanna \textbf{ e cerca la mansedunbre conuiene de saber } que la sanna algunans vezes va ante la razon e ante el entendimiento . & sciendum quod ira aliquando rationem praecedit , \textbf{ et tunc est inordinata et cauenda , } aliquando sequitur ordinem rationis , \\\hline
1.3.7 & Et estonçe es desorde nada \textbf{ e non la deuen sufrir . } Ca si la sannava ante la razon & et tunc potest esse inordinata , \textbf{ et imitanda . } Si enim ira rationem praecedat , \\\hline
1.3.7 & e ante el entendimiento en dos maneras \textbf{ es de esquiuar e fuyr } ¶Lo primero por que non oye acabadamente la razon ¶ & Si enim ira rationem praecedat , \textbf{ dupliciter est cauenda . Primo , } quia non perfecte rationem audit . \\\hline
1.3.7 & ante que entiendan conplidamente el mandamiento del corren \textbf{ para fazer } e cunplir el su mandado . & antequam plene percipiant praeceptum eius , \textbf{ currunt , ut exequantur mandatum ipsius ; } quare contingit eos deficere , \\\hline
1.3.7 & para fazer \textbf{ e cunplir el su mandado . } Por la qual cosa les contesce algunas vegadas de errar & antequam plene percipiant praeceptum eius , \textbf{ currunt , ut exequantur mandatum ipsius ; } quare contingit eos deficere , \\\hline
1.3.7 & e cunplir el su mandado . \textbf{ Por la qual cosa les contesce algunas vegadas de errar } por que non entendieron conplidamente & currunt , ut exequantur mandatum ipsius ; \textbf{ quare contingit eos deficere , } quia non perfecte perceperunt quomodo exequendum sit mandatum illud . \\\hline
1.3.7 & por que non entendieron conplidamente \textbf{ en qual manera se auia de conplir aquel mandamiento . } En essa misma manera avn los canes & quare contingit eos deficere , \textbf{ quia non perfecte perceperunt quomodo exequendum sit mandatum illud . } Sic etiam \\\hline
1.3.7 & Ca luego que la razon e el entendimi ento dize \textbf{ que sea techa vengança luego quiere correr } por que sea fecha uengança & statim enim cum ratio dicit vindictam esse fiendam , \textbf{ statim vult currere , } ut vindictam exequatur , \\\hline
1.3.7 & deue ser esquiuada \textbf{ e deuemos foyr della } por que non oye acabadamente la razon ¶ & ø \\\hline
1.3.7 & por que non oye acabadamente la razon ¶ \textbf{ Lo segundo deuemos foyr } e esquiuar la sanna & Est igitur cauenda ira inordinata , \textbf{ quia non perfecte ratione audit . Secundo cauenda est : } quia rationem obnubilat . \\\hline
1.3.7 & Lo segundo deuemos foyr \textbf{ e esquiuar la sanna } por que cubre e oscuresçe la razon e el entendimiento & quia non perfecte ratione audit . Secundo cauenda est : \textbf{ quia rationem obnubilat . } Nam corpore non existente indebito temperamento , \\\hline
1.3.7 & por la saña se ençienda la sangre cerca el coraçon tornasse el cuerpo destenprado \textbf{ e non podemos conueniblemente vsar de la razon . } Ca maga el entendimiento non sea uirtud corporal & corpus redditur intemperatum , \textbf{ ut non possimus debite ratione uti . | Ratio enim , } vel intellectus licet \\\hline
1.3.7 & Ca maga el entendimiento non sea uirtud corporal \textbf{ enpero en su obra vsa de entender de organos e de mienbros corporales . } Por la qual cosa el cuerpo non estando bien ordenado & non sit virtus corporalis , \textbf{ utitur tamen in suo actu corporalibus organis ; } propter quod corpore existente indisposito , \\\hline
1.3.7 & commo deue non vsa el entendimiento liberalmente de su obra . \textbf{ Et ponen de si en cada vn omne es de esquiuar } que aya la razon e el entendimiento oscuresçido & non potest libere uti actu suo . \textbf{ Quare si in quolibet homine cauendum est habere rationem obnubilatam , } et non plene rationi obedire : \\\hline
1.3.7 & e non obedezca conplidamente ala razon \textbf{ a cada vno es de foyr } e de esquiuar la sanna desordenada . & et non plene rationi obedire : \textbf{ cauenda a quolibet est ira inordinata ; magis } tamen cauenda est a Regibus , \\\hline
1.3.7 & a cada vno es de foyr \textbf{ e de esquiuar la sanna desordenada . } Empero mas es de esquiuar & et non plene rationi obedire : \textbf{ cauenda a quolibet est ira inordinata ; magis } tamen cauenda est a Regibus , \\\hline
1.3.7 & e de esquiuar la sanna desordenada . \textbf{ Empero mas es de esquiuar } e de foyr la sana desordenada a los reyes e alos prinçipes & cauenda a quolibet est ira inordinata ; magis \textbf{ tamen cauenda est a Regibus , } et Princibus , \\\hline
1.3.7 & Empero mas es de esquiuar \textbf{ e de foyr la sana desordenada a los reyes e alos prinçipes } por que mucho mas conuiene aellos de segnir el iuyzio dela razon e del entendimiento . & tamen cauenda est a Regibus , \textbf{ et Princibus , } quia eos maxime decet sequi imperium rationis . Cauenda est ergo ira inordinata , \\\hline
1.3.7 & e de foyr la sana desordenada a los reyes e alos prinçipes \textbf{ por que mucho mas conuiene aellos de segnir el iuyzio dela razon e del entendimiento . } Et pues que assi es paresçe & et Princibus , \textbf{ quia eos maxime decet sequi imperium rationis . Cauenda est ergo ira inordinata , } et rationem praecedens . \\\hline
1.3.7 & Et pues que assi es paresçe \textbf{ que es de esquiuar la sanna desordenada } e aquella que viene ante iuyzio de la razon . & quia eos maxime decet sequi imperium rationis . Cauenda est ergo ira inordinata , \textbf{ et rationem praecedens . } Sed si sequatur imperium rationis , \\\hline
1.3.7 & e del entendimiento pueda ser ordenada \textbf{ e es de segnir . } Ca quando la sana es organo et estrumento de la razon e del entendimiento & potest esse ordinata , \textbf{ et imitanda . } Nam cum ira est rationis organum , \\\hline
1.3.7 & Et pues que assi es paresçe \textbf{ en qual manera nos deuemos auer çerca la saña } e çerca la manssedunbre . Ca por la manssedunbre es de repremir la sanna & quando ira est organum virtutis , et rationis . Patet ergo quomodo nos habere debemus circa iram , \textbf{ et mansuetudinem : } quia per mansuetudinem reprimenda est ira , ne praecedat iudicium rationis \\\hline
1.3.7 & en qual manera nos deuemos auer çerca la saña \textbf{ e çerca la manssedunbre . Ca por la manssedunbre es de repremir la sanna } por que non uaya ante el iuyzio dela razon . & quando ira est organum virtutis , et rationis . Patet ergo quomodo nos habere debemus circa iram , \textbf{ et mansuetudinem : } quia per mansuetudinem reprimenda est ira , ne praecedat iudicium rationis \\\hline
1.3.7 & por que non uaya ante el iuyzio dela razon . \textbf{ Et por la saña es de repremir la manssedunbre } porque non enbargue las obras de uirtudes & quia per mansuetudinem reprimenda est ira , ne praecedat iudicium rationis \textbf{ et per iram reprimenda est mansuetudo , } ne impediat virtutem opera , \\\hline
1.3.7 & por la razon \textbf{ que es aquello que deuemos fazer deuemos ser manssos . } Mas despues que fueriuisto conplidamente aquello que deuemos fazer & Ante ergo quam per rationem iudicemus plene quid agendum , \textbf{ debemus esse mansueti . } Sed postquam plene visum est , \\\hline
1.3.7 & que es aquello que deuemos fazer deuemos ser manssos . \textbf{ Mas despues que fueriuisto conplidamente aquello que deuemos fazer } podemos romar la sana & debemus esse mansueti . \textbf{ Sed postquam plene visum est , | quid facturi sumus , } possumus assumere iram tanquam seruam , \\\hline
1.3.7 & Mas despues que fueriuisto conplidamente aquello que deuemos fazer \textbf{ podemos romar la sana } assi commo sierua dela razon e del entendemiento & quid facturi sumus , \textbf{ possumus assumere iram tanquam seruam , } et ancillam rationis , \\\hline
1.3.7 & por que podamos \textbf{ por ella segnir mas esforcadamente aquellas cosas } que la razon iudgo . ¶ Et pues que assi es & et ancillam rationis , \textbf{ ut per eam exequamur virilius , } quae ratio iudicabit . \\\hline
1.3.7 & que la razon iudgo . ¶ Et pues que assi es \textbf{ si assi se deuen auer los omes çerca dela sanna e dela manssedunbre . } tanto mas conuiene a los Reyes & quae ratio iudicabit . \textbf{ Sic ergo se habere circa iram et mansuetudinem , } tanto magis decet Reges et Principes \\\hline
1.3.7 & tanto mas conuiene a los Reyes \textbf{ e alos prinçipes de se auer } assi commo dicho es & Sic ergo se habere circa iram et mansuetudinem , \textbf{ tanto magis decet Reges et Principes } quanto magis decet eos non impediri in usu rationis , \\\hline
1.3.7 & quanto mas conuiene aellos de non ser enbargados en el vso dela razon . \textbf{ Et otrossi quanto mas les conuiene de segnir esforcadamente } aquello que la razon e el entendemiento iudgua . ¶ & quanto magis decet eos non impediri in usu rationis , \textbf{ et viriliter exequi , quae ratio iudicabit . } Dicebatur enim supra , \\\hline
1.3.8 & ¶Et pues que assi es \textbf{ que dicho es de todas la s otras passiones finca de dezir } en qual manera los Reyes e los prinçipesse de una auer çerca las delectaçiones & vel ad delectationem . \textbf{ Dicto ergo de omnibus aliis passionibus restat dicere quomodo Reges , } et Principes se habere debeant circa delectationes , \\\hline
1.3.8 & que ninguna otra . \textbf{ Et todas las cosas dessean de se delectar } e auer delectaçion & quia quod ab omnibus appetitur maxime videtur esse bonum \textbf{ et eligibile : omnia autem appetunt delectari : } quod non esset , \\\hline
1.3.8 & Et todas las cosas dessean de se delectar \textbf{ e auer delectaçion } la qual cosa non seria & quia quod ab omnibus appetitur maxime videtur esse bonum \textbf{ et eligibile : omnia autem appetunt delectari : } quod non esset , \\\hline
1.3.8 & Et dizia \textbf{ assi que si qual si quier tristeza era de foyr } por que auia razon de mal & quae est delectationi contraria . \textbf{ Nam si tristitia quaelibet est fugienda , } et habet rationem mali : \\\hline
1.3.8 & por que auia razon de mal \textbf{ qual si quier delectacion era de seguir } por que auia razon de bien . & et habet rationem mali : \textbf{ quaelibet delectatio est prosequenda , } et habet rationem boni . \\\hline
1.3.8 & desto diziendo \textbf{ que toda delectaçion era mala de foyr e de esquiuar } Mas todos estos & Alii autem econtrario , \textbf{ dicebant omnem delectationem esse fugiendam . } Sed hi omnem delectationem condemnantes , \\\hline
1.3.8 & luego mostra una \textbf{ que la su posicion era de reprehender . } Ca segunt el philosofo ninguon non puede beuir sin alguna delectaçon & Sed hi omnem delectationem condemnantes , \textbf{ statim suam positionem ostendebant reprehensibilem : } quia \\\hline
1.3.8 & En essa misma manera el que pone \textbf{ que toda delectaçiones de esquiuar } e de foyr pone & ut patet per Philos 4 Metaphy’ ) \textbf{ sic ponens omnem delectationem esse fugiendam , } ponit aliquam delectationem esse prosequendam . \\\hline
1.3.8 & que toda delectaçiones de esquiuar \textbf{ e de foyr pone } que alguna delectaciones de segnir . & sic ponens omnem delectationem esse fugiendam , \textbf{ ponit aliquam delectationem esse prosequendam . } Nam cum loquela non possit negari , \\\hline
1.3.8 & e de foyr pone \textbf{ que alguna delectaciones de segnir . } Ca assy commo la fabla non puede ser negada & sic ponens omnem delectationem esse fugiendam , \textbf{ ponit aliquam delectationem esse prosequendam . } Nam cum loquela non possit negari , \\\hline
1.3.8 & En essa misma manera \textbf{ por que ninguno non puede foyr toda delectaçion } si non fuere a el delectable de foyr toda delectaçion & quare negando loquelam , \textbf{ concedit loquelam . Sic quia nullus omnem delectationem fugit , } nisi delectabile sit ei omnem delectationem fugere ; \\\hline
1.3.8 & por que ninguno non puede foyr toda delectaçion \textbf{ si non fuere a el delectable de foyr toda delectaçion } siguese que aquel que fuye toda delectaçion sigue algunan delectaçion . & concedit loquelam . Sic quia nullus omnem delectationem fugit , \textbf{ nisi delectabile sit ei omnem delectationem fugere ; } sequitur quod fugiens omnem delectationem , \\\hline
1.3.8 & Ca assi commo ueemos en el gostar \textbf{ assi podemos iuidgar en el appetito e en el desseo . } Ca assi commo algunos han el gosto desordenado & Sicut enim videmus in gustu , \textbf{ sic et in appetitu iudicare possumus . Aliqui enim habent gustum infectum , } ut infirmi , \\\hline
1.3.8 & Et pues que assi es \textbf{ assi commo non son de dezir uerdaderamente cosas dulçes } aquellas que paresçen dulçes alos enfermos & ut homines boni , \textbf{ et virtuosi . Sicut ergo non sunt dicenda vere dulcia , } quae videntur dulcia infirmis , et habentibus gustum infectum : sed quae videntur dulcia sanis , \\\hline
1.3.8 & e en buena disposicion . \textbf{ Et essa misma manera non son de dezir uerdaderamente cosas delectables } aquellas que son delectables alos uiçiosos e alos malos & et habentibus linguam bene dispositam . \textbf{ Sic non sunt dicenda vere delectabilia , } quae sunt delectabilia vitiosis , \\\hline
1.3.8 & e aquellos que han el appetito corrupto . \textbf{ Mas aquellas cosas son de dezir uerdaderamente delectables } que son delectables alos bueons & et habentibus appetitum infectum : \textbf{ sed quae sunt delectabilia bonis , } et habentibus voluntatem rectam . \\\hline
1.3.8 & Mas las delectaçonnes luxiosas e senssibles son conuenientes alas bestias . \textbf{ Et si los omes ouieren de vsar de tales delecta connes } esto non deue ser sinplemente & et sensibiles sunt conuenientes bestiis . \textbf{ Si autem homines talibus delectationibus uti debent : } hoc non est \\\hline
1.3.8 & Pues que assi es paresçe \textbf{ en qual manera nos deuemos auer a estas delecta connes . } Ca commo cosa denostable sea a cada vn çibdadano & et prout deseruiunt actionibus virtuosis . \textbf{ Patet ergo quomodo nos habere debemus ad ipsas delectationes . } Nam cum detestabile sit cuilibet \\\hline
1.3.8 & e que aya costunbres bestiales siguese \textbf{ que pertenesçe a cada vno de seguir } non aquellas cosas & et quod mores habeat bestiales , \textbf{ spectat ad quemlibet sequi non quae sunt delectabilia bestiis , } et hominibus vitiosis : \\\hline
1.3.8 & Et por ende quanto mas de deno stares alos Reyes \textbf{ e alos prinçipes de escoger uida de bestias . } Tanto mas de deno stares a ellos segnir las delecta connes bestiales ¶ & Quanto ergo detestabilius est Reges , \textbf{ et Principes eligere vitam pecudum , } tanto detestabilius est eos sequi bestiales delectationes . \\\hline
1.3.8 & e alos prinçipes de escoger uida de bestias . \textbf{ Tanto mas de deno stares a ellos segnir las delecta connes bestiales ¶ } Et pues que assi es paresçe & et Principes eligere vitam pecudum , \textbf{ tanto detestabilius est eos sequi bestiales delectationes . } Patet igitur quomodo Reges , \\\hline
1.3.8 & en qual manera los Reyes \textbf{ e los prinçipes se deuen auer alas delectaçonnes . } Ca prinçipalmente & Patet igitur quomodo Reges , \textbf{ et Principes se ad delectationes habere debent , } quia principaliter , \\\hline
1.3.8 & Ca prinçipalmente \textbf{ e por si se deuen delectar en las obras uirtuosas } por que la delectaçion sienpre acaba & quia principaliter , \textbf{ et per se delectari debent in operibus virtuosis . } Nam delectatio semper perficit , et expeditiorem reddit operationem conuenientem . \\\hline
1.3.8 & mas desenbargadamente \textbf{ e mas acabadamente podrian fazer estas tales obras . } Ca quando alguno mas fuertemente se delecta en las obras uirtuosas & expeditiori modo \textbf{ et magis perfecte efficere poterunt huiusmodi opera . Nam quanto quis vehementiori modo delectatur in actibus virtuosis , } tanto excellentius efficit illos actus . \\\hline
1.3.8 & e los prinçipes \textbf{ non se deuen delectar prinçipalmente } e por si mas deuen dellas vsar tenpradamente & et per se , \textbf{ sed uti debent eis moderate , } et prout habent ordinem ad opera virtuosa . \\\hline
1.3.8 & non se deuen delectar prinçipalmente \textbf{ e por si mas deuen dellas vsar tenpradamente } e en quanto son ordenadas alas obras uirtuosas . & sed uti debent eis moderate , \textbf{ et prout habent ordinem ad opera virtuosa . } Nam si tales delectationes vehementes sint , \\\hline
1.3.8 & Visto en qual manera los Reyes \textbf{ e los prinçipes se deuen auer alas delectaçiones } finça deuer en qual maneras & et operationes impediunt virtuosas . Viso , \textbf{ quomodo Reges , et Principes se habere debeant ad delectationes : } videre restat , \\\hline
1.3.8 & e los prinçipes se deuen auer alas delectaçiones \textbf{ finça deuer en qual maneras } e de una auer alas tristezas . & quomodo Reges , et Principes se habere debeant ad delectationes : \textbf{ videre restat , } quomodo se habere debent ad tristitias . Tristitia autem nunquam est assumenda , \\\hline
1.3.8 & e de una auer alas tristezas . \textbf{ Mas la tristeza nunca es de tomar } nin de loar & videre restat , \textbf{ quomodo se habere debent ad tristitias . Tristitia autem nunquam est assumenda , } nec est laudabilis , \\\hline
1.3.8 & Mas la tristeza nunca es de tomar \textbf{ nin de loar } si non fuere por alguna cosa torpe triste . & quomodo se habere debent ad tristitias . Tristitia autem nunquam est assumenda , \textbf{ nec est laudabilis , } nisi supposito aliquo turpi . \\\hline
1.3.8 & Ca si alguno vee \textbf{ que el fizo en algua manera cosas torpes deue se doler } et entsteçer se dello . & nisi supposito aliquo turpi . \textbf{ Si quis enim videt se in aliquo turpia egisse , } debet dolere et tristari . \\\hline
1.3.8 & que el fizo en algua manera cosas torpes deue se doler \textbf{ et entsteçer se dello . } ¶ pues que assi es delas cosas torpes se deuen los omes entristeçer & Si quis enim videt se in aliquo turpia egisse , \textbf{ debet dolere et tristari . } De turpibus igitur est tristandum , \\\hline
1.3.8 & et entsteçer se dello . \textbf{ ¶ pues que assi es delas cosas torpes se deuen los omes entristeçer } e auer tristeza & debet dolere et tristari . \textbf{ De turpibus igitur est tristandum , } sed omnis alia tristitia est moderanda , \\\hline
1.3.8 & ¶ pues que assi es delas cosas torpes se deuen los omes entristeçer \textbf{ e auer tristeza } e toda otra tristeza es destenpda & De turpibus igitur est tristandum , \textbf{ sed omnis alia tristitia est moderanda , } et vitanda ; \\\hline
1.3.8 & e es escusadera . \textbf{ Et por ende por que esta tal tristeza sea tenprada deuemos dar remedios } por los quales podamos escusar esta tal tristeza & et vitanda ; \textbf{ ut ergo huiusmodi tristitia moderetur , } danda sunt remedia , \\\hline
1.3.8 & Et por ende por que esta tal tristeza sea tenprada deuemos dar remedios \textbf{ por los quales podamos escusar esta tal tristeza } e esquiuar la . & et vitanda ; \textbf{ ut ergo huiusmodi tristitia moderetur , } danda sunt remedia , \\\hline
1.3.8 & por los quales podamos escusar esta tal tristeza \textbf{ e esquiuar la . } Et paresçe que el philosofo tanne tres remedios & ut ergo huiusmodi tristitia moderetur , \textbf{ danda sunt remedia , } per quae huiusmodi tristitia vitari possit . Videtur autem Philosophus tria remedia tangere , \\\hline
1.3.8 & Et paresçe que el philosofo tanne tres remedios \textbf{ por los quales la tristeza se puede esquiuar . } Conuiene saber las uirtudes . & per quae huiusmodi tristitia vitari possit . Videtur autem Philosophus tria remedia tangere , \textbf{ per quae tristitia vitatur ; videlicet , } virtutes , \\\hline
1.3.8 & por los quales la tristeza se puede esquiuar . \textbf{ Conuiene saber las uirtudes . } los amigos & per quae tristitia vitatur ; videlicet , \textbf{ virtutes , } amicos , \\\hline
1.3.8 & por que non fallan en si mismos \textbf{ onde se puedan delectar . } Ca los malos son enemigos assi mismos & quia ( ut vult Phil’ 9 Ethicor’ ) Mali et vitiosi seipsis non gaudent , \textbf{ non enim inueniunt unde delectari possint in seipsis . } Nam mali sibi ipsis inimicantur , et in seipsis dissentiunt , \\\hline
1.3.8 & e en ssi mismos han discordia \textbf{ assi conmo en aquel logar da a entender el philosofo . } porque lo vno iudgan por razon e por entendimiento & Nam mali sibi ipsis inimicantur , et in seipsis dissentiunt , \textbf{ ut ibidem innuitur . } Nam unum ratione iudicant , \\\hline
1.3.8 & que en nos mismos fallemos delectaçion . \textbf{ Et este remedio es escusar los pecados e auer uirtudes¶ } El segundo remedio es la consolaçion de los amigos . & et ut in nobis ipsis delectationem inueniamus , \textbf{ est fugere vitia , | et habere virtutes . } Secundum remedium est consolatio amicorum , \\\hline
1.3.8 & assi commo en el peso corporal \textbf{ quando muchos nos ayudan a leuar aquel peso } menos nos aguauiamos del peso . & videtur enim tristitia esse quoddam pondus aggrauans animam . \textbf{ Sicut ergo in pondere corporali cum multi iuuant nos } ad portandum illud , minus grauamur : sic cum videmus multitudinem amicorum condolore nobis , \\\hline
1.3.8 & dolemos nos nos \textbf{ e veemos los amigos doler . } Et assi paresçe & cum nos dolemus , \textbf{ et videmus amicos dolere , } ut videtur , \\\hline
1.3.8 & mas acresçienta se¶ \textbf{ Et pues que assi es podemos dezir } que quando nos veemos los amigos doler se & non debet dolor minui , \textbf{ sed augeri . Possumus ergo dicere } quod cum videmus eos dolere de dolore nostro , \\\hline
1.3.8 & Et pues que assi es podemos dezir \textbf{ que quando nos veemos los amigos doler se } del nuestro dolor non es menguado el nuestro dolor & sed augeri . Possumus ergo dicere \textbf{ quod cum videmus eos dolere de dolore nostro , } non quia ipsi dolent de dolore nostro minuitur dolor noster , \\\hline
1.3.8 & por aquellos que se duellen de nos \textbf{ mas por que ueemos aellos doler } se engendrase en nos vna firme fantasia & non quia ipsi dolent de dolore nostro minuitur dolor noster , \textbf{ sed quia videmus eos dolere , adgeneratur nobis quaedam firma fantasia quod sint amici : } et quia delectabile est habere amicos , delectamur : et delectando , \\\hline
1.3.8 & que son nros amigos . \textbf{ Et por que es cosa delectable auer amigos delectamos nos } e en delectado nos menguase el nuestro dolor & sed quia videmus eos dolere , adgeneratur nobis quaedam firma fantasia quod sint amici : \textbf{ et quia delectabile est habere amicos , delectamur : et delectando , } minuitur dolor noster : \\\hline
1.3.8 & e el penssamiento dela uerdat . \textbf{ Ca commo quier que nos deuemos doler delas cosastorpes empero } de los bienes dela uentura & est consideratio veritatis . \textbf{ Nam licet de turpibus sit dolendum , } de bonis tamen fortunae , \\\hline
1.3.8 & o de otros bienes \textbf{ que puedan anos contesçer } avn que non obremos cosas corpes non nos deuemos doler . & de bonis tamen fortunae , \textbf{ vel de aliis quae in nobis contingere possunt , } absque eo quod operemur turpia , \\\hline
1.3.8 & que puedan anos contesçer \textbf{ avn que non obremos cosas corpes non nos deuemos doler . } Mas para esto muy grand temedio es la consideraçion & vel de aliis quae in nobis contingere possunt , \textbf{ absque eo quod operemur turpia , | dolere non debemus . } Ad hoc autem , \\\hline
1.3.8 & Et pues que assi es paresçe \textbf{ que non nos deuemos doler } sinon delas cosas torpes & forte per accidens , \textbf{ inquantum per amissionem eorum impedimur ab operibus virtuosis . Patet ergo non esse dolendum , } nisi de turpibus , \\\hline
1.3.8 & o por conosçimiento de uerdat . \textbf{ Mas avn suel en dar otro remedio quarto a esto . } Conuien e saber remedios corporales . & vel per amicos , \textbf{ vel per cognitionem veritatis . Consueuit etiam ad hoc dari quartum subsidium , } videlicet , \\\hline
1.3.8 & Mas avn suel en dar otro remedio quarto a esto . \textbf{ Conuien e saber remedios corporales . } assi conmo el suenno e vannos & vel per cognitionem veritatis . Consueuit etiam ad hoc dari quartum subsidium , \textbf{ videlicet , | remedia corporalia , } ut somnus , \\\hline
1.3.8 & e otras cosas cales \textbf{ que suelen alongar e arredrar la tristeza Et pues que assi es commo tales obras de tristeza enbarguen las obras de uirtudes } tanto mas conuiene alos Reyes & balneum , \textbf{ et talia quae tristitiam fugare solent . | Cum ergo tales tristitiae impediant operationes virtuosas , } tanto magis decet Reges , et Principes tales tristitias moderare , \\\hline
1.3.8 & tanto mas conuiene alos Reyes \textbf{ e alos prinçipes de tenprar tales tristezas } quanto mas cosa conueniente es a ellos & Cum ergo tales tristitiae impediant operationes virtuosas , \textbf{ tanto magis decet Reges , et Principes tales tristitias moderare , } quanto decentius est eos excellere in operibus virtuosis . \\\hline
1.3.8 & e ala su magnificençia Real \textbf{ de sobrepuiar los otros en obras uirtuosas } ssi commo fueron contadas dessuso doze uirtudes & tanto magis decet Reges , et Principes tales tristitias moderare , \textbf{ quanto decentius est eos excellere in operibus virtuosis . } Sicut enumerabantur superius duodecim virtutes , \\\hline
1.3.9 & que son cotadas de ssuso \textbf{ signiendo la doctrina de nuestros anteçessores podemos dezer } que son las quatro prinçipales . & sic inter has duodecim passiones enumeratas , \textbf{ sequendo praedecessorum doctrinam , | dicere possumus , } quod sunt quatuor principales ; \\\hline
1.3.9 & Mas que estas sean mas prinçipales \textbf{ que las otras podemos lo prouar } por tres maneras & Has autem esse magis principales , \textbf{ triplici via venari possumus . Primo , } ut comparantur ad passiones alias . Secundo , \\\hline
1.3.9 & e despues va al desseo \textbf{ e terminasse en la esꝑança quando aquel bien ha de uenir . } Mas apostremas terminas se en gozo e en delectaçion . & postea vadit in desiderium , \textbf{ et terminatur in spem , | dum bonum illud est futurum : } ultimo autem terminatur \\\hline
1.3.9 & Mas quando es en conparaçion de algun mal comiença en la mal querençia \textbf{ e vayendo para la foyr } e para lo aborrescer & quando bonum illud est praesens , et adeptum . \textbf{ Respectu vero mali incipit ab odio , et procedit in fugam , vel abominationem , et terminatur in timorem , } si malum illud sit futurum : \\\hline
1.3.9 & e vayendo para la foyr \textbf{ e para lo aborrescer } e terminasse en temor & ø \\\hline
1.3.9 & si aquel mal fuerefuturo \textbf{ que ha de uenir . } mas apostremas terminasse en tristeza & Respectu vero mali incipit ab odio , et procedit in fugam , vel abominationem , et terminatur in timorem , \textbf{ si malum illud sit futurum : } ultimo autem terminatur in tristitiam , \\\hline
1.3.9 & ¶ La segunda manera \textbf{ para mostrar que estas passiones son prinçipales } se puede tomar de parte dela materia & sicut ad spem et gaudium ordinatur passiones sumptae respectu boni . \textbf{ Secunda via ad inuestigandum has passiones esse principales , } sumi potest ex parte materiae , \\\hline
1.3.9 & para mostrar que estas passiones son prinçipales \textbf{ se puede tomar de parte dela materia } cerca la qual obran tales passiones . & Secunda via ad inuestigandum has passiones esse principales , \textbf{ sumi potest ex parte materiae , } circa quam tales passiones versantur . Nam omnis passio , \\\hline
1.3.9 & o en conparaçion de algun mal . \textbf{ Otrossi el bien o el mal se pueden penssar } o tomar & vel respectu mali . Rursus bonum , \textbf{ vel malum considerari potest , } vel ut est futurum , vel ut est iam praesens . \\\hline
1.3.9 & Otrossi el bien o el mal se pueden penssar \textbf{ o tomar } o en quanto es futuro & vel respectu mali . Rursus bonum , \textbf{ vel malum considerari potest , } vel ut est futurum , vel ut est iam praesens . \\\hline
1.3.9 & o en quanto es futuro \textbf{ que es de venir o en quanto es presente . } Et por ende segunt esto se han de tomar estas quetro passions & vel malum considerari potest , \textbf{ vel ut est futurum , vel ut est iam praesens . } Secundum hoc ergo sumi habent hae quatuor passiones ; \\\hline
1.3.9 & que es de venir o en quanto es presente . \textbf{ Et por ende segunt esto se han de tomar estas quetro passions } por que del bien futuro es la esperança & vel ut est futurum , vel ut est iam praesens . \textbf{ Secundum hoc ergo sumi habent hae quatuor passiones ; } quia de bono futuro est spes , \\\hline
1.3.9 & Et del mal futuro \textbf{ que es de venir es el temor . } Mas del presente es latsteza¶ & de praesenti est gaudium : \textbf{ de malo futuro est timor , } de praesenti vero est tristitia . \\\hline
1.3.9 & que es de venir es el temor . \textbf{ Mas del presente es latsteza¶ } La terçera manera es que estas passiones se pueden tomar & de malo futuro est timor , \textbf{ de praesenti vero est tristitia . } Tertio modo hae passiones sumi possunt respectu potentiarum animae , \\\hline
1.3.9 & Mas del presente es latsteza¶ \textbf{ La terçera manera es que estas passiones se pueden tomar } en conparaçion delas potençias del alma . & de praesenti vero est tristitia . \textbf{ Tertio modo hae passiones sumi possunt respectu potentiarum animae , } ut respectu irascibilis , \\\hline
1.3.9 & quando es futuro \textbf{ e ha de uenir } e es esparado . & cum est futurum \textbf{ et speratur : } et malum arduum , \\\hline
1.3.9 & quando es futuro \textbf{ que es de uenir } e es tenido . & et malum arduum , \textbf{ cum est futurum et timetur : } spes et timor sunt principales passiones respectu irascibilis . \\\hline
1.3.9 & por estas passiones . \textbf{ Cconuiene a nos de acuçiosamente entender } en quales cosas nos deuemos delec tar & ø \\\hline
1.3.9 & Cconuiene a nos de acuçiosamente entender \textbf{ en quales cosas nos deuemos delec tar } e en quales cosas nos deuemos entsteçer e quales cosas deuemos espar & Sed cum ex passionibus diuersificari habeant opera nostra , decet nos diligenter intendere , in quibus delectemur , et tristemur , \textbf{ et quae speremus , } et quae timeamus . \\\hline
1.3.9 & en quales cosas nos deuemos delec tar \textbf{ e en quales cosas nos deuemos entsteçer e quales cosas deuemos espar } e quales cosas deuemos temer . & Sed cum ex passionibus diuersificari habeant opera nostra , decet nos diligenter intendere , in quibus delectemur , et tristemur , \textbf{ et quae speremus , } et quae timeamus . \\\hline
1.3.9 & e en quales cosas nos deuemos entsteçer e quales cosas deuemos espar \textbf{ e quales cosas deuemos temer . } Mas esto tanto mas conuiene alos Reyes et alos prinçipes & et quae speremus , \textbf{ et quae timeamus . } Sed hoc tanto magis decet Reges et Principes , \\\hline
1.3.10 & cuenta otras seys passiones . \textbf{ Conuiene saber Relo . } gera Njemesim & Sed praeter omnes has passiones Philosop’ \textbf{ 2 Rhetor’ sex alias passiones enumerare videtur , } videlicet , zelum , gratiam , nemesin \\\hline
1.3.10 & gera Njemesim \textbf{ que quiere dezir tanto commo indignacion dela buena andança de los malos . } Misericordia e jnuidia . & videlicet , zelum , gratiam , nemesin \textbf{ ( quod idem est quod indignatio de prosperitatibus malorum ) misericordiam , inuidiam , et erubescentiam siue verecundia . } Sed omnes hae passiones reducuntur ad aliquas passiones praedictarum : \\\hline
1.3.10 & e erubesçençia o uerguença . \textbf{ Mas todas estas passiones se pueden adozir a algunas delas passiones ya dichos . } Ca el zelo e la gera se pueden adozir al amor . & ( quod idem est quod indignatio de prosperitatibus malorum ) misericordiam , inuidiam , et erubescentiam siue verecundia . \textbf{ Sed omnes hae passiones reducuntur ad aliquas passiones praedictarum : } quia zelus , et gratia reducuntur ad amorem : \\\hline
1.3.10 & Mas todas estas passiones se pueden adozir a algunas delas passiones ya dichos . \textbf{ Ca el zelo e la gera se pueden adozir al amor . } Et la uerguença se puede adozir al temor e al esꝑanto . & Sed omnes hae passiones reducuntur ad aliquas passiones praedictarum : \textbf{ quia zelus , et gratia reducuntur ad amorem : } verecundia ad timorem : inuidia , et misericordia , et nemesis siue indignatio de prosperitatibus malorum reducuntur ad tristitiam . Zelus reducitur ad amorem : \\\hline
1.3.10 & Ca el zelo e la gera se pueden adozir al amor . \textbf{ Et la uerguença se puede adozir al temor e al esꝑanto . } La inuidia e la miscderia en emessis & quia zelus , et gratia reducuntur ad amorem : \textbf{ verecundia ad timorem : inuidia , et misericordia , et nemesis siue indignatio de prosperitatibus malorum reducuntur ad tristitiam . Zelus reducitur ad amorem : } quia zelus nihil est aliud , \\\hline
1.3.10 & que es descennamiento dela buean andança \textbf{ de los malos pueden se adozir ala tristeza . } Ca el zelo es aducho al amor & quia zelus nihil est aliud , \textbf{ quam quidam amor intensus . Ea autem , } quae intense diligimus , \\\hline
1.3.10 & por que tales quando son auidas de vno non son auidas de otro . \textbf{ Por ende se suele difinir } e declarar el zelo en conparaçion destas cosas & non habentur ab alio , \textbf{ ideo zelus respectu horum diffiniri consueuit , } quod est amor intensus non patiens consortium in amato . \\\hline
1.3.10 & Por ende se suele difinir \textbf{ e declarar el zelo en conparaçion destas cosas } diziendo & non habentur ab alio , \textbf{ ideo zelus respectu horum diffiniri consueuit , } quod est amor intensus non patiens consortium in amato . \\\hline
1.3.10 & assi que el zelo es amor muy grande \textbf{ que non puede sofrir conpanma ninguna enla cosa } que el ama . & ideo zelus respectu horum diffiniri consueuit , \textbf{ quod est amor intensus non patiens consortium in amato . } Inde ergo inoleuit , \\\hline
1.3.10 & que algunos son dichos çelosos \textbf{ de alguna personasi non quieren auer alguna conparia en ella ¶ } Et pues que assi es el amor grande de las colas corporales parelçe & Inde ergo inoleuit , \textbf{ quod aliqui dicuntur Zelotypi de persona aliqua , } si noluerint in ea habere aliquod consortium . Intensus ergo amor corporalium videtur esse amor priuatus , et reprehensibilis , \\\hline
1.3.10 & que es preuad amor \textbf{ e de rephender } por que non sufre & ø \\\hline
1.3.10 & Mas en conparaçion de los bienes intellectuales e en conpaçion delas uirtudes \textbf{ si fuere el amor grande es de loar } e es assi conmocomun . & et respectu virtutum , \textbf{ si sit intensus amor , | est laudabilis , } et quasi communis . \\\hline
1.3.10 & nin amaria propiamente las \textbf{ uirtudessi non quisiesse auer conpania en aquellas uirtudes . } Et por ende este tal zelo e amor en conparaçion delons bienes honrrables es difinido & nec proprie virtutes diligeret , \textbf{ si nollet in eis habere consortium . } Huiusmodi ergo zelus respectu bonorum honorabilium diffinitur a Philosopho 2 Rheto’ \\\hline
1.3.10 & si non vn mouemiento de coraçon \textbf{ por el qual es inclinado alguno a dar benefiçios } a otroo es inclinado a fazer bien a otro . & quam quidam motus animi , \textbf{ per quem inclinatur aliquis ad beneficia conferendum . Zelus ergo et gratia reducuntur ad amorem . } Sed verecundia reducitur ad timorem . Dupliciter autem quis timere potest , \\\hline
1.3.10 & por el qual es inclinado alguno a dar benefiçios \textbf{ a otroo es inclinado a fazer bien a otro . } ¶ Et pues que assi es el zelo & quam quidam motus animi , \textbf{ per quem inclinatur aliquis ad beneficia conferendum . Zelus ergo et gratia reducuntur ad amorem . } Sed verecundia reducitur ad timorem . Dupliciter autem quis timere potest , \\\hline
1.3.10 & Mas la uerguença reduze se al temor \textbf{ por que dos cosas pue de cada vno temer } Conuiene saber . Corrinpimientos e desonrras . & Sed verecundia reducitur ad timorem . Dupliciter autem quis timere potest , \textbf{ videlicet , } corruptiones , \\\hline
1.3.10 & por que dos cosas pue de cada vno temer \textbf{ Conuiene saber . Corrinpimientos e desonrras . } Ca aquel que se espanta de los males & videlicet , \textbf{ corruptiones , | et inhonorationes , } qui enim expauescit mala corruptiua , \\\hline
1.3.10 & que es b̃me iura en la cara \textbf{ suele se nonbrar uerguença paresçida en el rostre̊ . } Ca los uergonosos comunalmente se tornan bmeios & unde verecundia erubescentia nominari consueuit , \textbf{ quia verecundantes communiter erubescunt , } sicut timentes pallescunt . \\\hline
1.3.10 & que es bien de dentro teme \textbf{ mas por que alguno teme perder la honrra e la eglesia } que son bienes de fuera ha uerguença . & timet : \textbf{ sed ex eo , | quod credit se amittere gloriam et honorem , } quae sunt bona exteriora , verecundatur . \\\hline
1.3.10 & Et por ende la sangre core al coraçon \textbf{ para esforçar los mienbros de dentro . } Mas quando alguno ha uerguença la sangre corre alos mienbros de fuera & ø \\\hline
1.3.10 & El vno es \textbf{ que te meꝑ̃der la uida e los bienes de dentro } por los quales se amarellesçe¶ & unus amittendi vitam , \textbf{ et bona interiora , } per quae quis pallescit , \\\hline
1.3.10 & por los quales se amarellesçe¶ \textbf{ Et otro temor es temor de perder la eglesia e la honrra } que son los bienes de fuera & per quae quis pallescit , \textbf{ et alius amittendi gloriam , | et honorem , } quae sunt bona exteriora , \\\hline
1.3.10 & Et pues que assi es el temor coirupartiuo \textbf{ que es temor de perder los bienes de dentro } por que non ha nonbre espeçial Retiene en ssi nonbre comun & Timor ergo corruptiui , \textbf{ et amittendi bona interiora } quia non habet speciale nomen , \\\hline
1.3.10 & Mas el temor \textbf{ que es temor de perder la eglesia } e la honrra ha nonbre espeçial & et dicitur timor , \textbf{ sed timor amittendi gloriam , et honorem , } habet speciale nomen , \\\hline
1.3.10 & e es dicha uerguença o herubesçençia \textbf{ que quiere dezir en bermegecimiento . } Et pues que assi es la uergunença es temor espeçial & et dicitur verecundia , \textbf{ vel erubescentia . } Verecundia ergo est quidam timor , \\\hline
1.3.10 & al temor fica \textbf{ deuer en qual manera la inuidia e la miscderia enemessis } que es indignaçion delas bien andanças de los malos son aduchos ala tristeza . & Restat videre , \textbf{ quomodo inuidia , } misericordia , nemesis siue indignatio reducuntur ad tristitiam . \\\hline
1.3.10 & que es indignaçion delas bien andanças de los malos son aduchos ala tristeza . \textbf{ Et pues que assi es deuedes saber } que aqui son tres maneras de tristeza . & misericordia , nemesis siue indignatio reducuntur ad tristitiam . \textbf{ Sciendum ergo haec tria esse species tristitiae . } Nam aliquis tristari putest , \\\hline
1.3.10 & si non alguna tristeza sobre mala paresçiente corruptiono con tristatiuo de \textbf{ aquel que non meresçe aquel mal de sofrir . } Mas si la tristeza non es de mal de otro & quam tristitia quaedam super apparenti malo corruptiuo , vel contristatio eius , \textbf{ qui indigne patitur malum illud . } Si vero sit tristitia non de malo alterius sed de bono , \\\hline
1.3.10 & non en qual siquier manera \textbf{ mas en quanto el non meresçe de auer aquel bien } assi es dicha enemessis o indignaçion & non quocunque modo , \textbf{ sed ut indigne habetur ab eo : } sic est nemesis , \\\hline
1.3.10 & assi es dicha enemessis o indignaçion \textbf{ que quiere dezir desden . } Ca segunt el philosofo en el segundo libro dela rectorica . Nemessis o indignaçiones auer tristeza de aquel & vel indignatio . \textbf{ Nam ( secundum Philosophum 2 Rhetoricorum ) nemesis vel indignatio , } est tristari de eo qui indigne videtur bene prosperari . \\\hline
1.3.10 & que quiere dezir desden . \textbf{ Ca segunt el philosofo en el segundo libro dela rectorica . Nemessis o indignaçiones auer tristeza de aquel } que ha algun bien & vel indignatio . \textbf{ Nam ( secundum Philosophum 2 Rhetoricorum ) nemesis vel indignatio , } est tristari de eo qui indigne videtur bene prosperari . \\\hline
1.3.10 & que ha algun bien \textbf{ et non lo meresçe auer ¶ } Et pues que assi es & Nam ( secundum Philosophum 2 Rhetoricorum ) nemesis vel indignatio , \textbf{ est tristari de eo qui indigne videtur bene prosperari . } Si ergo omnes hae passiones diuersificare habent omnes operationes nostras , \\\hline
1.3.10 & Et pues que assi es \textbf{ si todas estas passiones han de partir } todasnr̃as obras conuiene a nos delas cognosçer todas . & est tristari de eo qui indigne videtur bene prosperari . \textbf{ Si ergo omnes hae passiones diuersificare habent omnes operationes nostras , } decet nos omnes eas cognoscere ; \\\hline
1.3.10 & si todas estas passiones han de partir \textbf{ todasnr̃as obras conuiene a nos delas cognosçer todas . } Et tanto mas esta conuiene alos Reyes e alos prinçipes & Si ergo omnes hae passiones diuersificare habent omnes operationes nostras , \textbf{ decet nos omnes eas cognoscere ; } et tanto magis hoc decet Reges et Principes , \\\hline
1.3.10 & Et tanto mas esta conuiene alos Reyes e alos prinçipes \textbf{ quantomas deuen auer las obras mas altas e mas nobles . } lgunas delas passiones sobredichas paresçen ser de loar & et tanto magis hoc decet Reges et Principes , \textbf{ quanto habere debent operationes maxime excellentes . } Praedictarum passionum quaedam videntur esse laudabiles : \\\hline
1.3.11 & quantomas deuen auer las obras mas altas e mas nobles . \textbf{ lgunas delas passiones sobredichas paresçen ser de loar } assi conmo la miscderia e la uerguença & quanto habere debent operationes maxime excellentes . \textbf{ Praedictarum passionum quaedam videntur esse laudabiles : } ut misericordia , et verecundia . \\\hline
1.3.11 & commo quier que non sea uirtud \textbf{ empero es passion de loar a vn } en essa misma manera la gera enemessis paresçen ser passiones de loar & secundum Philosophum , \textbf{ licet non sit virtus est tamen laudabilis passio . } Sic etiam gratia , \\\hline
1.3.11 & empero es passion de loar a vn \textbf{ en essa misma manera la gera enemessis paresçen ser passiones de loar } Mas otras algunas ay & licet non sit virtus est tamen laudabilis passio . \textbf{ Sic etiam gratia , } et nemesis laudabiles passiones esse videntur . Quaedam autem sunt vituperabiles : \\\hline
1.3.11 & Mas otras algunas ay \textbf{ que son de denostar } assi commo la inuidia & Sic etiam gratia , \textbf{ et nemesis laudabiles passiones esse videntur . Quaedam autem sunt vituperabiles : } ut inuidia , \\\hline
1.3.11 & e ahun la mal querençia \textbf{ que es cosa de denostar } si non fuer mal querençia de los pecados . & ut inuidia , \textbf{ et odium , } vituperabile est odium , \\\hline
1.3.11 & que se han a amas las partes \textbf{ por que pueden ser de loar } o pueden ser de deno star Empero deuedes parar mientes con grant acuçia & Aliae autem passiones videntur se habere ad utrunque , \textbf{ quia possunt esse laudabiles , } et vituperabiles . Est enim diligenter aduertendum , \\\hline
1.3.11 & por que pueden ser de loar \textbf{ o pueden ser de deno star Empero deuedes parar mientes con grant acuçia } que sienpre en las costunbres las meytades son de loar & quia possunt esse laudabiles , \textbf{ et vituperabiles . Est enim diligenter aduertendum , } quod semper in moribus laudantur media , et vituperantur extrema . Passiones ergo illae , \\\hline
1.3.11 & o pueden ser de deno star Empero deuedes parar mientes con grant acuçia \textbf{ que sienpre en las costunbres las meytades son de loar } e las estremidades son de denostar Et pues que assi es aquellas passiones & et vituperabiles . Est enim diligenter aduertendum , \textbf{ quod semper in moribus laudantur media , et vituperantur extrema . Passiones ergo illae , } quae videntur de se esse laudabiles , important rationem medii . \\\hline
1.3.11 & que sienpre en las costunbres las meytades son de loar \textbf{ e las estremidades son de denostar Et pues que assi es aquellas passiones } que paresçen de ser loadas de ssi trahen consigo razon de medio & et vituperabiles . Est enim diligenter aduertendum , \textbf{ quod semper in moribus laudantur media , et vituperantur extrema . Passiones ergo illae , } quae videntur de se esse laudabiles , important rationem medii . \\\hline
1.3.11 & Mas aquel que de njngua cosa non ha uerguença es dicho desuergonçado \textbf{ e ninguno destos non es de loar } por que son estremidades . & dicitur inuerecundus ; \textbf{ nullus autem horum est laudabilis . } Qui vero medio modo se habet , \\\hline
1.3.11 & este es dicho uirgonçoso \textbf{ e es de loar . } En essa misma manera avn la miscderia es mediana entre la crueldat e la molleza . & ut non debet , dicitur verecundus : \textbf{ et est laudabilis . Sic etiam misericordia media est inter crudelitatem , et molliciem . } Nam qui nulli compatitur , est crudelis , \\\hline
1.3.11 & Mas aquel que ha piedat de los que sufren algun mala tuerto tiene el medio \textbf{ e es de loar } e es dicho miscderioso . & tenet medium , \textbf{ et laudatur , } et dicitur misericors . \\\hline
1.3.11 & En essa misma manera avn la gran en \textbf{ quanto es de loar es medianera entre lo sobeio e lo menguado } por que aquel que es graçioso a todos tan bien alos dignos commo alos non dignos es de denostar . & Sic etiam gratia ut est quid laudabile , \textbf{ media est inter superfluum , | et diminutum . } Nam \\\hline
1.3.11 & quanto es de loar es medianera entre lo sobeio e lo menguado \textbf{ por que aquel que es graçioso a todos tan bien alos dignos commo alos non dignos es de denostar . } Mas aquel que non es guaçioso a ningund es otr̃ossi de denostar . & et diminutum . \textbf{ Nam | qui omnibus est gratiosus tam dignis quam indignis , vituperabilis est ; } qui autem nullis est gratiosus , vituperari debet . Sed qui est gratiosus dignis et non dignis , \\\hline
1.3.11 & por que aquel que es graçioso a todos tan bien alos dignos commo alos non dignos es de denostar . \textbf{ Mas aquel que non es guaçioso a ningund es otr̃ossi de denostar . } Et aquel que es graçioso alos dignos & qui omnibus est gratiosus tam dignis quam indignis , vituperabilis est ; \textbf{ qui autem nullis est gratiosus , vituperari debet . Sed qui est gratiosus dignis et non dignis , } hic tenet medium , \\\hline
1.3.11 & este el medio tiene \textbf{ e es de loar . } ¶ Et pues que assi es las passiones & hic tenet medium , \textbf{ et laudatur . } Passiones ergo laudabiles tenent medium : \\\hline
1.3.11 & ¶ Et pues que assi es las passiones \textbf{ que lon de loar tienen el medio } mas las que son de denostartienen el estremo . & et laudatur . \textbf{ Passiones ergo laudabiles tenent medium : } vituperabiles vero tenent extremum . \\\hline
1.3.11 & En essa misma manera ahun la mal querençia \textbf{ en quanto es de denostar tiene el estremo . Mas las passiones } que se han a amas las partes & quae de omni prosperitate dolet . Sic etiam odium ut est vituperabile , \textbf{ extremum tenet . | Passiones vero , quae se habent } ad utrunque , possunt esse laudabiles , \\\hline
1.3.11 & que se han a amas las partes \textbf{ e pueden ser de loar } e pueden ser de denostar & Passiones vero , quae se habent \textbf{ ad utrunque , possunt esse laudabiles , } et vituperabiles , \\\hline
1.3.11 & e pueden ser de loar \textbf{ e pueden ser de denostar } en quanto son de denostar tienen el estremo & ad utrunque , possunt esse laudabiles , \textbf{ et vituperabiles , } ut sunt vituperabiles , \\\hline
1.3.11 & e pueden ser de denostar \textbf{ en quanto son de denostar tienen el estremo } e en quanto son de loar tienen el medio . & et vituperabiles , \textbf{ ut sunt vituperabiles , | tenent extremum , } ut sunt laudabiles , tenent medium . His visis , \\\hline
1.3.11 & en quanto son de denostar tienen el estremo \textbf{ e en quanto son de loar tienen el medio . } ¶ Estas cosas iustas conuiene de veer & tenent extremum , \textbf{ ut sunt laudabiles , tenent medium . His visis , } videndum est , \\\hline
1.3.11 & e en quanto son de loar tienen el medio . \textbf{ ¶ Estas cosas iustas conuiene de veer } en qual manera los Reyes & ut sunt laudabiles , tenent medium . His visis , \textbf{ videndum est , } quomodo Reges et Principes ad has passiones se habere debeant . \\\hline
1.3.11 & e los prinçipesse de una auer a estas passiones . \textbf{ Ca deuen segnir la gera e milcderia } en quanto son passiones de loar & quomodo Reges et Principes ad has passiones se habere debeant . \textbf{ Nam gratiam , et misericordiam , } ut sunt passiones laudabiles , imitari debent . \\\hline
1.3.11 & Ca deuen segnir la gera e milcderia \textbf{ en quanto son passiones de loar } Por que conuiene a ellos de ser guaçiosos & Nam gratiam , et misericordiam , \textbf{ ut sunt passiones laudabiles , imitari debent . } Decet enim eos esse gratiosos , \\\hline
1.3.11 & quando los Reyes e los prinçipes \textbf{ e aquellos a quien conuiene de partir tales benefiçios son graçiosos conueniblemente alos buenos e alos dignos . } Mas los males dela pena estonçe son dados conueniblemente & et ii \textbf{ quorum talia est distribuere , | bonis et dignis sunt debite gratiosi , } mala autem poenae tunc debite infliguntur , \\\hline
1.3.11 & e non sobre los que sufren mal a su culpa . \textbf{ Mas la uerguença e la nemessis commo quier que parescan passiones de loar } Enpero non son sinplemente de loar & quando semper indigne patientibus ad misericordiam commouentur . Verecundia autem , \textbf{ et nemesis , licet videantur esse laudabiles passiones , } non tamen simpliciter \\\hline
1.3.11 & Mas la uerguença e la nemessis commo quier que parescan passiones de loar \textbf{ Enpero non son sinplemente de loar } nin en toda manera son de auer alos Reyes e alos prinçipes . Ca non les conuiene alos Reyes & et nemesis , licet videantur esse laudabiles passiones , \textbf{ non tamen simpliciter } et per omnem modum attribuendae sunt Regibus \\\hline
1.3.11 & Enpero non son sinplemente de loar \textbf{ nin en toda manera son de auer alos Reyes e alos prinçipes . Ca non les conuiene alos Reyes } e alos prinçipes de ser uergonçosos & non tamen simpliciter \textbf{ et per omnem modum attribuendae sunt Regibus } et Principibus . Non decet enim Reges verecundos esse , quia non decet eos talia operari unde verecundari possint ; \\\hline
1.3.11 & e alos prinçipes de ser uergonçosos \textbf{ por que non les conuiene a ellos de obrar tales obras } donde puedan resçebir uirguença & et per omnem modum attribuendae sunt Regibus \textbf{ et Principibus . Non decet enim Reges verecundos esse , quia non decet eos talia operari unde verecundari possint ; } propter quod dicitur 4 Ethicorum , \\\hline
1.3.11 & por que non les conuiene a ellos de obrar tales obras \textbf{ donde puedan resçebir uirguença } por la qual cosa es dicho en el quarto libro delas ethicas & et per omnem modum attribuendae sunt Regibus \textbf{ et Principibus . Non decet enim Reges verecundos esse , quia non decet eos talia operari unde verecundari possint ; } propter quod dicitur 4 Ethicorum , \\\hline
1.3.11 & Ca non cuydamos \textbf{ que conuenga aellos de obrar ninguna cosa } en que caya uerguença . & si verecundabilis sit : \textbf{ quia non existimamus ipsum oportere operari , } in quibus est verecundia . \\\hline
1.3.11 & y que los estudiosos non deuen ser uirgon cosos . \textbf{ por que la uerguença es delas cosas malas . Mas al estudioso non conuiene obrar ningunas cosas malas . } Por la qual cosa si conuiene alos Reyes & quod studiosi non est verecundari , \textbf{ quia verecundia est in prauis : eius autem non est praua operari . } Quare si decet Reges esse studiosos , et esse senes moribus , non decet ipsos verecundari , \\\hline
1.3.11 & si non por alguna condiçion . \textbf{ Ca si les contesçiesse a ellos de obrar algunas cosas torpes e malas } deuen se enuergonçar . & nisi ex suppositione : \textbf{ nam si contingeret eos operari turpia , verecundari deberent . Nemesis etiam non multum videtur esse laudabilis , } nec multum laudatur \\\hline
1.3.11 & Ca si les contesçiesse a ellos de obrar algunas cosas torpes e malas \textbf{ deuen se enuergonçar . } Ahun la nemessis non paresçe mucho & nisi ex suppositione : \textbf{ nam si contingeret eos operari turpia , verecundari deberent . Nemesis etiam non multum videtur esse laudabilis , } nec multum laudatur \\\hline
1.3.11 & Ahun la nemessis non paresçe mucho \textbf{ que sea de loar en los Reyes . } por que non es mucho de loar & ø \\\hline
1.3.11 & que sea de loar en los Reyes . \textbf{ por que non es mucho de loar } aquel que es muy desdennos & nam si contingeret eos operari turpia , verecundari deberent . Nemesis etiam non multum videtur esse laudabilis , \textbf{ nec multum laudatur } qui nimis indignatur de prosperitatibus malorum . \\\hline
1.3.11 & o delas bien andanças de lons malos \textbf{ porque los malos non pueden auer grandes bienes } assi commo son las uirtudes . & qui nimis indignatur de prosperitatibus malorum . \textbf{ Nam mali non possunt possidere maxima bona , } cuiusmodi sunt virtutes : \\\hline
1.3.11 & assi commo son las uirtudes . \textbf{ Mas por auentra a pueden auer bienes medianeros o bienes muy pequanos } los quales son bienes de fuera . & cuiusmodi sunt virtutes : \textbf{ sed forte possidere possunt bona media , | vel bona minima , } cuiusmodi sunt bona exteriora . \\\hline
1.3.11 & e por que son en linage de los bienes \textbf{ non deuemos auer cura dellos } si los ouieten los malos & quae minima sunt in genere bonorum , \textbf{ non multum curandum est } si possidentur a malis : \\\hline
1.3.11 & por su culpa propria \textbf{ Porque si parte nesçiesse a alguno de dispensar } e partir estos bienes & dum tamen hoc accidat sine culpa propria ; \textbf{ nam si alicuius esset haec bona dispensare , } non deberet ea retribuere indignis , \\\hline
1.3.11 & Porque si parte nesçiesse a alguno de dispensar \textbf{ e partir estos bienes } non los deuia dar alos non dignos & dum tamen hoc accidat sine culpa propria ; \textbf{ nam si alicuius esset haec bona dispensare , } non deberet ea retribuere indignis , \\\hline
1.3.11 & e partir estos bienes \textbf{ non los deuia dar alos non dignos } mas alos dignos ¶ & nam si alicuius esset haec bona dispensare , \textbf{ non deberet ea retribuere indignis , } sed dignis . Reges ergo et Principes in tantum debent esse nemesiti , et indignari debent \\\hline
1.3.11 & e los prinçipes en tanto deuen ser nemessicos \textbf{ que quiere dezer desdennosos . } Et en tanto deuen ser manssos & ø \\\hline
1.3.11 & que sean desdennosos dela bien andança de los malos \textbf{ en quanto ellos non deuen partir los sus bienes alos malos . } nin alos que non son dignos . & de prosperitatibus malorum , \textbf{ ne indignis distribuant sua bona . } Sic ergo se habere debent ad praedictas passiones , \\\hline
1.3.11 & Et por ende \textbf{ assi se deuen auer los Reyes alas pasiones sobredichos } que paresçen de ser loadas & ne indignis distribuant sua bona . \textbf{ Sic ergo se habere debent ad praedictas passiones , } quae videntur esse laudabiles . Inuidiam autem , \\\hline
1.3.11 & que paresçen de ser loadas \textbf{ Mas deuen ellos foyr en todas maneras dela inuidia } que es passion de denostar & Sic ergo se habere debent ad praedictas passiones , \textbf{ quae videntur esse laudabiles . Inuidiam autem , } quae est vituperabilis passio , \\\hline
1.3.11 & Mas deuen ellos foyr en todas maneras dela inuidia \textbf{ que es passion de denostar } Et avn deuen foyr dela mal querençia & quae videntur esse laudabiles . Inuidiam autem , \textbf{ quae est vituperabilis passio , } penitus fugere debent : \\\hline
1.3.11 & que es passion de denostar \textbf{ Et avn deuen foyr dela mal querençia } si non fuesse de pecados o de males . & quae est vituperabilis passio , \textbf{ penitus fugere debent : | et etiam odium , } nisi esset vitiorum : \\\hline
1.3.11 & si non fuesse de pecados o de males . \textbf{ Ca los pecados londe mal querer e aborresçer } e son de deraygar por toda su fuerça & nisi esset vitiorum : \textbf{ nam vitia sunt odienda , } et sunt pro viribus extirpanda . In aliis autem passionibus , \\\hline
1.3.11 & Ca los pecados londe mal querer e aborresçer \textbf{ e son de deraygar por toda su fuerça } Mas en las o tris passiones & nam vitia sunt odienda , \textbf{ et sunt pro viribus extirpanda . In aliis autem passionibus , } quae possunt esse laudabiles , \\\hline
1.3.11 & Mas en las o tris passiones \textbf{ que pueden ser de loar } e de denostar & et sunt pro viribus extirpanda . In aliis autem passionibus , \textbf{ quae possunt esse laudabiles , } et vituperabiles , debite se habebunt , \\\hline
1.3.11 & que pueden ser de loar \textbf{ e de denostar } conueniblemente se aueran los Reyes & et sunt pro viribus extirpanda . In aliis autem passionibus , \textbf{ quae possunt esse laudabiles , } et vituperabiles , debite se habebunt , \\\hline
1.3.11 & que es osado en las cosas que deue e es temeroso en las cosas \textbf{ que deue temer . } Mas por la grandeza & qui audet audenda , \textbf{ et timet timenda . Per magnanimitatem , } et humilitatem bene se habebunt circa spem , \\\hline
1.3.11 & nin nos tiremos de los bienes grandes e altos \textbf{ por guaueza de lons ganar . } Mas la humildat tienpra la espança & Nam magnanimitas reprimit desperationem , \textbf{ ne retrahantur a bonis arduis propter difficultatem . Humilitas vero moderabit spem , } ne nimis se ingerant ad illa , propter bonitatem , \\\hline
1.3.11 & Mas la humildat tienpra la espança \textbf{ por que non se ꝑueda el omne mucho temer adelante en aquellas cosas } por la bondat e delectaçion & ne retrahantur a bonis arduis propter difficultatem . Humilitas vero moderabit spem , \textbf{ ne nimis se ingerant ad illa , propter bonitatem , } et delectationem , \\\hline
1.3.11 & e aborrezcan las colas \textbf{ que son de abortesçer } e se delecten e se entristezcan en las cosas & et abominentur abominanda : \textbf{ delectentur , } et tristentur , \\\hline
1.3.11 & e se delecten e se entristezcan en las cosas \textbf{ que deuen delectar e entsteçer . } Et a todas estas cosas se deuen auer & et tristentur , \textbf{ ut est delectandum , et tristandum : } et ad haec omnia se habeant , \\\hline
1.3.11 & que deuen delectar e entsteçer . \textbf{ Et a todas estas cosas se deuen auer } assi commo la orden & ut est delectandum , et tristandum : \textbf{ et ad haec omnia se habeant , } ut requirit ordo , \\\hline
1.3.11 & segunt que conuiene a cada cosa . \textbf{ Cay mostraremos en qual manera los Reyes se deuen auer } por que sean temidos de lons pueblos e amados . & Agetur enim in tertio de timore , amore , misericordia , et de aliis , \textbf{ ut rei cognoscentia postulabit . Ibi enim ostendemus , quomodo Reges et Principes se habere debeant , ut a populis timeantur , et amentur : } et quomodo debent esse misericordes : \\\hline
1.3.11 & e tctadas generalmente \textbf{ e vniuersalmente ally seran mostradas particular menter cada vna } por si & quae hic uniuersaliter sunt tradita , \textbf{ ibi particulariter ostendentur . | QUARTA PARS Primi Libri de regimine Principum . In qua tractatur , } qui sunt mores Iuuenum , Senum , \\\hline
1.4.1 & de quales utudes deuen ser honrrados \textbf{ e quels passiones deuen segnir ¶ } finça de dezir dela quarta parte . & et quibus virtutibus debeant esse ornati : \textbf{ et quas passiones debent sequi . Restat exequi de parte quarta , } videlicet quos , \\\hline
1.4.1 & e quels passiones deuen segnir ¶ \textbf{ finça de dezir dela quarta parte . } Conuiene sab quales costunbres deuen auer & et quibus virtutibus debeant esse ornati : \textbf{ et quas passiones debent sequi . Restat exequi de parte quarta , } videlicet quos , \\\hline
1.4.1 & finça de dezir dela quarta parte . \textbf{ Conuiene sab quales costunbres deuen auer } e segnir los prinçipes e los Reyes . & et quas passiones debent sequi . Restat exequi de parte quarta , \textbf{ videlicet quos , } mores debeant imitari . \\\hline
1.4.1 & Conuiene sab quales costunbres deuen auer \textbf{ e segnir los prinçipes e los Reyes . } Mas las costunbt̃s en dos maneras se pueden departir & videlicet quos , \textbf{ mores debeant imitari . } Mores autem dupliciter diuersificari possunt : \\\hline
1.4.1 & e segnir los prinçipes e los Reyes . \textbf{ Mas las costunbt̃s en dos maneras se pueden departir } ¶ & mores debeant imitari . \textbf{ Mores autem dupliciter diuersificari possunt : } ab aetate , \\\hline
1.4.1 & Por la hedat ca vnas costunbres han los mançebos \textbf{ e otras costunbres han los uieios . Mas avn por la uentura se pueden departir las costunbres . } Ca o tris costunbres han aquellos que estan en estado de buena uentura . & quia alios mores habent iuuenes , \textbf{ alios senes , } alios illi qui sunt in statu fortunae . \\\hline
1.4.1 & Mas primero diremos delas costunbres de los mançebos \textbf{ de los quales alguas costunbres son de loar } e algunas de denostar . & Sed primo de moribus iuuenum , \textbf{ quorum mores quidam sunt laudabiles , } quidam vituperabiles . \\\hline
1.4.1 & de los quales alguas costunbres son de loar \textbf{ e algunas de denostar . } Mas entre las otras cosas & quorum mores quidam sunt laudabiles , \textbf{ quidam vituperabiles . } Inter alia quidem quae tangit Philosophus de iuuenibus 2 Rhetoricorum , \\\hline
1.4.1 & en el segundo de la rectorica \textbf{ tanne seys costunbres de loar } e seys de denostar ¶ & ø \\\hline
1.4.1 & tanne seys costunbres de loar \textbf{ e seys de denostar ¶ } Ca primero son las costunbres de los mançebos de loar & Inter alia quidem quae tangit Philosophus de iuuenibus 2 Rhetoricorum , \textbf{ tangit sex mores laudabiles , et sex vituperabiles . Primo enim sunt iuuenes moris laudabilis , } quia sunt liberales . Secundo , \\\hline
1.4.1 & e seys de denostar ¶ \textbf{ Ca primero son las costunbres de los mançebos de loar } por que son liberales ¶ & Inter alia quidem quae tangit Philosophus de iuuenibus 2 Rhetoricorum , \textbf{ tangit sex mores laudabiles , et sex vituperabiles . Primo enim sunt iuuenes moris laudabilis , } quia sunt liberales . Secundo , \\\hline
1.4.1 & e segunt cursso natural deuen much beuir en el tienpo \textbf{ que ha de uenir . Et pues que assi es commo la memoria sea en conparaçion del tienpo passado } por que es recordaçion delas cosas que passaron & secundum cursum naturalem debent multum viuere in futuro . \textbf{ Cum ergo memoria sit respectu praeteritorum , } et spes respectu futurorum : \\\hline
1.4.1 & e la esperança es en conparacion deltp̃o \textbf{ que ha de uenir . } por ende los mançebos poco biuen en memoria & Cum ergo memoria sit respectu praeteritorum , \textbf{ et spes respectu futurorum : } iuuenes parum viuunt memoria , \\\hline
1.4.1 & Mas mucho se delectan en cuydando aquellas cosas \textbf{ que han de fazer } por que esperan que han de fazer grandes cosas . & sed multum delectantur in cogitando , \textbf{ quae facturi sunt . } Sperant enim se magna facere , \\\hline
1.4.1 & que han de fazer \textbf{ por que esperan que han de fazer grandes cosas . } Por la qual cosa contesce & quae facturi sunt . \textbf{ Sperant enim se magna facere , } quare contingit eos animosos esse , \\\hline
1.4.1 & por que se tiene por digno para g̃ndescolas \textbf{ e entremetesse de fazer grandes cosas . } Et pues que assi es commo los mancebos non ayan ninguna cosa & cuius causa ex praecedentibus assignatur . Nam ex hoc est quis magnanimus , \textbf{ quia dignificat se magnis , et ingerit se ad faciendum magna . Iuuenes ergo , cum sint liberales , et cum sint animosi } et bonae spei , \\\hline
1.4.1 & por ende son animolos e de grand esperança . \textbf{ Et avn podemos aesto adozir otra razon espeçial . } Ca por que los mançebos son muy calientes & unde retrahantur quin sint magnanimi . \textbf{ Posset | etiam ad hoc specialis ratio assignari . } Nam cum iuuenes sint percalidi , \\\hline
1.4.1 & Ca por que los mançebos son muy calientes \textbf{ e la calentura quiere sienpre sobir arriba e sobrepuiar . } Por ende los mançebos sienpre quieren sobir e sobrepuiar & Nam cum iuuenes sint percalidi , \textbf{ et calidi sit superferri : } iuuenes semper volunt superferri , \\\hline
1.4.1 & e la calentura quiere sienpre sobir arriba e sobrepuiar . \textbf{ Por ende los mançebos sienpre quieren sobir e sobrepuiar } Ca & et calidi sit superferri : \textbf{ iuuenes semper volunt superferri , } et excellere . Sic enim videmus in ordine uniuersi , \\\hline
1.4.1 & Et pues que assi es conmo entre todas las cosas \textbf{ por las quales cada vno quiere sobir } e sobrepuiar se a la eglesia e la honrra los mançebos & Cum ergo inter caetera , \textbf{ per quae quis videtur superferri et excellere , } sit honor , et gloria : \\\hline
1.4.1 & por las quales cada vno quiere sobir \textbf{ e sobrepuiar se a la eglesia e la honrra los mançebos } por que son muy calientes & per quae quis videtur superferri et excellere , \textbf{ sit honor , et gloria : } iuuenes \\\hline
1.4.1 & por que son muy calientes \textbf{ e codiçian sobrepuiar . } mayormente dessean la eglesia e la honrra . & iuuenes \textbf{ quia sunt percalidi , et cupiunt excellere , maxime desiderant gloriam , et honorem : } et per consequens aliquo modo sunt magnanimi , \\\hline
1.4.1 & delos quales mag̃nimos la proprea materia es la honrra \textbf{ ¶Lo quarto los mançebos son de loar } quando non son maliçiosos de uoluntad & cuius propria materia videtur esse honor . \textbf{ Quarto iuuenes sunt laudabiles , } quia non sunt maligni moris , \\\hline
1.4.1 & ¶Lo sexto los mançebos son uergonosos \textbf{ por que cada vno teme perder aquello } que mucho dessea . & erubescitiui et verecundi . \textbf{ Nam quilibet timet perdere , } quod nimis affectat . \\\hline
1.4.1 & por que los mançebos \textbf{ por la calentura natural desse an sobrepuiar } los otros temen & Cum ergo iuuenes , \textbf{ qui percalidi nimis affectent excellere , } timent inglorificari , \\\hline
1.4.1 & Et por que la uerguença es temor \textbf{ de non auer eglesia } nin honrra los mançebos de ligero toman uerguença & et quia erubescentia est timor inglorificationis , \textbf{ iuuenes de facili erubescunt . Viso } qui mores sunt laudabiles de iuuenibus , \\\hline
1.4.1 & nin honrra los mançebos de ligero toman uerguença \textbf{ ¶ visto quales costunbres son de loar } en los mançebos de ligero puede paresçer & iuuenes de facili erubescunt . Viso \textbf{ qui mores sunt laudabiles de iuuenibus , } de leui patere potest , \\\hline
1.4.1 & ¶ visto quales costunbres son de loar \textbf{ en los mançebos de ligero puede paresçer } en qual manera los Reyes & qui mores sunt laudabiles de iuuenibus , \textbf{ de leui patere potest , } quomodo Reges et Principes se debeant habere ad illos . \\\hline
1.4.1 & e los prinçipesse de una auera aquellas costunbres . \textbf{ Ca algunas costunbres son de loar en los mançebos } que non son de loar en los uieios nin en los Rleyes . & ø \\\hline
1.4.1 & Ca algunas costunbres son de loar en los mançebos \textbf{ que non son de loar en los uieios nin en los Rleyes . } Mas si qual quier cosa & quomodo Reges et Principes se debeant habere ad illos . \textbf{ Nam non quicquid est laudabile in hoc , } est laudabile simpliciter : \\\hline
1.4.1 & Mas si qual quier cosa \textbf{ que es de loar en ssi e sinplemente en el omne es de loar } e en qual si quier omne . & Nam non quicquid est laudabile in hoc , \textbf{ est laudabile simpliciter : } uidemus enim quod esse furibundum , \\\hline
1.4.1 & Mas qual si quier cosa \textbf{ que sea de loar en vno } e non en otro o es de loar & est laudabile simpliciter : \textbf{ uidemus enim quod esse furibundum , } est laudabile in cane , \\\hline
1.4.1 & que sea de loar en vno \textbf{ e non en otro o es de loar } por alguna condicion non es de loar sinple mente . & est laudabile simpliciter : \textbf{ uidemus enim quod esse furibundum , } est laudabile in cane , \\\hline
1.4.1 & e non en otro o es de loar \textbf{ por alguna condicion non es de loar sinple mente . } Ca ueemos que ser sanudo es de loar en el can . & uidemus enim quod esse furibundum , \textbf{ est laudabile in cane , } non tamen est laudabile in homine . Sic , \\\hline
1.4.1 & por alguna condicion non es de loar sinple mente . \textbf{ Ca ueemos que ser sanudo es de loar en el can . } Empero non es de loar en el omne & uidemus enim quod esse furibundum , \textbf{ est laudabile in cane , } non tamen est laudabile in homine . Sic , \\\hline
1.4.1 & Ca ueemos que ser sanudo es de loar en el can . \textbf{ Empero non es de loar en el omne } En essa misma manera & est laudabile in cane , \textbf{ non tamen est laudabile in homine . Sic , } licet uerecundari sit laudabile in iuuenibus , \\\hline
1.4.1 & En essa misma manera \textbf{ maguer ser uergonçoso sea de loar en los mançebos } por razon delan hedat & non tamen est laudabile in homine . Sic , \textbf{ licet uerecundari sit laudabile in iuuenibus , } quia ratione aetatis se continere non possunt quin committant aliqua turpia , de quibus decet eos uerecundari : \\\hline
1.4.1 & por razon delan hedat \textbf{ que se non pueden contener } que non acometan algunas cosas torpes & licet uerecundari sit laudabile in iuuenibus , \textbf{ quia ratione aetatis se continere non possunt quin committant aliqua turpia , de quibus decet eos uerecundari : } Reges tamen et Principes , \\\hline
1.4.1 & que tomne uerguença . \textbf{ Enpero esto non es de loar en los uieios nin en los Reyes . } por que los Reyes e los prinçipes alos quales conuiene de ser & ø \\\hline
1.4.1 & assi commo medios dioses \textbf{ non solamente non les conuiene de fazer cosas torpes } mas avn deuen aborresçer delas oyr nonbrar & quasi semideos , \textbf{ non solum quod turpia committant , } sed abominabile eis esse debet quod audiant turpia nominari : \\\hline
1.4.1 & non solamente non les conuiene de fazer cosas torpes \textbf{ mas avn deuen aborresçer delas oyr nonbrar } por que las malas palauras & non solum quod turpia committant , \textbf{ sed abominabile eis esse debet quod audiant turpia nominari : } quia corrumpunt bonos mores colloquia praua . \\\hline
1.4.1 & que ellos obrassen algunas cosas \textbf{ torꝑes deuen auer uerguença ahun } mas que los otros & quia si contingeret eos operari turpia , \textbf{ uerecundari deberent } etiam plus quam alii , \\\hline
1.4.1 & mas que los otros \textbf{ por que peor cae a ellos de fazer mal que a otros . } Et pues que assi es las primeras çinco cosas & etiam plus quam alii , \textbf{ eo quod magis indecenter se gererent . Prima ergo quinque , } quae diximus laudabilia in iuuenibus , \\\hline
1.4.1 & que dixiemos \textbf{ que eran de loar en los mançebos podemos las } e conueniblemente apprear alos Reyes e alos prinçipes & eo quod magis indecenter se gererent . Prima ergo quinque , \textbf{ quae diximus laudabilia in iuuenibus , } adaptare possumus Regibus et Principibus : \\\hline
1.4.1 & que eran de loar en los mançebos podemos las \textbf{ e conueniblemente apprear alos Reyes e alos prinçipes } por que conuiene aellos de ser liberales & quae diximus laudabilia in iuuenibus , \textbf{ adaptare possumus Regibus et Principibus : } quia decet eos esse liberales , \\\hline
1.4.1 & e de ser mibicordiosos . \textbf{ Mas la sexta condicion conuiene a saber ser uergoncosos . } Esta non conuiene nin pertenesçe & ø \\\hline
1.4.1 & e por las quales los preçian \textbf{ en las non espender en vsos o en obras conuenibles e piadosas } assi commo dessuso dixiemos & qua pollent , \textbf{ non multiplicarent in debitos | et pios usus , } ut supra in tractatu de liberalitate sufficienter tetigimus . \\\hline
1.4.1 & en quanto los fechos comunes çerca \textbf{ los quales ellos se deuen trabaiar } son mas dignos que los otros ¶ & tanto magis esse bonae spei quam alios , \textbf{ quanto facta communia circa quae insudare debent , } sunt digniora quam alia . \\\hline
1.4.1 & por que mucho conuiene a ellos \textbf{ de obrar grandes cosas } e entender cerca las cosas altas ¶ & ( ut dicebatur in quodam capitulo de magnanimitate ) maxime magnanimitas competit Regibus et Principibus , \textbf{ quia eos maxime magna decet operari , } et in ardua tendere . Sic etiam congruum est eos non esse maligni moris , \\\hline
1.4.1 & de obrar grandes cosas \textbf{ e entender cerca las cosas altas ¶ } An en essa misma manera conuiene aellos & ø \\\hline
1.4.2 & ssi commo dessuso contamos seys costunbres de los mançebos \textbf{ que sonb de loar . } Assi podemos contar seys costunbres & Sicut supra enumerauimus \textbf{ ipsorum iuuenum sex mores laudabiles : } sic enumerare possumus sex vituperabiles : \\\hline
1.4.2 & que sonb de loar . \textbf{ Assi podemos contar seys costunbres } que son de denostar & ipsorum iuuenum sex mores laudabiles : \textbf{ sic enumerare possumus sex vituperabiles : } quas etiam tangit Philosophus 2 Rhetoricorum . \\\hline
1.4.2 & Assi podemos contar seys costunbres \textbf{ que son de denostar } las quales pone el philosofo en el segundo libro de la rectorica¶ & ipsorum iuuenum sex mores laudabiles : \textbf{ sic enumerare possumus sex vituperabiles : } quas etiam tangit Philosophus 2 Rhetoricorum . \\\hline
1.4.2 & e mayormente siguen las cobdiçias dela carne \textbf{ por que non se pueden contener } e siguen la lux̉ia & sed omnia faciunt valde . Sunt enim primo iuuenes insecutores passionum , \textbf{ et maxime insequuntur concupiscentias circa corpus . Sunt enim incontinentes } et insecutores venereorum , \\\hline
1.4.2 & e muy afincadamente quieren \textbf{ mas de ligero se tristornan¶ } Lo tercero son muy creyentes & vehementer volunt , \textbf{ sed de facili permutantur . } Tertio sunt nimis creditiui : \\\hline
1.4.2 & e les proponen algun negoçio \textbf{ non pueden catar a muchͣs cosas } por que non son sabios de muchos negoçios & statim cum eis aliquod proponitur negocium , \textbf{ non valentes ad multa respicere , } eo quod sint multorum ignari , \\\hline
1.4.2 & visto quales son las costunbres delos mançebos \textbf{ que son de denostar } de ligero pue de omne ueren qual manera los Reyes et los prinçipes se de una auer atales costunbres . & sed omnia faciunt valde . Viso \textbf{ qui sunt mores iuuenum vituperabiles ; de facili videri potest , } quomodo Reges et Principes \\\hline
1.4.2 & de ligero pue de omne ueren qual manera los Reyes et los prinçipes se de una auer atales costunbres . \textbf{ Ca si tales costunbres son de denostar en los mançebos mucho } mas son de deno star & ad huiusmodi mores debeant se habere . \textbf{ Nam si talia sunt vituperabilia in iuuenibus , } multo magis sunt vituperabilia in adultis : \\\hline
1.4.2 & Et por ende cosa desconuenible es alos Reyes de ser segnidores delas passiones \textbf{ e de auer cobdiçias afincadas de lux̉ia } por que en ellos mucho mas se deue apoderar la razon & qui debent esse caput et regula aliorum . Indecens enim est Reges et Principes esse passionum insecutores , \textbf{ et venereorum habere concupiscentias vehementes : } quia in eis maxime dominari habet ratio , \\\hline
1.4.2 & e de auer cobdiçias afincadas de lux̉ia \textbf{ por que en ellos mucho mas se deue apoderar la razon } e el entendimientoque la passion . & et venereorum habere concupiscentias vehementes : \textbf{ quia in eis maxime dominari habet ratio , } et intellectus . \\\hline
1.4.2 & Lo terçero cosa desconuenible es alos Reyes \textbf{ e alos prinçipes de çreer de ligero . } Ca commo ellos ayan muchos lisongeros & ø \\\hline
1.4.2 & Ca commo ellos ayan muchos lisongeros \textbf{ e muchos les estenruyendo alas oreias deuen penssar con grand acuçia } commo les fabla cada vno & Nam cum multos habeant adulatores , \textbf{ et plurimi sint in eorum auribus susurrantes , | cum maxima diligentia cogitare debent , } qui sunt \\\hline
1.4.2 & e deno stadores \textbf{ por que ellos deuen dar penas } e non deuen dar iniurias & Quarto indecens est eos esse iniuriatores \textbf{ et contumeliosos . Nam poenas inferre debent , } non iniuriam , \\\hline
1.4.2 & por que ellos deuen dar penas \textbf{ e non deuen dar iniurias } nin tuertos nin denuestos nin maldades & et contumeliosos . Nam poenas inferre debent , \textbf{ non iniuriam , } uel contumeliam , \\\hline
1.4.2 & e por el bien comun de todos ¶ \textbf{ Lo quinto es mucho de elquiuar la mentira alos reyes e alos prinçipes } e generalmente a todos los omes & et propter commune bonum . \textbf{ Quinto maxime a Regibus et Principibus , } et uniuersaliter ab omnibus dominantibus , \\\hline
1.4.2 & quanto mas es cosa desconuenible ala Real magestad de ser despreçiada \textbf{ tanto con mayor cautella deue estudiar } que se alleguen ala uerdat & quanto magis indecens est regiam maiestatem contemptibilem esse , \textbf{ tanto maiori cautela studere debent , } ut inhaereant ueritati . \\\hline
1.4.2 & Le seyto cosa desconuenible es alos Reyes \textbf{ non auer manera en las sus obras . } Ca commo todas las sus obras de una ser tenp̃das & ut inhaereant ueritati . \textbf{ Sexto indecens est eos non habere modum in actionibus suis : } quia cum alia sint moderanda per mensuram , \\\hline
1.4.3 & ontadas las costunbres de los mançebos \textbf{ e mostrado en qual manera los Reyes e los prinçipes se deuen auer alas tales costunbres } finca de uer quales son las costunbres & Enumeratis moribus iuuenum , \textbf{ et ostenso quomodo ad mores illos Reges | et Principes se debeant habere . } Restat uidere , \\\hline
1.4.3 & e mostrado en qual manera los Reyes e los prinçipes se deuen auer alas tales costunbres \textbf{ finca de uer quales son las costunbres } que son de denostar en los vieios & et Principes se debeant habere . \textbf{ Restat uidere , } qui sunt mores senum , \\\hline
1.4.3 & finca de uer quales son las costunbres \textbf{ que son de denostar en los vieios } e en qual manera los Reyes e los prinçipes se deuan auer a aquellas costunbres . & Restat uidere , \textbf{ qui sunt mores senum , } et quomodo Reges et Principes se debeant habere ad mores illos . \\\hline
1.4.3 & que son de denostar en los vieios \textbf{ e en qual manera los Reyes e los prinçipes se deuan auer a aquellas costunbres . } Et deuedessaber & qui sunt mores senum , \textbf{ et quomodo Reges et Principes se debeant habere ad mores illos . } Senum autem quidam mores sunt laudabiles , \\\hline
1.4.3 & e en qual manera los Reyes e los prinçipes se deuan auer a aquellas costunbres . \textbf{ Et deuedessaber } que delas costunbres de los uieios algunas son de loar & et quomodo Reges et Principes se debeant habere ad mores illos . \textbf{ Senum autem quidam mores sunt laudabiles , } quidam uituperabiles . Philosophus autem 2 Rhetoricorum , \\\hline
1.4.3 & Et deuedessaber \textbf{ que delas costunbres de los uieios algunas son de loar } e algunas de denostar . & et quomodo Reges et Principes se debeant habere ad mores illos . \textbf{ Senum autem quidam mores sunt laudabiles , } quidam uituperabiles . Philosophus autem 2 Rhetoricorum , \\\hline
1.4.3 & que delas costunbres de los uieios algunas son de loar \textbf{ e algunas de denostar . } Ca el philosofo en el segundo libro de la rectorica & Senum autem quidam mores sunt laudabiles , \textbf{ quidam uituperabiles . Philosophus autem 2 Rhetoricorum , } inter alios mores \\\hline
1.4.3 & que tanne de los uieios cuenta seys costunbres \textbf{ que son de denostar } ¶La primera costunbre denostadera que pone de los uieios es que son mal creyentes & inter alios mores \textbf{ quos tangit de senibus , } enumerat sex uituperabiles mores . \\\hline
1.4.3 & e de los humores apareia la carrera \textbf{ para temer . } Ca segunt el philosofo & quia ( ut dicitur 2 Rhetoricorum . ) Infrigidatio praeparat viam formidini . \textbf{ Nam secundum Philosophum , } Quicunque naturaliter sic disponitur , \\\hline
1.4.3 & que uern que amengua son escassos \textbf{ e non osan espender } Ante ueyendo se assi fallesçer en los cuerpos & quod omnia eis deficiant . Timentes ergo defectum pati , sunt illiberales , \textbf{ et non audent expendere : } immo videntes sic se deficere , non confidunt de propriis viribus , \\\hline
1.4.3 & e non osan espender \textbf{ Ante ueyendo se assi fallesçer en los cuerpos } non fian de su fuerça pprea & et non audent expendere : \textbf{ immo videntes sic se deficere , non confidunt de propriis viribus , } sed solum confidunt \\\hline
1.4.3 & Et por ende poniendo en lo que han su esperança \textbf{ e su fiuza non osan fazer espenssas . } Etrossi son escassos & Ponentes ergo in eis suam spem et confidentiam , \textbf{ non audent expensas facere . } Rursus illiberales sunt ex experientia temporis : \\\hline
1.4.3 & Ca por que biuieron muchos \textbf{ a nons de creer } es que ellos sufrieron muchͣs menguas . & credibile est eos fuisse passos indigentias multas . Timentes ergo indigentiam pati , \textbf{ illiberales fiunt . } Contingit \\\hline
1.4.3 & que han much viuido en elt pon tris passado \textbf{ e han poco de buir en el tr̃o } que ha de uenir & quam spe . Cogitant enim se multum vixisse in praeterito , \textbf{ et parum victuros in futuro . } Ideo \\\hline
1.4.3 & e han poco de buir en el tr̃o \textbf{ que ha de uenir } Et por ende por que la memoria es delas cosas passadas & et parum victuros in futuro . \textbf{ Ideo } quia memoria est praeteritorum , \\\hline
1.4.3 & e la esperança es delas cosas \textbf{ que han de uenir } non biuen por esperança & ø \\\hline
1.4.3 & nin fian de aquellas cosas \textbf{ que deuen ganar en el tienpo } que ha de uenir . & ø \\\hline
1.4.3 & que deuen ganar en el tienpo \textbf{ que ha de uenir . } Mas biuen por memoria & et spes est futurorum : non viuunt spe , nec confidunt de iis quae debent acquirere in futuro , \textbf{ sed viuunt memoria , } et confidunt \\\hline
1.4.3 & por que non fian delas cosas \textbf{ que han de ganar son escassos } e non francos & de iis quae acquisiuerunt in praeterito ; \textbf{ quare non confidentes de acquirendis , } fiunt illiberales , \\\hline
1.4.3 & por que si la espança es delas cosas \textbf{ que han de uenir } e la memoria delas cosas passadas . & sed circa omnia deficere credunt . Causa autem huiusmodi assignata est in praecedentibus . \textbf{ Nam si spes est futurorum , } et memoria praeteritorum , \\\hline
1.4.3 & e creen que viuran poco enel tienpo \textbf{ que ha deuenir . } Por ende los uieios fallesçen en espando & et senes multum vixerunt in praeterito , \textbf{ et parum credunt se viuere in futuro : } senes in sperando deficiunt , \\\hline
1.4.3 & e cuydan \textbf{ que son pocas cosas aquellas que pueden fazer } ca non biuen & senes in sperando deficiunt , \textbf{ et modica se cogitant facturos ; non enim viuunt , } nec delectantur in sperando , \\\hline
1.4.3 & Ca nos veemos \textbf{ que quando los uieios se ayuntan en vno } sienpre cuentan las cosas passadas & quia quilibet libenter tractat ea , \textbf{ in quibus delectatur . Videmus autem quod cum senes adinuicem congregantur , } semper recitant res gestas , \\\hline
1.4.3 & mas non se delectan en contando las cosas \textbf{ que son de fazer . } Las quales cosas ellos han de fazer & non autem delectantur in recitando res fiendas , \textbf{ quas sunt facturi , } eo quod videant se multa fecisse , \\\hline
1.4.3 & que son de fazer . \textbf{ Las quales cosas ellos han de fazer } por que parezca alos omes & non autem delectantur in recitando res fiendas , \textbf{ quas sunt facturi , } eo quod videant se multa fecisse , \\\hline
1.4.3 & e cuyden \textbf{ que han pocas por fazer . } Et por ende son de poca esperança & eo quod videant se multa fecisse , \textbf{ et cogitent se pauca facturos . Sunt ergo difficilis spei , } quia in sperando deficiunt , \\\hline
1.4.3 & por que en esperando fallesçen \textbf{ e esperan de fazer pocas cosas ¶ } Lo sexto los uieios son desuergonçados & quia in sperando deficiunt , \textbf{ et pauca se facere sperant . Sexto senex sunt inuerecundi , } et inerubescitiui . Nam senes quia illiberales sunt , magis curant de utili quam de honesto . \\\hline
1.4.3 & Et mas estudian al prouecho \textbf{ que a auer honrra o estado honrado . } Et assi lo dize el philosofo en el segundo libro dela Rectorica & Magis enim student ad utilitatem , \textbf{ quam ad ea quae requirit honoris status , } ut vult Philosophus 2 Rhetoricorum . \\\hline
1.4.3 & que dela honrra Ca toda la razon \textbf{ por que el omne es uir gonçoso } assi commo dize el philosofo es & quam de honore . Tota enim causa , \textbf{ quare quis est verecundus } ( ut vult Philosophus ) est , \\\hline
1.4.3 & e apretandolas torna las colas mas pesadas \textbf{ e faz las dessear el logar mas bayo . } Ca nos veemos & et constringendo ea , \textbf{ reddit ipsa grauiora , | et facit } ea appetere inferiorem locum . Videmus enim quod elementa frigida et grauia in inferiori loco collocantur : \\\hline
1.4.3 & e son fecho pesados \textbf{ que se non pueden mouer en tal manera } que non osan fazer ninguna cosa & et redduntur immobiles , \textbf{ ut } nihil audeant vel credant ; \\\hline
1.4.3 & que se non pueden mouer en tal manera \textbf{ que non osan fazer ninguna cosa } nin creen ninguna cosa & ut \textbf{ nihil audeant vel credant ; } nihil sperent , \\\hline
1.4.3 & nin de ser tenidos en muchͣ \textbf{ por que la cosa fria non ha de querer logar alto } mas baxo . & nec curent reputari : \textbf{ quia frigidi non est appetere locum superiorem , } sed inferiorem . \\\hline
1.4.3 & visto quales son las costunbres de los uieios \textbf{ que son de denostar de ligero puede paresçe } en qual manera los Reyes e los prinçipes se de una auer atales costunbres . & Viso qui sunt mores senum vituperabiles ; \textbf{ de leui patere potest quomodo Reges et Principes ad huiusmodi se debeant habere . } Nam constat \\\hline
1.4.3 & Mas penssadas las condiconnes delas personas \textbf{ deuen dar fe a aquellas cosas } que les dizen segunt orden de razon & sed consideratis conditionibus personarum , \textbf{ debent adhibere fidem iis quae eis dicuntur } secundum dictamen \\\hline
1.4.3 & las quales los Reyes \textbf{ e los prinçipes se deuen trabaiar } son grandes e altos & circa quae Reges \textbf{ et Principes insudare debent , } sint magna et ardua , \\\hline
1.4.3 & Ca por esto son denostados los uieios \textbf{ e assi non les conuiene aellos de auer uerguenna } por que non les conuiene de obrar cosas torꝑes & Non decet tamen eos verecundari : \textbf{ quia indecens est ipsos operari turpia , } ex quibus verecundia consurgit . \\\hline
1.4.3 & e assi non les conuiene aellos de auer uerguenna \textbf{ por que non les conuiene de obrar cosas torꝑes } delas quales se le unata la uerguença . & Non decet tamen eos verecundari : \textbf{ quia indecens est ipsos operari turpia , } ex quibus verecundia consurgit . \\\hline
1.4.4 & nestas las costunbres de los uieios \textbf{ que non son de loar fincanos de poner las costunbres dellos qson de loar } Mas paresçe que el philosofo en el segundo libro dela rectorica pone quatro costunbres de los uieios & Positis moribus senum vituperabilibus , \textbf{ restat enumerare mores ipsorum laudabiles . } Videtur autem Philosophus 2 Rhetoricorum , \\\hline
1.4.4 & Mas paresçe que el philosofo en el segundo libro dela rectorica pone quatro costunbres de los uieios \textbf{ que pueden ser de loar ¶ } La primera es que los uieios han las cobdiçias botas e tenpradas e flacas ¶ & circa senes tangere quatuor mores , \textbf{ qui possunt esse laudabiles . } Primo enim senes habent concupiscentias remissas , \\\hline
1.4.4 & Et por ende han las cobdiçias dela luxia menguadas e abaxadas e tenpradas . \textbf{ Et desto puede paresçer } en qual manera los uieios son escassos . & et moderatas . \textbf{ Ex hoc autem apparere potest , } quomodo senes sunt illiberales . \\\hline
1.4.4 & en qual manera los uieios son escassos . \textbf{ Ca commo contezca alos omes de pecar } por la escasseza en dos maneras ¶ & ø \\\hline
1.4.4 & mas de quanto demanda la razon¶ \textbf{ La segunda si contra razon dessean auer } lo que non han . & si ultra quam ratio dictet , \textbf{ teneat quod habet . Secundo , } si praeter rationem concupiscat habere quod non habet . \\\hline
1.4.4 & por el appetito \textbf{ para segnir la manera del frio . } Et por que el frio se restune & anima per appetitum inclinatur \textbf{ ut sequatur modum frigidi ; } et \\\hline
1.4.4 & e por ende de ligero se mueuen \textbf{ a auer misicordia dellos . } Mas los iueios non son en esta manera milcdiolos & propter quod de facili \textbf{ miserentur . Senes vero non sunt miseratiui , } quia credant alios bonos esse , \\\hline
1.4.4 & Por ende el se inclina de ligero \textbf{ a auer misconia e piadat delos otros . } Pues que assi es los vieios & ut alii compatiantur ei , et miserentur eius ; propter quod et ipsi de facili inclinantur , \textbf{ ut misereantur , | et compatiantur aliis . } Senes ergo propter imbecillitatem , \\\hline
1.4.4 & por ende de ligero se mueuen ellos \textbf{ a auer piadat e miscderia de los otros ¶ } Lo terçero los uieios non afirman ningunan cosa dubdosa afinçada mente . & et misereri , \textbf{ de facili compatiuntur et miserentur aliis . Tertio nihil dubium pertinaciter affirmant . } Nam \\\hline
1.4.4 & e vieron que fueron muchͣs uezes engannados \textbf{ non osan afirmar ningunan cosa afincandamente } temiendo que el fecho saldra de otra guisa & quod saepe sunt decepti : \textbf{ non audent pertinaciter aliquid asserere , } timentes , \\\hline
1.4.4 & Lo quarto non fazen ninguna cosa con sobrepunaça \textbf{ mas en todas las sus obras quieren paresçer tenprados . } Ca assi conmo los mançebos & Quarto nihil agunt valde , \textbf{ sed in omnibus operibus suis videntur esse temperati . } Nam sicut iuuenes , \\\hline
1.4.4 & Visto quales son las costunbres de los mançebos \textbf{ e de los uieios de ligero puede paresçer } quales son las costunbres de aquellos & qui sunt mores iuuenum , \textbf{ et senum : | de leui apparere potest , } qui sunt mores eorum , \\\hline
1.4.4 & Ca han todas las cosas \textbf{ que son de loar tan bien en los uieios } commo en los mançebos & qui sunt in statu , \textbf{ et sunt medii inter senes , } et iuuenes , \\\hline
1.4.4 & Por ende se siguet̃yendo lo todo a vno \textbf{ que todas las cosas que son de loar en los uieios } e todas las cosas que son de loar en los mançebos todas son falladas & ad unum dicere , \textbf{ quicquid laudabilitatis est in senibus , } vel in iuuenibus , \\\hline
1.4.4 & que todas las cosas que son de loar en los uieios \textbf{ e todas las cosas que son de loar en los mançebos todas son falladas } en los que son en el estado medianero . & quicquid laudabilitatis est in senibus , \textbf{ vel in iuuenibus , } totum reperitur in iis qui sunt in statu . \\\hline
1.4.4 & E todas las colas \textbf{ que son de denostar en ellos todas son arredradas delas medianeras . } Ca assi commo dicho es de suso muchͣs uezes & totum reperitur in iis qui sunt in statu . \textbf{ Et quicquid vituperabilitatis est in eis totum remouetur ab illis . } Nam \\\hline
1.4.4 & Ca assi commo dicho es de suso muchͣs uezes \textbf{ sienpre las cosas estremas son de denostar } e las medianerasson de loar . & Nam \textbf{ ( ut supra pluries dicebatur ) semper extrema sunt vituperabilia , et medium est laudabile . } Si ergo in senibus , \\\hline
1.4.4 & sienpre las cosas estremas son de denostar \textbf{ e las medianerasson de loar . } Et por ende si en los vieios o en los mançebos es alg̃ cosa de loar & Nam \textbf{ ( ut supra pluries dicebatur ) semper extrema sunt vituperabilia , et medium est laudabile . } Si ergo in senibus , \\\hline
1.4.4 & e las medianerasson de loar . \textbf{ Et por ende si en los vieios o en los mançebos es alg̃ cosa de loar } esto es & ( ut supra pluries dicebatur ) semper extrema sunt vituperabilia , et medium est laudabile . \textbf{ Si ergo in senibus , | vel in iuuenibus est aliquid laudabile , } hoc est , \\\hline
1.4.4 & por que non son arredrados del todo del medio . \textbf{ Et assi ahun alguna cosa es en ellos de denostar } esto es porque se parten del medio & quia non omnino recedunt a medio . \textbf{ Si autem est in eis aliquid vituperabile , } hoc est , \\\hline
1.4.4 & Ca qual si quier cosa \textbf{ que es de denostar en ellos } deuemos lo alongar & totum reperiri debet in iis qui sunt in statu . \textbf{ Rursus quia quicquid extremitatis est in eis , } remouetur ab eis \\\hline
1.4.4 & que es de denostar en ellos \textbf{ deuemos lo alongar } e tirar de aquellos que son en estado medianero . & Rursus quia quicquid extremitatis est in eis , \textbf{ remouetur ab eis } qui sunt in statu : quicquid vituperabilitatis est in illis , \\\hline
1.4.4 & deuemos lo alongar \textbf{ e tirar de aquellos que son en estado medianero . } Et pues que & remouetur ab eis \textbf{ qui sunt in statu : quicquid vituperabilitatis est in illis , | remouetur } ab iis \\\hline
1.4.4 & Et pues que \textbf{ assi es en tal manera deuemos fablar delas costunbres delons omes . } Empero non se deue entender & ab iis \textbf{ qui sunt in statu . | Sic ergo loquendum est de moribus hominum . } Non tamen intelligenda sunt \\\hline
1.4.4 & assi es en tal manera deuemos fablar delas costunbres delons omes . \textbf{ Empero non se deue entender } que ayan neçessidat del todo & Sic ergo loquendum est de moribus hominum . \textbf{ Non tamen intelligenda sunt } talia omnino necessitatem habentia , \\\hline
1.4.4 & e que los mançebos non puo dan ser tenpdos e firmes . \textbf{ Mas deuen se entender estas cosas } segunt alguna manera e alguna inclinaçion & et quod iuuenes non possint esse temperati et stabiles : \textbf{ sed intelligenda sunt } secundum quandam pronitatem , \\\hline
1.4.4 & por las cosas que son ya dichans de suso . \textbf{ Et estas cosas uistas de ligero puede paresçer } en qual manera se de una auer los reyes & ad mores eis conuenientes , \textbf{ ut est per habita manifestum . His visis , de leui patere potest , } quomodo Reges et Principes ad huiusmodi mores debeant se habere . \\\hline
1.4.4 & e los prinçipes a estas costunbres . \textbf{ Ca conuiene les de auer las costunbres } que son de loar en los vieios & quomodo Reges et Principes ad huiusmodi mores debeant se habere . \textbf{ Nam mores laudabiles senum } ( secundum quod huiusmodi sunt ) eos habere decet . \\\hline
1.4.4 & Ca conuiene les de auer las costunbres \textbf{ que son de loar en los vieios } en quanto talos costunbres son de loar . & Nam mores laudabiles senum \textbf{ ( secundum quod huiusmodi sunt ) eos habere decet . } Nam cum Reges , \\\hline
1.4.4 & que son de loar en los vieios \textbf{ en quanto talos costunbres son de loar . } Ca commo los Reyes e los prinçipes & Nam mores laudabiles senum \textbf{ ( secundum quod huiusmodi sunt ) eos habere decet . } Nam cum Reges , \\\hline
1.4.4 & por razon \textbf{ que por passion dela carne conuiene les aellos de auer las cobdiçias tenpdas . } Ca assy commo es dicho desuso las cobdiçias & et Principes magis debeant viuere ratione quam passione , \textbf{ decet eos habere concupiscentias temperatas : } quia ( ut supra dicebatur ) concupiscentiae si vehementes sint , \\\hline
1.4.4 & ¶ Lo terçero non conuiene alos Reyes \textbf{ e alos prinçipes de afirmar afincadamente las cosas dubdosas ala vna parte } por que por esto non sean iudgados liuianos & et misericordem . Tertio , \textbf{ non decet Reges , et Principes dubia pertinaciter in alteram partem asserere , } ne per hoc iudicentur leues et indiscreti . \\\hline
1.4.4 & lo quarto los Reyes \textbf{ e los prinçipes deuen auer en las sus obras mesura e tenpramiento } por que assi commo dicho es ellos deuen ser forma de beuir & ne per hoc iudicentur leues et indiscreti . \textbf{ Quarto in suis actionibus debent habere moderationem et temperamentum : } quia ( ut dictum est ) ipsi esse debent forma viuendi , \\\hline
1.4.4 & Et pues que assi es alguas costunbres delons uieios \textbf{ e de los mançebosson bueans e de segnir } e algunas malas e de escusar . & et iuuenum aliqui mores sunt imitandi , \textbf{ aliqui fugiendi . } Sed eorum \\\hline
1.4.4 & e de los mançebosson bueans e de segnir \textbf{ e algunas malas e de escusar . } Mas todas las costunbres de aquellos & et iuuenum aliqui mores sunt imitandi , \textbf{ aliqui fugiendi . } Sed eorum \\\hline
1.4.4 & que son en estado medianero \textbf{ en alguna manera son bueans e de seguir . } Et dezimos en alguna manera & qui sunt in statu , \textbf{ quodammodo omnes mores sunt imitandi . Dicimus autem , quodammodo : } quia sicut senes , \\\hline
1.4.4 & e de deno star . \textbf{ Empero pueden fazer contra aquella disposiconn } e inclina conn natural & et inclinationem ad mores vituperabiles : \textbf{ possunt tamen contra illam pronitatem facere consequi laudabiles mores . } Sic et illi \\\hline
1.4.4 & e inclina conn natural \textbf{ e segnir bueans costunbrs e de loar . } Vien assi ahun aquellos que son en estado medianero maguera & possunt tamen contra illam pronitatem facere consequi laudabiles mores . \textbf{ Sic et illi } qui sunt in statu , \\\hline
1.4.4 & de ssi ayan disposiconn \textbf{ e indinaçion a costunbres bueans e de loar } enpero pueden uenir & qui sunt in statu , \textbf{ et si de se pronitatem habent ad mores laudabiles , } possunt tamen contra istam pronitatem facere , \\\hline
1.4.4 & e indinaçion a costunbres bueans e de loar \textbf{ enpero pueden uenir } e fazer contra esta disposiçion natural & et si de se pronitatem habent ad mores laudabiles , \textbf{ possunt tamen contra istam pronitatem facere , } ut per corruptionem appetitus sequantur \\\hline
1.4.4 & enpero pueden uenir \textbf{ e fazer contra esta disposiçion natural } assi que por la corrupçion del appetito pueden seguir malas costunbres & et si de se pronitatem habent ad mores laudabiles , \textbf{ possunt tamen contra istam pronitatem facere , } ut per corruptionem appetitus sequantur \\\hline
1.4.4 & e fazer contra esta disposiçion natural \textbf{ assi que por la corrupçion del appetito pueden seguir malas costunbres } e de denostar . & possunt tamen contra istam pronitatem facere , \textbf{ ut per corruptionem appetitus sequantur } vituperabiles mores . \\\hline
1.4.4 & assi que por la corrupçion del appetito pueden seguir malas costunbres \textbf{ e de denostar . } Por la qual cosa sinoble cosa es & ut per corruptionem appetitus sequantur \textbf{ vituperabiles mores . } Quare si dignum est dominari rationi , \\\hline
1.4.4 & Por la qual cosa sinoble cosa es \textbf{ e muy digna de enssenorear } por razon e por entendemiento Conuiene alos Reyes e alos prinçipes que son senores de los otros segnir costunbres bueans & Quare si dignum est dominari rationi , \textbf{ et intellectui , } decet Reges , et Principes , \\\hline
1.4.4 & e muy digna de enssenorear \textbf{ por razon e por entendemiento Conuiene alos Reyes e alos prinçipes que son senores de los otros segnir costunbres bueans } e de loar & et intellectui , \textbf{ decet Reges , et Principes , | qui aliis dominantur , } sequi mores laudabiles \\\hline
1.4.4 & por razon e por entendemiento Conuiene alos Reyes e alos prinçipes que son senores de los otros segnir costunbres bueans \textbf{ e de loar } segunt que muestra el entendemiento & qui aliis dominantur , \textbf{ sequi mores laudabiles } secundum dictamen , et ordinem rationis . \\\hline
1.4.5 & e ordena la razon . \textbf{ quanto parte nesçe alo prasente podemos dezir } que quatro son las costunbres bueans & secundum dictamen , et ordinem rationis . \textbf{ Prout ad praesens spectat , | dicere possumus , } ipsorum nobilium \\\hline
1.4.5 & que quatro son las costunbres bueans \textbf{ e de lapña esboar delos nobles omes ¶ } que son de grand coraçon ¶ & dicere possumus , \textbf{ ipsorum nobilium } esse quatuor mores laudabiles . \\\hline
1.4.5 & Ca natural cosa es \textbf{ que sienpre la fechura quiera semeiar a su fazedor } por que los fijos son fechuras de los padron natural cosa es & Naturale est enim , \textbf{ quod semper effectus vult assimilari causae : } cum filii sint quidam effectus parentum , \\\hline
1.4.5 & si de antigo tienpo abondo en riquezas . \textbf{ Et pues que assi es comm sienpre ayamos de dar comienço } en que los padres de alguons comne caron de se enrriqueçer & si ab antiquo affluebat diuitiis . \textbf{ Cum ergo semper sit dare initium , } in quo genitores alicuius ditari inceperunt : \\\hline
1.4.5 & Et pues que assi es comm sienpre ayamos de dar comienço \textbf{ en que los padres de alguons comne caron de se enrriqueçer } quanto mas va descendiendo la generaçion de los fijos & Cum ergo semper sit dare initium , \textbf{ in quo genitores alicuius ditari inceperunt : } quanto magis proceditur per creationem filiorum , \\\hline
1.4.5 & commo la nobleza sienpre incline el coraçon de los nobles \textbf{ para fazer grandes cosas } siguese que los nobles han de ser magnificos & magis antiquatae diuitiae in filiis quam in parentibus : \textbf{ quare cum nobilitas semper inclinet animum nobilium ut faciant magna , } sequitur nobiles esse magnificos , \\\hline
1.4.5 & Et por ende non solamente son magnificos \textbf{ mas avn esfuercan se de acometer mayores fechos } que los padres . & quia quodammodo sunt nobiliores illis . Ideo nobiles non solum sunt magnifici , \textbf{ sed etiam nituntur maiora facere quam parentes . } Unde Philos’ 4 Eth’ ait , \\\hline
1.4.5 & contesçe alos nobles \textbf{ de auer el alma mas apareiada e de ser ellos mas enssennados e mas engennosos que los otros } por que en ellos es la buean disposiconn dela carne & 2 de Anima : \textbf{ contingit nobiles habere mentem aptam , | et esse dociles et industres , } quia in eis viget carnis mollicies , \\\hline
1.4.5 & e escodrinnadores sotilmente de todo aquello \textbf{ que les conuiene de fazer } por que las sus obras & ut sint viri meditatiui , subtiliter inuestigantes \textbf{ quid decet eos facere , } ne opera eorum , \\\hline
1.4.5 & por que las sus obras \textbf{ que veen todos non parescan de reprehender . } Por la qual razon les contesçe a ellos de ser muyen ssenados e sabidores & ne opera eorum , \textbf{ quae multi desiderant , | reprehensibilia videantur ; } quare contingit eos ex diligenti consideratione suorum agibilium esse dociles , \\\hline
1.4.5 & por el penssamiento que han muy acuçioso en todas sus obras \textbf{ que deuen fazer . } Et desto puede paresçer & quare contingit eos ex diligenti consideratione suorum agibilium esse dociles , \textbf{ et industres . } Ex hoc autem apparere potest , \\\hline
1.4.5 & que deuen fazer . \textbf{ Et desto puede paresçer } quanto son de aborresçer los traydores lisongeros & et industres . \textbf{ Ex hoc autem apparere potest , } quanto odiendi sunt adulatores , \\\hline
1.4.5 & Et desto puede paresçer \textbf{ quanto son de aborresçer los traydores lisongeros } que alaban falsamente los fechos de los senores . & Ex hoc autem apparere potest , \textbf{ quanto odiendi sunt adulatores , } omnia aliorum facta commendantes . \\\hline
1.4.5 & si temieren de ser reprehendidos \textbf{ e si temieren de fazer cosas reprehenssibles e si cuydaren sotilmente todo lo que han de fazer } ¶La quatta condicion de los nobles es que son corteses e amigables . & et industres , \textbf{ si timentes reprehensibilia facere , diligenter considerent } quid agendum . Quarto nobiles contingit esse politicos , et affabiles . \\\hline
1.4.5 & ¶La quatta condicion de los nobles es que son corteses e amigables . \textbf{ Ca porque en la mayor parte en las cortes delos nobles es acostunbrado de auer grandes } con p̃anas acahesçeles de ser corteses e aconpanables & quid agendum . Quarto nobiles contingit esse politicos , et affabiles . \textbf{ Nam quia ut plurimum in curiis nobilium consueuit esse magna societas , } conuenit eos esse politicos et sociales , \\\hline
1.4.5 & por que biuen en conpania fazen se conpanonnes e bien fablantes . \textbf{ Mas estas costunbres las quales pueden ser bueans e de loar } e en qual manera conuiene alos Reyes & fiunt sociales et affibiles . Hos autem mores , \textbf{ qui possunt esse boni et laudabiles , } quomodo habere oporteat Reges , et Principes , in antehabitis sufficienter est dictum . Diximus enim supra , \\\hline
1.4.5 & e en qual manera conuiene alos Reyes \textbf{ e alos prinçipes delas auer conplidamente es dich̃o en las cosas sobredichͣs . } Ca ya dixiemos de ssuso & qui possunt esse boni et laudabiles , \textbf{ quomodo habere oporteat Reges , et Principes , in antehabitis sufficienter est dictum . Diximus enim supra , } cum de virtute tractauimus , quomodo decet Reges , et Principes esse magnanimos , quomodo magnificos , \\\hline
1.4.5 & Et en qual manera bien razonados e corteses¶ \textbf{ Visto quales costunbres delos nobłs son de loar } finca de uer quales costunbres son de denostar . & et dociles , \textbf{ et quomodo affabiles et sociales . Viso qui mores nobilium sunt laudabiles : } videre restat \\\hline
1.4.5 & Visto quales costunbres delos nobłs son de loar \textbf{ finca de uer quales costunbres son de denostar . } Ca el philosofo cuenta enel segundo libro dela rectorica dos costunbres & et quomodo affabiles et sociales . Viso qui mores nobilium sunt laudabiles : \textbf{ videre restat | qui mores sunt vituperabiles . } Narrat autem Philosophus 2 Rhetoricorum duos mores vituperabiles , \\\hline
1.4.5 & Ca el philosofo cuenta enel segundo libro dela rectorica dos costunbres \textbf{ que son de denostar . } las quales dize que pertenesçena los nobles ¶ & qui mores sunt vituperabiles . \textbf{ Narrat autem Philosophus 2 Rhetoricorum duos mores vituperabiles , } quos dicit competere ipsius nobilibus . Primus est , \\\hline
1.4.5 & Ca natural cosa es \textbf{ que cada vno quiere ayuntar } e a montonar alguna cosa & Naturale est enim , \textbf{ quod quilibet vult accumulare } ad id quod habet : \\\hline
1.4.5 & que cada vno quiere ayuntar \textbf{ e a montonar alguna cosa } a aquello que ha . & Naturale est enim , \textbf{ quod quilibet vult accumulare } ad id quod habet : \\\hline
1.4.5 & a aquello que ha . \textbf{ Et por ende aquellos que comiençan a enrriqueçerquieten se fazer mas ricos . } Et aquellos que son honrrados quieren se fazer mas honrrados . & ad id quod habet : \textbf{ ideo qui incipiunt ditari , | volunt fieri ditiores : } et qui sunt honorabiles , volunt esse honorabiliores . Nobiles ergo , \\\hline
1.4.5 & Et por ende aquellos que comiençan a enrriqueçerquieten se fazer mas ricos . \textbf{ Et aquellos que son honrrados quieren se fazer mas honrrados . } Et por ende los nobles & volunt fieri ditiores : \textbf{ et qui sunt honorabiles , volunt esse honorabiliores . Nobiles ergo , } quia ex suo genere videntur esse honorabiles , ideo volunt accumulare \\\hline
1.4.5 & por el su linage \textbf{ por ende quieren acresçentar aquella honrra } que han & et qui sunt honorabiles , volunt esse honorabiliores . Nobiles ergo , \textbf{ quia ex suo genere videntur esse honorabiles , ideo volunt accumulare } ad id quod habent , \\\hline
1.4.5 & que han \textbf{ e quieren se fazer mas honrrados et por ende son much desseadores de grant } honrra¶Lo segundo son sobuios & ad id quod habent , \textbf{ et volunt fieri honorabiliores ; | ideo sunt nimis honoris appetitiui . } Secundo sunt elati et despectatores progenitorum : \\\hline
1.4.5 & Mas ser sobrauios \textbf{ e despreçiar los sus engendradores } e ser muy cobdiçiosos de honrra & quia semper est magis antiqua . Esse autem elatum , \textbf{ et despicere suos progenitores , } et nimis esse honoris cupidi , \\\hline
1.4.5 & paresçe de ser malas costunbres \textbf{ por que non deuemos dessear las honrras en lli . } Ca esto fazen los orgullolos et los sob̃uios . & videtur esse mali moris . \textbf{ Non enim debemus appetere ipsos honores in se , } quia hoc faciunt elati et superbi : \\\hline
1.4.5 & Ca esto fazen los orgullolos et los sob̃uios . \textbf{ Mas deuemos dessear las obras } que son dignas de honrra & quia hoc faciunt elati et superbi : \textbf{ sed debemus appetere opera honore digna , } quod faciunt virtuosi et magnanimi . \\\hline
1.4.5 & Et pues que assi es commo los Reyes \textbf{ e los prinçipes non puedan naturalmente ensseñorear } si non fueren bueons e uirtuosos & Reges ergo et Principes , \textbf{ cum non possint naturaliter dominari , } nisi sint boni \\\hline
1.4.5 & si non fueren bueons e uirtuosos \textbf{ conuiene les aellos de segnir las bueans costunbres de los nobles } por que sean de grand coraçon e de grand fazienda & et virtuosi , \textbf{ decet eos sequi bonos mores nobilium , } ut sint magnanimi et magnifici , \\\hline
1.4.5 & e muy sabios e bien razonados . \textbf{ Otrossi les conuiene de foyr malas costunbres } por que non senas obrauios & prudentes et affabiles : \textbf{ et fugere malos mores , } ut non sint elati , \\\hline
1.4.6 & ca por que ellos son loçanos e peleadores e sobrauios \textbf{ e quieren paresçer mas altos } que todos los otros . & Nam si sunt elati , \textbf{ et volunt videri esse excellentes : } cum ex hoc quis excellere videatur , \\\hline
1.4.6 & Et commo por tal razon commo esta alguno parezca de ser mas alto \textbf{ si puede fazer tuertos alos otros } e denostar los . & cum ex hoc quis excellere videatur , \textbf{ si potest aliis contumelias inferre : diuites , } ut videantur aliis praeferri , \\\hline
1.4.6 & si puede fazer tuertos alos otros \textbf{ e denostar los . } Por ende los ricos & cum ex hoc quis excellere videatur , \textbf{ si potest aliis contumelias inferre : diuites , } ut videantur aliis praeferri , \\\hline
1.4.6 & e por que parezcan que son mayores que los otros \textbf{ por ende se mueuen a fazer e a dezer tuertos } e a fablar palabras de denuesto alos otros ¶ & ut videantur aliis praeferri , \textbf{ et ut appareat eos esse excellentiores illis , mouentur , } ut aliis contumelias inferant . Tertio diuites sunt molles \\\hline
1.4.6 & por ende se mueuen a fazer e a dezer tuertos \textbf{ e a fablar palabras de denuesto alos otros ¶ } Lo terçero los ricos son delicados & et ut appareat eos esse excellentiores illis , mouentur , \textbf{ ut aliis contumelias inferant . Tertio diuites sunt molles } et intemperati ; quod eis contingit ex deliciis viuendi . Assueti enim sunt viuere adeo delicate , \\\hline
1.4.6 & por que biuen en deleytes Ca por que son acostunbrados de beuir tan delicadamente \textbf{ que non pueden sofrir } nin gͦs trabaios . & et intemperati ; quod eis contingit ex deliciis viuendi . Assueti enim sunt viuere adeo delicate , \textbf{ quod non possunt aliquas molestias sufferre . } Ideo statim compassionantur , \\\hline
1.4.6 & e siguen sura alas passiones \textbf{ por que non pueden corͣdezir a sus passiones } por que son acostunbrados de beuir delicadamente & ruunt , et fiunt insecutores passionum , \textbf{ non valentes passionibus resistere : } quia assueti sunt viuere delicate , \\\hline
1.4.6 & Et ella respondio \textbf{ que ma veye yr los sabios alas puertas de los ricos } que los ricos alas puertas de los sabios . & fieri diuitem , quam sapientem . Respondit , \textbf{ quod magis videat sapientes frequentare ianuas diuitum , } quam diuites ianuas sapientum . Diuitiae ergo , \\\hline
1.4.6 & assi commo dize el philosofo \textbf{ por que cuydan auer aquello } por que el omne es digno de ser señor . & Nam ( ut Philosophus ait ) \textbf{ habere putant } id quo quis efficitur dignus principari : \\\hline
1.4.6 & Et por ende conuiene alos Reyes \textbf{ e alos prinçipes de foyr } e arredrar dessi estas malas costunbres & Decet ergo Reges , \textbf{ et Principes hos malos mores fugere . } Nam si diuitiis affluunt , \\\hline
1.4.6 & e alos prinçipes de foyr \textbf{ e arredrar dessi estas malas costunbres } por que si ellos son mas abondados de riquezas & Decet ergo Reges , \textbf{ et Principes hos malos mores fugere . } Nam si diuitiis affluunt , \\\hline
1.4.6 & por esso non deuen leuna tarse nin enssoƀueçerse \textbf{ nin por esso non deuen creer } que son dignos de ser sennores . & non debent propter hoc extolli . \textbf{ Nec propter hoc credere debent , } se esse dignos principari . \\\hline
1.4.6 & e ayudan al anr̃a bien andança \textbf{ e son instrumentos e organos para ganarla . } Et ahun fazen las riquezas alguna claridat dela bien andança . & deseruiunt felicitati nostrae , \textbf{ et sunt organa ad ipsam , } et faciunt ad quandam eius claritatem . \\\hline
1.4.6 & Por la qual cosa los Reyes \textbf{ e los prinçipes non son dignos de prinçipar } nin de ser sennores & si sit bonus \textbf{ et prudens , non si diuitiis affluat . Quare Reges et Principes non sunt digni principari , } nisi fugiant malos mores ipsorum diuitum , \\\hline
1.4.6 & nin de ser sennores \textbf{ si non se arte draten delas malas costunbres de los ricos e si non ordenar en las sus riquezas abien o a obras de uirtud ¶ } Visto quales son las malas costunbres de los ricos & nisi fugiant malos mores ipsorum diuitum , \textbf{ et nisi suas diuitias ordinent ad bonum , | et ad opera virtuosa . } Viso \\\hline
1.4.6 & e que conuiene alos Reyes \textbf{ e alos prinçipesarredrar se de tales costunbrs finca de ueer } quales son las bueans costunbres de los ricos . & et quod decet Reges , \textbf{ et Principes fugere tales mores : | videre restat , } qui sunt boni mores eorum . Philosophus autem 2 Rhetoricorum , \\\hline
1.4.6 & por que paresçen biens de auentura non paresçe \textbf{ que abonde la sabiduria del omne para tanto que el se pueda fazer rico } porque veenmos algunas vezes & quod bene se habent circa diuina . Diuitiae enim , \textbf{ quia videntur esse bona fortunae , non videtur sufficere industria humana ad hoc quod aliquis fiat diues : } videmus enim aliquando homines magis industres , \\\hline
1.4.6 & e mas labidores lon menosncos . \textbf{ Et pues que assi es el ganar delas riquezas } assi commo dize el philosofo en el segundo libro dela rectorica & minus ditari . \textbf{ Quia ergo sic est , | acquisitio diuitiarum , } ut dicitur 2 Rhetoricor’ attribuenda est fato , \\\hline
1.4.6 & Et por ende los ricos dignamente \textbf{ e bien se deuen auer çerca las cosas diuinales } e esto es lo que dize el philosofo en el segundo libro de la rectorica & quam sagacitati humanae . \textbf{ Digne ergo diuites circa diuina bene se habere debent . } Hoc est ergo quod dicitur 2 Rhetorico . \\\hline
1.4.6 & e estas riquezas . \textbf{ Et por ende este dicho tan sotil del philosofo deuemos le estudiar con grand acuçia . } Ca las riquezas & tredentes aliqualiter per fata , idest , \textbf{ per ordinationem diuinam habere huiusmodi bona . Hoc autem dictum Philosophicum diligenter est considerandum . } Nam diuitias , \\\hline
1.4.6 & e quales si quier bienes de los omes \textbf{ mas las deuemos aproprear } e retornar al ordenamiento de dios & Nam diuitias , \textbf{ et quaecunque bona habemus , magis habemus referre in diuinum ordinem , et in diuinam prouidentiam , } quam in propriam industriam : \\\hline
1.4.6 & mas las deuemos aproprear \textbf{ e retornar al ordenamiento de dios } e ala prouidençia diuinal & Nam diuitias , \textbf{ et quaecunque bona habemus , magis habemus referre in diuinum ordinem , et in diuinam prouidentiam , } quam in propriam industriam : \\\hline
1.4.6 & tanto mas conuiene alos Reyes e alos prinçipes \textbf{ quanto ellos han de dar cuenta } de mas cosas a dios & et Principes , \textbf{ quanto summo Deo iudici } de pluribus debent reddere rationem . \\\hline
1.4.7 & entre ser noble e ser rico \textbf{ e ahun disferençia ay entre ler noble e rico } e entre ser poderoso . & Differunt ergo esse nobilem , et esse diuitem . Differt etiam esse nobilem , \textbf{ et esse diuitem , } ab esse potentem . \\\hline
1.4.7 & finça \textbf{ de ueer quales son las costunbres delos poderosos . } Mas los poderosos & et qui diuitum restat videre , \textbf{ qui sunt mores potentum . Potentes autem } ( ut vult Philosophus 2 Rhetor’ ) omnino habent meliores mores , \\\hline
1.4.7 & e mas bueons que los ricos . \textbf{ Ca segunt el philosofo costrinnidos son de entender } en aquellas cosas & et magis boni quam diuites , \textbf{ quia secundum Philosophum coacti sunt intendere in ea quae sunt circa potentatum . } Nam qui est in aliquo potentatu , \\\hline
1.4.7 & las quales catan muchos uergunença \textbf{ han en toda manera de arredrar se del medio } en que esta la uirtud & quia est persona communis et publica , \textbf{ ad quam multi respiciunt , verecundatur omnino declinare a medio , } et \\\hline
1.4.7 & en que esta la uirtud \textbf{ e de non fazer obras uirtuosas } e pues que assi es el prinçipado & et \textbf{ non agere opera virtuosa . Ipse ergo principatus quodammodo inducit hominem ad bonitatem , et ad iustitiam . } Studiosiores igitur sunt potentes , \\\hline
1.4.7 & en alguna manera aduze al omne abondat e a iustiçia . Et por ende los poderosos son mas estudiosos que los ricos \textbf{ por que son costrintados de entender çerca las obras del poderio } que deuen ser buenas e estudiosas ¶ & quam diuites : \textbf{ quia coacti sunt intendere circa opera potentatus , } quae debent esse bona et studiosa . \\\hline
1.4.7 & Et por ende los ricos sinon abondan en poderio çiuil \textbf{ e non son costrennidos de entender en obras de iustiçiavagan } e dansse a occiosidat & si non abundent in ciuili potentia , \textbf{ et non cogantur intendere operibus iustitiae , vacant ocio , et de leui inclinantur , } ut dent operam rebus venereis , \\\hline
1.4.7 & e dansse a occiosidat \textbf{ e muy de ligero se inclinan a fazer obras de luxia } et fazen se destenprados . & et non cogantur intendere operibus iustitiae , vacant ocio , et de leui inclinantur , \textbf{ ut dent operam rebus venereis , } et fiant intemperati . Potentes vero et principantes , \\\hline
1.4.7 & Mas los poderosos et los prinçipes \textbf{ por que les conuiene de entender } e auer cuydados de muchͣs cosas retrahen se & et fiant intemperati . Potentes vero et principantes , \textbf{ quia oportet eos intendere exterioribus curis , } retrahuntur , \\\hline
1.4.7 & por que les conuiene de entender \textbf{ e auer cuydados de muchͣs cosas retrahen se } e tiran se & et fiant intemperati . Potentes vero et principantes , \textbf{ quia oportet eos intendere exterioribus curis , } retrahuntur , \\\hline
1.4.7 & non les fazen tuerto en pequennas mas en grandes \textbf{ por que non curan de fazer pequano tuerto } nin pequeno danno . & non iniuriabuntur in paruis , \textbf{ sed in magnis . Non enim curabunt facere paruam offensam , } sed vel in nullo damnificabunt alios , \\\hline
1.4.7 & que los nicos \textbf{ por que non curan de fazer pelea } nin denuesto a ninguno en las pequennas cosas & quam diuites : \textbf{ quia quamlibet contumeliam inferre non curant , } sed solummodo contumeliosi in magnis . \\\hline
1.4.7 & aca es dicho bien auentra adolin lelo . \textbf{ por que non sabe husar delas riquezas } que son instrumentos para auer la bien auenturança & dicitur esse insensatus felix , \textbf{ quia aescit uti diuitiis , } quae sunt organa ad felicitatem ; \\\hline
1.4.7 & por que non sabe husar delas riquezas \textbf{ que son instrumentos para auer la bien auenturança } o es dicho sin seso auenturado & quia aescit uti diuitiis , \textbf{ quae sunt organa ad felicitatem ; } vel dicitur insensatus fortunatus , \\\hline
1.4.7 & o es dicho sin seso auenturado \textbf{ por que non sabesofrir las bien auenturanças } o por que non se sabe auer conueniblemente en las riquezas & vel dicitur insensatus fortunatus , \textbf{ quia nescit fortunas ferre vel nescit debite se habere in diuitiis , } quae sunt bona fortunae . \\\hline
1.4.7 & por que non sabesofrir las bien auenturanças \textbf{ o por que non se sabe auer conueniblemente en las riquezas } que son bienes de auentra a . & vel dicitur insensatus fortunatus , \textbf{ quia nescit fortunas ferre vel nescit debite se habere in diuitiis , } quae sunt bona fortunae . \\\hline
1.4.7 & non son enssennados \textbf{ en qual manera puedan usar de las riquezas . } Por la qual cosa dize el philosofo en el segundo libro dela rectorica & non sunt eruditi , \textbf{ quomodo possunt uti diuitiis ; propter quod in eodem 2 Rhetor’ scribitur , } quod mores diuitiarum \\\hline
1.4.7 & que las costunbres de los ricos \textbf{ por que lo podamos todo dezir en sunma son costunbres de bien auenturͣado sin seso . } Et pues que assi es la nobleza bien se aconpanna alas riquezas & quod mores diuitiarum \textbf{ ( ut in summa sit dicere ) sunt mores insensati felicis . } Nobilitas ergo bene associatur diuitiis : \\\hline
1.4.7 & que es noble \textbf{ e de antigo tienpo los sus auuelos fueron ricos meior sabe sofrir las riquezas } e por ellas non se leu nata en so ƀͣuia & quod tamen est nobilis , \textbf{ et ab antiquo sui progenitores diuites extiterunt , | melius nouit diuitias supportare , } et propter eas non tantum extollitur . \\\hline
1.4.7 & pues que las ha ganadas \textbf{ por que aquel que comiença agostar dela bondat delas uirtudes e dela dulçedunbre delas sçiençias } luego entiende & Nam prius quam habeantur , non reputantur tam magna , sicut postquam habita sunt : \textbf{ qui enim gustare incipit de suauitate virtutum , | et dulcedine scientiarum , } statim percipit ea esse maiora bona , \\\hline
1.4.7 & Et por ende por que ninguno non sea bien auentraado sin seso \textbf{ e non sepa sotrir la buena uentra } a conuiene & Ne igitur quis sit insensatus felix , \textbf{ et nesciat fortunas ferre , } expedit ut diuitias concomitetur nobilitas . Diues enim nobilis et ab antiquo , in omnibus scit melius se habere , quam rusticus ex nouo ditatus . Bene ergo dictum est , \\\hline
1.4.7 & que con las riquezas sea aconpanada la nobleza \textbf{ porque el rico e el noble de antiguedat en todas las cosas se sabe meior auer } que el rustico e el aldeano enrriqueçido de nueuo . & requiritur ut sit nobilis ; \textbf{ sed ne sit intemperatus , } requiritur \\\hline
1.4.7 & por el su prinçipado non pueden \textbf{ assi entender nin se dar alos deleytes dela lux̉ia¶ } pues que assi es los Reyes e los prinçipes & quia diuersis curis intendunt propter principatum , \textbf{ non ita possunt vacare venereis . Reges ergo et Principes , } quia ut plurimum his tribus exterioribus affluunt , \\\hline
1.4.7 & en rriquezas \textbf{ saben meior sofrir las bueans uenturas } e saben mas sabiamente se auer çerca delas rriquezas . & Nam per nobilitatem eo quod ab antiquo abundauerunt diuitiis , \textbf{ sciunt magis fortunas ferre , } et circa diuitias sciunt \\\hline
1.4.7 & e por su poderio \textbf{ conuiene les de auer grandes cuydados } e de se tyrar de obras lux̉iosas & Rursus quia pollent principatu et potentia , \textbf{ et oportet eos diuersis curis intendere , } retrahuntur a venereis , \\\hline
1.4.7 & conuiene les de auer grandes cuydados \textbf{ e de se tyrar de obras lux̉iosas } Et por ende son enduzidos a ser tenprados . & et oportet eos diuersis curis intendere , \textbf{ retrahuntur a venereis , } et inducuntur , \\\hline
1.4.7 & e han grand disposiçion \textbf{ para segnir las costunbres sobredichͣs . } Et por ende non se deuen enssonnar los mançebos & et magnam pronitatem habent , \textbf{ ut sequantur praedictos mores . } Iuuenes ergo \\\hline
1.4.7 & para segnir las costunbres sobredichͣs . \textbf{ Et por ende non se deuen enssonnar los mançebos } nin los vieios & ut sequantur praedictos mores . \textbf{ Iuuenes ergo } et senes non indignentur , \\\hline
1.4.7 & si nos contamos algunos males dellos \textbf{ Ca non entendemos poner ninguna neçessidat aellos } por estas costunbres sobredichͣs & de ipsis narrauimus : \textbf{ quia nullam eis per huiusmodi mores necessitatem imponimus , } quin possint omnes malos mores vitare , \\\hline
1.4.7 & por estas costunbres sobredichͣs \textbf{ que non puedan ellos esquiuar todas estas malas costunbres } e segnir orden de razon e de entendemiento & quia nullam eis per huiusmodi mores necessitatem imponimus , \textbf{ quin possint omnes malos mores vitare , } et sequi ordinem rationis . Sic etiam nec nobiles , \\\hline
1.4.7 & que non puedan ellos esquiuar todas estas malas costunbres \textbf{ e segnir orden de razon e de entendemiento } e auer las bueans . & quin possint omnes malos mores vitare , \textbf{ et sequi ordinem rationis . Sic etiam nec nobiles , } vel diuites indignari debent , \\\hline
1.4.7 & e segnir orden de razon e de entendemiento \textbf{ e auer las bueans . } Ahun en essa mis ma guasa non se deuen enssannar los nobłs & quin possint omnes malos mores vitare , \textbf{ et sequi ordinem rationis . Sic etiam nec nobiles , } vel diuites indignari debent , \\\hline
1.4.7 & e auer las bueans . \textbf{ Ahun en essa mis ma guasa non se deuen enssannar los nobłs } nin los ricos li dellos contamos algunas malas costunbres & et sequi ordinem rationis . Sic etiam nec nobiles , \textbf{ vel diuites indignari debent , } si ipsorum narrauimus aliquos malos mores : \\\hline
1.4.7 & que todos seantales . \textbf{ Mas abasta que aquellas costunbres sean falladas en muchos por que non entendiemosponer en ellos } por estas costunbres neçessidat ninguna mas alguna disposicion e inclinaçion & quia non oportet omnes esse tales , \textbf{ sed sufficit reperiri illud in pluribus : pronitatem enim quandam , et non necessitatem , } per huiusmodi mores in eis ponere intendebamus . Narratis ergo moribus laudabilibus , \\\hline
1.4.7 & por estas costunbres neçessidat ninguna mas alguna disposicion e inclinaçion \textbf{ para auer las ¶ } Et pues que assi es contadas & sed sufficit reperiri illud in pluribus : pronitatem enim quandam , et non necessitatem , \textbf{ per huiusmodi mores in eis ponere intendebamus . Narratis ergo moribus laudabilibus , } et vituperabilibus \\\hline
1.4.7 & Et pues que assi es contadas \textbf{ e dichas las costunbres de loar } e de denostar & per huiusmodi mores in eis ponere intendebamus . Narratis ergo moribus laudabilibus , \textbf{ et vituperabilibus } secundum diuersas aetates , \\\hline
1.4.7 & e dichas las costunbres de loar \textbf{ e de denostar } segunt departidos estados e departidas edades . & per huiusmodi mores in eis ponere intendebamus . Narratis ergo moribus laudabilibus , \textbf{ et vituperabilibus } secundum diuersas aetates , \\\hline
1.4.7 & segunt departidos estados e departidas edades . \textbf{ Conuiene a todos los omes de segnir las costunbres } que son de loar & et secundum diuersos status : \textbf{ decet omnes homines sequi mores laudabiles , } et fugere vituperabiles . \\\hline
1.4.7 & Conuiene a todos los omes de segnir las costunbres \textbf{ que son de loar } e desse arredrar delas & et secundum diuersos status : \textbf{ decet omnes homines sequi mores laudabiles , } et fugere vituperabiles . \\\hline
1.4.7 & que son de loar \textbf{ e desse arredrar delas } que son de denostar . & decet omnes homines sequi mores laudabiles , \textbf{ et fugere vituperabiles . } Sed tanto magis hoc decet Reges , \\\hline
1.4.7 & e desse arredrar delas \textbf{ que son de denostar . } Et tanto mas conuiene esto alos Reyes e alos prinçipes & decet omnes homines sequi mores laudabiles , \textbf{ et fugere vituperabiles . } Sed tanto magis hoc decet Reges , \\\hline
1.4.7 & qual si quier cosa \textbf{ que sea ded en estar loar en las buenas costunbres de los otros } omes todo deue ser fallado en ellos & et forma viuendi ; \textbf{ quicquid ergo laudabilitatis est in moribus singulorum , totum debet in ipsis peramplius et perfectius reperiri . Primi libri de regimine Principis finis , in quo traditum fuit , } quomodo Princeps seipsum regere debeat . D . AEGIDII ROMANI Ordinis Fratrum Eremitarum S . Augustini , \\\hline
2.1.1 & en qual cosa de un a los Reyes \textbf{ e los prinçipes poner su bien andança . } Et quales uirtudes deuen auer . & quia ostensum est , \textbf{ in quo Reges , et Principes debeant suam felicitatem ponere : } quas virtutes habere : \\\hline
2.1.1 & e los prinçipes poner su bien andança . \textbf{ Et quales uirtudes deuen auer . } Et quales costunbres deuen segnir & in quo Reges , et Principes debeant suam felicitatem ponere : \textbf{ quas virtutes habere : } quas passiones sequi : \\\hline
2.1.1 & Et quales uirtudes deuen auer . \textbf{ Et quales costunbres deuen segnir } Et quales costunbres deuen guardar . & quas virtutes habere : \textbf{ quas passiones sequi : } et quos mores debeant imitari . \\\hline
2.1.1 & Et quales costunbres deuen segnir \textbf{ Et quales costunbres deuen guardar . } Ca por estas quatro cosas conplidamente se puede auer & quas passiones sequi : \textbf{ et quos mores debeant imitari . } Per haec enim quatuor , \\\hline
2.1.1 & Et quales costunbres deuen guardar . \textbf{ Ca por estas quatro cosas conplidamente se puede auer } en qual manera deua cada vno gouernar assi mesmo & et quos mores debeant imitari . \textbf{ Per haec enim quatuor , | sufficienter habetur , } qualiter quilibet debeat seipsum regere , \\\hline
2.1.1 & Ca por estas quatro cosas conplidamente se puede auer \textbf{ en qual manera deua cada vno gouernar assi mesmo } e qual deue ser en si mesmo¶ & sufficienter habetur , \textbf{ qualiter quilibet debeat seipsum regere , } et qualis debeat in seipso esse . \\\hline
2.1.1 & e qual deue ser en si mesmo¶ \textbf{ Et pues que assi es finca de dezir del gouernamiento delan conpanna o del gouernamiento dela casa . } Ca abasta que los Reyes e los prinçipes sean bue ons en ssi mismos si non fueren bueons alos otros & et qualis debeat in seipso esse . \textbf{ Restat ergo dicere de regimine familiae , | siue de regimine ipsius domus . } Non enim sufficit , \\\hline
2.1.1 & Ca abasta que los Reyes e los prinçipes sean bue ons en ssi mismos si non fueren bueons alos otros \textbf{ e si non sopiet en gouernar alos otros . } Otrossi non abasta & quod Reges , et Principes sint boni in seipsis , \textbf{ nisi sint boni quo ad alios , } et nisi sciant alios gubernare : \\\hline
2.1.1 & assi mismos conueniblemente \textbf{ sinon sopieren gouernar su casa } e la çibdat e el regno . & nec sufficit , quod Reges debite seipsos regant , \textbf{ nisi regere sciant domum , } ciuitatem , et regnum . \\\hline
2.1.1 & e sea comunidat natraal \textbf{ si queremos fablar dela casa } conuiene nos de ver & et sit communitas naturalis : \textbf{ si de domo determinare volumus , } videndum est \\\hline
2.1.1 & si queremos fablar dela casa \textbf{ conuiene nos de ver } commo se deue auer el omne & si de domo determinare volumus , \textbf{ videndum est } quomodo se habeat homo adesse communicatiuum , \\\hline
2.1.1 & conuiene nos de ver \textbf{ commo se deue auer el omne } para ser comunal con todos e conpanenro . & videndum est \textbf{ quomodo se habeat homo adesse communicatiuum , } et sociale . \\\hline
2.1.1 & para ser comunal con todos e conpanenro . \textbf{ Et por ende conuiene de saber } que el omne sobre todas las otras ainalias ha menester quatro cosas & et sociale . \textbf{ Sciendum igitur , } quod homo ultra alia animalia quatuor indigere videtur ex quibus quadruplici via venari possumus , ipsum esse communicatiuum \\\hline
2.1.1 & que el omne sobre todas las otras ainalias ha menester quatro cosas \textbf{ por las quales podemos prouar en quatro maneras } que el omne es natra alnen te comunal a todos e conpannero ¶ & Sciendum igitur , \textbf{ quod homo ultra alia animalia quatuor indigere videtur ex quibus quadruplici via venari possumus , ipsum esse communicatiuum } et sociale . Prima via sumitur ex victu , \\\hline
2.1.1 & de que se ha omne auestir¶ \textbf{ La terçera por tirar de si todo enbargo } parel qual se puede guardar e defender de los nemigos ¶ & quo homo indiget . Secunda ex vestitu , \textbf{ quo tegitur . Tertia ex remotione prohibentium , per quam ab hostibus liberetur . } Quarta ex disciplina et sermone , \\\hline
2.1.1 & La terçera por tirar de si todo enbargo \textbf{ parel qual se puede guardar e defender de los nemigos ¶ } La quarta es del castigo et dela palaura & quo homo indiget . Secunda ex vestitu , \textbf{ quo tegitur . Tertia ex remotione prohibentium , per quam ab hostibus liberetur . } Quarta ex disciplina et sermone , \\\hline
2.1.1 & por la qual el omne ha de ser castigado e enssennado ¶ \textbf{ Lo primero se declara assi . Ca assi deuemos ymaginar } que la natura non faze ninguna cosa en vano . & Quarta ex disciplina et sermone , \textbf{ per quae instruitur . Sic enim imaginari debemus , } quod natura nihil facit frustra ; \\\hline
2.1.1 & que naturalmente es fecha todas aquellas cosas le son naturales \textbf{ sin las quales non se puede bien guardar en su ser . } Ca la nata en vano faria las cosas & naturalia sunt ea , \textbf{ sine quibus non potest bene conseruari in esse . } Frustra enim natura ageret , \\\hline
2.1.1 & Ca la nata en vano faria las cosas \textbf{ si las cosas naturales en ninguna manera non se pudiessen guardar } en si mesmas e en su ser . & Frustra enim natura ageret , \textbf{ si res naturales nullo modo conseruarentur in esse , } sed statim , \\\hline
2.1.1 & Et en vano si ansi luego \textbf{ que son fechͣs començassen afallesçer . } Por la qual cosa commo el beuir & sed statim , \textbf{ postquam essent factae , esse desinerent ; } quare cum viuere sit homini naturale , omnia illa , \\\hline
2.1.1 & Por la qual cosa commo el beuir \textbf{ sean atuer tal cosa al omne todas aquellas cosas } que fazen a bien beuir & postquam essent factae , esse desinerent ; \textbf{ quare cum viuere sit homini naturale , omnia illa , } quae faciunt ad bene viuere , et sine quibus non potest sibi in vita sufficere , sunt homini naturalia : \\\hline
2.1.1 & que fazen a bien beuir \textbf{ e sin las quales non puede el omne abondar } assi mesmo en la uida son colas naturales al omne & quare cum viuere sit homini naturale , omnia illa , \textbf{ quae faciunt ad bene viuere , et sine quibus non potest sibi in vita sufficere , sunt homini naturalia : } inter alia autem , \\\hline
2.1.1 & Et por ende naturalmente el omne esaianlia aconpanable e conpanera \textbf{ Mas que la conpannia mucho faga a conplimien to deuida de omne puede se demostrar } por aquellas quatro cosas & naturaliter ergo homo est animal sociabile . \textbf{ Quod autem societas maxime faciat ad sufficientiam vitae humanae , } patere potest ex his quatuor supra enumeratis : \\\hline
2.1.1 & Da las yeruas e los fruos \textbf{ de los quales se pueden mantener conplidamente } sin ningun otro apareiamiento . & et caetera talia , ministrat herbas et fructus ; \textbf{ qui absque alia praeparatione sufficiunt ad nutrimentum . } Animalibus vero viuentibus ex rapina , \\\hline
2.1.1 & por que cada vno dellos cunpla el fallimiento del otro \textbf{ Et esto que dicho es del trigo deuemos entender de todas las otras uiandas } por que nunca el omne estando señero puede conplir & ut homo ratione victus sufficiat sibi in vita , indiget societate , ut unusquisque suppleat alterius defectum . \textbf{ Et quod dictum est de frumento , | intelligendum est de cibariis aliis . } numquam cum homo existens solus sufficit sibi ad habendum congrua cibaria , \\\hline
2.1.1 & Et esto que dicho es del trigo deuemos entender de todas las otras uiandas \textbf{ por que nunca el omne estando señero puede conplir } assy mismo & intelligendum est de cibariis aliis . \textbf{ numquam cum homo existens solus sufficit sibi ad habendum congrua cibaria , } quae requiruntur ad vitam . \\\hline
2.1.1 & assy mismo \textbf{ para auer viandas conuenibles } que son men ester para la uida . & numquam cum homo existens solus sufficit sibi ad habendum congrua cibaria , \textbf{ quae requiruntur ad vitam . } Bene ergo dictum est , \\\hline
2.1.1 & La segunda manera \textbf{ para prouar esto mesmo se toma de parte delas vestiduras } de quanos cobrimos . & Secunda via ad inuestigandum hoc idem , \textbf{ sumitur | ex parte indumentorum , } quibus tegimur . \\\hline
2.1.1 & e de los otros destensaientos del tienpo \textbf{ por la qual cosa commo sea conuenible ala uida humanal de auer uianda } e vestida conuenible & quam illa ; \textbf{ quare cum habere victum et vestitum congruat | ad vitam humanam , } et nullus sibi sufficiat sine societate alterius , \\\hline
2.1.1 & Mas si estas cosas son neçessarias \textbf{ para guardar la uida natural delos omes } siguese que beuir en conpannia & ad hoc quod homines perfecte sibi in vita sufficiant . \textbf{ Sed si haec sunt necessaria ad conseruandam hominis naturalem vitam , } viuere in communitate \\\hline
2.1.1 & La tercera manera \textbf{ para prouar esto mesmo se toma por arredrar de ssi aquellas cosas } que enbargarian & et in societate est quodammodo homini naturale . \textbf{ Tertia via ad inuestigandum hoc idem , } sumitur ex remotione prohibentium , \\\hline
2.1.1 & que enbargarian \textbf{ para se guardar e defender de los enemi gos . } Ca por que la natura dio a algunas anmalias & sumitur ex remotione prohibentium , \textbf{ prout ab inimicis , | et ab hostibus defendimur . } Natura cum aliquibus animalibus ad sui tuitionem dedit cornua , \\\hline
2.1.1 & Ca las corcas e las liebres saben \textbf{ que no pueden escapar } por otro camino los periglos dela muerte & ut leporibus \textbf{ et capreis . Sicut enim capreae et lepores } quia non per aliam viam possunt euadere mortis pericula nisi per corporis agilitatem \\\hline
2.1.1 & si non por lignieza del su cuerpo \textbf{ e por foyr . } Et por ende luego que oy en algun roydo luego comiençana foyr . & quia non per aliam viam possunt euadere mortis pericula nisi per corporis agilitatem \textbf{ et per fugam , } ideo statim cum audiunt strepitum , \\\hline
2.1.1 & e por foyr . \textbf{ Et por ende luego que oy en algun roydo luego comiençana foyr . } Mas la natura al omne & et per fugam , \textbf{ ideo statim cum audiunt strepitum , | fugam arripiunt . } Sed natura homini , tanquam excellentiori animalium , \\\hline
2.1.1 & La qual segunt el philosofo en el terçero libro del alma es organo e instrumento sobre todos los instrumentos . \textbf{ Ca por la mano podemos fazer todos los instrumentos } e todo aquello que puede ser & est organum organorum . \textbf{ Nam per manum omnia organa , } et quicquid ad defensionem facit , \\\hline
2.1.1 & para nuestro defendemiento . \textbf{ Par la qual cosa si natural cosa es al omne de dessear conseruaçion } e guarda de su uida & fabricare valemus . \textbf{ Quare si naturale est homini desiderare conseruationem vitae , } cum homo solitarius non sufficiat sibi ad habendum congruum victum \\\hline
2.1.1 & assi mismo \textbf{ para auer uianda conueinble nin uestidura nin para fazer } para si armas & cum homo solitarius non sufficiat sibi ad habendum congruum victum \textbf{ et vestitum , } et ad fabricandum sibi arma et organa , per quae a contrariis defendatur : \\\hline
2.1.1 & e jnstrumentos \textbf{ por los quales se pueda defender de los enemigos . } por ende natural cosa es ael & et vestitum , \textbf{ et ad fabricandum sibi arma et organa , per quae a contrariis defendatur : } naturale est ei , ut desideret viuere in communitate , \\\hline
2.1.1 & por inclinaçion natra al faze su tela conuenible \textbf{ avn que nunca aya visto otras arannas texer } en essa misma manera & ut aranea ex instinctu naturae debitam telam faceret , \textbf{ si nunquam vidisset araneas alias texuisse . Sic etiam et hirundines debite facerent nidum , } si nunquam vidissent alias nidificasse . \\\hline
2.1.1 & avn las golondrinas fazen su nido conueniblemente \textbf{ avn que nunca ayan visto a otras golondinas fazer su nido . Et la perra } por inclinaçion de natura es enssennada & si nunquam vidisset araneas alias texuisse . Sic etiam et hirundines debite facerent nidum , \textbf{ si nunquam vidissent alias nidificasse . } Et canis ex instinctu naturae instruitur , qualiter se debeat habere in partu ; \\\hline
2.1.1 & por inclinaçion de natura es enssennada \textbf{ en qual manera se deua auer en el parto } avn que nunca uiesse o trisperras parir . & si nunquam vidissent alias nidificasse . \textbf{ Et canis ex instinctu naturae instruitur , qualiter se debeat habere in partu ; } si nunquam vidisset canes alias peperisse . Mulier autem cum parit , nescit qualiter se debeat habere in partu , \\\hline
2.1.1 & en qual manera se deua auer en el parto \textbf{ avn que nunca uiesse o trisperras parir . } Mas la mug̃r quan do pare non labe & Et canis ex instinctu naturae instruitur , qualiter se debeat habere in partu ; \textbf{ si nunquam vidisset canes alias peperisse . Mulier autem cum parit , nescit qualiter se debeat habere in partu , } nisi per obstetrices sit sufficienter edocta . \\\hline
2.1.1 & Mas la mug̃r quan do pare non labe \textbf{ en qual manera se deua auer en el parto } si non fuere enssennada conuenibłmente & si nunquam vidisset canes alias peperisse . Mulier autem cum parit , nescit qualiter se debeat habere in partu , \textbf{ nisi per obstetrices sit sufficienter edocta . } Quia ergo homo non sufficienter ex instinctu naturae inclinatur \\\hline
2.1.1 & e el vno resçiba doctrina e enssenança del otro . \textbf{ Et por que esto non se puede fazer } si non biuieremos en vno con los otros omes . & et unus ab alio suscipiat disciplinam . \textbf{ Et quia hoc fieri non potest , } nisi simul \\\hline
2.1.1 & mas es tal commo bestia o tal commo dios . \textbf{ Ca por çierto aquellos que escogen de non beuir con los otros } o esto fazen & aut Deus . \textbf{ Ad literam enim , | eligentes non conuiuere aliis , } vel hoc est , \\\hline
2.1.1 & por que son muy pecadores o muy menguados \textbf{ e non pueden sofrir conpanna de los otros } estonçe son tales commo bestias o por que son muy bueons & quia nimis sunt scelerati , \textbf{ et non possunt societatem aliorum supportare , } et tunc sunt quasi bestiae ; \\\hline
2.1.1 & e mayormente alos Reyes \textbf{ e alos prinçipes parar mientes muy acuçiosamente } en quanto siruela conpannia ala uida humanal & et tunc sunt quasi diuini . Decet ergo omnes homines , \textbf{ et maxime Reges et Principes diligenter aduertere , } quantum deseruiat societas humanae vitae : ut ad bene viuere sciant se , \\\hline
2.1.1 & en quanto siruela conpannia ala uida humanal \textbf{ por que sepan ordenar assi } e alos otros a bien beuir & et maxime Reges et Principes diligenter aduertere , \textbf{ quantum deseruiat societas humanae vitae : ut ad bene viuere sciant se , } et alios ordinare . \\\hline
2.1.2 & Et pues que assi es \textbf{ si la çibdat es derechamente ordenada deue auer } en ssi todas las cosas & Ciuitas ergo , \textbf{ si sit } recte ordinata , continere debet expedientia in tota vita , \\\hline
2.1.2 & e ante ponen esta comunidat dela casa ¶ \textbf{ Et pues que assy es deuedes saber } que si nos cuydamos con grant acuçia los dichos del philosofo en las politicas paresçra & quia per hoc manifeste ostenditur necessariam esse communitatem domesticam : \textbf{ cum omnis alia communitas communitatem illam praesupponat . Aduertendum ergo quod si dicta Politica diligenter consideremus , } apparebit quadruplicem esse communitatem ; videlicet , domus , \\\hline
2.1.2 & que si nos cuydamos con grant acuçia los dichos del philosofo en las politicas paresçra \textbf{ que son quatro las comuindades Conuiene a saber . } Comuidat dela casa Et comunidat de uarrio . & cum omnis alia communitas communitatem illam praesupponat . Aduertendum ergo quod si dicta Politica diligenter consideremus , \textbf{ apparebit quadruplicem esse communitatem ; videlicet , domus , } vici , ciuitatis , et regni . \\\hline
2.1.2 & por que faziendo uarrio siguese \textbf{ que pueden fazer çibdat e regno . } Et por ende la comunidat dela casa sea alas trͣs comunidades & quia constituendo vicum , \textbf{ ex consequenti constituere possunt ciuitatem , | et regnum . } Hoc ergo modo communitas domus se habet ad communitates alias : \\\hline
2.1.2 & e despues cresçiendo los fijos e las fijas \textbf{ e por que non podieron por muchedunbre dellos morar todos en vna cala } e por ende fizieron ellos otras casas iuntas ala primera casa & sed crescentibus filiis et filiabus , \textbf{ et non valentibus praemultitudine habitare in domo illa , } construxerunt sibi domos annexas : \\\hline
2.1.2 & Ca assi commo dicho es de suso cresçiendo los nietos e los fijos de los fijos \textbf{ por que non podien todos morar en vna casa } por fuerça ouieron de fazer muchas casas & crescentibus collectaneis idest nepotibus , \textbf{ et filiis , et filiorum filiis : et non valentibus habitare in una domo , compulsi sunt facere domos plures , } et constituere vicum . \\\hline
2.1.2 & por que non podien todos morar en vna casa \textbf{ por fuerça ouieron de fazer muchas casas } e fizieron vn uarrio . & crescentibus collectaneis idest nepotibus , \textbf{ et filiis , et filiorum filiis : et non valentibus habitare in una domo , compulsi sunt facere domos plures , } et constituere vicum . \\\hline
2.1.2 & Mas si ay otra manera \textbf{ en que se pueda fazer uarrio o çibdat o regno } sin esta & et regnum . \textbf{ Utrum autem alio modo sit possibilis generatio vici , ciuitatis , } vel regni , \\\hline
2.1.2 & ante si¶ visto \textbf{ en qual manera la comunidat dela casa se ha alas otras comuidades de ligero puede parescer } en qual manera esta comunidat es neçessaria ala uida humanal . & et aliquo modo eam omnes aliae praesupponunt . Viso , \textbf{ quomodo communitas domus se habeat ad communitates alias : | de leui patet , } quomodo huiusmodi communitas sit necessaria in humana vita . \\\hline
2.1.2 & Conuiene que la comunidat dela casa sea mas neçessaria Et pues que assi es los Reyes e los prinçipes \textbf{ cuyo ofiçio es de gerar } e enderescar los otros & Nam si omnes communitates aliae domum praesupponunt : \textbf{ si aliqua communitas est necessaria ad per se sufficientiam vitae , oportet communitatem domus necessariam esse . Reges ergo et Principes , quorum officium est dirigere alios ad bene viuere , ignorare non debent , } quomodo domus et ciuitates deseruiunt \\\hline
2.1.2 & cuyo ofiçio es de gerar \textbf{ e enderescar los otros } a bien beuir deuen saber & ø \\\hline
2.1.2 & e enderescar los otros \textbf{ a bien beuir deuen saber } en qual manera las casas e las çibdades & si aliqua communitas est necessaria ad per se sufficientiam vitae , oportet communitatem domus necessariam esse . Reges ergo et Principes , quorum officium est dirigere alios ad bene viuere , ignorare non debent , \textbf{ quomodo domus et ciuitates deseruiunt } ad sufficientiam humanae vitae . \\\hline
2.1.3 & or que non trabaiemos en vano fablando dela casa \textbf{ conuiene de saber } que la casa algunas uezes puede ser dicha costruymiento fech̃o de paredes e de techo e de paredesçimientos & cum de domo loquimur , \textbf{ sciendum quod domus nominari potest aedificium constitutum ex pariete , } tecto , \\\hline
2.1.3 & mas por que lo fizieron los que moran en la çibdat . \textbf{ Bien assi alguons suelen dezir } que las sus casas fizieron esto o aquello & non quod muri vel aedificia hoc egerint , \textbf{ sed quia in colae ciuitatis fecerunt illud . Sic aliqui dicere consueuerunt domos suas hoc operatas esse , } non quia lapides illud egerint , \\\hline
2.1.3 & Et pues que assi es la casa \textbf{ de que prinçipalmente deuemos fablar en esta sçiençia moral } e de costunbres non es el hedifiçio & et communicatio personarum habitantium in una domo domus nuncupari potest . \textbf{ Domus ergo de qua principaliter intenditur in morali negocio non est ipsum aedificium , } sed est communicatio domesticarum personarum ; \\\hline
2.1.3 & assi commo alinconicos nico \textbf{ que quiere dezir ordenador de casa de deter minar de los heditiçios delas calas } por que ael parte nesçe generalmente demostrar & spectat enim ad moralem Philosophum , \textbf{ ut ad oeconomicum , | determinare de aedificiis domorum : } quia spectat ad ipsum uniuersaliter et typo ostendere , \\\hline
2.1.3 & que quiere dezir ordenador de casa de deter minar de los heditiçios delas calas \textbf{ por que ael parte nesçe generalmente demostrar } por figera e por exienplo & determinare de aedificiis domorum : \textbf{ quia spectat ad ipsum uniuersaliter et typo ostendere , } quod decet homines habere habitationes decentes \\\hline
2.1.3 & por figera e por exienplo \textbf{ que conuiene alos omes de auer conueibles moradas } segunt el su poder e la su riqueza . & quia spectat ad ipsum uniuersaliter et typo ostendere , \textbf{ quod decet homines habere habitationes decentes } secundum suam possibilem facultatem ; \\\hline
2.1.3 & segunt el su poder e la su riqueza . \textbf{ Enpero non pertenesçea el de tractar delan casa prinçipalmente } segunt que la casa nonbra hedifiçio fecho de paredes . & secundum suam possibilem facultatem ; \textbf{ non tamen spectat ad ipsum principaliter determinare de domo , } ut nominat aedificium constructum : \\\hline
2.1.3 & segunt que la casa nonbra hedifiçio fecho de paredes . \textbf{ mas deue tractar dela casa } que es fecha de paredes & sed determinare \textbf{ debet de domo , quae est aedificium , } prout habet ordinem ad domum , quae est communitas personarum ; \\\hline
2.1.3 & ala ca la que es comuidat delas perssonas assy commo parte nesçe al politico \textbf{ que quiere dezir ordenador de çibdat } de tractar de la orden delas casas & prout habet ordinem ad domum , quae est communitas personarum ; \textbf{ sicut spectat ad Politicum determinare de ordine domorum , } et de constructione vici , \\\hline
2.1.3 & que quiere dezir ordenador de çibdat \textbf{ de tractar de la orden delas casas } e dela costruyçion de los varrios & prout habet ordinem ad domum , quae est communitas personarum ; \textbf{ sicut spectat ad Politicum determinare de ordine domorum , } et de constructione vici , \\\hline
2.1.3 & e ala polliçia e ordenamiento de los çibdadanos . \textbf{ Et pues que assi es nos entendemos de determinar dela casa } que es comunidat delas perssonas dela casa & et ad politiam ciuium . \textbf{ Intendimus ergo ostendere de domo , } quae est communitas personarum domesticarum , \\\hline
2.1.3 & en qual manera ella es la comunidat primera . \textbf{ Et por ende deuemos notar } que lo que dizen los omes primero & quomodo sit communitas prima . \textbf{ Notandum ergo , } quod Primum multipliciter distingui potest . Est enim primum in opere : \\\hline
2.1.3 & que lo que dizen los omes primero \textbf{ que se puede departir en muchas maneras . } Ca puede ser primero en obra et puede ser primero en entençion & Notandum ergo , \textbf{ quod Primum multipliciter distingui potest . Est enim primum in opere : } et primum in intentione : \\\hline
2.1.3 & que faz de los maderos \textbf{ ¶as por que non puede alcançar la fin } sin aquellas cosas & Agens enim primo et principaliter intendit finem . \textbf{ Verum quia non potest habere finem , } nisi per ea , \\\hline
2.1.3 & assi que aquellas cosas que son ordenadas ala fin \textbf{ ma guer que le an postrimeras en la uoluntad e enla entençion . } Enpero son primeras en la execuçion e en la obra & et exequendo est econtrario . Nam per opus consequitur finem , operando ea quae sunt ad finem , \textbf{ ita quod ea quae sunt ad finem licet sint posteriora et in voluntate et in intentione , } sunt tamen priora in executione et opere : \\\hline
2.1.3 & que son ordenadas ala fin . \textbf{ Et pues que assi es vna manera de departir la cosa primera } es & quae sunt ad finem . \textbf{ Unus ergo modus distinguendi prioritatem , est , } quia aliquid est primum in voluntate , \\\hline
2.1.3 & Et alguna cosa es dicha primera en obra e en execucion \textbf{ mas otra manera puede se de departir la cosa } que es dicha primera . & et in intentione : \textbf{ aliquid vero in executione , et opere . Alius vero modus distinguendi prioritatem esse potest : } quia aliqua est prioritas in via generationis \\\hline
2.1.3 & que en el moço . \textbf{ Et pues que assi es departidas las cosas primeras en esta manera de ligero puede paresçer } en qual manera la comunidat dela casa se ha ala comunindat dela çibdat & per amplius et perfectius reseruantur in viro quam in puero . \textbf{ Sic ergo distinctis prioritatibus : | de leui patet , } quomodo communitas domus se habet ad communitatem ciuitatis , et ad communitates alias . \\\hline
2.1.3 & assi commo assufi \textbf{ Ca la casa es para fazer eluarrio } e el varrio & et regnum . \textbf{ Est enim domus propter uicum , } uicus propter ciuitatem , \\\hline
2.1.3 & e el varrio \textbf{ para fazer la çibdat } e la çibdat & Est enim domus propter uicum , \textbf{ uicus propter ciuitatem , } ciuitas propter regnum . \\\hline
2.1.3 & e la çibdat \textbf{ para fazer el regno . } Et pues que & uicus propter ciuitatem , \textbf{ ciuitas propter regnum . } Communitas ergo vici est finis communitatis domus , \\\hline
2.1.3 & quela comunidat dela çibdat es la primera . \textbf{ Et esto non se deue entender } que es primera por generaçion & quod prima communitas est communitas ciuitatis ; \textbf{ quod non est intelligendum de prioritate generationis vel temporis , } cum ipsemet dicat ciuitatem procedere ex multiplicatione vici , \\\hline
2.1.3 & assi comm̃el uarrio se faze de muchas casas . \textbf{ Et por ende deue se entender } que la çibdat es primero & ex multiplicatione domorum , \textbf{ intelligendum est ergo hoc de prioritate perfectionis , } et complementi . \\\hline
2.1.3 & mas ahun a cada vno de los çibdadanos \textbf{ enparar mientes primero } abalbien dela çibdat e del regno & spectat enim non solum ad Principem siue ad legislatorem , \textbf{ sed etiam ad quemlibet ciuem prius intendere bonum ciuitatis et regni , } quam etiam bonum propriae domus , \\\hline
2.1.3 & mas conplidamente en el terçero libro \textbf{ Visto en qual manera la comunidat dela casa es primero en alguna manera que las otras comuni dades de ligero puede paresçer } en alguna manera esta comunidat dela cała es natural & ut in tertio libro plenius ostendetur . Viso , \textbf{ quomodo communitas domus aliquo modo est prior , | quam communitates aliae : } de leui videri potest , \\\hline
2.1.3 & e alos prinçipes \textbf{ de saber gouernar las cosas dela casa } e gouernar la conpanna dela casa & Reges ergo et Principes decet \textbf{ scire gubernare domestica , } et regere familiam siue domum , \\\hline
2.1.3 & de saber gouernar las cosas dela casa \textbf{ e gouernar la conpanna dela casa } non solamente en quanto deuen ser uarones aconpannables & scire gubernare domestica , \textbf{ et regere familiam siue domum , } non solum inquantum esse debent viri sociales \\\hline
2.1.3 & e bien acostunbrados . \textbf{ Ca en esta manera saber gouernar la casa } par tenesçe a todos los çibradadanos . & et politici , \textbf{ quia sic scire gubernationem domus pertinet ad omnes ciues : } sed spectat specialiter ad Reges \\\hline
2.1.3 & Et ela conosçimiento del gouernamento dela casa ¶ Por que nunca ninguno puede ser buen gouernador del regno o dela çibdat \textbf{ si non sopiere bien gouernar } assi e ala su conpana . & nunquam enim quis debitus rector regni vel ciuitatis efficitur , \textbf{ nisi se et suam familiam sciat debite gubernare . } Quare si specialiter spectat ad Reges et Principes regere regnum et ciuitates , \\\hline
2.1.3 & Por la qual cosasi espeçialmente pertenesçe alos Reyes \textbf{ e alos prinçipes de gouernar el regno } e la çibdat espeçialmente pertenesçe a ellos & nisi se et suam familiam sciat debite gubernare . \textbf{ Quare si specialiter spectat ad Reges et Principes regere regnum et ciuitates , } specialiter spectat ad eos , \\\hline
2.1.3 & e la çibdat espeçialmente pertenesçe a ellos \textbf{ que sepan gouernar su casa proprea } e que conoscan que cosa & specialiter spectat ad eos , \textbf{ ut sciant domum propriam gubernare , } et ut cognoscant quae et qualis est communitas domus \\\hline
2.1.3 & assi commo es declarado en este presente capitulo . \textbf{ Ca por esta aur̃a muy grant manera para saber et buscar el gouernamiento dela çibdat e del regno . } Enpero deuedes saber con grant acuçia & ut est in praesenti capitulo declaratum : \textbf{ nam per hoc magnam viam habebunt ad inuestigandum regimen ciuitatis } et regni . Est tamen diligenter notandum quod licet quodam speciali \\\hline
2.1.3 & Ca por esta aur̃a muy grant manera para saber et buscar el gouernamiento dela çibdat e del regno . \textbf{ Enpero deuedes saber con grant acuçia } que commo quier que en alguna manera espeçial & ut est in praesenti capitulo declaratum : \textbf{ nam per hoc magnam viam habebunt ad inuestigandum regimen ciuitatis } et regni . Est tamen diligenter notandum quod licet quodam speciali \\\hline
2.1.3 & e alta part enesça alos Reyes \textbf{ e alos prinçipes de entender } e cuydar çerca & ø \\\hline
2.1.3 & e alos prinçipes de entender \textbf{ e cuydar çerca } el bien del regno & et regni . Est tamen diligenter notandum quod licet quodam speciali \textbf{ et excellenti modo } ad Reges et Principes spectat intendere bonum regni et principatus : \\\hline
2.1.3 & el bien del regno \textbf{ e del prinçipado empero pertenesçe entender este bien } a qual si quier çibdadano & et excellenti modo \textbf{ ad Reges et Principes spectat intendere bonum regni et principatus : } attamen huiusmodi bonum intendere spectat ad unumquemque ciuem , \\\hline
2.1.3 & Et pues que assi es lo vno \textbf{ por que cada vno deue estudiar de ser digno para gouernar e para enssennorear . } Lo otro avn por que parte nesçe a cada vn çibdadano & Ergo \textbf{ tum quia quilibet studere debet | ut sit dignus regere } et principari , \\\hline
2.1.3 & de entenderal bien del regno . \textbf{ Por ende pertenesçe a cada vn çibdadano saber gouernar su casa } non solamente en quanto este gouernamiento es bien propreo suyo & quia spectat ad omnes ciues intendere bonum regni : \textbf{ spectat ad unumquemque ciuem scire regere domum suam , } non solum inquantum huiusmodi regimen est bonum proprium , \\\hline
2.1.4 & qual es esta comuidat dela \textbf{ casapor ende entendemos de dezir algunas cosas dela comunidat dela casa . } Pues que assi es deuedes saber & qualis sit huiusmodi communitas , \textbf{ ideo intendimus aliqua dicere | de communitate domestica . } Sciendum ergo , \\\hline
2.1.4 & casapor ende entendemos de dezir algunas cosas dela comunidat dela casa . \textbf{ Pues que assi es deuedes saber } que el philosofo en el primero libro delas politicasasse declara & de communitate domestica . \textbf{ Sciendum ergo , } Philosophum 1 Politicorum sic describere communitatem domus : \\\hline
2.1.4 & segunt natura construyda e fecha para cada dia \textbf{ que quiere dezir } para las obras & quod domus est communitas \textbf{ secundum naturam , } constituta quidem in omnem diem . \\\hline
2.1.4 & Mas en esta declaraçion alguna cosa es ya declarada en el capitulo sobredich̃o \textbf{ Et alguna cola finca adelante de declarar . } Ca que la casa sea comunidat & constituta quidem in omnem diem . \textbf{ In hac autem descriptione aliquid declaratum est per praecedens capitulum , et aliquid restat ulterius declarandum . } Nam quod domus sic communitas \\\hline
2.1.4 & e prouaremos que cada vna ꝑtetal dela casa es cosa natural . \textbf{ Pues que assi es finça de declarar en la difiniçion sobredichͣ } en qual manera la casa sea comunidat establesçida para cada dia . & quod quaelibet talis pars est aliquid naturale . \textbf{ Restat ergo declarare in descriptione praedicta , } quomodo domus sit communitas constituta in omnem diem . \\\hline
2.1.4 & en qual manera la casa sea comunidat establesçida para cada dia . \textbf{ Et para auer esto deuedes saber } que assi comm̃ departe el philosofo el philosofo en el primero libro delas politicas & quomodo domus sit communitas constituta in omnem diem . \textbf{ Ad cuius euidentiam aduertendum , } quod humanorum operum , \\\hline
2.1.4 & que auemos meester de cada dia . \textbf{ Las quales son comer e beuer e vestir } e o tristales & quibus indigemus omni die , \textbf{ cuiusmodi est comedere , | bibere , } et \\\hline
2.1.4 & las quales non auemos mester de cada dia \textbf{ assy commo es el conp̃r e el uender . } Ca si algunos de los que estan en alguna casa han menester de cada dia & quibus non quotidie indigemus , \textbf{ ut emptio , et venditio . } Nam et si aliqui existentes in aliqua domo , \\\hline
2.1.4 & Ca si algunos de los que estan en alguna casa han menester de cada dia \textbf{ para conplimiento de su vida conprar e vender } esto es por fallestemiento & Nam et si aliqui existentes in aliqua domo , \textbf{ ad sustentationem vitae quotidie indigent emptione , et venditione , } videtur hoc esse ex defectu , \\\hline
2.1.4 & Mas paresçen mas ser ꝑegrinos e uiandantes \textbf{ si para conplimiento de su uida han mester cada dia conprar e vender . } pues que assi es la comunidat dela casa fue fecha para aquellas cosas & sed magis ut peregrini \textbf{ et ut viatores , } si ad sustentationem vitae emptione vel venditione continue egeant . Communitas ergo domus facta fuit propter ea , quibus quotidie indigemus . Verum quia in una domo non reperiuntur omnia necessaria ad vitam , \\\hline
2.1.4 & para la uida non cunplie la comunidat de vna casa \textbf{ mas conuiene de dar comunidat de varrio . } Por que commo el uarrio sea fech̃ de muchas casas & non sufficiebat communitas domestica , \textbf{ sed oportuit dare communitatem vici , } ita quod cum vicus constet ex pluribus domibus , \\\hline
2.1.4 & aquello que non es fallado en vna casa \textbf{ puede se fallar en otra } Et por ende dize el philosofo en el primero libro delas politicas & quod non reperitur in una domo , \textbf{ reperiatur in alia . } Propter quod Philosophus 1 Politicorum ait , \\\hline
2.1.4 & para cada dia \textbf{ commo son conprar e uender . } Mas ahun por que en vnuarrio non son falladas todas las cosas & sic in opera non diurnalia constituta est communitas vici . \textbf{ Verum } quia etiam in uno vico non reperiuntur omnia \\\hline
2.1.4 & neçessarias ala uida \textbf{ conuiene de dar comunidat ala çibdat sobre la comunidat deluarrio . } Et por ende paresçeque la comunindat dela çibdat es & necessaria ad vitam , \textbf{ praeter communitatem } vici oportuit dare communitatem ciuitatis . Communitas ergo ciuitatis esse videtur ad supplendam indigentiam in tota vita . \\\hline
2.1.4 & Et por ende paresçeque la comunindat dela çibdat es \textbf{ para conplir la mengua } que puede ser en toda la uida ¶ & praeter communitatem \textbf{ vici oportuit dare communitatem ciuitatis . Communitas ergo ciuitatis esse videtur ad supplendam indigentiam in tota vita . } Illa ergo videtur esse perfecta ciuitas \\\hline
2.1.4 & assi commo dicho es de suso \textbf{ que çibdar acabada es aquella } en la qual se pueden fallar todas aquellas cosas generalmente & Illa ergo videtur esse perfecta ciuitas \textbf{ ( ut superius dicebatur ) } in qua reperiri possunt \\\hline
2.1.4 & que çibdar acabada es aquella \textbf{ en la qual se pueden fallar todas aquellas cosas generalmente } que son neçessarias para toda la uida . & ( ut superius dicebatur ) \textbf{ in qua reperiri possunt | quae sunt } necessaria uniuersaliter \\\hline
2.1.4 & Otrossi . por que contesçe que las çibdades han entressi guerras prouechosa cosa es a vna çibdat \textbf{ para que pueda lidiar con otra çibdat } que aya conpanna & quia contingit ciuitates habere guerras , \textbf{ utile est uni ciuitati ad expugnandam ciuitatem } aliam confoederare se alteri ciuitati ; \\\hline
2.1.4 & e amistança con otra çibdat \textbf{ que la pueda ayudar . } Por la qual consa el amistança delas çibdades es prouechosa & utile est uni ciuitati ad expugnandam ciuitatem \textbf{ aliam confoederare se alteri ciuitati ; } quare cum confoederatio ciuitatum utilis sit ad bellandum hostes , \\\hline
2.1.4 & Por la qual consa el amistança delas çibdades es prouechosa \textbf{ para vençer los enemigos } e para tirar e arredrar todas las cosas & aliam confoederare se alteri ciuitati ; \textbf{ quare cum confoederatio ciuitatum utilis sit ad bellandum hostes , } et ad remouendum prohibentia corruptiua , \\\hline
2.1.4 & para vençer los enemigos \textbf{ e para tirar e arredrar todas las cosas } que pueden enbargar e corronper los bienes de la çibdat . & quare cum confoederatio ciuitatum utilis sit ad bellandum hostes , \textbf{ et ad remouendum prohibentia corruptiua , } praeter communitatem domus , vici , \\\hline
2.1.4 & e para tirar e arredrar todas las cosas \textbf{ que pueden enbargar e corronper los bienes de la çibdat . } por ende sobre la comunindat dela casa e del uatrio & quare cum confoederatio ciuitatum utilis sit ad bellandum hostes , \textbf{ et ad remouendum prohibentia corruptiua , } praeter communitatem domus , vici , \\\hline
2.1.4 & Mas el regno es comunidat establesçida \textbf{ non solamente para conplir las menguas dela uida } mas avn para tirar & ad sufficientiam in vita tota . \textbf{ Sed regnum est communitas constituta non solum ad supplendum indigentias vitae , } sed etiam ad remouendum prohibentia corruptiua ; \\\hline
2.1.4 & non solamente para conplir las menguas dela uida \textbf{ mas avn para tirar } e arredrat todas las cosas & Sed regnum est communitas constituta non solum ad supplendum indigentias vitae , \textbf{ sed etiam ad remouendum prohibentia corruptiua ; } ad quae remouendum una ciuitas non potest plene sufficere , \\\hline
2.1.4 & e corronpen las cosas neçessarias ala uida . \textbf{ Et para tirar e arredrar estas cosas non puede abastar conplidamente vna çibdat } si non le fueren ayuntadas otras muchas çibdades & sed etiam ad remouendum prohibentia corruptiua ; \textbf{ ad quae remouendum una ciuitas non potest plene sufficere , } nisi ei sint adiunctae aliae plurimae ciuitates \\\hline
2.1.4 & para las obras cotidianas e de cada dia \textbf{ mas non es cosa fuerte de veer } que la casa sea establesçida de muchas perssonas . & quia est communitas naturalis constituita propter opera diurnalia et quotidiana . \textbf{ Quod autem oporteat domum ex pluribus constare personis , videre non est difficile . } Nam cum domus \\\hline
2.1.4 & Et commo non sea propreamente comunidat nin conpannia de vno \textbf{ assi commo si queremos saluar la comuidat dela casa conuiene } que ella sea establesçida de muchͣs perssonas & cum non sit proprie communitas nec societas ad seipsum , \textbf{ si in domo communitatem saluare volumus , } oportet eam ex pluribus constare personis ; \\\hline
2.1.4 & non solamente la casa es vna comiundat \textbf{ mas en la casa conuiene de dar muchͣs } comunidades la qual cosa non puede ser sin muchͣs perssonas . & non solum domus est communitas quaedam , \textbf{ sed in domo oportet dare plures communitates : } quod sine pluralitate personarum esse non potest . Patet ergo quod domus ex pluribus constat personis . Patet \\\hline
2.1.4 & mas conplidamente en el terçero libro . \textbf{ Mas quantoalo presente abasta de dezir } en tantodel regno e dela çibdat & in tertio libro plenius ostendetur . \textbf{ Ad praesens autem sufficiat in tantum tangere de regno et ciuitate , } in quantum eorum notitia aliquo modo deseruit ad cognoscendum domum , \\\hline
2.1.4 & en quanto el conosçimiento dellos siruen el en alguna manera al conosçimiento dela casa . \textbf{ Et para saber } en qual manera se ha de gouernar la casaca & in quantum eorum notitia aliquo modo deseruit ad cognoscendum domum , \textbf{ et ad sciendum qualiter sit regenda . } Nam \\\hline
2.1.4 & Et para saber \textbf{ en qual manera se ha de gouernar la casaca } assi commo es dicho de suso en este segundo libro & in quantum eorum notitia aliquo modo deseruit ad cognoscendum domum , \textbf{ et ad sciendum qualiter sit regenda . } Nam \\\hline
2.1.4 & assi commo es dicho de suso en este segundo libro \textbf{ entendemos prinçipalmente fablar del gouernamiento dela casa } e non del gouernamiento dela çibdat nin del regno . & et ad sciendum qualiter sit regenda . \textbf{ Nam } ( ut superius tangebatur ) in hoc secundo libro intenditur principaliter regimen domus , non autem regimen ciuitatis . His sic pertractatis , \\\hline
2.1.4 & pertenesçe a cada vn çibdada \textbf{ no de laber gouernar conueniblemente lucasa . } Et tanto mas esto parte nesçe alos Reyes e alos prinçipes & tam necessaria in vita ciuili , \textbf{ spectat ad quemlibet ciuem scire debite regere suam domum : } tanto tamen magis hoc spectat ad Reges et Principes , \\\hline
2.1.4 & quanto por mal gouernamiento de su casa proprea \textbf{ mas se puede leuna tar piuyzio ala çibdat e al regno } por mal gouernamiento de los otros . & tanto tamen magis hoc spectat ad Reges et Principes , \textbf{ quanto ex incuria propriae domus magis potest insurgere praeiudicium ciuitati et regno , } quam ex incuria aliorum . \\\hline
2.1.5 & dize que la casa primera es establesçida de dos comuindades . \textbf{ Conuiene a saber } De comunidat de uaron et de muger . & quod ex duabus communitatibus , \textbf{ videlicet , } ex communitate viri et uxoris , \\\hline
2.1.5 & mas que sin varon e sin mugnỉ e sin sennor e sin sieruo . ¶ La casa de que aqui fablamos non pueda ser conueniblemente \textbf{ assi se puede mostrar . } Ca la casa assi conmo es dicho dessuso & sine quibus congrue esse non potest . Quod vero sine viro et uxore , et domino et seruo , \textbf{ domus ( | de qua hic loquimur ) congrue non possit existere , sic potest ostendi . } Nam domus \\\hline
2.1.5 & e guardada en su essençia e en su ser . \textbf{ Et pues que assi es estas dos obras de natura conuiene a saber . } La generaçion delas cosas & si non posset in esse conseruari . \textbf{ Haec ergo duo naturae opera , } videlicet , rerum generatio , \\\hline
2.1.5 & non deuen ser departidas vna de otra . \textbf{ Ca non es de departir la conseruaçion dela generaçion } por que la conleruaçion presupone e antepone la generaçion . & non debent ab inuicem separari . \textbf{ Nam non est separanda conseruatio a generatione : } quia haec illam praesupponit . \\\hline
2.1.5 & assi manifiestamente \textbf{ por la qual cosa deuedes saber } que si alguon conueinblemente es señor e ontro , & et conseruationem , \textbf{ non videtur adeo manifestum . Sciendum ergo } quod si aliquis debite dominatur \\\hline
2.1.5 & Pues que assi es \textbf{ por que el sieruo non sabe enderesçar } nin go uernar & pollet fortitudine corporali Seruus ergo , \textbf{ quia seipsum nescit dirigere , } expedit ei ut obtemperet \\\hline
2.1.5 & por que el sieruo non sabe enderesçar \textbf{ nin go uernar } assi milmo conuiene le & pollet fortitudine corporali Seruus ergo , \textbf{ quia seipsum nescit dirigere , } expedit ei ut obtemperet \\\hline
2.1.5 & assi commo çiego \textbf{ e non sabe en que manera deua andar . } Et por ende aquel que esta quanto mas es esforçado en las uirtudes del cuerpo & est quasi coecus , \textbf{ et nescit qualiter sit eundum . } Qui ergo est huiusmodi , \\\hline
2.1.5 & que lognie \textbf{ por que mas ayna se puede ferir } en essa misma manera el sieruo & tanto magis indiget tali dirigente , \textbf{ quia magis offendi potest . Sic quia naturaliter seruus pollet viribus , } et deficit scientia , \\\hline
2.1.5 & e fallesçe enla sabiduria e enl entendemiento \textbf{ para se saluar conuiene } que obedezca & quia magis offendi potest . Sic quia naturaliter seruus pollet viribus , \textbf{ et deficit scientia , } et cognitione : \\\hline
2.1.5 & Ca por mengua dela fuerça corporal \textbf{ ensegniendo las cosas neçessarias ala uida non pueden abastar } assimesmos & quia propter defectum fortitudinis corporalis , \textbf{ in exequendo necessaria ad vitam , | sibi ipsis non possunt sufficere . } Quare si dominus saluatur propter seruum , \\\hline
2.1.5 & para la casa . \textbf{ Ca dellas segunt que dize el philosofo se faze la primera casa de ligero puede paresçer } commo para el establesçemiento desta casa & et seruus propter dominum , infra clarius ostendetur . Viso , quomodo saltem hae duae communitates requiruntur ad domum , \textbf{ quia secundum Philosophum ex eis constare dicitur domus prima : | de leui videri potest , } quomodo ad constitutionem huius domus saltem requiruntur ibi tria genera personarum . \\\hline
2.1.5 & Ca assi commo alli dize los omes pobres \textbf{ non pue den auer sienpre sieruos e aministdores } que los siruna . & Nam ut ait , pauperes homines , \textbf{ qui non possunt habere seruum } et ministrum rationalem , \\\hline
2.1.5 & Et avn algunos son tan pobres \textbf{ que non solamente pueden auer seruidor razonable } mas avn non pueden auer huidor con alma . & immo aliqui sunt ita pauperes , \textbf{ qui non solum non possunt habere ministrum rationalem , } sed etiam habere non possunt ministrum animatum : \\\hline
2.1.5 & que non solamente pueden auer seruidor razonable \textbf{ mas avn non pueden auer huidor con alma . } assi commo bestia anas en logar de tal seruidor ponen alguna cosa & qui non solum non possunt habere ministrum rationalem , \textbf{ sed etiam habere non possunt ministrum animatum : } sed loco eius habent aliquid inanimatum ; \\\hline
2.1.5 & assi commo son los tan pobres \textbf{ que ni pueden auer bueye nin cauallo nin asno } para fazer sus obras & ut qui sunt adeo pauperes , \textbf{ qui neque boues , | neque equos habere possunt tanquam ministros arantes et scindentes terram , } loco bonis vel equi habent ligonem quandam \\\hline
2.1.5 & que ni pueden auer bueye nin cauallo nin asno \textbf{ para fazer sus obras } nin para arar nin para cauar la trraen logar de bueyes e de bestias han açadones & neque equos habere possunt tanquam ministros arantes et scindentes terram , \textbf{ loco bonis vel equi habent ligonem quandam } vel aliquod aliud instrumentum : \\\hline
2.1.5 & para fazer sus obras \textbf{ nin para arar nin para cauar la trraen logar de bueyes e de bestias han açadones } o otros tales instramentos & neque equos habere possunt tanquam ministros arantes et scindentes terram , \textbf{ loco bonis vel equi habent ligonem quandam } vel aliquod aliud instrumentum : \\\hline
2.1.5 & o otros tales instramentos \textbf{ para abrir la tr̃ra } assi que alos muy pobres & loco bonis vel equi habent ligonem quandam \textbf{ vel aliquod aliud instrumentum : } ita quod nimis pauperibus ad scindendam terram , est pro instrumento \\\hline
2.1.5 & assi que alos muy pobres \textbf{ para labrar la trarra es menester el açadon pos instrumento } e por sieruo la qual cosa alos que non son & vel aliquod aliud instrumentum : \textbf{ ita quod nimis pauperibus ad scindendam terram , est pro instrumento } vel ministro ligo ; \\\hline
2.1.5 & por que los omes pobres \textbf{ assi commo esse mismo philosofo dio a entender } que en logar de sieruo han el buey & quia pauperes homines \textbf{ ( ut idem Philosophus innuit ) loco serui habent bouem , } vel habent \\\hline
2.1.5 & e . o alguna otra cosa en logar de bueye . \textbf{ Et conuiene a todos los çibdadanos de conosçer } e saber las partes & aliquid aliud loco bonis . \textbf{ Decet autem omnes ciues cognoscere partes , } ex quibus componitur domus : \\\hline
2.1.5 & Et conuiene a todos los çibdadanos de conosçer \textbf{ e saber las partes } de que se conpone la casa & aliquid aliud loco bonis . \textbf{ Decet autem omnes ciues cognoscere partes , } ex quibus componitur domus : \\\hline
2.1.5 & ca si las non sopieren \textbf{ non sabran bien gouernar la casa . } Empero esto much̃ & ex quibus componitur domus : \textbf{ quia eis ignoratis ignorabitur regimen ipsius domus ; } multo magis \\\hline
2.1.5 & mas conuiene alos Reyes e alos prinçipes . \textbf{ Ca saber las partes dela casa } e saƀ & tamen hoc decet Reges et Principes . \textbf{ Quia ergo cognitio partium domus , } et scire quot genera personarum , \\\hline
2.1.6 & sobredicho dos comuni dades \textbf{ fazen la primera casa o nuene de saber et uaron } e dela mu ger & duas communitates , \textbf{ videlicet , viri et uxoris , domini et serui , | facere domum primam . } Sed tamen , \\\hline
2.1.6 & Emposi la casa fuere acabada \textbf{ conuiene de dar } y la terçera comunidat & si domus debet esse perfecta , \textbf{ oportet ibi dare communitatem tertiam , } scilicet patris et filii . \\\hline
2.1.6 & que luego que la cosa es engendrada la natura es acuçiosa çerca de su salud . \textbf{ Empero faze engendrar cosa semeiante } de ssi non es assi conpado alas cosas natraales & quia statim cum res est genita , solicitatur natura circa salutem eius ; \textbf{ producere tamen sibi simile , } non sic comparatur ad res naturales : \\\hline
2.1.6 & por que non puede la cosa natural \textbf{ luego que es fecha fazer otra semeiante } assi mas conuiene & quia non statim cum est res naturalis , \textbf{ potest sibi simile producere , } sed oportet prius ipsam esse perfectam . \\\hline
2.1.6 & ca luego quando nasçe el omne la natura es acuçiosaçerca la conseruaçion e dela salud del Enpero \textbf{ non puede luegero engendrar } nin puede fazer luego otro su semeinante & cum natus est homo , solicitatur natura circa conseruationem ipsius : \textbf{ non | tamen statim potest generare , } nec statim potest sibi simile producere , \\\hline
2.1.6 & non puede luegero engendrar \textbf{ nin puede fazer luego otro su semeinante } mas conuiene & tamen statim potest generare , \textbf{ nec statim potest sibi simile producere , } sed oportet prius ipsum esse perfectum : \\\hline
2.1.6 & que primeramente el sea acabado . \textbf{ Et pues que assi es engendrar su semeiante } non pertenesçe a cosa natural tomada & sed oportet prius ipsum esse perfectum : \textbf{ producere ergo sibi similem , } non est \\\hline
2.1.6 & e las cosas \textbf{ que veemos en la casa queremos traer } a razones naturales & Si ergo domus est quid naturale , \textbf{ et ea quae videmus in domo , } reducere volumus in naturales causas , \\\hline
2.1.6 & Ca assi commo veemos en vna perssona singular \textbf{ assi avn deuemos iudgar en toda la casa } que sienpre la muchedunbre uiene dela vnidat . & ut habet esse perfectum . \textbf{ Sicut enim videmus in una persona singulari , sic et in domo tota debemus aduertere . } Nam semper multitudo ab unitate procedit , \\\hline
2.1.6 & que ser engendrado e ser saluo son de natraa del primer omne . \textbf{ Mas fazer e engendrar semeiante desi es de natura de omne ya acabado . } Ca quando el ome es primero conuiene & et saluari sunt de ratione hominis primi : \textbf{ sed producere sibi simile , | est de ratione hominis iam perfecti . } Cum enim primo homo est , \\\hline
2.1.6 & Et la natura luego es acuçiosa de su salud . \textbf{ Enpero non puede fazer su semeiante } si non quando el omne es ya acabado . & et natura statim est solicita de salute eius ; \textbf{ non potest tamen sibi simile producere , } nisi sit iam perfectus . \\\hline
2.1.6 & Mas que ala perfectiuo dela casafaga menester esta terçera comunidat \textbf{ pademos lo prouar } por tres razones & Quod autem ad perfectionem domus requiratur haec tertia communitas , \textbf{ triplici via venari possumus . Videmus enim } quod potentia fructificare et agere , sunt perfectiora non potentibus . Rursus sunt perfectiora perpetua non perpetuis . \\\hline
2.1.6 & por tres razones \textbf{ Ca veemos que las cosas que pueden fructificar e engendrar son mas acabadas } que las cosas que non pueden fructificar nin engendrar . & triplici via venari possumus . Videmus enim \textbf{ quod potentia fructificare et agere , sunt perfectiora non potentibus . Rursus sunt perfectiora perpetua non perpetuis . } Sic et feliciora sunt perfectiora infelicibus . \\\hline
2.1.6 & Ca veemos que las cosas que pueden fructificar e engendrar son mas acabadas \textbf{ que las cosas que non pueden fructificar nin engendrar . } Otrossi las cosas que duran & triplici via venari possumus . Videmus enim \textbf{ quod potentia fructificare et agere , sunt perfectiora non potentibus . Rursus sunt perfectiora perpetua non perpetuis . } Sic et feliciora sunt perfectiora infelicibus . \\\hline
2.1.6 & e del fijo pertenescan a conplimiento dela casa \textbf{ podemos lo prouar . } ¶Lo primero de parte dela generaçion & quae est patris et filii , \textbf{ primo possumus probare ex parte generationis } et fructificationis naturalis . \\\hline
2.1.6 & Ca segunt el philosofo estonçe toda cosa es acabada \textbf{ quan do puede fazer } e engendrar su semeiante & Tunc unumquodque perfectum est , \textbf{ cum potest sibi simile producere . } Ad hoc enim quod aliquid sit perfectum , \\\hline
2.1.6 & quan do puede fazer \textbf{ e engendrar su semeiante } Ca para ser la cosa acabada non conuiene & Tunc unumquodque perfectum est , \textbf{ cum potest sibi simile producere . } Ad hoc enim quod aliquid sit perfectum , \\\hline
2.1.6 & e engendre su semeiante \textbf{ mas que aya poder delo fazer . } Ca la perfection e el cunplimiento deuemos la tomar dela natura & quod sibi simile producat , \textbf{ sed quod possit sibi simile producere : perfectio enim consideranda est ex natura et ex forma } rei , per quam aliquid est in actu , \\\hline
2.1.6 & mas que aya poder delo fazer . \textbf{ Ca la perfection e el cunplimiento deuemos la tomar dela natura } e dela forma de essa misma cosa . & quod sibi simile producat , \textbf{ sed quod possit sibi simile producere : perfectio enim consideranda est ex natura et ex forma } rei , per quam aliquid est in actu , \\\hline
2.1.6 & por la qual toda cosa es en su ser conplido \textbf{ e puede obrar e engendrar } por la qual cosa si non ha poderio de obrar & rei , per quam aliquid est in actu , \textbf{ et potest agere ; } quare si est impotens ad agendum , \\\hline
2.1.6 & e puede obrar e engendrar \textbf{ por la qual cosa si non ha poderio de obrar } siguese qual mengua alguna forma o alguna perfection e conplimiento & et potest agere ; \textbf{ quare si est impotens ad agendum , } sequitur quod ei deficiat aliqua forma vel aliqua perfectio , \\\hline
2.1.6 & mas alguno es dicho \textbf{ que non es poderoso de obrar } quando tiene su materia proprea presente & quae sit principium actionis . \textbf{ Impotens autem ad agendum dicitur aliquid , } cum praesente proprio passiuo , \\\hline
2.1.6 & quando tiene su materia proprea presente \textbf{ en que puede obrar } e non puede fazer su semeiante . & cum praesente proprio passiuo , \textbf{ non producat sibi simile ; } quare cum proprium actiuum generationis sit masculus , \\\hline
2.1.6 & en que puede obrar \textbf{ e non puede fazer su semeiante . } Por la qual razon commo el propre o fazedor & cum praesente proprio passiuo , \textbf{ non producat sibi simile ; } quare cum proprium actiuum generationis sit masculus , \\\hline
2.1.6 & e la muger \textbf{ sea la primera parte dela casar la primera comunidat } que es menester para la uida dela casa & vir et uxor sit prima pars domus \textbf{ et prima communitas , } quae requiritur in vita domestica : \\\hline
2.1.6 & en que non ay engendraçion de fijos . ¶ La segunda razon \textbf{ para prouar esto mesmo se toma de acabamiento de comunindat natural duradera } por sienpre . & Secunda via ad inuestigandum hoc idem , \textbf{ sumitur ex parte naturalis perpetuitatis . } Nam cum homines non possunt per seipsos perpetuari in vita , \\\hline
2.1.6 & Ca commo los omes non pueden \textbf{ por si mesmos durar } por sienpre en la uida . & sumitur ex parte naturalis perpetuitatis . \textbf{ Nam cum homines non possunt per seipsos perpetuari in vita , } sed quodammodo perpetuatur humana vita per successionem filiorum : domus \\\hline
2.1.6 & Et pues que assi es la morada natural dela casa \textbf{ non puede durar naturalmente } para sienpre & noua familia inhabitet domum illam . \textbf{ Naturalis ergo habitatio domestica naturaliter perpetuari non potest , } nisi per generationem , \\\hline
2.1.6 & fallesçen los fijos por los quales contesçe que la morada dela casa \textbf{ naturalmente en alguna manera puede durar } para sienpre¶ & imperfecta est domus , \textbf{ ubi desunt filii , } per quos habitationem domesticam naturaliter quodammodo contingit perpetuari . Tertio autem declarari potest ex parte ipsius felicitatis . \\\hline
2.1.6 & para sienpre¶ \textbf{ Lo terçero esso mesmo se puede declarar de parte dela bien andança } ca los fijos e el poderio çiuil & ubi desunt filii , \textbf{ per quos habitationem domesticam naturaliter quodammodo contingit perpetuari . Tertio autem declarari potest ex parte ipsius felicitatis . } Nam filii , et ciuilis potentia , \\\hline
2.1.6 & La terçera del padre e del fij̉o . \textbf{ Et destas cosas puede paresçer } que en la casa acabada deuen ser tres gouernamientos . & tertiam patris et filii . \textbf{ Ex his autem patere potest , } quod oportet in domo perfecta esse tria regimina . \\\hline
2.1.6 & que en la casa acabada deuen ser tres gouernamientos . \textbf{ Ca nunca podemos dar comunindat ninguna bien ordenada } si non fuere . & quod oportet in domo perfecta esse tria regimina . \textbf{ Nam nunquam est dare communitatem aliquam bene ordinatam , } nisi aliquid \\\hline
2.1.6 & e del fijo \textbf{ el padre deua sienpre mandar } e el fij̉o ser obediente . & in communitate vero patris et filii , \textbf{ pater debet esse imperans , } et filius obtemperans ; in communitate quidem domini \\\hline
2.1.6 & Et en la comunidat del señor \textbf{ e del sieruo el señor deue mandar } e el sieruo seruir e ministrar . & et filius obtemperans ; in communitate quidem domini \textbf{ et serui , dominus debet esse praecipiens , et seruus ministrans et seruiens , } in domo perfecta \\\hline
2.1.6 & e del sieruo el señor deue mandar \textbf{ e el sieruo seruir e ministrar . } Siguese que en la casa acabada & et filius obtemperans ; in communitate quidem domini \textbf{ et serui , dominus debet esse praecipiens , et seruus ministrans et seruiens , } in domo perfecta \\\hline
2.1.6 & ian de ser tres comuindades \textbf{ e tres gouernamientos de ligero puede paresçer } que conuiene que sean y . quatro linages de perssonas & in domo perfecta esse communitates tres , \textbf{ et tria regimina : | de leui patere potest , } quod ibi oportet esse quatuor genera personarum . Videretur \\\hline
2.1.6 & que conuiene que sean y . quatro linages de perssonas \textbf{ Empero podrie paresçer a alguno por auentura que deuen y ser seys linages de perssonas alssi que la primera perssona deue ser ꝑ el uaron¶ La segunda dela muger ¶ } Et la terçera el padre . & quod ibi oportet esse quatuor genera personarum . Videretur \textbf{ tamen forte alicui ibi debere esse sex genera personarum , | ita quod prima persona sit ibi vir , secunda uxor , } tertia pater , quarta filius , \\\hline
2.1.6 & e quantos los gouernamientos et quantos los linages delas perssonas . \textbf{ Et por ende de ligero puede paresçer quantas partes deuen auer este segundo libro } en que trảctaremos eł gouertiamiento dela casa & et quot genera personarum . \textbf{ De leui ergo patere potest , | quot partes habere debeat hic secundus Liber , } in quo tractatur de regimine domus . Nam cum in domo perfecta sint tria regimina , oportet hunc librum tres habere partes ; \\\hline
2.1.6 & ¶ Empero mucho conuiene alos Reyes \textbf{ e alos prinçipes de conosçer bien estos tres gouernamientos } por que estos gouernamientos catados con grant acuçia auran grant ayuda & in tertia , de dominatiuo . \textbf{ Haec autem tria regimina bene cognoscere maxime decet Reges et Principes ; } quia eis diligenter inspectis , \\\hline
2.1.6 & por que estos gouernamientos catados con grant acuçia auran grant ayuda \textbf{ por que sepan bien gouernar sus regnos et sus çibdades } icho es en el capitulo sobredicho & magnum adminiculum habebunt , \textbf{ ut bene sciant regere regnum , et ciuitatem . } Dicebatur in praecedenti capitulo , \\\hline
2.1.7 & icho es en el capitulo sobredicho \textbf{ que tres cosas son de determinar en este segundo libro } en el qual tractaremos del gouernamiento dela casa & Dicebatur in praecedenti capitulo , \textbf{ tria esse determinanda in hoc secundo libro , } in quo agitur de regimine domus , \\\hline
2.1.7 & segunt que enla casa han de ser tres gouernamientos . \textbf{ Conuiene a saber que el vno es conuigal . } El otro es paternal . & secundum quod in ipsa domo tria contingit esse regimina ; \textbf{ videlicet , } coniugale , \\\hline
2.1.7 & primeramente \textbf{ conuiene de ayuntar el uaron con la mugni } e esta orden es muy con razon . & primum oportet congregare marem , \textbf{ et foeminam . Est autem hic ordo rationabilis . } Nam primo aliquid generatur , secundo conseruatur in esse ; tertio perficitur , \\\hline
2.1.7 & Lo terçero es acabada \textbf{ e puede engendrar semeiable dessi ¶ } Et pues que assi es la comunidat del uaton e dela muger & Nam primo aliquid generatur , secundo conseruatur in esse ; tertio perficitur , \textbf{ et potest sibi simile producere . Communitas ergo maris } et foeminae , quae est propter generationem , \\\hline
2.1.7 & qual es este ayun tamiento \textbf{ e quales mugieres deuen tomar } quales si quier çibdadanos & quale sit ipsum coniugium , \textbf{ et quales uxores , ciues , | et maxime Reges et Principes deceat sumere . } Deinde ostendemus , \\\hline
2.1.7 & quales si quier çibdadanos \textbf{ e mayormente los Reyes e los prinçipes . Despues mostraremosen qual manera los uarones deuen gouernar sus mugers } e a quales uirtudes e aquales obras las de una ordenar . & Deinde ostendemus , \textbf{ qualiter viri suas uxores regere debeant | et ad quas virtutes , } et ad quae opera eas debeant ordinare . \\\hline
2.1.7 & e mayormente los Reyes e los prinçipes . Despues mostraremosen qual manera los uarones deuen gouernar sus mugers \textbf{ e a quales uirtudes e aquales obras las de una ordenar . } Mas en demostrando quales el ayuntamiento del uaron & et ad quas virtutes , \textbf{ et ad quae opera eas debeant ordinare . } In ostendendo quidem \\\hline
2.1.7 & Mas en demostrando quales el ayuntamiento del uaron \textbf{ e dela muger pmeramente nos conuiene de declarar } en qual manera el mater moino es alguna cosa segunt natura . & quale sit ipsum coniugium , \textbf{ primo declarandum occurrit , } coniugium esse aliquid \\\hline
2.1.7 & que el philosofo en el octauo delas ethicas \textbf{ quariendo mostrar qual es el amistança del uaron a la muger prueua } que aquella amistança es segunt natura . & et quod homo naturaliter est animal coniugale . Sciendum ergo quod Philosophus 8 Ethic’ volens ostendere qualis amicitia sit viri \textbf{ ad uxorem , } probat amicitiam illam esse \\\hline
2.1.7 & La terçera de parte de obras \textbf{ que han de fazer . } Ca prouado es en el primero deste segundo libro & Tertia ex parte operum . \textbf{ Probabatur enim in primo capitulo huius secundi libri , } hominem esse naturaliter animal sociale et communicatiuum . Communitas autem in vita humana \\\hline
2.1.7 & que el omne es naturalmente aina l aconpannable e comun incatiuo \textbf{ que quiere dezir ꝑtiçipante con otro } Mas la comunidat en la uida humanal & hominem esse naturaliter animal sociale et communicatiuum . Communitas autem in vita humana \textbf{ ( } ut supra tangebatur ) \\\hline
2.1.7 & mas coniuigable que politico \textbf{ que quier dezer } que el omne mas es de casa que de çibdat & ø \\\hline
2.1.7 & por la qual cosa commo todas las ainalias naturalmente sean inclinadas \textbf{ para querer engendrar otro semeiable } assi por que entre los omes esto se faze conueniblemente & et omnia animalia naturaliter inclinentur , \textbf{ ut velint producere sibi simile , quia in hominibus hoc debite sit per coniugium , } homo naturaliter est animal coniugale . \\\hline
2.1.7 & por que natural cola es al omne \textbf{ e atondas las aianlias auer natural inclinaçion } e appetito para engendrar cosa semeiable & quia naturale est homini , \textbf{ et omnibus animalibus , } habere naturalem impetum ad producendum sibi simile . \\\hline
2.1.7 & e atondas las aianlias auer natural inclinaçion \textbf{ e appetito para engendrar cosa semeiable } assi ¶ & et omnibus animalibus , \textbf{ habere naturalem impetum ad producendum sibi simile . } Tertia ratio sumitur ex parte operum : \\\hline
2.1.7 & luego man amano son departidas las obras del uaron \textbf{ e dela mugnỉca las obras del uaron son en fazer aquellas cosas } que son de fazer fuera de casa . & ( ut dicitur 8 Ethicorum ) \textbf{ confestim enim diuisa sunt opera viri , et uxoris . Opera enim uiri uidentur esse in agendo , } quae sunt fienda extra domum : opera uero uxoris in conseruando suppellectilia , uel in operando aliqua intra domum . \\\hline
2.1.7 & e dela mugnỉca las obras del uaron son en fazer aquellas cosas \textbf{ que son de fazer fuera de casa . } Mas las obras dela muger son en guardando las alfaias dela casa & confestim enim diuisa sunt opera viri , et uxoris . Opera enim uiri uidentur esse in agendo , \textbf{ quae sunt fienda extra domum : opera uero uxoris in conseruando suppellectilia , uel in operando aliqua intra domum . } Ponentes ergo propria ad commune , \\\hline
2.1.7 & assi commo dize el philosofo en el viij ̊ libro delas ethicas . \textbf{ Ca ordenar assi los bienes propreos al bien comun fazen avn abastamientode uida . } Por la qual cosa si natural cosa es al omne de auer inclinaçion & ut dicitur 8 Ethic’ . \textbf{ Nam sic propria ordinare ad bonum commune , | facit ad quandam sufficientiam vitae . } Quare si naturale est homini , \\\hline
2.1.7 & Ca ordenar assi los bienes propreos al bien comun fazen avn abastamientode uida . \textbf{ Por la qual cosa si natural cosa es al omne de auer inclinaçion } e appetito al abastamiento dela uida natural cosa es a el de querer ser a i al conuigable & facit ad quandam sufficientiam vitae . \textbf{ Quare si naturale est homini , } habere impetum ad sufficientiam vitae : \\\hline
2.1.7 & Por la qual cosa si natural cosa es al omne de auer inclinaçion \textbf{ e appetito al abastamiento dela uida natural cosa es a el de querer ser a i al conuigable } e ayuntable a su muger & Quare si naturale est homini , \textbf{ habere impetum ad sufficientiam vitae : | naturale est ei , } quod velit esse animal coniugale . \\\hline
2.1.7 & que es contraria al mater moino \textbf{ es generalmente de esquiuar alos çibdadanos } assi conmo aquella que es contraria ala cosa natural . & quid naturale , sequitur quod fornicatio , \textbf{ quae contrariatur coniugio , sit uniuersaliter a ciuibus vitanda , } tanquam aliquid contrarium rei naturali : \\\hline
2.1.7 & mas conuiene alos Reyes \textbf{ e alos prinçipes delo esquiuar quanto mas conuiene aellos de ser meiores } e mas uirtuosos que los otros . & et uniuersaliter omnem venereorum usum illicitum , \textbf{ tanto magis decet fugere Reges et Principes , } quanto decet eos meliores et virtuosiores esse . \\\hline
2.1.7 & ¶ Estas cosas dichͣs \textbf{ assi paresçe nasçer vna dubda delas cosas sobredichas . } Ca si el casamiento es al omne natural & quanto decet eos meliores et virtuosiores esse . \textbf{ His visis , quaedam dubitatio videtur ex dictis oriri . } Nam si coniugium est homini naturale , \\\hline
2.1.7 & por ende de reprehenderes \textbf{ qual quier que non da obra a se ayuntar } por matermoni oo por casamiento . & reprehensibilis est igitur quicunque \textbf{ non dat operam , } ut coniugio copuletur . \\\hline
2.1.7 & si pararemos mientes \textbf{ alo que ya dicho es de ligero se puede soluer . } Ca si natural cosa es al omne & si considerentur iam dicta , \textbf{ de leui refellitur . } Nam si naturale est homini esse animal coniugale , \\\hline
2.1.7 & de ser aianlia conuigable \textbf{ qualquier que esqua de tomar mugni non biue } assi commo omne . & Nam si naturale est homini esse animal coniugale , \textbf{ quicunque renuit coniugem ducere , non viuit } ut homo . Sed non viuere ut homo , \\\hline
2.1.7 & assi commo omne . \textbf{ mas non bruir commo omne puede ser en dos maneras . } O por que escogeuida sobre ome & quicunque renuit coniugem ducere , non viuit \textbf{ ut homo . Sed non viuere ut homo , | potest esse dupliciter . } Vel non viuit ut homo , \\\hline
2.1.7 & e quiere conteuersse \textbf{ para dar se a contenplaçion e a sçiençia e a obras diuinales . } O non biue commo omne & et vult continere \textbf{ ut vacet contemplationi veritatis | et operibus diuinis . } Vel non viuit \\\hline
2.1.7 & e de çibdat . \textbf{ Conuiene a saber que el que escoge beuir solo } e non quiere beuir & Quare sicut dicebamus de societate politica , \textbf{ videlicet quod eligens solitudinem , } et nolens ciuiliter viuere , \\\hline
2.1.7 & este tal o es bestia o es assi commo dios \textbf{ En essa misma manera podemos dezir del mater moion } que aquel que non quiere beuir conuigalmente e con su muger & uel est deus : \textbf{ sic et de coniugio dicere possumus . Nam nolens coniugaliter uiuere , } uel hoc est , \\\hline
2.1.7 & que aquel que non quiere beuir conuigalmente e con su muger \textbf{ o esto es por que quiere mas libremente fazer forniçio . } Et por ende escoge assi uida mas baxa & sic et de coniugio dicere possumus . Nam nolens coniugaliter uiuere , \textbf{ uel hoc est , } quia uult liberius fornicari ; quare eligit sibi uitam infra hominem , \\\hline
2.1.7 & Et por ende es assy commo bestia . \textbf{ O esto es por que se quiere dar a sçiençias o a obras diuinales . } Et por ende escoge & et est quasi bestia ; \textbf{ uel hoc est , | quia uult se dare speculationi ueritatis , et diuinis operibus : } quare eligit sibi uitam supra hominem , \\\hline
2.1.7 & e asi commo dios o assi commo angł . \textbf{ Et pues que assi es los que non quieren casar } e se dana mayores bienes & et est quasi Deus . \textbf{ Non nubentes ergo , } si dent se potioribus bonis quam sint bona coniugii , \\\hline
2.1.8 & que los casamientos sean sin departimiento ninguno \textbf{ e que non le puedan partir . } Et para esto mostrar & Probant autem Philosophi , \textbf{ quod decet coniugia indiuisibilia esse . } Ad quod ostendendum adducere possumus duas vias , \\\hline
2.1.8 & e que non le puedan partir . \textbf{ Et para esto mostrar } podemos dezir dos razones & quod decet coniugia indiuisibilia esse . \textbf{ Ad quod ostendendum adducere possumus duas vias , } quas philosophi tetigerunt . \\\hline
2.1.8 & Et para esto mostrar \textbf{ podemos dezir dos razones } las quales posieron los philosofos & quod decet coniugia indiuisibilia esse . \textbf{ Ad quod ostendendum adducere possumus duas vias , } quas philosophi tetigerunt . \\\hline
2.1.8 & Ca commo . nunca ninguno fiel e leal se ayunte fielmente a otro \textbf{ por amistança si se departe della . Si entre el marido e la mugier queremos saluar fe conuenible } e amistan ça leal & Nam cum nunquam aliquis fideliter amicetur alicui , \textbf{ si ab amicitia eius discedat : | si inter virum } et uxorem debitam fidem , \\\hline
2.1.8 & e para que entre el uaron e la muger sea amistança natural conuiene que guarden vno a otro fe e lealtad \textbf{ assi que non se puedan partir vno de otro . } Et esta razon tanne vałio el grande & oportet quod sibi inuicem seruent fidem , \textbf{ ita quod ab inuicem non discedant . } Hanc autem rationem videtur tangere Valerius Maximus in libro de factis memorabilibus , \\\hline
2.1.8 & por alguna razon \textbf{ que era de sofrir . Enpero non dexo } por ello de ser reprehendido & qui quamquam tolerabili ratione motus videretur , \textbf{ reprehensione tamen non caruit : } quia cuncti arbitrabantur cupiditatem liberorum coniugali fidei non debere praeponi . Fides ergo , \\\hline
2.1.8 & por que todos iudga una \textbf{ que la cobdiçia de los fijos non se deue ante poter ala fe del casamiento . } ¶ Et pues que assi es la fe segunt segunt uałio el grande & reprehensione tamen non caruit : \textbf{ quia cuncti arbitrabantur cupiditatem liberorum coniugali fidei non debere praeponi . Fides ergo , } secundum Valerium Maximum , \\\hline
2.1.8 & ¶ Et pues que assi es la fe segunt segunt uałio el grande \textbf{ la qual fe deue guardar el marido ala muger } e la muger al marido paresçe ser vna razon & secundum Valerium Maximum , \textbf{ quam obseruari debet vir uxori , } et econuerso , \\\hline
2.1.8 & Et pues que assy es conuiene a todos los çibdadanos \textbf{ de se ayuntar asus muger } ssin departimiento ninguno e sin repoyamiento . & propter quam coniugium manere debet indiuisibile , et inrepudiatum . \textbf{ Decet ergo omnes ciues coniungi suis uxoribus indiuisibiliter absque repudiatione , } tanto magis hoc decet reges et principes , \\\hline
2.1.8 & Mas esto tantomas parte nesçe alos Reyes e alos prinçipes \textbf{ quanto mas deue en ellos reluzir la fialdat } e todas las otras bondades ¶ & tanto magis hoc decet reges et principes , \textbf{ quanto magis in eis relucere } debet fidelitas , et ceterae bonitates . \\\hline
2.1.8 & La segunda razon \textbf{ para prouar esto mesmo se toma de parte dela generaçion de los fijos . } Ca commo quier que el casamiento sea mannero & Secunda via ad inuestigandum \textbf{ hoc idem sumitur ex parte prolis . } Nam licet coniugium , \\\hline
2.1.8 & por que es bien comunal dellos \textbf{ assi los fijos ayuntan } e tienen los padres e las madres & ne a ciuilitate recedant , eo quod sit quoddam commune bonum ipsorum : \textbf{ sic filii coniungunt } et continent ipsos parentes , \\\hline
2.1.8 & e tienen los padres e las madres \textbf{ que non se pueden partir vno de otro } por que son bien comunal dellos . & et continent ipsos parentes , \textbf{ ne ab inuicem recedant , } eo quod sint quoddam commune bonum ipsorum : \\\hline
2.1.8 & por la fe \textbf{ que es de guardar enel casamiento son inclinados los casados } por que se non departan de en vno . & non solum ex fide , \textbf{ quae in coniugio est seruanda , | inclinantur coniuges , } ut sibi inuicem inseparabiliter adhaereant : \\\hline
2.1.8 & Mas coͣtra odo amor aya alguna fuerça \textbf{ para ayuncar los omes el actes çentamiento del amor } por la genneraçion de los fijos acresçiental a uoluntad dellos & augmentatur eorum amicitia naturalis . \textbf{ Sed cum omnis amor vim quandam unitiuam dicat , } augmentato amore propter prolem genitam , \\\hline
2.1.8 & por la genneraçion de los fijos acresçiental a uoluntad dellos \textbf{ por que puedan fincar en amor sin departimiento ninguno . } Et pues que assi es conuiene a todos los çibda danos & augumentatur quoque eorum propositum , \textbf{ ut velint inseparabiliter permanere . } Patet ergo , \\\hline
2.1.8 & mas alos Reyes e alos prinçipes \textbf{ quanto a ellos conuiene de auer may or cuydado de sus fijos } que alos otros & Tanto tamen hoc magis decet Reges et Principes , \textbf{ quanto de prole suscepta prae omnibus aliis debent diligentiorem habere curam . Incuria enim regiae prolis } plus potest \\\hline
2.1.8 & que alos otros \textbf{ ca non auer cuydado de los fijos del Rey } mas puede fazer danno a todo el regno & quanto de prole suscepta prae omnibus aliis debent diligentiorem habere curam . Incuria enim regiae prolis \textbf{ plus potest } inferre nocumenti ipsi regno , quam incuria cuiuscunque alterius . \\\hline
2.1.8 & ca non auer cuydado de los fijos del Rey \textbf{ mas puede fazer danno a todo el regno } que non auer cuydado de los fijos & quanto de prole suscepta prae omnibus aliis debent diligentiorem habere curam . Incuria enim regiae prolis \textbf{ plus potest } inferre nocumenti ipsi regno , quam incuria cuiuscunque alterius . \\\hline
2.1.8 & mas puede fazer danno a todo el regno \textbf{ que non auer cuydado de los fijos } de qual quier otro & plus potest \textbf{ inferre nocumenti ipsi regno , quam incuria cuiuscunque alterius . } Quare si amor et diligentia circa prolem facit \\\hline
2.1.8 & si el amor e el acuçia de los fijos faza mayor ayuntamiento del casamiento \textbf{ quanto mayor es de tomar la cura } e el acuçia de los fijos de los Reyes & maiorem unionem coniugum : \textbf{ quanto maior cura et diligentia adhibenda est circa prolem regiam quam circa alias , } tanto magis decet Reges , et Principes , \\\hline
2.1.8 & que de los fiios de los otros tanto mas conuiene alos Reyes e alos prinçipes \textbf{ mientre sus mugers biuieren ayuntar se a ellas } sin ningun departimiento . & tanto magis decet Reges , et Principes , \textbf{ quam diu suae uxores vixerint , eis inseparabiliter adhaerere . } Apud aliquas sectas non reputatur contra dictamen rationis , \\\hline
2.1.9 & e mayormente los Reyes e los prinçipes deuen ser pagados de vna muger sola \textbf{ e esto podemos prouar } por tres razones ¶ & et maxime Reges et Principes unica coniuge debere esse contentos , \textbf{ triplici via venari possumus . } Prima sumitur ex parte ipsius viri . \\\hline
2.1.9 & La segunda departe dela \textbf{ mugr¶ la terçera de parte de los fijos } ¶ & Secunda ex parte ipsius uxoris . \textbf{ Tertia est ex parte prolis . } Prima via sic patet . \\\hline
2.1.9 & si non es cosa conuenible a todos los çibdadanos \textbf{ de dar se mucho alos deleytes de lux̉ia } e arredrar se delas obras dela sabiduria & ( si fortes sint ) obnubilent mentem , \textbf{ et rationem percutiant ; si indecens est omnibus ciuibus nimis vacare venereis , } et retrahere se ab actibus prudentiae , \\\hline
2.1.9 & de dar se mucho alos deleytes de lux̉ia \textbf{ e arredrar se delas obras dela sabiduria } e delas obras ciuiles & et rationem percutiant ; si indecens est omnibus ciuibus nimis vacare venereis , \textbf{ et retrahere se ab actibus prudentiae , } et ab operibus ciuilibus , \\\hline
2.1.9 & non es cosa conuenible a ellos \textbf{ de auer muchͣs mugieres . } Enpero tanto esto es mas desconuenible alos Reyes e alos prinçipes & et ab operibus ciuilibus , \textbf{ indecens est eos plures habere coniuges . } Tamen tanto hoc indecens est magis Regibus , \\\hline
2.1.9 & Enpero tanto esto es mas desconuenible alos Reyes e alos prinçipes \textbf{ quanto mas deueser fallada en ellos razon e entendemiento para gouernar } e quanto mas se deuen dar & et Principibus , \textbf{ quanto plus in eis vigere debet prudentia | et intellectus , } et quanto plus vacare debent \\\hline
2.1.9 & quanto mas deueser fallada en ellos razon e entendemiento para gouernar \textbf{ e quanto mas se deuen dar } e ser mas acuçiosos cerca las obras çiuiles & et intellectus , \textbf{ et quanto plus vacare debent } et magis esse soliciti circa opera ciuilia , \\\hline
2.1.9 & por el grand huso de luxa a del cuydado \textbf{ que deuen tomar en el gouernar aiento del regno non les conuiene de auer muchͣs mugiers¶ } La segunda razon se toma de parte dela muger . & quam aliqui aliorum . \textbf{ Ne ergo per nimiam operam venereorum nimis retrahantur ab huiusmodi cura , indecens est eos plures habere uxores . } Secunda via sumitur ex parte ipsius uxoris . \\\hline
2.1.9 & Ca assi commo de parte del uaron \textbf{ es cosa desconuenible de auer muchͣs mugiers } por que por el guaadhuso dela luxia el marido non sea enbargado en el cuydado & Secunda via sumitur ex parte ipsius uxoris . \textbf{ Nam sicut ex parte viri indecens est uxorum pluralitas , } ne propter nimiam operam venereorum vir a cura debita retrahatur : \\\hline
2.1.9 & por que por el guaadhuso dela luxia el marido non sea enbargado en el cuydado \textbf{ quel conuiene de auer . } En essa misma manera esto es desconueinble de parte dela muger & Nam sicut ex parte viri indecens est uxorum pluralitas , \textbf{ ne propter nimiam operam venereorum vir a cura debita retrahatur : } si hoc indecens est parte uxoris , \\\hline
2.1.9 & a quales si quier çibdadanos \textbf{ e a quales se quier uatones de auer muchͣs mugieres } por que non las amarien de tan grand amor quanto deue ser entre los maridos e las mugers . & ut vult Philosophus 9 Ethicor’ , indecens est quoscunque ciues plures habere uxores : \textbf{ quia eas non tanta amicitia diligerent , } quanta \\\hline
2.1.9 & ¶ La terçera razon \textbf{ para prouar esto mesmo se toma de parte dela cerazon delos fijos . } Ca commo el matermonio sea cosa natural & Tertia via ad inuestigandum hoc idem , \textbf{ sumitur ex nutritione filiorum . } Nam cum coniugium sit \\\hline
2.1.9 & Ca commo el matermonio sea cosa natural \textbf{ en qual manera se deua fazer } conueniblemente puede se muy bien demostrar & quid naturale : \textbf{ quomodo debito modo fieri debeat maxime inuestigari potest } per ea quae in aliis animalibus conspicimus . Videmus autem \\\hline
2.1.9 & en qual manera se deua fazer \textbf{ conueniblemente puede se muy bien demostrar } por aquellas cosas que veemos en las otrasaianlias . & quid naturale : \textbf{ quomodo debito modo fieri debeat maxime inuestigari potest } per ea quae in aliis animalibus conspicimus . Videmus autem \\\hline
2.1.9 & en las quales vna fenbra sola non abasta \textbf{ para dar conuenible nudͣmiento alos fijos } vn mas lo non se ayunta sinon a vna fenbra & Sed in illis , \textbf{ in quibus sola foemina non sufficit ad praestandum filiis debitum nutrimentum , } unus masculus non adhaeret \\\hline
2.1.9 & e parte los mas los . \textbf{ Et pues que assi es commo para sofrir las cargas del matermoion } en los omes & et partem masculus . \textbf{ Cum igitur ad supportandum onera coniugii in hominibus non sufficiat sola foemina , } naturale est hominibus \\\hline
2.1.9 & que vn uaron case con vna muger . \textbf{ Ca nos deuemos iudgar las cosas naturales } segunt que son en la mayor parte & ut unus vir uni mulieri nubat . \textbf{ Ea enim naturalia iudicare debemus } quae sunt \\\hline
2.1.9 & commo quier que contezca a algunos de ser esquierdos en essa manera \textbf{ por que en la mayor parte vna sola fenbra non puede sofrir las cargas del matermonio } nin abonda & licet contingat aliquos esse sinistros . \textbf{ Sic quia ut in pluribus sola foemina non potest portare onera matrimonii , } nec sufficit ad praestandum filiis omnia necessaria \\\hline
2.1.9 & nin abonda \textbf{ para dar todas las cosas neçessarias alos fijos } e el nudermiento conuenible . & Sic quia ut in pluribus sola foemina non potest portare onera matrimonii , \textbf{ nec sufficit ad praestandum filiis omnia necessaria } et debitum nutrimentum : \\\hline
2.1.9 & Por ende commo quier que por auentura algunas muger s abondan en rriquezas abastarian \textbf{ para dar conuenible nudermiento alos fijos . } Empero por que non es iudgar la cosa natural & quia facultatibus abundant sufficerent \textbf{ ad praestandum filiis debitum nutrimentum , } quia tamen naturale non est iudicandum illud quod est in paucioribus , \\\hline
2.1.9 & para dar conuenible nudermiento alos fijos . \textbf{ Empero por que non es iudgar la cosa natural } segunt aquello que es en pocas cosas & ad praestandum filiis debitum nutrimentum , \textbf{ quia tamen naturale non est iudicandum illud quod est in paucioribus , } sed quod est ut in pluribus : \\\hline
2.1.9 & en todo esse tienpo vn \textbf{ mas lo se deue ayuntar a vna fenbra } por matrimoino . & ut quam diu filii indigent parentibus , \textbf{ tam diu unus masculus uni foeminae per coniugium copuletur . } Sed filii quam diu viuunt indigent ope parentum , decens est \\\hline
2.1.9 & e vinieren acres çentamiento conueni e e ble \textbf{ por si mesmos pueden buscar su iuda conuenible } e non han menester dende adelante ayuda del padre e dela madre . & et peruenerunt ad debitum incrementum : \textbf{ per seipsos possunt sibi debitum cibum quaerere , } et non ulterius egent ope parentum ; \\\hline
2.1.9 & por que a ellos la uianda conuenible non es apareiada conplidamente por natura . \textbf{ assi se deuen auer el maslo e la fenbraen los omes } por mater moino commo las aues & a natura paratur , \textbf{ sic se debent habere mas et foemina per coniugium in hominibus , } sicut in auibus alternatim supportantibus onera filiorum se habent masculus \\\hline
2.1.9 & por la qual cosa \textbf{ si en los que se quieren ayuntar } por casamiento es cosa conuenible & et unus uni adhaereat . \textbf{ Quare si nolentes adhaerere coniugio , decens est eos adhaerere } secundum modum , \\\hline
2.1.9 & por casamiento es cosa conuenible \textbf{ que se ellos ayunten segunt manera conuenible } e segunt ordenna traal . & Quare si nolentes adhaerere coniugio , decens est eos adhaerere \textbf{ secundum modum , } et ordinem naturalem , decet omnes ciues una sola uxore esse contentos . \\\hline
2.1.9 & Et tanto esto mas pertenesçe a los Reyes \textbf{ e alos prinçipes de segnir mas orden natural } quanto mas conuiene aellos de ser meiores & et ordinem naturalem , decet omnes ciues una sola uxore esse contentos . \textbf{ Et tanto magis hoc decet Reges et Principes , } quanto decet eos meliores esse aliis , \\\hline
2.1.10 & e por alguna cosa de razonnes consentida \textbf{ non deue estoruar el mandamento } nin la ley comun . & vel ex aliqua rationabili causa permissum , \textbf{ communem legem turbare non debet . } Secundum enim commune dictamen rationis detestabile est unum virum simul plures habere uxores : \\\hline
2.1.10 & que vna muger aya muchos maridos en vno . Ca el casamiento puede ser conparado a quatro cosas \textbf{ delas quales podemos tomar quatro razones } Por las quales podemos prouar & si una foemina per coniugium simul pluribus copularetur viris . Coniugium enim ad quatuor comparari potest , \textbf{ ex quibus sumi possunt quatuor rationes , } per quas inuestigare possumus , \\\hline
2.1.10 & delas quales podemos tomar quatro razones \textbf{ Por las quales podemos prouar } que es de denostar en todo en todo & ex quibus sumi possunt quatuor rationes , \textbf{ per quas inuestigare possumus , } omnino detestabile esse unam foeminam nuptam esse pluribus viris . In coniugio enim primo reseruatur ordo naturalis : \\\hline
2.1.10 & Por las quales podemos prouar \textbf{ que es de denostar en todo en todo } que vna muger sea casada con muchos uarones & ø \\\hline
2.1.10 & que por esto se tire la orden natural \textbf{ esto non es guaue de prouar . } Ca segunt la orden natural & Quod autem ex hoc tollatur naturalis ordo , \textbf{ videre non est difficile . } Nam \\\hline
2.1.10 & natraales que sea subiecta a otro uaron . \textbf{ Ca si vno puede enssenorear a muchos en quanto son muchͣ̃s . } Enpero que vno obedezca a muchos prinçipantes & ut viro alteri sit subiecta . \textbf{ Nam etsi idem potest dominari pluribus | ut plures sunt , } eundem tamen obedire pluribus principantibus \\\hline
2.1.10 & que muchͣs fenbras sean mugers de vn uaron . \textbf{ Enpero mas de denostar es } que vna muger case en vno con mugons varones . & secundum naturalem ordinem esse non potest . Quare et si detestabile est plures foeminas coniuges esse unius viri , \textbf{ unam tamen nubere simul viris pluribus detestabilius esse debet . } Decet ergo coniuges omnium ciuium uno viro esse contentas : \\\hline
2.1.10 & Enpero mucho mas conuiene esto alas mugers de los Reyes e delos prinçipes \textbf{ por que en el casamiento dellos conuiene de guardar la orden natural } mas que en otro ninguno . & multo magis tamen hoc decet coniuges Regum et Principum , \textbf{ quia in eorum coniugio magis quam in alio decet naturalem ordinem conseruare . } Secundo hoc idem inuestigari potest ex ipsa pace et concordia , \\\hline
2.1.10 & quanto la contienda e la discordia de los prinçipes es mas periglosa \textbf{ e mas de esquiuar } que de todos los otros ¶ Lo terçero se puede prouar esto mesmo & quanto lis \textbf{ et discordia principantium est periculosior , } et magis cauenda quam omnium aliorum . Tertio hoc idem patet ex generatione prolis , \\\hline
2.1.10 & e mas de esquiuar \textbf{ que de todos los otros ¶ Lo terçero se puede prouar esto mesmo } por la generacion de los fijos & et discordia principantium est periculosior , \textbf{ et magis cauenda quam omnium aliorum . Tertio hoc idem patet ex generatione prolis , } ad quam coniugium ordinatur . \\\hline
2.1.10 & en qua vna fenbra aya muchos maridos \textbf{ ca commo quier que vn mas lo puede enprenniar muchͣs fenbras . } En pero vna muger non puede & quam alias mulieres . Igitur ex parte procreationis filiorum omnino indecens est unam foeminam plures habere uiros . \textbf{ Nam etsi unus masculus potest plures foecundare foeminas : } una tamen foemina non sic foecundari potest a pluribus viris , \\\hline
2.1.10 & quanto en tal casamiento es mas periglosa la manneria de los fijos ¶ \textbf{ Et pues que assi es cada vnas mugers deuen tener mientes con grand acuçia } que las que son ayuntadas a sus maridos & quanto in tali coniugio periculosior est sterilitas filiorum . \textbf{ Diligenter ergo aduertere debent singulae mulieres , } quae suis viris per coniugium copulantur , \\\hline
2.1.10 & por matmonio \textbf{ con quanta diligençia deuen guardar la su honestad e con quanto esfuerço deuen guardar la fialdat } que prometieron a sus maridos & quae suis viris per coniugium copulantur , \textbf{ quanta diligentia debeant ad suam pudicitiam obseruare , } et quanto conatu fidem suis viris obseruent : \\\hline
2.1.10 & assi cosa desconuenible es \textbf{ a qual si quier mug̃r de ser ayuntada a otro uar̃o } por fornicaçion ¶ & ( viuente viro suo ) viro alio copulari , \textbf{ magis detestabile est alicui viro fornicarie commiscere . } Quarto hoc inuestigare possumus ex filiorum debito nutrimento . \\\hline
2.1.10 & por fornicaçion ¶ \textbf{ Lo quarto podemos mostrar esso mismo } por el nudermiento conueinble de los fuos . & magis detestabile est alicui viro fornicarie commiscere . \textbf{ Quarto hoc inuestigare possumus ex filiorum debito nutrimento . } Nam ex hoc parentes solicitantur circa pueros , \\\hline
2.1.10 & commo deurien en el nudermiento conuenible de sus fijos \textbf{ nin en proueer los dela hedat ¶ } Et pues que assi es cosa de denostar & ut suis filiis debite in nutrimento \textbf{ et in haereditate prouideant . } Detestabile est ergo unum virum \\\hline
2.1.10 & nin en proueer los dela hedat ¶ \textbf{ Et pues que assi es cosa de denostar } que vn ome aya muchas mugers . & et in haereditate prouideant . \textbf{ Detestabile est ergo unum virum } plures habere uxores : \\\hline
2.1.10 & Por la qual cosa sy conuiene a todos los çibdadanos ser çier tos de lus fios \textbf{ por que los puedan proueer con grand acuçia en las hedades e en el nudrimiento . } Conuiene alas mugers de todos los çibdadanos & de suis filiis , \textbf{ ut eis diligenter prouideant in haereditate et in nutrimento : } decet coniuges omnium ciuium uno viro esse contentas ; \\\hline
2.1.10 & en quanto es mas periglo mayor \textbf{ de non auer cuydado de los fijos de los Reyes et de los prinçipes } que de los fijos delons otros . ¶ & et Principum , \textbf{ in quantum incuria circa eorum filios periculosior est , } quam incuria aliorum . \\\hline
2.1.11 & nin con parientes ayuntados \textbf{ por grand parentesco . Esto podemos prouar } por tres razones ¶ & et quod cum parentibus et consanguineis nimia consanguineitate coniunctis non sit ineundum coniugium , \textbf{ triplici via venari possumus . } Prima sumitur ex debita reuerentia , \\\hline
2.1.11 & e muy conuenible \textbf{ que auemos de fazer al padre e ala madre e alos parientes muy çercanos ¶ } La segunda del bien que se leunata del mater moion & Prima sumitur ex debita reuerentia , \textbf{ quae est parentibus et consanguineis exhibenda . Secunda , } ex bono \\\hline
2.1.11 & ¶La terçera del mal \textbf{ que se deude escusa¶ } La primera razon se declara assi . & ex malo \textbf{ quod inde vitatur . Prima via sic patet . } Nam cum ex naturali ordine debeamus parentibus debitam subiectionem , \\\hline
2.1.11 & La primera razon se declara assi . \textbf{ Ca commo por la orden natural deuamos auer subiectiuo al padre e ala madre } e reuerençia conueible alos parientes & quod inde vitatur . Prima via sic patet . \textbf{ Nam cum ex naturali ordine debeamus parentibus debitam subiectionem , } et consanguineis debitam reuerentiam , \\\hline
2.1.11 & e muy çercana \textbf{ por su linage a algun uaron non es de tomar } para en mater moion . & dicta naturalis ratio , \textbf{ quod nimis propinqua ex suo genere non est per coniugium socianda , } immo adeo videtur \\\hline
2.1.11 & ¶ Et pues que assi es non conuiene alas fijas \textbf{ de casar con su padre nin alos fijos con su madre } por la grand reuerençia & inconueniens esset sic matrem filio esse subiectam . \textbf{ Non licet ergo filiis contrahere cum parentibus propter mutuam reuerentiam , } quam sibi inuicem debent . \\\hline
2.1.11 & que les deuen \textbf{ e les son tenudos de fazer . } Avn essa misma manera non les conuiene de casar con los parientes & Non licet ergo filiis contrahere cum parentibus propter mutuam reuerentiam , \textbf{ quam sibi inuicem debent . } Sic etiam non licet eis contrahere cum consanguineis aliis , \\\hline
2.1.11 & e les son tenudos de fazer . \textbf{ Avn essa misma manera non les conuiene de casar con los parientes } que les son muy çercanos & quam sibi inuicem debent . \textbf{ Sic etiam non licet eis contrahere cum consanguineis aliis , } si sint eis nimia consanguineitate coniuncti , \\\hline
2.1.11 & Ca por algun grand bien \textbf{ que dende podrie nasçer } es algunas uezes otorgado a alguno en alguncaso & Nam propter aliquod magnum bonum , \textbf{ quod inde posset consurgere , } conceditur aliquando alicui in aliquo casur , \\\hline
2.1.11 & por razon de la reuerençia deuida alos parientes \textbf{ que non se puede guardar conueniblemente entre el marido e la muger en sus obrassacado con dispenssaçion } e en algun caso non se deuen fazer matermonios entre tales perssonas & quae in agendis \textbf{ inter coniuges congrue reseruari non possunt } ( nisi ex dispensatione et in casu ) \\\hline
2.1.11 & que non se puede guardar conueniblemente entre el marido e la muger en sus obrassacado con dispenssaçion \textbf{ e en algun caso non se deuen fazer matermonios entre tales perssonas } que son muy cercanas por parentesto . & inter coniuges congrue reseruari non possunt \textbf{ ( nisi ex dispensatione et in casu ) | inter personas nimia consanguineitate coniunctas non sunt } connubia contrahenda . \\\hline
2.1.11 & Et pues que assi es conuiene a todos los çibdadanos \textbf{ de non fazer matermonios con quales quier perssonas . } Enpero tanto & connubia contrahenda . \textbf{ Decet ergo omnes ciues non contrahere coniugia cum quibuscunque personis ; } tanto tamen hoc magis decet Reges , \\\hline
2.1.11 & mas esto conuiene alos Reyes e alos prinçipes \textbf{ quanto mas conuiene a ellos de guardar la orden natural } ¶La segunda razon & et Principes , \textbf{ quanto magis eos obseruare decet ordinem naturalem . } Secunda via ad inuestigandum \\\hline
2.1.11 & ¶La segunda razon \textbf{ para prouar esto mesmo se toma del bien } que se leunata del matermonio & Secunda via ad inuestigandum \textbf{ hoc idem , | sumitur ex quodam bono } quod in matrimonio consurgit . \\\hline
2.1.11 & por razon del parentescoparesca de ser amistança grande \textbf{ Por ende la razon natural dize que los matermonios non son de fazer entre estos } tales que son muy allegados por parentesco & Sed cum inter consanguineos ex ipsa proximitate carnis sufficiens amicitia esse videatur , dictat naturalis ratio coniugia contrahenda esse \textbf{ inter illos } qui non sunt nimia consanguineitate coniuncti : \\\hline
2.1.11 & Et pues que assi es conuiene a todos losçibdadanos \textbf{ de non fazer mater monio entre perssonas muy ayuntadas por parentesço . } Empero esto & coniungat contractio copulae coniugalis . \textbf{ Decet ergo omnes ciues non contrahere cum personis nimia consanguineitate coniunctis : } magis tamen hoc decet Reges , \\\hline
2.1.11 & ¶La terçera razon \textbf{ para prouar esto mesmo se toma del mal } que se puede escusar & et plures tam affines quam consanguineos et amicos . \textbf{ Tertia via ad inuestigandum hoc idem , } sumitur ex malo quod per coniugium vitatur . \\\hline
2.1.11 & para prouar esto mesmo se toma del mal \textbf{ que se puede escusar } por el casamiento . & Tertia via ad inuestigandum hoc idem , \textbf{ sumitur ex malo quod per coniugium vitatur . } Per coniugium enim non solum producitur \\\hline
2.1.11 & para el mal . \textbf{ Ca los que non pueden guardar castidat } por que non sean muy destenpdos e sueltos mezclandose & sed etiam vitatur intemperantiae malum : \textbf{ qui enim castitatem seruare non possunt , } ne sint nimis intemperati , \\\hline
2.1.11 & por que non sean muy destenpdos e sueltos mezclandose \textbf{ a quales quier muger sconuiene les de auer sus casamientos } por que sean pagados cada vno de su muger & ne sint nimis intemperati , \textbf{ quibuslibet foeminis se miscendo ; } expedit eis inire connubia , \\\hline
2.1.11 & assi commo dixiemos muchͣs uezes de suso . \textbf{ Conuienea todos aquellos que quieren vsar de razon e de entendemiento } que non den grand obra adelectaçiones dela catue . Et pues que & ut supra pluries diximus ; \textbf{ expedit quibuslibet volentibus vigere ratione et intellectu , non nimiam operam dare venereis . } Cum ergo ad personas nimia affinitate coniunctas habeatur naturalis amor , \\\hline
2.1.11 & tanto se acrescentarie el amor entre ellos \textbf{ que les conuernie de enteder } e de darse mucho alas obras de lux̉ia . & inter coniuges sic se habentes tanta multiplicaretur dilectio , \textbf{ quod oporteret eos nimium vacare venereis . } Decet ergo omnes ciues non inire \\\hline
2.1.11 & que les conuernie de enteder \textbf{ e de darse mucho alas obras de lux̉ia . } Et pues que assi es conuiene a todos los çibdadanos & inter coniuges sic se habentes tanta multiplicaretur dilectio , \textbf{ quod oporteret eos nimium vacare venereis . } Decet ergo omnes ciues non inire \\\hline
2.1.11 & Et pues que assi es conuiene a todos los çibdadanos \textbf{ de non fazer casamientos entre perssonas } que son muy ayuntadas en parentesço & Decet ergo omnes ciues non inire \textbf{ connubia cum personis nimia consanguinitate coniunctis ; } ne dando nimis operam venereis , \\\hline
2.1.11 & nin el su entendimientollagado \textbf{ e se ayan de tirar de los cuydados conuenibles e delas obras çiuiles dando se mucho a obras lux̉iosas . } pues que assi es tanto & percutiatur eorum ratio , \textbf{ et retrahantur a curis debitis | et a ciuilibus operibus . } Tanto hoc ergo magis decet Reges , et Principes , \\\hline
2.1.11 & quanto ellos mas deuen ser conplidos de razon e de entendemiento \textbf{ e quanto mayor periglo se podeleunatar al regno } si los Reyes e los prinçipes non entendiessen con grand acuçia & quanto ipsi plus vigere debent prudentia et intellectu : \textbf{ et quanto maius periculum potest regno consurgere , } si Reges , et Principes circa salutem regni \\\hline
2.1.11 & e enlas obras çiuiles . \textbf{ Et por ende non se deue fazer mater moion en grado muy cercano de parentesco . } Enpero en el terçero e en el quarto guado & et circa ciuilia opera non diligenter intendant . \textbf{ In nimis ergo propinquo gradu consanguineitatis | non est matrimonium contrahendum . } Tertius tamen et quartus gradus , \\\hline
2.1.11 & Enpero en el terçero e en el quarto guado \textbf{ por que ally comiençan de se arredrar del parentesco sy ouieren dispenssacion } por algun bien & Tertius tamen et quartus gradus , \textbf{ qui a propinquitate deuiare incipiunt , | si dispensatio adsit , } propter bonum aliquod prosequendum , \\\hline
2.1.11 & por algun bien \textbf{ que se puede dende segnir } o por algun mal & si dispensatio adsit , \textbf{ propter bonum aliquod prosequendum , } vel magnum malum vitandum , \\\hline
2.1.11 & o por algun mal \textbf{ que se puede escusar } puede se fazer casamiento e mater moion ¶ & vel magnum malum vitandum , \textbf{ contrahi poterit copula coniugalis . } Bonorum autem quaedam sunt bona animae , \\\hline
2.1.11 & que se puede escusar \textbf{ puede se fazer casamiento e mater moion ¶ } onuiene de saber & vel magnum malum vitandum , \textbf{ contrahi poterit copula coniugalis . } Bonorum autem quaedam sunt bona animae , \\\hline
2.1.12 & puede se fazer casamiento e mater moion ¶ \textbf{ onuiene de saber } que algunos de los bienes son bienes del alma & contrahi poterit copula coniugalis . \textbf{ Bonorum autem quaedam sunt bona animae , } ut virtututes , \\\hline
2.1.12 & quando los Reyes e los prinçipes \textbf{ si quisieren ayuntar a alguna muger } por casamiento deuen catar & Cum ergo Reges , \textbf{ et Principes volunt alicui per coniugium copulari , } attendere debent \\\hline
2.1.12 & si quisieren ayuntar a alguna muger \textbf{ por casamiento deuen catar } e entender que aquella perssona & et Principes volunt alicui per coniugium copulari , \textbf{ attendere debent } ut persona illa , quam sibi in coniugem optant , \\\hline
2.1.12 & por casamiento deuen catar \textbf{ e entender que aquella perssona } que dessean auer en muger sea honrrada e conpuesta de todos estos biens . & attendere debent \textbf{ ut persona illa , quam sibi in coniugem optant , } sit omnibus his bonis ornata . Non tamen aeque principaliter intendere debent \\\hline
2.1.12 & e entender que aquella perssona \textbf{ que dessean auer en muger sea honrrada e conpuesta de todos estos biens . } Enpero non deuen egualmente & ut persona illa , quam sibi in coniugem optant , \textbf{ sit omnibus his bonis ornata . Non tamen aeque principaliter intendere debent } ad tria praedicta bona . \\\hline
2.1.12 & Enpero non deuen egualmente \textbf{ nin prinçipalmente entender a estos tres bienes sobredichos } mas deuen entenderentre los bienes de fuera & sit omnibus his bonis ornata . Non tamen aeque principaliter intendere debent \textbf{ ad tria praedicta bona . } Sed ad nobilitatem generis , \\\hline
2.1.12 & Mas primeramente \textbf{ e por si deuen entender a la nobleza del linage } e ala muchedunbre de los amigos & et ad multitudinem amicorum \textbf{ inter exteriora bona debent intendere } quasi primo et per se : \\\hline
2.1.12 & assi commo a cosa que se sigue . \textbf{ Conuiene a ellos de tomar tal muger } que sea noble por linage & sed pluralitas diuitiarum est intendenda quasi ex consequenti . \textbf{ Decet enim eos talem uxorem acceptare , } quae sit nobilis genere , \\\hline
2.1.12 & que se siguiesse alo primero . \textbf{ Ca mas deuen entender enla } mugniassi commo paresçra & ( ut in prosequendo patebit ) honorabilitas generis , \textbf{ et pluralitas amicorum , quam multitudo diuitiarum . Omnia tamen haec tria aliquo modo sunt attendenda . } Ordinantur enim coniugium \\\hline
2.1.12 & que ala muchedunbre delas rrianzas . \textbf{ Enpero a todas estas tres cosas deuen entender en algua manera . Ca el mater moion es ordenado } assi commo paresçe en las cosas sobredichͣs aconpannia conuenible & et ad sufficientiam vitae . \textbf{ Prout ergo coniugium ordinatur } ad debitam societatem , \\\hline
2.1.12 & conuiene alos Reyes \textbf{ e alos prinçipes de querer } en sus mugi eres nobleza de liuage & ad esse pacificum , \textbf{ quaerenda est multitudo amicorum : } prout vero ordinatur \\\hline
2.1.12 & en sus mugi eres nobleza de liuage \textbf{ mas en quanto el matermoino es ordenado abien de paz deuen querer en el muchedunbre de amigos } Et en quanto es ordenado el casamiento a abondamiento deuida deuen querer en el muchedunbre de riquezas . & quaerenda est multitudo amicorum : \textbf{ prout vero ordinatur } ad sufficientiam vitae , \\\hline
2.1.12 & mas en quanto el matermoino es ordenado abien de paz deuen querer en el muchedunbre de amigos \textbf{ Et en quanto es ordenado el casamiento a abondamiento deuida deuen querer en el muchedunbre de riquezas . } Ca prouado es de suso & prout vero ordinatur \textbf{ ad sufficientiam vitae , } quaerenda est pluralitas diuitiarum . Probabatur enim supra , \\\hline
2.1.12 & que son nobles por linage \textbf{ en quant el casamiento es ordenado aconpanni . digna e conuenible deuen querer para si } e da mandar mugers & quos constat esse nobiles genere , \textbf{ prout coniugium ordinatur in societatem dignam } et congruam , \\\hline
2.1.12 & en quant el casamiento es ordenado aconpanni . digna e conuenible deuen querer para si \textbf{ e da mandar mugers } que sean de noble linage¶ & prout coniugium ordinatur in societatem dignam \textbf{ et congruam , } debent \\\hline
2.1.12 & Lo segundo \textbf{ por el bien dela paz es de querer en el mater moino la muchedunbe de los amigos . } Ca la paz se ha entre los omes & debent \textbf{ sibi uxores quaerere quae sint ex nobili genere . Secundo propter esse pacificum quaerenda est amicorum multitudo . } Nam pax \\\hline
2.1.12 & que aya la nata fuerte \textbf{ por que pueda arredrar dessi las cosas quel enpesçen } assi & quod quis habeat naturam fortem , \textbf{ ut possit nociua expellere , } sic ad esse pacificum requiritur abundantia ciuilis potentiae \\\hline
2.1.12 & e non es apoderado con amigos \textbf{ de ligero pue de sofrir tuertos e desaguisados } e non le dexan beuir en paz . & et non est munitus amicis , \textbf{ de leui iniuriam patitur , } et non sinitur viuere in esse pacifico . Coniugium igitur prout ordinatur ad esse pacificum , \\\hline
2.1.12 & Et pues que assy es el casamiento en quanto es ordenado a bien de paz \textbf{ por ende es de querer en ellos bienes } e la muchedunbre de los amigos . & et non sinitur viuere in esse pacifico . Coniugium igitur prout ordinatur ad esse pacificum , \textbf{ quaerenda est ex eo amicorum pluralitas . } Hoc autem ( ut patet ex habitis ) \\\hline
2.1.12 & e mas han menester sustinimiento de cosas \textbf{ con que se puedan defender } en quanto pueden ser conbatidos de mayores periglos & quanto eorum status , \textbf{ quia altior magis indiget sustentamentis , } et pluribus potest concuti infortuniis . \\\hline
2.1.12 & en qual manera enel casamiento de los Reyes e de los prinçipes \textbf{ e de los nobles es de demandar nobleza de linage . } Et en qual manera por tal casamiento jes de demandar muchedunbre de amigos finca & Patet ergo quomodo in coniuge regum , \textbf{ et nobilium quaerenda est nobilitas generis , } et quomodo ex tali coniugio quaerenda est pluralitas amicorum . Restat ostendere quomodo ex eo quaeri debeat diuitiarum multitudo . Quaeruntur enim ex coniuge dotes \\\hline
2.1.12 & e de los nobles es de demandar nobleza de linage . \textbf{ Et en qual manera por tal casamiento jes de demandar muchedunbre de amigos finca } de demostrar & et nobilium quaerenda est nobilitas generis , \textbf{ et quomodo ex tali coniugio quaerenda est pluralitas amicorum . Restat ostendere quomodo ex eo quaeri debeat diuitiarum multitudo . Quaeruntur enim ex coniuge dotes } et diuitiae ad supportandum onera matrimonii siue coniugii , \\\hline
2.1.12 & Et en qual manera por tal casamiento jes de demandar muchedunbre de amigos finca \textbf{ de demostrar } en qual manera por ende deua ser demandada la muchedunbre de las riquezas . & et nobilium quaerenda est nobilitas generis , \textbf{ et quomodo ex tali coniugio quaerenda est pluralitas amicorum . Restat ostendere quomodo ex eo quaeri debeat diuitiarum multitudo . Quaeruntur enim ex coniuge dotes } et diuitiae ad supportandum onera matrimonii siue coniugii , \\\hline
2.1.12 & Ca son demandadas de parte dela muger las arras e las riquezas \textbf{ para sofrir las cargas del casamiento } e para abastamiento dela uida . & et quomodo ex tali coniugio quaerenda est pluralitas amicorum . Restat ostendere quomodo ex eo quaeri debeat diuitiarum multitudo . Quaeruntur enim ex coniuge dotes \textbf{ et diuitiae ad supportandum onera matrimonii siue coniugii , } et propter sufficientiam vitae . \\\hline
2.1.12 & que son cosas que siruen a abastamiento dela uida . \textbf{ Conuiene aellos de demandar en las sus mugers } mas prinçipalmente & quae deseruiunt ad sufficientiam vitae : \textbf{ decet eos in suis coniugibus principalius quaerere , } quod sint nobiles genere , \\\hline
2.1.12 & Mas que por tal casamiento ganen muchedunbre de riquezas e de dineros ¶ \textbf{ Visto en qual manera los Reyes e los prinçipes deuen demandar con sus mugers los bienes de fuera } que pertenesçen a honrramiento del cuerpo . & quam ex tali coniugio acquiratur multitudo numismatis . \textbf{ Viso quomodo Reges , | et Principes in suis coniugibus debent quaerere exteriora bona : } de leui patet quales coniuges singuli ciues accipere debeant . \\\hline
2.1.12 & De ligo paresçe \textbf{ quales mugers deuen tomar cada vnos de los çibdadanos . } Ca si conuiene alos Reyes & et Principes in suis coniugibus debent quaerere exteriora bona : \textbf{ de leui patet quales coniuges singuli ciues accipere debeant . } Nam si Reges , \\\hline
2.1.12 & Ca si conuiene alos Reyes \textbf{ e alos prinçipes de tomar mugieres nobles } por que entre ellos sea paz e conpannia digna & Nam si Reges , \textbf{ et Principes decet accipere nobiles coniuges , } ut inter eos sit pax \\\hline
2.1.12 & non sera entre ellos conpanni a cenuenible \textbf{ Mas vno se esforçara de enssennorear al otro } mas que demanda la ley del mater moino . En essa misma manera avn si el uieio casare con la moça & inter eos digna societas , \textbf{ sed unus alteri , } ultra quam leges coniugii requirant , dominari conabitur . Sic etiam si nimis senes iuuenculae nubat , \\\hline
2.1.12 & que son bienes de fuera \textbf{ so de demandar enla muger . } Conuiene a saberhonrra de linage . & ø \\\hline
2.1.13 & e es cosa \textbf{ que se non deue departir } Et que es de vno avna . & quia est quid naturale , \textbf{ et est quid indiuisibile , } et est unius ad unam . \\\hline
2.1.13 & por grand parentesco \textbf{ e avn adelante declararemos quales bienes de fuera son de demandar enla muger . } Esto uisto finca nos de dezir & inter quas personas debet esse coniugium , \textbf{ quia non inter nimia propinquitate coniunctos . Ulterius autem declarauimus , } qualia bona exteriora sunt quaerenda in coniuge . \\\hline
2.1.13 & e avn adelante declararemos quales bienes de fuera son de demandar enla muger . \textbf{ Esto uisto finca nos de dezir } en qual manera las mugers de los çibdadanos deuen ser honrradas & inter quas personas debet esse coniugium , \textbf{ quia non inter nimia propinquitate coniunctos . Ulterius autem declarauimus , } qualia bona exteriora sunt quaerenda in coniuge . \\\hline
2.1.13 & commo delons del alma . \textbf{ Pues que assi es deuedes saber } que el philosofo en el primero libro de la rectorica & quomodo coniuges ipsorum virorum bonis tam corporis quam animae debent esse ornatae . \textbf{ Sciendum igitur quod Philosophus 1 Rhetoricorum enumerando } bona foeminarum , ait , \\\hline
2.1.13 & e amor de obras sinsiudunbre . \textbf{ Et por ende quanto alos bienes del cuerpo es de demandar } en la mugier fermosura e grandeza . & quod bona corporis foeminarum sunt pulchritudo , et magnitudo : bona vero animae , temperantia , et amor operositatis siue seruilitatis . \textbf{ Quantum ergo ad bona corporis , } quaerenda est in uxore pulchritudo , \\\hline
2.1.13 & e que non ame ser uagarosa . \textbf{ mas que ame fazer obras non seruiles nin de sieruo . } Mas quales son estas obras & et quod non amet esse ociosa , \textbf{ sed diligat | facere opera non seruilia . } Quae autem sunt opera non seruilia quae quaerenda sunt in coniugibus , \\\hline
2.1.13 & que non son seruiles \textbf{ las quales son de demandar enlas mugt̃s adelante parescra en signỉendo esta materia . } Mas que estos bienes del cuerpo & Quae autem sunt opera non seruilia quae quaerenda sunt in coniugibus , \textbf{ multa in prosequendo patebit . } Quod autem haec bona corporis , magnitudo videlicet , \\\hline
2.1.13 & Mas que estos bienes del cuerpo \textbf{ que son grandeza e fermosura sean de demandar en la muger } assi se puede prouar . & Quod autem haec bona corporis , magnitudo videlicet , \textbf{ et pulchritudo quaerenda sint in uxore , sic potest ostendi . } Nam licet coniugium , \\\hline
2.1.13 & que son grandeza e fermosura sean de demandar en la muger \textbf{ assi se puede prouar . } Ca commo quier que el casamiento & Quod autem haec bona corporis , magnitudo videlicet , \textbf{ et pulchritudo quaerenda sint in uxore , sic potest ostendi . } Nam licet coniugium , \\\hline
2.1.13 & e avn el bien dela generaçion de los fijos . \textbf{ Mas paresçe parte nesçer derechamente al casamiento } que aquellas cosas que dixiemos en el capitulo sobredich̃o . & quam seruando fornicationem vitant ) \textbf{ et bonum prolis magis directe pertinere videntur ad coniugium , } quam ea quae in praecedenti capitulo diximus . omnia ergo illa , \\\hline
2.1.13 & que assi es todas aquellas cosas \textbf{ que paresçen de fazer } para escusar la fornicaçion & quam ea quae in praecedenti capitulo diximus . omnia ergo illa , \textbf{ quae videntur facere ad fornicationem vitandam , ad fidem coniugum conseruandam , } et ad prolem debite producendam , in coniuge quaeri debent . Videmus autem quod magnitudo corporis facit \\\hline
2.1.13 & que paresçen de fazer \textbf{ para escusar la fornicaçion } e para guardar la fialdat del casamiento & quam ea quae in praecedenti capitulo diximus . omnia ergo illa , \textbf{ quae videntur facere ad fornicationem vitandam , ad fidem coniugum conseruandam , } et ad prolem debite producendam , in coniuge quaeri debent . Videmus autem quod magnitudo corporis facit \\\hline
2.1.13 & para escusar la fornicaçion \textbf{ e para guardar la fialdat del casamiento } e para engendrar conueniblemente los fijos & quae videntur facere ad fornicationem vitandam , ad fidem coniugum conseruandam , \textbf{ et ad prolem debite producendam , in coniuge quaeri debent . Videmus autem quod magnitudo corporis facit } ad bonum prolis . \\\hline
2.1.13 & e para guardar la fialdat del casamiento \textbf{ e para engendrar conueniblemente los fijos } deuen ser demandadas enla muger . & quae videntur facere ad fornicationem vitandam , ad fidem coniugum conseruandam , \textbf{ et ad prolem debite producendam , in coniuge quaeri debent . Videmus autem quod magnitudo corporis facit } ad bonum prolis . \\\hline
2.1.13 & e por que los fijos dellos resplandezcan \textbf{ por grandeza de cuepo de demandar } en las sus mugers grandeza de cuerpo . & ut filii polleant magnitudine corporali , \textbf{ quaerere in suis uxoribus magnitudinem corporis : } tanto tamen magis hoc decet Reges et Principes , \\\hline
2.1.13 & mas esto conuiene alos Reyes e alos prinçipes \textbf{ quanto ellos deuen auer mayor cuydado de sus fijos propreos } por que dellos cuelga el bien comun & tanto tamen magis hoc decet Reges et Principes , \textbf{ quanto ipsi circa proprios filios , } eo quod ex eis dependeat bonum commune \\\hline
2.1.13 & e la salud del regno ¶ \textbf{ Lo segundo entre los bienes del cuerpo es de demandar } en la muger fermosura e apostura & et salus regni , \textbf{ plus solicitari debent , | quam alii . } Secundo inter bona corporis quaerenda est in uxore formositas et pulchritudo : \\\hline
2.1.13 & por fiios grandes e fermosos . \textbf{ Conuiene a ellos de demandar en las sus mugieres grandeza e fermosura corporal . } Ca paresçe que la fermosura dela muger & ut polleant filiis pulchris et magnis ; \textbf{ decet eos in suis uxoribus quaerere magnitudinem , | et pulchritudinem corporalem : } videtur enim pulchritudo coniugis non solum facere ad bonitatem prolis , \\\hline
2.1.13 & non solamente faze ala bondat de los fijos \textbf{ mas avn faze para esquiuar la fortcaçion } ala qual escusadera & ø \\\hline
2.1.13 & ¶ Visto en qual manera \textbf{ quanto alos bienes del cuerpo son de demandar en la muger la grandeza } e la fermosura finca de veer & sed etiam ad fornicationem vitandam , ad quam vitandam est ipsum coniugium ordinatum . Viso , \textbf{ quomodo quantum ad bona corporis quaerenda sit in coniuge magnitudo , } et pulchritudo . \\\hline
2.1.13 & quanto alos bienes del cuerpo son de demandar en la muger la grandeza \textbf{ e la fermosura finca de veer } en qual manera quanto alos bienes del alma son de demandar en ella tenprança & quomodo quantum ad bona corporis quaerenda sit in coniuge magnitudo , \textbf{ et pulchritudo . } Restat videre quomodo quantum ad bona animae quaerenda sunt in ea temperantia , \\\hline
2.1.13 & e la fermosura finca de veer \textbf{ en qual manera quanto alos bienes del alma son de demandar en ella tenprança } e amor de bien obrar . & et pulchritudo . \textbf{ Restat videre quomodo quantum ad bona animae quaerenda sunt in ea temperantia , } et amor operositatis . \\\hline
2.1.13 & en qual manera quanto alos bienes del alma son de demandar en ella tenprança \textbf{ e amor de bien obrar . } Ca aquel bien paresçe de ser demandado prinçipalmente en la fenbra & Restat videre quomodo quantum ad bona animae quaerenda sunt in ea temperantia , \textbf{ et amor operositatis . } Illud enim bonum maxime videtur esse quaerendum in foemina , \\\hline
2.1.13 & Et dicho es de ssuso \textbf{ que obrar segunt razon } e seguir las passiones lonco las contrarias & ad cuius oppositum maxime incitatur . Dicebatur autem supra , \textbf{ quod agere | secundum rationem , } et insequi passiones , modo opposito se habent , \\\hline
2.1.13 & que obrar segunt razon \textbf{ e seguir las passiones lonco las contrarias } alli que quanto alguno mas sigue las passiones menos obra segunt razon . & secundum rationem , \textbf{ et insequi passiones , modo opposito se habent , } ita quod quanto magis insequitur passiones , \\\hline
2.1.13 & alas quales las mugers son mucho iclinadas . \textbf{ Et commo quier que las mugers de una resplandesçer } entondas las uirtudes & ad quas mulieres maxime incitantur ; \textbf{ licet singulis virtutibus } secundum modum eis congruum foeminas pollere deceat , \\\hline
2.1.13 & segunt la manera que les conuiene . \textbf{ Enpero quando la fenbra es de dar a algun marido mayormente deuemos tener mientessi resplandesçe por tenprança } por que las fenbras son mucho inclinadas a destenpramiento e a cosas destenpradas . & secundum modum eis congruum foeminas pollere deceat , \textbf{ tamen cum tradenda est aliqua nuptui , } potissime inquirendum est , \\\hline
2.1.13 & Et pues que assi es conuiene a todos los çibdadanos \textbf{ de demandar esto en las sus mugers } Empero tanto conuiene esto & utrum polleat temperantia , \textbf{ eo quod ad intemperantiam foeminae maxime incitentur . Decet ergo omnes ciues hoc in suis coniugibus quaerere : } tanto tamen hoc decet Reges et Principes , \\\hline
2.1.13 & mas alos Reyes e alos prinçipes \textbf{ quanto la destenprança delas mugers dellos puede fazer mayor danno e enpeçemiento } que la destenprança delas mugers de los otros . & tanto tamen hoc decet Reges et Principes , \textbf{ quanto intemperantia coniugum ipsorum | plus nocumenti inferre potest , } quam intemperantia coniugum aliorum . \\\hline
2.1.13 & que las muger ssean tenpradas . \textbf{ Et avn les conuiene aellas de amar fazer buenas obras . } Ca quando alguna persona esta de uagar mas ligeramente es inclinada a aquellas cosas & Decet ergo coniuges temperatas esse . \textbf{ Decet eas etiam amare operositatem : } quia cum aliqua persona ociosa existat , \\\hline
2.1.13 & Et en qual manera todos los çibdadanos \textbf{ e mayormente los Reyes e los prinçipes se deuen auer en tomar sus mugers . } Ca ante que las tomne deuen primero catar con grand diligençia & et qualiter omnes ciues , \textbf{ et maxime Reges et Principes se habere debent in ducendis uxoribus . } Nam priusquam eas ducant , \\\hline
2.1.13 & e mayormente los Reyes e los prinçipes se deuen auer en tomar sus mugers . \textbf{ Ca ante que las tomne deuen primero catar con grand diligençia } en qual manera son honrradas e conpuestas de los biens de fuera . & et maxime Reges et Principes se habere debent in ducendis uxoribus . \textbf{ Nam priusquam eas ducant , | diligenter debent primo inquirere , } qualiter sint ornatae exterioribus bonis : \\\hline
2.1.13 & Et en qual manera \textbf{ por tal casamiento pueden auer poderio çiuil } e muchedunbre de amigos . & ut quomodo sint nobiles genere , \textbf{ et quomodo per tale coniugium consequi possint ciuilem potentiam , } et multitudinem amicorum . Secundo inuestigare debent , \\\hline
2.1.13 & e muchedunbre de amigos . \textbf{ ¶ Lo segundo deuen catar } en qual manera resplandescan por bienes corporales & et quomodo per tale coniugium consequi possint ciuilem potentiam , \textbf{ et multitudinem amicorum . Secundo inuestigare debent , } quomodo polleant corporalibus bonis : \\\hline
2.1.13 & e que ayan fermosura e apostura del cuerpo ¶ \textbf{ Lo terçero por que non abasta de abondar en los bienes de fuera } e resplandescer en los bienes corporales si non fueren & et quod habeant formositatem et pulchritudinem corporalem . Tertia , \textbf{ quia non sufficit affluere bonis exterioribus , } et pollere corporalibus bonis , \\\hline
2.1.13 & Lo terçero por que non abasta de abondar en los bienes de fuera \textbf{ e resplandescer en los bienes corporales si non fueren } y los bienes dela uoluntad & quia non sufficit affluere bonis exterioribus , \textbf{ et pollere corporalibus bonis , } nisi adsint ibi bona mentis et animae , debent inquirere , quomodo uxores ducendae sint temperatae , \\\hline
2.1.13 & y los bienes dela uoluntad \textbf{ e del alma deuen demandar } en qual manera las mugers & et pollere corporalibus bonis , \textbf{ nisi adsint ibi bona mentis et animae , debent inquirere , quomodo uxores ducendae sint temperatae , } et quomodo sint operosae circa exercitia licita et honesta . \\\hline
2.1.13 & que han de tomarsean tenpradas . \textbf{ Otrossi deuen demandar } en qual manera sean acuçiosas & ø \\\hline
2.1.13 & en qual manera sean acuçiosas \textbf{ para obtar tales obras } que sean conuenibles e honestas . & nisi adsint ibi bona mentis et animae , debent inquirere , quomodo uxores ducendae sint temperatae , \textbf{ et quomodo sint operosae circa exercitia licita et honesta . } Quia non sufficit scire quale coniugium , \\\hline
2.1.14 & que sean conuenibles e honestas . \textbf{ or que non abasta saber } qual es el casamiento & et quomodo sint operosae circa exercitia licita et honesta . \textbf{ Quia non sufficit scire quale coniugium , } et qualiter quis se habere debeat in uxore ducenda , \\\hline
2.1.14 & qual es el casamiento \textbf{ e en qual manera se deue cada vno auer en tomar sumus } si non se sopiere auer conueinblemente çerca & Quia non sufficit scire quale coniugium , \textbf{ et qualiter quis se habere debeat in uxore ducenda , } nisi circa eam iam ductam sciat debite se habere . \\\hline
2.1.14 & e en qual manera se deue cada vno auer en tomar sumus \textbf{ si non se sopiere auer conueinblemente çerca } ella despues que fuere tomada & et qualiter quis se habere debeat in uxore ducenda , \textbf{ nisi circa eam iam ductam sciat debite se habere . } Ideo praemissis capitulis rationabiliter hoc annectitur , \\\hline
2.1.14 & por ende a estos capitulos sobredichos \textbf{ con razon deuemos ayuntar este capitulo } que se sigue & nisi circa eam iam ductam sciat debite se habere . \textbf{ Ideo praemissis capitulis rationabiliter hoc annectitur , } ut sciamus quomodo regimen nuptiale , \\\hline
2.1.14 & por el qual ellas deuen ser gouernadas es apartado de los otros gouernamientos . \textbf{ Et esto podemos prouar } por dos razones & ab aliis regiminibus est distinctum . \textbf{ Possumus autem } duplici via inuestigare , \\\hline
2.1.14 & por dos razones \textbf{ que en otra manera son las mugers de gouernar } e en otra los fijos ¶ & duplici via inuestigare , \textbf{ quod alio regimine regendae sunt coniuges , } et alio filii . Prima via sumitur ex parte modi regendi . \\\hline
2.1.14 & e en otra los fijos ¶ \textbf{ La primera razon se toma de parte dela manera de gouernar ¶ } La segunda de parte delas obras & quod alio regimine regendae sunt coniuges , \textbf{ et alio filii . Prima via sumitur ex parte modi regendi . } Secunda vero ex parte operum fiendorum . Prima via sic patet . \\\hline
2.1.14 & La segunda de parte delas obras \textbf{ que son de fazer ¶ } La primera razon paresçe assi . & et alio filii . Prima via sumitur ex parte modi regendi . \textbf{ Secunda vero ex parte operum fiendorum . Prima via sic patet . } Nam \\\hline
2.1.14 & si deuen ser bien gouernadas han de ser gouernadas \textbf{ por entendemiento e por razon Et la manera de gouernar el mundo } en tanto es fallada en cada vno de los omes & quae sunt in homine , \textbf{ si debite regi debent , regenda sunt intellectu et ratione . } Immo adeo modus uniuersi reseruatur in quolibet homine , \\\hline
2.1.14 & Ca si la casa es mas \textbf{ que vn omne singłar } la çibdat es menos que todo el mundo . & Nam \textbf{ si domus est plus quam unus homo singularis , } et ciuitas est minus quam uniuersum : \\\hline
2.1.14 & quanto parte nesçe alo presente \textbf{ por dos gouernamientos se ha de gouernar . } Conuiene a saber por gouernamientoçiuil . & ( quantum ad praesens spectat ) \textbf{ duplici regimine regi potest , politico scilicet } et regali . Dicitur autem quis praeesse regali dominio , \\\hline
2.1.14 & por dos gouernamientos se ha de gouernar . \textbf{ Conuiene a saber por gouernamientoçiuil . } Et por gouernamiento real . & ( quantum ad praesens spectat ) \textbf{ duplici regimine regi potest , politico scilicet } et regali . Dicitur autem quis praeesse regali dominio , \\\hline
2.1.14 & mas por los çibdadanos \textbf{ a qual gouernamiento non se deue nonbrar del regñate nin del prinçipante . } Mas deue se nonbrar dela çibdat e de los lus çibdadanos & sed a ciuibus , \textbf{ illud regimen non est denominandum ab ipso regnante | et principante , } sed magis ab ipsa politia \\\hline
2.1.14 & a qual gouernamiento non se deue nonbrar del regñate nin del prinçipante . \textbf{ Mas deue se nonbrar dela çibdat e de los lus çibdadanos } e es dicho tal gouernamiento politico e çiuil . & et principante , \textbf{ sed magis ab ipsa politia } et ab ipsis ciuibus . Dicitur ergo tale regimen politicum \\\hline
2.1.14 & por gouernamiento çiuil . \textbf{ Ca deueen ssennorear a ella segunt leyes çiertas e segunt leyes de matermoion } e segunt las condiçiones & secundum certas leges , \textbf{ et secundum leges matrimonii , } et secundum conuentiones et pacta . \\\hline
2.1.14 & e los pleitos del matermoino \textbf{ Mas el padre deue enssennorear alos fiion } s segunt aluedrio & et secundum conuentiones et pacta . \textbf{ Sed pater debet praeesse filiis } secundum arbitrium , \\\hline
2.1.14 & nin pleitos \textbf{ en qual manera lo deua gouernar . } Mas el padre segunt su aluedrio & Inter patrem enim et filium non interueniunt conuentiones et pacta , \textbf{ quomodo eum regere debeat : } sed pater \\\hline
2.1.14 & en quanto viere \textbf{ que conuiene meior a su fijo lo deue gouernar . } Et otssi el Rey deue gouernar la gente & sed pater \textbf{ secundum suum arbitrium prout melius viderit filio expedire , ipsum gubernat et regit : | sicut } et Rex gentem sibi subiectam regere debet \\\hline
2.1.14 & que conuiene meior a su fijo lo deue gouernar . \textbf{ Et otssi el Rey deue gouernar la gente } que es subiecta a el & sicut \textbf{ et Rex gentem sibi subiectam regere debet } secundum suum arbitrium , prout melius viderit illi genti expedire . \\\hline
2.1.14 & e algunos plertos e algunas palauras \textbf{ en qual manera el marido se deua auer çerca la muger . } Et por ende es dicho tal gouernamiento politico e çiuil & et sermones quidam , \textbf{ quomodo vir habere se debeat circa ipsam . } Dicitur ergo tale regimen politicum : \\\hline
2.1.14 & muestran le los pleitos e las con diconnes \textbf{ que deue el guardar en el su gouernamiento . } Et pues que assi es dela manera del gouernar & ostendunt ei pacta \textbf{ et conuentiones quasdam in suo regimine obseruare . Ex ipso ergo modo regendi , } quia unum est politicum , \\\hline
2.1.14 & que deue el guardar en el su gouernamiento . \textbf{ Et pues que assi es dela manera del gouernar } por que vno es el gouernamientoçiuil & ostendunt ei pacta \textbf{ et conuentiones quasdam in suo regimine obseruare . Ex ipso ergo modo regendi , } quia unum est politicum , \\\hline
2.1.14 & e por election . \textbf{ Ca enssennorear realmente es senoreat en toda manera } e seg̃t aluedrio . & magis totale et naturale : regimen vero politicum est magis paternale et ex electione . \textbf{ Nam praeesse regaliter est praeesse totaliter et secundum arbitrium : } praeesse vero politice est praeesse non totaliter nec simpliciter , \\\hline
2.1.14 & e seg̃t aluedrio . \textbf{ mas enllennore ar çiuilmente es enssennorear } non en toda manera & praeesse vero politice est praeesse non totaliter nec simpliciter , \textbf{ sed secundum quasdam conuentiones | et pacta . Rursus , } licet omne regimen \\\hline
2.1.14 & Ca commo quier que el omne sean atal mente aina laconpannable en matrimonio \textbf{ Enpero auer esta muger o aquella esto es segunt ueluntad } e por el ectiuo & tamen quod habeat hanc coniugem \textbf{ vel illam est | secundum placitum } et ex electione : \\\hline
2.1.14 & e por el ectiuo \textbf{ assi commo el omne naturalmente es despuesto a fablar } Enpero que fable en este lenguaie o en aquel estos es segunt uoluntad & et ex electione : \textbf{ sicut homo naturaliter est aptus ad loquendum , } tamen quod loquatur \\\hline
2.1.14 & Visto en qual manera se departe el gouernamientoma termoianl del paternal \textbf{ por la manera de el gouernar } por qvno es mas general & quia filii nullo modo eligunt sibi patrem . Viso , \textbf{ quomodo differt regimen coniugale a paternali ex modo regendi , } quia unum est magis simpliciter et naturale ; \\\hline
2.1.14 & El otro es assi commo particular \textbf{ e por electiuo de ligero puede paresçer } en qual manera se departe este gouernamiento de aquel & aliud vero est \textbf{ quasi particulariter et ex electione : | de leui videri potest , } quomodo differt hoc regimen ab illo ex parte operum fiendorum . \\\hline
2.1.14 & de parte delas obras \textbf{ que se han de fazer . } Ca el padre assi deue enssennorear alos fijes & quomodo differt hoc regimen ab illo ex parte operum fiendorum . \textbf{ Nam pater sic debet praeesse filiis , } ut ordinet eos \\\hline
2.1.14 & que se han de fazer . \textbf{ Ca el padre assi deue enssennorear alos fijes } que los ordene a otrasobras & quomodo differt hoc regimen ab illo ex parte operum fiendorum . \textbf{ Nam pater sic debet praeesse filiis , } ut ordinet eos \\\hline
2.1.14 & aque non ordena ala mus . \textbf{ Ca los fijos son de enssennar alas obras de caualleria } e alas obras çiuiles alas quales deuen entender & ut ordinet eos \textbf{ ad alia opera , quam uxorem . Nam filii instruendi sunt ad opera militaria , } vel ciuilia , quibus vacare debeant cum sint adulti : \\\hline
2.1.14 & Ca los fijos son de enssennar alas obras de caualleria \textbf{ e alas obras çiuiles alas quales deuen entender } quando fueren criados e mayores alas quales cosas non son de enssennar las mugers & ut ordinet eos \textbf{ ad alia opera , quam uxorem . Nam filii instruendi sunt ad opera militaria , } vel ciuilia , quibus vacare debeant cum sint adulti : \\\hline
2.1.14 & e alas obras çiuiles alas quales deuen entender \textbf{ quando fueren criados e mayores alas quales cosas non son de enssennar las mugers } por que non deuen entender atales cosas . & ad alia opera , quam uxorem . Nam filii instruendi sunt ad opera militaria , \textbf{ vel ciuilia , quibus vacare debeant cum sint adulti : | ad quae non sunt instruendae uxores , } quia talibus vacare non debent . \\\hline
2.1.14 & quando fueren criados e mayores alas quales cosas non son de enssennar las mugers \textbf{ por que non deuen entender atales cosas . } Et pues que assi es paresçe & ad quae non sunt instruendae uxores , \textbf{ quia talibus vacare non debent . } Patet ergo , \\\hline
2.1.14 & mas esto pertenesçe alos Reyes e alos prinçipes \textbf{ quanto mas ellos deuen guardar aquellas cosas } que la orden e la razon natural muestra . & tanto tamen hoc magis decet Reges et Principes , \textbf{ quanto ipsi plus obseruare debent quae dictat ordo } et ratio naturalis . \\\hline
2.1.15 & Et otrossi por que para ot̃ sobm ses el matermoinal \textbf{ e para otras el paternal fincanos de demostrar } en qual manera el gauernamiento matermoianl se departe del gouernamiento seruil & rursus , \textbf{ quia ad alia opera est hoc quam illud . Restat ostendere , } quomodo coniugale regimen differt a seruili . Quod autem non debeat uti sua coniuge tanquam serua , triplici via venari possumus . Prima sumitur \\\hline
2.1.15 & en qual manera el gauernamiento matermoianl se departe del gouernamiento seruil \textbf{ mas que el uaron non deua vsar de su mug̃r } assi commo de sierua esto podemos prouar & quia ad alia opera est hoc quam illud . Restat ostendere , \textbf{ quomodo coniugale regimen differt a seruili . Quod autem non debeat uti sua coniuge tanquam serua , triplici via venari possumus . Prima sumitur } ex ipso ordine naturali . \\\hline
2.1.15 & mas que el uaron non deua vsar de su mug̃r \textbf{ assi commo de sierua esto podemos prouar } por tres razones ¶ & ø \\\hline
2.1.15 & assi que vn cuchiello sirue a muchos ofiçios . \textbf{ Conuiene saber } que por que los pobres non podian auer muchos instrumentos & ita quod unus gladius deseruiebat pluribus officiis : \textbf{ utputa pauperes } non valentes plura habere instrumenta , \\\hline
2.1.15 & Conuiene saber \textbf{ que por que los pobres non podian auer muchos instrumentos } fazian fazer vn instrͤde & utputa pauperes \textbf{ non valentes plura habere instrumenta , } faciebant aliquod instrumentum fabricari , \\\hline
2.1.15 & que por que los pobres non podian auer muchos instrumentos \textbf{ fazian fazer vn instrͤde } que podiesen vsar en muchos ofiçies & non valentes plura habere instrumenta , \textbf{ faciebant aliquod instrumentum fabricari , } quo possent ad plura uti officia . \\\hline
2.1.15 & fazian fazer vn instrͤde \textbf{ que podiesen vsar en muchos ofiçies } mas la natura non faze & faciebant aliquod instrumentum fabricari , \textbf{ quo possent ad plura uti officia . } Natura autem non sic agit , \\\hline
2.1.15 & non conuiene \textbf{ que sea ordenada a seruir . } ¶ Et pues que assi es non es orden natural & ø \\\hline
2.1.15 & e de entendemiento sea naturalmente sieruo \textbf{ por que non sabe gniar assi mismo } e conuiene que sea gado de otro & Sed cum carens rationis usu sit naturaliter seruus , \textbf{ quia nescit seipsum dirigere , } et expedit ei quod ab aliquo alio dirigatur , \\\hline
2.1.15 & e el sieruo esto es por fallesçemiento de razon e de entendimiento \textbf{ por que non saben departir } entre el gouernamiento del marido e dela muger & hoc est propter rationis defectum , \textbf{ quia nesciunt | distinguere } inter regimen coniugale , \\\hline
2.1.15 & Et por ende si conuiene alos çibdadanos de ser sabidores \textbf{ e conosçer la manerar la orden natural } cosa muy desconuenble es a ellos de vsar delas muger & Quare si decet ciues esse industres , \textbf{ et cognoscere modum | et ordinem naturalem ; } indecens est eos uti uxoribus tanquam seruis . \\\hline
2.1.15 & e conosçer la manerar la orden natural \textbf{ cosa muy desconuenble es a ellos de vsar delas muger } sassi conmode sieruos . & et ordinem naturalem ; \textbf{ indecens est eos uti uxoribus tanquam seruis . } Tanto tamen hoc magis indecens est \\\hline
2.1.15 & e de ser menguados de razon e de encendemiento . \textbf{ Et pues que assi es de parte de la orden natural paresçe que otra cosa es el gouernamiento del marido ala mug̃r } que del señor al sieruo . & et carere ratione et intellectu . \textbf{ Ex parte igitur ordinis naturalis patet aliud esse regimen coniugale quam seruile : } et non esse utendum uxoribus tanquam seruis . \\\hline
2.1.15 & que del señor al sieruo . \textbf{ Et paresçe que non deuen vsar los omes delas mugers } assi commo de sieruas ¶ & Ex parte igitur ordinis naturalis patet aliud esse regimen coniugale quam seruile : \textbf{ et non esse utendum uxoribus tanquam seruis . } Secunda via ad inuestigandum hoc idem , sumitur ex parte perfectionis domus . \\\hline
2.1.15 & La segunda razon \textbf{ para prouar esto melmo le toma de parte del conplimiento } e del abastamiento dela casa . & et non esse utendum uxoribus tanquam seruis . \textbf{ Secunda via ad inuestigandum hoc idem , sumitur ex parte perfectionis domus . } Videtur enim domus esse imperfecta , et habere penuriam rerum , \\\hline
2.1.15 & que deue ser entre la muger e el marido . \textbf{ Ca commo quier que el marido deua ensseñorear ala mug̃r } por que es mas enoblesçido e mayor en razon & debet inter virum et uxorem . \textbf{ Nam licet vir debeat praeesse uxori , } eo quod ratione praestantior : \\\hline
2.1.15 & e entendimientoen erpero non deue ser tanta desigualeza entre la muger e el marido \textbf{ que deua vsar della } assi commo de sierua & eo quod ratione praestantior : \textbf{ non tamen debet esse tanta imparitas inter virum et uxorem , quod ea uti debeat tanquam serua , } sed magis tanquam socia . Non enim est \\\hline
2.1.15 & assi commo de sierua \textbf{ mas deue vsar della } assi commo de conpannera . & non tamen debet esse tanta imparitas inter virum et uxorem , quod ea uti debeat tanquam serua , \textbf{ sed magis tanquam socia . Non enim est } tanta imparitas \\\hline
2.1.16 & en el primero libromenos aprouecha en el neqocio moral e de costunbres . \textbf{ Et pues que assi es determinar del casamiento } e que el gouernamiento matermoian les otro que el paternal & circa morale negocium minus proficiunt : \textbf{ determinare ergo de coniugio , } quia regimen coniugale est aliud a paternali et seruili : \\\hline
2.1.16 & e que el suil \textbf{ e mostrar } que en otra manera se deua auer el uaron cerca la mugni & quia regimen coniugale est aliud a paternali et seruili : \textbf{ et ostendere quod aliter debet se habere vir tam erga uxorem , } quam erga filios , \\\hline
2.1.16 & e mostrar \textbf{ que en otra manera se deua auer el uaron cerca la mugni } que çerca los fijos e cerca los sieruos & quia regimen coniugale est aliud a paternali et seruili : \textbf{ et ostendere quod aliter debet se habere vir tam erga uxorem , } quam erga filios , \\\hline
2.1.16 & que çerca los fijos e cerca los sieruos \textbf{ esto es dar conosçimiento mucho en general } qual sea el casamiento & et seruos , \textbf{ est valde in uniuersali notitiam tradere , } quale sit ipsum coniugium , \\\hline
2.1.16 & qual sea el casamiento \textbf{ e en qual manera de una vsar del . } Et pues que assi es conuiene de desçender & quale sit ipsum coniugium , \textbf{ et quomodo utendum sit eo . } Oportet ergo magis in particulari descendere , \\\hline
2.1.16 & e en qual manera de una vsar del . \textbf{ Et pues que assi es conuiene de desçender } mas en particular mostrando & et quomodo utendum sit eo . \textbf{ Oportet ergo magis in particulari descendere , } qualiter omnes ciues \\\hline
2.1.16 & Et pues que assi es los smones \textbf{ genera les cerca las costunbres non son de despreçiar . } Ca el non saber delas cosas genera les nos faria muchͣs uezes & et maxime Reges et Principes debent uti copula coniugali . Sermones enim uniuersales circa mores despiciendi non sunt : \textbf{ quia } ignorantia uniuersalium saepe facit particularia ignorare : \\\hline
2.1.16 & genera les cerca las costunbres non son de despreçiar . \textbf{ Ca el non saber delas cosas genera les nos faria muchͣs uezes } que non sopiessemos las particulares . & quia \textbf{ ignorantia uniuersalium saepe facit particularia ignorare : } ipsis \\\hline
2.1.16 & Et por ende alas palauras \textbf{ e alos ymones generales deuemos añader los sermones particulares . } Ca commo el negoçio moral & ipsis \textbf{ tamen uniuersalibus sermonibus sunt particularia addenda , } quia cum negocium morale circa particularia consistat \\\hline
2.1.16 & en tales cosas los smones particulares \textbf{ mas proprouecha que los generales . Et por ende deuemos determiuar particulariente } en qual hedat deua ser vsado el casamiento . & secundum doctrinam Philosophi 2 Ethicorum ) \textbf{ in talibus particulares sermones plus proficiunt . Determinandum est ergo particulariter , } in qua aetate sit utendum coniugio . \\\hline
2.1.16 & por que praeua \textbf{ que enla hedat de grand moçedat non deuamos vsar del casamiento ¶ } La primera razon se toma de parte del dannamiento delos fijos ¶ & quatuor rationes probantes \textbf{ quod in aetate nimis iuuenili non est utendum coniugio . } Prima ratio sumitur ex electione filiorum . Secunda , \\\hline
2.1.16 & conuiene que aquella obra non sea acabada . \textbf{ Ca assi commo para escalentar es menester calentura } si aquella calentura non es calentura acabada siguese que non es caliente acabada mente . En essa misma nanera avn pero que algunan cosa sea escalençada es mester & si illud sit imperfectum oportet effectum imperfectum esse : \textbf{ ut si ad calefactionem requiritur calidum , } si illud sit imperfecte calidum , \\\hline
2.1.16 & que sea apareiada \textbf{ para resçebir aquella calentura . } Mas si alguna cosa fuesse non apareiada acabadamente & etiam quia \textbf{ ad hoc quod aliquid calefaciat , | requiritur quod sit dispositum ad susceptionem caloris ; } si aliquid sit imperfecte dispositum ad huiusmodi susceptionem , \\\hline
2.1.16 & Mas si alguna cosa fuesse non apareiada acabadamente \textbf{ para resçebir esta calentura siguese } que non resçibrie esta calentura acabada mente . & requiritur quod sit dispositum ad susceptionem caloris ; \textbf{ si aliquid sit imperfecte dispositum ad huiusmodi susceptionem , } sequitur quod imperfecte calorem suscipiat . \\\hline
2.1.16 & nin es de buean conplission el alma es enbargada \textbf{ por que non pueda bien entender } e que non pueda faze sus obras libremente . & non est bonae complexionis ) impeditur anima \textbf{ ne possit bene speculari , } et ne possit libere exequi actiones suas . Nascentes ergo ex tali coniugio non solum sunt imperfecti corpore , \\\hline
2.1.16 & ¶ Et pues que assi es conuiene a todos los çibdadanos \textbf{ de non vsar de casamiento en hedat de grand moçedat . } Et esto tantomas conuiene alos Reyes e alos prinçipes & Decet ergo omnes ciues \textbf{ non uti coniugio in aetate nimis iuuenili ; } hoc tamen tanto magis decet Reges et Principes , \\\hline
2.1.16 & Et pues que assi es puesto \textbf{ que ellos ayan de tomar mugni muy moça } enpo non deuen vsar de tal casamiento si non en hedat conplida e conueinble¶ & Dato ergo \textbf{ quod contingat eos uxorem ducere in aetate nimis iuuenili , | non tamen debent } uti coniugio \\\hline
2.1.16 & que ellos ayan de tomar mugni muy moça \textbf{ enpo non deuen vsar de tal casamiento si non en hedat conplida e conueinble¶ } La segunda razon & non tamen debent \textbf{ uti coniugio | nisi in aetate debita . } Secunda via ad inuestigandum hoc idem , \\\hline
2.1.16 & La segunda razon \textbf{ para prouar esto mesmo se toma dela destenprança delas mugers . } Ca si en la hedat de grand moçedat las mugers se ayuntaren a sus maridos & Secunda via ad inuestigandum hoc idem , \textbf{ sumitur ex intemperantia mulierum . } Nam si in aetate valde iuuenili uxores suis viris copulentur , \\\hline
2.1.16 & e muchͣs dellas periglan Onde avn en el tienpo antigo \textbf{ assi commo el philosofo dize fue costunbre entre los gentiles de fazer logar speçial de oracion } por el parto delas moças en sennal & Unde et antiquitus \textbf{ ( ut Philosophus recitat ) fuit consuetudo apud gentiles speciale oraculum facere pro partu iuuencularum , } in signum , \\\hline
2.1.16 & mas los resçiben danno \textbf{ si en el t pon del cresçer } e cresçiendo el cuerpo vsaren de lux̉ia . & dicitur quod masculorum corpora laeduntur , \textbf{ si tempore augmenti et crescente corpore utantur venereis . } Quare \\\hline
2.1.16 & Por la qual cosa \textbf{ si estos males que pueden acahesçer } por la grand mançebia & Quare \textbf{ si haec mala , quae ex nimia iuuentute coniugum accidere possunt } tam coniugibus quam eorum filiis , \\\hline
2.1.16 & commo alos fijos dellas estos males \textbf{ mas son de escusar en los Reyes e en los prinçipes } que en los otros & magis vitanda sunt in Princibus \textbf{ et Regibus quam in aliis , | potissime non decet } eos \\\hline
2.1.16 & que en los otros \textbf{ Et por ende mayormente conuiene aellos de non vsar de casamiento en grand moçedat . } Mas si fuere demandado quanto tienpo ha menester & eos \textbf{ uti coniugio in nimia iuuentute . } Sed si quaeratur \\\hline
2.1.16 & Mas si fuere demandado quanto tienpo ha menester \textbf{ para tomar el casamiento delas mugers paresçe } que el philosofo quiere & Sed si quaeratur \textbf{ quantum tempus requiratur in ipsis coniugibus . } Videtur velle Philosophus , \\\hline
2.1.16 & Mas en el uaron ha menester mayor tienpo . \textbf{ Ca si por todo elt podel cresçer es muy enpesçible a los alos } mas los vsar de casamiento . & In viro vero plus temporis requiritur . \textbf{ Nam si per totum tempus augmenti nociuum est masculis uti coniugio , } cum ad tempus augmenti communiter in hominibus requirantur tria septennia , \\\hline
2.1.16 & Ca si por todo elt podel cresçer es muy enpesçible a los alos \textbf{ mas los vsar de casamiento . } Como el tp̃o dela cresçençia demande communalmente en los omes tres setenas de años . & In viro vero plus temporis requiritur . \textbf{ Nam si per totum tempus augmenti nociuum est masculis uti coniugio , } cum ad tempus augmenti communiter in hominibus requirantur tria septennia , \\\hline
2.1.16 & Paresçe que despues del terçerosetenario de los uarones es tienpo conueinble \textbf{ para dar obra al ayuntamientod el casamiento . } Empero por que la fu erça de engendtar es muy corrupta puede se este tienpo antuar & post tertium septennium in viris videtur esse debitum tempus dare operam copulae coniugali : \textbf{ verum } quia vis generatiua est nimis corrupta , \\\hline
2.1.16 & para dar obra al ayuntamientod el casamiento . \textbf{ Empero por que la fu erça de engendtar es muy corrupta puede se este tienpo antuar } si uieren los omes & verum \textbf{ quia vis generatiua est nimis corrupta , | huiusmodi tempus anticipari poterit , } si videbitur expedire . \\\hline
2.1.17 & que prouo por muchos razones \textbf{ que los omes non deue dar obra al casamiento } en la he perdat de grand mançebia demanda & postquam probauit per rationes plurimas , \textbf{ non esse dandam } operam coniugio in aetate nimis iuuenili : \\\hline
2.1.17 & en la he perdat de grand mançebia demanda \textbf{ en quet pon deuen dar mas obra ala generaçion delos fijos . } Et dizen que esto otorgan tan bien los naturales & operam coniugio in aetate nimis iuuenili : \textbf{ inquirit quo tempore magis insistendum est procreationi filiorum , } et ait , \\\hline
2.1.17 & en que vientan los vientos del çierco \textbf{ es meior de dar obra al casamiento . } que en el tp̃o caliente & quo flant venti boreales , \textbf{ melius est dare operam coniugio , } quam calido tempore quo flant australes . \\\hline
2.1.17 & en que vientan los alvientos del abrigo . \textbf{ Et esto podemos prouar } por tres razones . & quam calido tempore quo flant australes . \textbf{ Possumus autem hoc triplici via venari . } Prima sumitur ex parte mulierum . Secunda ex laesione virorum . \\\hline
2.1.17 & mas los mas los que las fenbras . \textbf{ Onde paresçe que el philosofo quiere dezir } que las oueias mas conçiben mallos & Unde \textbf{ et Philosophus videtur velle , quod oues , flantibus borealibus ventis , } magis concipiunt masculos : \\\hline
2.1.17 & quando vientan los uietos de çierço \textbf{ por el esforçamiento dela calentraa del uientre de la madremas pueden guardar las ceraturas } e fazer las mas fuertes & magis foeminas . \textbf{ Omnino enim flantibus borealibus ventis propter roborationem caloris materni uteri , | magis possunt conseruare suos foetus , } et eos perfectiores faciunt . \\\hline
2.1.17 & por el esforçamiento dela calentraa del uientre de la madremas pueden guardar las ceraturas \textbf{ e fazer las mas fuertes } ¶La segunda razon & magis possunt conseruare suos foetus , \textbf{ et eos perfectiores faciunt . } Secunda via ad inuestigandum hoc idem , \\\hline
2.1.17 & ¶La segunda razon \textbf{ para demostrar esto } mesmo se toma del danno delos uarones & et eos perfectiores faciunt . \textbf{ Secunda via ad inuestigandum hoc idem , } sumitur \\\hline
2.1.17 & mas guardada \textbf{ e tornada adentromas podemos conuter en nuestros mienbros dela uianda . Por la qual cosa el vso del ayuntamiento del casamiento } en tal tp̃o & quam frigido flante borea . Tempore enim boreali et frigido \textbf{ quia calor naturalis magis reseruatur interius , plus possumus conuertere } de alimento . Quare usus coniugalis copulae in tali tempore non sic laedit corpora virorum , nec sic attenuat ea , \\\hline
2.1.17 & por que se faze mayorconuerssion dela uianda en el cuerpo del omne . \textbf{ ¶ La terçera razon para mostrar esso mismo se toma de parte dela disposiçion del ayre . } Ca el çierço faze el ayre puro & eo quod maior sit ibi conuersio alimenti . \textbf{ Tertia via ad inuestigandum hoc idem , | sumitur ex aeris dispositione . } Nam boreas reddit aerem purum : \\\hline
2.1.17 & Por la qual cosa en taltp̃o \textbf{ mas es de dar obra al ayuntamiento del casamiento Et pues que assi es conuiene a todos los çibdadanos vsar } mas del casamiento en elt pon & Quare tali tempore magis est danda opera coniugali copulae . \textbf{ Decet ergo omnes ciues uti magis coniugio tempore , } quo sit melior procreatio filiorum : \\\hline
2.1.17 & mas conuiene alos Reyes e alos prinçipes \textbf{ quanto mas les conuiene aellos de auer los fijos grandes e esforcados de cuerpo } euedes saber que las costunbres delas mugers en la mayor parte son & quo sit melior procreatio filiorum : \textbf{ tanto tamen hoc magis decet Reges et Principes , quanto decet eos elegantiores habere filios . } Mulierum autem mores \\\hline
2.1.18 & quanto mas les conuiene aellos de auer los fijos grandes e esforcados de cuerpo \textbf{ euedes saber que las costunbres delas mugers en la mayor parte son } assi commo las costunbres delos moços e de los mançebos . & tanto tamen hoc magis decet Reges et Principes , quanto decet eos elegantiores habere filios . \textbf{ Mulierum autem mores } ut plurimum \\\hline
2.1.18 & en el qual tractamos delas costunbres \textbf{ en general non ouiemos cuydado de fazer capitulo espeçial delas costunbres delas mugers } mas diemos lo a entender & ubi uniuersaliter tractabamus de moribus , \textbf{ non curauimus speciale capitulum } facere de moribus mulierum : \\\hline
2.1.18 & en general non ouiemos cuydado de fazer capitulo espeçial delas costunbres delas mugers \textbf{ mas diemos lo a entender } e dexamoslo a penssar & non curauimus speciale capitulum \textbf{ facere de moribus mulierum : } sed supposuimus coniecturandum esse de huiusmodi moribus ex moribus puerorum . \\\hline
2.1.18 & mas diemos lo a entender \textbf{ e dexamoslo a penssar } que tales son las costunbres delas mugers & facere de moribus mulierum : \textbf{ sed supposuimus coniecturandum esse de huiusmodi moribus ex moribus puerorum . } Verumtamen quia in hoc secundo libro de regimine coniugum specialem requirit tractatum , \\\hline
2.1.18 & por que sepamos \textbf{ por qual gouernamiento son de gouerenar las mugieres . } Conuiene de contar breuemente & ut sciamus \textbf{ quo regimine regendae sint coniuges , } narrandum est sub compendio \\\hline
2.1.18 & por qual gouernamiento son de gouerenar las mugieres . \textbf{ Conuiene de contar breuemente } e en suma quales costunbres son de loar & quo regimine regendae sint coniuges , \textbf{ narrandum est sub compendio } et succincte , \\\hline
2.1.18 & Conuiene de contar breuemente \textbf{ e en suma quales costunbres son de loar } e quales de denostar en las mugers ¶ & narrandum est sub compendio \textbf{ et succincte , | quae sunt laudabilia , } et quae vituperabilia in ipsis foeminis . Est autem primo laudabile in eis , \\\hline
2.1.18 & e en suma quales costunbres son de loar \textbf{ e quales de denostar en las mugers ¶ } Mas lo primero que es de loar en ellas & quae sunt laudabilia , \textbf{ et quae vituperabilia in ipsis foeminis . Est autem primo laudabile in eis , } quia \\\hline
2.1.18 & e quales de denostar en las mugers ¶ \textbf{ Mas lo primero que es de loar en ellas } es que en la mayor parte acaesçe ala mugieres de ser uergonçosas & quae sunt laudabilia , \textbf{ et quae vituperabilia in ipsis foeminis . Est autem primo laudabile in eis , } quia \\\hline
2.1.18 & por que non veen en si mismos sçiençia \textbf{ donde se puedan gozar } lo que non han en uerdat quieren paresçerdelo auer en opinion de los omes ¶ & Sed qui imperfecte cognoscunt , \textbf{ quia non uident in seipsis scientiam unde gaudere possint ; quod non habent in rei ueritate , uolunt habere in hominum opinione . } Quod ergo dictum est de scientia perfecta \\\hline
2.1.18 & donde se puedan gozar \textbf{ lo que non han en uerdat quieren paresçerdelo auer en opinion de los omes ¶ } Et pues que assi es aquello & Sed qui imperfecte cognoscunt , \textbf{ quia non uident in seipsis scientiam unde gaudere possint ; quod non habent in rei ueritate , uolunt habere in hominum opinione . } Quod ergo dictum est de scientia perfecta \\\hline
2.1.18 & e non acabada \textbf{ deuemos entender dela bondat acabada e non acabada . } Ca aquellos que son acabadamente bueons non dessean & et imperfecta , \textbf{ intelligendum est de bonitate completa et incompleta . } Nam qui sunt perfecte boni , \\\hline
2.1.18 & en si mesmos fallan \textbf{ donde se puedan gozar Por la qual cosa non han grant cuydado } de se gozar dela opinion de los omes . & in seipsis inueniunt \textbf{ unde gaudere possint : } propter quod non multum curant gaudere de hominum opinione . \\\hline
2.1.18 & donde se puedan gozar Por la qual cosa non han grant cuydado \textbf{ de se gozar dela opinion de los omes . } por ende commo las mugers comunalmente non sean ennoblesçidas & unde gaudere possint : \textbf{ propter quod non multum curant gaudere de hominum opinione . } Quare cum mulieres communiter non tanta bonitate polleant sicut uiri ; \\\hline
2.1.18 & qen la bondat \textbf{ que los omes por la qual cosa commo la uirguença sea temor de non auer eglesia o de ꝑder alabança } e las mugers son comunalmente uergonçosas & quanto in bonitate sunt imperfectiores illis . \textbf{ Propter quod , } cum uerecundia \\\hline
2.1.18 & por que temen de non ser gliadas \textbf{ o temen perder alabança } la qual dessean con muy grand apetito ¶ pueᷤ & de inglorificatione \textbf{ et de amissione laudis , } mulieres communiter sunt uerecundae , \\\hline
2.1.18 & que las mugers son uergonçosas \textbf{ Otrossi esto mismo se puede prouar } por el temor del coraçon . & Ex ipsa igitur cupiditate laudis probatur mulieres uerecundas esse . \textbf{ Rursus hoc idem probari potest ex timiditate cordis . } Nam cum mulieres sint naturaliter adeo timidae , \\\hline
2.1.18 & por que la uerguença es algun temor assi commo dicho es de suso . \textbf{ Et pues que assi es por muchos razones podemos prouar } que las muger sson uergonçosas . & ut superius dicebatur . \textbf{ Ex diuersis ergo causis probare possumus mulieres uerecundas esse : quicquid } tamen sit de eius causis , \\\hline
2.1.18 & que las muger sson uergonçosas . \textbf{ Enpero que quier que sea destas razones mucho es de alabar en ellas ser uergon cosas } ca por la uerguença dexan de fazer muchs cosas torpes & Ex diuersis ergo causis probare possumus mulieres uerecundas esse : quicquid \textbf{ tamen sit de eius causis , | laudabile est in ipsis esse uerecundas : } quia propter uerecundiam multa turpia dimittunt \\\hline
2.1.18 & Enpero que quier que sea destas razones mucho es de alabar en ellas ser uergon cosas \textbf{ ca por la uerguença dexan de fazer muchs cosas torpes } que non dexarien de fazer & laudabile est in ipsis esse uerecundas : \textbf{ quia propter uerecundiam multa turpia dimittunt } quae non dimitterent , \\\hline
2.1.18 & ca por la uerguença dexan de fazer muchs cosas torpes \textbf{ que non dexarien de fazer } si non las constriniessen la cadena dela uerguença . & quia propter uerecundiam multa turpia dimittunt \textbf{ quae non dimitterent , } nisi eas uerecundiae cathena constringeret . \\\hline
2.1.18 & si non las constriniessen la cadena dela uerguença . \textbf{ ¶ Lo segundo es de loar en las mugers } que son comunalmente piadosas e mibicordiosas & nisi eas uerecundiae cathena constringeret . \textbf{ Secundo est laudabile in mulieribus , } quia communiter sunt piae et misericordes . \\\hline
2.1.18 & que los otros ayan miscderia e piadat dellos \textbf{ por que cada vno de ligero se inclina a fazer alos otros lo que el quarne que los otros fiziessen a el . } Et por ende los uicios de ligero se enpiadan sobre los otros . & uolunt aliis misereri et compati ipsis : \textbf{ quare cum de facili quis inclinetur ad faciendum aliis , | quod ab eis uult fieri sibi ; } senes de facili super aliis miserentur . \\\hline
2.1.18 & Et por ende los uicios de ligero se enpiadan sobre los otros . \textbf{ Mas las mugers por etra razon son mis cordiosas Conuiene a saber } por blandura de coraçon . & senes de facili super aliis miserentur . \textbf{ Mulieres quidem miseratiuae sunt ex mollitie cordis . } Nam habentes cor molle , \\\hline
2.1.18 & Ca los que han el coraçon blando \textbf{ non pueden softir ninguna cosa dura } e por ende luego que veen a algunos sofrir cosas duras han piadat sobre ellos ¶ & Nam habentes cor molle , \textbf{ non possunt aliquid sustinere : } ideo statim miserentur , \\\hline
2.1.18 & non pueden softir ninguna cosa dura \textbf{ e por ende luego que veen a algunos sofrir cosas duras han piadat sobre ellos ¶ } Lo terçero deuemos penssar en las mugers & non possunt aliquid sustinere : \textbf{ ideo statim miserentur , } cum vident aliquos dura pati . Tertio considerandum est in mulieribus , \\\hline
2.1.18 & e por ende luego que veen a algunos sofrir cosas duras han piadat sobre ellos ¶ \textbf{ Lo terçero deuemos penssar en las mugers } que comunalmente tienen a el tremo & ideo statim miserentur , \textbf{ cum vident aliquos dura pati . Tertio considerandum est in mulieribus , } quia communiter nimis excedunt . \\\hline
2.1.18 & que apenas auria omes en el mundo tan desuergonçados \textbf{ que pudiessen fazer cosas tan torpes } Mas esta terçera cosa maguera pueda ser alabada en los buenos . & quod vix inuenirentur viri adeo inuerecundi \textbf{ ut possint tanta turpia operari . } Hoc autem tertium \\\hline
2.1.18 & Mas esta terçera cosa maguera pueda ser alabada en los buenos . \textbf{ Enpero enlos otros es muchos de denostar ¶ } Visto quales cosas son de alabar & ut possint tanta turpia operari . \textbf{ Hoc autem tertium } et si in bonis potest esse laudabile , \\\hline
2.1.18 & Enpero enlos otros es muchos de denostar ¶ \textbf{ Visto quales cosas son de alabar } en las muger s finca de dezir & Hoc autem tertium \textbf{ et si in bonis potest esse laudabile , } in malis vero est vituperabile . Viso , \\\hline
2.1.18 & Visto quales cosas son de alabar \textbf{ en las muger s finca de dezir } que cosas son de denostar en ellas . & et si in bonis potest esse laudabile , \textbf{ in malis vero est vituperabile . Viso , } quae sunt laudabilia in foeminis : \\\hline
2.1.18 & en las muger s finca de dezir \textbf{ que cosas son de denostar en ellas . } Et podemos dez & in malis vero est vituperabile . Viso , \textbf{ quae sunt laudabilia in foeminis : } restat narrare quae sunt vituperabilia in eis . Possumus autem narrare tria in mulieribus vituperabilia . Primo , \\\hline
2.1.18 & Et podemos dez \textbf{ que tres cosas son de denostar en ellas ¶ } Lo primero que por la mayor parte son destep̃das e seguidoras delas passiones ¶ & quae sunt laudabilia in foeminis : \textbf{ restat narrare quae sunt vituperabilia in eis . Possumus autem narrare tria in mulieribus vituperabilia . Primo , } quia ut plurimum sunt intemperatae , \\\hline
2.1.18 & e el entendimiont non pueden \textbf{ esforçar se } para se arredrar de las cobdiçias & Nam quia in eis ratio deficit , \textbf{ non sic habent } ut retrahantur a concupiscentiis : \\\hline
2.1.18 & esforçar se \textbf{ para se arredrar de las cobdiçias } e de los desseos & non sic habent \textbf{ ut retrahantur a concupiscentiis : } sicut vir , \\\hline
2.1.18 & Por la qual cosa quando se mueuen \textbf{ non sabenertenprar assy mesmas . } Mas sin freno varaian e parlan . & magis hoc faciunt ex verecundia quam ex ratione . Quare cum motae sunt , \textbf{ nesciunt se moderare , } sed sine fraeno garriunt et litigant . Videmus enim foeminas plus perseuerare in litigando \\\hline
2.1.19 & icho es de suso \textbf{ que las mugers non se deuen gouernar } por aquel gouernamiento mismo & Dicebatur in praecedentibus \textbf{ non eodem regimine regendas esse coniuges } quo regendi sunt serui : \\\hline
2.1.19 & por aquel gouernamiento mismo \textbf{ por que se deue gouernar los fijos } nin por aquel gouernamiento & ø \\\hline
2.1.19 & nin por aquel gouernamiento \textbf{ por que se deuen gouernar los sieruos . } Ca dicho es & non eodem regimine regendas esse coniuges \textbf{ quo regendi sunt serui : } aliud enim est regimen coniugales a paternali , \\\hline
2.1.19 & Et otro el sul o de señora sieruo \textbf{ mas por que non abasta de tractar } e dezir & et etiam a seruili . \textbf{ Sed } quia non sufficit sic in uniuersali tractare de regimine coniugali , \\\hline
2.1.19 & mas por que non abasta de tractar \textbf{ e dezir } assi en general del gouernamiento matermonial del marido ala mugni & et etiam a seruili . \textbf{ Sed } quia non sufficit sic in uniuersali tractare de regimine coniugali , \\\hline
2.1.19 & si non dixieremos rtractaremos en espeçial \textbf{ por qual gouernamiento se han de gouernar las mugers . Por ende deuemos dezir en espeçial algunas cosas del gouernamiento del casamiento . } Et pues que assi es deuedes saber & nisi in speciali tractetur , \textbf{ quo regimine regendae sunt coniuges : | ideo oportet in speciali dicere aliqua de regimine coniugum . } Sciendum ergo unam esse communem regulam ad omne regimen . \\\hline
2.1.19 & por qual gouernamiento se han de gouernar las mugers . Por ende deuemos dezir en espeçial algunas cosas del gouernamiento del casamiento . \textbf{ Et pues que assi es deuedes saber } que es vna regla comunal & ideo oportet in speciali dicere aliqua de regimine coniugum . \textbf{ Sciendum ergo unam esse communem regulam ad omne regimen . } Nam quicunque vult aliquid bene regere , \\\hline
2.1.19 & para todo gouernamiento \textbf{ ca qual siquier que quier gouernar bien algunas cosas conuiene } que el aya algunas cautelas espeçiales & Sciendum ergo unam esse communem regulam ad omne regimen . \textbf{ Nam quicunque vult aliquid bene regere , } oportet ipsum speciales habere cautelas ad ea , circa quae videt ipsum magis deficere . Nam sicut est in locutionibus , \\\hline
2.1.19 & en las quales vee \textbf{ que mas ayna puede fallesçer . } Ca assi commo es en las fablas & ø \\\hline
2.1.19 & assi en su manera es en las obras propreas . \textbf{ Ca veemos alguon sauer las lenguas escorrechas } Et ueemos algunos ser tartamudos & sic suo modo est in ipsis operibus . \textbf{ Videmus enim aliquos habere linguas disertas , } aliquos vero balbutientes esse : \\\hline
2.1.19 & mas guauemente vna palaura que otra . \textbf{ Et pues que assi es aquel que qualiesse enderesçar los tartamudos en la fabla conuenir le ha } que los ensseñasse & sed aliqui difficilius proferunt unum verbum , aliqui vero difficilius aliud . \textbf{ Qui ergo balbutientes vellet in loquela dirigere , } oporteret eos \\\hline
2.1.19 & e espeçial esfuerço cerca aquellas palauras \textbf{ que peor pueden pronunçiar . } Ende leemos que algunos philosofos lo fizieron & ut specialem pugnam et specialem conatum acciperent circa ea verba quae deterius proferre possent . \textbf{ Unde et aliquos Philosophos legimus sic fecisse , } qui cum essent impeditae linguae , \\\hline
2.1.19 & e assi fueron fechos bien fablantes . \textbf{ Et pues que assi es en essa misma manera deuemos fazer çerca las obras } por que quando alguno vee & accipientes specialem conatum circa illas literas quas deterius proferebant , facti sunt eloquentes . \textbf{ Hoc ergo modo | et circa opera se habet . } Cum enim quis se vel alium videt circa aliqua deficere , \\\hline
2.1.19 & por que quando alguno vee \textbf{ assi o a otro fallesçer en algunas cosas } si quiere & et circa opera se habet . \textbf{ Cum enim quis se vel alium videt circa aliqua deficere , } si se \\\hline
2.1.19 & si quiere \textbf{ assi o a otro gouernar } derechamente deue tomar espeçial esfuerco çerca aquellas cosas & si se \textbf{ vel alium vult recte regere , } specialem conatum assumere debet circa ea in quibus esse contingit facilior casus . \\\hline
2.1.19 & assi o a otro gouernar \textbf{ derechamente deue tomar espeçial esfuerco çerca aquellas cosas } en que puede caer & vel alium vult recte regere , \textbf{ specialem conatum assumere debet circa ea in quibus esse contingit facilior casus . } Quare cum mulieres \\\hline
2.1.19 & derechamente deue tomar espeçial esfuerco çerca aquellas cosas \textbf{ en que puede caer } e fallesçer mas ligera mente . & vel alium vult recte regere , \textbf{ specialem conatum assumere debet circa ea in quibus esse contingit facilior casus . } Quare cum mulieres \\\hline
2.1.19 & en que puede caer \textbf{ e fallesçer mas ligera mente . } por la qual cosa & specialem conatum assumere debet circa ea in quibus esse contingit facilior casus . \textbf{ Quare cum mulieres } ( ut in praecedenti capitulo dicebatur ) communiter sint intemperatae , \\\hline
2.1.19 & que las aduga atenprança \textbf{ e a callar } e a estar firmes & ut inducantur ad temperantiam , \textbf{ et ad taciturnitatem , } et ad stabilitatem . Partes autem temperantiae \\\hline
2.1.19 & assi commo es dicho en el primer libre son quatro . \textbf{ Conuiene a saber . } ¶ La castidat ¶ La linpieza ¶ honestad ¶abstinençia ¶ Et mesura . & et ad stabilitatem . Partes autem temperantiae \textbf{ ( ut dicebatur in primo libro ) sunt quatuor . videlicet , castitas pudicitia siue honestas , abstinentia , et sobrietas . } Tunc ergo mulieres sunt temperatae , \\\hline
2.1.19 & de ser castas \textbf{ non solamente por guardar fe a sus maridos } mas avn por la generaçion de los fijos . & et sobriae . \textbf{ Decet enim coniuges esse castas non solum propter fidem seruandam suis viris , } sed etiam propter procreandam prolem . \\\hline
2.1.19 & que non seria propreo de su marido uirnia \textbf{ a heredar la heredat de aquel } que non seria su padre . Et pues que & Nam si coniux castitatem non seruat , \textbf{ de facili filius proprius ipsius viri non succedit in haereditatem patris . } Decet ergo coniuges omnium ciuium esse castas : \\\hline
2.1.19 & quanto de los fijos non legitimos dellas \textbf{ podria nasçer mayor contienda e mayor discordia } que lo segundo couiene a el de los otros . & quanto ex earum illegitima prole potest maior lis \textbf{ et discordia , } vel dissensio oriri . Secundo decet eas esse pudicas \\\hline
2.1.19 & por que se guarden de sobrepuiança de vino \textbf{ e de mucho beuer } por que la destenprança tan bien dela uianda commo del beuer aduze desseo para luxia . & Quarto decet eas esse sobrias , \textbf{ ut caueant sibi a superfluitate potus . } Nam tam immoderantia cibi quam potus venerea prouocat . \\\hline
2.1.19 & e de mucho beuer \textbf{ por que la destenprança tan bien dela uianda commo del beuer aduze desseo para luxia . } Onde en el tienpo antigo & ut caueant sibi a superfluitate potus . \textbf{ Nam tam immoderantia cibi quam potus venerea prouocat . } Unde et antiquitus \\\hline
2.1.19 & en el capitulo delas constitucon nes antigas entre las mugers \textbf{ romana sera grand denuesto beuer el vino . } Ende dize & de Institutis antiquis ) \textbf{ quodammodo nefas erat bibere vinum . } Unde ait , \\\hline
2.1.19 & Et pues que assi es \textbf{ assi son de gouernar las mugers } por que sean castas e honestas e abstinentes e mesuradas . & quia proximus a Libero patre intemperantiae gradus ad inconcessam Venerem esse consueuit . Sic igitur regendae sunt foeminae , \textbf{ ut sint castae , honestae , abstinentes , } et sobriae . Modus autem ; quo inducendae sunt \\\hline
2.1.19 & e dela riqueza de sus maridos . \textbf{ Ca los çibdadanos que fallesçen en nobleza e en riqueza deuen enssennar a sus mugers } por si mesmos & secundum diuersitatem nobilitatis diuitiarum ipsorum virorum . \textbf{ Nam ciues in nobilitate et diuitiis deficientes , | debent per seipsos suas instruere coniuges , } et debitas cautelas adhibere , \\\hline
2.1.19 & por si mesmos \textbf{ e dar les castigos conuenibles } por que puedan resplandesçer en las bondades sobredichͣ̃s . & debent per seipsos suas instruere coniuges , \textbf{ et debitas cautelas adhibere , } ut polleant bonitatibus supradictis . Abundantes vero nobilitate , \\\hline
2.1.19 & e dar les castigos conuenibles \textbf{ por que puedan resplandesçer en las bondades sobredichͣ̃s . } Mas los que abondan en nobleza e en rianza e en poderio çiuil & et debitas cautelas adhibere , \textbf{ ut polleant bonitatibus supradictis . Abundantes vero nobilitate , } et diuitiis , \\\hline
2.1.19 & Mas los que abondan en nobleza e en rianza e en poderio çiuil \textbf{ conuiene les de bulcar buenas mugers } e antiguas de buen testimoino prouadas & et ciuili potentia , \textbf{ decet inquirere matronas aliquas boni testimonii per diuturna tempora prudentia } et bonis moribus approbatas , \\\hline
2.1.19 & sobredich̃ͣs¶ \textbf{ Visto en qual manera se deua gouernar la muger } por que ella sea tenprada & et inducentes eam per monitiones debitas ad bonitates praehabitas . Viso , \textbf{ coniugem sic regendam esse , ut sit debite temperata : } restat ostendere , quomodo regenda \\\hline
2.1.19 & por que ella sea tenprada \textbf{ commo cunple finca de demostrar } en qual manera se deua gouernar & coniugem sic regendam esse , ut sit debite temperata : \textbf{ restat ostendere , quomodo regenda } sit , ut debite sit taciturna . Nam , \\\hline
2.1.19 & commo cunple finca de demostrar \textbf{ en qual manera se deua gouernar } por que sea callada conueniblemente . & coniugem sic regendam esse , ut sit debite temperata : \textbf{ restat ostendere , quomodo regenda } sit , ut debite sit taciturna . Nam , \\\hline
2.1.19 & Ca assi commo dize el philosofo \textbf{ en el pramer libro delas politicas } grand conponimiento es delas mugers el silençio . & sit , ut debite sit taciturna . Nam , \textbf{ ut scribitur 1 Polit’ ornamentum mulieris est taciturnitas . } Si enim mulieres debite se habeant , \\\hline
2.1.19 & por las seys bondades sobredichͣs . \textbf{ Conuiene a saber } que sean castas e honestas e abstinentes et mesuradas e calladas e estables e firmes & ut polleant praedictis sex bonitatibus , \textbf{ videlicet , } ut sint castae , honestae , abstinentes , sobriae , taciturnae , et stabiles . \\\hline
2.1.19 & que sean castas e honestas e abstinentes et mesuradas e calladas e estables e firmes \textbf{ Mas a todas estas cosas las pueden adozir los maridos } por si mismos & ut sint castae , honestae , abstinentes , sobriae , taciturnae , et stabiles . \textbf{ Ad haec autem viri eas inducere poterunt } vel per seipsos , \\\hline
2.1.19 & por la qual cosa conuiene a todos los çibdadanos \textbf{ de gouernar a sus mugers assi . } Et esto tanto mas conuiene alos Reyes e alos prinçipes & vel per cautelas alias adhibendo . \textbf{ Quare decet omnes ciues sic suas coniuges regere : } et tanto magis hoc decet Reges , \\\hline
2.1.19 & quanto por el gouernamiento desconuenible dellos \textbf{ puede acaesçer mayor periglo } cerca el gouernamiento del regno & et Principes , \textbf{ quanto ex eorum indebito regimine potest circa regnum maius periculum imminere . } Non sufficit scire , \\\hline
2.1.20 & cerca el gouernamiento del regno \textbf{ en cunple de saber } en qual manera los Reyes e los prinçipes & quanto ex eorum indebito regimine potest circa regnum maius periculum imminere . \textbf{ Non sufficit scire , } quomodo Reges et Principes , \\\hline
2.1.20 & e generalmente todos los çibdadanos \textbf{ y deuen gouernar sus mugers } e aquales bonda deslas de una enduzir & quomodo Reges et Principes , \textbf{ et uniuersaliter omnes ciues debeant suas coniuges regere , } et ad quas bonitates debeant eas inducere : \\\hline
2.1.20 & y deuen gouernar sus mugers \textbf{ e aquales bonda deslas de una enduzir } e traher linon lo pieren & et uniuersaliter omnes ciues debeant suas coniuges regere , \textbf{ et ad quas bonitates debeant eas inducere : } nisi sciatur , \\\hline
2.1.20 & e aquales bonda deslas de una enduzir \textbf{ e traher linon lo pieren } en qual manera se de una auer çerca ellas . & et ad quas bonitates debeant eas inducere : \textbf{ nisi sciatur , } quomodo circa eas debeant se habere . Sunt autem tria \\\hline
2.1.20 & quanto parte nesçe alo present \textbf{ en que deuemos cuydar con grand acuçia } en las quales conuiene alos uarones de se auer & ( quantum ad praesens spectat ) \textbf{ diligenter consideranda , } in quibus viros circa proprias coniuges decet debite se habere . \\\hline
2.1.20 & en que deuemos cuydar con grand acuçia \textbf{ en las quales conuiene alos uarones de se auer } con ueinblemente çerca sus mugers . & diligenter consideranda , \textbf{ in quibus viros circa proprias coniuges decet debite se habere . } Nam primo debent eis moderate \\\hline
2.1.20 & con ueinblemente çerca sus mugers . \textbf{ Ca primero deuen vsar con ellas tenpradamente e sabiamente ¶ } Lo segundo deuen las tractar honrradamente¶ & in quibus viros circa proprias coniuges decet debite se habere . \textbf{ Nam primo debent eis moderate | et discrete uti . } Secundo debent eas honorifice tractare . \\\hline
2.1.20 & Ca primero deuen vsar con ellas tenpradamente e sabiamente ¶ \textbf{ Lo segundo deuen las tractar honrradamente¶ } Lo terçero deuen beuir con ellas conueniblemente . ¶ Lo primero se praeua & et discrete uti . \textbf{ Secundo debent eas honorifice tractare . } Tertio debent \\\hline
2.1.20 & Lo segundo deuen las tractar honrradamente¶ \textbf{ Lo terçero deuen beuir con ellas conueniblemente . ¶ Lo primero se praeua } assi que conuien e alos uarones de vsar con sus muger stenpdamente e sabia mente . & Secundo debent eas honorifice tractare . \textbf{ Tertio debent | cum eis debite conuersari . } Decet enim eos suis coniugibus moderate \\\hline
2.1.20 & Lo terçero deuen beuir con ellas conueniblemente . ¶ Lo primero se praeua \textbf{ assi que conuien e alos uarones de vsar con sus muger stenpdamente e sabia mente . } Ca el vso tenpdo & cum eis debite conuersari . \textbf{ Decet enim eos suis coniugibus moderate | et discrete uti : } nam immoderatus \\\hline
2.1.20 & mas de aquello que abasta \textbf{ para restaurar e cobrar aquello queꝑ } dio la qual cosa non puede ser & et non moderaret desiderium copulae coniugalis ; ageret ultra quam natura requirat , \textbf{ et ultra quam sufficiat ad restaurandum : } quod sine debilitatione proprii corporis esse non poterit . Unde cerebrum , \\\hline
2.1.20 & mienbros nobles el alma es enbargada con el bso dela razon e del entendimiento \textbf{ por que non puede penslar conplidamente } lo que ha de fazer & ø \\\hline
2.1.20 & por que non puede penslar conplidamente \textbf{ lo que ha de fazer } Onde es prouado de suso por la autoridat del philosofo & et aliis membris nobilibus impeditur anima a rationis usu , \textbf{ ut non possit sufficienter considerare ; unde et supra per auctoritatem Philosophi probabatur , } quod talia , \\\hline
2.1.20 & por la qual cosa se tira la opim̃o de algunos omes bestiales \textbf{ que dizen quiero yo agora conplir mi uoluntad . } Ca despues me guardare . & quorundam brutalis cogitatio dicentium , \textbf{ Satisfaciam nunc concupiscentiae , } et postea de caetero abstinebo . \\\hline
2.1.20 & mas se abiua el apetito \textbf{ para vsar della } e sienpra se faz mas destenprado . & et quanto quis plus ea utitur , \textbf{ tanto magis incitatur ad utendum , } et semper intemperantior redditur . \\\hline
2.1.20 & Pues que assi es conuiene a todos los çibdadanos \textbf{ de vsar tenpradamente e mesuradamente del ayuntamiento matermoinal . } Et tanto mas conuiene esto alos Reyes e alos prinçipes & et semper intemperantior redditur . \textbf{ Decet ergo omnes ciues uti moderate coniugali copula , } et tanto magis hoc decet Reges , \\\hline
2.1.20 & quanto mas desconuenible es aellos \textbf{ por tales obras carnales auer el cuerpo enflaqueçido } e auer el alma menguada & quanto indecentius est eos propter \textbf{ huiusmodi actus habere corpus debilitatum , mentem depressam , } et intemperatum appetitum . \\\hline
2.1.20 & por tales obras carnales auer el cuerpo enflaqueçido \textbf{ e auer el alma menguada } e el apetito destenprado nin les cunple a ellos usar de tal ayuntamiento carnal tenpradamente & quanto indecentius est eos propter \textbf{ huiusmodi actus habere corpus debilitatum , mentem depressam , } et intemperatum appetitum . \\\hline
2.1.20 & e auer el alma menguada \textbf{ e el apetito destenprado nin les cunple a ellos usar de tal ayuntamiento carnal tenpradamente } si non vsaren del sabia mente . & huiusmodi actus habere corpus debilitatum , mentem depressam , \textbf{ et intemperatum appetitum . | Nec sufficit eos tali copula uti temperate , } nisi utantur ea discrete . \\\hline
2.1.20 & ca en los t pons \textbf{ en que deuen estar en oraçion conuiene les de se arredrar de tales obras . } Et avn assi en los trons & Nam temporibus , \textbf{ quibus est orationibus vacandum , | decet a talibus abstinere : } sic etiam temporibus , \\\hline
2.1.20 & en que se puede le una tardanno alos fijos \textbf{ conuiene les de guardar se de allegar se mucho alas mugers . Et pues que assi es conuiene les alos casados de guardar tienpo conuenible } e avn assi les conuiene de guardar logar conuenible en manera conuenible & quibus posset insurgere nocumentum proli , \textbf{ decet abstinere a tali copula : | est ergo obseruandum tempus debitum . Sic etiam obseruandus est locus congruus } et modus conueniens , \\\hline
2.1.20 & conuiene les de guardar se de allegar se mucho alas mugers . Et pues que assi es conuiene les alos casados de guardar tienpo conuenible \textbf{ e avn assi les conuiene de guardar logar conuenible en manera conuenible } por que sea entre los casados non solamente amestança de delectaçion & est ergo obseruandum tempus debitum . Sic etiam obseruandus est locus congruus \textbf{ et modus conueniens , } ut sit \\\hline
2.1.20 & ¶ Visto \textbf{ en qual manera conuiene alos uarons de vlar labiamente } e tenpradamente de sus muger sfinca de ver & restat videre , \textbf{ quomodo eas debeant honorifice pertractare . } Quemlibet enim virum \\\hline
2.1.20 & en qual manera conuiene alos uarons de vlar labiamente \textbf{ e tenpradamente de sus muger sfinca de ver } en qual manera conuiene alos uarones de tractar sus mugers honradamente . & quomodo eas debeant honorifice pertractare . \textbf{ Quemlibet enim virum } secundum possibilem facultatem decet \\\hline
2.1.20 & e tenpradamente de sus muger sfinca de ver \textbf{ en qual manera conuiene alos uarones de tractar sus mugers honradamente . } Ca conuiene aquel si quier uaron de tener su muger honrradamente & quomodo eas debeant honorifice pertractare . \textbf{ Quemlibet enim virum } secundum possibilem facultatem decet \\\hline
2.1.20 & en qual manera conuiene alos uarones de tractar sus mugers honradamente . \textbf{ Ca conuiene aquel si quier uaron de tener su muger honrradamente } segunt su poder e sus riquezas & Quemlibet enim virum \textbf{ secundum possibilem facultatem decet } suam uxorem honorifice retinere in debito apparatu , \\\hline
2.1.20 & Mas por que fue mostrado de suso \textbf{ que la mugni non se deue auer al marido } assi commo sierua & redundat in persona ipsius viri . \textbf{ Immo cum ostensum sit supra uxorem non se habere ad virum quasi seruam , } sed quasi sociam , \\\hline
2.1.20 & Por ende conuiene a cada vn marido \textbf{ segunt su estado de tractar muy honrradamente a su muger . } ¶ Mostrado & decet quamlibet \textbf{ secundum suum statum uxorem propriam honorifice pertractare . Ostenso , } quomodo decet viros suis uxoribus moderate \\\hline
2.1.20 & e tenprada mente . \textbf{ Et en qual manera las deuen tener honrradamente } finca de demostrar & quomodo decet viros suis uxoribus moderate \textbf{ et discrete uti , et quomodo debent eas honorifice retinere : } restat ostendere , \\\hline
2.1.20 & Et en qual manera las deuen tener honrradamente \textbf{ finca de demostrar } en qual manera deuen beuir conellas & et discrete uti , et quomodo debent eas honorifice retinere : \textbf{ restat ostendere , } qualiter cum eis debeant conuersari . \\\hline
2.1.20 & por conuenibles castigos . \textbf{ Mas declarar quales son las señales conueinbles dela mistança } e quales son las moniçonnes & si ei ostendat debita signa amicitiae , \textbf{ et si eas per debitas monitiones instruat . Declarare autem quae sunt signa amicitiae debita , } et quae sunt monitiones congruae , \\\hline
2.1.20 & Et si non fueren penssadas las condiconnes delas perssonas . \textbf{ ca los maridos deuen catar con grand acuçia } si las muger sson soƀuias o si son homildosas & et consideratis conditionibus personarum : \textbf{ debent enim viri diligenter aduertere , } utrum uxores sint superbae , \\\hline
2.1.20 & Ca assi deuemos beuir con las mugers \textbf{ que les deuemos mostrar muchͣs señales de amistança } si fueren humildosas & aut fatuae . \textbf{ Nam sic conuersandum est cum uxoribus , quod plura signa amicitiae ostendenda sunt eis , } si sint humiles , \\\hline
2.1.20 & si les fuere mostrada grand amistança \textbf{ que avn quieren enssenorear a sus maridos . } ¶ Otrossi en tal manera deuemos beuir con ella sperando mientes & si eis multa amicitia ostendatur , \textbf{ ut velint | etiam viris propriis dominari . } Rursus sic conuersandum est cum eis , \\\hline
2.1.20 & que avn quieren enssenorear a sus maridos . \textbf{ ¶ Otrossi en tal manera deuemos beuir con ella sperando mientes } que en otra manera son de enssennar las sabias & etiam viris propriis dominari . \textbf{ Rursus sic conuersandum est cum eis , } quod aliter instruendae sunt prudentes , \\\hline
2.1.20 & ¶ Otrossi en tal manera deuemos beuir con ella sperando mientes \textbf{ que en otra manera son de enssennar las sabias } e en otra manera las locas . & Rursus sic conuersandum est cum eis , \textbf{ quod aliter instruendae sunt prudentes , } aliter fatuae . \\\hline
2.1.20 & Et para correction e castigamiento delas sabias abastan palauras amorosas e blandas . \textbf{ Mas alas locas es menester de dar denuesto mas apero } de que ayan algun miedo ¶ & Nam prudentibus ad correptionem leuia verba , \textbf{ et blanda sufficiunt : fatuis vero est asperior increpatio adhibenda . } Decet ergo quoslibet viros , considerato proprio statu , \\\hline
2.1.20 & penssando el su estado propreo \textbf{ e catadas las condiconnes delas perssonas mostrar a sus mugersseñales conueibles de amor } e enssennarlas & Decet ergo quoslibet viros , considerato proprio statu , \textbf{ et inspectis conditionibus personarum , suis uxoribus ostendere debita amicitiae signa , } et eas \\\hline
2.1.20 & e catadas las condiconnes delas perssonas mostrar a sus mugersseñales conueibles de amor \textbf{ e enssennarlas } assi commo conuiene & et inspectis conditionibus personarum , suis uxoribus ostendere debita amicitiae signa , \textbf{ et eas } ( ut expedit ) per debitas monitiones instruere . \\\hline
2.1.21 & e generalmente a todos los çibdadanos \textbf{ saber } en qual manera couiene alas sus mugers & Decet Reges , et Principes , \textbf{ et uniuersaliter omnes ciues scire , } quomodo circa ornatum corporis deceat suas coniuges debite se habere . \\\hline
2.1.21 & en qual manera couiene alas sus mugers \textbf{ de se auer conueniblemente } en el conponimiento e honrramiento de sus cuerpos . & et uniuersaliter omnes ciues scire , \textbf{ quomodo circa ornatum corporis deceat suas coniuges debite se habere . } Nam cum vir suam uxorem regere debeat , \\\hline
2.1.21 & e castiga a su muger \textbf{ deue la castigar } a obras honestas & Nam cum vir suam uxorem regere debeat , \textbf{ eam dirigendo ad actiones honestas , } et ad opera virtuosa : \\\hline
2.1.21 & en todas aquellas cosas \textbf{ que pueden fallesçer } e errar las sus mugers en la mayor parte & expedit quoslibet viros in iis , \textbf{ in quibus ut plurimum delinquunt foeminae , } diligenter aduertere quae sunt ibi licita , \\\hline
2.1.21 & que pueden fallesçer \textbf{ e errar las sus mugers en la mayor parte } e que cosas les son conuenibles & expedit quoslibet viros in iis , \textbf{ in quibus ut plurimum delinquunt foeminae , } diligenter aduertere quae sunt ibi licita , \\\hline
2.1.21 & mayormente pecan en el conponimiento de los cuerpos . \textbf{ Por la qual cosa conuiene alos uarones de saber } qual conponimiento es conuenible alas mugieres & potissime delinquunt circa ornatum corporis ; \textbf{ quare decet viros } cognoscere quis mulierum ornatus sit licitus , \\\hline
2.1.21 & en la meytad de su fazienda \textbf{ deuen retrenar conueiblemente } e castigaras Ꝯ mugeres & vel cuiuscunque ciuis \textbf{ secundum medium sit infelix , } quantum ad ornatum \\\hline
2.1.21 & quanto al conpongmiento e honrramiento de sus cuerpos . \textbf{ Et ahun deuen las refrenar } quanto a todas las otras cosas & quantum ad ornatum \textbf{ et etiam quantum ad omnia alia , } circa quae foeminae consueuerunt delinquere debent \\\hline
2.1.21 & quanto a todas las otras cosas \textbf{ en que pueden las muger serrar ¶ } Et pues que assi es conuiene de sabra & et etiam quantum ad omnia alia , \textbf{ circa quae foeminae consueuerunt delinquere debent } eas debite admonere . \\\hline
2.1.21 & quanto ꝑtenesçe alo presente esta en dos cosas \textbf{ ca el vn conponimiento delas mugerses enfinto para paresçer . } El otro non es infintomas es uerdadero . Mas el conponume ninfinto es dicho & Sciendum ergo ornatum foemineum in duobus consistere , \textbf{ quorum unus potest dici fictitius , alius non fictitius . Fictitius autem ornatus dicitur fucatio , } ut appositio coloris albi vel rubei : \\\hline
2.1.21 & e engannosas son desconueinbles \textbf{ e son les de defender . } Mas otro conponimiento ay & Talia autem quae sunt fictitia et sophistica sunt illicita \textbf{ et prohibenda . Alius autem est ornatus non fictitius , } qui consistit in debitis indumentis , \\\hline
2.1.21 & e commo cunple . \textbf{ Ca conuiene alos maridos de proueer conueniblemente a sus mugerssegunt sus estados } e en vestiduras conuenibles & quae si considerato proprio statu et conditionibus personarum debite et ordinate fiant , sunt licita \textbf{ et honesta . Decet enim viros } secundum suum statum , suis uxoribus , in debitis vestimentis , et in aliis ornamentis , debite prouidere . Unde et Valerius Maximus ciues Romanos commendat , \\\hline
2.1.21 & mas espeçialmente \textbf{ en qual manera se deuen auer conueiblemente en sus uestiduras e en los otros conponimientos del cuerpo . } C suiene de saber & Sed ut uxores ipsas magis specialiter instruamus qualiter circa indumenta \textbf{ et circa alia corporis ornamenta debeant se habere , aduertendum quod circa ornamentum vestimentorum contingit dupliciter peccare . } Primo ex superabundantia . Secundo ex defectu . Superabundantia vero \\\hline
2.1.21 & en qual manera se deuen auer conueiblemente en sus uestiduras e en los otros conponimientos del cuerpo . \textbf{ C suiene de saber } que çerca los conponimientos delas uestiduras podemos pecar en dos maneras ¶ & Sed ut uxores ipsas magis specialiter instruamus qualiter circa indumenta \textbf{ et circa alia corporis ornamenta debeant se habere , aduertendum quod circa ornamentum vestimentorum contingit dupliciter peccare . } Primo ex superabundantia . Secundo ex defectu . Superabundantia vero \\\hline
2.1.21 & C suiene de saber \textbf{ que çerca los conponimientos delas uestiduras podemos pecar en dos maneras ¶ } Lo primero por sobrepiuamiento ¶ & ø \\\hline
2.1.21 & nin se afeytan por vana eglesia \textbf{ mas esto fazen por fazer plazer a sus maridos e por los tirar de forncacion e de luxuria . } Mas estonçe son tenpradas & sed agunt \textbf{ ut suis viris placentes , | eos a fornicatione retrahant . } Tunc vero sunt moderatae , \\\hline
2.1.21 & e non se posie con sse \textbf{ por uana eglesia podria pecar en el conponimiento del cuerpo } si non fuese tenprada . & vel etiam Regis decet magis ornatam esse . \textbf{ Dato igitur quod uxor alicuius viri esset humilis non ornans se propter vanam gloriam , posset delinquere in ornatum , } si non esset moderata , \\\hline
2.1.21 & mas de quanto demanda su estado . \textbf{ Enpero avn podria pecar } si fuesse muy acunçiosa de su conponimiento ¶ & nec ultra suum statum ornamenta appeteret : \textbf{ posset delinquere , } si nimis esset solicita circa ipsa . \\\hline
2.1.21 & si fuesse muy acunçiosa de su conponimiento ¶ \textbf{ Et pues que assi es en estas tres maneras contesçe de pecar } çerca la sobrepuiança de los honrramientos del cuerpo . & si nimis esset solicita circa ipsa . \textbf{ His ergo tribus modis contingit } delinquere circa superfluitatem ornamentorum . \\\hline
2.1.21 & Mas en dos maneras \textbf{ segunt la opimon comun pueden las muger specar } en el fallesçimiento del su conponimiento & Sed duobus modis \textbf{ ( ut communiter ponitur ) contingit delinquere circa defectum . Primo } si hoc fiat ex pigritia \\\hline
2.1.21 & o çerca las uestiduras del lu cudrpo . \textbf{ ¶ Lo segundo en tales cosas pueden pecar } si por el fallesçimiento demandar en loor e eglesia . & si ex ipso defectu quaeratur laus et gloria . \textbf{ Contingit enim aliquando aliquem efferri et superbiri ex ipsa miseria , } quam sustinet . \\\hline
2.1.21 & ¶ Lo segundo en tales cosas pueden pecar \textbf{ si por el fallesçimiento demandar en loor e eglesia . } Ca contesçe alguas uegadas & si ex ipso defectu quaeratur laus et gloria . \textbf{ Contingit enim aliquando aliquem efferri et superbiri ex ipsa miseria , } quam sustinet . \\\hline
2.1.21 & que otro . \textbf{ Et pues que assi es puoden pecar las muger } ssi por uileza de sus uestiduras & si credat ex hoc ab hominibus commendari . Delinquunt igitur mulieres , \textbf{ si ex vilitate habitus , } et ex defectu vestium gloriam quaerant : \\\hline
2.1.21 & Et por ende enesta manera \textbf{ que dicho es son de gouernar las mugt̃s } quanto al conponimiento del cuerpo & et iactantiam mouebantur . \textbf{ Hoc ergo modo regendae sunt coniuges quantum ad ornatum corporis , } ut circa illa sex quae tetigimus , \\\hline
2.1.21 & que non sean sofisticas \textbf{ e engannolas en querer a feytes e aposturas } que non son suyas ¶ Lo segundo que sean homillosas & ne sint sophisticae , \textbf{ quaerentes fucum et figmentum . Secundo , } ut sint humiles , \\\hline
2.1.22 & por que son muy çelosos de sus mugrs . \textbf{ Mas que el grand çelo en los omnes non sea de alabar } esto podemos prouar & quia circa uxores proprias sunt nimis zelotypi . \textbf{ Sed quod nimis zelus | non sit laudabilis , } triplici via ostendere possumus . \\\hline
2.1.22 & Mas que el grand çelo en los omnes non sea de alabar \textbf{ esto podemos prouar } por tres razones . & non sit laudabilis , \textbf{ triplici via ostendere possumus . } Nam cum quis erga suam coniugem est nimis zelotypus , \\\hline
2.1.22 & por que los çelosos son acostunbrados \textbf{ de sospechar las cosas a mal Ca por la sospecha del çelo } que ellos toman & etiam bene acta suspicantur in peius . \textbf{ Ex suspitione autem ipsius zelotypi , } si nimis sit eius zelus , \\\hline
2.1.22 & tres males se le una tan dende \textbf{ de los quales podemos tomar tres razones } para prouar & si nimis sit eius zelus , \textbf{ tria mala consurgunt ; ex quibus tres rationes sumi possunt , } ostendentes nimis zelotypos \\\hline
2.1.22 & de los quales podemos tomar tres razones \textbf{ para prouar } que los muy çelosos non son de loar ¶ & tria mala consurgunt ; ex quibus tres rationes sumi possunt , \textbf{ ostendentes nimis zelotypos } non esse laudandos . Primum est , \\\hline
2.1.22 & para prouar \textbf{ que los muy çelosos non son de loar ¶ } La primera se toma de esto & tria mala consurgunt ; ex quibus tres rationes sumi possunt , \textbf{ ostendentes nimis zelotypos } non esse laudandos . Primum est , \\\hline
2.1.22 & Conuiene alos tales çelosos de ser enbargados enlos cuydados conueibles \textbf{ que han de auer de su casa } e avn de ser enbargados enlas obras çiuiles ¶ & ø \\\hline
2.1.22 & Et tanto mas esto conuiene alos Reyes e alos prinçipes \textbf{ quanto mayor mal e mayor preiizio se puede leunatar al regno } si los Reyes fueren en grand angostura de su coraçon & et tanto magis hoc decet Reges et Principes , \textbf{ quanto maius praeiudicium potest insurgere regno , } si Reges sint in anxietate cordis , \\\hline
2.1.22 & La segunda razon \textbf{ para prouar esto mismo se toma desto } que por ende las muger sson mas abiuadas a mal & et retrahantur a debita cura regni . \textbf{ Secunda via ad inuestigandum hoc idem , } sumitur \\\hline
2.1.22 & la qual cosa fazen los maridos muy çelosos . \textbf{ Por ende non lo pueden sofrir las mugersen paciençia . } por la qual cosa leunatase en aquella casa muchͣs uezes varaias e contiendas . & quod faciunt uiri zelotypi ; \textbf{ non possunt patienter sufferre : } propter quod in domo illa \\\hline
2.1.22 & nin avn les conuiene \textbf{ de non poner alguna guarda en sus mugers } nin les conuiene & Non ergo decet uiros de suis coniugibus esse nimis zelotypos . \textbf{ Nec etiam decet eos circa suas coniuges nullam habere custodiam } et nullum habere zelum , \\\hline
2.1.22 & nin les conuiene \textbf{ avn de non auer algun çelo dellas } Mas penssadas las condiconnes delas perssonas & Nec etiam decet eos circa suas coniuges nullam habere custodiam \textbf{ et nullum habere zelum , } sed consideratis conditionibus personarum , \\\hline
2.1.22 & e catadas las costunbres dela tr̃ra \textbf{ ca da vno deue auer cura } e cuydado conuenible de su muger & sed consideratis conditionibus personarum , \textbf{ et inspectis consuetudinibus patriae quilibet circa propriam coniugem debet debitam curam , } et debitam diligentiam adhibere . Sic enim decet uirum quemlibet erga suam coniugem ornatum habere zelum , \\\hline
2.1.22 & e cuydado conuenible de su muger \textbf{ e deue auer acuçia conuenible de su casa . } Ca assi conuiene a cada vn marido de auer çelo ordenado de su mugni & et inspectis consuetudinibus patriae quilibet circa propriam coniugem debet debitam curam , \textbf{ et debitam diligentiam adhibere . Sic enim decet uirum quemlibet erga suam coniugem ornatum habere zelum , } ut sit \\\hline
2.1.22 & e deue auer acuçia conuenible de su casa . \textbf{ Ca assi conuiene a cada vn marido de auer çelo ordenado de su mugni } por que sea entre ellos amistança natural delectable e honesta & et debitam diligentiam adhibere . Sic enim decet uirum quemlibet erga suam coniugem ornatum habere zelum , \textbf{ ut sit } inter eos amicitia naturalis delectabilis , et honesta . \\\hline
2.1.23 & ¶ Et pues que assi es lo primero \textbf{ que nos deuemos penssar } en el consseio delas mugieres & et expeditus habet rationis usum . Primum igitur , \textbf{ quod est attendendum in consilio mulierum , est , } quia est inualidum . \\\hline
2.1.23 & es que el es muy flaco . \textbf{ Mas lo segundo a que deuemos parar mientes } es que el es apressurado e arrebatado . & quia est inualidum . \textbf{ Secundum vero , | quod est attendendum , } est , \\\hline
2.1.23 & et dela muger \textbf{ si alguno quisiesse obrar adesora } e non podiesse auer deliberacion conplida & citius venit ad suum complementum . Ceteris ergo paribus \textbf{ si quis statim operari deberet , } nec posset \\\hline
2.1.23 & si alguno quisiesse obrar adesora \textbf{ e non podiesse auer deliberacion conplida } para aquel fecho & si quis statim operari deberet , \textbf{ nec posset | ad illud negocium sufficientem deliberationem habere ; } elegibilius esset \\\hline
2.1.23 & para aquel fecho \textbf{ mas de escoger seria el consseio dela mugnỉ } que del uaron & ad illud negocium sufficientem deliberationem habere ; \textbf{ elegibilius esset } consilium muliebre quam virile . \\\hline
2.1.23 & e con sabiduria Ca al sabio pertenesçe \textbf{ mas ayna desenbargar se } e mas pequano tienpo poner en las cosas uiles & quod agat ordinate et prudenter . Prudentis est enim cito se expedire , \textbf{ et modicum tempus apponere in rebus vilioribus } de quibus est minus curandum : \\\hline
2.1.23 & mas ayna desenbargar se \textbf{ e mas pequano tienpo poner en las cosas uiles } delas quales deue auer menor cuydado . & quod agat ordinate et prudenter . Prudentis est enim cito se expedire , \textbf{ et modicum tempus apponere in rebus vilioribus } de quibus est minus curandum : \\\hline
2.1.23 & Por la qual cosa commo el alma sigua ala conplission del cuerpo . \textbf{ A assi commo el cuerpo dela muzer } por que es mas uil & cuius habet esse perfectum quam vir . Quare cum anima sequatur complexionem corporis , \textbf{ sicut ipsum corpus muliebre } eo quod sit vilius , \\\hline
2.1.23 & que deluats . \textbf{ por que si acaesçiesse de obrar alguna cosa adesora } por auentra a seria & ut quia illud est citius in suo complemento , \textbf{ sic oporteret repentino operari , } forte elegibilius esset huiusmodi consilium . \\\hline
2.1.23 & por auentra a seria \textbf{ mas de escoger el su conseio } que del uaron . & sic oporteret repentino operari , \textbf{ forte elegibilius esset huiusmodi consilium . } Triplici via inuestigare possumus , \\\hline
2.1.24 & or tres razones podemos prouarque las mugeres comunalmente \textbf{ e por la mayor parte non puede guardar los secretos ¶ } La primera razon se toma del fallesçimiento dela razon . & quod mulieres communiter , \textbf{ et ut plurimum secreta retinere non possunt . Prima via sumitur ex defectu rationis . } Secunda ex mollicie cordis . \\\hline
2.1.24 & e lanr̃a cobdiçia \textbf{ para obrar aquella cosa . } Por la qual razon commo poner alguna cosa en poridat se a uedar & augetur concupiscentia nostra , \textbf{ ut operemur illud . } Quare cum ponere aliquid in praecepto , \\\hline
2.1.24 & para obrar aquella cosa . \textbf{ Por la qual razon commo poner alguna cosa en poridat se a uedar } que non se diga aquella cosa & ut operemur illud . \textbf{ Quare cum ponere aliquid in praecepto , | sit prohibere , } ne dicatur illud : \\\hline
2.1.24 & que non se diga aquella cosa \textbf{ por este defendemiento sea biua mas la cobdiçia para dezir aquellas poridades } que las otras cosas . & ne dicatur illud : \textbf{ ex ipsa prohibitione incitatur concupiscentia magis ad dicendum secreta , } quam alia ; \\\hline
2.1.24 & por la razon \textbf{ e por el entendimiento non se pueden guardar las poridades conueniblemente . } ¶ Et pues que assi es las mugieres & nisi ergo a ratione huiusmodi concupiscentia refraenetur , \textbf{ non congrue possunt retinere secreta . Mulieres igitur } eo quod ab usu rationis deficiunt , \\\hline
2.1.24 & por que fallesçen de vso de razon non pueden \textbf{ assi refrenar sus cobdiçias e sus appetitos } e por ende son mas reueladoras delas poridades & eo quod ab usu rationis deficiunt , \textbf{ nec possunt sic refraenare concupiscentias , } sunt magis propalatiuae secretorum , \\\hline
2.1.24 & tanto son las mugers mas prestas \textbf{ para fazer mal . } Las cosas defendidas & Nam si prohibitio auget concupiscentiam : \textbf{ tanto sunt proniores mulieres ad faciendum prohibita quam masculi , } quanto ab usu rationis deficientes , \\\hline
2.1.24 & quanto ellas mas fallesçen de vso de razon \textbf{ e tan comenos pueden refrenar los abiuamientos de las cobdiçias } que los uarones & quanto ab usu rationis deficientes , \textbf{ minus possunt refraenare incitamenta concupiscentiarum quam viri . } Secunda via ad inuestigandum hoc idem , \\\hline
2.1.24 & que en los omes ¶ la segunda razon \textbf{ para prouar esto mesmo se toma dela blandura } e del mouimientod el coraçon . & minus possunt refraenare incitamenta concupiscentiarum quam viri . \textbf{ Secunda via ad inuestigandum hoc idem , } sumitur ex molicie cordis . \\\hline
2.1.24 & e mas de ligero son mouibles \textbf{ luego que algunas ꝑssonas les comiençan a lisongar e a Reyr en su faz dellas } luego ellas a aquella perssona tienen por amiga & ut plurimum sunt molles et ductibiles , \textbf{ statim cum aliqua persona eis applaudet , } et ridet in facie earum , credunt ipsam esse amicam , \\\hline
2.1.24 & ¶la terçera razon \textbf{ para prouar esto mismo se toma dela cobdiçia del loor . } Ca en los capitulos sobredichos es dicho & Tertia via ad hoc probandum , \textbf{ sumitur ex appetitu laudis . } Dicebatur enim in praecedentibus , \\\hline
2.1.24 & e algun loor dellas \textbf{ si lo pieren contar alas otras las poridades de sus maridos } por que parescan ser amadas de sus maridos & et laudari \textbf{ apud eas ; | si sciant referre virorum secreta . } Nam quia videntur a suis viris diligi ; \\\hline
2.1.24 & Et por ende de ligero paresçe \textbf{ en qual manera los maridos de una descobrir a sus mugieres los sus secretos . } Ca quando nos dezimos & Ex hoc autem de facili apparet , \textbf{ qualiter viri suis coniugibus debeant reuelare secreta . } Nam cum dicimus hos esse mores iuuenum , \\\hline
2.1.24 & e estas son de e son las mugers \textbf{ non entendemos dezir } que estas cosas tales vienen por neçesidat e por fuerça & hos senum . \textbf{ Non est intelligendum talia necessitatem imponere , } sed solummodo \\\hline
2.1.24 & si quisieren ser constantes e firmes \textbf{ e vençer estos appetitos natraales } e estas iclinaçiones . & ( si velint ) esse constantes , \textbf{ et vincere huiusmodi impetus et inclinationes . } Nam licet sit difficile superare incitamenta concupiscentiarum , \\\hline
2.1.24 & e estas iclinaçiones . \textbf{ Ca conmoquier que sea cosa guaue sobrepuiar } e vençer los entendimientos delas cobdiçias & et vincere huiusmodi impetus et inclinationes . \textbf{ Nam licet sit difficile superare incitamenta concupiscentiarum , } et sit hoc magis difficile in foeminis quam in viris , eo quod magis deficiant a rationis usu . \\\hline
2.1.24 & Ca conmoquier que sea cosa guaue sobrepuiar \textbf{ e vençer los entendimientos delas cobdiçias } et maguer esto sea mas guaue en las mugers & et vincere huiusmodi impetus et inclinationes . \textbf{ Nam licet sit difficile superare incitamenta concupiscentiarum , } et sit hoc magis difficile in foeminis quam in viris , eo quod magis deficiant a rationis usu . \\\hline
2.1.24 & Ca pueden tan bien las mugers \textbf{ commo los uarones rençer estos appetuos } por la qual cosa commo quier que las mugers comunalmente sean descobridas de los secercos enpero non es cosa desconueible & possunt enim tam viri \textbf{ quam mulieres huiusmodi impetus vincere . } Quare licet communiter mulieres sint propalatiuae secretorum , \\\hline
2.1.24 & que algicans mugers quarden los secretos de sus maridos ¶ \textbf{ Et pues que assi es los maridos non deuen descobrir a sus mugers } las sus poridades & non est tamen inconueniens aliquas mulieres esse debita secreta tenentes . \textbf{ Viri igitur non debent suis coniugibus secreta aperire , } nisi per diuturna tempora sint experti , \\\hline
2.1.24 & de ser uagarosas \textbf{ assy commo dicho es de suso deuemos mostrar quales obras conuiene que vsen las mugers } Mas por que dellas diremos adelante & ut superius dicebatur , \textbf{ describenda sunt opera , | quae decet ipsas coniuges exercere . } Sed quia de eis infra dicetur , volumus ea hic silentio praeterire . Primae partis secundi libri de regimine Principum finis , \\\hline
2.1.24 & Mas por que dellas diremos adelante \textbf{ non quegremos a àmas dellas dezir . } cabada la primera parte deste libro segundo & in qua traditum fuit , \textbf{ quo regimine Reges | et Principes debeant suas coniuges regere . } SECUNDA PARS Secundi Libri de regimine Principum : \\\hline
2.2.1 & enla qual dixiemos del gouernamiento del casamiento \textbf{ fincanos de dezir dela segunda parte } en la qual ¶ diremos del gouernamiento del padre alos fijos . & in qua determinatum est de regimine nuptiali , \textbf{ restat exequi de secunda , } in qua agetur de regimine paternali : \\\hline
2.2.1 & en la qual ¶ diremos del gouernamiento del padre alos fijos . \textbf{ Ende non abasta al padre dela casa saber gouernar a su muger } si non sopiere gouernar conueniblemente asus fijos . & in qua agetur de regimine paternali : \textbf{ non enim sufficit patrifamilias , | scire coniugem regere , } nisi nouerit filios debite gubernare . \\\hline
2.2.1 & Ende non abasta al padre dela casa saber gouernar a su muger \textbf{ si non sopiere gouernar conueniblemente asus fijos . } Et pues que assi es deuedessaber & scire coniugem regere , \textbf{ nisi nouerit filios debite gubernare . } Sciendum igitur , \\\hline
2.2.1 & del gouernamiento del casamiento \textbf{ deuiemos determinar del gouernamiento de los sieruos . } Enpero assi commo dize el philosofo & quia domus prima praecedit domum iam in esse perfectam , \textbf{ ideo forte videretur alicui statim post determinationem de regimine nuptiali , determinandum esse de regimine seruorum . Verum quia , } ut dicitur primo Politicorum , oeconomiae \\\hline
2.2.1 & que non delos sieruos . \textbf{ Et por ende nos propusie mos de determinar primeramente del gouernamiento de los fiios } que de los sieruos & et circa liberos quam circa seruos , \textbf{ ideo decreuimus prius determinare de regimine filiali quam de regimine seruili , } tanquam de eo circa quod esse debet amplior cura . In determinando quidem de regimine filiorum , \\\hline
2.2.1 & assi commo de aquello \textbf{ de que deuemos auer mayor cuydado . } Et en determinando del gouernamiento de los fijos primeramente queremos mostrar & ideo decreuimus prius determinare de regimine filiali quam de regimine seruili , \textbf{ tanquam de eo circa quod esse debet amplior cura . In determinando quidem de regimine filiorum , } primo ostendere uolumus , \\\hline
2.2.1 & de que deuemos auer mayor cuydado . \textbf{ Et en determinando del gouernamiento de los fijos primeramente queremos mostrar } que conuiene a todos los padres de ser muy cuydadosos de los fiios . & tanquam de eo circa quod esse debet amplior cura . In determinando quidem de regimine filiorum , \textbf{ primo ostendere uolumus , } quod decet \\\hline
2.2.1 & mas sian mouidos los padres \textbf{ para gouernar bien sus fijos } Et nos podemos mostrar & magis incitabuntur parentes \textbf{ ut suos filios bene regant . Possumus autem triplici via venari , } quod decet huiusmodi solicitudinem habere parentes . Prima via sumitur \\\hline
2.2.1 & para gouernar bien sus fijos \textbf{ Et nos podemos mostrar } por tres razones & magis incitabuntur parentes \textbf{ ut suos filios bene regant . Possumus autem triplici via venari , } quod decet huiusmodi solicitudinem habere parentes . Prima via sumitur \\\hline
2.2.1 & por tres razones \textbf{ que conuienen a todos los padres de auer grand cuydado de sus fijos ¶ } La primera se toma & ut suos filios bene regant . Possumus autem triplici via venari , \textbf{ quod decet huiusmodi solicitudinem habere parentes . Prima via sumitur } ex eo quod patres sunt causa filiorum , \\\hline
2.2.1 & que son menester \textbf{ para la mantener en su ser . } Ca assi commo la naturada ser al fuego & nisi sit solicita circa ea per quae illud valet conseruari in esse , \textbf{ ut si natura dat esse igni , } statim est solicita dare ei leuitatem quandam , \\\hline
2.2.1 & Ca assi commo la naturada ser al fuego \textbf{ luego es cuydadosa de dar le liuiandat } por que pueda sobir suso Por que en logar de suso & ut si natura dat esse igni , \textbf{ statim est solicita dare ei leuitatem quandam , } per quam feratur sursum , \\\hline
2.2.1 & luego es cuydadosa de dar le liuiandat \textbf{ por que pueda sobir suso Por que en logar de suso } mas ha de ser guardado el su ser & statim est solicita dare ei leuitatem quandam , \textbf{ per quam feratur sursum , } eo quod in loco superiori magis habet in esse conseruari quam in inferiori : \\\hline
2.2.1 & Et en essa misma manera avn si la naturada alas ainalias ser . \textbf{ Et por que la aina lia non puede beuir sin comer } luego es cuydados a de dar & sic etiam si natura dat animalibus esse , \textbf{ quia animal sine cibo conseruari non potest , } natura est solicita dare animalibus ora \\\hline
2.2.1 & Et por que la aina lia non puede beuir sin comer \textbf{ luego es cuydados a de dar } a todas las aian lias bocas & quia animal sine cibo conseruari non potest , \textbf{ natura est solicita dare animalibus ora } et alia organa , \\\hline
2.2.1 & e todos los organos e instrumentos \textbf{ por los quales puedan tomar la uianda } qual les conuiene . & et alia organa , \textbf{ per quae possunt sumere cibum } et nutrimentum . \\\hline
2.2.1 & e razon de los fijos \textbf{ e los fijos naturalmente han el ser de los padres . Conuiene alos padres de auer cuydado de los fijos } e ser cuydadosos dellos & Quare si patres sunt causa filiorum , \textbf{ et filii naturaliter a patribus esse habent , | decet patres habere curam filiorum , } et solicitari erga eos , \\\hline
2.2.1 & La segunda razon \textbf{ para prouar esto mismo } s en toma de aquello & et per quae valeant conseruari in esse . Secunda via ad inuestigandum hoc idem , sumitur ex eo quod patres sunt praestantiores filiis , \textbf{ et debent praeesse eis . } Naturaliter enim semper superiora in inferiora influunt , \\\hline
2.2.1 & que los fijos \textbf{ e deuen enssennorear a ellos } por que naturalmente las cosas de suso & et ea regulant \textbf{ et conseruant : } videmus enim super caelestia corpora influere \\\hline
2.2.2 & que han alos fijos sean cuy dadosos della \textbf{ aguer que todos los padres de una auer cuydado de sus fijos } assi commo paresçe & quem habent ad filios , solicitari circa eos . \textbf{ Licet omnes patres deceat solicitari circa proprios filios , } ut patet per rationes superius assignatas : \\\hline
2.2.2 & por las razones ya dichos . Mayormente conuiene alos Reyes \textbf{ e alos prinçipeᷤ de auer tal cuydado . } Et esto podemos prouar & maxime tamen decet Reges , \textbf{ et Principes talem solicitudinem gerere . } Quod triplici via venari possumus . \\\hline
2.2.2 & e alos prinçipeᷤ de auer tal cuydado . \textbf{ Et esto podemos prouar } por tres razones . & et Principes talem solicitudinem gerere . \textbf{ Quod triplici via venari possumus . } Prima via sumitur , \\\hline
2.2.2 & tanto con mayor cuydado \textbf{ e con mayor amor se deue mouer a ella . } Et por ende los padres & et magis cognoscit proprium opus , \textbf{ tanto maiori solicitudine et dilectione mouetur circa illud . } Patres ergo \\\hline
2.2.2 & Et por ende los padres \textbf{ tanto mayor cuydado deuen auer de los fijos } quanto mas sabios son & Patres ergo \textbf{ tanto magis debent solicitari circa filios , } quanto predentiores sunt , \\\hline
2.2.2 & que los Reyes e los prinçipes \textbf{ e generalmente todos los señores sy de una naturalmente ensseñorear conuiene les } que ayan sabidia e entendimiento . & et Principes \textbf{ et uniuersaliter omnes dominantes , | si debeant naturaliter dominari , } oportet quod polleant prudentia et intellectu : \\\hline
2.2.2 & Et tanto mas conuiene alos Reyes \textbf{ e alos prinçipes de auer mayor cuydado de sus fijos } que alos otros & oportet quod polleant prudentia et intellectu : \textbf{ tanto decet Reges et Principes magis solicitari circa proprios filios quam ceteri , } quanto in eis magis vigere \\\hline
2.2.2 & mas deue ser de sabiduria e de entendimiento \textbf{ para gouernar } que en los otros ¶ & ø \\\hline
2.2.2 & La segunda razon \textbf{ para propuar esto mismo se toma dela bondat de los fijos . } Ca conuiene alos fijos de los Reyes & quanto in eis magis vigere \textbf{ debet mentis industria et prudentia regitiua . Secunda via ad inuestigandum hoc idem , sumitur ex bonitate filiorum . } Decet enim filios Regum et Principum maiori bonitate \\\hline
2.2.2 & e de los prinçipes \textbf{ de auer mayor bondat e mayor nobleza que los otros . } Ca segunt el philosofo enlas politicas . & Decet enim filios Regum et Principum maiori bonitate \textbf{ pollere quam alios : } quia secundum Philosophum in Politic’ \\\hline
2.2.2 & Ca conuenible cosa es \textbf{ que aquel que quiere gouernar los otros sea tan sabio } e ran bueno & et esse magis perfecti scientia et virtutibus . Congruum enim est \textbf{ qui alios regere cupit , | ut si adeo prudens } et bonus , \\\hline
2.2.2 & e ran bueno \textbf{ que todos los otros puedan tomar del exienplo de beuir . } Et pues que assi es los fijos de los Reyes & et bonus , \textbf{ ut ceteri ex ipso possint sumere viuendi exemplum . Filii ergo regum licet non omnes sint reges , } tamen \\\hline
2.2.2 & e en algun sennorio \textbf{ en quel conuiene gouernar los otros } mucho les conuiene de ser sabios e buenos . & et in aliquo dominio , \textbf{ in quo oportet eos alios gubernare ; } maxime decet eos esse prudentes \\\hline
2.2.2 & si los padres ouieron cuydado dellos \textbf{ mas que si se ouieren negligenter te çerca dellos } tanto mas conuiene alo Reyes & si patres circa eos sint soliciti , \textbf{ quam circa ipsos se habeant negligenter , } tanto decet Reges \\\hline
2.2.2 & La terçera razon \textbf{ para prouar esto mismo se toma del prouecho del regno } que desçende dela bondat de aquellos que son en el regno . & et ampliori bonitate . \textbf{ Tertia via ad hoc ostendendum sumitur ex utilitate regni . Bonitas enim regni dependet ex bonitate eorum } qui sunt in regno , \\\hline
2.2.2 & Ca bien commo la sanidat na tra᷑al del cuerpo desçende dela sanidat de todos los mienbros e mayormente dela samdat del coraçon e de los mienbros prinçipales \textbf{ por que el coraçon e los mienbros prinçipales han de dar uirtud alos otros mienbros } e enderescarlos e garlos . & et membrorum principalium , \textbf{ eo quod cor et principalia membra habent influere in alia et rectificare ipsa : } sic bonitas regni dependet ex bonitate omnium ciuium ; \\\hline
2.2.2 & pues que assi es prouechosa cosa es a todo el regno \textbf{ de auer bueon sçibdadanos . } Mas mas prouechosa cosa es de auer bueons prinçipes & et dominantur in regno . \textbf{ Utile est ergo toti regno habere bonos ciues , } sed utilius est habere bonos principantes , \\\hline
2.2.2 & de auer bueon sçibdadanos . \textbf{ Mas mas prouechosa cosa es de auer bueons prinçipes } por que alos prinçipes parte nesçe de gouernar & Utile est ergo toti regno habere bonos ciues , \textbf{ sed utilius est habere bonos principantes , } eo quod principantis sit alios regere et gubernare : \\\hline
2.2.2 & Mas mas prouechosa cosa es de auer bueons prinçipes \textbf{ por que alos prinçipes parte nesçe de gouernar } e de garalo sots . & sed utilius est habere bonos principantes , \textbf{ eo quod principantis sit alios regere et gubernare : } tanto ergo magis decet Reges et Principes solicitari circa proprios filios , \\\hline
2.2.2 & quanto mayor prouecho seleunata al regno dela bodat de los fijos delos Reyes \textbf{ que deuen auer el prinçipado e el senorio en el regno } que dela bondat e dela sabiduria de los otros & quanto maior utilitas consurgit ipsi regno ex bonitate filiorum Regum , \textbf{ qui debent habere principatum et dominium in regno ; } quam ex bonitate et prudentia aliorum . \\\hline
2.2.3 & auemos atrattardel gouernamiento paternal . \textbf{ Conuienne de uer onde toma comienço el gouernamiento paternal . } Et por qual gduernamiento son de gouernar los fijos ¶ & determinandum est de regimine paternali : \textbf{ videndum est , | unde sumit originem regimen paternum , } et quo regimine regendi sunt filii . \\\hline
2.2.3 & Conuienne de uer onde toma comienço el gouernamiento paternal . \textbf{ Et por qual gduernamiento son de gouernar los fijos ¶ } pues que assi es conuiene de saber & unde sumit originem regimen paternum , \textbf{ et quo regimine regendi sunt filii . } Sciendum ergo triplex esse regimen . \\\hline
2.2.3 & Et por qual gduernamiento son de gouernar los fijos ¶ \textbf{ pues que assi es conuiene de saber } que tres son los gouernamientos . & et quo regimine regendi sunt filii . \textbf{ Sciendum ergo triplex esse regimen . } Nam omnis regens alios , \\\hline
2.2.3 & e segunt çiertos abenemientos . \textbf{ Et tal gouer namiento } segunt dicho es de lulo es nonbrado politico o çiuil o los gouenena segunt albedrio . & secundum certa pacta . \textbf{ Et tale regimen } ( ut superius dicebatur ) nominatur politicum \\\hline
2.2.3 & Et tal gouernamiento es dicho real . \textbf{ Mas estos tres gouernamientos seguñ los quales veemos algunos regnar en las çibdades } e en las villas son semeiantes trs gouernamientos & et tale regimen dicitur regale . \textbf{ His autem tribus regiminibus , | secundum quod videmus aliquos regnare in ciuitatibus et castris , } assimilantur tria regimina reperta in una domo . \\\hline
2.2.3 & assi conmo dicho es de suso \textbf{ non deue ninguno ensennorear sinplemente } e por aluedrio & Nam regnum coniugale assimilatur regimini politico : \textbf{ quia uxori ( ut in praehabitis tangebatur ) non quis debet praeesse simpliciter ex arbitrio , } sed ei praeesse debet \\\hline
2.2.3 & e por aluedrio \textbf{ mas deue enssennorear la segunt } que demandan las leyes del casamiento e del matermoino o assi commo demandan los abenemientos conueibles e honestos & quia uxori ( ut in praehabitis tangebatur ) non quis debet praeesse simpliciter ex arbitrio , \textbf{ sed ei praeesse debet } ut requirunt leges matrimonii , \\\hline
2.2.3 & e non por establesçimientos e pleitos \textbf{ que caen entre el subdito e el sennar } si non fuere en poderio del subdito & Nam pacta et conuentiones non interueniunt \textbf{ inter subditum et praeeminentem , } nisi sit in potestate subiecti eligere sibi rectorem : \\\hline
2.2.3 & si non fuere en poderio del subdito \textbf{ de esceger su gouernador . } Mas non es en poderio de los fijos de escoger assi mismos padres & inter subditum et praeeminentem , \textbf{ nisi sit in potestate subiecti eligere sibi rectorem : } non est autem in potestate filiorum eligere sibi patrem , \\\hline
2.2.3 & de esceger su gouernador . \textbf{ Mas non es en poderio de los fijos de escoger assi mismos padres } mas por natra al nasçençia los fijos desçenden de los padres . & nisi sit in potestate subiecti eligere sibi rectorem : \textbf{ non est autem in potestate filiorum eligere sibi patrem , } si ex naturali origine filii procederent a parentibus . \\\hline
2.2.3 & Onde el philosofo en el primo libro delas politicas dize \textbf{ que el uaron deue ensennorear ala mug̃ } e alos fiios . & sed regali . Unde et Philosophus 1 Politicorum ait , \textbf{ virum praeesse mulieri , } et natis tanquam liberis . \\\hline
2.2.3 & Enpero non les deue \textbf{ e assennorear de vna manera } nin de vna guasa . & Non \textbf{ tamen eodem modo principandi : } sed mulieri quidem politice , natis autem regaliter . Tertium autem regimen quod est in domo , \\\hline
2.2.3 & nin de vna guasa . \textbf{ Mas ala muger deue enssennorear çibdadana niete } e alos fues real mente & tamen eodem modo principandi : \textbf{ sed mulieri quidem politice , natis autem regaliter . Tertium autem regimen quod est in domo , } vel regimen \\\hline
2.2.3 & que dixiemos en comienço deste capitulo . \textbf{ Ca si el padre deue enssennorear alos fiios realmente } e por el bien dollos commo amar a alguno sea esso mismo & manifeste apparent vera esse quae in principio capituli proponebantur . \textbf{ Nam si pater debet praeesse filiis regaliter } et propter bonum ipsorum : \\\hline
2.2.3 & Ca si el padre deue enssennorear alos fiios realmente \textbf{ e por el bien dollos commo amar a alguno sea esso mismo } que querer qual bien . & Nam si pater debet praeesse filiis regaliter \textbf{ et propter bonum ipsorum : | cum amare aliquod , } idem sit \\\hline
2.2.3 & e por el bien dollos commo amar a alguno sea esso mismo \textbf{ que querer qual bien . } Et el padre deue enssennoreas alos funos & idem sit \textbf{ quod velle ei bonum , } pater \\\hline
2.2.3 & de aquel gouernamiento \textbf{ de que deuer ser gouernados los sieruos . } Ca el senor enssennorea alos sieruos & quod non eodem regimine debent regi filii , \textbf{ quo regendi sunt serui . Nam dominus praeest seruis propter bonum proprium , } non propter bonum seruorum : \\\hline
2.2.3 & e non por el bien de los sieruos . \textbf{ Et esto es segunt el philosofo enssennarear a algunos seruilmente } e non çibdadanamente non entender enssennorear el bien de los sieruos & non propter bonum seruorum : \textbf{ hoc est enim secundum Philosophum } praeesse aliquibus dominatiue , \\\hline
2.2.3 & Et esto es segunt el philosofo enssennarear a algunos seruilmente \textbf{ e non çibdadanamente non entender enssennorear el bien de los sieruos } mas por el luyo propreo . & hoc est enim secundum Philosophum \textbf{ praeesse aliquibus dominatiue , | non intendere bonum ipsorum , } sed proprium : \\\hline
2.2.3 & mas por el luyo propreo . \textbf{ Empero el padre deue enssennorear alos fuos } cor el bien dellos . & sed proprium : \textbf{ pater tamen debet praeesse filiis propter bonum ipsorum filiorum . } Ex ipsa ergo distinctione regiminum : \\\hline
2.2.3 & Enpero por que mas mani fiestamente paresca \textbf{ lo que dezimos todemos prouar pardas rasones } quel gouernamiento qł padre toma comie y de amor¶ & ut manifestius appareat quod dicitur , \textbf{ possumus duplici via venare , } paternale regimen trahere originem ex amore . Prima via sumitur ex ordine naturali . \\\hline
2.2.3 & La primera razon se torna de la orden natural \textbf{ ¶La segunda da partertian del padre¶ } La primera razon paresçe & paternale regimen trahere originem ex amore . Prima via sumitur ex ordine naturali . \textbf{ Secunda } ex ipsa perfectione patris . Prima via sic patet . Nam secundum Philosophum ideo natura dedit vim generatiuam rebus , \\\hline
2.2.3 & assy que segunt el philosofo \textbf{ por ende la natura die uirtud de engendrar alas cosas } por que lo qua nonpite de ser durable & Secunda \textbf{ ex ipsa perfectione patris . Prima via sic patet . Nam secundum Philosophum ideo natura dedit vim generatiuam rebus , } ut quae non possunt perpetuari in seipsis , \\\hline
2.2.3 & por que el fijo naturalmente es vria semerança que desçende del cadre . \textbf{ Canmo segunr nacsta semeianca sea ardenado el amor } per la ore & quia filius naturaliter est quaedam similitudo procedens a patre : \textbf{ cum secundum naturam ad huiusmodi similia fit dilectio , } ex ipso ordine naturali arguere possumus , \\\hline
2.2.3 & per la ore \textbf{ enna tal podemos i prouar } que el çouernamiento del tadre al fiio so funda en amor & cum secundum naturam ad huiusmodi similia fit dilectio , \textbf{ ex ipso ordine naturali arguere possumus , } paternum regimen in amore fundari , \\\hline
2.2.3 & Ca bien commo enlos omes es natra al appetito \textbf{ para engendrar su semeiante } e para cerar sus fijos & et ex amore oriri . \textbf{ Sicut enim est in hominibus naturalis impetus ad producendum sibi simile , } et ad filios procreandum : \\\hline
2.2.3 & para engendrar su semeiante \textbf{ e para cerar sus fijos } assi en ellos es natural appetito & Sicut enim est in hominibus naturalis impetus ad producendum sibi simile , \textbf{ et ad filios procreandum : | sic est in eis } naturalis impetus ad eos diligendum , \\\hline
2.2.3 & assi en ellos es natural appetito \textbf{ para los amar } e para los gouernar & sic est in eis \textbf{ naturalis impetus ad eos diligendum , } et per consequens ad eos gubernandum et regendum , \\\hline
2.2.3 & para los amar \textbf{ e para los gouernar } e para auer grant cuydado dellos ¶ & naturalis impetus ad eos diligendum , \textbf{ et per consequens ad eos gubernandum et regendum , } et ad habendum solicitudinem circa ipsos . \\\hline
2.2.3 & e para los gouernar \textbf{ e para auer grant cuydado dellos ¶ } Et pues que assi es esta es natra al orde & et per consequens ad eos gubernandum et regendum , \textbf{ et ad habendum solicitudinem circa ipsos . } Est ergo hic naturalis ordo , \\\hline
2.2.3 & que el gouernamiento e el cuydado del paradreal fijo toma comienço de amor ¶la segunda razon \textbf{ para prouar esto mismo se toma dela perfecçion del padre . } Ca toda cosa estonçe es acabada & Secunda via ad inuestigandum hoc idem , \textbf{ sumitur ex ipsa perfectione patris . } Nam unumquodque perfectum est \\\hline
2.2.3 & segunt el philosofo \textbf{ quando ꝑuede engendrar su semeiante . } Et commo quier que cada vno ame su perfecçion . & secundum Philosophum , \textbf{ quando potest sibi simile generare . } Quare cum quilibet suam perfectionem diligat , \\\hline
2.2.3 & Visto que el gouernamiento del padre nasçe de amor paresçe \textbf{ que el padre deue enssennorear alos fiios } por el bien de los fijos . & quod paternum regimen ex amore nascitur , \textbf{ patet quod filiis debet } praeesse pater propter bonum filiorum . \\\hline
2.2.3 & por el bien de los fijos . \textbf{ Et por ende non son de gouernar los fuos } por aquel gouernamiento & praeesse pater propter bonum filiorum . \textbf{ Non ergo regendi sunt filii eodem regimine , } quo serui : \\\hline
2.2.3 & por aquel gouernamiento \textbf{ por do son de gouernar los sieruos } Ca los sieruos segunt que el señor se apoderadellos enssennore a los & Non ergo regendi sunt filii eodem regimine , \textbf{ quo serui : | quia seruis } ( \\\hline
2.2.3 & assi pobre \textbf{ que non puede auer conplimiento de sieruos } Ende el philosofo en el sexto libro delas politicas dize & ut quia forte sic pauper est , \textbf{ quod seruorum copiam habere non potest . Unde et Philosophus 6 Polit’ ait , } quod egenis necesse est uti mulieribus et filiis , tanquam seruis . \\\hline
2.2.3 & Ende el philosofo en el sexto libro delas politicas dize \textbf{ que los robres han menester de vsar de sus mugers e de sus fiios } assi commo de sieruos & quod seruorum copiam habere non potest . Unde et Philosophus 6 Polit’ ait , \textbf{ quod egenis necesse est uti mulieribus et filiis , tanquam seruis . } Dicebatur in praecedenti capitulo , \\\hline
2.2.4 & El gouernamiento patrinal toma comienço del amor \textbf{ Et pues que assi es deuemos uer } quanto es el amor de los padres alos fijos & paternale regimen sumere originem ex amore . \textbf{ Videndum est igitur quantus sit amor patrum ad filios , } et filiorum ad patres , \\\hline
2.2.4 & por que nos conosca mos \textbf{ en qual manera de una los padres gouernar alos fijos } e los fujos obedesçer alos padres¶ & ut nobis innotescat , \textbf{ quomodo patres debeant regere filios , } et filii patribus obedire . \\\hline
2.2.4 & en qual manera de una los padres gouernar alos fijos \textbf{ e los fujos obedesçer alos padres¶ } Et pues que assi es deuedes saber segt̃ & quomodo patres debeant regere filios , \textbf{ et filii patribus obedire . } Sciendum ergo per Philosophum 8 Ethic’ \\\hline
2.2.4 & e los fujos obedesçer alos padres¶ \textbf{ Et pues que assi es deuedes saber segt̃ } que dize el philosofo en el . viij̊ . de las ethicas & et filii patribus obedire . \textbf{ Sciendum ergo per Philosophum 8 Ethic’ } triplici ratione probare , \\\hline
2.2.4 & que dize el philosofo en el . viij̊ . de las ethicas \textbf{ que por tres razones podemos prouar } que los padres aman mas alos fijos & Sciendum ergo per Philosophum 8 Ethic’ \textbf{ triplici ratione probare , } parentes plus diligere filios quam econtra . Prima via sumitur ex diuturnitate temporis . Secunda , \\\hline
2.2.4 & La terçera del ayuntamiento de los padres alos fijos ¶ \textbf{ La primera razon se puede prouar assi . } Ca quanto el amor mas dura tanto se faze mayor . & ex certitudine prolis . Tertia , \textbf{ ex unione parentum ad filios . Prima via sic patet . } Nam quanto amor magis durat , \\\hline
2.2.4 & Ca luego que los fijos nasçen los aman los padres . \textbf{ Enpero los fijos luego que nasçen non comiençan a amar los padres } por que luego non son de tan grand conosçimiento & parentes diligunt eos : \textbf{ non tamen filii statim incipiunt amare parentes , } quia statim non sunt tantae cognitionis \\\hline
2.2.4 & por que luego non son de tan grand conosçimiento \textbf{ que pueden conoscer } qual cosa deuen amar . Ca los moços luego que nasçen & quia statim non sunt tantae cognitionis \textbf{ ut possint discernere } quod sit diligendum . \\\hline
2.2.4 & que pueden conoscer \textbf{ qual cosa deuen amar . Ca los moços luego que nasçen } non saben apartar los padres de los otros omes & ut possint discernere \textbf{ quod sit diligendum . | Immo } quia pueri mox nati nesciunt discernere parentes ab aliis , \\\hline
2.2.4 & qual cosa deuen amar . Ca los moços luego que nasçen \textbf{ non saben apartar los padres de los otros omes } nin se inclinan luego a ellos & Immo \textbf{ quia pueri mox nati nesciunt discernere parentes ab aliis , } non statim per amorem efficiuntur ad illos ; \\\hline
2.2.4 & mas quando van cresçiendo \textbf{ e passando el tp̃o pueden departir } e conosçer sus padres entre los otros & non statim per amorem efficiuntur ad illos ; \textbf{ sed per processum temporis , } quando possunt discernere parentes ab aliis , \\\hline
2.2.4 & e passando el tp̃o pueden departir \textbf{ e conosçer sus padres entre los otros } estonçe comiençan atunar los Et & sed per processum temporis , \textbf{ quando possunt discernere parentes ab aliis , } incipiunt eos diligere . Diuturnior est ergo amor parentum ad filios , \\\hline
2.2.4 & e conosçer sus padres entre los otros \textbf{ estonçe comiençan atunar los Et } assi paresçe & sed per processum temporis , \textbf{ quando possunt discernere parentes ab aliis , } incipiunt eos diligere . Diuturnior est ergo amor parentum ad filios , \\\hline
2.2.4 & La segunda razon \textbf{ para puar esto mismo se toma dela c̀tidunbre de los fueros } Ca los padres & quam econuerso : \textbf{ quare fortior et vehementior . Secunda via ad inuestigandum hoc idem , sumitur ex certitudine prolis . } Nam parentes magis sunt certi de sua prole quam proles de suis parentibus : \\\hline
2.2.4 & non es el de tan grand conosçimiento \textbf{ que pueda conosçer } de qual madre salio & non est illius cognitionis \textbf{ ut possit cognoscere a qua matre procedit , } vel a quo patre . \\\hline
2.2.4 & quanto en los padres es mayor çertidunbre de los fuos . \textbf{ Et por aquesta razon se puede mostrar } que las madres aman mas los fiios que los padres & quanto apud parentes est maior certitudo de prole . \textbf{ Ex hac autem ratione ostendi potest , } quod et matres plus diligunt filios , \\\hline
2.2.4 & e conphendido della . \textbf{ Por que en el todo non se puede sennalar alguna cosa } que es alongada mucho dela parte . & quod ab ea comprehendi non potest . \textbf{ In toto enim est assignare aliquid , } quod multum distat a parte : \\\hline
2.2.4 & Mas destas razones sobredichͣs \textbf{ assi commo paresçe podriamos mostrar conplidamente } que parte nesçe alos padres & Ex his autem viis Philosophi \textbf{ ( ut videtur ) | sufficienter arguere possumus , } quod ad parentes spectat solicitari circa regimen filiorum , \\\hline
2.2.4 & o de aquellas cosas que ama con grant amor . \textbf{ Mas alos fiios parte nesçe de obedesçer alos padres } por que cada vno deue obedesçer a aquel que sabe & quia quilibet solicitus esse debet circa ea quae vehementi amore diligit . \textbf{ Ad filios vero pertinet obedire parentibus : } quia quilibet illis obedire debet , \\\hline
2.2.4 & Mas alos fiios parte nesçe de obedesçer alos padres \textbf{ por que cada vno deue obedesçer a aquel que sabe } que lo ama conplida mente . & Ad filios vero pertinet obedire parentibus : \textbf{ quia quilibet illis obedire debet , } quos scit eum vehementer diligere , \\\hline
2.2.4 & si non enel su bien ¶ Et pues que assi es \textbf{ assi podemos prouar } por las razones del philosofo las cosas & quos scit eum vehementer diligere , \textbf{ et non intendere nisi ipsius bonum . Sic ergo ex viis Philosophi probare possumus , } quae in principio capituli dicebantur . \\\hline
2.2.4 & que dicho son en el comienço del capitulo \textbf{ Empero por que mas spanlmente paresca la su entençion deuedes saber } que commo quier que los padres mas sear inclinados alas fijos & quae in principio capituli dicebantur . \textbf{ Tamen ut magis specialiter appareat intentum , } sciendum quod licet parentes magis afficiantur circa filios , et magis intense velint bonum filiorum quam econuerso : \\\hline
2.2.4 & Empero por que mas spanlmente paresca la su entençion deuedes saber \textbf{ que commo quier que los padres mas sear inclinados alas fijos } e mas les pertenesca de querer el bien de los fiios & quae in principio capituli dicebantur . \textbf{ Tamen ut magis specialiter appareat intentum , } sciendum quod licet parentes magis afficiantur circa filios , et magis intense velint bonum filiorum quam econuerso : \\\hline
2.2.4 & que commo quier que los padres mas sear inclinados alas fijos \textbf{ e mas les pertenesca de querer el bien de los fiios } que los fijos de los padres . & Tamen ut magis specialiter appareat intentum , \textbf{ sciendum quod licet parentes magis afficiantur circa filios , et magis intense velint bonum filiorum quam econuerso : } non est tamen inconueniens quantum ad aliquod bonum filios magis diligere quam econuerso . \\\hline
2.2.4 & quanto ha algun bien de los fijos . \textbf{ Mas amar alos padres } que los padres alos fijos . & non est tamen inconueniens quantum ad aliquod bonum filios magis diligere quam econuerso . \textbf{ Nam ut dicitur 8 Ethicorum parentes diligunt filios } ut existentes aliquid ipsorum : \\\hline
2.2.4 & assi commo son possesiones e riquezas e dineros \textbf{ por los quals puedan abondar } assi mismos en la uida & et numismata , \textbf{ per quae sufficiant sibi ad vitam , } et conseruentur in esse . \\\hline
2.2.4 & assi mismos en la uida \textbf{ e mantener le en ella . } ¶ Et pues que assi es los padres allegan & per quae sufficiant sibi ad vitam , \textbf{ et conseruentur in esse . } Parentes ergo congregant pro filiis , \\\hline
2.2.4 & commo amara alguno sea essa misma cosa \textbf{ que querer bien } para el deuemos departir deste bien . & cum diligere aliquem , \textbf{ idem sit quod velle ei bonum , } distinguendum est de ipso bono . \\\hline
2.2.4 & que querer bien \textbf{ para el deuemos departir deste bien . } Ca si fablaremos del bien aprouechable & idem sit quod velle ei bonum , \textbf{ distinguendum est de ipso bono . } Nam si loqueris de bono utili , \\\hline
2.2.4 & que los fiios alos padres \textbf{ por que ayuntan algo para ellos } e non los fiios para los padres . & quam econuerso ; \textbf{ quia congregant pro eis , } non ipsi pro illis . \\\hline
2.2.4 & Porque natra al cosa es \textbf{ que avn los fiios non pue den oyr } nin sofrir los deuuestos que fazen los otros a sus padres . & quam econuerso . Naturale est enim quod etiam usque ad auditum contumelias parentum sustinere non possint . Nam sic indignantur parentes de contumelia filiorum , \textbf{ quam econuerso . Viso , } qualis est amor \\\hline
2.2.4 & que avn los fiios non pue den oyr \textbf{ nin sofrir los deuuestos que fazen los otros a sus padres . } Mas los padres non toman tan grand indignaçion & quam econuerso . Naturale est enim quod etiam usque ad auditum contumelias parentum sustinere non possint . Nam sic indignantur parentes de contumelia filiorum , \textbf{ quam econuerso . Viso , } qualis est amor \\\hline
2.2.4 & ¶ Visto qual es el amor de los radres alos fijos . \textbf{ Ca los padres son inclia a dos a amar los fijos } para allegar les los bienes & et filium , \textbf{ quia parentes afficiuntur ad filios ut congregant eis bona : } et congregare aliis bona , \\\hline
2.2.4 & Ca los padres son inclia a dos a amar los fijos \textbf{ para allegar les los bienes } que les faz menestra . & et filium , \textbf{ quia parentes afficiuntur ad filios ut congregant eis bona : } et congregare aliis bona , \\\hline
2.2.4 & que les faz menestra . \textbf{ Commo allegar les los bienes } e ser cuydadosos & et congregare aliis bona , \textbf{ et solicitari circa eorum vitam , } sit eos regere et gubernare , \\\hline
2.2.4 & e ser cuydadosos \textbf{ çerca la uida dellos sea gouernar les e gouernar los } por el amor & et solicitari circa eorum vitam , \textbf{ sit eos regere et gubernare , } ex amore \\\hline
2.2.4 & por el amor \textbf{ que han los padres alos fijos los deuen gouernar } Mas commo los fijos sean inclinados alos padres . & ex amore \textbf{ quem habent patres ad filios debent eos regere et gubernare . } Sed cum filii afficiantur ad parentes , \\\hline
2.2.4 & Mas commo los fijos sean inclinados alos padres . \textbf{ assi commo aquellos que quieren auer en honrra e en reuerençia . } Commo honrrar & Sed cum filii afficiantur ad parentes , \textbf{ tanquam ad eos , | quos volunt esse in honore et reuerentia : } cum honorari et reuereri alium sit quodammodo subiici illi ; \\\hline
2.2.4 & assi commo aquellos que quieren auer en honrra e en reuerençia . \textbf{ Commo honrrar } e auer reuerençia a otro sea en alguna manera ser subiecto a el . & quos volunt esse in honore et reuerentia : \textbf{ cum honorari et reuereri alium sit quodammodo subiici illi ; } sicut ex amore quem habent patres ad filios debent eos regere \\\hline
2.2.4 & Commo honrrar \textbf{ e auer reuerençia a otro sea en alguna manera ser subiecto a el . } Por ende & quos volunt esse in honore et reuerentia : \textbf{ cum honorari et reuereri alium sit quodammodo subiici illi ; } sicut ex amore quem habent patres ad filios debent eos regere \\\hline
2.2.4 & que han los padres alos fijos \textbf{ los deuen gouernar } ben & sicut ex amore quem habent patres ad filios debent eos regere \textbf{ et gubernare , sic ex dilectione quam filii habent ad patres , } debent eis obedire et esse subiecti . \\\hline
2.2.4 & ben \textbf{ assy por el amor que han los fijnos alos padins deuen les obedesçer } e ser subiectos a ellos . & et gubernare , sic ex dilectione quam filii habent ad patres , \textbf{ debent eis obedire et esse subiecti . } Patet igitur quod quantum ad bonum quod est utile , \\\hline
2.2.5 & assegunt que paresçe tres cosas conuienen ala fe \textbf{ las quales podemos mostrar } por tres razoño & Videntur autem tria ipsi fidei conuenire , \textbf{ per quae triplici via venari possumus , } quod omnes ciues , \\\hline
2.2.5 & Por que la fe lo primero es sobre toda razon \textbf{ e aquellas cosas que son de fe non se pueden prouar } por razon ¶ & ut ab ipsa infantia instruantur in fide . Fides enim primo supra rationem est : \textbf{ et ea , | quae sunt fidei , } ratione comprehendi non possunt . \\\hline
2.2.5 & deuen ser en toda manera creydas ¶ \textbf{ Lo terçero deuen se los omes llegar firmemente } a aquellas cosas que creen . ¶ La primera razon paresçe assi . & quae sunt fidei , \textbf{ simpliciter sunt credenda . Tertio eis est firmiter adhaerendum . } Prima autem via sic patet . \\\hline
2.2.5 & Ca si la fe es sobre razon \textbf{ e aquellas cosas que son de fe non se pueden prouar } por razon prouechosa cosa es que en aquella hedat sean enssennadas las cosas & Nam sic fides supra rationem est , \textbf{ et ea quae sunt fidei ratione comprehendi non possunt : } utile est ut in illa aetate proponantur ea quae sunt fidei , \\\hline
2.2.5 & a las cosas \textbf{ que han de creer non les demandan razon delo que dizen } ca non han vso de razon & non quaerunt rationem dictorum , \textbf{ qui usum rationis non habent , } sed simpliciter acquiescunt dictis . \\\hline
2.2.5 & por lo que les dizen ¶la segunda razon \textbf{ para prouar esto mismo se declara assi . } Ca las cosas que son de fe & Secunda via ad inuestigandum \textbf{ hoc idem sic ostendi potest . } Nam ea quae sunt fidei \\\hline
2.2.5 & por que son sobre razon \textbf{ por ende las deuemos creer sinplemente } e deuemos estar por ellas & quia supra rationem sunt , \textbf{ ideo eis simpliciter est credendum . Acquiescendum est autem iis quae sunt fidei ex auctoritate diuina , } et ex simplici credulitate , \\\hline
2.2.5 & Enpero non es esto sin razon . \textbf{ Ca ninguon non deue dudar } que la sabiduria de dios & non tamen hoc inconuenienter fit , \textbf{ quia nulli dubium esse debet diuinam prudentiam } et eius auctoritatem , \\\hline
2.2.5 & que la sabiduria de dios \textbf{ e la su auctoridat sobrepiua toda sotileza de engennio humanal . por la qual cosa mas prouechosa cosa es de creer } sinplemente la auctoridat de dios & quia nulli dubium esse debet diuinam prudentiam \textbf{ et eius auctoritatem , | omnem perspicaciam humani generis superare . } Quare utilius auctoritati diuinae simpliciter creditur , \\\hline
2.2.5 & ¶ Pues que assi es \textbf{ si aquellas cosas que son de fe son de creer suplemente . } Conueinble cosa es & et demonstrationibus hominum . \textbf{ Si ergo ea quae sunt fidei simpliciter sunt credenda , } conuenienter talia in illa aetate proponuntur , \\\hline
2.2.5 & Ca assi commo dixiemos en el primero libro \textbf{ quando tractauamos delas costunbres de los mançebos } los mançebos son sinples en creer . & talis autem est aetas infantiae . \textbf{ Nam ut in primo libro diximus cum tractauimus de moribus iuuenum , Iuuenes sunt simpliciter creditiui : } unde et in 2 Rhetor’ dicitur , \\\hline
2.2.5 & quando tractauamos delas costunbres de los mançebos \textbf{ los mançebos son sinples en creer . } Onde en el segundo de la rectorica dize el philosofo & Nam ut in primo libro diximus cum tractauimus de moribus iuuenum , Iuuenes sunt simpliciter creditiui : \textbf{ unde et in 2 Rhetor’ dicitur , } quod iuuenes eruditi sunt solum a lege . Appellat autem ibi Philosophus eruditionem a lege , \\\hline
2.2.5 & Ca si \textbf{ e las otras leyes los padres son acuçiosos de enssennar sus fijos } en aquellas cosas & et documenta illius fidei quam parentes tenent . \textbf{ Si enim in aliis legibus parentes } statim sunt soliciti erudire proprios filios in iis quae sunt fidei suae , \\\hline
2.2.5 & que cree de ligero esto \textbf{ tanto mas lo deuen fazer } los que tienen la ley e la fex̉ana & eo quod illa aetas sit simpliciter creditiua : \textbf{ tanto magis hoc debent efficere profitentes christianam fidem , } quanto lex nostra superat caeteras leges . \\\hline
2.2.5 & Ca la ley yana sola es quita de todo ensuziamiento de herror . \textbf{ Mas en todas las otras leyes ay alguas enfer me dades } e algunos herrores ayuntados a ellas ¶ & quanto lex nostra superat caeteras leges . \textbf{ Sola enim christiana lex est immunis ab omni errorum contagio : in caeteris aliis legibus falsitates aliquae sunt admixtae . } Tertia via \\\hline
2.2.5 & La terçera razon \textbf{ para prouar esto se tomad esto } que aquellas cosas que son dela fe nos deuemos allegar firmemente . & Tertia via \textbf{ ad hoc idem probandum , sumitur ex eo quod iis quae sunt fidei est firmiter adhaerendum . } Videmus enim quod consuetudo est \\\hline
2.2.5 & para prouar esto se tomad esto \textbf{ que aquellas cosas que son dela fe nos deuemos allegar firmemente . } por que veemos que la costunbre es assi conmo otra nafa . & Tertia via \textbf{ ad hoc idem probandum , sumitur ex eo quod iis quae sunt fidei est firmiter adhaerendum . } Videmus enim quod consuetudo est \\\hline
2.2.5 & por que podamos mas firmemente \textbf{ e sin ninguna dubda llegar nos a aquellas que son dela fe . } Por ende tales cosas nos deuen anos dezir & ut firmiter et absque haesitatione adhaerere possimus \textbf{ iis quae sunt fidei ; } talia proponenda sunt nobis etiam ab ipsa infantia . \\\hline
2.2.5 & e sin ninguna dubda llegar nos a aquellas que son dela fe . \textbf{ Por ende tales cosas nos deuen anos dezir } e demostrar en tienpo dela moçedat . & iis quae sunt fidei ; \textbf{ talia proponenda sunt nobis etiam ab ipsa infantia . } Unde et Philosophus 2 Meta’ \\\hline
2.2.5 & Por ende tales cosas nos deuen anos dezir \textbf{ e demostrar en tienpo dela moçedat . } Onde el philosofo en el segundo libro dela methafisica & iis quae sunt fidei ; \textbf{ talia proponenda sunt nobis etiam ab ipsa infantia . } Unde et Philosophus 2 Meta’ \\\hline
2.2.5 & Onde el philosofo en el segundo libro dela methafisica \textbf{ quariendo prouar } que la costunbre es de grand & Unde et Philosophus 2 Meta’ \textbf{ volens probare consuetudinem esse magnae efficaciae , } ait , \\\hline
2.2.5 & fuerça dize assi . \textbf{ Tu puedes ver } quanto faze la costunbre & ait , \textbf{ quantam vero vim habeat quod consuetum est , } leges ostendunt , \\\hline
2.2.5 & si las leyes de los gentiles que contienen en si muchͣs fabliellas \textbf{ enlas quales leyes son muchͣs cosas falssas e de escarneçer } e estas son allegadas al coraçon & Si ergo leges Gentilium continentes multas fabulas \textbf{ et apologos idest multa fabulatoria et derisoria , | plus possunt propter consuetudinem , } et sunt sic applicabiles animo , \\\hline
2.2.5 & e sanca en el tronde la moçedat \textbf{ deuen mostrar alos moços todas aquellas cosas } que parte nesçenalafe . & ab ipsa infantia \textbf{ ea quae sunt fidei sunt iuuenibus proponenda . } Verum quia distincte articulos fidei cognoscere , \\\hline
2.2.5 & que parte nesçenalafe . \textbf{ Et por que saber claramente } e departidamente los articulos dela fe & ea quae sunt fidei sunt iuuenibus proponenda . \textbf{ Verum quia distincte articulos fidei cognoscere , } et subtiliter ea quae sunt fidei pertractare , \\\hline
2.2.5 & e departidamente los articulos dela fe \textbf{ et tractar lotilmente aquellas colas } que son dela fe esto pertenesçe alos cłigos doctors & Verum quia distincte articulos fidei cognoscere , \textbf{ et subtiliter ea quae sunt fidei pertractare , } spectat ad clericos doctores instruentes alios in ipsa fide , \\\hline
2.2.5 & que son dela fe esto pertenesçe alos cłigos doctors \textbf{ que han de enformar los otros en la fe . } El qual escudrinnamiento sotil non pueden auer los legos & et subtiliter ea quae sunt fidei pertractare , \textbf{ spectat ad clericos doctores instruentes alios in ipsa fide , } quam subtilem perscrutationem laici , \\\hline
2.2.5 & que han de enformar los otros en la fe . \textbf{ El qual escudrinnamiento sotil non pueden auer los legos } e menos los moços . & ø \\\hline
2.2.5 & e que see ala diestra de dios padre \textbf{ e que dende deue venir a uiyzio . } Et que todos resuçitaremos & sedet ad dexteram Dei patris . \textbf{ Quod iterum venturus est ad iudicium : } et omnes resurgemus , \\\hline
2.2.5 & quanto por el feruor \textbf{ e por la deuoçion dela fe dellos se puede seguir mayor bien en toda la x̉andat . } Et por la mengua dellos podria venir mayor periglo alos xanos & tanto tamen hoc magis decet Reges , et Principes , \textbf{ quanto ex feruore fidei ipsorum potest maius bonum consequi religio christiana , } et ex eorum tepiditate potest ei maius periculum imminere . \\\hline
2.2.5 & e por la deuoçion dela fe dellos se puede seguir mayor bien en toda la x̉andat . \textbf{ Et por la mengua dellos podria venir mayor periglo alos xanos } N quanto el alma es mas noble que el cuerpo & quanto ex feruore fidei ipsorum potest maius bonum consequi religio christiana , \textbf{ et ex eorum tepiditate potest ei maius periculum imminere . } Quanto anima est nobilior corpore , \\\hline
2.2.6 & e mayormente los Reyes \textbf{ e los prinçipes ler mas acuçiosos } por que los lo fijos sean acabados & et maxime Reges , \textbf{ et Principes | magis solicitari debent } ut proprii filii sint perfecti in anima , \\\hline
2.2.6 & si ellos son acuçiosos en los h̃edamientos e en los aueres del mundo \textbf{ por que puedan fazer asus fijos Ricos } e acorrer los quanto ala menguadel cuerpo & et circa numismata , \textbf{ ut possint subuenire filiis quantum ad indigentiam corporalem : } multo magis solicitari debent , \\\hline
2.2.6 & por que puedan fazer asus fijos Ricos \textbf{ e acorrer los quanto ala menguadel cuerpo } mucho mas deuen ser acuçiosos & et circa numismata , \textbf{ ut possint subuenire filiis quantum ad indigentiam corporalem : } multo magis solicitari debent , \\\hline
2.2.6 & por que ellos sean acabados en el alma e en uirtudes \textbf{ e por que sean enformados en bueans costunbres . Et por que esto es vn grant bien non lo deuen dexar peresçer } por negligençia en ningunan manera . & et ut virtutibus et bonis moribus imbuantur . \textbf{ Et quia hoc , | tantum existit bonum , non } debet per negligentiam praeteriri : \\\hline
2.2.6 & por negligençia en ningunan manera . \textbf{ Mas luego en su ninnez son de enssenñar los moços } por que dexen la locania & debet per negligentiam praeteriri : \textbf{ sed ab ipsa infantia | instruendi sunt pueri , } ut relinquentes lasciuiam sequantur bonos mores . Possumus autem quadruplici via venari , \\\hline
2.2.6 & e siguna bueans costunbres \textbf{ Et nos podemos esto mostrar } por quatro razones & ut relinquentes lasciuiam sequantur bonos mores . Possumus autem quadruplici via venari , \textbf{ quod ab ipsa puerilitate } instruendi sunt pueri ad bonos mores . \\\hline
2.2.6 & por quatro razones \textbf{ que los moços son de enssennar en su ninnes en bueans costunbres ¶ } La primera se toma dela naturaleza dela delectacion & quod ab ipsa puerilitate \textbf{ instruendi sunt pueri ad bonos mores . } Prima via sumitur \\\hline
2.2.6 & Ca segunt dize el philosofo en las ethicas \textbf{ en tanto es natural anos de nos delectar enla ninnes } assi que los moços luego se delectan & secundum Philosophum in Ethic’ \textbf{ adeo connaturale est nobis delectari , } quod ab ipsa infantia delectari incipimus : \\\hline
2.2.6 & assi que los moços luego se delectan \textbf{ e quaeçentes si que los mocos luero de mamar . } Et pues que assi es & adeo connaturale est nobis delectari , \textbf{ quod ab ipsa infantia delectari incipimus : } nam et pueri statim delectantur , \\\hline
2.2.6 & para las cosas delectables \textbf{ luego en la moçedat deuemos poner freno } e contradezir ala cobdiçia & cum incipiunt suggere mammas . \textbf{ Si ergo sic ab ipsa infantia nobiscum crescit concupiscentia delectabilium , } ab ipsa infantia est tali concupiscentiae resistendum : \\\hline
2.2.6 & luego en la moçedat deuemos poner freno \textbf{ e contradezir ala cobdiçia } enla nr̃a moçedat . & Si ergo sic ab ipsa infantia nobiscum crescit concupiscentia delectabilium , \textbf{ ab ipsa infantia est tali concupiscentiae resistendum : } ex ipsa ergo connaturalitate delectationis , statim cum pueri sunt sermonum capaces , \\\hline
2.2.6 & que han entendemiento \textbf{ para tomar razon } estonçe son de enssennar en bueans costunbres & ex ipsa ergo connaturalitate delectationis , statim cum pueri sunt sermonum capaces , \textbf{ sunt instruendi ad bonos mores , } et debent eis fieri monitiones debitae . \\\hline
2.2.6 & para tomar razon \textbf{ estonçe son de enssennar en bueans costunbres } e deuen les ser fechos amonestamientos conuenibles . & ø \\\hline
2.2.6 & ¶ La segunda razon \textbf{ para prouar esto mismo se toma del fallesçimiento dela razon } ca estonce son algunos de amonestar a buenas costunbres & Secunda via ad inuestigandum hoc idem , \textbf{ sumitur ex rationis defectu . } Nam tunc aliqui sunt magis mouendi ad bonos mores , \\\hline
2.2.6 & para prouar esto mismo se toma del fallesçimiento dela razon \textbf{ ca estonce son algunos de amonestar a buenas costunbres } quando mas son mouidos ala loçania & sumitur ex rationis defectu . \textbf{ Nam tunc aliqui sunt magis mouendi ad bonos mores , } quando magis incitantur \\\hline
2.2.6 & e siguen sus passiones e sus desseos . \textbf{ Por la qual cosa estonçe los deuemos mucho mas acorrer } assi que por moniconnes conueinbles & et passionum insecutores : \textbf{ ergo tunc maxime est subueniendum , } ut per monitiones debitas , \\\hline
2.2.6 & assi que por moniconnes conueinbles \textbf{ e por castigos conueinbles los podamos refrenar de aquellas loçanias } e de aquellos orgullos & ut per monitiones debitas , \textbf{ et per correctiones conuenientes retrahantur a lasciuiis . } Quare cum rationis sit concupiscentias refraenare \\\hline
2.2.6 & e de aquellos orgullos \textbf{ por que de la razon e del entendimiento es de refrenar los desseos e las locanias . } Et por ende quanto alguon mas fallesçe en razon e en entendimiento & et per correctiones conuenientes retrahantur a lasciuiis . \textbf{ Quare cum rationis sit concupiscentias refraenare | et lasciuias , } quanto aliquis magis a ratione deficit , \\\hline
2.2.6 & tanto mas es inclinado \textbf{ para seguir las passiones e los desseos . } Et pues que assi es en la hedat dela moçedat & tanto magis inclinatur \textbf{ ut sequatur passiones . } In iuuenili ergo aetate sunt pueri instruendi ad bonos mores : \\\hline
2.2.6 & Et pues que assi es en la hedat dela moçedat \textbf{ son los moços de enformar en bueans costunbres } por que entonçe fallesçen & ut sequatur passiones . \textbf{ In iuuenili ergo aetate sunt pueri instruendi ad bonos mores : } quia tunc magis ab usu rationis deficiunt , \\\hline
2.2.6 & en la qual manera se enderesça la piertega tuerta . \textbf{ Ca aquel que quiere endereçar la pierte ga tuerta } inclina la mucho ala parte contraria & quo dirigitur virga tortuosa . \textbf{ Volens enim virgam tortuosam rectificare , } inclinat eam ad partem contrariam valde , \\\hline
2.2.6 & e inclinacion a mal e a delecta connes non conueinbles deuemos \textbf{ por much otp̃o foyr } e guardar nos delas delecta connes non conueninbles & Sic nos , \textbf{ quia obliquitatem et pronitatem habemus ad malum , } et ad delectationes illicitas , \\\hline
2.2.6 & por much otp̃o foyr \textbf{ e guardar nos delas delecta connes non conueninbles } por que podamos escusar e foyr desta inclinaçion . & quia obliquitatem et pronitatem habemus ad malum , \textbf{ et ad delectationes illicitas , } debemus per multum tempus ab illicitis delectationibus abstinere , \\\hline
2.2.6 & e guardar nos delas delecta connes non conueninbles \textbf{ por que podamos escusar e foyr desta inclinaçion . } Mas assi commo la piertega & et ad delectationes illicitas , \textbf{ debemus per multum tempus ab illicitis delectationibus abstinere , } ut possimus hanc pronitatem vitare . Immo sicut virga rectificanda ad partem contrariam , \\\hline
2.2.6 & Mas assi commo la piertega \textbf{ que es de endereçar inclinamos la ala parte contraria } mas que ala meatad & debemus per multum tempus ab illicitis delectationibus abstinere , \textbf{ ut possimus hanc pronitatem vitare . Immo sicut virga rectificanda ad partem contrariam , } ultra medium inclinatur \\\hline
2.2.6 & mas que ala meatad \textbf{ por que la podamos fazer uenir al medio . } En essa misma manera nos en fuyendo delas cosas delectabłs deuemos tris passar & ultra medium inclinatur \textbf{ ut possit ad medium redire : } sic \\\hline
2.2.6 & por que la podamos fazer uenir al medio . \textbf{ En essa misma manera nos en fuyendo delas cosas delectabłs deuemos tris passar } allende del medio onde deuemos foyr & ut possit ad medium redire : \textbf{ sic } et nos in fugiendo delectabilia , debemus ultra medium nos facere , \\\hline
2.2.6 & En essa misma manera nos en fuyendo delas cosas delectabłs deuemos tris passar \textbf{ allende del medio onde deuemos foyr } e esquiuar muchͣs delectaçonnes & sic \textbf{ et nos in fugiendo delectabilia , debemus ultra medium nos facere , } idest debemus multas delectationes \\\hline
2.2.6 & allende del medio onde deuemos foyr \textbf{ e esquiuar muchͣs delectaçonnes } que son conueibles & sic \textbf{ et nos in fugiendo delectabilia , debemus ultra medium nos facere , } idest debemus multas delectationes \\\hline
2.2.6 & que son conueibles \textbf{ por que nos podamos guardar mas ligeramente delas delecta con nes non conueinbles . } Et pues que assi es & idest debemus multas delectationes \textbf{ etiam licitas cauere , ut faciliter ab illicitis abstinere possimus . } Si ergo tantam pronitatem habemus ad malum , \\\hline
2.2.6 & Et pues que assi es \textbf{ si nos auemos tanta inclinaçion a mal conuiene nos de acostunbrar nos } por luengost pons al contrario e al bien & etiam licitas cauere , ut faciliter ab illicitis abstinere possimus . \textbf{ Si ergo tantam pronitatem habemus ad malum , } et oportet nos sic per diuturna tempora assuescere ad contrarium , \\\hline
2.2.6 & por luengost pons al contrario e al bien \textbf{ por que podamos escusar } mas ligeramente esta inclinaçion & ut ad bonum ; \textbf{ ut facilius hanc pronitatem vitare possimus , } est ab ipsa infantia inchoandum , \\\hline
2.2.6 & por que vengamos al bien \textbf{ por ende deuemos comneçar luego enla moçedat } por que dexando las locanias siguamos bueans costunbres . & ut facilius hanc pronitatem vitare possimus , \textbf{ est ab ipsa infantia inchoandum , } ut relinquentes lasciuias sequamur bonos mores , \\\hline
2.2.6 & por que dexando las locanias siguamos bueans costunbres . \textbf{ Et en esto non deuemos poner alongamiento . } ¶ la quarta razon se toma del esquiuamiento delas malas costunbres . & ut relinquentes lasciuias sequamur bonos mores , \textbf{ nec est ulterius differendum . } Quarta via sumitur \\\hline
2.2.6 & e de pecado \textbf{ luego en su moçedat son de amonestar } e de castigar & Ne ergo iuuenes informentur habitibus vitiosis , \textbf{ statim ab ipsa infantia sunt monendi et corrigendi ; } ut per monitiones et correctiones debitas a lasciuiis retrahantur . \\\hline
2.2.6 & luego en su moçedat son de amonestar \textbf{ e de castigar } assi que por bueons amonestamientos & Ne ergo iuuenes informentur habitibus vitiosis , \textbf{ statim ab ipsa infantia sunt monendi et corrigendi ; } ut per monitiones et correctiones debitas a lasciuiis retrahantur . \\\hline
2.2.6 & assi que luego en su moçedat sean acuçiosos \textbf{ en en poter los en bueans costunbres . } Enpero esto & instruentur \textbf{ ad bonos mores . } Tanto tamen hoc magis decet Reges et Principes , \\\hline
2.2.7 & Et quanto dela maliçia dellos vernie mayor periglo a todo el regno \textbf{ omo quier que conuenga a todos los omes de saber letras } por que por ellas puedan ser mas sabios & et quanto ex eorum malitia potest in regno maius periculum imminere . \textbf{ Licet deceret omnes homines cognoscere literas ; } ut per eas prudentiores effecti , \\\hline
2.2.7 & e pueda mas las cosas desconueinbles . \textbf{ Empero paresçe que alguons pueden auer escusacion conueinble } si non trabaiaten en el estudio delas letris . & magis possent illicita praecauere : \textbf{ videntur tamen aliqui licitam excusationem habere , } si non insudant studio literarum . \\\hline
2.2.7 & e estos son tirados delas sciençias libales \textbf{ por que han de buscar las cosas } quales son menester para su uida & qui si retrahantur a liberalibus disciplinis , \textbf{ ut quaerant sibi necessaria vitae ; } videntur excusabiles esse . \\\hline
2.2.7 & e los prinçipes \textbf{ que abondan en riquezas e en possessiones en toda manera son de reprehender } si non fueren acuçiosos en el gouernamiento de sus fijos & et maxime Reges et Principes in diuitiis \textbf{ et possessionibus abundantes , | omnino reprehensibiles existunt , } si non sic solicitantur erga regimen filiorum , \\\hline
2.2.7 & asp que en su moçedat luego sean puestos alas siete artes liberales . \textbf{ Mas que luego enla moçedat de una trabaiar } e comneçar enlas letras esto podemos prouar & ut etiam ab ipsa infantia \textbf{ tradantur liberalibus disciplinis . Quod autem studium literarum sit ab ipsa infantia inchoandum , } possumus triplici via venari . \\\hline
2.2.7 & Mas que luego enla moçedat de una trabaiar \textbf{ e comneçar enlas letras esto podemos prouar } por tres razones ¶ & tradantur liberalibus disciplinis . Quod autem studium literarum sit ab ipsa infantia inchoandum , \textbf{ possumus triplici via venari . } Prima sumitur ex parte eloquentiae . \\\hline
2.2.7 & ¶La segunda de parte dela acuçia \textbf{ que deue auer en el estudio ¶ } La terçera de parte dela perfectiuo & Prima sumitur ex parte eloquentiae . \textbf{ Secunda ex parte attentionis . } Tertia ex parte perfectionis , \\\hline
2.2.7 & Lo primero paresce assi . \textbf{ Ca conuiene que los que quieren aprinder sçiençia de letris } que aprendan pronunçiar departidamente las palabras delas letras ¶ & quae est ex scientia acquirenda . \textbf{ Decet enim volentes literas discere , } literales sermones scire distincte proferre . \\\hline
2.2.7 & Ca conuiene que los que quieren aprinder sçiençia de letris \textbf{ que aprendan pronunçiar departidamente las palabras delas letras ¶ } Et avn conuiene les de ser acuçiosos & Decet enim volentes literas discere , \textbf{ literales sermones scire distincte proferre . } Decet \\\hline
2.2.7 & Lo terçero les conuiene \textbf{ quanto ellos pudieren de venir ala perfecçion } e al conplimiento dela sçiençia ¶ & Tertio decet ipsos , \textbf{ secundum modum sibi possibilem , } ad perfectionem scientiae peruenire . Prima via sic patet . \\\hline
2.2.7 & que muy pocas vezes \textbf{ e apenas puede ninguon pronunçiar con ueinblemente } e departidamente algun lenageiaie & Nam videmus in idiomatibus vulgaribus , \textbf{ quod raro potest quis debite et distincte proferre aliquod idioma , } nisi sit in eo in ipsa infantia assuefactus ; \\\hline
2.2.7 & que esten luengo tienpo en aquellas tierrasapenas \textbf{ o nunca pueden fablar derechamente aquella lengua . } Mas luego son conosçidos de los moradores de aqual la tierra & etiam si per multa tempora in partibus illis existat , \textbf{ vix aut nunquam potest recte loqui linguam illam ; } et ab incolis illius terrae semper cognoscitur ipsum fuisse aduenam , \\\hline
2.2.7 & que nigun lenguage del pueblo non es conplido nin acabado \textbf{ por al qual pudiessen conplidamente pronunçiar las natraas delas cosas e las costunbres de los omes } e los mouimientos delas estrellas & nullum idioma vulgare esse completum \textbf{ et perfectum , per quod perfecte exprimere possent naturas rerum , } et mores hominum , et cursus astrorum , \\\hline
2.2.7 & e los mouimientos delas estrellas \textbf{ e las otras cosas delas quales quarrian disputar fallaron } para si propre o lenguaie & et mores hominum , et cursus astrorum , \textbf{ et alia de quibus disputare volebant , } inuenerunt sibi quasi proprium idioma , \\\hline
2.2.7 & el qual fizieron tan ancho e tan conplido \textbf{ que por el pudiessen razonar } e mostrar acabadamente & vel idioma literale : \textbf{ quod constituerunt adeo latum et copiosum , } ut per ipsum possent omnes suos conceptus sufficienter exprimere . \\\hline
2.2.7 & que por el pudiessen razonar \textbf{ e mostrar acabadamente } te dos los sus conçibimientos . & quod constituerunt adeo latum et copiosum , \textbf{ ut per ipsum possent omnes suos conceptus sufficienter exprimere . } Quare si hoc idioma est completum , \\\hline
2.2.7 & Por la qual cosasi este lenguage es assi conplido \textbf{ e los otros lenguages non los podemos fablar acabadamente } nin departidamente & Quare si hoc idioma est completum , \textbf{ et alia idiomata non possumus recte et distincte loqui , } nisi ab ipsa infantia assuescamus ad illa : \\\hline
2.2.7 & Et por ende paresçe de parte dela fabla \textbf{ si quisieremos fablar derechamente e conplidamente el lenguage del latin } que aprendamos letris & ex parte eloquentiae , \textbf{ videlicet | ut recte et distincte loquamur idioma latinum , } si volumus literas discere , debemus ab ipsa infantia literis insudare . Secunda via ad inuestigandum hoc idem , sumitur ex parte attentionis \\\hline
2.2.7 & ¶ La segunda razon \textbf{ para mostrar esto mismo se toma de parte dela acuçia } que deuemos poner en el estudio . & si volumus literas discere , debemus ab ipsa infantia literis insudare . Secunda via ad inuestigandum hoc idem , sumitur ex parte attentionis \textbf{ et feruoris , } qui est in studio adhibendus : \\\hline
2.2.7 & para mostrar esto mismo se toma de parte dela acuçia \textbf{ que deuemos poner en el estudio . } Ca nunca puede ninguno bien estudiar & et feruoris , \textbf{ qui est in studio adhibendus : } nunquam autem quis bene studet , \\\hline
2.2.7 & que deuemos poner en el estudio . \textbf{ Ca nunca puede ninguno bien estudiar } si non fuere muy cuydado so çerca el estudio . & qui est in studio adhibendus : \textbf{ nunquam autem quis bene studet , } nisi sit feruens , \\\hline
2.2.7 & por que nos podamos ser mas cuydadosos \textbf{ e mas acuçiosos cerca el estudio delas letris deuemos trabaiar en el tpon dela moçedat en las sciençias liberales . } ¶ La terçera razon se toma de parte dela perfecçion & ut possimus esse attenti \textbf{ et feruentes circa studium literale , | ab infantia insudandum est literalibus disciplinis . } Tertia via sumitur ex parte perfectionis , quae est ad scientiam acquirandam . Nam raro aut nunquam quis peruenit \\\hline
2.2.7 & por la sçiençia . \textbf{ Ca pocas vezes o nunca puede ninguno venir a perfectiuo de sçiençia } si non lo comne care de pequanon . & ad perfectionem scientiae , \textbf{ nisi quasi ab ipsis cunabulis vacare incipiat ad ipsam . } Nam licet intelligentiae \\\hline
2.2.7 & luego que fue ton cerados fueron bien ordenados \textbf{ para entender e conosçer las uaturas delas cosas . } Enpero el ome comneco de su nasçimiento es mal despuesto & etiam ab ipsa creatione sint bene dispositae ad intelligendum \textbf{ et ad cognoscendum naturas rerum : } homo tamen a sui natiuitate est male dispositus \\\hline
2.2.7 & e mal ordenado \textbf{ para tomar sçiençia } e para rescebit castigo . & homo tamen a sui natiuitate est male dispositus \textbf{ ad capiendam disciplinam . } Licet enim unus homo sit melioris ingenii quam alius , \\\hline
2.2.7 & e mal a pareiados \textbf{ para laber } Por que el nuestro conosçimiento comiença enl seso & uniuersaliter \textbf{ tamen homines male nati sunt ad sciendum : } quia nostra cognitio incipit a sensu \\\hline
2.2.7 & Onde el philosofo en el primero libro del alma dize \textbf{ que el alma mayor tienpo pone enł non saber } que en la sabiduria . & ø \\\hline
2.2.7 & Ca por muy granttp̃o el omne trabaia eñł estudio \textbf{ ante que pueda venir a perfectiuo e conosçimiento dela sçiençia . } Por la qual cosa si lanr̃a uida es breue & quod anima plus temporis apponat in ignorantia , quam in scientia . Per multum enim temporis quis insudat studio , \textbf{ antequam peruenire possit ad perfectionem scientiae . } Quare si vita nostra est breuis , \\\hline
2.2.7 & e los omes comunalmente son mal ordenados \textbf{ para resçebir la sçiençia . } Si quisieremos venir a alguna perfectiuo de sçiençia deuemos comneçar luego ennraninnes & et scientiae sunt difficiles et longae , \textbf{ et homines communiter male nati sunt ad capescendam scientiam ; } si volumus ad aliquam perfectionem scientiae peruenire , \\\hline
2.2.7 & para resçebir la sçiençia . \textbf{ Si quisieremos venir a alguna perfectiuo de sçiençia deuemos comneçar luego ennraninnes } assi commo quando nos tirassen delas amas . & et homines communiter male nati sunt ad capescendam scientiam ; \textbf{ si volumus ad aliquam perfectionem scientiae peruenire , | debemus ab ipsa infantia } et quasi ab ipsis cunabulis inchoare . \\\hline
2.2.7 & Et si quieren \textbf{ que ellos sean acuçiosos çerca dellas e que puedan venir a alguna perfeççion } e a acabamiento de sçiençia & et si volunt eos esse feruentes , \textbf{ et attentos circa ipsos , | et peruenire ad aliquam perfectionem scientiae , } ab ipsa infantia \\\hline
2.2.7 & e a acabamiento de sçiençia \textbf{ deuenlos luego poner en su moçedat alas letros e alas sçiençias liberales . } Ca assi conmodicho es de suso ninguno non es dich̃ sennor naturalmente & ab ipsa infantia \textbf{ eos tradere literalibus disciplinis . } Nam ( ut superius dicebatur ) nullus est naturaliter dominus , \\\hline
2.2.7 & tanto mas conuiene alos fijos de los Reyes \textbf{ luego en su moçedat de trabaiar } se en las letris e en las sçiençias liberales & tanto magis decet filios Regum , \textbf{ et Principum } etiam ab ipsa infantia insudare literalibus disciplinis , \\\hline
2.2.7 & quanto mas les conuiene de ser mas entendudos e mas sabios que los otros \textbf{ por que puedan enssennorear mas sabiamente } e mas natural mente . & etiam ab ipsa infantia insudare literalibus disciplinis , \textbf{ quanto decet eos intelligentiores } et prudentiores esse , \\\hline
2.2.7 & e mas natural mente . \textbf{ Mas podriemos para prouar esto mismo } adozir otra razon . & et prudentiores esse , \textbf{ ut possint naturaliter dominari . Posset autem ad hoc idem alia ratio adduci . } Nam nisi princeps vigeat prudentia \\\hline
2.2.7 & Mas podriemos para prouar esto mismo \textbf{ adozir otra razon . } Ca si los prinçipes non fueren ennoblesçidos & et prudentiores esse , \textbf{ ut possint naturaliter dominari . Posset autem ad hoc idem alia ratio adduci . } Nam nisi princeps vigeat prudentia \\\hline
2.2.7 & e los bienes tenporales \textbf{ mas de quanto de una preçiar . } Et por esta razd̃ serian tir annos e robadores del pueblo . & sed appretiabitur nummismata , \textbf{ et exteriora bona ultra quam debeat . } Erit ergo Tyrannus , \\\hline
2.2.7 & nin sean tirannos \textbf{ Conuiene les avn de trabaiar entp̃o dela moçedat en sciençias } e ensseñaça de deletris & et Principum cum ponuntur in aliquo dominio tyrannizent , \textbf{ decet ipsos etiam ab ipsa infantia insudare literis , } ut vigere possint prudentia et intellectu . \\\hline
2.2.7 & e ensseñaça de deletris \textbf{ por que se pueda enoblesçer } por sabiduria e por entendimiento . ¶ & decet ipsos etiam ab ipsa infantia insudare literis , \textbf{ ut vigere possint prudentia et intellectu . } Septem scientias esse famosas \\\hline
2.2.8 & e de los francos \textbf{ e delos nobles se ponian a aprender las . } Ca primeramente los fijos de los nobłs aprindian la guamatica & eo quod filii liberorum , \textbf{ et nobilium ponebantur ad illas . } Addiscebant enim primo filii nobilium grammaticam . Nam grammatica secundum Alpharabium inuenta est , \\\hline
2.2.8 & por la qual cosa \textbf{ si non podemos abastar por nos mismos } para fallar e auer todas las sçiençias . & sub tali enim sermone Philosophi suam scientiam tradiderunt . Quare si per nosipsos non sufficimus ad inuenienda omnia scientifica , \textbf{ sed indigemus } ad hoc auxilio Philosophorum et Doctorum , \\\hline
2.2.8 & si non podemos abastar por nos mismos \textbf{ para fallar e auer todas las sçiençias . } Mas para esto auemos men ester ayuda de los philosofos & sub tali enim sermone Philosophi suam scientiam tradiderunt . Quare si per nosipsos non sufficimus ad inuenienda omnia scientifica , \textbf{ sed indigemus } ad hoc auxilio Philosophorum et Doctorum , \\\hline
2.2.8 & e de los doctores . \textbf{ Conuiene nos de saber } e de aprender aquel lenguage & ad hoc auxilio Philosophorum et Doctorum , \textbf{ expedit nos scire idioma illud , } in quo doctores \\\hline
2.2.8 & Conuiene nos de saber \textbf{ e de aprender aquel lenguage } en que fablaron los doctors e los philosofos . & ad hoc auxilio Philosophorum et Doctorum , \textbf{ expedit nos scire idioma illud , } in quo doctores \\\hline
2.2.8 & La segunda sciencia liberal es dicha logica . \textbf{ la qual sçiençia enssenna manera de argumentar e de contradezir . } Ca lanr̃a manera de saber es & Secunda liberalis scientia dicitur esse dialectica , \textbf{ quae docet modum arguendi et opponendi . Nam modus sciendi noster , est } ut per debita argumenta , \\\hline
2.2.8 & e por razones derechas i anifestamos nr̃a uoluntad e nr̃a entençion . \textbf{ Et por ende conuiene de fallar algua sçiençia } que nos mostrasse & et per debitas rationes manifestemus propositum . \textbf{ Oportuit ergo inuenire aliquam scientiam docentem modum , } quo formanda sunt argumenta , \\\hline
2.2.8 & que nos mostrasse \textbf{ en qual manera son de enformar los argumentos e las razones . } Ca si nos non sopiessemos la manera de argumentar & Oportuit ergo inuenire aliquam scientiam docentem modum , \textbf{ quo formanda sunt argumenta , | et rationes . } Nam nisi modum arguendi sciremus , \\\hline
2.2.8 & en qual manera son de enformar los argumentos e las razones . \textbf{ Ca si nos non sopiessemos la manera de argumentar } e de razonar & et rationes . \textbf{ Nam nisi modum arguendi sciremus , } possemus in arguendo peccare , \\\hline
2.2.8 & Ca si nos non sopiessemos la manera de argumentar \textbf{ e de razonar } podriemos pecar en argumentando e en razonando . & Nam nisi modum arguendi sciremus , \textbf{ possemus in arguendo peccare , } et per consequens decipi . Crederemus aliquando bene concludere , \\\hline
2.2.8 & e de razonar \textbf{ podriemos pecar en argumentando e en razonando . } Et por ende podriamos ser engannados & Nam nisi modum arguendi sciremus , \textbf{ possemus in arguendo peccare , } et per consequens decipi . Crederemus aliquando bene concludere , \\\hline
2.2.8 & Et por ende podriamos ser engannados \textbf{ ca cuydariemos alguas vezes bien ençerrar razones e ençerrar las yamos falsamente } por que non sabriamos la manera de argumentar . & et per consequens decipi . Crederemus aliquando bene concludere , \textbf{ et concluderemus falsum : } eo quod ignoraremus arguendi modum . \\\hline
2.2.8 & ca cuydariemos alguas vezes bien ençerrar razones e ençerrar las yamos falsamente \textbf{ por que non sabriamos la manera de argumentar . } Et por ende & et concluderemus falsum : \textbf{ eo quod ignoraremus arguendi modum . } Sicut igitur \\\hline
2.2.8 & que endereça la lengua \textbf{ que non erremos en fablar } assi es neçessaria la logica & quae est directio linguae , \textbf{ ne erremus in loquendo : | sic } secundum Alpharabium necessaria est Dialectica , \\\hline
2.2.8 & que guie e enderesçe el entendimiento \textbf{ que non erremos en argumentar } e en razonar ¶ & quae est directio intellectus , \textbf{ ne erretur in arguendo . } Tertia scientia liberalis dicitur esse Rhetorica . Est autem Rhetorica , \\\hline
2.2.8 & que non erremos en argumentar \textbf{ e en razonar ¶ } La tercera sçiençia liberales dicha rectorica . & quae est directio intellectus , \textbf{ ne erretur in arguendo . } Tertia scientia liberalis dicitur esse Rhetorica . Est autem Rhetorica , \\\hline
2.2.8 & assi commo vna gruessa logica . \textbf{ Ca assi commo son de fazer razones sotiles en las sçiençias especulatinas } assi son de fazer razones gruessas en las sciençias morales & quasi quaedam grossa dialectica . \textbf{ Nam sicut fiendae sunt rationes subtiles in scientiis naturalibus } et in aliis scientiis speculabilibus , sic fiendae sunt rationes grossae in scientiis moralibus , \\\hline
2.2.8 & Ca assi commo son de fazer razones sotiles en las sçiençias especulatinas \textbf{ assi son de fazer razones gruessas en las sciençias morales } que tractan delas obras & Nam sicut fiendae sunt rationes subtiles in scientiis naturalibus \textbf{ et in aliis scientiis speculabilibus , sic fiendae sunt rationes grossae in scientiis moralibus , } quae tractant de agibilibus . \\\hline
2.2.8 & que tractan delas obras \textbf{ que auemos de fazer . } por la qual cosa & et in aliis scientiis speculabilibus , sic fiendae sunt rationes grossae in scientiis moralibus , \textbf{ quae tractant de agibilibus . } Quare sicut necessaria fuit dialectica , \\\hline
2.2.8 & que nos muestra manera sotil \textbf{ e muy afincada de argumentar } assi fue neçessaria la rectorica & Quare sicut necessaria fuit dialectica , \textbf{ quae docet modum arguendi subtilem et violentiorem : } sic necessaria fuit rhetorica , \\\hline
2.2.8 & que nos mostrasse manera gruessa e figural \textbf{ para argumentar gruessa mente . } Et esta es neçessaria alos fijos de los nobles e de los libres & ø \\\hline
2.2.8 & Ca a estos buenos parte nesçe de beuir entre las gentes \textbf{ e de enssennorear al pueblo el qual pueblo non puede entender } si non gruesas razones e exenplarias ¶ & quia horum est conuersari \textbf{ inter gentes et dominari populo , | qui non potest percipere } nisi rationes grossas et figurales . \\\hline
2.2.8 & por muchos razones delas quales vna es \textbf{ que los moços non pueden sofrir ninguna cosa de tristeza . } Por la qual cosa & Quarum una est , \textbf{ quia pueri nihil tristabile sustinere possunt : } quare si debent eis aliqua delectabilia concedi , \\\hline
2.2.8 & La segunda razon \textbf{ para prouar esto mismo puede ser esta } que pone el philosofo en el vii i̊ libro delas politicas & quia habent innocuas delectationes . \textbf{ Secunda ratio ad hoc idem esse potest , } quia ( ut Philosophus innuit in eodem 8 Poli’ ) \\\hline
2.2.8 & conuiene le algunas vezes \textbf{ de en reponer algunas delectaciones musicales e de cantares } que son delectaconnes conuenibles e sin danno . & mens humana nescit ociosa esse . Ideo \textbf{ ( ut videtur ) ocium bonum est aliquando interponere inter delectationes musicales , quae sunt licitae } et innocuae . Maxime autem hoc decens est filiis liberorum et nobilium , \\\hline
2.2.8 & Mas el philosofo tanne muchͣs razones en las politicas \textbf{ por las quales se podrie mostrar } que conuienea los fijos de los nobles & Tangit enim Philosophus multas rationes in Politicis , \textbf{ per quas ostendi posset , } quod filios nobilium decet addiscere musicam . \\\hline
2.2.8 & que conuienea los fijos de los nobles \textbf{ de aprender la musica } mas destas razones por auentura fablaremos adelante ¶ & per quas ostendi posset , \textbf{ quod filios nobilium decet addiscere musicam . } Sed de his forte infra tangetur . Quinta scientia liberalis dicitur esse arithmetica , \\\hline
2.2.8 & por auentura los fijos de los nobles eran puestos \textbf{ por quela musica non se podia saber acabadamente sin ella ¶ } La sexta sçiençia çia libales geometera & ad quam forte filii liberorum ideo tradebantur , \textbf{ quia sine ea musica sciri non potest . Sexta scientia liberalis est geometria , } quae docet \\\hline
2.2.8 & La sexta sçiençia çia libales geometera \textbf{ que muestra conosçer las mesuras e las quantidades delas cosas . } Et aesta eran puestos por auentura los fijos delos nobles & quae docet \textbf{ cognoscere mensuras | et quantitates rerum . } Ad hanc autem filii nobilium , \\\hline
2.2.8 & por que los gentiles e tan muy acuçiosos çerca los iuyzios delas estrellas . \textbf{ Ca nunca quarien comneçar batallas } nin quarien comneçar otras obras ningunas & circa iudicia astrorum nimis erant curiosi . \textbf{ Nunquam enim volebant bella inchoare , } nec aliqua opera incipere , \\\hline
2.2.8 & Ca nunca quarien comneçar batallas \textbf{ nin quarien comneçar otras obras ningunas } si non catassen primeramente el grado del ascondente & Nunquam enim volebant bella inchoare , \textbf{ nec aliqua opera incipere , } nisi considerato gradu ascendente , \\\hline
2.2.8 & Ca la natural ph̃ia \textbf{ que muestra conosçer las naturas delas cosas muy meior es } que ninguna de las dichos artes . & Nam Naturalis Philosophia docens cognoscere naturas rerum , \textbf{ longe melior est , } quam aliqua praedictarum . \\\hline
2.2.8 & que ninguna de las dichos artes . \textbf{ Mas declarar esto non parte nesçe a esta sçiençia } que deue ser gruessa e fig᷑al . & quam aliqua praedictarum . \textbf{ Sed hoc declarare non patitur perscrutatio praesens , } quae debet esse grossa et figuralis . \\\hline
2.2.8 & de la qual sçiençia el philosofo en el primero libro dela methaphisica dize \textbf{ que ninguna sçiençia non es mas digna que ella la qual cosa se deue entender destas sçiençias } que son falladas por los omes & de qua Philosophus ait primo Meta’ \textbf{ quia nulla est dignior ipsa } quod intelligendum est de scientiis humanitus inuentis , \\\hline
2.2.8 & e de los prinçipes \textbf{ li quisieren beuir uida politica e çiuil } e quasieren gouernar los otro & et maxime filii Regum et Principum , \textbf{ si velint politice viuere , } et velint alios regere et gubernare , \\\hline
2.2.8 & li quisieren beuir uida politica e çiuil \textbf{ e quasieren gouernar los otro } o mayormente se deuen trabaiar çerca destas sçiençias morales . & si velint politice viuere , \textbf{ et velint alios regere et gubernare , } maxime circa has debent insistere . Sunt autem et aliae scientiae subalternatae et suppositae istis : \\\hline
2.2.8 & e quasieren gouernar los otro \textbf{ o mayormente se deuen trabaiar çerca destas sçiençias morales . } Et avn son o trissçiençias subal ternadas e subiectas destas & et velint alios regere et gubernare , \textbf{ maxime circa has debent insistere . Sunt autem et aliae scientiae subalternatae et suppositae istis : } ut perspectiua , \\\hline
2.2.8 & por que argumentan e forman sus razones rudamente e sin arte . \textbf{ La qual manera de argumentar muestra la logica . } Por ende aquellos que non quieren argumentar artifiçialmente por sçiençia de logica estos son llamados del philosofo logicos & et formant rationes suas , \textbf{ quem modum arguendi docet dialectica , } ideo ipsi quia non arguunt artificialiter et dialectice , \\\hline
2.2.8 & La qual manera de argumentar muestra la logica . \textbf{ Por ende aquellos que non quieren argumentar artifiçialmente por sçiençia de logica estos son llamados del philosofo logicos } nesçios en essa misma manera los legistas & quem modum arguendi docet dialectica , \textbf{ ideo ipsi quia non arguunt artificialiter et dialectice , | appellantur a Philosopho idiotae dialectici : } sic Legistae , \\\hline
2.2.8 & por ende pueden ser llamados nesçios politicos . \textbf{ Et desto puede parescer } en qual manera & appellari possunt idiotae politici . \textbf{ Ex hoc autem patere potest } quod magis honorandi sunt scientes politicam \\\hline
2.2.8 & en qual manera \textbf{ mas son de honrrar } los que saben la politica e las sçiençias morales & Ex hoc autem patere potest \textbf{ quod magis honorandi sunt scientes politicam } et morales scientias , \\\hline
2.2.8 & non dando razon de sus dichos tanto estos tales son mas honrrados que los otros . \textbf{ Et avn desto puede paresçer } que los sabios en la theologia son mas de honrrar . & et non reddentibus causam dicti : \textbf{ tanto tales honorobiliores sunt illis . Ex hoc autem apparere potest , qui scientes magis sint honorandi . } Nam primo honorandi sunt diuini \\\hline
2.2.8 & Et avn desto puede paresçer \textbf{ que los sabios en la theologia son mas de honrrar . } Ca primero son de honrrar los omes diuinales & et non reddentibus causam dicti : \textbf{ tanto tales honorobiliores sunt illis . Ex hoc autem apparere potest , qui scientes magis sint honorandi . } Nam primo honorandi sunt diuini \\\hline
2.2.8 & que los sabios en la theologia son mas de honrrar . \textbf{ Ca primero son de honrrar los omes diuinales } e los que saben la theologia . & tanto tales honorobiliores sunt illis . Ex hoc autem apparere potest , qui scientes magis sint honorandi . \textbf{ Nam primo honorandi sunt diuini } et scientes theologiam ; \\\hline
2.2.8 & que es prinçipalmente de dios es sennora de todas las otras sçiençias humanales . \textbf{ Et despues de los theologos son mas de honrrar los methaphisicos } los quales & quae principaliter est de Deo , \textbf{ est dea omnium , et domina humanarum scientiarum . Post theologos vero magis honorandi Metaphysici : } quia ( ut superius dicebatur ) \\\hline
2.2.8 & Et pues que assi es \textbf{ assi los omes sabios son de honrrar cada vno en su grado . } Et pues que assi es estas sçiençias & tamen principatum \textbf{ inter omnes alias scientias inuentas ab homine . Sic ergo scientes gradatim sunt honorandi . } His ergo scientiis sic diuisis , \\\hline
2.2.8 & Et pues que assi es estas sçiençias \textbf{ assi departidas de ligero puede paresçer } cerça quales sçiençias deuen trabaiar los fijos de los nobles & His ergo scientiis sic diuisis , \textbf{ de leui patere potest , } circa quas scientias filii nobilium , \\\hline
2.2.8 & assi departidas de ligero puede paresçer \textbf{ cerça quales sçiençias deuen trabaiar los fijos de los nobles } e mayormente delos Reyes e de los prinçipes . & de leui patere potest , \textbf{ circa quas scientias filii nobilium , } et maxime Regum , \\\hline
2.2.8 & Ca commo les conuenga a ellos de ser \textbf{ assi commo medios dioses e de entender conueinblemente } e sin ninguna negligençia en los negoçios del regno . & Nam cum oporteat eos esse quasi semideos , \textbf{ et debite et absque negligentia negotium regni intendere , } non vacat eis subtiliter perscrutari scientias : \\\hline
2.2.8 & e sin ninguna negligençia en los negoçios del regno . \textbf{ Et por que non le suaga a ellos de escodrinnar sotilmente las sçiençias } mucho les conuiene aellos & et debite et absque negligentia negotium regni intendere , \textbf{ non vacat eis subtiliter perscrutari scientias : } maxime igitur decet ipsos bene se habere circa diuina , \\\hline
2.2.8 & mucho les conuiene aellos \textbf{ de se auer bien cerca las cosas diuinales } e ser enssennados e firmes en la fe & non vacat eis subtiliter perscrutari scientias : \textbf{ maxime igitur decet ipsos bene se habere circa diuina , } et esse instructos et firmos in fide , \\\hline
2.2.8 & e ser enssennados e firmes en la fe \textbf{ e conuiene les de saber aquellas sçiençias } por las quales cada vno puede gouernar assi e alos otros & et esse instructos et firmos in fide , \textbf{ et illas scientias scire , } per quas quis se et alios nouit regere et gubernare . Huiusmodi autem sunt scientiae morales . Alias ergo scientias in tantum decet eos scire , inquantum deseruiunt morali negocio . \\\hline
2.2.8 & e conuiene les de saber aquellas sçiençias \textbf{ por las quales cada vno puede gouernar assi e alos otros } Et estas sçiençias son las sçiençias i trorales . & et illas scientias scire , \textbf{ per quas quis se et alios nouit regere et gubernare . Huiusmodi autem sunt scientiae morales . Alias ergo scientias in tantum decet eos scire , inquantum deseruiunt morali negocio . } Decet igitur eos scire grammaticam , \\\hline
2.2.8 & en quanto siruen ala ph̃ia moral . \textbf{ Et pues que assi es conuiene les a ellos de saber la guamatica } por que entiendan el lenguage delas letras & per quas quis se et alios nouit regere et gubernare . Huiusmodi autem sunt scientiae morales . Alias ergo scientias in tantum decet eos scire , inquantum deseruiunt morali negocio . \textbf{ Decet igitur eos scire grammaticam , } ut intelligant idioma literale : \\\hline
2.2.8 & Et por ende conuiene alos Reyes \textbf{ e a los prinçipes de saber el lenguage delas letris } por que puedan alos otros escuirles sus poridades & et Principes \textbf{ scire idioma literale , } ut possint secreta sua alii scribere \\\hline
2.2.8 & por que puedan alos otros escuirles sus poridades \textbf{ e leerlas sin sabiduria de los otros . } Et pues que assi es los fijos de los nobłs & ut possint secreta sua alii scribere \textbf{ et legere absque aliorum scitu . } Filii ergo nobilium \\\hline
2.2.8 & maguera que entiendan ser caualleros \textbf{ e entender en el fecho politico dela çibdat deuen trabaiar } por que sepan el lenguage delas letras . & quantumcunque intendant esse milites , \textbf{ et vacare negocio politico , debent insudare , } ut sciant idioma literale . Debent etiam aliquid addiscere de dialectica et rhetorica , \\\hline
2.2.8 & por que sepan el lenguage delas letras . \textbf{ Otrossi deuen aprender algunan cosa dela logica e dela rectorica } por que por esto sean mas sotiles & et vacare negocio politico , debent insudare , \textbf{ ut sciant idioma literale . Debent etiam aliquid addiscere de dialectica et rhetorica , } ut ex hoc subtiliores fiant ad intelligendum quaecunque proposita : quo facto totum suum ingenium debent exponere , ut bene intelligant moralia \\\hline
2.2.8 & por que por esto sean mas sotiles \textbf{ para entender } quales si quier cosas que les sean propuestas & ø \\\hline
2.2.8 & quales si quier cosas que les sean propuestas \textbf{ la qual casa fechͣ deuen poner todo su en gennio } porque puedan bien entender las sçiençias m orales & ut sciant idioma literale . Debent etiam aliquid addiscere de dialectica et rhetorica , \textbf{ ut ex hoc subtiliores fiant ad intelligendum quaecunque proposita : quo facto totum suum ingenium debent exponere , ut bene intelligant moralia } et ut sciant se et alios regere . \\\hline
2.2.8 & la qual casa fechͣ deuen poner todo su en gennio \textbf{ porque puedan bien entender las sçiençias m orales } por que sepan gouernar & ut sciant idioma literale . Debent etiam aliquid addiscere de dialectica et rhetorica , \textbf{ ut ex hoc subtiliores fiant ad intelligendum quaecunque proposita : quo facto totum suum ingenium debent exponere , ut bene intelligant moralia } et ut sciant se et alios regere . \\\hline
2.2.8 & porque puedan bien entender las sçiençias m orales \textbf{ por que sepan gouernar } assi e alo . s otros . & ut ex hoc subtiliores fiant ad intelligendum quaecunque proposita : quo facto totum suum ingenium debent exponere , ut bene intelligant moralia \textbf{ et ut sciant se et alios regere . } Nam et de musica \\\hline
2.2.8 & Otrossi segunt el philosofo en las politicas \textbf{ conuiene les alguacosado saber dela musica } en quanto ella sirue alas bueans costunbres . & Nam et de musica \textbf{ secundum Philosophum in politicis eos scire decet , } inquantum deseruit ad bonos mores . \\\hline
2.2.8 & Et pues que \textbf{ assi es si conuiene de saber la sçiençia moral a aquellos que dessean enssennorear } assi commo si non sopiessen todas las otras sçiençias deuen estudiar en esto & inquantum deseruit ad bonos mores . \textbf{ Sic ergo morale negocium scire expedit ab iis | qui cupiunt principari : } ut si omnes alias scientias ignorarent , \\\hline
2.2.8 & assi es si conuiene de saber la sçiençia moral a aquellos que dessean enssennorear \textbf{ assi commo si non sopiessen todas las otras sçiençias deuen estudiar en esto } por que manifiestamente e gruessamente sea mostrada a ellos la sciençia moral & qui cupiunt principari : \textbf{ ut si omnes alias scientias ignorarent , | adhuc studere debent , } ut eis moralia vulgariter et grosse proportionentur : \\\hline
2.2.8 & por la qual los prinçipes son ensseñados conplidamente \textbf{ en qual manera de una enssennorear } et en qual manera de una endozir assia alos sy çibdadanos a uirtudes & quia per ea princeps sufficienter instruitur , \textbf{ qualiter debeat principari , } et quo se et ciues inducere debeat ad virtutes . \\\hline
2.2.8 & en qual manera de una enssennorear \textbf{ et en qual manera de una endozir assia alos sy çibdadanos a uirtudes } l philosofo çerca la fin del terçero libro delas ethicas & qualiter debeat principari , \textbf{ et quo se et ciues inducere debeat ad virtutes . } Philosophus circa finem 3 Ethicorum , \\\hline
2.2.9 & assi commo es dicho en esse mismo terçero libro delas ethicas . \textbf{ Assi avn el maestro de los moços deue sienpre enduzir e acuçiar los mocos a muy buenas costunbres . } Et pues que assi es & ut in eodem tertio Ethicorum dicitur : \textbf{ sic et magister puerorum semper debet eos | ad optima instigare . } Si igitur in doctrina puerorum finis intentus est , \\\hline
2.2.9 & si la fin \textbf{ que entiende el maestro en enssennar los moços } es enduzir los a muy buenas costunbrs e asçiençia & ad optima instigare . \textbf{ Si igitur in doctrina puerorum finis intentus est , } instigatio ad optima ; \\\hline
2.2.9 & que entiende el maestro en enssennar los moços \textbf{ es enduzir los a muy buenas costunbrs e asçiençia } por que dela fin se deue tomar razon de todas las otras cosas & Si igitur in doctrina puerorum finis intentus est , \textbf{ instigatio ad optima ; } quia ex fine accipienda est ratio omnium aliorum , \\\hline
2.2.9 & es enduzir los a muy buenas costunbrs e asçiençia \textbf{ por que dela fin se deue tomar razon de todas las otras cosas } que son ordenadas ala fin . & instigatio ad optima ; \textbf{ quia ex fine accipienda est ratio omnium aliorum , } talis quaerendus est magister , \\\hline
2.2.9 & Por ende tal deue ser tomado el maestro \textbf{ que pueda endozir los moços } quel son en comnedados a muy grandes bienes . & talis quaerendus est magister , \textbf{ qui possit iuuenes sibi commissos ad optima inducere . Optimum autem } et bonum , ad quod inducendi sunt iuuenes , \\\hline
2.2.9 & a que deuen ser endozidos los moços . \textbf{ Conuiene a saber . Sçiençia . Et bueans costun bres . } Mas a buenas costunbres es adozido cada vno en dos ianeras . & est duplex : \textbf{ scientia scilicet , et mores . } Ad bonos autem mores inducitur dupliciter quis , \\\hline
2.2.9 & Mas a buenas costunbres es adozido cada vno en dos ianeras . \textbf{ Couiene a saber . } por exienplo e por bondat de uida . & scientia scilicet , et mores . \textbf{ Ad bonos autem mores inducitur dupliciter quis , } exemplo per bonitatem vitae , \\\hline
2.2.9 & e de los prinçipes \textbf{ tres cosas deue auer enssi . } Ca deue ser sabio en las sçiençias especulatiuas & et maxime Regum et Principum , \textbf{ tria in se habere debet . } Debet esse sciens in speculabilibus . Prudens in agibilibus . \\\hline
2.2.9 & e deue ser entendudo en las obras \textbf{ que ha de fazer } e deue ser bueno en uida & Debet esse sciens in speculabilibus . Prudens in agibilibus . \textbf{ Et bonus in vita . } Ad hoc autem quod sit sciens in speculabilibus , requiruntur tria . \\\hline
2.2.9 & Lo terçero que se \textbf{ para iudgar tan bien delas cosas falladas } commo de aquellas que entendio de los otros ¶ & Quod intelligat aliorum dicta . \textbf{ Et tertio quod sit iudicatiuus tam de inuentis , } quam de iis quae ab aliis intellexit . Inuentiuus enim esse debet : \\\hline
2.2.9 & Lo primero que deue ser fallador \textbf{ por que el que non sabe en ninguna manera fallar alguna cosa . } Mas solamente sabe contar los dichos de los otros & quam de iis quae ab aliis intellexit . Inuentiuus enim esse debet : \textbf{ quia qui nullo modo scit aliqua inuenire , } sed solum nouit aliorum dicta referre , \\\hline
2.2.9 & por que el que non sabe en ninguna manera fallar alguna cosa . \textbf{ Mas solamente sabe contar los dichos de los otros } este tal mas es rezador que doctor ¶ & quia qui nullo modo scit aliqua inuenire , \textbf{ sed solum nouit aliorum dicta referre , } magis est recitator , \\\hline
2.2.9 & que non tan solamente sea fallador delas cosas \textbf{ mas que sea entendido e sotil . Ca assi commo ninguon non puede abastar } asi en la uida bien e conplidamente & et perspicacem . \textbf{ Nam sicut nullus bene et perfecte sibi sufficit in vita sed ad hoc quod habeamus sufficientiam in vita , } oportet nos in societate viuere , \\\hline
2.2.9 & que sea iudgador \textbf{ e que aya razon para iudgar . } Ca la perfection dela sçiençia prinçipalmen te esta en el iuyzio . & Tertio oportet ipsum esse iudicatiuum : \textbf{ nam perfectio scientiae potissime in iudicio consistit . } Non enim satis est aliorum dicta intelligere , \\\hline
2.2.9 & Ca la perfection dela sçiençia prinçipalmen te esta en el iuyzio . \textbf{ Ca non abondade entender los dichos de los otros } e fallar de si mismo muchͣs & nam perfectio scientiae potissime in iudicio consistit . \textbf{ Non enim satis est aliorum dicta intelligere , } et de se multa inuenire , \\\hline
2.2.9 & Ca non abondade entender los dichos de los otros \textbf{ e fallar de si mismo muchͣs } cosas si non sopiereiudgar tan bien delas cosas entendidas & Non enim satis est aliorum dicta intelligere , \textbf{ et de se multa inuenire , } nisi tam de intellectis quam de inuentis nouerit iudicare quae sunt tenenda , et quae respuenda . \\\hline
2.2.9 & e fallar de si mismo muchͣs \textbf{ cosas si non sopiereiudgar tan bien delas cosas entendidas } commo delas falladas & Non enim satis est aliorum dicta intelligere , \textbf{ et de se multa inuenire , } nisi tam de intellectis quam de inuentis nouerit iudicare quae sunt tenenda , et quae respuenda . \\\hline
2.2.9 & commo delas falladas \textbf{ quales cosas son de retener } e quales de refusar . & et de se multa inuenire , \textbf{ nisi tam de intellectis quam de inuentis nouerit iudicare quae sunt tenenda , et quae respuenda . } Quantum ergo ad scientiam talis debet quaeri doctor , \\\hline
2.2.9 & quales cosas son de retener \textbf{ e quales de refusar . } Et pueᷤ & et de se multa inuenire , \textbf{ nisi tam de intellectis quam de inuentis nouerit iudicare quae sunt tenenda , et quae respuenda . } Quantum ergo ad scientiam talis debet quaeri doctor , \\\hline
2.2.9 & e de los prinçipes \textbf{ non quieran en ninguna manera entrar ala profundidat o ala alteza delas sçiençias } mas tienen que les cunple aellos de saber alguna cosa delas sçiençias . & et maxime Regum , \textbf{ et Principum nolint omnino ad profunditatem scientiae ingredi , } sed sufficiat eis aliqua de his cognoscere : \\\hline
2.2.9 & non quieran en ninguna manera entrar ala profundidat o ala alteza delas sçiençias \textbf{ mas tienen que les cunple aellos de saber alguna cosa delas sçiençias . } Enpero deuen demandar & et Principum nolint omnino ad profunditatem scientiae ingredi , \textbf{ sed sufficiat eis aliqua de his cognoscere : } nihilominus tamen doctorem bene scientem debent inquirere , \\\hline
2.2.9 & mas tienen que les cunple aellos de saber alguna cosa delas sçiençias . \textbf{ Enpero deuen demandar } e querer maestro e doctor muy sabio & sed sufficiat eis aliqua de his cognoscere : \textbf{ nihilominus tamen doctorem bene scientem debent inquirere , } eo quod doctrina prudentium facilis , \\\hline
2.2.9 & Enpero deuen demandar \textbf{ e querer maestro e doctor muy sabio } porque la sçiençia de los bueons maestros et sabios es ligera & sed sufficiat eis aliqua de his cognoscere : \textbf{ nihilominus tamen doctorem bene scientem debent inquirere , } eo quod doctrina prudentium facilis , \\\hline
2.2.9 & Por ende aquellas cosas pocas \textbf{ que cobdiçian saber los fijos de los nobles mas ligeramente e mas claramente } e mas derechamente las pueden entender del sabio & Illa ergo modica \textbf{ quae scire cupiunt , facilius , clarius , et rectius intelligent a sciente , } quam ab inscio . Viso qualis debet esse doctor , \\\hline
2.2.9 & que cobdiçian saber los fijos de los nobles mas ligeramente e mas claramente \textbf{ e mas derechamente las pueden entender del sabio } que del nesçio & Illa ergo modica \textbf{ quae scire cupiunt , facilius , clarius , et rectius intelligent a sciente , } quam ab inscio . Viso qualis debet esse doctor , \\\hline
2.2.9 & e en qual manera deue ser sabio \textbf{ para enssennar los maços en la sçiençia finca de uer } en qual manera deue ser prudente e sabio en las obras & et quomodo debet esse sciens \textbf{ ut doceat in scientia : restat videre , } qualiter debet esse prudens in agibilibus ut instruat in bonis moribus . \\\hline
2.2.9 & en qual manera deue ser prudente e sabio en las obras \textbf{ que ha de fazer } por que enssenne los moços en bueans costunbres . & ut doceat in scientia : restat videre , \textbf{ qualiter debet esse prudens in agibilibus ut instruat in bonis moribus . } Ad huiusmodi autem prudentiam describendam , \\\hline
2.2.9 & Mas esta sabiduria \textbf{ que es de notar } e de escuir & qualiter debet esse prudens in agibilibus ut instruat in bonis moribus . \textbf{ Ad huiusmodi autem prudentiam describendam , } licet enumerare possemus \\\hline
2.2.9 & que es de notar \textbf{ e de escuir } commo quier que la podamos contar & Ad huiusmodi autem prudentiam describendam , \textbf{ licet enumerare possemus } omnia illa octo \\\hline
2.2.9 & e de escuir \textbf{ commo quier que la podamos contar } entre aquellas ocho cosas & Ad huiusmodi autem prudentiam describendam , \textbf{ licet enumerare possemus } omnia illa octo \\\hline
2.2.9 & que dixiemos enel primero libro dela sabiduria . \textbf{ Enpero cunple nos de contar las quatro cosas de aquellas ocho . } ¶ Et pues que assi es el maestro & quae in primo libro de prudentia tetigimus , \textbf{ sufficiat } tamen ad praesens quatuor \\\hline
2.2.9 & nobles deue ser \textbf{ assi sabio en las cosas que ha de fazer } por que sea menbrado delas cosas & tamen ad praesens quatuor \textbf{ de illis enumerare . Doctor enim puerorum nobilium debet esse sic prudens in agibilibus , } ut sit memor , cautus , prouidus , \\\hline
2.2.9 & e prouiso en las cosas \textbf{ que han de uenir Sabio } en departir el mal & de illis enumerare . Doctor enim puerorum nobilium debet esse sic prudens in agibilibus , \textbf{ ut sit memor , cautus , prouidus , } et circumspectus . \\\hline
2.2.9 & que han de uenir Sabio \textbf{ en departir el mal } del bien & de illis enumerare . Doctor enim puerorum nobilium debet esse sic prudens in agibilibus , \textbf{ ut sit memor , cautus , prouidus , } et circumspectus . \\\hline
2.2.9 & e acordado e prouado \textbf{ enlo que ha de fazer ¶ } Ca primero deue ser menbrado & et circumspectus . \textbf{ Debet enim esse memor , } recolendo praeterita . \\\hline
2.2.9 & e acordado delas colas passadas . \textbf{ Ca assi commo aquel que quiere enderesçar la pierte } ga nunca la puede enderesçar & recolendo praeterita . \textbf{ Nam sicut volens rectificare virgam , } nunquam eam rectificare posset \\\hline
2.2.9 & Ca assi commo aquel que quiere enderesçar la pierte \textbf{ ga nunca la puede enderesçar } si non conosçiere de qual parte esta tuerta . En essa misma manera aquel que quiere enderesçar los otros & Nam sicut volens rectificare virgam , \textbf{ nunquam eam rectificare posset } nisi cognosceret \\\hline
2.2.9 & ga nunca la puede enderesçar \textbf{ si non conosçiere de qual parte esta tuerta . En essa misma manera aquel que quiere enderesçar los otros } nunca los podria enderesçar & nunquam eam rectificare posset \textbf{ nisi cognosceret | ex qua parte esset obliquata : } sic volens alios rectificare nunquam eos congrue rectificare posset \\\hline
2.2.9 & si non conosçiere de qual parte esta tuerta . En essa misma manera aquel que quiere enderesçar los otros \textbf{ nunca los podria enderesçar } si non ouiesse conosçimiento delas cosas passadas & ex qua parte esset obliquata : \textbf{ sic volens alios rectificare nunquam eos congrue rectificare posset } nisi haberet praeteritorum notitiam , \\\hline
2.2.9 & Et pues que assi es conuiene \textbf{ que el que ha degniar los otros } que sea acordado delas cosas passadas ¶ & per quae cognosceret quomodo obliquata essent . \textbf{ Decet igitur aliorum directorem memorem esse praeteritorum . } Secundo decet ipsum esse prouidum futurorum . \\\hline
2.2.9 & que sea prouiso en las cosas \textbf{ que han de uenir . } Ca assi commo el que ha degniar los otros deue penssar & ø \\\hline
2.2.9 & que han de uenir . \textbf{ Ca assi commo el que ha degniar los otros deue penssar } lo que es passado & Secundo decet ipsum esse prouidum futurorum . \textbf{ Nam sicut aliorum director debet cogitare praeterita , } ut sciat quomodo per tempora praeterita obliquati fuerint \\\hline
2.2.9 & e errados los quales deuen ser guardados por el ¶ Bien \textbf{ assi deue proueer las cosas } que han de uenir & qui ab eo sunt dirigendi : \textbf{ sic debet prouidere futura , } ut adhibeat medicamenta \\\hline
2.2.9 & assi deue proueer las cosas \textbf{ que han de uenir } por que ponaga melezmas a aquellas cosas & qui ab eo sunt dirigendi : \textbf{ sic debet prouidere futura , } ut adhibeat medicamenta \\\hline
2.2.9 & por que ponaga melezmas a aquellas cosas \textbf{ por que podrien los omes de ligero errar ¶ } Lo terçero le conuiene al maestro & ad ea , \textbf{ per quae in posterum facilius obliquari possint . } Tertio decet ipsum esse cautum , \\\hline
2.2.9 & Lo terçero le conuiene al maestro \textbf{ que sea sabio en departir el manl del bien . } Ca assi commo dixiemos en el primer libro & per quae in posterum facilius obliquari possint . \textbf{ Tertio decet ipsum esse cautum , } quia ut in primo libro tetigimus , \\\hline
2.2.9 & assi commo en conosçiendo \textbf{ e en contenplando es de tomar cautela } por que las cosas falssas non sean mezcladas alas uerdaderas . & quia ut in primo libro tetigimus , \textbf{ sicut in cognoscendo et speculando est adhibenda cautela , } ne falsa admisceantur veris : \\\hline
2.2.9 & Assi en las obras \textbf{ que son de fazer conuiene al omne de ser sabio } porque los males non sean mesclados alos bienes . & ne falsa admisceantur veris : \textbf{ sic in agendis conuenit hominem esse cautum , } ne mala admisceantur bonis : \\\hline
2.2.9 & sin ningun mezclamiento de cosas falssas . \textbf{ En essa misma manera el que quiere enderesçar } e enformar los moços deue ser sabio & ø \\\hline
2.2.9 & En essa misma manera el que quiere enderesçar \textbf{ e enformar los moços deue ser sabio } aponiendo les los bienes & ø \\\hline
2.2.9 & Lo quarto este doctor atal deue ser acatado e prouado por esperiençia . Ca aquel que prueua las cosas es \textbf{ mas çierto en conosçer las cosas particulares e speçiales . } Et este dector tal deue conosçer las condiçiones speçiales delos moços & vel expertus . Nam experti , \textbf{ est particularia cognoscere : sic et huiusmodi doctor debet } cognoscere particulares conditiones illorum iuuenum , \\\hline
2.2.9 & mas çierto en conosçer las cosas particulares e speçiales . \textbf{ Et este dector tal deue conosçer las condiçiones speçiales delos moços } a & est particularia cognoscere : sic et huiusmodi doctor debet \textbf{ cognoscere particulares conditiones illorum iuuenum , } quos debet dirigere . \\\hline
2.2.9 & a \textbf{ que ha de castigar e de enssennar . } Ca segunt que los moços han departidas condiçiones & cognoscere particulares conditiones illorum iuuenum , \textbf{ quos debet dirigere . } Nam \\\hline
2.2.9 & e el maestro de los mocos \textbf{ que los pueda endozir } por conueinbles castigos abien . & Talis ergo debet esse doctor iuuenum , \textbf{ ut eos per debitos sermones , } et per debitas monitiones inducat ad bonum . \\\hline
2.2.9 & que este doctor e maestro sea enssi bueno e honesto en su uida . \textbf{ Ca por que la he dar de los moços es muy inclinada a destenpramiento e a loçania } Como quier que el doctor de los moços les gponga & et honestus . \textbf{ Nam quia aetas iuuenilis valde est prona ad intemperantiam | et lasciuiam ; } quantumcunque puerorum doctor eis verba bona proponeret , \\\hline
2.2.9 & por buenas palauras \textbf{ lo que han de fazer . } Enpero si fiziere por obra el contrario delo & quantumcunque puerorum doctor eis verba bona proponeret , \textbf{ si tamen opere contraria faceret , iuuenes illi exemplo inducti de facili ad illicita declinarent . } Patet igitur talem quaerendum esse doctorem , \\\hline
2.2.9 & Et pues que assi es paresçe \textbf{ que los moços deuen tomar e buscar tal doctor e tal maestro } quanto alas sçiençias speculatiuas & intellectiuus aliorum , \textbf{ et iudicatiuus tam inuentorum quam intellectorum . } Quantum vero ad prudentiam agibilium , \\\hline
2.2.9 & e sea entendido \textbf{ para entender los dichs de los otros } e sea iudgadoͬ tan bien delas cosas falladas & decet ipsum esse memorem , prouidum , \textbf{ cautum , } et circumspectum . \\\hline
2.2.9 & Enpero quanto ala sabiduria moral delas obras \textbf{ que son de fazer conuiene le que el doctor sea menbrado e prouado e sabio e acatado . } Mas quanto ala uida deue ser honesto e bueno . & et circumspectum . \textbf{ Quantum autem ad vitam , } decet ipsum esse honestum , \\\hline
2.2.9 & si los Reyes e los prinçipes \textbf{ e generalmente rodos los çibdadanos deuen ser muy acuçiosos en catar qual mayordomo deuen poner en sus riquezas e en sus posessiones } e enlas o triscosas & Si ergo Reges et Principes \textbf{ et uniuersaliter omnes ciues valde solicitantur , } qualem proponant suis numismatibus , possessionibus , et rebus inanimatis : \\\hline
2.2.9 & que delas otras cosas . \textbf{ por cuydadolos e tener mientes con grand acuçia } qual maestro deuen poner en gouernamiento de sus fijos & et de filiis quam de aliis ; \textbf{ valde deberent esse soliciti , | et cum magna diligentia attendere , } qualem magistrum proponerent in regimine filiorum . \\\hline
2.2.9 & por cuydadolos e tener mientes con grand acuçia \textbf{ qual maestro deuen poner en gouernamiento de sus fijos } erca la fin del septimo delas politicas muestra el philosofo & et cum magna diligentia attendere , \textbf{ qualem magistrum proponerent in regimine filiorum . } Circa finem 7 Politicor’ \\\hline
2.2.10 & que alos moços deuen ser defendidas alzs cosas çerca las fablas \textbf{ e cerca la vista e cerca el oyr . } Ca non conuiene alos moços de fablar & docet Philosophus iuuenes cohibendos esse circa locutionem , \textbf{ visionem , | et auditum : } non enim decet pueros qualitercunque loqui , \\\hline
2.2.10 & e cerca la vista e cerca el oyr . \textbf{ Ca non conuiene alos moços de fablar } qual quier manera nin de oyr & et auditum : \textbf{ non enim decet pueros qualitercunque loqui , | nec decet eos qualiacunque videre , } vel qualiacunque audire ; \\\hline
2.2.10 & Ca non conuiene alos moços de fablar \textbf{ qual quier manera nin de oyr } a quales si quier cosas . & nec decet eos qualiacunque videre , \textbf{ vel qualiacunque audire ; } sed est ibi modus aliquis adhibendus . Circa locutionem quidem iuuenes tripliciter peccare videntur . Primo , \\\hline
2.2.10 & a quales si quier cosas . \textbf{ Mas deuen y tomar alguna manera çerca la fabla . } Ca çierto es & vel qualiacunque audire ; \textbf{ sed est ibi modus aliquis adhibendus . Circa locutionem quidem iuuenes tripliciter peccare videntur . Primo , } quia de facili loquuntur lasciua . Secundo , \\\hline
2.2.10 & Ca çierto es \textbf{ que los moços pueden pecar en tres maneras en fablar } ¶ & quia de facili loquuntur lasciua . Secundo , \textbf{ quia de leui loquuntur falsa . Tertio , } quia ut plurimum loquuntur fatua et imprae meditata . Loquuntur enim de leui lasciua , \\\hline
2.2.10 & e son inclinados a orgullos e aloçania . \textbf{ Por la qual cosa commo sienpre deuemos dar algunan cautellado paresçe el peligro . } por ende deuemos defender alos moços & ut superius in primo libro dicebatur ) iuuenes sunt insecutores passionum , \textbf{ et ad lasciuiam proni . Quare cum semper sit adhibenda cautela ubi periculum imminet , } cohibendi sunt iuuenes a locutione lasciua , \\\hline
2.2.10 & Por la qual cosa commo sienpre deuemos dar algunan cautellado paresçe el peligro . \textbf{ por ende deuemos defender alos moços } e alos mançebos la fabla orgullosa & et ad lasciuiam proni . Quare cum semper sit adhibenda cautela ubi periculum imminet , \textbf{ cohibendi sunt iuuenes a locutione lasciua , } et a sermonibus turpibus : \\\hline
2.2.10 & e las palauras torpes . \textbf{ Et son mucho de denostar } e avn de castigar & et a sermonibus turpibus : \textbf{ et sunt increpandi } et etiam corrigendi , \\\hline
2.2.10 & Et son mucho de denostar \textbf{ e avn de castigar } por ello & et sunt increpandi \textbf{ et etiam corrigendi , } si eos talia loqui contingat . \\\hline
2.2.10 & assi commo otra nafa \textbf{ pues que la ho dat iuu enillos enduze a dezir falsedat e mentira } la qual cosa segunt el . & quasi altera natura , \textbf{ ex quo iuuenilis aetas inclinat eos | ad dicendum falsum et mendacium , } quod secundum Philosophum 4 Ethicor’ est per se prauum et fugiendum , \\\hline
2.2.10 & es muy mala dessi \textbf{ e muy de esquiuar . } Por ende son de endozir & ø \\\hline
2.2.10 & e muy de esquiuar . \textbf{ Por ende son de endozir } por castigos e por conseios conuenibles & quod secundum Philosophum 4 Ethicor’ est per se prauum et fugiendum , \textbf{ per debitas monitiones et correptiones inducendi sunt } ut relinquentes mendacium adhaereant veritati , \\\hline
2.2.10 & es dessi buean \textbf{ e mucho de loar . ¶ Lo . iij . son de castigar los moços } que non fablen & secundum Philosophum in eodem 4 Ethic’ est per se bona \textbf{ et laudabilis . Tertio cohibendi sunt , } ne absque praemeditatione loquantur . Nam iuuenes sunt inexperti , \\\hline
2.2.10 & fasta que ayan penssado \textbf{ lo que han de fablar . } Ca los moços non han prouado las cosas & ne absque praemeditatione loquantur . Nam iuuenes sunt inexperti , \textbf{ et pauca cognouerunt : } quia ergo pauca cognoscunt , \\\hline
2.2.10 & e dizen de ligero \textbf{ lo que han de dezer . } Por la qual cosa avn deuen ser amonestados & idest enunciant cito et debiliter , \textbf{ quare monendi sunt , } ne statim ad interrogata respondeant . \\\hline
2.2.10 & e las moços non pueden ser luego acabados nin sabios . \textbf{ Enpero si se acostunbraten a responder con penssamiento } e cuydando ante en las palabras que han de dezer & et iuuenes non statim possint esse perfecti et prudentes ; \textbf{ tamen si assuefiant , | ut praemeditati respondeant , } et ut prae cogitent in sermonibus proferendis , \\\hline
2.2.10 & Enpero si se acostunbraten a responder con penssamiento \textbf{ e cuydando ante en las palabras que han de dezer } por el derenemiento del tpo son ordenades & ut praemeditati respondeant , \textbf{ et ut prae cogitent in sermonibus proferendis , } per successionem temporis disponentur \\\hline
2.2.10 & por el derenemiento del tpo son ordenades \textbf{ para fablar palabras sin reprehenssion . } ¶ Visto en qual manera los ayos & per successionem temporis disponentur \textbf{ ut proferant sermones irreprehensibiles . Viso qualiter paedagogi } et doctores iuuenum debent eos instruere quomodo se habeant ad loquelam . Restat videre , quomodo sunt instruendi , ut se habeant circa visum . \\\hline
2.2.10 & e los maestros de los moços \textbf{ los deuen enssennar e enformar } en qual manera se de una auer enla fabla finca de ver & ut proferant sermones irreprehensibiles . Viso qualiter paedagogi \textbf{ et doctores iuuenum debent eos instruere quomodo se habeant ad loquelam . Restat videre , quomodo sunt instruendi , ut se habeant circa visum . } In visione autem iuuenum duplex cautela est adhibenda . \\\hline
2.2.10 & los deuen enssennar e enformar \textbf{ en qual manera se de una auer enla fabla finca de ver } en qual manera son de enssennar & et doctores iuuenum debent eos instruere quomodo se habeant ad loquelam . Restat videre , quomodo sunt instruendi , ut se habeant circa visum . \textbf{ In visione autem iuuenum duplex cautela est adhibenda . } Primo quantum ad visibilia . \\\hline
2.2.10 & en qual manera se de una auer enla fabla finca de ver \textbf{ en qual manera son de enssennar } e commo se deuen auer en uista . & In visione autem iuuenum duplex cautela est adhibenda . \textbf{ Primo quantum ad visibilia . } Secundo quantum ad modum videndi . \\\hline
2.2.10 & en qual manera son de enssennar \textbf{ e commo se deuen auer en uista . } Ca en la uista de los moços & In visione autem iuuenum duplex cautela est adhibenda . \textbf{ Primo quantum ad visibilia . } Secundo quantum ad modum videndi . \\\hline
2.2.10 & Ca en la uista de los moços \textbf{ e de los mançebos dos cautellas son de tomar . } ¶ La primera quanto alas cosas uisibles & Primo quantum ad visibilia . \textbf{ Secundo quantum ad modum videndi . } Quantum ad visibilia quidem , \\\hline
2.2.10 & ¶ La primera quanto alas cosas uisibles \textbf{ que assi commo non les conuiene de fablar cosas torpes } Et la razon desto pone el philosofo en łvij̊ libro delas ethicas & Quantum ad visibilia quidem , \textbf{ quia sicut non decet eos turpia sequi : | sic indecens est eos turpia videre . } Ratio autem eius assignatur a Philosopho 7 Polit’ \\\hline
2.2.10 & mas que conuiene \textbf{ Et desto nos viene de fazer cosas estrannas } e de fazer cosas torpes & quod omnia prima amamus magis : \textbf{ propter quod oportet ab ipsis iuuenibus extranea facere } quaecunque sunt turpia , \\\hline
2.2.10 & Et desto nos viene de fazer cosas estrannas \textbf{ e de fazer cosas torpes } e de faz cosas que han manziella . & propter quod oportet ab ipsis iuuenibus extranea facere \textbf{ quaecunque sunt turpia , } et quaecunque infectionem habent : \\\hline
2.2.10 & e por ende se sigue \textbf{ que se inclinan mas a cobdiçiar las . } Onde el philosofo defiende & magis recordantur de illis , \textbf{ et per consequens inclinantur ad concupiscendum ea . } Unde et Philosophus prohibet , \\\hline
2.2.10 & que non solamente sea defendido alos mançebos \textbf{ de ver cosas torpes } mas avn vieda & Unde et Philosophus prohibet , \textbf{ ut non solum prohibeantur iuuenes ad videndum turpia in re , } sed etiam in picturis \\\hline
2.2.10 & assi que si las mugers fuessen pintadas desnuas \textbf{ non serien de demostrar alos moços nin alos mançebos } ca aquella es hedat & ut si mulieres nudae essent depictae vel sculptae , \textbf{ non essent iuuenibus ostendendae . } Nam quia satis illa aetas de se prouocatur ad lasciuiam \\\hline
2.2.10 & ca aquella es hedat \textbf{ por que assaz es inclinada de ssi a cosas locauas e orgullosas e aseguir sus passiones . } Et por ende non conuiene en aquella hedat & non essent iuuenibus ostendendae . \textbf{ Nam quia satis illa aetas de se prouocatur ad lasciuiam } et ad passiones insequendas , non oportet ipsam per visionem turpium ad ulteriorem prouocare . \\\hline
2.2.10 & por uision de cosas torpes \textbf{ de inclinar los amayores males . } ¶ Lo segundo deuemos dar cautella alos moços & Nam quia satis illa aetas de se prouocatur ad lasciuiam \textbf{ et ad passiones insequendas , non oportet ipsam per visionem turpium ad ulteriorem prouocare . } Secundo adhibenda est cautela in iuuenibus , \\\hline
2.2.10 & de inclinar los amayores males . \textbf{ ¶ Lo segundo deuemos dar cautella alos moços } que sean enssennados & et ad passiones insequendas , non oportet ipsam per visionem turpium ad ulteriorem prouocare . \textbf{ Secundo adhibenda est cautela in iuuenibus , } ut instruantur quod palpebras oculorum cum maturitate eleuent , \\\hline
2.2.10 & que sean enssennados \textbf{ quanto ala manera de ver } assi que alçen las palpebras de los oios con grand madureza & Secundo adhibenda est cautela in iuuenibus , \textbf{ ut instruantur quod palpebras oculorum cum maturitate eleuent , } ut non habeant oculos vagabundos . Inclinatur enim aetas illa ( eo quod omnia respiciat tanquam noua ) \\\hline
2.2.10 & e que non echen los oios a cada parte con locura . \textbf{ C aquella hedat es inclinada para catar todas las cosas } assi commo si fuessen nueuas & ut instruantur quod palpebras oculorum cum maturitate eleuent , \textbf{ ut non habeant oculos vagabundos . Inclinatur enim aetas illa ( eo quod omnia respiciat tanquam noua ) } ut omnia videre velit , \\\hline
2.2.10 & assi commo si fuessen nueuas \textbf{ e quieren uer todas las cosas } e andan uagado çerca todas las cosas & ut non habeant oculos vagabundos . Inclinatur enim aetas illa ( eo quod omnia respiciat tanquam noua ) \textbf{ ut omnia videre velit , } et circa omnia visu vagatur : \\\hline
2.2.10 & que se marauilla de todas las cosas . \textbf{ Et pues que assi es conuiene a aquellos que han de prinçipar } e de enssennorear & eo quod videantur de omnibus admirari . Eos igitur \textbf{ qui debent principari } et dominari etiam ab ipsa infantia monendi sunt de modo videndi , \\\hline
2.2.10 & Et pues que assi es conuiene a aquellos que han de prinçipar \textbf{ e de enssennorear } que sean amonestados & eo quod videantur de omnibus admirari . Eos igitur \textbf{ qui debent principari } et dominari etiam ab ipsa infantia monendi sunt de modo videndi , \\\hline
2.2.10 & luego en el comienço de su moçedat \textbf{ en qual manera han de ver e de catar } por que se ayan con discreçion e con entendemiento çerca el leuamiento de los oios & et dominari etiam ab ipsa infantia monendi sunt de modo videndi , \textbf{ ut mature se habeant circa eleuationem oculorum , } et circa videndi modum : \\\hline
2.2.10 & por que se ayan con discreçion e con entendemiento çerca el leuamiento de los oios \textbf{ e çerca la manera de catar . } Ca mas ligeramente guarda cada vno & ut mature se habeant circa eleuationem oculorum , \textbf{ et circa videndi modum : } nam facilius quis vir factus obseruat quod ab infantia assueuit . Ostenso , \\\hline
2.2.10 & en que es acostunbrado de su moçedat . \textbf{ ¶ Mostrado en qual manera son de enssennar los moços } e los mançebos & nam facilius quis vir factus obseruat quod ab infantia assueuit . Ostenso , \textbf{ quomodo instruendi sunt iuuenes quantum ad loquelam , } et visionem : \\\hline
2.2.10 & quento ala fabla \textbf{ e quanto ala iusta finca de demostrar } en qual manera son de enssennar & et visionem : \textbf{ restat ostendere , } quomodo sunt instruendi , \\\hline
2.2.10 & e quanto ala iusta finca de demostrar \textbf{ en qual manera son de enssennar } e de castigar & restat ostendere , \textbf{ quomodo sunt instruendi , } ut se habeant ad auditum . \\\hline
2.2.10 & en qual manera son de enssennar \textbf{ e de castigar } e en commo se deuen auer çerca & quomodo sunt instruendi , \textbf{ ut se habeant ad auditum . } Circa quem ( quantum ad praesens spectat ) \\\hline
2.2.10 & e de castigar \textbf{ e en commo se deuen auer çerca } las cosas que oyen & quomodo sunt instruendi , \textbf{ ut se habeant ad auditum . } Circa quem ( quantum ad praesens spectat ) \\\hline
2.2.10 & las cosas que oyen \textbf{ e çerca el oyr . } Et quanto pertenesçe alo presente dos cautellas son de tomar & ut se habeant ad auditum . \textbf{ Circa quem ( quantum ad praesens spectat ) } etiam duplex cautela est adhibenda . Primo , \\\hline
2.2.10 & e çerca el oyr . \textbf{ Et quanto pertenesçe alo presente dos cautellas son de tomar } ¶La primera quanto alas cosas que oyen ¶ La segunda & Circa quem ( quantum ad praesens spectat ) \textbf{ etiam duplex cautela est adhibenda . Primo , } quantum ad res auditas . Secundo , \\\hline
2.2.10 & Mas quanto alas cosas oydas se guarda cautella en los mançebos \textbf{ si les fuer defendido de oyr cosas torpes . } Ca segunt el philosofo en el vi̊ libro delas politicas & quantum ad eos quos audit . In rebus autem auditis obseruatur cautela quantum ad iuuenes , \textbf{ si prohibeantur | ab auditione turpium . } Nam secundum philosophum vii Polit’ \\\hline
2.2.10 & Ca segunt el philosofo en el vi̊ libro delas politicas \textbf{ do fabla desta materia es deue dar alos mançebos } que non oyan cosas torpes & Nam secundum philosophum vii Polit’ \textbf{ ubi de hac materia loquitur , | prohibendi sunt iuuenes , } ne audiant quodcunque turpium : \\\hline
2.2.10 & que non oyan cosas torpes \textbf{ por que el oyr es muy çerca del obrar ¶ } Et pues que assi es segunt el philosofo los mançebos son de castigar & ne audiant quodcunque turpium : \textbf{ quia audire , | est prope ad ipsum facere . } Ideo ergo \\\hline
2.2.10 & por que el oyr es muy çerca del obrar ¶ \textbf{ Et pues que assi es segunt el philosofo los mançebos son de castigar } que non oyan cosas torpes & est prope ad ipsum facere . \textbf{ Ideo ergo | secundum Philosophum cohibendi sunt iuuenes ab auditione turpium : } quia ex hoc de facili inclinantur ad opus . Secundo , \\\hline
2.2.10 & que non oyan cosas torpes \textbf{ por que por el oyr de ligeros } e inclinarien ala obra . & secundum Philosophum cohibendi sunt iuuenes ab auditione turpium : \textbf{ quia ex hoc de facili inclinantur ad opus . Secundo , } adhibenda est cautela in ipsis iuuenibus \\\hline
2.2.10 & e inclinarien ala obra . \textbf{ ¶ Lo segundo deuemos dar otra cautela alos mocos e alos mançebos } quanto a aquellos que oyen . & quia ex hoc de facili inclinantur ad opus . Secundo , \textbf{ adhibenda est cautela in ipsis iuuenibus } quantum ad eos quos audiunt : \\\hline
2.2.10 & Ca assi commo es cosa conuenible a ellos \textbf{ de oyr cosas honestas e fermosas } e desconuenible de oyr cosas torpes & quia sicut decens est audire eos honesta , \textbf{ et pulchra , et indecens audire turpia : } sic decet eos audire viros bonos \\\hline
2.2.10 & de oyr cosas honestas e fermosas \textbf{ e desconuenible de oyr cosas torpes } assi les conuienea ellos de oyr a bueons omes e honestos & quia sicut decens est audire eos honesta , \textbf{ et pulchra , et indecens audire turpia : } sic decet eos audire viros bonos \\\hline
2.2.10 & e desconuenible de oyr cosas torpes \textbf{ assi les conuienea ellos de oyr a bueons omes e honestos } e son de refrenar & et pulchra , et indecens audire turpia : \textbf{ sic decet eos audire viros bonos | et honestos , } et cohibendi sunt \\\hline
2.2.10 & assi les conuienea ellos de oyr a bueons omes e honestos \textbf{ e son de refrenar } que non oyan alos maldizientes & et honestos , \textbf{ et cohibendi sunt } ne audiant maliloquos et inhonestos . \\\hline
2.2.11 & nin alos desonestos e cacurros . \textbf{ a dixiemos de suso en comm̃ los moços } luego en su moçedat deuia ser enssennados en buenas costunbres . & ne audiant maliloquos et inhonestos . \textbf{ Diximus superius , iuuenes ab infantia } instruendos esse in bonis moribus : \\\hline
2.2.11 & que las spanles \textbf{ por ende conuiene nos de dezir speçialmente } en qual manera los moços deuen ser enssennados & in morali negocio sermones uniuersaliores minus proficiunt , \textbf{ oportet specialiter tradere , } quomodo iuuenes sunt in bonis moribus instruendi . \\\hline
2.2.11 & Por la qual cosa despues que dixiemos \textbf{ en qual manera se deuen auer çerca la fabla } e cerca la uista & Quare postquam diximus , \textbf{ quomodo se habere debent circa loquelam , } et circa visum , \\\hline
2.2.11 & lo que oyen finca \textbf{ deuer } en qual manera se de una auer çerca el comer & et auditum . \textbf{ Restat ostendere , } quomodo se habere debeant circa cibum , et circa potum , et circa venerea , siue circa matrimonium contrahendum . \\\hline
2.2.11 & en qual manera se de una auer çerca el comer \textbf{ e çerca el bener e c̃ca las cosas de luxia o çerca del casamiento } que deuen tomar . & Restat ostendere , \textbf{ quomodo se habere debeant circa cibum , et circa potum , et circa venerea , siue circa matrimonium contrahendum . } Sed primo dicemus , \\\hline
2.2.11 & e çerca el bener e c̃ca las cosas de luxia o çerca del casamiento \textbf{ que deuen tomar . } Mas primero diremos & ø \\\hline
2.2.11 & en quantas maneras pecan los omes \textbf{ en resçebir el comer } e en qual manera se de una auer los moços & Sed primo dicemus , \textbf{ quot modis peccatur circa sumptionem cibi : } et qualiter se debeant habere iuuenes circa ipsum . Circa cibum autem contingit sex modis peccare , \\\hline
2.2.11 & e los mançebos çerta el comer \textbf{ Mas conuiene saber } que cerça el comer pueden los omes errar en seys maneras . & et qualiter se debeant habere iuuenes circa ipsum . Circa cibum autem contingit sex modis peccare , \textbf{ vel delinquere . } Primo si sumatur ardenter . Secundo , si nimis . Tertio , \\\hline
2.2.11 & Mas conuiene saber \textbf{ que cerça el comer pueden los omes errar en seys maneras . } ¶ Lo primero si comieren con grand garganteria ¶ Lo segundo si comieren much¶ & et qualiter se debeant habere iuuenes circa ipsum . Circa cibum autem contingit sex modis peccare , \textbf{ vel delinquere . } Primo si sumatur ardenter . Secundo , si nimis . Tertio , \\\hline
2.2.11 & e con grand golosia . \textbf{ Ca assi commo da a entender el philosofo } en el terçero libro delas ethicas pequana delectaçiones & ut plurimum non obseruant sumentes cibum auide . \textbf{ Nam ( ut innuit Philosophus 3 Ethicorum ) modica delectatio est , } cum cibus attingit linguam : \\\hline
2.2.11 & por que enbarga el cozimiento conuenible . \textbf{ Ca si la vianda se ouiere bien a cozer } conuiene que sea bien proporçionada ala calentura natural & quia impedit digestionem debitam . \textbf{ Si enim cibus digeri debeat , oportet ipsum esse proportionatum calori naturali . } Quare si in tanta quantitate sumatur , quod calor naturalis ei dominari non possit , non bene digeritur , \\\hline
2.2.11 & Por la qual cosa \textbf{ si en tan grand quantia se tomaque la calentura natural non pueda enssennorar } sobrella non se puede bien moler nin cozer . & Si enim cibus digeri debeat , oportet ipsum esse proportionatum calori naturali . \textbf{ Quare si in tanta quantitate sumatur , quod calor naturalis ei dominari non possit , non bene digeritur , } et per consequens non causat debitum nutrimentum . \\\hline
2.2.11 & si en tan grand quantia se tomaque la calentura natural non pueda enssennorar \textbf{ sobrella non se puede bien moler nin cozer . } Et assi se sigue & Si enim cibus digeri debeat , oportet ipsum esse proportionatum calori naturali . \textbf{ Quare si in tanta quantitate sumatur , quod calor naturalis ei dominari non possit , non bene digeritur , } et per consequens non causat debitum nutrimentum . \\\hline
2.2.11 & Et assi se sigue \textbf{ que non faga nudrimiento conueinble¶ Lo . } iij . pecan si toma la uianda torpemente & Quare si in tanta quantitate sumatur , quod calor naturalis ei dominari non possit , non bene digeritur , \textbf{ et per consequens non causat debitum nutrimentum . } Tertio delinquitur , \\\hline
2.2.11 & e suzia mente . \textbf{ Ca son muchos que non saben gouernar assi mismos . } Los quales abeso nunca pueden comer que non enlixen sus vestiduras . & Tertio delinquitur , \textbf{ si sumatur turpiter . Sunt enim plurimi seipsos pascere nescientes , } quod vix aut nunquam comedere possunt , \\\hline
2.2.11 & Ca son muchos que non saben gouernar assi mismos . \textbf{ Los quales abeso nunca pueden comer que non enlixen sus vestiduras . } Mas la torpedat del cuerpo commo quier & si sumatur turpiter . Sunt enim plurimi seipsos pascere nescientes , \textbf{ quod vix aut nunquam comedere possunt , | quin sua vestimenta deturpent : } turpitudo autem corporalaris licet \\\hline
2.2.11 & si contesçiere por desordenamiento del alma . \textbf{ Por la qual cosa commo la manera torpe de resçebir la vianda sea señal de golosina } de aquel & si contingat ex inordinatione animae . \textbf{ Quare cum turpis modus sumendi cibum } signum sit cuiusdam gulositatis , \\\hline
2.2.11 & en el resçibimiento del maniar \textbf{ non solamente deuemos esquiuar la golosina e la grand cobdiçia } mas avn deuemose squiuar de comer la vianda torpemente & signum sit cuiusdam gulositatis , \textbf{ vel inordinationis mentis in sumptione cibi , non solum cauendus est ardor et nimietas , } sed \\\hline
2.2.11 & non solamente deuemos esquiuar la golosina e la grand cobdiçia \textbf{ mas avn deuemose squiuar de comer la vianda torpemente } e suziamente ¶ & signum sit cuiusdam gulositatis , \textbf{ vel inordinationis mentis in sumptione cibi , non solum cauendus est ardor et nimietas , } sed \\\hline
2.2.11 & Ca si mucho ante ora \textbf{ e mucho desordenadamente tomar enla uianda pecan } por que por tal resçibimiento se fagen algunos golosos e destenprados & Quarto circa cibum delinquitur ex inordinatione temporis : ut si nimis ante horam , \textbf{ vel nimis inordinate sumatur cibus . } Nam ex tali sumptione efficitur quis gulosus \\\hline
2.2.11 & assi commo otra nata por ende \textbf{ quando alguno se acostunbra a tomar la uianda en algua ora desordenada } por la mayor parte dessea dela tomar en aquella misma ora . & ideo \textbf{ cum quis assuescit , | sumere cibum in aliqua hora , } ut plurimum appetit sumptionem eius in eadem hora . \\\hline
2.2.11 & quando alguno se acostunbra a tomar la uianda en algua ora desordenada \textbf{ por la mayor parte dessea dela tomar en aquella misma ora . } Et pues que assi es & sumere cibum in aliqua hora , \textbf{ ut plurimum appetit sumptionem eius in eadem hora . } Si ergo inordinate \\\hline
2.2.11 & si del ordenadamente toma la uianda \textbf{ e vsare comer ante de tien po en la mayor parte } assi que ante dela digestiuo & Si ergo inordinate \textbf{ et praeter consuetam horam utatur quis cibum sumere , } ut plurimum ante digestionem primi cibi sumitur secundus cibus , laeditur ergo inde corpus . Hora ergo debita et determinata , non solum propter bonitatem animae , \\\hline
2.2.11 & non solamente por la bondat del alma \textbf{ mas avn por sanidat del cuerpo deuemos guardar ora conuenible } en que deuemos rescebir la vianda . & sed etiam propter sanitatem corporis , \textbf{ obseruanda est in sumptione cibi . } Quinto peccatur circa sumptionem cibi , \\\hline
2.2.11 & mas avn por sanidat del cuerpo deuemos guardar ora conuenible \textbf{ en que deuemos rescebir la vianda . } ¶ Lo quinto pecan los omes & sed etiam propter sanitatem corporis , \textbf{ obseruanda est in sumptione cibi . } Quinto peccatur circa sumptionem cibi , \\\hline
2.2.11 & si demandare viandas apareiadas con grant estudio \textbf{ por que avn en las viles viandas cada vno se puede mostrar } por muy goloso & si quaerantur cibaria nimis studiose parata . Nam \textbf{ etiam in vilibus cibariis potest } quis ostendere se nimis gulosum , \\\hline
2.2.11 & por muy goloso \textbf{ si las quisiere auer apareiadas con quant estudio . } Ca paresçeque tales quieren beuir & quis ostendere se nimis gulosum , \textbf{ si nimio studio velit ea esse parata . } Videntur enim tales viuere \\\hline
2.2.11 & por que coma \textbf{ et non quieren comer } por que biuna por que ponen grand estudio e grant cuydado çerca los apareiamientos delas viandas . & ut comedant , \textbf{ non comedere ut viuant , } cum nimium studium \\\hline
2.2.11 & por que biuna por que ponen grand estudio e grant cuydado çerca los apareiamientos delas viandas . \textbf{ Et pues que assi es estas cosas vistas de ligero pueden paresçer } en qual manera son de enssennar los moços & et nimiam \textbf{ curam apponant circa praeparamenta ciborum . | His ergo visis de leui apparet , } qualiter instruendi sunt pueri , \\\hline
2.2.11 & Et pues que assi es estas cosas vistas de ligero pueden paresçer \textbf{ en qual manera son de enssennar los moços } commo se deuan auer çerca las viandas . & His ergo visis de leui apparet , \textbf{ qualiter instruendi sunt pueri , } ut se habeant circa cibos . \\\hline
2.2.11 & en qual manera son de enssennar los moços \textbf{ commo se deuan auer çerca las viandas . } Ca ninguno adesora non se faze grande . & qualiter instruendi sunt pueri , \textbf{ ut se habeant circa cibos . } Nam nullus repente fit summus . \\\hline
2.2.12 & es pues que enł capitulo sobredich̃ dixiemos \textbf{ en qual manera pecan los omes çerca del comer } finca nons de dezer & Postquam in praecedenti capitulo diximus , \textbf{ qualiter delinquitur circa cibum . } Restat dicere , \\\hline
2.2.12 & en qual manera pecan los omes çerca del comer \textbf{ finca nons de dezer } en qual manera pecan çerca el beuer . & qualiter delinquitur circa cibum . \textbf{ Restat dicere , } quomodo delinquitur circa potum . Dicebatur enim superius quod iuuenilis aetas \\\hline
2.2.12 & finca nons de dezer \textbf{ en qual manera pecan çerca el beuer . } Ca dicho es de suso & Restat dicere , \textbf{ quomodo delinquitur circa potum . Dicebatur enim superius quod iuuenilis aetas } maxime est prona ad intemperantiam , \\\hline
2.2.12 & Ca do mayor es el periglo \textbf{ alli deue omne poner mayor remedio } Et por ende en la hedat de los moços deuemos guardar & maxime est prona ad intemperantiam , \textbf{ quare cum semper sit adhibenda cautela , } ubi maius periculum imminet , \\\hline
2.2.12 & alli deue omne poner mayor remedio \textbf{ Et por ende en la hedat de los moços deuemos guardar } que non se fagan destenprados . & quare cum semper sit adhibenda cautela , \textbf{ ubi maius periculum imminet , | in puerili aetate cauendum est } ne iuuenes efficiantur intemperati . \\\hline
2.2.12 & Ca la tenprança ha de ser puesta çerca de tres cosas . \textbf{ Conuiene de saber . } Cerca el comer . Cerca el beuer . & Temperantia autem circa tria est adhibenda : \textbf{ circa cibum , potum , } et venerea . \\\hline
2.2.12 & Conuiene de saber . \textbf{ Cerca el comer . Cerca el beuer . } Et çerca el vso dela luxia . & Temperantia autem circa tria est adhibenda : \textbf{ circa cibum , potum , } et venerea . \\\hline
2.2.12 & commo non conuiene faze destenprança \textbf{ mas avn el beuer faze esso mismo } Pues que assi es conuiene alos moços & Nam non solum cibus indebite sumptus intemperantiam causa , \textbf{ sed etiam potus . Decet ergo iuuenes non solum esse abstinentes , } ut non efficiantur gulosi ex sumptione cibi : \\\hline
2.2.12 & que non se fagan beodos \textbf{ por el beuer o por el much vino . } Ca quanto pertenesçe alo presente el vino tomado sin mesura faze tres males . & sed etiam decet eos esse sobrios , \textbf{ ut non efficiantur ebrii ex sumptione potus . } Vinum enim immoderate sumptum \\\hline
2.2.12 & e abiualo a destenprança de lux̉ia . \textbf{ Et por ende el mucho beuer del vino } mas es de guardar en la bedat de los moços & incitat ad incontinentiam nimiam . \textbf{ Ergo sumptio vini , } in quantum venerea prouocat , \\\hline
2.2.12 & Et por ende el mucho beuer del vino \textbf{ mas es de guardar en la bedat de los moços } que de los vieios & Ergo sumptio vini , \textbf{ in quantum venerea prouocat , } tanto magis in aetate iuuenili quam senili cauenda est , \\\hline
2.2.12 & El segundo mal \textbf{ que viene del tomar mucho el vino } es çegamiento de la razon e del entendimiento . & Secundum malum , \textbf{ quod inducit nimia sumptio vini , est depressio rationis . Nam ascendentibus fumositatibus vini ad caput , } turbatur cerebrum : \\\hline
2.2.12 & quanto alas sus obras \textbf{ por que non podemos libremente vsar de razon . } Et pues que assi es el tomar del vino destenpradamente & quo turbato deprimitur ratio nostra quantum ad suos actus , \textbf{ quia non possumus libere ratione uti . } Immoderata ergo sumptio vini in tantum impedit rationis usum , \\\hline
2.2.12 & por que non podemos libremente vsar de razon . \textbf{ Et pues que assi es el tomar del vino destenpradamente } en tanto enbar gar el vlo de la razon & quia non possumus libere ratione uti . \textbf{ Immoderata ergo sumptio vini in tantum impedit rationis usum , } in quantum turbat cerebrum . \\\hline
2.2.12 & Et pues que assi es el tomar del vino destenpradamente \textbf{ en tanto enbar gar el vlo de la razon } en quanto turba el meollo . & quia non possumus libere ratione uti . \textbf{ Immoderata ergo sumptio vini in tantum impedit rationis usum , } in quantum turbat cerebrum . \\\hline
2.2.12 & e mas ayna se turba el meollo dellos \textbf{ Et pues que assi es de defender es alos mocos } que non beuna much vino . & quia habent debilius caput , \textbf{ et citius turbatur eorum cerebrum . Prohibendi sunt ergo iuuenes a nimia sumptione vini , } quia propter debilitatem cerebri citius offenduntur a vino . Tertium malum quod ex vino consurgit , est lis \\\hline
2.2.12 & de ligero salta el omne \textbf{ en dezer palabras desordenadas } Et por ende se le una tan discordias e varaiasas & Immo quia consurgit ex inflammatione sanguinis , \textbf{ vinum , } quod propter sui caliditatem inflammat sanguinem , \\\hline
2.2.12 & que non be una much bino \textbf{ por el qual beuer se mueuen alas peleas e alas varaias . } Et pues que assi es en toda hedat deuemos escusar el mucho comer & prohibendi sunt a nimia sumptione vini , \textbf{ per quam quis ad lites | et contumelias prouocatur . In omni } ergo aetate cauendum est a nimietate cibi , \\\hline
2.2.12 & por el qual beuer se mueuen alas peleas e alas varaias . \textbf{ Et pues que assi es en toda hedat deuemos escusar el mucho comer } e el mucho beuer . & et contumelias prouocatur . In omni \textbf{ ergo aetate cauendum est a nimietate cibi , } et ab immoderatione potus : \\\hline
2.2.12 & Et pues que assi es en toda hedat deuemos escusar el mucho comer \textbf{ e el mucho beuer . } Enpero por que mas ligeramente nos allegamos a aquellas cosas & ergo aetate cauendum est a nimietate cibi , \textbf{ et ab immoderatione potus : | veruntamen } quia facilius adhaeremus iis , ad quae ab infantia assueti sumus , \\\hline
2.2.12 & e mayormente alos Reyes \textbf{ e alos prinçipes de auer grand cuydado çerca el gouernamientode los fijos } por que en tal manera sean gouernados en sumo oçedat & decet omnes patres \textbf{ et maxime reges et Principes solicitari circa regimen filiorum , } ut ab ipsa infantia sic regantur , \\\hline
2.2.12 & por que sean guardados e mesurados ¶ \textbf{ Visto en qual manera los moços se deuen auer çerca el comer } e el beuer finca de ver & et sobrii . Viso , \textbf{ qualiter iuuenes debeant se habere circa cibum et potum . } Restat videre , \\\hline
2.2.12 & Visto en qual manera los moços se deuen auer çerca el comer \textbf{ e el beuer finca de ver } en qual manera se deuen auer çerca las cosas de luxuria & qualiter iuuenes debeant se habere circa cibum et potum . \textbf{ Restat videre , } quomodo se debeant habere circa venerea , \\\hline
2.2.12 & e el beuer finca de ver \textbf{ en qual manera se deuen auer çerca las cosas de luxuria } e cerca el casamiento & Restat videre , \textbf{ quomodo se debeant habere circa venerea , } et circa coniugia contrahenda . \\\hline
2.2.12 & e cerca el casamiento \textbf{ que deuen tomar } por que la luxuria nasce dela garganteria & quomodo se debeant habere circa venerea , \textbf{ et circa coniugia contrahenda . } Oritur enim luxuria ex gula , \\\hline
2.2.12 & que los mançebos deuen ser enssennados \textbf{ por que non sean golosos finca de dezer } en qual manera deuen ler enssennados & iuuenes ipsos instruendos esse , \textbf{ ne sint gulosi : | restat dicere , } quomodo instruendi sunt , \\\hline
2.2.12 & por que non sean golosos finca de dezer \textbf{ en qual manera deuen ler enssennados } por qua non sean locanos e orgullosos nin luxiosos . & restat dicere , \textbf{ quomodo instruendi sunt , } ne sint lasciui . Cum ergo omnis actus venereus , \\\hline
2.2.12 & En essa misma manera los mançebos \textbf{ que non quieren guardar sus cuerpos deuen los en dozir } que sean pagados de sus mugers ppas . & ut sint virtuosi , \textbf{ iuuenes continere nolentes , inducendi sunt } ut propria coniuge sint contenti . \\\hline
2.2.12 & que sean pagados de sus mugers ppas . \textbf{ Mas en qual hedat deuen los omes vsar del casamiento } muestra lo el philosofo en el septimo libro delas politicas & ut propria coniuge sint contenti . \textbf{ In qua autem aetate debeant uti coniugio , } ostendit Philosophus 7 Poli’ dicens , \\\hline
2.2.12 & o dize \textbf{ que enla muger es de demandar hedat de dize ocho a nons e el vaton de veynte e dos } porque en tal hedat se engendran los fijos & ostendit Philosophus 7 Poli’ dicens , \textbf{ quod in muliere requiritur aetas decem et octo annorum , | in masculo sex et triginta : } in tali enim aetate \\\hline
2.2.12 & segunt que dize el philosofo . \textbf{ Mas porque la uirtud de engendrar } assi common dixiemos de suso es muy corrupta abasta & secundum ipsum ) procreantur filii magis perfecti . \textbf{ Sed quia vis generatiua } ( ut superius diximus ) est nimis corrupta , \\\hline
2.2.12 & en qual tp̃o dura comunalmente \textbf{ fasta xxvn año deuen se guardar los mocos } e los mançebos del ayuntamiento carneral & quod durat communiter usque ad vigesimum primum annum , \textbf{ abstinere iuuenes a carnali copula : } quod si infra tale tempus utantur coniugio , \\\hline
2.2.12 & assi commo dize el philosofo en esse mismo libro dela o politicas \textbf{ Et pues que assi es en esta misma manera deuemos vsar del casamiento } si lanr̃afuerça del appetito desseador non fuere muy corrupta . & ut vult Philosophus in eisdem Poli’ . \textbf{ Sic ergo utendum est coniugio , } si nostra vis concupiscibilis \\\hline
2.2.12 & si lanr̃afuerça del appetito desseador non fuere muy corrupta . \textbf{ Enpero por que deuemostemer dela corrupçion del ippetito desseador } si los maestros de los moços entienden & non esset nimis corrupta : \textbf{ quia tamen timendum est de corruptione concupiscibili , } si doctores puerorum percipiant iuuenes tantum tempus expectare non posse , \\\hline
2.2.12 & que los moços o los mançebos non pue den \textbf{ espar tanto tienpo pueden abreuiar aquel tienpo } e poner lo ante & si doctores puerorum percipiant iuuenes tantum tempus expectare non posse , \textbf{ poterit illud tempus anticipari prout eis videbitur expedire . Qualiter autem se debeant habere iuuenes cum uxore iam ducta , } et quae sunt consideranda in uxore ducenda : supra , \\\hline
2.2.12 & espar tanto tienpo pueden abreuiar aquel tienpo \textbf{ e poner lo ante } assi commo vieren & si doctores puerorum percipiant iuuenes tantum tempus expectare non posse , \textbf{ poterit illud tempus anticipari prout eis videbitur expedire . Qualiter autem se debeant habere iuuenes cum uxore iam ducta , } et quae sunt consideranda in uxore ducenda : supra , \\\hline
2.2.12 & que ya tienen tomadas \textbf{ e quales cosas son aquellas que deuen cuydar en las mugers } que deuen tomar de suso lo dixiemos mas largamente & poterit illud tempus anticipari prout eis videbitur expedire . Qualiter autem se debeant habere iuuenes cum uxore iam ducta , \textbf{ et quae sunt consideranda in uxore ducenda : supra , } cum egimus de regimine coniugali , diffusius diximus . \\\hline
2.2.12 & e quales cosas son aquellas que deuen cuydar en las mugers \textbf{ que deuen tomar de suso lo dixiemos mas largamente } quando dixiemos del gouernamiento del casamiento & et quae sunt consideranda in uxore ducenda : supra , \textbf{ cum egimus de regimine coniugali , diffusius diximus . } Ostenso , \\\hline
2.2.13 & ostrado en qual manera los as . moços deuen ser guardados enla vianda \textbf{ e mesurados enel beuer } e tenpdos en la lux̉ia & quomodo iuuenes debent esse abstinentes in cibo , \textbf{ sobrii in potu , } temperati in venereis , \\\hline
2.2.13 & en tomando su casamiento en hedat conuenible \textbf{ e commo se deuen auer tenpradamente con sus mugers } que han ya tomadas finca de demostrar & contrahendo coniugium in aetate debita , \textbf{ et modeste se habere cum uxore iam ducta . Restat ostendere , } quomodo circa ludos , \\\hline
2.2.13 & e commo se deuen auer tenpradamente con sus mugers \textbf{ que han ya tomadas finca de demostrar } en qual manera se deuen auer çerca los trebeios & contrahendo coniugium in aetate debita , \textbf{ et modeste se habere cum uxore iam ducta . Restat ostendere , } quomodo circa ludos , \\\hline
2.2.13 & que han ya tomadas finca de demostrar \textbf{ en qual manera se deuen auer çerca los trebeios } e çerca los gestos & et modeste se habere cum uxore iam ducta . Restat ostendere , \textbf{ quomodo circa ludos , } circa gestus , \\\hline
2.2.13 & es neçessario enla vida humanal \textbf{ la qual cosa podemos declarar } quanto parte nesçe a lo presente & ø \\\hline
2.2.13 & por dos razones . \textbf{ ¶ Lo pripreo por escusar cuydado desconneible ¶ Lo segundo por alcançar fin conuenible ¶ } La primera razon paresçe assi . & ut probat Philosophus 8 Poli’ est necessarius in vita quod ( quantum ad praesens spectat ) duplici via declarari potest . Primo , \textbf{ ex vitatione illicitae solicitudinis . Secundo , } ex adeptione finis intenti . Prima via sic patet . \\\hline
2.2.13 & Ca la uoluntad del omne non sabe ser ocçiosa \textbf{ nin estar de vagar Et pues que assi es } quando alguon se da adagar & ex adeptione finis intenti . Prima via sic patet . \textbf{ Nam mens humana nescit ociosa esse : cum ergo quis vacat ocio , } et non intendit aliquibus delectationibus licitis , \\\hline
2.2.13 & nin estar de vagar Et pues que assi es \textbf{ quando alguon se da adagar } e non entiende en algunas delectaçiones conuenibles & ex adeptione finis intenti . Prima via sic patet . \textbf{ Nam mens humana nescit ociosa esse : cum ergo quis vacat ocio , } et non intendit aliquibus delectationibus licitis , \\\hline
2.2.13 & e non entiende en algunas delectaçiones conuenibles \textbf{ luego comiença a andar vagando cuydando enlas cosas desconueibles . } Onde el philosofo enłviiij libro delas politicas dize & et non intendit aliquibus delectationibus licitis , \textbf{ statim incipit vagari cogitando de illicitis : } unde Philosophus 8 Polit’ \\\hline
2.2.13 & Conuiene a nos algunas uegadas \textbf{ de auer algunos trebeios } e algunos solazes conuenibles e honestos . & et vitemus delectationes illicitas , \textbf{ expedit aliquando habere aliquos ludos , } et habere aliquas deductiones licitas \\\hline
2.2.13 & e quales son estos solazes çerca \textbf{ los queles los moços deuen entender adelante se dira ¶ } La segunda razon & Qui sunt autem illi ludi , \textbf{ et quae sunt illae deductiones , circa quas debent vacare pueri , infra dicetur . Secunda via ad ostendendum hoc idem , sumitur ex adeptione finis intenti . } Nam non semper statim quis habere potest finem intentum : \\\hline
2.2.13 & La segunda razon \textbf{ para prouar esto mismo se toma del alcançamiento } dela fin & et quae sunt illae deductiones , circa quas debent vacare pueri , infra dicetur . Secunda via ad ostendendum hoc idem , sumitur ex adeptione finis intenti . \textbf{ Nam non semper statim quis habere potest finem intentum : } ne ergo propter continuos labores deficiat a consecutione finis , \\\hline
2.2.13 & que ente demos . \textbf{ Ca non puede ninguno sienpreauer luego la fin } que entiende . & Nam non semper statim quis habere potest finem intentum : \textbf{ ne ergo propter continuos labores deficiat a consecutione finis , } expedit aliquos ludos \\\hline
2.2.13 & por que non fallezca el omne \textbf{ por los grandes trabaios de alcançar su fin } conuienel de entreponer alguons trebeios & expedit aliquos ludos \textbf{ et aliquas deductiones interponere suis curis , } ut ex hoc aliquam requiem recipientes , \\\hline
2.2.13 & por los grandes trabaios de alcançar su fin \textbf{ conuienel de entreponer alguons trebeios } e algunos solazes en sus cuydados . & et aliquas deductiones interponere suis curis , \textbf{ ut ex hoc aliquam requiem recipientes , } magis possint laborare in consecutione finis . Unde et Philosophus 8 Politicorum ait , \\\hline
2.2.13 & e algunos solazes en sus cuydados . \textbf{ assi que en esto resçibiendo alguno folgua a puedan mas trabaiar } para alcançar su fin . & ut ex hoc aliquam requiem recipientes , \textbf{ magis possint laborare in consecutione finis . Unde et Philosophus 8 Politicorum ait , } quod \\\hline
2.2.13 & assi que en esto resçibiendo alguno folgua a puedan mas trabaiar \textbf{ para alcançar su fin . } Onde el philosofo enłviij̊ delas politicas dize & magis possint laborare in consecutione finis . Unde et Philosophus 8 Politicorum ait , \textbf{ quod } quia homo non potest requiescere in fine adepto , \\\hline
2.2.13 & Onde el philosofo enłviij̊ delas politicas dize \textbf{ que por que el ome non puede sienpre folgar } en la fin ganada . & quod \textbf{ quia homo non potest requiescere in fine adepto , } et aliquando quis constituit sibi finem , \\\hline
2.2.13 & que alcançe aquella fin \textbf{ por ende conuienele de entroponer alguons trebeios } e algunas delectaconnes & antequam consequatur illum , \textbf{ ideo oportet interponere aliquos ludos , } et aliquas delectationes , \\\hline
2.2.13 & e algunas delectaconnes \textbf{ por que non fallesca de alcançar } a qual la fin . & et aliquas delectationes , \textbf{ ne deficiat a consecutione finis . } Sic ergo instruendi sunt pueri erga ludos , \\\hline
2.2.13 & Ca los trebeios torpes \textbf{ e los solazes desonestos son de defender alos moços } assi commo dize el philosofo enł . viij̊ delas politicas ¶ & Nam ludi turpes , \textbf{ et eloquia turpia , | et deductiones inhonestae prohibendae sunt a iuuenibus , } ut vult Philosophus 7 Politicorum . \\\hline
2.2.13 & assi commo dize el philosofo enł . viij̊ delas politicas ¶ \textbf{ Visto en qual manera los moços se deuen auer çerca } los trebeios & ut vult Philosophus 7 Politicorum . \textbf{ Viso qualiter iuuenes se habere debeant circa ludos . Restat videre , } qualiter se habere debeant circa gestus . Gestus autem dicuntur quilibet motus membrorum , \\\hline
2.2.13 & los trebeios \textbf{ finca de ver } en qual manera se deuen auer çerca los gestos ¶ & ø \\\hline
2.2.13 & finca de ver \textbf{ en qual manera se deuen auer çerca los gestos ¶ } Los gestos son dichos & Viso qualiter iuuenes se habere debeant circa ludos . Restat videre , \textbf{ qualiter se habere debeant circa gestus . Gestus autem dicuntur quilibet motus membrorum , } ex quibus iudicari possunt motus animae . Videmus enim prudentes \\\hline
2.2.13 & por que non ayan algun mouimiento \textbf{ del qual alguno pueda presumir en ellos } soƀua del coraçon & ex \textbf{ quo quis coniecturari possit elationem animi , } vel insipientiam mentis , vel intemperantiam appetitus . \\\hline
2.2.13 & que vsa de razon e de entendimiento . \textbf{ Et pues que assi es para fazer el omne obras conueinbles } non es inclinado conplidamente & qui utitur ratione et intellectu ; \textbf{ et ad agendum sibi opera debita , } non sufficienter inclinatur ex natura . Disciplina autem , \\\hline
2.2.13 & Et la doctrina e la enssenança \textbf{ que es de dar en los gestos alos moços } es tal que cada vn mienbro sea ordenado ala obra & non sufficienter inclinatur ex natura . Disciplina autem , \textbf{ quae est danda in gestibus , est , } ut quodlibet membrum ordinetur ad opus sibi debitum . Homo enim non audit per os , \\\hline
2.2.13 & Et pues que assi es \textbf{ quando alguno quiere oyr al otro } en vano tiene la boca abierta . & Frustra ergo , \textbf{ cum quis vult audire alium , } retinet os apertum . Sic etiam homo non loquitur pedibus , \\\hline
2.2.13 & que assi es \textbf{ assi commo aquellos que quieren oyr alos otros } e tienen las bocas abiertas & Sicut ergo habent indisciplinatos gestus , \textbf{ qui cum volunt audire alios , } tenent ora aperta : \\\hline
2.2.13 & En essa misma manera son desenssennados segunt los gestos \textbf{ aquellos que quando que eren fablar estienden los pies } e las prinas o mueuen los braços & secundum gestus , \textbf{ qui cum volunt loqui , | extendunt pedes et crura , } vel mouent nimis spissim brachia , \\\hline
2.2.13 & que non siruen en ninguna cosa ala fabla . \textbf{ En esta misma manera son de enssennar los moços } que ayan tales gestos & vel erigunt humeros , vel faciunt alia , \textbf{ quae ad locutionem nihil deseruiunt . Sic ergo disciplinandi sunt iuuenes , } ut habeant tales gestus , et ut sic utantur motibus membrorum , \\\hline
2.2.13 & por que assi vsen de los mouimientos de los mienbros \textbf{ por que sir una alas obras } que entienden fazer . & ut habeant tales gestus , et ut sic utantur motibus membrorum , \textbf{ ut deseruiant ad opera quae intendunt . } Nam agere aliquos motus membrorum non deseruientes operi intento , \\\hline
2.2.13 & por que sir una alas obras \textbf{ que entienden fazer . } Ca fazer alguons mouimientos de los mienbros & ut habeant tales gestus , et ut sic utantur motibus membrorum , \textbf{ ut deseruiant ad opera quae intendunt . } Nam agere aliquos motus membrorum non deseruientes operi intento , \\\hline
2.2.13 & que entienden fazer . \textbf{ Ca fazer alguons mouimientos de los mienbros } que non siruen ala obra que entienden fazer . & ut deseruiant ad opera quae intendunt . \textbf{ Nam agere aliquos motus membrorum non deseruientes operi intento , } vel procedit ex insipientia mentis , \\\hline
2.2.13 & Ca fazer alguons mouimientos de los mienbros \textbf{ que non siruen ala obra que entienden fazer . } Et estos salle & ut deseruiant ad opera quae intendunt . \textbf{ Nam agere aliquos motus membrorum non deseruientes operi intento , } vel procedit ex insipientia mentis , \\\hline
2.2.13 & ¶ Estas cosas \textbf{ assi vistas finca de seguir } e de tractar delo terçero & vel ex aliquo alio vitio . Hiis visis , \textbf{ restat exequi de tertio , } quod proponebatur in principio capituli , \\\hline
2.2.13 & assi vistas finca de seguir \textbf{ e de tractar delo terçero } que era propuesto en el comienço del capitulo . & vel ex aliquo alio vitio . Hiis visis , \textbf{ restat exequi de tertio , } quod proponebatur in principio capituli , \\\hline
2.2.13 & que era propuesto en el comienço del capitulo . \textbf{ Conuienea saber } en qual manera se de una auer los moços & ø \\\hline
2.2.13 & por bien de honrra \textbf{ assi las deuemos querer fermosas e apuestas . } Ca cosa desconuenible es al omne ser & ø \\\hline
2.2.13 & por que las armas del fierro han en ssi alguna dureza \textbf{ por ende aquellos que han cuydado çerca tales vestiduras muelles dubdan de tomar las armas } e fazen semedrosos . & Nam cum arma ferrea in se quandam duriciem habeant , \textbf{ qui semper solicitantur circa mollia vestimenta , | dubitant arma arripere , } et efficiuntur timidi . Iuuenes , maxime cum ad aliam aetatem venerint , \\\hline
2.2.13 & que sean bien ordenados e bien apareiados \textbf{ para entender çerca los trabaios delas batallas } e que non aborrezcan las armas & et efficiuntur timidi . Iuuenes , maxime cum ad aliam aetatem venerint , \textbf{ ad hoc quod sint habiles ad vacandum circa labores bellicos , } ne abhorreant arma , instruendi sunt , \\\hline
2.2.13 & e que non aborrezcan las armas \textbf{ por ende son de enssennar } que non se delecten mucho en las vestiduras muelles . & ad hoc quod sint habiles ad vacandum circa labores bellicos , \textbf{ ne abhorreant arma , instruendi sunt , } ut non nimis delectentur in mollicie vestium . Dicto , \\\hline
2.2.13 & e de una auer los mançebos en las uestidans \textbf{ en quanto ellas siruen ala delectaçion finca de demostrar } en qual manerase de una auer çerca las vestidas & quomodo se habere debeant iuuenes in vestibus , \textbf{ ut deseruiunt ad delectationem . | Restat ostendere , } quomodo se habere debeant circa ipsa , \\\hline
2.2.13 & en quanto ellas siruen al prouecho \textbf{ la qual cosa se puede ver departiendo entre las conplissiones e los tienpos e las hedades . } Ca los que han las conplissiones espessas & ut deseruiunt ad utilitatem . Quod videri habet , \textbf{ distinguendo inter complexiones , tempora , | et aetates . } Nam habentes complexiones magis depressas \\\hline
2.2.13 & por ende menos resçiben daño dela calentura nin del frio que los omes . \textbf{ Bien en essa manera conuiene avn de departir entre los tienpos } ca en los tienpos frios & et minus porosam , \textbf{ quod patet ex carentia pilorum ; minus offenduntur a calore et frigore , quam viri . Sic etiam distinguendum est inter tempora : } nam temporibus frigidis , \\\hline
2.2.13 & ca en los tienpos frios \textbf{ quando vienta el çierço deuemos vsar de uestiduras calientes } mas que quando vienta el abrego . & nam temporibus frigidis , \textbf{ et flante borea utendum est aliis indumentis , } quam temporibus calidis flante austro . \\\hline
2.2.13 & mas que quando vienta el abrego . \textbf{ avn deuemos departir entre las hedades } ca la hedat dela vegez & quam temporibus calidis flante austro . \textbf{ Distinguendum est | etiam inter aetates : } quia senilis aetas , \\\hline
2.2.13 & e delas conplissiones \textbf{ e delas hedades las vestiduras son de departir } en quanto siruen al bien aprouechable . & complexionum , \textbf{ et aetatum , diuersificanda sunt , } ut deseruiunt ad bonum utile . \\\hline
2.2.13 & en quanto siruen al bien aprouechable . \textbf{ Mas en quanto siruen al bien de honrra es de catar la costunbre dela tierra } e la condiçion delas perssonas . & ut deseruiunt ad bonum utile . \textbf{ Sed ut deseruiunt ad bonum honorabile , attendenda est consuetudo patriae , } et conditio personarum . Sic igitur instruendi sunt iuuenes circa vestitum , \\\hline
2.2.13 & Et por ende \textbf{ assi son de enssennar los mançebos çerca las uestiduras } que non sean muy cuydado & Sed ut deseruiunt ad bonum honorabile , attendenda est consuetudo patriae , \textbf{ et conditio personarum . Sic igitur instruendi sunt iuuenes circa vestitum , } ut non nimis soliciti sint circa molliciem vestium , \\\hline
2.2.14 & por que assi vsen de las uestiduras conuenibles aprouecho del cuerpo . \textbf{ en abesta dezir } en qual manera los mançebos se de una auere & utantur debitis indumentis ad utilitatem corporis . \textbf{ Non sufficit scire , } qualiter iuuenes se habere debeant circa ludos , \\\hline
2.2.14 & en abesta dezir \textbf{ en qual manera los mançebos se de una auere } ca los trebeios & Non sufficit scire , \textbf{ qualiter iuuenes se habere debeant circa ludos , } gestus , \\\hline
2.2.14 & que son enlos mançebos \textbf{ delas quales se pueden tomar quatro razones } para prouar & quatuor videntur inesse ipsis iuuenibus , \textbf{ ex quibus quatuor rationes sumi possunt , } quod maxime in iuuenili aetate fugienda sit puerorum societas . Iuuenes enim primo sunt molles , \\\hline
2.2.14 & delas quales se pueden tomar quatro razones \textbf{ para prouar } que en la hedat dela mançebia deue ser mucho escusada la mala conpannia & ex quibus quatuor rationes sumi possunt , \textbf{ quod maxime in iuuenili aetate fugienda sit puerorum societas . Iuuenes enim primo sunt molles , } et ductiles . Secundo sunt passionum \\\hline
2.2.14 & que en la hedat dela mançebia deue ser mucho escusada la mala conpannia \textbf{ e deuen foyr della . } ¶ Ca los mançebos primeramente son muelles e tristornables & quod maxime in iuuenili aetate fugienda sit puerorum societas . Iuuenes enim primo sunt molles , \textbf{ et ductiles . Secundo sunt passionum } insecutores , \\\hline
2.2.14 & que assi es la primera razon \textbf{ para mostrar } que conuiene alos mançebos de foyr la mala conpannia se toma & Quarto sunt nimis creditivi . \textbf{ Prima ergo via ad ostendendum maxime competere iuuenibus , } fugere societatem prauam , sumitur ex eo quod iuuenes sunt nimis molles \\\hline
2.2.14 & para mostrar \textbf{ que conuiene alos mançebos de foyr la mala conpannia se toma } desto que los mançebos son muy muelles e muy tristor nabłs & Prima ergo via ad ostendendum maxime competere iuuenibus , \textbf{ fugere societatem prauam , sumitur ex eo quod iuuenes sunt nimis molles } et ductiles . \\\hline
2.2.14 & que en otra hedat . \textbf{ Et pues que assi es mucho son de castigar en aquella hedat } que non se alleguen a mala conpannia . & quam in alia . \textbf{ Maxime ergo tunc prohibendi sunt a societate praua . } Secunda via ad inuestigandum hoc idem , \\\hline
2.2.14 & ¶ La segunda razon \textbf{ para prouar esto mismo se tomadesto } que la hedat dela mançebia es inclinada a mal & Maxime ergo tunc prohibendi sunt a societate praua . \textbf{ Secunda via ad inuestigandum hoc idem , } sumitur ex eo quod iuuenilis aetas maxime est prona ad malum , \\\hline
2.2.14 & que la hedat dela mançebia es inclinada a mal \textbf{ e a segnir las passiones . } Ca de ligero cada vno se puede endozir & sumitur ex eo quod iuuenilis aetas maxime est prona ad malum , \textbf{ et insecutiua passionum . } Nam de facili quis inducitur ad illud , \\\hline
2.2.14 & e a segnir las passiones . \textbf{ Ca de ligero cada vno se puede endozir } a aquello que es inclinado . & et insecutiua passionum . \textbf{ Nam de facili quis inducitur ad illud , } ad quod est pronus . \\\hline
2.2.14 & assi commo fue dicho \textbf{ quando determinamos delas costunbres de los moços mayormente es de esquiuar en la hedat dela mançebia } de se llegara mala conpannia . & ut patuit cum determinauimus de moribus iuuenum ; \textbf{ maxime in iuuenili aetate cauendum est a societate praua . } Tertia via ad probandum hoc idem , \\\hline
2.2.14 & ¶ La terçera razon \textbf{ para prouar esto mismo se toma de aquello que los mançebos son mucho amadores de amistança . } Ca assi commo dize el philosofo en el segundo libro dela rectorica & Tertia via ad probandum hoc idem , \textbf{ sumitur ex eo quod iuuenes sunt nimis amatores amicitiae . } Nam ( ut dicitur secundo Rhetoricorum , \\\hline
2.2.14 & e asser amiga de los conpannones Et pues que assi es en la hedat dela mançebia \textbf{ mucho son de costrennir } e arredrar los mançebos dela mala conpannia & aetas illa gaudet conuiuere in societate , \textbf{ et amicari sodalibus . In iuuenili ergo aetate maxime cohibendi sunt iuuenes a societate praua : } quia tunc maxime se conformant moribus sociorum , \\\hline
2.2.14 & mucho son de costrennir \textbf{ e arredrar los mançebos dela mala conpannia } por que estonçe se enforman mayormente en las costunbres de los conpannones & ø \\\hline
2.2.14 & Et por ende ninguon non goza de beuir en conpanna \textbf{ si non tomare plazer } de se conformar alos conpanneros . & Nullus ergo gaudet in societate viuere , \textbf{ nisi gaudeat se sociis conformare . } Quarta via sumitur \\\hline
2.2.14 & si non tomare plazer \textbf{ de se conformar alos conpanneros . } ¶ La quartarazon se toma & Nullus ergo gaudet in societate viuere , \textbf{ nisi gaudeat se sociis conformare . } Quarta via sumitur \\\hline
2.2.14 & por que las delecta con nes̃ senssibles e desconuenibles \textbf{ commo quier que dessi sean malas e las deuamos foyr . } Enpero pueden ser bueans segunt la paresçençia & Delectationes enim sensibiles illicitae licet simpliciter sint prauae \textbf{ et fugiendae : } possunt tamen esse bonae \\\hline
2.2.14 & por que ciean \textbf{ que los bienes senssibłs son de segnir } mas que los otros & et de facili persuadetur eis , \textbf{ ut credant bona sensibilia esse sequenda . } Quia sermones particulares valde videntur esse proficui morali negocio , \\\hline
2.2.15 & Lo primero mostraremos \textbf{ qual cuydado auemos de tomar de los fijos } fasta los siete años & Primo enim declarabimus qualis cura habenda \textbf{ sit de filiis usque } ad septem annos . \\\hline
2.2.15 & Mas el pho tanne enł septimo libro delas politicas seys cosas \textbf{ que se deuen guardar çerca los moços enla primera hedat . } ¶ la primera es que deuen ser criados fasta los siete años de cosas humidas . & Tangit autem Philosophus 7 Polit’ sex circa ipsos pueros , \textbf{ quae seruanda sunt in aetate primitiua . Primum est , | quia } ad septimum debent pasci mollibus , \\\hline
2.2.15 & Enpero assy que en el comienço dela nasçençia mayormente sean cados con leche ¶ \textbf{ La segunda es que les deuen defender el vino¶ } La terçera es que los deuen acostunbrar alos frios & Secundum , \textbf{ quia sunt prohibendi a vino . Tertium , sunt assuescendi ad frigora . Quartum , } sunt assuescendi ad conuenientes \\\hline
2.2.15 & La segunda es que les deuen defender el vino¶ \textbf{ La terçera es que los deuen acostunbrar alos frios } ¶la quarta es que son de acostunbrar a mouimientos conuenibles e tenpdos . & Secundum , \textbf{ quia sunt prohibendi a vino . Tertium , sunt assuescendi ad frigora . Quartum , } sunt assuescendi ad conuenientes \\\hline
2.2.15 & La terçera es que los deuen acostunbrar alos frios \textbf{ ¶la quarta es que son de acostunbrar a mouimientos conuenibles e tenpdos . } Et esto es prouechoso en todas las hedades ¶ & quia sunt prohibendi a vino . Tertium , sunt assuescendi ad frigora . Quartum , \textbf{ sunt assuescendi ad conuenientes | et temperatos motus , } quod in omni aetate videtur esse proficuum . Quintum , \\\hline
2.2.15 & Et esto es prouechoso en todas las hedades ¶ \textbf{ La quinta es que lon de recrear por trebeios conuenibles . } Et deuen rezar ante ellos algunas bueans estorias . & quod in omni aetate videtur esse proficuum . Quintum , \textbf{ sunt recreandi per debitos ludos , } et sunt eis recitandae aliquae historiae , \\\hline
2.2.15 & La quinta es que lon de recrear por trebeios conuenibles . \textbf{ Et deuen rezar ante ellos algunas bueans estorias . } Et algunas bueans fablasen & sunt recreandi per debitos ludos , \textbf{ et sunt eis recitandae aliquae historiae , } et aliquae fabulae , \\\hline
2.2.15 & Et esto les es prouechoso mayormente \textbf{ quando comiençan a entender las significa connes delas palabras . } ¶ La sexta es & et hoc maxime , \textbf{ cum incipiunt percipere significationes verborum . Sextum , } a ploratu sunt cohibendi . \\\hline
2.2.15 & ¶ La sexta es \textbf{ que deuen ser guardados de llorar . } Et pues que assi es los moços & cum incipiunt percipere significationes verborum . Sextum , \textbf{ a ploratu sunt cohibendi . } Iuuenes ergo usque ad septennium alendi sunt mollibus ; \\\hline
2.2.15 & Et pues que assi es en aquella hedat tierna \textbf{ fasta que sean de siete años son de criar con cosas humidas e muelles } por que tales cosas commo estas ligeramenter se cuezen & quod nutrimentum lactis maxime videtur esse familiare corporibus puerorum . In illa ergo aetate tenera usque \textbf{ quo sunt circa septem annos , | alendi sunt mollibus , } et humidis : \\\hline
2.2.15 & e se conuierten en la sustançia e en el nudrimiento del cuerpo . \textbf{ Enpero deuemos guardar } que quando los mocos maman la leche & quia talia faciliter patiuntur , \textbf{ et faciliter conuertuntur in nutrimentum . Obseruandum est tamen in iuuenibus } cum aluntur lacte , \\\hline
2.2.15 & si contesçe \textbf{ que ayan de mamar otra leche } que la de su madre deuen catar ama & cum aluntur lacte , \textbf{ quod si contingat eos suggere aliud lac quam maternum , } quaerenda est foemina similis matri \\\hline
2.2.15 & que ayan de mamar otra leche \textbf{ que la de su madre deuen catar ama } que semeie ala madre & quod si contingat eos suggere aliud lac quam maternum , \textbf{ quaerenda est foemina similis matri } quantum ad complexionem , \\\hline
2.2.15 & si en el tienpo \textbf{ en que manian se acostunbraren a beuer vino } Et dize algs & secundum Philosophum propter aegritudines . \textbf{ De facili enim aegrotantur pueri et efficiuntur male dispositi in corpore , si tempore quo ut plurimum pascuntur lacte assuescant bibere vinum . } Immo dicunt aliqui , \\\hline
2.2.15 & que si en aquel tienpo se acostunbraren los mocos \textbf{ a beuer vino resçiben ende apareiamiento } para ser gafos . & Immo dicunt aliqui , \textbf{ quod si eo tempore ad vinum assuescant , } disponuntur ad lepram . \\\hline
2.2.15 & Onde el philosofo en el septimo libro delas politicas \textbf{ dizeque luego conuiene alos mocos pequanos de acostunbrar los alos frios } por que es aprouechable a dos cosas ¶ & ait , \textbf{ quod mox expedit pueris paruis consuescere ad frigora . Assuescere enim pueros ad frigora utile est ad duo . Primo ad sanitatem , } unde idem Philosophus ait , \\\hline
2.2.15 & assi commo son los alimanes \textbf{ e los de nuruega de vannar los sus fijos en los trios muy frios } por que los fagan muy fuertes & quod apud aliquas Barbaras nationes consuetudo est in fluminibus frigidis balneare filios , \textbf{ ut eos fortiores reddant . Attendendum est tamen , } quod cum dicimus pueros paruos assuescendos esse ad hoc vel ad illud , \\\hline
2.2.15 & por que los fagan muy fuertes \textbf{ Enpero deuemos entender } que quando dezimos & ut eos fortiores reddant . Attendendum est tamen , \textbf{ quod cum dicimus pueros paruos assuescendos esse ad hoc vel ad illud , } intelligendum est moderate \\\hline
2.2.15 & que quando dezimos \textbf{ que los moços pequanos son de acostunbrara esto o aquello deue se entender tenpradamente } e por grados & quod cum dicimus pueros paruos assuescendos esse ad hoc vel ad illud , \textbf{ intelligendum est moderate } et gradatim , \\\hline
2.2.15 & que si ouie remouimiento tenprado \textbf{ e se acostunbrar } e a trabaios del cuerpo tenprados los mienbros del su cuerpo seran mas firmes & quilibet enim in seipso experitur , \textbf{ quod si se moderate exercitet ad corporales labores , } membra corporis eius solidantur , \\\hline
2.2.15 & Et pues que assi es los moços \textbf{ por que han los mienbros muy tiernos deuen los acostunbrar a algers mouimientos pequanos et tenpdos } por que los mienbros dellos sean mas firmes . & et fiunt fortiora . Pueri ergo \textbf{ quia nimis habent tenera membra , | ad aliquos motus modicos } et temperatos sunt assuescendi , \\\hline
2.2.15 & que conuiene alos moços de faz quales quier mouimientos pequanos \textbf{ para soldar los mienbros } por que non los dexen caer & quod expedit in pueris facere motus \textbf{ quoscunque } et tantillos ad solidandum membra , \\\hline
2.2.15 & para soldar los mienbros \textbf{ por que non los dexen caer } por que son tiernos . & quod expedit in pueris facere motus \textbf{ quoscunque } et tantillos ad solidandum membra , \\\hline
2.2.15 & en tanto lo alaba el philosofo \textbf{ que diz que luego enł comienço de su nasçençia deuen fazer alguons instrumentos } en que se mueun a los moços & et ad non defluere propter teneritudinem : moderatum enim motum in pueris adeo laudat Philosophus , \textbf{ ut ab ipso primordio natiuitatis dicat , | fienda esse aliqua instrumenta , } in quibus pueri vertantur , \\\hline
2.2.15 & assi commo los biercos e los cartiellos pequanos \textbf{ con que se vezan los moços a andar . } ¶ Lo quinto deuen ser los mocos recreados e asolazados & in quibus pueri vertantur , \textbf{ et moueantur . Quinto , } recreandi sunt pueri per aliquos ludos , \\\hline
2.2.15 & e fazen se los cuerpos mas ligeros \textbf{ Otrossi avn deuen rezar alos moços alguas estorias } despues que comiençan at entender las significaçiones delas palabras . & et redduntur corpora agiliora . Sunt \textbf{ etiam pueris recitandae aliquae fabulae , | vel aliquae historiae , } postquam incipiunt percipere significationes verborum . \\\hline
2.2.15 & Otrossi avn deuen rezar alos moços alguas estorias \textbf{ despues que comiençan at entender las significaçiones delas palabras . } Et avn deuen les dezir algunos cantos & vel aliquae historiae , \textbf{ postquam incipiunt percipere significationes verborum . } Vel etiam aliqui cantus honesti sunt eis cantandi . \\\hline
2.2.15 & despues que comiençan at entender las significaçiones delas palabras . \textbf{ Et avn deuen les dezir algunos cantos } ca los cantos honestos son de cantar alos moços & postquam incipiunt percipere significationes verborum . \textbf{ Vel etiam aliqui cantus honesti sunt eis cantandi . } Nam ipsi nihil tristes sustinere possunt : ideo bonum est , eos assuescere ad aliquos moderatos ludos , \\\hline
2.2.15 & Et avn deuen les dezir algunos cantos \textbf{ ca los cantos honestos son de cantar alos moços } por que los moços non pueden sostir ninguna cosa triste . & postquam incipiunt percipere significationes verborum . \textbf{ Vel etiam aliqui cantus honesti sunt eis cantandi . } Nam ipsi nihil tristes sustinere possunt : ideo bonum est , eos assuescere ad aliquos moderatos ludos , \\\hline
2.2.15 & ca los cantos honestos son de cantar alos moços \textbf{ por que los moços non pueden sostir ninguna cosa triste . } Por ende es bien de los acostunbrara algs trebeios tenprados & Vel etiam aliqui cantus honesti sunt eis cantandi . \textbf{ Nam ipsi nihil tristes sustinere possunt : ideo bonum est , eos assuescere ad aliquos moderatos ludos , } et ad honestas aliquas et innocuas delectationes . \\\hline
2.2.15 & que retengan en ssi el spun e el eneldo . \textbf{ Ca assy commo quando los dexan llorar enbian el spuer e el eneldo . } assi quando les defienden & Nam \textbf{ sicut cum plorare permittuntur , | emittunt spiritum } et anhelitum : \\\hline
2.2.15 & por que los moços sean mas fuertes \textbf{ e mas rezios deuen les defender } que non lloren . ¶ & secundum Philosophum septimo Politicorum , facit ad robur corporis . \textbf{ Ut ergo pueri robustiores fiant , sunt a ploratu illo cohibendi . } Cum distinguimus aetates filiorum per septennia , \\\hline
2.2.16 & assi deuian ser enssennados . \textbf{ Et estos setenarios son de encortar o de al ougar } segunt el departimiento delas ꝑssonas . & ad decimumquartum sic esse instruendos , \textbf{ huiusmodi septennia sunt abbreuianda et elonganda } secundum diuersitatem personarum . \\\hline
2.2.16 & que otros en el seyto deçimo . \textbf{ Et por ende por que de tales cosas commo estas non podemos dar regla en punto . } Por ende son de dexar alguas cosas & quam alii in sedecim . Ideo \textbf{ quia de talibus punctualem regulam dare | non possumus , } aliqua relinquenda sunt iudicio paedagogi , \\\hline
2.2.16 & Et por ende por que de tales cosas commo estas non podemos dar regla en punto . \textbf{ Por ende son de dexar alguas cosas } al iuyzio del maestro e del ayo que deue enssennar los moços & non possumus , \textbf{ aliqua relinquenda sunt iudicio paedagogi , } qui debet pueros instruere , \\\hline
2.2.16 & Por ende son de dexar alguas cosas \textbf{ al iuyzio del maestro e del ayo que deue enssennar los moços } por que el pueda este tienpo tal acortar o alongar & aliqua relinquenda sunt iudicio paedagogi , \textbf{ qui debet pueros instruere , } ut possint huiusmodi tempus anticipare et prolongare , \\\hline
2.2.16 & al iuyzio del maestro e del ayo que deue enssennar los moços \textbf{ por que el pueda este tienpo tal acortar o alongar } segunt que a el paresçiere & qui debet pueros instruere , \textbf{ ut possint huiusmodi tempus anticipare et prolongare , } ut ei videbitur expedire . In hoc autem tempore , quod est a septimo usque ad \\\hline
2.2.16 & fasta el año xiiij̊ . \textbf{ son de penssar tres cosas enl gouernamiento de los fijos . } Ca el omne enł primero departimiento departesse enl alma e enel cuerpo . & ut ei videbitur expedire . In hoc autem tempore , quod est a septimo usque ad \textbf{ decimumquartum annum , tria sunt consideranda circa regimen filiorum . } Nam homo prima diuisione diuiditur in animam , \\\hline
2.2.16 & que es la uoluntad . \textbf{ Et por ende tres cosas deuemos cuydar en los fijos ¶ } La primera qual cuerpo han . ¶ La segunda qual uoluntad ¶ La terçera qual entendemiento . & habet intellectum , \textbf{ et appetitum . Tria ergo attendenda sunt in filiis . Primo , } quale habeant corpus . \\\hline
2.2.16 & por que los mocosayan el cuerpo bien dispuesto \textbf{ e bien ordenado son de vsar } por bsos e por mouimientos conuenibles . & qualem intellectum . \textbf{ Ut ergo iuuenes habeant corpus bene dispositum , } exercitandi sunt per debita exercitia , et per debitos motus . \\\hline
2.2.16 & Mas por que ayan la uoluntad bien dispuesta e bien ordenada \textbf{ deuen se enduzir a uertudes conuenibles } e a obras uirtuosas . & exercitandi sunt per debita exercitia , et per debitos motus . \textbf{ Ut habeant voluntatem bene ordinatam , inducendi sunt ad debitas virtutes , et ad virtutum opera . } Sed ut habeant intellectum perfectum , \\\hline
2.2.16 & Et porque sea escusada la ꝑeza en los moços \textbf{ deuen los acostunbrar enl comienço dela su nasçençia } a algunos mouimientos conuenibles . & ut vitetur inertia puerorum , \textbf{ assuescendi sunt pueri ad aliquos motus . } Sed cum impleuerunt septennium usque ad annum decimum quartum , \\\hline
2.2.16 & Mas quando ouieren conplido el vii̊ año fasta el xiiij̊ . \textbf{ deuen se acostun brar poco a poco } e de grado en grado amas altos trabaios & Sed cum impleuerunt septennium usque ad annum decimum quartum , \textbf{ debent gradatim assuescere ad ulteriores labores , } et ad fortiora exercitia . \\\hline
2.2.16 & por que tal hedat es muy tierna \textbf{ non deuen tomar obras de caualłia } nin otras obras fuertes . & eo \textbf{ quod nimis sit tenera , non sunt assumenda opera militaria nec opera ardua . } Unde Philosophus 8 Polit’ ait , \\\hline
2.2.16 & que fasta la hedat dela pubeçençia \textbf{ que quiere dezer } fasta la hedat de los xiuf ͤ años deuen vsar de los vsos et mouimientos mas ligeros & quod usque ad pubescentiam , \textbf{ idest usque } ad decimumquartum annum , leuiora quaedam exercitia sunt assumenda , \\\hline
2.2.16 & que quiere dezer \textbf{ fasta la hedat de los xiuf ͤ años deuen vsar de los vsos et mouimientos mas ligeros } por que non enbarguen la cresçençia delos mienbros ¶ & idest usque \textbf{ ad decimumquartum annum , leuiora quaedam exercitia sunt assumenda , } ne impediatur incrementum . Viso , \\\hline
2.2.16 & quanto al cuerpo del vi̊ año fasta xuiij . \textbf{ Et que deuen tomar del xiiij año adelante trebeios mas fuertes que enl primero setenario . } finca de demostrar & ut iuuenes sint bene dispositi quantum ad corpus , \textbf{ a septimo usque ad quartumdecimum annum debent fortiores labores assumere , | quam in primo septennio . } Restat videre , \\\hline
2.2.16 & Et que deuen tomar del xiiij año adelante trebeios mas fuertes que enl primero setenario . \textbf{ finca de demostrar } en qual manera conuiene alos moços & quam in primo septennio . \textbf{ Restat videre , } quomodo oporteat eos ordinari ad virtutes , \\\hline
2.2.16 & porque ayan bien dispuesta e bien ordenada la uoluntad e el entendimiento . \textbf{ Et pues que assi łes deuedes saber } que el philosofo enl vij̊ . & quomodo oporteat eos ordinari ad virtutes , \textbf{ ut habeant dispositam voluntatem . Sciendum ergo , } quod Philosophus 5 Polit’ ait , \\\hline
2.2.16 & que el philosofo enl vij̊ . \textbf{ libro delas politicas dizeque muy mala cosa es de non enssennar } e de non enduzir los mocos a uirtudes & quod Philosophus 5 Polit’ ait , \textbf{ quod pessimum est non instruere pueros ad virtutem , et ad obseruantiam legum utilium . Inquirit enim Philosophus 8 Polit’ } utrum prius curandum sit de pueris , \\\hline
2.2.16 & libro delas politicas dizeque muy mala cosa es de non enssennar \textbf{ e de non enduzir los mocos a uirtudes } e aguardar las leyes bueans e aprouechosas . & quod Philosophus 5 Polit’ ait , \textbf{ quod pessimum est non instruere pueros ad virtutem , et ad obseruantiam legum utilium . Inquirit enim Philosophus 8 Polit’ } utrum prius curandum sit de pueris , \\\hline
2.2.16 & e de non enduzir los mocos a uirtudes \textbf{ e aguardar las leyes bueans e aprouechosas . } Ca el philosofo enłvij̊ libͤ delas politicas & ø \\\hline
2.2.16 & Et pues que assi es en el segundo setenario \textbf{ por que los mocos comiençan a cobdiçiar e a dessear . } Enpero non han vso acabado de razon e de entendimiento . & In secundo autem septennio \textbf{ quia pueri iam incipiunt concupiscere , } non tamen habent perfectum rationis usum , potissime videtur esse curandum circa ipsos \\\hline
2.2.16 & por que el caerpo es primero por generaçion \textbf{ que el alma primero deuemos entender } e auer cuydado & quia generatione \textbf{ corpus est prius anima , } prius intendendum est quomodo habeamus moderatas concupiscentias \\\hline
2.2.16 & que el alma primero deuemos entender \textbf{ e auer cuydado } en qual manera ayamos las cobdiçiastenpradas & corpus est prius anima , \textbf{ prius intendendum est quomodo habeamus moderatas concupiscentias } et ordinatam voluntatem , \\\hline
2.2.16 & que ayamos cuydado \textbf{ en qual manera ayamos de alunbrar el entendimiento } Mas la manera & et ordinatam voluntatem , \textbf{ quam quomodo habeamus illuminatum intellectum . Modus autem , } quo moderandae sunt concupiscentiae iuuenum , \\\hline
2.2.16 & por la qual deuen ser tenpradas . \textbf{ las cobdiçias delos mançebos es en poner speçial cautela çerca aquellas cosas } en que ellos se acostunbraron comunalmente de fallesçer . & quo moderandae sunt concupiscentiae iuuenum , \textbf{ est | ut specialis cautela adhibeatur circa illa , } circa quae maximae consueuerunt deficere . \\\hline
2.2.16 & las cobdiçias delos mançebos es en poner speçial cautela çerca aquellas cosas \textbf{ en que ellos se acostunbraron comunalmente de fallesçer . } Et pues que assi es & ut specialis cautela adhibeatur circa illa , \textbf{ circa quae maximae consueuerunt deficere . } Si ergo iuuenes sunt insecutores passionum et concupiscentiarum , \\\hline
2.2.16 & que fazen fazen las mucho \textbf{ mas que deuen asi que quando aman am̃a much̃ . Etrͣndo comiençan de trebeiar trebeian much̃ . } Et assi que todas las cosas que fazen fazen la ssienpre con sobrepuiança . & et omnia faciunt valde , \textbf{ ita quod cum amant nimis amant , | cum incipiunt ludere nimis ludunt , } et in caeteris aliis semper excessum faciunt , \\\hline
2.2.16 & Et assi que todas las cosas que fazen fazen la ssienpre con sobrepuiança . \textbf{ Por ende deuemos poner cautela } por que non siguna sus cobdiçias & et in caeteris aliis semper excessum faciunt , \textbf{ adhibenda est cautela ne insequantur concupiscentias : } sed sint abstinentes \\\hline
2.2.16 & ¶ Vistas estas cosas \textbf{ finca de demostrar } en qual manera son de enformar los moços & nec omnia agant valde sed in suis actibus et sermonibus moderationem accipiant . \textbf{ His visis restat videre , } quomodo sunt bene disponendi quantum ad intellectum . \\\hline
2.2.16 & finca de demostrar \textbf{ en qual manera son de enformar los moços } quanto al entendimiento . & His visis restat videre , \textbf{ quomodo sunt bene disponendi quantum ad intellectum . } Nam cum dicimus \\\hline
2.2.16 & quanto al entendimiento . \textbf{ Ca quando dixiemos que enłsegundo setenario deuiemos auer cuydado } mas prinçipal del ordenamiento dela uoluntad & quomodo sunt bene disponendi quantum ad intellectum . \textbf{ Nam cum dicimus } quod in secundo septennio principalius curandum est de ordinatione appetitus , quam de perfectione intellectus : \\\hline
2.2.16 & Esto non se deue \textbf{ assi entender } que en ninguna manera non deuamos auer cuydado & quod in secundo septennio principalius curandum est de ordinatione appetitus , quam de perfectione intellectus : \textbf{ non sic intelligendum est quod nullo modo curandum sit , } quomodo habeant intellectum perfectum , \\\hline
2.2.16 & assi entender \textbf{ que en ninguna manera non deuamos auer cuydado } en qual manera ayan los moços el entendimiento acabado & quod in secundo septennio principalius curandum est de ordinatione appetitus , quam de perfectione intellectus : \textbf{ non sic intelligendum est quod nullo modo curandum sit , } quomodo habeant intellectum perfectum , \\\hline
2.2.16 & por ende \textbf{ mas prinçipalmente deuemos entender } e cuydar çerca la ordeuaçion dela uoluntad & et deficiunt ab usu rationis , \textbf{ ideo principalius est insistendum circa ordinationem appetitus ; } quasi enim per totum \\\hline
2.2.16 & mas prinçipalmente deuemos entender \textbf{ e cuydar çerca la ordeuaçion dela uoluntad } por que por la mayor parte en todo el segundo se tenatio & ideo principalius est insistendum circa ordinationem appetitus ; \textbf{ quasi enim per totum } secundum septennium pueri non addiscunt \\\hline
2.2.16 & si non palabras por que non son de tan grand entendimiento \textbf{ que puedan penssar en las cosas . } Por ende enł segundo setenario pueden ser enssennados los moços en la guamatica & nondum enim sunt tanti intellectus , \textbf{ ut de ipsis rebus considerare possint : } ideo in secundo septennio possunt \\\hline
2.2.16 & Et otrossi en la logica \textbf{ que es mas manera de saber que sçiençia . } Et en la musica pratica & et in logica , \textbf{ quae est magis modus sciendi quam sit scientia : } et in practica musicali , \\\hline
2.2.16 & Et por que en aquel tienpo los moços fallesçe de vso de razon \textbf{ non pueden saber las sçiençias acabada mente . } Enpero por que quando comiençan a auer vso de razon non seanda todo mal apareiados ala sçiençia & quae consistit in quadam modulantia vocum : quia enim illo tempore pueri deficiunt a rationis usu , \textbf{ perfecte scire non possunt . } Ne tamen cum incipiunt habere rationis usum , \\\hline
2.2.16 & non pueden saber las sçiençias acabada mente . \textbf{ Enpero por que quando comiençan a auer vso de razon non seanda todo mal apareiados ala sçiençia } deuen ser acostunbrados alas otras artes & perfecte scire non possunt . \textbf{ Ne tamen cum incipiunt habere rationis usum , | omnino sint indispositi ad scientiam , } assuescendi sunt ad alias artes , \\\hline
2.2.17 & ni emos dessuso \textbf{ que trs cosas deuemos entender c̃ca los fijos . } Conuiene a saber & de quibus fecimus mentionem . \textbf{ Dicebatur supra circa filios tria intendenda esse , } videlicet quomodo habeant bene dispositum corpus , \\\hline
2.2.17 & que trs cosas deuemos entender c̃ca los fijos . \textbf{ Conuiene a saber } en qual manera ayamos el cuerpo bien apareiado . & Dicebatur supra circa filios tria intendenda esse , \textbf{ videlicet quomodo habeant bene dispositum corpus , } bene ordinatum appetitum , \\\hline
2.2.17 & e de los sacͣmentos de la eglesia \textbf{ prinçipalmente deuemos trabaiar } en qual manera los moços ayan el cuerpo bien apareiado e bien ordenado . & et sacramentorum Ecclesiae , \textbf{ principaliter est insistendum quomodo habeant bene dispositum corpus . } Nam \\\hline
2.2.17 & nin a los entendimientos delas sciençias . \textbf{ Et si algua cosa pudiero aprender en aquelt p̃o } esto solo es los lenguages delos omes . & nec ad scientificas considerationes . \textbf{ Si enim aliquid illo tempore addiscere possunt , } hoc est idiomata vulgaria . Sed a septimo usque \\\hline
2.2.17 & fasta los xiiij̊ . años \textbf{ por que ya comiencan a auer algunas cobdiçias desordenadas . } Et en alguna manera comiençan a partiçipar vso de razon e de entendimiento & ad quartumdecimum \textbf{ quia iam incipiunt habere concupiscentias aliquas illicitas , } et aliquo modo \\\hline
2.2.17 & por que ya comiencan a auer algunas cobdiçias desordenadas . \textbf{ Et en alguna manera comiençan a partiçipar vso de razon e de entendimiento } commo quier que non acabada mente & quia iam incipiunt habere concupiscentias aliquas illicitas , \textbf{ et aliquo modo } ( licet imperfecte ) incipiunt participare rationis usum , \\\hline
2.2.17 & por ende en aqł tienpo \textbf{ non solamente auemos de auer cuydado } en qual manera los moços ayan el cuer pero acabado & ( licet imperfecte ) incipiunt participare rationis usum , \textbf{ ideo in illo tempore non solum curandum est quomodo habeant perfectum corpus , } sed etiam quomodo habeant ordinatum appetitum : \\\hline
2.2.17 & en qual manera los moços ayan el cuer pero acabado \textbf{ mas avn auemos de tomar cuydado } en qual manera ayan el appetito ordenado & ideo in illo tempore non solum curandum est quomodo habeant perfectum corpus , \textbf{ sed etiam quomodo habeant ordinatum appetitum : } usque enim ad quartumdecimum annum \\\hline
2.2.17 & segunt el philosofo en las politicas \textbf{ mas son de enduzir los moços a bien } por costunbramiento & ø \\\hline
2.2.17 & Mas despues del . xiiij̊ año \textbf{ por que estonçe comiençan a partiçipar de vso de razon e de entendimiento } mas acabadamente non tan solamente deuen auer cuydado los padres de los fijos & Sed a quartodecimo anno , \textbf{ quia tunc perfectius participare incipiunt rationis usum , } non solum curandum est quomodo habeant bene dispositum corpus , \\\hline
2.2.17 & en qual manera ayan el cuerpo e el appetito bien ordenado . \textbf{ Mas avn deuen auer cuydado } que sean sabios e bien apareiados e que ayan el entendimiento bien aluibrado & et bene ordinatum appetitum , \textbf{ sed etiam quod sint prudentes , } et quod habeant bene illuminatum intellectum : ex tunc enim addiscere possunt \\\hline
2.2.17 & que sean sabios e bien apareiados e que ayan el entendimiento bien aluibrado \textbf{ por que estonçe pueden bien aprender } non solamente la guamatica que paresçe & sed etiam quod sint prudentes , \textbf{ et quod habeant bene illuminatum intellectum : ex tunc enim addiscere possunt } non solum grammaticam \\\hline
2.2.17 & que es assi comm̃ sciençia dela signiłicaçion delas palabras o la logica \textbf{ que es assi commo vna manera de saber } o la pratica dela musica & vel dialecticam \textbf{ quae est quidam modus sciendi , } vel practicam musicae quae consistit in consonantia vocum : \\\hline
2.2.17 & Mas avn pueden ser enssennados en aquellas sçiençias \textbf{ alas quales nos deuemos acoger } para saber el entendimiento delas cosas & sed possunt instrui in illis scientiis , \textbf{ ad quas sciendas oportet recurrere ad intellectum rerum , } et per quarum cognitionem possumus fieri sapientes et prouidi . \\\hline
2.2.17 & alas quales nos deuemos acoger \textbf{ para saber el entendimiento delas cosas } por el cono sçimiento & sed possunt instrui in illis scientiis , \textbf{ ad quas sciendas oportet recurrere ad intellectum rerum , } et per quarum cognitionem possumus fieri sapientes et prouidi . \\\hline
2.2.17 & despues del resçebemientodel bautismo \textbf{ e de los sacͣmentos deuen entender prinçipalmente çerca vna cosa } assi commo çerca buena disposiconn del cuerpo . & et sacramentorum Ecclesiae , \textbf{ intendendum est principaliter | quasi circa unum , } ut circa bonam dispositionem corporis . \\\hline
2.2.17 & que es de los siete años fasta los . \textbf{ xiiij̊ deuen entender prinçipalmente çercados cosas } assi commo çerca la disposiconn del cuerpo & ut a duodecimo anno usque ad quartumdecimum , est \textbf{ quasi intendendum principaliter circa duo , ut circa dispositionem corporis , } et circa ordinationem appetitus . \\\hline
2.2.17 & que comiença del xiiii año dende adelante \textbf{ deuen entender çerca tres cosas } assi commo çerca la buean disposi conn del cuerpo . & Sed in tertio septennio , \textbf{ ut a quartodecimo anno et deinceps intendendum est circa tria , } ut circa bonam dispositionem corporis , \\\hline
2.2.17 & Ca todos los trabaios que tomaron en los xiiiij años passados deuen ser liuianos e ligeros . \textbf{ Et destonçe adelante deuen tomar trabaios mas fuertes } Et esto dize elpho enł viij̊ libro delas politicas & et quasi debiles : \textbf{ ex tunc autem assumendi sunt labores fortiores . } Nam et Philosophus 8 Poli’ \\\hline
2.2.17 & que fasta los . xiiij años los moços deuen ser acostunbrados a trabaios ligeros \textbf{ mas dende adelante se deuen acostunbrar a trabaios mas fuertes . } Et en tanto que segunt el philosofo desde los . xiiij̊ años & quod usque ad quartumdecimum annum pueri assuescendi sunt ad labores leues : \textbf{ sed deinde debent assumere labores fortes . Adeo enim } secundum ipsum a quartodecimo anno assuescendi sunt pueri ad labores fortes , \\\hline
2.2.17 & Et en tanto que segunt el philosofo desde los . xiiij̊ años \textbf{ se deuen acostunbrar los mocos a trabaios fuertes . } Assi commo al vso dela lucha & secundum ipsum a quartodecimo anno assuescendi sunt pueri ad labores fortes , \textbf{ ut ad exercitationem luctatiuam , } vel ad aliquam aliam exercitationem similem exercitationi bellicae ; \\\hline
2.2.17 & o a otro trabaio semeiable de trabaio de batallas . \textbf{ Et despues enł . xviij ̊anero estando enssennados los mocos en trabaio dela lucha e del caualgar } e en los trabaios & vel ad aliquam aliam exercitationem similem exercitationi bellicae ; \textbf{ ut postea in quartodecimo anno instructi in luctatiua } et in equitativa , \\\hline
2.2.17 & que pertenesçen ala caualłia \textbf{ estonçe se pue den poner alos trabaios dela caualłia } por que estonçe es alguno bien ordenado & et in equitativa , \textbf{ et in aliis quae ad militiam requiruntur , subire possint labores militares : tunc enim est quis bene dispositus } quantum ad corpus , \\\hline
2.2.17 & la qual cosa non puede ser sin fuerte trabaio de su cuerpo . \textbf{ ¶ Et pues que assi es commo todos aquellos que quieren beuir uida çiuil } conuiene les de sofrir alguas uegadas fuertes trabaios & Quod sine forti exercitatione corporis esse non potest . \textbf{ Cum ergo omnes volentes viuere vita politica , | oporteat } aliquando sustinere fortes labores pro defensione reipublicae : \\\hline
2.2.17 & ¶ Et pues que assi es commo todos aquellos que quieren beuir uida çiuil \textbf{ conuiene les de sofrir alguas uegadas fuertes trabaios } por defendemiento dela tierra . & oporteat \textbf{ aliquando sustinere fortes labores pro defensione reipublicae : } omnes volentes viuere tali vita a quartodecimo anno ultra sic debent assuefieri ad aliqua officia robusta , \\\hline
2.2.17 & Et por ende desde los xiiij̊ annos adelante \textbf{ deuen se acostunbrar a fuertes trabaios } assi que si viniere tienpo & aliquando sustinere fortes labores pro defensione reipublicae : \textbf{ omnes volentes viuere tali vita a quartodecimo anno ultra sic debent assuefieri ad aliqua officia robusta , } quod si aduenerit tempus \\\hline
2.2.17 & ayan el cuerpo o bien ordenado \textbf{ por que puedan tomar trabaios } e pueda defender la tierra . & et congruitas quod respublica defensione indigeat , habeant corpus sic dispositum , \textbf{ ut possint tales subire labores , } ut per eos respublica possit defendi . \\\hline
2.2.17 & por que puedan tomar trabaios \textbf{ e pueda defender la tierra . } Mas quales son de acostunbrar & ut possint tales subire labores , \textbf{ ut per eos respublica possit defendi . } Qui autem magis et qui minus sunt assuescendi ad tales labores , \\\hline
2.2.17 & e pueda defender la tierra . \textbf{ Mas quales son de acostunbrar } mas o menos atales trabaios adelante paresçra & ut per eos respublica possit defendi . \textbf{ Qui autem magis et qui minus sunt assuescendi ad tales labores , } in prosequendo patebit . \\\hline
2.2.17 & por que ayan el cuerpo bien ordenado \textbf{ por que puedan tomar trabaios conuenibles } la qual cosa mayormente se pue de fazer & ut habeant sic bene dispositum corpus , \textbf{ ut possint debitos subire labores , } quod maxime fieri contingit , \\\hline
2.2.17 & por que puedan tomar trabaios conuenibles \textbf{ la qual cosa mayormente se pue de fazer } si vsaren los fijos a ex̉çiçios & ut possint debitos subire labores , \textbf{ quod maxime fieri contingit , } si ad debita exercitia assuescant . Viso , \\\hline
2.2.17 & Visto en qual manera del . xiiij . \textbf{ año adelante deuen los padres auer cuydado de los fijos } por que ayan el cuerpo bien ordenado & si ad debita exercitia assuescant . Viso , \textbf{ quomodo a quartodecimo anno ultra solicitari debent patres erga filios , } ut habeant dispositum corpus . Restat videre , \\\hline
2.2.17 & por que ayan el cuerpo bien ordenado \textbf{ finca de ver } en qual manera de una auer cuydado dellos & quomodo a quartodecimo anno ultra solicitari debent patres erga filios , \textbf{ ut habeant dispositum corpus . Restat videre , } quomodo solicitari debeant circa eos , \\\hline
2.2.17 & quanto al orgullo e ala locama . \textbf{ paresce que estonçe comiençan a auer vso de razon acabada paresçe } les que son dignos de enssennorear & Primo quantum ad elationem , \textbf{ quia cum ex tunc incipiant habere perfectum rationis usum , | videtur eis } quod digni sint dominari , \\\hline
2.2.17 & paresce que estonçe comiençan a auer vso de razon acabada paresçe \textbf{ les que son dignos de enssennorear } e de ser senneres & videtur eis \textbf{ quod digni sint dominari , } et dedignantur aliis esse subiecti . Secundo delinquunt in prosequendo venerea , \\\hline
2.2.17 & por que estonçe comiençan \textbf{ mas cobdiçiosamente de segnir las obras de luxia Et } pues que assi es do paresçe mayor periglo & et dedignantur aliis esse subiecti . Secundo delinquunt in prosequendo venerea , \textbf{ quia tunc incipiunt ardentius circa venerea incitari . } Quia ergo semper cautela est adhibenda \\\hline
2.2.17 & pues que assi es do paresçe mayor periglo \textbf{ alli deuemos poner mayor cautella e mayor remedio . } por ende del xiiij & Quia ergo semper cautela est adhibenda \textbf{ ubi maius periculum imminet : } a quartodecimo anno ultra specialiter monendi sunt iuuenes , \\\hline
2.2.17 & por el bien dellos . \textbf{ Et cada vno deue obedesçer } a aquel & quia seniores \textbf{ et patres imperant iuuenibus propter eorum bonum : } quilibet autem obedire debet ei quem scit non percipere aliqua nisi ad bonum eius . \\\hline
2.2.17 & quando venieren ahedat acabada sean sennores . Et por ende ellos deuen mientra son enla hedat dela mançebia \textbf{ assi obedesçer alos padres e alos vieios } por que den alos otros exienplo & quia cum filii venerint ad aetatem perfectam futuri sunt dominari : \textbf{ debent ergo ipsi dum sunt in aetate sic iuuenili patribus et senioribus obedire , } ut dent eis exemplum , \\\hline
2.2.17 & La terçera razon es \textbf{ por que aquel que quiere aprender de ser señor } deue primero aprender de ser subiecto . & et obediant . Tertia ratio est : \textbf{ quia qui vult discere principari , } debet prius discere subiici . \\\hline
2.2.17 & por que aquel que quiere aprender de ser señor \textbf{ deue primero aprender de ser subiecto . } Onde el philosofo dize & quia qui vult discere principari , \textbf{ debet prius discere subiici . } Unde ait Philosophus , \\\hline
2.2.17 & Et pues que assi es \textbf{ por que los mançebos puedan enssenñorear non deuen del dennar de ser subiectos } e obedesçer alos uieios e alos padres & nisi prius extiterit bonus discipulus : \textbf{ ut ergo ipsi valeant bene principari , } indignari non debent subiici senioribus \\\hline
2.2.17 & por que los mançebos puedan enssenñorear non deuen del dennar de ser subiectos \textbf{ e obedesçer alos uieios e alos padres } Et por ende deuemos enderesçar el appetito & ut ergo ipsi valeant bene principari , \textbf{ indignari non debent subiici senioribus } et patribus . \\\hline
2.2.17 & e obedesçer alos uieios e alos padres \textbf{ Et por ende deuemos enderesçar el appetito } e el desseo de los mançebos & ut ergo ipsi valeant bene principari , \textbf{ indignari non debent subiici senioribus } et patribus . \\\hline
2.2.17 & de que fablamos \textbf{ por que quieran obedesçer alos uieios e alos padres } ¶lo segundo es de enderesçar el appetito de los mançebos & Est ergo rectificandus appetitus iuuenum existentium in aetate de qua loquimur , \textbf{ ut velint patribus et senioribus obedire . Secundo rectificandus est , } ne venerea illicita prosequantur . Iuuenes a decimoquarto anno ultra non solum inducendi sunt , \\\hline
2.2.17 & por que quieran obedesçer alos uieios e alos padres \textbf{ ¶lo segundo es de enderesçar el appetito de los mançebos } por que non siguna las cosas luxiosas & Est ergo rectificandus appetitus iuuenum existentium in aetate de qua loquimur , \textbf{ ut velint patribus et senioribus obedire . Secundo rectificandus est , } ne venerea illicita prosequantur . Iuuenes a decimoquarto anno ultra non solum inducendi sunt , \\\hline
2.2.17 & Ca los mançebos de los xiiij̊ anos adelante \textbf{ non solamente son de enduzir } por que sean guardados e mesurados & ut velint patribus et senioribus obedire . Secundo rectificandus est , \textbf{ ne venerea illicita prosequantur . Iuuenes a decimoquarto anno ultra non solum inducendi sunt , } ut sint abstinentes \\\hline
2.2.17 & por que sean guardados e mesurados \textbf{ quanto al comer e al beuermas } avn que sean continentes e castos quanto alas obras de luyia . & et sobrii \textbf{ quantum ad cibum et potum , } sed \\\hline
2.2.17 & e sean continentes e pagados de sus mugers propraas ¶ \textbf{ Mostrado en qual manera es de ordenar bien el cuerpo } e enderescar el appetito & vel usu propriae coniugis sint contenti . Ostenso , \textbf{ quomodo in iuuenibus a quartodecimo anno ultra est bene disponendum corpus , } et rectificandus appetitus : \\\hline
2.2.17 & Mostrado en qual manera es de ordenar bien el cuerpo \textbf{ e enderescar el appetito } en los mançebos de los xiiij̊ años & quomodo in iuuenibus a quartodecimo anno ultra est bene disponendum corpus , \textbf{ et rectificandus appetitus : } reliquum est \\\hline
2.2.17 & adelante \textbf{ fincanos de demostrar } en qual manera ha de ser alunbrado el su entendimiento . & reliquum est \textbf{ ut ostendatur , } quomodo recte illuminandus sit intellectus . \\\hline
2.2.17 & por las cosas ya dichͣs \textbf{ por que del xiiij ̊ año adelante comiença los mançebos de auer } mas acabadamente vso de razon . & Hoc autem satis esse potest per habita manifestum . \textbf{ Nam | quia a decimoquarto anno ultra incipiunt habere perfectum rationis usum , } ut dicebatur , \\\hline
2.2.17 & que es assi commo sçiençia de palabras o en logica \textbf{ que es assi commo vna manera de saber . } Et en la pratica dela musita & vel in dialectica quae est \textbf{ quasi quidam modus sciendi , } vel in practica musicae quae consistit in consonantia vocum : \\\hline
2.2.17 & Ca si quieren beuir uida politica \textbf{ e de çibdat e de caualłia deuen estudiar mayormente enlas sciençias morales } por que por ellas podran saber & Nam si volunt viuere vita politica et militari , \textbf{ potissime studere debent in moralibus scientiis : } quia per eos scire poterunt , \\\hline
2.2.17 & e de çibdat e de caualłia deuen estudiar mayormente enlas sciençias morales \textbf{ por que por ellas podran saber } en qual manera de una gouernar & potissime studere debent in moralibus scientiis : \textbf{ quia per eos scire poterunt , } quomodo se \\\hline
2.2.17 & por que por ellas podran saber \textbf{ en qual manera de una gouernar } assi e alos otros . & quia per eos scire poterunt , \textbf{ quomodo se } et alios debeant gubernare : \\\hline
2.2.17 & Et por ende en la hedat muy de moço \textbf{ que dura fasta los . xiiii̊ años tałs ̃ sçiençias altas non son de proponer } nin de enssennar a ellos & et insecutor passionum non est sufficiens auditor . In aetate ergo nimis puerili , quae durat usque \textbf{ ad decimumquartum annum , tales scientiae non sunt proponendae illis : } sed a decimoquarto anno ultra \\\hline
2.2.17 & que dura fasta los . xiiii̊ años tałs ̃ sçiençias altas non son de proponer \textbf{ nin de enssennar a ellos } Mas del . xiiiij̊ . año adelante & et insecutor passionum non est sufficiens auditor . In aetate ergo nimis puerili , quae durat usque \textbf{ ad decimumquartum annum , tales scientiae non sunt proponendae illis : } sed a decimoquarto anno ultra \\\hline
2.2.17 & para ser conplidos oydores delas sciençias morales \textbf{ por las quales sabran gouernar assi e alos otros¶ } Et pues que assi es & ut sunt sufficientes auditores moralium , \textbf{ per quae se } et alios gubernare cognoscant . Sic ergo regendi sunt iuuenes a decimoquarto anno ultra : \\\hline
2.2.17 & Et pues que assi es \textbf{ assi son de gouernar los mançebos de xiiij̊ . } años adelante & per quae se \textbf{ et alios gubernare cognoscant . Sic ergo regendi sunt iuuenes a decimoquarto anno ultra : } ut a decimoquarto anno usque ad vigesimumprimum , \\\hline
2.2.17 & Mas si fuere demandado \textbf{ en qual manera deste tienpo adelante son de gouernar los mançebos respondremos } e diremos & Si vero quaeratur , \textbf{ qualiter ab illo tempore ultra regendi sunt homines : } quia tunc quasi peruenerunt omnimode ad suam perfectionem , debent esse tales , ut sciant seipsos regere et gubernare . Ex tunc ergo non indigent paedagogo , \\\hline
2.2.17 & e deuen ser tałs \textbf{ que de una gouernar alłi mismos . } Et pues que assi es estonçe non han menester ayo nin maestro . & ø \\\hline
2.2.17 & que dixiemos en el primero libro \textbf{ que es del gouerna mi Mait de su milmo pueden tomar enssennamientos } en qual manera reglen assi mismos . & sed per ea quae diximus in primo libro qui est de regimine sui , \textbf{ possunt documenta accipere , } qualiter seipsos regant . \\\hline
2.2.18 & e de çibdat \textbf{ en alguna manera se deuen bsar en los trabaios corporałs } por que el vso cor por al si fuere tenprado a todos es aprouechable & Omnes iuuenes volentes viuere vita politica , \textbf{ aliquo modo exercitandi sunt ad corporales labores . } Corporalis enim exercitatio si moderata sit , \\\hline
2.2.18 & por que fazen en algua manera \textbf{ para auer salud } e alguna buean disposiçion del cuerpo & omnibus videtur esse proficua , \textbf{ eo quod faciat ad quandam sanitatem , } et ad quandam bonam dispositionem corporis : \\\hline
2.2.18 & nin alas delectaçiones spunales conuiene les \textbf{ para escusar la ꝑeza } e para escusar cuydado desconuenible & nec in spiritualibus delectationibus ; \textbf{ expedit eis } ut vitent inertiam , \\\hline
2.2.18 & para escusar la ꝑeza \textbf{ e para escusar cuydado desconuenible } de se dar & nec in spiritualibus delectationibus ; \textbf{ expedit eis } ut vitent inertiam , \\\hline
2.2.18 & e para escusar cuydado desconuenible \textbf{ de se dar } a algunos trabaios corporales & expedit eis \textbf{ ut vitent inertiam , } et ut vitent solicitudinem illicitam , exercitari aliquibus laboribus corporalibus licitis . Expedit enim volentibus politice viuere tam ciuibus quam nobilibus , \\\hline
2.2.18 & e alos prinçipes commo alos otros \textbf{ de non quedar del todo de los trabaios corporales } e de non quedar de se vsar de trabaiar enłbso delas armas & quam aliis , \textbf{ non omnino cessare a corporalibus actibus vel laboribus , } nec omnino inexercitatos esse circa armorum usum . \\\hline
2.2.18 & de non quedar del todo de los trabaios corporales \textbf{ e de non quedar de se vsar de trabaiar enłbso delas armas } por que el mouimiento conuenible del cuerpo faze el cuerpo mas fuerte & non omnino cessare a corporalibus actibus vel laboribus , \textbf{ nec omnino inexercitatos esse circa armorum usum . } Exercitatio enim corporalis debita reddit corpus robustius , \\\hline
2.2.18 & e mas rezio \textbf{ para que pueda sofrir mas ligeramente la dureza delas armas } por la qual cosa si el vso delas armas non solamente es conuenible en ssi . & Exercitatio enim corporalis debita reddit corpus robustius , \textbf{ ut facilius duriciem armorum sustinere possit . } Quare si usus armorum non solum aliquando est licitus , \\\hline
2.2.18 & Enpero a estos trabaiostales todos los mançebos \textbf{ non se deuen vsar egualmente . } Ca los fijos delos Reyes & non \textbf{ debet omnino ignotus esse . } Non tamen omnes iuuenes \\\hline
2.2.18 & tra lasios corporales \textbf{ merenos son doma e devlar } que los otros . & Non tamen omnes iuuenes \textbf{ ad huiusmodi labores sunt equaliter exercitandi : } nam filii Regum et Principum ad huiusmodi labores corporales minus sunt exercitandi quam alii , et adhuc primogeniti \\\hline
2.2.18 & Et avn el primero gento \textbf{ que deue regnar } conuiene de tomar menores trabaios & nam filii Regum et Principum ad huiusmodi labores corporales minus sunt exercitandi quam alii , et adhuc primogeniti \textbf{ qui regere debent decet minores labores assumere . } Nam secundum Philosophum 8 Politicorum labor corporalis , \\\hline
2.2.18 & que deue regnar \textbf{ conuiene de tomar menores trabaios } ca segunt el pho en łviij̊ libro delas politicas los trabaios corporales & nam filii Regum et Principum ad huiusmodi labores corporales minus sunt exercitandi quam alii , et adhuc primogeniti \textbf{ qui regere debent decet minores labores assumere . } Nam secundum Philosophum 8 Politicorum labor corporalis , \\\hline
2.2.18 & ca segunt el pho en łviij̊ libro delas politicas los trabaios corporales \textbf{ e el cuydar del entendimiento } et bargasse vno a otro . & Nam secundum Philosophum 8 Politicorum labor corporalis , \textbf{ et consideratio per intellectum , } se impedire videntur : \\\hline
2.2.18 & et bargasse vno a otro . \textbf{ Et la razon deste dicho se puede tomar } de aquellas cosas & se impedire videntur : \textbf{ ratio autem huiusmodi sumi potest ex iis , } quae habentur in 2 de anima , \\\hline
2.2.18 & e mas apareiados para sçiençia . \textbf{ Et pues que assi es para auer entendimiento claro } e sotil auemos menester blandura de la carne & ubi dicitur , \textbf{ quod molles carne aptos mente dicimus . Ad habendam igitur intellectualem industriam indigemus mollicie carnis ; } corporales igitur labores , \\\hline
2.2.18 & que el alma en seyendo e enfolgado se faze sabia \textbf{ ca por se assesegar } e por folgar se faze la carne muelle & et quiescendo fit prudens . \textbf{ Nam per sessionem et quietem redditur caro mollis , } per quam sumus apti ad speculandum ; \\\hline
2.2.18 & ca por se assesegar \textbf{ e por folgar se faze la carne muelle } por la qual somos despuestos & et quiescendo fit prudens . \textbf{ Nam per sessionem et quietem redditur caro mollis , } per quam sumus apti ad speculandum ; \\\hline
2.2.18 & por la qual somos despuestos \textbf{ para estudiar enlas sçiençias especulatiuas . } Mas por el trabaio & Nam per sessionem et quietem redditur caro mollis , \textbf{ per quam sumus apti ad speculandum ; } per laborem vero \\\hline
2.2.18 & por la qual cosa se enbarga la sotileza del entendimiento . \textbf{ Et aquellos que deuen gouernar el regno } mas les conuiene de ser sabios & per quam impeditur mentis sublimitas . \textbf{ Eos autem qui debent regere regnum , } magis expedit esse prudentes , \\\hline
2.2.18 & menos que vno otro omne . \textbf{ Enpero por la sabidia puede mas valer a todo el pueblo } que les acomne dado & vel aliquando minus quam unus homo : \textbf{ tamen per prudentiam praeualere potest toti populo sibi commisso . } Nam totus populus , \\\hline
2.2.18 & si non fuere bien ayuntada e bien ordenada es muy pequana mas ayuntada e orde nada \textbf{ por la sabidia del rey puede obrar grandes cosas . } Et pues que assy es commo quier que los Reyes e los prinçipeᷤ non de una de todo dexar el vso delas armas & modica possunt : \textbf{ unita vero | et ordinata per reges maxima operari valent . } Licet ergo Reges et Principes non omnino ignorare debeant armorum usum , \\\hline
2.2.18 & por la sabidia del rey puede obrar grandes cosas . \textbf{ Et pues que assy es commo quier que los Reyes e los prinçipeᷤ non de una de todo dexar el vso delas armas } nin de una assi escusar los trabaios del cuerpo & et ordinata per reges maxima operari valent . \textbf{ Licet ergo Reges et Principes non omnino ignorare debeant armorum usum , } nec sic debeant fugere corporales labores ; \\\hline
2.2.18 & Et pues que assy es commo quier que los Reyes e los prinçipeᷤ non de una de todo dexar el vso delas armas \textbf{ nin de una assi escusar los trabaios del cuerpo } por que le fagan mugeriles en tal manera & Licet ergo Reges et Principes non omnino ignorare debeant armorum usum , \textbf{ nec sic debeant fugere corporales labores ; } ut effecti quasi muliebres , \\\hline
2.2.18 & que nin por defension nin por defendimiento del regno \textbf{ nin por otra uentra a qual si quier non osen tomar armas . } Enpero por que mas conuiene de ser sabios & ut effecti quasi muliebres , \textbf{ nec pro defensione regni nec pro alio casu audeant arma assumere ; } attamen quia decet eos esse magis prudentes quam bellatores , \\\hline
2.2.18 & e mayormente alos primogenitos \textbf{ que deuen regnar } por ende menos son de acostunbrar alos trabaios del cuerpo & et maxime primogeniti \textbf{ qui regnare debent , } minus sunt assuescendi ad corporales labores \\\hline
2.2.18 & que deuen regnar \textbf{ por ende menos son de acostunbrar alos trabaios del cuerpo } que los otros & qui regnare debent , \textbf{ minus sunt assuescendi ad corporales labores } quam alii , \\\hline
2.2.18 & Et por ende estos tales \textbf{ mas deuen entender çerca la sabidia } que cerca la fortaleza del cuerpo & ne propter huiusmodi labores caro eorum indurata impediat subtilitatem mentis ; tales ergo plus debent vacare prudentiae , \textbf{ quam fortitudini corporali . Vacabunt autem prudentiae , } si diligenter insistant circa morales scientias , \\\hline
2.2.18 & que cerca la fortaleza del cuerpo \textbf{ e pueden se llegar ala sabiduria } si con grand acuçia estudiaren enlas sçiençias morales & quam fortitudini corporali . Vacabunt autem prudentiae , \textbf{ si diligenter insistant circa morales scientias , } ut possint mores hominum \\\hline
2.2.18 & si con grand acuçia estudiaren enlas sçiençias morales \textbf{ por que puedan conosçer las costunbres de los omes } et las obras dellos . & si diligenter insistant circa morales scientias , \textbf{ ut possint mores hominum | et agibilia cognoscere . } Decet ergo eos , \\\hline
2.2.18 & ¶ Et pues que assi es conuiene \textbf{ aquellos que deuen gouernar los otros } de escusar la ꝑeza & Decet ergo eos , \textbf{ qui debent alios regere , } vitare inertiam \\\hline
2.2.18 & aquellos que deuen gouernar los otros \textbf{ de escusar la ꝑeza } e el cuydado desconueinble & qui debent alios regere , \textbf{ vitare inertiam } et solicitudinem illicitam , \\\hline
2.2.18 & e sus herederos \textbf{ e todos gouernadores meior escusar la peza et el uagar } que non por el vso corporal & et a suis haeredibus magis est vitanda inertia , \textbf{ quam per laborem , } et exercitium corporalem . \\\hline
2.2.19 & non sola mente nazcanfuos \textbf{ mas pueden nasçer fijos e fijas spues } que dixiemos & Cum ex usu coniugii non solum oriantur filii et mares , \textbf{ sed oriri possunt filiae | et foeminae : } postquam diximus qualis cura gerenda est circa filios , \\\hline
2.2.19 & que dixiemos \textbf{ qual cuydado deuen auer los padres } cerca los fiios fincanos & et foeminae : \textbf{ postquam diximus qualis cura gerenda est circa filios , } restat dicere , \\\hline
2.2.19 & cerca los fiios fincanos \textbf{ de dez qual cuydado deuen auer los padres çerca las fijas } mas esto ha menester muy pequano tractado & restat dicere , \textbf{ qualis gerenda sit circa filias . } Sed hoc breui tractatu indiget : \\\hline
2.2.19 & ca quando dixiemos \textbf{ e determinamos del gouer namiento del casamiento } e mostramos en qual manera son de gouernar las muger s casadas & Sed hoc breui tractatu indiget : \textbf{ quia cum determinauimus de regimine coniugali , } et ostendimus qualiter regendae sunt foeminae ; \\\hline
2.2.19 & e determinamos del gouer namiento del casamiento \textbf{ e mostramos en qual manera son de gouernar las muger s casadas } alli diemos doctrina conplidamente & quia cum determinauimus de regimine coniugali , \textbf{ et ostendimus qualiter regendae sunt foeminae ; } quasi sufficienter tradidimus qualis cura gerenda sit circa filias . \\\hline
2.2.19 & alli diemos doctrina conplidamente \textbf{ qual cuydado deuen auer los padt̃s } e las madres çerca delas fijas & et ostendimus qualiter regendae sunt foeminae ; \textbf{ quasi sufficienter tradidimus qualis cura gerenda sit circa filias . } Nam sicut decet coniuges esse continentes , pudicas , abstinentes , et sobrias : sic decet \\\hline
2.2.19 & mas que sean ençerradas \textbf{ la qual cosa podemos prouar } por tres razones ¶ La primera se toma & et discursu , \textbf{ quod triplici via venari possumus . Prima sumitur , } ut tollatur filiabus commoditas malefaciendi . Secunda , \\\hline
2.2.19 & por tres razones ¶ La primera se toma \textbf{ por que sea tirado alas fijas manera de mal fazer . } ¶ La segunda por que non se fagan desuergonçadas ¶ la terçera & quod triplici via venari possumus . Prima sumitur , \textbf{ ut tollatur filiabus commoditas malefaciendi . Secunda , } ne fiant inuerecundae . Tertiae , \\\hline
2.2.19 & por que tan grand inclinaçion auemos alas delecta con nes senssibles \textbf{ en la mayor parte pecamos en tales cosas si nos fuere dada manera de pecar } Onde el philosofo en la rectorica dize & Quia ergo tantum habemus impetum ad delectationes sensibiles , \textbf{ ut plurimum deliquimus in talibus ; | si adsit nobis commoditas delinquendi , } unde et Philosophus in Rheto’ vult quod homines \\\hline
2.2.19 & que los omes en la mayor parte fazen mal quando pueden . \textbf{ Et pues que assi es muy grant cautela es de poter } para guardar la sinpieza & cum possunt . \textbf{ Maxima ergo cautela ad conseruandam puritatem et innocentiam , } est vitare commoditates malefaciendi , \\\hline
2.2.19 & Et pues que assi es muy grant cautela es de poter \textbf{ para guardar la sinpieza } e la inoçençia de non pecar & cum possunt . \textbf{ Maxima ergo cautela ad conseruandam puritatem et innocentiam , } est vitare commoditates malefaciendi , \\\hline
2.2.19 & para guardar la sinpieza \textbf{ e la inoçençia de non pecar } e para guardar las maneras de mal Razer & Maxima ergo cautela ad conseruandam puritatem et innocentiam , \textbf{ est vitare commoditates malefaciendi , } propter quod et prouerbialiter dicitur , \\\hline
2.2.19 & e la inoçençia de non pecar \textbf{ e para guardar las maneras de mal Razer } por la qual cosa se dize vn prouerbio & Maxima ergo cautela ad conseruandam puritatem et innocentiam , \textbf{ est vitare commoditates malefaciendi , } propter quod et prouerbialiter dicitur , \\\hline
2.2.19 & por la qual cosa se dize vn prouerbio \textbf{ que el azma de furtar faze el ladron } e por ende si en los omes & propter quod et prouerbialiter dicitur , \textbf{ Furandi commoditas facit furem . } Si ergo in viris , \\\hline
2.2.19 & en los quales es la razon e el entendimiento mayor es grant peligro \textbf{ de non escusar las azinas de los pecados much mas es esto de escusar en las mugers . } Et avn mas es en las fiias e en las moças & in quibus est ratio praestantior , \textbf{ est magnum periculum non vitare commoditates delictorum : multo magis hoc est in foeminis , } et adhuc magis est in filiabus vel in puellis : \\\hline
2.2.19 & e por ende \textbf{ por que non sea dada a ella sazina de mal fazer son de guardar conueniblemente } e deuenles defender & et adhuc magis est in filiabus vel in puellis : \textbf{ ne ergo eis detur commoditas malefaciendi , } sunt debite custodiendae , \\\hline
2.2.19 & por que non sea dada a ella sazina de mal fazer son de guardar conueniblemente \textbf{ e deuenles defender } que non anden por las placas & ne ergo eis detur commoditas malefaciendi , \textbf{ sunt debite custodiendae , } et prohibendae sunt a circuitu et discursu . \\\hline
2.2.19 & La segunda razon \textbf{ para mostrar esto mismo se toma } por que non se fagan desuergonçadas & et prohibendae sunt a circuitu et discursu . \textbf{ Secunda via ad inuestigandum hoc idem , sumitur , } ne fiant inuerecundae . \\\hline
2.2.19 & en acatamientode los uarones \textbf{ es de non las acostunbrar entre los uarones nin entre las gentes . } por la qual cosa las moças uagando e andando por la tierra acostunbran se auer los omes & Inter cetera ergo , \textbf{ quae reddunt puellas verecundas in aspectu virorum , est non assuescere eas inter gentes . Quare cum puellae circemeundo , } et vagando per patriam assuescant virorum aspectibus , \\\hline
2.2.19 & es de non las acostunbrar entre los uarones nin entre las gentes . \textbf{ por la qual cosa las moças uagando e andando por la tierra acostunbran se auer los omes } e fazen se familiares dellos & quae reddunt puellas verecundas in aspectu virorum , est non assuescere eas inter gentes . Quare cum puellae circemeundo , \textbf{ et vagando per patriam assuescant virorum aspectibus , } fiunt familiares eis , \\\hline
2.2.19 & por la conpannia de los uarones \textbf{ mas toller alas moças } la uerguença es toller el freno dellas & et tollitur ab ipsis verecundia \textbf{ ex virorum consortio . Tollere autem a puellis verecundiam , } est tollere ab eis fraenum , \\\hline
2.2.19 & mas toller alas moças \textbf{ la uerguença es toller el freno dellas } por el qual freno se retrahen & ex virorum consortio . Tollere autem a puellis verecundiam , \textbf{ est tollere ab eis fraenum , } quo trahuntur , \\\hline
2.2.19 & e mayormente delas moças es la uerguença \textbf{ por que non puedan sallir a fazer cosas torpes . } Et pues que assi es cosa conuenible es de defender es alas mocas & et potissime puellarum , \textbf{ ne prorumpant in turpia , } videtur esse verecundia . Decens ergo est cohibere puellas \\\hline
2.2.19 & por que non puedan sallir a fazer cosas torpes . \textbf{ Et pues que assi es cosa conuenible es de defender es alas mocas } que non corran & ne prorumpant in turpia , \textbf{ videtur esse verecundia . Decens ergo est cohibere puellas } a discursu \\\hline
2.2.19 & si en manera conuenible fueren tenidas en guarda \textbf{ e non fueren dexadas andar a su uoluntad a quande } e assende non lo lamente le tazen uergontiosas & si debito modo sub custodia teneantur , \textbf{ et non permittantur discurrere et circuire , } non solum efficiuntur verecundae , \\\hline
2.2.19 & el qual es muy bueno \textbf{ para guardar la castidat delas moças } ca ueemos que las aian lias montes & quandam syluestreitatem , \textbf{ quae optima est ad saluandam pudicitiam puellarum . } Videmus enim quod animalia \\\hline
2.2.19 & ca ueemos que las aian lias montes \textbf{ mas si se acostunbran a vsar con los omes } fazen se mansas en tal manera & Videmus enim quod animalia \textbf{ etiam valde syluestria | si assuescant conuersationibus hominum , } domesticantur , \\\hline
2.2.19 & fazen se mansas en tal manera \textbf{ que se dexan tanner e palpar } mas si fueren alongadas de vsar con los omes & domesticantur , \textbf{ et permittunt se tangi | et palpari : } si vero a conuersationibus hominum sint remotae , \\\hline
2.2.19 & que se dexan tanner e palpar \textbf{ mas si fueren alongadas de vsar con los omes } assi conmo cosas montes & et palpari : \textbf{ si vero a conuersationibus hominum sint remotae , } quasi syluestria tactum \\\hline
2.2.19 & e non se acostunbraren \textbf{ de se aconpannar con los omes } o de beuir conellos & Quod ergo in animalibus aliis aspicimus , reperimus \textbf{ etiam in foeminis . } Si igitur foeminae non discurrant , \\\hline
2.2.19 & fazen se assi commo montes \textbf{ mas delar la conpannanan de los omes } e mas desdennan la loca ma conpannia dellos . & Si igitur foeminae non discurrant , \textbf{ et a conuersatione virorum sint inconsuetae , quasi syluestres ab ipsorum societate difficilius ad lasciuiam } et ad impudicitiam inclinantur . Uniuersaliter igitur omnes ciues , \\\hline
2.2.19 & e mucho mas alos nobles \textbf{ e mayoͬmente alos Reyes e alos prinçipes de auer tanto mayor cuydado çerca las sus fiias propraas } que conueniblemente anden & et potissime Reges \textbf{ et Principes | tanto maiorem curam circa proprias filias adhibere debent , } ne indebite circuant \\\hline
2.2.19 & e dela desuergonança delas sus fijas \textbf{ puede contesçer mayor mal e mayor periglo . } ssi commo es dich de suso el philosofo enl primero libro de la rectorica & quanto ex impudicitia \textbf{ et lasciuia suarum filiarum potest maius malum vel periculum imminere . } Ut superius dicebatur , Philosophus 1 Rheto’ commendat in foeminis amorem operositatis . \\\hline
2.2.20 & para ser fazenderas e obraderas \textbf{ mas podemos prouar } por tres razones & Ut superius dicebatur , Philosophus 1 Rheto’ commendat in foeminis amorem operositatis . \textbf{ Possumus autem triplici via venari , } quod decet omnes ciues , \\\hline
2.2.20 & assi ser acuçiosos del gouernamientode las fijas \textbf{ por que non las dexen beuir oçiosamente } e en uagar & et maxime Reges \textbf{ et Principes sic solicitari erga regimen filiarum } ut nolint ociosae viuere , \\\hline
2.2.20 & e en uagar \textbf{ mas que las fagan vsar en alguas obras conuenibles e honestas ¶ } La primera razon se toma del so las honesto & ut nolint ociosae viuere , \textbf{ sed ament se exercitare circa opera aliqua licita et honesta . } Prima via sumitur ex honesto solatio , \\\hline
2.2.20 & segunt el philosofo enl Rlibro delas ethicas \textbf{ la uida de los omes non puede durar sin alguna delectaçion } e por ende son reprehendidos & Nam \textbf{ secundum Philosophum 10 Ethicorum absque omni delectatione vita nostra durare non potest : } ideo reprehenduntur dicentes , \\\hline
2.2.20 & los que dizen \textbf{ que deuemos escusar toda aai las delecta connes } por la qual cosa & ideo reprehenduntur dicentes , \textbf{ omnem delectationem fugiendam esse . } Quare si non possumus commode durare in vita , \\\hline
2.2.20 & si non nos delectaremos en alguas cosas \textbf{ conuiene de tomar alguas obras conuenibles e honestas çerca } las quales entendamos e estudiemos & nisi in aliquibus delectemur : \textbf{ decet nos assumere aliqua opera licita et honesta , } circa quae vacantes , \\\hline
2.2.20 & por que non biuna ociosamente \textbf{ e en vagar } e por que se delecten en las obras conuenibles & ne ociose viuant , \textbf{ et ut delectentur in operibus licitis , } insistere debent circa aliqua opera ciuilia , \\\hline
2.2.20 & e por que se delecten en las obras conuenibles \textbf{ deuen estudiar çerca algunas obras dela çibdat } o çerra aquellas cosas & et ut delectentur in operibus licitis , \textbf{ insistere debent circa aliqua opera ciuilia , } vel circa ea quae spectant ad gubernationem regni , \\\hline
2.2.20 & en essa misma manera avn las mugers \textbf{ por que non buian en vagar deuen amar } de ser azendosassienpre en algua cosa conueinble & vel circa regimen domus , vel circa aliqua alia exercitia licita . Sic et mulieres , \textbf{ ne ociose viuant , debent operositatem amare : et decet eas exercitati circa aliqua opera licita et honesta . } Quia igitur omnes delectantur in propriis operibus , \\\hline
2.2.20 & de ser azendosassienpre en algua cosa conueinble \textbf{ e conuiene les de vsar en algunas obras } que sean conueinbles e honestas . Et pues que assi es & ne ociose viuant , debent operositatem amare : et decet eas exercitati circa aliqua opera licita et honesta . \textbf{ Quia igitur omnes delectantur in propriis operibus , } et omnes diligunt sua opera , \\\hline
2.2.20 & assi commo dize el philosofo eñł quarto libro delas ethicas \textbf{ por que las muger sspue dan resçebir recreaçion en alguas delecta connes } conuenbłs & ut vult Philosophus 9 Ethicorum , \textbf{ ut mulieres ipse recreari possint aliquibus delectationibus licitis , } decet eas intentas esse circa aliqua opera licita \\\hline
2.2.20 & La segunda razon \textbf{ para prouar esto mismo se toma por arredrar dessi cuydado desconueinble } ca por que las mugers & circa quae mulieres insudare decet , \textbf{ in prosequendo patebit . Secunda via ad inuestigandum hoc idem , } sumitur ex vitatione solicitudinis illicitae . \\\hline
2.2.20 & por la mayor parte deuen estar sienpre en casa \textbf{ e non se deuen entremeter de vagar nin cuydar çerca } quales si quier obras & ut plurimum domi stant , \textbf{ et non vacant ciuilibus operibus , } nec regiminibus reipublicae ; \\\hline
2.2.20 & e esto tanto \textbf{ mas es de escusar en las mugers } que en los omes & statim cum quis non dat se licitis exercitiis , vagatur eius mens circa illicitas occupationes , \textbf{ quod tanto magis cauendum est in mulieribus quam in viris , } quanto molliores sunt illis , \\\hline
2.2.20 & mas ligeramente se inclinan \textbf{ para conplir las } por las obras & et quanto cogitando illicita facilius trahuntur , \textbf{ ut ea } ( \\\hline
2.2.20 & La terçera razon se toma del fructo e del prouech \textbf{ que dende se le unataca nunca las muger sse pueden dar a obras conueibles } si dende non se seunata algun bien dela obra de fuera & si adsit commoditas ) opere compleant . Tertia via sumitur ex fructu et utilitate quae inde consurgit . \textbf{ Nam nunquam mulieres possunt se dare licitis exercitiis , } nisi inde consurgat aliquod bonum exterioris operis \\\hline
2.2.20 & o alguna buean disposiconn dela uoluntad de dentro . \textbf{ Et pues que assi es deuen auer los padres grant acuçia } e auer grant cuydado & vel aliqua bona dispositio interioris mentis . \textbf{ Debet ergo diligentia } et cura adhiberi , \\\hline
2.2.20 & Et pues que assi es deuen auer los padres grant acuçia \textbf{ e auer grant cuydado } por que las muger ssean bueans e uirtuosas por la qual cosa el philosofo enł primero libro de la rectorica denuesta los gniegos & Debet ergo diligentia \textbf{ et cura adhiberi , } ut mulieres sint bonae et virtuosae , \\\hline
2.2.20 & en qual manera las sus muger ssean uirtuosas e bueans . \textbf{ Pues que assi es deuemos auer grand cuydado } por que las mugers non biuna en uagar & quomodo earum foeminae essent virtuosae \textbf{ et bonae . Adhibenda est ergo solicitudo , } ne foeminae ociose viuant , \\\hline
2.2.20 & Mas si alguno demandare \textbf{ de que se deuen trabaiar las mugers conuiene de fablar en tales cosas departidamente } segunt el departimiento delas perssonas cateyer & Si autem quaeratur circa qualia opera solicitari debent : \textbf{ oportet in talibus differenter loqui } secundum diuersitatem personarum . \\\hline
2.2.20 & de que se deuen trabaiar las mugers conuiene de fablar en tales cosas departidamente \textbf{ segunt el departimiento delas perssonas cateyer } e filar e obrar e coser & oportet in talibus differenter loqui \textbf{ secundum diuersitatem personarum . } Texere enim et filare , \\\hline
2.2.20 & segunt el departimiento delas perssonas cateyer \textbf{ e filar e obrar e coser } e taiar algunas cosas sotiles & secundum diuersitatem personarum . \textbf{ Texere enim et filare , } et operari sericum , \\\hline
2.2.20 & e filar e obrar e coser \textbf{ e taiar algunas cosas sotiles } asaz paresçen obras & Texere enim et filare , \textbf{ et operari sericum , } satis videntur opera competentia foeminis . \\\hline
2.2.20 & que se trabaiasse en tales obras \textbf{ commo estas avn non la deurien dexar } que beuiesse baldia & ut se circa talia exercitaret : \textbf{ adhuc non esset dimittendum , } ut non viueret ociosa , \\\hline
2.2.20 & e en uagar \textbf{ mas deuen la poner a estudio de letris } e a prinder leer e rezar & ut non viueret ociosa , \textbf{ sed tradenda esset studio literarum , } ut ad amorem literarum affecta , non vacaret ociose , \\\hline
2.2.20 & mas deuen la poner a estudio de letris \textbf{ e a prinder leer e rezar } por que inclinada al amor delas letris non estudiesse baldia & sed tradenda esset studio literarum , \textbf{ ut ad amorem literarum affecta , non vacaret ociose , } sed saepe saepius librum arripiens , se lectionibus occuparet . \\\hline
2.2.20 & nin oçiosa mas muchͣs e muchos uegadas tomando el libro s \textbf{ e trabaiasse en rezar sus leçiones o sus salmos o sus oronnes . } Et por esto & ut ad amorem literarum affecta , non vacaret ociose , \textbf{ sed saepe saepius librum arripiens , se lectionibus occuparet . } Ex hoc autem declaratur \\\hline
2.2.20 & y dixiemos \textbf{ que adelante serie de declarar cerca quales obras conuenia } que las mugers fuesen acuçiosas . & infra declarandum esse , \textbf{ circa quae opera deceat foeminas esse intentas . } Ostenso , \\\hline
2.2.21 & que las mugers fuesen acuçiosas . \textbf{ ostrado que non conuiene alas moças de andar uagarosas a quande } e allende & circa quae opera deceat foeminas esse intentas . \textbf{ Ostenso , | quod non decet puellas esse vagabundas , } nec decet eas viuere otiose : \\\hline
2.2.21 & que deuen ser callantias \textbf{ e non parleras la qual cosa podemos mostrar } por tres razons . & ø \\\hline
2.2.21 & que non ha \textbf{ e dessea de auer } assi commo dize el philosofo en el ij̊ libro de la rectorica & ne sint pronae ad iurgia et ad lites . Prima via sic patet \textbf{ nam cum desiderium sit eius quod abest , } ut vult Philosophus 2 Retor’ quanto aliquid quod est possibile haberi , \\\hline
2.2.21 & assi commo dize el philosofo en el ij̊ libro de la rectorica \textbf{ quanto alguna cosa que se puede auer } mas patesçegue e alta de alcançar & nam cum desiderium sit eius quod abest , \textbf{ ut vult Philosophus 2 Retor’ quanto aliquid quod est possibile haberi , } magis videtur arduum et inacessibile ; tanto magis videtur abesse \\\hline
2.2.21 & quanto alguna cosa que se puede auer \textbf{ mas patesçegue e alta de alcançar } tanto mas parełçe & ø \\\hline
2.2.21 & tanto mas parełçe \textbf{ que se non puede auer } e por ende mueue la cobdiçia & ut vult Philosophus 2 Retor’ quanto aliquid quod est possibile haberi , \textbf{ magis videtur arduum et inacessibile ; tanto magis videtur abesse } et magis concupiscentia mouet ; \\\hline
2.2.21 & en el primero libro delas poliricas \textbf{ dizeque el honrramiento delas mugers es callar } por que si las mugers fueren callantias en manera conueinble & unde et Philosophus primo Politicorum ait , \textbf{ quod ornamentum mulierum est silentium . } Si enim mulieres sint modo debito taciturnae , \\\hline
2.2.21 & por ende paresçe \textbf{ que los omes non pue den auer su conpanma } nin pueden alcançar & quia se sic non familiares exhibent , \textbf{ eorum consortium videtur magis abesse , } et videntur ipsae magis inaccessibiles : \\\hline
2.2.21 & que los omes non pue den auer su conpanma \textbf{ nin pueden alcançar } lo que quieren con ellas & eorum consortium videtur magis abesse , \textbf{ et videntur ipsae magis inaccessibiles : } propter quod non sic vilipenduntur \\\hline
2.2.21 & que assi es luego en la hedat dela moçedat \textbf{ deuen ser enssennadas las moças a callar } e non aparlar & et ornatae . \textbf{ Ab ipsa ergo puerili aetate instituendae sunt puellae } ad taciturnitatem : \\\hline
2.2.21 & deuen ser enssennadas las moças a callar \textbf{ e non aparlar } ca si contesçiere despues que sean ayuntadas a sus maridos & Ab ipsa ergo puerili aetate instituendae sunt puellae \textbf{ ad taciturnitatem : } quia si contingat eas postmodum per connubium suis viris copulari , \\\hline
2.2.21 & ¶ La segunda razon \textbf{ para prouar esto mismo se toma } por que non fablen commo non deuen & si sint debite taciturnae , feruentius diligentur . \textbf{ Secunda via ad inuestigandum hoc idem , } sumitur \\\hline
2.2.21 & por que dicho es de ssuso \textbf{ que sienpre deuemos poner } y mayor cautelao se acostunbro & Dicebatur enim supra , \textbf{ ibi semper maiorem cautelam adhibendam esse } ubi consueuit maior defectus consurgere . \\\hline
2.2.21 & y mayor cautelao se acostunbro \textbf{ de se leunatar mayor fallesçimiento o mayor pecado . } Et pues que assi es commo & ibi semper maiorem cautelam adhibendam esse \textbf{ ubi consueuit maior defectus consurgere . } Cum ergo ex hoc quis loquatur prudenter , et caute , \\\hline
2.2.21 & por ende cerca las mugers \textbf{ e mayormente çera las moças deuemos auer grant cuydado } por que non fablen sin sabiduria . & circa foeminas , \textbf{ et maxime circa puellas sollicitandum est , } ne incaute loquantur . \\\hline
2.2.21 & Mas entre todas las cautelas por que algunan non se entremeta \textbf{ para fablar la cosa } que non ha penssada esta paresçe muy grande & ne quis prorumpat in locutionem incautam , \textbf{ haec videtur esse potissima , | ut nullum sermonem proferat , } nisi prius ipsum diligenter examinet . \\\hline
2.2.21 & La terçera razon \textbf{ para prouar esto se tomadesto } que las mugrͣ̃s non sean prestas avaraias e apeleas & sed oportet ipsas esse debite taciturnitas , ut possint omnem sermonem prolatum diligenter excutere . \textbf{ Tertia via ad inuestigandum hoc idem , } sumitur , \\\hline
2.2.21 & e si non examinaten con grand cordura las palabras \textbf{ que han de dezer } assi commo por fallesçimento de razon & nisi sint modo debito taciturnae , \textbf{ et nisi sermones dicendos diligenter examinent : } sicut propter rationis defectum de facili loqui possunt pertinentia ad simplicitatem , et imprudentiam , \\\hline
2.2.21 & e de entendimiento \textbf{ de ligero puede fablar cosas } que pertenesçen a sinplicidat e a neçedat e palabras sin sabiduria & et nisi sermones dicendos diligenter examinent : \textbf{ sicut propter rationis defectum de facili loqui possunt pertinentia ad simplicitatem , et imprudentiam , } sic de facili loqui possunt pertinentes ad lites , \\\hline
2.2.21 & que pertenesçen a sinplicidat e a neçedat e palabras sin sabiduria \textbf{ assi de ligero pue den fablar cosas } que pertenesçen a peleas e abaraias & sicut propter rationis defectum de facili loqui possunt pertinentia ad simplicitatem , et imprudentiam , \textbf{ sic de facili loqui possunt pertinentes ad lites , } et ad iurgia . Decet ergo ipsas per debitam taciturnitatem adeo examinare dicenda , \\\hline
2.2.21 & e por ende les conuiene aellas de ser callanţias en manera conuenible \textbf{ e en tanto examinar las cosas que han de dezer } por que non digan alguas cosas & sic de facili loqui possunt pertinentes ad lites , \textbf{ et ad iurgia . Decet ergo ipsas per debitam taciturnitatem adeo examinare dicenda , } ut nec dicant aliqua , per quae iudicentur imprudentes ; nec dicant aliqua , \\\hline
2.2.21 & nin digan alguas cosas \textbf{ que puedan turbar alos } que las oyen & ut nec dicant aliqua , per quae iudicentur imprudentes ; nec dicant aliqua , \textbf{ quae possint turbare audientes , } propter quod iudicentur litigiosae , \\\hline
2.2.21 & por la qual razon les conuiene a ellas de ser callantias \textbf{ e que non se entremetan de fablar palabras de contienda } ca mayormente las mugers se deuen guardar de palabras de baraia e de pelea & quare decet ipsas esse taciturnas , \textbf{ ne in verba litigiosa prorumpant . } A verbis autem litigiosis potissime foeminae sibi cauere debent ; \\\hline
2.2.21 & e que non se entremetan de fablar palabras de contienda \textbf{ ca mayormente las mugers se deuen guardar de palabras de baraia e de pelea } por que despues que comiençan a pelear non saben partirse dela contienda & ne in verba litigiosa prorumpant . \textbf{ A verbis autem litigiosis potissime foeminae sibi cauere debent ; } quia postquam litigare incipiunt nesciunt \\\hline
2.2.21 & ca mayormente las mugers se deuen guardar de palabras de baraia e de pelea \textbf{ por que despues que comiençan a pelear non saben partirse dela contienda } por que fallesçe en ellas vso de razon e de entendimiento & A verbis autem litigiosis potissime foeminae sibi cauere debent ; \textbf{ quia postquam litigare incipiunt nesciunt | a litigio se abstinere : } deficit enim in eis rationis usus , \\\hline
2.2.21 & e por ende despues que son mouidas las mugers \textbf{ e comiençan a pelear } acresçientas se en ellas la cobdiçia delas baraias & postquam motae sunt , \textbf{ et litigare coeperunt , augetur in eis concupiscentia litium , } quam per rationem \\\hline
2.2.21 & acresçientas se en ellas la cobdiçia delas baraias \textbf{ la qual cobdiçia de ligero non pueden refrenar } por que fallesçe de vso de razon e de entendimiento & et litigare coeperunt , augetur in eis concupiscentia litium , \textbf{ quam per rationem | de facili refraenare non possunt , } eo \\\hline
2.3.1 & Ca es mostrai ser \textbf{ do en qual manera conuiene a los maridos de gouernar a sus mugers . } Et en qual maneta los padres deuen regir e gouernar a sus fijos . & quia ostensum est , \textbf{ qualiter decet uiros suas coniuges regere , } et qualiter patres suos filios gubernare . \\\hline
2.3.1 & do en qual manera conuiene a los maridos de gouernar a sus mugers . \textbf{ Et en qual maneta los padres deuen regir e gouernar a sus fijos . } finca de dezer dela terçera ꝑte & qualiter decet uiros suas coniuges regere , \textbf{ et qualiter patres suos filios gubernare . } Restat exequi de parte tertia , \\\hline
2.3.1 & Et en qual maneta los padres deuen regir e gouernar a sus fijos . \textbf{ finca de dezer dela terçera ꝑte } en que se tractara del gouernamiento de los siruientes & et qualiter patres suos filios gubernare . \textbf{ Restat exequi de parte tertia , } in qua agetur de regimine ministrorum , \\\hline
2.3.1 & enlo que se sigue \textbf{ se puede ayuntar aquella materia en que tracta de aquellas cosas } que cunplen la mengua corporal & et familiae caeterae \textbf{ ( ut in prosequendo patebit ) connecti potest materia illa , } in qua agitur de iis quae supplent indigentiam corporalem , \\\hline
2.3.1 & por que estas materias son ayuntadas \textbf{ en vno entendemos de enssennar } a aquellos que quisieren conueinblemente gouernar sus calas & eo quod hae materiae sunt connexae , \textbf{ intendimus instruere uolentem suas domus debite gubernare , } non solum quantum ad regimen ministrorum et familiae , \\\hline
2.3.1 & en vno entendemos de enssennar \textbf{ a aquellos que quisieren conueinblemente gouernar sus calas } non lo lamente & eo quod hae materiae sunt connexae , \textbf{ intendimus instruere uolentem suas domus debite gubernare , } non solum quantum ad regimen ministrorum et familiae , \\\hline
2.3.1 & quan honrradas \textbf{ e quan conueinbles deuen auer las moradas los Reyes e los prinçipes } e generalmente todos los çibdadanos . & quae supplere videntur indigentiam corporalem . Determinabimus igitur in hac parte tertia huius secundi libri , \textbf{ quam decentem habitationem Reges et Principes } et uniuersaliter omnes ciues habere debeant : \\\hline
2.3.1 & e generalmente todos los çibdadanos . \textbf{ Et en qual manera se deuan auer çerca las possesiones } e çerca las riquezas e los dineros & et uniuersaliter omnes ciues habere debeant : \textbf{ quomodo deceat ipsos se habere circa possessiones , et numismata , } et circa regnum familiae , \\\hline
2.3.1 & e de los sirmientes . \textbf{ Mas que conuenga de cuydar todas estas cosas al sabio padre familias } que es padre dela conpanna & et ministrorum . \textbf{ Quod autem omnia haec considerare deceat prudentem patremfamilias , } vel doctum gubernatorem familiae : \\\hline
2.3.1 & que es el gouernador dela conpanna \textbf{ e dela casa cuydar } e penssar delas posessiones & quod spectat ad oeconomicum \textbf{ et ad gubernationem familiae considerare de possessionibus : } ut de domibus , \\\hline
2.3.1 & e dela casa cuydar \textbf{ e penssar delas posessiones } assi commo delas casas & quod spectat ad oeconomicum \textbf{ et ad gubernationem familiae considerare de possessionibus : } ut de domibus , \\\hline
2.3.1 & por que el padre familias \textbf{ que es senor dela casa aquien parte nesçe gouernar la casa } conueinblemente pueda auer grant cuydado çerca aquellas cosas & ut patrifamilias \textbf{ cuius est domum gubernare debite solicitetur } circa ea quae faciunt ad bene viuere , \\\hline
2.3.1 & que es senor dela casa aquien parte nesçe gouernar la casa \textbf{ conueinblemente pueda auer grant cuydado çerca aquellas cosas } que fazen para beuir bien & cuius est domum gubernare debite solicitetur \textbf{ circa ea quae faciunt ad bene viuere , } et quae requiruntur ad sufficientiam vitae : \\\hline
2.3.1 & conueinblemente pueda auer grant cuydado çerca aquellas cosas \textbf{ que fazen para beuir bien } e que son meneste para conplimiento dela uida & cuius est domum gubernare debite solicitetur \textbf{ circa ea quae faciunt ad bene viuere , } et quae requiruntur ad sufficientiam vitae : \\\hline
2.3.1 & ¶La segunda razon \textbf{ para prouar esto mismo } assi commo paresçe & spectat ad gubernatorem domus considerare de talibus . \textbf{ Secunda via ad inuestigandum hoc idem } ( ut patet per Philosophum ibidem ) sumitur \\\hline
2.3.1 & En essa misma manera el arte del gouernamiento dela casa demanda sus estrumentos \textbf{ por los quales pueda conplir sus obras . } Et por ende los que quieren dar conosçimiento dela arte del ferrero & et textoria , habent sua organa , per quae perficiunt actiones suas : sic et gubernatio domus requirit sua organa , \textbf{ per quae opera sua complere possit . } Volens ergo tradere notitiam de arte fabrili , \\\hline
2.3.1 & por los quales pueda conplir sus obras . \textbf{ Et por ende los que quieren dar conosçimiento dela arte del ferrero } conuiene les de determinar del martiello & per quae opera sua complere possit . \textbf{ Volens ergo tradere notitiam de arte fabrili , } oportet ipsum determinare de martello , \\\hline
2.3.1 & Et por ende los que quieren dar conosçimiento dela arte del ferrero \textbf{ conuiene les de determinar del martiello } e dela yunque & Volens ergo tradere notitiam de arte fabrili , \textbf{ oportet ipsum determinare de martello , } et incude , \\\hline
2.3.1 & Et al ferero pertenesçe de cognosçertales estrumentos . \textbf{ Et dessa misma manera el que quiere dar conosçimiento del arte del texer } deue determinar de los peinnes e de los otros estrumentos de aquella arte & et spectat ad fabrum talia instrumenta cognoscere . \textbf{ Sic volens tradere notitiam de arte textoria , } debet determinare de pectinibus , \\\hline
2.3.1 & Et dessa misma manera el que quiere dar conosçimiento del arte del texer \textbf{ deue determinar de los peinnes e de los otros estrumentos de aquella arte } e pertenesçe al texedor de conosçer tales estrumentos . & Sic volens tradere notitiam de arte textoria , \textbf{ debet determinare de pectinibus , | et aliis organis illius artis , } et spectat ad textorem talia instrumenta cognoscere . \\\hline
2.3.1 & deue determinar de los peinnes e de los otros estrumentos de aquella arte \textbf{ e pertenesçe al texedor de conosçer tales estrumentos . } Por la qual cosa el que quisiere dar conosçimiento del arte del gouernamiento dela casa deue determinar de los hedifiçios & et aliis organis illius artis , \textbf{ et spectat ad textorem talia instrumenta cognoscere . } Quare volens tradere notitiam de arte gubernationis domus , \\\hline
2.3.1 & e pertenesçe al texedor de conosçer tales estrumentos . \textbf{ Por la qual cosa el que quisiere dar conosçimiento del arte del gouernamiento dela casa deue determinar de los hedifiçios } e delas posessiones & et spectat ad textorem talia instrumenta cognoscere . \textbf{ Quare volens tradere notitiam de arte gubernationis domus , | debet determinare de aedificiis , possessionibus , } et numismatibus : \\\hline
2.3.1 & commo estas son estrumentos desta arte . \textbf{ Et pues que assi es ꝑtenesçe al gouernador dela casa de conoscer tales cosas } por que por estas cosas & et numismatibus : \textbf{ quia talia sunt organa huius artis . Spectat ergo ad gubernationem domus talia cognoscere : } quia per haec tanquam propria organa consequi poterit , \\\hline
2.3.1 & assi conmo \textbf{ por prop̃os estrumentos pueda alcançar estas cosas } que fazen al abastamiento dela uida . & quia talia sunt organa huius artis . Spectat ergo ad gubernationem domus talia cognoscere : \textbf{ quia per haec tanquam propria organa consequi poterit , } quae faciunt ad sufficientiam vitae : \\\hline
2.3.1 & que es derecha razon de todas las cosas \textbf{ que ha de fazer . } Et pues que alsi es & ø \\\hline
2.3.1 & por que el arte mecanica es derecha razon delas cosas \textbf{ que ha de fazer de fuera . } Et por tal arte sale algcosa fechͣ en la materia de fuera . & Differt autem prudentia ab arte , \textbf{ quia ars est recta ratio factibilium et per artem resultat aliquid factum in materia extra : } sed prudentia est recta ratio agibilium , \\\hline
2.3.1 & Mas la sabiduria es derecha razon delas cosas \textbf{ que son de fazer } e por ella non sale propreamente ninguna cosa fechͣ de fuera Massale alguna accioono alguna perfection en aquella & quia ars est recta ratio factibilium et per artem resultat aliquid factum in materia extra : \textbf{ sed prudentia est recta ratio agibilium , } et per eam non proprie resultat aliquid factum extra : \\\hline
2.3.1 & que ha de ser çerca \textbf{ las cosas que se han de fazer } e fincan en el alma . & quia sunt organa prudentiae , \textbf{ quae circa agibilia habet esse . } Ars ergo gubernationionis domus licet largo modo possit dici ars , \\\hline
2.3.1 & que fazen a gouernamiento e a conplimiento dela uida dela uida dela casa . \textbf{ Mas por que aquellas mismas razones sen podia prouar } que parte nesçe al gouernamiento dela casa & et ad sufficientiam vitae . \textbf{ Per illas autem easdem rationes probari posset , } quod spectat ad gubernatorem domus scire debite se habere circa ministros \\\hline
2.3.1 & que parte nesçe al gouernamiento dela casa \textbf{ saber se auer conueinblemente çerca los ofiçiales } e çerca los sieruos & Per illas autem easdem rationes probari posset , \textbf{ quod spectat ad gubernatorem domus scire debite se habere circa ministros } et seruos : \\\hline
2.3.1 & Por la qual cosa es declarada la primera parte del capitulo \textbf{ do es dicho queꝑ tenesçe al gouernador dela casa } de auer cuydado de los siruientos & propter quod declarata est prima pars capituli , \textbf{ ubi dicebatur quod specta : | ad gubernationem domus considerare de ministris } et seruis , \\\hline
2.3.1 & do es dicho queꝑ tenesçe al gouernador dela casa \textbf{ de auer cuydado de los siruientos } e de los sieruos de aquellas cosas & ad gubernationem domus considerare de ministris \textbf{ et seruis , } et de his quae supplent indigentiam corporalem . \\\hline
2.3.1 & Mas que estas dos materias sean ayuntadas en vno \textbf{ e que determinar de los hedifiçios } e delas possessiones se deua ayuntar al tractado & Quod autem hae duae materiae sint connexae ; \textbf{ et quod determinare de aedificiis , } et possessionibus connecti debeat . Tractatui , \\\hline
2.3.1 & e que determinar de los hedifiçios \textbf{ e delas possessiones se deua ayuntar al tractado } en que es do terminar de los siruientes e de los sieruos & et quod determinare de aedificiis , \textbf{ et possessionibus connecti debeat . Tractatui , } in quo determinatur de ministris et seruis , \\\hline
2.3.1 & e delas possessiones se deua ayuntar al tractado \textbf{ en que es do terminar de los siruientes e de los sieruos } Esto puede paresçer de ligero . & et possessionibus connecti debeat . Tractatui , \textbf{ in quo determinatur de ministris et seruis , } de leui patere potest . Nam possessiones , domus , \\\hline
2.3.1 & en que es do terminar de los siruientes e de los sieruos \textbf{ Esto puede paresçer de ligero . } Ca las possessiones e las casas & in quo determinatur de ministris et seruis , \textbf{ de leui patere potest . Nam possessiones , domus , } et numismata sunt organa gubernationis domus : \\\hline
2.3.1 & e los dsson esteumentos del gouernamiento dela casa . \textbf{ Et por ender cada vn sieruo es vn estrumento } e cada vn estrumento es vn sieruo . & et numismata sunt organa gubernationis domus : \textbf{ quilibet ergo seruus est quoddam organum , } et quodlibet organum est quidam seruus . \\\hline
2.3.2 & assi commo es en el arte \textbf{ para gouernar las naues } que vsa de estremento sin alma & et inanimatum , \textbf{ ut in arte gubernatiua nauium , tamquam organum inanimatum , est gubernaculum siue remus : tanquam animatum , est ibi prorarius siue remigator . Sic in gubernatione domus , tanquam organa inanimata , } sunt ibi indumenta , \\\hline
2.3.2 & finca \textbf{ de ver } en qual manera son conparados los vnos alos otros . & quomodo distinguuntur organa gubernationis domus : \textbf{ restat ostendere , } quomodo ad inuicem comparantur . Oportet enim \\\hline
2.3.2 & e en la natura todas las cosas son ordenadas \textbf{ e sienpre las cosas mas baxas deuen seruir alas cosas mas altas e meiores . } Et resçiben dellas manera de seruiçio e mesura . & et in natura ordinata sunt omnia , \textbf{ ut semper inferiora administrentur per superiora , } et recipiant modum \\\hline
2.3.2 & por cosa semeiable en las otras artes \textbf{ do da a entender que el instrumento dela itulero non es } por si çitulador & quod declarat Philosophus 1 Polit’ per simile in aliis artibus , \textbf{ ubi innuit quod plectra non per se cytharizant , } et pectines non per se ipsos pectinant . Ideo ad cytharizandum plectrum indiget ministro mouente , \\\hline
2.3.2 & Et el penne \textbf{ para peyndar ha menester algun mouedor } que lo mueua & et pecten \textbf{ ad pectinandum indiget mouente ipsum . Sic } ( \\\hline
2.3.2 & por que los instrumentos sin alma \textbf{ por si non pueden fazer las obras } para que son fechos . & quia organa inanimata per se ipsa exercere \textbf{ non possunt illud , | ad quod sunt facta . } Nam si tripodes \\\hline
2.3.2 & por si se abrerien e se cerrarien . \textbf{ Mas por que non es assi que los istrumentos sin alma non se pueden mouer } por si por ende los sennores & et clauderent : \textbf{ sed quia non sic est , organa inanimata a se ipsis moueri non possunt . Ideo domini } et architectores indigent ministris \\\hline
2.3.2 & que non han razon \textbf{ por que puedan conplir su obra propraa } e fazer sus ofiçios . & vel organa carentia ratione , \textbf{ ut opus proprium possint implere . } Indignum est enim \\\hline
2.3.2 & por que puedan conplir su obra propraa \textbf{ e fazer sus ofiçios . } Ca non es cosa conueinble & vel organa carentia ratione , \textbf{ ut opus proprium possint implere . } Indignum est enim \\\hline
2.3.2 & por aquellas que ban alma \textbf{ por que puedan conplir las lus obras propia } e las cosas ya dichͣs puede paresçer & et mouenda , \textbf{ ut possint propria opera adimplere . } Ex praedictis patere potest , \\\hline
2.3.3 & por que puedan conplir las lus obras propia \textbf{ e las cosas ya dichͣs puede paresçer } que auemos de dezir enesta terçera parte deste segundo libro & ut possint propria opera adimplere . \textbf{ Ex praedictis patere potest , } in hac tertia parte huius secundi \\\hline
2.3.3 & e las cosas ya dichͣs puede paresçer \textbf{ que auemos de dezir enesta terçera parte deste segundo libro } ta bien de los siruientes & Ex praedictis patere potest , \textbf{ in hac tertia parte huius secundi | libri dicendum } esse tam de ministris \\\hline
2.3.3 & las quales son moradas e possessions e dineros . \textbf{ Et por ende destas tres cosas auemos a dezir ¶ } Ca declaramos primero quales casas & et numismata . \textbf{ De quatuor ergo dicendum erit . } Declarabitur enim primo , \\\hline
2.3.3 & Ca declaramos primero quales casas \textbf{ e quales moradas deuen auer los Reyes e los prinçipes } e generalmente todos los çibdadanos & Declarabitur enim primo , \textbf{ quales domos , et quales habitationes Reges et Principes , } et uniuersaliter omnes ciues habere debent . Secundo determinabitur de ipsis possessionibus . Tertio de numismatibus . Quarto de ministris . \\\hline
2.3.3 & mas entre todas las otras cosas \textbf{ que son de cuydar en las moradas } assi commo dize paladio enel libro dela agnitul & et uniuersaliter omnes ciues habere debent . Secundo determinabitur de ipsis possessionibus . Tertio de numismatibus . Quarto de ministris . \textbf{ Inter alia autem , quae consideranda sunt in aedificiis , } ut tradit Pallad’ in lib de Agric’ \\\hline
2.3.3 & delas quales la primera se toma de parte dela grandeza real . ¶ La segunda de parte del pueblo \textbf{ mas nos podemos annader la terçera razon } que se toma de parte dela conpannia & Secunda ex parte populi . Possumus autem \textbf{ et nos tertiam rationem addere , } ex parte familiae et ministrorum . Prima via sic patet : \\\hline
2.3.3 & en esse mismo quarto libro delas ethicas \textbf{ al magnifico parte nesçe de apareiar morada conuenible } mas non es conuenible morada al magnifico & Sed magnifici , \textbf{ ut dicitur in eodem 4 Ethicor’ est praepare habitationem decentem : } non est autem decens habitatio , \\\hline
2.3.3 & quanto ala maestera dela obra pertenesçe auer moradas matauillosas \textbf{ e por ende los otros çibdadanos deuen auer tales moradas mas o menos marauillosas } segunt que pertenesçe a cada vno en su estado & decet habere habitationes mirabiles . \textbf{ Alii vero ciues tales habitationes habere debent magis | et minus mirabiles , } ut competit propriae facultati . \\\hline
2.3.3 & ¶La segunda razon \textbf{ para prouar esto mismo se toma de parte del pueblo } e esta razon tanne el philosofo enł vi̊ libro delas politicas & Secunda via ad inuestigandum hoc idem , \textbf{ sumitur ex parte ipsius populi : } et hanc tangit Philosophus 6 Politicorum , \\\hline
2.3.3 & do dize \textbf{ que alos Reyes e alos prinçipes parte nesçe de fazer tan grandes cosas } e de estroyr e de fazer tales moradas & ubi ait , \textbf{ quod Principes decet sic magnifica facere , } et talia aedificia construere , \\\hline
2.3.3 & que alos Reyes e alos prinçipes parte nesçe de fazer tan grandes cosas \textbf{ e de estroyr e de fazer tales moradas } que el pueblo que lo viere finque & quod Principes decet sic magnifica facere , \textbf{ et talia aedificia construere , } quod populus ea videns , \\\hline
2.3.3 & que el prinçipe es tan grande \textbf{ que en ninguna manera non podria yr contra el } e por que en las cosas que non pueden ser non cae hellecçion & quilibet enim de populo hoc viso opinatur principem esse tantum , quod quasi impossibile sit ipsum inuadere : \textbf{ et quia circa impossibilia } non cadit electio neque consilium , \\\hline
2.3.3 & enl terçero libro delas ethicas \textbf{ cada vno del pueblo se guar da } que non faga bolliçio contra el prinçipe & ut vult Philosophus 3 Ethicorum , \textbf{ quilibet ex populo retrahitur , } ne dissensionem faciat contra Principem , \\\hline
2.3.3 & ca la grandeza delas moradas \textbf{ maguer non se de una fazer aparesçençia } nin a grandezaauana eglesia & et tam magnificum . Magnitudo enim aedificiorum \textbf{ licet non sit fienda ad ostentationem et inanem gloriam : } decet tamen Reges et Principes , \\\hline
2.3.3 & enpero conuiene alos Reyes \textbf{ e alos prinçipes de fazer moradas costosas e nobles } assi commo el su estado demanda & decet tamen Reges et Principes , \textbf{ ne in contemptum habeantur a populo , | facere aedificia magnifica , } prout requirit decentia status , \\\hline
2.3.3 & por que non solamente la persona del Rey o del prinçipe \textbf{ mas avn por que la muchedunbre de los siruientes pue dan morar con ueinblemente en las casas } que ellos fazen a conuiene & ut ergo non solum personas Regis et Principis , \textbf{ sed etiam multitudo ministrorum debite commorari possint in aedificiis constructis , } oportet ipsa esse magnifica . Viso , \\\hline
2.3.3 & quanto ala grandeza \textbf{ e quanto ala maestera dela obra finca de ver } quales deuen ser & quantum ad magnificentiam \textbf{ et industriam operis : | restat videre , } qualia esse debent , \\\hline
2.3.3 & mas tanne peladio enłlibro dela agnicultura tres cosas \textbf{ por las quales podemos conosçer } en que ayre es de fazer la morada & quantum ad aeris temperamentum . Tangit autem Palladius in libro de Agricultura , tria , \textbf{ ex quibus cognoscere possumus , } in quo aere sit aedificium construendum . Dicit enim salubritatem aeris primo declarare loca a vallibus infimis libera . Si enim in vallibus infimis aedificia construantur , \\\hline
2.3.3 & por las quales podemos conosçer \textbf{ en que ayre es de fazer la morada } o dize & ø \\\hline
2.3.3 & que el ayre y non sea sano \textbf{ porque deuemos assi ymaginar } que assi commo el agua corre es mas ssana & propter circumstantiam montium contingit ipsum non esse salubrem . \textbf{ Sic enim imaginari debemus , } quod sicut aqua currens sanior est quam stans , \\\hline
2.3.3 & e non sano . \textbf{ Et pues que assi es deuemos cuydar } por la sanidat del ayre en las moradas & quasi ingrossatur , \textbf{ et efficitur non salubris ; } est ergo propter salubritatem aeris considerandum in aedificiis construendis , \\\hline
2.3.3 & por la sanidat del ayre en las moradas \textbf{ que auemos de fazer } que non se fagan & et efficitur non salubris ; \textbf{ est ergo propter salubritatem aeris considerandum in aedificiis construendis , } ut non fiat \\\hline
2.3.3 & nin se costruyan en los valłs muy baxos ¶ \textbf{ Lo segundo deuemos cuydarl } que aquel lugar en que deuemos fazer la morada sea guardado delas timebras dela meblaca & ut non fiat \textbf{ talis constructio in vallibus infimis . Secundo considerandum est , } ut locus ille , in quo est aedificium construendum , sit a nebularum tenebris absolutus . \\\hline
2.3.3 & Lo segundo deuemos cuydarl \textbf{ que aquel lugar en que deuemos fazer la morada sea guardado delas timebras dela meblaca } en alguna parte dela tierra & talis constructio in vallibus infimis . Secundo considerandum est , \textbf{ ut locus ille , in quo est aedificium construendum , sit a nebularum tenebris absolutus . } Nam in aliqua parte terrarum , \\\hline
2.3.3 & por la qual cosa \textbf{ si lo podemos mudar non son de fazer las moradas en aquel loguar . } ¶ Lo terçero & Ideo si vitari potest , \textbf{ non sunt ibi aedificia construenda . Tercium , } quod declarat salubritatem aeris , \\\hline
2.3.3 & ¶ Lo terçero \textbf{ que muestra la sanidat del ayte es cuydar enlos moradores } que estan enł & non sunt ibi aedificia construenda . Tercium , \textbf{ quod declarat salubritatem aeris , } est consideratio habitatorum existentium in ipso si enim alicubi aedificare volumus , \\\hline
2.3.3 & que estan enł \textbf{ por que si en algun loguar queremos fazer algunas moradas } si contezca que alguos moren cerca aquella t rrason de uer & quod declarat salubritatem aeris , \textbf{ est consideratio habitatorum existentium in ipso si enim alicubi aedificare volumus , } si contingat circa regionem illam aliquos habitare consideranda sunt habitatorum corpora , \\\hline
2.3.3 & por que si en algun loguar queremos fazer algunas moradas \textbf{ si contezca que alguos moren cerca aquella t rrason de uer } e de catar los cuerpos & est consideratio habitatorum existentium in ipso si enim alicubi aedificare volumus , \textbf{ si contingat circa regionem illam aliquos habitare consideranda sunt habitatorum corpora , } si eis \\\hline
2.3.3 & si contezca que alguos moren cerca aquella t rrason de uer \textbf{ e de catar los cuerpos } de los que moran en ella & est consideratio habitatorum existentium in ipso si enim alicubi aedificare volumus , \textbf{ si contingat circa regionem illam aliquos habitare consideranda sunt habitatorum corpora , } si eis \\\hline
2.3.4 & e la tenprinça del ayre \textbf{ son otras dos colas de pensar en las moradas } assi comm̃es la sanidat del agua & et aeris temperamentum : \textbf{ sunt duo alia in aedificiis attendenda , } ut aquae salubritas , \\\hline
2.3.4 & e en much ͣs cosas sirue ala neçessidat dela iuida \textbf{ e por ende mucho es de cuydar } que la morada sea assentada en tal logar & Aqua enim ( secundum Philosophum ) est valde communis , \textbf{ et in multis deseruit ad necessaria vitae . Ideo valde considerandum est , } ut sic aedificium situetur , \\\hline
2.3.4 & por mengua de agua \textbf{ nin por corronpemiento della non ayan de caer en enfermedades } e por ende tanne paladio en el libro de la agnicultura seys cosas & ne habitatores eius \textbf{ ob infectionem aquae infirmitatem contrahant . Tangit autem Palladius in libro De agricultura sex } quae ait esse consideranda in cognitione aquae salubritatis . Primum est : \\\hline
2.3.4 & que deuen ser penssadas enl comneçamiento del agua sana \textbf{ ¶La primera es que el agua non deue correr } nin uasçer de peçinas nin de lagu nas & quae ait esse consideranda in cognitione aquae salubritatis . Primum est : \textbf{ quia aqua illa deriuari non debet a paludibus et a lacunis . } Paludes enim et lacunae , \\\hline
2.3.4 & e por ende por la mayor parte non han el agua sana¶ \textbf{ Lo segundo deuemos cuydar } que aquella agua non tome nasçençia delos metalles & ut plurimum habent aquam non salubrem . \textbf{ Secundo considerandum est , } ne aqua illa sumat originem ex metallis , \\\hline
2.3.4 & ¶Lo terçero \textbf{ que es de penssar en las aguas } es que sean de color claro de gnisa & sed simul cum metallis sunt aliquae admixtae putredines . Tertium , \textbf{ quod considerandum est in aquis , } est quod sit coloris perspicui . \\\hline
2.3.4 & por que es corrupta non puede ser sana . \textbf{ Et pues que assi es estas çinco cosas deuemos penssar } en la sanidat delas aguas enpero & sana esse non potest . \textbf{ Haec ergo quinque attendenda sunt in salubritate aquarum . } Verum quia contingit aliquando nos in his signis decidi , \\\hline
2.3.4 & que vsen de aquellas aguas \textbf{ e por ende deuemos parar mientes sy los dientes e las gengibas } que vsan de aquestas aguas & et dispositio corporum utentium illis aquis . Est ergo aspiciendum , \textbf{ si dentes et gingiuae utentium illis aquis , sint puri : } si utentes habeant capita sana et inperturbata : \\\hline
2.3.4 & por que por la maliçia delas aguas estas cosas \textbf{ o algua dellas suele contesçer . } Et pues que assi es la morada deue ser fechͣ en tal logar & vel omnia mala haec , \textbf{ vel aliqua horum consueuerunt contingere . } Aedificium ergo construendum aedificari debet in tali situ , \\\hline
2.3.4 & Enpero si tanta fuere la neçesidat \textbf{ que nos costranga de fazer alli moradera } e non podamos auer & de quibus fecimus mentionem . \textbf{ Quod si tamen aedificandi necessitas urgeat , } nec tamen ibi aquae salubris sit copia , \\\hline
2.3.4 & que nos costranga de fazer alli moradera \textbf{ e non podamos auer } y abastamiento de agua sana deuemos & Quod si tamen aedificandi necessitas urgeat , \textbf{ nec tamen ibi aquae salubris sit copia , } est ibi \\\hline
2.3.4 & e algibes \textbf{ en que se puedan coger las aguas dela luuia } por que segunt este philosofo el agua dela luuia çelestiales fallada & secundum Palladium ) construenda cisterna , \textbf{ in qua pluuiales aquae colligendae sunt . } Nam \\\hline
2.3.4 & por mas sana \textbf{ para beuer entre todas las otras aguas } mas en estas çisternas o en estos algibes deuemos poner peçes de rio & ( \textbf{ secundum eundem ) aqua caelestis et pluuialis ad bibendum } quasi omnibus amefertur , sunt autem in cisterna illa pisces fluuiales apponendi , \\\hline
2.3.4 & para beuer entre todas las otras aguas \textbf{ mas en estas çisternas o en estos algibes deuemos poner peçes de rio } por que por el nadamiento de los peces el agua estante semeie en lignieza & secundum eundem ) aqua caelestis et pluuialis ad bibendum \textbf{ quasi omnibus amefertur , sunt autem in cisterna illa pisces fluuiales apponendi , } ut horum natatu aqua stans agilitatem currentis imitetur . Viso , qualiter est aedificium construendum quantum ad salubritatem aquae : \\\hline
2.3.4 & por que por el nadamiento de los peces el agua estante semeie en lignieza \textbf{ al agua que corre ¶ Visto en qual manera son de fazer las moradas } quanto ala salud delas agunas finca de ver & quasi omnibus amefertur , sunt autem in cisterna illa pisces fluuiales apponendi , \textbf{ ut horum natatu aqua stans agilitatem currentis imitetur . Viso , qualiter est aedificium construendum quantum ad salubritatem aquae : } restat videre , \\\hline
2.3.4 & al agua que corre ¶ Visto en qual manera son de fazer las moradas \textbf{ quanto ala salud delas agunas finca de ver } en qual manera son de fazer & quasi omnibus amefertur , sunt autem in cisterna illa pisces fluuiales apponendi , \textbf{ ut horum natatu aqua stans agilitatem currentis imitetur . Viso , qualiter est aedificium construendum quantum ad salubritatem aquae : } restat videre , \\\hline
2.3.4 & quanto ala salud delas agunas finca de ver \textbf{ en qual manera son de fazer } quanto ala orden del mundo & ut horum natatu aqua stans agilitatem currentis imitetur . Viso , qualiter est aedificium construendum quantum ad salubritatem aquae : \textbf{ restat videre , } qualiter construendum sit quantum ad ordinem Uniuersi . \\\hline
2.3.4 & segunt que demanda la morada \textbf{ que es de fazer son de penssar tres cosas } conuiene a saber la condicion del çielo & prout requiri aedificium construendum , \textbf{ sunt tria consideranda , } videlicet conditio caelestis : \\\hline
2.3.4 & que es de fazer son de penssar tres cosas \textbf{ conuiene a saber la condicion del çielo } e el departimiento de los uientos & sunt tria consideranda , \textbf{ videlicet conditio caelestis : } diuersitas ventorum : \\\hline
2.3.4 & Mas quanto ala condiconn del çielo \textbf{ segunt dize paladio son de penssar dos cosas . } ¶ La primera que enł yuier no sean alunbradas las moradas de claridat conueinble ¶ & diuersitas ventorum : \textbf{ et dispositio terrarum . Quantum ad conditionem caelestem duo sunt attendenda . Primo ut hyeme debita claritate illustretur . Secundo , } ne in aestate immoderato calore opprimatur , \\\hline
2.3.4 & por grant calentura¶ \textbf{ Lo primero puede contesçer } si la morada & ne in aestate immoderato calore opprimatur , \textbf{ quod fieri contingit , si aedificium } secundum suam ampliorem partem respiciat oriens \\\hline
2.3.4 & que el derecho ¶ \textbf{ Lo segundo en fazer la morada auemos de penssar el departimiento de los uientos } e esto quanto al departimiento delas camaras . & nam semper radius obliquus minorem calorem generat , \textbf{ quam directus . Secundo in aedificando aedificio attendenda est diuersitas ventorum : } et hoc quantum ad diuersitatem camerarum . \\\hline
2.3.4 & e por ende enel tienpo del estiuo \textbf{ en que los omes muy ligeramente enferman son de fazer alguas camaras contrarias al uiento set enteronal e del çierço } por que sea enllas guardada la uida con mayor salud . & propter tempus ergo aestiuum , \textbf{ in quo homines facilius infirmantur , aedificandae sunt aliquae camerae oppositae vento septentrionali , } ut in eis salubrior custodiatur vita . Tertio \\\hline
2.3.4 & ¶ Lo terçero \textbf{ quanto ala orden del mundo auemos de penssar la disposiconn de las tierras } por que en tal loguar sean fechos las moradas & ut in eis salubrior custodiatur vita . Tertio \textbf{ quantum ad ordinem uniuersi | consideranda est dispositio terrarum , } ut in tali loco aedificium construatur , \\\hline
2.3.4 & por que en tal loguar sean fechos las moradas \textbf{ por que puedan auer uergeles e fructales e huertas } que sean çerca dellas & ut in tali loco aedificium construatur , \textbf{ cui viridaria | et pomeria esse possunt connexa : } aspectus enim talium et deambulatio per ea ad hylaritatem \\\hline
2.3.4 & que sean çerca dellas \textbf{ por que catar tales cosas } e andar por ellas vale mucha alegera & et pomeria esse possunt connexa : \textbf{ aspectus enim talium et deambulatio per ea ad hylaritatem } et sanitatem confert . \\\hline
2.3.4 & por que catar tales cosas \textbf{ e andar por ellas vale mucha alegera } e a salud del cuerpo mas podrien se dezer alguas otras cosas mas particulares & aspectus enim talium et deambulatio per ea ad hylaritatem \textbf{ et sanitatem confert . } Essent autem in aedificiis construendis quaedam alia particularia dicenda ; \\\hline
2.3.4 & e andar por ellas vale mucha alegera \textbf{ e a salud del cuerpo mas podrien se dezer alguas otras cosas mas particulares } en el fazer delas moradas & et sanitatem confert . \textbf{ Essent autem in aedificiis construendis quaedam alia particularia dicenda ; } ut qualis deberet esse cella vinaria , \\\hline
2.3.4 & e a salud del cuerpo mas podrien se dezer alguas otras cosas mas particulares \textbf{ en el fazer delas moradas } assi commo diremos & et sanitatem confert . \textbf{ Essent autem in aedificiis construendis quaedam alia particularia dicenda ; } ut qualis deberet esse cella vinaria , \\\hline
2.3.4 & e de los muradales e del estiercol . \textbf{ ¶ Avn en essa misma manera se pueden departir o triscondiconnes particulares delas moradas . } mas por que tales cosas son much̉ particular & et fluminibus : et longe a stabulis , fimo , \textbf{ et sterquiliniis . Sic etiam aliae particulares conditiones aedificiorum distingui possent . } Sed \\\hline
2.3.5 & segunt la orden \textbf{ que contamos de suso finca de dezer delas possessiones } maspodemos prouar & secundum ordinem superius enarratum , \textbf{ restat dicere de possessionibus . Possumus autem triplici via venari , } quod rerum possessio est quodammodo naturalis . \\\hline
2.3.5 & que contamos de suso finca de dezer delas possessiones \textbf{ maspodemos prouar } por tres razons la possession delas cosas es natural en algua manera al omne & secundum ordinem superius enarratum , \textbf{ restat dicere de possessionibus . Possumus autem triplici via venari , } quod rerum possessio est quodammodo naturalis . \\\hline
2.3.5 & La segunda razon \textbf{ para mostrar esto mismo se toma dela dignidat del omne } por que por esso mismo que el omne es c̀atura & est quodammodo homini naturalis . \textbf{ Secunda via ad inuestigandum hoc idem sumitur } ex dignitate humana : \\\hline
2.3.5 & e a estas cosas senssibles \textbf{ e que pueda vsar dellas } e resçebir seruiçio dellas & quod dominetur istis sensibilibus , \textbf{ et quod possit eis uti in suum obsequium , } et \\\hline
2.3.5 & e que pueda vsar dellas \textbf{ e resçebir seruiçio dellas } segunt quel fuere uisto & et quod possit eis uti in suum obsequium , \textbf{ et } quia hoc est quodammodo possidere ea , \\\hline
2.3.5 & segunt quel fuere uisto \textbf{ e por que esto es en alguna manera auer possession de estas cosas } por ende la possession dellas es al omenatanlonde el philosofo enł primer libro delas politicas & et \textbf{ quia hoc est quodammodo possidere ea , } ideo eorum possessio est ei naturalis . \\\hline
2.3.5 & nud̀miento conuenible . \textbf{ mucho mas deue apareiar alas ainalias ya acabadas } mas delas aian lias alguas son & et ipsis imperfectis animalibus praeparatur debitum alimentum a natura , \textbf{ multo magis hoc praeparabitur eis iam inesse perfectis . } Animalium autem quaedam ouificant , \\\hline
2.3.5 & quanto al nutermiento dellas \textbf{ por que luego que nasçe es acuçiosa de aduzer leche enlas teras delas madres } assi que de aquella lech̃e las aian las enrendradas se pueden cerar & quia \textbf{ statim solicita est inducere lac in mamillis matris , ut ex eo animalia genita nutriri possint . } Quare si animalibus imperfectis natura praeparat nutrimentum \\\hline
2.3.5 & por que luego que nasçe es acuçiosa de aduzer leche enlas teras delas madres \textbf{ assi que de aquella lech̃e las aian las enrendradas se pueden cerar } por la qual cofa & ø \\\hline
2.3.5 & e cosas neçessarias ala uida alas aianlias non acabadas \textbf{ much mas esto faze e deue fazer alas aianlias acabadas . } Et por ende el philosofo dize en el primero libro delas politicas & et necessaria vitae , \textbf{ multo magis hoc facit animalibus perfectis . } Ideo ait Philosophus primo Polit’ \\\hline
2.3.5 & que estas cosas de que tomamosnudmiento son dadas a nos por nata . \textbf{ Et pues que assi es natal cosa es a nos de auer las cosas de fuera } e por ende el sennorio delas cosas de fuera es en algua manera natural al omne . & datae sunt nobis a natura . \textbf{ Naturale est ergo nobis habere res exteriores . } Habere ergo dominium rerum exteriorum est \\\hline
2.3.5 & que es mas alta que de ome \textbf{ ca assi commo cassar } e engendrar otros semeiables & et supra hominem . \textbf{ Sicut nubere , } et per generationem producere sibi similia , \\\hline
2.3.5 & ca assi commo cassar \textbf{ e engendrar otros semeiables } assi es cosa natural & Sicut nubere , \textbf{ et per generationem producere sibi similia , } est homini naturale : \\\hline
2.3.5 & en quanto es omne \textbf{ segunt dize el philosofo enl primero libro delas politicas de auer possession } e sennorio de algers cosas de fuera & ut homo est , \textbf{ ut vult Philosophus primo Polit’ habere possessionem , } et dominium aliquarum rerum exteriorum propter sufficientiam vitae . Natura ergo sicut dedit boni viuere , \\\hline
2.3.5 & assi fizo las aianlias e las plantas e las yerbas \textbf{ por que el omne se enssenorear se dellas } e por qualas possediesse & sic fecit animalia , plantas , et herbas : \textbf{ ut homo eis dominaretur , } et ut possideret ea , \\\hline
2.3.5 & e que resçebiesse ende nudermiento conuenible \textbf{ sin el qual non puede durar lanr̃a uida ¶ } ne opinion de socrates e de platon & et ut susciperet inde debitum nutrimentum , \textbf{ sine quo vita nostra durare non potest . } Fuit opinio Socratis et Platonis , \\\hline
2.3.6 & que quisiessen ser pagados de beuir en tal uida comun \textbf{ mas nos podemos tomar de departidos logares } enł & homines enim communiter non sunt adeo perfecti , \textbf{ quod essent contenti viuere tali vita . Possumus autem ex diuersis locis in libro Polit’ accipere tria , } per quae triplici via venari possumus , \\\hline
2.3.6 & libro delas politicas tres cosas \textbf{ por las quales podemos prouar } por tres razons & quod essent contenti viuere tali vita . Possumus autem ex diuersis locis in libro Polit’ accipere tria , \textbf{ per quae triplici via venari possumus , } quod expedit ciuitati ciues habere proprias possessiones . Prima via sumitur , \\\hline
2.3.6 & por que sea tirada la peza \textbf{ e el non cuydar delas cosas ¶ La segunda por que sea tirada la uaraia ¶ La terçera } por que sea tirado el desordenamiento & quod expedit ciuitati ciues habere proprias possessiones . Prima via sumitur , \textbf{ ut remoueatur inertia et ignauia . Secunda , ut prohibeatur litigium . Tertio , ut tollatur inordinatio et confusio . } Prima via sic patet : \\\hline
2.3.6 & e se desgastarien \textbf{ e en tanto seria guaue cosa de trabaiar cada vno } por su ganançia & et deuastare , \textbf{ et adeo est difficile laborare } et proprio lucro substantiam \\\hline
2.3.6 & por su ganançia \textbf{ para ganar possessiones e sustaçias } quando sopies & et proprio lucro substantiam \textbf{ et possessiones acquirere , } quod in ciuitate contingit multos egere et esse pauperes , \\\hline
2.3.6 & que todas las cosas son comunes en la çibdat \textbf{ que ninguon non quarrie trabaiar por ellas ca agora en la çibdat lon muchs pobres } avn que non contradigamos & et possessiones acquirere , \textbf{ quod in ciuitate contingit multos egere et esse pauperes , } non obstante quod ciues possunt gaudere possessionibus propriis , \\\hline
2.3.6 & avn que non contradigamos \textbf{ que los çibdada nos puedan auer possessiones proprias } e que sean acuçiosos çerca dellas & quod in ciuitate contingit multos egere et esse pauperes , \textbf{ non obstante quod ciues possunt gaudere possessionibus propriis , } et quod solicitantur circa ea tanquam circa propria bona . \\\hline
2.3.6 & mas todas las possessiones fuessen comunes alos çibdadanos \textbf{ por que los çibdadanos non aurien cuydado de labrar las possessiones comunes } assi commo han cuydado de labrar las propreas & sed omnibus ciuibus essent possessiones communes , \textbf{ quia ciues non sic essent soliciti ad colendas possessiones communes , } sicut sunt soliciti ad proprias , \\\hline
2.3.6 & por que los çibdadanos non aurien cuydado de labrar las possessiones comunes \textbf{ assi commo han cuydado de labrar las propreas } por ende contesçeria en la mayor parte & quia ciues non sic essent soliciti ad colendas possessiones communes , \textbf{ sicut sunt soliciti ad proprias , } ut plurimum contingeret ciuitatem \\\hline
2.3.6 & assi orde nada uerme a grant pobreza \textbf{ por que los çibdadanos non podrien abondar } assi enla uida & illam sic ordinatam venire ad inopiam , \textbf{ ut ciues non possent } sibi in vita sufficere ; \\\hline
2.3.6 & ¶La segunda razon \textbf{ para prouar esto mismo se toma } por tirar la contienda de entre los omnes & ne propter ignauiam circa communia , domus ciuium patiantur inopiam . \textbf{ Secunda via ad inuestigandum hoc idem , sumitur ex remotione litigii : } ut plurimum enim consurgunt lites \\\hline
2.3.6 & para prouar esto mismo se toma \textbf{ por tirar la contienda de entre los omnes } ca por la mayor parte se le una tan contiendas e uaraias & Secunda via ad inuestigandum hoc idem , sumitur ex remotione litigii : \textbf{ ut plurimum enim consurgunt lites } et bella inter participantes aliquid commune : videmus enim ipsos fratres \\\hline
2.3.6 & viij̊ libro delas ethicas es amistança natural \textbf{ en la mayor parte veemos los contender } e uaraiar sobre la heredat & inter quos secundum Philosophum 8 Ethicorum est amicitia naturalis , \textbf{ ut plurimum bellare ad inuicem , } eo quod sit eis communis haereditas : \\\hline
2.3.6 & en la mayor parte veemos los contender \textbf{ e uaraiar sobre la heredat } que han en comun . & ut plurimum bellare ad inuicem , \textbf{ eo quod sit eis communis haereditas : } quanto ergo magis esset dissentio \\\hline
2.3.7 & el pho enł primero libro delas politicas muestra \textbf{ que por departidos usos de usar delas cosas } de fuera se le una tan departidas uidas & Philosophus primo Politicorum ostendit , \textbf{ quod ex alio et alio usu exteriorum rerum , | consurgit alia } et alia vita , \\\hline
2.3.7 & assi commo le couiene a cunplimiento de su uida \textbf{ e en la manera que meior pueden buscar la uianda e el nudermiento } que les conuiene . & ut melius sibi possint cibum \textbf{ et nutrimentum quaerere . } Homines etiam diuersimode utuntur exterioribus rebus , \\\hline
2.3.7 & ¶ Et uida de ca ça¶ \textbf{ E uida de tescar ¶ } Et uida de robar e de furtar . & vel aliae vitae . Est enim vita quadruplex , \textbf{ videlicet pascualis , venatiua , } piscatiua , et furatiua . Pascualem autem vitam ducunt viuentes ex agricultura , \\\hline
2.3.7 & E uida de tescar ¶ \textbf{ Et uida de robar e de furtar . } Mas la uida del pasto aduzen aquellos que biuen & videlicet pascualis , venatiua , \textbf{ piscatiua , et furatiua . Pascualem autem vitam ducunt viuentes ex agricultura , } vel ex animalibus domesticis : venatiuam autem , \\\hline
2.3.7 & aquellos quebiuen uida de aianlias montanne sas . \textbf{ ¶ Et los que biuen uida de pescar biuien de los peçes ¶ } Mas aquellos que biuen uida de furtar biue uida de rapina e de furto . & viuentes ex syluestribus : \textbf{ piscatiuam autem , | ex piscibus : } sed furatiuam vitam ducunt viuentes ex rapina et furto . \\\hline
2.3.7 & ¶ Et los que biuen uida de pescar biuien de los peçes ¶ \textbf{ Mas aquellos que biuen uida de furtar biue uida de rapina e de furto . } ¶ Et pueden estas maneras de beuir ler ayuntadas en vno & ex piscibus : \textbf{ sed furatiuam vitam ducunt viuentes ex rapina et furto . } Contingit autem hos modos viuendi conbinare : \\\hline
2.3.7 & Mas aquellos que biuen uida de furtar biue uida de rapina e de furto . \textbf{ ¶ Et pueden estas maneras de beuir ler ayuntadas en vno } por que algunos buien de pasto e de caça & sed furatiuam vitam ducunt viuentes ex rapina et furto . \textbf{ Contingit autem hos modos viuendi conbinare : } quia quidam viuunt pascualiter et venatiue : \\\hline
2.3.7 & e algunos biuen de pasto e de pesca . \textbf{ Mas algunos biuende caçar } e de furtar & quidam pascualiter et piscatiue : \textbf{ quidam vero venatiue et furtiue : } quidam autem ex omnibus , \\\hline
2.3.7 & Mas algunos biuende caçar \textbf{ e de furtar } e algers buien de todas estas cosas cabuscan la uida . & quidam pascualiter et piscatiue : \textbf{ quidam vero venatiue et furtiue : } quidam autem ex omnibus , \\\hline
2.3.7 & ¶ Et pues que assi es cosa conuenible \textbf{ es de tomar nudermiento delos canpos } e delas aianlias de casa & quod natura dedit nobis talia , ordinauit enim ea ad usum \textbf{ et dominium nostrum ; licitum est ergo sumere nutrimentum ex agris , et animalibus domesticis quare vita pascualis est licita . } Sic etiam venatiua \\\hline
2.3.7 & et por ende la uida de los pastos es conuenible a nos . \textbf{ avn en essa misma manera la uida de caçar } e de pescar des si non son desconueibles & et dominium nostrum ; licitum est ergo sumere nutrimentum ex agris , et animalibus domesticis quare vita pascualis est licita . \textbf{ Sic etiam venatiua } et piscatiua \\\hline
2.3.7 & avn en essa misma manera la uida de caçar \textbf{ e de pescar des si non son desconueibles } por que el omne deue enssennorearna falmente & Sic etiam venatiua \textbf{ et piscatiua | de se non sunt illicita ; } quia enim homo naturaliter dominari debet \\\hline
2.3.7 & por la qual cosa el philosofo enl ꝑ̀mo libro delas politicas dize \textbf{ que la uida de caçar } e la uida de pescar son uidas conuenibles & habet contra talia iustum bellum propter quod Philosophus 1 Politic’ vult venatiuam \textbf{ et piscatiuam esse vitas licitas . } Ait enim , \\\hline
2.3.7 & que la uida de caçar \textbf{ e la uida de pescar son uidas conuenibles } por que dize que es derecha batalla de los omes alas bestias & habet contra talia iustum bellum propter quod Philosophus 1 Politic’ vult venatiuam \textbf{ et piscatiuam esse vitas licitas . } Ait enim , \\\hline
2.3.7 & ca fablando omne uerdaderamente \textbf{ por si los omes pueden prender } por sitałs̃aianlias & et ad alia animalia est iustum bellum ; \textbf{ per se enim loquendo homines iuste possunt talia facere , } et ordinare \\\hline
2.3.7 & por sitałs̃aianlias \textbf{ e ordenar las asu uso propreo } Mas la uida de furtar fablado sinplemente de ssi es desconueible & et ordinare \textbf{ ea in usum proprium . Furtiua autem vita per se loquendo est illicita , } quia hominum ad homines per se non est iustum bellum . \\\hline
2.3.7 & e ordenar las asu uso propreo \textbf{ Mas la uida de furtar fablado sinplemente de ssi es desconueible } por que los os contra los omes & et ordinare \textbf{ ea in usum proprium . Furtiua autem vita per se loquendo est illicita , } quia hominum ad homines per se non est iustum bellum . \\\hline
2.3.7 & si non quisieren sorsus subiectos . \textbf{ Et segunt esta manera tal de fablar los çibdadanos que son mas noblesçidos } por sabiduria e por entendimiento han batalla derecha contra los rusticos & si eis nolint esse subiecti . \textbf{ Secundum quem modum loquendi , | ciues , } qui magis uigent prudentia et intellectu , \\\hline
2.3.7 & Et esta misma manera las cosas assi entendidas paresçe \textbf{ que la uida del robar es conuenible } e que non solamente los omes deuen tomar et robar las sus cosas alos otros & secundum sententiam Philosophi sic intellectam , uidetur esse licita praedatiua uita ; \textbf{ ut quod licitum esset non solum hos depraedari } et accipere sua , \\\hline
2.3.7 & que la uida del robar es conuenible \textbf{ e que non solamente los omes deuen tomar et robar las sus cosas alos otros } mas avn deuen tomar a ellos en su perssona & ut quod licitum esset non solum hos depraedari \textbf{ et accipere sua , } sed eos \\\hline
2.3.7 & e que non solamente los omes deuen tomar et robar las sus cosas alos otros \textbf{ mas avn deuen tomar a ellos en su perssona } por que recusan de fazer aquello a que son teriuidos . & et accipere sua , \textbf{ sed eos | etiam accipere in praeda , } ex quo recusant facere quod tenentur . \\\hline
2.3.7 & mas avn deuen tomar a ellos en su perssona \textbf{ por que recusan de fazer aquello a que son teriuidos . } mas por que non es de fazer tuerto & etiam accipere in praeda , \textbf{ ex quo recusant facere quod tenentur . } Verum quia nulli est iniuria facienda , \\\hline
2.3.7 & por que recusan de fazer aquello a que son teriuidos . \textbf{ mas por que non es de fazer tuerto } nin eniuria a ninguno fablado uerdaderamente la uida de robar & ex quo recusant facere quod tenentur . \textbf{ Verum quia nulli est iniuria facienda , } per se loquendo , \\\hline
2.3.7 & mas por que non es de fazer tuerto \textbf{ nin eniuria a ninguno fablado uerdaderamente la uida de robar } e de furtar deuen ser iudgadas desconuenbles & Verum quia nulli est iniuria facienda , \textbf{ per se loquendo , | vita furatiua } uel praedatiua debet illicita iudicari : \\\hline
2.3.7 & nin eniuria a ninguno fablado uerdaderamente la uida de robar \textbf{ e de furtar deuen ser iudgadas desconuenbles } por que cada vno deue beuir delo suyo propreo & vita furatiua \textbf{ uel praedatiua debet illicita iudicari : } debet enim quis de proprio uiuere , \\\hline
2.3.7 & por que cada vno deue beuir delo suyo propreo \textbf{ e non deue tomar lo ageno¶ } Et pues que assi es & debet enim quis de proprio uiuere , \textbf{ non de usurpatione alieni . } Quia igitur decet ipsos ciues , \\\hline
2.3.7 & e alos prinçipes de beuir uida uirtuosa \textbf{ si quieren gouernar conueniblemente en su uida las sus casas propreas } conuiene les de saber & et Principes uiuere uita uirtuosa ; \textbf{ si uolunt domus proprias debite gubernare , } decet eos scire quot sunt uitae , \\\hline
2.3.7 & si quieren gouernar conueniblemente en su uida las sus casas propreas \textbf{ conuiene les de saber } quantas son las uidas & si uolunt domus proprias debite gubernare , \textbf{ decet eos scire quot sunt uitae , } uel quot sunt modi uiuendi , \\\hline
2.3.8 & ca nunça los omes son fartos comunalmente de possessiones e de riquezas \textbf{ por que nunca pueden auer tantas riquezas } que non quisiessen ante muchͣs mas ¶ & nunquam enim communiter homines satiantur possessionibus et diuitiis : \textbf{ nam nunquam possunt tot habere , } quin plura velint . \\\hline
2.3.8 & assi que por ellas cada vno cuyda \textbf{ que podra alcancar aquello que dessea . } por ende los omes non se fartan de riquezas nin de possessions & cum diuitiae maxime videantur hoc efficere , \textbf{ ut per eas quilibet consequi possit quod appetit , } ut melius homines possint explere \\\hline
2.3.8 & por ende los omes non se fartan de riquezas nin de possessions \textbf{ por que meior puedan por ellas conplir } lo que dessean ¶ & ut per eas quilibet consequi possit quod appetit , \textbf{ ut melius homines possint explere } quod volunt , \\\hline
2.3.8 & La segunda razon \textbf{ para prouar esto mismo se toma dela falsa estimaçion } e informaçion dela fin . & quod volunt , \textbf{ diuitiis } et possessionibus non satiantur . \\\hline
2.3.8 & e segunt mesura de aquella fin \textbf{ que ha de alcançar } assi commo si dixiessemos & ea vero quae sunt ad finem , \textbf{ secundum modum et mensuram ipsius finis . } Ut si finis medicinae est sanare , \\\hline
2.3.8 & assi commo si dixiessemos \textbf{ quela fin del fisico es sanar } e en esta fin nunca tiene el fisico mesuta & secundum modum et mensuram ipsius finis . \textbf{ Ut si finis medicinae est sanare , } non posset medicus tantam sanitatem inducere , \\\hline
2.3.8 & e en esta fin nunca tiene el fisico mesuta \textbf{ ca nunca podrie el fisico tanta salud aduzir al enfermo } que si podiesse ser & Ut si finis medicinae est sanare , \textbf{ non posset medicus tantam sanitatem inducere , | quin } ( si esset possibile ) vellet eam maiorem efficere : \\\hline
2.3.8 & que si podiesse ser \textbf{ que avn mayor non gela quasiese dar . } Et pues que assi es el fisico dessea aduzir salud sin mesura e sin fin & quin \textbf{ ( si esset possibile ) vellet eam maiorem efficere : } medicus ergo sanitatem \\\hline
2.3.8 & que avn mayor non gela quasiese dar . \textbf{ Et pues que assi es el fisico dessea aduzir salud sin mesura e sin fin } mas la melezina dessea de dar & ( si esset possibile ) vellet eam maiorem efficere : \textbf{ medicus ergo sanitatem | quasi appetit inducere infinitam , } sed potionem appetit dare \\\hline
2.3.8 & Et pues que assi es el fisico dessea aduzir salud sin mesura e sin fin \textbf{ mas la melezina dessea de dar } segunt manera & quasi appetit inducere infinitam , \textbf{ sed potionem appetit dare } secundum modum \\\hline
2.3.8 & e segunt mesura dela sabud \textbf{ que ha de fazer enł entermo . } Et pues que assi es los omes comunalmente & secundum modum \textbf{ et mensuram sanitatis . } Communiter ergo homines quia falsam aestimationem habent de fine , \\\hline
2.3.8 & e por que cuydan \textbf{ que han de poner su fin e su bien andança } en las riquezas dessean las sin mesura e sin fin . & Communiter ergo homines quia falsam aestimationem habent de fine , \textbf{ et putant ipsum finem in diuitiis esse ponendum , } appetunt eas in infinitum . \\\hline
2.3.8 & Mas que pertenezca al gouernamiento dela casa \textbf{ non dessear las riquesas } e las possessiones sin mesura e sin fin & Sed quod ad gubernationem domus pertineat \textbf{ non appetere infinitas possessiones , } duplici via venari possumus . \\\hline
2.3.8 & e las possessiones sin mesura e sin fin \textbf{ esto podemos mostrar } por dos razones ¶ & non appetere infinitas possessiones , \textbf{ duplici via venari possumus . } Prima sumitur ex similitudine quam habent oeconomica \\\hline
2.3.8 & para nudermiento de la catura \textbf{ si el gouernador dela casa quiere fazer contra natura } mas si quiere gouernar su casa & et lac in uberibus ad nutrimentum foetus : \textbf{ si gubernator domus | non vult contra naturam agere , } sed vult suam domum regere \\\hline
2.3.8 & si el gouernador dela casa quiere fazer contra natura \textbf{ mas si quiere gouernar su casa } segunt manera natural non deue dessear possessiones nin riquezas sin mesura e sin fin . & non vult contra naturam agere , \textbf{ sed vult suam domum regere | secundum modum } et ordinem naturalem , \\\hline
2.3.8 & mas si quiere gouernar su casa \textbf{ segunt manera natural non deue dessear possessiones nin riquezas sin mesura e sin fin . } Mas quando ha tantas riquezas & secundum modum \textbf{ et ordinem naturalem , | non debet infinitas diuitias } et possessiones appetere ; \\\hline
2.3.8 & Et pues que assi es \textbf{ nin el arte del gouernamiento dela casa non deue demandar possessiones et riquezas sin mesura e sin fin . } ¶ Et por ende conuiene a todos los çibdadanos & ut ait Philosophus 1 Politicor’ habet organa infinita , \textbf{ ergo nec gubernatiua debet quaerere infinitas possessiones . } Decet igitur omnes ciues \\\hline
2.3.8 & quantas demanda el menester de su estado \textbf{ ca non se fartar omne de possessiones nin de riquezas } assi commo paresçe & quantas requirit exigentia sui status . \textbf{ Nam non satiari possessionibus } et diuitiis \\\hline
2.3.8 & Et pues que assi es tanto \textbf{ mas es estode demostrar en los Reyes e enlos prinçipes } que en los otros & uel ex inordinatione uoluptatis . \textbf{ Tanto ergo detestabilius est hoc in Regibus , | et Principibus } quam in aliis , \\\hline
2.3.8 & que en los otros \textbf{ quanto mas conuiene aellos de auer mayor ordenamiento dela uoluntad } e meior estimacion dela finca & quam in aliis , \textbf{ quanto decet habere ordinatiorem uoluptatem , } et meliorem aestimationem finis : \\\hline
2.3.8 & assi commo dixiemos enł primero libro \textbf{ mas de denostares enl Rey de non auer uerdadera estimaçonn dela fin } que enl pueblo & nam sicut in primo libro dicebatur , \textbf{ detestabilius est in Rege non habere ueram aestimationem de fine quam in populo , } eo quod populus a Rege dirigitur , \\\hline
2.3.8 & assi commo \textbf{ mas de denostar es enł liallero } de non conosçer la señal & eo quod populus a Rege dirigitur , \textbf{ sicut detestabilius est in sagittante non cognoscere signum , } quam in sagitta : \\\hline
2.3.8 & mas de denostar es enł liallero \textbf{ de non conosçer la señal } que en la saeta & eo quod populus a Rege dirigitur , \textbf{ sicut detestabilius est in sagittante non cognoscere signum , } quam in sagitta : \\\hline
2.3.9 & que estas tales muda connes fuessen puestas en la tiecra \textbf{ deuedes saber } que si non fuesse sinon la comiundat dela casa & Ut ergo sciamus quomodo huiusmodi commutationes oportuit introduci , sciendum quod si non esset \textbf{ nisi communitas domus quae est communitas prima , } nulla commutatio esset necessaria . Nam in domo dominatur paterfamilias , \\\hline
2.3.9 & que es padre dela conpanna dela casa \textbf{ a quien parte nesçe de proueher a todos aquellos daçion } que es fechͣ para acorrer ala mantenençi & nulla commutatio esset necessaria . Nam in domo dominatur paterfamilias , \textbf{ cuius est prouidere omnibus existentibus in domo , } quare omnis commutatio facta ad subiectionem domus fit a patrefamilias , \\\hline
2.3.9 & a quien parte nesçe de proueher a todos aquellos daçion \textbf{ que es fechͣ para acorrer ala mantenençi } e estan enla casa & ø \\\hline
2.3.9 & o por si o por sus procuradores entre medianos . \textbf{ Ca al padre familias parte nesçe dereleuar toda la menguadela casa . } Mas por que non puede ser conpra & uel per se uel per procuratores intermedios , \textbf{ nam ipsius patrisfamilias est totam indigentiam subleuare domesticam . } Sed cum eiusdem ad seipsum \\\hline
2.3.9 & que non abonda ca lentura \textbf{ Por la qual cosa non solamente conuiene alos omes morar } e conuerssar los vnos con los otros & aliquibus abundant partes calidae quibus non abundant frigidae \textbf{ et econuerso . Propter quod non solum oportet communicare } et conuersari ad inuicem homines unius vici , vel unius ciuitatis , aut unius prouinciae , \\\hline
2.3.9 & Por la qual cosa non solamente conuiene alos omes morar \textbf{ e conuerssar los vnos con los otros } los que son de vn uarrio o de vna çibdat o de vna prouinçia & et econuerso . Propter quod non solum oportet communicare \textbf{ et conuersari ad inuicem homines unius vici , vel unius ciuitatis , aut unius prouinciae , } sed \\\hline
2.3.9 & por eltas comiundades sobredichͣs fueron puestas aquellas tres maneras de mudaçonnes \textbf{ ca si ala comuidat de vn barrio o de vna çibdatur llgua manera abastasse la mudaçion delas cosas alas cosas enpero ala comunidat } que hades es en todo el regno conuiene de poner m̃udaçion delas cosas alos dineros & Nam et si ad communitatem vici , \textbf{ vel ciuitatis , | aliquo modo sufficeret commutatio rerum ad res : } tamen ad communicationem \\\hline
2.3.9 & ca si ala comuidat de vn barrio o de vna çibdatur llgua manera abastasse la mudaçion delas cosas alas cosas enpero ala comunidat \textbf{ que hades es en todo el regno conuiene de poner m̃udaçion delas cosas alos dineros } e de los diueros alas cosas & tamen ad communicationem \textbf{ quod habetur in toto regno , | oportuit introduci commutationem rerum ad denarios , } et econuerso . \\\hline
2.3.9 & e de departidas \textbf{ prouinçias conuiene de poner non sola mente mudaçion delas cosas alas cosas } o delas cosas alos dineros & in commutatione tamen , quae est diuersorum regnorum , et prouinciarum , \textbf{ oportuit introduci non solum commutationem rerum ad res , vel rerum ad denarios ; } sed \\\hline
2.3.9 & o delas cosas alos dineros \textbf{ mas avn conuiene de auer mudaçion } e canbio de diueros a dineros & oportuit introduci non solum commutationem rerum ad res , vel rerum ad denarios ; \textbf{ sed } etiam denariorum ad denarios . Antiquitus enim homines \\\hline
2.3.9 & e canbio de diueros a dineros \textbf{ ca en el tienpo antiguo los oens assi commo da a entender el philosofo } en el prim̃o libro delas politicas & etiam denariorum ad denarios . Antiquitus enim homines \textbf{ ( ut satis innuit Philosophus primo Politicorum ) } in simplicitate viuentes \\\hline
2.3.9 & mas tan solamente inudan vnas cosas por otras . \textbf{ Mas commo quier que esta manera atal se podiesse guardar en vn barrio o en vna uilla non se poda guardar conueinblemente } en todo vn regno o en toda vna prouinçia & quae non habentes denariorum usum , \textbf{ solum res ipsas commutant . Hic autem modus forte in uno vico , | vel in una villa obseruari posset : } sed in toto uno regno , \\\hline
2.3.9 & por que son de grand peso \textbf{ non las poderemos leuar } conueniblemente a luengas tierras . & commode \textbf{ ad partes longinquas portari non possunt . } Oportuit ergo inuenire aliquid quod esset portabile , \\\hline
2.3.9 & conueniblemente a luengas tierras . \textbf{ Et pues que assi es conuiene de fablar alguna cosa } que se podiesse leuar & ad partes longinquas portari non possunt . \textbf{ Oportuit ergo inuenire aliquid quod esset portabile , } et quod esset pulchrum , \\\hline
2.3.9 & Et pues que assi es conuiene de fablar alguna cosa \textbf{ que se podiesse leuar } e que fues fermosa e aprouechable & ad partes longinquas portari non possunt . \textbf{ Oportuit ergo inuenire aliquid quod esset portabile , } et quod esset pulchrum , \\\hline
2.3.9 & e que fues fermosa e aprouechable \textbf{ por que se podiessen fallar las uiandas . } Mas entre todas las otras cosas & et utile , \textbf{ pro quo inueniri possent victualia . } Huiusmodi autem maxime est argentum , \\\hline
2.3.9 & e mas aprouechables e mas honrrados \textbf{ que dellos se pueden fazer basos } que son aprouechables & quae inter caetera metalla sunt pulchriora , \textbf{ et sunt utilia , } et honorabilia : \\\hline
2.3.9 & tan solamente segt̃ sus pesos \textbf{ assi que los que quirien auer tunerto de vino conuimeles a dar } tanto de peso de plata o de oro & Primitus ergo inuentae fuerunt commutationes ad metalla solum secundum pondera : \textbf{ ut volentes habere tantum vini , | oportebat dare tantum ponderis argenti , } vel auri , \\\hline
2.3.9 & o avn de otro metal \textbf{ assi commo plazia de establesçer en aquel tienpo alos pueblos e alos Reyes . } Mas por que era cosa guaue en toda conpra & etiam alterius metalli , \textbf{ ut placebat tunc temporis populis | et regibus instituere . } Sed quia difficile erat in omni emptione vel venditione , \\\hline
2.3.9 & que es \textbf{ assi commo vn fiador nuestro por el qual luego podemos rescebir } segunt el ualor de aquellas cosas & et numisma , \textbf{ qui est quasi quidam fideiussor noster , } pro quo statim secundum ipsius valorem \\\hline
2.3.9 & que cunplen \textbf{ para cunplir la mengua dela uida . } Et pues que assi es & pro quo statim secundum ipsius valorem \textbf{ recipere possumus supplentia indigentiam vitae . } In toto ergo uno regno \\\hline
2.3.9 & e que estan en vn logar del regno commo contesçe alas vezes \textbf{ e los que estan en vna parte del regno an de yr } ala otra parte del regno & ø \\\hline
2.3.9 & por ende fue fallado el dinero \textbf{ que es ligero de leuar } e es tal cosa & inuentus fuit denarius , \textbf{ qui est de facili portatilis : } per cuius commutationem victualia inueniuntur . \\\hline
2.3.9 & e es tal cosa \textbf{ que por la mudaçion del se pueden fallar las uiandas } por la qual cosa si la mudacion de las cosas alos dineros & qui est de facili portatilis : \textbf{ per cuius commutationem victualia inueniuntur . } Quare commutatio rerum ad denarios , et econuerso , \\\hline
2.3.9 & por ende conniene \textbf{ que los que mora una en apartadas prouinçias ouiessen de fallar } sin la mudaçion delas cosas alas cosas . & praeter commutationem rerum ad res , \textbf{ et rerum ad numismata , } oportuit inuenire commutationem numismatum ad numismata . \\\hline
2.3.9 & et qual fue la neçessidat \textbf{ para fallar los des e por ende conuiene al sabio padre familias } e al sabio gouernador dela casa & et quae fuit necessitas inuenire denarios . \textbf{ Decet ergo prudentem patremfamilias , } et doctum gubernatorem cognoscere , \\\hline
2.3.9 & e al sabio gouernador dela casa \textbf{ saber } en qual manera fueron falladas tałs mudaconnes e canbios & Decet ergo prudentem patremfamilias , \textbf{ et doctum gubernatorem cognoscere , } quomodo ortae sunt tales commutationes , \\\hline
2.3.9 & e aque siruen \textbf{ por que sabien do esto sepa meior proueer su casa } onueinblemente depues del tractado de las possessiones & et ad quid deseruiunt , \textbf{ ut cognoscendo , melius sciat suae domui prouidere . } Conuenienter post tractatum de possessionibus tractatur \\\hline
2.3.10 & ca aun possessiones \textbf{ e abondar en vino } e entgo es abondar en riquezas e possessiones naturales & nam habere possessiones \textbf{ et abundare vino et frumento , } est abundare in diuitiis naturalibus : \\\hline
2.3.10 & e abondar en vino \textbf{ e entgo es abondar en riquezas e possessiones naturales } mas abondar endmeros & et abundare vino et frumento , \textbf{ est abundare in diuitiis naturalibus : } sed abundare in denariis et numismatibus , \\\hline
2.3.10 & e entgo es abondar en riquezas e possessiones naturales \textbf{ mas abondar endmeros } e en riquezas es abondar en riquezasartifiçiales & est abundare in diuitiis naturalibus : \textbf{ sed abundare in denariis et numismatibus , | ut patet per Philos’ 1 Pol’ } est abundare in diuitiis artificialibus . \\\hline
2.3.10 & mas abondar endmeros \textbf{ e en riquezas es abondar en riquezasartifiçiales } e por ende si el arte presupone la natura & ut patet per Philos’ 1 Pol’ \textbf{ est abundare in diuitiis artificialibus . } Si ergo ars naturam praesupponit , \\\hline
2.3.10 & que son riquezas artifiçiales \textbf{ assi que despues delas possessiones deuemos determinar delas monedas . } Et pues que & ex quibus oriuntur diuitiae naturales , annectitur tractatur de numismatibus , quae sunt diuitiae artificiales . \textbf{ Quare si post possessiones determinandum est de numismatibus : } postquam diximus quae fuit necessitas inuenire \\\hline
2.3.10 & Et pues que \textbf{ assi es despues que dixiemos qual fue la neçessidat de fallar las monedas e los dineros } finca & Quare si post possessiones determinandum est de numismatibus : \textbf{ postquam diximus quae fuit necessitas inuenire } numismata et pecuniam , restat dicere , quot sunt species pecuniatiuae . \\\hline
2.3.10 & finca \textbf{ de dezer quantas son las maneras de los dineros . } Et el philosofo en las politicas & postquam diximus quae fuit necessitas inuenire \textbf{ numismata et pecuniam , restat dicere , quot sunt species pecuniatiuae . } Distinguit autem Philos’ in Poli’ quatuor species pecuniatiuae : \\\hline
2.3.10 & Et el philosofo en las politicas \textbf{ pone quatro maneras de dineros conuiene saber . Natural . } Et canssoria de canbio . & numismata et pecuniam , restat dicere , quot sunt species pecuniatiuae . \textbf{ Distinguit autem Philos’ in Poli’ quatuor species pecuniatiuae : } videlicet naturalem , \\\hline
2.3.10 & Obolostica \textbf{ que quiere dezer maunera } de tornar los dineros en pasta . & obolostaticam , \textbf{ et tacos siue usuram : } his enim quatuor modis possideri consueuit multitudo pecuniae . \\\hline
2.3.10 & que quiere dezer maunera \textbf{ de tornar los dineros en pasta . } Et talzes que es husura¶ & obolostaticam , \textbf{ et tacos siue usuram : } his enim quatuor modis possideri consueuit multitudo pecuniae . \\\hline
2.3.10 & Et talzes que es husura¶ \textbf{ En estas quetro maneras de auer se suele departir la muchedunbre de la moneda e de los dineros } Et pues que assi es La primera manera dela pecunia es dichͣ & et tacos siue usuram : \textbf{ his enim quatuor modis possideri consueuit multitudo pecuniae . } Prima ergo species ipsius pecuniatiuae dicitur esse \\\hline
2.3.10 & Et por ende quando acaesçe \textbf{ que algs han de yr a otros regnos } en los quales son propas aquellas monedas & eo quod non esset propria regioni illi : \textbf{ illis ergo casu euntibus ad regiones illas , } quibus illa numismata erant propria , \\\hline
2.3.10 & en los quales son propas aquellas monedas \textbf{ que lie una contesçe les de resçebir } mas por ellas & quibus illa numismata erant propria , \textbf{ et portantibus numismata illa , accidit eos plus recipere pro numismatibus illis , } quam in partibus propriis : propter quod casu campsoria usi sunt . \\\hline
2.3.10 & e en quales partes se espienden \textbf{ por que despues el arte camiadora artifiçial niente fuesse fechͣ por razon de ganer ardiños } e Mas esta arte pecumatiua de dineros non deue ser dicha natural & quae numismata in quibus partibus expenduntur , \textbf{ ut postea campsoria artificialiter effecta , | esset causa lucrandi pecuniam . } Haec enim pecuniatiua , \\\hline
2.3.10 & ¶La terçera manera del arte pecuniatiua de dineros es obolostica \textbf{ que quiere dezer arte de peso sobrepuiante que por auentura fue fallada assi . } Ca assi commo la massa del metal es partida en los dineros & obolostatica , \textbf{ vel ponderis excessiua : | quae forte sic inuenta fuit . } Nam sicut massa metalli in denarios diuiditur , et imprimitur ibi signum publicum ; \\\hline
2.3.10 & que acaesçe \textbf{ assi commo por fazer uasos o escudiellas } e alguas otras cosas . & sic aliquando aliqua necessitate interueniente , \textbf{ ut propter vasa fienda , } vel propter aliquid aliud , \\\hline
2.3.10 & que dizen obolostica \textbf{ que quieredezer arte meaial } por que se pesassen todos los dineros & confici massam maioris ponderis : \textbf{ ex quo casu ars sumpsit originem , } ut omnes denarii ponderarentur , \\\hline
2.3.10 & que los tornassen en massa \textbf{ porque por esta manera pudiessen auer ganançia } Et pues que assi es esta arte meai & et qui essent maioris ponderis resoluerentur in massam , \textbf{ ut ex hoc lucrum haberi posset . Ars ergo ista obolostatica , siue ponderis excessiua , ex excessu ponderis , } qui inuenitur in denariis sumpsit originem . \\\hline
2.3.10 & por que aquellas oueias engendraron e parieron . \textbf{ Et pues que assi es siꝑ alguno de diez dineros depues de algun } e tienpo quissiesse auer doze la qual cosa faze el arte pecumatiua dela usura & et pepererunt . \textbf{ Si quis ergo ex decem denariis post aliquod tempus vult habere duodecim quod facit pecuniatiua usuraria , ut plane patet , } vult quod denarii illi pariant \\\hline
2.3.10 & Et pues que assi es siꝑ alguno de diez dineros depues de algun \textbf{ e tienpo quissiesse auer doze la qual cosa faze el arte pecumatiua dela usura } assi con moclaramente paresçe & et pepererunt . \textbf{ Si quis ergo ex decem denariis post aliquod tempus vult habere duodecim quod facit pecuniatiua usuraria , ut plane patet , } vult quod denarii illi pariant \\\hline
2.3.10 & e es de loarmas las otrastres \textbf{ segunt el philosofo son con derecho de denostar } por que toda arte & quae est quasi oeconomica et naturaliter , \textbf{ est laudabilis . } Aliae vero tres \\\hline
2.3.10 & Et por ende el arte canssoria \textbf{ que es arte de canbiar } e el arte obolostica & secundum ipsum ) merito vituperantur . \textbf{ Nam omnis ars quae a denariis incipit , } et ad denarios terminatur , \\\hline
2.3.10 & por que estan del todo en los dineros por que comiença en dineros \textbf{ e terminansse en dineros son de denostar } segunt dize el philosofo & quia a pecunia incipiunt , \textbf{ et ad pecuniam terminantur , uituperandae sunt } secundum Philosophum : nimis enim videtur esse denariorum cupidus , \\\hline
2.3.10 & e dela o bolostica \textbf{ que es delas meaias . La usura es de denostar sienpre e en todas cosas } assi commo parezcra enł capitulo & et abolostatica : \textbf{ usura tamen est simpliciter | et in omnibus detestanda , } ut in sequenti capitulo apparebit . Usuras enim nemo exercere debet : campsoria autem , \\\hline
2.3.10 & que se sigiͤ \textbf{ por que ninguno non deue usar delas usuras } Avon la canssoria e la obolostica & ut in sequenti capitulo apparebit . Usuras enim nemo exercere debet : campsoria autem , \textbf{ et obolostatica } et si mercatoribus , \\\hline
2.3.10 & Enpero alos Reyes e alos prinçipes los quales deuen ser medios dioses \textbf{ non los conuiene de usar dellas } saluo dela primera manera pecumatiua & quod decet esse quasi semideos , \textbf{ exercere non congruit . } Nam primam speciem pecuniatiuae , \\\hline
2.3.10 & ca conuiene alos Rey \textbf{ e de abondar en possessiones e en rentas del fact̃o } delas quales pueden abondar en dineros & ø \\\hline
2.3.10 & e de abondar en possessiones e en rentas del fact̃o \textbf{ delas quales pueden abondar en dineros } para defenssion del regno & quae est oeconomica et quasi naturalis , decet . Decet enim ipsos abundare in possessionibus \textbf{ et in redditibus , ex quorum fructu pro defensione regni } et aliis necessariis possunt abundare pecunia . \\\hline
2.3.11 & assi commo ella ha dos nonbres \textbf{ assi podemos prouar } por dos razones & sicut duplici nomine nominatur , \textbf{ sic duplici via inuestigare possumus eam detestabilem esse . } Vocatur enim primo denariorum partus , \\\hline
2.3.11 & por dos razones \textbf{ que ella es de denostar . } Ca primeramente la llamamos parto de dineros & ø \\\hline
2.3.11 & diziendo que es contra natura \textbf{ ca parir e engendrar } e amuchiguar se las cosas en si mismas & ex quo nomine arguit Philosophus 1 Polit’ eam contra naturam esse . \textbf{ Nam parere , | et generare , } et multiplicari in seipsis , \\\hline
2.3.11 & ca parir e engendrar \textbf{ e amuchiguar se las cosas en si mismas } es cosa proprea en las cosas naturales & et generare , \textbf{ et multiplicari in seipsis , } est proprium naturalibus , \\\hline
2.3.11 & lo que dize el ph̃co en el primero libro delas politicas \textbf{ que la usuraes de denostar } por que es contra natura ¶ & quod dicitur 1 Poli’ usuram esse \textbf{ quid detestabile } et contra naturam . \\\hline
2.3.11 & por que es contra natura ¶ \textbf{ Lo segundo podemos mostrar } que esta manera de arte pecumatiua es de denostar & et contra naturam . \textbf{ Secundo huiusmodi pecuniatiuam possumus ostendere detestabilem esse ex alio nomine quo nominatur , } ut quia dicitur usura , \\\hline
2.3.11 & Lo segundo podemos mostrar \textbf{ que esta manera de arte pecumatiua es de denostar } por el otro nonbre & et contra naturam . \textbf{ Secundo huiusmodi pecuniatiuam possumus ostendere detestabilem esse ex alio nomine quo nominatur , } ut quia dicitur usura , \\\hline
2.3.11 & que ha ella quela llaman usura \textbf{ que quiere tanto dezer commo uso de rapina e de robo } por que en la usura se ro ba & ut quia dicitur usura , \textbf{ quod quasi idem est } quod rapina usus . In usura enim usus rapitur et usurpatur , \\\hline
2.3.11 & y que non es suyo dela cosa . \textbf{ Et para esto entender conuiene de saber } que otra cosa es la cosa & qui non est suus . \textbf{ Ad cuius euidentiam sciendum , } quod licet aliud sit \\\hline
2.3.11 & assi commo dezimos que otra cosa es la casa \textbf{ e otra cosa es morar enella } enpero en alguas cosas nunca se puede otorgar el uso dellas & ut aliud est domus , \textbf{ et aliud inhabitare ipsam : } in aliquibus \\\hline
2.3.11 & e otra cosa es morar enella \textbf{ enpero en alguas cosas nunca se puede otorgar el uso dellas } sinon otorgandose la sustançia & et aliud inhabitare ipsam : \textbf{ in aliquibus } tamen nunquam concessio usus separari potest a concessione substantiae . \\\hline
2.3.11 & sinon otorgandose la sustançia \textbf{ ca non se puede partir el uso dela sustançia . } Et por ende en quales se quier cosas & in aliquibus \textbf{ tamen nunquam concessio usus separari potest a concessione substantiae . } In quibuscunque igitur potest concedi usus rei absque eo quod concedatur eius substantia , \\\hline
2.3.11 & Et por ende en quales se quier cosas \textbf{ en que se puede otorgar el uso dela cosa } non se otorgando la sustançia della & tamen nunquam concessio usus separari potest a concessione substantiae . \textbf{ In quibuscunque igitur potest concedi usus rei absque eo quod concedatur eius substantia , } potest inde accipi pensio , \\\hline
2.3.11 & non se otorgando la sustançia della \textbf{ ally en aquella cosa se puede tomar loguer o alquiler della } puesto que aquella cosa se enpeor & In quibuscunque igitur potest concedi usus rei absque eo quod concedatur eius substantia , \textbf{ potest inde accipi pensio , } dato quod res illa in nullo deterioraretur . \\\hline
2.3.11 & assi commo paresçe en las bestias e en las casas . \textbf{ mas si non se puede otorgar el uso dela cosa sin otorgamiento dela sustançia della } por tal uso conmoeste non deuemos tomar alquiler & dato quod res illa in nullo deterioraretur . \textbf{ Sed si non potest concedi usus absque concessione substantiae , } quantum ad talem usum non est pensio aliqua accipienda ; \\\hline
2.3.11 & mas si non se puede otorgar el uso dela cosa sin otorgamiento dela sustançia della \textbf{ por tal uso conmoeste non deuemos tomar alquiler } ca si lo tomassemos serie y usura & Sed si non potest concedi usus absque concessione substantiae , \textbf{ quantum ad talem usum non est pensio aliqua accipienda ; } quia si accipiatur , erit ibi usura , id est , rapina usus . \\\hline
2.3.11 & ca si lo tomassemos serie y usura \textbf{ que quieredezer robo de uso . } Et por ende commo el uso dela casa sea morar en la casa & quantum ad talem usum non est pensio aliqua accipienda ; \textbf{ quia si accipiatur , erit ibi usura , id est , rapina usus . } Cum ergo usus ipsius domus sit domum inhabitare , \\\hline
2.3.11 & que quieredezer robo de uso . \textbf{ Et por ende commo el uso dela casa sea morar en la casa } et non enagenar la casa & quia si accipiatur , erit ibi usura , id est , rapina usus . \textbf{ Cum ergo usus ipsius domus sit domum inhabitare , } non domum alienare ; \\\hline
2.3.11 & Et por ende commo el uso dela casa sea morar en la casa \textbf{ et non enagenar la casa } el señor dela casa puede otorgar el uso dela casa & Cum ergo usus ipsius domus sit domum inhabitare , \textbf{ non domum alienare ; } possessor domorum potest concedere usum domus \\\hline
2.3.11 & et non enagenar la casa \textbf{ el señor dela casa puede otorgar el uso dela casa } para morar & non domum alienare ; \textbf{ possessor domorum potest concedere usum domus } ut inhabitationem absque eo quod concedat substantiam eius : \\\hline
2.3.11 & el señor dela casa puede otorgar el uso dela casa \textbf{ para morar } sin que otorgue la sustançia della . & possessor domorum potest concedere usum domus \textbf{ ut inhabitationem absque eo quod concedat substantiam eius : } et quia cuius est substantia , \\\hline
2.3.11 & e lo que parte nesçe a el \textbf{ conueiblemente lo pue de fazer } non faziendo tuerto a ninguno . & et quod pertinet ad ipsum , \textbf{ licite potest , } et nulli iniuriatur . \\\hline
2.3.11 & assi ca el uso propo de los dineros \textbf{ es despender los } e enagenar los . & ø \\\hline
2.3.11 & es despender los \textbf{ e enagenar los . } Et por ende nunca se puede otorgar el uso propreo de los dineros & et usum eius . In denariis autem non sic : \textbf{ nam usus proprius denariorum , est expendere et alienare denarios nunquam ergo potest concedi usus proprius denarii , } nisi concedatur eius substantia : \\\hline
2.3.11 & e enagenar los . \textbf{ Et por ende nunca se puede otorgar el uso propreo de los dineros } si non se otorgare la sustançia dellos & et usum eius . In denariis autem non sic : \textbf{ nam usus proprius denariorum , est expendere et alienare denarios nunquam ergo potest concedi usus proprius denarii , } nisi concedatur eius substantia : \\\hline
2.3.11 & si non se otorgare la sustançia dellos \textbf{ commo atal uso pertenezca de enagenar la sustançia . Et pues que assi es } por que el açidente desçende dela sustançia del subieto & nisi concedatur eius substantia : \textbf{ cum ad talem usum oporteat | ipsam substantiam alienare . } Quia ergo accidens a substantia dependet , \\\hline
2.3.11 & que cuya es la sustaçia del esalulo della . \textbf{ Et pues que assi es el que quiere tomar ganançia de lisso de los dineros dezimos } que comete usura & eius est usus . \textbf{ Volens ergo accipere pensionem de usu denariorum , } dicitur committere usuram , \\\hline
2.3.11 & que comete usura \textbf{ e tal es dichusurar e robar uso } por que los que otorgan el uso del dinero otorgan la sustançia & dicitur committere usuram , \textbf{ uel dicitur usurpare , | et rapere ipsum usum : } quia concedendo usum denarii , \\\hline
2.3.11 & despues que otorgae la sustaçia del . \textbf{ Enpero deuedes saber } que assi commo dize el pho & ad eum usus denarii , \textbf{ ex quo eius substantiam iam concessit . Aduertendum tamen quod ( ut ait Philos’ 1 Politicorum ) } quasi cuiuslibet rei est duplex usus : \\\hline
2.3.11 & e otro non prop̃oso propreo del dinero \textbf{ es mudar lo o espender lo o enagenarlo . } Vso non prop̃o es paresçer con ellos & Usus proprius denarii , \textbf{ est ipsum commutare , | uel expendere , et alienare . } Usus uero non proprius , est apparere : \\\hline
2.3.11 & es mudar lo o espender lo o enagenarlo . \textbf{ Vso non prop̃o es paresçer con ellos } Ca muchͣs demuestran sus dineros & uel expendere , et alienare . \textbf{ Usus uero non proprius , est apparere : } multi enim ostendunt denarios suos non ad expendendum , \\\hline
2.3.11 & Ca muchͣs demuestran sus dineros \textbf{ non para despender los } mas para paresçer con ellos & Usus uero non proprius , est apparere : \textbf{ multi enim ostendunt denarios suos non ad expendendum , } sed ad apparendum \\\hline
2.3.11 & non para despender los \textbf{ mas para paresçer con ellos } por que semeien ricos . Et avn assi el uso propreo dela casa es morar en lła . & multi enim ostendunt denarios suos non ad expendendum , \textbf{ sed ad apparendum | et ut uideantur diuites . Sic } etiam usus proprius domus , est inhabitare . \\\hline
2.3.11 & mas para paresçer con ellos \textbf{ por que semeien ricos . Et avn assi el uso propreo dela casa es morar en lła . } Et uso noppreo es uender la o canbiar la . & et ut uideantur diuites . Sic \textbf{ etiam usus proprius domus , est inhabitare . | Usus non proprius , } est ipsam uendere , \\\hline
2.3.11 & por que semeien ricos . Et avn assi el uso propreo dela casa es morar en lła . \textbf{ Et uso noppreo es uender la o canbiar la . } Ca muchos fazen casas & Usus non proprius , \textbf{ est ipsam uendere , | uel commutare : } multi enim domos fabricant non ad inhabitandum , \\\hline
2.3.11 & Ca muchos fazen casas \textbf{ non para morar en ellas } mas para vender las . & uel commutare : \textbf{ multi enim domos fabricant non ad inhabitandum , } sed ad uendendum . \\\hline
2.3.11 & non para morar en ellas \textbf{ mas para vender las . } Mas de todo uso & multi enim domos fabricant non ad inhabitandum , \textbf{ sed ad uendendum . } De omni autem usu , \\\hline
2.3.11 & si quier sea propreo \textbf{ si quier non propreo se puede tomar ganançia } commo quier que aquel uso se pueda otorgar & siue sit proprius , \textbf{ siue non proprius , | potest accipi pensio , } si usus ille concedi posset absque concessione substantiae . Propter quod plane patet , \\\hline
2.3.11 & si quier non propreo se puede tomar ganançia \textbf{ commo quier que aquel uso se pueda otorgar } sin otorgamiento dela sustançia . & potest accipi pensio , \textbf{ si usus ille concedi posset absque concessione substantiae . Propter quod plane patet , } econtrario esse de denariis , \\\hline
2.3.11 & Ca del uso p preo de los dineros \textbf{ non se puede tomar ganançia sin usura ca tal uso non se puede otorgar } sin otorgamiento dela sustançia . & et de aliis rebus . \textbf{ Nam de usu proprio denariorum non potest accipi pensio absque usura : } quia talis usus concedi non potest absque concessione substantiae . \\\hline
2.3.11 & sin otorgamiento dela sustançia . \textbf{ Mas del uso non prop̃o puede tomar ganançia sin usura . } assi commo si alguno quesiesse tomar mone das & Nam de usu proprio denariorum non potest accipi pensio absque usura : \textbf{ quia talis usus concedi non potest absque concessione substantiae . } Sed de usu non proprio , potest ; ut si quis vellet numismata non ad expendendum , \\\hline
2.3.11 & Mas del uso non prop̃o puede tomar ganançia sin usura . \textbf{ assi commo si alguno quesiesse tomar mone das } non para despender las & quia talis usus concedi non potest absque concessione substantiae . \textbf{ Sed de usu non proprio , potest ; ut si quis vellet numismata non ad expendendum , } sed ad apparendum : \\\hline
2.3.11 & assi commo si alguno quesiesse tomar mone das \textbf{ non para despender las } mas & quia talis usus concedi non potest absque concessione substantiae . \textbf{ Sed de usu non proprio , potest ; ut si quis vellet numismata non ad expendendum , } sed ad apparendum : \\\hline
2.3.11 & mas \textbf{ para paresçer con ellas . } La qual cosa fazen los rảcadores muchͣs uezes & Sed de usu non proprio , potest ; ut si quis vellet numismata non ad expendendum , \textbf{ sed ad apparendum : } quod forte multotiens mercatores faciunt , \\\hline
2.3.11 & ante ssi muchedunbre de dineros . \textbf{ Et pues que assi es commo alguon pueda otorgar los dineros } para tal uso . & habent coram se multitudinem pecuniae . \textbf{ Cum ergo quis possit concedere denarios } ad talem usum \\\hline
2.3.11 & Conuiene a sabra \textbf{ para paresçer con ellos } sin que non otorgue la sustançia dellos si en esta manera tomassen ganançia de los dineros & ad talem usum \textbf{ videlicet ad apparendum absque eo quod concedat substantiam eorum , } si hoc modo de denariis pensionem acciperet , \\\hline
2.3.11 & por que la morada \textbf{ que es uso propreo dela casa puede se otorgar } sin otorgamiento dela sustançia dela casa & ut in usu proprio non committitur usura , \textbf{ nam quia inhabitatio , quae est proprius usus domus , concedi potest absque concessione substantiae eius ; } si de hoc pensio accipiatur , \\\hline
2.3.11 & la qual cosa non es uso ppreo dela casa \textbf{ puesto que non resçebiesse luego los dineros mas por el uso dela casa quisiesse tomar dineros acometrie usura } por que ya el uso dela casa non parte nesçrie a el & quod non est proprius usus eius , \textbf{ dato quod non statim pecuniam acciperet , | si propter usum domus vellet ulteriorem pecuniam accipere , } usuram committeret : \\\hline
2.3.11 & Et pues que assi es conuiene alos Reyes e a los prinçipeᷤ \textbf{ si quisieren ser señors natalmente de defender las usuras } que non se fagan & qui non spectat ad creditorem , contra naturam est . Decet ergo Reges et Principes , \textbf{ si volunt naturaliter Dominari , | prohibere usuras , } ne fiant eo quod iuri naturali contradicant . \\\hline
2.3.12 & Et esto contesçe \textbf{ que se parue de fazer } assi commo en çinco maneras ¶ & quibus numismata acquiruntur . \textbf{ Contingit enim hoc fieri } quasi quinque viis . \\\hline
2.3.12 & La terçera merçenaria o al que la dera . \textbf{ ¶ la quarta espunental e de praeua¶ } La quinta artifiçial e por artifiçio ¶ & Secunda mercatiua . \textbf{ Tertia mercenaria vel conducta . } Quarta experimentalis . Quinta artificia . Via autem possessoria acquiritur pecunia , \\\hline
2.3.12 & sabien do quales son de mayor fructo \textbf{ e de quales puede meior acorrer ala mengua dela cala . } Et esto le puede fazer & sciendo quae sunt magis fructiferae , \textbf{ et ex quibus potest melius subueniri indigentiae corporali domesticae siue gubernationi domus . Hoc autem fieri contingit , } si sciatur quae in quibus partibus abundant , \\\hline
2.3.12 & e de quales puede meior acorrer ala mengua dela cala . \textbf{ Et esto le puede fazer } si lopieren quales aianlias & sciendo quae sunt magis fructiferae , \textbf{ et ex quibus potest melius subueniri indigentiae corporali domesticae siue gubernationi domus . Hoc autem fieri contingit , } si sciatur quae in quibus partibus abundant , \\\hline
2.3.12 & e quales cosas son aquellas \textbf{ que en aquellas tranrras del gouernador dela casa biue meior se puedan guardar . } Mas esto en qual manera se puede saber & ut quis illis animalibus abundet , \textbf{ quae in partibus illis in quibus existit melius conseruantur . } Haec autem quomodo sciri possint , \\\hline
2.3.12 & que en aquellas tranrras del gouernador dela casa biue meior se puedan guardar . \textbf{ Mas esto en qual manera se puede saber } e en qual manera cada vno se deua auer çerca las possessiones . Et en qual manera las aues e las aialas de quatro pies se pueden guardar . & quae in partibus illis in quibus existit melius conseruantur . \textbf{ Haec autem quomodo sciri possint , } et qualiter circa possessiones quis se habere debeat , \\\hline
2.3.12 & Mas esto en qual manera se puede saber \textbf{ e en qual manera cada vno se deua auer çerca las possessiones . Et en qual manera las aues e las aialas de quatro pies se pueden guardar . } Et qual tr̃ra vale para labrar & Haec autem quomodo sciri possint , \textbf{ et qualiter circa possessiones quis se habere debeat , } ut qualiter aues , \\\hline
2.3.12 & e en qual manera cada vno se deua auer çerca las possessiones . Et en qual manera las aues e las aialas de quatro pies se pueden guardar . \textbf{ Et qual tr̃ra vale para labrar } e qual mas para plantar las uinnas . & Haec autem quomodo sciri possint , \textbf{ et qualiter circa possessiones quis se habere debeat , } ut qualiter aues , \\\hline
2.3.12 & Et qual tr̃ra vale para labrar \textbf{ e qual mas para plantar las uinnas . } Et qual cuydado deuen tomar çerca los arboles & et qualiter circa possessiones quis se habere debeat , \textbf{ ut qualiter aues , } et animalia quadrupedia conseruentur , \\\hline
2.3.12 & e qual mas para plantar las uinnas . \textbf{ Et qual cuydado deuen tomar çerca los arboles } que plantan & et qualiter circa possessiones quis se habere debeat , \textbf{ ut qualiter aues , } et animalia quadrupedia conseruentur , \\\hline
2.3.12 & que plantan \textbf{ propusiemos de pasar todas estas cosas en silençio e callando } por que paresçe que otros muchs tractaron conplidamente de tales cosas & ut qualiter aues , \textbf{ et animalia quadrupedia conseruentur , } et quae terra magis valeat pro agricultura : \\\hline
2.3.12 & ¶ la segunda manera aprouechable \textbf{ para ganar las riquezas } es dichͣ mercaduria & quae magis ad vinearum plantationem , \textbf{ et qualis cura circa arbores sit gerenda , disposuimus silentio pertransire , } eo quod alii de talibus sufficienter tradidisse videntur . Palladius enim multa huiusmodi enarrauit . Secunda via utilis ad pecuniam acquirendam , dicitur esse mercatiua , cum quis per mare aut per terram defert mercationes aliquas , \\\hline
2.3.12 & que llamamos cabdaleria ¶ \textbf{ La terçera manera para ganar riquezas es llamada merçendera o logadera } assi commo quando alguno por esꝑança de merçed o de preçio se aluega & dicitur esse mercenaria \textbf{ vel conducta : } ut cum quis spe mercedis , \\\hline
2.3.12 & assi commo quando alguno por esꝑança de merçed o de preçio se aluega \textbf{ para obrar alguas cosas } en que gana algo ¶ & vel precio conductus aliqua operatur . \textbf{ Quarta via dicitur : | experimentalis : } experimentum enim particularium est . \\\hline
2.3.12 & que fizo talez mi les io vno de los siete sabios \textbf{ que primeramente comneçara a philosofo far . Este mille sio commo fuesse muy pobre } e le denostassen sus amigos diziendol & quod fecit Thales Milesius unus de septem sapientibus , \textbf{ qui primo philosophari coeperunt . | Ipse enim cum esset pauper , } et improperaretur sibi a multis cur philosopharetur , \\\hline
2.3.12 & Mas por que mostrasse \textbf{ que ligera cosa seria alos philosofos de se enrriquesçer signind } cuydado ouiessen delas riquezas . & sed ut ostenderet \textbf{ quod facile esset Philosophis ditari , } si circa talia curam gererent ; \\\hline
2.3.12 & que aquel anero \textbf{ que auie de uenir } que auie de ser grant cunplimiento de oliuas e de olio . & vidit per astronomiam , \textbf{ futuram esse magnam copiam oliuarum : } et ab omnibus incolis regionis illius emit tantum oleum , \\\hline
2.3.12 & Et el por ende conpro todo el olio \textbf{ que auien de coger todos los labradores de aquel regno en el ano } que auie de uenir . & futuram esse magnam copiam oliuarum : \textbf{ et ab omnibus incolis regionis illius emit tantum oleum , } quod recollecturi erant in anno futuro . Mutuata ergo pecunia , et data atra bona pro futuro oleo : \\\hline
2.3.12 & que auien de coger todos los labradores de aquel regno en el ano \textbf{ que auie de uenir . } Et por estarazon demando dineros prestados & et ab omnibus incolis regionis illius emit tantum oleum , \textbf{ quod recollecturi erant in anno futuro . Mutuata ergo pecunia , et data atra bona pro futuro oleo : } tum quia nullus poterat vendere oleum , \\\hline
2.3.12 & et dio bueons fiadores \textbf{ por todo el olio que auie de venir . } Et lo vno & ø \\\hline
2.3.12 & Et lo vno \textbf{ por que ninguon non podie vender olio } si non el solo . & quod recollecturi erant in anno futuro . Mutuata ergo pecunia , et data atra bona pro futuro oleo : \textbf{ tum quia nullus poterat vendere oleum , } nisi ipse : \\\hline
2.3.12 & Et en esto mostro \textbf{ que ligera cosaes alos philosofos de se enrriquesçer } quando quisieren . & et ostendit \textbf{ quod facile erat Philosophis ditari . } Secundum particulare gestum , \\\hline
2.3.12 & quantos quarie \textbf{ por que segunt el philosofo entre todas las cosas que acresçientan las riquezas es fazer monopolia } que quiere dezer vendiconn de vno solo . & Inter caetera autem augentia diuitias \textbf{ ( } secundum Philos’ ) est facere monopoliam , idest facere vendationem unius : \\\hline
2.3.12 & por que segunt el philosofo entre todas las cosas que acresçientan las riquezas es fazer monopolia \textbf{ que quiere dezer vendiconn de vno solo . } Ca quando vno solo uende taxa el preçio & ( \textbf{ secundum Philos’ ) est facere monopoliam , idest facere vendationem unius : } nam \\\hline
2.3.12 & Et por ende el que quiere gana rriqueza \textbf{ conuiene le de tener enla memoria estos fechs particulares } e otros semeiantes & quia unus solus vendit , taxat precium pro suae voluntatis arbitrio : volentem ergo pecuniam acquirere , \textbf{ oportet haec et similia particularia gesta , } per quae aliqui pecuniam sunt lucrati , \\\hline
2.3.12 & Et pues que assi es conuiene a cada vno \textbf{ que quiere proueer ala mengua dela casa } segunt uida politica de auer cuydado de ganar dineros segunt que requiere & Decet ergo quemlibet \textbf{ secundum vitam politicam volentem prouidere indigentiae domesticae , } habere curam de acquisitione pecuniae , \\\hline
2.3.12 & que quiere proueer ala mengua dela casa \textbf{ segunt uida politica de auer cuydado de ganar dineros segunt que requiere } e demanda el su estado de cada vno . & Decet ergo quemlibet \textbf{ secundum vitam politicam volentem prouidere indigentiae domesticae , } habere curam de acquisitione pecuniae , \\\hline
2.3.12 & Mas alos Reyes e alos prinçipes entre las otras maneras dichͣs paresçen dos maneras tan solamente conueinbles \textbf{ para auer dineros . } Conuiene a saber . & et Principes \textbf{ inter vias tactas solae duae viae videntur esse utiles : } videlicet possessoria , \\\hline
2.3.12 & para auer dineros . \textbf{ Conuiene a saber . } La possessoria & et Principes \textbf{ inter vias tactas solae duae viae videntur esse utiles : } videlicet possessoria , \\\hline
2.3.12 & por que conuiene a ellos \textbf{ que por si o por otros ayan prouada de saber las condiconnes particulares del regno } e los fechs particulares de los sus anteçessor ssegunt & vel per se , \textbf{ vel per alios esse expertos , | sciendo particulares conditiones regni , } et gesta particularia praedecessorum suorum , \\\hline
2.3.12 & los quales gana una conueniblemente so dineros . \textbf{ Ca deuen guardar las costunbres del regno } que son de loar segunt las quales acresçenta una sus rentas conueiblemente & et gesta particularia praedecessorum suorum , \textbf{ secundum quae licite pecuniam acquirebant . Debent enim conseruare laudabiles consuetudines regni , } secundum quas capiant licitos redditus , non usurpando aliorum bona . Consideratio ergo gestorum particularium , \\\hline
2.3.12 & Ca deuen guardar las costunbres del regno \textbf{ que son de loar segunt las quales acresçenta una sus rentas conueiblemente } non tomando los bienes de los otros por fuerça . & secundum quae licite pecuniam acquirebant . Debent enim conseruare laudabiles consuetudines regni , \textbf{ secundum quas capiant licitos redditus , non usurpando aliorum bona . Consideratio ergo gestorum particularium , } et consuetudinum approbatarum , \\\hline
2.3.12 & e dela prueua son aprouechables alos Reyes \textbf{ e alos prinçipes en ganar riquezas } avn en esta misma manera es aprouechable la manera de possessiones & et via experimentalis ; \textbf{ utilis est Regibus et Principibus in acquisitione pecuniae . } Sic etiam utilis est via possessionalis non solum in possessionibus immobilibus , \\\hline
2.3.12 & mas avn en las posessiones muebles . \textbf{ Ca conuiene a ellos de resplandesçer } por muchedunbre de bestias e avn de aues . & etiam in possessionibus mobilibus . \textbf{ Decet enim ipsos pollere multitudine bestiarum , } et etiam auium , \\\hline
2.3.12 & por muchedunbre de bestias e avn de aues . \textbf{ por las quales pueden satisfazer ala mengua dela uida . } Ca viemos que el enparador fadrique & Decet enim ipsos pollere multitudine bestiarum , \textbf{ et etiam auium , | per quae satisfaciant indigentiae vitae . } Vidimus enim Federicum Imperatorem , \\\hline
2.3.12 & Enpero con todo esto \textbf{ quaria tomar gouierno de carnes } sienpre tx de sus greyes & habuisse massaritias multas . Non obstante enim quod terrae fertilissimae dominabatur , \textbf{ ubi victualia modici precii existebant ; nihilominus quasi tamen semper ex propriis alimenta carnium volebat assumere , } et hoc tam in bestiis , \\\hline
2.3.12 & Por la qual cosa conuiene alos Reyes \textbf{ e alos principes de auer omes acuçiosos } e sabidores & Qui enim singula alimenta pecunia emit , \textbf{ magis viuit } ut aduena , \\\hline
2.3.12 & assi commo veemos \textbf{ que en algunos logars han costunbres de auer palomares } e muchedunbre de palomas & et aliorum animalium deseruientium ad indigentiam vitae , non solum quadrupedium sed etiam volucrum : \textbf{ sicut alicubi consuetudo est habere multitudinem columbarum } vel aliarum auium , \\\hline
2.3.12 & segunt el philosofo dize \textbf{ enlas politicas deuemos cuydar cerca el labramiento delas tr̃ras } por que deuemos cuydar & ut Philosophus in Polit’ ait , \textbf{ insistendum est circa apum culturam , } si partes illae aptae essent ad talem cultum : \\\hline
2.3.12 & enlas politicas deuemos cuydar cerca el labramiento delas tr̃ras \textbf{ por que deuemos cuydar } si aquellas partes dela tr̃ra & ut Philosophus in Polit’ ait , \textbf{ insistendum est circa apum culturam , } si partes illae aptae essent ad talem cultum : \\\hline
2.3.12 & en que beuimos son ordenadas e apareiadas \textbf{ para resçebir tal labrança . } Otrossi ueemos que delas abeias en algunas trras se cogen muy grand fructo & ø \\\hline
2.3.12 & que la uida pelegnina \textbf{ e auer uiandas de propreo es mas de loar } que çonprar cada vna cosa & et habere alimenta ex propriis , \textbf{ laudabilius est , } quam singula pecunia emere . \\\hline
2.3.12 & e auer uiandas de propreo es mas de loar \textbf{ que çonprar cada vna cosa } por su dinero . ¶ & laudabilius est , \textbf{ quam singula pecunia emere . } Dicebatur supra in hac tertia parte huius secundi libri , \\\hline
2.3.13 & a dixiemos dessuso \textbf{ que en esta terçera parte deste segundo libro auemos de tractar } e de dezir de quatro cosas . & quam singula pecunia emere . \textbf{ Dicebatur supra in hac tertia parte huius secundi libri , } de quatuor esse dicendum : \\\hline
2.3.13 & que en esta terçera parte deste segundo libro auemos de tractar \textbf{ e de dezir de quatro cosas . } Conuiene a saber Delos hedifiçios & Dicebatur supra in hac tertia parte huius secundi libri , \textbf{ de quatuor esse dicendum : } videlicet de aedificiis , \\\hline
2.3.13 & e de dezir de quatro cosas . \textbf{ Conuiene a saber Delos hedifiçios } e delas moradas . & Dicebatur supra in hac tertia parte huius secundi libri , \textbf{ de quatuor esse dicendum : } videlicet de aedificiis , \\\hline
2.3.13 & e determinadas las tres cosas \textbf{ finca de dezer dela quarta } assi commo son los sieruos & Expeditis ergo tribus , \textbf{ restat exequi de quarto , } ut de seruis . Ostendemus enim primo seruitutem aliquam naturalem esse , \\\hline
2.3.13 & Assi commo si muchas uozes fiziess en alguna armonia o concordança de canto . \textbf{ Conuerna de dar y alguna bos } que enssennoreasse sobre las otras & ut si plures voces efficiunt aliquam harmoniam , \textbf{ oportet ibi dare aliquam vocem praedominantem , } secundum quam tota harmonia diiudicatur . \\\hline
2.3.13 & de algun cuerpo mezclado \textbf{ conuiene de dar } ay algun helemento & ø \\\hline
2.3.13 & La segunda razon \textbf{ para prouar esto mismo se toma de aquellas cosas } que ueemos enlas partes de vna aian lia . & Sunt ergo aliqui naturaliter domini , \textbf{ et aliqui naturaliter serui . Secunda via ad inuestigandum hoc idem sumitur } ex his quae videmus in partibus eiusdem animalis : \\\hline
2.3.13 & assi en la poliçia \textbf{ e enla çibdat bien ordenadlos sabios deuen enssennorear } e los non si bios deuen obedesçer & anima dominatur , \textbf{ et corpus obedit : sic in politia bene ordinata sapientes debent dominari , } et insipientes obedire : \\\hline
2.3.13 & e enla çibdat bien ordenadlos sabios deuen enssennorear \textbf{ e los non si bios deuen obedesçer } por que estos son conparados a aquellos & et corpus obedit : sic in politia bene ordinata sapientes debent dominari , \textbf{ et insipientes obedire : } quia hi comparantur ad alios quasi corpus ad animam , \\\hline
2.3.13 & por la sabiduria de los omes \textbf{ la qual non podrian auer } por su propia acuçia . & propter prudentiam hominum , \textbf{ quam ex propria industria consequi non possent . } Expedit ergo eis , \\\hline
2.3.13 & nin razon \textbf{ assi commo las bestias non saben enderesçar } nin gniar assi mismos . & eo quod carentes prudentia nesciant seipsos dirigere : \textbf{ sicut naturale est bestias seruire hominibus , } sic naturale est ignorantes subiici prudentibus expedit enim eis sic esse subiectos , \\\hline
2.3.13 & assi commo las bestias non saben enderesçar \textbf{ nin gniar assi mismos . } Por ende assi commo natural cosa es alas bestias de seruir alos omes & sicut naturale est bestias seruire hominibus , \textbf{ sic naturale est ignorantes subiici prudentibus expedit enim eis sic esse subiectos , } ut per eorum industriam dirigantur \\\hline
2.3.13 & nin gniar assi mismos . \textbf{ Por ende assi commo natural cosa es alas bestias de seruir alos omes } assi cosa naturales & sicut naturale est bestias seruire hominibus , \textbf{ sic naturale est ignorantes subiici prudentibus expedit enim eis sic esse subiectos , } ut per eorum industriam dirigantur \\\hline
2.3.13 & que algs omes en conparaçion de los otros pueden \textbf{ mas fallesçer de uso de razon } que las fenbras de los uarones & quod habet consilium inualidum : \textbf{ cum ergo videamus aliquos homines respectu aliorum plus deficere a rationis usu quam foeminae a viris , } sequitur eos naturaliter esse subiectos . \\\hline
2.3.14 & ssi commo sin el derecho natraal \textbf{ por el bien comun connino de dar } e de fazer alg̃s leyes pointiuas legunt & ut in principio capituli dicebatur . \textbf{ Sicut praeter ius naturale propter commune } bonum oportuit dare leges aliquas positiuas , \\\hline
2.3.14 & por el bien comun connino de dar \textbf{ e de fazer alg̃s leyes pointiuas legunt } las quales se gouernassen los regnos e las çibdades & Sicut praeter ius naturale propter commune \textbf{ bonum oportuit dare leges aliquas positiuas , } secundum quas regentur regna et ciuitates : \\\hline
2.3.14 & segunt la qual los nesçios \textbf{ e sin sabiduria deuen puir a los sabios } . & quod praeter seruitutem naturalem , \textbf{ secundum quam ignorantes debent seruire sapientibus , } esset dare seruitutem legalem , \\\hline
2.3.14 & . \textbf{ es de dar serudunbre legal de ley puesta por los omes } segunt la qual los flacos e los vençidos deuen seruir alos fuertes e alos beçedores . & secundum quam ignorantes debent seruire sapientibus , \textbf{ esset dare seruitutem legalem , | et quasi positiuam , } secundum quam debiles \\\hline
2.3.14 & es de dar serudunbre legal de ley puesta por los omes \textbf{ segunt la qual los flacos e los vençidos deuen seruir alos fuertes e alos beçedores . } Ca decho eslegal & et quasi positiuam , \textbf{ secundum quam debiles | et victi seruirent victoribus } et potentibus . Est enim iustum legale , \\\hline
2.3.14 & segunt que dize el philosofo enl primero libro delas politicas \textbf{ que los uençidos en la fazienda deuen seruir alos uençedores . } Et paresçe que este derecho & et potentibus . Est enim iustum legale , \textbf{ ut recitat Philosophus 1 Politicorum superatos in bello seruire superantibus . } Videtur autem huiusmodi iustum , \\\hline
2.3.14 & que segunt el philosofo es derecho en parte e positiuo \textbf{ que puede auer tres razones . } ¶ la primera se toma de los fazedores delas leyes ¶ & secundum \textbf{ quid et positiuum , triplicem congruitatem habere . } Quarum prima sumitur ex conditoribus legum . Secunda ex defensione patriae . \\\hline
2.3.14 & Ca aquel que ha auentaia en los bienes del cuerpo \textbf{ assi commo en fortaleza o en poderio ciuil deue enssennorear } a aquellos que vençio & Quod enim superans in bonis corporis , \textbf{ ut in fortitudine , | vel in ciuili potentia , } dominetur iis quos debellauit , \\\hline
2.3.14 & que segunt los biens del cuerpo Mas assi commo el dize \textbf{ non es semeiante cosa fazer paresçer la fermosura del alma } e la fermosura del cuerpo ¶ & sed ( ut ait ) \textbf{ non similiter esse facile , } videre pulchritudinem animae , et corporis . \\\hline
2.3.14 & por que los otros oms vençedores serique \textbf{ mas enclinados a matar e a fazer honuçidio } si soperiessen que nigunt pro non aurian de tal uençimiento . & nam propter hanc legem multotiens superati in bello saluantur : \textbf{ homines enim alios debellantes proniores essent ad homicidium , } si scirent se ex eis nullam utilitatem consecuturos ; \\\hline
2.3.14 & que los gana \textbf{ por sieruos quardan los de matar } por enl pro & sed cum cogitant eos acquirere in seruos , \textbf{ reseruant ipsos propter utilitatem quam inde consequi sperant . } Unde seruus \\\hline
2.3.14 & por enl pro \textbf{ que es pan de auer dellos . } ¶ Oude sieruo segunt vna echunologia qͥe redez irgidado & reseruant ipsos propter utilitatem quam inde consequi sperant . \textbf{ Unde seruus } secundum unam etymologiam dicitur a seruando : \\\hline
2.3.14 & por que tales son guardados enla batalla delos vençedores e non los matan \textbf{ por qualos puedan auer en possession } assi conmo a sieruos e acatiuos . & et non occiduntur , \textbf{ ut possint ipsos possidere } quasi seruos \\\hline
2.3.15 & que son siluestres e montaneses \textbf{ et non saben gouernar assi mismos . } Por la qual cosa conuiene & quasi syluestres , \textbf{ et nesciunt seipsos dirigere : } propter quod tales contingit esse naturaliter seruos , \\\hline
2.3.15 & assi commo si algunos fallesciessen enl poderio \textbf{ e non se podiessen defender de los lidiadores } por mandamiento dela ley pregonada & si qui in potentia deficientes ; \textbf{ et non valentes debellantibus resistere , } ex promulgatione legis debent eis seruire et ministrare . Mercenarios vero contingit esse ministros ex conducto : \\\hline
2.3.15 & por mandamiento dela ley pregonada \textbf{ deuen seruir } e miistrar & et non valentes debellantibus resistere , \textbf{ ex promulgatione legis debent eis seruire et ministrare . Mercenarios vero contingit esse ministros ex conducto : } ille enim mercenarius dicitur , \\\hline
2.3.15 & deuen seruir \textbf{ e miistrar } a aquellos que los defienden & ex promulgatione legis debent eis seruire et ministrare . Mercenarios vero contingit esse ministros ex conducto : \textbf{ ille enim mercenarius dicitur , } qui principaliter mercedem intendens , \\\hline
2.3.15 & que sean ministros por alquiler \textbf{ por que aquel es dich merçenario que prinçipalmente entiende resçebir merçed } e por ende obedesçe a aquel de que es para auer merçed . & obsequitur ei , \textbf{ a quo mercedem expectat . } Sed boni et virtuosi sunt ministri \\\hline
2.3.15 & por que aquel es dich merçenario que prinçipalmente entiende resçebir merçed \textbf{ e por ende obedesçe a aquel de que es para auer merçed . } Mas los bueons e los utuosos son siruientes & obsequitur ei , \textbf{ a quo mercedem expectat . } Sed boni et virtuosi sunt ministri \\\hline
2.3.15 & e en tal scruiçio \textbf{ deue cada vno entender algun bien } mas si entiende & tamen in ministerio \textbf{ debet quis intendere bonum : } si autem intendat ibi mercedem aliquam temporalem , \\\hline
2.3.15 & mas si entiende \textbf{ y auer alguna merçed tenporal esto deue ser despues de aquel bien que entiende . } Mas conuiene de dar a ministraçion de alquiler e de amor sin la ministt̃ion natural et segunt ley . & si autem intendat ibi mercedem aliquam temporalem , \textbf{ hoc debet esse ex consequenti . Oportuit autem dare ministrationem conductam et dilectiuam } praeter ministrationem naturalem \\\hline
2.3.15 & y auer alguna merçed tenporal esto deue ser despues de aquel bien que entiende . \textbf{ Mas conuiene de dar a ministraçion de alquiler e de amor sin la ministt̃ion natural et segunt ley . } Ca por que en nos es el appetito corrupto & hoc debet esse ex consequenti . Oportuit autem dare ministrationem conductam et dilectiuam \textbf{ praeter ministrationem naturalem | et secundum legem : } nam quia est in nobis corruptio appetitus , \\\hline
2.3.15 & e de entendimiento \textbf{ non consienten de ministr } nin de seruir & nam ignoratis priuatim bonis animae , \textbf{ non acquiescunt ministrare } et seruire pollentibus prudentia \\\hline
2.3.15 & non consienten de ministr \textbf{ nin de seruir } a aquellos que resplandesçen & non acquiescunt ministrare \textbf{ et seruire pollentibus prudentia } et intellectu . Rursus , \\\hline
2.3.15 & que en todo tienpo de su uida non fazen ningua batalla iusta \textbf{ por que por ella puedan ganar algunos seruientes e sieruos . } Et pues que assi es & quia contingit \textbf{ aliquando plures } etiam ex nobili genere ortos toto tempore vitae suae non agere aliquod iustum bellum , \\\hline
2.3.15 & por el gualardon e por la merçed \textbf{ que dende aurian de resçebir } Et conuiene que ouiessen algunos otros seruientesamadores e uirtuosos & et seruos : \textbf{ ne ergo tales omnino priuentur ministris , ad supplendum indigentiam domesticam oportuit esse aliquos ministros conductos seruientes intuitu mercedis , } et aliquos dilectiuos virtuosos ministrantes ex amore boni . \\\hline
2.3.15 & los quales la uirtud \textbf{ e el amor de bien los inclina asuir . } Conuiene que los prinçipes se ayan çerca ellos & quos virtus \textbf{ et amor boni inclinat | ad seruiendum , } decet principantes se habere quasi ad filios , \\\hline
2.3.15 & assi commo cerca de fijos . \textbf{ Et conuiene les alos prinçipes delos gouernar } non por gouernamiento seruil & decet principantes se habere quasi ad filios , \textbf{ et decet eos regere non regimine seruili , } sed magis quasi paternali et regali . \\\hline
2.3.15 & non por gouernamiento seruil \textbf{ mas por gouernamiento paternal e real Esto podemos mostrar } por dos razones & et decet eos regere non regimine seruili , \textbf{ sed magis quasi paternali et regali . } Possumus autem duplici via ostendere , \\\hline
2.3.15 & La primera razon se prueua assi . \textbf{ Ca sienpre alos mas digunos son de dar los mayores benefiçios . } Et por ende commo el seruiente uirtuoso & ad principantem . Prima via sic patet . \textbf{ Nam dignioribus semper sunt ampliora beneficia tribuenda : cum ergo virtuosus seruiens ex amore honesti , } et ex dilectione boni , \\\hline
2.3.15 & por taçed o por gualatdon \textbf{ que entiende auer . } Ca este tal prinçipalmente sirue & dignior sit mercenario , \textbf{ qui principaliter seruit ex conducto } et ex mercede , \\\hline
2.3.15 & por alquileo o por merçed \textbf{ que entiende auer . } Otrossi es mas digno que el seruiente & qui principaliter seruit ex conducto \textbf{ et ex mercede , } et dignior sit seruiente non amore honesti \\\hline
2.3.15 & que el siruente barburo que sirue \textbf{ por que non sabe gouernar asi mismo } e fanllesçe de vso de razon e de encendimiento . & et omnino sit melior Barbaro qui ministrat , \textbf{ eo quod nesciat seipsum dirigere } et deficiat a rationis usu : \\\hline
2.3.15 & e fanllesçe de vso de razon e de encendimiento . \textbf{ Et por ende ha de fuir a otro } que lo mude gouernar . & eo quod nesciat seipsum dirigere \textbf{ et deficiat a rationis usu : } ex ipsa dignitate ministrantium patet huiusmodi ministros a principante esse magis honorandos \\\hline
2.3.15 & Et por ende ha de fuir a otro \textbf{ que lo mude gouernar . } Et pues que assi espesçe & ø \\\hline
2.3.15 & por la qual cosa . \textbf{ si todo amor ha de auer alguna fuerça de ayuntamiento e de vnidat . . } assi commo quiere dionisio enł . & quam habent huiusmodi ministri ad principantem . Dignum est enim ut partes propinquiores fonti plus profundantur aqua : \textbf{ quare si omnis amor quandam virtutem unitiuam } et coniunctiuam habere dicitur , \\\hline
2.3.16 & Enpero quanto pertenesçe alo psente . \textbf{ tres cosas deuemos pessar en esto . } Conuiene de saber . & ad praesens spectat , \textbf{ tria sunt attendenda , } videlicet ordo ministrandi , facilitas exequendi , et conditio ministrantium . \\\hline
2.3.16 & tres cosas deuemos pessar en esto . \textbf{ Conuiene de saber . } la orden del ministrͣ e del seruir ¶ & ad praesens spectat , \textbf{ tria sunt attendenda , } videlicet ordo ministrandi , facilitas exequendi , et conditio ministrantium . \\\hline
2.3.16 & Conuiene de saber . \textbf{ la orden del ministrͣ e del seruir ¶ } Et la ligere ca de los seruientes ¶ & tria sunt attendenda , \textbf{ videlicet ordo ministrandi , facilitas exequendi , et conditio ministrantium . } Primo enim attendenda sunt officia domestica sic esse committenda ministris , \\\hline
2.3.16 & Et la condiconn de los que siruen ¶ \textbf{ Lo primero deuemos tener mientes } que los ofiçios dela casa sean & videlicet ordo ministrandi , facilitas exequendi , et conditio ministrantium . \textbf{ Primo enim attendenda sunt officia domestica sic esse committenda ministris , } ut reseruetur ibi debitus ordo ministrandi : \\\hline
2.3.16 & por que sea y guardada la orden conuenible del seruiçio \textbf{ la qual cosa se pue de bien conplir } si nunca fuere a comnedado vn ofiçio a muchs oficiales & ut reseruetur ibi debitus ordo ministrandi : \textbf{ quod maxime fieri contingit , } si nunquam officium aliquod committatur ministris pluribus nisi illi plures sint sub aliquo uno , \\\hline
2.3.16 & e non se faga confusamente \textbf{ e desordenadamente conuiene de poner } y vn mayoral & ne illud negligatur , \textbf{ et ne fiat confuse et inordinate , } praeficiendus est unus architector ministris illis , \\\hline
2.3.16 & que sea ordenador e mandador de todos los seruientes \textbf{ a quien parte nezca de acuçiar } e de ordenar todos los otros & praeficiendus est unus architector ministris illis , \textbf{ cuius sit solicitare } et ordinare illos . \\\hline
2.3.16 & a quien parte nezca de acuçiar \textbf{ e de ordenar todos los otros } Et esta regla es muy neçessaria & cuius sit solicitare \textbf{ et ordinare illos . } Est autem hoc documentum maxime necessarium in gubernatione domorum regalium , \\\hline
2.3.16 & en el gouernamiento delas casas de los Reyes \textbf{ En las quales por la grandeza de los offiçios conuiene de acomne dar vn ofiçio a muchos seruientes } por que vno non cunpliria & Est autem hoc documentum maxime necessarium in gubernatione domorum regalium , \textbf{ ubi propter magnitudinem officiorum oportet idem ministerium committi ministris multis , } eo quod unus non sufficeret exequi opus illud . Est igitur in commissione officiorum attendendus ordo ministrandi . Secundo debet ibi attendi facilitas exequendi : \\\hline
2.3.16 & por que vno non cunpliria \textbf{ para fazer aquel oficio e aquella obra . } ¶ Pues que assi es en a comne dar estos ofiçios & ø \\\hline
2.3.16 & para fazer aquel oficio e aquella obra . \textbf{ ¶ Pues que assi es en a comne dar estos ofiçios } deuemos tener mientes ala orden del seruiçio . & ubi propter magnitudinem officiorum oportet idem ministerium committi ministris multis , \textbf{ eo quod unus non sufficeret exequi opus illud . Est igitur in commissione officiorum attendendus ordo ministrandi . Secundo debet ibi attendi facilitas exequendi : } quod maxime fieri contingit , \\\hline
2.3.16 & ¶ Pues que assi es en a comne dar estos ofiçios \textbf{ deuemos tener mientes ala orden del seruiçio . } ¶ Lo segundo deuemos & ubi propter magnitudinem officiorum oportet idem ministerium committi ministris multis , \textbf{ eo quod unus non sufficeret exequi opus illud . Est igitur in commissione officiorum attendendus ordo ministrandi . Secundo debet ibi attendi facilitas exequendi : } quod maxime fieri contingit , \\\hline
2.3.16 & ¶ Lo segundo deuemos \textbf{ y penssar la ligereza de segnir el ofiçio } lo que puede ser fech conueniblemente & ø \\\hline
2.3.16 & e departidos a vna perssona . \textbf{ Et la razon desto se puede tomar } de aquellas cosas & si eidem non committantur \textbf{ officia plura et diuersa . Ratio autem huius haberi potest } ex iis quae dicuntur 4 Polit’ . \\\hline
2.3.16 & que son dichͣs en el quarto libro delas politicas . \textbf{ Ca deuemoos assi ymaginar que comm̃o se ha la guand çibdat ala pequeña . } Assi se ha la grand casa ala pequana . & ex iis quae dicuntur 4 Polit’ . \textbf{ Debemus enim sic imaginari quod sicut se habet magna ciuitas paruam , } sic se habet magna domus ad paruam . \\\hline
2.3.16 & assi commo el dize \textbf{ çerca la fin del quarto librdelas politicas non son de ayuntar muchsofiçios } nin muchos prinçipados avn apssona & sic se habet magna domus ad paruam . \textbf{ In magna enim ciuitate non sunt congreganda officia et principatus , } ita quod eidem committantur diuersa officia , \\\hline
2.3.16 & tan grand cura hananexa e ayuntada \textbf{ que vna perssona non puede conplir } para segnir ligeramente muchs ofiçios . & tantam curam habent annexam , \textbf{ ut eadem persona sufficere non posset } ad faciliter exequendum officia multa . \\\hline
2.3.16 & que vna perssona non puede conplir \textbf{ para segnir ligeramente muchs ofiçios . } Mas en la pequanan çibdat & ut eadem persona sufficere non posset \textbf{ ad faciliter exequendum officia multa . } Sed in parua ciuitate vel in parua villa , \\\hline
2.3.16 & o avn enla pequana uilla \textbf{ do non pueden muchs auer los ofiçios } por la poquedat de los moradores . & Sed in parua ciuitate vel in parua villa , \textbf{ ubi propter habitantium paucitatem non multi possunt praesidere in officiis et ubi officia commissa non magnam curam habent annexam , } congregari possunt officia et magistratus , \\\hline
2.3.16 & pueden se muchos ofiçios \textbf{ e muchos maestradgos ayuntar en vno . } Assi que avna perssona sean acomnedados departidos ofiçios o departidos mahestradgos . & ubi propter habitantium paucitatem non multi possunt praesidere in officiis et ubi officia commissa non magnam curam habent annexam , \textbf{ congregari possunt officia et magistratus , } ita quod eidem diuersa officia committantur . \\\hline
2.3.16 & que es dicho dela çibdat pequanan \textbf{ e grande deue se entender } avn dela casa pequana e grande & Quod ergo dictum est de ciuitate parua \textbf{ et magna , | intelligendum est de parua } et magna domo . \\\hline
2.3.16 & por que podria la casa ser tan pequanan \textbf{ que muy ligeramente vno podria fazer ser seruidor dela mesa } e guardador dela puerta . & posset enim domus esse adeo modica , \textbf{ quod ( et faciliter ) | idem posset esse totus minister mensae , } et custos portae . \\\hline
2.3.16 & e do son muy guatdes ofiçios \textbf{ e han grand cura anexa son en toda manera los ofiçios de partir } e de dar & ubi est multitudo ministrantium , et ubi officia maximam curam habent , \textbf{ sunt omnino officia particulanda , } et distinguenda , \\\hline
2.3.16 & e han grand cura anexa son en toda manera los ofiçios de partir \textbf{ e de dar } a much s ofiçiales . & sunt omnino officia particulanda , \textbf{ et distinguenda , } et non sunt plura committenda eidem , \\\hline
2.3.16 & a much s ofiçiales . \textbf{ Et non son muchs ofiçios de a comne dar a vno } por que non sea enbarguada la ligereza dela enecuçion & et distinguenda , \textbf{ et non sunt plura committenda eidem , } ne impediatur facilitas exequendi . \\\hline
2.3.16 & o el seruiçio del ofiçio ¶ \textbf{ Lo terçero en acomne dar los o siçios es de penssar la con diconn de los seruientes . } por que comunalmente los ofiçiales acostunbraron de fallesçer en dos cosas & ne impediatur facilitas exequendi . \textbf{ Tertio in commissione officiorum consideranda est conditio ministrantium . } Communiter enim ministri in duobus consueuerunt deficere : \\\hline
2.3.16 & Lo terçero en acomne dar los o siçios es de penssar la con diconn de los seruientes . \textbf{ por que comunalmente los ofiçiales acostunbraron de fallesçer en dos cosas } Ca algunos fazen mal los ofiçios & Tertio in commissione officiorum consideranda est conditio ministrantium . \textbf{ Communiter enim ministri in duobus consueuerunt deficere : } nam aliqui male exequuntur opus iniunctum , \\\hline
2.3.16 & mas esto non es por maliçia de uoluntad . \textbf{ ca ellos quarrien de ssi bien seruir } mas esto uiene por enbotamiento del entendimiento & sed non ex malitia voluntatis , \textbf{ quia ipsi de se bonum volunt , } sed hoc contingit \\\hline
2.3.16 & que ellos non engannen \textbf{ nin amenguen los derechs de los sennores . enpero son engannados e menguados en ssi por non lo entender ¶ Et pues que assi es las condiconnes de los seruientes deuen ser tales } que ellos sean fieles e sabios conuiene saber & nec fraudent , decipiuntur \textbf{ tamen et defraudantur . | Conditio ergo ministrantium esse debet , } ut sint fideles , \\\hline
2.3.16 & nin amenguen los derechs de los sennores . enpero son engannados e menguados en ssi por non lo entender ¶ Et pues que assi es las condiconnes de los seruientes deuen ser tales \textbf{ que ellos sean fieles e sabios conuiene saber } que sean fieles & Conditio ergo ministrantium esse debet , \textbf{ ut sint fideles , | et prudentes : } fideles quidem quantum ad rectitudinem voluntatis , \\\hline
2.3.16 & por que non sean engannados \textbf{ por non saber . } Mas la fiesdat se puede conosçer & prudentes vero quantum ad industriam intellectus , \textbf{ ne per insipientiam defraudentur . } Fidelitas autem cognosci habet per diuturnitatem : \\\hline
2.3.16 & por non saber . \textbf{ Mas la fiesdat se puede conosçer } por luengo tienpo & ne per insipientiam defraudentur . \textbf{ Fidelitas autem cognosci habet per diuturnitatem : } ipsum enim cor hominis videre non possumus ; \\\hline
2.3.16 & por luengo tienpo \textbf{ por que nos non podemos ver el coraçon del omne } mas si por luengos tienpos en departidos offiçios & Fidelitas autem cognosci habet per diuturnitatem : \textbf{ ipsum enim cor hominis videre non possumus ; } sed si per diuturna tempora , et in diuersis officiis commissis fideliter se gessit , \\\hline
2.3.16 & si fielmente se ouiere en ellos tal deue ser iudgado por fiel . \textbf{ Mas la sabidina se puede conosçer } por aquellas cosas & ø \\\hline
2.3.16 & e bien acatado e bien aguardado delas cosas \textbf{ que ha de fazer } e si ouiere las o triscosas & ut si aliquis sit memor , prouidus , cautus , et circumspectus , et alia quae ibi diximus , \textbf{ prudens est reputandus : } et secundum magis et minus , \\\hline
2.3.17 & en qual manera cerca esto se de una auer honiradamente \textbf{ e sabiamente deuen uer } e sabrar & et Principes , \textbf{ qualiter circa hoc se habeant honorifice et prudenter : } videndum est qualiter sunt exhibenda indumenta ministris . \\\hline
2.3.17 & e sabiamente deuen uer \textbf{ e sabrar } en qual manera son de dar e departir las vestiduras alos seruientes . & qualiter circa hoc se habeant honorifice et prudenter : \textbf{ videndum est qualiter sunt exhibenda indumenta ministris . } Ad cuius euidentiam sciendum quod circa hoc ( quantum ad praesens spectat ) quinque sunt attendenda , \\\hline
2.3.17 & e sabrar \textbf{ en qual manera son de dar e departir las vestiduras alos seruientes . } e para conosçimiento desto conuieneles de saber & qualiter circa hoc se habeant honorifice et prudenter : \textbf{ videndum est qualiter sunt exhibenda indumenta ministris . } Ad cuius euidentiam sciendum quod circa hoc ( quantum ad praesens spectat ) quinque sunt attendenda , \\\hline
2.3.17 & en qual manera son de dar e departir las vestiduras alos seruientes . \textbf{ e para conosçimiento desto conuieneles de saber } quanto pertenesçe alo presente & videndum est qualiter sunt exhibenda indumenta ministris . \textbf{ Ad cuius euidentiam sciendum quod circa hoc ( quantum ad praesens spectat ) quinque sunt attendenda , } videlicet regis magnificentia , \\\hline
2.3.17 & quanto pertenesçe alo presente \textbf{ que cinco cosas son de penssar en esto . Conuiene a saber la magnificençia et gran dia del Rey . } La ordenança e semeiança de los siruientes . & Ad cuius euidentiam sciendum quod circa hoc ( quantum ad praesens spectat ) quinque sunt attendenda , \textbf{ videlicet regis magnificentia , } uniformitas ministrantium , conditio personarum , consuetudo patriae , \\\hline
2.3.17 & libro \textbf{ conuiene les de auer sus siruientes apareiados conueniblemente } en el parescer de fuera & ut supra in primo libro diffusius probabatur , \textbf{ decet ipsum erga suos ministros decenter } se habere in apparatu debito , \\\hline
2.3.17 & nin por aparesçençia uanas \textbf{ e de una fazer tales cosas . } Enpero por que los Reyes e los prinçipes sean guardados en su estado granado & Nam licet non ad inanem gloriam , \textbf{ nec ad ostentationem talia sint fienda : } tamen ut Reges et Principes conseruent se in statu suo magnifico , \\\hline
2.3.17 & Enpero por que los Reyes e los prinçipes sean guardados en su estado granado \textbf{ e por qua non sean despreçiados de los pueblos conuieneles de fazer grandes } assi commo prueua el philosofo en el sesto libro delas politicas¶ & tamen ut Reges et Principes conseruent se in statu suo magnifico , \textbf{ et ne a populis condemnantur , | decet eos magnifica facere , } ut probat Philosophus 7 Poli’ . Secundo circa vestitum consideranda est uniformitas ministrantium . \\\hline
2.3.17 & assi commo prueua el philosofo en el sesto libro delas politicas¶ \textbf{ Lo segundo çerca las uestidas es de penssar la semeiança de los siruientes . } ca por que parescan los seruientes & decet eos magnifica facere , \textbf{ ut probat Philosophus 7 Poli’ . Secundo circa vestitum consideranda est uniformitas ministrantium . } Nam \\\hline
2.3.17 & ca por que parescan los seruientes \textbf{ que pertenesçen avn prinçipe o avn señor guardada la condiçion delas personas deue entender mayormente } por que sean uestidos ordenadamente e semeiablemente & Nam \textbf{ ut appareant ministri pertinere ad unum Principem siue ad unum dominantem , } reseruata condictione personarum , maxime debet attendi \\\hline
2.3.17 & Lo terçero çerca la prouision delas uestidas \textbf{ es de penssar la condiçion delas personas } por que non conuiene & ut ex conformitate indumentorum cognoscatur eos \textbf{ esse unius Principis ministros . Tertio circa prouisionem indumentorum consideranda est conditio personarum . } Nam non omnes decet habere aequalia indumenta . \\\hline
2.3.17 & que es dios \textbf{ assi commo llanamente lo da a ̧entender el philosofo en łxij̊ dela methaph̃ica } e en esta casa & Dei scilicet , \textbf{ ut plane innuit Philosophus 12 Meta’ } in hac domo non omnia gaudent aequali apparatu , \\\hline
2.3.17 & assi commo vn mundo \textbf{ e por ende non deuen los seruientes gozar egualmente de vn apareiamiento fermoso } nin deuen gozar egualmente de uestiduras fermosas . & quasi quoddam uniuersum dici potest , \textbf{ non omnes ministrantes aequae pulchro apparatu , } nec aeque pulchris indumentis gaudere debent , \\\hline
2.3.17 & e por ende non deuen los seruientes gozar egualmente de vn apareiamiento fermoso \textbf{ nin deuen gozar egualmente de uestiduras fermosas . } Mas penssada la condiçion delas personas & non omnes ministrantes aequae pulchro apparatu , \textbf{ nec aeque pulchris indumentis gaudere debent , } sed considerata conditione personarum sic \\\hline
2.3.17 & Mas penssada la condiçion delas personas \textbf{ assi se deue partir acadera vno dellos } segunt el su estado & sed considerata conditione personarum sic \textbf{ secundum suum statum cuilibet sunt talia tribuenda , } ut in hoc appareat prouidentia \\\hline
2.3.17 & e la sotileza de los Reyes e de los prinçipes ¶ \textbf{ Lo quarto esto es de penssar la costunbre dela tr̃ra catada cosaque non es acostunbrada en alguna tr̃ra paresce } assi commo torpe e desordenada enella & et industria principantis . \textbf{ Quarto circa hoc consideranda est consuetudo patriae . | Nam omne inconsuetum videtur } quasi turpe et inordinatum : \\\hline
2.3.17 & assi commo torpe e desordenada enella \textbf{ por que aquellas cosas que solemos veer en la moçedat } mucħ nos inclinamos aellas & quasi turpe et inordinatum : \textbf{ quae enim a pueritia soliti sumus videre , } nimis afficimur ad illa . \\\hline
2.3.17 & conuersaçiones \textbf{ alguna cosa es de dar ala costunbre dela tierra } si aquellas costunbres non fuessen en toda manera malas corruptas . & et in omnibus conuersationibus aliquid dandum est consuetudini regionum , nisi consuetudines \textbf{ illae sint penitus corruptelae . } Quinto circa hoc considerandum occurrit congruentia temporum . \\\hline
2.3.17 & si aquellas costunbres non fuessen en toda manera malas corruptas . \textbf{ ¶ Lo quinto que es de penssar çerca } desto que son las cosas conuemientes alos tienpos . & illae sint penitus corruptelae . \textbf{ Quinto circa hoc considerandum occurrit congruentia temporum . } Nam cum haec inferiora corpora per super caelestia regantur prout variantur tempora , \\\hline
2.3.17 & e se departe la condicion del mouimiento del çielo \textbf{ assi en tal manera se deuen departir } e mudar las uestiduras & et prout diuersificatur conditio caelestis : \textbf{ potus , cibi , } et indumenta \\\hline
2.3.17 & assi en tal manera se deuen departir \textbf{ e mudar las uestiduras } e las otras cosas & et prout diuersificatur conditio caelestis : \textbf{ potus , cibi , } et indumenta \\\hline
2.3.18 & en lo que se sigue bien commo la iustiçia legales en alguna manera toda uirtud \textbf{ por que la ley manda cunplir toda uirtud } assi la cortesia es en alguna manera toda uirtud & sicut legalis iustitia est quodammodo omnis virtus , \textbf{ quia omnem virtutem lex implere iubet : } sic curialitas est quodammodo omnis virtus , \\\hline
2.3.18 & por que toda uirtud deue en algua manera ser aconpannada ala nobleza delas costunbres . \textbf{ Et nos podemos departir en dos maneras la nobleza ¶ } Vna segunt opinion de los omes & quia nobilitatem morum \textbf{ quasi omnis virtus concomitari debet . Possumus enim distinguere duplicem nobilitatem : } unam secundum opinionem , \\\hline
2.3.18 & assi que non sea memoria en el pueblo que los sus padres nin los sus auuelos fueron pobres \textbf{ e estos tales son dichos auer nobleza de linage } e por ende se sigue & ita quod non sit memoria in populo progenitores suos fuisse pauperes , \textbf{ dicitur habere nobilitates generis , } et per consequens est nobilis \\\hline
2.3.18 & por que muchos oios catan aellos comunalmente mas uerguença toman que los otros \textbf{ e mas desdennan de obrar cosas torpes } que los otros Et pues & communiter plus verecundantur , \textbf{ et dedignantur operati turpia , } quam alii . \\\hline
2.3.18 & ca assi commo dize el philosofo \textbf{ en las politicas la natura quiere sienpte fazer alguna cosa . } Enpero muchͣs uezes non la puede fazer & quia ( ut dicitur in Politicis ) \textbf{ natura vult | qui de hoc facere multotiens , } tamen non potest , \\\hline
2.3.18 & en las politicas la natura quiere sienpte fazer alguna cosa . \textbf{ Enpero muchͣs uezes non la puede fazer } mas fallesçe por algun enbargo & qui de hoc facere multotiens , \textbf{ tamen non potest , } sed deficit . \\\hline
2.3.18 & en essa misma manera son dichos algunos curiales \textbf{ si conueiblemente se ouieren en fazer guaades despessas } la qual cosa es obra de magnifiçençia . En essa misma manera avn los comedores son dichos curiales & Sic et liberales dicuntur , \textbf{ si decenter se habeant in magnis sumptibus , } quod est opus magnificentiae . Sic et curiales dicuntur comestores , \\\hline
2.3.18 & por que todas estas cosas sean aduchos \textbf{ avn dezer } en algua manera la curialidat es toda uirtud & Ut ergo sit ad unum dicere , \textbf{ quodammodo curialitas est omnis virtus . } Idem ergo opus esse potest a virtute speciali , \\\hline
2.3.18 & por que cunpla la ley es iusto legual \textbf{ mas aquel que lo faze por guardar las costunbron dela corte } e de los nobles es dicho curial & iustus legalis erit : \textbf{ qui vero | ut seruet mores curiae } et nobilium , curialis esse dicetur . Sunt enim multi facientes opera virtutum \\\hline
2.3.18 & e esto non lo fazen \textbf{ por que les plaze de despender } nin por que se delecten en dar la gual cosa faze el omne liberal & ø \\\hline
2.3.18 & por que les plaze de despender \textbf{ nin por que se delecten en dar la gual cosa faze el omne liberal } nin otrossi non lo faze & ut bona sua aliis largientes , non agentes hoc quia eis placeat expendere ; \textbf{ nec quod delectentur in dando , | quod facit liberalis ; } nec quod ex hoc velint implere \\\hline
2.3.18 & nin otrossi non lo faze \textbf{ por que quiera cunplir la ley } que lo manda & nec quod ex hoc velint implere \textbf{ legem hoc precipientem , } quod facit iustus legalis : \\\hline
2.3.18 & la qual cosa faze el iusto legal . \textbf{ Mas por que el quiere retener las costunbres dela corte } e de los no nobles omes & quod facit iustus legalis : \textbf{ sed quia volunt retinere mores curiae et nobilium , } quos decet datiuos esse ; \\\hline
2.3.18 & por la qual cosa tales deuen ser dichos cunales . \textbf{ Enpero deuedes saber } que commo quier que la curialidat sea toda uirtud en alguna manera . & quos decet datiuos esse ; \textbf{ propter quod tales curiales dici debent . Aduertendum tamen quod licet curialitas sit omnis virtus , } per quandam tamen antonomasiam largi , \\\hline
2.3.18 & que han con los omes Visto \textbf{ que cosa es la curialidat et la cortesia de ligero puede paresçer } que los seruientes de los Reyes & Nam si decet Reges et Principes \textbf{ eo quod sunt in maximo nobilitatis gradu , } habere mores nobiles et curiales , \\\hline
2.3.18 & assi conuienea los Reyes e alos prinçipes \textbf{ por que son en muy grand grado de nobleza auer buenas costunbres } e de ser curiales e nobles & habere mores nobiles et curiales , \textbf{ ministros , } quos in bonis decet suos dominos imitari , \\\hline
2.3.18 & assi conuiene alos seruientes dellos \textbf{ los que quieren semeiar a sus sennors } de ser buenos e mesurados e corteses . ¶ & ministros , \textbf{ quos in bonis decet suos dominos imitari , } oportet curiales esse . \\\hline
2.3.19 & e delos prinçipes \textbf{ ca los deuen auer nobles e curiales e corteses } ca assi conmo conuiene alos çibdadanos de ser iustos e legales & quales debent esse ministri Regum et Principum , \textbf{ quia debent habere nobiles | et curiales . } Nam sicut decet ciues \\\hline
2.3.19 & ca assi conmo conuiene alos çibdadanos de ser iustos e legales \textbf{ para guardar su poliçia conueniblemente } assi conuiene alos sermient s̃ de los sennores de ser curiales & Nam sicut decet ciues \textbf{ ut debitam politiam seruent } esse iustos legales , \\\hline
2.3.19 & por ende si sabemos quales deuen ser los seruientes fincanos \textbf{ de demostrar } en qual manera los Reyes e los prinçipes & quare si scimus quales oportet esse ministros , \textbf{ restat ostendere qualiter Reges et Principes } et uniuersaliter dominantes erga eos debeant se habere . Debitus autem modus se habendi circa ipsos quasi in quinque videtur consistere . Primo , \\\hline
2.3.19 & se de una auer cerca los seruientes \textbf{ mas la manera conuenible de se auer cerca ellos paresçe } que esta en çinco cosas . & restat ostendere qualiter Reges et Principes \textbf{ et uniuersaliter dominantes erga eos debeant se habere . Debitus autem modus se habendi circa ipsos quasi in quinque videtur consistere . Primo , } ut eis debite officia committantur . Secundo , \\\hline
2.3.19 & que lep̃a los prinçipes \textbf{ en qual manera han de beuir con ellos } lo quarto & ut ad officia commissa debite solicitentur . Tertio , \textbf{ ut sciatur qualiter cum eis est conuersandum . } Quarto , \\\hline
2.3.19 & lo quarto \textbf{ en qual manera son de poner los conseios en ellos } e en qual manera les son de descobrar las poridades ¶ & Quarto , \textbf{ qualiter ipsis communicanda consilia } et aperienda sunt secreta . \\\hline
2.3.19 & en qual manera son de poner los conseios en ellos \textbf{ e en qual manera les son de descobrar las poridades ¶ } Lo quinto e lo postrimero conuiene de saber & qualiter ipsis communicanda consilia \textbf{ et aperienda sunt secreta . } Quinto et ultimo oportet cognoscere , \\\hline
2.3.19 & e en qual manera les son de descobrar las poridades ¶ \textbf{ Lo quinto e lo postrimero conuiene de saber } en qual manera los señores les han de fazer bien & et aperienda sunt secreta . \textbf{ Quinto et ultimo oportet cognoscere , } qualiter sunt beneficiandi , \\\hline
2.3.19 & Lo quinto e lo postrimero conuiene de saber \textbf{ en qual manera los señores les han de fazer bien } e en qual manera les deuen fazer grans ¶ & Quinto et ultimo oportet cognoscere , \textbf{ qualiter sunt beneficiandi , } et quomodo sunt eis gratiae impendendae . Primum autem horum aliquo modo est ex habitis manifestum . \\\hline
2.3.19 & en qual manera los señores les han de fazer bien \textbf{ e en qual manera les deuen fazer grans ¶ } Mas lo primero destas cosas en alguna manera es ya magnifiesto & qualiter sunt beneficiandi , \textbf{ et quomodo sunt eis gratiae impendendae . Primum autem horum aliquo modo est ex habitis manifestum . } nam si minister constituendus in aliquo officio vel in aliquo magistratu debet esse fidelis \\\hline
2.3.19 & por las cosas ya dichos \textbf{ ca si el seruiente es de poner en algun ofiçio o en algun mahestradgo deue ser fiel } por que non enganne e sabio & et quomodo sunt eis gratiae impendendae . Primum autem horum aliquo modo est ex habitis manifestum . \textbf{ nam si minister constituendus in aliquo officio vel in aliquo magistratu debet esse fidelis } ne defraudet , \\\hline
2.3.19 & e de su sabiduria tantol parte nesçen \textbf{ e le son de acomendar mayores ofiçios } ca ninguno non pue de auer praena conplidamente dela bondat e dela fieldat de otro & quanto plus constat de eius fidelitate et prudentia , \textbf{ tanto sunt ei maiora officia committenda . nullus autem ad plenum potest experiri de beniuolentia } et fidelitate alterius nisi per diuturnitatem temporis , \\\hline
2.3.19 & e le son de acomendar mayores ofiçios \textbf{ ca ninguno non pue de auer praena conplidamente dela bondat e dela fieldat de otro } si non fuere & quanto plus constat de eius fidelitate et prudentia , \textbf{ tanto sunt ei maiora officia committenda . nullus autem ad plenum potest experiri de beniuolentia } et fidelitate alterius nisi per diuturnitatem temporis , \\\hline
2.3.19 & assi commo en prouerbio \textbf{ que non se puede saber la condiçion del omne } fasta que sea despendidos muchos moyos de sal . & in prouerbio est dicere , \textbf{ Non est scire ad inuicem , } priusquam simul multos modios salis consumant . \\\hline
2.3.19 & En essa misma manera avn dela sabiduria de alguno non pue de ser el omne conplidamente çierto \textbf{ si non le uieremos fazer las cosas sabiamente } por luengo tienpo . & Sic etiam et de prudentia alicuius plene non constat , \textbf{ nisi per diuturnum tempus viderimus ipsum prudenter egisse . } Hoc ergo modo sunt ministris officia committenda , \\\hline
2.3.19 & que ellos bien se ayan en aquellos ofiçios menores \textbf{ que les puedan acomne dar otros mayores ofiçios . } Enpero deuedes tener mientes con grand diligençia & et prudentia experimentum habeatur : \textbf{ quod si contingat eos bene se habere in illis , poterunt eis ulteriora committi . Est tamen diligenter considerandum , } quod quia mores nuper ditatorum , \\\hline
2.3.19 & que les puedan acomne dar otros mayores ofiçios . \textbf{ Enpero deuedes tener mientes con grand diligençia } que por que las costunbres & quod si contingat eos bene se habere in illis , poterunt eis ulteriora committi . Est tamen diligenter considerandum , \textbf{ quod quia mores nuper ditatorum , } et de nouo ascendentium ad altum statum , \\\hline
2.3.19 & otrossi los que son puestos \textbf{ por mayores en los ofiçios son tomados de villoguar } puesto que en algunos pequanos maestradgos e ofiçios parezçan sabios e que se han sabia mente e fielmente & ut si praepositi officiorum \textbf{ ex vili genere sunt assumpti , } dato quod in aliquibus paruis magistratibus videantur prudenter \\\hline
2.3.19 & puesto que en algunos pequanos maestradgos e ofiçios parezçan sabios e que se han sabia mente e fielmente \textbf{ enpero non se deue luego tomar } para otros ofiçios mas altos & et fideliter se gessisse , \textbf{ non statim assumendi sunt ad officia nimis alta , } sed gradatim \\\hline
2.3.19 & para otros ofiçios mas altos \textbf{ mas deuen sobir de guado enguado } e deue ser tomada experiençia & non statim assumendi sunt ad officia nimis alta , \textbf{ sed gradatim } et per diuturna tempora est habenda de ipsis experientia , \\\hline
2.3.19 & que ellos suban atan al cogrado . \textbf{ ¶ Visto commo son de a comne dar los ofiçios alos ofiçiales } finca deuer en qual manera son de acuçiar & prius quam ad aliud altum ascendant . \textbf{ Viso quomodo ministris sunt officia committenda , } restat videre quomodo sunt in commissis officiis solicitandi . Per se enim ipsos habere curam \\\hline
2.3.19 & ¶ Visto commo son de a comne dar los ofiçios alos ofiçiales \textbf{ finca deuer en qual manera son de acuçiar } por que cunplan bien sus ofiçios & Viso quomodo ministris sunt officia committenda , \textbf{ restat videre quomodo sunt in commissis officiis solicitandi . Per se enim ipsos habere curam } et solicitudinem de ministris , \\\hline
2.3.19 & nin a Reyes \textbf{ nin a prinçipes de auer cuydado } por si mismos & non decet magnos dominos , nec Reges et Principes . \textbf{ Nam ( ut dicitur primo Polit’ ) } et hoc non habet aliquod magnum \\\hline
2.3.19 & que han tan grand poder \textbf{ por que esto puedan escusar el su procurador } deue tomar esta carga e esta honrra & ut hoc vitent , \textbf{ procurator accipit hunc honorem : } ipsi vero ciuiliter viuunt , \\\hline
2.3.19 & por que esto puedan escusar el su procurador \textbf{ deue tomar esta carga e esta honrra } e ellos deuen beuir çiuilmente & ut hoc vitent , \textbf{ procurator accipit hunc honorem : } ipsi vero ciuiliter viuunt , \\\hline
2.3.19 & deue tomar esta carga e esta honrra \textbf{ e ellos deuen beuir çiuilmente } e darse a grandes cosas o a sabiduria . & procurator accipit hunc honorem : \textbf{ ipsi vero ciuiliter viuunt , } aut philosophantur . \\\hline
2.3.19 & e ellos deuen beuir çiuilmente \textbf{ e darse a grandes cosas o a sabiduria . } ¶ Et pues que assi es alos Reyes e alos prinçipes & ipsi vero ciuiliter viuunt , \textbf{ aut philosophantur . } Reges ergo et Principes \\\hline
2.3.19 & ¶ Et pues que assi es alos Reyes e alos prinçipes \textbf{ alos quales couiene de auer altos coraçones } conuiene les de obrar pocas cosas e grandes & Reges ergo et Principes \textbf{ quos decet esse magnanimos decet operari pauca et magna , } ut decet ipsos solicitari \\\hline
2.3.19 & alos quales couiene de auer altos coraçones \textbf{ conuiene les de obrar pocas cosas e grandes } ca les conuiene de ser acuçiosos çerca aquellas cosas que derechamente parte nesçen al bien comun e al gouernamiento del regno & Reges ergo et Principes \textbf{ quos decet esse magnanimos decet operari pauca et magna , } ut decet ipsos solicitari \\\hline
2.3.19 & nin cerca de sus ofiçios \textbf{ nin se deuen entremeter de quales quier ofiçiales } nin de sus ofiçios & et ad regnum regni : \textbf{ quae autem sunt illa in tertio Libro patebit . Solicitari vero circa quosdam ministros } et velle se de quibuscumque inimicis intrommittere , nullatenus decet ipsos . Hoc viso restat ostendere tertium , \\\hline
2.3.19 & ca esto ꝑtenesçe alos menores . \textbf{ ¶ Esto iusto finça de demostrar lo terçero } que es en qual manera han de beuir los sennores con sus ofiçiales & quae autem sunt illa in tertio Libro patebit . Solicitari vero circa quosdam ministros \textbf{ et velle se de quibuscumque inimicis intrommittere , nullatenus decet ipsos . Hoc viso restat ostendere tertium , } videlicet qualiter cum ipsis sit conuersandum . \\\hline
2.3.19 & ¶ Esto iusto finça de demostrar lo terçero \textbf{ que es en qual manera han de beuir los sennores con sus ofiçiales } la qual cosa muestra el philosofo en alguna manera & et velle se de quibuscumque inimicis intrommittere , nullatenus decet ipsos . Hoc viso restat ostendere tertium , \textbf{ videlicet qualiter cum ipsis sit conuersandum . } Quod aliquomodo tradit Philosophus 4 Ethi’ \\\hline
2.3.19 & e de altos coraçones \textbf{ de se auer tenpradamente alos hunul lodos } mas a aquellos que son en grandes dignidades los magnanimos se deuen mostrar grandes . & ubi vult , \textbf{ quod ad humiles decet magnanimos se habere moderate , } sed ad eos \\\hline
2.3.19 & de se auer tenpradamente alos hunul lodos \textbf{ mas a aquellos que son en grandes dignidades los magnanimos se deuen mostrar grandes . } ¶ Et pues que assi es los Reyes e los prinçipes & quod ad humiles decet magnanimos se habere moderate , \textbf{ sed ad eos | qui sunt in dignitatibus decet magnanimos ostendere se magnos . } Reges ergo et Principes , \\\hline
2.3.19 & alos quales conuiene de ser magn animos \textbf{ deuen se mostrar tonprados } a los sus seruientes propreos & qui respectu eorum sunt inferiores et humiles , \textbf{ debent se ostendere moderatos : } quia erga eos velle se habere in nimia excellentia \\\hline
2.3.19 & los quales en conparacion dellos son humillosos e baxos \textbf{ por que contra ellos non se deuen mostrar en grand alteza } assi commo dize el philosofo llanamente & quia erga eos velle se habere in nimia excellentia \textbf{ ( } ut plane tradit Philosophos ) \\\hline
2.3.19 & mas es cosa de grand carga . \textbf{ Mas que en tales cosas de uemeros tener el medio e manera e ser tenprados puedese tomar de aquellas cosas } que son dichͣs en el quinto libro delas politicas & sed onerosum . \textbf{ Quid sit autem tenere in talibus medium , | et esse moderatum , } sumi potest \\\hline
2.3.19 & do dize \textbf{ que la persona del prinçipe non deue paresçer cruel } mas honrrada e mesurada . & ubi dicitur , \textbf{ quod persona Principis non debet apparere seuera , } sed reuerenda . \\\hline
2.3.19 & Et pues que assi es non conuiene al prinçipe \textbf{ de se fazer tan familiar alos sus siruientes } por que sea despreçiado de los & Non ergo decet Principem \textbf{ tam familiarem se exhibere ministris , } ut habeatur in contemptu , \\\hline
2.3.19 & e que non aparesca persona honrrada e de reuerençia ¶ \textbf{ Etrossi non se deue mostrar tantlto el su estado } por que en toda manera parezca cruel & et ut non appareat persona reuerenda : \textbf{ nec debet se sic excellentem ostendere , } ut omnino appareat austerus \\\hline
2.3.19 & por que en todas estas cosas assi commo es dicho en las ethicas \textbf{ el medio es de loar } e los estremos son de denostar . & ut omnino appareat austerus \textbf{ et onerosus . In omnibus enim ( ut traditur in Ethi’ ) medium laudatur , } et extrema vituperantur . Est tamen aduertendum , \\\hline
2.3.19 & el medio es de loar \textbf{ e los estremos son de denostar . } Enpero es de penssar & et onerosus . In omnibus enim ( ut traditur in Ethi’ ) medium laudatur , \textbf{ et extrema vituperantur . Est tamen aduertendum , } quod aliqua familiaritas esset laudabilis in ciue vel milite , \\\hline
2.3.19 & e los estremos son de denostar . \textbf{ Enpero es de penssar } que algua familiaridat es de loar en el çibdadano & et onerosus . In omnibus enim ( ut traditur in Ethi’ ) medium laudatur , \textbf{ et extrema vituperantur . Est tamen aduertendum , } quod aliqua familiaritas esset laudabilis in ciue vel milite , \\\hline
2.3.19 & Enpero es de penssar \textbf{ que algua familiaridat es de loar en el çibdadano } e en el cauallero & et extrema vituperantur . Est tamen aduertendum , \textbf{ quod aliqua familiaritas esset laudabilis in ciue vel milite , } quae non esset laudabilis in Rege : omnino enim decet Reges et Principes minus se exhibere quam caeteros , \\\hline
2.3.19 & e en el cauallero \textbf{ que non es de loar en el Rey e en el prinçipe } por que en toda manera conuiene alos Reyes & quod aliqua familiaritas esset laudabilis in ciue vel milite , \textbf{ quae non esset laudabilis in Rege : omnino enim decet Reges et Principes minus se exhibere quam caeteros , } et ostendere se esse personas \\\hline
2.3.19 & que los otros \textbf{ e de se mostrar } que son personas mas pesadas & quae non esset laudabilis in Rege : omnino enim decet Reges et Principes minus se exhibere quam caeteros , \textbf{ et ostendere se esse personas } magis graues \\\hline
2.3.19 & mas por que non sea despreçiada la dignidat real mostrado \textbf{ en qual manera se deuen dar los ofiçios a los seruientes } e en qual manera deuen ser acuçiosos çerca ellos & non ad ostentationem , \textbf{ sed ne regia dignitas contemnatur . Ostenso qualiter ministris sunt officia committenda , } et quomodo sunt erga eos solicitandi , \\\hline
2.3.19 & e en qual manera deuen ser acuçiosos çerca ellos \textbf{ e en qual manera deuen conuersar } e veuir con ellos & et quomodo sunt erga eos solicitandi , \textbf{ et qualitater est cum ipsis conuersandum : } reliquum est ostendere , \\\hline
2.3.19 & e en qual manera deuen conuersar \textbf{ e veuir con ellos } ¶ finca de demostrar & et quomodo sunt erga eos solicitandi , \textbf{ et qualitater est cum ipsis conuersandum : } reliquum est ostendere , \\\hline
2.3.19 & e veuir con ellos \textbf{ ¶ finca de demostrar } en qual manera deuen tractar sus conseios con ellos & et qualitater est cum ipsis conuersandum : \textbf{ reliquum est ostendere , } qualiter sunt eis communicanda consilia , \\\hline
2.3.19 & ¶ finca de demostrar \textbf{ en qual manera deuen tractar sus conseios con ellos } e en qual manera deuen dar los benefiçios a ellos & reliquum est ostendere , \textbf{ qualiter sunt eis communicanda consilia , } et quomodo sunt eis beneficia exhibenda . Dicebatur enim supra , \\\hline
2.3.19 & en qual manera deuen tractar sus conseios con ellos \textbf{ e en qual manera deuen dar los benefiçios a ellos } ca dicho es de suso & qualiter sunt eis communicanda consilia , \textbf{ et quomodo sunt eis beneficia exhibenda . Dicebatur enim supra , } diuersa esse maneries seruientium , \\\hline
2.3.19 & por amor e por delectaçion del cuerpo \textbf{ mas que por merçed alguaque ende espaauer . } Et pues que assi es commo alos que son sieruos naturalmente & et dilectio Principis magis quam merces aliqua \textbf{ quam inde habituri essent . } Cum seruis ergo naturaliter non sunt communicanda secreta neque consilia : \\\hline
2.3.19 & Et pues que assi es commo alos que son sieruos naturalmente \textbf{ non londe descobrar las poridades nin los conseios } por que assi commo es dicho dessuso & quam inde habituri essent . \textbf{ Cum seruis ergo naturaliter non sunt communicanda secreta neque consilia : } quia ( ut supra dicebatur ) \\\hline
2.3.19 & e aquel que fallesçe de uso de razon e de entendimiento \textbf{ e atal non deuen descobrir sus poridades . } Mas fablando elpho destas cosas & nisi sit inscius , \textbf{ et ab usu rationis deficiat . } De his autem loquens Philosophus circa finem primi Politicorum ait , \\\hline
2.3.19 & e por merçed \textbf{ que atienden non son de descobrir los conseios } nin las poridades & quam ex amore . \textbf{ Nec etiam mercenariis sunt communicanda secreta : } quia tales non personam Principis , \\\hline
2.3.19 & por tpoluengo \textbf{ que ellos son beniuolos e fieles e sabios atales podran descobrir sus poridades mas o menos seg̃t } que dela fieldat dellos & et prudentes , \textbf{ poterunt aperire secreta magis , } et minus , \\\hline
2.3.19 & mas en qual manera \textbf{ por los trabaios son de dar a los seruientes los bñfiçios . } la qual cosa era postrimera de tractar & et prudentia , \textbf{ personae Principis plus constabit . Qualiter autem pro laboribus sunt beneficia exhibenda ministris , } quod ultimo dicebatur esse tractandum , de leui potest patere . \\\hline
2.3.19 & por los trabaios son de dar a los seruientes los bñfiçios . \textbf{ la qual cosa era postrimera de tractar } de ligero puede paresçerca los Reyes e los prinçipes & personae Principis plus constabit . Qualiter autem pro laboribus sunt beneficia exhibenda ministris , \textbf{ quod ultimo dicebatur esse tractandum , de leui potest patere . } Nam Reges et Principes , \\\hline
2.3.19 & de ligero puede paresçerca los Reyes e los prinçipes \textbf{ en cuyo poderio es departir muchͣs } cosas & Nam Reges et Principes , \textbf{ in quorum potestate est multa tribuere , } mercedes ministrorum retinere non debent : \\\hline
2.3.19 & cosas \textbf{ non deuen retener la merçed } e el gualardon de los siruientes & in quorum potestate est multa tribuere , \textbf{ mercedes ministrorum retinere non debent : } sed debent ipsis maiora \\\hline
2.3.19 & e el gualardon de los siruientes \textbf{ mas deuen les partir mayores e menores bñfiçios } segunt que les paresçiere & mercedes ministrorum retinere non debent : \textbf{ sed debent ipsis maiora | et minora beneficia tribuere , } prout apparebit ipsos minus vel amplius meruisse . \\\hline
2.3.20 & que digamos en qual manera en las mesas de los prinçipes \textbf{ se de una auer enl fablar tan bien los Reyes e los prinçipes } e aquellos que estan assentados con ellos & restat ut dicamus qualiter in mensis . \textbf{ Principum circa eloquia habere se debeant | tam ipsi Reges et Principes } et omnes recumbentes \\\hline
2.3.20 & commo avn aquellos que los siruen . \textbf{ Mas podemos mostrar por dos razones } que non conuiene de fablar mucho en las mesas de los Reyes & cum eis , \textbf{ quam etiam ministrantes . Possumus autem duplici via ostendere , } quod non decet in mensis Regum et Principum \\\hline
2.3.20 & Mas podemos mostrar por dos razones \textbf{ que non conuiene de fablar mucho en las mesas de los Reyes } nin de los prinçipes & quam etiam ministrantes . Possumus autem duplici via ostendere , \textbf{ quod non decet in mensis Regum et Principum } et uniuersaliter omnium nobilium \\\hline
2.3.20 & quando vn estrumento es ordenado a vna obra \textbf{ por el qual estrumento se deue fazer } e acabar aquella obra & tunc secundum naturam unumquodque perficitur , \textbf{ quando unum organum ordinatur } ad unum opus . \\\hline
2.3.20 & por el qual estrumento se deue fazer \textbf{ e acabar aquella obra } ca en las obras dela natura non deue ser confusion & quando unum organum ordinatur \textbf{ ad unum opus . } Immo quia in operibus naturae non debet esse confusio , \\\hline
2.3.20 & para dos obras de natura \textbf{ assi conmoparagostar e para fablar . } Et por ende contra orden natural es & lingua congruat in duo opera naturae , \textbf{ ut in gustum , } et locutionem contra naturalem ordinem est , \\\hline
2.3.20 & La qual cosa se faze enla mesa \textbf{ en el tienpo del comer } si estonçe non quedaremos de aquella obra & ad illud opus quod est gustare , \textbf{ quod sit in mensa tempore comestionis , } si tunc temporis non cessemus ab opere alio quod est loqui , \\\hline
2.3.20 & si estonçe non quedaremos de aquella obra \textbf{ que es el fablar contradize ala orden natural . } Lo segundo contradize ala bondat delas costunbres . & si tunc temporis non cessemus ab opere alio quod est loqui , \textbf{ repugnat ergo hoc ordini naturali . } Secundo repugnat bonitati morum . \\\hline
2.3.20 & Lo segundo contradize ala bondat delas costunbres . \textbf{ Ca si los que se assientan enla mesa sedana mucho fablar } por que paresçe & Secundo repugnat bonitati morum . \textbf{ Nam si recumbentes in mensa erga multiloquium insistant , } quia vinum videtur eloquia multiplicare , \\\hline
2.3.20 & alos quales conuiene ser muy tenprados \textbf{ e guardar la orden natural en toda meranera } deuen ordenar en sus mesas & et Principes , \textbf{ quos decet maxime temperatos esse , et obseruare ordinem naturalem omnino in suis mensis , } ordinare debent \\\hline
2.3.20 & e guardar la orden natural en toda meranera \textbf{ deuen ordenar en sus mesas } que los que se assentaren non se estiendan a fablar mucho . & quos decet maxime temperatos esse , et obseruare ordinem naturalem omnino in suis mensis , \textbf{ ordinare debent } ut recumbentes in multiloquia non prorumpant decet \\\hline
2.3.20 & deuen ordenar en sus mesas \textbf{ que los que se assentaren non se estiendan a fablar mucho . } Et avn esto mismo conuiene generalmente a todos los çibdadanos & ordinare debent \textbf{ ut recumbentes in multiloquia non prorumpant decet } etiam hoc uniuersaliter nobiles \\\hline
2.3.20 & Et avn esto mismo conuiene generalmente a todos los çibdadanos \textbf{ por que avn cosa conuenible es aellos de auer uirtudes } e bueans costunbres . & et omnes ciues , \textbf{ quia congruum est etiam et ipsos participare virtutes } et bonos mores . \\\hline
2.3.20 & e bueans costunbres . \textbf{ Mas alos que son assentados en las mesas conuiene de escusar muchedunbre de palabras } por que non sea tirada la ordenn natural & et bonos mores . \textbf{ Sed si recumbentes , } ne tollatur naturalis ordo , \\\hline
2.3.20 & assi commo dicho es . \textbf{ Avn esto mismo conuiene alos seruientes por que la orden e la manera del seruir non sea despreçiada nin enbargada . } Mas si alas mesas de los Reyes & decet \textbf{ etiam hoc ipsos ministrantes , | ne negligatur , et impediatur ordo , et dispositio ministrandi . } Immo si ad mensas Regum \\\hline
2.3.20 & Et por ende deuen los Reyes \textbf{ e los prinçipes ordenar } por que se lean ala mesa algunas costunbres de loar del regno & cum fauces recumbentium cibum sumunt , earum aures doctrinam perciperent ; esset omnino decens et congruum . \textbf{ Ordinare igitur deberent Reges et Principes , } ut laudabiles consuetudines regni , \\\hline
2.3.20 & e los prinçipes ordenar \textbf{ por que se lean ala mesa algunas costunbres de loar del regno } si tales cosas son puestas en escͥpto & Ordinare igitur deberent Reges et Principes , \textbf{ ut laudabiles consuetudines regni , } si tales sunt in scripto redactae , \\\hline
2.3.20 & o avn los fechos \textbf{ que son de loar de los sus anteçessores } e mayormente de aquellos que scanmente e religiosamente se oueron çerca las cosas diuinales & si tales sunt in scripto redactae , \textbf{ vel etiam laudabilia gesta praedecessorum suorum , } et maxime eorum \\\hline
2.3.20 & e de aquellos que iustamente e conueinblemente gouernaron el regno \textbf{ o se deue leer a la mesa delos prinçipes } el libro del gouernamiento delos prinçipes & et debite regnum rexerunt , legerentur ad mensam ; \textbf{ vel legeretur ad mensam liber de regimine Principum , } ut etiam ipsi principantes instruerentur , \\\hline
2.3.20 & por que los prinçipes sean enssennados \textbf{ en qual manera de una prinçipar } e ensseñorear alos otros . & ut etiam ipsi principantes instruerentur , \textbf{ qualiter principari deberent ; } et alii docerentur , \\\hline
2.3.20 & en qual manera de una prinçipar \textbf{ e ensseñorear alos otros . } Et los otros que los oyeren sean enssennados & qualiter principari deberent ; \textbf{ et alii docerentur , } quomodo est principibus obediendum . Haec ergo , \\\hline
2.3.20 & Et los otros que los oyeren sean enssennados \textbf{ en qual manera deuen obedesçer alos prinçipes . } Et pues que & et alii docerentur , \textbf{ quomodo est principibus obediendum . Haec ergo , } vel alia utilia tradita \\\hline
2.3.20 & assi es esto \textbf{ e o triscosas aprouechosas deuen leer alas mesas de los Reyes } e de los prinçipes dela trra & quomodo est principibus obediendum . Haec ergo , \textbf{ vel alia utilia tradita } secundum vulgare idioma , \\\hline
2.3.20 & por que todos sean enssennados por ellas . Et estas cosas assi tractadas çerca el gouernamiento dela casa \textbf{ commo quier que al gunas o trisco las particulares se podiessen tractar . } Enpero por que todas las cosas ꝑticulares non caen sonairaçion & His ergo sic pertractatis circa regnum domus , \textbf{ licet quaedam alia particularia tractari possent , } tamen quia non omnia particularia sub narratione cadunt , \\\hline
2.3.20 & Enpero por que todas las cosas ꝑticulares non caen sonairaçion \textbf{ e so cuento propusiemos de passar las en silençio } e callando las e poniendo fin a este segundo libro & tamen quia non omnia particularia sub narratione cadunt , \textbf{ proponimus } ea silentio pertransire , imponentes finem huic secundo Libro , \\\hline
3.1.1 & assi ca assi commo dize el philosofo en el primero libro delas ethicas \textbf{ que toda obra e toda electiuo dessea de auer algun bien . } Et en el primero libro delas politicas dize & Prima via sic patet . \textbf{ Quia } ( ut dicitur primo Ethicorum ) \\\hline
3.1.1 & por grande alguna cosa que paresçe buena \textbf{ ni esto es de entender } assi que la çibdat sea establesçida por grande aquello que paresçe bueno & quod videtur bonum ; \textbf{ nec sic est intelligendum , } ciuitatem constitutam esse gratia eius quod videtur bonum , \\\hline
3.1.1 & La segunda razon \textbf{ para prouar esto se toma de parte dela çibdat establesçida } por conparaçion alas otras comuidades . & Secunda via ad inuestigandum hoc idem , \textbf{ sumitur ex parte ciuitatis constitutae per comparationem } ad ciuitates alias . \\\hline
3.1.1 & por conparaçion alas otras comuidades . \textbf{ Ca commo quier que toda comun dar natural sea ordenada a bien } enpero mayormente es ordenada aquel bien la comunidat & ad ciuitates alias . \textbf{ Nam licet omnis communitas naturalis ordinetur } ad bonum , \\\hline
3.1.1 & por nonbre comunal çibdat . \textbf{ Enpero conuiene de saber } que la comuidat dela çibdat non es la mas prinçipal sinplemente & quae communi nomine vocatur ciuitas . Aduertendum tamen , \textbf{ communitatem ciuitatis esse principalissimam non simpliciter } et per omnem modum , \\\hline
3.1.2 & que la comunidat dela çibdat \textbf{ on a basta de dezer } que la çibdat es establesçida & quam communitas illa . \textbf{ Non sufficit dicere ciuitatem constitutam } esse gratia alicuius boni , \\\hline
3.1.2 & quanto parte nesçe alo presente son tres biemes \textbf{ por que la çibdat es ordenada abenir } e a beuir conplidamente & quae ex constitutione ciuitatis homines consequuntur . \textbf{ Huiusmodi autem bona ( quantum ad praesens spectat ) tria esse contingit . Ordinatur enim ciuitas ad viuere , } ad sufficienter viuere , \\\hline
3.1.2 & Enpero todas las cosas non partiçipan el ser conplidamente \textbf{ mas solamente a qual las cosas son dichos auer el ser conplidamente } segunt su natura & Non tamen omnia participant sufficienter esse , \textbf{ sed solum illa dicuntur habere sufficiens esse } secundum naturam suam , \\\hline
3.1.2 & por que si a algunan cosa fallesçiere algun acabamiento que pertenezca ala suspeno ala su semeiança \textbf{ commo quier que puede auer aquella cosa aquel ser menguado en alguna manera . } Enpero non es dichͣ & quae secundum suam speciem habent esse completum . \textbf{ Si enim alicui rei deficiat aliqua perfectio competens suae speciei , } licet possit habere illa res esse aliquod , \\\hline
3.1.2 & que el ser uirtuosamente \textbf{ ca non puede auer la uirtud } dela qual fablamos aqui & latius est ergo esse , \textbf{ quam sufficienter esse . Sic etiam latius est sufficienter esse , quam virtuose esse : nam virtute , } quae est habitus electiuus \\\hline
3.1.2 & que es disposicion moral de alma \textbf{ para escoger entre el bien e el mal } si non las aianlias & et moralis , \textbf{ de qua hic loqui intendimus , participare non possunt nisi rationalia } et habentia intellectum . Irrationalia ergo \\\hline
3.1.2 & e las cosas que non han alma \textbf{ segunt su natura pueden auer su ser conplido } si ouieren sus passiones et sus propiedades & et etiam inanimata \textbf{ secundum naturam suam possunt habere esse sufficiens , } si habeant perfectiones competentes propriae speciei : \\\hline
3.1.2 & que parte nesçen ala su semeiançaprop̃a . \textbf{ Enpero non puede auer el ser uirtuoso } por que non puede partiçipar la uirtud ¶ & si habeant perfectiones competentes propriae speciei : \textbf{ esse tamen virtuosum habere non possunt , } quia nequeunt participare virtute . \\\hline
3.1.2 & Enpero non puede auer el ser uirtuoso \textbf{ por que non puede partiçipar la uirtud ¶ } Et pues que assi es en aquella manera & esse tamen virtuosum habere non possunt , \textbf{ quia nequeunt participare virtute . } Illo ergo modo \\\hline
3.1.2 & e el ser conplidamente \textbf{ e el ser uirtuosamente podemos depa rtir el beuir sinplemente } e el beuir conplidamente & et virtuose esse ; \textbf{ distinguere possumus viuere , sufficienter viuere , } et virtuose viuere . Qualitercumque \\\hline
3.1.2 & e el beuir conplidamente \textbf{ e el benir uirtuosa mente . } Ca en qual si quier manera & distinguere possumus viuere , sufficienter viuere , \textbf{ et virtuose viuere . Qualitercumque } etiam homo habeat esse , viuit : non tamen dicitur sufficienter viuere , \\\hline
3.1.2 & que el omne aya el ser biue enpero non es dicho beuir conplidamente \textbf{ e auer uida conplida } si non ouiere aquellas cosas & etiam homo habeat esse , viuit : non tamen dicitur sufficienter viuere , \textbf{ et habere vitam sufficientem , nisi habeat ea quae congrue sufficiunt } ad supplendam indigentiam corporalem , \\\hline
3.1.2 & si non ouiere aquellas cosas \textbf{ que abastan conueinblemente a conplir la mengua corporal . } Et por ende otra cosa es beuir & et habere vitam sufficientem , nisi habeat ea quae congrue sufficiunt \textbf{ ad supplendam indigentiam corporalem , } aliud est ergo viuere aliud sufficienter viuere . \\\hline
3.1.2 & Et por ende estendiendo el beuir politico \textbf{ segunt alguas leys de loar } e segunt alguas ordenaconnes de loar & ut homo est . Ostendendo ergo viuere politicum \textbf{ secundum aliquas leges et } secundum aliquas laudabiles ordinationes , \\\hline
3.1.2 & segunt alguas leys de loar \textbf{ e segunt alguas ordenaconnes de loar } el que refusare de beuir & secundum aliquas leges et \textbf{ secundum aliquas laudabiles ordinationes , } recusans sic viuere , \\\hline
3.1.2 & por que nunca la çibdat es acabada \textbf{ si los moradores della non puede fallar todas las cosas en ella que cunplen para la uida del omne ¶ } Et pues que assi es nos alcançamos & nunquam enim est ciuitas perfecta \textbf{ nisi habitatores eius ibi inuenire possint omnia sufficientia ad vitam . Consequimur ergo ex ciuitate non solum viuere } ut homo , \\\hline
3.1.2 & et tales comunidat acabada \textbf{ ca esto se sigͤel dezir } que tal comunidat es la que ha termino por si & quae est ciuitas constans ex pluribus vicis , est communitas perfecta : \textbf{ quia iam | ( ut consequens est dicere ) } huiusmodi communitas est habens terminum omnis per se sufficientiae vitae . \\\hline
3.1.2 & que establesçieron la çibdat fue en beuir \textbf{ e auer abastamiento en la uida . } Et veyendo & quod facta quidem igitur est ciuitas viuendi gratia : \textbf{ existens autem gratia bene viuendi forte primum motiuum hominum constituentium ciuitatem , } fuit ipsum viuere , \\\hline
3.1.2 & que por grant cuydado \textbf{ que ouiessen non podrian abastar assi mesmos en la uida } por ende establesçieron çibdat & et habere sufficientia in vita . Videntes autem \textbf{ quod solitarie non poterant sibi in vita sufficere , constituerunt ciuitatem , } ut unus alterius defectum supplens haberet sufficientia in vita . \\\hline
3.1.2 & mas sotilmente vevendo \textbf{ que non era asaz auer cunplimiento enla uida } si non biuiessen bien & et homines perspicatius intuentes \textbf{ et videntes } quod non satis est habere sufficientia in vita nisi viuant bene et virtuose , \\\hline
3.1.2 & e sin iustiçia non pueda estar \textbf{ nin durar otdenaron la comunidat politica } que era fechͣ para beuir e para abastamiento dela uida & quod non satis est habere sufficientia in vita nisi viuant bene et virtuose , \textbf{ cum sine lege et iustitia constituta ciuitas stare non posset , ordinauerunt communitatem politicam , } quae facta erat ad viuere , \\\hline
3.1.3 & ca non es esto \textbf{ assi natural al omne conmoes cosa natural al fuego de escalentar } e ala piedra de desçender ayuso & Non enim hoc est sic homini naturale , \textbf{ sicut est naturale igni calefacere , } et lapidi deorsum tendere : \\\hline
3.1.3 & assi natural al omne conmoes cosa natural al fuego de escalentar \textbf{ e ala piedra de desçender ayuso } por que estas cosas tales & sicut est naturale igni calefacere , \textbf{ et lapidi deorsum tendere : } quia talia sic eis naturaliter competunt quod ad contrarium assuefieri non possunt , \\\hline
3.1.3 & e ala piedra \textbf{ que non se pueden acostunbrar al contrario . } Et conuienen les sienpre & quia talia sic eis naturaliter competunt quod ad contrarium assuefieri non possunt , \textbf{ et competunt eis semper } et ubique nunquam enim totiens posset lapis sursum proiici , \\\hline
3.1.3 & e do quier que son can nunca podria ser \textbf{ que tantas uezes fuesse echada la piedra aniba que se pudiesse acostunbrar de yr arriba } e por ende la pied̃ sienpre & et ubique nunquam enim totiens posset lapis sursum proiici , \textbf{ ut assuesceret sursum ire : } semper ergo lapis \\\hline
3.1.3 & e algun appetito natural \textbf{ para benir ciuilmente e en conpanna e las cosas } que assi son naturales & et quandam aptitudinem naturalem , \textbf{ ut ciuiliter viuat . } Quae autem sic sunt naturalia , \\\hline
3.1.3 & por algun caso . \textbf{ o por algun enbargo o por algua otra cosa se pueden enbargar } assi commo dezimos & Quae autem sic sunt naturalia , \textbf{ ex casu vel ex aliquo impedimento siue ex aliqua causa impediri possunt , } ut licet naturale sit homini esse dextrum , \\\hline
3.1.3 & assi commo por que son muy pobres non pueden beuir çiuilmente \textbf{ mas son costrennidos de sallir dela çibdat } e apartar se a labrar los canpos & non possunt ciuiliter viuere , \textbf{ sed coguntur ciuitatem exire , } et agros excolere . \\\hline
3.1.3 & mas son costrennidos de sallir dela çibdat \textbf{ e apartar se a labrar los canpos } por do biuna . & sed coguntur ciuitatem exire , \textbf{ et agros excolere . } Secundum propter quod homo aliquando redditur non ciuilis est nimia prauitas : \\\hline
3.1.3 & por que es malo . \textbf{ o pecadar e sin yugo de ley } que non quiere conplir les & vel ex fortuna hoc accidat ) vel est bestia et sceleratus et sine iugo , \textbf{ non valens legem et societatem supportare , } vel est quasi deus idest diuinus , eligens altiorem vitam : \\\hline
3.1.3 & o pecadar e sin yugo de ley \textbf{ que non quiere conplir les } nin sofrir conpannia o por que es diuinal & vel ex fortuna hoc accidat ) vel est bestia et sceleratus et sine iugo , \textbf{ non valens legem et societatem supportare , } vel est quasi deus idest diuinus , eligens altiorem vitam : \\\hline
3.1.3 & que non quiere conplir les \textbf{ nin sofrir conpannia o por que es diuinal } assi commo dios & ø \\\hline
3.1.4 & e que el oen non era naturalmente aianlçiuil . \textbf{ Et pues que assi es commo non cunpla soluer la sobiecconnes contrarias } por las quales se desfaze la uerdat & et hominem non esse naturaliter animal ciuile . \textbf{ Cum ergo non satis sit remouere omnes obiectiones contrarias veritatem aliquam impugnantes , } nisi adducantur rationes propriae \\\hline
3.1.4 & por las quales la uerdat sea confirmada . \textbf{ Por ende entendemos en este capitulo de adozir razones } que muestren que la çibdat es cosa natural & per quas illa ueritas confirmetur : \textbf{ intendimus | in hoc capitulo } adducere rationes ostendentes ciuitatem esse \\\hline
3.1.4 & que beuir es cosa natural al omne \textbf{ e por que la nafa non pueda fallesçer en las cosas } neçessarias & secundum naturam ; \textbf{ ut natura non deficiat in necessariis , } oportet quid naturale esse quicquid \\\hline
3.1.4 & La segunda razon \textbf{ para prouar esto mismo se toma desto } que la çibdat es fin et conplimiento de las dichͣs dos comunidades & secundum naturam . Secunda uia ad inuestigandum \textbf{ hoc idem , sumitur ex eo quod ciuitas est illarum communitatum finis et complementum . } Nam \\\hline
3.1.4 & que la çibdat es cosa natural . \textbf{ finca de demostrar que el omne es naturalmente aianl politicas } e ciuilla qual cosa podemos demostrar & reliquum est ostendere , \textbf{ hominem esse naturaliter animal politicum et ciuile , } quod etiam duplici via inuestigare possumus . Prima via sumitur ex parte sermonis . \\\hline
3.1.4 & finca de demostrar que el omne es naturalmente aianl politicas \textbf{ e ciuilla qual cosa podemos demostrar } por dos razones ¶ & hominem esse naturaliter animal politicum et ciuile , \textbf{ quod etiam duplici via inuestigare possumus . Prima via sumitur ex parte sermonis . } Secunda ex parte impetus naturalis . Probatur enim in principio secundi libri , \\\hline
3.1.4 & por que por la palabra tioma el omne costunbres e disçiplina . \textbf{ En essa misma manera aqui deꝑte dela palabra podemos demostrar } que el omne es natraalmente aianl politicas e ciuil & et disciplinam . \textbf{ Hoc autem ex parte sermonis ostendere possumus hominem esse naturaliter animal politicum et ciuile , } ex eo quod vox humana , \\\hline
3.1.4 & e en otra manera \textbf{ quando se trista puede a otro can de mostrar } por su ladrado sutsteza o su delecta conn & aliter latrat cum delectatur : \textbf{ et cum tristatur potest alteri cani per suum latratum significare tristitiam , } vel delectationem quam habet . \\\hline
3.1.4 & que ha mas al omne sobre todo esto le es dada la palabra \textbf{ por la qual puede demostrar departidamente } qual cosa le es delectable & Sed hominibus ultra hoc datus est sermo , \textbf{ per quem distincte significatur quid conferens , } quid nociuum , \\\hline
3.1.4 & e qual non iusta por ende \textbf{ si la comunidat dela casa es ordenada a alcançar lo que es delectable } e para foyr lo que es enpeçible . & et \textbf{ quid iniustum . Si ergo communitas domestica ordinatur ad prosequendum conferens , } et ad fugiendum nociuum : \\\hline
3.1.4 & si la comunidat dela casa es ordenada a alcançar lo que es delectable \textbf{ e para foyr lo que es enpeçible . } Et la comunidat dela çibdat & quid iniustum . Si ergo communitas domestica ordinatur ad prosequendum conferens , \textbf{ et ad fugiendum nociuum : } communitas vero ciuitatis ultra hoc ordinatur \\\hline
3.1.4 & Et la comunidat dela çibdat \textbf{ sobre esto es ordenada a segnir } lo que es iusto & et ad fugiendum nociuum : \textbf{ communitas vero ciuitatis ultra hoc ordinatur } ad prosequendum iustum , \\\hline
3.1.4 & lo que es iusto \textbf{ e a foyr delo } que non es iusto conuiene & ad prosequendum iustum , \textbf{ et ad fugiendum iniustum , } oportet communitatem domesticam \\\hline
3.1.4 & que es ordenada a aquellas cosas \textbf{ que se han de demostrar conueinblemente } por la palabra conuiene & naturalis est illa communitas quae ordinatur ad illa , \textbf{ quae sunt apta nata exprimi per sermonem : } iustum enim et iniustum non proprie habet esse in communitate domestica , \\\hline
3.1.4 & e departidas se leuna tan lides e uaraias \textbf{ e alli es de poner } lo que es iusto & consurgunt lites \textbf{ et litigia , } quid iustum \\\hline
3.1.4 & por quela palabra anos en dada naturalmente \textbf{ demostrar qual cosa nos es delectable } e qual enpesçible & utraque tamen communitas erit naturalis tam domestica quam ciuilis , \textbf{ eo quod per sermonem nobis datum a natura repraesentatur conferens } et nociuum , \\\hline
3.1.4 & por la qual cosa si la natura dio al omne el beuir diol natural inclinaçion \textbf{ para fazer aquellas cosas } por las quales se pueden manteñ en la uida & dedit ei naturalem impetum \textbf{ ad faciendum ea per quae possit sibi in vita sufficere . } Hoc autem maxime contingit \\\hline
3.1.4 & para fazer aquellas cosas \textbf{ por las quales se pueden manteñ en la uida } e esto contesçe mayormente & dedit ei naturalem impetum \textbf{ ad faciendum ea per quae possit sibi in vita sufficere . } Hoc autem maxime contingit \\\hline
3.1.4 & por la comunidat dela çibdat \textbf{ por que deue contener en ssi todas aquellas cosas } que cunplen para la vida del omne & per communitatem ciuilem , \textbf{ eo quod ciuitas debeat esse contentiua omnium , } quae ad vitam sufficiunt . Inerit ergo hominibus impetus naturalis \\\hline
3.1.4 & para beuir politicas miente en çibdat \textbf{ e para fazer çibdat } mas commo aquello a que auemos inclinaçion natural sea cosa natural & ad viuendum politice , \textbf{ et ad constituendum ciuitatem . } Sed cum id , \\\hline
3.1.5 & e sea alguna cosa segunt natura \textbf{ e podemos mostrar por tres razones } que sin la comunidat dela çibdat cosa aprouechosa fue ala uida humanal de establesçer comunidat de regno ¶ & secundum naturam . \textbf{ Possumus autem triplici via ostendere , } quod praeter communitatem ciuitatis , \\\hline
3.1.5 & e podemos mostrar por tres razones \textbf{ que sin la comunidat dela çibdat cosa aprouechosa fue ala uida humanal de establesçer comunidat de regno ¶ } La primera razon se toma de parte del conplimiento dela uida & Possumus autem triplici via ostendere , \textbf{ quod praeter communitatem ciuitatis , | utile est humanae vitae statuere communitatem regni . Prima via sumitur ex parte sufficientiae vitae . } Secunda ex parte adeptionis virtutis . \\\hline
3.1.5 & La primera razon se toma de parte del conplimiento dela uida \textbf{ ¶la segunda de parte de ganar uirtud ¶ } La terçera de parte del defendimiento & utile est humanae vitae statuere communitatem regni . Prima via sumitur ex parte sufficientiae vitae . \textbf{ Secunda ex parte adeptionis virtutis . } Tertia ex parte defensionis \\\hline
3.1.5 & e aya en ssi conplimiento delas cosas \textbf{ que ꝑtenesçen ala uida esto non es assi de entender } que sienpre conuenga & est habens terminum omnis per se sufficientiae vitae . \textbf{ Non sic intelligendum est , } quod semper oporteat ciuitatem \\\hline
3.1.5 & e lo que es mester \textbf{ e por sabiduria hu manal se puedan auer ligeramente las cosas } que cunplen ala uida . & et ponderis portatiuam , \textbf{ et per humanam industriam faciliter habere possit sufficientia vitae . } Sicut ergo vicus , \\\hline
3.1.5 & por que mas ligeramente \textbf{ e meior se puedan acoirer los vnos alos otros } quanto aquellas cosas & ut facilius \textbf{ et melius sibi inuicem | subueniant } quantum ad ea quibus indigemus in vita . \\\hline
3.1.5 & prouechosa cosa es alas çibdades de ser ay uirtadas so vn regno \textbf{ por que meior se puedan acorrer las vnas alas otras } en aquellas cosas que son meester & non omnes ciuitates abundant in eisdem , \textbf{ utile est eis congregari } sub uno regno , \\\hline
3.1.5 & La segunda razon \textbf{ para prouar esto se toma de parte de ganar uirtud } ca la entencion & ut melius possint sibi inuicem \textbf{ subuenire in his quae requiruntur ad sufficientiam vitae . Secunda via ad inuestigandum hoc idem , sumitur ex parte adoptionis virtutis . } Intentio enim legislatoris \\\hline
3.1.5 & que son mester \textbf{ para conplir la mengua dela uida corporal } mas avn por que biuna los omes segunt ley & non solum esse debet , \textbf{ ut ciues habeant quae requiruntur ad supplendam indigentiam corporalem : } sed etiam ut viuant \\\hline
3.1.5 & ante el que faze la ley cantomas \textbf{ prinçipalmente deue tener mientes a esto } que aquello quante el alma es meior & secundum legem et virtuose . \textbf{ Immo tanto principalius debet intendere hoc quam illud , } quanto anima est potior corpore , \\\hline
3.1.5 & que los gouernadores de la çibdat ayan poderio çiuil \textbf{ por que puedan costrennir e fazer iustiçia } en los que non quieren beuiruirtuosamente & ut possint cogere \textbf{ et punire nolentes virtuose viuere , } et turbantes pacem et bonum statum aliorum ciuium . \\\hline
3.1.5 & de beuir uirtu cosamente \textbf{ e de se ayuntar so vn regno } por espantar los malos . & expedit ciuitatibus propter virtuose viuere , \textbf{ et propter corruptionem peruersorum congregari sub uno regno } quod si tamen Princeps tyrannizare vellet , \\\hline
3.1.5 & e de se ayuntar so vn regno \textbf{ por espantar los malos . } Enpero si contesçiesse por auenta & expedit ciuitatibus propter virtuose viuere , \textbf{ et propter corruptionem peruersorum congregari sub uno regno } quod si tamen Princeps tyrannizare vellet , \\\hline
3.1.5 & los que prinçipes \textbf{ quesiessen tiranizar e ser malos en todo quanto menor poderio ellos ouieren tanto } mas es pro dela çibdat ¶ & quod si tamen Princeps tyrannizare vellet , \textbf{ quanto minorem haberet potentiam , } tanto magis esset expediens ciuitati . \\\hline
3.1.5 & La terçera razon \textbf{ para prouar esto mismo } se toma de parte dela defenssion & tanto magis esset expediens ciuitati . \textbf{ Tertia via ad inuestigandum hoc idem , } sumitur ex parte defensionis \\\hline
3.1.5 & faze amistança con otra çibdat \textbf{ por que pueda meior cotra dezir } e con tristar alos enemigos & et remotionis impedimentorum hostium . Videmus enim quod cum aliqua ciuitas impugnatur confoederat se ciuitati alii , \textbf{ ut melius possit resistere impugnationem hostium : } cum ergo regnum sit \\\hline
3.1.5 & por que pueda meior cotra dezir \textbf{ e con tristar alos enemigos } que la conbaten . & et remotionis impedimentorum hostium . Videmus enim quod cum aliqua ciuitas impugnatur confoederat se ciuitati alii , \textbf{ ut melius possit resistere impugnationem hostium : } cum ergo regnum sit \\\hline
3.1.5 & por que ellas son ayuntadas \textbf{ so vn Rey aqui pertenesçe de defender a cada vna parte del regno } e ordener el poderio çiuil delas otras çibdades a defendimiento de cada vna delas çibdades del regno & eo quod uniantur sub uno rege , \textbf{ cuius est quemlibet partem regni defendere , } et ordinare ciuilem potentiam aliarum ciuitatum \\\hline
3.1.5 & so vn Rey aqui pertenesçe de defender a cada vna parte del regno \textbf{ e ordener el poderio çiuil delas otras çibdades a defendimiento de cada vna delas çibdades del regno } si contesca que vna çibdat sea conbatida de los enemigos & cuius est quemlibet partem regni defendere , \textbf{ et ordinare ciuilem potentiam aliarum ciuitatum | ad defensionem cuiuslibet ciuitatis regni ; } si contingat eam ab extraneis impugnari , \\\hline
3.1.6 & se ssziese vna comuidat de regno \textbf{ os podemos sennalar dos maneras del fazemiento dela çibdat e del regno } e cada vna dlłas es en alguna manera natural & ex pluribus communitatibus politicis constituere communitatem unam regni . \textbf{ Generationis ciuitatis | et regni duos modos possumus assignare , } quorum quilibet est aliquo modo naturalis , \\\hline
3.1.6 & para ssi algua çibdat \textbf{ en la qual morando en vno podiessen auer } mas conplidamente aquellas cosas & constituentes sibi ciuitatem aliqua , \textbf{ in qua simul morantes habere possent sufficientius quae requiruntur ad indigentiam vitae . } Hoc ergo modo posset fieri constitutio regni , \\\hline
3.1.6 & para la mengua dela uida ¶ \textbf{ Et pues que assi es en esta manera se podia fazer el establesçemiento del regno } assi commo sy muchͣs çibdades & in qua simul morantes habere possent sufficientius quae requiruntur ad indigentiam vitae . \textbf{ Hoc ergo modo posset fieri constitutio regni , } ut si multae ciuitates \\\hline
3.1.6 & ¶ la segunda es natural \textbf{ por que los omes naturalmente han inclinaçion a establesçer çibdat e regno } por que si por generaçion cresçiendo los fijos e los mietos en vna casa & quae est opus naturae . Secundo naturalis existit , \textbf{ quia homines naturalem habent impetum ad constituendam ciuitatem | et regnum . } Si enim per generationem ex crescentibus filiis \\\hline
3.1.6 & por que si por generaçion cresçiendo los fijos e los mietos en vna casa \textbf{ e non podiendo morar en vno fagan } para ssi muchos casas & et nepotibus in eadem domo , \textbf{ et non valentibus simul habitare , | faciant } sibi plures domos , \\\hline
3.1.6 & Et despues mas adelante cresçiendo \textbf{ e non podiendo morar en vn uarrio fagan mas adelante } para si much suarrios e establescandellos çibdat . & et ulterius excrescentibus \textbf{ et non valentibus habitare in uno vico , } faciant sibi vicos plures , \\\hline
3.1.6 & Et despues mas adelante cresçiendo \textbf{ e noo podiendo morar en vna çibdat fagan } para si muchͣs çibdades & amplius autem ipsis excrescentibus \textbf{ et non valentibus habitare in una ciuitate , fabricent sibi ciuitates plures , } et constituant regnum : \\\hline
3.1.6 & por que los omes han natal inclinaçion para beuir \textbf{ por que concuerdan de establesçer çibdat } enla qual sean falladas aquellas cosas & quia homines propter viuere habent naturalem impetum , \textbf{ ut concordent ciuitatem constituere , } in qua reperiuntur sufficientia ad vitam . Sic etiam habent naturalem impetum , \\\hline
3.1.6 & Avn en essa misma manera han natural inclinaçion \textbf{ para establesçer prinçipado e regno } por que por tal establesçimiento pueden beuir & in qua reperiuntur sufficientia ad vitam . Sic etiam habent naturalem impetum , \textbf{ ut constituant principatum | et regnum ; } quia per talem constitutionem possunt magis pacifice viuere , \\\hline
3.1.6 & mas en paz \textbf{ e pueden mas defender se de los enemigos } que les quieren mal fazer et esta tal inclinaçion es natural & quia per talem constitutionem possunt magis pacifice viuere , \textbf{ et magis resistere hostibus volentibus impugnare ipsos . Est enim huius impetus naturalis : } nam sicut homines naturalem habent impetum ut viuant , \\\hline
3.1.6 & e pueden mas defender se de los enemigos \textbf{ que les quieren mal fazer et esta tal inclinaçion es natural } que assi commo los omes han naturͣal inclinaçion & quia per talem constitutionem possunt magis pacifice viuere , \textbf{ et magis resistere hostibus volentibus impugnare ipsos . Est enim huius impetus naturalis : } nam sicut homines naturalem habent impetum ut viuant , \\\hline
3.1.6 & para que biuna en paz \textbf{ e por que puedan mas ligeramente defender se de sus enemigos } la qual cosa se faze & sic naturalem habent impetum ut pacifice \textbf{ viuant | et ut naturaliter resistant hostibus , } quod fit per societatem principatus et regni ; \\\hline
3.1.6 & e del tegno delas quales cada vna puede ser dichͣ natural \textbf{ Podemos eñader la terçera manera } que es sinplemente & quorum quilibet dici potest naturalis : \textbf{ possumus addere modum tertium , } qui quasi est simpliciter violentus . \\\hline
3.1.6 & por tirania e por fuerça \textbf{ e ser sennar dellos . } Et por que mas ligeramente sennoreasse sobre ellos & et per violentiam , \textbf{ et dominari eis : } et ut facilius eis dominaretur , \\\hline
3.1.6 & Et por que mas ligeramente sennoreasse sobre ellos \textbf{ podrie ayuntarlos en vno por fuerça } e establesçer ende çibdat . & et ut facilius eis dominaretur , \textbf{ poterat eos uiolenter congregare in unum , } et constituere inde ciuitatem . \\\hline
3.1.6 & podrie ayuntarlos en vno por fuerça \textbf{ e establesçer ende çibdat . } En essa misma manera avn el Regno por fuerça & poterat eos uiolenter congregare in unum , \textbf{ et constituere inde ciuitatem . } Sic etiam et regnum per uiolentiam \\\hline
3.1.6 & En essa misma manera avn el Regno por fuerça \textbf{ e por tirania se podia establesçer } assi como si alguno quisiesse senorear alguna çibdat & Sic etiam et regnum per uiolentiam \textbf{ et per tyrannidem constitui posset , } ut si quis dominans alicui ciuitati per tyrannidem \\\hline
3.1.6 & e por tirania se podia establesçer \textbf{ assi como si alguno quisiesse senorear alguna çibdat } por tirania e por poderio ciuil apremiasse las otras çibdades & et per tyrannidem constitui posset , \textbf{ ut si quis dominans alicui ciuitati per tyrannidem } et per ciuilem potentiam opprimat ciuitates alias , \\\hline
3.1.6 & que son maneras departidas de establesçimiento de çibdat e de Regno . \textbf{ finca de ver en quantas partes conuiene de partir este terçero libro } enl qual auemos de tractardeste tal gouernamiento & se Regem constitui super illas . Viso diuersos esse modos generationis ciuitatis et regni , \textbf{ restat uidere in quot partes oportet hunc tertium librum diuidere , } in quo de huiusmodi regimine tractaturi sumus . Patet enim ex dictis naturalem generationem ciuitatis \\\hline
3.1.6 & e a beuir en paz \textbf{ e para yr } contra los que quisieren turbar la paz & et ad pacifice uiuere , \textbf{ et ad resistendum uolentibus turbare pacem , } et impugnare ciues . \\\hline
3.1.6 & e para yr \textbf{ contra los que quisieren turbar la paz } e qualieren lidiar contra los çibdadanos o contra los del regno . & et ad pacifice uiuere , \textbf{ et ad resistendum uolentibus turbare pacem , } et impugnare ciues . \\\hline
3.1.6 & contra los que quisieren turbar la paz \textbf{ e qualieren lidiar contra los çibdadanos o contra los del regno . } ¶ Et pues que assi es conuiene nos de mostrar & et ad resistendum uolentibus turbare pacem , \textbf{ et impugnare ciues . } Ostendendum est ergo qualiter possit bene regi \\\hline
3.1.6 & e qualieren lidiar contra los çibdadanos o contra los del regno . \textbf{ ¶ Et pues que assi es conuiene nos de mostrar } qual es la meior manera de gouernamiento de çibdat e de regno & et impugnare ciues . \textbf{ Ostendendum est ergo qualiter possit bene regi } ciuitas \\\hline
3.1.6 & qual es la meior manera de gouernamiento de çibdat e de regno \textbf{ e en qual manera se puede bien gouernar la çibdat } e el regno en tp̃o de paz & Ostendendum est ergo qualiter possit bene regi \textbf{ ciuitas } siue regnum tempore pacis , \\\hline
3.1.6 & e el regno en tp̃o de paz \textbf{ e en qual manera deuemos lidiar } contra los enemigos entp̃o de guerra . & siue regnum tempore pacis , \textbf{ et qualiter impugnandi sint hostes tempore belli . } Verum \\\hline
3.1.6 & e cerca la fin de los elencos niguno non abasta \textbf{ assi mismo en fallar algunan arte } mas conuiene & et circa finem Elenchorum , \textbf{ nullus sibi sufficit in inueniendo artem aliquam , } sed oportet \\\hline
3.1.6 & que dieron algun conosçimiento et algun entendimiento de aquella arte por ende conuiene \textbf{ para auer arte conplida de gouernamiento dela çibdat } e del regno de dezer & ad hoc iuuari per auxilium praecedentium tradentium notitiam aliquam de arte illa : \textbf{ ideo oportet propter sufficientiam artis regiminis ciuitatis } et regni citare \\\hline
3.1.6 & para auer arte conplida de gouernamiento dela çibdat \textbf{ e del regno de dezer } e de contar & ideo oportet propter sufficientiam artis regiminis ciuitatis \textbf{ et regni citare } quid senserunt Philosophi de huiusmodi regimine , \\\hline
3.1.6 & e del regno de dezer \textbf{ e de contar } que sintieron los philosofos antiguos de tal gouernamiento & ideo oportet propter sufficientiam artis regiminis ciuitatis \textbf{ et regni citare } quid senserunt Philosophi de huiusmodi regimine , \\\hline
3.1.6 & por que por esto seamos en algun manera endozidos \textbf{ a sabrque cosa deuemos esquiuar enł gouernamiento del rogno e dela çibdat . } Et pues que assy es todo este re terçero & ex hoc aliqualiter manducemur ad sciendum \textbf{ quid vitandum , et quid imitandum sit in regimine regni et ciuitatis . } Totum ergo hunc librum tertium diuidemus in tres partes . \\\hline
3.1.6 & por que por el conosçimiento dellos sean endozidos los Reyes e los prinçipes \textbf{ en qual manera de una gouernar las çibdades e los regnos ¶ } Lo segundo mostraremos & ut per eorum notitiam manuducantur Reges et Principes , \textbf{ quomodo debeant regere ciuitates } et regna . Secundo ostendetur , \\\hline
3.1.6 & qual es la muy buean politica o çibdat o muy vuen regno \textbf{ e de quales cautelas deuen usar los prinçipes e los . Reyes } e en qual tra nera es de gouernar la çibdat e el regno & quae sit optima politia siue optimum regnum , \textbf{ et quibus cautelis uti debeant principantes , } et quomodo tempore pacis propter uiuere sufficienter \\\hline
3.1.6 & e de quales cautelas deuen usar los prinçipes e los . Reyes \textbf{ e en qual tra nera es de gouernar la çibdat e el regno } para beuir conplidamente e uirtuosamente . & et quibus cautelis uti debeant principantes , \textbf{ et quomodo tempore pacis propter uiuere sufficienter } et uirtuose regenda sit ciuitas \\\hline
3.1.6 & Lo terçero manifestaremos \textbf{ en qual manera son de escoier los lidiadors } e en qual manera deuen lidiar contra los enemigos & et uirtuose regenda sit ciuitas \textbf{ et regnum . Tertio manifestabitur quomodo eligendi sunt pugnatores , } et qualiter impugnandi sunt hostes , \\\hline
3.1.6 & en qual manera son de escoier los lidiadors \textbf{ e en qual manera deuen lidiar contra los enemigos } e en qual manera se deuen ordenar las hazes delas batallas & et regnum . Tertio manifestabitur quomodo eligendi sunt pugnatores , \textbf{ et qualiter impugnandi sunt hostes , } et quomodo ordinandae sunt acies bellorum , \\\hline
3.1.6 & e en qual manera deuen lidiar contra los enemigos \textbf{ e en qual manera se deuen ordenar las hazes delas batallas } e de quales cautelas deuen vsar los lidiadores & et qualiter impugnandi sunt hostes , \textbf{ et quomodo ordinandae sunt acies bellorum , } et quibus cautelis uti debeant bellantes . \\\hline
3.1.6 & e en qual manera se deuen ordenar las hazes delas batallas \textbf{ e de quales cautelas deuen vsar los lidiadores } as socrates commo ouiesse phophado luengo tienpo çerca las naturas delas cosas & et quomodo ordinandae sunt acies bellorum , \textbf{ et quibus cautelis uti debeant bellantes . } Socrates autem quandiu philosophatus esset circa naturas rerum , \\\hline
3.1.7 & que toda la discordia de los çibdadanos se leunataua dela propreedat delas possessiones \textbf{ e por que cada vno se esforcaua ha dezer esto es mio . } por la qual cosa & et ex eo quod quilibet utitur dicere , \textbf{ Hoc est meum . } Quare ( ut arguebat ) \\\hline
3.1.7 & que las mugers deuian ser enssennadas alas obras dela batalla \textbf{ e que deuian batallar e guerrear } assi commo los maridos & quia dixerunt mulieres instruendas esse ad opera bellica , \textbf{ et debere bellare sicut } et viros . Inducebantur enim ad hoc \\\hline
3.1.7 & en la qual partiçipamos con las otras aianlias \textbf{ que las mugeres deuen lidiar tan bien commo los omes . } e por ende natalmente paresçe alos dichos philosofos & secundum ordinem naturalem in quo communicamus \textbf{ cum animalibus aliis , } videtur ciuitas maxime naturaliter ordinata \\\hline
3.1.7 & que cada vna çibdat demaser partida en çinco partes . \textbf{ Conuiene a saber en labradores e en maestros de algunas artes } assi commo ferreros e texedores e carpentos e capateros . & quia dixerunt ciuitatem quamlibet diuidendam esse in quinque partes , \textbf{ videlicet in agricolas , | artifices , bellatores , } consiliarios , \\\hline
3.1.7 & para la defenssion de la çibdat \textbf{ e para ref̉tenar los enemigos } por que dizian & et alia quae requiruntur ad vitam . Bellatores quidem necessarii existebant , \textbf{ propter defensionem ciuitatis et reprimendos hostes dicebant autem ciuitatem } ( si bene ordinata erat ) \\\hline
3.1.7 & Mas el prinçipe e los conseieros eran nescessarios en la çibdat \textbf{ para enderesçar los çibdadanos } por que biuiessen uirtuosamente & et consiliarii necessarii erant \textbf{ in ciuitate ad dirigendum ciues } ut virtuose viuerent , \\\hline
3.1.7 & delas quales \textbf{ si acaesçiere logar nos podremos dellas fazer mençion . } on conuiene de demandar en todas las cosas & de quibus si locus occurrat \textbf{ mentio fieri poterit . } Maximam unitatem et aequalitatem non oportet quaerere in omnibus rebus . \\\hline
3.1.8 & si acaesçiere logar nos podremos dellas fazer mençion . \textbf{ on conuiene de demandar en todas las cosas } grant egualdat cosas fuessen eguales & mentio fieri poterit . \textbf{ Maximam unitatem et aequalitatem non oportet quaerere in omnibus rebus . } Nam si omnia essent aequalia , \\\hline
3.1.8 & e por que el mundo \textbf{ segunt su estado sea muy acabado conuietie de dar } en el departidas speçias e departidas semeianças & et ad hoc quod uniuersum \textbf{ secundum suum statum sit maxime perfectum , } oportet ibi dare diuersa \\\hline
3.1.8 & ca por que toda la bondat del mundo non puede ser fallada en vna espeçie \textbf{ nin en vna semeiança conuiene de dar } y deꝑ tidas espeçies e departidas semeianças & reseruari in una specie , \textbf{ oportet ibi dare species diuersas ; } ut in pluribus speciebus entium \\\hline
3.1.8 & nin en vna semeiança conuiene de dar \textbf{ y deꝑ tidas espeçies e departidas semeianças } ca en muchos espeçies e semeianças delas cosas se salua mayor perfectiuo que en vna tan sola mente . & reseruari in una specie , \textbf{ oportet ibi dare species diuersas ; } ut in pluribus speciebus entium \\\hline
3.1.8 & En essa misma manera avn en la çibdat \textbf{ para que aya ser acabada conuiene de dar ay algun departimiento } nin conuiene de ser & quam in una tantum . Sic etiam in ciuitate \textbf{ ad hoc quod habeat esse perfectum , oportet dare diuersitatem aliquam , nec oportet ibi esse omnimodam conformitatem } et aequalitatem , \\\hline
3.1.8 & e si se estendiere a mayor vnidat paresçra el ser dela çibdat . \textbf{ Et pues que assi es dezer } que enla çibdat o en el regno deua ser tan grant vnidat & ultra quam si descendatur , perit esse eius . \textbf{ Dicere ergo in ciuitate } vel in regno esse debere omnem unitatem , \\\hline
3.1.8 & que enla çibdat o en el regno deua ser tan grant vnidat \textbf{ commo dizian socrates e platones dezer } que la çibdat non sea çibdat & vel in regno esse debere omnem unitatem , \textbf{ est dicere ciuitatem non esse ciuitatem , } et regnum non esse regnum . \\\hline
3.1.8 & La segunda razon \textbf{ para prouar esto mismo se toma } por conpara çiconala hueste o ala inpugnacion de los enemi gosca & et regnum non esse regnum . \textbf{ Secunda via ad inuestigandum hoc idem , } sumitur per comparationem ad exercitum , \\\hline
3.1.8 & ca assi commo vn cuerpo ha mester departidas obras \textbf{ assi commo de andar e de tanner } e de oyr e deuer . & quia idem corpus diuersis indiget operibus , \textbf{ ut ambulatione , tactu , visione , } et auditus ideo oportet ibi dare diuersa membra exercentia diuersos actus : \\\hline
3.1.8 & assi commo de andar e de tanner \textbf{ e de oyr e deuer . } por ende conuiene de dar . & quia idem corpus diuersis indiget operibus , \textbf{ ut ambulatione , tactu , visione , } et auditus ideo oportet ibi dare diuersa membra exercentia diuersos actus : \\\hline
3.1.8 & e de oyr e deuer . \textbf{ por ende conuiene de dar . } y departidos mienbros que fagan estas obras departidas . & ut ambulatione , tactu , visione , \textbf{ et auditus ideo oportet ibi dare diuersa membra exercentia diuersos actus : } sic quia ad indigentiam vitae indigemus domibus , vestimentis ; et victualibus , et aliis huiusmodi ; \\\hline
3.1.8 & por que \textbf{ para conplir la mengua dela uida auemos mester casas e uestid̃as e viandas e otras cosas } tales & et auditus ideo oportet ibi dare diuersa membra exercentia diuersos actus : \textbf{ sic quia ad indigentiam vitae indigemus domibus , vestimentis ; et victualibus , et aliis huiusmodi ; } oportet in ciuitate dare diuersitatem aliqua , \\\hline
3.1.8 & tales \textbf{ por ende conuiene de dar algun departimiento en la çibdat por que en ella sean falladas todas las cosas } que cunplen ala uida . & sic quia ad indigentiam vitae indigemus domibus , vestimentis ; et victualibus , et aliis huiusmodi ; \textbf{ oportet in ciuitate dare diuersitatem aliqua , } ut in ea reperiatur sufficientia ad vitam . \\\hline
3.1.8 & el cuerpo non seria acabado \textbf{ ca commo quier que uiesse non podria oyr } nin andar . & corpus imperfectum esset , \textbf{ quia licet videret , | non posset audire , } nec ambulare : \\\hline
3.1.8 & ca commo quier que uiesse non podria oyr \textbf{ nin andar . } En essa misma manera si fuesse grant ayuntamiento & non posset audire , \textbf{ nec ambulare : } sic si esset maxima conformitas in ciuitate , \\\hline
3.1.8 & a algun prinçipe o algun sennor \textbf{ e commo en la çibdat conuenga de dar alguons ofiçioso alguons maestradgos o algunas alcaldias } la qual cosa non seria & ut cum in ciuitate oporteat \textbf{ dare aliquos magistratus , | et aliquas praeposituras , } quod non esset , \\\hline
3.1.8 & por ende commo estas cosas demanden departimiento \textbf{ conuiene de dar en la çibdat algun departimiento . } La quanta razon se toma & Quare cum hoc diuersitatem requirat , \textbf{ oportet in ciuitate dare diuersitatem aliquam . Quinta uia sumitur } per comparationem ad finem . Nam finis ciuitatis est bene viuere , \\\hline
3.1.8 & La quanta razon se toma \textbf{ por conpaçion dela finca la fin dela çibdat es bien beuir } e auer abastamiento en la uida & oportet in ciuitate dare diuersitatem aliquam . Quinta uia sumitur \textbf{ per comparationem ad finem . Nam finis ciuitatis est bene viuere , } et habere sufficientiam in vita ; \\\hline
3.1.8 & por conpaçion dela finca la fin dela çibdat es bien beuir \textbf{ e auer abastamiento en la uida } ca la çibdat escomunidat & per comparationem ad finem . Nam finis ciuitatis est bene viuere , \textbf{ et habere sufficientiam in vita ; } nam ciuitas est communitas habens terminum omnis per se sufficientiae vitae : \\\hline
3.1.8 & son meester muchͣs cosas departidas \textbf{ por ende conuiene enla çibdat de auer en ssi algun departimiento } e de auer departidos uarrios & sed ut dictum est ad sufficientiam vitae requiruntur diuersa , ideo oportet ciuitatem habere aliquam diuersitatem in se , \textbf{ et diuersos habere vicos , } ut expediens \\\hline
3.1.8 & por ende conuiene enla çibdat de auer en ssi algun departimiento \textbf{ e de auer departidos uarrios } assi que abonden a la uida & et diuersos habere vicos , \textbf{ ut expediens } ad vitam quod non neperitur \\\hline
3.1.8 & que es muchedunbre de bozes nunca es bien proporçionada si non fueren y todas las bozes eguales \textbf{ mas ala derecha consonançia delas bozes conuiene de dar } y departimiento de los tonos & ut nunquam melodia ; \textbf{ quae est multitudo vocum , } est bene proportionata , \\\hline
3.1.8 & e alos prinçipes de sabesto \textbf{ por que munca ninguno sopo bien gouernar çibdat } si non sopiere en qual manera es establesçida la çibdat & nisi sit ibi diuersitas officiorum . Decet ergo hoc Reges , et Principes cognoscere , \textbf{ quod nunquam quis bene nouit regere ciuitatem , } nisi sciuerit qualiter constituitur ; \\\hline
3.1.8 & si non sopiere en qual manera es establesçida la çibdat \textbf{ e si non sopiere en qual manera conuiene de auer en ella departimiento de ofiçios e de ofiçiales } l sermon en los comienços deueser luengo e bien examinado & nisi sciuerit qualiter constituitur ; \textbf{ et nisi cognoscat quod oportet in ea diuersitatem esse . } Sermo in principiis debet esse longus , \\\hline
3.1.9 & por que pequano yerro en el comienço es muy grande en la fin . \textbf{ Et pues que assi es luengamente son los comienços de terctar } e por luengos sermones son de escodrinnar & maximus est in fine . \textbf{ Diu ergo sunt principia pertractanda , } et longis sermonibus sunt excutienda , \\\hline
3.1.9 & Et pues que assi es luengamente son los comienços de terctar \textbf{ e por luengos sermones son de escodrinnar } por que çerca ellos non contezca yerro . & Diu ergo sunt principia pertractanda , \textbf{ et longis sermonibus sunt excutienda , } ne circa ipsa contingat error . \\\hline
3.1.9 & por la qual cosa commo en el gouernamiento dela çibdat \textbf{ primeramente se ha de ordenar la poliçia } muy luengamente es de buscar e de escodrinnar & ne circa ipsa contingat error . \textbf{ Quare cum in regimine ciuitatis primo sit politia ordinanda , } diu inuestigandum est , \\\hline
3.1.9 & primeramente se ha de ordenar la poliçia \textbf{ muy luengamente es de buscar e de escodrinnar } en qual manera la çibdat conuiene de ser vna & Quare cum in regimine ciuitatis primo sit politia ordinanda , \textbf{ diu inuestigandum est , } qualiter ciuitatem oportet esse unam , \\\hline
3.1.9 & en qual manera la çibdat conuiene de ser vna \textbf{ e qual departimiento deue auer enlła } e en qual manera conuiene alos çibdadanos de se auer los vnos con los otros & qualiter ciuitatem oportet esse unam , \textbf{ et quam diuersitatem habere debet , } et quomodo ciues decet se habere ad inuicem , \\\hline
3.1.9 & e qual departimiento deue auer enlła \textbf{ e en qual manera conuiene alos çibdadanos de se auer los vnos con los otros } e en quales cosas deuen partiçipar & et quam diuersitatem habere debet , \textbf{ et quomodo ciues decet se habere ad inuicem , } quibus communicare debent , \\\hline
3.1.9 & e en qual manera conuiene alos çibdadanos de se auer los vnos con los otros \textbf{ e en quales cosas deuen partiçipar } e si deuen todas las cosas ser comunes a todos & et quomodo ciues decet se habere ad inuicem , \textbf{ quibus communicare debent , } utrum omnia deberent \\\hline
3.1.9 & para las cosas que se siguen \textbf{ mas nos queremos en este capitulo mostrar primeramente } que conuiene ala çibdat & quasi principia \textbf{ et quasi quaedam praeambula ad sequentia . Volumus autem in hoc capitulo ostendere , } quod non expedit ciuitati habere omnia communia \\\hline
3.1.9 & que conuiene ala çibdat \textbf{ quer todas las cosas comunes } assi commo socrates ordeno . & et quasi quaedam praeambula ad sequentia . Volumus autem in hoc capitulo ostendere , \textbf{ quod non expedit ciuitati habere omnia communia } ut Socrates ordinauit : \\\hline
3.1.9 & e por ende en la çibdat seria muy grant amor \textbf{ Et pues que assi es nos podemos mostrar } por tres razones & ciues omnes pueros esse filios suos , \textbf{ et sic esset in ciuitate maximus amor . Possumus ergo triplici uia ostendere , } quod hoc non expedit ciuitati . Primo , \\\hline
3.1.9 & Et pues que assi es estando las possessiones comunes \textbf{ conuernia a cada vno de departir aquellas cosas } que fuessen meester & nutriat corpus alterius . \textbf{ Existentibus ergo possessionibus communibus oporteret } cuilibet distribui quae requiruntur ad supplendam indigentiam corporalem : \\\hline
3.1.9 & que fuessen meester \textbf{ para conplir la mengua del cuerpo } e por que el ome es engannado en su prouecho prop̃o & Existentibus ergo possessionibus communibus oporteret \textbf{ cuilibet distribui quae requiruntur ad supplendam indigentiam corporalem : } et quia homo nimis decipitur in proprio commodo , \\\hline
3.1.9 & sienpre le paresçria ael \textbf{ que deuia tomar mas que tomaua mas por que esto es cosa qua non puede ser } que todos los çibdadanos sean egualmente sabios & et quia homo nimis decipitur in proprio commodo , \textbf{ semper uidetur ei se plus debere recipere quam accipiat . Immo } quia impossibile est omnes ciues aequaliter esse prudentes \\\hline
3.1.9 & por que cada vno cuydaria \textbf{ que deuia mas resçebir } de quanto resçibe camientra & statim insurgeret rixa inter ciues , \textbf{ quia quilibet crederet plus esse accepturum : } ut dum unus ciuis iudicaret \\\hline
3.1.9 & que segunt la su dignỉdat le diessen mayor gualardo delas cosas comunes \textbf{ mas esta egualdat non se podria guardar de ligero entre los çibdadanos } sin contienda e sin uaraia & secundum dignitatem suam ei fieri retributionem . \textbf{ Hanc autem aequalitatem non de facili esset possibile reseruari } inter ciues absque dissensione et lite , \\\hline
3.1.9 & sin contienda e sin uaraia \textbf{ por que cada vno seyendo engannado en iuyzio de ssi mesmo preçiarse ya } mas de quanto valie . & inter ciues absque dissensione et lite , \textbf{ quia quilibet deceptus in iudicando de seipso , } appreciatur se plus ualere quam ualeat . Communitas ergo possessiones \\\hline
3.1.9 & que ordenaua socrates \textbf{ para escusar las contiendas } et las uaraias seria mas contienda de razon et de uaraia & appreciatur se plus ualere quam ualeat . Communitas ergo possessiones \textbf{ quam ordinabat Socrates ad uitandum lites , potius esset causa litigii quam pacis . } Nam \\\hline
3.1.9 & en el segundo libro \textbf{ esta es uida de o ensacabados non auer prỏo } e por ende los acabados son pocos¶ & ( \textbf{ ut supra in secundo libro tetigimus ) haec est uita perfectorum : non habere proprium : } perfecti autem sunt pauci . \\\hline
3.1.9 & e por ende los acabados son pocos¶ \textbf{ Et pues que assi es la ley se deue poner a todos los çibdadanos } e a todo el pueblo & perfecti autem sunt pauci . \textbf{ Lex ergo imponenda omnibus ciuibus } et toti populo debet esse talis , \\\hline
3.1.9 & e non fue tal ordenamiento de socrates \textbf{ por que se non podria guardar entre las gentes } que biuien comunalmente & cuiusmodi non fuit ordinatio Socratis , \textbf{ quia inter gentes communiter uiuentes absque multis litigiis obseruari non posset . Dato tamen communitatem possessionum posse obseruare absque litigiis , communitatem tamen foeminarum , propter quam uolebat Socrates filios esse communes , } non est possibile obseruare absque litigiis . \\\hline
3.1.9 & e muchas contiendas . \textbf{ Enpero puesto que la comunindat delas possessiones se podiesse guardar sinuamias } non se podria guardar la comuidat delas mugieres & ø \\\hline
3.1.9 & Enpero puesto que la comunindat delas possessiones se podiesse guardar sinuamias \textbf{ non se podria guardar la comuidat delas mugieres } por la qual quarie socrates que los fijos fuessen comunes esto non podria ser & ø \\\hline
3.1.9 & por la qual quarie socrates que los fijos fuessen comunes esto non podria ser \textbf{ nin se podria guardar sin muy grandes uaraias ¶ la segunda razon paresçe } assi ca si todas las cosas fuessen assi comunes & non est possibile obseruare absque litigiis . \textbf{ Secunda uia sic patet : } nam si omnia sic essent communia , \\\hline
3.1.9 & La terçera razon paresçe \textbf{ assi ca puesta tal comuindar delas mugers } non seria tanto amor en la çibdat commo cuydaua socrat̃s & ut ciues omnes pueros crederent esse proprios filios . \textbf{ Tertia via sic patet . Nam supposita communitate uxorum non esset tantus amor in ciuitate } ut opinabatur Socrates : \\\hline
3.1.9 & Mas conuiene alos Reyes \textbf{ e alos prençipes̃ de tener mientes a esto con grant acuçia } por que sepan ordenar la çibdat & Decet autem hoc Reges , \textbf{ et Principes diligenter aduertere , } ut sciant sic ciuitatem ordinare , \\\hline
3.1.9 & e alos prençipes̃ de tener mientes a esto con grant acuçia \textbf{ por que sepan ordenar la çibdat } assi commo conuiene ala comunidat de los çibdadanos & et Principes diligenter aduertere , \textbf{ ut sciant sic ciuitatem ordinare , } ut expedit communitati ciuium . \\\hline
3.1.10 & si las mugers e los fuos fueren puestos de ser comunes \textbf{ e quanto parte nesçe a lo presente podemos contar c̃co males } que se sigune de tal comuidat & si uxores et filii ponantur esse communes . \textbf{ Quantum autem ad praesens spectat , | enumerare possumus quinque mala , } quae Philosophus tangit ibidem sequentia ex tali communitate . Primum est , \\\hline
3.1.10 & El segundo es abiltamiento de los malos omes . \textbf{ El terçero es non auer cuydado delos fijos ¶ } El quarto es destenpremiento en las cosas dela lururia . & Secundum , vilificatio nobilium . Tertium , \textbf{ iniuria filiorum . Quartum , intemperantia venereorum . Quintum , } abusio parentum . Primum sic patet . \\\hline
3.1.10 & que en la çibdat se leunatenlides e feridas e deniestos \textbf{ las quales cosas tanto son mas de denostar } quento son acometidas en perssona mas cercana . & et contumelias , \textbf{ quae tanto detestabiliores sunt , } quanto committuntur in personam magis coniunctam . \\\hline
3.1.10 & e cada vnos sean çiertos de sus parientes \textbf{ por que por esta non sabiduria los fiios non ayan de fazer miurias } nin tuertosa sus padres e a sus parientes & et quoslibet certificari de eorum consanguineis , \textbf{ ne propter ignorantiam filii in proprios parentes } et consanguineos \\\hline
3.1.10 & segunt la manera del seruiçio \textbf{ e por ende querer } que los fijos de los çibdadanos sean comunes & et nobiles eis retribuunt mercedem \textbf{ secundum proportionem seruitii . Velle ergo filios ciuium communes esse , } et aequalem curam geri de filiis nobilium et ignobilium , est vilificare nobiles , \\\hline
3.1.10 & e de los fijos de aquellos que non son nobles . \textbf{ esto es vilificar los nobles } e enxalcar los viles & et aequalem curam geri de filiis nobilium et ignobilium , est vilificare nobiles , \textbf{ et exaltare ignobiles , } et non saluare amicitiam inter eos . Tertium malum sic declaratur . \\\hline
3.1.10 & esto es vilificar los nobles \textbf{ e enxalcar los viles } e assi se salua la amistança entre ellos & et exaltare ignobiles , \textbf{ et non saluare amicitiam inter eos . Tertium malum sic declaratur . } Nam supposita praedicta communitate , \\\hline
3.1.10 & non puede ser amistança acabada de vno a muchos \textbf{ nin puede vno amar mucha muchas personas en vno } por que mucha mar dize sobre puiqueça de amor & secundum perfectam amicitiam non contingit , \textbf{ nec contingit multas personas simul amare valde . } Nam valde amare , \\\hline
3.1.10 & si non fuesse loco \textbf{ en ninguna manera non podria sospechͣr } que todos los moços fuessen sus fijos . & nisi esset fatuus , \textbf{ nullo modo suspicari posset } omnes pueros esse suos filios . \\\hline
3.1.10 & e por que aquellos nonl serian conosçidos çiertamente por ende cuydaua son crates \textbf{ que todos los otros mocos aurie de amar muy de coraçon } por aquellos dos otres & et quia illi non essent eis certitudinaliter noti , opinabatur Socrates \textbf{ quod omnes alios intime diligerent propter illos . } Sed ut arguit Philosophus 2 Politic’ propter duos \\\hline
3.1.10 & assi conmo a fijos propreos \textbf{ esto es poner poco de miel en muchͣ agua . } Et pues que assi es & vel tres vel propter paucos pueros velle magnam multitudinem diligere puerorum tanquam proprios filios , hoc est ponere parum de melle in multa aqua . \textbf{ Sicut ergo parum mellis totum unum fluuium } non posset facere dulcem , \\\hline
3.1.10 & Et pues que assi es \textbf{ assi commo poca miel puesta en vn grant rio non puede fazer todo el rio dulçe } assi amor de dos o de tro fiios non puede faz & Sicut ergo parum mellis totum unum fluuium \textbf{ non posset facere dulcem , } sic amor duorum \\\hline
3.1.10 & de que son en vna çibdat \textbf{ nin puede fazer } que aquella muchedunbre sea plazible & existentium in ciuitate una , \textbf{ non posset reddere placibilem } et dilectam . \\\hline
3.1.10 & non estando el amost de los çibdadanos alos moços \textbf{ non se puede auer er cuydado conuenible de los moços } e dende se sigue & Sed non existente dilectione ciuium ad pueros , \textbf{ non habebitur eorum cura debita : } sequitur quod supposita communitate , \\\hline
3.1.10 & que puesta tal comunidat commo ordeno soctateᷤ \textbf{ non se puede auer cuydado conuenible } nin ç̉ança qual conuiene delos fijos . El quarto mal se puede assi mostrar & quam ordinauerat Socrates , \textbf{ non habeatur debita cura , } nec diligentia debita erga filios . Quartum malum sic ostendi potest quia virtus generatiua est ita corrupta , \\\hline
3.1.10 & non se puede auer cuydado conuenible \textbf{ nin ç̉ança qual conuiene delos fijos . El quarto mal se puede assi mostrar } ca la uertud de engendrar en el omne es assi corrupta & non habeatur debita cura , \textbf{ nec diligentia debita erga filios . Quartum malum sic ostendi potest quia virtus generatiua est ita corrupta , } et sic est insatiabilis concupiscentiae appetitus , \\\hline
3.1.10 & nin ç̉ança qual conuiene delos fijos . El quarto mal se puede assi mostrar \textbf{ ca la uertud de engendrar en el omne es assi corrupta } e la cobdiçia del appetito del omne estan sin fartura & non habeatur debita cura , \textbf{ nec diligentia debita erga filios . Quartum malum sic ostendi potest quia virtus generatiua est ita corrupta , } et sic est insatiabilis concupiscentiae appetitus , \\\hline
3.1.10 & que el se ouiesse conueinblemente \textbf{ e tenpradamente en vsar della } e por ende & quod non habente viro nisi unam uxorem , adhuc est valde difficile debite \textbf{ et temperate se habere erga illam . } Sicut ergo prouocata gula per multitudinem ciborum difficile est esse abstinentes , \\\hline
3.1.10 & por grant muchedunbre de uiandas \textbf{ es cosa guaue de fazer al omne astinençia . } assi el appetito de luxuria & ø \\\hline
3.1.10 & que sea el o entenprado ¶ \textbf{ El quinto mal se puede assy manifestar } ca quando los padres e las madres non ouiessen conosçimiento de sus propios fijos e de sus fijas de ligero auria & per multitudinem foeminarum difficile est \textbf{ et quasi impossibile est esse temperatum . Quintum autem malum sic manifestari potest . } Nam non habentibus parentibus cognitionem \\\hline
3.1.10 & e los padres con sus fijas . \textbf{ Empero socrates quariendo escusar este mal dix̉o } que al prinçipe dela çibdat pertenesçia de auer cuydado e acuçia & et patres filias . \textbf{ Socrates volens hoc inconueniens vitare , dixit , } quod spectabat ad Principem ciuitatis habere curam et diligentiam , \\\hline
3.1.10 & Empero socrates quariendo escusar este mal dix̉o \textbf{ que al prinçipe dela çibdat pertenesçia de auer cuydado e acuçia } por que los fijos non yoguiessen con sus madres & Socrates volens hoc inconueniens vitare , dixit , \textbf{ quod spectabat ad Principem ciuitatis habere curam et diligentiam , } ne filii coirent cum matribus , et patres cum filiabus . \\\hline
3.1.10 & e los padres con sus fiias \textbf{ non les podrian defender el amor luxurioso } e la cobdiçia carnal & et patri circa filiam , \textbf{ non prohibebatur eis amor libidinosus et concupiscentia , } ex quo non manifestabatur eis parentela illa . \\\hline
3.1.10 & por la qual cosa commo en las personas tan ayuntadas \textbf{ non solamente es de denostar el ayuntamiento carnal } mas avn la cobdiçia & Quare cum in personis \textbf{ tam coniunctis non solum detestabilis sit actualis commistio , } sed etiam abominabilis sit concupiscentia \\\hline
3.1.10 & mas avn la cobdiçia \textbf{ e el amor luyioso es much de aborresçer en tales personas } por ende much era de reprehender la opimon de socrates dela comunidat & sed etiam abominabilis sit concupiscentia \textbf{ et amor libidinosus , reprehensibilis erat opinio Socratis de communitate uxorum et filiorum . Decet ergo Reges } et Principes sic ordinare ciuitatem , ut prohibita communitate foeminarum \\\hline
3.1.10 & e el amor luyioso es much de aborresçer en tales personas \textbf{ por ende much era de reprehender la opimon de socrates dela comunidat } que puso delas mugeres e de los fijos . & sed etiam abominabilis sit concupiscentia \textbf{ et amor libidinosus , reprehensibilis erat opinio Socratis de communitate uxorum et filiorum . Decet ergo Reges } et Principes sic ordinare ciuitatem , ut prohibita communitate foeminarum \\\hline
3.1.10 & Et pues que assi es conuiene alos Reyes \textbf{ e alos prinçipes de ordenar assi la çibdat } por que defendia la comunidat delas fenbras & et amor libidinosus , reprehensibilis erat opinio Socratis de communitate uxorum et filiorum . Decet ergo Reges \textbf{ et Principes sic ordinare ciuitatem , ut prohibita communitate foeminarum } et uxorum certificentur parentes \\\hline
3.1.11 & en el segundo libro delas politicas \textbf{ en tres maneras se puede entender } ca en las cosas que siruen ala uida e ala uianda del omne auemos de penssar dos cosas & ut ait Philosophus 2 Politic’ , \textbf{ tripliciter potest intelligi . } Nam in rebus deseruientibus ad victum est considerare duo : \\\hline
3.1.11 & en tres maneras se puede entender \textbf{ ca en las cosas que siruen ala uida e ala uianda del omne auemos de penssar dos cosas } Conuiene a saber las cosas & tripliciter potest intelligi . \textbf{ Nam in rebus deseruientibus ad victum est considerare duo : } videlicet res fructiferas , \\\hline
3.1.11 & ca en las cosas que siruen ala uida e ala uianda del omne auemos de penssar dos cosas \textbf{ Conuiene a saber las cosas } que lie una fructo & tripliciter potest intelligi . \textbf{ Nam in rebus deseruientibus ad victum est considerare duo : } videlicet res fructiferas , \\\hline
3.1.11 & mas assi commo dize el philosofo en el segundo libro delas politicas \textbf{ en los fechos particulares conuiene de venir ala prueua } ca veemos e prouamos & inter quos tanta communitas obseruatur . \textbf{ Sed ut dicitur secundo Politicorum in actibus particularibus oportet ad experientiam recurrere : experti enim sumus } quod habentes aliqua communia , \\\hline
3.1.11 & ssi han mayores contiendas que si cada vno ouiesse sus cosas proprias \textbf{ e esto podemos prouar } por tres razones & plura litigia inter se habent , \textbf{ quam si quilibet propria possideret . Possumus autem triplici via inuestigare , } quod si res ciuium communes essent , \\\hline
3.1.11 & commo paresçe al otro \textbf{ que della deuia usar . } Et pues que assi es & dum unus haereditate illa non utitur \textbf{ prout alteri expedire videtur . } Si ergo fratres uterini \\\hline
3.1.11 & que non es çierto \textbf{ non podria tirar tales varaias entre muchos } assi commo entre los çibdadanos . & non posset tollere \textbf{ inter multos , } ut inter omnes ciues . Secunda via ad inuestigandum \\\hline
3.1.11 & La segunda razon \textbf{ para prouar esto mesmo se toma de parte de los que han la heredat en comun } ca quanto alguons han mas cosas en comun & hoc idem , \textbf{ sumitur ex parte communicantium in haereditate communi . Nam quanto aliqui plura habent communia , } tanto magis ad inuicem conuersantur : sed cum esse non possit , \\\hline
3.1.11 & ca quanto alguons han mas cosas en comun \textbf{ tanto mas han de beuir en vno } mas commo non pueda ser & sumitur ex parte communicantium in haereditate communi . Nam quanto aliqui plura habent communia , \textbf{ tanto magis ad inuicem conuersantur : sed cum esse non possit , } aliquos valde et diu conuersari ad inuicem , \\\hline
3.1.11 & los que han la heredat en comun paresçe \textbf{ por que han de beuir en vno } que por la mayor parte han contiendas e uaraias por la qual cosa dize el philosofo en el segundo libro de las politicas & ex parte ipsorum communicantium in haereditate communi , \textbf{ eo quod oporteat eos valde ad inuicem conuersari , | ostenditur } ut plurimum homines habere lites \\\hline
3.1.11 & e nos enssannamos contra ellos muchͣs uezes \textbf{ por que nos conuiene de fablar muchͣs uezes con ellos } e de beuir conellos non los podiendo escusar & maxime offendimur , et indignamur erga illos , \textbf{ quia oportet nos habere ad illos multa colloquia , } et diu conuersari cum illis . \\\hline
3.1.11 & por que nos conuiene de fablar muchͣs uezes con ellos \textbf{ e de beuir conellos non los podiendo escusar } que si entre los sennoron e los sus sieruos & quia oportet nos habere ad illos multa colloquia , \textbf{ et diu conuersari cum illis . } Quare si \\\hline
3.1.11 & por que serie muy guaue \textbf{ e muy costoso de buscar tanta muchedunbre de labradores } que pudiessen tanta tierra labrar conuerne & quia valde difficile esset \textbf{ et valde sumptuosom tantam multitudinem extraneorum quaerere | qui possent } tantam terram colere : expediret ciuitati sic diuidi , \\\hline
3.1.11 & e muy costoso de buscar tanta muchedunbre de labradores \textbf{ que pudiessen tanta tierra labrar conuerne } que la çibdat fuesse assi partida & qui possent \textbf{ tantam terram colere : expediret ciuitati sic diuidi , } ut aliqui ciuium terram colerent , \\\hline
3.1.11 & e algunos estudiessen en guarda e en defendimiento dela çibdat \textbf{ e esto otorgauas ocrates . } Et pues que & ut aliqui ciuium terram colerent , \textbf{ aliqui vero insisterent circa custodiam ciuitatis , quae et Socrates concedebat . } Cum ergo custodes ciuitatis nobiliores sint agricolis , tanquam meliores et nobiliores estimarent se plus esse accepturos \\\hline
3.1.11 & e mas nobles cuydaria \textbf{ que aurian de reçebir } mas de los fructos delas possessiones que los labradores & Cum ergo custodes ciuitatis nobiliores sint agricolis , tanquam meliores et nobiliores estimarent se plus esse accepturos \textbf{ de fructibus possessionum , } quam illi . Insurgeret igitur dissensio inter eos , \\\hline
3.1.11 & e comunes por uirtud de liberalidat \textbf{ que deuen ser francos en las partir . } Et pues que assi es catada la condicion de los omes acuçiosamente & ø \\\hline
3.1.11 & Et pues que assi es catada la condicion de los omes acuçiosamente \textbf{ en quanto el pueblo deue guardar la ley } e las buenos ordenamientos & Diligenter igitur inspecta humana conditione , \textbf{ prout communiter populus observatiuus est legum } et laudabilium ordinationum , \\\hline
3.1.11 & e las buenos ordenamientos \textbf{ conuiene alos çibdadanos de auer las cosas } e las possessiones prop̃as & et laudabilium ordinationum , \textbf{ expedit cuilibet habere res et possessiones proprias } quantum ad dominum : \\\hline
3.1.11 & ca cada vno de los çibdadanos \textbf{ quando auia meester alguna cosa sin la demandar al otro vsauad los cauallos } e de los canes & quilibet enim ciuium cum indigebat , \textbf{ absque alia requisitione utebatur alterius equis , canibus , et seruis . } Ut in praecedentibus dicebatur , \\\hline
3.1.12 & ordeno que las mugers deuian ser enssennadas alas obras dlas batallas \textbf{ en manera que pudiessen lidiar . } e dizie que deuien lidiar & Ut in praecedentibus dicebatur , \textbf{ Socrates statuit mulieres instruendas esse ad opera bellica , } et debere bellare , \\\hline
3.1.12 & en manera que pudiessen lidiar . \textbf{ e dizie que deuien lidiar } assi commo los omes & Socrates statuit mulieres instruendas esse ad opera bellica , \textbf{ et debere bellare , } sicut et viri : \\\hline
3.1.12 & a quien parte nesçe \textbf{ de ordenar la çibdat } non deuen ordenar las muger & Sed quod Reges et Principes , \textbf{ et uniuersaliter illi quorum est ciuitatem disponere , } non debeant ordinare mulieres ad opera bellica , \\\hline
3.1.12 & de ordenar la çibdat \textbf{ non deuen ordenar las muger } sa obras de batallas & et uniuersaliter illi quorum est ciuitatem disponere , \textbf{ non debeant ordinare mulieres ad opera bellica , } sed viros , \\\hline
3.1.12 & sa obras de batallas \textbf{ nin a lidiar } mas los omes esto podemos prouar & non debeant ordinare mulieres ad opera bellica , \textbf{ sed viros , } triplici via venari possumus , \\\hline
3.1.12 & nin a lidiar \textbf{ mas los omes esto podemos prouar } por tres razones & non debeant ordinare mulieres ad opera bellica , \textbf{ sed viros , } triplici via venari possumus , \\\hline
3.1.12 & si fueren arteros \textbf{ e engennosos en lidiar vençena muchos } non por fortaleza mas por sabiduria . & pauci enim bellatores si sint sagaces \textbf{ et industres superant multos non fortitudine } sed prudentia . \\\hline
3.1.12 & nin tan sabias commo los omes . \textbf{ Et por ende non son de poner en las } batallasca grant cautela e grant sabiduria & et prouidae sicut viri , \textbf{ non sunt ordinandae ad opera bellica . } Nam in bellis magna cautela et industria est adhibenda , \\\hline
3.1.12 & segunt dizeuegeçio en el libro del negoçio dela caualleria \textbf{ ca si las otras cosas mal fechͣs se pueden cobrar . } Enpero las auenturas delas batallas non han remedio ninguno . & quia secundum Vegetium in De re militari , \textbf{ si alia male acta recuperari possunt , } casus \\\hline
3.1.12 & Et por ende dela sabiduria \textbf{ que es meester en las batallas podemos tomar argumento } que las mugers non son de enssennar & casus \textbf{ tamen bellorum irremediabiles sunt . Ex ipsa igitur industria , quae requiritur in bellantibus , arguere possumus mulieres instruendas non esse ad opera bellica . } Secunda via ad inuestigandum hoc idem , \\\hline
3.1.12 & que es meester en las batallas podemos tomar argumento \textbf{ que las mugers non son de enssennar } nin de pouer alas batallas & casus \textbf{ tamen bellorum irremediabiles sunt . Ex ipsa igitur industria , quae requiritur in bellantibus , arguere possumus mulieres instruendas non esse ad opera bellica . } Secunda via ad inuestigandum hoc idem , \\\hline
3.1.12 & que las mugers non son de enssennar \textbf{ nin de pouer alas batallas } La segunda razon & tamen bellorum irremediabiles sunt . Ex ipsa igitur industria , quae requiritur in bellantibus , arguere possumus mulieres instruendas non esse ad opera bellica . \textbf{ Secunda via ad inuestigandum hoc idem , } sumitur ex virilitate \\\hline
3.1.12 & La segunda razon \textbf{ para prouar estomesmo se toma dela fortaleza de uirtud } e del buen coraçon & Secunda via ad inuestigandum hoc idem , \textbf{ sumitur ex virilitate } et animositate , \\\hline
3.1.12 & ca assi commo es dicho de suso la frialdat apareia carrera al temor \textbf{ por que el frio a restrennir e apretar } mas los aionsos e de grant coraçon e los esforçados han se de estendera muchos cosas . & nam \textbf{ ( ut dicebatur supra ) frigiditas viam timori praeparat ; frigidi enim est costringere } et retrahere ; animosi vero et virilis est ad alia se extendere , \\\hline
3.1.12 & e de flaco coraçon \textbf{ e non se deuen enbiar alas batallas } ca meior cosa es de echar los temerosos dela batalla & et pusillanimes , \textbf{ ad opera bellica destinari non debent . } In bellis enim melius est pauidos expellere , \\\hline
3.1.12 & e non se deuen enbiar alas batallas \textbf{ ca meior cosa es de echar los temerosos dela batalla } que auerlos en su conpannia & ad opera bellica destinari non debent . \textbf{ In bellis enim melius est pauidos expellere , } quam eos in societate habere \\\hline
3.1.12 & ca meior cosa es de echar los temerosos dela batalla \textbf{ que auerlos en su conpannia } ca commo todos los omes teman la muerte los esforçados & In bellis enim melius est pauidos expellere , \textbf{ quam eos in societate habere } nam \\\hline
3.1.12 & Et por ende por que los lidiadores non se enflaquezcan en las batallas \textbf{ conuiene de echar dela batalla } e dela fazienda alos de flaco coraçon & cum humanum sit timere mortem , viriles etiam et animosi trepidant videntes timidos trepidare : \textbf{ ne igitur reddantur bellantes pusillanimes , } quos constat esse timidos oportet ab exercitu expelli . \\\hline
3.1.12 & La terçera razon se toma de parte dela fortaleza corporal \textbf{ ca commo los lidiadores ayan de sofrir el peso delas armas } e ayan de dar grandes colpes & Tertia via sumitur ex parte fortitudinis corporalis . \textbf{ Nam cum bellantes oporteat diu sustinere armorum pondera , et dare magnos ictus , } expedit eos habere magnos humeros et renes ad sustinendum armorum grauedinem , \\\hline
3.1.12 & ca commo los lidiadores ayan de sofrir el peso delas armas \textbf{ e ayan de dar grandes colpes } conuieneles de auer fuertes honbros e fuertes rennes & Tertia via sumitur ex parte fortitudinis corporalis . \textbf{ Nam cum bellantes oporteat diu sustinere armorum pondera , et dare magnos ictus , } expedit eos habere magnos humeros et renes ad sustinendum armorum grauedinem , \\\hline
3.1.12 & e ayan de dar grandes colpes \textbf{ conuieneles de auer fuertes honbros e fuertes rennes } para sofrir la pesadura delas armas & Nam cum bellantes oporteat diu sustinere armorum pondera , et dare magnos ictus , \textbf{ expedit eos habere magnos humeros et renes ad sustinendum armorum grauedinem , } et habere fortia brachia ad faciendum percussiones fortes : \\\hline
3.1.12 & conuieneles de auer fuertes honbros e fuertes rennes \textbf{ para sofrir la pesadura delas armas } e conuiene les de auer fuertes braços & Nam cum bellantes oporteat diu sustinere armorum pondera , et dare magnos ictus , \textbf{ expedit eos habere magnos humeros et renes ad sustinendum armorum grauedinem , } et habere fortia brachia ad faciendum percussiones fortes : \\\hline
3.1.12 & para sofrir la pesadura delas armas \textbf{ e conuiene les de auer fuertes braços } para fazer fuertes colpes . & expedit eos habere magnos humeros et renes ad sustinendum armorum grauedinem , \textbf{ et habere fortia brachia ad faciendum percussiones fortes : } mulieres igitur \\\hline
3.1.12 & e conuiene les de auer fuertes braços \textbf{ para fazer fuertes colpes . } Et por que las mugers esto non pueden auer & expedit eos habere magnos humeros et renes ad sustinendum armorum grauedinem , \textbf{ et habere fortia brachia ad faciendum percussiones fortes : } mulieres igitur \\\hline
3.1.12 & para fazer fuertes colpes . \textbf{ Et por que las mugers esto non pueden auer } por que han las carnes muelles & et habere fortia brachia ad faciendum percussiones fortes : \textbf{ mulieres igitur } eo quod habent carnes molles \\\hline
3.1.12 & que monio a socrates \textbf{ para poner esto } que las mugers auian de yr ala batalla & ø \\\hline
3.1.12 & para poner esto \textbf{ que las mugers auian de yr ala batalla } la qual razon tomo delas semesaças delas bestias & Ratio autem quae mouit Socratem \textbf{ ad hoc ponendum sumpta a similitudine bestiarum , } insufficiens est : \\\hline
3.1.12 & por que segunt el philosofo en el segundo libro delas politicas alas bestias \textbf{ non parte nesçe de ordenar casanin çibdat } por que non han razon nin entendimiento . & secundum Philosophum 2 Poli’ bestiis nihil attinet oeconomice , \textbf{ nec participant ratione . Hominis ergo est } secundum debitam oeconomiam \\\hline
3.1.12 & Et pues que assi es los omes \textbf{ a quien parte nesçe de ordenar la casa e la çibdat } segunt ordenamiento conueinble en aquellas cosas & secundum debitam oeconomiam \textbf{ et | secundum debitam dispensationem ordinare domum } et ciuitatem . \\\hline
3.1.12 & segunt ordenamiento conueinble en aquellas cosas \textbf{ que las bestias fazen sin razon los omes non las deuen segnir } ize socrates & et ciuitatem . \textbf{ Quare in iis , in quibus bestiae | praeter rationem agunt , eas sequi non debent . } Dicebat Socrates semper eosdem debere esse Principes in ciuitate . \\\hline
3.1.13 & por cada vn sennorio o por cada vn maestradgo o por cada vn ofiçio \textbf{ mas podemos mostrar por tons razones } que non es conuenible ala çibdat & siue pro qualibet praepositura . \textbf{ Possumus autem triplici via ostendere , } quod non sit expediens ciuitati semper praeponere eosdem in eisdem magistratibus . Prima via sumitur \\\hline
3.1.13 & o de qual quier otro \textbf{ que aya de partir } e de dar las dignidades e los ofiçios & cuiuscunque alterius \textbf{ cuius est tales dignitates tribuere . } Secunda ex parte ipsorum praepositorum . \\\hline
3.1.13 & que aya de partir \textbf{ e de dar las dignidades e los ofiçios } La segunda de parte de los prebostes e de los mayorales & cuiuscunque alterius \textbf{ cuius est tales dignitates tribuere . } Secunda ex parte ipsorum praepositorum . \\\hline
3.1.13 & por que non paran mientes tantos oios en la persona priuada commo en la publica por ende non se puede \textbf{ assi saber } quales el uaron & et quia non tot oculi respiciunt personam priuatam quam publicam , \textbf{ non scitur qualis sit vir , } donec est persona priuata non habens potestatem , \\\hline
3.1.13 & por la qual razon commo venga ala real magestad \textbf{ e generalmente a qual quier que ha de dar } e partir los maestradgos & Quare cum deceat regia maiestatem \textbf{ et uniuersaliter omnem ciuem , } cuius est magistratus et praeposituras distribuere , \\\hline
3.1.13 & e generalmente a qual quier que ha de dar \textbf{ e partir los maestradgos } e las diguidades de conosçer & et uniuersaliter omnem ciuem , \textbf{ cuius est magistratus et praeposituras distribuere , } cognoscere quales praeficiant praepositos \\\hline
3.1.13 & e partir los maestradgos \textbf{ e las diguidades de conosçer } quales pone en los prinçipados e enlos maestradgos & et uniuersaliter omnem ciuem , \textbf{ cuius est magistratus et praeposituras distribuere , } cognoscere quales praeficiant praepositos \\\hline
3.1.13 & e de los maestradgos \textbf{ de auer primero prueua de los çibdadanos } quales son aquellos que entienden poner en los ofiçioso en los maestradgos o en las dignidades & et manifestat , expedit tribuentem praeposituras \textbf{ et magistratus super ciues prius experiri quales sint , } quos praefecit in praepositos vel magistros , \\\hline
3.1.13 & de auer primero prueua de los çibdadanos \textbf{ quales son aquellos que entienden poner en los ofiçioso en los maestradgos o en las dignidades } e assi auida prueua çierta & et magistratus super ciues prius experiri quales sint , \textbf{ quos praefecit in praepositos vel magistros , } et postea experimento assumpto \\\hline
3.1.13 & e assi auida prueua çierta \textbf{ despueᷤlos deue poner en los dichos ofiçios } e confirmar los enlleros & et postea experimento assumpto \textbf{ debet | eos in dictis officiis affirmare , } vel ad maiora assumere prout ei videbitur melius expedire . \\\hline
3.1.13 & despueᷤlos deue poner en los dichos ofiçios \textbf{ e confirmar los enlleros } o si son dignos proueer los a mayores & ø \\\hline
3.1.13 & e confirmar los enlleros \textbf{ o si son dignos proueer los a mayores } segunt que uiere que cunple . & eos in dictis officiis affirmare , \textbf{ vel ad maiora assumere prout ei videbitur melius expedire . } Secunda via ad inuestigandum hoc idem , \\\hline
3.1.13 & ¶ La segunda razon \textbf{ para prouar esto mismo se toma de parte de los prebostes } e de losi mayorales & vel ad maiora assumere prout ei videbitur melius expedire . \textbf{ Secunda via ad inuestigandum hoc idem , } sumitur ex parte ipsorum praepositorum . \\\hline
3.1.13 & e mucho tuertos acometrian los ofiçiales delas çibdades \textbf{ si sopiesse que sienpre les auian de durar los ofiçios } e que nunca les auian de tirar dellos ¶ & et multas iniustitias committerent praepositi ciuitatem , \textbf{ si scirent se esse perpetuo tales , } a quibus cauerent \\\hline
3.1.13 & si sopiesse que sienpre les auian de durar los ofiçios \textbf{ e que nunca les auian de tirar dellos ¶ } La terçera razon se toma de parte de los çibdadanos & si scirent se esse perpetuo tales , \textbf{ a quibus cauerent | si considerarent se esse ab huiusmodi officio remouendos . } Tertia via sumitur ex parte ipsorum ciuium , \\\hline
3.1.13 & sobre los quales son puestos tales ofiçiales \textbf{ ca assi commo el fisico non ha de tomar conseio dela sanidat } mas ha de tener mienteᷤ en la sanidat & quibus huiusmodi praepositi praeponuntur . \textbf{ Nam sicut medici non est consiliari de sanitate , } sed eius est intendere sanitatem tanquam finem : \\\hline
3.1.13 & ca assi commo el fisico non ha de tomar conseio dela sanidat \textbf{ mas ha de tener mienteᷤ en la sanidat } assi comm̃en su fin & Nam sicut medici non est consiliari de sanitate , \textbf{ sed eius est intendere sanitatem tanquam finem : } sic cuius est ciuitatem ordinare , \\\hline
3.1.13 & bien \textbf{ assi aquel que ha de ordenar la çibdat } non ha de tomar conseio dela paz & sed eius est intendere sanitatem tanquam finem : \textbf{ sic cuius est ciuitatem ordinare , } non est consiliari de pace et de concordia ciuium . \\\hline
3.1.13 & assi aquel que ha de ordenar la çibdat \textbf{ non ha de tomar conseio dela paz } e dela concordia de los çibdadanos & sic cuius est ciuitatem ordinare , \textbf{ non est consiliari de pace et de concordia ciuium . } Nam ut dicitur 3 Ethic’ \\\hline
3.1.13 & e la paz e la concordia dela çibdat es fin \textbf{ que deue entender todo gouernador dela çibdat } e por ende non ha de tomar conseio sobre ella & ad finem pacem enim \textbf{ et concordiam ciuium debet intendere rector ciuitatis tanquam finem . } Sic ergo disponenda est ciuitas , \\\hline
3.1.13 & que deue entender todo gouernador dela çibdat \textbf{ e por ende non ha de tomar conseio sobre ella } mas assi ha de ordenar la çibdat & et concordiam ciuium debet intendere rector ciuitatis tanquam finem . \textbf{ Sic ergo disponenda est ciuitas , } ut conseruetur in pace , \\\hline
3.1.13 & e por ende non ha de tomar conseio sobre ella \textbf{ mas assi ha de ordenar la çibdat } que sienpre sea guardada en pas e en concordia & et concordiam ciuium debet intendere rector ciuitatis tanquam finem . \textbf{ Sic ergo disponenda est ciuitas , } ut conseruetur in pace , \\\hline
3.1.13 & ca quando veen \textbf{ que non pueden auer nin gͤdignidat en la çibdat } si contezca & apud nullam dignitatem possidentes : \textbf{ videntes enim nullam dignitatem possidere , } si contingat eos esse viriles et animosos , \\\hline
3.1.14 & que non fablaron uirazon \textbf{ por ende bienes de contar sus o pimones } non por razon de vana gloria & non sine ratione locutos fuisset \textbf{ ideo bene se habet eorum opiniones tractare , } non ostentationis causa , \\\hline
3.1.14 & mas por que esto demanda esta arte presente \textbf{ que entendemos dar del gouernamiento dela çibdat } e del regno & sed quia hoc requirit praesens methodus , \textbf{ quam intendimus tradere de regimine ciuitatis , et regni . } Viso enim quid circa hoc senserunt Philosophi , \\\hline
3.1.14 & mas claramente paresçra \textbf{ que auemos de iudgar en este gouernamiento } por la quel cosa commo sea manifiesto & clarius apparebit \textbf{ quid circa huiusmodi regimen sit censendum . Quare cum patefactum sit in praecedentibus , } non expedire ciuitati possessiones , uxores , \\\hline
3.1.14 & assi commo paresçia \textbf{ que ordenauas octates . } pues que assi es finca de tractar del sdepartimiento & eosdem semper in eisdem magistratibus praefici , \textbf{ ut Socrates statuisse videbatur . } Restat ergo exequi de iussione et ordine ciuitatis , \\\hline
3.1.14 & que ordenauas octates . \textbf{ pues que assi es finca de tractar del sdepartimiento } e de la ordenaçion dela çibdat & ut Socrates statuisse videbatur . \textbf{ Restat ergo exequi de iussione et ordine ciuitatis , } quam Socrates ordinauit . \\\hline
3.1.14 & assi commo es dicho desuso \textbf{ que toda çibdat deue auer enssi çinco cosas conuiene a saber . } El prinçipe ¶ Los conseieros . & Dicebat enim Socrates , \textbf{ ut supra tangebam , | ciuitatem quinque in se debere habere , } videlicet Principem , \\\hline
3.1.14 & Et este establesçimiento de socrates \textbf{ quanto alos lidiadores en tres maneras se puede reprehender } segunt aquellas tres cosas & ad minus mille . \textbf{ Huiusmodi autem statutum quantum ad bellatores tripliciter improbari potest , } secundum tria quae de ipsis bellantibus Socrates statuebat . \\\hline
3.1.14 & ¶ \textbf{ Lo terçero les poner } e en cuento determinado . & Secundo statuebat magnam multitudinem bellatorum . \textbf{ Tertio ponebat eos in quodam determinato numero . } Prima via sit patet . \\\hline
3.1.14 & nin de los otros çibdadanos \textbf{ que sienpre los çibdadanos non les conuenga de lidiar } por defendimiento de su tierra & et ab aliis ciuibus , \textbf{ quod ciues alii pro defensione patriae bellare non oporteat melius est ergo dicere in ciuitate tot esse bellatores } et defensores patriae , \\\hline
3.1.14 & quantos son y çibdadanos \textbf{ que pue dan tomar armas } e esto es meior & et defensores patriae , \textbf{ quot sunt ibi ciues valentes portare arma , } quam seperare bellatores ab aliis ciuibus . \\\hline
3.1.14 & e esto es meior \textbf{ que dezir que sean apartados los bdiadores de los otros çibdadanos ¶ } La segunda razon se prueua & quot sunt ibi ciues valentes portare arma , \textbf{ quam seperare bellatores ab aliis ciuibus . } Secunda via sic patet , \\\hline
3.1.14 & La segunda razon se prueua \textbf{ assy ca establesçer tan grant muchedunbre de lidiadores en cada vna delas çibdades } assi que sean çinco miłl omin lesto serie muy graue & Secunda via sic patet , \textbf{ nam constituere tantam multitudinem bellatorum in qualibet ciuitate } ut quinque milia , \\\hline
3.1.14 & e graue cosa serie alos çibdadanos de vna çibdat \textbf{ mantener mill caualleros de las rentas comunes de vna çibdat } los quales caualleros non ouiessen otro ofiçio ninguno si non lidiar & esse valde difficile \textbf{ et onerosum ipsis ciuibus . Onerosum enim et difficile esset ciuibus unius ciuitatis sustentare mille viros in stipendiis communibus , } quorum nullum esset aliud officium , nisi bellare , \\\hline
3.1.14 & mantener mill caualleros de las rentas comunes de vna çibdat \textbf{ los quales caualleros non ouiessen otro ofiçio ninguno si non lidiar } quando fuesse me este & et onerosum ipsis ciuibus . Onerosum enim et difficile esset ciuibus unius ciuitatis sustentare mille viros in stipendiis communibus , \textbf{ quorum nullum esset aliud officium , nisi bellare , } cum adesset oportunitas : \\\hline
3.1.14 & Et muy mayor carga \textbf{ e peor de sofrir serie } si ouiessen a mantener cinco mill caualleros & cum adesset oportunitas : \textbf{ et onerosius et quasi omnino importabile esset sustentare sic quinque milia : } oporteret enim ciuitatem illam habere possessiones quasi ad votum , \\\hline
3.1.14 & e peor de sofrir serie \textbf{ si ouiessen a mantener cinco mill caualleros } ca conuerne que aquella çibdat ouiesse tantas possessiones & cum adesset oportunitas : \textbf{ et onerosius et quasi omnino importabile esset sustentare sic quinque milia : } oporteret enim ciuitatem illam habere possessiones quasi ad votum , \\\hline
3.1.14 & quantas quisiesse a ssu uoluntad \textbf{ por que pudiesse de las rentas comunes abondar atanta muchedunbre } por la qual cosa el philosofo en el libro delas politicas reprehende a socrates deste tal gouernamiento de çibdat & oporteret enim ciuitatem illam habere possessiones quasi ad votum , \textbf{ ut posset ex communibus sumptibus tantam multitudinem pascere . } Propter quod Philosophus 2 Politicorum reprehendens Socratem de huiusmodi ordine ciuitatis , \\\hline
3.1.14 & ay grant espaçio de tierras \textbf{ de que se podria gouernar grant muchedunbre de gente } ca guaue cosa es & ubi forte propter magnitudinem desertorum est magnum spatium terrarum , \textbf{ ex quo posset magna multitudo pasci . } Difficile est enim , \\\hline
3.1.14 & ca guaue cosa es \textbf{ assi commo dize el philosofo gouernar atanta muchedunbre de lidiadores sin çibdadanos e sin mugers e sin siruientes e sin fijos } Lo terçero erraua socrates en el ordenamiento dela çibdat & Difficile est enim , \textbf{ ut Philosophus innuit , | praeter ipsos ciues } et praeter mulieres , \\\hline
3.1.14 & ca segunt dize el philosofo en el segundo libro delas politicas \textbf{ el que quiere poner leyes o fazer ordenaçion alguna en la çibdat } a tres co sas deue deuer mietes . & Nam secundum Philosophum secundo Politicorum , \textbf{ volens ponere leges | vel facere ordinationem aliquam in ciuitate , } ad tria debet respicere , \\\hline
3.1.14 & el que quiere poner leyes o fazer ordenaçion alguna en la çibdat \textbf{ a tres co sas deue deuer mietes . } ¶ Conuiene a saber alos omes Et al regno . & vel facere ordinationem aliquam in ciuitate , \textbf{ ad tria debet respicere , } scilicet ad homines , \\\hline
3.1.14 & a tres co sas deue deuer mietes . \textbf{ ¶ Conuiene a saber alos omes Et al regno . } Et alos logares & ad tria debet respicere , \textbf{ scilicet ad homines , | ad regionem , } et ad loca vicina . \\\hline
3.1.14 & que son uezinos . \textbf{ Et pues que assi es si alguno quisiesse sin los çibdadanos establesçer otros lidiadores } que uiuiessen de los fructos e de los bienes comunes & et ad loca vicina . \textbf{ Si quis ergo vellet praeter ciues statuere aliquos bellatores viuentes ex communibus fructibus ciuitatis , } ad tria deberet respicere . \\\hline
3.1.14 & que uiuiessen de los fructos e de los bienes comunes \textbf{ dela çibdat a tres cosas deuia tener mientes } conuiene a saber . a los çibdadanos & Si quis ergo vellet praeter ciues statuere aliquos bellatores viuentes ex communibus fructibus ciuitatis , \textbf{ ad tria deberet respicere . } Primo ad ciues : \\\hline
3.1.14 & dela çibdat a tres cosas deuia tener mientes \textbf{ conuiene a saber . a los çibdadanos } ca si los çibdadanos fuessen de flacos coraçones & Si quis ergo vellet praeter ciues statuere aliquos bellatores viuentes ex communibus fructibus ciuitatis , \textbf{ ad tria deberet respicere . } Primo ad ciues : \\\hline
3.1.14 & ca si los çibdadanos fuessen de flacos coraçones \textbf{ e sin prouech para lidiar } e para se defender aurian meester mayor conpanna delidiadores & ad tria deberet respicere . \textbf{ Primo ad ciues : } nam si ciues ipsi essent pusillanimes et inutiles ad bellum , indigeret maiori copia bellatorum sic se habentium . Secundo inspiciendum esset ad regionem : \\\hline
3.1.14 & e sin prouech para lidiar \textbf{ e para se defender aurian meester mayor conpanna delidiadores } que los pudiessen ayudar & ad tria deberet respicere . \textbf{ Primo ad ciues : } nam si ciues ipsi essent pusillanimes et inutiles ad bellum , indigeret maiori copia bellatorum sic se habentium . Secundo inspiciendum esset ad regionem : \\\hline
3.1.14 & e para se defender aurian meester mayor conpanna delidiadores \textbf{ que los pudiessen ayudar } Lo segundo deuen tener mientes al regno & Primo ad ciues : \textbf{ nam si ciues ipsi essent pusillanimes et inutiles ad bellum , indigeret maiori copia bellatorum sic se habentium . Secundo inspiciendum esset ad regionem : } nam quanto ciuitas illa maiori regione \\\hline
3.1.14 & que los pudiessen ayudar \textbf{ Lo segundo deuen tener mientes al regno } ca quanto la çibdat esta en mayor regnado & Primo ad ciues : \textbf{ nam si ciues ipsi essent pusillanimes et inutiles ad bellum , indigeret maiori copia bellatorum sic se habentium . Secundo inspiciendum esset ad regionem : } nam quanto ciuitas illa maiori regione \\\hline
3.1.14 & e vsa de mayor espaçio de tierras \textbf{ tanto mayor cuento de lidiadores pueden mantener ¶ } Lo terçero deuen tener mientes alos logares & et maiori terrarum spatio potiretur , \textbf{ tanto sustentare posset maiorem numerum bellantium . } Tertio aspiciendum esset \\\hline
3.1.14 & tanto mayor cuento de lidiadores pueden mantener ¶ \textbf{ Lo terçero deuen tener mientes alos logares } quel son uezinos assi commo si aquella çibdat ouiesse çerca & tanto sustentare posset maiorem numerum bellantium . \textbf{ Tertio aspiciendum esset } ad loca vicina , \\\hline
3.1.14 & Ca departidas las condiçonnes de los uezinos \textbf{ departidamente se deue tomar el cuento de los lidiadores } por la qual cosa la arte e la sçiençia non pueden ser cerca las cosas particulares e senñaladas . El que quiere dar arte e sçiençia de gouernamiento dela çibdat & Nam variatis conditionibus vicinorum , \textbf{ oportet aliter | et aliter determinare de conditionibus bellantium . } Quare cum ars \\\hline
3.1.14 & departidamente se deue tomar el cuento de los lidiadores \textbf{ por la qual cosa la arte e la sçiençia non pueden ser cerca las cosas particulares e senñaladas . El que quiere dar arte e sçiençia de gouernamiento dela çibdat } non puede escablesçer cuento determinado de los lidiadores & et aliter determinare de conditionibus bellantium . \textbf{ Quare cum ars | et scientia non possit esse circa particularia signata , } volens tradere artem de regimine ciuitatum , \\\hline
3.1.14 & por la qual cosa la arte e la sçiençia non pueden ser cerca las cosas particulares e senñaladas . El que quiere dar arte e sçiençia de gouernamiento dela çibdat \textbf{ non puede escablesçer cuento determinado de los lidiadores } mas tales cosas commo estas son de dexar auyzio de gouernador sabio & volens tradere artem de regimine ciuitatum , \textbf{ non potest statuere determinatum numerum bellatorum : } sed talia relinquenda sunt iudicio prudentis rectoris considerantis conditionem ciuium , modum regionis , \\\hline
3.1.14 & non puede escablesçer cuento determinado de los lidiadores \textbf{ mas tales cosas commo estas son de dexar auyzio de gouernador sabio } que ha de penssar la condicion de los çibdadanos & non potest statuere determinatum numerum bellatorum : \textbf{ sed talia relinquenda sunt iudicio prudentis rectoris considerantis conditionem ciuium , modum regionis , } et circumstantias vicinorum . \\\hline
3.1.14 & mas tales cosas commo estas son de dexar auyzio de gouernador sabio \textbf{ que ha de penssar la condicion de los çibdadanos } e la manera de los regnos e las cercunstaçias de los uezinos & non potest statuere determinatum numerum bellatorum : \textbf{ sed talia relinquenda sunt iudicio prudentis rectoris considerantis conditionem ciuium , modum regionis , } et circumstantias vicinorum . \\\hline
3.1.15 & assi ser gouernada \textbf{ que todos los çibdadanos deuian auer las possessionos e las mugers e los fijos comunes } la qual cosa assi entendida & ciuitatem sic esse regendam et gubernandam , \textbf{ ut ciuibus communes essent uxores , | et filii , et possessiones . } Quod si intelligitur \\\hline
3.1.15 & por las cosas ya dichͣs . \textbf{ Enpero por quela costunbre delas flatonicos fue de fablar } por methado phoras & ut est per habita manifestum . \textbf{ Quia modus fuit Platonicorum metaphorice loqui } quem modum loquendi forte ipse Socrates habebat , \\\hline
3.1.15 & e semeianças \textbf{ la qual manera de fablar } por auentra a auia socrates & Quia modus fuit Platonicorum metaphorice loqui \textbf{ quem modum loquendi forte ipse Socrates habebat , } cum Plato eius discipulus fuisset . \\\hline
3.1.15 & por que platon fue su disçipulo . \textbf{ Si quisieremos entender los dichos de socrates } non & cum Plato eius discipulus fuisset . \textbf{ Si volumus } non ut verba sonant intelligere dicta Socratica , saluare poterimus positionem eius . Omnia enim esse ciuibus communia \\\hline
3.1.15 & non \textbf{ assi conmo suena las palabras podremos entender la su opinion } diziendo que non es cosa que pueda ser & Si volumus \textbf{ non ut verba sonant intelligere dicta Socratica , saluare poterimus positionem eius . Omnia enim esse ciuibus communia } secundum \\\hline
3.1.15 & por amor \textbf{ e por dilectiuo puede se saluar } y la comu indat & tamen \textbf{ secundum amorem , et dilectionem debet ibi saluari communitas . } Sicut enim quilibet ciuis \\\hline
3.1.15 & y la comu indat \textbf{ por que si cada vn çibdada no deue amar tos otros çibdảdanos } assy commo assi mesmo . & secundum amorem , et dilectionem debet ibi saluari communitas . \textbf{ Sicut enim quilibet ciuis | debet diligere ciues alios , } sicut seipsum : \\\hline
3.1.15 & assy commo assi mesmo . \textbf{ En essa misma manera deue amar los fijos e las mugers e las possessiones de los otros çibdadanos } assi commo las suyas propreas . & sicut seipsum : \textbf{ sic debent diligere uxores , | filios , } et possessiones aliorum , \\\hline
3.1.15 & e quanto ala libalidat e franqueza \textbf{ por quacada vno de los çibdadanos deue enprestar alos otros çibdadanos los sus bienes } assi conmoiudga la razon . & et quantum ad liberalitatem : \textbf{ quia quilibet ciuis debet aliis ciuibus } ( ut dictat ratio ) \\\hline
3.1.15 & quanto ala comunidat de los çibdadanos \textbf{ En essa misma manera podemos saluar el su dicho } del & quantum ad communitatem ciuium : \textbf{ sic etiam saluare possumus dictum eius } quantum ad unitatem ciuitatis . \\\hline
3.1.15 & much vna por auentura non entendio de vnidat dela morada \textbf{ assi que todos los çibdadanos deuiessen morar en vna casa } assi que en vna çibdat non fuessen muchͣs casas o much suarrios . & forte non intellexit de unitate habitationis , \textbf{ ut quod omnes ciues habitare deberent in una domo , } ita quod in ciuitate non essent plures domus \\\hline
3.1.15 & para conplimiento dela uida \textbf{ mas por auentura quariesoctates retornar esta vnidat a amor } e adilectiuo de los çibdadanos & et diuersa prout requirit indigentia vitae . \textbf{ Sed forte unitatem | huiusmodi ad amorem } et dilectionem referre volebat , \\\hline
3.1.15 & mas aquello que eñadia socrates delas mugers \textbf{ que deuian ser ordena a obras de batalla puedese saluar } non entendiendo esto sinplemente & et concordia . Quod vero Socrates addebat de mulieribus , \textbf{ quod ordinandae essent ad opera bellica . } Saluari potest non intelligendo hoc simpliciter , \\\hline
3.1.15 & mas en algun caso . \textbf{ por que podria contesçer algun caso } que las mugers deuian batallar & sed in casu . \textbf{ Posset enim casus contingere , } quod et mulieres bellare oporteret . \\\hline
3.1.15 & por que podria contesçer algun caso \textbf{ que las mugers deuian batallar } ca muchͣs uegadas & Posset enim casus contingere , \textbf{ quod et mulieres bellare oporteret . } Multotiens autem circa partes Italiae hoc contigit , \\\hline
3.1.15 & por la qual cosa conuenio alas mugers \textbf{ por mengua de los çibdadanos de defender la çibdat mas lo que enandio adelante diziendo } que sienpre conuenia & propter quod oportuit \textbf{ et mulieres propter penuriam ciuium defendere ciuitatem . Quod autem ulterius addebat , } quod semper oportet eosdem in magistratibus praeferri , \\\hline
3.1.15 & por auentura se deue \textbf{ assi entender } que sienpie enssennoreassen vnos & esse idem praepositi et baliui : \textbf{ forte sic debet intelligi , } ut semper principentur iidem , \\\hline
3.1.15 & por los batalladores entendia nobles omes \textbf{ alos quales non parte nesçia de obrar ningunan otra cosa con sus manos } ca commo en la çibdat alguon sobren por manos & Forte per bellatores intendebat nobiles , \textbf{ quorum non est manibus operari . } Cum enim in ciuitate quidam manibus opererentur \\\hline
3.1.15 & Por ende conuiene \textbf{ que estos nobles de una prinçipalmente defender la tierra entre los otros } e a ellos parte nesçe mayormente & ut nobiles : \textbf{ hi videlicet nobiles potissime debent defendere patriam , } et eorum maxime est vacare circa armorum industriam . Volebat ergo Socrates politiam aliquam non debere nominari ciuitatem , nisi saltem contineret mille nobiles , \\\hline
3.1.15 & e a ellos parte nesçe mayormente \textbf{ de entender çerca la sabiduria delas armas . } Et pues que assi esquariasocrates & ø \\\hline
3.1.16 & que entre todas las cosas \textbf{ aque deue parar mientes el fazedor della es } en qual manera los çibdadanos de una auer las possesipnes igualadas & Dicebat autem , \textbf{ quod inter caetera quae debet intendere legislator } et rector politiae , est , \\\hline
3.1.16 & e contado la muchedunbre de los canpos \textbf{ de ligero podria partir el rectoor dela çibdat } egualmente aquellas possessions entre los çibdadanos & et computata multitudine camporum , \textbf{ de facili rector ciuitatis posset diuidere aequaliter possessiones illas inter ciues . } Sed ciuitate iam constituta , \\\hline
3.1.16 & deseguales \textbf{ mayor trabaio delas traer despues a ygualdat } Et por ende establesçio felleas & et ciuibus iam habentibus possessiones inaequales , \textbf{ difficilius erat hoc adaequalitatem reducere . } Statuit enim Phaleas ciuitatis rectorem \\\hline
3.1.16 & en esta manera aduxiesse esta desegualdat a egualdat \textbf{ Conuiene a saber } por las arras & Statuit enim Phaleas ciuitatis rectorem \textbf{ hoc modo reducere hanc inaequalitatem ad aequalitatem mediantibus dotibus statuendo quod pauperes contrahant } cum diuitibus : \\\hline
3.1.16 & de los ricos \textbf{ podria se ygualar alos ricos en las possessiones . } Et este pho felleas pudo se mouer a establesçer esto & ø \\\hline
3.1.16 & podria se ygualar alos ricos en las possessiones . \textbf{ Et este pho felleas pudo se mouer a establesçer esto } por tres razones & et non dent pauperes ergo accipiendo magnas dotes a diuitibus poterunt aequari eis in possessionibus . \textbf{ Potuit autem Phaleas triplici via moueri } ad hoc statuendum . Primo quidem moueri potuit , \\\hline
3.1.16 & por las possessiones \textbf{ por que comunalmente son muy codiçiosos para auer grandes rentas e grandes possessiones } por la qual cosa nasçen entre ellos muchͣs contiendas e muchͣs uaraias . & Nam ciues valde litigant pro possessionibus : \textbf{ sunt enim communiter nimis cupidi ad habendum redditus , | et possessiones ; } unde et propter hoc multa litigia oriuntur . \\\hline
3.1.16 & e sopiessen \textbf{ que vno non podia sobrepuiar los otros çibdadanos sus uesnos en possessiones } por esto debalde se leunatarian entre ellos las contiendas e las uaraias & Quare si ciues haberent possessiones aequales , \textbf{ et scirent se non posse excedere suos conciues in possessionibus , } frustra propter hoc insurgerent lites \\\hline
3.1.16 & por esto debalde se leunatarian entre ellos las contiendas e las uaraias \textbf{ ca puesto que vno de los contendores uençiesse el pleito non podia much gozar } por esto & et placita . \textbf{ Nam dato | quod alter litigantium causam obtineret , } non multum ex hoc gaudere posset \\\hline
3.1.16 & por que sean tiradas dela çibdat las contiendas e las naraias \textbf{ ca por ganar las possessions fazen los çibdadanos desagnisados } e tuercos vnos a otros & ut tollantur de ciuitate iniuriae et contumeliae : \textbf{ nam ex eo quod ciues libenter sibi possessiones appropriant , dicente , } Hoc est meum , \\\hline
3.1.16 & Et por ende non solamente se le una tan lides e pleitos \textbf{ mas avn iuicias et tuertos e uaraias ca para ganar possessiones fazen los çibdadanos tuerto vno a otro en sus ppreas personas . } An fazense en la çibdat furtos e robos i home çie & et placita ; \textbf{ sed etiam iniuriae et contumeliae . | Nam pro acquirendis possessionibus inferunt sibi ciues iniurias et contumelias in personis propriis ; } fiunt autem in ciuitate furta , rapinae et homicidia propter cupiditatem possidendi diuitias . \\\hline
3.1.16 & An fazense en la çibdat furtos e robos i home çie \textbf{ doi dios por cobdiçia de ganar riquezas } e por ende si los çibdadanos non podiessen sobrepuiar vnos a otros en las possessiones & fiunt autem in ciuitate furta , rapinae et homicidia propter cupiditatem possidendi diuitias . \textbf{ Quare } si ciues non possent se excellere in possessionibus , et haberent aequales diuitias , ut videtur , \\\hline
3.1.16 & doi dios por cobdiçia de ganar riquezas \textbf{ e por ende si los çibdadanos non podiessen sobrepuiar vnos a otros en las possessiones } et las ouiessen todas eguales & Quare \textbf{ si ciues non possent se excellere in possessionibus , et haberent aequales diuitias , ut videtur , } omnia haec cessarent . \\\hline
3.1.17 & quanto pertenesçe alo presente \textbf{ por tres razones podemos prouar } que non conuiene que las possessiones sean egualadas en aquella manera & quantum ad praesens spectat , \textbf{ possumus triplici via venari , } quod non oportet possessiones aequatas esse , \\\hline
3.1.17 & e delos tuertos \textbf{ que se podrian leunatar } e recrescer en la çibdat ¶ & ex parte iniuriarum . \textbf{ Tertia , } ex parte virtutum , \\\hline
3.1.17 & que se podrian leunatar \textbf{ e recrescer en la çibdat ¶ } La terçera de parte delas & Tertia , \textbf{ ex parte virtutum , } quas decens est habere ciues . Prima via sic patet . \\\hline
3.1.17 & La terçera de parte delas \textbf{ uirtudesque deuen auer los çibdadanos . La primera razon paresçe } assi ca ha gouernamiento de cecho de la çibdat & ex parte virtutum , \textbf{ quas decens est habere ciues . Prima via sic patet . } Nam ad rectum regimen ciuitatis \\\hline
3.1.17 & assi ca ha gouernamiento de cecho de la çibdat \textbf{ non se puede poner ley } por que los çibdadanos ayan possessiones eguales & Nam ad rectum regimen ciuitatis \textbf{ non potest imponi lex , } ut ciues habeant possessiones aequatas , \\\hline
3.1.17 & por que en vano se pone la ley \textbf{ que se non puede guardar } ca segunt elpho en el quarto libro delas politicas & quod habeant aequales filios : \textbf{ frustra enim ponitur lex quae conseruari non potest . } Nam \\\hline
3.1.17 & entre los fiios \textbf{ destruyr se ya la ley de de felleas } que non aurien las possessiones yguales . Pues que assi es commo & ø \\\hline
3.1.17 & mas en fijos que otros \textbf{ por ende non se puede poner ley en la çibdat } que todos los çibdadanos ayan ygual cuento de fijos & sed quia aliqua connubia sunt omnino sterilia , aliqua vero sunt foecundiora quam alia , non est possibile \textbf{ statuere in ciuitate omnes ciues habere aequalem numerum filiorum . } Propter quod ex parte procreationis prolis manifeste ostenditur praedictam \\\hline
3.1.17 & quela ley puesta por felleas non es conuenible \textbf{ por que se non puede guardar conueinblemente ¶ } La segunda razon se toma de parte delas & legem non esse congruentem ; \textbf{ eo quod congrue obseruari non possit . } Secunda via sumitur ex parte iniuriarum , \\\hline
3.1.17 & e de los tuertos \textbf{ que pueden contesçer en la çibdat } las quales miurias deue escusar muchel & Secunda via sumitur ex parte iniuriarum , \textbf{ quae possunt in ciuitate consurgere , } quas legislator summo studio cauere debet . \\\hline
3.1.17 & que pueden contesçer en la çibdat \textbf{ las quales miurias deue escusar muchel } que faze la ley & quae possunt in ciuitate consurgere , \textbf{ quas legislator summo studio cauere debet . } Spectat enim ad principem , \\\hline
3.1.17 & que faze la ley \textbf{ ca pertenesçe al prinçipe de auer grant cuydado } que los çibdadanos non sean bulliçiosos ni turbadores dela paz & quas legislator summo studio cauere debet . \textbf{ Spectat enim ad principem , | omnem } curam habere \\\hline
3.1.17 & ca si algunos qiessen mas fijos \textbf{ que los pobres partir lłeyan } en mas partes las possessiones de los ricos & et econuerso . \textbf{ Nam si diuites aliqui plures habent filios quam pauperes , } diuiditur possessio diuitum in plures partes , \\\hline
3.1.17 & Lo primero quando los pobres se fazen ricos \textbf{ non saben sofrir la su buena uentura } assi commo el philosofo muestra llanamente en el segundo libro de la rectoriça & et iurgia in ciuitate . \textbf{ Primo quia pauperes cum ditantur nesciunt fortunas ferre , } ut plane ostendit Philosop’ 2 Rhet’ . Iniuriabuntur ergo aliis . Filii enim pauperum inflati , \\\hline
3.1.17 & que los otros \textbf{ por ende mueuensse a fazer les tuerto } e por ende se farian iniurias e tuertos en la çibdat & eo quod videant se plus aliis in diuitiis abundare , iniustificabunt in eos , et fient iniuriae in ciuitate . Rursus , \textbf{ huiusmodi iniuriae et contumeliae orirentur } inter ciues non solum ex parte filiorum pauperum , \\\hline
3.1.17 & ca assi commo dize el philosofo \textbf{ en el segundo libro delas politicas meesteres ala pazer dela çibdat } que los fijos de los ricos non sean sobuios & sed etiam ex parte filiorum diuitum : \textbf{ quia ut dicitur 2 Polit’ } opus est ad pacem ciuitatis filios diuitum non esse insolentes . \\\hline
3.1.17 & si uieren que son despreçiados \textbf{ e los pobres enxalçados non lo pue den sofrir } e por ende fazer se yan sobuios & se esse despectos \textbf{ et pauperes exaltatos non valentes se continere , } fient insolentes , \\\hline
3.1.17 & e los pobres enxalçados non lo pue den sofrir \textbf{ e por ende fazer se yan sobuios } e turbarian los otros . & et pauperes exaltatos non valentes se continere , \textbf{ fient insolentes , } et turbabunt alios . \\\hline
3.1.17 & ¶ La terçera razon \textbf{ para mostrar } que la ley de felleas del ordenamiento delas possessiones & ø \\\hline
3.1.17 & se toma de parte delas uirtudes \textbf{ que deuen auer los çibdadanos } por que conuiene que los çibdadanos sean liberales e francos & Tertia via ad ostendendum legem Phaleae non esse decentem de aequatione possessionum , \textbf{ sumitur ex parte virtutum quas decet habere ciues : } decet enim ipsos esse liberales et temperatos : \\\hline
3.1.17 & si algua cosa non fue determinada dela cantidat de aquellas possessiones \textbf{ por que podrian los çibdadanos auer tan pocas possessiones } que les conuenia de beuir & nisi aliquid determinetur de quantitate possessionum illarum : \textbf{ possent enim ciues adeo modicas possessiones habere , } quod oporteret \\\hline
3.1.17 & que non podrien ser lo ƀales \textbf{ nin fazer ligeramente obras de largueza } Otrossi auiendo las possessiones ygualadas podrian assi abondar en ellas & eos \textbf{ ita parce viuere quod opera liberalitatis } de facili exercere non valerent . Rursus habendo possessiones aequatas possent \\\hline
3.1.17 & nin fazer ligeramente obras de largueza \textbf{ Otrossi auiendo las possessiones ygualadas podrian assi abondar en ellas } e beuir delicadamente & ita parce viuere quod opera liberalitatis \textbf{ de facili exercere non valerent . Rursus habendo possessiones aequatas possent | ita abundare in eis , } et adeo deliciose viuere , \\\hline
3.1.18 & e generalmente a todos aquellos a quien parte nesçe \textbf{ de poner leyes de establesçer algunas leyes cerca las possessiones de los çibdadanos } casolon aquel philosofo & et uniuersaliter eos \textbf{ quorum est leges ferre , | decet aliquas leges statuere circa possessiones ciuium . } Nam et Solon , \\\hline
3.1.18 & que para que las suertes antiguas fuessen guardadas sin dan \textbf{ no non conuerma a ninguon de uender sus possessiones } si non podiessen mostrar conplidamente qual contesçiera alguna grant ocasion o algua mala uentura & quod ad hoc ut antiquae sortes seruarentur illesae , \textbf{ nulli licebat possessiones vendere , } nisi posset sufficienter ei ostendere aliquod magnum infortunium accidisse . Sunt enim multa determinanda in legibus circa possessiones , \\\hline
3.1.18 & no non conuerma a ninguon de uender sus possessiones \textbf{ si non podiessen mostrar conplidamente qual contesçiera alguna grant ocasion o algua mala uentura } por que son muchͣs cosas de determinar & nulli licebat possessiones vendere , \textbf{ nisi posset sufficienter ei ostendere aliquod magnum infortunium accidisse . Sunt enim multa determinanda in legibus circa possessiones , } sed non expedit hoc lege statui circa ipsas , \\\hline
3.1.18 & si non podiessen mostrar conplidamente qual contesçiera alguna grant ocasion o algua mala uentura \textbf{ por que son muchͣs cosas de determinar } en las leyes çerca las possession s & nulli licebat possessiones vendere , \textbf{ nisi posset sufficienter ei ostendere aliquod magnum infortunium accidisse . Sunt enim multa determinanda in legibus circa possessiones , } sed non expedit hoc lege statui circa ipsas , \\\hline
3.1.18 & del que faze la ley deue ser çerca delas possessiones \textbf{ mas mayormente deue entender } en la reprehension delas cobdiçias & esse debet circa possessiones , \textbf{ sed principalius } debet intendere reprehensionem concupiscentiarum , \\\hline
3.1.18 & mas la cobdiçia de dentro \textbf{ por que la rayz de todos los males es cobdiçia Et pues que assi es si la enfermedat mas se deue guaresçer en la rays e en el comienço } donde nasçe & sed concupiscentia : \textbf{ radix enim omnium malorum est cupiditas . | Si ergo morbus magis curari } debet in radice et in causa quam alibi , \\\hline
3.1.18 & que en otra cosa los rectores delas çibdades \textbf{ mas deuen entender en repreheder las cobdiçias } que en otra cosa & debet in radice et in causa quam alibi , \textbf{ magis debent intendere rectores ciuium circa reprimendas concupiscentias quam circa alia , } ut plane probat Philosophus 2 Polit’ . \\\hline
3.1.18 & en el qual si los dichs fueren penssados superfiçialmeᷤte e sin sotileza paresçe \textbf{ que quiere contradezir a estas cosas } que agora dixiemos & nos edidisse quendam tractatum . De differentia Ethicae Rhetoricae \textbf{ et Politicae , } ubi dicta superficialiter considerata contradicere videntur his quae nunc diximus . Sed illa controuersia infra tolletur . \\\hline
3.1.18 & Mas aquella contradiçion adelante se declarara . \textbf{ mas quanto alo presente cunple de saber } que la prinçipal entençion del que faze la ley non deue ser & ubi dicta superficialiter considerata contradicere videntur his quae nunc diximus . Sed illa controuersia infra tolletur . \textbf{ Ad praesens autem scire sufficiat , } principalem intentionem legislatoris non debere esse circa possessiones exteriores mensurandas . \\\hline
3.1.18 & que la prinçipal entençion del que faze la ley non deue ser \textbf{ en mesurarlas possessiones de fuera } mas las cobdiçias de dentro . & Ad praesens autem scire sufficiat , \textbf{ principalem intentionem legislatoris non debere esse circa possessiones exteriores mensurandas . } Quod triplici via venari possumus . \\\hline
3.1.18 & mas las cobdiçias de dentro . \textbf{ la qual cosa podemos prouar } por tres razones . & principalem intentionem legislatoris non debere esse circa possessiones exteriores mensurandas . \textbf{ Quod triplici via venari possumus . } Prima sumitur ex parte bonorum , \\\hline
3.1.18 & e mas se traban \textbf{ si non pueden alcançar la honrra digna } e con ueinble a ellos . & ø \\\hline
3.1.18 & e graçiosas pertenesçe \textbf{ al que faze la ley non solamente de establesçer } que las possessiones sean ordenadas conueinblemente & et honorabiles : \textbf{ spectat ad legislatorem non solum statuere , } ut possessiones debite ordinentur , \\\hline
3.1.18 & mas avn que las honras sean partidas derechamente e conueiblemente \textbf{ e tanto mas prinçipalmente deue entender en partir las honrras } quanto las peleas entre las perssonas honrradas son de mayor periglo & et iuste distribuantur . \textbf{ Et tanto principalius debet hoc intendere circa honores , } quanto litigia \\\hline
3.1.18 & La segunda razon \textbf{ para prouar esto mismo se toma de parte delas delecta connes } las quales los omes siguen en la mayor parte & inter personas honorabiles sunt magis detestanda , quam inter personas alias . Secunda via ad inuestigandum hoc idem sumitur \textbf{ ex parte delectationum , } quas homines ut plurimum insequuntur . Non enim homines solum iniustificant , \\\hline
3.1.18 & e alos prinçipes \textbf{ non poner todas las leyes } en mesurar las possessiones & et Principes \textbf{ non omnes leges ferre circa possessiones mensurandas , } ut quod quilibet quod suum est possideat : \\\hline
3.1.18 & non poner todas las leyes \textbf{ en mesurar las possessiones } assi que cada vno aya & et Principes \textbf{ non omnes leges ferre circa possessiones mensurandas , } ut quod quilibet quod suum est possideat : \\\hline
3.1.18 & lo que suyo es \textbf{ Mas conuiene les de ordenar muchͣs cosas } para repremir los desseos desordenados & ut quod quilibet quod suum est possideat : \textbf{ sed multa ordinare decet circa possessiones reprimendas , } ne ciues sint intemperati , \\\hline
3.1.18 & Mas conuiene les de ordenar muchͣs cosas \textbf{ para repremir los desseos desordenados } por que los çibdadanos non sean destenprados & ut quod quilibet quod suum est possideat : \textbf{ sed multa ordinare decet circa possessiones reprimendas , } ne ciues sint intemperati , \\\hline
3.1.18 & por su poder . \textbf{ ca lo s ons quieren gozar de delecta çonnes sin tristezas } e por ende fazen tuertos alos otros & Tertia via sumitur ex parte tristitiarum , quas homines pro viribus fugiunt . \textbf{ Volunt enim homines gaudere delectationibus absque tristitiis , } ideo iniuriantur aliis in se non solum propter supplendam indigentiam , \\\hline
3.1.18 & e por ende fazen tuertos alos otros \textbf{ non por conplir grant mengua } assi commo por tirar fanbre o por tirar frio dessi . & Volunt enim homines gaudere delectationibus absque tristitiis , \textbf{ ideo iniuriantur aliis in se non solum propter supplendam indigentiam , } ut propter tollendam famem , \\\hline
3.1.18 & non por conplir grant mengua \textbf{ assi commo por tirar fanbre o por tirar frio dessi . } Mas por que cuydan & ideo iniuriantur aliis in se non solum propter supplendam indigentiam , \textbf{ ut propter tollendam famem , | et repellendum frigus : } sed quia existimant alios posse eorum delectationibus impedire , \\\hline
3.1.18 & Mas por que cuydan \textbf{ que los otros pueden enbargar sus delecta connes } o porque cuydan & et repellendum frigus : \textbf{ sed quia existimant alios posse eorum delectationibus impedire , } vel quia existimant eis posse tristitiam inferre . \\\hline
3.1.18 & o porque cuydan \textbf{ que les pueden fazer tristeza . } Et pues que assi es & sed quia existimant alios posse eorum delectationibus impedire , \textbf{ vel quia existimant eis posse tristitiam inferre . } non ergo solum propter possessiones sunt instituendae leges , \\\hline
3.1.19 & ¶Lo quarto del departimiento delos que iudgan . \textbf{ Lo quinto dela manera de iudgar ¶ Lo sesto e lo postrimo establesçio algunas leyes } que tannian alguons linages de personas dezimos & Quarto de distinctione iudicantium . \textbf{ Quinto de modo iudicandi . Sexto et ultimo statuit quasdam leges tangentes diuersa genera personarum . Hippodamus autem statuens suam politiam , } primo intromisit se de multitudine \\\hline
3.1.19 & e esta quantidat departia en tres partes \textbf{ conuiene a saber en lidiadores e en menestrales } e en labradores caquaria & quod optima quantitas ciuium est circa decem millia virorum . \textbf{ Hanc autem quantitatem distinxit in tres partes , } videlicet in bellatores , artifices , et agricolas . Volebat autem bellatores debere habere arma , \\\hline
3.1.19 & quanto pertenesçe alo presente ha mͣester tres cosas \textbf{ ¶Conuienea saber comer e beuer } por la calentura natural & ( quantum ad praesens spectat ) tribus indigere , \textbf{ videlicet victu , potu , } et cibo propter calorem naturalem consumentem huiusmodi radicale . \\\hline
3.1.19 & e por las discordias e uaraias \textbf{ que pueden contesçer en la çibdat . } Et para estas tres cosas siruen los tres linages sobredichos de los uarones & et litigia , \textbf{ quae possunt in ciuitate contingere . } Ad haec enim tria deseruiunt praedicta tria genera virorum . \\\hline
3.1.19 & partiendo todo el regno o todo el terretorio de la çibdat en tres partes . \textbf{ Conuiene a saber en parte sagrada } e en parte comun e en parte proprea . & diuidens totam regionem idest totum territorium ciuitatis in tres partes \textbf{ videlicet in partem sacram , } communem , et propriam . Partem sacram attribuebat cultui diuino , \\\hline
3.1.19 & e esta llama una comun \textbf{ por que los lidiadores deuian entender el bien comun } assi commo al defendimiento dela tr̃ra . & quam appellabat communem , \textbf{ eo quod bellatores communi bono } ut defensioni patriae vacare debeant : \\\hline
3.1.19 & que algunte rrectorio deuie ser comun \textbf{ del qual deuian beuir los lidiadores } assi commo de cosa comun¶ & sed solum tribuebat eis arma . \textbf{ Dicebat autem debere esse aliquod territorium commune , de quo bellatores viuerent } quasi de communi aerario . \\\hline
3.1.19 & segunt que por tres cosas contienden los çibdadanos . \textbf{ Conuiene a saber por enpesçimiento e por iniuria e por muerte . } Ca qual se quier que faze tuerto a otro & videlicet de nocumento , \textbf{ iniuria , | et morte : } quicunque enim iniustificat in alium , \\\hline
3.1.19 & e enderesçassen aquellos pleytos \textbf{ ¶Lo quinto se entremetio el dichpho dela manera de iudgar } ca quiere que en cada vna destas audiençias & senes illi electi \textbf{ et boni testimonii rectificant ipsas . Quinto intromisit se de modo iudicandi ; } volebat enim in utrisque praetoriis tam in principali quam in ordinario iudicia fieri debere sine collatione iudicum \\\hline
3.1.19 & que oyda la querella cada vno de los uiezes \textbf{ por si deuia penssar } e despues poner sus nina en esc̀pto & Dicebat autem quod audita causa quilibet iudex per se cogitaret , \textbf{ et postea in pugillaribus scriptam adduceret suam sententiam : } ut si incusatus simpliciter condemnandus esset , \\\hline
3.1.19 & por si deuia penssar \textbf{ e despues poner sus nina en esc̀pto } assi que si el acusado sin ninguna condiçion fuesse de condepnar el iuezes & Dicebat autem quod audita causa quilibet iudex per se cogitaret , \textbf{ et postea in pugillaribus scriptam adduceret suam sententiam : } ut si incusatus simpliciter condemnandus esset , \\\hline
3.1.19 & e despues poner sus nina en esc̀pto \textbf{ assi que si el acusado sin ninguna condiçion fuesse de condepnar el iuezes } ceruiesse su condenaçion sin ninguna condiçion & et postea in pugillaribus scriptam adduceret suam sententiam : \textbf{ ut si incusatus simpliciter condemnandus esset , } iudex simpliciter condemnationem scriberet : \\\hline
3.1.19 & ceruiesse su condenaçion sin ninguna condiçion \textbf{ mas si fuesse de soltar sin ninguna condiçion } troxiesse la tabla uazia sin ninguna esc̀ptura & iudex simpliciter condemnationem scriberet : \textbf{ si uero simpliciter absoluendus , } pugillarem portaret uacuum : \\\hline
3.1.19 & mas si en algua manera \textbf{ e o fuesse de condenar } e en algunan manera de soltar deuie lo traer determinado & pugillarem portaret uacuum : \textbf{ sed si aliquo modo condemnandus } et aliquo absoluendus , per scripturam determinaret illud . Philosophus autem 2 Politicor’ \\\hline
3.1.19 & e o fuesse de condenar \textbf{ e en algunan manera de soltar deuie lo traer determinado } por su estriptura . & ø \\\hline
3.1.19 & que diessen lo que sentiessen e entendiessen \textbf{ ca por auentura negarien de dezer } lo que sienten & ø \\\hline
3.1.19 & e destas tales leyes establesçio quatro . \textbf{ Lapmera tannia alos sabios¶ } La segunda alos lidiaderes . & Statuit autem quatuor , \textbf{ quarum prima tangebat sapientes , } secunda bellatores , \\\hline
3.1.19 & que ouiesse el prinçipe prinçipalmente cuydado de tres cosas . \textbf{ Conuiene a saber delas cosa comunes ¶ } Et de los pelegninos Et de los huerfanos . & ut princeps principalem curam haberet de tribus , \textbf{ videlicet de rebus communibus , } de peregrinis , et de orphanis . Appellabat autem orphanos uniuersaliter omnes personas impotentes , \\\hline
3.1.19 & que non auian poder \textbf{ nin podian por si mismas guardar su derechca parte nesçe al Rey e al prinçipeque deue ser guardador dela iustiçia de auer cuydado espeçial delas cosas comunes } e delos pelegninos & de peregrinis , et de orphanis . Appellabat autem orphanos uniuersaliter omnes personas impotentes , \textbf{ non valentes per se ipsas sua iura conquirere . Spectat enim ad Regem et Principem , } qui debet esse custos iusti , \\\hline
3.1.19 & e delas perssonas \textbf{ que non pueden guardar su derecho } por que tales perssonas los otros de ligero les fazen tuerto & et etiam de peregrinis , \textbf{ et personis impotentibus specialem curam gere : } eo quod talibus alii de facili iniuriantur , \\\hline
3.1.19 & por que tales perssonas los otros de ligero les fazen tuerto \textbf{ por que non pueden defender su derecho } uchos bienes se nos siguen delas opiniones de los phos antigos & eo quod talibus alii de facili iniuriantur , \textbf{ cum non possint defendere iura sua . } Multa bona consequimur \\\hline
3.1.20 & Enpero puesto que non dixiessen algua cosa uerdadera \textbf{ bien es de contar las opiniones dellos . } Et segunt la suia del philosofo & Dato tamen quod nihil veri dixissent , \textbf{ recitandae tamen sunt opiniones eorum , } et secundum sententiam Philosophi 2 Metaphysicae debemus gratias \\\hline
3.1.20 & en el segundo libro dela mecha phisica \textbf{ deuemos dar grans aquellos } que se desuian de la uerdat & et secundum sententiam Philosophi 2 Metaphysicae debemus gratias \textbf{ reddere eis qui a veritate deuiant , } et discordant ab opinionibus nostris : \\\hline
3.1.20 & enel segundo libro delas politicas \textbf{ rephender a ipodomio } quanto a tres cosas & quantum \textbf{ ad praesens spectat , sequendo dicta Philos’ 2 Pol’ increpare Hippodamum quantum ad tria . Primo , } quantum ad impossibilitatem statutorum . \\\hline
3.1.20 & Lo primero quanto ala inpossibilidat de los sus establesçimientos¶ \textbf{ Lo segundo quanto ala manera que establesçio en iudgar . } Lo traçero quanto alas leyes & quantum ad impossibilitatem statutorum . \textbf{ Secundo quantum ad modum , | quem statuit in iudicando . } Tertio quantum ad modum , \\\hline
3.1.20 & segunt el dixo \textbf{ se deuia partir en tres partes . } Oon uiene a saber enlidiadores e en menestrales e en labradores . & Si enim ciuitas \textbf{ secundum ipsum distingui debeat in tres partes , } videlicet in bellatores , \\\hline
3.1.20 & se deuia partir en tres partes . \textbf{ Oon uiene a saber enlidiadores e en menestrales e en labradores . } Et los lidiadores so los auian de traher las armas & secundum ipsum distingui debeat in tres partes , \textbf{ videlicet in bellatores , | artifices , et agricolas ; } et soli bellatores habeant arma , \\\hline
3.1.20 & Oon uiene a saber enlidiadores e en menestrales e en labradores . \textbf{ Et los lidiadores so los auian de traher las armas } por que podiessen tirar las discordias e las uaraias & artifices , et agricolas ; \textbf{ et soli bellatores habeant arma , } ut possent seditiones \\\hline
3.1.20 & Et los lidiadores so los auian de traher las armas \textbf{ por que podiessen tirar las discordias e las uaraias } que se fazen en la çibdat & et soli bellatores habeant arma , \textbf{ ut possent seditiones } et iurgia facta in ciuitate remouere , et patriam ab hostibus defendere : \\\hline
3.1.20 & que se fazen en la çibdat \textbf{ e por que podiessen defender latr̃ra delos enemigos . } Et segunt esto conuiene & ut possent seditiones \textbf{ et iurgia facta in ciuitate remouere , et patriam ab hostibus defendere : } oportebat bellatores habere maiorem potentiam , \\\hline
3.1.20 & Puesto que el dizie siguese \textbf{ que los menestrales e los labradores non los dexanien auer parte en la elecçion del prinçipe . } Et por ende establesçer & hoc posito artifices , \textbf{ et agricolae non haberent partem in politia ; et bellatores | non permitterent eos participare in electione principis . } Statuere ergo \\\hline
3.1.20 & que los menestrales e los labradores non los dexanien auer parte en la elecçion del prinçipe . \textbf{ Et por ende establesçer } que todo el pueblo escoia el prinçipe non puede estar con el establesçimiento & non permitterent eos participare in electione principis . \textbf{ Statuere ergo } quod totus populus eligat principem , \\\hline
3.1.20 & quanto ala manera que establesçia en iudgando \textbf{ ca quarie que los iuezes non deuian auer acuerdo enla audiençia } sobre las uian & et sint soli habentes arma . Secundo deficiebat Hippodamus quantum ad modum quem statuit in iudicando ; \textbf{ volebat enim iudices non debere conferre in praetorio de sententia ferenda : } tamen ( ut ait Philosophus ) \\\hline
3.1.20 & sobre las uian \textbf{ que auian de dar . } Enpero assi commo dize el philosofo non nego & volebat enim iudices non debere conferre in praetorio de sententia ferenda : \textbf{ tamen ( ut ait Philosophus ) } non negabat quin domi priuatim conferre possint ; \\\hline
3.1.20 & que en su casa priuadamente \textbf{ e en pondat non pudiessen auer tonseio sobre las iuͣ } Mas esto podemos reprehender & tamen ( ut ait Philosophus ) \textbf{ non negabat quin domi priuatim conferre possint ; } hoc autem duplici via improbari potest . Primo , \\\hline
3.1.20 & e en pondat non pudiessen auer tonseio sobre las iuͣ \textbf{ Mas esto podemos reprehender } por dos razones . & non negabat quin domi priuatim conferre possint ; \textbf{ hoc autem duplici via improbari potest . Primo , } quia ex isto modo iudicandi sequitur maior peruersio \\\hline
3.1.20 & Lo primero \textbf{ que desta manera de iudgar se sigue mayor discordia } e mayor desaguisado de los uiezes & hoc autem duplici via improbari potest . Primo , \textbf{ quia ex isto modo iudicandi sequitur maior peruersio } et maior degeneratio iudicum . \\\hline
3.1.20 & por que mas ayna se trastornan los uezes \textbf{ si pueden fablar } e auer conseio en poridat vno con otro & Nam citius peruertuntur iudices , \textbf{ si possunt sibi loqui in priuato , } quam si loquantur publice in praetorio : \\\hline
3.1.20 & si pueden fablar \textbf{ e auer conseio en poridat vno con otro } que si fablassen en publico en audiençia & Nam citius peruertuntur iudices , \textbf{ si possunt sibi loqui in priuato , } quam si loquantur publice in praetorio : \\\hline
3.1.20 & que si fablassen en publico en audiençia \textbf{ e sy iuraren de dezer lo que sienten mas ayna proui raran } por esta manera priuada & quam si loquantur publice in praetorio : \textbf{ et si iurauerunt dicere quod sentiunt , } citius degenerabunt hoc modo quam aliter . Unde et Philosophus innuit , \\\hline
3.1.20 & que non por sa publica . \textbf{ Onde el phoda a entender } que en algunos buenos ordenamientos de çibdat el contra no es establesçido delo & et si iurauerunt dicere quod sentiunt , \textbf{ citius degenerabunt hoc modo quam aliter . Unde et Philosophus innuit , } quod in quibusdam bonis politiis econtrario statuitur quam ordinauerit Hippodamus : \\\hline
3.1.20 & enlos quales ordenamientos es ordenado \textbf{ que los iezes puedan fablar vno con otro en publico } e non puedan auer conseio vno con otro en ascondido . & quod in quibusdam bonis politiis econtrario statuitur quam ordinauerit Hippodamus : \textbf{ ubi ordinantur iudices posse loqui sibi inuicem publice , } non tamen posse ad inuicem habere consilium in priuato . \\\hline
3.1.20 & que los iezes puedan fablar vno con otro en publico \textbf{ e non puedan auer conseio vno con otro en ascondido . } Otrossi fallesçe la dichͣ manera & ubi ordinantur iudices posse loqui sibi inuicem publice , \textbf{ non tamen posse ad inuicem habere consilium in priuato . } Rursus deficit dictus modus , \\\hline
3.1.20 & Otrossi fallesçe la dichͣ manera \textbf{ por que en alguons iuyzios conuiene de auer fablas los iiezes vno con otro } assi que si los uiezes descordassen en publico & Rursus deficit dictus modus , \textbf{ quia in aliquibus iudiciis oportet collationem habere ad inuicem ; } ut si iudices discordarent , \\\hline
3.1.20 & assi que si los uiezes descordassen en publico \textbf{ conuernia de auer fablas los vnos con los otros de quales miezes serie de tener la sentençia¶ } Lo terçero fallesçie el dicho philosofo & ut si iudices discordarent , \textbf{ oporteret collationem habere ad inuicem , | quorum sententia tenenda esset . } Tertio deficiebat dictus Philosophus quantum ad leges quas statuit , \\\hline
3.1.20 & para la çibdat deue \textbf{ por ende resçebir grant honrra } por ende se esforcarian los sabios de fallar nueuas leyes & deberet \textbf{ ex hoc debitum honorem accipere ; conarentur sapientes ad inueniendum nouas leges , } et ad ostendendum nouas leges inuentas esse proficuas ciuitati : \\\hline
3.1.20 & por ende resçebir grant honrra \textbf{ por ende se esforcarian los sabios de fallar nueuas leyes } para mostrar que las leys nueuas & deberet \textbf{ ex hoc debitum honorem accipere ; conarentur sapientes ad inueniendum nouas leges , } et ad ostendendum nouas leges inuentas esse proficuas ciuitati : \\\hline
3.1.20 & por ende se esforcarian los sabios de fallar nueuas leyes \textbf{ para mostrar que las leys nueuas } que ellos fallan son muy prouechosas ala çibdat & ex hoc debitum honorem accipere ; conarentur sapientes ad inueniendum nouas leges , \textbf{ et ad ostendendum nouas leges inuentas esse proficuas ciuitati : } quare continue mutarentur leges , \\\hline
3.1.20 & que ellos fallan son muy prouechosas ala çibdat \textbf{ por la qual cosa cadal dia se aurian de mudar las leyes } la qual cosa seria muy dannosa e muy peligrosa ala çibdat & et ad ostendendum nouas leges inuentas esse proficuas ciuitati : \textbf{ quare continue mutarentur leges , } quod est valde in ciuitate periculosum , \\\hline
3.1.20 & si cada dia se renouassen las leyes \textbf{ tirar se ya la uirtud e la fuerça delas leyes Et pues } que assi es el establesçimiento de ipodomio & Quare si assidue innouentur leges , \textbf{ tolletur virtus } et efficacia ipsarum . \\\hline
3.1.20 & Mas si el fallesçio en las otras cosas que dixo \textbf{ e si los prinçipes deuen yr por elecçion o por suçession de heredamiento . } Et si las leyes son & Utrum autem defecerit in aliis , \textbf{ et utrum principatus debeat ire per electionem | vel per haereditatem , } et utrum leges sint innouandae dato quod aliquem defectum contineant , \\\hline
3.1.20 & Et si las leyes son \textbf{ assi de renouar } commo el dixo & vel per haereditatem , \textbf{ et utrum leges sint innouandae dato quod aliquem defectum contineant , } et alia quae circa istam materiam sunt quaerenda , infra diffusius tractabuntur : \\\hline
3.1.20 & e las otras cosas \textbf{ que en esta materia son de dezer adelante lo tractaremos } mas conplidamente . & ad praesens tamen sufficiat hoc tetigisse de opinionibus Philosophorum . \textbf{ Et in hoc terminetur prima pars huius tertii libri , } in quo agitur de regimine ciuitatis et regni . Primae partis , \\\hline
3.1.20 & mas conplidamente . \textbf{ Enpero quanto alo presente c̃ple de auer tranniendo esto } que diches delas opimones de los philosofos & Et in hoc terminetur prima pars huius tertii libri , \textbf{ in quo agitur de regimine ciuitatis et regni . Primae partis , } in qua dictum fuit , \\\hline
3.2.1 & que establesçieron poliçias \textbf{ e dieron arte del gouernamiento dela çibdar } e del regno & et recitando opinionem diuersorum Philosophorum instituentium politiam , \textbf{ et tradentium artem de regimine ciuitatis et regni : } restat nunc exequi de aliis duabus partibus \\\hline
3.2.1 & e del regno \textbf{ finca nos de dezir delas otras dos partes . } Conuiene a saber del gouernamiento dela çibdat & et tradentium artem de regimine ciuitatis et regni : \textbf{ restat nunc exequi de aliis duabus partibus } videlicet de regimine regni et ciuitatis tempore pacis , \\\hline
3.2.1 & finca nos de dezir delas otras dos partes . \textbf{ Conuiene a saber del gouernamiento dela çibdat } e del regno entp̃o de paz . & ø \\\hline
3.2.1 & e del regno entr̃o de guerra . \textbf{ Pues que assi es conuiene de saber } que assi commo entp̃o de guerra es de defender la cibdat por armas & videlicet de regimine regni et ciuitatis tempore pacis , \textbf{ et de huiusmodi regimine tempore belli . Sciendum igitur , } quod tempore belli defendenda est ciuitas per arma ; \\\hline
3.2.1 & Pues que assi es conuiene de saber \textbf{ que assi commo entp̃o de guerra es de defender la cibdat por armas } assi en tr̃o de pazes de gouernar & et de huiusmodi regimine tempore belli . Sciendum igitur , \textbf{ quod tempore belli defendenda est ciuitas per arma ; } sicut tempore pacis gubernanda est per leges iustas , \\\hline
3.2.1 & que assi commo entp̃o de guerra es de defender la cibdat por armas \textbf{ assi en tr̃o de pazes de gouernar } por leyes derechͣs & quod tempore belli defendenda est ciuitas per arma ; \textbf{ sicut tempore pacis gubernanda est per leges iustas , } et per consuetudines approbatas , \\\hline
3.2.1 & Et pues que assi es visto \textbf{ que en elt pon dela paz es de gouernar la çibdat } e el regno & sicut leges ad tempus pacis . Viso igitur , \textbf{ tempore pacis gubernandam esse ciuitatem } et regnum per leges iustas , \\\hline
3.2.1 & por leyes derechos \textbf{ e por costunbres aprouadas de ligero puede paresçer } quales cosas & et regnum per leges iustas , \textbf{ et consuetudines approbatas ; | de leui patere potest , } quae et quot consideranda sunt in tali regimine . \\\hline
3.2.1 & quales cosas \textbf{ e quantas son de penssar } en el gouernamiento del regno e dela çibdat & de leui patere potest , \textbf{ quae et quot consideranda sunt in tali regimine . } Videtur autem Philos’ 3 Polit’ tangere , \\\hline
3.2.1 & tanne quatro cosas \textbf{ que son de penssar } enł gouernamiento del regno et dela çibdat & Videtur autem Philos’ 3 Polit’ tangere , \textbf{ quatuor quae consideranda sunt in regimine ciuitatis . } Haec autem sunt princeps , \\\hline
3.2.1 & Enpero podemos de aquellas cosas \textbf{ por las quales es de gouernar la çibdat entp̃o de paz prouar } por dos razones & Haec autem sunt princeps , \textbf{ consilium , praetorium , } et populus . Possumus autem ex ipsis legibus quibus regenda est ciuitas tempore pacis duplici via inuestigare , \\\hline
3.2.1 & por dos razones \textbf{ que destas quatro cosas dichͣ sauemos de tractar enł gouernamiento } por el qual es de gouernar la çibdat en el tp̃o dela guerra ¶ & consilium , praetorium , \textbf{ et populus . Possumus autem ex ipsis legibus quibus regenda est ciuitas tempore pacis duplici via inuestigare , } quod de praedictis quatuor considerandum est in regimine quo regenda est ciuitas tempore pacis . Prima via sumitur ex his quae requiruntur ad legem : \\\hline
3.2.1 & que destas quatro cosas dichͣ sauemos de tractar enł gouernamiento \textbf{ por el qual es de gouernar la çibdat en el tp̃o dela guerra ¶ } La primera razon se toma de aquellas cosas & consilium , praetorium , \textbf{ et populus . Possumus autem ex ipsis legibus quibus regenda est ciuitas tempore pacis duplici via inuestigare , } quod de praedictis quatuor considerandum est in regimine quo regenda est ciuitas tempore pacis . Prima via sumitur ex his quae requiruntur ad legem : \\\hline
3.2.1 & en las leyes sean guardadas commo deuen . \textbf{ Mas guardar bien las leyes } por el poderio çiuil esto parte nesçe alos prinçipes onde el philosofo enłqnto libbo delas ethicas dize & ut propter pacificum statum populi quae traduntur \textbf{ in legibus sint debite obseruata . Bene autem custodire leges } per ciuilem potentiam spectat ad principem . \\\hline
3.2.1 & ca los establesçimientos delas leyes pueden ser dichͣs leyes \textbf{ mas fallar bien las leyes pertenesçe al conseio } por sabiduria & nam statuta ipsa legalia leges quaedam dici possunt . \textbf{ Bene quidem inuenire leges per sapientiam spectat ad consilium : } decet enim princeps adeo sapientes consiliarios habere , \\\hline
3.2.1 & por sabiduria \textbf{ ca pertenesçe alos prinçipes de auer conseieros tan sabios } por que pueidan fallar aquellas cosas & Bene quidem inuenire leges per sapientiam spectat ad consilium : \textbf{ decet enim princeps adeo sapientes consiliarios habere , } ut debite inuenire possint quae populus obseruare debet . Bene vero iudicare \\\hline
3.2.1 & ca pertenesçe alos prinçipes de auer conseieros tan sabios \textbf{ por que pueidan fallar aquellas cosas } commo cunple & Bene quidem inuenire leges per sapientiam spectat ad consilium : \textbf{ decet enim princeps adeo sapientes consiliarios habere , } ut debite inuenire possint quae populus obseruare debet . Bene vero iudicare \\\hline
3.2.1 & commo cunple \textbf{ que el pueblo ha de guardar } mas bien iudgar & decet enim princeps adeo sapientes consiliarios habere , \textbf{ ut debite inuenire possint quae populus obseruare debet . Bene vero iudicare } secundum leges inuentas per consiliarios , \\\hline
3.2.1 & que el pueblo ha de guardar \textbf{ mas bien iudgar } segt las leyes falladas por los conseieros & ut debite inuenire possint quae populus obseruare debet . Bene vero iudicare \textbf{ secundum leges inuentas per consiliarios , } et custoditas per principem , \\\hline
3.2.1 & esto parte nesçe al alcaldia o alos uezes \textbf{ por que ellos son aqual los que segunt tales leyes deuen iudgar los fechs de los çibdadanos . Mas bien guardar las leyes puestas } eston pertenesce a todos los çibdadanos o a todo el pueblo & spectat ad praetorium siue ad iudices : \textbf{ ipsi enim sunt | qui secundum huiusmodi leges acta ciuium iudicare debent . } Sed bene obseruare leges spectat ad omnes ciues , siue ad totum populum . \\\hline
3.2.1 & que la çibdat sea gouernada entp̃o dela paz \textbf{ por las leyes conuiene de fazer tractado destas quatro cosas } sobredichͣs en este gouernamiento ¶ & per leges bene gubernetur ciuitas , \textbf{ oportet in huiusmodi regimine de praedictis quatuor considerationem facere . } Secunda via ad inuestigandum hoc idem sumitur ex fine qui in legibus intenditur : \\\hline
3.2.1 & La segunda razon \textbf{ para prouar esto mismo se toma dela fin } que entienden los fazedores delas leyes & oportet in huiusmodi regimine de praedictis quatuor considerationem facere . \textbf{ Secunda via ad inuestigandum hoc idem sumitur ex fine qui in legibus intenditur : } debet enim intendere legislator \\\hline
3.2.1 & que entienden los fazedores delas leyes \textbf{ ca deue entender el fazedor dela ley } que por las leyes alcançemos & Secunda via ad inuestigandum hoc idem sumitur ex fine qui in legibus intenditur : \textbf{ debet enim intendere legislator } ut per leges consequamur conferens , \\\hline
3.2.1 & Et sigamos la cosa \textbf{ que es de loar } e fuyamos la cosa & et consequamur laudabile , \textbf{ et fugiamus vituperabile . } De conferenti autem \\\hline
3.2.1 & e fuyamos la cosa \textbf{ que es de denostar . } Mas delo que nos cunple & et fugiamus vituperabile . \textbf{ De conferenti autem } et nociuo est consilium : de iusto et iniusto est praetorium ; \\\hline
3.2.1 & e del tuerto es el alealłia . \textbf{ Et dela cosa que es de loar } e de denostar es llamamiento e conuiramiento & et nociuo est consilium : de iusto et iniusto est praetorium ; \textbf{ sed de laudabili } et vituperabili est exclamatio siue concionatio , \\\hline
3.2.1 & Et dela cosa que es de loar \textbf{ e de denostar es llamamiento e conuiramiento } que parte nesçe a todo el pueblo & sed de laudabili \textbf{ et vituperabili est exclamatio siue concionatio , } quae potest respicere totum populum : populus enim ad bene agendum , \\\hline
3.2.1 & que parte nesçe a todo el pueblo \textbf{ ca el pueblo es de abiuar } por las leyes & et vituperabili est exclamatio siue concionatio , \textbf{ quae potest respicere totum populum : populus enim ad bene agendum , } per leges maxime inducendus est , \\\hline
3.2.1 & por las leyes \textbf{ para fazer buenas obras } por que en las leyes son mandadas las cosas & quae potest respicere totum populum : populus enim ad bene agendum , \textbf{ per leges maxime inducendus est , } eo quod ipsis legibus praecipiuntur laudabilia , \\\hline
3.2.1 & por que en las leyes son mandadas las cosas \textbf{ que sende loar } e son defendidas las cosas torpes & per leges maxime inducendus est , \textbf{ eo quod ipsis legibus praecipiuntur laudabilia , } et prohibentur turpia . \\\hline
3.2.1 & e son defendidas las cosas torpes \textbf{ que son de denostar . } Pues que assi es el que quiere determinar & eo quod ipsis legibus praecipiuntur laudabilia , \textbf{ et prohibentur turpia . } Volentem ergo determinare qualiter regenda sit ciuitas tempore pacis , \\\hline
3.2.1 & que son de denostar . \textbf{ Pues que assi es el que quiere determinar } en qual manera se ha de gouernar la çibdat & et prohibentur turpia . \textbf{ Volentem ergo determinare qualiter regenda sit ciuitas tempore pacis , } non solum considerare \\\hline
3.2.1 & Pues que assi es el que quiere determinar \textbf{ en qual manera se ha de gouernar la çibdat } ent p̃o de paz & et prohibentur turpia . \textbf{ Volentem ergo determinare qualiter regenda sit ciuitas tempore pacis , } non solum considerare \\\hline
3.2.1 & ent p̃o de paz \textbf{ non solamente deue catar } qual deue ser el prinçipe aquien parte nesçe & Volentem ergo determinare qualiter regenda sit ciuitas tempore pacis , \textbf{ non solum considerare } debet qualis debeat esse princeps \\\hline
3.2.1 & qual deue ser el prinçipe aquien parte nesçe \textbf{ de pouer las leyes } e delas guardar & debet qualis debeat esse princeps \textbf{ cuius est leges ferre et custodire , } sed etiam quales debeant esse consiliarii quorum est cognoscere quid conferens \\\hline
3.2.1 & de pouer las leyes \textbf{ e delas guardar } mas avn deue catar & debet qualis debeat esse princeps \textbf{ cuius est leges ferre et custodire , } sed etiam quales debeant esse consiliarii quorum est cognoscere quid conferens \\\hline
3.2.1 & e delas guardar \textbf{ mas avn deue catar } quales deuen ser los conleieros & ø \\\hline
3.2.1 & quales deuen ser los conleieros \textbf{ alos quales parte nesçe de conosçer } que cosa non es buena & cuius est leges ferre et custodire , \textbf{ sed etiam quales debeant esse consiliarii quorum est cognoscere quid conferens } et quid nociuum , \\\hline
3.2.1 & e quales deuen ser los iuezes \textbf{ alos que parte nesçe de iudgar } lo que es derecho & et quid nociuum , \textbf{ et quales debeant esse iudices quorum est iudicare } quid iustum \\\hline
3.2.1 & que sigua lo bueno e lo fermoso \textbf{ e lo de loar } e fuya lo malo e lo feo e lo de denostar . & qui inducendus est ut sequatur pulchrum \textbf{ et laudabile et fugiat vituperabile et turpe . } De omnibus ergo his quatuor , videlicet de principatu , consilio , \\\hline
3.2.1 & e lo de loar \textbf{ e fuya lo malo e lo feo e lo de denostar . } Pues que assi es de todas estas quatro cosas & qui inducendus est ut sequatur pulchrum \textbf{ et laudabile et fugiat vituperabile et turpe . } De omnibus ergo his quatuor , videlicet de principatu , consilio , \\\hline
3.2.2 & ca el regno e la aristo carçia \textbf{ que quiere dezer sennorio de buenos } e la poliçia que quiere dezer pueblo bien enssenoreante son bueons prinçipados . & Nam regnum aristocratia , \textbf{ et politia sunt principatus boni : } tyrannides , \\\hline
3.2.2 & que quiere dezer sennorio de buenos \textbf{ e la poliçia que quiere dezer pueblo bien enssenoreante son bueons prinçipados . } La thirama que quiere dezer sennorio malo & Nam regnum aristocratia , \textbf{ et politia sunt principatus boni : } tyrannides , \\\hline
3.2.2 & e la poliçia que quiere dezer pueblo bien enssenoreante son bueons prinçipados . \textbf{ La thirama que quiere dezer sennorio malo } e la obligaçia que quiere dezer sennorio duro . & et politia sunt principatus boni : \textbf{ tyrannides , } oligarchia , \\\hline
3.2.2 & La thirama que quiere dezer sennorio malo \textbf{ e la obligaçia que quiere dezer sennorio duro . } Et la democraçia que quiere dez maldat del pueblo & tyrannides , \textbf{ oligarchia , } et democratia sunt mali . \\\hline
3.2.2 & enssennoreante son malos prinçipados . \textbf{ Et alli muestra el pho departir el buen prinçipado del malo . } ca si en algun señorio o prinçipado es entendido el bien comun & et democratia sunt mali . \textbf{ Docet enim idem ibidem discernere bonum principatum a malo . } Nam si in aliquo dominio \\\hline
3.2.2 & e en esta manera segunt derecho e tuerto commo pone el pho . \textbf{ podemos tomar la suficiençia de los prinçipados } tan biende los malos commo delos luienos & et aliorum oppressio , sic est corruptum et peruersum . Hoc enim modo \textbf{ secundum viam Philosophi accipere possumus sufficientiam principantium tam peruersorum quam rectorum . Nam in ciuitate , } vel in aliqua gente vel dominatur unus , \\\hline
3.2.2 & e estonçe es dicho tal sennorio monarch̃ia o e egno \textbf{ ca al Rey parte nesçe de enteder el bien comun . } Et li aquel vno & tunc dicitur Monarchia siue Regnum : \textbf{ regis autem est intendere commune bonum . } Si vero \\\hline
3.2.2 & assi enssennoreante non entiende el bien comun \textbf{ Mas entiende por poderio çiuil apremiar los otros } e todas las cosas ordena al su bien propio & ille unus dominans non intendit commune bonum , \textbf{ sed per ciuilem potentiam opprimens alios , } omnia ordinat in bonum proprium et priuatum , \\\hline
3.2.2 & e tal prinçipado es dicħa ristrocaçia \textbf{ que quiere dezer prinçipado de buenos omes e uir̉tuosos } e dende vienen & et tunc talis principatus dicitur Aristocratia , \textbf{ quod idem est quod principatus bonorum et virtuosorum . Inde autem venit } ut maiores in populo , \\\hline
3.2.2 & que los mayores en el pueblo \textbf{ e los que deuen gouernar el pueblo son dich sobtimates } que quiere dezir muy buenos ca muy buenos deuen ser aquellos & ut maiores in populo , \textbf{ et qui debent populum regere vocati sunt optimates , } quia optimi debent esse \\\hline
3.2.2 & e los que deuen gouernar el pueblo son dich sobtimates \textbf{ que quiere dezir muy buenos ca muy buenos deuen ser aquellos } que dessean ser mayores que los otros . & et qui debent populum regere vocati sunt optimates , \textbf{ quia optimi debent esse } qui aliis praeesse desiderant . \\\hline
3.2.2 & tal es dich obligartia \textbf{ que quiere dezir prançipado de ricos . } Et pues que assi es dos prinçipados se leuna & et opprimentes alios intendunt proprium lucrum huiusmodi principatus Oligarchia dicitur , \textbf{ quod idem est quod principatus diuitum . } Consurgit igitur duplex principatus ex dominio paucorum : \\\hline
3.2.2 & mas por que son ricos e apmian los otros ¶ \textbf{ Lo tercero se pueden departir los prinçipados } por que enla çibdat enssennorean muchsca comunal . & et alios opprimentes . \textbf{ Tertio possunt distingui principatus , } ex eo quod in ciuitate dominantur multi . \\\hline
3.2.2 & para los establesçimientos \textbf{ que han de fazer } ca maguera sienpre llamen ally alguna potespado algun sennor & ut totus populus : ibi enim requiritur consensus totius populi in statutis condendis , in potestatibus eligendis , \textbf{ et etiam in potestatibus corrigendis . } Licet enim semper ibi adnotetur potestas , \\\hline
3.2.2 & que aquel que es llamado sennor \textbf{ por que a todo el pueblo ꝑtenesçe de escoger le } e de castigar le si mal feziere . & quam dominus adnotatus , \textbf{ eo quod totius populi est eum eligere et corrigere , } si male agat : \\\hline
3.2.2 & por que a todo el pueblo ꝑtenesçe de escoger le \textbf{ e de castigar le si mal feziere . } Et a todo el pueblo pertenesçe de fazer los establesçimientos & eo quod totius populi est eum eligere et corrigere , \textbf{ si male agat : } etiam eius totius est statuta condere , \\\hline
3.2.2 & e de castigar le si mal feziere . \textbf{ Et a todo el pueblo pertenesçe de fazer los establesçimientos } los quales establesçimientos non puede passar el sennor & si male agat : \textbf{ etiam eius totius est statuta condere , } quae non licet dominium transgredi . \\\hline
3.2.2 & Et a todo el pueblo pertenesçe de fazer los establesçimientos \textbf{ los quales establesçimientos non puede passar el sennor } e por ende en tal prinçipado & etiam eius totius est statuta condere , \textbf{ quae non licet dominium transgredi . } In tali ergo principatu , \\\hline
3.2.2 & Et por que tal prinçipado non ha nonbre propio \textbf{ llamalle el philosofo nonbre comun } e diz el poliçia & et tunc est rectus et aequalis : \textbf{ et quia talis principatus non habet nomen proprium , vocat eum Philosophus nomine communi , } et dicit ipsum esse Politiam . Politia enim quasi idem est , \\\hline
3.2.2 & por que non ha nonbre comun es dich poliçia \textbf{ e nos podemos llamar atal prinçipado gouernamiento del pueblo } si derecho es . & si rectus sit , \textbf{ eo quod non habeat commune nomen , Politia dicitur . Nos autem talem principatum appellare possumus gubernationem populi , } si rectus sit . \\\hline
3.2.2 & segunt su estado \textbf{ mas quiere tirnizar los ricos } es dich prinçipado de malos & Sed si populus sic dominans non intendit bonum omnium secundum suum statum , \textbf{ sed vult tyrannizare | et opprimere diuites , } est principatus peruersus , \\\hline
3.2.2 & es dich prinçipado de malos \textbf{ e en nonbre gniego es dicho democraçia mas nos podemos le llamar destruymiento e desordenamiento del pueblo . } Et pues que assi es paresçe & est principatus peruersus , \textbf{ et in graeco nomine dicitur Democratia . | Nos autem ipsum appellare possumus peruersionem populi . } Patet ergo quot sunt principatus , \\\hline
3.2.3 & e quales malos \textbf{ ir que entendemos mostrar } qual es la muy buena poliçia & et qui peruersi . \textbf{ Quia intendimus ostendere quae sit optima politia , } et quis sit optimus principatus . \\\hline
3.2.3 & e mostramos quales dellos son buenos \textbf{ e quales son malos finca de demostrar entre los prinçipados derechs e buenos } qualdlleros es el meior & qui illorum sunt recti , \textbf{ et qui peruersi : | restat ostendere } inter principatus rectos quis sit melior . \\\hline
3.2.3 & si quier sean muchos ¶ La primera razon se toma de ayuntamiento e de paz \textbf{ que se deue entender en el regno } e enla çibdat assi conmofin . & siue illi plures sint pauci , siue multi . Prima via sumitur ex unitate \textbf{ et pace , quae debet intendi in regno et ciuitate tanquam finis . Secunda ex ciuili potentia , } quae requiritur \\\hline
3.2.3 & e la egualdat de los humores es entendida del fisico finalmente mas esta vnidat e esta concordia \textbf{ mas la puede fazer } aquello que es vna cosa & et concordiam \textbf{ magis efficere potest quod est per se unum ; } magis autem per se unitas reperitur \\\hline
3.2.3 & que la uoluntad es mas vna enssi \textbf{ que non los que se ayuntan en ella . } Et por ende si es paz e concordia entre los çibdadanos & Et propter quod autem unumquodque , \textbf{ et illud magis . } Si ergo est pax et concordia inter ciues , \\\hline
3.2.3 & antes por ende siguese \textbf{ que si fuere vno lo lo el que enssennoreaua aura mayor paz en la çibdat } ca non se puede turbar la paz tan de ligero & si plures principes sint quam unum : \textbf{ ergo si solus unus principaretur | inter eos , } non sic de facili turbari posset pax ipsorum ciuium . \\\hline
3.2.3 & que si fuere vno lo lo el que enssennoreaua aura mayor paz en la çibdat \textbf{ ca non se puede turbar la paz tan de ligero } quando es vno commo quando son muchs . & inter eos , \textbf{ non sic de facili turbari posset pax ipsorum ciuium . } Secunda via ad inuestigandum hoc idem , \\\hline
3.2.3 & La segunda razon \textbf{ para prouar esto se toma del poderio çiuil } que es meester en el gouernamiento dela çibdat & Secunda via ad inuestigandum hoc idem , \textbf{ sumitur ex ciuili potentia , } quae requiritur in regimine ciuitatis . \\\hline
3.2.3 & e seria mas uirtuoso \textbf{ para traer la naueque muchs } por la qual razon & congregarentur in uno , \textbf{ quia ille magis unite traheret , virtuosior esset in trahendo . } Quare si tota ciuilis potentia , \\\hline
3.2.3 & Et aquel prinçipe \textbf{ por ma . yor cunplimiento de poderio meior podria gouernar la çibdat } que muchos ¶ & quae est in pluribus principantibus , congregaretur in uno Principe , efficacior esset ; \textbf{ et ille principans propter abundantiorem potentiam melius posset politiam gubernare . } Tertia via sumitur ex his quae videmus in natura . \\\hline
3.2.3 & assi commo si en vn cuerpo son departidos mienbros ordenados a departidos ofiçios \textbf{ e departidos mouimientos conuiene de dar algun mienbro vno } assi commo es el coraçon & ubi sunt diuersa membra ordinata ad diuersa officia et diuersos motus , \textbf{ est dare aliquod unum membrum } ut cor , \\\hline
3.2.3 & que son en todo el cuerpo \textbf{ Otrossi si a conposicion de vn cuerpo vienen departidos helementos conuiene de dar } y alguna cosa vna & omnis motus animalis factus in toto corpore . \textbf{ Rursus si ad constitutionem eiusdem concurrunt diuersa elementa , } est dare ibi unum aliquid , \\\hline
3.2.4 & por las quales paresçe \textbf{ que se puede prouar } que meinor cosa es & Philosophus 3 Politicorum videtur tangere tres rationes , \textbf{ per quas probari videtur , } quod melius sit ciuitatem \\\hline
3.2.4 & que lees acomnedado . \textbf{ Lo primero deue auer razon abiuada e sotil } ¶Lo segundo entencion derecha . & Videntur enim in Principe tria esse necessaria , \textbf{ ut bene regat populum sibi commissum . Primo enim debet habere perspicacem rationem . Secundo rectam intentionem . } Tertio perfectam stabilitatem . \\\hline
3.2.4 & Lo terçero firmeza acabada . \textbf{ Et destas trs cosas se pueden tomar tres razons } por las quales podemos prouar & Tertio perfectam stabilitatem . \textbf{ Ex his autem sumi possunt tres viae , } ex quibus venari possumus , \\\hline
3.2.4 & Et destas trs cosas se pueden tomar tres razons \textbf{ por las quales podemos prouar } que buena cosa es & Ex his autem sumi possunt tres viae , \textbf{ ex quibus venari possumus , } quod bonum sit principari plures , \\\hline
3.2.4 & e mas afincada en enssennoreado La segunda razon \textbf{ para prouar esto mismo se toma dela entençion derecha } que deue ser en el prinçipe & sumitur \textbf{ ex recta intentione quae requiritur in principante . } Tunc enim principans rectam habet intentionem , \\\hline
3.2.4 & e en que vno \textbf{ Et para soltar estas razones deuedes saber } que esta duda & quam unum . \textbf{ Sciendum igitur quod hanc dubitationem } quam Philosophus 3 Politicorum venatur , \\\hline
3.2.4 & Despues en esse mismo terçero tanne algunas cosas \textbf{ por las quales se pueden solcar aquellas razones } e aquellas ob iecçiones & postea in eodem 3 tangit quaedam , \textbf{ per quae obiectiones huiusmodi solui possunt . } Non enim intelligendum est , \\\hline
3.2.4 & e aquellas ob iecçiones \textbf{ ca non deuemos entender sinplemente } que fue la entençion del philosofo & per quae obiectiones huiusmodi solui possunt . \textbf{ Non enim intelligendum est , } simpliciter fuisse de intentione Philosophi , \\\hline
3.2.4 & que el ssennorio de muchos es \textbf{ mas de alabar } que el ssennorio de vno & simpliciter fuisse de intentione Philosophi , \textbf{ dominium plurium esse comendabilius dominio unius , } dum tamen utrunque sit rectum , \\\hline
3.2.4 & que si enssennoreas se vno . \textbf{ ca nunca pueden much senssennorear derechͣmente } si non en quanto tienen logar de vno & cum nunquam plures \textbf{ recte dominari possint , } nisi inquantum tenent locum unius , \\\hline
3.2.4 & si el sennorio de muchos en tanto es derecho e digno \textbf{ en quanto ellos tienen logar de vno el sennorio de vno es meior e fazer tal monarchia de vno } si se faze en manera conueinble & Quare si dominari plures in tantum est rectum et dignum , \textbf{ inquantum tenent locum unius : | dominari unum et facere monarchiam , } si debito modo fiat , \\\hline
3.2.4 & que la de muchs . \textbf{ Et pues̃ que assi es deuemos otorgar } que el regno es prinçipado muy digno & et dignior . Censendum est igitur , regnum esse dignissimum principatum , \textbf{ et } secundum rectum dominium melius est dominari unum , \\\hline
3.2.4 & que muchos . \textbf{ Mas para soltar las razones } e las obiectiones sobredichͣs deuedes saber & quam plures . \textbf{ Sed ut soluantur obiectiones praetactae , } sciendum quod quia plura cognoscunt plures quam unus , \\\hline
3.2.4 & Mas para soltar las razones \textbf{ e las obiectiones sobredichͣs deuedes saber } que la razon que dizia & quam plures . \textbf{ Sed ut soluantur obiectiones praetactae , } sciendum quod quia plura cognoscunt plures quam unus , \\\hline
3.2.4 & Et la terçera que dizia \textbf{ que muchs non se pueden assi artedrar dela entencion del bien comun commo por auentra a se arredraria vno solo . } Por ende si fuere la monarchia e el sennorio & et citius corrumpitur unus quam plures , \textbf{ et multi non sic deuiare possunt ab intentione communis boni , | sicut forte faceret unus solus : } ideo si fiat monarchia \\\hline
3.2.4 & que dize el philosofo enel terçero libro delas politicas \textbf{ que deue aconpannar assi } e tomar consigo muchos sabios & secundum Philosophum 3 Politicor’ ) \textbf{ debet sibi associare multos sapientes , } ut habeat multos oculos \\\hline
3.2.4 & que deue aconpannar assi \textbf{ e tomar consigo muchos sabios } por que ayan muchs oios & secundum Philosophum 3 Politicor’ ) \textbf{ debet sibi associare multos sapientes , } ut habeat multos oculos \\\hline
3.2.4 & por que ayan muchs oios \textbf{ e deue tomar muchs buenos e uirtuosos omes } por que ayan muchos pies e muhͣs manos & ut habeat multos oculos \textbf{ et multos bonos | et virtuosos , } ut habeat multos pedes \\\hline
3.2.4 & e en esta manera se fara el prinçipe vn omne de muchs oios e de muchͣs manos e de muchs pies . \textbf{ Et pues que assi es non se puede dezer } que vn tal monarchia o tal prinçipe & ut habeat multos pedes \textbf{ et multas manus : } et sic fiet unus homo multorum oculorum , multarum manuum , \\\hline
3.2.4 & que aquellos sabios conosçen \textbf{ e saben todo lo deue conosçer } e saber el Rey . & quia quantum spectat ad regimen regni , quicquid omnes illi sapientes cognoscunt , \textbf{ totum ipse Rex cognoscere dicitur . } Nec etiam dici poterit ipsum de leui posse corrumpi et peruerti : \\\hline
3.2.4 & e saben todo lo deue conosçer \textbf{ e saber el Rey . } nin avn se puede dezer & quia quantum spectat ad regimen regni , quicquid omnes illi sapientes cognoscunt , \textbf{ totum ipse Rex cognoscere dicitur . } Nec etiam dici poterit ipsum de leui posse corrumpi et peruerti : \\\hline
3.2.4 & e saber el Rey . \textbf{ nin avn se puede dezer } que aquel vn prinçipe de ligero se pueda trastornar & totum ipse Rex cognoscere dicitur . \textbf{ Nec etiam dici poterit ipsum de leui posse corrumpi et peruerti : } nam si Rex recte dominari desiderat , \\\hline
3.2.4 & nin avn se puede dezer \textbf{ que aquel vn prinçipe de ligero se pueda trastornar } e coe ronper se & totum ipse Rex cognoscere dicitur . \textbf{ Nec etiam dici poterit ipsum de leui posse corrumpi et peruerti : } nam si Rex recte dominari desiderat , \\\hline
3.2.4 & que aquel vn prinçipe de ligero se pueda trastornar \textbf{ e coe ronper se } ca si el Rey dessea enssennorear & Nec etiam dici poterit ipsum de leui posse corrumpi et peruerti : \textbf{ nam si Rex recte dominari desiderat , } non est possibile ipsum peruerti , \\\hline
3.2.4 & e coe ronper se \textbf{ ca si el Rey dessea enssennorear } derechͣmente & Nec etiam dici poterit ipsum de leui posse corrumpi et peruerti : \textbf{ nam si Rex recte dominari desiderat , } non est possibile ipsum peruerti , \\\hline
3.2.4 & e de los omes buenos \textbf{ e quisiere seguir su cabeça proprea } e su appetito corrupto e malo & ut si spreto consilio , et dimissa societate sapientum et bonorum , \textbf{ vellet sequi caput proprium , et appetitum priuatum , iam non esset Rex sed tyrannus : } talem ergo dominari non esset melius quam plures : \\\hline
3.2.5 & que ꝑ non por heredamiento paresçe \textbf{ que duda si es meior de el tablesçer el sennor } por suerte & an per haereditatem : \textbf{ dubitare videtur , | an melius sit constituere dominium arte , } vel sorte . \\\hline
3.2.5 & qual sera el fijo \textbf{ al que pertenesçe de tegnar } e de auer la dignidat real . & nam incertum est qualis debeat esse filius , \textbf{ ad quem spectabit habere regiam dignitatem . } Absolute ergo loquendo , \\\hline
3.2.5 & al que pertenesçe de tegnar \textbf{ e de auer la dignidat real . } Et por ende paresça e a alguons & nam incertum est qualis debeat esse filius , \textbf{ ad quem spectabit habere regiam dignitatem . } Absolute ergo loquendo , \\\hline
3.2.5 & por pruiena patesçe \textbf{ que deuemos iudgar } que mas conuiene al regno & quas experimentaliter videmus , \textbf{ videtur esse censendum magis expedire regno vel ciuitati , } ut dominus praeficiatur per haereditatem , \\\hline
3.2.5 & La segunda de parte del fijo \textbf{ que deue heredar el regno . } la terçera de parte del pueblo que deue ser gouernado & Tertia ex parte populi , \textbf{ qui debet per tale regimen gubernari . } Prima via sic patet . \\\hline
3.2.5 & ca segunt elpho en el segundo libro delas politicas \textbf{ non se puede contar } quanto plaz ha el omne & Prima via sic patet . \textbf{ Nam secundum Philosophum 2 Politic’ inenarrabile est quantam dilectionem habeat , } et quantum differat patare aliquid proprium : \\\hline
3.2.5 & Por la qual cosa si el Rey viere \textbf{ que deue regnar sobre el regno } non solamente en su uida & et bonum proprium : \textbf{ quare si Rex videat debere se principari super regnum non solum ad vitam , } sed etiam per haereditatem in propriis filiis , \\\hline
3.2.5 & por ende el Rey en toda cosa \textbf{ que pudiere mouerse a procurar todo buen estado del regno } si penssare & et nimio ardore mouentur patres erga dilectionem filiorum : \textbf{ ideo omni cura qua poterit mouebitur ad procurandum bonum statum regni , } si cogitet ipsum prouenire ad dominium filiorum . Philosophus tamen 3 Politic’ \\\hline
3.2.5 & si penssare \textbf{ que el regno ha de venir a señorio de los fijos enpero el pho enel terçero libro delas politicas } do fabla destamateria dize & ideo omni cura qua poterit mouebitur ad procurandum bonum statum regni , \textbf{ si cogitet ipsum prouenire ad dominium filiorum . Philosophus tamen 3 Politic’ } ubi de hac materia determinat , ait , \\\hline
3.2.5 & do fabla destamateria dize \textbf{ que esto es muy guaue conuene saber } que los padres pueden dar & si cogitet ipsum prouenire ad dominium filiorum . Philosophus tamen 3 Politic’ \textbf{ ubi de hac materia determinat , ait , } quod hoc est difficile , \\\hline
3.2.5 & que esto es muy guaue conuene saber \textbf{ que los padres pueden dar } assi el gouierno del regno & ubi de hac materia determinat , ait , \textbf{ quod hoc est difficile , } videlicet \\\hline
3.2.5 & assi que los fijos en esta manera ouiessen la heredat de los padres \textbf{ o podemos dezir sinplemente que es por la uirtud de dios . } si los Reyes e los prinçipes non regnaren & ut filii hoc modo succederent in haereditatem paternam . \textbf{ Vel simpliciter dicitur hoc esse diuinum , } quia nisi Reges et Principes bene se habeant erga diuina , \\\hline
3.2.5 & pues que assi es departe del Rey \textbf{ por que mayor cuydado aya del bien del regno podemos prouar } que el gouernamiento real & vix aut nunquam regnabunt filii filiorum . Ex parte ergo regis \textbf{ ut magis solicitetur circa bonum regni , } arguere possumus , \\\hline
3.2.5 & e non por eleccion . \textbf{ La segunda razon para puar esto mismo se tora de parte del fino } a quien parte nesçe de regnar & arguere possumus , \textbf{ quod expediat regale regimen in filios per haereditatem succedere . Secunda via ad inuestigandum hoc idem , sumitur ex parte filii , } ad quem spectat suscipere curam regni . \\\hline
3.2.5 & La segunda razon para puar esto mismo se tora de parte del fino \textbf{ a quien parte nesçe de regnar } e de tomar cuydado del regno . & quod expediat regale regimen in filios per haereditatem succedere . Secunda via ad inuestigandum hoc idem , sumitur ex parte filii , \textbf{ ad quem spectat suscipere curam regni . } Nam sicut mores nuper ditatorum \\\hline
3.2.5 & a quien parte nesçe de regnar \textbf{ e de tomar cuydado del regno . } ca assi commo las costunbres de los que en el otro dia fueron enrrequeçidos & quod expediat regale regimen in filios per haereditatem succedere . Secunda via ad inuestigandum hoc idem , sumitur ex parte filii , \textbf{ ad quem spectat suscipere curam regni . } Nam sicut mores nuper ditatorum \\\hline
3.2.5 & que los otros \textbf{ ca estos tales non saben sofrir las uenturas } por que ser leunatado en rey del otro dia es & sunt moribus aliorum . \textbf{ Nesciunt enim tales fortunas ferre , } nuper enim esse exaltatum in Regem , \\\hline
3.2.5 & por que ser leunatado en rey del otro dia es \textbf{ assi commo non saber qual es la maiestad nin la dignidat Real } ca estos tales & nuper enim esse exaltatum in Regem , \textbf{ est quasi quaedam ineruditio regiae dignitatis tales quidem } ut plurimum tyrannizant , \\\hline
3.2.5 & ca non tienen por desaguisado \textbf{ si heredar en aquello que heredaron sus padres } e si omerenaqllo que omeron sus padres & si illud habeant quod patres possederant : \textbf{ quare ex parte filiorum debentium succedere in haereditatem paternam , } expedit regno \\\hline
3.2.5 & por la qual razon de parte de los fijos \textbf{ que deuen heredar los bienes de los padres conuiene al regno } por que non sea mal gouernado & quare ex parte filiorum debentium succedere in haereditatem paternam , \textbf{ expedit regno } ne inerudite regatur , \\\hline
3.2.5 & si por costunbre muy prolongada obedesçe alos padres . \textbf{ ha por aguisado de obedesçer alos fijos } e alos fijos de sus fijos & quasi naturalia . \textbf{ Populus ergo si per diuturnam consuetudinem obediuit patribus , filiis , } et filiorum filiis , \\\hline
3.2.5 & conuiene que la Real dignidat vaya por hedamiento \textbf{ Pues que assi es determinar la casa } e el linage & et facilius obediat populus mandatis regis , \textbf{ expedit regiae dignitati per haereditatem succedere . Determinare igitur domum et prosapiam , } ex qua assumendus est rex , sedat litigia , tollit tyrannidem , \\\hline
3.2.5 & e vn por que esta es condiçion de thirano \textbf{ non tener mientes } en el bien del regno & non sic diligunt bonum regni sicut haereditarie principantes : \textbf{ et } quia hoc est esse tyrannum , non intendere bonum regni , tales facilius tyrannizant . Facit \\\hline
3.2.5 & que el ssennorio sea natural . \textbf{ ca el pueblo inclinase naturalmente a obedesçer los mandamientos de tal Rey } e desto puede paresçer & etiam hoc dominium naturale , \textbf{ quia populus quasi naturaliter inclinatur | ut obediat iussionibus talis Regis . } Ex hoc autem patere potest , \\\hline
3.2.5 & ca el pueblo inclinase naturalmente a obedesçer los mandamientos de tal Rey \textbf{ e desto puede paresçer } que non solamente parte nesçe al regno & ut obediat iussionibus talis Regis . \textbf{ Ex hoc autem patere potest , } quod non solum expedit regno determinare prosapiam , \\\hline
3.2.5 & que non solamente parte nesçe al regno \textbf{ de determinar el linage donde ha de ser tomado el sennor . } Mas avn conuiene de determinar la perssona . & quod non solum expedit regno determinare prosapiam , \textbf{ ex qua praeficiendus est dominus , } sed etiam oportet determinare personam . \\\hline
3.2.5 & de determinar el linage donde ha de ser tomado el sennor . \textbf{ Mas avn conuiene de determinar la perssona . } Ca assi commo nasçen discordias e lides & ex qua praeficiendus est dominus , \textbf{ sed etiam oportet determinare personam . } Nam sicut oriuntur dissentiones \\\hline
3.2.5 & en qual linage deua ser prinçipe \textbf{ e auer el senorio . } Mas determinar tal perssona non hagniueza ninguna & ø \\\hline
3.2.5 & e auer el senorio . \textbf{ Mas determinar tal perssona non hagniueza ninguna } do el sennorio viene por hedamiento & si ignoretur ex qua prosapia assumendus sit Rex : sic etiam litigia oriuntur , \textbf{ si non determinetur quae persona in illa prosapia debeat principari . Talem autem determinare personam , difficultatem non habet : } nam si dignitas regia per haereditatem transferatur ad posteros , oportet eam transferre in filios , quia secundum lineam consanguinitatis filii parentibus maxime sunt coniuncti : \\\hline
3.2.5 & mas aman alos primogenitos \textbf{ por que el Rey aya mayor cuy dado del bien del regno conuiene de establesçer } que el regno pertenesca al primo genito & quia patres plus communiter primogenitos diligunt ; \textbf{ ut magis sit curae regi de bono regni , } expedit statuere regnum succedere primogenito : \\\hline
3.2.5 & que alos mayores atales \textbf{ argunentos de ligero podemos responder } ca los fe cho humanales & quod contingit \textbf{ aliquando magis diligere minores . Talibus obiectionibus de facili respondetur : } quia facta humana et gesta particularia omnino sub recta regula , \\\hline
3.2.5 & nin so çierta cuenta . \textbf{ Pues que assi es cunple en tales cosas yr } por aquello que prouamos & et sub certa narratione non cadunt : \textbf{ sufficit ergo in talibus pertransire probabiliter , } et leges statuere quae ut in pluribus continent veritatem . \\\hline
3.2.5 & por aquello que prouamos \textbf{ e establesçer leyes por que en la mayor parte ayan uerdat e sean uerdaderas . } Mas aqual lo que dessuso fue dich conuiene saber & sufficit ergo in talibus pertransire probabiliter , \textbf{ et leges statuere quae ut in pluribus continent veritatem . } Quod vero superius tangebatur , \\\hline
3.2.5 & e establesçer leyes por que en la mayor parte ayan uerdat e sean uerdaderas . \textbf{ Mas aqual lo que dessuso fue dich conuiene saber } que quando va el regno & et leges statuere quae ut in pluribus continent veritatem . \textbf{ Quod vero superius tangebatur , } videlicet quod ire per haereditatem , \\\hline
3.2.5 & por que non saben qual ha de ser el Rey \textbf{ que ha de uenir } a esto podemos dezir & dignitatem regiam , est exponere fortunae , \textbf{ eo quod ignoretur qualis regius filius sit futurus . Dici debet quod vix sunt aliqua gesta humana , } quae ex aliqua parte non exponantur periculis . Illa tamen magis cauenda sunt , \\\hline
3.2.5 & que ha de uenir \textbf{ a esto podemos dezir } que apenas ay ningunos fechs de los omes & ø \\\hline
3.2.5 & que en alguna parte non sean espuestos a peligros e auenturas . \textbf{ Enpero aquellas cosas son mas de escusar } que pueden auer mayores peligros & eo quod ignoretur qualis regius filius sit futurus . Dici debet quod vix sunt aliqua gesta humana , \textbf{ quae ex aliqua parte non exponantur periculis . Illa tamen magis cauenda sunt , } quae pluribus periculis exponuntur . \\\hline
3.2.5 & Enpero aquellas cosas son mas de escusar \textbf{ que pueden auer mayores peligros } ca veeraos & quae ex aliqua parte non exponantur periculis . Illa tamen magis cauenda sunt , \textbf{ quae pluribus periculis exponuntur . } Videmus autem per experientiam plura mala oriri in ciuitatibus et regnis , \\\hline
3.2.5 & Onde muchs males auemos visto en tales gouernamientos \textbf{ que non podemos contar } de los quales non podemos fablar & plura enim huiusmodi mala in talibus regiminibus vidimus , \textbf{ quae enumerare } per singula longum esset . \\\hline
3.2.5 & que non podemos contar \textbf{ de los quales non podemos fablar } nin contar de cada vno & plura enim huiusmodi mala in talibus regiminibus vidimus , \textbf{ quae enumerare } per singula longum esset . \\\hline
3.2.5 & de los quales non podemos fablar \textbf{ nin contar de cada vno } por menudo pues que assi es podemos dezir & quae enumerare \textbf{ per singula longum esset . } Dicere ergo possumus \\\hline
3.2.5 & nin contar de cada vno \textbf{ por menudo pues que assi es podemos dezir } que conuiene al regno & per singula longum esset . \textbf{ Dicere ergo possumus } quod expedit regno , filium succedere in regimine patris : \\\hline
3.2.5 & a quien parte nesçe auer cuydado del regno \textbf{ este fallesçimiento se puede conplir por sabios et por buenos omes } lo quales deue el Rey ayuntar assi & et si aliquis defectus esset in filio regio , ad quem deberet regia cura peruenire , suppleri poterit per sapientes \textbf{ et bonos , } quos tanquam manus et oculos debet sibi Rex in societatem coniungere : \\\hline
3.2.5 & este fallesçimiento se puede conplir por sabios et por buenos omes \textbf{ lo quales deue el Rey ayuntar assi } e auer en su conpannia & et si aliquis defectus esset in filio regio , ad quem deberet regia cura peruenire , suppleri poterit per sapientes \textbf{ et bonos , } quos tanquam manus et oculos debet sibi Rex in societatem coniungere : \\\hline
3.2.5 & lo quales deue el Rey ayuntar assi \textbf{ e auer en su conpannia } assi comm̃ sus manos e sus oios . & et bonos , \textbf{ quos tanquam manus et oculos debet sibi Rex in societatem coniungere : } valde tamen reprehensibiles sunt Reges et Principes , \\\hline
3.2.5 & assi comm̃ sus manos e sus oios . \textbf{ Enpero much son de reprehender los Reyes e los prinçipes } si non fueren muy acuçiosos & quos tanquam manus et oculos debet sibi Rex in societatem coniungere : \textbf{ valde tamen reprehensibiles sunt Reges et Principes , } si non nimia cura solicitentur \\\hline
3.2.5 & Et non cunple \textbf{ que por el primogenito deue regnar } que del solo deue ser tomada acuçia & et totius regni in hoc consistat . \textbf{ Nec sufficit quod quia solus primogenitus regnare debet , } ut de eo solo cura habeatur diligens : \\\hline
3.2.5 & Por ende por que el bien comun non sea puesto a peligro de todos los fios \textbf{ deuen auer los padres grant cuydado } pho en el quanto libro delas poluenta tres cosas & ne periclitetur bonum commune , \textbf{ de omnibus filiis Regis cura diligens est habenda . } Philosophus 5 Politic’ narrat tria , \\\hline
3.2.6 & en las quales auataia soƀre los otros . \textbf{ ca dize que antiel Rey deue sobrepiuar } e auerguamente los Reyes eran establesçidos enlos sennorios & Philosophus 5 Politic’ narrat tria , \textbf{ in quibus Rex alios debet excedere . Dicit enim , } quod antiquitus Reges a triplici excessu constituebantur . \\\hline
3.2.6 & por ende aquellos que vee ser liberales \textbf{ e bien fechores mueuense con grant ardor alos amar } e dessean de los auer por sennores & quos videt esse liberales et beneficos , \textbf{ nimis ardenter mouetur in eorum amorem , } et optat eos habere in dominos . Inde est quod antiquitus plures sic praeficiebantur in Reges . \\\hline
3.2.6 & por su Rey \textbf{ ¶Lo segundo puede alguno sobrepuiar a otro } por a unataia de obras uirtuosas . & Nam si aliquis fuerat primo beneficus , \textbf{ gens illa tracta ad amorem eius praeficiebat ipsum in Regem . Secundo potest aliquis praefici in Regem ab excessu virtuosarum actionum : } nam \\\hline
3.2.6 & ca por que de los buenos \textbf{ e de los uirtuosos es de amar } mas el bien comun & nam \textbf{ quia bonorum virtuosorum est diligere bonum commune potius quam priuatum , } ideo reputatur dignus \\\hline
3.2.6 & que sea tomado por Rey \textbf{ Lo terçero fue acostunbrado de tomar alguno por Rey } pora una taia de poderio e de dignidat & qui a populo creditur virtuosus . \textbf{ Tertio consueuit praefici aliquis in Regem ab excessu potentiae } et dignitatis . \\\hline
3.2.6 & ca por que es cosa prouada \textbf{ que los nobles e los poderosos toman mayor uerguença de obrar cosas torpes e feas } que los otros & et dignitatis . \textbf{ Nam quia probabile est nobiles et potentes , } magis verecundari operari turpia quam alios : \\\hline
3.2.6 & que el Rey \textbf{ que quiere bien gouernar su regno } quanto parte nesçe alo presente & decens est tales excessus in ipsa monarchia perfectius reperiri . Decet enim ipsum regem volentem \textbf{ recte regere } ( quantum ad praesens spectat ) ad tria solicitari . Primo , \\\hline
3.2.6 & que el rey aya aquellas tres aun ataias \textbf{ e aquellas tres condiçiones buenas sobredichͣs . ca si abondare en bien fazer } seria muy amado del pueblo & Quare expedit regem habere praedictos tres excessus . \textbf{ Nam si abundet } in beneficiis tribuendis , \\\hline
3.2.6 & en obras uirtuosas procurara el bien comun . \textbf{ ca si la uirtud parte nesçede se estender a mayor bien } e en mayor bien dela gente & procurabit commune bonum : \textbf{ quia si virtutis est , } tendere in bonum , \\\hline
3.2.6 & que el Rey abonde en poderio çiuil \textbf{ por que pueda castigar los que se quisieren le una tar } contra la paz del regno & et virtuosus , quam bonum aliquod proprium \textbf{ et priuatum . Tertio expedit eum abundare in ciuili potentia , ut possit corrigere volentes insurgere , } et turbare pacem regni . \\\hline
3.2.6 & contra la paz del regno \textbf{ Visto en quales cosas el Rey deue sobrepuiar } e auer a una taia de los otros . & et turbare pacem regni . \textbf{ Viso in quibus Rex alios debet excedere : } restat ostendere , \\\hline
3.2.6 & Visto en quales cosas el Rey deue sobrepuiar \textbf{ e auer a una taia de los otros . } finca deuer en qual manera ay departimiento entre el Rey e el thirano . & Viso in quibus Rex alios debet excedere : \textbf{ restat ostendere , } quomodo differat a Tyranno . \\\hline
3.2.6 & e auer a una taia de los otros . \textbf{ finca deuer en qual manera ay departimiento entre el Rey e el thirano . } Mas el philosofo en el quinto libro delas pol . & restat ostendere , \textbf{ quomodo differat a Tyranno . } Tangit autem Philosophus 5 Politicorum quatuor differentias \\\hline
3.2.6 & e deste departimiento primero se sigue el segundo . \textbf{ Conuiene a saber } que el thiranno entiende en el bien delectable . & Ex hac autem differentia prima sequitur secunda : \textbf{ videlicet } quod tyrannus intendit bonum delectabile : Rex vero bonum honorificum . \\\hline
3.2.6 & Mas el rey en tiede el bien de honrraca \textbf{ assi commo non se puede contar } quanto se delecta cada vno en el bien propreo & quod tyrannus intendit bonum delectabile : Rex vero bonum honorificum . \textbf{ Nam sicut inenarrabile est } quanto quis delectatur in bono proprio , \\\hline
3.2.6 & quanto se delecta cada vno en el bien propreo \textbf{ assi non se puede dezir } nin contar & quanto quis delectatur in bono proprio , \textbf{ sic quasi inenarrabile est quantus honor sequitur , } et quanto honore est dignius intendens commune bonum . \\\hline
3.2.6 & assi non se puede dezir \textbf{ nin contar } quanta honrra se sigue & quanto quis delectatur in bono proprio , \textbf{ sic quasi inenarrabile est quantus honor sequitur , } et quanto honore est dignius intendens commune bonum . \\\hline
3.2.6 & veste departimiento segundo se sigue el terçero . \textbf{ Conuiene de saber } que la entencion del tiran no es en auer riquezas o dineros & Ex hac autem secunda differentia sequitur tertia ; \textbf{ videlicet quod intentio tyrannica est circa pecuniam . } Tyrannus \\\hline
3.2.6 & Conuiene de saber \textbf{ que la entencion del tiran no es en auer riquezas o dineros } Ca el tirano & Ex hac autem secunda differentia sequitur tertia ; \textbf{ videlicet quod intentio tyrannica est circa pecuniam . } Tyrannus \\\hline
3.2.6 & Et por ende la su entençion toda se pone \textbf{ en el auer o en los dinos creyendo que por ellos puede auer las otras cosas delectables . } Mas la entençion del Rey esta & quia spreto communi bono non curat \textbf{ nisi de delectationibus propriis , maxime versatur sua intentio circa pecuniam , credens se per eam posse huiusmodi delectabilia obtinere . } Sed regis intentio versatur circa virtutem , \\\hline
3.2.6 & Et deste departimiento terçero se sigue el quarto . \textbf{ Conuiene de saber } que el tiranno non ha cuydado de ser guardado de los çibdadanos & et commune . \textbf{ Ex hac autem differentia } tertia sequitur quarta videlicet quod tyrannus non curat custodiri a ciuibus , \\\hline
3.2.6 & e del bien comun fia muchͣ de aquellos que son en el su regno . \textbf{ Et por ende se faze guardar de los sus çibdadanos propreos } e non de los estrannos . & eo quod videat se maximam curam habere de bono regni et communi , \textbf{ maxime confidit de his qui sunt in regno . Ideo facit se custodiri a propriis , } non ab extraneis . \\\hline
3.2.7 & e non de los estrannos . \textbf{ or quatro razones podemos prouar } que la thirama es muy mal prinçipado & non ab extraneis . \textbf{ Quadruplici via venari possumus , } tyrannidem esse pessimum principatum . \\\hline
3.2.7 & La terçera se toma \textbf{ por razon que tal prinçipado es muy afincado por enpesçer . } la quarta & Tertia , \textbf{ ex eo quod est efficacissimum ad nocendum . Quarta , } ex eo quod impedire habet maxime bona ipsorum ciuium . Prima via sic patet . \\\hline
3.2.7 & la quarta \textbf{ por razon que tal prinçipado ha de enbargar muy grandeᷤ bienes delos çibdadanos ¶ } La primera se declara & ex eo quod est efficacissimum ad nocendum . Quarta , \textbf{ ex eo quod impedire habet maxime bona ipsorum ciuium . Prima via sic patet . } Quia si dominatur Rex , \\\hline
3.2.7 & dize el philosofo \textbf{ que tal sennor auer prinçipado es } assi commo partir vn prinçipado en muchs . & Unde 3 Polit’ dicitur , quod principari talem , est quasi partiri principatum in multos . Vel ( quod idem est ) \textbf{ est quasi principari multitudinem , } eo quod in tali principatu intendatur bonum multorum . \\\hline
3.2.7 & que tal sennor auer prinçipado es \textbf{ assi commo partir vn prinçipado en muchs . } o que es esso mismo que auer muchs el prinçipado . & Unde 3 Polit’ dicitur , quod principari talem , est quasi partiri principatum in multos . Vel ( quod idem est ) \textbf{ est quasi principari multitudinem , } eo quod in tali principatu intendatur bonum multorum . \\\hline
3.2.7 & assi commo partir vn prinçipado en muchs . \textbf{ o que es esso mismo que auer muchs el prinçipado . } por que en tal prinçipado es entendido el bien de muchs . & est quasi principari multitudinem , \textbf{ eo quod in tali principatu intendatur bonum multorum . } Sic etiam si dominentur plures , \\\hline
3.2.7 & si non fuere en alguͣ manera cosa diuinal \textbf{ e por que en el prinçipalmente se deue entender el bien comun } que es mas diuinal que ningun bien singular . & quid diuinum , \textbf{ et quia in eo principaliter | debet intendi bonum commune } quod est diuinius quam aliquod singulare ; \\\hline
3.2.7 & la segunda manera \textbf{ para prouar esto mismo se toma . } por razon que tal sennorio es muy desnatural & quia tyrannis plurimum distat a politia , \textbf{ idest a communi bono . Secunda via ad inuestigandum hoc idem , sumitur ex eo quod tale dominium maxime est naturale . } Nam illa est naturalis operatio erga aliquid , \\\hline
3.2.7 & quando cada cosa se faze \textbf{ assi commo se deue fazer } por la qual cosa el regno estonçe es naturalmente gouernado & quando sic agitur \textbf{ ut est aptum natum agi : } quare tunc regnum naturaliter agitur , \\\hline
3.2.7 & assi gouernados \textbf{ commo se deuen gouernar . } Mas el omne & quando homines existentes in ipso sic reguntur \textbf{ ut sunt apti nati regi : } homo autem quia libero arbitrio et ratione participat , \\\hline
3.2.7 & e ha razon estonçe es naturalmente gouernado \textbf{ assi commo se deue gouernar } quando de uoluntad sirue & tunc naturaliter regitur \textbf{ et ut est aptus natus regi , } quando voluntarie seruit \\\hline
3.2.7 & ¶La terçera razon se toma \textbf{ por que tal prinçipado es muy afincado para enpeesçer . casi commo el prinçipado del Rey } por que es muy vno es muy afincado para aprouechar . & Tertia via sumitur ex eo quod talis principatus est efficacissimus ad nocendum . \textbf{ Nam sicut principatus Regis eo quod sit maxime unitus , } est efficacissimus ad proficiendum : \\\hline
3.2.7 & por que tal prinçipado es muy afincado para enpeesçer . casi commo el prinçipado del Rey \textbf{ por que es muy vno es muy afincado para aprouechar . } Assi la tirania es muy afincada para enpees çer & Nam sicut principatus Regis eo quod sit maxime unitus , \textbf{ est efficacissimus ad proficiendum : } sic tyrannis efficacissima ad nocendum . Monarchia enim \\\hline
3.2.7 & por que es muy vno es muy afincado para aprouechar . \textbf{ Assi la tirania es muy afincada para enpees çer } ca el senñorio de vno enssennorea & est efficacissimus ad proficiendum : \textbf{ sic tyrannis efficacissima ad nocendum . Monarchia enim } quia ibi dominatur unus , \\\hline
3.2.7 & e el su prinçipado es muy bueno . \textbf{ ca por la uirtud ayuntada en el puede fazer muchͣs bueans cosas . } Et si por auentura el prinçipe ha la entençion tuerta estonçe es tirano & tunc est Rex et est optimus principatus : \textbf{ quia propter unitatem virtutes potest multa bona efficere : } si vero monarchia habet intentionem peruersam , \\\hline
3.2.7 & ca por el su poderio muy grande \textbf{ que es ayuntado en vno puede fazer muchs males } e esta razon tanne el philosofo en el quinto libro delas politicas & et est pessimus , \textbf{ quia propter suam unitam potentiam potest multa mala efficere . } Hanc autem rationem tangit Philosophus quinto Politicorum \\\hline
3.2.7 & que la tirnia es la postrimera obligarçia \textbf{ que quiere dezer muy mala obligacion } por que es muy enpesçedera alos subditos ¶ & ubi ait , \textbf{ tyrannidem esse oligarchiam extremam idest pessimam : } quia est maxime nociua subditis . \\\hline
3.2.7 & mas avn esfuercasse \textbf{ para enbargar los bienes dellos } e tanne espho en el quinto libro delas politicas muy grandes tres bienes & sed \textbf{ etiam satagit impedire eorum maxima bona . Tangit autem Philosophus 5 Polit’ tria maxima bona , } quae satagit impedire tyrannus , \\\hline
3.2.7 & e tanne espho en el quinto libro delas politicas muy grandes tres bienes \textbf{ que puna de enbargar el tirano . } Conuiene de saber . & etiam satagit impedire eorum maxima bona . Tangit autem Philosophus 5 Polit’ tria maxima bona , \textbf{ quae satagit impedire tyrannus , } videlicet pacem , virtutes , \\\hline
3.2.7 & que puna de enbargar el tirano . \textbf{ Conuiene de saber . } Paz ¶ Virtudes . & videlicet pacem , virtutes , \textbf{ et scientias . } Tyranni enim nolunt ciues habere pacem et concordiam adinuicem : \\\hline
3.2.7 & ayuso se dira . \textbf{ Et cunpla agora de saber } que la tirania es muy mal prinçipado & Quare autem tyranni praedicta bona impediunt in ciuibus ; infra dicetur . \textbf{ Sufficiat autem ad praesens scire , } tyrannidem esse pessimum principatum propter rationes tactas . \\\hline
3.2.7 & por las razones sobredichͣs . \textbf{ Mas en commo los Reyes en toda manera de una esquiuar } de non enssennorear con sennorio de tirania & tyrannidem esse pessimum principatum propter rationes tactas . \textbf{ Quod autem reges summo opere cauere debeant , } ne principentur principatu tyrannico , \\\hline
3.2.7 & Mas en commo los Reyes en toda manera de una esquiuar \textbf{ de non enssennorear con sennorio de tirania } esto non es guaue de mostrar & Quod autem reges summo opere cauere debeant , \textbf{ ne principentur principatu tyrannico , } videre non est difficile . Nam tanto peior est Princeps , \\\hline
3.2.7 & de non enssennorear con sennorio de tirania \textbf{ esto non es guaue de mostrar } ca qtanto peor es el prinçipe & ne principentur principatu tyrannico , \textbf{ videre non est difficile . Nam tanto peior est Princeps , } quanto peiori dominio principatur : quare si omnem diligentiam adhibere debet Rex \\\hline
3.2.7 & quanto peor es el su sennorio \textbf{ por la qual cosa el Rey deue poner muy grant acuçia } por non ser tiranno & videre non est difficile . Nam tanto peior est Princeps , \textbf{ quanto peiori dominio principatur : quare si omnem diligentiam adhibere debet Rex } ne sit pessimus , \\\hline
3.2.7 & e por non ser mal prinçipe \textbf{ e much deue escusar } de non enssennorear con señorio de tirania & ne sit pessimus , \textbf{ summe curare debet } ne dominetur \\\hline
3.2.7 & e much deue escusar \textbf{ de non enssennorear con señorio de tirania } que es muy mal prinçipado & summe curare debet \textbf{ ne dominetur | per tyrannidem } qui est pessimus principatus . \\\hline
3.2.8 & que es muy mal prinçipado \textbf{ el Rey o el prinçipe quiere gouernar conueniblemente la gente } que le es a comnedada & qui est pessimus principatus . \textbf{ Si Rex aut Princeps gentem sibi commissam vult debite gubernare , } et scire desiderat \\\hline
3.2.8 & que le es a comnedada \textbf{ e si dessea saber } qual es el su ofiçio deue penssar con grant acuçia en las cosas naturales & Si Rex aut Princeps gentem sibi commissam vult debite gubernare , \textbf{ et scire desiderat } quod sit eius officium : \\\hline
3.2.8 & e si dessea saber \textbf{ qual es el su ofiçio deue penssar con grant acuçia en las cosas naturales } ca si toda la natura es gouernada & et scire desiderat \textbf{ quod sit eius officium : | diligenter considerare debet in naturalibus rebus . } Nam si natura tota administratur per ipsum Deum , \\\hline
3.2.8 & Por ende del gouernamiento \textbf{ que veemos enlas cosas naturales deue descender el gouernamiento } que es de dar & quare a regimine , \textbf{ quod videmus in naturalibus , deriuari debet regimen , } quod trahendum est in arte de regimine regum ; \\\hline
3.2.8 & que veemos enlas cosas naturales deue descender el gouernamiento \textbf{ que es de dar } en el arte del gouernamiento de los Reyes & quod videmus in naturalibus , deriuari debet regimen , \textbf{ quod trahendum est in arte de regimine regum ; } est enim ars imitatrix naturae . \\\hline
3.2.8 & que la natura primeramente da a todas las cosas aquello \textbf{ por que pueden alcançar su fin . } ¶ Lo segundo les da aquellas cosas & ø \\\hline
3.2.8 & ¶ Lo segundo les da aquellas cosas \textbf{ por que pueden arredrar dessi las cosas que les enpesçen . } lo terçero las cosas naturales & quod natura primo dat rebus \textbf{ ea per quae possunt consequi finem suum . Secundo dat eis ea per quae possunt prohibentia remouere . } Tertio per huiusmodi collata naturaliter intendunt in suos fines siue in suos terminos . \\\hline
3.2.8 & que la natura da al fuego luunadat \textbf{ por la qual puede sobir arriba } ¶ & Ut natura dat igni leuitatem , \textbf{ per quam potest tendere sursum . } Secundo dat ei calorem , \\\hline
3.2.8 & que en tal manera sea el pueblo apareiado e ordenado \textbf{ por que pue da alcançar su fin } que entiende . & ut populus taliter disponatur et ordinetur , \textbf{ ut possit consequi finem intentum . Secundo , } ut remoueantur prohibentia \\\hline
3.2.8 & Lo segundo conuiene que sean arredradas todas aquellas cosas \textbf{ que enbargan de alcançar aquella fin } ¶Lo terçero conuiene & ut possit consequi finem intentum . Secundo , \textbf{ ut remoueantur prohibentia } et deuiantia ab huiusmodi fine . Tertio , ut dirigatur et promoueatur in finem praedictum . \\\hline
3.2.8 & que primero la fazemos derecha \textbf{ por que pueda yr meior ala señal . } Lo segundo la fazemos enpennolada & quia primo efficitur recta , \textbf{ ut possit melius in finem tendere : } secundo efficitur pennata , \\\hline
3.2.8 & Lo segundo la fazemos enpennolada \textbf{ por que pueda meiorfender el ayre } por que non sea enbargada de yr a su señal ¶ & secundo efficitur pennata , \textbf{ ut melius aerem scindat } ne prohibeatur tendere in ipsum signum : \\\hline
3.2.8 & por que pueda meiorfender el ayre \textbf{ por que non sea enbargada de yr a su señal ¶ } Lo terçero es puesta en la ballesta del saetero & ut melius aerem scindat \textbf{ ne prohibeatur tendere in ipsum signum : } tertio a sagittante sagittatur \\\hline
3.2.8 & assi commo vna saeta \textbf{ que es de enderesçar } e gara la fin e al bien comun . & ø \\\hline
3.2.8 & por que la gente que les acomnedada aya aquellas cosas \textbf{ por las quales puede alcançar la fin } que entiende . & Quia primo solicitari debet , \textbf{ ut gens sibi commissa habeat per quae possit consequi finem intentum . } Secundo debet prohibentia remouere . \\\hline
3.2.8 & que entiende . \textbf{ ¶ Lo segundo deue arredrar todas aquellas cosas } que enbargan de alcançar la fin . & ut gens sibi commissa habeat per quae possit consequi finem intentum . \textbf{ Secundo debet prohibentia remouere . } Tertio gentem ipsam debet in finem dirigere . \\\hline
3.2.8 & ¶ Lo segundo deue arredrar todas aquellas cosas \textbf{ que enbargan de alcançar la fin . } Lo terçero deue el Rey enderesçar & Secundo debet prohibentia remouere . \textbf{ Tertio gentem ipsam debet in finem dirigere . } Ea vero quae deseruiunt \\\hline
3.2.8 & que enbargan de alcançar la fin . \textbf{ Lo terçero deue el Rey enderesçar } e gniar su gente & Secundo debet prohibentia remouere . \textbf{ Tertio gentem ipsam debet in finem dirigere . } Ea vero quae deseruiunt \\\hline
3.2.8 & Lo terçero deue el Rey enderesçar \textbf{ e gniar su gente } e su pueblo a su fin & ø \\\hline
3.2.8 & Mas aquellas cosas que siruen aesto \textbf{ por que el pueblo pueda alcançar su fin } e pueda bien beuir son estas . & Ea vero quae deseruiunt \textbf{ ut populus possit consequi finem intentum } et bene viuere , \\\hline
3.2.8 & e pueda bien beuir son estas . \textbf{ Conuiene de saber ¶ Las uirtudes e las sçiençias } e los bienes de fuera . & sunt tria , \textbf{ videlicet , virtutes , scientia , } et bona exteriora . \\\hline
3.2.8 & mas es tirano . \textbf{ Lo segundo para alcançar la fin } que entiende siruen buenos apareiamientos del alma e uirtudes & sed tyrannus . \textbf{ Secundo ad consequendum finem intentum deseruiunt boni habitus et virtutes . } Non enim sufficit finem cognoscere , \\\hline
3.2.8 & que entiende siruen buenos apareiamientos del alma e uirtudes \textbf{ ca non cunple conosçer la fin } e auer el entendimiento alunbrado & Secundo ad consequendum finem intentum deseruiunt boni habitus et virtutes . \textbf{ Non enim sufficit finem cognoscere , } et habere illuminatum intellectum ; \\\hline
3.2.8 & ca non cunple conosçer la fin \textbf{ e auer el entendimiento alunbrado } si non fuere el ome uirtuoso & Non enim sufficit finem cognoscere , \textbf{ et habere illuminatum intellectum ; } nisi sit quis virtuosus , \\\hline
3.2.8 & en tal manera \textbf{ que quiera e pueda alcançar aquella fin . } Et por ende parte nesçe al gouernador del regno de otdenar sus subditosa uirtudes & et habeat ordinatum appetitum , \textbf{ ut velit consequi | finem illum : } spectat igitur ad rectorem regni ordinare suos subditos ad virtutes . Tertio ad consequendum \\\hline
3.2.8 & que quiera e pueda alcançar aquella fin . \textbf{ Et por ende parte nesçe al gouernador del regno de otdenar sus subditosa uirtudes } e a buenas costunbres ¶ & finem illum : \textbf{ spectat igitur ad rectorem regni ordinare suos subditos ad virtutes . Tertio ad consequendum } finem \\\hline
3.2.8 & e a buenas costunbres ¶ \textbf{ Lo terçero para alcançar la fin } que entiende enla uida çiuil & spectat igitur ad rectorem regni ordinare suos subditos ad virtutes . Tertio ad consequendum \textbf{ finem } qui intenditur in vita politica , organice deseruiunt \\\hline
3.2.8 & Et por ende conuiene alos Reyes \textbf{ e alos prinçipes de gouernar } assi las çibdades e los regnos & res exteriores . Decet ergo Reges et Principes sic regere ciuitates \textbf{ et regna , } ut sibi subiecti abundent rebus exterioribus \\\hline
3.2.8 & en quanto ellas siruen a bien beuir \textbf{ e para alcançar la fin } que entienden en la uida çiuil . & ad bene viuere , \textbf{ et ad consequendum finem intentum in vita politica . } Hoc autem quomodo fieri possit , \\\hline
3.2.8 & que entienden en la uida çiuil . \textbf{ Mas esto conmose puede fazer } e en qual manera se deua ordenar la çibdat por que enlła sean falladas todas aquellas cosas & et ad consequendum finem intentum in vita politica . \textbf{ Hoc autem quomodo fieri possit , } et quomodo ordinanda fit ciuitas \\\hline
3.2.8 & Mas esto conmose puede fazer \textbf{ e en qual manera se deua ordenar la çibdat por que enlła sean falladas todas aquellas cosas } que siruen a abastamiento dela uida . & Hoc autem quomodo fieri possit , \textbf{ et quomodo ordinanda fit ciuitas } ut in ea reperiri valeant \\\hline
3.2.8 & de ser acuçioso çerca aquellas cosas \textbf{ por las quales el pueblo puede alcançar su fin } que entiende finca de demostrar & licet per praecedentia sit aliqualiter manifestum , \textbf{ clarius tamen infra dicetur . Viso quod spectat ad Regis officium solicitari circa ea per quae possit populus consequi finem intentum : } restat ostendere , \\\hline
3.2.8 & por las quales el pueblo puede alcançar su fin \textbf{ que entiende finca de demostrar } en qual manera parte nesçe alos Reyes & clarius tamen infra dicetur . Viso quod spectat ad Regis officium solicitari circa ea per quae possit populus consequi finem intentum : \textbf{ restat ostendere , } quomodo spectat ad Reges et Principes \\\hline
3.2.8 & en qual manera parte nesçe alos Reyes \textbf{ et alos prinçipes de tirar todos los enbargos } que enbargan al pueblo & restat ostendere , \textbf{ quomodo spectat ad Reges et Principes } huiusmodi prohibentia remouere . Quae \\\hline
3.2.8 & que enbargan al pueblo \textbf{ de alcançar su fin } los quales enbargos son tres de los quales . ¶ El vno toma nasçençia dela natura & quomodo spectat ad Reges et Principes \textbf{ huiusmodi prohibentia remouere . Quae } etiam tria sunt , \\\hline
3.2.8 & Ca lo omes son en ssi mismos naturalmente corruptibles \textbf{ e por ende por que en ssi mismos non pueden durar } dessean naturalmente de durar en sus fijos . & Nam homines in seipsis sunt naturaliter corruptibiles : \textbf{ inde est ergo quod quia in seipsis durare non possunt , } naturaliter appetunt perpetuari in suis filiis siue sint naturales siue adoptiui . \\\hline
3.2.8 & e por ende por que en ssi mismos non pueden durar \textbf{ dessean naturalmente de durar en sus fijos . } si quier sean naturales siquier por fuados . & inde est ergo quod quia in seipsis durare non possunt , \textbf{ naturaliter appetunt perpetuari in suis filiis siue sint naturales siue adoptiui . } Videtur enim homini \\\hline
3.2.8 & segunt su establesçimientero otro aya hedat en sucession \textbf{ pues que as susi es much se puede turbar el buen estado e la paz dela çibdat } e la fin & secundum suam institutionem alius in haereditatem succedat . \textbf{ Valde ergo turbari potest tranquillus status | et pax ciuitatis } et finis intentus in politica , \\\hline
3.2.8 & en qual manera los fijos ayan la hͣedat de los padres \textbf{ e los postrimeros de los primeros Et pues que assi es tirar vna cosa } que enbarga & si Reges \textbf{ et Principes } non solicitentur qualiter posteriores succedant in haereditatem priorum : remouere igitur unum maxime prohibentium bonam vitam politicam , \\\hline
3.2.8 & much la buena uida çiuil \textbf{ es ordenar bien } en qual manera las hedades & non solicitentur qualiter posteriores succedant in haereditatem priorum : remouere igitur unum maxime prohibentium bonam vitam politicam , \textbf{ est bene ordinare quomodo haereditates decedentium perueniant ad posteros . Secundo , } status tranquillus ciuitatis \\\hline
3.2.8 & en qual manera las hedades \textbf{ de los que mueren vengan alos hederos } que fincan & ø \\\hline
3.2.8 & por la maldat de los çibdadanos . \textbf{ Ca algunos son tan malos que sienpre quieren trauar } e enbargar los otros . & et regni aliquando impeditur ex peruersitate ciuium : \textbf{ sunt enim aliqui adeo peruersi ut semper velint alios molestare . Solicitari ergo debent Reges et Princeps } ne impediatur pax regni \\\hline
3.2.8 & Ca algunos son tan malos que sienpre quieren trauar \textbf{ e enbargar los otros . } Et por ende mucho deuen ser acuçiosos los Reyes & et regni aliquando impeditur ex peruersitate ciuium : \textbf{ sunt enim aliqui adeo peruersi ut semper velint alios molestare . Solicitari ergo debent Reges et Princeps } ne impediatur pax regni \\\hline
3.2.8 & toma nasçençia dela mal querençia de los enemigos \textbf{ por que non seria nada escusar los males de dentro del alma } e los pecados & et corrigantur delinquentes . Tertium huiusmodi impeditiuum sumit originem ex malitia hostium : \textbf{ quasi enim nihil esset vitare interiora discrimina , } nisi prohibentur exteriora pericula . Spectat igitur ad regis officium sic solicitari circa ciuilem potentiam ; \\\hline
3.2.8 & assy acuçioso çerca el poderio çiuil e çerca la sabiduria delas almas \textbf{ por que pueda defender lo suyo } e arredrar la rauia de los enemigos . & nisi prohibentur exteriora pericula . Spectat igitur ad regis officium sic solicitari circa ciuilem potentiam ; \textbf{ et circa industriam armatorum , ut possint hostium rabiem prohibere . } Quomodo autem hoc fieri habeat , \\\hline
3.2.8 & por que pueda defender lo suyo \textbf{ e arredrar la rauia de los enemigos . } Mas en qual manera esto se aya de fazer & nisi prohibentur exteriora pericula . Spectat igitur ad regis officium sic solicitari circa ciuilem potentiam ; \textbf{ et circa industriam armatorum , ut possint hostium rabiem prohibere . } Quomodo autem hoc fieri habeat , \\\hline
3.2.8 & e arredrar la rauia de los enemigos . \textbf{ Mas en qual manera esto se aya de fazer } en la terçera parte deste terçero libro & et circa industriam armatorum , ut possint hostium rabiem prohibere . \textbf{ Quomodo autem hoc fieri habeat , } in tertia parte huius tertii libri , \\\hline
3.2.8 & que le es a comnedado aya aquellas cosas \textbf{ por que pueda alcançar su fin . } Et en qual manera deuen arredr e tirar los enbargos & plenius ostendetur . Ostenso quomodo Reges et Principes solicitari debent , \textbf{ ut populus sibi commissus habeat per quae possit finem consequi , } et quomodo debeant prohibentia remouere : \\\hline
3.2.8 & por que pueda alcançar su fin . \textbf{ Et en qual manera deuen arredr e tirar los enbargos } que enbargan la fin finca & ut populus sibi commissus habeat per quae possit finem consequi , \textbf{ et quomodo debeant prohibentia remouere : } restat ostendere quomodo eos debeant in finem dirigere . \\\hline
3.2.8 & que enbargan la fin finca \textbf{ de demostrar } en qual manera ellos deuengar e enderesçar su pueblo ala fin que entienden . & et quomodo debeant prohibentia remouere : \textbf{ restat ostendere quomodo eos debeant in finem dirigere . } Haec etiam tria sunt . \\\hline
3.2.8 & de demostrar \textbf{ en qual manera ellos deuengar e enderesçar su pueblo ala fin que entienden . } Et para esto ver son menester tres cosas . & et quomodo debeant prohibentia remouere : \textbf{ restat ostendere quomodo eos debeant in finem dirigere . } Haec etiam tria sunt . \\\hline
3.2.8 & en qual manera ellos deuengar e enderesçar su pueblo ala fin que entienden . \textbf{ Et para esto ver son menester tres cosas . } ¶ Ca lo primero las cosas & restat ostendere quomodo eos debeant in finem dirigere . \textbf{ Haec etiam tria sunt . } Nam primo commissa sunt supplenda : \\\hline
3.2.8 & ¶ Ca lo primero las cosas \textbf{ que menguna a buen gouernamiento son de ennader } assi que si viessen & Haec etiam tria sunt . \textbf{ Nam primo commissa sunt supplenda : } ut si viderint aliquid deesse ad bonum regimen ciuitatis , illud est supplendum : \\\hline
3.2.8 & para buen gouernamiento dela çibdat \textbf{ aquello deuen cunplir } e esto se puede fazer & ut si viderint aliquid deesse ad bonum regimen ciuitatis , illud est supplendum : \textbf{ hoc autem fieri potest } per consilium sapientum , \\\hline
3.2.8 & aquello deuen cunplir \textbf{ e esto se puede fazer } por conseio de sabios & ut si viderint aliquid deesse ad bonum regimen ciuitatis , illud est supplendum : \textbf{ hoc autem fieri potest } per consilium sapientum , \\\hline
3.2.8 & Lo segundo los bueons ordenamientos \textbf{ e los bueons establesçimientos son de guardar ¶ } Lo terçero los que bien obran & de quo infra dicetur . Secundo , bonae ordinationes , \textbf{ et bona statuta debent esse obseruanda . Tertio , bene operantes } et maxime facientes \\\hline
3.2.8 & e mayormente los que fazen aquellas cosas \textbf{ por que viene grant bien al comun son de gualardonar } e de les fazer bien . Ca commo quier que las cosas menguadas se cunplan & et maxime facientes \textbf{ ea per quae resultat commune bonum , sunt remunerandi et praemiandi : } nam licet commissa supplere , \\\hline
3.2.8 & por que viene grant bien al comun son de gualardonar \textbf{ e de les fazer bien . Ca commo quier que las cosas menguadas se cunplan } e los bienes ordenados se guarden . & ea per quae resultat commune bonum , sunt remunerandi et praemiandi : \textbf{ nam licet commissa supplere , } et bene ordinata conseruare , \\\hline
3.2.8 & muchualeni para que el pueblo seagniado derechamente en su fin . \textbf{ Enpero mucho uale galardonar las buean sobras } e los que bien fazen . & ut populus recte dirigatur in finem , \textbf{ maxime tamen deseruire videtur ; } bene apta remunerare . \\\hline
3.2.8 & los que lon labios son gualardonados e honrrados . \textbf{ Et pues que assi es auer acuçia çerca las cosas } que son dichas & apud illos sunt sapientes et boni , \textbf{ apud } quos tales remunerantur , \\\hline
3.2.9 & parte nesçe al ofiçio del Rey \textbf{ asnos podemos contar dies cosas } que deue obrar el uerdadero rey & et honorantur . Solicitari igitur circa praedicta nouem , ad Regis officium pertinere videtur . \textbf{ Narrare autem possumus decem quae debet operari bonus Rex , } et quae Tyrannus se facere simulat . \\\hline
3.2.9 & asnos podemos contar dies cosas \textbf{ que deue obrar el uerdadero rey } Et estas diez cosas el tirano se enfinze delas fazer & et honorantur . Solicitari igitur circa praedicta nouem , ad Regis officium pertinere videtur . \textbf{ Narrare autem possumus decem quae debet operari bonus Rex , } et quae Tyrannus se facere simulat . \\\hline
3.2.9 & que deue obrar el uerdadero rey \textbf{ Et estas diez cosas el tirano se enfinze delas fazer } e non las faze & Narrare autem possumus decem quae debet operari bonus Rex , \textbf{ et quae Tyrannus se facere simulat . } Illa enim decem licet aliquo modo in uniuersali contineantur in dictis , \\\hline
3.2.9 & los sermones generales poco proprouechan . \textbf{ por ende sera bien de contar estas diez cosas cada vna } por si¶ & circa morale negocium uniuersales sermones proficiunt minus , \textbf{ ideo bene se habet illa decem narrare per singula . } Est autem primum quod spectat ad verum Regem facere , \\\hline
3.2.9 & Et la primera \textbf{ que parte nesçe de fazer aludadero Reyes } que mucha ya cuydado de procurar e acresçentar los bienes comunes & ideo bene se habet illa decem narrare per singula . \textbf{ Est autem primum quod spectat ad verum Regem facere , } ut maxime procuret bona communia , et regni redditus studeat expendere in bonum commune , \\\hline
3.2.9 & que parte nesçe de fazer aludadero Reyes \textbf{ que mucha ya cuydado de procurar e acresçentar los bienes comunes } e estudie & Est autem primum quod spectat ad verum Regem facere , \textbf{ ut maxime procuret bona communia , et regni redditus studeat expendere in bonum commune , } vel in bonum regni : \\\hline
3.2.9 & e en el bien del regno \textbf{ Mas los tirannos fingen se de fazer estas cosas . } Enpero non las fazen . & vel in bonum regni : \textbf{ tyranni vero hoc simulant facere , } non tamen faciunt : \\\hline
3.2.9 & Lo segundo parte nesçe a derecho gouernador de regño \textbf{ non solamente de ordenar las rentas del regno } e las donaçiones & et aliis personis inutilibus . \textbf{ Secundo spectat ad rectum rectorem regni non solum redditus } et oblationes ordinare in bonum commune regni , \\\hline
3.2.9 & e en el bien del regno . \textbf{ Mas avn parte nesçel much mas de guardar e mantener los bienes comunes et los derechs del regno . } Et maguera que los tiranos se enfingan de fazer estas cosas & sed etiam bona communia \textbf{ et iura regni debet maxime custodire et obseruare . Quod tyranni licet se facere simulant , } non tamen faciunt , \\\hline
3.2.9 & Mas avn parte nesçel much mas de guardar e mantener los bienes comunes et los derechs del regno . \textbf{ Et maguera que los tiranos se enfingan de fazer estas cosas } enpero non las fazen . & sed etiam bona communia \textbf{ et iura regni debet maxime custodire et obseruare . Quod tyranni licet se facere simulant , } non tamen faciunt , \\\hline
3.2.9 & ¶ Lo terçeto conuiene al Rey et al prinçipe \textbf{ de non mostrarsse muy espantable nin muy cruel . } nin le conuiene otrosi de se fazer muy familiar alos omnes & Tertio decet Regem , \textbf{ et Principem non ostendere se nimis terribilem et seuerum , } nec decet se nimis familiarem exhibere , \\\hline
3.2.9 & de non mostrarsse muy espantable nin muy cruel . \textbf{ nin le conuiene otrosi de se fazer muy familiar alos omnes } Mas deue paresçer perssona pesada & et Principem non ostendere se nimis terribilem et seuerum , \textbf{ nec decet se nimis familiarem exhibere , } sed apparere debet persona grauius \\\hline
3.2.9 & nin le conuiene otrosi de se fazer muy familiar alos omnes \textbf{ Mas deue paresçer perssona pesada } e de muy grant reuerençia . & nec decet se nimis familiarem exhibere , \textbf{ sed apparere debet persona grauius } et reuerenda , quod congrue sine virtute fieri non potest : ideo verus Rex vere virtuosus existit : \\\hline
3.2.9 & e de muy grant reuerençia . \textbf{ la qual cosa non se pue de fazer conueniblemente sin uirtud . } Et por ende el uerdadero Rey deue ser uirtuoso & sed apparere debet persona grauius \textbf{ et reuerenda , quod congrue sine virtute fieri non potest : ideo verus Rex vere virtuosus existit : } tyrannus autem non est , \\\hline
3.2.9 & mas quiere pare sçertal . lo quarto parte nesçe a uerdadero Rey \textbf{ non despreciar a ninguon de los subditos } nin fazer tuerto a ninguno & Quarto spectat ad Regem , \textbf{ nullum subditorum contemnere , } nulli iniuriari , \\\hline
3.2.9 & non despreciar a ninguon de los subditos \textbf{ nin fazer tuerto a ninguno } nin en las fij̉as nin en las mugers & nullum subditorum contemnere , \textbf{ nulli iniuriari , } nec in filiabus , \\\hline
3.2.9 & Et si contesçiesse \textbf{ que alguno en el regno feziesse algun mal deuel castigar } e dar pena & nec in aliquibus aliis : \textbf{ et si contingeret aliquem ex regno fore facere , } non propter contumeliam \\\hline
3.2.9 & que alguno en el regno feziesse algun mal deuel castigar \textbf{ e dar pena } non en razon de uaraia & et si contingeret aliquem ex regno fore facere , \textbf{ non propter contumeliam } vel propter aliquam libidinosam voluntatem explendam , \\\hline
3.2.9 & non en razon de uaraia \textbf{ nin por segnir su appetito } nin por conplir su uoluntad . & non propter contumeliam \textbf{ vel propter aliquam libidinosam voluntatem explendam , } sed propter bonum commune et iustitiae ipsum punire debet . \\\hline
3.2.9 & nin por segnir su appetito \textbf{ nin por conplir su uoluntad . } Mas deu el castigar & non propter contumeliam \textbf{ vel propter aliquam libidinosam voluntatem explendam , } sed propter bonum commune et iustitiae ipsum punire debet . \\\hline
3.2.9 & nin por conplir su uoluntad . \textbf{ Mas deu el castigar } por el bien conun & vel propter aliquam libidinosam voluntatem explendam , \textbf{ sed propter bonum commune et iustitiae ipsum punire debet . } Hoc autem tyranni non faciunt , \\\hline
3.2.9 & por el bien conun \textbf{ e por conplir iustiçia . } mas esto non fazen los tiranos . & ø \\\hline
3.2.9 & si non su bien \textbf{ e por auer algo e por conplir sus delecta connes . } por ende fazen muchs tuertosa los çibdadanos tan bien en las mugers & Hoc autem tyranni non faciunt , \textbf{ sed quia ipsi intendunt bonum pecuniosum et delectabile , iniuriantur ciuibus in uxoribus , } et in filiabus , et rapiunt eorum bona . \\\hline
3.2.9 & e alos prinçipes \textbf{ non solamente de auer buenos familiares } e de amar los nobles & ø \\\hline
3.2.9 & non solamente de auer buenos familiares \textbf{ e de amar los nobles } e los ricos omes & Quinto decet Reges et Principes non solum habere familiares , \textbf{ et diligere nobiles , } et barones , \\\hline
3.2.9 & e todos los otros omes . \textbf{ por los quales se puede guardar el buen estado del regno . } Mas avn assi commo dize el philosofo & et alios \textbf{ per quos bonus status regni conseruari potest , } sed etiam \\\hline
3.2.9 & e de los otros \textbf{ por los quales se deue guardar el buen estado vieren } que son menospreçiadas dela muger del prinçipe o del Rey enduzirian a sus maridos & Quare si uxores nobilium \textbf{ et aliorum per quos bonus status regni conseruari habet , } viderent se contemni ab uxore Regis aut Principis , inducerent viros \\\hline
3.2.9 & que assi es . \textbf{ assi se deue auer buen Rey e buen gouernador deregas e de çibdat } Mas el tirano non se ha assi & sic ergo gerere se debet \textbf{ bonus rector regni aut ciuitatis . } Tyrannus autem non sic se habet , \\\hline
3.2.9 & si non el su bien propo \textbf{ assi commo auer dineros o auer delecta connes . } Et por que es muy grant delectaçion sensible enlas uiandas e enlas luyias . & nisi commodum proprium \textbf{ ut bonum pecuniosum | et delectabile , } quia maxima delectatio sensibilis est in cibis \\\hline
3.2.9 & los tiranos sin fre no vsan delas delecta connes \textbf{ e de tomar lo ageno } ante lo que es peor & et venereis , \textbf{ tyranni absque fraeno fruuntur voluptatibus illis . } Immo ( quod peius est ) \\\hline
3.2.9 & e en grandes beueres e en grandes conuides . \textbf{ Enpero non se deuen assi auer . } Ca puesto que ellos non sean tenprados & non \textbf{ tamen sic se habere deberent . Dato enim quod temperamento carerent } ne contemnantur a populis , non deberent suam intemperantiam ostendere : \\\hline
3.2.9 & por qua non sean menospreçiados de los pueblos \textbf{ non deuen mostrar su destenpramiento alos otros . } Ca sienpte es de alabar la mesura e la tenprança . & tamen sic se habere deberent . Dato enim quod temperamento carerent \textbf{ ne contemnantur a populis , non deberent suam intemperantiam ostendere : } laudatur enim sobrietas et temperantia , \\\hline
3.2.9 & non deuen mostrar su destenpramiento alos otros . \textbf{ Ca sienpte es de alabar la mesura e la tenprança . } Et es de deuostar la glotonia e la destenprança e la auariçia¶ & ne contemnantur a populis , non deberent suam intemperantiam ostendere : \textbf{ laudatur enim sobrietas et temperantia , } vituperatur autem auaritia et gulositas . Septimo decet verum Regem ornare \\\hline
3.2.9 & Ca sienpte es de alabar la mesura e la tenprança . \textbf{ Et es de deuostar la glotonia e la destenprança e la auariçia¶ } Lo septimo conuiene al uerdadero Rey de conponer & laudatur enim sobrietas et temperantia , \textbf{ vituperatur autem auaritia et gulositas . Septimo decet verum Regem ornare } et munire ciuitates \\\hline
3.2.9 & Et es de deuostar la glotonia e la destenprança e la auariçia¶ \textbf{ Lo septimo conuiene al uerdadero Rey de conponer } e guarnesçer las çibdades e los castiellos & laudatur enim sobrietas et temperantia , \textbf{ vituperatur autem auaritia et gulositas . Septimo decet verum Regem ornare } et munire ciuitates \\\hline
3.2.9 & Lo septimo conuiene al uerdadero Rey de conponer \textbf{ e guarnesçer las çibdades e los castiellos } que son en el su regno & vituperatur autem auaritia et gulositas . Septimo decet verum Regem ornare \textbf{ et munire ciuitates } et castra existentia in regno , \\\hline
3.2.9 & Lo octauo conuiene al uerdadero . \textbf{ Rey assi commo dize el philosofo de honrrar alos sabios } e alos bue nos & quam tyrannus quaerens utilitatem propriam . Octauo decet \textbf{ verum Regem | ( ut ait Philosophus ) sapientes et bonos , } etiam extraneos adeo honorare , \\\hline
3.2.9 & mas matan los \textbf{ e destierran los ¶ Loye conuiene al Rey uerdadero de non enssanchar su regno } por tomar lo ageno por fuerça e sin iustiçia . & non honorant , \textbf{ sed perimunt . Nono decet verum Regem per usurpationem } et iniustitiam non dilatare suum dominium . \\\hline
3.2.9 & e destierran los ¶ Loye conuiene al Rey uerdadero de non enssanchar su regno \textbf{ por tomar lo ageno por fuerça e sin iustiçia . } Ca assi commo dize el philosofo & non honorant , \textbf{ sed perimunt . Nono decet verum Regem per usurpationem } et iniustitiam non dilatare suum dominium . \\\hline
3.2.9 & en el tercero libro delas politicas \textbf{ mas durable es regnar sobre pocos } que sobre muchos . & et iniustitiam non dilatare suum dominium . \textbf{ Nam ut dicitur Polit’ durabilius est regnare super paucos , } quam super multos . \\\hline
3.2.9 & e lo postrimero conuiene alos uerdaderos \textbf{ Reyes de se auer bien çerca las cosa de dios . } Ca el pueblo segunt que dize el philosofo es del todo subiecto al Rey & sine ratione usurpant . Decimo \textbf{ et ultimo decet veros reges bene se habere circa diuina . } Populus enim ( ut recitat Philosophus ) \\\hline
3.2.9 & e non faze ninguna cosa contuerto . \textbf{ Enpero nos podemos traher otra razon meiora esto diziendo } que si el Rey ha a dios por amigo . & existimat enim talem semper iuste agere , \textbf{ et nihil iniquum exercere . Possumus tamen ad hoc aliam meliorem rationem adducere dicentes } quod si Rex habeat amicum Deum diuina prouidentia \\\hline
3.2.9 & que conuiene ala Real maiestad \textbf{ de se auer bien çerca aquellas cosas } que son de dios & Ultimo autem diximus \textbf{ quod decet regiam maiestatem bene se habere circa diuina : } quia hoc debet esse finis \\\hline
3.2.9 & Mas el tirano non es tal . \textbf{ mas enfinnes e por auentura de paresçer tal . } nchas cautelas tanne el philosofo en el quinto libro delas politicas & secundum veritatem bene se habet erga diuina ; tyrannus vero non talis est , \textbf{ sed simulat se talem esse . } Multas cautelas tangit Philosophus 5 Polit’ ex quibus quantum ad praesens spectat , \\\hline
3.2.10 & delas quales \textbf{ quanto par tenesçe alo presente podemos tomar diez . } por las qualose esfuerça el tiranno & Multas cautelas tangit Philosophus 5 Polit’ ex quibus quantum ad praesens spectat , \textbf{ possumus enumerare decem per quas nititur } tyrannus se in suo dominio praeseruare . \\\hline
3.2.10 & por las qualose esfuerça el tiranno \textbf{ de se mantener en su sennorio . } La primera cautela del tirano es matar los grandes omes e los poderosos . & possumus enumerare decem per quas nititur \textbf{ tyrannus se in suo dominio praeseruare . } Prima cautela tyrannica , \\\hline
3.2.10 & de se mantener en su sennorio . \textbf{ La primera cautela del tirano es matar los grandes omes e los poderosos . } Ca commo el tirano non ame & tyrannus se in suo dominio praeseruare . \textbf{ Prima cautela tyrannica , | est excellentes perimere . } Cum enim tyrannus non diligat \\\hline
3.2.10 & ̃en el su regno \textbf{ non quariendo sofrir sus males leuna tanse contra el . Et el tirano de que conosçe } que el es tirano & insurgunt contra ipsum : \textbf{ tyrannus autem ex quo talem se esse cognoscit , } non cogitat \\\hline
3.2.10 & e malo non pienssa \textbf{ si non commo podra matar los grandes e los nobles . } Ante quando alguons cobdician de fazer tiranias & non cogitat \textbf{ nisi quomodo possit excellentes perimere . } Immo \\\hline
3.2.10 & si non commo podra matar los grandes e los nobles . \textbf{ Ante quando alguons cobdician de fazer tiranias } non solamente matan los grandes & nisi quomodo possit excellentes perimere . \textbf{ Immo | cum aliqui tyrannizare cupiunt , } non solum excellentes alios , \\\hline
3.2.10 & e much menos los sus parientes propreos . \textbf{ por los quales el buen estado del regno se puede guardar mas saluales } e mantienelos en su honrra . & et multo magis cognatos proprios , \textbf{ per quos bonus status regni conseruari potest , } non perimit , \\\hline
3.2.10 & e mantienelos en su honrra . \textbf{ ¶ La segunda cautela de los tiranos es destroyr los sabios . } Ca ueyendo que aquello que fazen es contra razon derech̃tu & sed saluat . \textbf{ Secunda cautela tyrannica , | est sapientes destruere . } Vident enim se contra dictamen rectae rationis agere , \\\hline
3.2.10 & e enduzen el pueblo al amor del Rey . \textbf{ la terçera cautela del tirano ̧eᷤ destroyr la sçiençia e el estudio . } Ca el tirano & populum commouent ad amorem eius . Tertia , \textbf{ est disciplinam | et studium non permittere . Tyrannus enim ( } secundum quod huiusmodi est ) \\\hline
3.2.10 & La quarta cautela del tyrano es \textbf{ non con lentir ningunas conpannias } nin ayuntamientos ser fechs . & meliorari habet . Quarta , \textbf{ est nullas sodalitates , } nec etiam aliquas congregationes permittere . Vult enim tyrannus ciues , \\\hline
3.2.10 & La quinta cautela del tirano \textbf{ es auer muchs assechadores } e escodrinnar & ø \\\hline
3.2.10 & es auer muchs assechadores \textbf{ e escodrinnar } que non se le encubra ninguna cosa delo & ø \\\hline
3.2.10 & que non lon amados del pueblo . \textbf{ por que en muchͣs cosas le aguauian quieren auer muchs assechadores } por que si vieren & Cum enim tyranni sciant se non diligi a populo , \textbf{ eo quod in multis offendant ipsum , | volunt habere exploratores multos , } ut si viderent aliquos ex populo machinari aliquid contra eos , \\\hline
3.2.10 & que alguon ssele una tan contra ellos \textbf{ que los puedan contradezer ante } por esso mismo que los çibdadanos creen & ut si viderent aliquos ex populo machinari aliquid contra eos , \textbf{ possint obuiare illis . } Immo eo ipso quod ciues credunt tyrannum habere exploratores multos , \\\hline
3.2.10 & por esso mismo que los çibdadanos creen \textbf{ que ay muchos assechadores non se osan ayuntar } a ordenar alguna cosa contra el & possint obuiare illis . \textbf{ Immo eo ipso quod ciues credunt tyrannum habere exploratores multos , } non audent congregari \\\hline
3.2.10 & que ay muchos assechadores non se osan ayuntar \textbf{ a ordenar alguna cosa contra el } nin commo se defiendan del . & non audent congregari \textbf{ ut machinentur aliquid contra eum : } nam \\\hline
3.2.10 & Enpero sienpre temen que y entre ellos assechadores e acusadores . \textbf{ Mas el uerdadero Rey non ha cuydado de auer tales assechadores entre los çibdadanos } nin entre los que son en el regno . & et si inter eos sic congregatos nullus exploratorum existeret , \textbf{ semper tamen timerent ibi exploratores esse . Huiusmodi autem exploratores verus Rex habere non curat ad ciues , } et ad eos qui sunt in regno : \\\hline
3.2.10 & e contra los estrannos . \textbf{ Mas si conuiene al Rey auer assechadores en el regno } por otra razon & et ad extraneos . \textbf{ Utrum autem deceat Reges habere exploratores in regno propter aliam causam , } quam ne populus insurgat in ipsum , \\\hline
3.2.10 & contra el adelante se dira . \textbf{ Mas quanto alo presente abasta de saber } que non ha menester el Rey en essa manera assechadores & infra dicetur : \textbf{ ad praesens autem scire sufficiat , } quod non sic Rex eget exploratoribus , \\\hline
3.2.10 & non solamente \textbf{ non conssentir las conpannias e las amistanças . } Mas avn las conpannias e las amistancas & Sexta cautela tyrannica , \textbf{ est non solum non permittere fieri sodalitates et amicitias , } sed etiam amicitias iam factas , et sodalitates turbare , \\\hline
3.2.10 & Mas avn las conpannias e las amistancas \textbf{ que ya son fechos turbar las } e destroyr las . & est non solum non permittere fieri sodalitates et amicitias , \textbf{ sed etiam amicitias iam factas , et sodalitates turbare , } et peruertere . Volunt enim tyranni turbare amicos cum amicis , populum cum insignibus , \\\hline
3.2.10 & que ya son fechos turbar las \textbf{ e destroyr las . } Ca quieren los tyranos turbar los amigos con los amigos & ø \\\hline
3.2.10 & e destroyr las . \textbf{ Ca quieren los tyranos turbar los amigos con los amigos } e el pueblo con los nobles & sed etiam amicitias iam factas , et sodalitates turbare , \textbf{ et peruertere . Volunt enim tyranni turbare amicos cum amicis , populum cum insignibus , } insignes cum seipsis . Vident autem quod quandiu ciues discordant a ciuibus , \\\hline
3.2.10 & e los ricos de los çibdadanos . \textbf{ Entre tanto non puede de ligero contradezer al su mal poderio } por que estonçe cada vna delas partes ha miedo dela otra & et diuites a diuitibus : \textbf{ tamdiu non potest aeque de facili eius potentiae resisti : } nam tunc quaelibet partium timens alteram , \\\hline
3.2.10 & ca non entendrie en el bien comun . \textbf{ La vij ͣ̊ cautela del tirano es fazer los subditos pobres } en tanto que el non aya menester guarda & quia non intenderet commune bonum . Septima , \textbf{ est pauperes facere subditos adeo } ut ipse tyrannus \\\hline
3.2.10 & que los subditos tantosean de pobres \textbf{ que sienpre se ayan de ocupar çerca sus menesteres } en que han de de beuir de cada dia & nulla custodia egeat . Volunt enim tyranni subditos esse intantum pauperes , \textbf{ ut sic occupentur circa cotidiana quibus indigent , } ut non vacet eis aliquid machinari contra ipsos , nec oporteat ipsos habere aliquam custodiam propter illos . \\\hline
3.2.10 & en que han de de beuir de cada dia \textbf{ por que no les uague de fazer ayuntamiento contra ellos } nin los tiranos non ayan menester ninguna guarda & ut sic occupentur circa cotidiana quibus indigent , \textbf{ ut non vacet eis aliquid machinari contra ipsos , nec oporteat ipsos habere aliquam custodiam propter illos . } Verus autem Rex quia intendit bonum subditorum \\\hline
3.2.10 & nin los enpobresçe . \textbf{ Mas ha cuydado de acrescentar sus bienes ¶ La . viij n . cautela del tirano es procurar guerras } e enbiar guerrasa partes estrannas & et depauperat ipsos , \textbf{ sed magis procurat eorum bona . Octaua , | est procurare bella , } mittere bellatores ad partes extraneas , \\\hline
3.2.10 & Mas ha cuydado de acrescentar sus bienes ¶ La . viij n . cautela del tirano es procurar guerras \textbf{ e enbiar guerrasa partes estrannas } e sienpre faze lidiar sus çibdadanos & est procurare bella , \textbf{ mittere bellatores ad partes extraneas , } et semper facere bellare eos \\\hline
3.2.10 & e enbiar guerrasa partes estrannas \textbf{ e sienpre faze lidiar sus çibdadanos } e los que son en el regno & mittere bellatores ad partes extraneas , \textbf{ et semper facere bellare eos } qui sunt in regno : \\\hline
3.2.10 & por razon que ellos en tal manera sean ocupados en las guercas \textbf{ que non les vague de seleunatar contra el tirano . Mas el uerdadero rey non entiende de atormentar los subditos } mouiendo les & quatenus semper circa bellorum onera intenti , \textbf{ non vacet eis aliquid machinari contra tyrannum . | Verus autem Rex non intendit affligere subditos , } suscitando \\\hline
3.2.10 & o por algun guerra derechurera . \textbf{ La nouena caute la del tirano es poner grant guarda en el su cuerpo } non por aquellos que son del regno & vel pro aliquo alio iusto bello . Nona , \textbf{ est custodiam corporis exercere } non per eos \\\hline
3.2.10 & mas si alguon sy ha punna \textbf{ por los destroyr } e por los desfazer . & et partes in regno , \textbf{ sed si quae ibi existunt , } eas amouere desiderat . \\\hline
3.2.10 & por los destroyr \textbf{ e por los desfazer . } o das estas cautelas de los tiranos & sed si quae ibi existunt , \textbf{ eas amouere desiderat . } Omnes cautelas tyrannicas , \\\hline
3.2.11 & para que alguno seleunate o faga algun ayuntamiento \textbf{ contra el tyrano de alguna destas quatro coosas puede sallir ¶ } La primera sale le de sabiduria e de grant entendimiento . & praedicta omnia reducuntur : \textbf{ nam quantum ad praesens spectat , ad hoc quod aliquis inuadat vel machinetur aliquid contra tyrannum ex aliquo praedictorum quatuor videtur procedere . Primo enim potest hoc accidere ex magnanimitate , } ut quia insurgens est tanti cordis \\\hline
3.2.11 & La primera sale le de sabiduria e de grant entendimiento . \textbf{ por que cuyda que podra fallar tantas carreras e tantas maneras } por que pueda matar el tyrano & nam quantum ad praesens spectat , ad hoc quod aliquis inuadat vel machinetur aliquid contra tyrannum ex aliquo praedictorum quatuor videtur procedere . Primo enim potest hoc accidere ex magnanimitate , \textbf{ ut quia insurgens est tanti cordis } ut nihil reputet magnum . \\\hline
3.2.11 & por que cuyda que podra fallar tantas carreras e tantas maneras \textbf{ por que pueda matar el tyrano } o echar lo dessi & ut quia insurgens est tanti cordis \textbf{ ut nihil reputet magnum . } Secundo ex industria et sagacitate , \\\hline
3.2.11 & por que pueda matar el tyrano \textbf{ o echar lo dessi } La segunda paresçe & ut quia insurgens est tanti cordis \textbf{ ut nihil reputet magnum . } Secundo ex industria et sagacitate , \\\hline
3.2.11 & por el poderio çiuil a cometera al tirano . \textbf{ La quarta puede acaesçer por grant uagar } assi que si los çibdadanos fueren muy ricos & propter ciuilem potentiam tyrannum inuadit . \textbf{ Quarto hoc poterit accidere ex nimio ocio , } ut si ciues nimis sint opulenti , \\\hline
3.2.11 & e estidieren en grant uagar \textbf{ que non ayan de obrar nada . } por que la uoluntad del omne non sabe estar de uagar . & et vacent nimio ocio \textbf{ quia mens humana nescit ociosa esse , } dum non occupantur ciues circa cotidianas indigentias , excogitant seditiones , quomodo possint turbare ciuitatem , et insurgere contra rectorem ciuium . \\\hline
3.2.11 & que han pienssan maneras e carreras \textbf{ por las quales podran defender su çibdat } e leunatarse & ø \\\hline
3.2.11 & contra el mal gouernador de los çibdadanos \textbf{ Et pues que assi es contra estas quatto cosas procuran los tyranos de matar los nobles e los grandes } por que los sus subditos non sean osados & dum non occupantur ciues circa cotidianas indigentias , excogitant seditiones , quomodo possint turbare ciuitatem , et insurgere contra rectorem ciuium . \textbf{ Contra haec ergo quatuor procurant tyranni perimere excellentes , } ne sui subditi sunt magnanimi : \\\hline
3.2.11 & nin de grandes coraçones . \textbf{ Otrossi procuran de destroyr los sabios } e enbargar las escuelas e el estudio . & ne sui subditi sunt magnanimi : \textbf{ destruere sapientes , } impedire scholas , \\\hline
3.2.11 & Otrossi procuran de destroyr los sabios \textbf{ e enbargar las escuelas e el estudio . } por que los que fueren en el regno sean nesçios e sin sabiduria & destruere sapientes , \textbf{ impedire scholas , | et studium , } ut existentes in regno sint ignorantes et inscii : \\\hline
3.2.11 & por que los que fueren en el regno sean nesçios e sin sabiduria \textbf{ Otrossi procuran de enbargar } e non consentir las conpannias & et studium , \textbf{ ut existentes in regno sint ignorantes et inscii : } non permittere sodalitates ; \\\hline
3.2.11 & Otrossi procuran de enbargar \textbf{ e non consentir las conpannias } e de turbar los çibdadanos entre ssi & ut existentes in regno sint ignorantes et inscii : \textbf{ non permittere sodalitates ; } turbare ciues \\\hline
3.2.11 & e non consentir las conpannias \textbf{ e de turbar los çibdadanos entre ssi } por que non fiende ssi mismos los vnos de los otros . & non permittere sodalitates ; \textbf{ turbare ciues } inter se ut de se inuicem non confidant : depauperare eos : \\\hline
3.2.11 & por que non fiende ssi mismos los vnos de los otros . \textbf{ Otrossi procuran de los en pobresçer } por que sean sienpre menesterosos & turbare ciues \textbf{ inter se ut de se inuicem non confidant : depauperare eos : } occupare eos in bello , et in aliis exercitiis , \\\hline
3.2.11 & por que sean sienpre menesterosos \textbf{ e delos poner en guerras } e en otros trabaios & inter se ut de se inuicem non confidant : depauperare eos : \textbf{ occupare eos in bello , et in aliis exercitiis , } ut non vacent ocio . Haec ergo sunt cautelae tyrannicae . Verum quia nullus forte est omnino tyrannus , \\\hline
3.2.11 & e en otros trabaios \textbf{ por que non puedan estar de vagar } para penssar del estado del regno . & ø \\\hline
3.2.11 & por que non puedan estar de vagar \textbf{ para penssar del estado del regno . } Et estas son las cautelas de los tiranos . & ø \\\hline
3.2.11 & ca el mal destruye assi melmo \textbf{ e si fuere el mal entrego non se puede sofrir } assi commo dize el philosofo & ut non vacent ocio . Haec ergo sunt cautelae tyrannicae . Verum quia nullus forte est omnino tyrannus , \textbf{ quia malum seipsum destruit , et si integrum sit , importabile fit } ut dicitur 4 Ethicorum , \\\hline
3.2.11 & e mas se arriedra de manera del tirano . \textbf{ Et nos por çierto queremos contar las obras de cada vno } tan bien del Rey commo del tirano & quanto plus accedit ad regnum , \textbf{ et est longius a tyranno . Voluimus quidem utriusque opera enarrare , et utrasque cautelas describere : } quia opposita iuxta se posita magis elucescunt : \\\hline
3.2.11 & tan bien del Rey commo del tirano \textbf{ e declarar las cautelas de cada vno dellos . } por que las cosas contrarias puestas çerca de ssi mismas maclaramente paresçen & ø \\\hline
3.2.11 & e desto paresçe manifiesta miente \textbf{ que la tirama es much de escusar alos Reyes } e mucho han de foyr della & Ex hoc autem manifeste patet , \textbf{ tyrannidem maxime esse fugiendam a regibus : } quia pessimum de se , \\\hline
3.2.11 & que la tirama es much de escusar alos Reyes \textbf{ e mucho han de foyr della } por que es . muy mala . & Ex hoc autem manifeste patet , \textbf{ tyrannidem maxime esse fugiendam a regibus : } quia pessimum de se , \\\hline
3.2.11 & por que la cosa muy mala \textbf{ de ssi misma es much de foyr . } Et la muy buena dessi misma es much de segnir . & quia pessimum de se , \textbf{ est maxime fugiendum , } et optimum maxime prosequendum . \\\hline
3.2.11 & de ssi misma es much de foyr . \textbf{ Et la muy buena dessi misma es much de segnir . } a tanto los sesos de los omes son enclinados a mal & est maxime fugiendum , \textbf{ et optimum maxime prosequendum . } Adeo sensus hominum sunt ad malum proni , \\\hline
3.2.12 & a tanto los sesos de los omes son enclinados a mal \textbf{ que mucho es prouechoso en la carrera de buenans costunbres mostrar } por muchͣs maneras & Adeo sensus hominum sunt ad malum proni , \textbf{ ut valde utile sit in via morum multis viis ostendere , } et multis rationibus probare malum , \\\hline
3.2.12 & por muchͣs maneras \textbf{ e prouar por muchͣs razones } que el mal e el pecado de si es cosa muy uil & ut valde utile sit in via morum multis viis ostendere , \textbf{ et multis rationibus probare malum , } et vitium de se esse quid vile et fugiendum , \\\hline
3.2.12 & que el mal e el pecado de si es cosa muy uil \textbf{ e es cosa de foyr } assi que aquel que la vna manera non enclina a escusar el mal . & et multis rationibus probare malum , \textbf{ et vitium de se esse quid vile et fugiendum , } ut quem una via non inclinat , \\\hline
3.2.12 & e es cosa de foyr \textbf{ assi que aquel que la vna manera non enclina a escusar el mal . } La otra la enduguapara foyr del mal . & et vitium de se esse quid vile et fugiendum , \textbf{ ut quem una via non inclinat , | ut malum vitet , } alia inducat ipsum \\\hline
3.2.12 & assi que aquel que la vna manera non enclina a escusar el mal . \textbf{ La otra la enduguapara foyr del mal . } ca despues que mostramos & ut malum vitet , \textbf{ alia inducat ipsum | ad fugiendum malum . } Postquam ergo ostendimus tyrannidem fugiendam esse a Regibus , \\\hline
3.2.12 & ca despues que mostramos \textbf{ que los Reyes han de foyr much dela tirania . } por que las obras del tirano son muy malas . & ad fugiendum malum . \textbf{ Postquam ergo ostendimus tyrannidem fugiendam esse a Regibus , } eo quod opera tyranni sunt pessima : \\\hline
3.2.12 & por que las obras del tirano son muy malas . \textbf{ que temos prouar } que los Reyes con grand acuçia se deuen guardar & eo quod opera tyranni sunt pessima : \textbf{ probare volumus Reges summa diligentia cauere debere } ne conuertantur in tyrannos , \\\hline
3.2.12 & que temos prouar \textbf{ que los Reyes con grand acuçia se deuen guardar } que non se tornen tiranos & eo quod opera tyranni sunt pessima : \textbf{ probare volumus Reges summa diligentia cauere debere } ne conuertantur in tyrannos , \\\hline
3.2.12 & e es llamado anstrocraçia \textbf{ que quiere dezer señorio de buenos . } Mas si enssennorear en pocos & quia boni et virtuosi , \textbf{ est rectus principatus , } et vocatur aristocratia siue principatus bonorum . Si vero dominentur non quia boni , \\\hline
3.2.12 & que quiere dezer señorio de buenos . \textbf{ Mas si enssennorear en pocos } non por que son buenos & est rectus principatus , \textbf{ et vocatur aristocratia siue principatus bonorum . Si vero dominentur non quia boni , } sed quia diuites , est peruersus et vocatur oligarchia . \\\hline
3.2.12 & mas por que son ricos es llamado obligarçia \textbf{ que quiere dezer señorio tuerto . } Mas quando enssennore a todo el pueblo & et vocatur aristocratia siue principatus bonorum . Si vero dominentur non quia boni , \textbf{ sed quia diuites , est peruersus et vocatur oligarchia . } Sed si dominatur totus populus et intendat bonum omnium tam insignium quam aliorum , est principatus rectus , et vocatur regimen populi . \\\hline
3.2.12 & Mas si el pueblo tiranizare \textbf{ e entienda de abaxar los ricos } es sennorio corrupto & Si vero populus tyrannizet \textbf{ et intendat opprimere diuites , } est principatus corruptus , \\\hline
3.2.12 & e tal sennorio es llamado de mocraçia \textbf{ que tanto quiere dezer commo corrupçion e maldat del pueblo . Et pues que assi es la tirania } e el sennorio corrupto de los ricos . & et vocatur Democratia , \textbf{ quod idem est quod quasi peruersio et corruptio populi . Tyrannis vero corruptus principatus diuitum , } et iniquum dominium populi , sunt regimina peruersa . Tyrannis \\\hline
3.2.12 & assy ca los ricos \textbf{ si cobdiçian de ensseñorear malamente } quanto parte nesçe alo presente tres cosas entienden . & nam diuites \textbf{ si inique dominari cupiant , } quantum , ad praesens spectat , tria intendunt , \\\hline
3.2.12 & quanto parte nesçe alo presente tres cosas entienden . \textbf{ Conuiene a saber riquezas de dineros . } Vv electa connes corporales . & quantum , ad praesens spectat , tria intendunt , \textbf{ videlicet pecuniam , corporales delicias , } et custodiam corporis . \\\hline
3.2.12 & por que es rico \textbf{ por que pueda mas enssennorear } sienpre entse de a acresçentamiento de riquezas e ayuntamiento de dineros . & quia diues , \textbf{ ut magis principari possit , } intendit augmentum diuitiarum , et congregationem pecuniae . \\\hline
3.2.12 & mas por que son ricos . \textbf{ toda su entençion se da apannar dineros } e ayuntar riquezas . & sed quia diuites , \textbf{ tota eorum intentio erit circa pecuniam congregandam . Rursus diuites inique principantes intendunt delicias corporales , et habere voluptates sensibiles . } Nam si quis vult aliquid , \\\hline
3.2.12 & toda su entençion se da apannar dineros \textbf{ e ayuntar riquezas . } Otrossi los ricos & sed quia diuites , \textbf{ tota eorum intentio erit circa pecuniam congregandam . Rursus diuites inique principantes intendunt delicias corporales , et habere voluptates sensibiles . } Nam si quis vult aliquid , \\\hline
3.2.12 & que mal enssenor ean entienden sienpte a delecta connes corporals \textbf{ e auer plazenterias senssibles . } ca si alguon quiere alguna cosa much & vult et ea multo magis , \textbf{ ad quae illud ordinatur : } ut si quis vellet potionem , \\\hline
3.2.12 & assi commo dicho es en bien de honrra \textbf{ mas en delecta conn escarnales . Lo terçero el tirano ha grant acuçia en poner guarda en su cuerpo } assi con mo paresçe por enxientlo . & sed pecuniam . Rursus ut supra dicebatur non intendit bonum honorificum , \textbf{ sed delectabile . Tertio tyrannus maxime delectatur circa custodiam corporis , } eo quod videat se plurimos offendisse . \\\hline
3.2.12 & que nunca mostraua la cara alegte \textbf{ e aquel tirano quariendo dar razon } desto fizo despoiar a su hͣmano & et quare nunquam hylarem vultum ostenderet . \textbf{ Tyrannus ille volens reddere causam quaesiti , } eum expoliari fecit , \\\hline
3.2.12 & e aquel tirano quariendo dar razon \textbf{ desto fizo despoiar a su hͣmano } e fizola tar & Tyrannus ille volens reddere causam quaesiti , \textbf{ eum expoliari fecit , } et ligari : \\\hline
3.2.12 & desto fizo despoiar a su hͣmano \textbf{ e fizola tar } e fizol colgar una espada sobre su cabesça & eum expoliari fecit , \textbf{ et ligari : } et supra caput eius acutissimum gladium pendentem tenuissimo filo apponi fecit : \\\hline
3.2.12 & e fizola tar \textbf{ e fizol colgar una espada sobre su cabesça } muy aguda & et ligari : \textbf{ et supra caput eius acutissimum gladium pendentem tenuissimo filo apponi fecit : } circa ipsum quosdam homines cum ballistis , \\\hline
3.2.12 & e el otro respodio \textbf{ que lo non podia fazer } por muchs peligros qual esta una aprestados . & timens ab acuto gladio vulnerari , et a ballistis perfodi , ait Tyrannus , Gaude frater , \textbf{ hylarem ostende vultum . Respondente illo quod non posset propter imminentia pericula : } inquit Tyrannus quod nec ipse gaudere poterat , \\\hline
3.2.12 & Et dixo el tirano \textbf{ que nin el esso mesmo non se podia gozar } por que estauen mayor peligro & hylarem ostende vultum . Respondente illo quod non posset propter imminentia pericula : \textbf{ inquit Tyrannus quod nec ipse gaudere poterat , } eo quod in maiori periculo existebat , \\\hline
3.2.12 & que sienrte temia la muerte . \textbf{ Et pues que assi es much conuiene al Rey de se guardar } que non se torne en tyrano . & quod semper sibi de morte dubitabat . \textbf{ Valde ergo cauere expedit Regi } ne conuertatur in tyrannum . \\\hline
3.2.12 & quantas han los uerdaderos Reyes ¶ \textbf{ Lo vno por que les conuiene a ellos responder muchͣs cosas superfluas . } lo otro por que alos uerdaderos Reyes mas cosas son dadas & tum \textbf{ quia oportet eos multa expendere superuacue , tum etiam quia veris regibus plus donatur } ex amore \\\hline
3.2.12 & quantas han los uerdaderos los Reyes \textbf{ por que auer amigos } e ser muy amado del pueblo es muy delectable & quot veri reges . \textbf{ Nam habere amicos , } et diligi a populo , \\\hline
3.2.12 & e ser muy amado del pueblo es muy delectable \textbf{ e non fiar de alguno } e creer que es odioso & est maxime delectabile : \textbf{ non confidere de aliquo } et credere se odiosum esse multitudini , \\\hline
3.2.12 & e non fiar de alguno \textbf{ e creer que es odioso } e aborresçido dela muchedunbre del pueblo & non confidere de aliquo \textbf{ et credere se odiosum esse multitudini , } est maxime tristabile . Priuatur ergo tyrannus a maxima delectatione , \\\hline
3.2.12 & que es aborresçido delos pueblos Disto \textbf{ que la tirauja es de esquiuar e de aborresçer } por que es mala & ø \\\hline
3.2.12 & e los malos señorios de los rricos son ayuntados en ella , \textbf{ finca de ver que es de foyr e de aborresçer avn } por que en ella son ayuntados los males del mal priͥnçipado del pueblo & quia iniqui principatus diuitum congregantur in ea : restat \textbf{ videre esse eam cauendam , } eo quod etiam in ipsa congregantur mala peruersi principatus populi . \\\hline
3.2.12 & por que en ella son ayuntados los males del mal priͥnçipado del pueblo \textbf{ por que quando el pueblo enseñorea malamente non entiende guaedar } a njnguno en su estado mas esfuercas & eo quod etiam in ipsa congregantur mala peruersi principatus populi . \textbf{ Cum enim populus principatur peruerse , } non intendit quodlibet seruare in suo statu , \\\hline
3.2.12 & e quanto puede \textbf{ para abaxar los nobles e los altos } E abn esto faze el tirano & non intendit quodlibet seruare in suo statu , \textbf{ sed satagit opprimere nobiles , } et insignes . Hoc etiam tyrannus facit , \\\hline
3.2.12 & ca asi commo paresçe \textbf{ por las cosas sobredh̃as el tiranᷤ con todo su poder se esfuerça en abaxar los grandes } e de matar los nobles e los rricos & et insignes . Hoc etiam tyrannus facit , \textbf{ quia ut patet ex habitis ipse pro viribus nititur opprimere excellentes , } et perimere diuites \\\hline
3.2.12 & por las cosas sobredh̃as el tiranᷤ con todo su poder se esfuerça en abaxar los grandes \textbf{ e de matar los nobles e los rricos } E pues que asi es muncho es de esquar alos rreyes la tiranja & quia ut patet ex habitis ipse pro viribus nititur opprimere excellentes , \textbf{ et perimere diuites | et insignes . } Summe ergo est cauenda tyrannis a Rege , \\\hline
3.2.12 & e de matar los nobles e los rricos \textbf{ E pues que asi es muncho es de esquar alos rreyes la tiranja } enla qual son ayuntados tantos males commo son dichos & et insignes . \textbf{ Summe ergo est cauenda tyrannis a Rege , } in qua tot mala congregantur . \\\hline
3.2.13 & e commo desinarse \textbf{ e arredrarse los rreyes } del gouernemjento derecho sea tiranizar & ut recte et debite gubernent populum sibi commissum : \textbf{ cum deuiare a recto regimine sit tyrannizare , } et iniuriari subditis , \\\hline
3.2.13 & e arredrarse los rreyes \textbf{ del gouernemjento derecho sea tiranizar } e fazer tuerto alos subditos & ut recte et debite gubernent populum sibi commissum : \textbf{ cum deuiare a recto regimine sit tyrannizare , } et iniuriari subditis , \\\hline
3.2.13 & del gouernemjento derecho sea tiranizar \textbf{ e fazer tuerto alos subditos } e non entender al bien comun & cum deuiare a recto regimine sit tyrannizare , \textbf{ et iniuriari subditis , } et non intendere commune bonum ; \\\hline
3.2.13 & e fazer tuerto alos subditos \textbf{ e non entender al bien comun } commo quier que por munchons rrazones mostramos & et iniuriari subditis , \textbf{ et non intendere commune bonum ; } licet pluribus viis ostenderimus per iam dicta detestabile \\\hline
3.2.13 & por lo que dicho ess \textbf{ que cosa peligrosa e aborresçible deue ser ala rreal maiestad tiranzar } e non gouernar derechamente el pueblo & licet pluribus viis ostenderimus per iam dicta detestabile \textbf{ et periculosum esse regiae maiestati tyrannizare , } et non recte gubernare populum : \\\hline
3.2.13 & que cosa peligrosa e aborresçible deue ser ala rreal maiestad tiranzar \textbf{ e non gouernar derechamente el pueblo } aun non tomamos pereza de adosjr nueuas s rrazones & et periculosum esse regiae maiestati tyrannizare , \textbf{ et non recte gubernare populum : } non piget adhuc nouas vias adducere ad ostendendum hoc idem . \\\hline
3.2.13 & e non gouernar derechamente el pueblo \textbf{ aun non tomamos pereza de adosjr nueuas s rrazones } para mostrar esto mismo & et non recte gubernare populum : \textbf{ non piget adhuc nouas vias adducere ad ostendendum hoc idem . } Volumus autem declarare in hoc capitulo \\\hline
3.2.13 & aun non tomamos pereza de adosjr nueuas s rrazones \textbf{ para mostrar esto mismo } mas queremos declarar en este capitulon & et non recte gubernare populum : \textbf{ non piget adhuc nouas vias adducere ad ostendendum hoc idem . } Volumus autem declarare in hoc capitulo \\\hline
3.2.13 & para mostrar esto mismo \textbf{ mas queremos declarar en este capitulon } que por munchans rrazones los sb̃ditos asechan alos tiranos & non piget adhuc nouas vias adducere ad ostendendum hoc idem . \textbf{ Volumus autem declarare in hoc capitulo } quod quia multis de causis subditi insidiantur tyrannis , \\\hline
3.2.13 & por que es muy pelignis ala vida del tirano \textbf{ en arredrarse del gouernamjento derecho } por ende los rreyes deuen estudiar con grand ciudado e con grand acnçia & et quia valde est periculosa vita tyrannica \textbf{ et deuiatio a recto regimine , reges cura peruigili studere debent , } ne delinquentes rectum gubernaculum , tyrannizent . \\\hline
3.2.13 & en arredrarse del gouernamjento derecho \textbf{ por ende los rreyes deuen estudiar con grand ciudado e con grand acnçia } que non tira njz en dexado el gouna mj̊ derecho & et quia valde est periculosa vita tyrannica \textbf{ et deuiatio a recto regimine , reges cura peruigili studere debent , } ne delinquentes rectum gubernaculum , tyrannizent . \\\hline
3.2.13 & e fazense muny buenos onde el prouerbio dize \textbf{ que quien muncho faze foyr al temeroso } por fuerça lo costrange desee oscido en essa misma manera & ø \\\hline
3.2.13 & que rresçibien dehcanatanl cosa \textbf{ e ᷤalos omes desear uengança del mal que rresçibe } Por la qual cosa omero aquel poeta dezia & est iniuria quam passi sunt ab ipso : \textbf{ naturale est enim desiderare vindictam , } propter quod Homerus dicebat , \\\hline
3.2.13 & e por ende \textbf{ e ssi por que es cosa delectable de querer los omes natanl mente vengança munchos asechan al tirano } e por muy gerad saña & quia est appetitus poenae in vindictam : \textbf{ quia ergo sic est delectabile vindictam exposcere , } multi insidiantur tyranno , \\\hline
3.2.13 & que han del algunas vezes la acometen \textbf{ quariendo vengar aquellos tuertos e aquellas miurias } que rresçibieron ¶ Lo tercep o algunos asecha al tirano & multi insidiantur tyranno , \textbf{ et propter vehementem iram aliquando inuadunt ipsum volentes latas iniurias vindicare . Tertio insidiantur aliqui tyranno , } et aliquando perimunt ipsum : \\\hline
3.2.13 & do dice que auemos exenplo en sardina palor en dionjsio \textbf{ ca el rrer̃sur danna palo despçiado } el bien tomun todo se dapla lururia & Exemplum horum recitat Philosophus 5 Politicor’ \textbf{ ubi habemus de Sardinapalo rege , } qui spreto communi bono totum se dedit venereis . Quidam vero dux contemnens eum , \\\hline
3.2.13 & e vn prinçipe despciandol \textbf{ por que suiera vida bescia lacomenol e matol aver en essa is mermana diomsio el tipano } que cupana mas dela garganta & qui spreto communi bono totum se dedit venereis . Quidam vero dux contemnens eum , \textbf{ eo quod vitam pecudum elegisset , inuasit , | et peremit ipsum . Sic etiam Dionysius posterior tyrannizans , } et curans magis de gula quam de bono communi , \\\hline
3.2.13 & e matol¶ \textbf{ Lo quarto esto puede contesçer } por ganar honrra o auer ganançia , & propter despectionem insurrexit in ipsum . \textbf{ Quarto hoc fieri contingit propter honorem } aut propter lucrum adipiscendum . \\\hline
3.2.13 & Lo quarto esto puede contesçer \textbf{ por ganar honrra o auer ganançia , } Ca commo la honrra e la gloria deste mundo sean muy grandes bienes entre los bienes & Quarto hoc fieri contingit propter honorem \textbf{ aut propter lucrum adipiscendum . } Nam cum honor et gloria \\\hline
3.2.13 & si non de su honrra e de sugłia propria \textbf{ e non quiere honrrar los subditos } njn quiere el bien comun & et gloriam propriam , \textbf{ et non honorare subditos , } et non quaerere commune bonum , \\\hline
3.2.13 & njn quiere el bien comun \textbf{ quariendo algunos alcançar la gloriar la honrra } que veen enel tirano acometen ler matanle , & et non quaerere commune bonum , \textbf{ volentes adipisci honorem } et gloriam quam conspiciunt in tyranno , inuadunt eum , \\\hline
3.2.13 & quariendo algunos alcançar la gloriar la honrra \textbf{ que veen enel tirano acometen ler matanle , } Ca essa misma manera avri & volentes adipisci honorem \textbf{ et gloriam quam conspiciunt in tyranno , inuadunt eum , } et perimunt ipsum . Sic etiam quia multi reputant pecuniam esse maximum bonum , \\\hline
3.2.13 & si non a grant ganançia propria \textbf{ e allegar grand auer acor̃tele } por quel tomne los tesoos & videntes tyrarannum non intendere nisi ad lucrum proprium , \textbf{ et ad congregandam pecuniam inuadunt ipsum , } et accipiunt thesauros eius . \\\hline
3.2.13 & segund que dize el philosofo \textbf{ ca conujene de dar a entender } que estos tales non han cuydado de saluar su vida ¶ Lo sexto contesçe & ( ut ait Philos’ ) \textbf{ sunt paucissimi numero , } supponi oportet eos nihil curare , \\\hline
3.2.13 & ca conujene de dar a entender \textbf{ que estos tales non han cuydado de saluar su vida ¶ Lo sexto contesçe } que al gunos a echa alos tiranos & sunt paucissimi numero , \textbf{ supponi oportet eos nihil curare , | ut saluentur . } Sexto contingit aliquos insidiari tyrannis \\\hline
3.2.13 & e por qua non tenjades muerto delos sbditos \textbf{ deue se guardar } com muy grand acuçiaqua non se faga tirano & et ne Rex semper dubitet \textbf{ ne perimatur a subditis : } summa diligentia cauere debet , \\\hline
3.2.13 & los que son en el regno a su amor \textbf{ e tipar le ha toda manera e toda rrazon } por que non le asech & inducit omnes existentes in regno ad amorem eius , \textbf{ et tollit ab eis omnem materiam et causam quare insidientur ipsi . } Licet per multa capitula praecedentia induxerimus Reges et Principes \\\hline
3.2.14 & que le es acomnedado , \textbf{ avn en este cpleo queremos adozjr otras rrazones } para mostrar que si los rrey e cobdiçian de duar muncho el su señorio es toda manera deuen estudiar & sed ut recte regant populum sibi commissum : \textbf{ adhuc in hoc capitulo volumus alias rationes adducere , | ostendentes } quod si reges cupiant suum durare dominium , \\\hline
3.2.14 & avn en este cpleo queremos adozjr otras rrazones \textbf{ para mostrar que si los rrey e cobdiçian de duar muncho el su señorio es toda manera deuen estudiar } por que se no fagantiranos . & ostendentes \textbf{ quod si reges cupiant suum durare dominium , | summo opere studere debent } ne efficiantur tyranni , \\\hline
3.2.14 & tres maneras dela corrupçion dela tiranja e dize que la tiranja corrope de si \textbf{ mismar coronpese desta çirana } e corronpese por El regno ¶ & et regius principatus . Narrat autem Philosophus 5 Politicorum tres modos corruptionis tyrannidis , dicens , \textbf{ Tyrannidem corrumpi a se , a tyrannide alia , } et a regno . Corrumpitur enim tyrannis a seipsa . \\\hline
3.2.14 & asi commo dicho es de suso el mal corronpeasi mi sino \textbf{ e si es mal entero non se puede sofrir . } Pues que asi es la tiranja yor las maldades & ut dicebatur supra ) malum seipsum corrumpit , \textbf{ et si integrum sit importabile sic . } Tyrannis ergo propter peruersitates \\\hline
3.2.14 & enella corronpese de si misma \textbf{ e non puede durar } E por ende dize el pho en el terçero libro delas politicas & corrumpitur \textbf{ et durare non potest . } Ideo dicitur in Politi’ quod tyrannis quanto intensior , \\\hline
3.2.14 & e por ende los Reyes e los prinçipes \textbf{ si quieren durar en su sennorio mucho deuen escusar } que non se arriedren dela iustiçia & et principes \textbf{ si volunt suum durare dominium , | summe cauere debent } ne a iustitia deuiantes in tyrannidem conuertantur : \\\hline
3.2.14 & Et si les contesçiere \textbf{ que en alguna manera ayan de tiranizar deue } por toda su fuerca atenprar la tirania & ne a iustitia deuiantes in tyrannidem conuertantur : \textbf{ et si eos aliquo modo tyrannizare contingat , } suam tyrannidem pro viribus moderare debent , \\\hline
3.2.14 & que en alguna manera ayan de tiranizar deue \textbf{ por toda su fuerca atenprar la tirania } ca quanto mas poco tiranzar en tanto mas dura el su sennorio ¶ & et si eos aliquo modo tyrannizare contingat , \textbf{ suam tyrannidem pro viribus moderare debent , } quia quanto remissius tyrannizabunt , \\\hline
3.2.14 & por toda su fuerca atenprar la tirania \textbf{ ca quanto mas poco tiranzar en tanto mas dura el su sennorio ¶ } Lo segundo la tirania se corronpe & suam tyrannidem pro viribus moderare debent , \textbf{ quia quanto remissius tyrannizabunt , | tanto durabilius principabuntur . } Secundo tyrannis corrumpitur ab alia tyrannide contraria . \\\hline
3.2.14 & quando lança \textbf{ por la qual cosa en el desuiar puede ser contrariedot } assi que lo mucho es cotrario alo poco . & sed contingit multipliciter deuiare ab ipso . \textbf{ Quare in deuiationibus contrarietas esse potest , } ut si multum opponitur pauco , proiectio ultra signum contrariatur proiectioni citra : et proiectio in dextrum proiectioni in sinistrum . \\\hline
3.2.14 & assi que lo mucho es cotrario alo poco . \textbf{ Et por ende lançar sobre la senal } e lançata quande la senal son cosas contrarias & ø \\\hline
3.2.14 & e lançata quande la senal son cosas contrarias \textbf{ e lançar ala mano derechͣ e ala mano desquierda son cosas contrarias . } Et en essa misma manera en este proponimiento & Quare in deuiationibus contrarietas esse potest , \textbf{ ut si multum opponitur pauco , proiectio ultra signum contrariatur proiectioni citra : et proiectio in dextrum proiectioni in sinistrum . } Sic etiam in proposito , \\\hline
3.2.14 & Et por ende vna tirama puede ser contraria a otra \textbf{ e corronper la } assi cotio la tirania del pueblo & Una ergo tyrannis potest contrariari alii , \textbf{ et corrumpere ipsam ; } ut tyrannis populi contrariatur tyrannidi monarchiae : \\\hline
3.2.14 & non podie \textbf{ do sofrir su tira maleunatasse } e tiraniza contrael prinçipe & Cum enim aliquis monarcha vel aliquis unus Princeps tyrannizet in populum , \textbf{ gens illa oppressa non valens sustinere tyrannidem Principis , } insurgit \\\hline
3.2.14 & por que gane el su prinçipado . \textbf{ Por ende mucho deuen escusar los Reyes e los prinçipes } que non tiraniz en commo en tantas maneras & ut obtineat principatum eius . \textbf{ Debent ergo cauere Reges et Principes } ne tyranizent , \\\hline
3.2.14 & que non tiraniz en commo en tantas maneras \textbf{ segunt dicho es se aya de destroyr el prinçipado tiranico . } Lo terçero se desfaze la tirani a non solamente por si misma o por otra tirama contraria . & ne tyranizent , \textbf{ cum tot modis dissoluatur tyrannicus principatus . } Tertio dissoluitur tyrannis non solum propter seipsam \\\hline
3.2.14 & Et pues atantos peligros es puesto el tirano \textbf{ e en tantas manerasse ha de desfazer el su prinçipado } non es bue no de tiranizar & ø \\\hline
3.2.14 & e en tantas manerasse ha de desfazer el su prinçipado \textbf{ non es bue no de tiranizar } mas el regno e el sennorio bueno non es puesto atantos peligros & ( si viderit expedire ) tyranno se opponit . Tot ergo discriminibus oppositus est tyrannus , \textbf{ et tot modis habet dissolui eius principatus . Regium autem dominium non tot periculis exponitur , } nec tot modis habet dissolui . \\\hline
3.2.14 & mas el regno e el sennorio bueno non es puesto atantos peligros \textbf{ nin en tantas maneras se puede desatar commo la tirana . } ca commo quier que el tiranno se esfuerçe & et tot modis habet dissolui eius principatus . Regium autem dominium non tot periculis exponitur , \textbf{ nec tot modis habet dissolui . } Nam licet tyrannus satagat pro viribus verum regem opprimere , \\\hline
3.2.14 & ca commo quier que el tiranno se esfuerçe \textbf{ por todo su poder a destroyr } e aprimiir el uerdadero Rey . & nec tot modis habet dissolui . \textbf{ Nam licet tyrannus satagat pro viribus verum regem opprimere , } nullus tamen verus Rex vero Regi se opponit : \\\hline
3.2.14 & por todo su poder a destroyr \textbf{ e aprimiir el uerdadero Rey . } Enpero ningun uerdadero rey non se pone contra otro uerdadero Rey & Nam licet tyrannus satagat pro viribus verum regem opprimere , \textbf{ nullus tamen verus Rex vero Regi se opponit : } nam nullus bonus et virtuosus insequitur alios bonos et virtuosos , \\\hline
3.2.14 & Et pues que assi es conuiene ala Real magestad \textbf{ de escusar con grant estudio } e con grant acuçia la tirama & et virtuosis se opponeret , deficeret esse bonus et virtuosus . \textbf{ Decet ergo regiam maiestatem summo studio cauere tyrannidem , } ne praedictis periculis exponatur . \\\hline
3.2.15 & e dela çibdat \textbf{ las quals conuiene al Rey de fazer } para que se pueda man tener en lu prinçipado & decem quae politiam saluant , \textbf{ et quae oportet facere Regem } ad hoc ut se in suo principatu praeseruet . \\\hline
3.2.15 & las quals conuiene al Rey de fazer \textbf{ para que se pueda man tener en lu prinçipado } e en lu lennorio ¶ & et quae oportet facere Regem \textbf{ ad hoc ut se in suo principatu praeseruet . } Primo est , \\\hline
3.2.15 & cahe en los grandes . \textbf{ Et pues que assi es deuemos contradezir en los comientos } e deuen avn ser defendidos los males pequanos . & qui minima negligit , \textbf{ paulatim defluit in maiora . Principiis ergo est obstandum , } et inhibendae sunt obliquitates , \\\hline
3.2.15 & e el gouernamiento del regno \textbf{ es bien vsar } e bien fazer & Secundum praeseruans politiam \textbf{ et regnum regium , est bene uti iis } qui sunt in regno , \\\hline
3.2.15 & es bien vsar \textbf{ e bien fazer } aquellos & ø \\\hline
3.2.15 & en el primero libro delas pol . \textbf{ bien vsar de los çibdadanos non solamente guarda la poliçia e el gouernamiento derecho . } Mas avn por esta razon el prinçipado se faze mas durable & et non iniuriando eis . \textbf{ Nam ut innuit Philosophus in Poli’ bene uti ciuibus non solum praeseruat politiam rectam , } sed etiam principatus ex hoc durabilior redditur , \\\hline
3.2.15 & La terçera cosa \textbf{ que guarda al gouernamiento del regno es meter mie do aquellos que son enla çibdat e en el regno } ca las corrupconnes alongadas de fecho e allegadas & dato quod in ipso sit aliquid obliquitatis ad mixtum . Tertium est , \textbf{ incutere timorem iis qui sunt in politia : } nam corruptiones longe \\\hline
3.2.15 & e meior le obedesçen \textbf{ e se temerien de los de auer danno de los enemigos de fuera } por quela guerra de fuera tira las discordias & et plus ei obediunt , \textbf{ si ex timoribus corruptionem timeant . Guerra enim exterius tollit seditiones intrinsecas , } et reddit ciues magis unanimes \\\hline
3.2.15 & Et desto auemos exenplo en los romanos \textbf{ en los quales depues que les fallesçieron las guerras de fuera comneçaron a auer guerra entre ssi . mismos . } Mas esta cautela es propro prouechable en dos prinçipados & Exemplum huius habemus in Romanis , \textbf{ qui postquam defecerunt exteriora bella , | intra seipsos bellare coeperunt . } Est autem haec cautela utilis solum duobus principatibus : \\\hline
3.2.15 & Et en el prinçipado \textbf{ en que alguno comiença nueuamente a regnar . } ca los omes lidiadores e acostunbrados alas batallas & ut principatui constituto ex hominibus assuetis ad bella , \textbf{ et principatui in quo quis nouiter principari coepit . Homines enim bellatores } et assueti ad praelia , \\\hline
3.2.15 & bien assi aquellos que han vso en las armas son obedientes al prinçipe \textbf{ mas los que son oçiosos e baldios non se saben auer conueniblemente } nin saben bien obedesçer al prinçipe . & sed si diu remanet inofficiosum , rubiginem contrahit . Sic et tales bellantes fiunt obedientes Principi : \textbf{ vacantes vero nesciunt debite se habere : } quare talibus ut debite se habeant semper incutiendi sunt timores bellici , \\\hline
3.2.15 & mas los que son oçiosos e baldios non se saben auer conueniblemente \textbf{ nin saben bien obedesçer al prinçipe . } por la qual cosa a estos tales & sed si diu remanet inofficiosum , rubiginem contrahit . Sic et tales bellantes fiunt obedientes Principi : \textbf{ vacantes vero nesciunt debite se habere : } quare talibus ut debite se habeant semper incutiendi sunt timores bellici , \\\hline
3.2.15 & por la qual cosa a estos tales \textbf{ por que se ayan conuenible sienpre les son de poner temores de guerras } e temores de dannos de fuera . & vacantes vero nesciunt debite se habere : \textbf{ quare talibus ut debite se habeant semper incutiendi sunt timores bellici , } et correptiones extrinsecae . \\\hline
3.2.15 & e temores de dannos de fuera . \textbf{ Otrossi si alguno comneçare de nueuo a regnar } por que contra tal prinçipe nueuo de ligero seleuna tan los cibdadanos & et correptiones extrinsecae . \textbf{ Rursus , | si quis super aliquos de nouo principari coepit , } quia contra talem principatum \\\hline
3.2.15 & e por que sean mas obedientes \textbf{ e mas ayuntados al obedesçer } sienpre les deuen poner temor de los peligros de fuera & et \textbf{ ut magis unanimiter obediant , } incutiendi sunt illis timores de extrinsecis periculis imminentibus : \\\hline
3.2.15 & e mas ayuntados al obedesçer \textbf{ sienpre les deuen poner temor de los peligros de fuera } que les pueden acaesçer & ut magis unanimiter obediant , \textbf{ incutiendi sunt illis timores de extrinsecis periculis imminentibus : } sed si regnum diu in statu perstiterit , \\\hline
3.2.15 & sienpre les deuen poner temor de los peligros de fuera \textbf{ que les pueden acaesçer } mas si el regnado fuere en estado de antiguedat & ut magis unanimiter obediant , \textbf{ incutiendi sunt illis timores de extrinsecis periculis imminentibus : } sed si regnum diu in statu perstiterit , \\\hline
3.2.15 & que salua la poliçia \textbf{ es escusar las discordias e las contiendas delos nobles } e esto poniendo les leyes & Quartum autem quod politiam saluare videtur , \textbf{ est cauere seditiones | et contentiones nobilium ; } et hoc ponendo eis leges , \\\hline
3.2.15 & e esto poniendo les leyes \textbf{ ca non pueden de ligero contradezir alas leyes } e son de poner tales leyes en el regno & et hoc ponendo eis leges , \textbf{ quia legibus non de facili contradicitur . } Sunt enim in regno tales leges instituendae , \\\hline
3.2.15 & ca non pueden de ligero contradezir alas leyes \textbf{ e son de poner tales leyes en el regno } que por ellas se puedan tirar las discordias & quia legibus non de facili contradicitur . \textbf{ Sunt enim in regno tales leges instituendae , } ut per eas sedari possint contentiones nobilium . Nam baronibus dissentientibus fiunt seditiones in regno , \\\hline
3.2.15 & e son de poner tales leyes en el regno \textbf{ que por ellas se puedan tirar las discordias } e las tales contiendas de los nobles & quia legibus non de facili contradicitur . \textbf{ Sunt enim in regno tales leges instituendae , } ut per eas sedari possint contentiones nobilium . Nam baronibus dissentientibus fiunt seditiones in regno , \\\hline
3.2.15 & que sea apareiado el principado Real \textbf{ para se destroyr La quinta cosa } que guardan la poliçia es catar con grant acuçia & et per consequens fit praeparamentum \textbf{ ut dissoluatur regius principatus . Quintum , est diligenter aspicere , } quomodo se habeant , \\\hline
3.2.15 & para se destroyr La quinta cosa \textbf{ que guardan la poliçia es catar con grant acuçia } en qual manera se han & ø \\\hline
3.2.15 & e la poliçia \textbf{ commo poner los bueons e los uirtuosos en las dignidades } e dar les los señorios e los prinçipados . & et politiam saluat , \textbf{ sicut praeficere homines bonos | et virtuosos , } et conferre eis dominia et principatus . \\\hline
3.2.15 & commo poner los bueons e los uirtuosos en las dignidades \textbf{ e dar les los señorios e los prinçipados . } por la qual cosalo & et virtuosos , \textbf{ et conferre eis dominia et principatus . } Quare maxime saluatiuum politiae est , \\\hline
3.2.15 & a quales dio los maestradgos e las dignidades . \textbf{ Et si bien se ouieren acresçentar les los sennorios } e si mal tirargelos . & regiam maiestatem considerare diligenter quos praeficit in aliquibus magistratibus : \textbf{ et si bene se habuerint , | augere eorum dominia : } si male , \\\hline
3.2.15 & Et si bien se ouieren acresçentar les los sennorios \textbf{ e si mal tirargelos . } Et en tanto se podrian auer mal & augere eorum dominia : \textbf{ si male , | minuere : } vel adeo male se habere possent , \\\hline
3.2.15 & e si mal tirargelos . \textbf{ Et en tanto se podrian auer mal } que serien de tirar de los sennorios & minuere : \textbf{ vel adeo male se habere possent , } quod essent totaliter a dominio remouendi , \\\hline
3.2.15 & Et en tanto se podrian auer mal \textbf{ que serien de tirar de los sennorios } o a ende condepnar & vel adeo male se habere possent , \textbf{ quod essent totaliter a dominio remouendi , } vel etiam capitali sententia condemnandi . \\\hline
3.2.15 & que serien de tirar de los sennorios \textbf{ o a ende condepnar } e de su unar de muerte ¶ & quod essent totaliter a dominio remouendi , \textbf{ vel etiam capitali sententia condemnandi . } Sextum est , \\\hline
3.2.15 & o a ende condepnar \textbf{ e de su unar de muerte ¶ } La vjͣ cosa & quod essent totaliter a dominio remouendi , \textbf{ vel etiam capitali sententia condemnandi . } Sextum est , \\\hline
3.2.15 & que guarda la poliçia . \textbf{ es non dara ninguer muy grant señorio } ca los grandes sennorios & Sextum est , \textbf{ nulli valde magnum dominium conferre . } Nam magna dominia ex nimia bonitate \\\hline
3.2.15 & e luenga prueua \textbf{ e por ende muches de guardar } que adesora non sea ninguno puesto en muy grant senorio . & de quibus Rex certam et diuturnam experientiam non accepit . \textbf{ Ideo potissime obseruandum est , } ne repente constituatur aliquis in maximo principatu . \\\hline
3.2.15 & aquello que ama \textbf{ ca el temor faze alos omes tomar conseio } e ser sabios & ne aliquod inconueniens accidat circa amatum . Timor autem consiliatiuos facit , \textbf{ ut dicitur 2 Rhet’ ibi enim est magna salus , } ubi \\\hline
3.2.15 & ca si el Rey amare el bien del regno saluat se ha el regno \textbf{ ca temiendo que cotezccan alguas cosas contrarias en el regno aura tomar muchs consseios } en qual manera pueda promouer el bien del regno & bonum regni diligat , saluabitur regnum ; \textbf{ quia timens ne in regno aduersa contingant , } adhibebit multa consilia qualiter possit bona regni promouere , \\\hline
3.2.15 & ca temiendo que cotezccan alguas cosas contrarias en el regno aura tomar muchs consseios \textbf{ en qual manera pueda promouer el bien del regno } e pueda contradezir alos peligros & quia timens ne in regno aduersa contingant , \textbf{ adhibebit multa consilia qualiter possit bona regni promouere , } et periculis imminentibus obuiare . \\\hline
3.2.15 & en qual manera pueda promouer el bien del regno \textbf{ e pueda contradezir alos peligros } que pueden acaesçer¶ La . viijn . & adhibebit multa consilia qualiter possit bona regni promouere , \textbf{ et periculis imminentibus obuiare . } Octauum saluans regnum et politiam , \\\hline
3.2.15 & e pueda contradezir alos peligros \textbf{ que pueden acaesçer¶ La . viijn . } cosa que salua el regno & et periculis imminentibus obuiare . \textbf{ Octauum saluans regnum et politiam , } est habere ciuilem potentiam . \\\hline
3.2.15 & cosa que salua el regno \textbf{ e la poliçia es auer poderio çiuilca } assi commo dize el philosofo en el libro delas grandes costunbres . & Octauum saluans regnum et politiam , \textbf{ est habere ciuilem potentiam . } Nam ( ut dicitur in Magnis moralibus ) \\\hline
3.2.15 & la iustiçia guarda las cortesias et las buenas maneras delas çibdades . \textbf{ Mas la iustiçia non se puede guardar en el regno } sinon dando pena poderio çiuil & iustitia urbanitates conseruat . \textbf{ Sed iustitia in regno conseruari non potest , } nisi per potentiam ciuilem puniantur transgressores iusti . \\\hline
3.2.15 & aquellos que traspassan la iustiçia . ca deue el Rey o el prinçipe \textbf{ si quisiere bien guardar la iustiçia } e si asi ere dar pena alos malos & nisi per potentiam ciuilem puniantur transgressores iusti . \textbf{ Debet enim Rex } aut Princeps si vult seruare iustitiam \\\hline
3.2.15 & si quisiere bien guardar la iustiçia \textbf{ e si asi ere dar pena alos malos } que trasgre en passando la iustiçia & Debet enim Rex \textbf{ aut Princeps si vult seruare iustitiam } et vult punire transgressores iusti , habere multos exploratores , \\\hline
3.2.15 & que trasgre en passando la iustiçia \textbf{ auer much sassechadores } e muchs pesquiridores & aut Princeps si vult seruare iustitiam \textbf{ et vult punire transgressores iusti , habere multos exploratores , } et multos inquisitores inuestigantes facta ciuium , et inquirentes unde ciues accipiunt \\\hline
3.2.15 & aquello que despienden \textbf{ et commo pueden dar razon de su uida } e de comm̃ se mantienen & et multos inquisitores inuestigantes facta ciuium , et inquirentes unde ciues accipiunt \textbf{ quod expendunt , et quomodo possunt reddere rationem sui victus : } nam qui huiusmodi rationem non potest reddere , \\\hline
3.2.15 & e de comm̃ se mantienen \textbf{ ca aquel que non puede dar razon desto } señal & quod expendunt , et quomodo possunt reddere rationem sui victus : \textbf{ nam qui huiusmodi rationem non potest reddere , } signum est quod ex furto \\\hline
3.2.15 & ca assi fazie \textbf{ do podra guardar la iustiçia } e guardar el regno de malefiçios e de los malos & Sic enim faciendo ista , \textbf{ poterit seruare iustitiam , } et praeseruare regnum a maleficis , \\\hline
3.2.15 & do podra guardar la iustiçia \textbf{ e guardar el regno de malefiçios e de los malos } que traspassan la iustiçia . & poterit seruare iustitiam , \textbf{ et praeseruare regnum a maleficis , } et transgressoribus iusti . Nonum maxime saluans regnum , \\\hline
3.2.15 & e es prinçipe \textbf{ e saber quales son aquellas cosas } que la pueden saluar e corronper & secundum quam principatur , \textbf{ et quae eam possunt saluare et corrumpere . Talia autem maxime sciri poterunt per experientiam : } nam \\\hline
3.2.15 & e saber quales son aquellas cosas \textbf{ que la pueden saluar e corronper } Mas si tales cosas se han de saber mucho & secundum quam principatur , \textbf{ et quae eam possunt saluare et corrumpere . Talia autem maxime sciri poterunt per experientiam : } nam \\\hline
3.2.15 & que la pueden saluar e corronper \textbf{ Mas si tales cosas se han de saber mucho } por esperiençia e por prueua & et quae eam possunt saluare et corrumpere . Talia autem maxime sciri poterunt per experientiam : \textbf{ nam } cum quis diu expertus est regni negocia , \\\hline
3.2.15 & por luengo tienpo los negoçios del regno \textbf{ de ligero puede penssar } qual cosa corronpe el buen estado del regno & ø \\\hline
3.2.15 & e qual cosa lo salua . \textbf{ Et pues que assi es conuiene al Rey de penssar mucha menudo } e muchͣs uezes delas cosas que passaron . & cum quis diu expertus est regni negocia , \textbf{ de leui arbitrari poteritque bonum statum regni corrumpunt , et saluant . Decet ergo Regem frequenter meditari et habere memoriam praeteritorum } quae contigerunt in regno , \\\hline
3.2.15 & e muchͣs uezes delas cosas que passaron . \textbf{ Et conuiene le de auer memoria de los fecho passados } que contesçieron en el regno & ø \\\hline
3.2.15 & fue meior guardado el buen estado del regno \textbf{ por que sepa entender } en qual manera deua enssennorear & bonus status regni , \textbf{ ut sciat cognoscere qualiter principari debeat , } et quae corrumpunt \\\hline
3.2.15 & por que sepa entender \textbf{ en qual manera deua enssennorear } e que sepa quales cosas corronpen el regno & bonus status regni , \textbf{ ut sciat cognoscere qualiter principari debeat , } et quae corrumpunt \\\hline
3.2.15 & ca estas cosas non sabidas \textbf{ non se puede derechamente gouernar el regno } rcho es de suso & quia his ignoratis \textbf{ recte regum gubernare non poterit . } Dicebatur supra , quatuor consideranda esse in regimine ciuitatis . \\\hline
3.2.16 & rcho es de suso \textbf{ que quatro cosas son de penssar } en el gouernamiento dela çibdat . & recte regum gubernare non poterit . \textbf{ Dicebatur supra , quatuor consideranda esse in regimine ciuitatis . } videlicet Principem , \\\hline
3.2.16 & en el gouernamiento dela çibdat . \textbf{ Conuiene a saber El prinçipe . } Et el conseio . & Dicebatur supra , quatuor consideranda esse in regimine ciuitatis . \textbf{ videlicet Principem , } consilium , praetorium , et populum . \\\hline
3.2.16 & Otrossi manifestamos \textbf{ quales el ofiçio del Reyr quales cosas le conuiene de fazer } para que derechamente gouierne el pueblo qual es acomendado . & et tyrannidem pessimum ; manifestauimus item quod sit Regis officium , \textbf{ et quae oporteat ipsum facere } ut recte regat populum sibi commissum : \\\hline
3.2.16 & que conuenia al Rey de ser acuçioso \textbf{ e de tomar grant cuydado en ssi para se non tomarentirano . } Et quanto parte nesçe a este negoçio presente & etiam multis viis decere Regem vigilem curam assumere , \textbf{ ne conuertatur in tyrannum : | et tandem , } quantum spectat ad praesens negocium , \\\hline
3.2.16 & conplidamente dixiemos aquellas cosas \textbf{ que eran de dezer } para enformaçion del prinçipe . & quantum spectat ad praesens negocium , \textbf{ sufficienter tractauimus quae circa Principem sunt dicenda . } Restat ergo de consilio pertransire quae tractanda sunt circa ipsum . \\\hline
3.2.16 & para enformaçion del prinçipe . \textbf{ Et pues que assi es fincanos de fablar del consseio } quales cosas son de trattrar çerca el . & sufficienter tractauimus quae circa Principem sunt dicenda . \textbf{ Restat ergo de consilio pertransire quae tractanda sunt circa ipsum . } Sed , \\\hline
3.2.16 & Et pues que assi es fincanos de fablar del consseio \textbf{ quales cosas son de trattrar çerca el . } Mas commo el pho diga en el segundo libro delas ethicas & sufficienter tractauimus quae circa Principem sunt dicenda . \textbf{ Restat ergo de consilio pertransire quae tractanda sunt circa ipsum . } Sed , \\\hline
3.2.16 & e el que ha buen entendimiento toma consseio \textbf{ por ende primero deuemos ver quales cosas son aconsseiables } e en quales deuemos torar consseio s . & et intellectum habens . \textbf{ Ideo primo videndum est quae sunt consiliatiua , } et circa quae debent fieri consilia , \\\hline
3.2.16 & por ende primero deuemos ver quales cosas son aconsseiables \textbf{ e en quales deuemos torar consseio s . } por qua non quaramos tomar conseios & Ideo primo videndum est quae sunt consiliatiua , \textbf{ et circa quae debent fieri consilia , } ne tanquam ignorantes consiliari velimus de quibus non sunt consilia adhibenda . Possumus autem tangere sex , \\\hline
3.2.16 & e en quales deuemos torar consseio s . \textbf{ por qua non quaramos tomar conseios } assi commo nesçios de aquellas cosas & Ideo primo videndum est quae sunt consiliatiua , \textbf{ et circa quae debent fieri consilia , } ne tanquam ignorantes consiliari velimus de quibus non sunt consilia adhibenda . Possumus autem tangere sex , \\\hline
3.2.16 & assi commo nesçios de aquellas cosas \textbf{ de que non deuemos tomar conseios . } Mas quanto alo presente nos podemos poner seys cosas & et circa quae debent fieri consilia , \textbf{ ne tanquam ignorantes consiliari velimus de quibus non sunt consilia adhibenda . Possumus autem tangere sex , } quantum ad praesens spectat , \\\hline
3.2.16 & de que non deuemos tomar conseios . \textbf{ Mas quanto alo presente nos podemos poner seys cosas } que non caen so consseio . & ne tanquam ignorantes consiliari velimus de quibus non sunt consilia adhibenda . Possumus autem tangere sex , \textbf{ quantum ad praesens spectat , } quae sub consilio non cadunt . Primo enim quaecunque sunt immutabilia consilium nostrum subterfugiunt . Nam ideo consiliamur , ut regulemur in actionibus nostris , \\\hline
3.2.16 & que non caen so consseio . \textbf{ la primera es que todas las cosas que non se pueden mudar non caen en nuestro consseio . } ca por ende nos aconseiamos & quantum ad praesens spectat , \textbf{ quae sub consilio non cadunt . Primo enim quaecunque sunt immutabilia consilium nostrum subterfugiunt . Nam ideo consiliamur , ut regulemur in actionibus nostris , } et ut vitemus mala , \\\hline
3.2.16 & Et pues que assi es aquellas cosas \textbf{ que se non pueden escusar } e que non se pueden mudar non caen ssonro conseio & et ut consequamur bona : \textbf{ quae ergo vitari non possunt , } et quae mutationi non subiacent , \\\hline
3.2.16 & que se non pueden escusar \textbf{ e que non se pueden mudar non caen ssonro conseio } e por ende dize elpho & quae ergo vitari non possunt , \textbf{ et quae mutationi non subiacent , | sub consilio non cadunt . } Ideo dicitur 3 Ethi’ \\\hline
3.2.16 & que son duraderas para sienpre \textbf{ e son cosas que se non pueden mudar } ninguon non toma consseio enllas & ø \\\hline
3.2.16 & por que non se ni de conla cuesta del quedrado \textbf{ nin toma consseio de ninguna otra cosa que seño puede mudar ¶ } Lo segundo non caen so consseio aquellas cosas & si est commensurabilis costae , \textbf{ vel de quocunque immutabili . Secundo } etiam consiliabilia non sunt quaecunque mobilia , \\\hline
3.2.16 & que assi se mueuen han nesçessidat en su mouimiento \textbf{ e non se pueden estorçer } assi commo los mouimientos del çielo . & Nam quae semper uniformiter mouentur \textbf{ ( } secundum quod huiusmodi sunt ) quandam necessitatem habent : consiliabilia autem non sunt necessaria , \\\hline
3.2.16 & nin estables sienpre \textbf{ mas son cosas que pueden contesçer } e non contesçer . & ø \\\hline
3.2.16 & mas son cosas que pueden contesçer \textbf{ e non contesçer . } Et por ende dize el philosofo enlas ethicas & secundum quod huiusmodi sunt ) quandam necessitatem habent : consiliabilia autem non sunt necessaria , \textbf{ sed contingentia . Ideo dicitur in Ethic’ } quod de his quae semper sunt in motu , \\\hline
3.2.16 & por que sienpre son en mouimiento \textbf{ e non se pueden mudar los sus mouimientos } por las nuestras obras & tamen quia semper sunt in motu , \textbf{ nec propter nostra opera immutari possunt eorum cursus , } ideo circa talia non est consilium adhibendum . \\\hline
3.2.16 & por las nuestras obras \textbf{ por ende nos deuemos tomar conseio sobre tales cosas } commo estas . & nec propter nostra opera immutari possunt eorum cursus , \textbf{ ideo circa talia non est consilium adhibendum . } Si autem circa talia cadit consilium ; \\\hline
3.2.16 & e otras en el tp̃ofrio mas . Las mouimientos delas estrellas \textbf{ que han a fazer calentura o frio } segunt departidost pons pueden caer en nuestro consseio & fiunt tempore calido , aliqua vero tempore frigido : cursus syderum quae inducere habet \textbf{ secundum diuersa tempora calorem } et frigiditatem , \\\hline
3.2.16 & que han a fazer calentura o frio \textbf{ segunt departidost pons pueden caer en nuestro consseio } non por si & secundum diuersa tempora calorem \textbf{ et frigiditatem , } non per se , \\\hline
3.2.16 & mas por algun acçidente \textbf{ por que sepamos en quetp̃o son de fazer algunas obras } e en que tp̃o orͣ̃s . & sed per accidens potest sub consilia cadere , \textbf{ ut sciamus | quo tempore quae opera sunt fienda . } Tertio non sunt consiliabilia etiam quae fiunt frequenter , \\\hline
3.2.16 & que muchas uezes se fazen en el tro del estiuo \textbf{ non auemos a tomar consseio } por que tales cosas commo estas son naturales & et de caumatibus quae sepe contingunt tempore aestiuali , \textbf{ non habet esse consilium : } quia talia naturalia sunt , \\\hline
3.2.16 & que delas securas \textbf{ e delas luuias non conuiene de tomar conseio ¶ La . iii in . } avn non caen so conseio en aquellas cosas & quod de siccitatibus , \textbf{ et imbribus non est consilium . } Quarto non sunt consiliabilia quae etiam fiunt raro , \\\hline
3.2.16 & e por su conseio \textbf{ e por su entençion auer alguna cosa . } Et pues que assi es aquellas cosas & vult ex electione \textbf{ vel ex intentione aliquid adipisci : } quae ergo praeter intentionem eueniunt , consiliabilia esse non possunt . \\\hline
3.2.16 & que uienen sin entençion del omne \textbf{ non pueden caer en conseio . } e por ende dize el philosofo enl terçero libro delas ethins & vel ex intentione aliquid adipisci : \textbf{ quae ergo praeter intentionem eueniunt , consiliabilia esse non possunt . } Ideo dicitur in Ethic’ \\\hline
3.2.16 & ca non caen so el nuestro conseio las obras de aquellos omes \textbf{ que por nuestros fech̃o non se pueden mudar . Et por ende dize el philosofo } en el terçero libro delas ethicas & etiam omnia humana opera : \textbf{ quia non cadunt sub consilio nostro opera illorum hominum , quae propter nostra facta mutari non possunt . } Ideo dicitur in Ethic’ \\\hline
3.2.16 & assi es los consseios non son cosas \textbf{ que se non puedan mudar } nin son sienpre de vna manera & nec etiam nulli Gallici consiliantur qualiter optime viuant Indii . Consiliabilia ergo non sunt immutabilia , \textbf{ nec uniformia semper , } nec quae sunt a natura , \\\hline
3.2.16 & nin todas las obras delos omes non caen so conseio . \textbf{ Mas solamente tomamos consseio de aquellas cosas que nos podemos obrar . } Ca assi commo dize el philosofo & nec omnia humana cadunt sub consilio , \textbf{ sed solum consiliamur | de iis quae sunt operabilia per nos : } nam ut dicitur 3 Ethicorum singuli autem hominum consiliantur \\\hline
3.2.16 & cada vne de los omes toma conseio de aquellas obras \textbf{ que se puden fazer } por el ¶Lo vi̊ non caen sosico consseio todas aqllas cosas & nam ut dicitur 3 Ethicorum singuli autem hominum consiliantur \textbf{ de iis operabilibus , } quae fieri possunt per ipsos . Sexto non sunt consiliabilia omnia quae per nos fieri possunt . \\\hline
3.2.16 & que por non seiseden ser fechas \textbf{ ca aquellas que nos podetas auer } por nuestras obras & quae fieri possunt per ipsos . Sexto non sunt consiliabilia omnia quae per nos fieri possunt . \textbf{ Nam quaecunque finaliter per opera nostra adipisci intendimus , } sub consilio non cadunt . \\\hline
3.2.16 & mas de aquellas cosas \textbf{ por que podemos alcançar aquella fin . } Ca el fisico & et non consiliari de ipso , \textbf{ sed de iis per quae consequi possumus illud . } Medicus enim quia finaliter intendit sanitatem , \\\hline
3.2.16 & por su fin non toma conseio \textbf{ si deua sanar el doliente } mas esto toma & non consiliatur \textbf{ utrum debeat sanare egrum } sed hoc accipit tanquam certum et notum , \\\hline
3.2.16 & assi commo cosa çierta e conosçida \textbf{ que el enfermo es de sanar } mas toma consseio de aquellas cosas & sed hoc accipit tanquam certum et notum , \textbf{ egrum sanandum esse : et consiliatur de uiis } per quas facilius \\\hline
3.2.16 & mas toma consseio de aquellas cosas \textbf{ por las quales se pueda sanar mas ligeramente e meior . } En essa misma manera el gouernador de la çibdat & egrum sanandum esse : et consiliatur de uiis \textbf{ per quas facilius } et melius sanetur . Sic rector ciuitatis et regni non consiliatur \\\hline
3.2.16 & e del regno non toma consseio \textbf{ si los çibdadanos deuen auer entre ssi paz . } Et si conuiene & et melius sanetur . Sic rector ciuitatis et regni non consiliatur \textbf{ utrum ciues inter se pacem debeant habere , } et utrum regnum oporteat esse in bono statu : \\\hline
3.2.16 & mas esto sopone \textbf{ e toma assi conmocosa çierta e conosçida e ha consseio en qual manera estas cosas se pueden meior fazer . } Et pues que assi es los consseios son de aquellas cosas & et utrum regnum oporteat esse in bono statu : \textbf{ sed haec accipit tanquam certa et nota , | et consiliatur quomodo melius fieri possint . } Sunt ergo consiliabilia quae dependent ex operibus nostris , \\\hline
3.2.17 & mas solamente de aquellas cosas \textbf{ que pueden obrar los omes } ca pueden se fazer questiones & consilium erit quaestio non de quibuscunque , \textbf{ sed solum de agibilibus humanis : } possunt autem circa speculabilia , et circa naturas rerum , et circa aeterna fieri quaestiones multae , \\\hline
3.2.17 & que pueden obrar los omes \textbf{ ca pueden se fazer questiones } en las sciençias especulatiuas & ø \\\hline
3.2.17 & ca es question delas obras \textbf{ que pueden fazer los omes } finca de ver & Viso quid est consilium , \textbf{ quia est quaestio agibilium humanorum : } restat videre qualiter est consiliandum , \\\hline
3.2.17 & que pueden fazer los omes \textbf{ finca de ver } en qual manera es de tomar el conseio & quia est quaestio agibilium humanorum : \textbf{ restat videre qualiter est consiliandum , } et quem modum in consiliis habere debemus . Sunt autem ( quantum ad praesens spectat ) sex obseruanda , \\\hline
3.2.17 & finca de ver \textbf{ en qual manera es de tomar el conseio } e qual manera deuemos tener en los conseios & quia est quaestio agibilium humanorum : \textbf{ restat videre qualiter est consiliandum , } et quem modum in consiliis habere debemus . Sunt autem ( quantum ad praesens spectat ) sex obseruanda , \\\hline
3.2.17 & en qual manera es de tomar el conseio \textbf{ e qual manera deuemos tener en los conseios } Mas quanto pertenesçe alo presente seys cosasson de guardar & restat videre qualiter est consiliandum , \textbf{ et quem modum in consiliis habere debemus . Sunt autem ( quantum ad praesens spectat ) sex obseruanda , } ut sciamus qualiter sit consiliandum . Primum est , \\\hline
3.2.17 & e qual manera deuemos tener en los conseios \textbf{ Mas quanto pertenesçe alo presente seys cosasson de guardar } para que sepamos en qual manera auemos de tomar conseio . & restat videre qualiter est consiliandum , \textbf{ et quem modum in consiliis habere debemus . Sunt autem ( quantum ad praesens spectat ) sex obseruanda , } ut sciamus qualiter sit consiliandum . Primum est , \\\hline
3.2.17 & Mas quanto pertenesçe alo presente seys cosasson de guardar \textbf{ para que sepamos en qual manera auemos de tomar conseio . } La primera cosa es & et quem modum in consiliis habere debemus . Sunt autem ( quantum ad praesens spectat ) sex obseruanda , \textbf{ ut sciamus qualiter sit consiliandum . Primum est , } quia quanto aliquid est magis determinatum , \\\hline
3.2.17 & Ca el esceruano non toma consseio \textbf{ commo esceruir a las letris } si non fuere del todo nesçio & non enim consiliatur scriptor \textbf{ ( nisi sit omnino ignorans ) qualiter debeat scribere litteras , } quia hoc sufficienter determinatum est per artem scribendi . \\\hline
3.2.17 & que non sepa \textbf{ en commo se han de escͥuir las letras . Ca esto conplidamente es demostrado } por el arte del esc̀uir en la gramatica & ( nisi sit omnino ignorans ) qualiter debeat scribere litteras , \textbf{ quia hoc sufficienter determinatum est per artem scribendi . } De his ergo sunt consilia , \\\hline
3.2.17 & en commo se han de escͥuir las letras . Ca esto conplidamente es demostrado \textbf{ por el arte del esc̀uir en la gramatica } Et por ende de aquellas cosas se pueden tomar los consseios & ( nisi sit omnino ignorans ) qualiter debeat scribere litteras , \textbf{ quia hoc sufficienter determinatum est per artem scribendi . } De his ergo sunt consilia , \\\hline
3.2.17 & por el arte del esc̀uir en la gramatica \textbf{ Et por ende de aquellas cosas se pueden tomar los consseios } que pueden ser fechas por nos & quia hoc sufficienter determinatum est per artem scribendi . \textbf{ De his ergo sunt consilia , } quae possunt fieri per nos ; \\\hline
3.2.17 & nin determinadas \textbf{ en qual manera se deue fazer . } Por ende dudamos & ø \\\hline
3.2.17 & Por ende dudamos \textbf{ e tomamos consseio de las Et pues que assi es vna manera es de toraar en los consseios } e es esta . & ideo dubitamus \textbf{ et consiliamur circa ipsa . Est ergo unus modus in consiliis adhibendus , } quod quando aliquod factum regni proponitur , \\\hline
3.2.17 & quanto al fech \textbf{ por mas maneras se puede fazer . } Et quanto menos ha çiertas & quod quando aliquod factum regni proponitur , \textbf{ quanto pluribus modis fieri potest } et quanto minus habet certas et determinatas vias , \\\hline
3.2.17 & e determinadas carreras \textbf{ para se fazer tanto mayor tienpo ha menester omne } para tomar consseio dello . & et quanto minus habet certas et determinatas vias , \textbf{ tanto per plus tempus est consiliandum , } ut de illis viis facilior \\\hline
3.2.17 & para se fazer tanto mayor tienpo ha menester omne \textbf{ para tomar consseio dello . } Porque de aquellas carreras escoia omne la meior & tanto per plus tempus est consiliandum , \textbf{ ut de illis viis facilior } et melior eligatur . Secundo est in consiliis attendendum , \\\hline
3.2.17 & e la mas ligera ¶ \textbf{ La segunda cosa es de guardar en los consseios } que non tome mas conseio & ut de illis viis facilior \textbf{ et melior eligatur . Secundo est in consiliis attendendum , } ut non consiliemur \\\hline
3.2.17 & Ca dicho fue desuso \textbf{ que el temor faze alos omes tomar conseio . } Et por ende paresçe & Dicebatur enim supra , \textbf{ quod timor consiliatiuos facit : } qui ergo consiliatur , \\\hline
3.2.17 & que contezca alguna dos auentura \textbf{ por que non pueda alcançar el bien } que desseao & et dubitare \textbf{ ne aliquo infortunio contingente deficiat a consecutione optati boni , } vel incurrat aliquod damnum , \\\hline
3.2.17 & que luenga algun danno o algun otro mal \textbf{ Et por ende las cosas que son muy pequan ans assi que pueden acarrear muy pequano mal } o enbargar pequano bien non son de poner en consseio . & vel aliquod aliud malum . \textbf{ Quae ergo sunt valde modica , } ut quae sunt apta nata efficere paruum bonum , \\\hline
3.2.17 & Et por ende las cosas que son muy pequan ans assi que pueden acarrear muy pequano mal \textbf{ o enbargar pequano bien non son de poner en consseio . } Et pues que & Quae ergo sunt valde modica , \textbf{ ut quae sunt apta nata efficere paruum bonum , } vel prohibere modicum malum , \\\hline
3.2.17 & Et pues que \textbf{ as si es auemos de tener manera en los conseios . } por que de grandes cosas tomemos consseio . & ut quae sunt apta nata efficere paruum bonum , \textbf{ vel prohibere modicum malum , } non sunt consiliabilia . Est ergo modus attendendus in consiliis , \\\hline
3.2.17 & la tercera cosa es \textbf{ que quando queremos tomar consseio } deuemos tomar connusco otros & non sunt consiliabilia . Est ergo modus attendendus in consiliis , \textbf{ ut de magnis consilietur negotiis . Tertio cum consiliari volumus , debemus alios assumere nobiscum , } inter quos conferamus de negociis fiendis . \\\hline
3.2.17 & que quando queremos tomar consseio \textbf{ deuemos tomar connusco otros } con los quales ayamos acuerdo delas cosas & non sunt consiliabilia . Est ergo modus attendendus in consiliis , \textbf{ ut de magnis consilietur negotiis . Tertio cum consiliari volumus , debemus alios assumere nobiscum , } inter quos conferamus de negociis fiendis . \\\hline
3.2.17 & con los quales ayamos acuerdo delas cosas \textbf{ que auemos de fazer } ca commo quier que el omne entre ssi mismo pueda fallar carreras e maneras & ut de magnis consilietur negotiis . Tertio cum consiliari volumus , debemus alios assumere nobiscum , \textbf{ inter quos conferamus de negociis fiendis . } Nam licet homo inter seipsum possit inuenire vias \\\hline
3.2.17 & que auemos de fazer \textbf{ ca commo quier que el omne entre ssi mismo pueda fallar carreras e maneras } para fazer alguna cosa & inter quos conferamus de negociis fiendis . \textbf{ Nam licet homo inter seipsum possit inuenire vias } et modos ad aliquid peragendum , \\\hline
3.2.17 & ca commo quier que el omne entre ssi mismo pueda fallar carreras e maneras \textbf{ para fazer alguna cosa } enpero non es sabio aquel que se esfuerça en su cabeça sola e menospreçia de oyr las suinas de los otros & Nam licet homo inter seipsum possit inuenire vias \textbf{ et modos ad aliquid peragendum , | attamen imprudens est } qui solo suo capiti innittitur , \\\hline
3.2.17 & para fazer alguna cosa \textbf{ enpero non es sabio aquel que se esfuerça en su cabeça sola e menospreçia de oyr las suinas de los otros } ca de grant sabiduria es en los consseios tener esta manera & attamen imprudens est \textbf{ qui solo suo capiti innittitur , | et renuit aliorum audire sententias . } Magnae enim prudentiae est in consiliis hunc habere modum : \\\hline
3.2.17 & enpero non es sabio aquel que se esfuerça en su cabeça sola e menospreçia de oyr las suinas de los otros \textbf{ ca de grant sabiduria es en los consseios tener esta manera } que con los otros ayamos acuerdo delo que auemos de fazer & et renuit aliorum audire sententias . \textbf{ Magnae enim prudentiae est in consiliis hunc habere modum : } ut cum aliis conferamus quid agendum , \\\hline
3.2.17 & ca de grant sabiduria es en los consseios tener esta manera \textbf{ que con los otros ayamos acuerdo delo que auemos de fazer } la qual cosa paresçe por dos cosas . & Magnae enim prudentiae est in consiliis hunc habere modum : \textbf{ ut cum aliis conferamus quid agendum , } quod ex duobus patet . \\\hline
3.2.17 & assi conmo dicho es deuen ser de grandes cosas \textbf{ e enlas tales cosas ninguno non deue creer } assi mismo del todo & esse debent de rebus magnis . \textbf{ In talibus autem nullus | debet omnino sibi credere , } et proprio sensui inniti : \\\hline
3.2.17 & assi mismo del todo \textbf{ nin se deue esforçar en su seso propreo } mas deue llamar otros a su consseio & debet omnino sibi credere , \textbf{ et proprio sensui inniti : } sed debet alios ad se vocare , \\\hline
3.2.17 & nin se deue esforçar en su seso propreo \textbf{ mas deue llamar otros a su consseio } por que sabe & et proprio sensui inniti : \textbf{ sed debet alios ad se vocare , } sciens \\\hline
3.2.17 & por que sabe \textbf{ que mas cosas puede sabeͬ muchos } que non vno . & sciens \textbf{ quod plura cognoscere possunt multi , } quam unus . Ideo dicitur 3 \\\hline
3.2.17 & desfunzando de nos mismos \textbf{ assi commo si non fuessemos sufiçientes para lo conosçer } Otrossi esto mismo paresçe por otta cosa & quam unus . Ideo dicitur 3 \textbf{ Ethicorum consiliatores assumimus in magna discernentes , nobis ipsis velut non sufficientibus dignoscere . } Rursus hoc idem patet \\\hline
3.2.17 & mas cosas ayan prouadas \textbf{ que vno solo conuiene de llamar otros } para los negoçios . por que por el conseio dellos pueda ser escogida la meior carrera & Quare cum plures plura experti sint , \textbf{ quam unus solus : decet ad huiusmodi negocia alios aduocare , } ut per eorum consilium possit eligi via melior quales autem esse debeant consiliarii aduocandi , \\\hline
3.2.17 & mas quales consseieros de una ser llamados adelante paresçra \textbf{ ¶Lo quarto es de guardar en los consseios } que sean grdados & ut per eorum consilium possit eligi via melior quales autem esse debeant consiliarii aduocandi , \textbf{ in prosequendo patebit . Quarto est in consiliis attendendum , } ut secreta habeantur , \\\hline
3.2.17 & ca muchos negoçios son enbargados \textbf{ por descobrir se los conseios } e por ende por auentura el conseio dende tomo nonbre & Nam multa negocia disturbantur ex reuelatione consiliorum . Inde forte consilium nomen accepit . \textbf{ Dicunt enim aliqui , } quod consilium \\\hline
3.2.17 & ca dizen alguons \textbf{ que conseio tanto quiere dezer commo cosa } en que deuen mucho petissar . & dictum est autem consilium , \textbf{ quia ibi plures simul consedere debent . } Sed forte melius dicere possumus , \\\hline
3.2.17 & que conseio tanto quiere dezer commo cosa \textbf{ en que deuen mucho petissar . } mas por auentura meior & dictum est autem consilium , \textbf{ quia ibi plures simul consedere debent . } Sed forte melius dicere possumus , \\\hline
3.2.17 & mas por auentura meior \textbf{ po demos dezir } que cosseio sea dicha conssilendo & quia ibi plures simul consedere debent . \textbf{ Sed forte melius dicere possumus , } quod consilium dictum sit a Con et Sileo ut illud dicatur esse Consilium , \\\hline
3.2.17 & que cosseio sea dicha conssilendo \textbf{ que quiere tanto dezir commo cosa que se deue callar entre muchs camuches } eston de guardar en los consseios & quod consilium dictum sit a Con et Sileo ut illud dicatur esse Consilium , \textbf{ quod simul aliqui plures silent | et tacent . } Nam maxime est hoc in consiliis attendendum , \\\hline
3.2.17 & que quiere tanto dezir commo cosa que se deue callar entre muchs camuches \textbf{ eston de guardar en los consseios } e mayormente enlos negoçios & et tacent . \textbf{ Nam maxime est hoc in consiliis attendendum , } et maxime in consiliis ubi tractantur negocia communia \\\hline
3.2.17 & e al prouecho del regno \textbf{ por que el prouecho del regno non se pueda enbargar } e por ende deue tener en poridat codas las cosas & solum aspiciat ad communem profectum : \textbf{ et ut regni profectus impediri non possit , secreta tenere debet } quae ibi sunt tradita . \\\hline
3.2.17 & por que el prouecho del regno non se pueda enbargar \textbf{ e por ende deue tener en poridat codas las cosas } que se dixieren en los consseios . & solum aspiciat ad communem profectum : \textbf{ et ut regni profectus impediri non possit , secreta tenere debet } quae ibi sunt tradita . \\\hline
3.2.17 & commo si nunca las ouiessen oydas . \textbf{ Onde ualerio maximo en el segundo libro de los fechs de recontar } en el cabillero delos establesçimientos antigos & adeo secretum erat ac si non audissent illud . \textbf{ Unde Valerius Maximus 2 libro de Factis memorabilibus , } capitulo de Institutis antiquis , \\\hline
3.2.17 & assi lo guarda una \textbf{ ¶Lo quintones de guardar en los consseios } que non fablen y cosas plazenteras mas uerdaderas & sed neminem audisse credere \textbf{ quod tam multorum auribus fuerat commissum . Quinto est in consiliis attendendum , } ut non loquantur ibi placentia , sed vera . \\\hline
3.2.17 & que non fablen y cosas plazenteras mas uerdaderas \textbf{ ca los lisongeros estudiando de fazer } plaza los prinçipes callan la uerdat & ut non loquantur ibi placentia , sed vera . \textbf{ Adulatores enim dum Principi placere student , } vera silentes , \\\hline
3.2.17 & que auie nonbre aristides dizie \textbf{ que los consseieros deuen auer enssi dos cosas . } la vna que non sean manifiestos & vel totum regnum . Inde est ergo \textbf{ quod quidam sapiens Aristides nomine dicebat consiliarios duo in se habere debere , } quod nec essent plani idest manifesti \\\hline
3.2.17 & ¶ La otra que non sean plazenteros \textbf{ assi que parezcan lisongeros auiendo mayor cuydado de fablar cosas plazenteras que uerdaderas . } En essa misma manera abn segunt dize el pho en el terçero libro de la rectorica & et propalatores consiliorum , \textbf{ nec essent placentes , } ut quod essent adulatores , \\\hline
3.2.17 & que los otros en conparaçion del non deuien ser dichos consseieros mas lisongeros . \textbf{ lo . vi̊ . es de guardar en los conseios } que por longadamente tomemos conseio delas cosas & sed magis forte adulatores . \textbf{ Sexto est in consiliis attendendum , } ut diu consiliemur ; \\\hline
3.2.17 & que luego lo pongamosen obra . ca quando viene el tp̃on coueinble \textbf{ para obrar } si derechͣmente queremos obrar & cito in opere exequamur . \textbf{ Nam cum adest opportunitas operandi , } et si recte volumus \\\hline
3.2.17 & para obrar \textbf{ si derechͣmente queremos obrar } e non lo fazemos & Nam cum adest opportunitas operandi , \textbf{ et si recte volumus } et non illud facimus , \\\hline
3.2.17 & por que non sabemos \textbf{ si auemos de fazer aquella cosa } o non ¶ & quia ignoramus \textbf{ an expediat illud fieri . } Bene ergo se habet diligenter quodlibet negocium discutere arduum , \\\hline
3.2.17 & o non ¶ \textbf{ pues que assi es muy bien es de escodrinnar con grant acuçia todo negoçio alto e noble } si es prouechoso delo fazer . & an expediat illud fieri . \textbf{ Bene ergo se habet diligenter quodlibet negocium discutere arduum , } an utile sit illud facere : \\\hline
3.2.17 & pues que assi es muy bien es de escodrinnar con grant acuçia todo negoçio alto e noble \textbf{ si es prouechoso delo fazer . } mas despues que fuere conosçido derechamente & Bene ergo se habet diligenter quodlibet negocium discutere arduum , \textbf{ an utile sit illud facere : } sed post quam per diuturnum consilium est recte cognitum \\\hline
3.2.17 & por el conse io prolongado \textbf{ que es lo que deuemos fazer } si ouieremos tp̃o conueinble & sed post quam per diuturnum consilium est recte cognitum \textbf{ quid fiendum , si adsit operandi facultas , } prompte operari debemus . \\\hline
3.2.17 & si ouieremos tp̃o conueinble \textbf{ e poder para obrar mano a mano } lo deuemos obrar & quid fiendum , si adsit operandi facultas , \textbf{ prompte operari debemus . } Bene ergo dictum est quod scribitur \\\hline
3.2.17 & e poder para obrar mano a mano \textbf{ lo deuemos obrar } Pues que assi es lo que dize el philosofo & quid fiendum , si adsit operandi facultas , \textbf{ prompte operari debemus . } Bene ergo dictum est quod scribitur \\\hline
3.2.17 & e luego \textbf{ e que conuiene de touiar conseio prolongadamente } mas conuiene de fazerl cosas conseiadas mucho ayna . & operamur autem prompte : \textbf{ et quod oportet consiliari tarde , } sed facere consiliata velociter . \\\hline
3.2.17 & e que conuiene de touiar conseio prolongadamente \textbf{ mas conuiene de fazerl cosas conseiadas mucho ayna . } mar ala Real magestado das aquellas cosas & et quod oportet consiliari tarde , \textbf{ sed facere consiliata velociter . } Omnia autem illa quae habere debet bene persuadens \\\hline
3.2.18 & mar ala Real magestado das aquellas cosas \textbf{ que deue auer aquel } que bien amonesta & sed facere consiliata velociter . \textbf{ Omnia autem illa quae habere debet bene persuadens } et bene creditiuus apparenter , \\\hline
3.2.18 & e bien razona \textbf{ e es bien de creer } en el paresçer de los omes & Omnia autem illa quae habere debet bene persuadens \textbf{ et bene creditiuus apparenter , } expedit ut habeat bonos consiliarios existenter . \\\hline
3.2.18 & ca por esto cada vno razon a bien en los consseios \textbf{ e es de creer en los sus dichos } por que cuydan los omes & Nam ex hoc aliquis persuadet \textbf{ in consiliis et creditur dictis eius , } quia existimatur bonus consiliarius esse ad persuadendum . \\\hline
3.2.18 & que es buen consseiero \textbf{ para dar razon de su consseio . } mas para que alguno sea bien de creer & in consiliis et creditur dictis eius , \textbf{ quia existimatur bonus consiliarius esse ad persuadendum . } Sed ad hoc quod aliquis sit bene creditiuus , \\\hline
3.2.18 & para dar razon de su consseio . \textbf{ mas para que alguno sea bien de creer } non conuiene & quia existimatur bonus consiliarius esse ad persuadendum . \textbf{ Sed ad hoc quod aliquis sit bene creditiuus , } non oportet ipsum esse existenter talem , \\\hline
3.2.18 & e es dado el ome \textbf{ por de creer } si cuydan los omes & et ex apparentibus . Ideo sit sufficienter homini fides , \textbf{ et redditur ei aliquis creditiuus , } si existimet illum bonum consiliatorem esse . \\\hline
3.2.18 & que ha el que bien razona en paresçençia \textbf{ todas deue auer en fech el buen consseiero . } por la qual cosa & habet apparenter , \textbf{ bonus consiliator existenter habere debet . } Quare si scire volumus quales consiliarios habere deceat regiam maiestatem , \\\hline
3.2.18 & por la qual cosa \textbf{ si queremos saber } quales consseieros deue auer la real magestad e quales e quantas cosas son menester & bonus consiliator existenter habere debet . \textbf{ Quare si scire volumus quales consiliarios habere deceat regiam maiestatem , } et quae et quot sunt in consiliis requirenda : \\\hline
3.2.18 & si queremos saber \textbf{ quales consseieros deue auer la real magestad e quales e quantas cosas son menester } en los consseios conuiene de saber & bonus consiliator existenter habere debet . \textbf{ Quare si scire volumus quales consiliarios habere deceat regiam maiestatem , } et quae et quot sunt in consiliis requirenda : \\\hline
3.2.18 & quales consseieros deue auer la real magestad e quales e quantas cosas son menester \textbf{ en los consseios conuiene de saber } en quantas maneras los omes son de endozir por razones & Quare si scire volumus quales consiliarios habere deceat regiam maiestatem , \textbf{ et quae et quot sunt in consiliis requirenda : } scire expedit quot modis persuadetur hominibus , \\\hline
3.2.18 & en los consseios conuiene de saber \textbf{ en quantas maneras los omes son de endozir por razones } e en quantas maneras les ha omne de fazer fe . & et quae et quot sunt in consiliis requirenda : \textbf{ scire expedit quot modis persuadetur hominibus , } vel quot modis fit eis fides \\\hline
3.2.18 & en quantas maneras los omes son de endozir por razones \textbf{ e en quantas maneras les ha omne de fazer fe . } e en quantas maneras son los omes enclinados & scire expedit quot modis persuadetur hominibus , \textbf{ vel quot modis fit eis fides } et inclinantur ad credendum sermones auditos . \\\hline
3.2.18 & e en quantas maneras son los omes enclinados \textbf{ e se enclinan a creer las palabras que oyen } e estas son tres cosas conuiene saber . & vel quot modis fit eis fides \textbf{ et inclinantur ad credendum sermones auditos . } Haec autem sunt tria , \\\hline
3.2.18 & e se enclinan a creer las palabras que oyen \textbf{ e estas son tres cosas conuiene saber . } ¶ el dezidor que fabla & et inclinantur ad credendum sermones auditos . \textbf{ Haec autem sunt tria , } secundum quod in omni locutione tria sunt consideranda , \\\hline
3.2.18 & e enclina alos oydores \textbf{ a creer las palabras que oyen puede ser de parte de aquel que las dize } e esto contesçe & quod persuadetur auditoribus , \textbf{ et inclinantur ad credendum sermones auditos ex parte dicentis , } quod contingit , \\\hline
3.2.18 & ca los buenos omes \textbf{ avn que non sepan dar razones } delo que dizen creen les los omes . & vel credatur bonus . \textbf{ Nam bonis hominibus etiam si nullas rationes assignare sciant , creditur eis , } et faciunt fidem auditoribus : \\\hline
3.2.18 & que es bueno \textbf{ e los bueons non quieren mentir } de ligero creen los omes asodichos . & nam quia dicens creditur esse bonus , \textbf{ cum tales mentiri nolint , } de facili creditur eorum dictis . Secundo potest fieri credulitas auditoribus ex parte ipsorum auditorum : \\\hline
3.2.18 & de ligero creen los omes asodichos . \textbf{ lo segundo se puede fazer creençia alos oydores de parte de los dizidores la qual cosa contesçe } si el que dize las palabras fuereentre los oydores bien quarido & cum tales mentiri nolint , \textbf{ de facili creditur eorum dictis . Secundo potest fieri credulitas auditoribus ex parte ipsorum auditorum : } quod contingit \\\hline
3.2.18 & de quanto son enlos negoçios \textbf{ que han de fazer comunalmente } creen los omes & quam valeant , \textbf{ et esse magis sapientes quam sint , } et in negotiis fiendis communiter credunt homines plus videre quam videant , \\\hline
3.2.18 & los quales oydores \textbf{ dessi son passionados e enclinados a creer aquellos que cuydan } que son sus amigos ¶ & qui ex se passionantur , \textbf{ et inclinantur | ut fidem adhibeant eis , } quos putant amicos esse . \\\hline
3.2.18 & ca del sabio es de sabra \textbf{ e de conosçer } daquellas cosas & Nam prudentis est , \textbf{ scire et cognoscere ipsas res , } et ipsa negocia agibilia : \\\hline
3.2.18 & de que fabla el sabio \textbf{ por que las sabe conosçer } e iudgar uiene la creençia & de quibus loquitur prudens , \textbf{ quia ea scit cognoscere et iudicare , } facit fidem auditoribus . \\\hline
3.2.18 & por que las sabe conosçer \textbf{ e iudgar uiene la creençia } e faze fe el sabio alos oydores . & de quibus loquitur prudens , \textbf{ quia ea scit cognoscere et iudicare , } facit fidem auditoribus . \\\hline
3.2.18 & et este amonestamiento es por si . \textbf{ Ca fazerse el omne digno de creer } e buen amonestador e razonador por si . & et haec persuasio est per se : \textbf{ nam reddere se credibilem } et bene persuadere per se , \\\hline
3.2.18 & de que fabla el amonestador \textbf{ e el razonadorca sabe tomar razones e argumentos } por los quales faga fe alos oydores & et ex ipsis negotiis de quibus loquitur \textbf{ scire assumere rationes et argumenta , } per quae fides fiat audientibus . \\\hline
3.2.18 & por los quales faga fe alos oydores \textbf{ e porque los sabios saben esto fazer } e aquellos que son tenidos por sabios son contados & per quae fides fiat audientibus . \textbf{ Et quia prudentes sciunt facere , } et qui existimantur prudentes , existimantur talia facere : ideo ad hoc quod aliquis ex rebus de quibus loquitur fidem faciat , vel oportet quod sit prudens \\\hline
3.2.18 & e aquellos que son tenidos por sabios son contados \textbf{ para fazer tales cosas Morende } para que alguno faga fe delas cosas & Et quia prudentes sciunt facere , \textbf{ et qui existimantur prudentes , existimantur talia facere : ideo ad hoc quod aliquis ex rebus de quibus loquitur fidem faciat , vel oportet quod sit prudens } vel quod credatur esse prudens . \\\hline
3.2.18 & que el que es buen amonestador e razonador \textbf{ e aquel a que los omes creen deue auer en el } e paresçer todas aquellas cosas & Itaque cum dictum sit quod qui bene persuadens , \textbf{ et ille cui fides adhibetur , } debet habere apparenter , \\\hline
3.2.18 & e aquel a que los omes creen deue auer en el \textbf{ e paresçer todas aquellas cosas } que ha todo buen conseiero en ssi de fecho . & et ille cui fides adhibetur , \textbf{ debet habere apparenter , } oportet quod bonus consiliator habeat existenter : \\\hline
3.2.18 & Et por ende assaz parelçe \textbf{ quales conseieros deue auer el rey } ca deue tomar tales que sean buenos e sean sabios & oportet quod bonus consiliator habeat existenter : \textbf{ satis apparet quales consiliatores deceat quaerere regiam maiestatem ; } quia debet quaerere tales qui sint boni , \\\hline
3.2.18 & quales conseieros deue auer el rey \textbf{ ca deue tomar tales que sean buenos e sean sabios } e sean amigos . & satis apparet quales consiliatores deceat quaerere regiam maiestatem ; \textbf{ quia debet quaerere tales qui sint boni , } et amici , et sapientes . \\\hline
3.2.18 & ca alos buenos pesa todo mal \textbf{ e toda cosa de denostar . } mas la mentira & quia bonis displicet omne malum , \textbf{ et omne detestabile : } mendacium autem ut dicitur 4 Ethic’ \\\hline
3.2.18 & en el quato libro delas etihͣses por si mala cosa \textbf{ e es de denostar ¶ } Lo segundo los consseieros & per se est malum \textbf{ et detestabile . } Secundo consiliarii debent esse non solum boni sed amici , \\\hline
3.2.18 & en quanto dan conseio \textbf{ as a vn conuiene les de non mentir } por razon de aquel & qui loquuntur \textbf{ et qui consilium praebent , } sed etiam ratione eius \\\hline
3.2.18 & a quien fablan o a quien dan conseio \textbf{ Ca de los amigos es bien conseiar a sus amigos } e de dar les bueons & sed etiam ratione eius \textbf{ ad quem loquuntur } et cui consilium praebent : \\\hline
3.2.18 & Ca de los amigos es bien conseiar a sus amigos \textbf{ e de dar les bueons } e uerdaderos conseios & ad quem loquuntur \textbf{ et cui consilium praebent : } quia amicorum est amicis vera et bona consulere . Tertio consiliarii debent esse sapientes , \\\hline
3.2.18 & de que fablan . ca conosçen los negoçios \textbf{ que han de fazer } e saben en qual manera los han de fazer . & quia non mentientur ex parte rerum de quibus loquuntur ; \textbf{ quia cognoscent negocia agibilia , } et scient qualiter sit agendum . Haec ergo tria quaerenda sunt in consiliariis : \\\hline
3.2.18 & que han de fazer \textbf{ e saben en qual manera los han de fazer . } Et pues que assi es estas tres cosas son menester en los consseieros & quia cognoscent negocia agibilia , \textbf{ et scient qualiter sit agendum . Haec ergo tria quaerenda sunt in consiliariis : } videlicet , bonitas , amicitia , et sapientia . Nam , \\\hline
3.2.19 & por lo que dicho es deuen los consseieros ser buenos e amigos e sabios \textbf{ les cosas son de dar e de tomar los conssei penssaremos } en los diioschos del philosofo en el primero libro & debent esse boni , et amici , et sapientes . \textbf{ Si consideretur dicta Philosophi 1 Rhet’ quinque sunt } de quibus consiliantur homines : \\\hline
3.2.19 & en los diioschos del philosofo en el primero libro \textbf{ de delas quales los omes han de dar } e de tomar la rectorica & Si consideretur dicta Philosophi 1 Rhet’ quinque sunt \textbf{ de quibus consiliantur homines : } videlicet , de prouentibus , \\\hline
3.2.19 & de delas quales los omes han de dar \textbf{ e de tomar la rectorica } çinco cosas son consseios & de quibus consiliantur homines : \textbf{ videlicet , de prouentibus , } de alimento , \\\hline
3.2.19 & Lo quarto dela paz e dela guerra ¶ \textbf{ Lo quanto commo se han de poner las leyes } e commo se han de guardar . & et de pace et bello , \textbf{ et de legislatore : } circa haec ergo quinque oportet consiliatores esse instructos . \\\hline
3.2.19 & Lo quanto commo se han de poner las leyes \textbf{ e commo se han de guardar . } Et en estas çinco cosas pueden ser enformados & et de pace et bello , \textbf{ et de legislatore : } circa haec ergo quinque oportet consiliatores esse instructos . \\\hline
3.2.19 & que el conseio del Rey sea cerca las sus rentas \textbf{ en la qual cosa dos cosas conuiene de penssar } ¶Lo primero couiene que el Rey non tome ningunas rentas & consilium circa prouentus , \textbf{ in quo duo sunt attendenda . Primo , } ne maiestas regia aliquos prouentus iniuste usurpet a suis conciuibus : \\\hline
3.2.19 & sin derech \textbf{ lo segundo ha de tener mient̃s el Rey de non ser engannado enlas sus rentas . } ca conuiene que el conseio del Rey sea bue no & Rursus est attendendum , \textbf{ ne in suis prouentibus defraudetur : } expedit enim regium consilium pro viribus saluare iura Regis , \\\hline
3.2.19 & ca conuiene que el conseio del Rey sea bue no \textbf{ para saluar } por todo su ponder los derechs del Rey . & ne in suis prouentibus defraudetur : \textbf{ expedit enim regium consilium pro viribus saluare iura Regis , } eo quod huiusmodi bona ordinanda sunt ad bonum commune , \\\hline
3.2.19 & para saluar \textbf{ por todo su ponder los derechs del Rey . } por que tales biens deuen ser ordenados & ne in suis prouentibus defraudetur : \textbf{ expedit enim regium consilium pro viribus saluare iura Regis , } eo quod huiusmodi bona ordinanda sunt ad bonum commune , \\\hline
3.2.19 & Et conuiene que sepan las rentas del regno \textbf{ las que han de venir al Rey quales e quantas son } por que si alguͣ cosa es superflua & qui et quanti sunt : \textbf{ quatenus } si quis est superfluus \\\hline
3.2.19 & para mantenençia dela uida corporal . \textbf{ Ca deue penssar el prinçipe } e los consseieros & ø \\\hline
3.2.19 & quanto uianda ha en el regno o en cada vna çibdat del regno \textbf{ e quantas cosas se pueden y aduzir } e por que manera se pueden aduzir & considerandum est enim quantum alimentum est in regno vel in ciuitate qualibet ipsius regni , \textbf{ et quantum est ibi adductibile , } et quorum inductione adduci valeat , \\\hline
3.2.19 & e quantas cosas se pueden y aduzir \textbf{ e por que manera se pueden aduzir } ca por que cerca tales cosas sean tomados conseios conueibles & et quantum est ibi adductibile , \textbf{ et quorum inductione adduci valeat , } ut circa haec debita consilia \\\hline
3.2.19 & e puestas ordena çonns \textbf{ quals deuen o quales se pueden fazer } ca non es pequana cosa de auer consseio er la mantenençia & ut circa haec debita consilia \textbf{ et debitae ordinationes fieri possint : } non enim modicum consiliandum est circa alimentum , \\\hline
3.2.19 & quals deuen o quales se pueden fazer \textbf{ ca non es pequana cosa de auer consseio er la mantenençia } e enlas uiandas & et debitae ordinationes fieri possint : \textbf{ non enim modicum consiliandum est circa alimentum , } ut quaelibet ciuitas habeat sufficientia ad vitam , \\\hline
3.2.19 & Et en estas cosas \textbf{ que parte nesçen para la uida deuense fazer mudaçiones e canbios conuenibles } assi que aya y vendidas e conpras conuenibles & ut in huiusmodi sufficientibus \textbf{ ad vitam fieri debent debitae commutationes , ut debitae emptiones , } et venditiones . Considerandae enim sunt mensurae \\\hline
3.2.19 & assi que aya y vendidas e conpras conuenibles \textbf{ Et por ende son de penssar las meluras } e los pesos de los vendedores e de los conpradores . & ad vitam fieri debent debitae commutationes , ut debitae emptiones , \textbf{ et venditiones . Considerandae enim sunt mensurae } et pondera vendentium : \\\hline
3.2.19 & e los pesos de los vendedores e de los conpradores . \textbf{ Et quando fuere menester tassar } e poner presçio alas cosas & et cum expedit taxandum est pretium venditionis , \textbf{ si ( ultra quam debent ) } venditores res suas vendere vellent . Tertio , \\\hline
3.2.19 & Et quando fuere menester tassar \textbf{ e poner presçio alas cosas } que se venden & et cum expedit taxandum est pretium venditionis , \textbf{ si ( ultra quam debent ) } venditores res suas vendere vellent . Tertio , \\\hline
3.2.19 & que se venden \textbf{ si los vendedores quisieren vender las cosas } mas de quando deuen ¶ & si ( ultra quam debent ) \textbf{ venditores res suas vendere vellent . Tertio , } est consilium adhibendum circa custodiam ciuitatis et regni , \\\hline
3.2.19 & mas de quando deuen ¶ \textbf{ Lo terçero el conseio es de tomar çerca la guarda dela çibdat } e del regno & venditores res suas vendere vellent . Tertio , \textbf{ est consilium adhibendum circa custodiam ciuitatis et regni , } quod dupliciter fieri habet : \\\hline
3.2.19 & e del regno \textbf{ la qual cosa se puede fazer en dos maueras . } ca lo primero es de poner guarda & est consilium adhibendum circa custodiam ciuitatis et regni , \textbf{ quod dupliciter fieri habet : } nam primo est custodia adhibenda \\\hline
3.2.19 & la qual cosa se puede fazer en dos maueras . \textbf{ ca lo primero es de poner guarda } por que se non le una ten discordias nin se fagan malefiçios entre los çibdadanos . & quod dupliciter fieri habet : \textbf{ nam primo est custodia adhibenda } ne insurgant seditiones et malitia \\\hline
3.2.19 & por que se non le una ten discordias nin se fagan malefiçios entre los çibdadanos . \textbf{ por ende deuemos cuydar con grant acuçia } quales delos çibdadanos son tenidos por bueons & ne insurgant seditiones et malitia \textbf{ inter ciues . Ideo attendendum est diligenter } qui ciuium reputantur boni , \\\hline
3.2.19 & o muertos . \textbf{ ca los Reyes e los prinçipes non deuen sofrir } que los malfechores bi una . & vel etiam totaliter extirpentur , \textbf{ quia Reges et Principes non debent pati maleficos viuere . Sunt } etiam consideranda loca in quibus consueuerunt magis maleficia perpetrari : \\\hline
3.2.19 & que los malfechores bi una . \textbf{ Avn son de penssar los logares } en los quales se suelen fazer malos malefiçios & quia Reges et Principes non debent pati maleficos viuere . Sunt \textbf{ etiam consideranda loca in quibus consueuerunt magis maleficia perpetrari : } nam sicut quidam hominum magis iniustificant \\\hline
3.2.19 & Avn son de penssar los logares \textbf{ en los quales se suelen fazer malos malefiçios } ca assi commo alguons delos omes fazen mayores tuertos & quia Reges et Principes non debent pati maleficos viuere . Sunt \textbf{ etiam consideranda loca in quibus consueuerunt magis maleficia perpetrari : } nam sicut quidam hominum magis iniustificant \\\hline
3.2.19 & assi son algunos logares mas apareiados \textbf{ para fazer mal } e mi usticia que los otros & nam sicut quidam hominum magis iniustificant \textbf{ quam alii sic sunt quaedam loca magis apta ad iniustificandum quam alia : } ut in ciuitate contingit esse vicos aliquos magis esse suspectos quam alios : \\\hline
3.2.19 & por que los mal fechores se acostunbraron \textbf{ de esconder se y mas que en los otros . } Et alli fuyen de la iustiçia & ut in ciuitate contingit esse vicos aliquos magis esse suspectos quam alios : \textbf{ quia iniustificantes ibidem possunt magis latere , } et effugere punientes : \\\hline
3.2.19 & e son brios mas apareiados \textbf{ para fazer mal que los otros . } e por ende deue ser tomado consseio & et umbrosa magis apta ad iniustificandum , \textbf{ quam alia : } debet ergo adhiberi consilium , \\\hline
3.2.19 & Otrossi çerca la guarda dela çibdat \textbf{ e del regno non solamente son de tomar consseios } por essos mismos çibdadanos & ut circa talia maior custodia praebeatur . Rursus circa custodiam ciuitatis \textbf{ et regni non solum sunt adhibenda consilia propter ipsos ciues vel propter eos } qui sunt in regno , \\\hline
3.2.19 & que el vno non faga tuerto contra el otro \textbf{ mas avn deuen tomar consseios } por los estrannos & ne unus iniustificet in alium : \textbf{ sed etiam propter ipsos extraneos : } considerandum enim est utrum regnum ex aliqua parte possit inuadi , \\\hline
3.2.19 & por los estrannos \textbf{ por la qual cosa deuemos penssar } si el regno puede ser cometido de alguna parte . & sed etiam propter ipsos extraneos : \textbf{ considerandum enim est utrum regnum ex aliqua parte possit inuadi , } et utrum aliqua ciuitas regni ab extraneis possit suscipere detrimentum : \\\hline
3.2.19 & si el regno puede ser cometido de alguna parte . \textbf{ Et si alguna çibdat del regno puede resçebir danno de los estran nos } por ende los passaies e los puertos e las entradas & considerandum enim est utrum regnum ex aliqua parte possit inuadi , \textbf{ et utrum aliqua ciuitas regni ab extraneis possit suscipere detrimentum : } ideo passagia , portus , \\\hline
3.2.19 & e las otras cosas tales \textbf{ por que pueden entrar los de fuera } non son de acomne dar & introitus \textbf{ et caetera talia unde possunt extrinseci ad venire , } non sunt extraneis committenda vel tribuenda , \\\hline
3.2.19 & por que pueden entrar los de fuera \textbf{ non son de acomne dar } nin de dar alos estrannos & et caetera talia unde possunt extrinseci ad venire , \textbf{ non sunt extraneis committenda vel tribuenda , } sed sunt diligenter custodienda \\\hline
3.2.19 & non son de acomne dar \textbf{ nin de dar alos estrannos } mas son con grant acuçia de guardat e de guamesçer . & et caetera talia unde possunt extrinseci ad venire , \textbf{ non sunt extraneis committenda vel tribuenda , } sed sunt diligenter custodienda \\\hline
3.2.19 & nin de dar alos estrannos \textbf{ mas son con grant acuçia de guardat e de guamesçer . } Lo quarto deue ser tomadon consseio dela paz e dela guerra & non sunt extraneis committenda vel tribuenda , \textbf{ sed sunt diligenter custodienda } et munienda . Quarto habet esse consilium circa pacem et bellum , \\\hline
3.2.19 & Lo quarto deue ser tomadon consseio dela paz e dela guerra \textbf{ la qual cosa non se deue entender } dela paz delas çibdadanos & et munienda . Quarto habet esse consilium circa pacem et bellum , \textbf{ quod non est intelligendum circa pacem ciuium } vel circa eorum bellum : \\\hline
3.2.19 & por que la paraz de los çibdadanos es fin \textbf{ que deuen todos entender } e la guerra es su contraria & ø \\\hline
3.2.19 & e la guerra es su contraria \textbf{ que deuen todos escusar } mas dela fin & vel circa eorum bellum : \textbf{ non pax ciuium est aliquid finaliter intentum , bellum autem est eius oppositum : } de fine autem et de eius opposito nullus sanae mentis consiliatur : \\\hline
3.2.19 & mas dela fin \textbf{ e del su contrario ninguno de sano entendimiento non deue tomar consseio . . } ca el consseio non es de tomar & non pax ciuium est aliquid finaliter intentum , bellum autem est eius oppositum : \textbf{ de fine autem et de eius opposito nullus sanae mentis consiliatur : } nam non est consilium \\\hline
3.2.19 & e del su contrario ninguno de sano entendimiento non deue tomar consseio . . \textbf{ ca el consseio non es de tomar } si non de aquellas cosas & de fine autem et de eius opposito nullus sanae mentis consiliatur : \textbf{ nam non est consilium } nisi de his de quibus est dubium , \\\hline
3.2.19 & e de aquello que prinçipalmente omne entiende \textbf{ ninguno non dubda delo segnir . } Et del contrario della cada vno sabe & et de hoc quod principaliter intenditur , \textbf{ nullus dubitat ipsum esse prosequendum . } De eius autem opposito quilibet cognoscit ipsum esse fugiendum . Ideo pax ciuium pro viribus est prosequenda , \\\hline
3.2.19 & Et del contrario della cada vno sabe \textbf{ que es de foyr . } Por la qual cosa la paz de los çibdadanos & ø \\\hline
3.2.19 & Por la qual cosa la paz de los çibdadanos \textbf{ con todo poder es de segnir . } Et las discordias dellos & nullus dubitat ipsum esse prosequendum . \textbf{ De eius autem opposito quilibet cognoscit ipsum esse fugiendum . Ideo pax ciuium pro viribus est prosequenda , } et eorum dissensiones \\\hline
3.2.19 & Et las discordias dellos \textbf{ e las guerras con todo su poder son de escusar } e en esto non cae questiuo nin consseio . & et eorum dissensiones \textbf{ et bella pro viribus fugienda : } et in hoc non est \\\hline
3.2.19 & e en esto non cae questiuo nin consseio . \textbf{ Mas si deuemos auer paz con los estrannos } o guerra pue de ser cosa dubdosa & quaestio nec consilium . \textbf{ Sed utrum cum extraneis debeamus habere pacem } vel bella dubitabile esse potest , \\\hline
3.2.19 & que non sea con razon e con derecho . \textbf{ por que fazer tuerto alos otros } e apremiar sos sin derecho es mala cosa por si & ut nunquam capiatur iniustum bellum , \textbf{ quia iniustificari in alios , } et eos indebite opprimere , \\\hline
3.2.19 & por que fazer tuerto alos otros \textbf{ e apremiar sos sin derecho es mala cosa por si } e es muchͣ de escusar & quia iniustificari in alios , \textbf{ et eos indebite opprimere , | per se est malum , } et fugiendum . Deinde , \\\hline
3.2.19 & e apremiar sos sin derecho es mala cosa por si \textbf{ e es muchͣ de escusar } despues si fuere iusto & per se est malum , \textbf{ et fugiendum . Deinde , } si visum sit bellum esse iustum , consideranda est potentia regni , \\\hline
3.2.19 & deue ser penssado el poderio del regno o dela çibdat \textbf{ que deue lidiar } quanto es & si visum sit bellum esse iustum , consideranda est potentia regni , \textbf{ vel ciuitatis , | quae bellare debet , } quanta sit , \\\hline
3.2.19 & quanto es \textbf{ e quanta aynda le puede uenir de fuera . } Avn deue ser penssado & quanta sit , \textbf{ et quanta ei potest aduenire extrinsecus . Consideranda est etiam potentia aduersariorum : } nam \\\hline
3.2.19 & por que segunt diz el philosofo en el primero libro de la rectorica \textbf{ con los meiores deuemos auer paz } mas con los peores & et quanta ei potest aduenire extrinsecus . Consideranda est etiam potentia aduersariorum : \textbf{ nam } secundum Philosophum 1 Rhetor’ ad meliores pacem debemus habere , ad deteriores autem nobis est expugnare vel non pugnare contra eos . \\\hline
3.2.19 & mas con los peores \textbf{ en nos es de lidiar o de non lidiar . } Ca puesto que los mas poderolos & nam \textbf{ secundum Philosophum 1 Rhetor’ ad meliores pacem debemus habere , ad deteriores autem nobis est expugnare vel non pugnare contra eos . } Posito enim quod potentiores , \\\hline
3.2.19 & Ca puesto que los mas poderolos \textbf{ alos quales non podemos contradezer } nos fagan alguna fuerca o algun tuerto grant sabiduria & Posito enim quod potentiores , \textbf{ ad quos resistere non valemus , } in nos forefaciant , \\\hline
3.2.19 & nos fagan alguna fuerca o algun tuerto grant sabiduria \textbf{ es non nos leunatar contra ellos } si non fuere en tien poconuenible & prudentiae est , \textbf{ non insurgere in ipsos , } nisi occurrat opportunitas temporis , \\\hline
3.2.19 & si non fuere en tien poconuenible \textbf{ en el qual podamos tomar conueinblemente uenganças dellos ¶ } Lo quinto çerca & nisi occurrat opportunitas temporis , \textbf{ in quo ex eis congrue possimus vindictas assumere . } Quintum circa quod sunt consilia adhibenda , \\\hline
3.2.19 & Lo quinto çerca \textbf{ lo que deuemos tomar el consse io } es el ponimiento delas leyes . & in quo ex eis congrue possimus vindictas assumere . \textbf{ Quintum circa quod sunt consilia adhibenda , } est Lator legum : \\\hline
3.2.19 & por que los regnos non pueden estar \textbf{ nin mucho durar sin iustiçia . } Et por ende conuiene al Rey de saber & et regna per leges iustas , \textbf{ quia absque iustitia nequeunt regna subsistere . Decet autem scire Regem quot sunt genera dominorum , siue quot sunt species principantium , } et \\\hline
3.2.19 & nin mucho durar sin iustiçia . \textbf{ Et por ende conuiene al Rey de saber } quantas son las . maneras de los senorios & et regna per leges iustas , \textbf{ quia absque iustitia nequeunt regna subsistere . Decet autem scire Regem quot sunt genera dominorum , siue quot sunt species principantium , } et \\\hline
3.2.19 & Et en qual manera el prinçipado \textbf{ segunt el qual alguno enssennorea se ha de saluar o corronper . } por que escogiendo la meior manera de prinçipar o de enssennorear ponga leyes muy derechos . & et qualiter principatus \textbf{ secundum quem dominatur habet saluari et corrumpi : } ut eligens optimum modum principandi , \\\hline
3.2.19 & segunt el qual alguno enssennorea se ha de saluar o corronper . \textbf{ por que escogiendo la meior manera de prinçipar o de enssennorear ponga leyes muy derechos . } segunt las quales aquel prinçipado se ha de saluar . & secundum quem dominatur habet saluari et corrumpi : \textbf{ ut eligens optimum modum principandi , | ferat leges iustissimas , } secundum quas saluari habet principatus ille . \\\hline
3.2.19 & por que escogiendo la meior manera de prinçipar o de enssennorear ponga leyes muy derechos . \textbf{ segunt las quales aquel prinçipado se ha de saluar . } Mas qual es la manera muy buena de prinçipar o de enssennorear . & ferat leges iustissimas , \textbf{ secundum quas saluari habet principatus ille . } Quis sit autem optimus modus principandi , \\\hline
3.2.19 & segunt las quales aquel prinçipado se ha de saluar . \textbf{ Mas qual es la manera muy buena de prinçipar o de enssennorear . } e qual deua ser el ofiçio del rey & secundum quas saluari habet principatus ille . \textbf{ Quis sit autem optimus modus principandi , } et quale debeat esse regis officium , \\\hline
3.2.19 & e qual deua ser el ofiçio del rey \textbf{ e en qual manera el Rey se deua guardar } en su sennorio & et quale debeat esse regis officium , \textbf{ et quomodo Rex se debeat in suo dominio praeseruare , } fuit in superioribus patefactum . \\\hline
3.2.19 & podra ser enssennado el gouernador dela çibdat e del regno . \textbf{ en qual manera se deue consseiar çerca el establesçimiento delas leyes } Et quales leyes son de establesçer e de poner & instrui poterit Rector ciuitatis aut regni , \textbf{ qualiter circa lationem legum sit consiliandum . } Quae autem leges sunt ferendae , ut saluetur principatus eius , \\\hline
3.2.19 & en qual manera se deue consseiar çerca el establesçimiento delas leyes \textbf{ Et quales leyes son de establesçer e de poner } por que se salue el prinçipado & qualiter circa lationem legum sit consiliandum . \textbf{ Quae autem leges sunt ferendae , ut saluetur principatus eius , } intendimus in capitulis sequentibus \\\hline
3.2.19 & Enpero entendemos en los capitulos \textbf{ que se siguen requarir muchͣs cosas delas leyes . } por que en vno de muchͣs cosas & intendimus in capitulis sequentibus \textbf{ de legibus multa inquirere , } ut simul ex dictis et dicendis melius veritas patere possit . \\\hline
3.2.19 & que son ya dichas \textbf{ e son por dezir m } or que en los capitulos sobredichos dixiemos del prinçipe demo & de legibus multa inquirere , \textbf{ ut simul ex dictis et dicendis melius veritas patere possit . } Quia in praecedentibus capitulis determinauimus de principe , \\\hline
3.2.20 & mostrando \textbf{ en qual manera deuemos iudgar } e quantas son las maneras delas leyes & siue de iudicio , \textbf{ inuestigando qualiter iudicandum sit , } et quot sunt genera legum , \\\hline
3.2.20 & Et las otras cosas \textbf{ que pueden acaesçer çerca desta materia } mas commo el iuyzio se deua fazer & et quot sunt genera legum , \textbf{ et alia quae circa istam occurrunt materiam . } Sed cum iudicium fiat per leges , \\\hline
3.2.20 & que pueden acaesçer çerca desta materia \textbf{ mas commo el iuyzio se deua fazer } por las leyes o por aluedrio o por amas estas cosas . & et alia quae circa istam occurrunt materiam . \textbf{ Sed cum iudicium fiat per leges , } aut per arbitrium , \\\hline
3.2.20 & por las leyes o por aluedrio o por amas estas cosas . \textbf{ ante que mostremos en qual manera deuemos iudgar . } queremos declarar & aut per arbitrium , \textbf{ aut per utrunque priusquam ostendamus qualiter sit iudicandum , } declarare volumus \\\hline
3.2.20 & ante que mostremos en qual manera deuemos iudgar . \textbf{ queremos declarar } que quanto puede ser todas las cosas son de determinar & aut per utrunque priusquam ostendamus qualiter sit iudicandum , \textbf{ declarare volumus } quod quantum possibile est sunt omnia legibus determinanda , et quam pauciora possunt sunt arbitrio iudicum committenda . \\\hline
3.2.20 & queremos declarar \textbf{ que quanto puede ser todas las cosas son de determinar } por las leyes & declarare volumus \textbf{ quod quantum possibile est sunt omnia legibus determinanda , et quam pauciora possunt sunt arbitrio iudicum committenda . } Quod quadruplici via inuestigare possumus , \\\hline
3.2.20 & que pueden ser deuen ser puestas en aluedrio de los uiezes . \textbf{ Et esto podemos demostrar } por quatro razones delas quales las tres tanne el philosofo en el primero libro de la rectorica & quod quantum possibile est sunt omnia legibus determinanda , et quam pauciora possunt sunt arbitrio iudicum committenda . \textbf{ Quod quadruplici via inuestigare possumus , } quarum tres tanguntur 1 Rhet’ quarta vero tangitur 1 Polit’ . \\\hline
3.2.20 & que aya vna alcalłia otdinaria \textbf{ ala qual deuen venir todos los pleitos } e todas las contiendas & secundum leges iam conditas . \textbf{ Nam in qualibet ciuitate oporteret esse aliquod praetorium ordinarium ad quod causae reducantur } et litigia exorta in ciuitate illa : \\\hline
3.2.20 & Empero por las leyes \textbf{ que son puestas en vna çibdat se pueden reglar muchas çibdadeᷤ } Et en aquella misma çibdat & et litigia exorta in ciuitate illa : \textbf{ per leges tamen conditas in una ciuitate regularis possunt ciuitates multae . Immo illa eadem ciuitate in } qua leges conduntur contingit plures esse iudices , \\\hline
3.2.20 & ca las leyes si son derecheras deuen ser tales \textbf{ que se non puedan corronper nin mudar . } ca en lłas non deue ser fecha mudaçion ninguna o muy pequana . & si iustae sint , debent esse quasi immortales \textbf{ et immutabiles : } quia circa eas nulla aut modica mutatio fieri debet . \\\hline
3.2.20 & que sin cornupçion e sin muerte los iuezes son tirados de sus oficlvii i̊ çios \textbf{ e son prouestos otros en sir logar } Et pues que assi es si & et morte iudices a suo officio remoueri , \textbf{ et alio in suum locum succedere . Igitur saltem per successionem ipsorum oportet in eadem ciuitate multos esse iudices : } eo quod ipsi quasi continue innouantur : \\\hline
3.2.20 & que assi se amuchiguen los fazedores delas leyes . \textbf{ por que las leyes non se deuen renouar muchas uezes . } por la qual cosa si los fazedores son pocos en conparaçion de los iuezes & non tamen sic oportet multiplicare legum conditores , \textbf{ eo quod leges | non si continue innouari debent . } Quare si legum conditores respectu iudicum sunt pauci , \\\hline
3.2.20 & por la qual cosa si los fazedores son pocos en conparaçion de los iuezes \textbf{ porque mas ligera cosa es de fallar pocos sabios que muchs . } por que todas las cosas sean ordenadas sabiamente & Quare si legum conditores respectu iudicum sunt pauci , \textbf{ quia facilius est inuenire paucos sapientes , } quam multos , ut omnia sapienter disponantur , \\\hline
3.2.20 & por ende commo los fazedores delas leyes en muchtp̃o \textbf{ e con grant conseio puedan et de una determinar quales leyes de una poner . } Mas los iuezes & statim iudicatiuam sententiam proferre . Itaque cum legum conditores \textbf{ multo tempore et magno consilio deliberare possint quales debeant leges fieri : } iudices vero propter instantiam partium , \\\hline
3.2.20 & por afincamiento delas partes \textbf{ et por que non deuen prolongar los pleitos } mas en cortallos non han tanto prolongamiento de tien po & iudices vero propter instantiam partium , \textbf{ et quia non debent lites prolongari , } sed deprimi , \\\hline
3.2.20 & et por que non deuen prolongar los pleitos \textbf{ mas en cortallos non han tanto prolongamiento de tien po } para ver qual es el derecho & et quia non debent lites prolongari , \textbf{ sed deprimi , | non tantam habent diuturnitatem ad videndum } quid iustum in proposito , \\\hline
3.2.20 & mas en cortallos non han tanto prolongamiento de tien po \textbf{ para ver qual es el derecho } e commo han de guardar iustiçia en el negoçio & non tantam habent diuturnitatem ad videndum \textbf{ quid iustum in proposito , } ne igitur in iudicando erretur , \\\hline
3.2.20 & para ver qual es el derecho \textbf{ e commo han de guardar iustiçia en el negoçio } que les es propuesto . & non tantam habent diuturnitatem ad videndum \textbf{ quid iustum in proposito , } ne igitur in iudicando erretur , \\\hline
3.2.20 & que les es propuesto . \textbf{ Et por ende por que non se yerre ninguͣ cosa en el iuyzio } quanto pudiere ser conuiene & quid iustum in proposito , \textbf{ ne igitur in iudicando erretur , } quantum possibile est sunt omnia legibus determinanda , \\\hline
3.2.20 & e delas cosas \textbf{ que auien de venir diziendo } que qual quier que tal cosa fiziere tal pena aura non sabiendo si serie amigo o enemigo & nam conditores legum leges ferunt in uniuersali et de futuris , \textbf{ dicentes | quicunque sic egerit , } sic puniatur , \\\hline
3.2.20 & aquel \textbf{ que auie de fazer aquella cosa } e deuie passar & ignorantes an amicus , \textbf{ vel inimicus } sit illa facturus , et debeat illam subire sententiam . \\\hline
3.2.20 & que auie de fazer aquella cosa \textbf{ e deuie passar } por tal suina . & vel inimicus \textbf{ sit illa facturus , et debeat illam subire sententiam . } Nam si scirent quod amicus , \\\hline
3.2.20 & ca si por auentra asopiessen ellos \textbf{ que su amigo auie de fazer aquella cosa } por auentraase torçerian en iudgando & sit illa facturus , et debeat illam subire sententiam . \textbf{ Nam si scirent quod amicus , } forte obliquerentur in iudicando , et poenam palliarent : \\\hline
3.2.20 & e encobririen la pena con algun color . \textbf{ Mas si fuesse su enemigo enclinar } sseyen por auentura ala parte contraria & forte obliquerentur in iudicando , et poenam palliarent : \textbf{ si vero inimicus , inclinarentur forte in partem oppositam , et punitionem augerent . Nunc autem } quia solum in hoc leges ferunt , et nesciunt \\\hline
3.2.20 & non son de las cosas \textbf{ que han de uenir } mas son delas cosas passadas & ø \\\hline
3.2.20 & e señaladas en coruna los iuezes \textbf{ para las amar o para las desamar . } Et muchas uezes tales perssonas tienen mientes a su bien propra o . por ende el iuez de ligero se t uerçeca & sed in particulari . Incusantur enim determinatae personae , \textbf{ ad quas est amare vel odire , } et saepe talia annexum habent proprium commodum . \\\hline
3.2.20 & et que muy pocas cosas sean dexadas alos iuezes ¶ \textbf{ Lo primero por que mas ligeramente pueden auer los omes vn sabio } o pocos que muchos . & quaecunque possibile est determinare : \textbf{ et quam paucissima committere iudicantibus . Primum quidem quia facilius est habere unam } aut paucos sapientes , \\\hline
3.2.20 & de las cosas \textbf{ que han de venir } et de las cosas generales & quia legislatores sunt de futuris \textbf{ et in uniuersali : } sed praefectus aut iudex iudicat de praesentibus et determinatis , \\\hline
3.2.20 & e delas perssonas señaladas . \textbf{ alas quales puede auer amor o abortençia . } Alos quales alcalłs muchos vezes se ayunta algun pro & sed praefectus aut iudex iudicat de praesentibus et determinatis , \textbf{ ad quos est amare , } et odire , et quibus proprium commodum annexum est saepe . \\\hline
3.2.20 & para si mesmos \textbf{ et por ende se pueden torçer en iudgando . } la quarta razon & et odire , et quibus proprium commodum annexum est saepe . \textbf{ Quarta via ad ostendendum hoc idem , } sic declarari potest . \\\hline
3.2.20 & la quarta razon \textbf{ para mostrar esto mismo se puede assi declarar . } ca assi commo paresçra adelançe conuiene & Quarta via ad ostendendum hoc idem , \textbf{ sic declarari potest . } Nam ( ut infra patebit ) \\\hline
3.2.20 & so recontamiento milo tuas çerominso palabrus gentas \textbf{ nin le pueden determinar conplidamente todas las cosas por leyes . } Enpero quanto pudieren ser todas las cosas & quia gesta particularia complete sub narratione non cadunt , \textbf{ nec lege omnia complete determinari possunt : } tamen quantum possibile est omnia legibus sunt determinanda , \\\hline
3.2.20 & Enpero quanto pudieren ser todas las cosas \textbf{ deuen ser determinadas por las leyes e muy pocas cosas se deuen dexar en aluedrio de los mezes } assi commo dize el pho en el vij̊ . libro de las politicas & tamen quantum possibile est omnia legibus sunt determinanda , \textbf{ et quanto pauciora possunt , } sunt arbitrio iudicum committenda : \\\hline
3.2.20 & que assi es \textbf{ quanto men or enemistaça fueren los que han de dar los iuyzios } tanto mas ayna uerna fazer essecuçion & quia ( ut dicitur \textbf{ 6 Politicorum ) | quanto utique minor inimicitia fuerit exequentibus iudicia , } tanto magis accipient finem executione iudiciorum . \\\hline
3.2.20 & quanto men or enemistaça fueren los que han de dar los iuyzios \textbf{ tanto mas ayna uerna fazer essecuçion } e dar fin a los iuyzios & quanto utique minor inimicitia fuerit exequentibus iudicia , \textbf{ tanto magis accipient finem executione iudiciorum . } Quare cum Iudex iudicando reos \\\hline
3.2.20 & tanto mas ayna uerna fazer essecuçion \textbf{ e dar fin a los iuyzios } e por la qual cosa commo el iues iudgando los culpados & quanto utique minor inimicitia fuerit exequentibus iudicia , \textbf{ tanto magis accipient finem executione iudiciorum . } Quare cum Iudex iudicando reos \\\hline
3.2.20 & por que el uiez non se incline . \textbf{ por temor de enemistança a alongar los iuyzios } e non los traer a su fin . & si iudicaret eos arbitrio proprio , \textbf{ ne iudex timore inimicitiae inclinatus differat debito } fini iudicia mancipare , \\\hline
3.2.20 & por temor de enemistança a alongar los iuyzios \textbf{ e non los traer a su fin . } por ende quanto puede ser todos los pleitos son de determinar & ne iudex timore inimicitiae inclinatus differat debito \textbf{ fini iudicia mancipare , } quantum possibile est , \\\hline
3.2.20 & e non los traer a su fin . \textbf{ por ende quanto puede ser todos los pleitos son de determinar } por las leyes & fini iudicia mancipare , \textbf{ quantum possibile est , } sunt omnia legibus determinanda , \\\hline
3.2.20 & por las leyes \textbf{ e muy pocas cosas son de dexar en aluedrio de los mueze ᷤ } quando iudga alguna cosa & sunt omnia legibus determinanda , \textbf{ et pauca arbitrio iudicum committenda : | sufficienter enim iudex excusatur , } cum secundum positas leges aliquid iudicat ; \\\hline
3.2.20 & mas que lo faze costrennido por la ley \textbf{ e por ende es dicho dar e publicar tal snïa } uien tener mientes los iueze ᷤ & secundum se hoc agere , \textbf{ sed lege compulsus dicitur talem sententiam promulgare . } Attendere debent iudices , \\\hline
3.2.21 & e por ende es dicho dar e publicar tal snïa \textbf{ uien tener mientes los iueze ᷤ } que assi passen el iuyzio & sed lege compulsus dicitur talem sententiam promulgare . \textbf{ Attendere debent iudices , } ut in iudicio procedant , \\\hline
3.2.21 & por las palabras malas e sannudas \textbf{ que pueden mouer los omes a mal } assi comm̃ayra a abortençia sean defendidas en łmyzio & ut sermones passionales prouocantes ad passiones , \textbf{ ut ad iram et odium ; } in iudicio prohibeantur : \\\hline
3.2.21 & o lo que non es fecho \textbf{ mas tornassea mouer el uuez a sana o a aborrençia } contra los sus contrarios & et quid non factum , \textbf{ sed conuertunt se ad commouendum iudicem ad iram } et odium circa aduersarios , \\\hline
3.2.21 & e pugnan \textbf{ por lo mouer amanssedunbre } e amiscderia contra si mismos . & et ad benignitatem \textbf{ et misericordiam erga seipsos . } Sed quod tales sermones sint prohibendi , \\\hline
3.2.21 & e amiscderia contra si mismos . \textbf{ Mas que estas palabras malas sean de defender } e de deue dar & et misericordiam erga seipsos . \textbf{ Sed quod tales sermones sint prohibendi , } triplici via inuestigare possumus . Prima sumitur ex eo quod huiusmodi sermones obligare habent iudicem , \\\hline
3.2.21 & Mas que estas palabras malas sean de defender \textbf{ e de deue dar } ante los alcalłs rodemos lo prouar por tres razones ¶ & et misericordiam erga seipsos . \textbf{ Sed quod tales sermones sint prohibendi , } triplici via inuestigare possumus . Prima sumitur ex eo quod huiusmodi sermones obligare habent iudicem , \\\hline
3.2.21 & e de deue dar \textbf{ ante los alcalłs rodemos lo prouar por tres razones ¶ } La primera seqma & ø \\\hline
3.2.21 & La primera seqma \textbf{ par aquello que tales palabras han de to terçeres desegualar eliez } el qual conuiene de ser & Sed quod tales sermones sint prohibendi , \textbf{ triplici via inuestigare possumus . Prima sumitur ex eo quod huiusmodi sermones obligare habent iudicem , } quem esse oportet \\\hline
3.2.21 & nin a la razon \textbf{ que han de dezir . } La primera razon paresçe assiça deuedessaber & quia sunt impertinentes ad propositum . \textbf{ Prima via sic patet . } Scire enim debemus quod iudex in iudicando de litigiis , \\\hline
3.2.21 & que han de dezir . \textbf{ La primera razon paresçe assiça deuedessaber } que el iuez en iudgando de los pleitos & quia sunt impertinentes ad propositum . \textbf{ Prima via sic patet . } Scire enim debemus quod iudex in iudicando de litigiis , \\\hline
3.2.21 & para que derechamente iudgue \textbf{ assi se deue auer entre las partes } que contienden commo la lengua & ut recte iudicet , \textbf{ sic debet se habere | inter partes litigantes , } sicut lingua volens discernere de saporibus , \\\hline
3.2.21 & que contienden commo la lengua \textbf{ quando quiere iudgar de los sabores } o si commo cada vno de los otros sesos & inter partes litigantes , \textbf{ sicut lingua volens discernere de saporibus , } vel sicut quilibet alius sensus volens discernere de proprii sensibilibus , \\\hline
3.2.21 & o si commo cada vno de los otros sesos \textbf{ quando quieren iudgar delas otras cosas } que sienten propriamente & sicut lingua volens discernere de saporibus , \textbf{ vel sicut quilibet alius sensus volens discernere de proprii sensibilibus , } habere se debet \\\hline
3.2.21 & que sienten propriamente \textbf{ ca la lengua se deue auer entre los sabores } o cada vno de los otros sesos & habere se debet \textbf{ inter ipsos sapores , } vel inter propria sensibilia . \\\hline
3.2.21 & assi commo regla tuerta \textbf{ e iudgar mal e desigual mente . } Et por que esto fazen las palabras desiguales & et amicitiam , ab altera vero recedat per iram \textbf{ et odium , quasi regula tortuosa peruerse iudicabit , } et quia hoc faciunt sermones passionales , \\\hline
3.2.21 & et malas \textbf{ consentir tales palabras en el iuyzio } non es otra cosa & et quia hoc faciunt sermones passionales , \textbf{ permittere talia in iudicio nihil est aliud quam regulam obliquare : } quasi si inconueniens est permittere obliquari regulam , \\\hline
3.2.21 & non es otra cosa \textbf{ si non torçer la regla } que non iudgue derecho & permittere talia in iudicio nihil est aliud quam regulam obliquare : \textbf{ quasi si inconueniens est permittere obliquari regulam , } inconueniens est sustinere in iudicio passionales sermones . \\\hline
3.2.21 & por la qual cosasi non es cosa conueible \textbf{ que la regla se tuerca non es cosa conuenible de sofrir } que enco el uuzio se digan palabras malas e desiguales & quasi si inconueniens est permittere obliquari regulam , \textbf{ inconueniens est sustinere in iudicio passionales sermones . } Dato tamen quod contingat sustinere aliquos passionales sermones , \\\hline
3.2.21 & ca assi commo parezçca \textbf{ en lo que es de dezer los mezes } mas enclinados deuen sera auer piedat & inconueniens est sustinere in iudicio passionales sermones . \textbf{ Dato tamen quod contingat sustinere aliquos passionales sermones , } quia ( ut in sequentibus patebit ) proniores debent esse iudices ad miserendum quam ad puniendum , potius sustinendi sunt sermones passionales prouocantes ad misericordiam \\\hline
3.2.21 & en lo que es de dezer los mezes \textbf{ mas enclinados deuen sera auer piedat } que ha de condenar o a dar pena & Dato tamen quod contingat sustinere aliquos passionales sermones , \textbf{ quia ( ut in sequentibus patebit ) proniores debent esse iudices ad miserendum quam ad puniendum , potius sustinendi sunt sermones passionales prouocantes ad misericordiam } vel beniuolentiam , \\\hline
3.2.21 & mas enclinados deuen sera auer piedat \textbf{ que ha de condenar o a dar pena } e mas son de sofrir las palabras & Dato tamen quod contingat sustinere aliquos passionales sermones , \textbf{ quia ( ut in sequentibus patebit ) proniores debent esse iudices ad miserendum quam ad puniendum , potius sustinendi sunt sermones passionales prouocantes ad misericordiam } vel beniuolentiam , \\\hline
3.2.21 & que ha de condenar o a dar pena \textbf{ e mas son de sofrir las palabras } que enclinan los omes a miscderia o a bien querençia & quia ( ut in sequentibus patebit ) proniores debent esse iudices ad miserendum quam ad puniendum , potius sustinendi sunt sermones passionales prouocantes ad misericordiam \textbf{ vel beniuolentiam , } quam ad odium \\\hline
3.2.21 & ¶La segunda razon \textbf{ que prueua que las palabras desiguales e malas non son de consentir } en el iuyzio se toma desto & quam ad odium \textbf{ vel ad iram . Secunda via omnes passionales sermones permittendos non esse , sumitur ex } eo \\\hline
3.2.21 & en el iuyzio se toma desto \textbf{ que tales palabras tristornan la orden del iudgar . } ca en iudgado es vna orden & eo \textbf{ quod talia peruertunt ordinem iudicandi . } In iucando enim est ordo quidam , \\\hline
3.2.21 & e entre los contendedores en el pleito . \textbf{ Et porque la cosa medianera deue tomar alguͣ cosa de cada vno de los estremos . } Et el iuez & inter legislatorem et litigantes , \textbf{ et quia medium aliquid | debet accipere ab utroque extremorum , } iudex tanquam medius aliquid accepit ab utrisque . \\\hline
3.2.21 & qual es derecho \textbf{ que ha de iudgar en las obras delos omes . } por las quales leyes se regla el iues en iudgando . & quid iuris ab illo addiscit iudex \textbf{ quid iustum in agibilibus humanis } per cuius leges regulatur in iudicando , \\\hline
3.2.21 & enclinan la uoluntad de los omes \textbf{ e fagan paresçer alguna cosa derecha . } por que los que assi son munnidos por passiones & Quare cum sermones passionabiles inclinent voluntatem , \textbf{ et faciant apparere aliquid iustum , } vel non iustum , \\\hline
3.2.21 & por que los que assi son munnidos por passiones \textbf{ asi commo a amar o aborresçer o a gozo o a tristeza . } a las quales cosas se mueu en los omes & et faciant apparere aliquid iustum , \textbf{ vel non iustum , } eo quod passionati , \\\hline
3.2.21 & Et por ende iudgan ygual mente . \textbf{ Et pues que assi es conssentir tales palabras } en iuyzio & eo quod passionati , \textbf{ ut amantes , et odientes , et gaudentes , et tristantes non pariter iudicamus , permittere passionales sermones in iudicio , } est peruertere ordinem iudicandi : \\\hline
3.2.21 & en iuyzio \textbf{ es trastornar la orden de iudgar . } Ca es iudgar las partes & ut amantes , et odientes , et gaudentes , et tristantes non pariter iudicamus , permittere passionales sermones in iudicio , \textbf{ est peruertere ordinem iudicandi : } quia est facere quod partes sint legislatores , \\\hline
3.2.21 & es trastornar la orden de iudgar . \textbf{ Ca es iudgar las partes } que contiende ser fazedores delas leyes & est peruertere ordinem iudicandi : \textbf{ quia est facere quod partes sint legislatores , } et quod teneant supremum gradum in iudicando , \\\hline
3.2.21 & que contiende ser fazedores delas leyes \textbf{ e tener elguado primero en iudgar } aquellos que deuen tener el postrimero & quia est facere quod partes sint legislatores , \textbf{ et quod teneant supremum gradum in iudicando , } quae debent tenere infimum . \\\hline
3.2.21 & e tener elguado primero en iudgar \textbf{ aquellos que deuen tener el postrimero } e assi se trastorna y la orden de iudgar . & et quod teneant supremum gradum in iudicando , \textbf{ quae debent tenere infimum . } Peruertitur ibi talis ordo , \\\hline
3.2.21 & aquellos que deuen tener el postrimero \textbf{ e assi se trastorna y la orden de iudgar . } Ca las partes mouiendo el iuez & quae debent tenere infimum . \textbf{ Peruertitur ibi talis ordo , } quia partes passionando iudicem , \\\hline
3.2.21 & Ca las partes mouiendo el iuez \textbf{ assi fazen paresçer alguͣ cosa derechͣo non de rethica . } la qual cosa es ofiçio del ponedor dela ley . & quia partes passionando iudicem , \textbf{ ei faciunt apparere aliquid iustum vel iniustum , } quod non est officium partium , \\\hline
3.2.21 & ca por el ponedor dela ley \textbf{ o por las leyes puestas por el deue paresçer } aliues & per legis enim latorem , \textbf{ vel per leges ab eo conditas oportet apparere iudici } quid iustum \\\hline
3.2.21 & e qual non es el derecho . \textbf{ Mas esto non se deue fazer } por las partes que contienden o por las palabras desiguales dichas delas partes . & et quid iniustum : \textbf{ non autem hoc debet fieri per partes litigantes , } vel per sermones passionales a partibus promulgatos . Tertia via sumitur ex \\\hline
3.2.21 & Ca commo la contienda sea de algun fech̃ \textbf{ o de alguna cosa non se deue dezir ninguna cosa en iuizio } si non lo que pertenesçe a aquel fecho o a aquel negoçio & Nam cum lis fit de aliquo facto \textbf{ vel de aliquare , | nihil oportet dici in iudicio nisi pertinens } ad rem vel ad negocium , \\\hline
3.2.21 & de que contienden \textbf{ Mas enduzir al iiez } por palabras contando le las miurias . & de quo est litigium : \textbf{ passionare autem iudicem , } aut narrare iniurias quas pars aduersa iudici intulit , \\\hline
3.2.21 & que ellos fizieron a liiez . \textbf{ Et en esta manera inclinar al iuez a malenconia } e a mal querençia dela parte contraria & vel narrare bona quae ipsi iudici contulerunt , \textbf{ et hoc modo prouocare iudicem ad maliuolentiam partis aduersae , } et ad beniuolentiam sui , \\\hline
3.2.21 & Esto non pertenesçe en ninguna guasa al proposito . \textbf{ por la qual razon non son de consentir tales palabras en iuyzio . } demos contar quatro cosas & est omnino impertinens ad propositum : \textbf{ quare | non sunt talia permittenda . } Possumus autem quatuor enumerare , \\\hline
3.2.22 & por la qual razon non son de consentir tales palabras en iuyzio . \textbf{ demos contar quatro cosas } que conuiene de auer alos iuezes & non sunt talia permittenda . \textbf{ Possumus autem quatuor enumerare , } quae oportet habere iudices , \\\hline
3.2.22 & demos contar quatro cosas \textbf{ que conuiene de auer alos iuezes } para que den uerdaderos iuisios & Possumus autem quatuor enumerare , \textbf{ quae oportet habere iudices , } ut vera iudicia proferant , \\\hline
3.2.22 & e para que iudguen derechamente \textbf{ lo primero es auctoridat de iudguar } lo segundo es sabiduria delas colas ¶ & et ut recte iudicent . Primum , \textbf{ est auctoritas iudicandi . } Secundum est prudentia legum . Tertium , \\\hline
3.2.22 & Ca en todo pleito \textbf{ quanto parte nesçe alo presente quatro cosas podemos peussar . } Conuiene de saber las partes que contienden . & In omni enim litigio \textbf{ ( quantum ad praesens spectat ) quatuor est considerare , } videlicet partes litigantes , negocium \\\hline
3.2.22 & quanto parte nesçe alo presente quatro cosas podemos peussar . \textbf{ Conuiene de saber las partes que contienden . } Et el negoçio de que contienden . & ( quantum ad praesens spectat ) quatuor est considerare , \textbf{ videlicet partes litigantes , negocium } de quo litigant , \\\hline
3.2.22 & Et las leyes \textbf{ segunt las quales se han de iudgar las contiendas . } Et el fazedor dela ley & leges \textbf{ secundum quas sunt litigia iudicanda , } et legislatorem aut Regem \\\hline
3.2.22 & Lo quarto si los uiezes non se ouieren conueniblemente alos negoçios \textbf{ que han de librar } assi commo aquellos que non son prouados en las obras de los omnes & si iudices non debite se habent \textbf{ ad negocia agibilia , } ut si sint inexperti humanorum actuum , \\\hline
3.2.22 & e pruena de los fechos . \textbf{ Ca assi commo en guaresçer las enfermedades de los cuerpos algunas uezes } mas aprouecha el que ha la prueua & nisi concurrant ibi auctoritas iudicandi , prudentia legum , zelus iustitiae , et experientia agibilium . \textbf{ Nam sicut in curando morbos | corporales } aliquando \\\hline
3.2.22 & Et todas estas cosas sobredichͣs son menester \textbf{ para iudgar derechamente e conuenible mente . } Et de aqui paresçe quales iuezes e quales examinadores de los pleitos deue tomar el Rey . & itaque quales iudices \textbf{ et quales discussores causarum quaerere deceat regiam maiestatem : } nam decet eos tales quaerere \\\hline
3.2.22 & para iudgar derechamente e conuenible mente . \textbf{ Et de aqui paresçe quales iuezes e quales examinadores de los pleitos deue tomar el Rey . } Ca deue tales tomar & itaque quales iudices \textbf{ et quales discussores causarum quaerere deceat regiam maiestatem : } nam decet eos tales quaerere \\\hline
3.2.22 & Et de aqui paresçe quales iuezes e quales examinadores de los pleitos deue tomar el Rey . \textbf{ Ca deue tales tomar } que sean omildosos & et quales discussores causarum quaerere deceat regiam maiestatem : \textbf{ nam decet eos tales quaerere } qui sint humiles , \\\hline
3.2.23 & para que meior escodrinnen los pleitos Car \textbf{ uanto pertenesçe alo presente podemos contar diez cosas de la rectorica . } Las quales dies cosas & ut cognoscentes particularia acta melius discutiant causas . \textbf{ Quantum ad praesens spectat decem numerare possumus , } quae videtur tangere Philos’ 1 Rhet’ \\\hline
3.2.23 & ¶L pramero \textbf{ que deue penssar es la natura humanal . } Lo segundo deue tener mientes al ponedor dela ley ¶ & ut humanis indulgeat , \textbf{ et ut sit clemens potius quam seuerus . Primum est ipsa natura humana , } secundum legislator , \\\hline
3.2.23 & que deue penssar es la natura humanal . \textbf{ Lo segundo deue tener mientes al ponedor dela ley ¶ } Lo terçero al piadoso entendimiento delas leyes ¶ & ut humanis indulgeat , \textbf{ et ut sit clemens potius quam seuerus . Primum est ipsa natura humana , } secundum legislator , \\\hline
3.2.23 & Lo quarto ala entençion del que obra . \textbf{ Lo quanto deue tener mientes } ala muchedunbre delas buean sobras ¶ Lo . vi̊ el alongamiento del tp̃o passado . lo vii̊ . & quartum operantis intentio , \textbf{ quintum multitudo bonorum operum , } sextum diuturnitas temporis retroacti , \\\hline
3.2.23 & la auantaia de bondat sobre la maliçia ¶ \textbf{ Lo viij̊ deue parar mientes } ala paçiençia de aquel que es acusado . & sextum diuturnitas temporis retroacti , \textbf{ septimum excessus bonitatis supra malitiam , } octauum patientia accusati , \\\hline
3.2.23 & ala paçiençia de aquel que es acusado . \textbf{ Lo ix̊ deue tener mientes } si se puede castigar el que peça¶ Lo x̊ . & septimum excessus bonitatis supra malitiam , \textbf{ octauum patientia accusati , } nonum corrigibilitas peccantis , \\\hline
3.2.23 & Lo ix̊ deue tener mientes \textbf{ si se puede castigar el que peça¶ Lo x̊ . } deue parar mientes el uiezala humildat & octauum patientia accusati , \textbf{ nonum corrigibilitas peccantis , } decimum subiectio delinquentis . Primo enim ipsa natura humana clamat pro clementia delinquentis . \\\hline
3.2.23 & si se puede castigar el que peça¶ Lo x̊ . \textbf{ deue parar mientes el uiezala humildat } e ala subiectiuo del que peca . & octauum patientia accusati , \textbf{ nonum corrigibilitas peccantis , } decimum subiectio delinquentis . Primo enim ipsa natura humana clamat pro clementia delinquentis . \\\hline
3.2.23 & mas que usto o que es cosa sobre iustiçia . \textbf{ Ca la piadat es mas de alabar } e mas de enssalças & qui est supra iustum siue supra iustitiam . Philosophus igitur parcentem humanis appellat supra iustum , \textbf{ quia clementia est extollenda supra veritatem , } et supra iustitiam . \\\hline
3.2.23 & nin que la iustiçia afincada ¶ \textbf{ Lo segundo que deue inclinar al iues a piedat es el establesçedor dela ley . } ca si por auentura el Rey o el prinçipe & Secundum quod inclinare debet iudicem ad clementiam , \textbf{ est ipse legislator . } Nam forte ipse Rex , \\\hline
3.2.23 & ca si por auentura el Rey o el prinçipe \textbf{ a quien pertenesçe de poner las leyes . } si penssassen las condiçiones del que peca perdonar leyen . & vel ipse Princeps \textbf{ cuius est leges ferre si consideraret conditiones peccantis , indulgeret eis : } quare si iudex hoc potest opinari \\\hline
3.2.23 & a quien pertenesçe de poner las leyes . \textbf{ si penssassen las condiçiones del que peca perdonar leyen . } Por la qual cosa & vel ipse Princeps \textbf{ cuius est leges ferre si consideraret conditiones peccantis , indulgeret eis : } quare si iudex hoc potest opinari \\\hline
3.2.23 & Por la qual cosa \textbf{ si el iuez puede esto penssar } que el del ponedor de las leyes penssarie en las leyes & cuius est leges ferre si consideraret conditiones peccantis , indulgeret eis : \textbf{ quare si iudex hoc potest opinari } quod legislator dispensaret in legibus et parceret reo , \\\hline
3.2.23 & e perdonarie al culpado . \textbf{ Mas deue el iues tener manera de mibicordia } que de crueldat contra el culpado . & quod legislator dispensaret in legibus et parceret reo , \textbf{ magis debeat agere } cum eo misericorditer quam crudeliter . \\\hline
3.2.23 & que el iuez \textbf{ mas deue tener mientes al ponedor dela ley } que alas leyes . & Ideo dicitur 1 Rheto’ \textbf{ quod iudicans potius debet respicere | ad legislatorem , } quam ad leges . \\\hline
3.2.23 & por que las leyes \textbf{ para espantar los que yerran contienen en ssi alguna grant asꝑeza . } por la qual cosa si las palabras delas leyes & Tertium inclinans ad pietatem est pius intellectus legum . \textbf{ Leges enim ad terrendum delinquentes } quandam ampliorem seueritatem continent : \\\hline
3.2.23 & e esto es lo que dize el pho enl primero libro de la rectorica \textbf{ que el iuez non deue parar mientes alas palabras delas leyes } mas al entendimientodellas ¶ & hoc est ergo \textbf{ quod dicitur 1 Rhetor’ quod iudicans non debet respicere ad verba legum , } sed ad intellectum legis . Quartum est intentio operantis . \\\hline
3.2.23 & commo muestre la obra \textbf{ e por que las cosas dubdosas son de iudgar ala meior parte } por ende si el iues en alguna manera puede entender & ut opus ostendit : \textbf{ et quia dubia iudicanda sunt in meliorem partem , } si aliquo modo potest percipere iudex peccantem non peccasse ex electione , \\\hline
3.2.23 & e por que las cosas dubdosas son de iudgar ala meior parte \textbf{ por ende si el iues en alguna manera puede entender } que el que peco non peco a sabiendas & et quia dubia iudicanda sunt in meliorem partem , \textbf{ si aliquo modo potest percipere iudex peccantem non peccasse ex electione , } sed ex ignorantia , vel ex infortunio , \\\hline
3.2.23 & mas por lo non sabero \textbf{ por alguna ocasion o por alguna desauentura deuesse inclinar a piedat . } Et por ende dize el pho en el primero libro de la rectorica & sed ex ignorantia , vel ex infortunio , \textbf{ debet ad clementiam declinare ideo dicitur 1 Rhet’ } quod iudicans debet aspicere non ad actionem , \\\hline
3.2.23 & Et por ende dize el pho en el primero libro de la rectorica \textbf{ que el que iudga non deue tener mientesa la obra mas ala entençion . Lo quinto } que enduze el iuez ami bicordia es muchedunbre de buean sobras . & debet ad clementiam declinare ideo dicitur 1 Rhet’ \textbf{ quod iudicans debet aspicere non ad actionem , } sed ad electionem . Quintum inducens ad misericordiam , est multitudo bonorum operum . \\\hline
3.2.23 & Et por ende dize el pho en el primero libro de la rectonca \textbf{ que el uiez non deue catar ala parte } mas al todo & sicut ad totum ut sicut ad multa bona opera quae prius fecit : \textbf{ ideo dicitur 1 Rhetor’ quod iudicans non debet respicere ad partem , } sed ad totum . \\\hline
3.2.23 & mas al todo \textbf{ nin deue tener mientes a vna obra } que fizo & ø \\\hline
3.2.23 & que alas uezes alguno en poco tp̃o faze muchas buenas obras . \textbf{ Et por ende dos cosas deuen endozir al Rey o al prinçipe } o a qual quier otro sennor & etiam in pauco tempore facere multa bona opera : \textbf{ duo ergo debent inducere Regem aut quemcunque alium dominum } ad dilectionem alicuius subditi , \\\hline
3.2.23 & cavno en non sirue a oł \textbf{ si non quando hatpo e oportunidat del seruir . } Et por ende puede contesçer & nam homo non seruit alteri \textbf{ nisi occurrat tempus et opportunitas seruiendi , } potest enim contingere \\\hline
3.2.23 & si non quando hatpo e oportunidat del seruir . \textbf{ Et por ende puede contesçer } que en mui cħtp̃o ouo pocas oportuni dades deur . & nisi occurrat tempus et opportunitas seruiendi , \textbf{ potest enim contingere } quod in multo tempore occurrant opportunitates paucae , \\\hline
3.2.23 & Et por ende puede contesçer \textbf{ que en mui cħtp̃o ouo pocas oportuni dades deur . } e en poco t p̃o o no muchͣ̃s para leruir & potest enim contingere \textbf{ quod in multo tempore occurrant opportunitates paucae , } et in pauco multae . Istud itaque sextum inclinatiuum \\\hline
3.2.23 & que en mui cħtp̃o ouo pocas oportuni dades deur . \textbf{ e en poco t p̃o o no muchͣ̃s para leruir } Et por ende estas esta razon & quod in multo tempore occurrant opportunitates paucae , \textbf{ et in pauco multae . Istud itaque sextum inclinatiuum } ad pietatem respiciens diuturnitatem temporis , \\\hline
3.2.23 & el que fizo bien en todo elt \textbf{ pon passado deue el iuez con el passar mibicordiosamente . } Et en tal cosa commo esta & Quare si contingat aliquem subditorum nunc in aliqua parte temporis delinquere : \textbf{ qui toto tempore se bene habuit praecedenti , } est cum ipso misericorditer agendum , et magis respiciendum est ad multum \\\hline
3.2.23 & Et en tal cosa commo esta \textbf{ mas deue omne tener mientes } alo much & qui toto tempore se bene habuit praecedenti , \textbf{ est cum ipso misericorditer agendum , et magis respiciendum est ad multum } et ad totum tempus praecedens , \\\hline
3.2.23 & Et por ende dize el pho en el primero libro de la rectorica \textbf{ que el ues deue catar } non qual es el acusado en este tpon agora & Ideo dicitur 1 Rhetor’ \textbf{ quod iudex non debet aspicere qualis nunc est incusatus , } sed qualis quidem fuit \\\hline
3.2.23 & ca assi commo el bien sobrepiua el mal \textbf{ e es mas de escoger que el mal } assi es mas de escoger & est excessus bonitatis supra malitiam . \textbf{ Nam sicut bonum excedit malum et est eligibilius ipso , } sic eligibilius est recordari bonorum et gratiarum , \\\hline
3.2.23 & e es mas de escoger que el mal \textbf{ assi es mas de escoger } de menbrar nos de los bienes & ø \\\hline
3.2.23 & assi es mas de escoger \textbf{ de menbrar nos de los bienes } e delas grans & Nam sicut bonum excedit malum et est eligibilius ipso , \textbf{ sic eligibilius est recordari bonorum et gratiarum , } quas suscepimus ab aliquo , \\\hline
3.2.23 & o pecasse contra nos del qual resçibiemos ya en lost pons passados \textbf{ muchs bienes deuemos nos mouer contra el } por mibicordia & Dato ergo aliquem in nos delinquere , \textbf{ a quo temporibus retroactis multa bona suscepimus , } debemus \\\hline
3.2.23 & por mibicordia \textbf{ e deuemos nos mas menbrar del bien } que del auemos resçebido & a quo temporibus retroactis multa bona suscepimus , \textbf{ debemus } ad illum misericorditer nos habere , \\\hline
3.2.23 & por ende el pho en el primero libro de la rectorica \textbf{ quariendo enduzir los iuezes } a misericordia & ad illum misericorditer nos habere , \textbf{ et magis memorari boni suscepti , } quam iniuriae illatae . Ideo Philos’ 1 Rhet’ volens iudicantes ad misericordiam adducere erga delinquentes in ipsos ; \\\hline
3.2.23 & contra los que yerran contra ellos dize \textbf{ que mas se deuen acordar de los biens } que resçibieron & quam iniuriae illatae . Ideo Philos’ 1 Rhet’ volens iudicantes ad misericordiam adducere erga delinquentes in ipsos ; \textbf{ ait } quod magis debent recordari bonorum quae passi sunt a delinquente , \\\hline
3.2.23 & que es acusado \textbf{ ca si alguno fuere acusado de algun pecado del qual deue resçebir pena del iuez } si aquella pena sufriere con paçiençia & quam iniuriae quam fecit . Octauum est patientia incusati . \textbf{ Nam si aliquis incusatur de aliquo delicto pro quo punitur a iudice , } si punitionem patienter sustinet , et non murmurat in poena sibi imposita , \\\hline
3.2.23 & e non murmurare dela pena \textbf{ qual es puesta es de passar contra el } mas misericordi osamente que con otro . & si punitionem patienter sustinet , et non murmurat in poena sibi imposita , \textbf{ est cum illo magis misericorditer agendum . } Ideo dicitur 1 Rhet’ \\\hline
3.2.23 & Et por ende dize el pho en el primero sibro de la rectorica \textbf{ que auemos de perdonar alas obras humanaleᷤ } si contesçiere & Ideo dicitur 1 Rhet’ \textbf{ quod indulgendum est humanis } si contingat patientem patienter esse , \\\hline
3.2.23 & Lo noueno es \textbf{ que deue parar mientes el uies } al que peca & qui poenam patitur , \textbf{ eam patienter sufferre . Nonum est corrigibilitas peccantis . } Nam sunt aliqui ita corrigibiles \\\hline
3.2.23 & al que peca \textbf{ si se puede castigar o non . } ca son alguons & ø \\\hline
3.2.23 & e por la sola palabrase meioran \textbf{ e dexan de fazer mal } e atalon deuemos mucho perdonar . & idest solo sermone meliorantur \textbf{ et desinunt praua agere : } talibus ergo est valde indulgendum , \\\hline
3.2.23 & e dexan de fazer mal \textbf{ e atalon deuemos mucho perdonar . } e tales son detractar muy benigna mente . & et desinunt praua agere : \textbf{ talibus ergo est valde indulgendum , } et tales sunt valde benigne tractandi . Ideo dicitur 1 Rhet’ \\\hline
3.2.23 & e atalon deuemos mucho perdonar . \textbf{ e tales son detractar muy benigna mente . } Et por ende dize el philosofo en el primero libro dela rectorica & talibus ergo est valde indulgendum , \textbf{ et tales sunt valde benigne tractandi . Ideo dicitur 1 Rhet’ } quod iudex \\\hline
3.2.23 & Et por ende dize el philosofo en el primero libro dela rectorica \textbf{ que el uiez deue perdonar alas obras humanales } si creyere que el que peca quiete ser mas iudgado & et tales sunt valde benigne tractandi . Ideo dicitur 1 Rhet’ \textbf{ quod iudex | debet indulgere humanis , } si credat magis peccantem iudicari velle sermone quam opere . \\\hline
3.2.23 & e se homilla e se pone todo en el aluedrio del iuez \textbf{ deuen passar contra el mas manssamente . } Et por ende dize el philosofo en el primero libro dela rectorica & et ponit se totaliter in arbitrio iudicantis , \textbf{ est cum eo mitius agendum . } Ideo dicitur 1 Rhet’ \\\hline
3.2.23 & que el iese pieques \textbf{ que quiere dezer } mas que iusto deue perdonar a las cosas humanales & Ideo dicitur 1 Rhet’ \textbf{ quod Iudex epiikis idest superiustus debet indulgere humanis , } si viderit delinquentem magis velle ire ad arbitrium , \\\hline
3.2.23 & que quiere dezer \textbf{ mas que iusto deue perdonar a las cosas humanales } si uiere & Ideo dicitur 1 Rhet’ \textbf{ quod Iudex epiikis idest superiustus debet indulgere humanis , } si viderit delinquentem magis velle ire ad arbitrium , \\\hline
3.2.23 & mas al aluedrio del iuez \textbf{ que non a escusar se } e a disputar con el . & si viderit delinquentem magis velle ire ad arbitrium , \textbf{ quam ad disceptationem : } omnino enim contra rationem est humilitati non parcere , \\\hline
3.2.23 & que non a escusar se \textbf{ e a disputar con el . } ca en toda manera es contra razon & si viderit delinquentem magis velle ire ad arbitrium , \textbf{ quam ad disceptationem : } omnino enim contra rationem est humilitati non parcere , \\\hline
3.2.23 & ca en toda manera es contra razon \textbf{ de non perdonar al que se homilla . } ca las bestias perdonan & quam ad disceptationem : \textbf{ omnino enim contra rationem est humilitati non parcere , } cum bestiae hoc agant : \\\hline
3.2.23 & que contra los que son homillosos \textbf{ deue quedar la sanna esto se muestra avn por las bestias crueles obravis } que non muerden a aquellos que yazen homillosamente ante ellos & quod autem ad humiliantes cesset ira \textbf{ etiam canes manifestant non mordentes eos } qui resident . \\\hline
3.2.23 & mas conuiene alos Reyes e alos prinçipes \textbf{ alos quales conuiene de resplandesçer } por mayor bondat . & multo magis decet Reges et Principes , \textbf{ quibus congruit ampliori bonitate pollere . } Decet \\\hline
3.2.23 & e la paz del regno \textbf{ en quanto puede ser deuen se inclinar a mi bicordia } Mas en qual manera con la piadat pueda estar la iustiçia adelante parezcra & sed saluato communi bono \textbf{ et pace regni quantum possibile est debent ad misericordiam declinare . } Qualiter autem clementia possit stare cum iustitia , infra patebit . \\\hline
3.2.24 & que nos departimos el derecho o la cosa \textbf{ e usta podemos departir las leyes } e por el contrario por las leyes podemos deꝑtir & ius siue iustum , \textbf{ distinguere possumus leges ipsas , } et econuerso . \\\hline
3.2.24 & e usta podemos departir las leyes \textbf{ e por el contrario por las leyes podemos deꝑtir } que cosa es derecho & ius siue iustum , \textbf{ distinguere possumus leges ipsas , } et econuerso . \\\hline
3.2.24 & Mas nos podemos tan bien dela ley \textbf{ commo del derecho fazer çinco departimientos } de los quales los dos pone el pho enel primero libro de la rectorica . & et econuerso . \textbf{ Possumus autem tam de lege quam de iusto quinque distinctiones facere , } quarum duae tanguntur 1 Rhet’ \\\hline
3.2.24 & El quarto ponen los iuristas . \textbf{ Et el quinto podemos nos mesmos en nader . } Ca primeramente el derecho se departe en esta manera & tertia ponitur 5 Ethic’ quarta traditur a Iuristis , \textbf{ quintam nos ipsi superaddere possumus . Distinguitur enim ius , } quia quoddam est scriptum , et quoddam non scriptum : \\\hline
3.2.24 & que los iuristas apartan el derecho natural del derecho delas gentes \textbf{ podemos nos apartar el derecho natural del derecho delas animalias } e darla quanta distinçion & ius naturale a iure gentium , \textbf{ possemus separare nos ius naturale a iure animalium : } et dare quintam distinctionem iuris , \\\hline
3.2.24 & podemos nos apartar el derecho natural del derecho delas animalias \textbf{ e darla quanta distinçion } e el quinto departimiento del derech̃ . & possemus separare nos ius naturale a iure animalium : \textbf{ et dare quintam distinctionem iuris , } dicendo quod quadruplex est ius , \\\hline
3.2.24 & diziendo que en quatro maneras se departe el derech . \textbf{ Conuiene a saber ende recħ natural } e en derecho delas ainalias & et dare quintam distinctionem iuris , \textbf{ dicendo quod quadruplex est ius , } videlicet naturale , animalium , gentium , \\\hline
3.2.24 & Et estos çinco departimientos \textbf{ que fiziemos del derecho dela cosa derechͣ podemos fazer dela ley . } Et por ende & et ciuile . Has ergo quinque distinctiones , \textbf{ quas fecimus de iure siue de iusto , facere possumus de ipsa lege . Ut ergo haec omnia melius patefiant , } et ut has diuersitates ad concordiam reducamus sciendum \\\hline
3.2.24 & Et por que estos departimientos adugamos a concordia . \textbf{ conuiene de saber } que la cosa derechurera es en dos maneras . & quas fecimus de iure siue de iusto , facere possumus de ipsa lege . Ut ergo haec omnia melius patefiant , \textbf{ et ut has diuersitates ad concordiam reducamus sciendum } quod duplex est iustum , \\\hline
3.2.24 & Et la ley es en dos maneras . \textbf{ Conuiene a saber natural e positiua puesta por omne . } Ca los derechsson dichos naturales & quod duplex est iustum , \textbf{ vel duplex est lex , naturalis , } et positiua . \\\hline
3.2.24 & assi o en otra manera . \textbf{ mas despues que es puesto a fuerça de obligar alos omes . } Mas la razon & nihil differt esse sic vel aliter , \textbf{ postquam autem est editum incipit habere ligandi efficaciam . Ratio autem , } quare iuri naturali oportuit superaddere positiuum , est : quia multa sunt sic iusta naturaliter , \\\hline
3.2.24 & Mas la razon \textbf{ por que al derech natural conuinio anneder derecho positiuo es esta } por que muchas cosas son derechas naturalmente & postquam autem est editum incipit habere ligandi efficaciam . Ratio autem , \textbf{ quare iuri naturali oportuit superaddere positiuum , est : quia multa sunt sic iusta naturaliter , } sicut est naturale homini loqui : \\\hline
3.2.24 & por que muchas cosas son derechas naturalmente \textbf{ assi commo natural cosa es al ome de fablar } ca auemos natural apetito e natural inclinacion para fablar & quare iuri naturali oportuit superaddere positiuum , est : quia multa sunt sic iusta naturaliter , \textbf{ sicut est naturale homini loqui : } habemus enim naturalem impetum \\\hline
3.2.24 & assi commo natural cosa es al ome de fablar \textbf{ ca auemos natural apetito e natural inclinacion para fablar } e para manifestar a otri & sicut est naturale homini loqui : \textbf{ habemus enim naturalem impetum | et naturalem inclinationem ut loquamur , } et ut per sermonem manifestemus \\\hline
3.2.24 & ca auemos natural apetito e natural inclinacion para fablar \textbf{ e para manifestar a otri } lo que conçebimos en la uoluntad por la palabra & et naturalem inclinationem ut loquamur , \textbf{ et ut per sermonem manifestemus } alteri quod mente concepimus : \\\hline
3.2.24 & Et pues que assi es \textbf{ assi commo fablar es cosa natural alos omes } assi fablar tal lenguaie & fures punire , \textbf{ non pati maleficos viuere , } et cetera huiusmodi sunt , \\\hline
3.2.24 & assi commo fablar es cosa natural alos omes \textbf{ assi fablar tal lenguaie } o otro es cosa positiua & fures punire , \textbf{ non pati maleficos viuere , } et cetera huiusmodi sunt , \\\hline
3.2.24 & e esa uoluntad de los omes . \textbf{ bien assi dar pena alos ladrones } e non sofrir beuir los malos & et cetera huiusmodi sunt , \textbf{ de iure naturali , } quia haec esse fienda dictat ratio naturalis , \\\hline
3.2.24 & bien assi dar pena alos ladrones \textbf{ e non sofrir beuir los malos } e o tristales cosas son de derechnatural & et cetera huiusmodi sunt , \textbf{ de iure naturali , } quia haec esse fienda dictat ratio naturalis , \\\hline
3.2.24 & e o tristales cosas son de derechnatural \textbf{ por que la razon natural muestra que se deuen fazer . } Et auemos natural inclinaçion & de iure naturali , \textbf{ quia haec esse fienda dictat ratio naturalis , } et habemus naturalem impetum \\\hline
3.2.24 & Et pues que assi es en aquel logar ose termina el derecho natural \textbf{ alli comiença a naçer el derech posituio } por que sienpre aquellas cosas que son falladas & Ubi ergo terminatur ius naturale , \textbf{ ibi incipit oriri ius positiuum : } quia semper quae sunt per artem hominum adinuenta fundantur in his quae tradita sunt a natura , \\\hline
3.2.24 & ¶ Esto uisto \textbf{ quanto pertenesçe alo presente podemos mostrar dos departimientos } e dos diferençias entre el derech natural e el positiuo . & determinans qua poena sint talia punienda . Hoc viso \textbf{ quantum ad praesens spectat duplicem differentiam assignare possumus } inter ius naturale , \\\hline
3.2.24 & Enpero cada vno destos dos derechs \textbf{ tan bien el natural commo el positiuo se puede escͥuir en algun libro } Enpero non es tanto meester de se escuir el natural commo el positiuo & oportuit ipsum scribi in aliqua exteriori substantia . \textbf{ Potest itaque utrunque ius scribi in aliqua exteriori substantia tam naturale quam positiuum : | naturale } tamen non sic indiget \\\hline
3.2.24 & tan bien el natural commo el positiuo se puede escͥuir en algun libro \textbf{ Enpero non es tanto meester de se escuir el natural commo el positiuo } por que non pue de assi caer dela memoria el natural commo el positiuo ¶ & naturale \textbf{ tamen non sic indiget | ut scribatur } sicut positiuum , \\\hline
3.2.24 & Enpero non es tanto meester de se escuir el natural commo el positiuo \textbf{ por que non pue de assi caer dela memoria el natural commo el positiuo ¶ } La segunda diferençia es que el derecho narurales vno a todo los omes . & ut scribatur \textbf{ sicut positiuum , | nam non sic potest a memoria recedere } sicut illud . \\\hline
3.2.24 & que continuadamente se estiende en claridat \textbf{ que se non puede mesurar . } Ca el fuego en su es para prõa & appellat ius naturale aetherem siue ignem , \textbf{ qui continuat protendere per inexplicabilem claritatem . Ignis enim in propria sphaera } quia plus se diffundit \\\hline
3.2.24 & e ser legal e de ley . \textbf{ Mas que deuemos sentir e iudgar del derecho delas gentes } e del derecho delas ainalias & et legale . \textbf{ Quid autem sentiendum sit de iure gentium , } et de iure animalium , \\\hline
3.2.25 & assi commo es el derecho delas gentes . \textbf{ Et segunt esta manera de fablar podemos ennader el quarto mienbro } que es derecho delas aian lias . & ut ius gentium : \textbf{ secundum quem modum loquendi potest ibi addi membrum quartum , } ut ius animalium . \\\hline
3.2.25 & que es derecho delas aian lias . \textbf{ Et para declaraçion desto conuiene de saber } que el omne en quanto es omne e penssando segunt su razon proprea & ut ius animalium . \textbf{ Ad cuius euidentiam sciendum quod homo } ut est homo et secudum \\\hline
3.2.25 & que el omne en quanto es omne e penssando segunt su razon proprea \textbf{ que es auer entendimiento e razon departesse delas otras aina } las que non han entendimiento . & Ad cuius euidentiam sciendum quod homo \textbf{ ut est homo et secudum } propriam rationem consideratus differt ab animalibus aliis , \\\hline
3.2.25 & Ca las otras aiquelias natraalmente son inclinadas \textbf{ para que los mas los se ayunten alas fenbras } e engendren fijos & nam et animalia naturaliter inclinantur , \textbf{ ut masculi coniungantur foeminis , } ut filios generent , \\\hline
3.2.25 & que son entre los omes \textbf{ assi commo conprar e vender logar de casas } e al qual es et otras cosas tales & Ex hoc ergo iure pene omnes contractus sunt introducti , \textbf{ ut emptio , | venditio , locatio , conductio } et cetera talia , \\\hline
3.2.25 & aquello que ha mester para la uida . \textbf{ Et por ende a este mismo derecho parte nesçe el enprestar } e el guardar & sine quibus societas humana non bene sufficit sibi ad vitam . \textbf{ Inde est ergo quod mutuum } et depositio , \\\hline
3.2.25 & Et por ende a este mismo derecho parte nesçe el enprestar \textbf{ e el guardar } e otras cosas tales & Inde est ergo quod mutuum \textbf{ et depositio , } quae etiam deseruiunt \\\hline
3.2.25 & e mas conosçido \textbf{ e menos se puede mudar } tanto mas meresçe de auer nonbre derech natural . Et por ende aquel derecho & Quanto ergo ius aliquod est communius , \textbf{ quam ius gentium : } tanto magis meretur nomen iuris naturalis . \\\hline
3.2.25 & e menos se puede mudar \textbf{ tanto mas meresçe de auer nonbre derech natural . Et por ende aquel derecho } que la natura mostro a todas las aianlas & quam ius gentium : \textbf{ tanto magis meretur nomen iuris naturalis . | Ius ergo , } quod omnia animalia docuit \\\hline
3.2.25 & Avn este derecho tal que es derecho delas aian lias \textbf{ menos se puede mudar } que los otros derechs . & tanto est intellectui nostro notius , et prius cadit in apprehensione nostra . Est etiam huiusmodi ius immutabilius , quia regulae iuris quanto magis applicantur ad materiam specialem , \textbf{ tanto plures defectus contrahunt , } et in pluribus casibus \\\hline
3.2.25 & quanto mas se allegan a algua materia espeçial tanto mas trahe consigo alguons desfallesçimientos \textbf{ e en muchos casos non son de guardar } e resçiben mayor mudamiento . & et in pluribus casibus \textbf{ non sunt obseruandae , } et maiorem mutationem suscipiunt : \\\hline
3.2.25 & Visto en qual manera el derecho delas gentes se departe \textbf{ del derecho natal de ligo puede paresçer } en qual manera el derecho delas aialias se departe del derech̉natiral . & Viso quomodo ius gentium differt a iure naturali , \textbf{ de leui patere potest quomodo ius animalium differt a iure naturali . } Nam sicut humana natura in quantum animal est , \\\hline
3.2.25 & que han ser . \textbf{ Et por ende la inclinacion natural puede seguir la natura del ome } o en quanto es omne & et cum substantiis aliis , \textbf{ et cum entibus omnibus . Poterit ergo inclinatio naturalis sequi naturam hominis } vel ut homo est , \\\hline
3.2.25 & Ra qual cola avn del sean todas las otras cosas que son . \textbf{ avn el omne natutalmente dessea de auer fijos } e de criar los . & quod et omnia entia alia appetunt : \textbf{ naturaliter appetit producere filios , } educare prolem , \\\hline
3.2.25 & avn el omne natutalmente dessea de auer fijos \textbf{ e de criar los . } Ca esto dessean todas las otras aina las natraalmente & naturaliter appetit producere filios , \textbf{ educare prolem , } quod et alia animalia concupiscunt : \\\hline
3.2.25 & en quanto la natura humanal es alguna sub̃a \textbf{ e haser } e conuiene con todas las cosas & sic huiusmodi regulae poterunt esse de iure naturali , \textbf{ prout natura humana est quaedam entitas , } et conuenit cum entibus omnibus . \\\hline
3.2.25 & que son e con todas las sustançias . \textbf{ Mas si aquellas reglas se tomaren en quanto el omne naturalmente dessea fazer fiios e carlos } assi podria ser de derecho natural & et conuenit cum entibus omnibus . \textbf{ Si vero regulae | illae sumantur ex eo quod homo naturaliter appetit filios producere et educare : } sic esse poterunt de iure naturali , \\\hline
3.2.25 & e es mas comun qual otro . \textbf{ Ca dessear bien } e dessear ser & et communius illo : \textbf{ nam appetere bonum et esse , } et fugere malum et non esse , \\\hline
3.2.25 & Ca dessear bien \textbf{ e dessear ser } e foyr el mal & et communius illo : \textbf{ nam appetere bonum et esse , } et fugere malum et non esse , \\\hline
3.2.25 & e dessear ser \textbf{ e foyr el mal } e foyr del non seres & nam appetere bonum et esse , \textbf{ et fugere malum et non esse , } est plus de iure naturali , \\\hline
3.2.25 & e foyr el mal \textbf{ e foyr del non seres } mas de derech natural & nam appetere bonum et esse , \textbf{ et fugere malum et non esse , } est plus de iure naturali , \\\hline
3.2.25 & mas de derech natural \textbf{ que dessear de engendrar fijos e criar los . Et pues que } assi es esta sera la orden entre estos de ti xu rechos & est plus de iure naturali , \textbf{ quam appetere procreare filios , et nutrire prolem . } Erit igitur hic ordo , \\\hline
3.2.25 & que es dicho natural pora un ataia de los otros derechos . \textbf{ Por que dessear el bien e el ser } e foyr el non ser & sic habet esse ius illud , \textbf{ quod per antonomasiam dicitur esse naturale . Appetere enim esse et bonum , } et fugere non esse et malum , \\\hline
3.2.25 & Por que dessear el bien e el ser \textbf{ e foyr el non ser } e el mal es aquello que desseamos naturalmente & quod per antonomasiam dicitur esse naturale . Appetere enim esse et bonum , \textbf{ et fugere non esse et malum , } quod naturaliter appetimus , \\\hline
3.2.25 & e en el se fuudaron . \textbf{ Ca en todos los derechos es entendido o alcançar } el bien o foyr del mal . & et in eo fundantur : \textbf{ nam in omnibus attenditur | vel consecutio boni , } vel fuga mali . \\\hline
3.2.25 & Ca en todos los derechos es entendido o alcançar \textbf{ el bien o foyr del mal . } Mas tractar esto & vel consecutio boni , \textbf{ vel fuga mali . } Sed hoc diffusius per tractare , \\\hline
3.2.25 & el bien o foyr del mal . \textbf{ Mas tractar esto } mas conplidamente pertenesçe a otra sçiençia . & vel fuga mali . \textbf{ Sed hoc diffusius per tractare , } alterius exposcit negocium . Sufficiat autem ad praesens scire , \\\hline
3.2.25 & mas conplidamente pertenesçe a otra sçiençia . \textbf{ Mas cunpla quanto alo presente de saber } en qual manera el derech delas genteᷤ & Sed hoc diffusius per tractare , \textbf{ alterius exposcit negocium . Sufficiat autem ad praesens scire , } quomodo ius gentium , \\\hline
3.2.26 & Saresçe que derecho çiuil o el derecho humanal e positiuo es conparado a tres cosas . \textbf{ Con uiene a saber . } al derecho natural o ala ley dela natura & siue ius humanum et positiuum ad tria comparari , \textbf{ videlicet ad ius naturale siue ad legem naturalem , } a qua suscipit fundamentum : \\\hline
3.2.26 & e fuidamiento . \textbf{ Et puede se conparar al bien comun } que se entiende enl ła & a qua suscipit fundamentum : \textbf{ ad bonum commune quod in ea intenditur : } et ad gentem ad quam applicatur , \\\hline
3.2.26 & que se entiende enl ła \textbf{ e puede se conparar ala gente } ala que es dada . & ad bonum commune quod in ea intenditur : \textbf{ et ad gentem ad quam applicatur , } quae per illam legem est regulanda . Tria igitur lex habere debet , \\\hline
3.2.26 & Et pues que al sy es tres cosas \textbf{ deue auer la ley } en quanto es conparada a estas tres cosas . & ø \\\hline
3.2.26 & Ca en las obras \textbf{ que fazemos algunan cosa auemos de dar ala costunbre } e alguna cosa ala tierra . & quod sit competens \textbf{ et compossibilis consuetudini patriae et tempori : } nam in agibilibus aliquid dandum est consuetudini , tempori , et patriae et moribus hominum : \\\hline
3.2.26 & e esto queremos \textbf{ alcançar conuiene que fagamos estas cosas . } Et pues que assi estales deuen ser las leyes & et hoc sequi volumus , \textbf{ oportet hoc agere . } Tales ergo debent esse leges , \\\hline
3.2.26 & al quien es dada deue ser \textbf{ tal que se pueda guardar } e sea conuenible ala tierra & ad populum cui est imponenda , \textbf{ debet esse compossibilis et competens regioni , } et consuetudini , \\\hline
3.2.26 & Et por ende dize el philosofo en el quarto libro delas politicas \textbf{ que non conuiene de apropar las comunidades delas çibdades alas leyes . } Mas las leyes alas comunidades de las çibdades & et consuetudini , \textbf{ et moribus illius gentis . Ideo dicitur 4 Politicorum } quod non oportet adaptare politias legibus , sed leges politiae , quas leges oportet diuersas esse \\\hline
3.2.26 & segunt el departimiento delas comunidades . \textbf{ Et pues que assi es el que quisiere poner leyes con grand acuçia } deue tener mientes & secundum diuersitatem politiarum . \textbf{ Volens ergo leges ferre , | diligenter debet attendere , } qualis sit populus , \\\hline
3.2.26 & Et pues que assi es el que quisiere poner leyes con grand acuçia \textbf{ deue tener mientes } qual es el pueblo & diligenter debet attendere , \textbf{ qualis sit populus , } cuius ritus et conditionis ; \\\hline
3.2.26 & Et assi commo viere \textbf{ que cunple al pueblo tales leyes les deue poner . } Voisto quales leyes los Reyes & et prout eis viderit expedire , \textbf{ tales debet eis leges imponere . } Viso quales leges Reges et Principes deceat ponere , \\\hline
3.2.26 & Voisto quales leyes los Reyes \textbf{ e los prinçipes deuen poner Ca deuen poner buenas e aprouechables } e parte nesçientes al pueblo & Viso quales leges Reges et Principes deceat ponere , \textbf{ quia condendae sunt leges , | quae sunt iustae , } utiles \\\hline
3.2.26 & e parte nesçientes al pueblo \textbf{ al qual son puestas de ligero puede paresçer } que conuiene al regno e a la çibdat & et competens populo , \textbf{ cui imponuntur : | de leui patere potest , } quod expedit regno et ciuitati huiusmodi leges condere . \\\hline
3.2.26 & que conuiene al regno e a la çibdat \textbf{ de pouer tales leyes . } Ca algs de los omes son de ssi bueons & de leui patere potest , \textbf{ quod expedit regno et ciuitati huiusmodi leges condere . } Nam aliqui sunt de se boni , \\\hline
3.2.26 & para fazerobras uirtuosas \textbf{ e para dexar de fazer obras malas . } Mas alguons otros son tales & ut agant opera virtuosa , \textbf{ et desistant agere peruerse . Aliqui vero si ex seipsis , } id est , \\\hline
3.2.26 & que han en ssi mesmos non son conplidamente inclinados \textbf{ para dexar el mal } e fazer el bien . & id est , \textbf{ ex habitibus quos habent in seipsis non sufficienter inclinantur ad bonum , } attamen sunt \\\hline
3.2.26 & para dexar el mal \textbf{ e fazer el bien . } Enpero son de ligero disçiplinables e corrigibles & ex habitibus quos habent in seipsis non sufficienter inclinantur ad bonum , \textbf{ attamen sunt } facile disciplinabiles per alios , \\\hline
3.2.26 & e los castigos solos los inclinan \textbf{ para faz̃ bien . } Mas otros son tan malos e tan desiguales & ita quod soli sermones \textbf{ et solae increpationes inclinant eos ad bonum . } Sed aliqui sunt adeo peruersi , \\\hline
3.2.26 & han uirtud \textbf{ e poderio de costrennir los malos . } Et pues que assi es las leyes son establesçidas & ø \\\hline
3.2.26 & Et pues que assi es las leyes son establesçidas \textbf{ e conuenia delas establesçer } por que si quier por miedo dela pena los que quesiessen enbargar la paz de los çibdadanos & quae ( ut dicitur 10 \textbf{ Ethicorum ) coactiuam habent potentiam . Condendae igitur sunt leges , et expediebat eas statuere : } ut saltem metu poenae volentes impedire pacem ciuium , desisterent agere peruerse . \\\hline
3.2.26 & e conuenia delas establesçer \textbf{ por que si quier por miedo dela pena los que quesiessen enbargar la paz de los çibdadanos } dexassen de obrar mal & Ethicorum ) coactiuam habent potentiam . Condendae igitur sunt leges , et expediebat eas statuere : \textbf{ ut saltem metu poenae volentes impedire pacem ciuium , desisterent agere peruerse . } Leges ( ut patet per habita ) \\\hline
3.2.26 & por que si quier por miedo dela pena los que quesiessen enbargar la paz de los çibdadanos \textbf{ dexassen de obrar mal } as leyes assi commo paresçe & Ethicorum ) coactiuam habent potentiam . Condendae igitur sunt leges , et expediebat eas statuere : \textbf{ ut saltem metu poenae volentes impedire pacem ciuium , desisterent agere peruerse . } Leges ( ut patet per habita ) \\\hline
3.2.27 & por las cosas ya dichas son algunas reglas delas obras \textbf{ que auemos de fazer } que nos ordenan a bien comun & sunt quaedam regulae agibilium , \textbf{ ordinantes nos in commune bonum , } habentes coactiuam potentiam . \\\hline
3.2.27 & que nos ordenan a bien comun \textbf{ e han poder de nos costrennir . } Et pues que assi es por dos razones podemos prouar & ordinantes nos in commune bonum , \textbf{ habentes coactiuam potentiam . } Duplici ergo via venari possumus , \\\hline
3.2.27 & e han poder de nos costrennir . \textbf{ Et pues que assi es por dos razones podemos prouar } que non parte nesçe aquel & habentes coactiuam potentiam . \textbf{ Duplici ergo via venari possumus , } quod non cuiuslibet est leges condere . Prima via sumitur \\\hline
3.2.27 & que non parte nesçe aquel \textbf{ si quier de fazer leyes ¶ } La primera se toma de aqual & Duplici ergo via venari possumus , \textbf{ quod non cuiuslibet est leges condere . Prima via sumitur } ex eo quod leges nos ordinant ad commune bonum . \\\hline
3.2.27 & lo que las leyes nos ordenan a bien comun . \textbf{ La segunda de aquello que han uirtud e poderio de costrennir ¶ } La primera razon assi . & ex eo quod leges nos ordinant ad commune bonum . \textbf{ Secundo ex eo quod habent potentiam coactiuam . } Prima via sic patet , \\\hline
3.2.27 & La primera razon assi . \textbf{ Ca aquel cuyo es de ordenar } e enderesçar a alguos en algun bien atlgun aquel parte nesçe establesçer leyes e reglas delas nuestras obras . & Prima via sic patet , \textbf{ nam | cuius est ordinare } et dirigere aliquos in aliquod bonum , \\\hline
3.2.27 & Ca aquel cuyo es de ordenar \textbf{ e enderesçar a alguos en algun bien atlgun aquel parte nesçe establesçer leyes e reglas delas nuestras obras . } por las quales leyes ymosa aquel bien . & cuius est ordinare \textbf{ et dirigere aliquos in aliquod bonum , } eiusdem est condere leges , \\\hline
3.2.27 & deuen ser establesçidas del prinçipe \textbf{ a quien parte nesçe ordenar } e enderesçar los otros atal bien & condendae sunt a Principe \textbf{ cuius est ordinare et dirigere alios in tale bonum , } vel condendae sunt a toto populo , \\\hline
3.2.27 & a quien parte nesçe ordenar \textbf{ e enderesçar los otros atal bien } o deuen ser establesçidas de todo el pueblo & condendae sunt a Principe \textbf{ cuius est ordinare et dirigere alios in tale bonum , } vel condendae sunt a toto populo , \\\hline
3.2.27 & si todo el pueblo en ssennorea \textbf{ e si en su poder es de escoger el prinçipe . Et pues que assi es la ley non es ninguna } si non es establesçida & si totus populus principetur , \textbf{ et sit in potestate eius eligere principantem : | Nulla est ergo lex , } quae non sit edita ab eo cuius est dirigere in bonum commune : \\\hline
3.2.27 & si non es establesçida \textbf{ por aquel a quien parte nesçe de enderesçar los omes al bien comun . } Ca si es ley diuinal e natural establesçida es de dios a quienꝑtenesçe enderesçar todas las cosas asimesmo . & Nulla est ergo lex , \textbf{ quae non sit edita ab eo cuius est dirigere in bonum commune : } nam si est lex diuina et naturalis , \\\hline
3.2.27 & por aquel a quien parte nesçe de enderesçar los omes al bien comun . \textbf{ Ca si es ley diuinal e natural establesçida es de dios a quienꝑtenesçe enderesçar todas las cosas asimesmo . } El qual mayormente es bien comun & quae non sit edita ab eo cuius est dirigere in bonum commune : \textbf{ nam si est lex diuina et naturalis , | condita est a Deo cuius est omnia dirigere in seipsum , } qui maxime est commune bonum , \\\hline
3.2.27 & Ca el prinçipe o avn todo el pueblo \textbf{ quando enssennorea ha de ordenar } e de enderesçar todos los otros al bien comun . & Princeps enim aut totus populus cum principatur , \textbf{ habet dirigere et ordinare alios in commune bonum . } Quaelibet ergo persona particularis , \\\hline
3.2.27 & quando enssennorea ha de ordenar \textbf{ e de enderesçar todos los otros al bien comun . } Et pues que assi es cada vna perssena singular & Princeps enim aut totus populus cum principatur , \textbf{ habet dirigere et ordinare alios in commune bonum . } Quaelibet ergo persona particularis , \\\hline
3.2.27 & mas non par tenesçe a \textbf{ qual si quier de establesçer leyes . } La segunda razon & debet esse obseruatiua legum , \textbf{ sed non cuiuslibet est leges condere . } Secunda via ad inuestigandum hoc idem , \\\hline
3.2.27 & La segunda razon \textbf{ para prouar esto mesmo se toma de aquello que las leyes han poderio de costrennir . } Ca puede cada vno del pueblo amonestar e enduzir a otro o a otros & Secunda via ad inuestigandum hoc idem , \textbf{ sumitur ex eo quod leges coactiuam habent potentiam . } Potest enim quilibet ex populo mouere \\\hline
3.2.27 & para prouar esto mesmo se toma de aquello que las leyes han poderio de costrennir . \textbf{ Ca puede cada vno del pueblo amonestar e enduzir a otro o a otros } que fagan bien . & sumitur ex eo quod leges coactiuam habent potentiam . \textbf{ Potest enim quilibet ex populo mouere } et persuadere alteri ut bene agat , \\\hline
3.2.27 & e estas condiçiones tales non son dich̃ͣs leyes \textbf{ por que non han ningun poderio para costrennir . } mas estendiendo e alargando el nonbre dela ley & sed huiusmodi monitiones \textbf{ et persuasiones non dicuntur leges , quia nihil habent coactiuum . } Extendendo autem nomen legis , \\\hline
3.2.27 & e quales si quier amonestamientos pueden ser dichas leyes \textbf{ segunt la qual manera de fablar } assi commo cuenta el philosofo & et quaelibet monitiones leges dici possunt . \textbf{ Secundum quem modum loquendi } ( ut recitat Philosophus 1 Politicor’ ) \\\hline
3.2.27 & dizia omero \textbf{ que cada vno podia fazer leyes } a sus fijos & dicebat Homerus \textbf{ quod unusquisque statuit legis pueris et uxoribus : } volebat enim Homerus quod monitiones et praecepta , \\\hline
3.2.27 & siguesse \textbf{ que non son de tomar las leyes } e las reglas delas nuestras obras & sed a bono quod intenditur in regno \textbf{ et ciuitate sumendae sunt leges et regulae agibilium . } Immo quia vim agendi \\\hline
3.2.27 & que es entendido en el regno e en la çibdat \textbf{ son de tomar las leyes } e las reglas delas nuestras obras . & Immo quia vim agendi \textbf{ et potentiam exterminandi maleficos non habet } proprie paterfamilias \\\hline
3.2.27 & nin ninguna perssona priuada \textbf{ non ha poder de costrennir } nin poder de matar los malfechores . & ø \\\hline
3.2.27 & non ha poder de costrennir \textbf{ nin poder de matar los malfechores . } mas este poder ha el prinçipe & nec persona priuata , \textbf{ sed Princeps } et persona publica , \\\hline
3.2.27 & Por ende las leyes \textbf{ que han poderio de costrennir } non son de establesçer & aut alii multitudini , ideo leges habentes coactiuam potentiam non sunt condendae nisi ab eo \textbf{ qui praeest ipsi multitudini , } vel a multitudine , \\\hline
3.2.27 & que han poderio de costrennir \textbf{ non son de establesçer } si non de aquel & aut alii multitudini , ideo leges habentes coactiuam potentiam non sunt condendae nisi ab eo \textbf{ qui praeest ipsi multitudini , } vel a multitudine , \\\hline
3.2.27 & si tal muchedunbre enssennoreare ¶ \textbf{ Visto que non parte nesçe a cada vno establesçer las leyes de ligo puede paresçer } que la ley non ha uirtud de obligar a & si tota huiusmodi multitudo principetur . \textbf{ Viso quod non est cuiuslibet leges condere , } de leui potest patere legem non habere vim obligandi , \\\hline
3.2.27 & Visto que non parte nesçe a cada vno establesçer las leyes de ligo puede paresçer \textbf{ que la ley non ha uirtud de obligar a } ningnon si non fuere obligada & Viso quod non est cuiuslibet leges condere , \textbf{ de leui potest patere legem non habere vim obligandi , } nisi sit promulgata . \\\hline
3.2.27 & Poque la ley aya uirtud \textbf{ e fuerça de obligar } conuiene que sea publicada e pregonada . & vel peruenire potuerit ad notitiam subditorum , \textbf{ ad hoc quod lex habeat vim obligandi , } oportet eam promulgatam esse . \\\hline
3.2.27 & Mas commo otra sea la ley natural e otra la positiua \textbf{ en vna manera se deue publicar la vna } e en otra manera la otra . & Sed cum alia sit lex naturalis , alia positiua : \textbf{ aliter propalatur haec , } aliter illa . \\\hline
3.2.27 & que en cada vn omne es publicada e manifestada \textbf{ quando comiença de auer uso de razon e de entendimiento } por el qual conosçe qual cosa ha de fazer & et propalatur , \textbf{ quando incipit habere rationis usum , } per quam cognoscit quid sequendum \\\hline
3.2.27 & quando comiença de auer uso de razon e de entendimiento \textbf{ por el qual conosçe qual cosa ha de fazer } e de escoger & quando incipit habere rationis usum , \textbf{ per quam cognoscit quid sequendum } et quid fugiendum , \\\hline
3.2.27 & por el qual conosçe qual cosa ha de fazer \textbf{ e de escoger } e qual cosa ha de foyr & ø \\\hline
3.2.27 & e de escoger \textbf{ e qual cosa ha de foyr } e de escusar & per quam cognoscit quid sequendum \textbf{ et quid fugiendum , } secundum quod haec pertinent \\\hline
3.2.27 & e qual cosa ha de foyr \textbf{ e de escusar } segunt que esto pertenesçe al derecho natural . & per quam cognoscit quid sequendum \textbf{ et quid fugiendum , } secundum quod haec pertinent \\\hline
3.2.27 & que sean muy acuçiosos cerca las leyes \textbf{ que han de poner al pueblo } a quien enssennorena . & decet Reges et Principes \textbf{ non modicum solicitari quas leges imponant populo } cui dominantur . \\\hline
3.2.27 & Et despues que cuydaren estas leyes \textbf{ que han de poner con acuçia . } por que tales leyes ayan uirtud & cui dominantur . \textbf{ Et postquam excogitauerunt leges imponendas , } ut huiusmodi leges vim obligandi habeant , \\\hline
3.2.27 & por que tales leyes ayan uirtud \textbf{ e fuerça de costrennir } e de obligar deuen las publicar & Et postquam excogitauerunt leges imponendas , \textbf{ ut huiusmodi leges vim obligandi habeant , } debent eas promulgare , \\\hline
3.2.27 & e fuerça de costrennir \textbf{ e de obligar deuen las publicar } e publicado las guardar las e mantenerlas . & ut huiusmodi leges vim obligandi habeant , \textbf{ debent eas promulgare , } et promulgatas custodire et obseruaret : \\\hline
3.2.27 & e de obligar deuen las publicar \textbf{ e publicado las guardar las e mantenerlas . } Ca segunt dize el philosofo & debent eas promulgare , \textbf{ et promulgatas custodire et obseruaret : } quia secundum Philosophum 4 Politicorum circa leges duplex cura esse debet : \\\hline
3.2.27 & Lo segundo que sean bien guardadas \textbf{ que quiere dezer tanto commo que atales leyes } assi establesçidas obedes tan bien los omes & ut bene custodiantur , \textbf{ vel ( quod idem est ) } ut legibus sic institutis bene obediatur . \\\hline
3.2.28 & quales deuen ser las leyes \textbf{ que son de poner } por los Reyes e por los prinçipes & Ostendimus in praecedentibus capitulis , \textbf{ quales debent esse leges condendae a Regibus et Principibus } quia debent esse iustae utiles , \\\hline
3.2.28 & a que son puestas . \textbf{ Mas en este capitulo queremos declarar } quales son los fechs delas leyes & et conuenientes populo cui imponuntur leges . \textbf{ Volumus autem in hoc capitulo declare , } qui sunt effectus legum , \\\hline
3.2.28 & quales son los fechs delas leyes \textbf{ e quales e quantas obras deuen contener estas leyes } e conuiene de sabra & qui sunt effectus legum , \textbf{ et quae et quot opera debent continere huiusmodi leges . } Dicuntur autem quinque esse effectus legum , \\\hline
3.2.28 & que çinco son los fechos o las obras delas leyes \textbf{ que son estas . Mandar ¶ Conssentir . } Vedar . ¶ Gualardonar & vel quinque esse opera legalia , \textbf{ videlicet praecipere , } permittere , prohibere , praemiare , et punire . Sunt enim leges \\\hline
3.2.28 & que son estas . Mandar ¶ Conssentir . \textbf{ Vedar . ¶ Gualardonar } Et dar pena . & videlicet praecipere , \textbf{ permittere , prohibere , praemiare , et punire . Sunt enim leges } ( ut supra dicebatur ) quaedam regulae actionum nostrarum : \\\hline
3.2.28 & Vedar . ¶ Gualardonar \textbf{ Et dar pena . } Ca las leyes & videlicet praecipere , \textbf{ permittere , prohibere , praemiare , et punire . Sunt enim leges } ( ut supra dicebatur ) quaedam regulae actionum nostrarum : \\\hline
3.2.28 & assi commo dicho es de suso son reglas delas nr̃as obras . \textbf{ Ca assi commo la fisica quiere reglar } e egualar los humores de los cuerpos de los omes & ( ut supra dicebatur ) quaedam regulae actionum nostrarum : \textbf{ sicut enim medicina per dietam , } et potionem et per \\\hline
3.2.28 & Ca assi commo la fisica quiere reglar \textbf{ e egualar los humores de los cuerpos de los omes } por dieta e por xarope & sicut enim medicina per dietam , \textbf{ et potionem et per } alia quae in ea traduntur , \\\hline
3.2.28 & que es del gouernamiento del regno \textbf{ e dela çibdat quiere reglar } e egualar las obras humanales & vult regulare et aequare humanos humores : \textbf{ sic scientia politica quae est de regimine regni et ciuitatis , } per leges \\\hline
3.2.28 & e dela çibdat quiere reglar \textbf{ e egualar las obras humanales } por las leyes & ø \\\hline
3.2.28 & por que los çibdadanos biuna derechamente \textbf{ e saayan conmose deuen auer . } Mas çerca las obras humanales & ut ciues iuste viuant , \textbf{ et debite se habeant . } Circa actiones autem humanas duplex cura esse potest : \\\hline
3.2.28 & Mas çerca las obras humanales \textbf{ se deuen tomar dos cuydados ¶ } El vno es delas obras que son de fazer & et debite se habeant . \textbf{ Circa actiones autem humanas duplex cura esse potest : } una cum opera sunt futura , \\\hline
3.2.28 & se deuen tomar dos cuydados ¶ \textbf{ El vno es delas obras que son de fazer } ante que sean fechͣs . & Circa actiones autem humanas duplex cura esse potest : \textbf{ una cum opera sunt futura , } priusquam sint effectui mancipata : \\\hline
3.2.28 & que algunas obras delas leyes se toman en conparaçion delas obras \textbf{ que son de fazer . } Et alguas se toman en conparaçion delas obras & quod aliqui effectus legum sumuntur respectu operum fiendorum , aliqui vero respectu operum iam factorum . \textbf{ Respectu fiendorum quidem tria possumos attribuere legibus , } videlicet praecipere , \\\hline
3.2.28 & que son ya fechͣs . \textbf{ Mas en conparaçion delas obras que son de fazer } podemos apodar alas leyes tres cosas conuiene de saber . Mandar e Consentir e vedar . & Respectu fiendorum quidem tria possumos attribuere legibus , \textbf{ videlicet praecipere , } permittere , et prohibere . \\\hline
3.2.28 & Mas en conparaçion delas obras que son de fazer \textbf{ podemos apodar alas leyes tres cosas conuiene de saber . Mandar e Consentir e vedar . } Mas en conparaçion delas obras & videlicet praecipere , \textbf{ permittere , et prohibere . } Respectu factorum vero duo legibus attribuimus , \\\hline
3.2.28 & Mas en conparaçion delas obras \textbf{ que son ya fechas dos cosas podemos a podar alas leyes . } Conuiene de saber gualardonar & Respectu factorum vero duo legibus attribuimus , \textbf{ videlicet punire , } praemiare . \\\hline
3.2.28 & que son ya fechas dos cosas podemos a podar alas leyes . \textbf{ Conuiene de saber gualardonar } e dar pena & Respectu factorum vero duo legibus attribuimus , \textbf{ videlicet punire , } praemiare . \\\hline
3.2.28 & Conuiene de saber gualardonar \textbf{ e dar pena } ca todas las obras de los omes & videlicet punire , \textbf{ praemiare . } Nam actiones humanae \\\hline
3.2.28 & e algunas malas e uiçiosas \textbf{ que son de denostar . } Mas otras son mediateras & et quaedam malae \textbf{ et vitiosae : } quaedam vero mediae , \\\hline
3.2.28 & por la entençion del que obra pueden ser buenas \textbf{ e de loar o malas } e de denostar & licet forte ex intentione operantium possint esse bona et laudabilia , \textbf{ vel mala } et vituperabilia . Ut sic eleuare festucam de terra , \\\hline
3.2.28 & e de loar o malas \textbf{ e de denostar } assi conmoleunatar la paia de tierra & vel mala \textbf{ et vituperabilia . Ut sic eleuare festucam de terra , } de se est opus indifferens : \\\hline
3.2.28 & e de denostar \textbf{ assi conmoleunatar la paia de tierra } de si es obra & vel mala \textbf{ et vituperabilia . Ut sic eleuare festucam de terra , } de se est opus indifferens : \\\hline
3.2.28 & Enpero si alguns le leunatare con mala entençion \textbf{ para poner la en el oio a su conpannon } es mala obra & si quis tamen mala intentione eleuaret illam , \textbf{ ut quia vellet ponere in oculum socii , } esset opus prauum et vituperabile : \\\hline
3.2.28 & para alinpiat la casa \textbf{ o para fazer alguna otra obra buean } por la buena entençion de aquel & esset opus prauum et vituperabile : \textbf{ si vero eleuando eam vellet purgare domum vel facere aliquod aliud opus pium , } propter bonam intentionem operantis , \\\hline
3.2.28 & que dessi non es buena nin mala \textbf{ enpero puede ser uirtuosa e de loar . } pues que assi es & quasi indifferens , \textbf{ esse potest virtuosum et laudabile . } Secundum igitur haec tria genera fiendorum , \\\hline
3.2.28 & pues que assi es \textbf{ segunt estas tres maneras delas obras que son de fazer podemos a podar } e a proprear tres cosas & esse potest virtuosum et laudabile . \textbf{ Secundum igitur haec tria genera fiendorum , } tria legibus attribuimus , \\\hline
3.2.28 & segunt estas tres maneras delas obras que son de fazer podemos a podar \textbf{ e a proprear tres cosas } alas leyes conuiene de saber . & Secundum igitur haec tria genera fiendorum , \textbf{ tria legibus attribuimus , } videlicet praecipere , \\\hline
3.2.28 & e a proprear tres cosas \textbf{ alas leyes conuiene de saber . } Mandar quanto alas buenas obras . & tria legibus attribuimus , \textbf{ videlicet praecipere , } quantum ad opera bona : \\\hline
3.2.28 & alas leyes conuiene de saber . \textbf{ Mandar quanto alas buenas obras . } vedar quanto alas malas & videlicet praecipere , \textbf{ quantum ad opera bona : } prohibere quantum ad mala : permittere quantum ad indifferentia . Aduertendum \\\hline
3.2.28 & Mandar quanto alas buenas obras . \textbf{ vedar quanto alas malas } Et conssentir quanto aquellas que nin son buenas nin malas . & quantum ad opera bona : \textbf{ prohibere quantum ad mala : permittere quantum ad indifferentia . Aduertendum } tamen quod lex imponitur quasi communiter omnibus , \\\hline
3.2.28 & vedar quanto alas malas \textbf{ Et conssentir quanto aquellas que nin son buenas nin malas . } Enpero deuedes saber & quantum ad opera bona : \textbf{ prohibere quantum ad mala : permittere quantum ad indifferentia . Aduertendum } tamen quod lex imponitur quasi communiter omnibus , \\\hline
3.2.28 & Et conssentir quanto aquellas que nin son buenas nin malas . \textbf{ Enpero deuedes saber } que por que la ley es puesta comunalmente & ø \\\hline
3.2.28 & a todos los omes \textbf{ e todos los omes comunalmente non pueden alcançar al medio } nin al conplimiento dela bondat nin dela uirtud & tamen quod lex imponitur quasi communiter omnibus , \textbf{ et omnes communiter punctaliter non attingunt medium bonitatis , } quia non omnes possunt esse omnino perfecti : \\\hline
3.2.28 & Por ende parte nesçe al ponedor dela ley \textbf{ non solamente conssentir aquellas cosas } que nin son bueans nin malas non las defendiendo & spectat ad legislatorem \textbf{ non solum permittere indifferentia non prohibendo ea , } nec puniendo , \\\hline
3.2.28 & nin dando pena por ellas . \textbf{ Mas avn parte nesçe al ponedor dela ley conssentir aquellas cosas } qua non se arriedran mucho del medio & nec puniendo , \textbf{ sed } etiam spectat ad ipsum permittere non solum quae non notabiliter recedunt a medio : \\\hline
3.2.28 & nin son muy malas . \textbf{ Ca si el ponedor dela ley quasi esse vedar } e dar pena & ø \\\hline
3.2.28 & Ca si el ponedor dela ley quasi esse vedar \textbf{ e dar pena } por todos los males & ø \\\hline
3.2.28 & o por quales quier culpas pequanas apenas o nunca poder \textbf{ e gouernar ningun pueblo . } Por ende non solamente son de conssentir aquellas cosas & nam si vellet legislator omnia mala quantumcunque modica prohibere et punire , \textbf{ vix } aut nunquam posset \\\hline
3.2.28 & e gouernar ningun pueblo . \textbf{ Por ende non solamente son de conssentir aquellas cosas } que nin son malas nin bueans . & nam si vellet legislator omnia mala quantumcunque modica prohibere et punire , \textbf{ vix } aut nunquam posset \\\hline
3.2.28 & e han poca maliçia . \textbf{ estas tales deue las conssentir el fazedor dela ley ¶ } Visto que obras son de apodar alas leyes & aliquem populum regere . Ideo non solum permittenda sunt indifferentia , \textbf{ sed etiam quae modicam malitiam habent annexam permitti possint a legislatore . Viso quae attribuenda sunt legibus respectu operum fiendorum : } de leui apparere potest quae attribuenda sunt eis respectu operum iam factorum . \\\hline
3.2.28 & estas tales deue las conssentir el fazedor dela ley ¶ \textbf{ Visto que obras son de apodar alas leyes } en conparaçion delas obras & aliquem populum regere . Ideo non solum permittenda sunt indifferentia , \textbf{ sed etiam quae modicam malitiam habent annexam permitti possint a legislatore . Viso quae attribuenda sunt legibus respectu operum fiendorum : } de leui apparere potest quae attribuenda sunt eis respectu operum iam factorum . \\\hline
3.2.28 & en conparaçion delas obras \textbf{ que son de fazer . } de ligero puede paresçer & ø \\\hline
3.2.28 & que son de fazer . \textbf{ de ligero puede paresçer } que obras son de apodar alas leyes & sed etiam quae modicam malitiam habent annexam permitti possint a legislatore . Viso quae attribuenda sunt legibus respectu operum fiendorum : \textbf{ de leui apparere potest quae attribuenda sunt eis respectu operum iam factorum . } Haec autem sunt duo , \\\hline
3.2.28 & de ligero puede paresçer \textbf{ que obras son de apodar alas leyes } en conparaçion delas obras & sed etiam quae modicam malitiam habent annexam permitti possint a legislatore . Viso quae attribuenda sunt legibus respectu operum fiendorum : \textbf{ de leui apparere potest quae attribuenda sunt eis respectu operum iam factorum . } Haec autem sunt duo , \\\hline
3.2.28 & que son ya fechas . \textbf{ Et estas son dos dar pena . } e dar gualardon . & Haec autem sunt duo , \textbf{ punire et praemiare . } Ut opera notabiliter mala , \\\hline
3.2.28 & Et estas son dos dar pena . \textbf{ e dar gualardon . } assi que las malas obras & Haec autem sunt duo , \textbf{ punire et praemiare . } Ut opera notabiliter mala , \\\hline
3.2.28 & assi que las malas obras \textbf{ ante que se fagan son mucho de defender . } mas despues que son ya fechas son de castigar . & Ut opera notabiliter mala , \textbf{ antequam fiant , sunt prohibenda : } sed postquam iam facta sunt , \\\hline
3.2.28 & ante que se fagan son mucho de defender . \textbf{ mas despues que son ya fechas son de castigar . } Et las obras buen asante & antequam fiant , sunt prohibenda : \textbf{ sed postquam iam facta sunt , } sunt punienda . Opera vero notabiliter bona , \\\hline
3.2.28 & Et las obras buen asante \textbf{ que se fagan son demandar } e de consseiar & sed postquam iam facta sunt , \textbf{ sunt punienda . Opera vero notabiliter bona , } antequam fiant , \\\hline
3.2.28 & que se fagan son demandar \textbf{ e de consseiar } por las leyes . Mas despues que son fechͣs son de gualardonar . & sunt punienda . Opera vero notabiliter bona , \textbf{ antequam fiant , } per leges sunt praecipienda \\\hline
3.2.28 & e de consseiar \textbf{ por las leyes . Mas despues que son fechͣs son de gualardonar . } Mas las que non son buenas nin malas o nin son muy buean so muy malas . & antequam fiant , \textbf{ per leges sunt praecipienda } et consulenda : \\\hline
3.2.28 & estas tales ante \textbf{ que se fagan son de conssentir } por las leyes humanales . & et consulenda : \textbf{ facta vero sunt praemianda . } Sed opera indifferentia , \\\hline
3.2.28 & Mas despues que son fechas \textbf{ nin son de gualardonar nin de condenpnar . } Et pues que assi es çinco cosas apodamos & Sed opera indifferentia , \textbf{ vel non notabiliter bona aut notabiliter mala , } dicuntur legibus humanis esse permissa ; facta vero nec puniuntur \\\hline
3.2.28 & Las dos en conparacion delas buean sobras . \textbf{ Assi commo es mandar } que se fagan e gualardonarlas & dicuntur legibus humanis esse permissa ; facta vero nec puniuntur \textbf{ nec praemiantur . Quinque igitur attribuimus legibus : duo respectu operum bonorum , } ut praecipere fienda , et praemiare facta : \\\hline
3.2.28 & Assi commo es mandar \textbf{ que se fagan e gualardonarlas } despues que son fechas . & nec praemiantur . Quinque igitur attribuimus legibus : duo respectu operum bonorum , \textbf{ ut praecipere fienda , et praemiare facta : } et duo respectu malorum , \\\hline
3.2.28 & despues que son fechas . \textbf{ Et otras dos en conparacion delas malas obras assi conmoes vedar que se non fagan . } Et despues que son fechas dar pena por ellas . & ut praecipere fienda , et praemiare facta : \textbf{ et duo respectu malorum , } ut prohibere fienda , et punire facta : \\\hline
3.2.28 & Et otras dos en conparacion delas malas obras assi conmoes vedar que se non fagan . \textbf{ Et despues que son fechas dar pena por ellas . } Mas bna obra sola apodamos & et duo respectu malorum , \textbf{ ut prohibere fienda , et punire facta : } sed unum attribuimus legibus respectu operum indifferentium \\\hline
3.2.28 & que nin son bueans nin malas \textbf{ e esto es conssentir } las que non es grant fuerça en dexar las passar . & vel quasi indifferentium , \textbf{ ut permittere . } His itaque sic pertractatis , \\\hline
3.2.28 & e esto es conssentir \textbf{ las que non es grant fuerça en dexar las passar . } Et por ende estas cosas & vel quasi indifferentium , \textbf{ ut permittere . } His itaque sic pertractatis , \\\hline
3.2.28 & e delas çibdadeᷤ \textbf{ assi que con grant cuydado e con grant estudio deuen trabaiar quales leyes } e quales establesçimientos pongan a sus çibdadanos . & circa regimen regni , \textbf{ et ciuitatis cura peruigili insudare quas leges , } et quae instituta imponant ciuibus , \\\hline
3.2.28 & asi que con grant acuçia por si \textbf{ e por sus consseieros examun en quales bueans obras son demandar } e de poner so mandamiento . & et quae instituta imponant ciuibus , \textbf{ et diligenter per se et suos consiliarios discutiant quae bona sunt praecipienda } et praemianda , \\\hline
3.2.28 & e por sus consseieros examun en quales bueans obras son demandar \textbf{ e de poner so mandamiento . } Et quales malas son de vedar & et diligenter per se et suos consiliarios discutiant quae bona sunt praecipienda \textbf{ et praemianda , } et quae mala sunt prohibenda , \\\hline
3.2.28 & e de poner so mandamiento . \textbf{ Et quales malas son de vedar } e poner so pena & et praemianda , \textbf{ et quae mala sunt prohibenda , } et quae dissimulanda , et permittenda . \\\hline
3.2.28 & Et quales malas son de vedar \textbf{ e poner so pena } et quales son de desseneiar & et praemianda , \textbf{ et quae mala sunt prohibenda , } et quae dissimulanda , et permittenda . \\\hline
3.2.28 & e poner so pena \textbf{ et quales son de desseneiar } e de sofrir & et quae mala sunt prohibenda , \textbf{ et quae dissimulanda , et permittenda . } Philosophus 3 Politicorum inquirit , \\\hline
3.2.28 & et quales son de desseneiar \textbf{ e de sofrir } o quales son de conssentir & et quae mala sunt prohibenda , \textbf{ et quae dissimulanda , et permittenda . } Philosophus 3 Politicorum inquirit , \\\hline
3.2.28 & e de sofrir \textbf{ o quales son de conssentir } L philosofo en el terçero delas politicas demanda & ø \\\hline
3.2.29 & por muy buen Rey o por muy buena ley . \textbf{ Mas para esto prouar aduze dos razones } que meior es de ser gouernado el regno & aut optima lege . \textbf{ Adducit autem rationes duas , } quod melius sit politiam regni Regi optima lege , \\\hline
3.2.29 & que mas ligera cosa es es \textbf{ de se corronper el rey } que la ley ¶ la primera razon paresçe & Secunda ex eo quod facilius est corrumpi Regem quam legem . \textbf{ Prima via sic patet . } Nam ut dicitur 5 Ethicorum Princeps debet esse custos iusti , \\\hline
3.2.29 & que sea fecho derechamente el Rey \textbf{ por su pode rio çiuil faga lo guardar . } Por la qual cosasi aquello que es mas prinçipales & ut quod iuste lex fieri praecipit , Rex per ciuilem potentiam obseruari facit . \textbf{ Quare si quod est principalius , eligibilius est in regimine , } quam organum et instrumentum : \\\hline
3.2.29 & Por la qual cosasi aquello que es mas prinçipales \textbf{ mas de escoger } commo en el gouernamiento del regno la ley sea mas prinçipal que el Rey & Quare si quod est principalius , eligibilius est in regimine , \textbf{ quam organum et instrumentum : } Regi optima lege eligibilius est , \\\hline
3.2.29 & commo en el gouernamiento del regno la ley sea mas prinçipal que el Rey \textbf{ que el instrumento dela ley mas es de escoger es que la çibdat e el regno sea gouernado } por muy buena ley & quam organum et instrumentum : \textbf{ Regi optima lege eligibilius est , } quam Regi optimo Rege . \\\hline
3.2.29 & Et esto es lo que dize el philosofo en el tercero delas politicas \textbf{ que mas de escoger es } que la ley enssennore & Hoc est ergo quod ait Philosophus 3 Politicorum , \textbf{ quod eligibilius est principari lege , } quia Reges aut Principes ita sunt instituendi , \\\hline
3.2.29 & La segunda razon \textbf{ para mostrar esto mismo se toma daquello } que mas ligera cosa es dese tris tornar & ut seruatores legis et ministri . \textbf{ Secunda via ad inuestigandum hoc idem , } sumitur \\\hline
3.2.29 & para mostrar esto mismo se toma daquello \textbf{ que mas ligera cosa es dese tris tornar } e de corconper el rey & Secunda via ad inuestigandum hoc idem , \textbf{ sumitur } ex eo quod facilius est peruerti Regem , \\\hline
3.2.29 & que mas ligera cosa es dese tris tornar \textbf{ e de corconper el rey } que la ley . & sumitur \textbf{ ex eo quod facilius est peruerti Regem , } quam legem . \\\hline
3.2.29 & por aquello que ha entendimiento \textbf{ Enpero puedese tristornar } por aquello que ha cobdiçia & ex eo quod est intellectus , \textbf{ peruerti } tamen potest ex eo quod habet concupiscentiam annexam . \\\hline
3.2.29 & quanto al ser muy bueno . \textbf{ por que quando el muy bueno en se en comiença de enssennar } e de cobdiçiar las cosas malas & tamen quantum ad esse optimum : \textbf{ quia cum optimus homo incipit furire } et concupiscere peruersa , \\\hline
3.2.29 & por que quando el muy bueno en se en comiença de enssennar \textbf{ e de cobdiçiar las cosas malas } si se non mata quantoal ser sinple mente . & quia cum optimus homo incipit furire \textbf{ et concupiscere peruersa , } et si non interimitur \\\hline
3.2.29 & Et por ende dize el pho en el terçero delas politicas \textbf{ que aquel que manda enssenerorear al entendimiento manda enssennorear a dios e ala ley . } as quien manda enssennorear al omne & ideo dicitur 3 Polit’ \textbf{ quod qui iubet principari intellectum , | iubet principari deum et legem ; } sed qui iubet principari hominem , \\\hline
3.2.29 & que aquel que manda enssenerorear al entendimiento manda enssennorear a dios e ala ley . \textbf{ as quien manda enssennorear al omne } por la cobdiçia se allega ael manda & iubet principari deum et legem ; \textbf{ sed qui iubet principari hominem , } propter concupiscentiam annexam apponit \\\hline
3.2.29 & que por Rey . \textbf{ Mas que esto non sea sinplemente de otorgar muestralo esse mismo pho en esse libro terçero } que assi commo el dize la ley dize generalmente & quam Rege . \textbf{ Sed quod hoc non sit simpliciter fatendum , } ostendit ibi Philosophus in eodem 3 . \\\hline
3.2.29 & Por que conuiene que las leyes humanales \textbf{ commo quier que sean examinadas de fallesçer en algun caso . } Et por ende meior es & ostendit ibi Philosophus in eodem 3 . \textbf{ Nam ( ut ait ) lex uniuersaliter dicit quod non est uniuersaliter : oportet enim humanas leges quantumcunque sint exquisitae in aliquo casu deficere : melius est igitur regnum Regi Rege , } quam lege , \\\hline
3.2.29 & Et por ende por que parezca meior \textbf{ que auemos de dezir } e de sentir en esta tal materia . & Itaque ut appareat \textbf{ quid circa hanc materiam sit dicendum , } sciendum est regem et quemlibet principantem esse medium \\\hline
3.2.29 & e de sentir en esta tal materia . \textbf{ conuiene de saber } que el rey & quid circa hanc materiam sit dicendum , \textbf{ sciendum est regem et quemlibet principantem esse medium } inter legem naturalem \\\hline
3.2.29 & Por la qual cosa sy el nonbre del Rey es tomado de gouernamiento . \textbf{ Conuiene al rey de gouernar los otros } e de ser regla de los otros . & Quare si nomen regis a regendo sumptum est , \textbf{ et decet Regem regere alios , } et esse regulam aliorum , oportet Regem in regendo alios sequi rectam rationem , \\\hline
3.2.29 & por la qual cosa la ley positiua es a quande del señor \textbf{ que la manda fazer } assi commo la ley naturales sobre el señor & propter auctoritatem ponentis differt : \textbf{ quare positiua , } lex est infra principantem , \\\hline
3.2.29 & assi commo la ley naturales sobre el señor \textbf{ por que la non puede mudar . } Et si dixiere alguno & quare positiua , \textbf{ lex est infra principantem , } sicut lex naturalis est supra . Et si dicatur legem aliquam positiuam esse supra principantem , \\\hline
3.2.29 & en el tercero delas politicas \textbf{ que en el derecho gouernamiento non deue enssennorear la bestia } nin el ome bestia la mas dios e el entendimiento . & Propterea bene dictum est quod innuit Philosophus tertio Politicorum \textbf{ quod in recto regimine principari non debet bestia , } sed Deus et intellectus . \\\hline
3.2.29 & que la bestia en ssennorea \textbf{ quando alguno non se esfuerça de gouernar los otros } por razon e por entendimiento & Nam tunc principatur bestia , \textbf{ cum quis non innititur regere alios ratione sed passione et concupiscentia , } in quibus communicamus cum bestiis . \\\hline
3.2.29 & en los quales tal ley fallesçe . \textbf{ Et manda generalmente guardar aquello } que non es de guardar general mente . & in quibus talis lex deficit , \textbf{ et dicit uniuersaliter } quod non est uniuersaliter obseruandum . \\\hline
3.2.29 & Et manda generalmente guardar aquello \textbf{ que non es de guardar general mente . } Et pues que assi es & et dicit uniuersaliter \textbf{ quod non est uniuersaliter obseruandum . } Secundum hoc ergo concludebat ratio in oppositum facta , \\\hline
3.2.29 & que por buena ley \textbf{ que por la ley non puede determinar todas los casos particulates . } Por ende conuiene que el Rey o otro prinçipe & Secundum hoc ergo concludebat ratio in oppositum facta , \textbf{ quod melius est Regi Rege , quam lege eo quod lex particularia determinare non potest . } Ideo expedit Regem aut alium principantem per rationem rectam , \\\hline
3.2.29 & e qua non guarde la ley positiua \textbf{ do non la deue guardar . } Et desto puede paresçer & et non obseruare legem , \textbf{ ubi non est obseruanda . } Ex hoc autem patere potest , \\\hline
3.2.29 & do non la deue guardar . \textbf{ Et desto puede paresçer } en qual manera el rigor o la fortaleza dela uistiçia & ubi non est obseruanda . \textbf{ Ex hoc autem patere potest , } quomodo seueritas \\\hline
3.2.29 & assi conmo dize el philosofo \textbf{ en el quinto delas ethicas non se pueden mesurar } por regla derecha & et propter eorum mutabilitates , \textbf{ ut vult Philosophus 5 Ethic’ non possunt mensurari regula inflexibili , } ut puta ferrea : \\\hline
3.2.29 & Mas conuiene que se reglen con regla de plommo \textbf{ que se pueda en coruar } e allegar alas obras delos omes . & quod mensurentur regula plumbea , \textbf{ quae sit applicabilis humanis actibus . } Oportet igitur aliquando legem plicare ad partem unam , \\\hline
3.2.29 & que se pueda en coruar \textbf{ e allegar alas obras delos omes . } Et por ende conuiene quela ley que se ençorue & quod mensurentur regula plumbea , \textbf{ quae sit applicabilis humanis actibus . } Oportet igitur aliquando legem plicare ad partem unam , \\\hline
3.2.29 & Ca las cercustançias particulares \textbf{ que non se pueden determinar } por la ley algunas uezes aliuian el pecado & Nam particulares circumstantiae , \textbf{ quae lege determinari non possunt , } aliquando alleuiant delictum : \\\hline
3.2.29 & Mas algunas uegadas tales cercunstançias agcauian el pecado \textbf{ e estonçe ha de passar } mas reziamente contra el que peca que mandan las leyes . & Aliquando tales circumstantiae aggrauant : \textbf{ et tunc est rigidius incedendum . Rigor igitur } et clementia licet videantur esse praeter iustitiam legalem et positiuam ; \\\hline
3.2.29 & e egual non se encorua ala vna parte los iuyzios me didos e mesurados \textbf{ segunt aquella regla son dichs de nasçer de egualdat } mas si segunt que meresçieren las condiçiones & non plicata ad aliquam partem , \textbf{ iudicia mensurata | secundum talem regulam dicuntur ex aequalitate procedere . } Si vero exigentibus conditionibus delinquentis legalis regula plicetur ad partem misericordiae , \\\hline
3.2.29 & del que peca la regla dela ley se encorua ala parte dela mibicordia . \textbf{ Estonçe el iuyzio fecho es dicho nasçer de gran o de piadat . } Mas si la dicha regla fuere encoruada ala parte contraria & Si vero exigentibus conditionibus delinquentis legalis regula plicetur ad partem misericordiae , \textbf{ iudicium tunc factum dicetur procedere ex gratia vel ex clementia . } Sed si dicta regula plicetur ad partem oppositam , \\\hline
3.2.29 & sera el iuizio de fortaleza o de iustiçia estrecha . \textbf{ Et por que todas estas cosas se pueden fazer derechamente } e con razon la reziedunbre dela iustiçia & Et \textbf{ quia haec omnia iuste et rationabiliter fieri possunt , clementia } et seueritas simul \\\hline
3.2.30 & que se defienden todos los pecados \textbf{ e se mandan fazer todas las uirtudes } Mas que sin la ley natural e humanal fue menester de dar ley & quae videntur omnia vitia prohibere , \textbf{ et omnes virtutes praecipere . } Sed quod praeter legem naturalem \\\hline
3.2.30 & e se mandan fazer todas las uirtudes \textbf{ Mas que sin la ley natural e humanal fue menester de dar ley } e un aglical e diuinal . & et omnes virtutes praecipere . \textbf{ Sed quod praeter legem naturalem | et humanam fuerit expediens dare legem euangelicam } et diuinam , \\\hline
3.2.30 & e un aglical e diuinal . \textbf{ podemos lo prouar } por tres razones & et diuinam , \textbf{ triplici via possumus venari } secundum ea quae communiter a doctoribus traduntur . \\\hline
3.2.30 & conosçimiento la terçera de parte dela fin o de parte del bien final \textbf{ que nos entendemos ganar } ¶ & Tertia ex parte finis , siue \textbf{ ex parte finalis boni , quod intendimus adipisci . } Prima via sic patet . \\\hline
3.2.30 & nin da pena por todos los pecados \textbf{ la qual cosa contesçe por dos razones La primera es por que el pueblo comunalmente non puede alcançar forma de beuir en punto . } Por ende conuiene & lex humana non omnia peccata punit , \textbf{ quod duplici de causa contingit . Prima est , | quia communiter populus non potest attingere punctalem formam viuendi , } ideo oportet aliqua peccata dissimulare \\\hline
3.2.30 & por la ley humanal \textbf{ por que el prinçipado pueda durar en el pueblo . } Ca non puede ser que ninguno pueda enssennorear prolongadamente & et non punire lege humana , \textbf{ ad hoc ut possit durare principatus in populo : } non enim esset possibile aliquem diuturne principari , \\\hline
3.2.30 & por que el prinçipado pueda durar en el pueblo . \textbf{ Ca non puede ser que ninguno pueda enssennorear prolongadamente } si quiliere condenpnar & ad hoc ut possit durare principatus in populo : \textbf{ non enim esset possibile aliquem diuturne principari , } si vellet omnia peccata punire , \\\hline
3.2.30 & Ca non puede ser que ninguno pueda enssennorear prolongadamente \textbf{ si quiliere condenpnar } e dar pena & ad hoc ut possit durare principatus in populo : \textbf{ non enim esset possibile aliquem diuturne principari , } si vellet omnia peccata punire , \\\hline
3.2.30 & si quiliere condenpnar \textbf{ e dar pena } por todos los pecados & ø \\\hline
3.2.30 & por todos los pecados \textbf{ e si quisiere correzir } e castigar todos los trasgreedores . & non enim esset possibile aliquem diuturne principari , \textbf{ si vellet omnia peccata punire , } et omnes transgressiones corrigere . Rursus \\\hline
3.2.30 & e si quisiere correzir \textbf{ e castigar todos los trasgreedores . } ¶ & si vellet omnia peccata punire , \textbf{ et omnes transgressiones corrigere . Rursus } et si omnes transgressiones possent puniri a principante , \\\hline
3.2.30 & assi commo son las cobdiçias del coraçon \textbf{ que non se pueden condepnar } por ley humanal . & ut interiores concupiscentiae , \textbf{ quae lege humana puniri non possunt . } Oportuit igitur praeter legem humanam dari aliquam legem , \\\hline
3.2.30 & e ningun bien non fincasse sia gualardon . \textbf{ Mas para esto fazer } non cunple la ley natural & et nullum bonum irremuneratum . \textbf{ Ad hoc autem faciendum non sufficit lex naturalis , } ut in prosequendo patebit : \\\hline
3.2.30 & assi commo paresçra adelante . \textbf{ Et por ende conuiene de dar ley diuinal } e e un agłical & ut in prosequendo patebit : \textbf{ oportuit igitur dare legem euangelicam et diuinam , } secundum quam prohiberentur tam delicta interiora \\\hline
3.2.30 & Et los tris passadores desta ley diuirial fuessen condenpnados en este siglo o en el otro \textbf{ Et desto puede paresçer } en qual manera la ley humanal vieda las cobdiçias & vel in hoc seculo , \textbf{ vel in alio punirentur . } Ex hoc autem apparere potest , quomodo lex humana prohibet concupiscentias , et quomodo non prohibet , \\\hline
3.2.30 & e non el coraçon \textbf{ e estonfico de suso de declarar . } Et para esto conuiene de saber & et non animum . \textbf{ Dicebatur enim supra , | hoc declarandum esse . } Sciendum ergo quod si consideretur intentio legislatoris , \\\hline
3.2.30 & e estonfico de suso de declarar . \textbf{ Et para esto conuiene de saber } que si fuere penssada la entençion & hoc declarandum esse . \textbf{ Sciendum ergo quod si consideretur intentio legislatoris , } lege humana omnia peccata prohibentur , \\\hline
3.2.30 & o deuen ser defendidos por ley humanal . \textbf{ Ca non ha de techͣ entençion el Rey } o qual quier otro fazendor de ley & vel prohiberi debent \textbf{ non enim habet rectam intentionem Rex } vel quilibet : \\\hline
3.2.30 & o qual quier otro fazendor de ley \textbf{ si non entendiere fazer a todos los sus çib } dadanos los mas uirtuosos & alius legislator , \textbf{ nisi intendat suos conciues facere quam virtuosiores potest . } Sed cum ad perfectam virtutem nullus inducatur \\\hline
3.2.30 & que pudiere \textbf{ ¶as commo ninguno non pueda venir a acatadas uirtudes } sim̃o entendiere escusar todos los pecados & nisi intendat suos conciues facere quam virtuosiores potest . \textbf{ Sed cum ad perfectam virtutem nullus inducatur } nisi intendat omnia peccata vitare : \\\hline
3.2.30 & ¶as commo ninguno non pueda venir a acatadas uirtudes \textbf{ sim̃o entendiere escusar todos los pecados } si fuer penssada la ley & Sed cum ad perfectam virtutem nullus inducatur \textbf{ nisi intendat omnia peccata vitare : } si consideretur lex quantum \\\hline
3.2.30 & la qual segunt el iuyzio de los otros non es derechͣ . \textbf{ Por la qual cosa commo en los iuyzios humanales pueda caer dubda e yerro . } cosa muy aprouechable & ut de eisdem \textbf{ apud diuersas gentes diuersi sint leges , } secundum iudicium enim quorundam aliquid est iustum , \\\hline
3.2.30 & e una gelical e diuinal \textbf{ en la qual non puede caer yerro . } la terçera razon se toma de parte del grant bien fin & quod secundum aliorum iudicium est iniustum . \textbf{ Quare cum in humanis iudiciis cadere possit dubieras et error , expediens fuit lex euangelica et diuina circa quam error esse non valet . Tertia via sumitur ex parte finalis boni , } quod intendimus adipisci . \\\hline
3.2.30 & la terçera razon se toma de parte del grant bien fin \textbf{ al que nos entendemos de aleançar . } Ca commo este tal bien sea sobre el poderio dela nuestra natura . & Quare cum in humanis iudiciis cadere possit dubieras et error , expediens fuit lex euangelica et diuina circa quam error esse non valet . Tertia via sumitur ex parte finalis boni , \textbf{ quod intendimus adipisci . } Nam cum huiusmodi bonum sit \\\hline
3.2.30 & la ley natural e la humanal \textbf{ que nos ayudan a alcançar este bien . el qual non podemos natural mente } alcançar non cunplen & lex naturalis \textbf{ et humana iuuantes nos ad consecutionem illius boni } quod possumus naturaliter adipisci , \\\hline
3.2.30 & que nos ayudan a alcançar este bien . el qual non podemos natural mente \textbf{ alcançar non cunplen } para alcançar este bien & lex naturalis \textbf{ et humana iuuantes nos ad consecutionem illius boni } quod possumus naturaliter adipisci , \\\hline
3.2.30 & alcançar non cunplen \textbf{ para alcançar este bien } que es sobre natural . & et humana iuuantes nos ad consecutionem illius boni \textbf{ quod possumus naturaliter adipisci , } non sufficiunt ad consequendum illud bonum supernaturale ; \\\hline
3.2.30 & assi commo medios dioses \textbf{ e de auer entendimiento sin cobdiçia } e de ser forma de beuir & Decet ergo reges et principes , \textbf{ quos competit esse quasi semideos , et esse intellectum sine concupiscentia , } et esse formam viuendi , et regulam agibilium , \\\hline
3.2.30 & e regla de todas las obras . \textbf{ assi se auer ala leyna traal } e diuinal e humanal & et esse formam viuendi , et regulam agibilium , \textbf{ sic se habere ad legem diuinam , naturalem , et humanam : } ut sicut excedunt alios potentia et dignitate , \\\hline
3.2.31 & si es cosa conuenible \textbf{ alas çibdades de renouar las leyes dela tierra } e de enduzir nueuas costunbres & cum disputat contra Hippodamum , \textbf{ utrum sit expediens ciuitatibus innouare patrias leges , } et inducere nouas consuetudines . \\\hline
3.2.31 & alas çibdades de renouar las leyes dela tierra \textbf{ e de enduzir nueuas costunbres } por que ypodomio ordenara & utrum sit expediens ciuitatibus innouare patrias leges , \textbf{ et inducere nouas consuetudines . } Ordinauerat enim Hippodamus \\\hline
3.2.31 & muchos eran enduzidos \textbf{ para fallar costunbres nueuas } diziendo que aquellas eran prouechosas ala çibdat . & inducebantur multi \textbf{ ut inuenientes consuetudines nouas , dicentes eas esse utiles } et proficuas ciuitati , soluerent leges patrias \\\hline
3.2.31 & Et por ende non sin razon dubdauna si la opinion de ypodomio era buena \textbf{ e si conuinie de renouar e mudar las leyes muchͣ suegadas } puesto avn que algunas leyes fuessen falladas & an positio Hippodami esset bona , \textbf{ et an expediat saepe saepius immutare leges : dato } etiam quod occurrant leges aliquae quae videantur esse magis proficuae et meliores . \\\hline
3.2.31 & por las quales para pesçe \textbf{ que muestra que conuiene de renouar las leyes ¶ } La primera se toma de parte delas sçiençias e delas artes . & per quas videtur ostendi , \textbf{ quod expediat innouare leges . } Prima sumitur ex parte scientiarum et artium . \\\hline
3.2.31 & que sea meior \textbf{ que los primeros los dichs de los primeros son de reprehender } assi commo en la sçiençia de la fisica son muchos cosas tiradas & ø \\\hline
3.2.31 & que dixieron los antigos . \textbf{ Et en la sçiençia del luchar o del torneamiento } assi commo dize el pho & reprobanda sunt dicta priorum , puta in medicinali amota sunt multa paterna eloquia , \textbf{ et in scientia gymnastica , siue in arte luctatiua } quae docet luctari , \\\hline
3.2.31 & que las que fallaron los primeros padres \textbf{ que non son de guardar las leyes antiguas dela tierra } dende adelante . & quam sint traditae a prioribus patribus , \textbf{ non sunt ulterius leges patriae obseruandae . } Secunda via ad ostendendum hoc idem , sumitur ex parte puritatis quarundam legum : \\\hline
3.2.31 & ¶ La segunda razon \textbf{ para mostrar esto mismo se toma de parte dela maldat de alguas leyes . } Ca contesçe que algunas leyes dela tr̃ra & non sunt ulterius leges patriae obseruandae . \textbf{ Secunda via ad ostendendum hoc idem , sumitur ex parte puritatis quarundam legum : } nam quasdam leges patrias \\\hline
3.2.31 & e nesçia . \textbf{ Ca nesçia cosa era establesçer tales leyes } por las quales los çibdadanos pudiessen vender so mugers . & omnino enim erat barbaricum , \textbf{ statuere leges , } ut ciues possent uxores suas vendere . Sic etiam contingit leges aliquas esse stultas , \\\hline
3.2.31 & Ca nesçia cosa era establesçer tales leyes \textbf{ por las quales los çibdadanos pudiessen vender so mugers . } assi avn contesçe & statuere leges , \textbf{ ut ciues possent uxores suas vendere . Sic etiam contingit leges aliquas esse stultas , } utputa legem illam quam ( ut recitat Philos’ ) \\\hline
3.2.31 & si non se sintiesse culpado \textbf{ mas establesçer esto fue locura } por que si quier alguno fuere culpado & nisi sentiret se culpabilem : \textbf{ sed hoc fuit stultum statuere , } quia siue aliquis sit culpabilis siue non , \\\hline
3.2.31 & por que algunas leyes dela tierra son malas . \textbf{ lon de renouar las leyes antiguas . } La tercera razon se toma dela sinpliçidat & ergo \textbf{ quia aliquae leges paternae sunt prauae , innouandae sunt . Tertia via sumitur } ex simplicitate condentium leges . Nam si aliquando condentes \\\hline
3.2.31 & Et por ende cosa sin razon seria \textbf{ si los sabios postrimos non pudiessen mudar las leyes delatrraque fueron establesçidas por los omes } sipulo postrimeros . & irrationale esset , \textbf{ si posteriores sapientiores | non possent immutare leges paternas per simpliciores conditas : } ergo in tali casu mutandae essent priorum leges . \\\hline
3.2.31 & sipulo postrimeros . \textbf{ Et por ende en tal caso mudar se deuen las leyes de los primeros } ¶La quarta razon se toma & non possent immutare leges paternas per simpliciores conditas : \textbf{ ergo in tali casu mutandae essent priorum leges . } Quarta via sumitur ex indeterminatione particularium actuum . \\\hline
3.2.31 & Ca las obras particulares non son determinadas \textbf{ nin se pueden saber acabada mente . } Por la qual cosa commo quier que los establesçedores delas leyes fuessen muy sabios . & Nam agibilia particularia indeterminata sunt , \textbf{ et perfecte comprehendi non possunt : } quare quantuncunque conditores legum fuerint sapientes , \\\hline
3.2.31 & Por la qual cosa commo quier que los establesçedores delas leyes fuessen muy sabios . \textbf{ poderan se les asconder las cercunstançias particulares } que son en las obras de todos los omes & quare quantuncunque conditores legum fuerint sapientes , \textbf{ potuerunt eos latere aliquae particulares circumstantiae circa agibilia hominum . } Si igitur posterioribus propter experientiam agibilium particularium occurrit aliquid melius , \\\hline
3.2.31 & Et por ende si los postrimeros sabios \textbf{ por la esperiençia delas obras particulares alguna cosa fallar en meior non es cosa sin razon de tirar las leyes dela tierra antiguas } por las meiores leyes falladas nueuamente por ellos . & potuerunt eos latere aliquae particulares circumstantiae circa agibilia hominum . \textbf{ Si igitur posterioribus propter experientiam agibilium particularium occurrit aliquid melius , } inconueniens est non remouere leges paternas \\\hline
3.2.31 & Et por ende paresçe \textbf{ que estas razones sobredichas prueuna que cada que acahesçiere algua cosa meior las leyes dela tierra son de mudar . } Mas afirmar esto sinplemente es muy perigloso ala çibdat e altegno . & et antiquas propter meliores leges nouiter inuentas . Videntur \textbf{ itaque hae rationes probare quod quotiescunque occurrit aliquid melius , | sunt leges paternae immutandae . } Sed hoc simpliciter afferre est valde periculosum ciuitati et regno . \\\hline
3.2.31 & que estas razones sobredichas prueuna que cada que acahesçiere algua cosa meior las leyes dela tierra son de mudar . \textbf{ Mas afirmar esto sinplemente es muy perigloso ala çibdat e altegno . } Ca acostunbrar se los omes afaznueuas leyes & sunt leges paternae immutandae . \textbf{ Sed hoc simpliciter afferre est valde periculosum ciuitati et regno . } Nam assuescere inducere nouas leges \\\hline
3.2.31 & Mas afirmar esto sinplemente es muy perigloso ala çibdat e altegno . \textbf{ Ca acostunbrar se los omes afaznueuas leyes } assi commo dize elpho en el segundo libro delas politicas & Sed hoc simpliciter afferre est valde periculosum ciuitati et regno . \textbf{ Nam assuescere inducere nouas leges } ( ut innuit Philosophus 2 Pol’ ) est assuescere non obedire legibus . \\\hline
3.2.31 & . \textbf{ es acostunbrar sea non obedesçer alas leyes } Ca las leyes grant fuerça toman dela costunbre . & ( ut innuit Philosophus 2 Pol’ ) est assuescere non obedire legibus . \textbf{ Nam leges magnam efficaciam habent ex consuetudine : } de difficili enim quis facit contra aliquid , \\\hline
3.2.31 & Et esto por que con grant guauezafaze cada vno contra aquello que es guardado por luengos tienpos \textbf{ mas acostunbrar se los omes } a non obedesçer las leyes & Nam leges magnam efficaciam habent ex consuetudine : \textbf{ de difficili enim quis facit contra aliquid , } quod est per diuturna tempora obseruatum . Assuescere autem non obedire legibus , est assuescere non obedire Regibus et Principibus , \\\hline
3.2.31 & mas acostunbrar se los omes \textbf{ a non obedesçer las leyes } es acostunbrar & de difficili enim quis facit contra aliquid , \textbf{ quod est per diuturna tempora obseruatum . Assuescere autem non obedire legibus , est assuescere non obedire Regibus et Principibus , } et per consequens est tollere principatum \\\hline
3.2.31 & a non obedesçer las leyes \textbf{ es acostunbrar } sea non obedesçer alos Reyes & de difficili enim quis facit contra aliquid , \textbf{ quod est per diuturna tempora obseruatum . Assuescere autem non obedire legibus , est assuescere non obedire Regibus et Principibus , } et per consequens est tollere principatum \\\hline
3.2.31 & es acostunbrar \textbf{ sea non obedesçer alos Reyes } e alos prinçipes & quod est per diuturna tempora obseruatum . Assuescere autem non obedire legibus , est assuescere non obedire Regibus et Principibus , \textbf{ et per consequens est tollere principatum } et regnum . \\\hline
3.2.31 & Mas quanto mal se se sigue \textbf{ de non obedesçer alas leyes } e alos reyes muestra lo el philosofo en el primero libro delaL rectorica & et regnum . \textbf{ Quantum autem malum sequitur non obedire Regibus } et legibus , \\\hline
3.2.31 & do dize \textbf{ que mas enpeesçe acostunbrar se los omes } de non obedesçer alos prinçipes & qui ait , \textbf{ magis nocere , } consuescere non obedire principibus , \\\hline
3.2.31 & que mas enpeesçe acostunbrar se los omes \textbf{ de non obedesçer alos prinçipes } que non de obedesçer alos fisicos & magis nocere , \textbf{ consuescere non obedire principibus , } quam non obedire medicis . \\\hline
3.2.31 & de non obedesçer alos prinçipes \textbf{ que non de obedesçer alos fisicos } Ca los fisicos entienden enla sanidat del cuerpo . & consuescere non obedire principibus , \textbf{ quam non obedire medicis . } Nam medici intendunt bonum corporis : \\\hline
3.2.31 & Ca los fisicos entienden enla sanidat del cuerpo . \textbf{ por que quieren adozir el cuerpo a sanidat . Mas los uerdaderos ordenadores delas leyes } e los uerdaderos Reyes sinplemente entienden en el bien del alma & Nam medici intendunt bonum corporis : \textbf{ volunt enim corpora inducere ad sanitatem . | Sed veri legislatores } et veri Reges principaliter intendunt bonum animae ; \\\hline
3.2.31 & e los uerdaderos Reyes sinplemente entienden en el bien del alma \textbf{ por que entienden de adozir los çibdadanos a uirtud } Et pues que assi es & et veri Reges principaliter intendunt bonum animae ; \textbf{ quia intendunt ciues inducere ad virtutem . } Ut ergo appareat \\\hline
3.2.31 & por que paresça \textbf{ lo que deuemostener desta question } e qual es la soluçion della . & Ut ergo appareat \textbf{ quid tenendum sit de quae sito , } sciendum quod lex positiua si recta sit , \\\hline
3.2.31 & e qual es la soluçion della . \textbf{ Conuiene de saber } que la ley politica sitiua & ø \\\hline
3.2.31 & Et conuiene que determine las obras e los fechos particulares de los omes . \textbf{ Et por ende en dos maneras puede la ley posiua fallesçer } o auer mengua¶ & et quod determinet gesta particularia hominum . Dupliciter ergo potest \textbf{ huiusmodi lex habere defectum . } Primum \\\hline
3.2.31 & Et por ende en dos maneras puede la ley posiua fallesçer \textbf{ o auer mengua¶ } Lo primero si fuerecontraria ala ley natural . & huiusmodi lex habere defectum . \textbf{ Primum } si sit contraria legi naturali . Secundo si non sufficienter determinaret particularia gesta . \\\hline
3.2.31 & Lo primero si fuerecontraria ala ley natural . \textbf{ Lo segundo si non determinar } e conplidamente los fechos e las obras particulares & Primum \textbf{ si sit contraria legi naturali . Secundo si non sufficienter determinaret particularia gesta . } Si primo modo deficiant leges paternae et positiuae , \\\hline
3.2.31 & mas son corronpimiento de leyes . \textbf{ por la qual cosa non se deuen guardar . } Ca las leyes positiuas & sed corruptiones legum , \textbf{ propter hoc obseruari non debent . } Nam leges positiuae \\\hline
3.2.31 & enpero non son contrarias dellas . \textbf{ saluo si quisiessemos dezir } que contrario del derecho natural es aquello que non es enduzido dela natura . & non tamen sunt contrariae illis , \textbf{ nisi vellemus appellare contrarium iuri naturali } quod non est a natura inductum , \\\hline
3.2.31 & Et es fallado por arte de los o omes \textbf{ segunt la qual manera de fablar el omne es desnudo natural mente . } Ca desnudo nasçe & et per artem omnium adinuentum : \textbf{ secundum quem modum loquendi homo est nudus naturaliter , } et vestimentum est contra naturam : \\\hline
3.2.31 & Ca ser el omne uestido paresçe contrario a aquello que es ser de sudo . \textbf{ Et segunt esta manera de fablar fablan los iuristas del derecho natural } assi commo paresçe en la institutado dize & nam esse vestitum videtur contrariari ei quod est esse nudum . \textbf{ Secundum hunc modum loquendi loquuntur Iuristae , ut patet ex Institutis de iure naturali , } ubi dicitur quod leges humanae contrariae sunt iuri naturali ; \\\hline
3.2.31 & en quanto la natan non enduze seruidunbre en los omes mas es puesta \textbf{ por las leyes aprouecho de los ons Mhas dezir } que alguna cosa es & inquantum natura non induxit seruitutem ; \textbf{ sed est ad utilitatem hominum per leges posita . } Sed sic appellare aliquid contra naturam esse , \\\hline
3.2.31 & que alguna cosa es \textbf{ assi contra natura el fablar rudamente e nesçiamente } por que aquello propreamente es dich ser contra natura & sed est ad utilitatem hominum per leges posita . \textbf{ Sed sic appellare aliquid contra naturam esse , } est ruditer loqui . \\\hline
3.2.31 & assi commo mostramos suso en el segudo libro \textbf{ que algunos seruir alos otros } e obedesçer alos otros & et propter commune bonum hominum est seruitus introducta , \textbf{ ut supra in lib’ 2 ostendimus aliquos seruire aliis et obedire licet sit iuri naturali additum et appositum , } sicut vestimentum est additum \\\hline
3.2.31 & que algunos seruir alos otros \textbf{ e obedesçer alos otros } maguer que sea ennadido e sobrepuesto al derecho natural & ut supra in lib’ 2 ostendimus aliquos seruire aliis et obedire licet sit iuri naturali additum et appositum , \textbf{ sicut vestimentum est additum } et adductum corpori nudo \\\hline
3.2.31 & que dizie \textbf{ que los çibdadanos podien vender sus mugiers } o otras & cuiusmodi erat lex illa , \textbf{ quod ciues possent suas uxores vendere , } vel quaecunque aliae leges sic prauae \\\hline
3.2.31 & o otras \textbf{ quales si quier leyes malas e non derechas non son de guardar } mas de tirar e de derraygar . & vel quaecunque aliae leges sic prauae \textbf{ et iniustae , | non sunt obseruandae , } sed extirpandae . \\\hline
3.2.31 & quales si quier leyes malas e non derechas non son de guardar \textbf{ mas de tirar e de derraygar . } Mas si las leyes positiuas fallesçieren & non sunt obseruandae , \textbf{ sed extirpandae . } Si vero leges sint defectiuae , \\\hline
3.2.31 & e mas conplidas . \textbf{ Enpero non nos auemos a acostunbrar a renouar las leyes . } Lo primero por que algunans vegadas contesçe & dato quod occurrant leges meliores \textbf{ et magis sufficientes , non est assuescendum innouare leges . Primo , quia aliquando contingit circa talia decipi , } quia creduntur meliores quae sunt peiores , \\\hline
3.2.31 & en alguna cosa fuessen mas sufiçientes . \textbf{ Enpero non deuen dexar de ser guardadas las leyes antiguas } e las leyes dela tierra . & Dato tamen quod in aliquo sufficientiores essent leges nouiter adinuentae , \textbf{ sunt tamen leges antiquae , et paternae obseruandae , } quia quanto ex una parte quis perficit , \\\hline
3.2.31 & Et por ende conuiene alos Reyes \textbf{ e alos prinçipes de guardar las bueans costunbres del prinçipado e del regno } e non renouar las leyes dela tierra & sed leges non sic : Immo magnam efficaciam habent ex diuturnitate et assuefactione . \textbf{ Decet ergo reges et principes obseruare bonas consuetudines principatus et regni , } et non innouare patrias leges , \\\hline
3.2.31 & e alos prinçipes de guardar las bueans costunbres del prinçipado e del regno \textbf{ e non renouar las leyes dela tierra } saluo si fuessen contrarias ala razon natural e ala razon derecha & Decet ergo reges et principes obseruare bonas consuetudines principatus et regni , \textbf{ et non innouare patrias leges , } nisi fuerit rectae rationi contrariae . \\\hline
3.2.32 & assi commo dixiemos \textbf{ de suso quatro cosas eran de tranctar . } Conuiene saber & ( ut superius dicebatur ) \textbf{ erant quatuor pertractanda , } videlicet qualis debet esse Rex siue Princeps , \\\hline
3.2.32 & de suso quatro cosas eran de tranctar . \textbf{ Conuiene saber } qual deue ser el Rey o el & erant quatuor pertractanda , \textbf{ videlicet qualis debet esse Rex siue Princeps , } quales consiliarii , \\\hline
3.2.32 & Et por ende desenbargadas las tres cosas \textbf{ finca de dezer de la quarta . } Conuiene de saber del pueblo . & Expeditis ergo tribus , \textbf{ restat dicere de quarto , } scilicet de populo . \\\hline
3.2.32 & finca de dezer de la quarta . \textbf{ Conuiene de saber del pueblo . } Mas commo para saber qual deua ser el puebło & restat dicere de quarto , \textbf{ scilicet de populo . } Sed cum ad sciendum qualis debet esse populus , \\\hline
3.2.32 & Conuiene de saber del pueblo . \textbf{ Mas commo para saber qual deua ser el puebło } e commo se deua auer al prinçipe & scilicet de populo . \textbf{ Sed cum ad sciendum qualis debet esse populus , } et quomodo debeat se habere ad principantem , \\\hline
3.2.32 & Mas commo para saber qual deua ser el puebło \textbf{ e commo se deua auer al prinçipe } e conuenga de saber & Sed cum ad sciendum qualis debet esse populus , \textbf{ et quomodo debeat se habere ad principantem , } non modicum amminiculetur \\\hline
3.2.32 & e commo se deua auer al prinçipe \textbf{ e conuenga de saber } que cosa es çibdat & Sed cum ad sciendum qualis debet esse populus , \textbf{ et quomodo debeat se habere ad principantem , } non modicum amminiculetur \\\hline
3.2.32 & e que posa es regno . Ca esto much aprouecha \textbf{ para saber que cosa es pueblo . } Por ende entendemos declarar en este capitulo & non modicum amminiculetur \textbf{ scire | quid sit ciuitas , } et quid regnum . Intendimus in hoc capitulo declarare et diffinire , \\\hline
3.2.32 & para saber que cosa es pueblo . \textbf{ Por ende entendemos declarar en este capitulo } que cosa es çibdat & quid sit ciuitas , \textbf{ et quid regnum . Intendimus in hoc capitulo declarare et diffinire , } quid sit ciuitas , \\\hline
3.2.32 & Et pueᷤ \textbf{ que assi es deuedes saber } que commo quier que la çibdat en alguna manera sea cosa natural & et quid regnum . Sciendum igitur \textbf{ quod cum ciuitas sit aliquo modo } quid naturale , eo quod naturalem habemus impetum ad ciuitatem constituendam : \\\hline
3.2.32 & por que auemos natural inclinaçion \textbf{ e desseo para establesçer } e fazer la çibdat . & quod cum ciuitas sit aliquo modo \textbf{ quid naturale , eo quod naturalem habemus impetum ad ciuitatem constituendam : } non tamen efficitur , \\\hline
3.2.32 & e desseo para establesçer \textbf{ e fazer la çibdat . } Enpero non se faze la çibdat & quid naturale , eo quod naturalem habemus impetum ad ciuitatem constituendam : \textbf{ non tamen efficitur , } nec perficitur ciuitas , \\\hline
3.2.32 & e delas o tris ten pestades del ayre \textbf{ Por la qual cosa si quisieremos saber } que cosa es çibdat & ut nos defendat a pluiis et caumatibus ; \textbf{ quare si scire volumus } quid est ciuitas , \\\hline
3.2.32 & que cosa es çibdat \textbf{ conuiene de contar todos aquellos bienes } aque sirue la fechura dela çibdat & quid est ciuitas , \textbf{ enumeranda sunt bona illa ad quae deseruit constitutio ciuitatis , } et animaduertendum est quod bonorum illorum sit potius . Narrat quidem Philosophus 3 Politic’ \\\hline
3.2.32 & aque sirue la fechura dela çibdat \textbf{ Et deuemos tener mientes } qual de aquellos bienes es el meior . & ø \\\hline
3.2.32 & Et cuenta el philosofo enel terçero libro delas politicas \textbf{ quariendo de el arar que cosa es la çibdat seys bienes } alos quales es ordenada la çibdat . & et animaduertendum est quod bonorum illorum sit potius . Narrat quidem Philosophus 3 Politic’ \textbf{ volens diffinire | quid sit ciuitas , } sex bona ad quae ciuitas ordinatur . \\\hline
3.2.32 & por que los otros sintiessen su conpannia e su magnifiçençia \textbf{ e por que les pudiesse dar de los sus bienes non termie todos aquellos bienes en much . } Et pues que assi es la çibdat fue fecha & ut alii suam magnificentiam perciperent , \textbf{ et ut eis sua bona communicare posset , non multum reputaret illa . Facta est ergo ciuitas , ut homines simul in uno loco viuentes , } iocunde \\\hline
3.2.32 & por defendimiento dessi mismos \textbf{ e por que non pudiessen resçebir tuerto delos enemigos . } Por que vn omne biuienda solo apartadamente & Tertio facta fuit ciuitas compugnationis gratia , \textbf{ et propter non iniustum pati . } Nam quia homo unus solitariam vitam ducens , \\\hline
3.2.32 & los que mal le quisiessen \textbf{ nin poder a escusar las imiurias } e los tuertos & non est sufficiens resistere impugnantibus , \textbf{ et vitare iniurias } et iniustitias sibi factas ; \\\hline
3.2.32 & Et por ende fue fechͣ la çibdat \textbf{ por que el omne estando solo non se podria defender de los enemigos . } Et seyendo parte de muchedunbre de çibdat puede benir seguro e sin temor ¶ & constituta fuit ciuitas , \textbf{ ut homo qui solitarius se non potest tueri } ab hostibus existens pars multitudinis , \\\hline
3.2.32 & por que el omne estando solo non se podria defender de los enemigos . \textbf{ Et seyendo parte de muchedunbre de çibdat puede benir seguro e sin temor ¶ } Lo quarto fue la çibdat ordenada e fecha por razon delos camios & ut homo qui solitarius se non potest tueri \textbf{ ab hostibus existens pars multitudinis , | tute et absque formidine viueret . } Quarto fuit ciuitas ordinata propter commutationes et contractus . \\\hline
3.2.32 & quando fablauamos delas leyes \textbf{ que fazer mudaçiones e contracto sera } segunt el derech delas gentes . & Dicebatur enim supra cum de legibus tractabamus , \textbf{ quod facere commutationes , et contractus erant } secundum ius gentium , \\\hline
3.2.32 & Et por ende fueron meester las conpras e las mudaçiones e los contractos . \textbf{ las quales cosas todas podian auer meior los omes biuiendo en vno } que si biuiessen apartados & ideo necessariae fuerunt emptiones , venditiones , commutationes , et contractus , \textbf{ quae omnia } quia facilius fiunt hominibus simul conuiuentibus , constituta fuit ciuitas , \\\hline
3.2.32 & que veen \textbf{ que se puede dende le una tar fazen sus casamientos } e fazen se cunnados e parientes los vnos con los otros . El sexto bien & vel propter aliquod aliud bonum , \textbf{ quod vident inde consurgere , } iungunt \\\hline
3.2.32 & por el qual fue establesçida la çibdat \textbf{ es benir escogidamente e uertuosamente } ca mas se pueden castigar los que yerran e fazen mal si los omes biuieren en vno en la çibdat & Sextum bonum propter quod est ciuitas constituta , \textbf{ est viuere eligibiliter } et virtuose . Nam magis possunt puniri delinquentes \\\hline
3.2.32 & es benir escogidamente e uertuosamente \textbf{ ca mas se pueden castigar los que yerran e fazen mal si los omes biuieren en vno en la çibdat } que si morassen apartados & est viuere eligibiliter \textbf{ et virtuose . Nam magis possunt puniri delinquentes | et malefici , } si homines simul conuiuant in ciuitate , \\\hline
3.2.32 & que por temor de pena muchs dexan de fazermal \textbf{ e acostunbran se a fazer buenas obras . } la qual cosa faziendo los omes ordenansse & quod timore poenae multi desinunt malefacere , \textbf{ et assuescunt ad operationes bonas : } quod faciendo , \\\hline
3.2.32 & que es entendido en cada cosa . \textbf{ es de tomar la declaraçion } e el conosçimiento de aquella cosa . & et a maiori bono \textbf{ quod intenditur in re accipienda est eius notitia , benedictum est } quod dicitur 3 Polit’ \\\hline
3.2.32 & por bien e por beuir los omes uirtuosamente e acabadamente . pues que assi es si alguno demandare \textbf{ que cosa es la çibdat podemos responder } e dez que es comunidat de çibdadanos & Si igitur quaeratur \textbf{ quid est ciuitas ? | Dici debet } quod est communicatio ciuium propter bene , \\\hline
3.2.32 & por beuir bien e uirtuosamente \textbf{ e por auer } por si uida acabada e conplida . & quod est communicatio ciuium propter bene , \textbf{ et virtuose viuere ; } et propter perfectam , \\\hline
3.2.32 & Visto que cosa es la çibdat \textbf{ de ligero se puede ver } que cosa es el regno . & Viso quid est ciuitas , \textbf{ de leui videri potest } quid est regnum . \\\hline
3.2.32 & que en vna çibdat . \textbf{ Et pues que assi es el regno puede se assi declarar } e demostrar diziendo & et sunt plures nobiles et ingenui , \textbf{ quam in ciuitate una . Potest ergo sic diffiniri regnum , quod est multitudo magna , } in qua sunt multi nobiles \\\hline
3.2.32 & Et pues que assi es el regno puede se assi declarar \textbf{ e demostrar diziendo } que el regno es grand muchedunbre & ø \\\hline
3.2.32 & e de toda la çibdat . \textbf{ Et assi avn podemos dezer } que vno es el bien & secundum Philos’ in Polit’ Idem finis est unius ciuis , \textbf{ et totius ciuitatis . Sic etiam dicere possumus , } quod idem est unius rectus ciuitatis , \\\hline
3.2.32 & Ca deue el Rey \textbf{ si es uerdadero e derecho Rey querer esse mismo bien en vn çibdadano } e en toda la çibdat & Debet enim Rex , \textbf{ si sit verus et rectus idem intendere in uno ciue , } et in tota ciuitate , et in regno toto . \\\hline
3.2.32 & e en todo el regno . \textbf{ Ca deue estudiar muy acuçiosamente } por que cada vn çibdadano se aya uirtuosamente & et in tota ciuitate , et in regno toto . \textbf{ Nam cura peruigili studere debet , } ut quilibet ciuis virtuose se habeat , \\\hline
3.2.32 & segunt que alguno sobrepuia alos otros en poderio e en dignidat \textbf{ assi los deue sobrepuiar en bondat e en uirtud . } por ende conuiene que los nobles e los altos & secundum quod aliquis excedit alios in potentia \textbf{ et dignitate , } sic debet eos excedere in bonitate et virtute : \\\hline
3.2.32 & Mostrado que cosa es la çibdat \textbf{ e que cosaes regno de ligero puede paresçer } qual deue ser el pueblo & Ostenso quid est ciuitas , \textbf{ et quid regnum : | de leui patere potest , } qualis debeat esse populus existens in ciuitate et regno . \\\hline
3.2.33 & que es departida la çibdat en tres partes \textbf{ se puede departir cada pueblo } et cada regno en tres partes . & alii vero sunt horum medii . \textbf{ Hoc ergo modo quo diuisa est ciuitas in tres partes , diuidi potest quilibet populus et quodlibet regnum . } Intentio autem huius capituli est ostendere optimam esse ciuitatem \\\hline
3.2.33 & et cada regno en tres partes . \textbf{ Mas la entençion deste capitulo es mostrar } que es muy buena la çibdat & Hoc ergo modo quo diuisa est ciuitas in tres partes , diuidi potest quilibet populus et quodlibet regnum . \textbf{ Intentio autem huius capituli est ostendere optimam esse ciuitatem } et regnum , \\\hline
3.2.33 & Et pone el pho en el quarto libro delas politicas quatro cosas \textbf{ delas quales se pueden tomar quatro razonnes } que muestran que meior es la poliçia & si ibi sit populus ex multis personis mediis constitutus . Tangit autem Philosophus 4 Politicorum , quatuor , \textbf{ ex quibus sumi possunt quatuor viae , ostendentes meliorem esse politiam , } vel melius esse regnum et ciuitatem , \\\hline
3.2.33 & nin con razon \textbf{ Ca los muy ricos non se saben auer } con razon alos muy pobres & vix aut nunquam rationabiliter viuet . \textbf{ Nam multum diuites ad multum pauperes nesciunt rationabiliter se habere , } sed modica occasione sumpta eis manifeste nocent . Sic etiam multum pauperes ad nimium diuites nesciunt \\\hline
3.2.33 & que ayan contra ellos manifiesta miente les enpeesçen . \textbf{ Assi avn los muy pobres non se saben auer } con razon alos muy ricos . & ø \\\hline
3.2.33 & Ca sienpre les asecha commo puedan faldridamente \textbf{ e encobiertamente tomar e robar de sus biens . } Mas si en el pueblo fueren muchas perssonas medianeras quedaran todo estos enpeesçimientos & se rationabiliter gereres insidiantur enim eis quomodo possint astute \textbf{ et latenter eorum depraedari bona . } Sed si in populo sint multae personae mediae , \\\hline
3.2.33 & e dela iustiçia \textbf{ que se deue guardar en la çibdat . } Ca assi commo dize el philosofo & ø \\\hline
3.2.33 & e en los otros bienes tenporales \textbf{ non se saben omillar . } Mas los otros que sobrepuian en grand mengua & et in aliis huiusmodi bonis , \textbf{ nesciunt subiici . } Qui autem \\\hline
3.2.33 & Mas los otros que sobrepuian en grand mengua \textbf{ e en grand pobreza non saben enssennorear } nin ser señores . & Qui autem \textbf{ secundum excessum sunt indigentes et pauperes , } nesciunt principari . \\\hline
3.2.33 & mente destas dos partes de muy ricos e de muy pobres \textbf{ e pocas perssonas o ningunas fueren y medianeras apenas se podran y guardar egualdat nin iustiçia . } Ca los ricos en toda manera quarran en ssennorear & et nimis pauperibus : \textbf{ et paucae aut nullae sint ibi personae mediae , | de difficili seruabitur ibi aequalitas et iustitia , } sed diuites penitus volent principari , \\\hline
3.2.33 & e pocas perssonas o ningunas fueren y medianeras apenas se podran y guardar egualdat nin iustiçia . \textbf{ Ca los ricos en toda manera quarran en ssennorear } e poner so pie alos otros & de difficili seruabitur ibi aequalitas et iustitia , \textbf{ sed diuites penitus volent principari , } et suppeditare alios . Alii vero contra nitentes dissensionem faciunt , \\\hline
3.2.33 & Ca los ricos en toda manera quarran en ssennorear \textbf{ e poner so pie alos otros } e los pobres contradiziendo alos ricos faran discordia en la çibdat & sed diuites penitus volent principari , \textbf{ et suppeditare alios . Alii vero contra nitentes dissensionem faciunt , } et si contingat pauperes obtinere , \\\hline
3.2.33 & que entre ellos non aya egualdat . \textbf{ la qual cosa se puede fazer si por qual quier razon los çibdadanos non ouieren liçençia de uender sus possessiones } nin ouiere cada vno liçençia de conprar qualsquier possessiones . & quod fieri poterit , \textbf{ si non pro quacunque causa liceat ciuibus paternas possessiones vendere ; } nec quibuslibet indifferenter liceat quascunque possessiones emere , adhibita enim debita diligentia circa emptionem , et venditionem agrorum \\\hline
3.2.33 & la qual cosa se puede fazer si por qual quier razon los çibdadanos non ouieren liçençia de uender sus possessiones \textbf{ nin ouiere cada vno liçençia de conprar qualsquier possessiones . } Ca auiendo diligençia e acuçia conuenible en las conpras & si non pro quacunque causa liceat ciuibus paternas possessiones vendere ; \textbf{ nec quibuslibet indifferenter liceat quascunque possessiones emere , adhibita enim debita diligentia circa emptionem , et venditionem agrorum } et terrarum , poterit aliqualis aequalitas reseruari inter ciues . \\\hline
3.2.34 & Et si guardare las leyes de los Reyes \textbf{ por las quals tres cosas podemos prouar } por tres razones & Ex quibus , \textbf{ triplici via venari possumus , } quantum sit utile et expediens populo obedire Regibus et Principibus , et obseruare leges . \\\hline
3.2.34 & por tres razones \textbf{ quanto es prouechoso e conuenible al pueblo de obedesçer alos Reyes } e guardar las leyes . & triplici via venari possumus , \textbf{ quantum sit utile et expediens populo obedire Regibus et Principibus , et obseruare leges . } Primo enim \\\hline
3.2.34 & quanto es prouechoso e conuenible al pueblo de obedesçer alos Reyes \textbf{ e guardar las leyes . } Ca lo primero desto alçança el pueblo uirtudes e grandes bienes & triplici via venari possumus , \textbf{ quantum sit utile et expediens populo obedire Regibus et Principibus , et obseruare leges . } Primo enim \\\hline
3.2.34 & la entençion del ponedor dela ley \textbf{ es enduzer los çibdadanos o uirtud . } Ca en la derecha poliçia & ut dicebatur in praecedentibus ) \textbf{ intentio legislatoris est inducere ciues ad virtutem . } In recta enim Politia \\\hline
3.2.34 & si ben obedesçiere al prinçipe \textbf{ e a aquel a quien pertenesçe de poner las leyes . } Por la qual cosa si el prinçipe gouernar e derechamente el pueblo & si bene obediat principanti , \textbf{ et ei | cuius est leges ferre . } Quare si principans \\\hline
3.2.34 & e a aquel a quien pertenesçe de poner las leyes . \textbf{ Por la qual cosa si el prinçipe gouernar e derechamente el pueblo } qual es acomne dado & cuius est leges ferre . \textbf{ Quare si principans | recte regat populum sibi commissum , } quia intentio eius est inducere alios ad virtutem , \\\hline
3.2.34 & qual es acomne dado \textbf{ por que la su entençion es enduzir los otros a uirtud . } Et la uirtud faze & recte regat populum sibi commissum , \textbf{ quia intentio eius est inducere alios ad virtutem , } cum virtus faciat habentem bonum ; \\\hline
3.2.34 & ya non serie Rey mas serie tyrano . \textbf{ Vien assi cada vno de los çibdadanos deue entender } por que sea bueno e uirtuoso . & sed tyrannus . \textbf{ Si ergo quilibet ciuis debet intendere } ut sit bonus \\\hline
3.2.34 & por que las uirtudes son muy grandes bienes . \textbf{ Et avn con grant diligençia deue estudiar cada vn çibdadano } por que obedezca al Rey & eo quod virtutes sunt maxima bona , \textbf{ cum summa diligentia studere debet , } ut Regi obediatur , \\\hline
3.2.34 & la segunda razon \textbf{ para prouar esto mesmo se tomadesto } que dela obediençia del Rey & quanto decentius est eos esse bonos , et virtuosos . \textbf{ Secunda via ad inuestigandum hoc idem , } sumitur ex eo quod ex obedientia Regis , \\\hline
3.2.34 & que guardan las leyes \textbf{ e obedesçer el Rey lea algunasiudunbre . } Enpero leguntel pho en el quanto libro delas politicas & quod obseruare leges , \textbf{ et obedire regi , | sit quaedam seruitus . } Sed secundum Philosoph’ 5 Politic’ \\\hline
3.2.34 & aquellos que dizen \textbf{ que guardar las leyes } e obedesçer alos Reyes es seruidunbre . & Ignorant enim quid est libertas , dicentes obseruare leges et obedire Regibus , \textbf{ esse seruitutem . } Cum enim bestiae sint naturae seruilis : \\\hline
3.2.34 & que guardar las leyes \textbf{ e obedesçer alos Reyes es seruidunbre . } Ca commo las bestias sean de natura seruil & Ignorant enim quid est libertas , dicentes obseruare leges et obedire Regibus , \textbf{ esse seruitutem . } Cum enim bestiae sint naturae seruilis : \\\hline
3.2.34 & e malo e turbador dela paz \textbf{ e querer beuir sin freno e sin ley } segunt lasuna delpho & et turbatorem pacis , \textbf{ velle viuere sine freno et sine lege , } secundum sententiam Philosophi , \\\hline
3.2.34 & los que non guardan las leyes \textbf{ nin quieren obedesçer alos Reyes } nin alos sus mayors & Quare non obseruantes leges , \textbf{ nolentes obedire regibus et superioribus , sunt magis bestiae quam homines : } et per consequens sunt magis serui , \\\hline
3.2.34 & Por la qual cosa \textbf{ assi commo es muy mala cosa al cuerpo desmanparar el alma } e non se gouernar por ella . & et vita regni . \textbf{ Quare sicut pessimum est corpori delinquere animam , } et non regi per eam , \\\hline
3.2.34 & assi commo es muy mala cosa al cuerpo desmanparar el alma \textbf{ e non se gouernar por ella . } assi es muy mala cosa & Quare sicut pessimum est corpori delinquere animam , \textbf{ et non regi per eam , } sic pessimum est regno deserere leges regias \\\hline
3.2.34 & La terçera razon \textbf{ para mostrar esto mismo se tomadesto } que dela obediençia del rey & et praecepta legalia , et non regi per Regem . \textbf{ Tertia via ad ostendendum hoc idem , sumitur ex eo quod ex obedientia regia , } et ex obseruatione legum oritur pax \\\hline
3.2.34 & Ca assi conmodicho es de ssuso las leyes \textbf{ e avn los ponedores dellas assi commo los Reyes e los prinçipes han poderio de costrennir alos omes } assi que aquellos que por amor de bien e de honestad non se quasiessen partir de malas obras & leges \textbf{ et etiam legislatores , | ut Reges et Principes coactiuam habent potentiam : } ut qui amore honesti per increpationes paternas , \\\hline
3.2.34 & e avn los ponedores dellas assi commo los Reyes e los prinçipes han poderio de costrennir alos omes \textbf{ assi que aquellos que por amor de bien e de honestad non se quasiessen partir de malas obras } por El xx castigos de sus padres & ut Reges et Principes coactiuam habent potentiam : \textbf{ ut qui amore honesti per increpationes paternas , } et amicorum non retrahitur ab operibus sceleratis , \\\hline
3.2.34 & Et por ende cosa conuenible fue al regno \textbf{ e ala çibdat de auer algun Rey o algun prinçipe } por que los malos non pudiessen turbar la paz de los çibdadanos . & saltem timore poenae retrahatur ab illis . Expediens enim fuit regno et ciuitati habere \textbf{ aliquem Regem vel aliquem principantem , } ne malefici turbarent pacem ciuium . Intendere enim debet quilibet legislator , \\\hline
3.2.34 & e ala çibdat de auer algun Rey o algun prinçipe \textbf{ por que los malos non pudiessen turbar la paz de los çibdadanos . } Ca deue cada vn ponedor dela ley entender en esto & aliquem Regem vel aliquem principantem , \textbf{ ne malefici turbarent pacem ciuium . Intendere enim debet quilibet legislator , } ut corda ciuium sint tranquilla , \\\hline
3.2.34 & por que los malos non pudiessen turbar la paz de los çibdadanos . \textbf{ Ca deue cada vn ponedor dela ley entender en esto } que los coraçones de los çibdadanos sean assessegados & aliquem Regem vel aliquem principantem , \textbf{ ne malefici turbarent pacem ciuium . Intendere enim debet quilibet legislator , } ut corda ciuium sint tranquilla , \\\hline
3.2.34 & commo los fisicos alos cuerpos . \textbf{ Ca assi commo el fisico entiende de amanssar } e de egualar los humores & sicut medicina ad corpora . \textbf{ Nam sicut medicus intendit sedare humores , } ne insurgat morbus \\\hline
3.2.34 & Ca assi commo el fisico entiende de amanssar \textbf{ e de egualar los humores } por que se non le una te enfermedat̃ & sicut medicina ad corpora . \textbf{ Nam sicut medicus intendit sedare humores , } ne insurgat morbus \\\hline
3.2.34 & nin batalla dellos en el cuerpo . \textbf{ assi el fazedor delas leyes entiende de amanssar los coraçones } e abenir las almas & et bellum in corpore : \textbf{ sic legislator intendit placare corda , } sedare animas , \\\hline
3.2.34 & assi el fazedor delas leyes entiende de amanssar los coraçones \textbf{ e abenir las almas } porque se non le uate pelea nin contienda en el regno o en la çibdat . & sic legislator intendit placare corda , \textbf{ sedare animas , } ne insurgat rixa et dissensio in regno , \\\hline
3.2.34 & Et dende nasçe aquello que dize el philosofo en el primero libro de la rectorica \textbf{ que non enpeesçe tanto pecar contra los mandamientos del fisico } quanto enpeesçe acostunbrar se & aut in ciuitate . Inde est ergo quod dicitur primo Rhet’ \textbf{ quod non tantum nocet peccare contra praecepta medici , } quantum consuescere non obedire Principi . Est enim anima maius bonum quam corpus , \\\hline
3.2.34 & que non enpeesçe tanto pecar contra los mandamientos del fisico \textbf{ quanto enpeesçe acostunbrar se } en non obedesçer al prinçipe . & aut in ciuitate . Inde est ergo quod dicitur primo Rhet’ \textbf{ quod non tantum nocet peccare contra praecepta medici , } quantum consuescere non obedire Principi . Est enim anima maius bonum quam corpus , \\\hline
3.2.34 & quanto enpeesçe acostunbrar se \textbf{ en non obedesçer al prinçipe . } Ca el alma es mayor bien que el cuerpo & quod non tantum nocet peccare contra praecepta medici , \textbf{ quantum consuescere non obedire Principi . Est enim anima maius bonum quam corpus , } et pax ciuium et eorum \\\hline
3.2.34 & que la egualdat de los humores o la sanidat de los cuerpos \textbf{ Et por ende con muy grant acuçia deue estudiar el pueblo } e todos los moradores del regno en obedesçer alos Reyes & qui sunt in regno potior est quam aequalitas humorum , \textbf{ vel quam sanitas corporum . Summo ergo opere studere } debet populus , et omnes habitatores regni circa obedientiam regiam , \\\hline
3.2.34 & Et por ende con muy grant acuçia deue estudiar el pueblo \textbf{ e todos los moradores del regno en obedesçer alos Reyes } e guardar las leyes . & vel quam sanitas corporum . Summo ergo opere studere \textbf{ debet populus , et omnes habitatores regni circa obedientiam regiam , } et obseruationem legum : \\\hline
3.2.34 & e todos los moradores del regno en obedesçer alos Reyes \textbf{ e guardar las leyes . } Commo desto se leunate tan grant bien & debet populus , et omnes habitatores regni circa obedientiam regiam , \textbf{ et obseruationem legum : } cum ex hoc consurgat tantum bonum , \\\hline
3.2.34 & non se labran las heredades \textbf{ mas fincan por labrar . } Et fazen se robos e furtos & Nam existente guerra in regno , \textbf{ terrae } manent incultae , \\\hline
3.2.34 & Et por ende si fuere penssado \textbf{ quanto bien viene al regno en obedesçer non solamente alos Re yes } que bien e derechamente gouiernan sus regnos & Si ergo consideretur \textbf{ quantum bonum aduenit ex rege , } non solum Regibus recte regentibus , \\\hline
3.2.34 & mas avn alos malos . \textbf{ Ca avn puesto que en alguna cosa tira nizen deue estu diar avn el pueblo de obedesçer los . } Ca mas sofridera es alguna tirama o desordenamiento del prinçipe & non solum Regibus recte regentibus , \textbf{ sed etiam dato quod in aliquo tyrannizarent , studeret populus obedire illis . } Nam magis est tolerabilis aliqualis tyrannis principantis , quam sit malum , quod consurgit ex inobedientia Principis , \\\hline
3.2.35 & e guarden las sus leyes \textbf{ queremos declarar en este capitulo } en qual manera se de una auer los que son en el regno & et obseruent eorum leges : \textbf{ volumus in hoc capitulo declarare , } qualiter se habere debeant existentes in regno , \\\hline
3.2.35 & segunt el philosofo enl comienço del segundo libro de la rectorica est steza \textbf{ que viene del appetito de dar pena manifiesta } por menospreçio aparesçiente de aquellas cosas & Ita autem \textbf{ secundum Philosophum in principio 2 Rhet’ est tristitia proueniens ex appetitu apparentis punitionis , } propter apparentem paruipensionem eorum quae in ipsum , aut in aliqua ipsius , \\\hline
3.2.35 & por que . nunca es sanna sin alguna tristeza . \textbf{ ca dessea el sannudo dar pena manifiestamente } a aquellos & aut in aliqua quae ordinantur ad ipsum . Nunquam enim ira sine tristitia est . Appetit autem iratus apparenter , \textbf{ idest manifeste punire eos } qui paruipendunt ipsum , vel aliqua quae sunt ipsius , \\\hline
3.2.35 & assana \textbf{ non deuen fazer ninguna cosa mala contra el Rey } nin contra aquellas cosas & Ciues itaque ut non prouocent Reges ad iram , \textbf{ non debent fore facere nec in Regem , } nec in ea quae sunt ipsius , \\\hline
3.2.35 & por que non cayan en sanna del reyes \textbf{ non fazer ninguna cosa mala } contra el Rey & qui sunt in regno , \textbf{ ut non incurrant regiam iram , non forefacere in ipsum Regem . Regi autem duo debentur , } honor et obedientia . Est enim Rex caput regni : caput autem ad alia membra dupliciter comparatur . Primo quidem , \\\hline
3.2.35 & que son en el regno . \textbf{ Otrossi porque el ha a poner las leyes } e por que en el Rey deue ser mayormente seso e sabiduria . & qui sunt in regno . \textbf{ Rursus quia eius est leges ferre , } et quia in Rege maxime vigere debet sensus et prudentia , \\\hline
3.2.35 & e por que en el Rey deue ser mayormente seso e sabiduria . \textbf{ Et por ende a el parte nesçe prinçipalmente gouernar e gniar todos los que son en el regno } por si o por otros . & et quia in Rege maxime vigere debet sensus et prudentia , \textbf{ ad ipsum spectat per se et per alios dirigere eos , } qui sunt in regno . Ratione ergo quia Rex est excellentior aliis , ei debetur honor , \\\hline
3.2.35 & ael deue ser dada honrra e reuerençia . \textbf{ Mas por razon que a el parte nesçe degniar los otros } ael deue ser fecha subiectiuo e obediençia & et reuerentia . Ratione vero , \textbf{ quia ipsius est dirigere alios , } debetur ei subiectio et obedientia . \\\hline
3.2.35 & Por la qual cosa en dos maneras pueden \textbf{ los que son en el regno errar contra el Rey¶ } La primera si non le fizieren honrra qual deuen e reuerençia conueinble . & Quare dupliciter potest forefieri ad Regem ab iis \textbf{ qui sunt in regno . Primo , } si non ei exhibeant honorem debitum , et reuerentiam dignam . Secundo , \\\hline
3.2.35 & Ca quando estas dos cosas . \textbf{ Conuiene a saber la honrra } e la obediençia son tiradas al Rey con grand derech se mueue a sanna . & Cum enim haec duo , \textbf{ honor videlicet , et obedientia subtrahuntur a Rege , } merito prouocatur ad iram . Ideo dicitur 2 Rhet’ \\\hline
3.2.35 & Et por ende dize el philosofo en el segundo libro de la rectorica \textbf{ que nos nos enssannamos contra aquellos que nos solian honrrar } o nos eran tenidos de honrrar & merito prouocatur ad iram . Ideo dicitur 2 Rhet’ \textbf{ quod irascimur iis qui tenentur , } vel consueuerunt nos honorare , \\\hline
3.2.35 & que nos nos enssannamos contra aquellos que nos solian honrrar \textbf{ o nos eran tenidos de honrrar } si non nos horraren & quod irascimur iis qui tenentur , \textbf{ vel consueuerunt nos honorare , } si non sic se habeant . \\\hline
3.2.35 & que nos despreçian . \textbf{ Ca si non nos despreçiassen fazernos } yan aquella honrra & quia sic se habentes videntur nos despicere . \textbf{ Nam si nos non despicerent , impenderent nobis honorem dignum . } Sic etiam ibidem dicitur , \\\hline
3.2.35 & si ellos non fizieren aquello \textbf{ que deuen fazer o si fizieren cosas contrarias de aquello que deuen fazer . } la qual cosa contesçe mayormente & si non sic se habeant ut debent , \textbf{ vel si faciant contraria eorum quae debent , } quod maxime contingit , \\\hline
3.2.35 & assi commo a su mayor ¶ \textbf{ Visto en qual manera los moradores del regno non deuen mouer el Rey a saña errando contra el } e non le faziendo obediençia & si transgrediuntur mandata superiorum , \textbf{ et non obediunt regi quasi praecellenti . Viso quomodo habitatores regni non debent prouocare Regem ad iram , } forefaciendo in ipsum , non exhibere ei debitum honorem et obedientiam condignam . Restat videre , quomodo non debent ipsum prouocare , \\\hline
3.2.35 & qual deuen e honrra conuenble . \textbf{ finca de uer } en qual manera non le deuen mouer a sanna al Rey errando & forefaciendo in ipsum , non exhibere ei debitum honorem et obedientiam condignam . Restat videre , quomodo non debent ipsum prouocare , \textbf{ forefaciendo in eos } qui sunt eius , \\\hline
3.2.35 & finca de uer \textbf{ en qual manera non le deuen mouer a sanna al Rey errando } en aquellas cosas & forefaciendo in eos \textbf{ qui sunt eius , } et qui pertinent ad ipsum . \\\hline
3.2.35 & que parte nesçen a el . \textbf{ Et deuedes saber } que al Rey parte nesçen quatro maneras de perssonas . & et qui pertinent ad ipsum . \textbf{ Ad Regem autem pertinere videntur quatuor genera personarum } videlicet parentes \\\hline
3.2.35 & que al Rey parte nesçen quatro maneras de perssonas . \textbf{ Conuiene a saber . El padir . Et la madre . } Et generalmente todos los sus parientes . & Ad Regem autem pertinere videntur quatuor genera personarum \textbf{ videlicet parentes } et uniuersaliter omnes cognati , \\\hline
3.2.35 & Et por ende parte nesçe alos moradores del regno \textbf{ sinon quasieren mouer al Rey a saña } non solamente de non fazer ningun tuerto & Spectat itaque ad habitatores regni \textbf{ si nolunt Regem ad iram prouocare , } non solum non forefacere in ipsum Regem , \\\hline
3.2.35 & sinon quasieren mouer al Rey a saña \textbf{ non solamente de non fazer ningun tuerto } contra el rey en su perssona . & si nolunt Regem ad iram prouocare , \textbf{ non solum non forefacere in ipsum Regem , } sed etiam non forefacere in cognatos , uxorem , filios , et uniuersaliter in omnes subiectos ipsi regi . \\\hline
3.2.35 & contra el rey en su perssona . \textbf{ Mas avn de non fazer contra sus parientes } nin contra su muger & non solum non forefacere in ipsum Regem , \textbf{ sed etiam non forefacere in cognatos , uxorem , filios , et uniuersaliter in omnes subiectos ipsi regi . } Et quia omnes \\\hline
3.2.35 & e avn alos otros \textbf{ de non defender lo suyo . } Lo terçero pertenesçe alos raoradores del regno & et subiectos \textbf{ eo quod est turpe ipsis non auxiliari . } Tertio spectat ad habitatores regni non forefacere in ea quae aliquo modo ordinantur ad regem . \\\hline
3.2.35 & Lo terçero pertenesçe alos raoradores del regno \textbf{ de non venir contra aquellas cosas } que pertenesçen al Rey en qualquier manera e estas tales cosas son las possessiones propas del Rey & eo quod est turpe ipsis non auxiliari . \textbf{ Tertio spectat ad habitatores regni non forefacere in ea quae aliquo modo ordinantur ad regem . } Huiusmodi autem sunt possessiones propriae , \\\hline
3.2.35 & que amen al Rey \textbf{ e queles enssennen en qual manera de una honnar al Rey } e obedescerle & et uniuersaliter omnes habitatores regni ab ipsa infantia prouocare filios ad dilectionem Regis : \textbf{ instruere eos quomodo debeant honorare Regem , obedire ei : } non forefacere in cognatos eius , \\\hline
3.2.35 & e queles enssennen en qual manera de una honnar al Rey \textbf{ e obedescerle } e non fazer tuerto contra los sus parientes & instruere eos quomodo debeant honorare Regem , obedire ei : \textbf{ non forefacere in cognatos eius , } nec in filios , \\\hline
3.2.35 & e obedescerle \textbf{ e non fazer tuerto contra los sus parientes } nin contra sus fijos & instruere eos quomodo debeant honorare Regem , obedire ei : \textbf{ non forefacere in cognatos eius , } nec in filios , \\\hline
3.2.35 & delas positicas \textbf{ non enssennar los mocos auertudes } e aguarda delas leyes prouechosas & nec in aliqua iura regni . Pessimum est enim ( ut dicitur 7 Pol’ ) \textbf{ non instruere pueros ad virtutem , } et obseruantiam legum utilium : \\\hline
3.2.35 & e aguarda delas leyes prouechosas \textbf{ e aguardar aquellas cosas } que demanda la poliçia o el gouernamiento del regno & non instruere pueros ad virtutem , \textbf{ et obseruantiam legum utilium : } et ad obseruandum ea quae requirit politia , \\\hline
3.2.35 & son mas inclinados \textbf{ para lo guardar } orque en el primero libro prometimos uos & ø \\\hline
3.2.36 & por que sean temidos . \textbf{ Nos queremos aqui dezer } en qual manera puede esto ser . & et quomodo timeantur : \textbf{ volumus hic exequi qualiter fieri hoc contingat . } Sciendum itaque quod ut Reges et Principes communiter amentur a populo , \\\hline
3.2.36 & en qual manera puede esto ser . \textbf{ Et por ende conuiene de saber } que para que los Reyes e los prinçipes comunalmente sean amados del pueblo deuen auer en ssi tres cosas prinçipalmente . & volumus hic exequi qualiter fieri hoc contingat . \textbf{ Sciendum itaque quod ut Reges et Principes communiter amentur a populo , } tria potissime in se habere debent . Primo quidem esse debent benefici , \\\hline
3.2.36 & Et por ende conuiene de saber \textbf{ que para que los Reyes e los prinçipes comunalmente sean amados del pueblo deuen auer en ssi tres cosas prinçipalmente . } La primera que sean bien fechores e liberales e francos ¶ & volumus hic exequi qualiter fieri hoc contingat . \textbf{ Sciendum itaque quod ut Reges et Principes communiter amentur a populo , } tria potissime in se habere debent . Primo quidem esse debent benefici , \\\hline
3.2.36 & e ha reuerençia alos bien fechores e alos liberales \textbf{ que son francos en dar dineros } o aquellas cosas & ideo beneficos , \textbf{ et liberales in numismata , } et in ea quae possunt numismate mensurari , amat , et reueretur . Ideo dicitur 2 Rheto’ \\\hline
3.2.36 & o aquellas cosas \textbf{ que se pueden auer por dineros } Et por ende dize el philosofo en el segundo libro de la rectorica & et liberales in numismata , \textbf{ et in ea quae possunt numismate mensurari , amat , et reueretur . Ideo dicitur 2 Rheto’ } cap’ \\\hline
3.2.36 & que el pueblo ama \textbf{ e honira alos bien fechores e alos liberales en dar dineros } e aueres la segunda cosa & quod populus amat , \textbf{ et honorat beneficos in pecunia , et liberales . } Secundo ut Reges amentur in populo , \\\hline
3.2.36 & que nos amamos alos bien fechores \textbf{ por la salur } que quiere dezer & credit enim per tales salutem consequi . Ideo dicitur 2 Rhetor’ \textbf{ quod quia diligimus beneficos in salutem , id est eos } qui possunt nobis benefacere nos saluando \\\hline
3.2.36 & por la salur \textbf{ que quiere dezer } que amamos a aquellos que nos pueden fazer bien saluandonos e librado nos . & quod quia diligimus beneficos in salutem , id est eos \textbf{ qui possunt nobis benefacere nos saluando } et liberando , ideo diligimus fortes \\\hline
3.2.36 & que quiere dezer \textbf{ que amamos a aquellos que nos pueden fazer bien saluandonos e librado nos . } Et por ende amamos los fuertes & quod quia diligimus beneficos in salutem , id est eos \textbf{ qui possunt nobis benefacere nos saluando } et liberando , ideo diligimus fortes \\\hline
3.2.36 & e los prinçipes se de una auer \textbf{ por que sean amados del pueblo . fiça deuer en qual manera se de una auer } porque sean temidos del pueblo . & Viso quomodo Reges et Principes debeant se habere ut amentur a populo : \textbf{ videre restat , } quomodo se habere debent \\\hline
3.2.36 & que fazen en los subditos . \textbf{ Mas tres cosas son de penssar } en la pena que dan los prinçipes . & ( ut patet in 2 Rhet’ ) propter punitiones , \textbf{ quas exercent in subditos . In punitione autem tria sunt consideranda videlicet punitionem ipsam , personam punitam , et modum puniendi . Quantum ergo ad punitionem , } timentur Reges et Principes ; \\\hline
3.2.36 & en la pena que dan los prinçipes . \textbf{ Conuiene a saber la pena misma quedan . } Et la perssona a quien la dan . & quas exercent in subditos . In punitione autem tria sunt consideranda videlicet punitionem ipsam , personam punitam , et modum puniendi . Quantum ergo ad punitionem , \textbf{ timentur Reges et Principes ; } si in eos \\\hline
3.2.36 & mas avn por razon delas perssonas a quien la dan . \textbf{ Ca el derechurero por iustiçia non deue perdonar a ninguno . Por ende dize el philosofo en el vij̊ . } libro delas politicas . & sed etiam ratione personarum punitarum . \textbf{ Nam iustus pro iustitia nulli parcere debet . Ideo dicitur 7 Politicorum } quod bene operans nulli parcit : \\\hline
3.2.36 & Ca nin por fijo nin por amigo \textbf{ nin por otro ninguon non deue dexar de fazer iustiçia } e de obrar derechureramente e bien Et pues que assi es estonçe son temidos los Reyes e los prinçipes & nec pro amico , \textbf{ nec pro aliquo alio dimittendum est operari iuste et bene . Tunc itaque ratione personarum punitarum timentur Reges et Principes , } quando nec amicis , \\\hline
3.2.36 & nin por otro ninguon non deue dexar de fazer iustiçia \textbf{ e de obrar derechureramente e bien Et pues que assi es estonçe son temidos los Reyes e los prinçipes } por razon delas perssonas & nec pro amico , \textbf{ nec pro aliquo alio dimittendum est operari iuste et bene . Tunc itaque ratione personarum punitarum timentur Reges et Principes , } quando nec amicis , \\\hline
3.2.36 & que mal fazen . \textbf{ Et pues que assi es cada vno del pueblo teme de mal fazer cuydando } que non podra escapar dela pena . & si viderint eos forefacere . \textbf{ Timet igitur tunc quilibet ex populo forefacere , } cogitans se non posse punitionem effugere . Imo , \\\hline
3.2.36 & Et pues que assi es cada vno del pueblo teme de mal fazer cuydando \textbf{ que non podra escapar dela pena . } Ante assi conmo dize el philosofo & Timet igitur tunc quilibet ex populo forefacere , \textbf{ cogitans se non posse punitionem effugere . Imo , } ut vult Philos’ 7 Polit’ \\\hline
3.2.36 & Lo terçero son temidos los Reyes e los prinçipes \textbf{ por razon dela manera de dar la pena } la qual cosa puede contesçer & quam contra alios . Tertio timentur Reges \textbf{ et Principes ratione modi puniendi : } quod fieri contingit , \\\hline
3.2.36 & por razon dela manera de dar la pena \textbf{ la qual cosa puede contesçer } quando assi se han los sus uiezes & et Principes ratione modi puniendi : \textbf{ quod fieri contingit , } cum ad eorum iudices et praepositos latenter et caute se gerunt in punitionibus exequendis , \\\hline
3.2.36 & quando assi se han los sus uiezes \textbf{ e los sus mjnos ascondidamente e cautelosamente en dar las penas } e en fazer la iustiçia & quod fieri contingit , \textbf{ cum ad eorum iudices et praepositos latenter et caute se gerunt in punitionibus exequendis , } et in iustitia facienda : \\\hline
3.2.36 & e los sus mjnos ascondidamente e cautelosamente en dar las penas \textbf{ e en fazer la iustiçia } que ninguno malo non pueda foyr de sus manos & cum ad eorum iudices et praepositos latenter et caute se gerunt in punitionibus exequendis , \textbf{ et in iustitia facienda : } quod mali effugere non possunt , \\\hline
3.2.36 & e en fazer la iustiçia \textbf{ que ninguno malo non pueda foyr de sus manos } que non aya pena & et in iustitia facienda : \textbf{ quod mali effugere non possunt , } quin puniantur . \\\hline
3.2.36 & que los manifiestos . \textbf{ Visto commo se deuen auer los Reyes e los prinçipes } para que sean amados . & quam manifesti . \textbf{ Viso quomodo Reges | et Principes se habere debeant } ut amentur , \\\hline
3.2.36 & para que sean amados . \textbf{ Et commo para que sean tenidos de ligero puede paresçer } que maguera estas dos cosas sean menester alos prinçipes . & ut amentur , \textbf{ et quomodo ut timeantur : } de leui patere potest quod licet utrunque sit necessarium , \\\hline
3.2.36 & que maguera estas dos cosas sean menester alos prinçipes . \textbf{ Et pero mas deuen ellos dessear de ser amados } que de ser temidos . & et quomodo ut timeantur : \textbf{ de leui patere potest quod licet utrunque sit necessarium , } amari \\\hline
3.2.36 & e de cada vn prinçipe deue ser \textbf{ e non duzir alos otros a uirtud . } Et pues que assi es todo bien fazer & tamen debent \textbf{ magis appetere quam timeri . Dicebatur enim supra quod principalis intentio Regis et cuiuscumque principantis esse debet , inducere alios ad virtutem . Omne ergo bonum per quod ciues sunt } magis boni et virtuosi , \\\hline
3.2.36 & e non duzir alos otros a uirtud . \textbf{ Et pues que assi es todo bien fazer } por el qual los çibdadanos son mas bueons & tamen debent \textbf{ magis appetere quam timeri . Dicebatur enim supra quod principalis intentio Regis et cuiuscumque principantis esse debet , inducere alios ad virtutem . Omne ergo bonum per quod ciues sunt } magis boni et virtuosi , \\\hline
3.2.36 & e por que non fuessen condep̃nados . \textbf{ Et por ende mas deuen dessear los Reyes } e los prinçipes & et ne punirentur ; \textbf{ magis debent appetere . Reges et Principes amari a populis : } et quod amore boni , \\\hline
3.2.36 & que por temor dellos \textbf{ nin que por temor de pena se escusen de fazer malas obras . } Et pues que & quam timeri ab eis , \textbf{ et quod timore poenae cauere sibi ab actibus malis . } Utrumque enim est necessarium , timeri , \\\hline
3.2.36 & e por amor del prinçipe ponedor dela ley \textbf{ cuya entençiones de tener mientes al bien comun } que por ende queden los omes de mal fazer . & et ex dilectione legislatoris , \textbf{ cuius est intendere commune bonum , } quiescant male agere : \\\hline
3.2.36 & cuya entençiones de tener mientes al bien comun \textbf{ que por ende queden los omes de mal fazer . } Por la qual cosa conuiene & cuius est intendere commune bonum , \textbf{ quiescant male agere : } oportuit ergo aliquos inducere ad bonum , \\\hline
3.2.36 & por temor de pena . \textbf{ Enpero mas de escoger es alos Reyes } e alos prinçipes de ser amados & et retrahere a malo timore poenae . Elegibilius \textbf{ tamen est amari , } quam timeri , \\\hline
3.3.1 & Et determinado es \textbf{ en qual manera es de gouernar la çibdat } e el regno en el tienpo de la pas . & quid senserunt antiqui Philosophi de regimine ciuitatis regni , \textbf{ et determinatum est qualiter sit ciuitas atque regnum tempore pacis . } Reliquum est tractare de tempore bellico , \\\hline
3.3.1 & e el regno en el tienpo de la pas . \textbf{ fincan nos de tractar lo terçero de la obra de la batalla . } por que sepan los Reyes e los prinçipes & et determinatum est qualiter sit ciuitas atque regnum tempore pacis . \textbf{ Reliquum est tractare de tempore bellico , } ut sciant Reges \\\hline
3.3.1 & por que sepan los Reyes e los prinçipes \textbf{ en qual manera se deuen acometer las batallas } ca muchas uezes la sabiduri̇a de los lidiadores & et Principes qualiter committenda sint bella . \textbf{ Nam ut plurimum bellorum industria , } ut patet per Vegetium in De re militari , \\\hline
3.3.1 & do tracta del Fecho de la . \textbf{ Mas uale para alcançar uictoria } que la muchedubre o ahun la fortaleza de los lidiadores & ut patet per Vegetium in De re militari , \textbf{ plus confert ad obtinendam victoriam , | quam faciat multitudo } vel fortitudo bellantium . Opus autem bellicum \\\hline
3.3.1 & en lo que se sigue es contienda so la caualleria . \textbf{ Por la qual cosa si queremos tractar de la obra de la batalla . } conuienenos de ver & ( ut patebit in sequendo ) continetur sub militari : \textbf{ quare si de opere bellico tractare volumus , } videndum est \\\hline
3.3.1 & Por la qual cosa si queremos tractar de la obra de la batalla . \textbf{ conuienenos de ver } que cosa es la caualleria & quare si de opere bellico tractare volumus , \textbf{ videndum est } quid sit militia , et ad quid sit instituta . \\\hline
3.3.1 & e para que es establesçida e ordenada . \textbf{ Et pues que assi es deuedes saber } que la caualleria es vna prudençia o vna manera de sabiduria . . & quid sit militia , et ad quid sit instituta . \textbf{ Sciendum igitur militiam esse quandam prudentiam , } siue quandam speciem prudentiae . Possumus autem , \\\hline
3.3.1 & Mas podemos \textbf{ quanto pertenesçe a lo presente departir çinco maneras de prudençia e de sabiduria . } Conuiene saber prudençia singular & Sciendum igitur militiam esse quandam prudentiam , \textbf{ siue quandam speciem prudentiae . Possumus autem , } quantum ad praesens spectat , \\\hline
3.3.1 & quanto pertenesçe a lo presente departir çinco maneras de prudençia e de sabiduria . \textbf{ Conuiene saber prudençia singular } para gouernar cada vno & ø \\\hline
3.3.1 & Conuiene saber prudençia singular \textbf{ para gouernar cada vno } assi mesmo . & siue quandam speciem prudentiae . Possumus autem , \textbf{ quantum ad praesens spectat , } distinguere quinque species prudentiae : \\\hline
3.3.1 & Et prudençia yconomica \textbf{ para gouernar la casa . } Et prudençia regnatiua & quantum ad praesens spectat , \textbf{ distinguere quinque species prudentiae : } videlicet prudentiam singularem , oeconomicam , regnatiuam , politicam siue ciuilem , et militarem . Dicitur enim aliquis habere singularem \\\hline
3.3.1 & Et prudençia regnatiua \textbf{ para gouernar el regno . } Et prudencia politica o çiuil & distinguere quinque species prudentiae : \textbf{ videlicet prudentiam singularem , oeconomicam , regnatiuam , politicam siue ciuilem , et militarem . Dicitur enim aliquis habere singularem } vel particularem prudentiam , \\\hline
3.3.1 & Et prudencia politica o çiuil \textbf{ para gouernar la çibdat . } Et prudençia caualleril & videlicet prudentiam singularem , oeconomicam , regnatiuam , politicam siue ciuilem , et militarem . Dicitur enim aliquis habere singularem \textbf{ vel particularem prudentiam , } quando seipsum \\\hline
3.3.1 & Et prudençia caualleril \textbf{ para gouernar la caualleria . } Ca dicho es & vel particularem prudentiam , \textbf{ quando seipsum } scit regere et gubernare : \\\hline
3.3.1 & que alguno ha sabiduria singular o particular \textbf{ quando sabe gouernar assi mesmo . } Et esta es menor sabiduria & quando seipsum \textbf{ scit regere et gubernare : } et haec est minor prudentia , quam oeconomica et regnatiua . \\\hline
3.3.1 & que es la sabiduria yconomica \textbf{ que es sabiduria de gouernar la casa } e es menor que la regnatiua & ø \\\hline
3.3.1 & e es menor que la regnatiua \textbf{ que es sabiduria de gouernar el regno . } Ca menos es saber gouernar a ssi mismo & et haec est minor prudentia , quam oeconomica et regnatiua . \textbf{ Nam minus est , } scire regere seipsum , quam scire regere familiam , et ciuitatem , \\\hline
3.3.1 & que es sabiduria de gouernar el regno . \textbf{ Ca menos es saber gouernar a ssi mismo } que saber gouernar la conpaña de casa & Nam minus est , \textbf{ scire regere seipsum , quam scire regere familiam , et ciuitatem , } aut regnum . Secunda species prudentiae dicitur esse oeconomica . Nam prudens ex hoc aliquis dicitur , \\\hline
3.3.1 & Ca menos es saber gouernar a ssi mismo \textbf{ que saber gouernar la conpaña de casa } o la cibdat o el regno . & Nam minus est , \textbf{ scire regere seipsum , quam scire regere familiam , et ciuitatem , } aut regnum . Secunda species prudentiae dicitur esse oeconomica . Nam prudens ex hoc aliquis dicitur , \\\hline
3.3.1 & assi commo dize el philosofo en el sesto libro de las Ethicas \textbf{ por que sabe bien consseiar } e bien guiar a buena fin . & ut patet ex septimo Ethicorum : \textbf{ quia scit bene consiliari , } et bene dirigere ad bonum finem : \\\hline
3.3.1 & por que sabe bien consseiar \textbf{ e bien guiar a buena fin . } Et pues que assi es do son falladas departidas razones de bien & quia scit bene consiliari , \textbf{ et bene dirigere ad bonum finem : } ubi ergo reperitur alia \\\hline
3.3.1 & Et por ende la sabiduria \textbf{ por la qual cada vno sabe gouernar la casa e la conpaña . } Conuiene que sea otra e departida de la sabiduria & sicut bonum commune est aliud a bono aliquo singulari , oeconomicam prudentiam , \textbf{ per quam quis scit regere domum et familiam , oportet esse aliam a prudentia , } qua quis nouit seipsum regere . \\\hline
3.3.1 & Conuiene que sea otra e departida de la sabiduria \textbf{ por la qual cada vno sabe gouernar a ssi mismo . } La terçera manera de la sabiduria & per quam quis scit regere domum et familiam , oportet esse aliam a prudentia , \textbf{ qua quis nouit seipsum regere . } Tertia species prudentiae dicitur esse regnatiua vel legum positiua . \\\hline
3.3.1 & que partenesçe al Rey o al prinçipe \textbf{ a quien pertenesçe de fazer leyes } e de gouernar el regno e la çibdat . & idest prudentia quae requiritur in Rege et principante , \textbf{ cuius est leges ferre , } et regere regnum et ciuitatem , \\\hline
3.3.1 & a quien pertenesçe de fazer leyes \textbf{ e de gouernar el regno e la çibdat . } esta es departida de la sabiduria yconomica & cuius est leges ferre , \textbf{ et regere regnum et ciuitatem , } est alia a prudentia oeconomica quae requiritur in patrefamilias , \\\hline
3.3.1 & que es menester en el padre familias \textbf{ a quien pertenesçe de gouernar la casa . } Mas en quanto el bien de la çibdat & ø \\\hline
3.3.1 & que pertenesçe al Rey \textbf{ deue sobrepuiar la sabiduria del gouernamiento de vna casa } o la sabiduria de algun omne particular . & tanto prudentia quae requiritur in Rege oportet \textbf{ excedere prudentiam patrisfamilias , } vel prudentiam alicuius particularis hominis . \\\hline
3.3.1 & por que la su doctrina \textbf{ e el su ensseñamiento pueda aprouechar a todos los sus subditos } Et pues que assi es por ende enssennando los Reyes & vel plura quam Principem , \textbf{ cuius doctrina omnibus prodesse potest . Inde est igitur } quod in erudiendo Reges et Principes , \\\hline
3.3.1 & en quanto el Rey e el prinçipe es vna perssona en ssi \textbf{ e en quanto ha de gouernar a ssi mismo . } Mas en el segundo libro le mostramos ser sabio & prout Rex aut Princeps est quaedam persona in se , \textbf{ et prout habet seipsum regere : } In secundo vero docuimus ipsum \\\hline
3.3.1 & en quanto es padre familias \textbf{ que ha de gouernar la casa } e ha de despenssar los bienes de la casa . & ut est paterfamilias , \textbf{ et ut habet dispensare bona domestica . } In tertio vero eruditur Rex aut Princeps ut est caput regni aut principatus , \\\hline
3.3.1 & que ha de gouernar la casa \textbf{ e ha de despenssar los bienes de la casa . } Et enel teçero libro ensseñamos al Rey o al prinçipe & ut est paterfamilias , \textbf{ et ut habet dispensare bona domestica . } In tertio vero eruditur Rex aut Princeps ut est caput regni aut principatus , \\\hline
3.3.1 & en quanto es cabeça del regno o del prinçipado \textbf{ e en quanto ha de poner leyes } e gouernar los çibdadanos . & In tertio vero eruditur Rex aut Princeps ut est caput regni aut principatus , \textbf{ et ut habet ferre leges } et gubernare ciues . \\\hline
3.3.1 & e en quanto ha de poner leyes \textbf{ e gouernar los çibdadanos . } Et todas estas tres sabidurias conuiene & et ut habet ferre leges \textbf{ et gubernare ciues . } Omnes autem tres prudentias decet habere Regem , \\\hline
3.3.1 & que aya el Rey . \textbf{ Conuiene a saber . } La particular & Omnes autem tres prudentias decet habere Regem , \textbf{ videlicet particularem , } oeconomicam \\\hline
3.3.1 & La particular \textbf{ que muestra gouernar la su perssona . } La yconomica & ø \\\hline
3.3.1 & La yconomica \textbf{ que muestra gouernar su casa . } Et la regnatiua & ø \\\hline
3.3.1 & Et la regnatiua \textbf{ que muestra gouernar el regno . } la quarta manera de sabiduria es dicha politica o çiuil . & et regnatiuam . \textbf{ Quarta species prudentiae dicitur esse politica siue ciuilis . } Nam sicut in principante requiritur excellens prudentia \\\hline
3.3.1 & Ca assi commo en el principe es neçessaria mayor sabiduria \textbf{ por la qual sepa guiar e gouernar los otros . } assi en cada cibdadno es meester alguna sabiduria & Nam sicut in principante requiritur excellens prudentia \textbf{ qua sciat alios regere , } sic in quolibet ciue requiritur prudentia aliqualis qua noscat adimplere leges \\\hline
3.3.1 & assi en cada cibdadno es meester alguna sabiduria \textbf{ por la qual sepa conplir las leyes e los mandamientos del principe } Ca non sirue & qua sciat alios regere , \textbf{ sic in quolibet ciue requiritur prudentia aliqualis qua noscat adimplere leges } et mandata principantis . Non enim sic obsequitur ciuis principanti aut Regi , \\\hline
3.3.1 & Ca otra cosa es \textbf{ que el omne se sepa gouernar } en quanto es alguna cosa en ssi . & quam collocauimus in prima specie . \textbf{ Nam aliud est quod sciat se regere } ut est aliquid in se , \\\hline
3.3.1 & en quanto es alguna cosa en ssi . \textbf{ Et otra cosa es que se sepa gouernar } en quanto es subiecto al prinçipe . & ut est aliquid in se , \textbf{ et aliud ut subiectus principanti . } Nam \\\hline
3.3.1 & Ca si alguno tomasse uida apartada \textbf{ e morasse solo avn conuenir le ya de auer alguna sabiduria } por la qual se sopiesse gouernar . & et si quis solitariam vitam duceret , \textbf{ adhuc oporteret ipsum habere aliqualem prudentiam qua sciret se regere } et gubernare : \\\hline
3.3.1 & e morasse solo avn conuenir le ya de auer alguna sabiduria \textbf{ por la qual se sopiesse gouernar . } Enpero non seria en el sabiduria çiuil & adhuc oporteret ipsum habere aliqualem prudentiam qua sciret se regere \textbf{ et gubernare : } non tamen esset in ipso prudentia ciuilis , \\\hline
3.3.1 & e venzca las cosas \textbf{ qual pueden enpeesçer e enbargar . } Et pues que assi es la sabiduria de la caualleria & et per operationem bellicam aggrediatur , \textbf{ et superet impedientia et prohibentia . } Militaris ergo est quaedam species prudentiae , \\\hline
3.3.1 & es vna manera de sabiduria \textbf{ por la qual se pueden uençer los enemigos } e los que enbarguan el bien comun & Militaris ergo est quaedam species prudentiae , \textbf{ per quam superantur hostes } et prohibentes bonum ciuile \\\hline
3.3.1 & por la turbacion e por el departimiento de los cibdadanos . \textbf{ Et otrossi por el agrauiamiento de las perssonas flacas podemos dezir } que assi commo el fuerte pertenesçe prinçipalmente de saber bien en obras de batallas . & ex consequenti vero ex seditione ipsorum ciuium , \textbf{ et ex oppressione debilium personarum , dicere possumus quod sicut ad fortem } principaliter spectat bene se habere in opere bellico , \\\hline
3.3.1 & Et otrossi por el agrauiamiento de las perssonas flacas podemos dezir \textbf{ que assi commo el fuerte pertenesçe prinçipalmente de saber bien en obras de batallas . } Et de si a esse mismo pertenesçe & et ex oppressione debilium personarum , dicere possumus quod sicut ad fortem \textbf{ principaliter spectat bene se habere in opere bellico , } ex consequenti vero spectat \\\hline
3.3.1 & Et de si a esse mismo pertenesçe \textbf{ de se auer bien en las otras obras espantables } que pueden poner miedo . & ex consequenti vero spectat \textbf{ ad ipsum bene se habere in aliis terribilibus sic etiam ad milites principaliter spectat bene se habere in opere bellico , | et per actiones bellicas opprimere impedimenta hostium : } ex consequenti vero spectat \\\hline
3.3.1 & de se auer bien en las otras obras espantables \textbf{ que pueden poner miedo . } assi a los caualleros pertenesçe principalmente de se auer bien en obras de batallas . & et per actiones bellicas opprimere impedimenta hostium : \textbf{ ex consequenti vero spectat } ad ipsos \\\hline
3.3.1 & que pueden poner miedo . \textbf{ assi a los caualleros pertenesçe principalmente de se auer bien en obras de batallas . } Et por las obras de batallas desenbargar todos aquellos enbargos & ex consequenti vero spectat \textbf{ ad ipsos } secundum iussionem regiam \\\hline
3.3.1 & assi a los caualleros pertenesçe principalmente de se auer bien en obras de batallas . \textbf{ Et por las obras de batallas desenbargar todos aquellos enbargos } que pueden venir de los enemigos . & ex consequenti vero spectat \textbf{ ad ipsos } secundum iussionem regiam \\\hline
3.3.1 & Et por las obras de batallas desenbargar todos aquellos enbargos \textbf{ que pueden venir de los enemigos . } Et de si a ellos pertenesce desenbargar & ad ipsos \textbf{ secundum iussionem regiam } et \\\hline
3.3.1 & que pueden venir de los enemigos . \textbf{ Et de si a ellos pertenesce desenbargar } e tirar todas las discordias de los çibdadanos & secundum iussionem regiam \textbf{ et | secundum mandata principantis impedire } omnes seditiones ciuium \\\hline
3.3.1 & Et de si a ellos pertenesce desenbargar \textbf{ e tirar todas las discordias de los çibdadanos } e todos agrauiamientos aquellos que son el regno segunt & secundum mandata principantis impedire \textbf{ omnes seditiones ciuium } et omnes oppressiones eorum \\\hline
3.3.1 & e todos agrauiamientos aquellos que son el regno segunt \textbf{ por las quales cosas se puede turbar la paz . } el assessiego de los çibdadanos e el bien comun . & qui sunt in regno , \textbf{ per quas turbari potest tranquillitas ciuium } et commune bonum . Hanc autem prudentiam \\\hline
3.3.1 & el assessiego de los çibdadanos e el bien comun . \textbf{ Et esto deue fazer los caualleros } segunt el mandamieto del Reyo & per quas turbari potest tranquillitas ciuium \textbf{ et commune bonum . Hanc autem prudentiam } videlicet militarem , \\\hline
3.3.1 & Ca commo quier que pertenezca a los caualleros la essecuçion de las batallas \textbf{ e tirar e arredrar los enbargos del bien comun . } Et tales cosas & maxime decet habere Regem . \textbf{ Nam licet executio bellorum , et remouere impedimenta ipsius communis boni , } spectet ad ipsos milites , \\\hline
3.3.1 & commo estas pertenezcan a aquellos \textbf{ a quien lo quisiere acomendar el rey o el prinçipe } Enpero saber & et etiam ad eos \textbf{ quibus ipse Rex aut Princeps voluerit committere talia : } scire \\\hline
3.3.1 & a quien lo quisiere acomendar el rey o el prinçipe \textbf{ Enpero saber } en qual manera son de acometer las batallas & quibus ipse Rex aut Princeps voluerit committere talia : \textbf{ scire } tamen quomodo committenda sint bella , \\\hline
3.3.1 & Enpero saber \textbf{ en qual manera son de acometer las batallas } e en qual manera se pueden sabiamente tirar & scire \textbf{ tamen quomodo committenda sint bella , } et qualiter caute remoueri possint impedientia commune bonum , \\\hline
3.3.1 & en qual manera son de acometer las batallas \textbf{ e en qual manera se pueden sabiamente tirar } e arredrar los enbargos que son contra el bien comun esto pertenesçe prinçipalmente al rey o al prinçipe . & tamen quomodo committenda sint bella , \textbf{ et qualiter caute remoueri possint impedientia commune bonum , } maxime spectat ad principantem . \\\hline
3.3.1 & e en qual manera se pueden sabiamente tirar \textbf{ e arredrar los enbargos que son contra el bien comun esto pertenesçe prinçipalmente al rey o al prinçipe . } Et pues que assi es desto puede paresçer & et qualiter caute remoueri possint impedientia commune bonum , \textbf{ maxime spectat ad principantem . } Ex hoc ergo patere potest , \\\hline
3.3.1 & e arredrar los enbargos que son contra el bien comun esto pertenesçe prinçipalmente al rey o al prinçipe . \textbf{ Et pues que assi es desto puede paresçer } quales son aquellos & maxime spectat ad principantem . \textbf{ Ex hoc ergo patere potest , } quales sint ad militiam admittendi . \\\hline
3.3.1 & quales son aquellos \textbf{ que son de tomar para caualleria . } Ca la caualleria paresçe ser alguna sabiduria de la obra de la batalla ordenada a bien comun & Ex hoc ergo patere potest , \textbf{ quales sint ad militiam admittendi . } Nam militia videtur esse quaedam prudentia operis bellici , \\\hline
3.3.1 & por que paresçe \textbf{ que assi se deuen auer los caualleros } en la obra de la batalla & ordinata ad commune bonum : \textbf{ videntur enim se habere milites in opere bellico , } sicut magistri et doctores in scientiis aliis . Quare \\\hline
3.3.1 & Por la qual cosa \textbf{ assi commo ninguno non es de tomar } para maestro en las otras sçiençias & sicut magistri et doctores in scientiis aliis . Quare \textbf{ sicut nullus efficiendus est magister in aliis scientiis , } nisi constet ipsum esse doctum in arte illa : \\\hline
3.3.1 & que el sea bueno en la obra de la batalla \textbf{ e que quiera segunt el mandamiento del prinçipe desenbargar las discordias de los cibdidanos } e lidiar por la iustiçia & quod sit bonus in opere bellico ; \textbf{ et quod velit | secundum iussionem principantis impedire seditiones ciuium , } pugnare pro iustitia \\\hline
3.3.1 & e que quiera segunt el mandamiento del prinçipe desenbargar las discordias de los cibdidanos \textbf{ e lidiar por la iustiçia } e con todo su poder tirar & secundum iussionem principantis impedire seditiones ciuium , \textbf{ pugnare pro iustitia } et pro iuribus , \\\hline
3.3.1 & e lidiar por la iustiçia \textbf{ e con todo su poder tirar } e arredrar & ø \\\hline
3.3.1 & e con todo su poder tirar \textbf{ e arredrar } quales se quier cosas & pugnare pro iustitia \textbf{ et pro iuribus , } remouere quaecunque impedire possunt commune bonum . \\\hline
3.3.1 & que enbarguen el bien comun . \textbf{ Et desto puede parescer } que toda obra de batalla e de guerra se contiene & et pro iuribus , \textbf{ remouere quaecunque impedire possunt commune bonum . } Ex hoc etiam patere potest omnem bellicam operationem contineri sub militari . \\\hline
3.3.1 & so la arte de la caualleria \textbf{ ca commo quier que contezca de lidiar los peones } e los omnes de cauallo & Ex hoc etiam patere potest omnem bellicam operationem contineri sub militari . \textbf{ Nam licet bellare contingat homines pedites , } vel etiam equestres non existentes milites : \\\hline
3.3.2 & p paresçe que quanto pertenesçe a lo psente dos cosas son meester en la obra de la batallas . \textbf{ Conuiene a saber fortaleza para lidiar et sabiduria en batallas . } por la qual cosa & in opere bellico duo necessaria esse , \textbf{ strenuitas bellandi , | et prudentia erga bella . } Quare si scire volumus in quibus regionibus meliores sunt bellatores , \\\hline
3.3.2 & por la qual cosa \textbf{ si queremos saber } en quales regnos o en quales tierras son meiores lidiadores . & et prudentia erga bella . \textbf{ Quare si scire volumus in quibus regionibus meliores sunt bellatores , } oportet attendere circa praedicta duo . In partibus igitur nimis propinquis soli , \\\hline
3.3.2 & en quales regnos o en quales tierras son meiores lidiadores . \textbf{ Conuiene de tener mientes en estas dos cosas sobredichas . } Et pues que asy es en las partes & Quare si scire volumus in quibus regionibus meliores sunt bellatores , \textbf{ oportet attendere circa praedicta duo . In partibus igitur nimis propinquis soli , } non sunt eligendi bellantes : \\\hline
3.3.2 & que son muy cercanas al sol \textbf{ non son de escoger los lidiadores } por que en aquellas tierrras fallesçe fortaleza & oportet attendere circa praedicta duo . In partibus igitur nimis propinquis soli , \textbf{ non sunt eligendi bellantes : } quia in eis deficit strenuitas \\\hline
3.3.2 & e han menos de sangre . \textbf{ Et por ende non han firmeza para lidiar } nin fiuza para vençer & nimio calore siccare ; amplius quidem sapiunt , sed modicum abundant in sanguine , \textbf{ et propterea non habent constantiam pugnandi neque fiduciam , } quia naturaliter metuunt vulnera . \\\hline
3.3.2 & Et por ende non han firmeza para lidiar \textbf{ nin fiuza para vençer } ca estos tales naturalmente han miedo de las feridas . & nimio calore siccare ; amplius quidem sapiunt , sed modicum abundant in sanguine , \textbf{ et propterea non habent constantiam pugnandi neque fiduciam , } quia naturaliter metuunt vulnera . \\\hline
3.3.2 & ca estos tales naturalmente han miedo de las feridas . \textbf{ Ca por que naturalmente han poca sangre naturalmente temen de perder la sangre . } Et por ende estos non son apareiados & quia naturaliter metuunt vulnera . \textbf{ Nam cum naturaliter habeant modicum sanguinis , | naturaliter timent sanguinis amissionem : } non ergo sunt prompta ad bella , \\\hline
3.3.2 & e muy arredradas del sol \textbf{ non son de escoger los lidiadores por que commo quier que en aquellas abonde mucho la sangre } assi que non teman las feridas . & et nimis a sole remotis , \textbf{ non sunt eligendi bellantes : | quia et si illis est sanguinis copia , } ut vulnera non metuant , \\\hline
3.3.2 & commo la sabiduria en las batallas siguese \textbf{ que de ningunas destas tierras non son de escoger los lidiadores } mas de las tierras medianeras son de escoger los lidiadores & necessaria est in bellis , \textbf{ ex neutris partibus eligendi sunt bellatores ; } sed ex regione media , \\\hline
3.3.2 & que de ningunas destas tierras non son de escoger los lidiadores \textbf{ mas de las tierras medianeras son de escoger los lidiadores } que non son muy arredrados & ex neutris partibus eligendi sunt bellatores ; \textbf{ sed ex regione media , } nec omnino a sole remota , \\\hline
3.3.2 & commo por fortaleza de coraçon ayan auantaia de los otros . \textbf{ Enpero deuemos tener mientes en estos tales ensseñamientos } que se deuen entender en la mayor parte . & nec omnino a sole remota , \textbf{ nec omnino soli propinqua eligendi sunt bellantes , } ut tam prudentia quam animositate participent . Aduertendum tamen circa talia , documenta accipienda esse ut in pluribus . Nam in omnibus partibus sunt aliqui industres , \\\hline
3.3.2 & Enpero deuemos tener mientes en estos tales ensseñamientos \textbf{ que se deuen entender en la mayor parte . } Ca en todas las partes son sabidores & nec omnino soli propinqua eligendi sunt bellantes , \textbf{ ut tam prudentia quam animositate participent . Aduertendum tamen circa talia , documenta accipienda esse ut in pluribus . Nam in omnibus partibus sunt aliqui industres , } et aliqui animosi : \\\hline
3.3.2 & Mas los que son arredrados del fallesçen en prudençia e en sabiduria . \textbf{ avn deuemos tener mientes } que commo quier & ut plurimum tamen soli propinqui animositate deficiunt , \textbf{ remoti vero prudentia . Aduertendum } etiam quod licet in bellis tam animositas , \\\hline
3.3.2 & Et entre las gentes medianeras \textbf{ mas de escoger son } para las obras de las batallas & et omnino remotae non sunt penitus utiles actibus bellicis : \textbf{ magis tamen inter medias regiones eligendi sunt ad opera bellica remotiores a sole , } quam propinquiores , \\\hline
3.3.2 & Visto de quales partes son los meiores lidiadores \textbf{ finca de ver } de quales artes son de escoger los lidiadores . & ex quibus partibus meliores sunt bellatores : \textbf{ videre restat , } ex quibus partibus eligendi sunt bellantes . \\\hline
3.3.2 & finca de ver \textbf{ de quales artes son de escoger los lidiadores . } Et por ende son de contar aquellas cosas & videre restat , \textbf{ ex quibus partibus eligendi sunt bellantes . } Enumeranda sunt igitur ea quae requiruntur in hominibus bellicosis , \\\hline
3.3.2 & de quales artes son de escoger los lidiadores . \textbf{ Et por ende son de contar aquellas cosas } que son menester en los omnes lidiadores & ex quibus partibus eligendi sunt bellantes . \textbf{ Enumeranda sunt igitur ea quae requiruntur in hominibus bellicosis , } ut sciamus quales homines sunt eligendi ad bellum , \\\hline
3.3.2 & que son menester en los omnes lidiadores \textbf{ por que sepamos quales omnes son de escoger para la batalla } e de quales artes son de tomar los lidiadores . & Enumeranda sunt igitur ea quae requiruntur in hominibus bellicosis , \textbf{ ut sciamus quales homines sunt eligendi ad bellum , } et ex quibus artibus sunt assumendi bellantes . \\\hline
3.3.2 & por que sepamos quales omnes son de escoger para la batalla \textbf{ e de quales artes son de tomar los lidiadores . } Et pues que assi es conuiene de saber & ut sciamus quales homines sunt eligendi ad bellum , \textbf{ et ex quibus artibus sunt assumendi bellantes . } Sciendum ergo quod cum bellantes debeant habere membra apta et assueta ad percutiendum , \\\hline
3.3.2 & e de quales artes son de tomar los lidiadores . \textbf{ Et pues que assi es conuiene de saber } que commo los lidiadores deuan auer los mienbros apareiados & et ex quibus artibus sunt assumendi bellantes . \textbf{ Sciendum ergo quod cum bellantes debeant habere membra apta et assueta ad percutiendum , } non debeant horrere sanguinis effusionem , \\\hline
3.3.2 & Et pues que assi es conuiene de saber \textbf{ que commo los lidiadores deuan auer los mienbros apareiados } e acostunbrados a ferir & et ex quibus artibus sunt assumendi bellantes . \textbf{ Sciendum ergo quod cum bellantes debeant habere membra apta et assueta ad percutiendum , } non debeant horrere sanguinis effusionem , \\\hline
3.3.2 & que commo los lidiadores deuan auer los mienbros apareiados \textbf{ e acostunbrados a ferir } et non deuan aborresçer el derramamiento de la sangre & ø \\\hline
3.3.2 & e acostunbrados a ferir \textbf{ et non deuan aborresçer el derramamiento de la sangre } e deuan ser animosos & Sciendum ergo quod cum bellantes debeant habere membra apta et assueta ad percutiendum , \textbf{ non debeant horrere sanguinis effusionem , } debeant esse animosi ad inuadendum , \\\hline
3.3.2 & e de grant coraçon \textbf{ para acometer } e deuan ser poderosos para sofrir los trabaios . & debeant esse animosi ad inuadendum , \textbf{ et etiam potentes ad tolerandum labores : } dicere possumus quod Fabriferrarii , \\\hline
3.3.2 & para acometer \textbf{ e deuan ser poderosos para sofrir los trabaios . } Podemos dezir que los ferreros e los carpenteros son aprouechables a las obras de la batalla & debeant esse animosi ad inuadendum , \textbf{ et etiam potentes ad tolerandum labores : } dicere possumus quod Fabriferrarii , \\\hline
3.3.2 & e deuan ser poderosos para sofrir los trabaios . \textbf{ Podemos dezir que los ferreros e los carpenteros son aprouechables a las obras de la batalla } por que por la su arte han los braços acostunbrados e apareiados para ferir . & et etiam potentes ad tolerandum labores : \textbf{ dicere possumus quod Fabriferrarii , | et carpentarii utiles sunt ad opera bellica : } quia ex arte sua habent brachia apta \\\hline
3.3.2 & Podemos dezir que los ferreros e los carpenteros son aprouechables a las obras de la batalla \textbf{ por que por la su arte han los braços acostunbrados e apareiados para ferir . } Avn en essa misma manera son aprouechables los carniceros & et carpentarii utiles sunt ad opera bellica : \textbf{ quia ex arte sua habent brachia apta | et assueta ad percutiendum . Sic } etiam utiles sunt Macellarii : \\\hline
3.3.2 & por que non aborresçen el derramamiento de la sangre \textbf{ por que son acostunbrados a matar las animalias } e a esparzer la sangre ellas . & etiam utiles sunt Macellarii : \textbf{ quia non horrent sanguinis effusionem , cum assueti sint ad occisionem animalium , } et ad effundendum sanguinem Venatores \\\hline
3.3.2 & por que son acostunbrados a matar las animalias \textbf{ e a esparzer la sangre ellas . } Avn los caçadores de los puercos monteses & quia non horrent sanguinis effusionem , cum assueti sint ad occisionem animalium , \textbf{ et ad effundendum sanguinem Venatores } etiam aprorum admittendi sunt ad huiusmodi opera : \\\hline
3.3.2 & Avn los caçadores de los puercos monteses \textbf{ son de resçebir a las obras de la batalla } por que non pueden sin grant osadia acometer los puercos monteses & et ad effundendum sanguinem Venatores \textbf{ etiam aprorum admittendi sunt ad huiusmodi opera : } quia non sine magna audacia contingit aliquos inuadere apros . \\\hline
3.3.2 & son de resçebir a las obras de la batalla \textbf{ por que non pueden sin grant osadia acometer los puercos monteses } e les otros fuertes venados . & etiam aprorum admittendi sunt ad huiusmodi opera : \textbf{ quia non sine magna audacia contingit aliquos inuadere apros . } Sunt ergo tales animosi \\\hline
3.3.2 & e estremados para la batalla \textbf{ ante por auentura non es menor peligro lidiar con el puerco montes } que lidiar con el enemigo . & et strenui ad bellandum . \textbf{ Imo forte non minus periculosum est bellare cum apro , } quam pugnare cum hoste . \\\hline
3.3.2 & ante por auentura non es menor peligro lidiar con el puerco montes \textbf{ que lidiar con el enemigo . } Por ende los que non temen los periglos de los puercos monteses & Imo forte non minus periculosum est bellare cum apro , \textbf{ quam pugnare cum hoste . } Nam non timentes aprorum pericula , \\\hline
3.3.2 & señal es que non temerien lasbatallas de los enemigos . \textbf{ Otrossi los caçadores de los çieruos non son de refusar } para las obras de la batalla & signum est eos non timere hostium bella . \textbf{ Rursus venatores ceruorum non sunt repudiandi ab actibus bellis : } eo quod tales assueti sunt ad labores nimios . \\\hline
3.3.2 & por que tales son acostunbrados a grandes trabaios . \textbf{ Et por ende destas tales artes son de escoger los lidiadores } por aquellas cosas que ya dixiemos . & eo quod tales assueti sunt ad labores nimios . \textbf{ Ex his ergo artibus propter ea quae diximus eligendi sunt bellatores . Barbitonsores autem et sutores , } si consideretur ars propria , \\\hline
3.3.2 & non son aprouechables para la batalla \textbf{ por que nunca bien leuantar a la maça } nin esgrimira la espada & ad pugnam sunt inutiles . \textbf{ Nam nunquam bene vibrant clauam , } aut ensem \\\hline
3.3.2 & nin esgrimira la espada \textbf{ aquel que deue auer la mano liuiana . } Et non es acostubrado de tener en la mano & aut ensem \textbf{ qui debet habere manum leuem , } et non est assuetus retinere in manibus nisi rasorium aut acum . \\\hline
3.3.2 & aquel que deue auer la mano liuiana . \textbf{ Et non es acostubrado de tener en la mano } si non su nauaia o su aguia & qui debet habere manum leuem , \textbf{ et non est assuetus retinere in manibus nisi rasorium aut acum . } Quae enim proportio est acus ad lanceam , \\\hline
3.3.2 & Avn en essa misma manera los apotecarios e los paxareros e los bretadores \textbf{ que toman los paxaros con el brete e los pescadores no son do escoger } para las obras de la batalla & et rasorii ad clauam ? \textbf{ Sic etiam Apothecarii , Aucupes , et Piscatores non sunt eligendi ad huiusmodi opera : } eo quod non habeant artem conformem operibus bellicosis . Potest ergo contingere \\\hline
3.3.2 & por que non han la arte semeiable a las obras de la batalla . \textbf{ Enpero puede contesçer } que en cada vna destas artes son algunos buenos lidiadores e atreuidos & eo quod non habeant artem conformem operibus bellicosis . Potest ergo contingere \textbf{ quod in qualibet arte sint aliqui bellicosi et audaces ; aliqui vero timidi } et pusillanimes . \\\hline
3.3.3 & e nos delectamos en ellas . \textbf{ Et si quisiere el ponedor de la ley fazer los çibdadanos buenos lidiadores } e fazer los apareiados & et delectamur in illis : \textbf{ si vult legislator ciues bellatores facere , } et reddere ipsos aptos ad pugnandum , potius debet praeuenire tempus quam praetermittere . \\\hline
3.3.3 & Et si quisiere el ponedor de la ley fazer los çibdadanos buenos lidiadores \textbf{ e fazer los apareiados } para la batalla deue ante tomar el tienpo de la mançebia & et delectamur in illis : \textbf{ si vult legislator ciues bellatores facere , } et reddere ipsos aptos ad pugnandum , potius debet praeuenire tempus quam praetermittere . \\\hline
3.3.3 & e fazer los apareiados \textbf{ para la batalla deue ante tomar el tienpo de la mançebia } que lo dexar passar . & si vult legislator ciues bellatores facere , \textbf{ et reddere ipsos aptos ad pugnandum , potius debet praeuenire tempus quam praetermittere . } Nam \\\hline
3.3.3 & para la batalla deue ante tomar el tienpo de la mançebia \textbf{ que lo dexar passar . } Ca assi commo dize & si vult legislator ciues bellatores facere , \textbf{ et reddere ipsos aptos ad pugnandum , potius debet praeuenire tempus quam praetermittere . } Nam \\\hline
3.3.3 & que el mançebo usado cuyde \textbf{ que la hedat de lidiar non es venida ante } que se duela & Nam \textbf{ ut ait Vegetius , melius est ut iuuenis exercitatus causetur aetatem nondum aduenisse pugnandi gratia , quam doleat praeteriisse . Est } etiam specialis ratio , \\\hline
3.3.3 & por que conuiene que los mançebos en el comienço de la su moçedat \textbf{ e de la su maçebia se acostunbren a la arte de lidiar } por que non es pequena & quare oporteat iuuenes \textbf{ ab ipsa iuuentute assuescere ad artem bellandi : } quia non parua nec leuis ars esse videtur armorum industria . \\\hline
3.3.3 & por que non es pequena \textbf{ nin ligera arte auer sabiduria de las armas . } Ca si quier sea cauallero si quier peon el que ha de lidiar paresçe & ab ipsa iuuentute assuescere ad artem bellandi : \textbf{ quia non parua nec leuis ars esse videtur armorum industria . } Nam siue equitem siue peditem oportet esse bellantem , \\\hline
3.3.3 & nin ligera arte auer sabiduria de las armas . \textbf{ Ca si quier sea cauallero si quier peon el que ha de lidiar paresçe } que por uentura alcaça uictoria & quia non parua nec leuis ars esse videtur armorum industria . \textbf{ Nam siue equitem siue peditem oportet esse bellantem , } quasi fortuito videtur peruenire ad palmam , \\\hline
3.3.3 & que por uentura alcaça uictoria \textbf{ si non ouiere sabiduria de lidiar } por que tan bien en la batalla de la caualleria & quasi fortuito videtur peruenire ad palmam , \textbf{ si caret industria bellandi . } Tam enim in pedestri quam etiam in equestri pugna , \\\hline
3.3.3 & por que tan bien en la batalla de la caualleria \textbf{ commo de los peones son de catar muchas cautelas } por que çiertamente grand locura es aquella hora & Tam enim in pedestri quam etiam in equestri pugna , \textbf{ sunt multae adhibendae cautelae . Fatuum est quidem non prius , } sed tunc velle addiscere bellare , quando imminet necessitas pugnandi , \\\hline
3.3.3 & por que çiertamente grand locura es aquella hora \textbf{ o aquel tienpo aprender de lidiar } e non ante que sea el periglo de la batalla & sunt multae adhibendae cautelae . Fatuum est quidem non prius , \textbf{ sed tunc velle addiscere bellare , quando imminet necessitas pugnandi , } ubi vita periculis mortis exponitur . \\\hline
3.3.3 & e non ante que sea el periglo de la batalla \textbf{ o la neçessidat de lidiar } do la uida se espone a periglo de la muerte & sed tunc velle addiscere bellare , quando imminet necessitas pugnandi , \textbf{ ubi vita periculis mortis exponitur . } Ut ergo bellatores habeant spatium addiscendi singula , \\\hline
3.3.3 & Pues que assi es \textbf{ por que los lidiadores ayan espaçio de aprender todas aquellas cosas } que son menester & ubi vita periculis mortis exponitur . \textbf{ Ut ergo bellatores habeant spatium addiscendi singula , } quae requiruntur ad bellum ; ab ipsa pubertate assuescendi sunt ad opera bellica . \\\hline
3.3.3 & que son menester \textbf{ para la batalla de su mancebia son de enssennar } e de se acostunbrar & Ut ergo bellatores habeant spatium addiscendi singula , \textbf{ quae requiruntur ad bellum ; ab ipsa pubertate assuescendi sunt ad opera bellica . } Quare si legislator \\\hline
3.3.3 & para la batalla de su mancebia son de enssennar \textbf{ e de se acostunbrar } e de se usar & ø \\\hline
3.3.3 & e de se acostunbrar \textbf{ e de se usar } a las obras de la batalla . & ø \\\hline
3.3.3 & si el fazedor de la ley \textbf{ assi commo el Rey o el prinçipe ouiere de acometer batalla deue tomar } e escoger varones vsados & Quare si legislator \textbf{ ut Rex aut Princeps debeat committere bellum , } viros exercitatos et bellatores strenuos debet assumere . \\\hline
3.3.3 & assi commo el Rey o el prinçipe ouiere de acometer batalla deue tomar \textbf{ e escoger varones vsados } e lidiadores escogidos e fuertes . & ut Rex aut Princeps debeat committere bellum , \textbf{ viros exercitatos et bellatores strenuos debet assumere . } Viso in qua aetate assuescendi sunt \\\hline
3.3.3 & e lidiadores escogidos e fuertes . \textbf{ Visto en qual hedat son de acostunbrar a obras de batalla } aquellos que se deuen fazir caualleros e ser lidiadores . & viros exercitatos et bellatores strenuos debet assumere . \textbf{ Viso in qua aetate assuescendi sunt } qui debent effici bellatores \\\hline
3.3.3 & Visto en qual hedat son de acostunbrar a obras de batalla \textbf{ aquellos que se deuen fazir caualleros e ser lidiadores . } finca de ver & Viso in qua aetate assuescendi sunt \textbf{ qui debent effici bellatores | ad actiones bellicas : } videre restat , \\\hline
3.3.3 & aquellos que se deuen fazir caualleros e ser lidiadores . \textbf{ finca de ver } por quales señales se han de conosçer los buenos lidiadores . & ad actiones bellicas : \textbf{ videre restat , } ex quibus signis cognosci habeant homines bellicosi . \\\hline
3.3.3 & finca de ver \textbf{ por quales señales se han de conosçer los buenos lidiadores . } Et para esto conuiene de saber & videre restat , \textbf{ ex quibus signis cognosci habeant homines bellicosi . } Sciendum igitur viros audaces et cordatos utiliores esse ad bellum , \\\hline
3.3.3 & por quales señales se han de conosçer los buenos lidiadores . \textbf{ Et para esto conuiene de saber } que los omnes osados eatreuidos & videre restat , \textbf{ ex quibus signis cognosci habeant homines bellicosi . } Sciendum igitur viros audaces et cordatos utiliores esse ad bellum , \\\hline
3.3.3 & por que son mas poderosos en las uirtudes corporales \textbf{ son mas de escoger } para la obra de la batalla . & ø \\\hline
3.3.3 & Et pues que asy es \textbf{ por tres maneras de señales podemos conosçer los omnes lidiadores } Lo primero por aquellas señales & Tribus igitur generibus signorum \textbf{ cognoscere possumus bellicosos viros . } Primo quidem per signa , \\\hline
3.3.3 & por las quales se muestra fortaleza de coraçon e osadia . \textbf{ Et estas son auer los oios bien biuos e bien despiertos } e la çeruiz leuantada e derecha & ø \\\hline
3.3.3 & que el mançebo \textbf{ que es de escoger } para la obra de la batalla deue ser tal que aya los oios bien abiertos e bien despiertos & secundum quae conformamur animalibus bellicosis . Sunt autem signa , \textbf{ per quae ostenditur animositas } et strenuitas cordis , vigilantia oculorum , et erectio ceruicis . Ideo dicit Vegetius , \\\hline
3.3.3 & Ca segunt el philosofo en el . viij° . libro de las . \textbf{ la obra de batallar } e la sabiduria del alma & Nam secundum Philos’ \textbf{ 8 Polit’ opus bellicosum , } et industria mentis , \\\hline
3.3.3 & e ha los braços luengos e los pechos anchos . \textbf{ estos tales deuemos iudgar } por lidiadores e apareiados para la batalla . & debemus arguere ipsum esse bellicosum , \textbf{ et aptum ad pugnam . } Tales ergo quaerendi sunt bellatores , \\\hline
3.3.3 & por lidiadores e apareiados para la batalla . \textbf{ Et pues que assi es tales son de escoger los lidiadores } que por la mayor parte sean apareiados & et aptum ad pugnam . \textbf{ Tales ergo quaerendi sunt bellatores , } quia ut plurimum contingit eos esse aptos ad actiones bellicas . \\\hline
3.3.4 & para las obras de la batalla . \textbf{ q quanto pertenesçe a lo presente ocho cosas podemos contar } que deuen auer los omnes lidiadores & quia ut plurimum contingit eos esse aptos ad actiones bellicas . \textbf{ Quantum ad praesens spectat , | enumerare possumus octo , } quae habere debent homines bellatores : \\\hline
3.3.4 & q quanto pertenesçe a lo presente ocho cosas podemos contar \textbf{ que deuen auer los omnes lidiadores } segunt las quales cosas & enumerare possumus octo , \textbf{ quae habere debent homines bellatores : } secundum quae ( quantum ad praesens spectat ) \\\hline
3.3.4 & segunt las quales cosas \textbf{ quanto pertenesçe alo presente podremos buscar } que omnes & quae habere debent homines bellatores : \textbf{ secundum quae ( quantum ad praesens spectat ) | inuestigare poterimus } quos aut quales bellatores debet Rex \\\hline
3.3.4 & que omnes \textbf{ e quales deuen escoger el Rey o el principe } para la batalla & inuestigare poterimus \textbf{ quos aut quales bellatores debet Rex } aut Princeps eligere . \\\hline
3.3.4 & para la batalla \textbf{ Lo primero conuiene que los omnes lidiadores puedan sofrir grandes pesos . } lo segundo que puedan sofrir grandes trabaios & aut Princeps eligere . \textbf{ Primo enim oportet pugnatiuos homines posse sustinere magnitudinem ponderis . Secundo posse sufferre } quasi assiduos membrorum motus , \\\hline
3.3.4 & Lo primero conuiene que los omnes lidiadores puedan sofrir grandes pesos . \textbf{ lo segundo que puedan sofrir grandes trabaios } e continuados mouimientos de los mienbros & aut Princeps eligere . \textbf{ Primo enim oportet pugnatiuos homines posse sustinere magnitudinem ponderis . Secundo posse sufferre } quasi assiduos membrorum motus , \\\hline
3.3.4 & e continuados mouimientos de los mienbros \textbf{ Lo terçero que puedan sofrir escasseza de vianda e fanbre e sed . } Lo quarto & quasi assiduos membrorum motus , \textbf{ et labores magnos . | Tertio posse tolerare parcitatem victus . } Quarto non curare de incommoditate iacendi \\\hline
3.3.4 & Lo quarto \textbf{ que non aya cuydado de mal yazer } nin de mal estar & Tertio posse tolerare parcitatem victus . \textbf{ Quarto non curare de incommoditate iacendi } et standi . \\\hline
3.3.4 & Lo sexto que non teman \textbf{ nin aborrezcan de derramar su sangreLo septimo conuiene } que ayan buena disposiçion e buena sabidura & Quinto \textbf{ quasi non appretiare corporalem vitam . Sexto non horrere sanguinis effusionem . Septimo habere aptitudinem , } et industriam ad protegendum se \\\hline
3.3.4 & que ayan buena disposiçion e buena sabidura \textbf{ para defender assi } e para ferir a los enemigos . & quasi non appretiare corporalem vitam . Sexto non horrere sanguinis effusionem . Septimo habere aptitudinem , \textbf{ et industriam ad protegendum se } et feriendum alios . Octauo verecundari et erubescere eligere turpem fugam . \\\hline
3.3.4 & para defender assi \textbf{ e para ferir a los enemigos . } lo octauo que ayan uerguença de foyr torpemente . & quasi non appretiare corporalem vitam . Sexto non horrere sanguinis effusionem . Septimo habere aptitudinem , \textbf{ et industriam ad protegendum se } et feriendum alios . Octauo verecundari et erubescere eligere turpem fugam . \\\hline
3.3.4 & e para ferir a los enemigos . \textbf{ lo octauo que ayan uerguença de foyr torpemente . } Pues que assi es & et industriam ad protegendum se \textbf{ et feriendum alios . Octauo verecundari et erubescere eligere turpem fugam . } Est enim primo necessarium bellantibus posse sustinere ponderis magnitudinem . \\\hline
3.3.4 & que es meester a los lidiadores \textbf{ es que puedan sofrir grandes pesos . } Ca los desarmados de qual quier parte & et feriendum alios . Octauo verecundari et erubescere eligere turpem fugam . \textbf{ Est enim primo necessarium bellantibus posse sustinere ponderis magnitudinem . } Nam inermes a quacunque parte foriantur , succumbunt : \\\hline
3.3.4 & que los fieran luego caen . \textbf{ Por la qual cosa si alguno non puede sofrir } el peso de las armas & Nam inermes a quacunque parte foriantur , succumbunt : \textbf{ quare nisi quis posset sustinere armorum pondera , } inutilis est ad bellum . \\\hline
3.3.4 & Lo segundo conuiene a los lidiadores \textbf{ que puedan sofrir continuados mouimientos de los mienbros . } Ca si alguno estudiere quedo en la batalla & inutilis est ad bellum . \textbf{ Secundo bellantibus expedis posse sufferre quasi assiduos membrorum motus . } Nam si quis in bello non continue se ducat , \\\hline
3.3.4 & e non se mouiere continuadamente \textbf{ el enemigo non errara el golpe en el ferir } por la qual cosa sienpre estara en peligro de resçebir mayores golpes . & Nam si quis in bello non continue se ducat , \textbf{ aduersarius non fallitur in percutiendo , } quare semper exponitur ad sustinendum fortiores ictus : \\\hline
3.3.4 & el enemigo non errara el golpe en el ferir \textbf{ por la qual cosa sienpre estara en peligro de resçebir mayores golpes . } ca cosa prouada es & aduersarius non fallitur in percutiendo , \textbf{ quare semper exponitur ad sustinendum fortiores ictus : } expertum est enim quod homine continue se ducente et mouente , \\\hline
3.3.4 & e se mueue de vna parte a otra \textbf{ apenas o nunca le puede alcançar ninguna ferida . } Mas por la mayor parte sienpre fuye los colpes . & expertum est enim quod homine continue se ducente et mouente , \textbf{ vix aut nunquam ad plenum aliqua percussio potest ipsum attingere , } sed semper vulnera subterfugit . \\\hline
3.3.4 & e non estudiesse queda \textbf{ non se podrie ferir tan de ligero del uallestero } assi el ome andando & et non staret fixum , \textbf{ non sic de facili percuteretur ab arcu , } sic homo se circumuoluens , \\\hline
3.3.4 & e volviendosse de vna parte a otra \textbf{ non puede el enemigo de ligero ferirle . } Et por ende el mouimiento continuado de los mienbros es meester & sic homo se circumuoluens , \textbf{ non sic de facili vulneratur ab hoste . } Continuus ergo membrorum motus est necessarius ad vitandum plagas . \\\hline
3.3.4 & Et por ende el mouimiento continuado de los mienbros es meester \textbf{ para escusar las feridas . Bien assi avn es menester mouimiento conuenible } para dar las feridas . & non sic de facili vulneratur ab hoste . \textbf{ Continuus ergo membrorum motus est necessarius ad vitandum plagas . } Sic etiam est necessarius ad incutiendum eas : \\\hline
3.3.4 & para escusar las feridas . Bien assi avn es menester mouimiento conuenible \textbf{ para dar las feridas . } Por la qual cosa tales deuen ser los omnes lidiadores & Continuus ergo membrorum motus est necessarius ad vitandum plagas . \textbf{ Sic etiam est necessarius ad incutiendum eas : } propter quod tales debent esse homines pugnatiui , \\\hline
3.3.4 & Por la qual cosa tales deuen ser los omnes lidiadores \textbf{ que prolongadamente puedan sofrir el andar } e el mouimiento continuado de los mienbros . & propter quod tales debent esse homines pugnatiui , \textbf{ ut diu tolerare possint assiduum membrorum motum . Tertio homines pugnatiuos decet non curare de parcitate victus . } Nam graue est , ultra armorum pondera \\\hline
3.3.4 & Lo terçero los omnes lidiadores \textbf{ non deue auer cuydado de escassa uianda . } Ca muy graue cosa & ø \\\hline
3.3.4 & e que lo lieuen en muchedunbre de uiandas \textbf{ Ca ante deuen dexar el peso de las talegas } que del peso de las armas . & deferre in abundantia victualium copiam : \textbf{ immo } et si adesset pugnantibus ciborum ubertas , \\\hline
3.3.4 & que del peso de las armas . \textbf{ mas avn puesto que e ningun agrauamiento pudiessen auer los lidiadores } gunand cunplimiento de viandas & immo \textbf{ et si adesset pugnantibus ciborum ubertas , } adhuc esset eis necessaria abstinentia , \\\hline
3.3.4 & gunand cunplimiento de viandas \textbf{ avn meestet les es en la batalla de coner poco e de fazer abstineçia } e non se finchir de mucha uianda & et si adesset pugnantibus ciborum ubertas , \textbf{ adhuc esset eis necessaria abstinentia , } et non grauari ex nimio cibo , \\\hline
3.3.4 & avn meestet les es en la batalla de coner poco e de fazer abstineçia \textbf{ e non se finchir de mucha uianda } por que pueda meior sofrir el trabaio de la batalla & adhuc esset eis necessaria abstinentia , \textbf{ et non grauari ex nimio cibo , } ut laborem pugnandi melius tolerare possent . \\\hline
3.3.4 & e non se finchir de mucha uianda \textbf{ por que pueda meior sofrir el trabaio de la batalla } Lo quarto conuiene a los lidiadores & et non grauari ex nimio cibo , \textbf{ ut laborem pugnandi melius tolerare possent . } Quarto decet eos non curare de incommoditate iacendi et standi . \\\hline
3.3.4 & Lo quarto conuiene a los lidiadores \textbf{ non auer cuydado de mal yazer } nin de mal estar . & ut laborem pugnandi melius tolerare possent . \textbf{ Quarto decet eos non curare de incommoditate iacendi et standi . } Nam expedit aliquando pugnantibus die noctuque esse in armis : \\\hline
3.3.4 & Ca conuiene algunas ueses a los lididores \textbf{ de tener las armas de dia e de noche . } Et por ende nin estando nin yaziendo non deuen auer cuydado de folgura . & Quarto decet eos non curare de incommoditate iacendi et standi . \textbf{ Nam expedit aliquando pugnantibus die noctuque esse in armis : } propter quod nec in stando , \\\hline
3.3.4 & de tener las armas de dia e de noche . \textbf{ Et por ende nin estando nin yaziendo non deuen auer cuydado de folgura . } Lo quinto conuiene a los lidadores & Nam expedit aliquando pugnantibus die noctuque esse in armis : \textbf{ propter quod nec in stando , | nec in iacendo est eis commodum aut requies . } Quinto decet ipsos propter iustitiam et commune bonum \\\hline
3.3.4 & Lo quinto conuiene a los lidadores \textbf{ de non preçiar la vianda corporal } por la iustiçia e por el bien comun . & Quinto decet ipsos propter iustitiam et commune bonum \textbf{ quasi non appretiari corporalem vitam . } Nam cum tota operatio bellica exposita sit periculis mortis , \\\hline
3.3.4 & nin buen lidiador \textbf{ si en alguna manera non fuere sin temer } que sin miedo en los periglos de la muerte . & et bonus bellator , \textbf{ nisi aliquo modo } sit impauidus circa pericula mortis . Spectat enim ad fortem \\\hline
3.3.4 & assi commo dize el philosofo en el terçero libro de las Ethicas \textbf{ de non auer cuydado } nin temor en la batalla de bien morir . & ut innuit Philosophus 3 Ethic’ \textbf{ non curare in bello bene mori . } Tunc enim quis dicitur bene mori in bello , \\\hline
3.3.4 & de non auer cuydado \textbf{ nin temor en la batalla de bien morir . } Ca estonçe es dicho cada vno bien morir en la batalla & ut innuit Philosophus 3 Ethic’ \textbf{ non curare in bello bene mori . } Tunc enim quis dicitur bene mori in bello , \\\hline
3.3.4 & nin temor en la batalla de bien morir . \textbf{ Ca estonçe es dicho cada vno bien morir en la batalla } quando lidia derechamente e assi commo por defendimiento de la tierra & non curare in bello bene mori . \textbf{ Tunc enim quis dicitur bene mori in bello , } quando iuste bellans , \\\hline
3.3.4 & de ligero fuye torpemente de la batalla . \textbf{ Lo sexto los lidiadores non deuen aborresçer el derramamiento de la sangre } por que si alguno ouiere el coraçon muele & diligit corporalem vitam , \textbf{ de facili eligit turpem fugam . Sexto pugnantes non debent horrere sanguinis effusionem . } Nam si quis cor molle habens , muliebris existens , horreat effundere sanguinem ; non audebit hostibus plagas infligere , et per consequens bene bellare non potest . \\\hline
3.3.4 & por que si alguno ouiere el coraçon muele \textbf{ e fuere assi commo mugeril aborresçra esparzer la sangre } e non osara fazer llagas a los enemigos . & de facili eligit turpem fugam . Sexto pugnantes non debent horrere sanguinis effusionem . \textbf{ Nam si quis cor molle habens , muliebris existens , horreat effundere sanguinem ; non audebit hostibus plagas infligere , et per consequens bene bellare non potest . } Septimo decet eos habere aptitudinem , \\\hline
3.3.4 & e fuere assi commo mugeril aborresçra esparzer la sangre \textbf{ e non osara fazer llagas a los enemigos . } Et assi se sigue & de facili eligit turpem fugam . Sexto pugnantes non debent horrere sanguinis effusionem . \textbf{ Nam si quis cor molle habens , muliebris existens , horreat effundere sanguinem ; non audebit hostibus plagas infligere , et per consequens bene bellare non potest . } Septimo decet eos habere aptitudinem , \\\hline
3.3.4 & Et assi se sigue \textbf{ que non podra bien lidiar . } Lo vij° . conuiene a los lidiadores de auer disposiçion e sabiduria & Nam si quis cor molle habens , muliebris existens , horreat effundere sanguinem ; non audebit hostibus plagas infligere , et per consequens bene bellare non potest . \textbf{ Septimo decet eos habere aptitudinem , } et industriam ad protegendum se , \\\hline
3.3.4 & que non podra bien lidiar . \textbf{ Lo vij° . conuiene a los lidiadores de auer disposiçion e sabiduria } para cobrirse e defender se & Nam si quis cor molle habens , muliebris existens , horreat effundere sanguinem ; non audebit hostibus plagas infligere , et per consequens bene bellare non potest . \textbf{ Septimo decet eos habere aptitudinem , } et industriam ad protegendum se , \\\hline
3.3.4 & Lo vij° . conuiene a los lidiadores de auer disposiçion e sabiduria \textbf{ para cobrirse e defender se } e para ferir a los otros . & Septimo decet eos habere aptitudinem , \textbf{ et industriam ad protegendum se , } et ad feriendum alios . \\\hline
3.3.4 & para cobrirse e defender se \textbf{ e para ferir a los otros . } Ca assi commo dize el philosofo en el primero libro de las Ethicas & et industriam ad protegendum se , \textbf{ et ad feriendum alios . } Nam ut dicitur Ethic’ primo , Finis militaris , \\\hline
3.3.4 & Ca assi commo dize el philosofo en el primero libro de las Ethicas \textbf{ la fin de la caualleria es victoria e vençer . } Mas conmo todas las obras de la batalla sean contenidas & Nam ut dicitur Ethic’ primo , Finis militaris , \textbf{ est victoria . } Sed cum omnis bellica operatio contineatur sub militari , \\\hline
3.3.4 & la uictoria es dicha fin de todas las obras de la batalla . \textbf{ Por la qual cosa commo mayormente contezca a los lidiadores vençer } si bien se sopieren cobrir e defender & omnis actionis bellicae dicetur victoria esse finis . \textbf{ Quare cum maxime contingat bellantes vincere , } si bene sciant se protegere \\\hline
3.3.4 & Por la qual cosa commo mayormente contezca a los lidiadores vençer \textbf{ si bien se sopieren cobrir e defender } e a los otros ferir la sabidura & Quare cum maxime contingat bellantes vincere , \textbf{ si bene sciant se protegere } et alios ferire ; industria protegendi , \\\hline
3.3.4 & si bien se sopieren cobrir e defender \textbf{ e a los otros ferir la sabidura } de se cobrir & si bene sciant se protegere \textbf{ et alios ferire ; industria protegendi , } et feriendi valde est expediens bellatoribus . Qualiter , autem talis industria habeatur , et qualiter sit feriendum hostem , \\\hline
3.3.4 & e a los otros ferir la sabidura \textbf{ de se cobrir } e de se defender & et alios ferire ; industria protegendi , \textbf{ et feriendi valde est expediens bellatoribus . Qualiter , autem talis industria habeatur , et qualiter sit feriendum hostem , } et quae adminiculantur ad ista , \\\hline
3.3.4 & de se cobrir \textbf{ e de se defender } e de ferir a los otros & et feriendi valde est expediens bellatoribus . Qualiter , autem talis industria habeatur , et qualiter sit feriendum hostem , \textbf{ et quae adminiculantur ad ista , } infra patebit . Octauo decet bellatores verecundari , \\\hline
3.3.4 & e de se defender \textbf{ e de ferir a los otros } mucho es conuenible a los lidiadores & et feriendi valde est expediens bellatoribus . Qualiter , autem talis industria habeatur , et qualiter sit feriendum hostem , \textbf{ et quae adminiculantur ad ista , } infra patebit . Octauo decet bellatores verecundari , \\\hline
3.3.4 & mucho es conuenible a los lidiadores \textbf{ Mas en qual manera tal sabiduria se puede auer } e en qual manera han de ferir alos enemigos & et quae adminiculantur ad ista , \textbf{ infra patebit . Octauo decet bellatores verecundari , } et erubescere turpem fugam . \\\hline
3.3.4 & Mas en qual manera tal sabiduria se puede auer \textbf{ e en qual manera han de ferir alos enemigos } e otras cosas & infra patebit . Octauo decet bellatores verecundari , \textbf{ et erubescere turpem fugam . } Nam , \\\hline
3.3.4 & e otras cosas \textbf{ que se ayuntan a estas adelante paresçra Lo . viij° que pertenesçe a los lidiadores es de auer uerguença } e de guardar se de non foyr torpemente de la batalla . & et erubescere turpem fugam . \textbf{ Nam , } ut dicitur 3 Ethic’ \\\hline
3.3.4 & que se ayuntan a estas adelante paresçra Lo . viij° que pertenesçe a los lidiadores es de auer uerguença \textbf{ e de guardar se de non foyr torpemente de la batalla . } Ca assi commo dize el philosofo en el tercero libro de las Ethicas & et erubescere turpem fugam . \textbf{ Nam , } ut dicitur 3 Ethic’ \\\hline
3.3.4 & que fazen al omne buen lidiador \textbf{ es dessear de ser honrado por batalla } e de auer uerguenna de foyr torpemente de la batalla . & Inter caetera autem quae reddunt hominem bellicosum , \textbf{ est diligere honorari expugna , } et erubescere turpem fugam . Aduertendum autem quod cum dicimus , \\\hline
3.3.4 & es dessear de ser honrado por batalla \textbf{ e de auer uerguenna de foyr torpemente de la batalla . } Mas deuemos parar mientes & est diligere honorari expugna , \textbf{ et erubescere turpem fugam . Aduertendum autem quod cum dicimus , } bellatores non habere effusionem sanguinis , \\\hline
3.3.4 & e de auer uerguenna de foyr torpemente de la batalla . \textbf{ Mas deuemos parar mientes } que quando dezimos & est diligere honorari expugna , \textbf{ et erubescere turpem fugam . Aduertendum autem quod cum dicimus , } bellatores non habere effusionem sanguinis , \\\hline
3.3.4 & que quando dezimos \textbf{ que los lidiadores non deuen aborresçer el esparzimiento de la sangre } nin deuen mucho preçiar la vida corporal & et erubescere turpem fugam . Aduertendum autem quod cum dicimus , \textbf{ bellatores non habere effusionem sanguinis , } non debere multum appretiari corporalem vitam , \\\hline
3.3.4 & que los lidiadores non deuen aborresçer el esparzimiento de la sangre \textbf{ nin deuen mucho preçiar la vida corporal } et las otras cosas tales & bellatores non habere effusionem sanguinis , \textbf{ non debere multum appretiari corporalem vitam , } et caetera quae diffusius connumerauimus : \\\hline
3.3.4 & e por defendimiento del bien comun \textbf{ es de poner la vida corporal a periglo de muerte } e non deue foyr & Nam pro defensione iustitiae \textbf{ et pro communi bono exponenda est periculo corporalis vita , } non est cauenda effusio sanguinis , \\\hline
3.3.4 & es de poner la vida corporal a periglo de muerte \textbf{ e non deue foyr } nin auer miedo de esparzer la sangre & et pro communi bono exponenda est periculo corporalis vita , \textbf{ non est cauenda effusio sanguinis , } et caetera alia sunt fienda , \\\hline
3.3.4 & e non deue foyr \textbf{ nin auer miedo de esparzer la sangre } e todas las otras cosas dichas & et pro communi bono exponenda est periculo corporalis vita , \textbf{ non est cauenda effusio sanguinis , } et caetera alia sunt fienda , \\\hline
3.3.4 & e todas las otras cosas dichas \textbf{ que son de fazer } por las quales la iustiçia e el bien comun se pueda defender Et destas cosas paresçe llanamente quales lidiadores & et caetera alia sunt fienda , \textbf{ per quae iustitia } et commune bonum defendi potest . \\\hline
3.3.4 & que son de fazer \textbf{ por las quales la iustiçia e el bien comun se pueda defender Et destas cosas paresçe llanamente quales lidiadores } e quales omnes & per quae iustitia \textbf{ et commune bonum defendi potest . | Ex his autem plane patet , } quales bellatores , \\\hline
3.3.4 & e quales omnes \textbf{ para lit deua escoger el Rey o el principe . } Ca aquellos son de escoger & quales bellatores , \textbf{ et quos viros pugnatiuos Rex aut Princeps eligere debeat . } Nam illi sunt eligendi in quibus plura reperiuntur \\\hline
3.3.4 & para lit deua escoger el Rey o el principe . \textbf{ Ca aquellos son de escoger } en los quales son falladas las mas cosas & et quos viros pugnatiuos Rex aut Princeps eligere debeat . \textbf{ Nam illi sunt eligendi in quibus plura reperiuntur } de iis quae requiruntur ad pugnam . \\\hline
3.3.5 & ontadasontadas cosas \textbf{ que los lidiadores deuen auer finca de demandar } quales son los meiores lidiadores . & de iis quae requiruntur ad pugnam . \textbf{ Numeratis iis quae habere debent bellatores viri : } restat inquirere , \\\hline
3.3.5 & que dichas son . \textbf{ los aldeanos son meiores para lidiar . } Et desta opinion fue vegeçio diziendo assi . & Videtur autem si considerentur praedicta , rurales meliores esse . \textbf{ Huiusmodi autem opinionis visus est esse Vegetius , dicens : } Numquam credo potuisse dubitari aptiorem armis esse rusticam plebem . \\\hline
3.3.5 & Et desta opinion fue vegeçio diziendo assi . \textbf{ Creo que ninguno nunca pudo dubdar } que los omnes rusticos e aldeanos non fuessen meiores & Huiusmodi autem opinionis visus est esse Vegetius , dicens : \textbf{ Numquam credo potuisse dubitari aptiorem armis esse rusticam plebem . } Ad hoc etiam videntur facere quae superius diximus . \\\hline
3.3.5 & que los omnes lidiadores sean tales \textbf{ que puedan sofrir grandes pesos } e grandes trabaios continuados en los sus cuerpos & Dicebatur enim , viros pugnatiuos tales esse debere , \textbf{ qui possent sustinere magnitudinem ponderis , } continuum laborem membrorum , \\\hline
3.3.5 & e grandes trabaios continuados en los sus cuerpos \textbf{ e que puedan sofrir escasseça de viandas } e sofrir mal yazer e mal estar & continuum laborem membrorum , \textbf{ parcitatem victus , } et incommoditatem iacendi et standi , \\\hline
3.3.5 & e que puedan sofrir escasseça de viandas \textbf{ e sofrir mal yazer e mal estar } e non temer la muerte & parcitatem victus , \textbf{ et incommoditatem iacendi et standi , } non timere mortem , \\\hline
3.3.5 & e sofrir mal yazer e mal estar \textbf{ e non temer la muerte } e non temer nin aborresçer el esparzimiento de la sangre & et incommoditatem iacendi et standi , \textbf{ non timere mortem , } non horrere sanguinis effusionem , \\\hline
3.3.5 & e non temer la muerte \textbf{ e non temer nin aborresçer el esparzimiento de la sangre } e las otras cosas & non timere mortem , \textbf{ non horrere sanguinis effusionem , } et cetera alia quae tetigimus in capitulo praecedenti . \\\hline
3.3.5 & que son criados viçiosamente . \textbf{ por que los aldeanos son acostunbrados a leuar grandes cargas e grandes pesos . Et por ende non se agrauiarian de grant carga de amas en las batallas } aquellos que continuadamente entienpo de la paz son acostunbrados a mayores cargas . & Sunt enim rustici assueti \textbf{ ad magnitudinem ponderum | non enim in bellis grauabit eos armorum sarcina , } qui assidue tempore pacis assueti sunt \\\hline
3.3.5 & aquellos que continuadamente entienpo de la paz son acostunbrados a mayores cargas . \textbf{ Et avn non canssan en correr } nin en mouer los braços & ad maiora pondera . \textbf{ Nec etiam eos fatigabit cursus } vel ductio brachiorum , \\\hline
3.3.5 & Et avn non canssan en correr \textbf{ nin en mouer los braços } nin vsar los mouimientos de los otros mienbros del cuerpo . & Nec etiam eos fatigabit cursus \textbf{ vel ductio brachiorum , } vel aliorum membrorum motus , \\\hline
3.3.5 & nin en mouer los braços \textbf{ nin vsar los mouimientos de los otros mienbros del cuerpo . } Ca a estos e a mayores trabaios son vsados de cadal dia . & vel ductio brachiorum , \textbf{ vel aliorum membrorum motus , } qui ad hos et ad maiores sunt continue assueti . Rursus , \\\hline
3.3.5 & a los quales abasta el agua en la sed \textbf{ e el pan grueso en comer . } Otrossi los aldeanos non se agranian en mal yazer nin en mal estar & quibus potus aquae satisfaciebat in siti , \textbf{ et grossus panis sufficiebat ad esum . Amplius , rurales } non affliguntur \\\hline
3.3.5 & e el pan grueso en comer . \textbf{ Otrossi los aldeanos non se agranian en mal yazer nin en mal estar } ca non temen de la grant calentura del sol & et grossus panis sufficiebat ad esum . Amplius , rurales \textbf{ non affliguntur | ex incommoditate iacendi vel standi , } qui solis ardorem non timent , \\\hline
3.3.5 & Et pues que assi es parando mientes \textbf{ a estas cosas podemos iudgar } que los rusticos e los villanos son meiores para las batallas & Ad haec igitur intendentibus videtur censendum esse meliores bellatores esse rurales . Sunt autem alia , \textbf{ per quae videtur ostendi , urbanos } et nobiles meliores esse pugnantes . \\\hline
3.3.5 & Mas ay otras cosas \textbf{ por las quales se puede prouar } que los nobles e los cibdadanos son meiores lidiadores . & Nam \textbf{ inter cetera , per quae quis redditur bonus pugnatiuus , } est \\\hline
3.3.5 & por las quales tenemos a alguno \textbf{ por buen lidiador es querer ser honrrado por la batalla } e tomar uerguença de foyr torpemente & ( ut dicebatur ) \textbf{ velle honorari ex pugna , } et erubescere turpem fugam . \\\hline
3.3.5 & por buen lidiador es querer ser honrrado por la batalla \textbf{ e tomar uerguença de foyr torpemente } assi commo dicho es dessuso . & velle honorari ex pugna , \textbf{ et erubescere turpem fugam . } Hoc est enim \\\hline
3.3.5 & que diomedes era su vençido \textbf{ Por la qual cosa commo querer auer honrra de la batalla } e tomar uergueña de torpe fecho & quia dicebat , \textbf{ Si in bello terga vertam , Hector cum concionabitur } inter Troianos , dicet , \\\hline
3.3.5 & Por la qual cosa commo querer auer honrra de la batalla \textbf{ e tomar uergueña de torpe fecho } mas partenezca a los nobles e a los fijosdalgo & quia dicebat , \textbf{ Si in bello terga vertam , Hector cum concionabitur } inter Troianos , dicet , \\\hline
3.3.5 & que los villanos e los aldeanos \textbf{ por que mayor uergueña toman de foyr } que ellos . & Quare cum velle honorari \textbf{ et erubescere de aliquo turpi facto , magis conueniat nobilibus quam rusticis , ii meliores esse videntur ad pugnam , } eo quod verecundentur fugere . \\\hline
3.3.5 & Otrossi en la batalla mucho vale sotileza e sabiduria de las armas . \textbf{ o arteria para lidiar } por que la sotileza e el arteria algunas vezes & ø \\\hline
3.3.5 & por que la sotileza e el arteria algunas vezes \textbf{ mas vale para auer victoria } que la fortaleza del pueblo . & Rursus , in bello multum valet industria et prudentia . \textbf{ Nam sagacitas et versutia aliquando plus faciunt ad obtinendam victoriam , } quam corporis fortitudo . \\\hline
3.3.5 & Ca paresçe que estas dos cosas son prinçipales \textbf{ para auer victoria . } Conuiene a saber uerguença de foyr . & Quare cum communiter nobiles homines industriores sint rusticis , \textbf{ sequitur hos meliores esse pugnantes . Videntur enim haec duo maxima esse ad obtinendam victoriam , } videlicet erubescentia fugiendi , \\\hline
3.3.5 & para auer victoria . \textbf{ Conuiene a saber uerguença de foyr . } Et sabiduria de lidiar . & sequitur hos meliores esse pugnantes . Videntur enim haec duo maxima esse ad obtinendam victoriam , \textbf{ videlicet erubescentia fugiendi , } et sagacitas bellandi . \\\hline
3.3.5 & Conuiene a saber uerguença de foyr . \textbf{ Et sabiduria de lidiar . } Et pues que & videlicet erubescentia fugiendi , \textbf{ et sagacitas bellandi . } Ut ergo sciatur \\\hline
3.3.5 & assi es \textbf{ para saber la uerdat } que auemos de tener desta question de tener mientes & et sagacitas bellandi . \textbf{ Ut ergo sciatur } quid sit de quaesito tenendum , \\\hline
3.3.5 & para saber la uerdat \textbf{ que auemos de tener desta question de tener mientes } que segunt el departimiento de las batallas son de escoger departidos lidiadores . & Ut ergo sciatur \textbf{ quid sit de quaesito tenendum , | oportet aduertere , } quod secundum diuersitatem pugnarum diuersi eligendi sunt bellatores . \\\hline
3.3.5 & que auemos de tener desta question de tener mientes \textbf{ que segunt el departimiento de las batallas son de escoger departidos lidiadores . } Ca la batalla puede ser de omnes de pie & oportet aduertere , \textbf{ quod secundum diuersitatem pugnarum diuersi eligendi sunt bellatores . } Potest enim esse pugna pedestris , \\\hline
3.3.5 & Et por ende en la batalla de los peones \textbf{ mas son de escoger aldeanos que nobles } por que y mucho vale uso de traer grandes cargas & In pedestri \textbf{ itaque certamine magis eligendi sunt rurales , quam nobiles : } eo quod illis maxime valet assuefactio ad portationem ponderum , \\\hline
3.3.5 & mas son de escoger aldeanos que nobles \textbf{ por que y mucho vale uso de traer grandes cargas } e de sofrir grandes trabaios . & itaque certamine magis eligendi sunt rurales , quam nobiles : \textbf{ eo quod illis maxime valet assuefactio ad portationem ponderum , } et tolerantiam laborum . In equestri vero magis eligendi sunt ipsi nobiles : \\\hline
3.3.5 & por que y mucho vale uso de traer grandes cargas \textbf{ e de sofrir grandes trabaios . } Mas en la batalla de los de cauallo & itaque certamine magis eligendi sunt rurales , quam nobiles : \textbf{ eo quod illis maxime valet assuefactio ad portationem ponderum , } et tolerantiam laborum . In equestri vero magis eligendi sunt ipsi nobiles : \\\hline
3.3.5 & Mas en la batalla de los de cauallo \textbf{ mas son de escoger los nobles . } por que la fortaleza de los cauallos cunple la mengua & eo quod illis maxime valet assuefactio ad portationem ponderum , \textbf{ et tolerantiam laborum . In equestri vero magis eligendi sunt ipsi nobiles : } eo quod equorum ipsorum fortitudo supplet defectum , \\\hline
3.3.5 & que han los nobles \textbf{ en non poder sofrir tantos trabaios } quantos se acostunbraron de sofrir los villanos . & eo quod equorum ipsorum fortitudo supplet defectum , \textbf{ quem patiuntur nobiles in non posse tantos sustinere labores , } quantos consueuerunt sustinere rurales . In huiusmodi enim \\\hline
3.3.5 & en non poder sofrir tantos trabaios \textbf{ quantos se acostunbraron de sofrir los villanos . } por que en tal batalla mucho vale la sabidura & quem patiuntur nobiles in non posse tantos sustinere labores , \textbf{ quantos consueuerunt sustinere rurales . In huiusmodi enim } pugna nimium valet bellandi sagacitas sociata erubescentiae fugiendi . \\\hline
3.3.5 & por que en tal batalla mucho vale la sabidura \textbf{ de lidiar aconpañada con la uerguenca del foyr . } Enpero conuiene de saber & quantos consueuerunt sustinere rurales . In huiusmodi enim \textbf{ pugna nimium valet bellandi sagacitas sociata erubescentiae fugiendi . } Sciendum tamen quod ut nobiles ex omni parte efficiantur strenui bellatores , \\\hline
3.3.5 & de lidiar aconpañada con la uerguenca del foyr . \textbf{ Enpero conuiene de saber } que para que los nobles de toda parte sean fechos estremados lidiadores & pugna nimium valet bellandi sagacitas sociata erubescentiae fugiendi . \textbf{ Sciendum tamen quod ut nobiles ex omni parte efficiantur strenui bellatores , } assuefaciendi sunt ad portandum armorum pondera , \\\hline
3.3.5 & que para que los nobles de toda parte sean fechos estremados lidiadores \textbf{ deuen se acostunbrar a sofrir el peso de las armas } que asofrir el trabaio & Sciendum tamen quod ut nobiles ex omni parte efficiantur strenui bellatores , \textbf{ assuefaciendi sunt ad portandum armorum pondera , } et ad sustinendum laborem : \\\hline
3.3.5 & deuen se acostunbrar a sofrir el peso de las armas \textbf{ que asofrir el trabaio } e el mouimiento de los braços & assuefaciendi sunt ad portandum armorum pondera , \textbf{ et ad sustinendum laborem : } et motum brachiorum , \\\hline
3.3.5 & Mas quales e quantas cosas son aquellas \textbf{ a que se deuen vsar los lidiadores } en el capitulo & et quot sunt illa , \textbf{ ad quae debeant exercitari bellantes ; } in sequenti capitulo ostendetur . \\\hline
3.3.6 & si aquellos pocos romanos non fueren primeramente muy vsados de las armas \textbf{ e si non ouieran grant sabiduria de lidiar . } Et non es cosa desconuenible & nisi plus illis fuissent exercitati in armis , \textbf{ et magis habuissent bellandi industriam . } Non est enim inconueniens virum prudentem \\\hline
3.3.6 & que vn omme sabidor e entendido en vna cosa . \textbf{ por non auer prueua de las cosas particulares } de non ser sabio en otra cosa . & et sagacem in uno , \textbf{ propter particularium inexperientiam esse imprudentem in aliquo . } Unde multotiens contingit , \\\hline
3.3.6 & que los omnes sabios en muchas otras cosas \textbf{ por non auer vso de las armas non son sabidores en las faziendas . } Ca el vso en cada vn negocio & Unde multotiens contingit , \textbf{ quod prudentes in rebus aliis propter inexercitium armorum non sunt industres in bellis . Exercitium enim in quolibet negocio praebet audaciam , } ut non metuat illud facere . \\\hline
3.3.6 & e en cada fecho de grant osadia \textbf{ por que non ayan miedo los omnes de fazer } aquello que han usado & quod prudentes in rebus aliis propter inexercitium armorum non sunt industres in bellis . Exercitium enim in quolibet negocio praebet audaciam , \textbf{ ut non metuat illud facere . } Nam quia secundum Vegetium : \\\hline
3.3.6 & aquello que han usado \textbf{ por que segunt que dize Uegeçio ninguno non tenie de fazer } aquello que ha bien aprendido e en que tieue fiuza & ut non metuat illud facere . \textbf{ Nam quia secundum Vegetium : | Nemo facere metuit , } quod se bene didicisse confidit . \\\hline
3.3.6 & que en la contienda de las batallas pocos omnes bien usados son apareiados \textbf{ para auer victoria de los muchos . } Et los muchos omnes non sabidores non usados & quod in bellorum certamine paucitas exercitata plus valet \textbf{ ( si sit prompta ) ad victoriam : } et multitudo rudis , \\\hline
3.3.6 & son apareiados \textbf{ para foyr e para morir . } Visto en qual manera el uso de las armas es muy prouechoso & et indocta semper sit exposita \textbf{ ad fugam et ad eaedem . } Viso armorum exercitium esse perutile ad opera bellica , \\\hline
3.3.6 & para las obras de la batalla \textbf{ finca de demostrar } en qual manera se deuen vsar los lidiadores & Viso armorum exercitium esse perutile ad opera bellica , \textbf{ restat ostendere quomodo exercitandi sunt bellantes ad incedendum gradatim , } et passim , \\\hline
3.3.6 & finca de demostrar \textbf{ en qual manera se deuen vsar los lidiadores } a andar ordenadamente e a passo . & Viso armorum exercitium esse perutile ad opera bellica , \textbf{ restat ostendere quomodo exercitandi sunt bellantes ad incedendum gradatim , } et passim , \\\hline
3.3.6 & en qual manera se deuen vsar los lidiadores \textbf{ a andar ordenadamente e a passo . } Et commo se deuen vsar acorrer e a saltar . & restat ostendere quomodo exercitandi sunt bellantes ad incedendum gradatim , \textbf{ et passim , } et ad cursum et ad saltum . \\\hline
3.3.6 & a andar ordenadamente e a passo . \textbf{ Et commo se deuen vsar acorrer e a saltar . } Ca estas cosas & et passim , \textbf{ et ad cursum et ad saltum . } Nam ista \\\hline
3.3.6 & assi commo paresçra adelante son neçessarias en las batallas \textbf{ e non auer uso en estas cosas } enpeesçe mucho a los lidiadores . & ( ut in prosequendo patebit ) necessaria sunt in bellis : \textbf{ et inexercitatio circa ipsa bellantibus est nociua . } Primo enim milites \\\hline
3.3.6 & Ca lo primero los caualleros e los peones \textbf{ e generalmente todos los lidiadores se deuen vsar a andar ordenadamente e a passo en la batalla . } por que cada vno tenga su orden & et ipsi pedites , \textbf{ et uniuersaliter bellantes assuescendi sunt ad gradum | et passum bellicum , } ut gradatim pergant ita , \\\hline
3.3.6 & Ca si el az si quier de peones \textbf{ si quier de caual leros non andudiere ordonadamentedos males se siguen dende . } Ca non guardando la orden & Nam \textbf{ si acies siue peditum siue militum non ordinate incedat , } duo mala inde consequuntur . Nam non seruato debito ordine , \\\hline
3.3.6 & Ca non guardando la orden \textbf{ que deuen guardar en la vna parte el az sera esparzida e rala } mas de quanto deue & duo mala inde consequuntur . Nam non seruato debito ordine , \textbf{ in una parte erit acies | quasi sparsa et peruia , } ultra quam debeat . \\\hline
3.3.6 & enbargan se los vnos a los otros \textbf{ para ferir . } Ca quando el lidiador esta muy apretado de su conpañon & et per consequens debellabitur . Secundo in parte illa in qua nimis arctum est , \textbf{ impedietur ad percutiendum . } Nam cum bellator a suo consocio nimis comprimitur , \\\hline
3.3.6 & Ca quando el lidiador esta muy apretado de su conpañon \textbf{ enbargansele los braços para ferir } e no puede dar colpes en los enemigos . & Nam cum bellator a suo consocio nimis comprimitur , \textbf{ sua impediuntur brachia , } ne possit hostibus plagas infligere . Haec enim duo in acie sunt necessaria : \\\hline
3.3.6 & enbargansele los braços para ferir \textbf{ e no puede dar colpes en los enemigos . } Ca estas dos cosas son menester en la az . & sua impediuntur brachia , \textbf{ ne possit hostibus plagas infligere . Haec enim duo in acie sunt necessaria : } ut scilicet non possit de facili perforari ab hostibus , \\\hline
3.3.6 & Ca estas dos cosas son menester en la az . \textbf{ Conuiene a saber que non puede ser de ligero foradada de los enemigos } e que non sea enbargada para ferir . & ne possit hostibus plagas infligere . Haec enim duo in acie sunt necessaria : \textbf{ ut scilicet non possit de facili perforari ab hostibus , } et non impediatur ad percutiendum . Quod , \\\hline
3.3.6 & Conuiene a saber que non puede ser de ligero foradada de los enemigos \textbf{ e que non sea enbargada para ferir . } La qual cosa non se puede fazer & ut scilicet non possit de facili perforari ab hostibus , \textbf{ et non impediatur ad percutiendum . Quod , } nisi seruato debito gradu , \\\hline
3.3.6 & e que non sea enbargada para ferir . \textbf{ La qual cosa non se puede fazer } non guardando grado conuenible & ut scilicet non possit de facili perforari ab hostibus , \textbf{ et non impediatur ad percutiendum . Quod , } nisi seruato debito gradu , \\\hline
3.3.6 & commo caualleros ante \textbf{ que vengan a la batalla son muchas vezes de ayuntar en vno et vsar en las armas } assi que cargados de armas puedan andar ordenadamente & Tam ergo pedites quam equites bellatores \textbf{ antequam bella exerceant sunt multotiens simul congregandi , et exercitandi , } ut onerat armis ordinate incedant , \\\hline
3.3.6 & que vengan a la batalla son muchas vezes de ayuntar en vno et vsar en las armas \textbf{ assi que cargados de armas puedan andar ordenadamente } assi commo si ouiessen de acometer la batalla . & antequam bella exerceant sunt multotiens simul congregandi , et exercitandi , \textbf{ ut onerat armis ordinate incedant , } ac si deberent pugnam committere . \\\hline
3.3.6 & assi que cargados de armas puedan andar ordenadamente \textbf{ assi commo si ouiessen de acometer la batalla . } Et quando vieren los caudiellos maestros de las batallas & ut onerat armis ordinate incedant , \textbf{ ac si deberent pugnam committere . } Et cum viderit magister bellorum \\\hline
3.3.6 & Et quando vieren los caudiellos maestros de las batallas \textbf{ que alguno non guarda orden en la az deuenle denostar e castigar } Et si muy desordenado andudiere deuen le echar de la az & Et cum viderit magister bellorum \textbf{ aliquem non tenere ordinem debitum in acie , | ipsum increpet } et corrigat : \\\hline
3.3.6 & que alguno non guarda orden en la az deuenle denostar e castigar \textbf{ Et si muy desordenado andudiere deuen le echar de la az } assi commo a aquel non es prouechoso lidiador . & ipsum increpet \textbf{ et corrigat : | vel si nimis delinquat , } ipsum omnino repellat ab acie tanquam inutilem bellatorem . \\\hline
3.3.6 & assi commo a aquel non es prouechoso lidiador . \textbf{ Lo segundo son de vsar los lidiadores } tan bien los peones & ipsum omnino repellat ab acie tanquam inutilem bellatorem . \textbf{ Secundo exercitandi sunt bellatores } tam pedites quam equites ad cursum , \\\hline
3.3.6 & tan bien los peones \textbf{ commo los caualleros a correr } por que sean despues mas ligeros & Secundo exercitandi sunt bellatores \textbf{ tam pedites quam equites ad cursum , } ut sint habiles in praecurrendo . \\\hline
3.3.6 & Ca paresçe que esto les vale atres cosas \textbf{ Lo primero para assechar e ascuchar el estado de los enemigos . } ca buena cosa es en la fazienda auer algunos mas ligeros que los otros & Videtur enim hoc valere ad tria . \textbf{ Primo ad explorandum inimicorum facta . } Nam bonum est in exercitu aliquos exiliores praecurrere , \\\hline
3.3.6 & Lo primero para assechar e ascuchar el estado de los enemigos . \textbf{ ca buena cosa es en la fazienda auer algunos mas ligeros que los otros } que non puedan ser alcançados do ligero de los enemigos & Primo ad explorandum inimicorum facta . \textbf{ Nam bonum est in exercitu aliquos exiliores praecurrere , } qui de facili non possint ab ipsis hostibus comprehendit , \\\hline
3.3.6 & que non puedan ser alcançados do ligero de los enemigos \textbf{ que vayan e escuchar e a saber las condiciones e el estado delos enemigos } Lo segundo esto es prouechoso & qui de facili non possint ab ipsis hostibus comprehendit , \textbf{ explorantes conditiones | et facta hostium . } Secundo hoc est utile ad obtinendum meliorem locum . \\\hline
3.3.6 & Lo segundo esto es prouechoso \textbf{ para ganar meior lugar en la batalla . por que el logar mucho ayuda a la batalla . } Et por ende si los lidiadores fueren usados a correr & Secundo hoc est utile ad obtinendum meliorem locum . \textbf{ Nam et locus multum facit ad pugnam . } Ideo si bellatores exercitati sunt ad cursum , \\\hline
3.3.6 & para ganar meior lugar en la batalla . por que el logar mucho ayuda a la batalla . \textbf{ Et por ende si los lidiadores fueren usados a correr } mas ligeramente ganaran meior logar parar lidiar & Nam et locus multum facit ad pugnam . \textbf{ Ideo si bellatores exercitati sunt ad cursum , } facilius obtinebunt aptiorem locum ad pugnandum . Est \\\hline
3.3.6 & Et por ende si los lidiadores fueren usados a correr \textbf{ mas ligeramente ganaran meior logar parar lidiar } Lo terçero esto es aprouechoso & Ideo si bellatores exercitati sunt ad cursum , \textbf{ facilius obtinebunt aptiorem locum ad pugnandum . Est } etiam hoc utile ad prosequendum hostes fugientes . \\\hline
3.3.6 & Lo terçero esto es aprouechoso \textbf{ para seguir } e alcançar los enemigos quando fuyen . & facilius obtinebunt aptiorem locum ad pugnandum . Est \textbf{ etiam hoc utile ad prosequendum hostes fugientes . } Nam non de facili quis potest euadere manus agilium \\\hline
3.3.6 & para seguir \textbf{ e alcançar los enemigos quando fuyen . } Ca no puede ninguno de ligero foyr de las manos de aquellos & facilius obtinebunt aptiorem locum ad pugnandum . Est \textbf{ etiam hoc utile ad prosequendum hostes fugientes . } Nam non de facili quis potest euadere manus agilium \\\hline
3.3.6 & e alcançar los enemigos quando fuyen . \textbf{ Ca no puede ninguno de ligero foyr de las manos de aquellos } que mucho corren . & etiam hoc utile ad prosequendum hostes fugientes . \textbf{ Nam non de facili quis potest euadere manus agilium } et praecurrentium . \\\hline
3.3.6 & que mucho corren . \textbf{ lo terçero son de vsar los lidiadores al salto } por que sepan andar saltando e por saltos . & et praecurrentium . \textbf{ Tertio exercitandi sunt bellatores ad saltum , } ut sciant saltim , \\\hline
3.3.6 & lo terçero son de vsar los lidiadores al salto \textbf{ por que sepan andar saltando e por saltos . } la qual cosa es prouechosa a tres cosas & Tertio exercitandi sunt bellatores ad saltum , \textbf{ ut sciant saltim , | vel per saltum incedere . } Quod etiam ad tria est utile . Primo ad remouendum impedimenta . \\\hline
3.3.6 & la qual cosa es prouechosa a tres cosas \textbf{ Lo primero para tirar los enbargos } Lo segundo para espantar los enemigos . & vel per saltum incedere . \textbf{ Quod etiam ad tria est utile . Primo ad remouendum impedimenta . } Secundo ad terrendum aduersarios . Tertio ad infligendum maiores plagas . \\\hline
3.3.6 & Lo primero para tirar los enbargos \textbf{ Lo segundo para espantar los enemigos . } Lo tercero para fazer mayores llagas . & Quod etiam ad tria est utile . Primo ad remouendum impedimenta . \textbf{ Secundo ad terrendum aduersarios . Tertio ad infligendum maiores plagas . } Contingit enim aliquando inuenire fossas \\\hline
3.3.6 & Lo segundo para espantar los enemigos . \textbf{ Lo tercero para fazer mayores llagas . } Ca contesçe algunas vezes de fallar algunas carcauas e arroyos e açequias & Quod etiam ad tria est utile . Primo ad remouendum impedimenta . \textbf{ Secundo ad terrendum aduersarios . Tertio ad infligendum maiores plagas . } Contingit enim aliquando inuenire fossas \\\hline
3.3.6 & Lo tercero para fazer mayores llagas . \textbf{ Ca contesçe algunas vezes de fallar algunas carcauas e arroyos e açequias } e algunos otros enbargos en la carrera & Secundo ad terrendum aduersarios . Tertio ad infligendum maiores plagas . \textbf{ Contingit enim aliquando inuenire fossas } et alia impedimenta in via , \\\hline
3.3.6 & e algunos otros enbargos en la carrera \textbf{ que sin salto non lo pueden saltar nin passar } por la qual cosa prouechosa cosa es el saltar & et alia impedimenta in via , \textbf{ quae sine saltu in via transire non possunt : } quare utile est ad remouenda impedimenta , \\\hline
3.3.6 & que sin salto non lo pueden saltar nin passar \textbf{ por la qual cosa prouechosa cosa es el saltar } para tirar estos enbargos . & quae sine saltu in via transire non possunt : \textbf{ quare utile est ad remouenda impedimenta , } ut equites sic sint docti , \\\hline
3.3.6 & por la qual cosa prouechosa cosa es el saltar \textbf{ para tirar estos enbargos . } Por que los caualleros & quae sine saltu in via transire non possunt : \textbf{ quare utile est ad remouenda impedimenta , } ut equites sic sint docti , \\\hline
3.3.6 & si fueren assi ensseñados \textbf{ que sepan assi a guiar los cauallos } por que salten carcauas & ut equites sic sint docti , \textbf{ ut sciant equum sic pungere , } ut per saltum foueas \\\hline
3.3.6 & si quieren ser buenos lidiadores \textbf{ conuiene que en su mançebia sean usados a saltar } por que puedan & si contingat eos pedestres esse , si volunt boni bellatores existere , \textbf{ sic ab ipsa iuuentute exercitandi sunt ad saliendum , } ut possint per saltum foueas , \\\hline
3.3.6 & por que puedan \textbf{ por el salto passar las carcauas } e los otros enbargos . & sic ab ipsa iuuentute exercitandi sunt ad saliendum , \textbf{ ut possint per saltum foueas , } et alia impedimenta transire . Terrentur \\\hline
3.3.6 & por razon del mouimiento vale \textbf{ para fazer mayores llagas . } e et podemos sin aquellas tres cosas aque dixiemos & ipse saltus ratione motus facit \textbf{ ut plaga amplior infligatur . } Possumus autem praeter tria praedicta , \\\hline
3.3.7 & e et podemos sin aquellas tres cosas aque dixiemos \textbf{ que son de vsar los lidiadores } contar otras ocho cosas & Possumus autem praeter tria praedicta , \textbf{ ad quae exercitandos diximus bellantes , } enumerare octo alia , \\\hline
3.3.7 & que son de vsar los lidiadores \textbf{ contar otras ocho cosas } aque se deuen vsar los lidiadores . & ad quae exercitandos diximus bellantes , \textbf{ enumerare octo alia , } ad quae exercitari debent homines bellicosi . \\\hline
3.3.7 & contar otras ocho cosas \textbf{ aque se deuen vsar los lidiadores . } Lo primero se deuen vsar aleuar grandes pesos . & enumerare octo alia , \textbf{ ad quae exercitari debent homines bellicosi . } Primo enim exercitandi sunt ad portandum pondera . \\\hline
3.3.7 & aque se deuen vsar los lidiadores . \textbf{ Lo primero se deuen vsar aleuar grandes pesos . } Lo segundo a acometer & ad quae exercitari debent homines bellicosi . \textbf{ Primo enim exercitandi sunt ad portandum pondera . } Secundo ad inuadendum \\\hline
3.3.7 & Lo primero se deuen vsar aleuar grandes pesos . \textbf{ Lo segundo a acometer } e a ferir con maças . & Primo enim exercitandi sunt ad portandum pondera . \textbf{ Secundo ad inuadendum } et percutiendum \\\hline
3.3.7 & Lo segundo a acometer \textbf{ e a ferir con maças . } Lo terçero son de vsar a lançar dardos & Secundo ad inuadendum \textbf{ et percutiendum } cum claua . Tertio ad emittendum tela siue iacula , \\\hline
3.3.7 & e a ferir con maças . \textbf{ Lo terçero son de vsar a lançar dardos } e a ferir con llan ças . & et percutiendum \textbf{ cum claua . Tertio ad emittendum tela siue iacula , } et ad percutiendum cum lancea . \\\hline
3.3.7 & Lo terçero son de vsar a lançar dardos \textbf{ e a ferir con llan ças . } lo quarto a a lançar saetas . & cum claua . Tertio ad emittendum tela siue iacula , \textbf{ et ad percutiendum cum lancea . } Quarto ad iaciendum sagittas . Quinto ad proiiciendum lapides cum fundis . Sexto ad percutiendum cum plumbatis . Septimo \\\hline
3.3.7 & e a ferir con llan ças . \textbf{ lo quarto a a lançar saetas . } lo quinto a a lançar piedras con fondas . & cum claua . Tertio ad emittendum tela siue iacula , \textbf{ et ad percutiendum cum lancea . } Quarto ad iaciendum sagittas . Quinto ad proiiciendum lapides cum fundis . Sexto ad percutiendum cum plumbatis . Septimo \\\hline
3.3.7 & lo quarto a a lançar saetas . \textbf{ lo quinto a a lançar piedras con fondas . } lo sexto a ferir con pellas de plomo o de fierro & et ad percutiendum cum lancea . \textbf{ Quarto ad iaciendum sagittas . Quinto ad proiiciendum lapides cum fundis . Sexto ad percutiendum cum plumbatis . Septimo } ad ascendendum equos . Octauo \\\hline
3.3.7 & lo quinto a a lançar piedras con fondas . \textbf{ lo sexto a ferir con pellas de plomo o de fierro } Lo vij° . & et ad percutiendum cum lancea . \textbf{ Quarto ad iaciendum sagittas . Quinto ad proiiciendum lapides cum fundis . Sexto ad percutiendum cum plumbatis . Septimo } ad ascendendum equos . Octauo \\\hline
3.3.7 & Lo vij° . \textbf{ se deuen usar a sobir ligeramente en los cauallos } Lo viij . se deuenusar a saber el arte de nadar . & Quarto ad iaciendum sagittas . Quinto ad proiiciendum lapides cum fundis . Sexto ad percutiendum cum plumbatis . Septimo \textbf{ ad ascendendum equos . Octauo } ad sciendum artem natandi . \\\hline
3.3.7 & se deuen usar a sobir ligeramente en los cauallos \textbf{ Lo viij . se deuenusar a saber el arte de nadar . } Avn a dezir mas adelante & ad ascendendum equos . Octauo \textbf{ ad sciendum artem natandi . | Esset } etiam ulterius dicendum , \\\hline
3.3.7 & Lo viij . se deuenusar a saber el arte de nadar . \textbf{ Avn a dezir mas adelante } en qual manera se auian de vsar los lidiadores a ferir con espadas e con cuchiellos & Esset \textbf{ etiam ulterius dicendum , } quomodo exercitandi sunt bellatores ad percutiendum \\\hline
3.3.7 & Avn a dezir mas adelante \textbf{ en qual manera se auian de vsar los lidiadores a ferir con espadas e con cuchiellos } por el arte del esgrima & etiam ulterius dicendum , \textbf{ quomodo exercitandi sunt bellatores ad percutiendum | cum gladiis } et ensibus . \\\hline
3.3.7 & lo primero dezimos \textbf{ que son de vsar los lidiadores a leuar grandes pesos } en tal manera que se acostubren a leuar mayor peso & Sed de hoc speciale capitulum faciemus . \textbf{ Primo enim sunt bellatores exercitandi } ad portandum pondera , \\\hline
3.3.7 & que son de vsar los lidiadores a leuar grandes pesos \textbf{ en tal manera que se acostubren a leuar mayor peso } que el peso de las armas & Primo enim sunt bellatores exercitandi \textbf{ ad portandum pondera , | ut plus ponderis portare assuescant } etiam quam sit armorum sarcina . \\\hline
3.3.7 & Et por ende \textbf{ quando alguno es acostunbrado a leuar mayor carga } que la carga de las armas semeial & quasi natura quaedam . \textbf{ Cum ergo quis assuetus ad portandum maius pondus , } videtur sibi \\\hline
3.3.7 & quando anda armado de sus armas . \textbf{ Otrossi non solamente son de vsar los lidiadores a leuar las armas . } mas avn a otras muchas cosas & quasi quod leuis incedat , \textbf{ si oneretur quodam minori pondere . Rursus , | non solum arma , } sed etiam plura alia sunt ferenda in bello : \\\hline
3.3.7 & mas avn a otras muchas cosas \textbf{ que son de leuar en las batallas . } Et por ende prouechosa cosa es de se acostunbrar los lidiadores a leuar grandes pesos . & non solum arma , \textbf{ sed etiam plura alia sunt ferenda in bello : } ideo \\\hline
3.3.7 & que son de leuar en las batallas . \textbf{ Et por ende prouechosa cosa es de se acostunbrar los lidiadores a leuar grandes pesos . } Lo segundo se deuen usar los lidiadores a acometer & sed etiam plura alia sunt ferenda in bello : \textbf{ ideo | etiam ad maiora pondera } non est inutile assuescere bellatores . \\\hline
3.3.7 & Et por ende prouechosa cosa es de se acostunbrar los lidiadores a leuar grandes pesos . \textbf{ Lo segundo se deuen usar los lidiadores a acometer } e a ferir con porras e con maças . & etiam ad maiora pondera \textbf{ non est inutile assuescere bellatores . } Secundo exercitandi sunt bellantes adinuadendum et percutiendum cum claua . Recitat enim Vegetius , \\\hline
3.3.7 & Lo segundo se deuen usar los lidiadores a acometer \textbf{ e a ferir con porras e con maças . } Ca cuenta vegeçio & non est inutile assuescere bellatores . \textbf{ Secundo exercitandi sunt bellantes adinuadendum et percutiendum cum claua . Recitat enim Vegetius , } quod antiquitus \\\hline
3.3.7 & Et los moços \textbf{ que querian acostunbrara fazer los buenos lidiadores . } vsauan los a ferir en aquellos palos & et iuuenes \textbf{ quos volebant } facere optimos bellatores exercitabant ad palos illos ita , \\\hline
3.3.7 & que querian acostunbrara fazer los buenos lidiadores . \textbf{ vsauan los a ferir en aquellos palos } assi que cada vno de aquellos moços tomaua escudo dos tanto pesado & quos volebant \textbf{ facere optimos bellatores exercitabant ad palos illos ita , } ut quilibet haberet scutum dupli ponderis quam sit scutum , \\\hline
3.3.7 & que el escudo \textbf{ que auia de leuar a la batalla . } Et tomar maça de fuste avn dos tanto pesada & ut quilibet haberet scutum dupli ponderis quam sit scutum , \textbf{ quod portatur in bello , } et clauam ligneam etiam dupli ponderis : \\\hline
3.3.7 & que auia de leuar a la batalla . \textbf{ Et tomar maça de fuste avn dos tanto pesada } que la otra & quod portatur in bello , \textbf{ et clauam ligneam etiam dupli ponderis : } et quilibet illorum iuuenum sic oneratus contra aliquem illorum palorum \\\hline
3.3.7 & que la otra \textbf{ que auia de leuar en la batalla } e cada vno de aquellos moços o maçebos assi cargado . & ø \\\hline
3.3.7 & assi usauan los mancebos prolongadamente en la mañana e en la tarde \textbf{ quando despues venien a la batalla non resçibien trabaio en ferir con la maca } nin en sofrir quales quier otros trabaios de la batalla . & cum postea veniebant ad bellum , \textbf{ non grauabantur in percutiendo cum claua , } vel in sustinendo quoscunque labores bellicos . \\\hline
3.3.7 & quando despues venien a la batalla non resçibien trabaio en ferir con la maca \textbf{ nin en sofrir quales quier otros trabaios de la batalla . } Lo terçero son de usar los lidiadores a alcançar dardos e azconetas & non grauabantur in percutiendo cum claua , \textbf{ vel in sustinendo quoscunque labores bellicos . } Tertio exercitandi sunt bellatores \\\hline
3.3.7 & nin en sofrir quales quier otros trabaios de la batalla . \textbf{ Lo terçero son de usar los lidiadores a alcançar dardos e azconetas } e a ferir con lanças & vel in sustinendo quoscunque labores bellicos . \textbf{ Tertio exercitandi sunt bellatores | admittendum tela et iacula , } et ad percutiendum cum lancea : \\\hline
3.3.7 & Lo terçero son de usar los lidiadores a alcançar dardos e azconetas \textbf{ e a ferir con lanças } la qual cosa avn fazien les mançebos al palo fincado . & admittendum tela et iacula , \textbf{ et ad percutiendum cum lancea : } quod etiam ad defixum palum fieri habet . \\\hline
3.3.7 & ca assi fazien antiguamente \textbf{ ca quando los mançebos eran usados a ferir en los palos fincados } con las maças & Fiebat enim antiquitus \textbf{ ut cum iuuenes exercitati erant ad percutiendum palos infixos } cum claua , \\\hline
3.3.7 & con las maças \textbf{ usauanse a ferir con las azconetas e con las lanças } e estauan alongados & cum claua , \textbf{ quod exercitabantur | ad percutiendum } cum telo \\\hline
3.3.7 & e usauan las braços \textbf{ assi que pudiessen ferir aquel palo o lançar cerca del } Mas conuiene de saber & vel cum iaculo , \textbf{ siue cum lancea . Stabant enim a remotis | et assuefaciebant brachia , } ut possent palum percutere , \\\hline
3.3.7 & assi que pudiessen ferir aquel palo o lançar cerca del \textbf{ Mas conuiene de saber } que en lançar dardo o lança conuiene de auer maestria & et assuefaciebant brachia , \textbf{ ut possent palum percutere , | vel saltem prope ipsum proiicere . Est autem attendendum } quod in proiiciendo telum , \\\hline
3.3.7 & Mas conuiene de saber \textbf{ que en lançar dardo o lança conuiene de auer maestria } ca primeramente es de esgrimir el dardo o la lança & vel saltem prope ipsum proiicere . Est autem attendendum \textbf{ quod in proiiciendo telum , } aut lanceam primo vibrandum est telum ipsum , \\\hline
3.3.7 & que en lançar dardo o lança conuiene de auer maestria \textbf{ ca primeramente es de esgrimir el dardo o la lança } e despues enbiarlo reziamente . & quod in proiiciendo telum , \textbf{ aut lanceam primo vibrandum est telum ipsum , } et postea fortiter impellendum : vibrato enim telo propter maiorem motum \\\hline
3.3.7 & ca primeramente es de esgrimir el dardo o la lança \textbf{ e despues enbiarlo reziamente . } ca esgrimiendo el dardo & aut lanceam primo vibrandum est telum ipsum , \textbf{ et postea fortiter impellendum : vibrato enim telo propter maiorem motum } quem efficit in aere , longius pergit \\\hline
3.3.7 & e mayor colpe faze . \textbf{ Lo quarto son de vsarlos lidiadores } a alançar saetas con arcos e con ballestas . & et amplius vulnus infligit . \textbf{ Quarto exercitandi sunt bellantes ad iaciendum sagittas , } vel cum arcubus , \\\hline
3.3.7 & Lo quarto son de vsarlos lidiadores \textbf{ a alançar saetas con arcos e con ballestas . } ca quando contesçe & Quarto exercitandi sunt bellantes ad iaciendum sagittas , \textbf{ vel cum arcubus , | vel cum ballistis . } Nam quia contingit quod ipsos hostes non possumus immediate attingere , \\\hline
3.3.7 & ca quando contesçe \textbf{ que non podemos de tan çerca llegar a los enemigos } para ferirlos . & vel cum ballistis . \textbf{ Nam quia contingit quod ipsos hostes non possumus immediate attingere , } utile est eos sagittis impugnare : \\\hline
3.3.7 & que non podemos de tan çerca llegar a los enemigos \textbf{ para ferirlos . } prouechosa cosa es lançar las saetas & Nam quia contingit quod ipsos hostes non possumus immediate attingere , \textbf{ utile est eos sagittis impugnare : } immo dato quod pugnantes se cum hostibus possint coniungere , \\\hline
3.3.7 & para ferirlos . \textbf{ prouechosa cosa es lançar las saetas } mas puesto que los lidiadores se puedan ayuntar con los enemigos & Nam quia contingit quod ipsos hostes non possumus immediate attingere , \textbf{ utile est eos sagittis impugnare : } immo dato quod pugnantes se cum hostibus possint coniungere , \\\hline
3.3.7 & prouechosa cosa es lançar las saetas \textbf{ mas puesto que los lidiadores se puedan ayuntar con los enemigos } ante que se apunte con ellos & utile est eos sagittis impugnare : \textbf{ immo dato quod pugnantes se cum hostibus possint coniungere , } antequam coniungantur proficuum est eos arcubus \\\hline
3.3.7 & ante que se apunte con ellos \textbf{ prouechosa cosa es de los espantar con los arcos e con las ballestas . } Ca leemos de çipion africano & antequam coniungantur proficuum est eos arcubus \textbf{ et ballistis terrere . } Legitur enim de Africano Scipione , \\\hline
3.3.7 & Ca leemos de çipion africano \textbf{ que quando auie de lidiar } por el pueblo de roma non cuydaua vençer en otra manera a los enemigos & Legitur enim de Africano Scipione , \textbf{ qui cum pro populo Romano certare deberet , } non aliter contra hostes se obtinere credebat , \\\hline
3.3.7 & que quando auie de lidiar \textbf{ por el pueblo de roma non cuydaua vençer en otra manera a los enemigos } si non poniendo arqueros e ballesteros mucho escogidos en todas las azes . & qui cum pro populo Romano certare deberet , \textbf{ non aliter contra hostes se obtinere credebat , } nisi in omnibus aciebus electos sagittarios miscuisset . \\\hline
3.3.7 & si non poniendo arqueros e ballesteros mucho escogidos en todas las azes . \textbf{ Lo quinto son los lidiadores de usar a a lançar piedras con fondas . } Ca esta manera de lidiar fue fallada en algunas yslas de la mar & nisi in omnibus aciebus electos sagittarios miscuisset . \textbf{ Quinto , sunt bellatores exercitandi ad iaciendum lapides cum fundis . Hic enim modus bellandi in quibusdam marinis } Insulis fuit inuentus , \\\hline
3.3.7 & Lo quinto son los lidiadores de usar a a lançar piedras con fondas . \textbf{ Ca esta manera de lidiar fue fallada en algunas yslas de la mar } do los moços & nisi in omnibus aciebus electos sagittarios miscuisset . \textbf{ Quinto , sunt bellatores exercitandi ad iaciendum lapides cum fundis . Hic enim modus bellandi in quibusdam marinis } Insulis fuit inuentus , \\\hline
3.3.7 & assi sabidores en esta arte \textbf{ que las madres nunca les querien dar de comer } fasta que ferien con la fonda en logar çierto . & adeo industres erant , \textbf{ ut matres nullum cibum eis exhiberent , } quem non primo cum funda percuterent . \\\hline
3.3.7 & Et este uso es muy prouechoso \textbf{ ca non es trabaio ninguno leuar fondas . } Ca algunas uezes contesçe & quia fundam portare , \textbf{ nullus est labor . Interdum tamen euenit , } ut in lapidosis locis habeatur conflictus , \\\hline
3.3.7 & e por ende por que el logar se defienda \textbf{ bueno es de tener fondas . Et avn para conbatimiento de los castiellos } e de las çibdades prouechoso & ut in lapidosis locis habeatur conflictus , \textbf{ et } ut mons si taliquis defendendus . In impugnatione \\\hline
3.3.7 & e de las çibdades prouechoso \textbf{ ca non es trabaio ninguno leuar fondas . } Ca algunas uezes es lançar piedras con fondas . & ut mons si taliquis defendendus . In impugnatione \textbf{ etiam castrorum et ciuitatum non inutile est lapides cum fundis eiicere . } Sexto , exercitandi sunt bellantes \\\hline
3.3.7 & ca non es trabaio ninguno leuar fondas . \textbf{ Ca algunas uezes es lançar piedras con fondas . } Lo . vi° son de usar los lidiadores & ut mons si taliquis defendendus . In impugnatione \textbf{ etiam castrorum et ciuitatum non inutile est lapides cum fundis eiicere . } Sexto , exercitandi sunt bellantes \\\hline
3.3.7 & Ca algunas uezes es lançar piedras con fondas . \textbf{ Lo . vi° son de usar los lidiadores } a ferir con pellas de fierro o de plomo . & etiam castrorum et ciuitatum non inutile est lapides cum fundis eiicere . \textbf{ Sexto , exercitandi sunt bellantes } ad percutiendum cum plumbatis . \\\hline
3.3.7 & Lo . vi° son de usar los lidiadores \textbf{ a ferir con pellas de fierro o de plomo . } Ca las pellas de plomo o de fierro atadas con alguna cadena a mango de madero & Sexto , exercitandi sunt bellantes \textbf{ ad percutiendum cum plumbatis . } Nam pila plumbea vel ferrea cum cathena aliqua coniuncta manubrio ligneo vehementem ictum reddit . \\\hline
3.3.7 & que si estudiesse ayuntada con el mango o con aste . \textbf{ Ca los lidiadores a todas maneras de colpes son de vsar } assi que contra departidos enemigos de partidamente lidien firiendo . & vel ipsi manubrio ligneo esset coniuncta . \textbf{ Ad omne enim genus percussionum exercitandi sunt bellantes , } ut contra alios et alios hostes , \\\hline
3.3.7 & Lo vij° . \textbf{ los lidiadores son de usar } a sobir ligeramente en los cauallos . & aliter \textbf{ et aliter percutiendo , dimicent . | Septimo , bellatores exercitandi sunt ad ascensiones equorum . } Nam , \\\hline
3.3.7 & los lidiadores son de usar \textbf{ a sobir ligeramente en los cauallos . } Ca assi commo cuenta vegeçio antiguamente fazien cauallos de madera . & Septimo , bellatores exercitandi sunt ad ascensiones equorum . \textbf{ Nam , } ut Vegetius recitat , \\\hline
3.3.7 & Et los mançebos vsauan en el yuierno \textbf{ a sobir en ellos solos techos . } Et en el uerano en el canpo primera miente & in hyeme exercitabantur sub tecto : \textbf{ aestate vero in campo ; } et primo equos illos ascendebant inermes , \\\hline
3.3.7 & Et tanto se usauan en esto \textbf{ que podien sobir en aquellos cauallos } a diestro e a siniestro & deinde armati : \textbf{ et adeo ad hoc assuescebant , } ut a sinistris \\\hline
3.3.7 & que en el tienpo de la batalla sin ningun detenimiento ligeramente suben en los cauallos . \textbf{ Lo . viij° son de acostunbrar los lidiadores } que sepan nadar & quod in tumultu praelii sine mora de facili ascendebant equos . \textbf{ Octauo , assuescendi sunt bellatores , } ut etiam natare sciant . \\\hline
3.3.7 & Lo . viij° son de acostunbrar los lidiadores \textbf{ que sepan nadar } ca non fallan sienpre puentes fechos & Octauo , assuescendi sunt bellatores , \textbf{ ut etiam natare sciant . } Nam non semper pontes sunt prompti : \\\hline
3.3.7 & ca non fallan sienpre puentes fechos \textbf{ por do puedan passar } e muchas uezes non saben quan fonda es el agua . & ut etiam natare sciant . \textbf{ Nam non semper pontes sunt prompti : } et multotiens ignoratur aquae profunditas ; \\\hline
3.3.7 & Por la qual cosa contesçe \textbf{ que por non saber nadar caen en muchos periglos . } Et por ende era costunbre antiguamente entre los romanos & propter quod ex ignorantia natandi , \textbf{ contingit multos periclitatos esse . } Inde est quod apud Romanos antiquitus consuetudo erat , \\\hline
3.3.7 & que por vna grant parte del dia eran usados en las armas \textbf{ si tienpo era conuenible para nadar } aduzienlos al rio & postquam per magnam partem dici exercitati essent ad arma , \textbf{ si tempus erat natationi congruum , } ducebantur ad fluuium , \\\hline
3.3.7 & aduzienlos al rio \textbf{ para que aprendiessen el arte del nadar } e non solamente los peones . mas avn los caualleros & ducebantur ad fluuium , \textbf{ ut artem natandi addiscerent . Immo non solum pedites , } sed equites , \\\hline
3.3.7 & e non solamente los peones . mas avn los caualleros \textbf{ e avn los caualleros usauan a nadar } Enpero conuiene de tener mientes & ut artem natandi addiscerent . Immo non solum pedites , \textbf{ sed equites , } et etiam ipsos equos ad natandum exercebant . Aduertendum autem quod praedictorum exercitiorum quaedam sunt magis propria equitibus , quaedam peditibus , quaedam utrisque . \\\hline
3.3.7 & e avn los caualleros usauan a nadar \textbf{ Enpero conuiene de tener mientes } que algunos destos usos sobredichos pertenesçen & ut artem natandi addiscerent . Immo non solum pedites , \textbf{ sed equites , } et etiam ipsos equos ad natandum exercebant . Aduertendum autem quod praedictorum exercitiorum quaedam sunt magis propria equitibus , quaedam peditibus , quaedam utrisque . \\\hline
3.3.7 & Et esto en qual manera sea non ha menester grant estudio . \textbf{ Ca non se puede asconder a ome sabio . } Ca sobir en los cauallos pertenesçe a los caualleros . & non magna consideratione eget , \textbf{ et solertem mentem latere non potest . } Nam ascendere equos , \\\hline
3.3.7 & Ca non se puede asconder a ome sabio . \textbf{ Ca sobir en los cauallos pertenesçe a los caualleros . } Et lançar piedas con fondas pertenesçe a los peones . & et solertem mentem latere non potest . \textbf{ Nam ascendere equos , | est proprium equitibus : } proiicere lapides cum funda , \\\hline
3.3.7 & Ca sobir en los cauallos pertenesçe a los caualleros . \textbf{ Et lançar piedas con fondas pertenesçe a los peones . } Mas las otras cosas en alguna manera puenden pertenesçer a todos . & est proprium equitibus : \textbf{ proiicere lapides cum funda , } videtur esse proprium peditibus . Alia vero sunt aliquo modo applicabilia ad utrosque . \\\hline
3.3.7 & Et lançar piedas con fondas pertenesçe a los peones . \textbf{ Mas las otras cosas en alguna manera puenden pertenesçer a todos . } P Paresçe que los negoçios de la batalla entre los otros son mas periglosos . & proiicere lapides cum funda , \textbf{ videtur esse proprium peditibus . Alia vero sunt aliquo modo applicabilia ad utrosque . } Negocia bellica \\\hline
3.3.8 & P Paresçe que los negoçios de la batalla entre los otros son mas periglosos . \textbf{ Et por ende es de poner grant acuçia en ellos . } Ca en tales cosas tan periglosas non puede auer omne tantas cautelas & inter caetera periculosiora esse videntur , \textbf{ ideo in eis est magna diligentia adhibenda . } In talibus igitur non potest quis superabundare cautelis . \\\hline
3.3.8 & Et por ende es de poner grant acuçia en ellos . \textbf{ Ca en tales cosas tan periglosas non puede auer omne tantas cautelas } que mas non sean menester . & ideo in eis est magna diligentia adhibenda . \textbf{ In talibus igitur non potest quis superabundare cautelis . } In pugna enim omnino est eligendum , \\\hline
3.3.8 & que mas non sean menester . \textbf{ Ca en la batalla sienpre auemos de tomar mayor acuçia } de que demandan las batallas . & In talibus igitur non potest quis superabundare cautelis . \textbf{ In pugna enim omnino est eligendum , } maiorem diligentiam habuisse quam bella commissa requirerent , \\\hline
3.3.8 & que en las otras cosas \textbf{ si alguna cosa es errada puedese despues emendar . } Mas los yerros de las batallas non resçiben emienda ninguna . & quod in aliis rebus \textbf{ si quid erratum est , } potest postmodum corrigi . Delicta vero bellorum emendationem non recipiunt : \\\hline
3.3.8 & que non pueden ser vençedores de sus enemigos \textbf{ e apenas o nunca osan acometer batalla contra ellos . } Por la qual cosa & vel in fugam versi adeo efficiuntur timidi , \textbf{ quod contra suos victores vix aut nunquam audent bella committere . } Quare si in bellis omnino est superabundandum cautelis , \\\hline
3.3.8 & Por la qual cosa \textbf{ si en las batallas auemos de auer muchas cautellas } non deuemos dexar & quod contra suos victores vix aut nunquam audent bella committere . \textbf{ Quare si in bellis omnino est superabundandum cautelis , } non est praetermittendum \\\hline
3.3.8 & si en las batallas auemos de auer muchas cautellas \textbf{ non deuemos dexar } qual si quier cosa & Quare si in bellis omnino est superabundandum cautelis , \textbf{ non est praetermittendum } quicquid in aliquo casu potest exercitui esse proficuum , \\\hline
3.3.8 & que la hueste ha conplido su iornada \textbf{ e quiere folgar de noche } en algun logar & ø \\\hline
3.3.8 & en algun logar \textbf{ o quiere y fazer mayor tardança } si aquel logar en algun caso o a auentura puedan a desora los enemigos venir & postquam exercitus suam dietam compleuit , \textbf{ alicubi vult pernoctare , vel ulteriorem moram contrahere , } si ad locum illum in aliquo casu , \\\hline
3.3.8 & o quiere y fazer mayor tardança \textbf{ si aquel logar en algun caso o a auentura puedan a desora los enemigos venir } luego que llegan al logar deuen fazer enderredor carcauas & alicubi vult pernoctare , vel ulteriorem moram contrahere , \textbf{ si ad locum illum in aliquo casu , | vel in aliquo euentu hostes superuenire possunt , } statim circa exercitum fiendae sunt fossae , \\\hline
3.3.8 & si aquel logar en algun caso o a auentura puedan a desora los enemigos venir \textbf{ luego que llegan al logar deuen fazer enderredor carcauas } e deuen leuantar algunas guarniçiones & vel in aliquo euentu hostes superuenire possunt , \textbf{ statim circa exercitum fiendae sunt fossae , } erigendae munitiones aliquae \\\hline
3.3.8 & luego que llegan al logar deuen fazer enderredor carcauas \textbf{ e deuen leuantar algunas guarniçiones } assi commo castiellos & statim circa exercitum fiendae sunt fossae , \textbf{ erigendae munitiones aliquae } quasi ad modum castrorum : \\\hline
3.3.8 & assi commo castiellos \textbf{ por que non se puede fallar ninguna cosa tan buena } nin tan prouechosa en la batalla & quasi ad modum castrorum : \textbf{ quia nihil neque tam salutare neque tam necessarium inuenitur } in bello , \\\hline
3.3.8 & en qual si quier tienpo \textbf{ e donde quier que vengan los enemigos acercar los } o acometer los . & quandocunque et undecunque \textbf{ superuenientes hostes obsideant . } Debet enim exercitus secum ferre munitiones congruas , \\\hline
3.3.8 & e donde quier que vengan los enemigos acercar los \textbf{ o acometer los . } Ca deue sienpre la hueste leuar consigo guarniciones conuenibles . & quandocunque et undecunque \textbf{ superuenientes hostes obsideant . } Debet enim exercitus secum ferre munitiones congruas , \\\hline
3.3.8 & o acometer los . \textbf{ Ca deue sienpre la hueste leuar consigo guarniciones conuenibles . } por que quando quisiere la hueste folgar en algun logar parezca que lieuan consigo & superuenientes hostes obsideant . \textbf{ Debet enim exercitus secum ferre munitiones congruas , } ut cum castrametari voluerit , \\\hline
3.3.8 & Ca deue sienpre la hueste leuar consigo guarniciones conuenibles . \textbf{ por que quando quisiere la hueste folgar en algun logar parezca que lieuan consigo } assi commo vna çibdat guarnida . & Debet enim exercitus secum ferre munitiones congruas , \textbf{ ut cum castrametari voluerit , } quasi quandam munitam ciuitatem secum portasse videatur . \\\hline
3.3.8 & assi commo vna çibdat guarnida . \textbf{ Visto commo es cosa prouechable a la hueste fazer carcauas } e costruir guarniçiones e castiellos . & quasi quandam munitam ciuitatem secum portasse videatur . \textbf{ Viso utile esse circa exercitum facere fossas } et construere castra : \\\hline
3.3.8 & Visto commo es cosa prouechable a la hueste fazer carcauas \textbf{ e costruir guarniçiones e castiellos . } finca de demostrar en qual manera las tales guarniciones & Viso utile esse circa exercitum facere fossas \textbf{ et construere castra : | restat ostendere , } quomodo huiusmodi monitiones et castra sunt construenda . \\\hline
3.3.8 & e costruir guarniçiones e castiellos . \textbf{ finca de demostrar en qual manera las tales guarniciones } et los tales castiellos se deuen fazer . & restat ostendere , \textbf{ quomodo huiusmodi monitiones et castra sunt construenda . } Nam si hostes sunt absentes \\\hline
3.3.8 & finca de demostrar en qual manera las tales guarniciones \textbf{ et los tales castiellos se deuen fazer . } Ca si los enemigos non estudieren cerca & restat ostendere , \textbf{ quomodo huiusmodi monitiones et castra sunt construenda . } Nam si hostes sunt absentes \\\hline
3.3.8 & Ca si los enemigos non estudieren cerca \textbf{ de ligero pueden fazer carcauas çerca de la hueste } e leuantar guarnicoñes e fazer castiellos . & Nam si hostes sunt absentes \textbf{ facile est fossas circa exercitum fodere , } munitiones erigere et castra construere . \\\hline
3.3.8 & de ligero pueden fazer carcauas çerca de la hueste \textbf{ e leuantar guarnicoñes e fazer castiellos . } Mas si los enemigos fueren cerca & facile est fossas circa exercitum fodere , \textbf{ munitiones erigere et castra construere . } Sed si aduersarii praesentes adsint , \\\hline
3.3.8 & Mas si los enemigos fueren cerca \textbf{ e fuere presentes graue cosa es de guarnescer la hueste } e de fazer castiellos . & munitiones erigere et castra construere . \textbf{ Sed si aduersarii praesentes adsint , } difficilius est castra munire . Sunt enim in tali casu duo necessaria , \\\hline
3.3.8 & e fuere presentes graue cosa es de guarnescer la hueste \textbf{ e de fazer castiellos . } Ca en tal caso commo este dos cosas son menester . & Sed si aduersarii praesentes adsint , \textbf{ difficilius est castra munire . Sunt enim in tali casu duo necessaria , } videlicet hostibus resistere , \\\hline
3.3.8 & Ca en tal caso commo este dos cosas son menester . \textbf{ Lo primero estar e lidiar contra los enemigos } Lo segundo fazer los castiellos . & difficilius est castra munire . Sunt enim in tali casu duo necessaria , \textbf{ videlicet hostibus resistere , } et castra construere . \\\hline
3.3.8 & Lo primero estar e lidiar contra los enemigos \textbf{ Lo segundo fazer los castiellos . } Pues que assi es en tal auenemiento & videlicet hostibus resistere , \textbf{ et castra construere . } In tali ergo euentu \\\hline
3.3.8 & Pues que assi es en tal auenemiento \textbf{ conmo este segunt la sentençia de los sabios es de departir la hueste en dos partes } assi que todos los caualleros & In tali ergo euentu \textbf{ secundum sapientum sententiam , est exercitus diuidendus , } ita quod omnes equites , \\\hline
3.3.8 & e alguna partida de peones deuen ser ordenados en vna az \textbf{ para refrenar e tornar a çaga el arrebato de los enemigos . } Mas la otra parte de lospeones & et una pars peditum \textbf{ debet ordinari in acie ad pellendum impetum hostium : } reliqua vero pars peditum quae possit sufficere ad celerem constructionem castrorum , \\\hline
3.3.8 & Mas la otra parte de lospeones \textbf{ que puede abastar } para fazer ligeramente los castiellos & debet ordinari in acie ad pellendum impetum hostium : \textbf{ reliqua vero pars peditum quae possit sufficere ad celerem constructionem castrorum , } debet celeriter castra construere . Oportet autem semper construendis castris , \\\hline
3.3.8 & que puede abastar \textbf{ para fazer ligeramente los castiellos } deue los costruyr & reliqua vero pars peditum quae possit sufficere ad celerem constructionem castrorum , \textbf{ debet celeriter castra construere . Oportet autem semper construendis castris , } et faciendis fossis aliquos magistros praestitui , \\\hline
3.3.8 & para fazer ligeramente los castiellos \textbf{ deue los costruyr } e fazer muy apriessa . & reliqua vero pars peditum quae possit sufficere ad celerem constructionem castrorum , \textbf{ debet celeriter castra construere . Oportet autem semper construendis castris , } et faciendis fossis aliquos magistros praestitui , \\\hline
3.3.8 & deue los costruyr \textbf{ e fazer muy apriessa . } Mas conuiene de poner algunos maestros & debet celeriter castra construere . Oportet autem semper construendis castris , \textbf{ et faciendis fossis aliquos magistros praestitui , } qui negligentes solicitent , \\\hline
3.3.8 & e fazer muy apriessa . \textbf{ Mas conuiene de poner algunos maestros } para costruyr los castiellos & debet celeriter castra construere . Oportet autem semper construendis castris , \textbf{ et faciendis fossis aliquos magistros praestitui , } qui negligentes solicitent , \\\hline
3.3.8 & Mas conuiene de poner algunos maestros \textbf{ para costruyr los castiellos } e fazer las carcauas & ø \\\hline
3.3.8 & para costruyr los castiellos \textbf{ e fazer las carcauas } que acuçien los negligentes & ø \\\hline
3.3.8 & que acuçien los negligentes \textbf{ e manden a cada vno qual cosa deua fazer . } Mostrado que prouechosa cosa es de fazer los castiellos . & qui negligentes solicitent , \textbf{ et unicuique iniungant | quod ipsum oporteat facere . } Ostenso utile esse castra construere , \\\hline
3.3.8 & e manden a cada vno qual cosa deua fazer . \textbf{ Mostrado que prouechosa cosa es de fazer los castiellos . } avn en qual manera los enemigos presentes son de fazer los castiellos & quod ipsum oporteat facere . \textbf{ Ostenso utile esse castra construere , } et qualiter \\\hline
3.3.8 & Mostrado que prouechosa cosa es de fazer los castiellos . \textbf{ avn en qual manera los enemigos presentes son de fazer los castiellos } Lo otro que nos finca de declarar quales cosas son de penssar & Ostenso utile esse castra construere , \textbf{ et qualiter | etiam praesentibus hostibus construenda sint castra : } reliquum est declarare quae sunt attendenda in constructione castrorum . In faciendis enim fossis , \\\hline
3.3.8 & avn en qual manera los enemigos presentes son de fazer los castiellos \textbf{ Lo otro que nos finca de declarar quales cosas son de penssar } en el fazimiento de los castiellos . & etiam praesentibus hostibus construenda sint castra : \textbf{ reliquum est declarare quae sunt attendenda in constructione castrorum . In faciendis enim fossis , } et in construendis castris , \\\hline
3.3.8 & deue estar assentada la hueste toda . \textbf{ Et por ende son de penssar tres cosas . } Conuiene a saber el assentamiento & inter quorum spatium est exercitus collocandus , \textbf{ tria sunt consideranda , } videlicet situs , forma ; \\\hline
3.3.8 & Et por ende son de penssar tres cosas . \textbf{ Conuiene a saber el assentamiento } e la forma e la manera de la guarnçion . & ø \\\hline
3.3.8 & Mas cerca del assentamiento \textbf{ quanto pertenesçe a lo persente son de penssar quatro cosas . } la primera que sea y abondamiento de agua & Circa situm \textbf{ ( quantum ad praesens spectat ) sunt quatuor attendenda . Primo } ut sit ibi copia aquae , \\\hline
3.3.8 & de que pueda ser acometida la hueste \textbf{ Lo terçero çerca del assentamiento es de penssar } e de cuydar el espaçio & et aliorum quae sunt exercitui necessaria . Secundo non debet esse ibi vicinus mons aliquis , \textbf{ a quo possit exercitus impugnari . Tertio circa situm considerandum est spatium ut pro numero bellatorum accipiendum est spatium , } circa quod sunt munitiones erigendae : \\\hline
3.3.8 & Lo terçero çerca del assentamiento es de penssar \textbf{ e de cuydar el espaçio } assi que por el cuento de los lidiadores es de tomar el espaçio çerca & ø \\\hline
3.3.8 & e de cuydar el espaçio \textbf{ assi que por el cuento de los lidiadores es de tomar el espaçio çerca } del qual son de leuantar las guarniçiones & ø \\\hline
3.3.8 & assi que por el cuento de los lidiadores es de tomar el espaçio çerca \textbf{ del qual son de leuantar las guarniçiones } assi que non sea tomado mayor espaçio & a quo possit exercitus impugnari . Tertio circa situm considerandum est spatium ut pro numero bellatorum accipiendum est spatium , \textbf{ circa quod sunt munitiones erigendae : } ut non accipiatur de spatio ultra quam requirat huiusmodi multitudo , \\\hline
3.3.8 & Lo quarto si conueniere \textbf{ que aquella hueste aya de fazer en aquel logar alguna tardança } e fuere cosa & Quarto si oporteat \textbf{ in loco illo exercitum moram contrahere , } et adsit possibilitas est eligenda circa situm salubritas aeris . \\\hline
3.3.8 & e fuere cosa \textbf{ que se puede fazer deuemos escoger çerca de aqual assentamiento } que sea el ayre sano . & in loco illo exercitum moram contrahere , \textbf{ et adsit possibilitas est eligenda circa situm salubritas aeris . } Nam in exercitu non solum cauenda sunt vulnera hostium , \\\hline
3.3.8 & que sea el ayre sano . \textbf{ Ca en la hueste non tan solamente deuemos escusar las feridas de los enemigos . } Mas avn si podieremos deuemos escusar las pestilençias de las enfermedades Et & et adsit possibilitas est eligenda circa situm salubritas aeris . \textbf{ Nam in exercitu non solum cauenda sunt vulnera hostium , } sed ut offert se facultas , cauendae sunt pestes morborum . Declarato ergo \\\hline
3.3.8 & Ca en la hueste non tan solamente deuemos escusar las feridas de los enemigos . \textbf{ Mas avn si podieremos deuemos escusar las pestilençias de las enfermedades Et } pues que assi es declarado quales cosas deuen ser penssadas & Nam in exercitu non solum cauenda sunt vulnera hostium , \textbf{ sed ut offert se facultas , cauendae sunt pestes morborum . Declarato ergo } quae sunt attendenda circa situm castrorum : \\\hline
3.3.8 & çerca de los assentamientos de los castiellos . \textbf{ Conuiene de declarar qual deue ser la folgura e las guarnicoñes de las carcauas } Et paresçe queUegeçio dize & quae sunt attendenda circa situm castrorum : \textbf{ declarandum est , qualis debeat esse eorum forma . Videtur autem velle Vegetius , } munitiones et fossas fiendas \\\hline
3.3.8 & que las guarniçiones e las carcauas \textbf{ que son de fazer çerca de la hueste deuen ser quadradas e luengas . } Enpero por que la forma redonda conprehende & munitiones et fossas fiendas \textbf{ circa exercitum debere habere formam quadrilateram oblongam . } Attamen quia figura circularis est capacissima , \\\hline
3.3.8 & mas que las otras \textbf{ por ende es mas de escoger de fazer las guarniçiones } segunt la figura redonda & Attamen quia figura circularis est capacissima , \textbf{ est elegibilius facere munitiones } secundum circularem formam , \\\hline
3.3.8 & por que si temen mucho del cometemiento de los enemigos \textbf{ coñuiene de fazer carcauas de muchos rencones } por que aquella figura es mas conuenible & quia si multum timeretur de impetu hostium , \textbf{ oporteret foueas facere | multorum angulorum , } eo quod illa est magis defensioni apta , \\\hline
3.3.8 & por que aquella figura es mas conuenible \textbf{ para defenderse de los enemigos } assi commo paresçra mas adelante . & multorum angulorum , \textbf{ eo quod illa est magis defensioni apta , } ut infra patebit . \\\hline
3.3.8 & que el assentamiento non sufre tal figura . \textbf{ Et por ende en tal caso deuen se fazer los castiellos } en figura de medio cerco o quedrados o de tres rencones & nisi loci situs impediat . \textbf{ Nam contingit aliquando situm illum non pati talem formam . In tali ergo casu construenda sunt castra semicircularia , triangularia , quadrata , } vel aliquam formam aliam \\\hline
3.3.8 & o çerca de aquella parte \textbf{ por do ha de salir la hueste . } Avn deuen se poner en los castiellos pendones algunos o algunas senales & ø \\\hline
3.3.8 & por do ha de salir la hueste . \textbf{ Avn deuen se poner en los castiellos pendones algunos o algunas senales } para espantar los enemigos & quae respicit hostes , \textbf{ vel circa quam protectus est exercitus . Sunt etiam in castris ponenda insignia ad terrendum hostes : } et etiam ad hoc , \\\hline
3.3.8 & Avn deuen se poner en los castiellos pendones algunos o algunas senales \textbf{ para espantar los enemigos } e avn si contesçiere & quae respicit hostes , \textbf{ vel circa quam protectus est exercitus . Sunt etiam in castris ponenda insignia ad terrendum hostes : } et etiam ad hoc , \\\hline
3.3.8 & e avn si contesçiere \textbf{ que algunos se ayan de alongar de la hueste de los castiellos . } vistas aquellas señales sepan meior tornar a la hueste o a los castiellos . & et etiam ad hoc , \textbf{ ut si contingat aliquos de exercitu elongare a castris , } visis insignis melius sciant ad castra redire . \\\hline
3.3.8 & que algunos se ayan de alongar de la hueste de los castiellos . \textbf{ vistas aquellas señales sepan meior tornar a la hueste o a los castiellos . } Et estas cosas & ut si contingat aliquos de exercitu elongare a castris , \textbf{ visis insignis melius sciant ad castra redire . } His itaque pertractatis superest \\\hline
3.3.8 & Et estas cosas \textbf{ assi dichas finca de uer } en qual manera de guarnimiento es de catar & His itaque pertractatis superest \textbf{ videre quis munitionis modus attendendus sit in constructione castrorum . } Nam si exercitus diu ibi morari intendat , eligendae sunt fortiores munitiones , et fiendae ampliores fossae . \\\hline
3.3.8 & assi dichas finca de uer \textbf{ en qual manera de guarnimiento es de catar } en el fazer de los castiellos . & His itaque pertractatis superest \textbf{ videre quis munitionis modus attendendus sit in constructione castrorum . } Nam si exercitus diu ibi morari intendat , eligendae sunt fortiores munitiones , et fiendae ampliores fossae . \\\hline
3.3.8 & en qual manera de guarnimiento es de catar \textbf{ en el fazer de los castiellos . } Ca si la hueste mucho ouiere de morar & videre quis munitionis modus attendendus sit in constructione castrorum . \textbf{ Nam si exercitus diu ibi morari intendat , eligendae sunt fortiores munitiones , et fiendae ampliores fossae . } Sed si solum ibi pernoctare cupit , \\\hline
3.3.8 & en el fazer de los castiellos . \textbf{ Ca si la hueste mucho ouiere de morar } ally son de catar & Nam si exercitus diu ibi morari intendat , eligendae sunt fortiores munitiones , et fiendae ampliores fossae . \textbf{ Sed si solum ibi pernoctare cupit , } aut ibi debet \\\hline
3.3.8 & Ca si la hueste mucho ouiere de morar \textbf{ ally son de catar } e de escoger mas fuertes guarniçiones & Sed si solum ibi pernoctare cupit , \textbf{ aut ibi debet } per modicum tempus existere , \\\hline
3.3.8 & ally son de catar \textbf{ e de escoger mas fuertes guarniçiones } e son de fazer mas anchas carcauas . & aut ibi debet \textbf{ per modicum tempus existere , } non oportet tantas munitiones expetere . Modum autem , \\\hline
3.3.8 & e de escoger mas fuertes guarniçiones \textbf{ e son de fazer mas anchas carcauas . } mas solamente quieren y estar vna noche o por poco tienpo non conuiene de fazer tantas guarniçiones . & per modicum tempus existere , \textbf{ non oportet tantas munitiones expetere . Modum autem , } et quantitatem fossarum tradit Vegetius dicens , \\\hline
3.3.8 & e son de fazer mas anchas carcauas . \textbf{ mas solamente quieren y estar vna noche o por poco tienpo non conuiene de fazer tantas guarniçiones . } Mas la manera e la quantidat de las carcauas pone la vegeçio & per modicum tempus existere , \textbf{ non oportet tantas munitiones expetere . Modum autem , } et quantitatem fossarum tradit Vegetius dicens , \\\hline
3.3.8 & Mas si la fuerça de los enemigos paresciere mas fuerte \textbf{ conuiene de fazer las carcauas mas anchas } et mas fondas & Sed si aduersariorum vis acrior imminet , \textbf{ contingit fossam ampliorem et altiorem facere ita , } ut sit lata pedes duodecim , \\\hline
3.3.8 & et mas fondas \textbf{ si han uagar para las fazer } assi que sea la carcaua ancha de doze pies & contingit fossam ampliorem et altiorem facere ita , \textbf{ ut sit lata pedes duodecim , } et alta nouem . Est tamen aduertendum quod si fossa sit alta pedum nouem , propter terram eiectam supra fossam crescit \\\hline
3.3.8 & e alta de nueue . \textbf{ Enpero conuiene de saber } que si la carcaua fuere fonda de nueue pies echando la tierra a la parte de la hueste fazese la carcaua mas alta de quatro pies & ut sit lata pedes duodecim , \textbf{ et alta nouem . Est tamen aduertendum quod si fossa sit alta pedum nouem , propter terram eiectam supra fossam crescit } quasi pedes quatuor : \\\hline
3.3.8 & Et en aquella tierra \textbf{ que echan faza dentro deuen fincar grandes palos e grandes maderos } e otras guarniçiones & et terra proiicienda est ad partem intra , \textbf{ ubi est exercitus collocandus . In terra autem illa figendi sunt stipites , et ligna , } et munitiones aliae ; \\\hline
3.3.8 & e otras guarniçiones \textbf{ las quales deueleuar consigo la hueste . } Et pues que assi es & et munitiones aliae ; \textbf{ quas secum exercitus portare debet . } Sic ergo castris constitutis , \\\hline
3.3.9 & de los fechos de las batallas \textbf{ es de poner muy grant cautella } e de tomar grand sabiduria . & Ut patet per habita , \textbf{ circa negocia bellica est cautela maxima adhibenda . } Nam quia bellorum casus irremediabiles sunt , \\\hline
3.3.9 & es de poner muy grant cautella \textbf{ e de tomar grand sabiduria . } por que los acaesçimientos de las batallas son su remedio . & circa negocia bellica est cautela maxima adhibenda . \textbf{ Nam quia bellorum casus irremediabiles sunt , } diligenter videnda sunt , \\\hline
3.3.9 & ante que la batalla publica se acometa . \textbf{ Ca meior cosa es non acometer la batalla } que se exponer sin prouision conuenible a auentura e a acaesçimiento . & prius quam pugna publica committatur : \textbf{ melius est enim pugnam non committere , } quam absque debita praeuisione fortunae \\\hline
3.3.9 & Ca meior cosa es non acometer la batalla \textbf{ que se exponer sin prouision conuenible a auentura e a acaesçimiento . } Ca nos veemos & melius est enim pugnam non committere , \textbf{ quam absque debita praeuisione fortunae | et casui se exponere . } Videmus autem in bello duo existere , \\\hline
3.3.9 & que en la batalla son dos cosas . \textbf{ Conuiene saber . Los omnes lidiadores } e las otras ayudas & Videmus autem in bello duo existere , \textbf{ videlicet viros pugnantes , } et auxilia alia quae requiruntur ad pugnam . \\\hline
3.3.9 & e las otras ayudas \textbf{ que son menester para lidiar . } Mas de parte de los omnes & videlicet viros pugnantes , \textbf{ et auxilia alia quae requiruntur ad pugnam . } Ex parte autem virorum pugnantium , \\\hline
3.3.9 & que lidian \textbf{ quanto pertenesçe a lo presente son seys cosas de penssar . } Assi commo avn de parte de las ayudas & Ex parte autem virorum pugnantium , \textbf{ quantum ad praesens spectat sex sunt attendenda : } sicut etiam ex parte auxiliorum \\\hline
3.3.9 & para la batalla \textbf{ se pueden contar otras seys cosas } las quales avn son de penssar . & et adminiculantium ad bellum , \textbf{ sex alia enumerari possunt , } quae etiam sunt attendenda . In uniuerso igitur rex , \\\hline
3.3.9 & se pueden contar otras seys cosas \textbf{ las quales avn son de penssar . } Et pues que assi es en general el rey o el prinçipe o el cabdiello de la hueste & sex alia enumerari possunt , \textbf{ quae etiam sunt attendenda . In uniuerso igitur rex , } aut Princeps , \\\hline
3.3.9 & que deue ser acucioso e mesurado e sabio \textbf{ e entendido deue penssar } doze cosas las seys de parte de los omnes & sobrius , prudens , \textbf{ et industris , } duodecim debet considerare : \\\hline
3.3.9 & doze cosas las seys de parte de los omnes \textbf{ que han de lidiar . } Et las seys de parte de aquellas cosas & et industris , \textbf{ duodecim debet considerare : } sex ex parte virorum bellatorum , \\\hline
3.3.9 & que son menester para la batalla . \textbf{ ante que venga acometer la batalla publicamente . } porque son seys cosas de parte de los enemigos lidiadores & et sex ex parte amminiculantium , \textbf{ prius quam eligat publicam pugnam committere . Sunt autem sex ex parte hominum bellatorum , } quae faciunt ad obtinendam victoriam . \\\hline
3.3.9 & porque son seys cosas de parte de los enemigos lidiadores \textbf{ que fazen ganar victoria . } Lo primero es el cuento de los lidiadores . & prius quam eligat publicam pugnam committere . Sunt autem sex ex parte hominum bellatorum , \textbf{ quae faciunt ad obtinendam victoriam . } Primum est , numerus bellantium . \\\hline
3.3.9 & ca do son mas lidiadores las otras cosas estando eguales \textbf{ segunt razon deuen auer uictoria . } Ca assi commo dize el philosofo en el segundo libro de las . & ( caeteris paribus aliis ) \textbf{ secundum quod huiusmodi sunt victoriam obtinere debent : } nam ut dicitur 2 Polit’ quantitas in compugnatione est utilis , \\\hline
3.3.9 & assi commo el mayor peso de la ualança trae al menor . \textbf{ Lo segundo de parte de los que lidian es de penssar } que sean usados a las batallas . & sicut maius pondus magis trahit . Secundo , \textbf{ ex parte bellatorum attendenda est exercitatio . } Nam habentes brachia inassueta ad percutiendum , \\\hline
3.3.9 & que sean usados a las batallas . \textbf{ Ca los que non han los braços acostunbrados para ferir } e non han los mienbros usados a la batalla & ex parte bellatorum attendenda est exercitatio . \textbf{ Nam habentes brachia inassueta ad percutiendum , } et membra inexercitata ad bellandum , \\\hline
3.3.9 & y algunos muelles e mugerilles \textbf{ que recusen de sofrir algunos trabaios } Ca estos tales vençidos & quare si sint ibi aliqui molles , \textbf{ et muliebres renuentes incommoditates aliquas sustinere , } deuicti propter incommoditates quas sustinent , \\\hline
3.3.9 & por los trabaios \textbf{ que sufren escusan de lidiar } e fuyen de la hueste . & deuicti propter incommoditates quas sustinent , \textbf{ bellare recusant } et exercitium fugiunt . \\\hline
3.3.9 & e fuyen de la hueste . \textbf{ Lo quarto es de penssar la fortaleza e la dureza del coraçon . } Ca grant deferençia es entre la dureza del fierro & et exercitium fugiunt . \textbf{ Quarto consideranda est fortitudo et durities corporis . } Multum enim interest \\\hline
3.3.9 & que los lidiadores sean de carnes blandas \textbf{ avn que ayan prouado las batallas pocas vezes quieren lidiar . } Ca assi commo dixiemos de suso & et si contingat molles carne , \textbf{ etiam postquam gustauerunt bella , | appetere pugnam ; } hoc est ut raro . \\\hline
3.3.9 & los que han las carnes muelles son mas apareiados \textbf{ para entender e para saber } mas en la mayor partida non son apareiados para lidiar . & Nam habenter carnes molles \textbf{ ( ut supra tangebatur ) sunt aptiores ad intelligendum , } sed ut plurimum sunt inepti ad pugnandum : \\\hline
3.3.9 & para entender e para saber \textbf{ mas en la mayor partida non son apareiados para lidiar . } Por que tales menos sufren el peso de las armas & ( ut supra tangebatur ) sunt aptiores ad intelligendum , \textbf{ sed ut plurimum sunt inepti ad pugnandum : } nam tales difficilius sustinent armorum pondus , \\\hline
3.3.9 & e mas se duelen de las feridas e de las llagas . \textbf{ Lo quinto es de penssar en los lidiadores arteria } e sabiduria para lidiar . & nam tales difficilius sustinent armorum pondus , \textbf{ vehementius dolent ex illatione vulnerum . Quinto , consideranda est in in bellantibus versutia } et industria ad bellandum . \\\hline
3.3.9 & Lo quinto es de penssar en los lidiadores arteria \textbf{ e sabiduria para lidiar . } Ca quanto mas sabios son los lidiadores & vehementius dolent ex illatione vulnerum . Quinto , consideranda est in in bellantibus versutia \textbf{ et industria ad bellandum . } Nam quanto cautiores sunt bellatores , \\\hline
3.3.9 & tanto mas ayna alcançan victoria . \textbf{ Lo sexto es de penssar el esfuerço } et la osadia del coraçon . & Nam quanto cautiores sunt bellatores , \textbf{ tanto citius victoriam obtinent . Sexto , attendenda est virilitas et audacia mentis , } quia audaciores \\\hline
3.3.9 & Et pues que assi es el rey o el prinçipeo el señor de la hueste \textbf{ ante que publicamente comiençen a lidiar } deue penssar seys cosas de parte de los omnes & aut princeps vel dux exercitus , \textbf{ priusquam publice dimicet , } ex parte hominum bellatorum septem considerare debet . Primo , \\\hline
3.3.9 & ante que publicamente comiençen a lidiar \textbf{ deue penssar seys cosas de parte de los omnes } que han de lidiar . & priusquam publice dimicet , \textbf{ ex parte hominum bellatorum septem considerare debet . Primo , } ex qua parte sunt plures bellatores . Secundo , \\\hline
3.3.9 & deue penssar seys cosas de parte de los omnes \textbf{ que han de lidiar . } la primera de qual parte son mas lidiadores . & priusquam publice dimicet , \textbf{ ex parte hominum bellatorum septem considerare debet . Primo , } ex qua parte sunt plures bellatores . Secundo , \\\hline
3.3.9 & La segunda quales son mas osados . \textbf{ La terçera quales son mas fuertes en sofrir los daños e las neçessidades de la batalla } Lo quarto quales son mas rezios e mas duros en el cuerpo . & ex qua parte sunt plures bellatores . Secundo , \textbf{ qui sunt magis exercitati . Tertio , | qui sunt fortiores in sustinendo necessitates , et incommoda . Quarto , } qui sunt robustiores , \\\hline
3.3.9 & Lo quinto quales son mas sabidores \textbf{ e mas arteros para lidiar . } Lo sexto quales son mas osados e mas fuertes de coracon . Estonçe el cabdiello de la hueste mesurado & et duriores corpore . Quinto , \textbf{ qui sunt industriores , } et sagaciores mente . Sexto , \\\hline
3.3.9 & que viere la su hueste ha conplimiento en estas seys condiçiones \textbf{ et fallesçe en ellas podra acometer la vatalla } mas ayna o prolongar la & prout viderit suum exercitum in his conditionibus abundare , \textbf{ aut deficere : | poterit accelerare pugnam , } vel differre : \\\hline
3.3.9 & et fallesçe en ellas podra acometer la vatalla \textbf{ mas ayna o prolongar la } e lidiar publicamente o manifiestamente o por assechos & poterit accelerare pugnam , \textbf{ vel differre : } et bellare publice et aperte , \\\hline
3.3.9 & mas ayna o prolongar la \textbf{ e lidiar publicamente o manifiestamente o por assechos } e por çeladas e ascondidamente . Contadas las seys condiçiones & vel differre : \textbf{ et bellare publice et aperte , } vel per insidias et latenter . \\\hline
3.3.9 & e por çeladas e ascondidamente . Contadas las seys condiçiones \textbf{ que son de penssar ante } que acometan publicamente de parte de los ommes lidiadores & Enumeratis septem conditionibus , \textbf{ quae considerandae sunt prius } quam committatur bellum publicum ex parte hominum bellatorum : \\\hline
3.3.9 & que acometan publicamente de parte de los ommes lidiadores \textbf{ finca nos de contar otras seys condiçiones } que son tomadas de parte de aquellas cosas & quam committatur bellum publicum ex parte hominum bellatorum : \textbf{ reliquum est enumerare sex alia , } quae sumuntur ex parte amminiculantium \\\hline
3.3.9 & Ca en la batalla ayudan los cauallos e las armas e las viandas \textbf{ e los logares de lidiar } e el tienpo e el ayuda demandada e prometida . & et eorum quae auxiliantur ad bellum . In bello quidem auxiliantur equi arma , victualia , \textbf{ loca pugnandi , } tempus , \\\hline
3.3.9 & e el tienpo e el ayuda demandada e prometida . \textbf{ Et por ende el cabdiello deue penssar todas estas cosasLo . } primero deue penssar el cabdiello de la hueste & et auxilium praestolatum . \textbf{ Debet igitur dux exercitus considerare primo ex qua parte sunt plures equi } et meliores . \\\hline
3.3.9 & Et por ende el cabdiello deue penssar todas estas cosasLo . \textbf{ primero deue penssar el cabdiello de la hueste } de qual parte son mas caualleros e meiores . & et auxilium praestolatum . \textbf{ Debet igitur dux exercitus considerare primo ex qua parte sunt plures equi } et meliores . \\\hline
3.3.9 & e por mengua e por pobreza non pueden estar en la hueste . \textbf{ Lo quarto es de penssar el logar de la batalla } e quales son assentados en mas alto o en meior lograr para lidiar . & non valentes moram contrahere . \textbf{ Quarto considerandus est pugnationis locus } qui sunt in altiori situ , \\\hline
3.3.9 & Lo quarto es de penssar el logar de la batalla \textbf{ e quales son assentados en mas alto o en meior lograr para lidiar . } Lo quinto en la batalla es de penssar el tienpo & Quarto considerandus est pugnationis locus \textbf{ qui sunt in altiori situ , | vel meliori ad pugnandum . } Quinto circa pugnam attendendum est tempus , \\\hline
3.3.9 & e quales son assentados en mas alto o en meior lograr para lidiar . \textbf{ Lo quinto en la batalla es de penssar el tienpo } si en aquel tienpo & vel meliori ad pugnandum . \textbf{ Quinto circa pugnam attendendum est tempus , } utrum tempore quo committenda est pugna , \\\hline
3.3.9 & si en aquel tienpo \textbf{ en que es de acometer la batalla } es el sol contrario a las caras de los enemigos & Quinto circa pugnam attendendum est tempus , \textbf{ utrum tempore quo committenda est pugna , } sol sit oppositus faciebus eorum , \\\hline
3.3.9 & ca los que han el uiento e el sol e el poluo \textbf{ contra si resçiben daño de los oios en manera que non pueden lidiar . } Lo vi° es de penssar & nam habentes solem et ventum siue puluerem contra se , offendentur in oculis \textbf{ ut dimicare non possint . } Sexto est attendendum , \\\hline
3.3.9 & contra si resçiben daño de los oios en manera que non pueden lidiar . \textbf{ Lo vi° es de penssar } quales esperan mayores ayudas . & ut dimicare non possint . \textbf{ Sexto est attendendum , } qui plures auxiliatores expectant . \\\hline
3.3.9 & o non conuiene de lidiaro \textbf{ conuiene de apressurar la batalla . } Mas si ellos esperan mayores ayudas deuen alongar la batalla . & si hostes plura expectant auxilia , vel non est bellandum , \textbf{ vel acceleranda est pugna . } Si autem ipsi plures auxiliatores expectant , \\\hline
3.3.9 & conuiene de apressurar la batalla . \textbf{ Mas si ellos esperan mayores ayudas deuen alongar la batalla . } Et pues que & vel acceleranda est pugna . \textbf{ Si autem ipsi plures auxiliatores expectant , } est compugnatio differenda . \\\hline
3.3.9 & el cabdello sabio de la hueste \textbf{ conplidamente puede entender } sil conuiene de acometer batalla publica o non . & His itaque igitur omnibus diligenter inspectis , \textbf{ prudens dux exercitus sufficienter aduertere potest , } utrum debeat publicam pugnam committere . \\\hline
3.3.9 & conplidamente puede entender \textbf{ sil conuiene de acometer batalla publica o non . } Ca segunt que viere & prudens dux exercitus sufficienter aduertere potest , \textbf{ utrum debeat publicam pugnam committere . } Nam \\\hline
3.3.9 & que abonda o fallesçe en las mas destas condiçiones \textbf{ assi se podra auer en la batalla } e por auentura nunca contezçra que todas estas condiçones puedan ser de la vna parte . & vel deficere , \textbf{ sic se habere poterit erga bellum : } forte enim nunquam contingeret omnes conditiones praefatas concurrere ex una parte : \\\hline
3.3.9 & Enpero do mas e meiores condiçiones fueren falladas \textbf{ aquella parte es la meior para lidiar . } s sienpre la uirtud ayuntada e ordenada es mas fuerte & et meliores conditiones concurrunt , \textbf{ est pars potior ad bellandum . } Semper virtus unita fortior est seipsa dispersa \\\hline
3.3.10 & Pues que assi es \textbf{ por que esto non pudiesse contesçer } los antiguos guardauan esto & et confundi . \textbf{ Ne igitur hoc posset accidere , } obseruabatur antiquitus , \\\hline
3.3.10 & señalPor ende prouechosa cosa fue \textbf{ e es en las batallas de leuar pendones e sobreseñales } por que se non desordenasse la hueste . & ut si contingeret aliquem bellatorem deuiare a propria acie , \textbf{ de facili rediret ad illam ; utile ergo fuit in bellis insignia et vexilla deferre , } ne confunderetur exercitus . \\\hline
3.3.10 & a la qual catando los deanes conosçian al su senora propreo \textbf{ e sabian aqual dean seguir } o a qual dean acometer . & quod respicientes decani agnoscebant centurionem proprium , \textbf{ et sciebant } quem sequi debebant . Sic \\\hline
3.3.10 & e sabian aqual dean seguir \textbf{ o a qual dean acometer . } Avn en essa misma manera en el yelmo de cada dean & et sciebant \textbf{ quem sequi debebant . Sic } etiam in galea cuiuslibet \\\hline
3.3.10 & tan bien en la az de los caualleros \textbf{ commo de los peones son de establesçer cabdiellos e mayorales e alferezes } que lieuen los pendones & vel alio tam in aciebus equitum , \textbf{ quam etiam peditum constituendi sunt duces et praepositi , } et ferentes vexilla : \\\hline
3.3.10 & por que cada vno sepa \textbf{ lo que ha de fazer . } por que tan grande es el espanto en la batalla & et ferentes vexilla : \textbf{ ut quilibet sciat quid debeat agere . } Est enim tantus terror in bello propter armorum strepitum et percussiones illatas , \\\hline
3.3.10 & por que tan grande es el espanto en la batalla \textbf{ ponr el roydo de las armas } e por los colpes que se dan & ut quilibet sciat quid debeat agere . \textbf{ Est enim tantus terror in bello propter armorum strepitum et percussiones illatas , } quod verba et monitiones non sufficiunt ad dirigendum bellantes , \\\hline
3.3.10 & que las palabras e las guarniçiones non abastan \textbf{ para guiar los lidiadores . } Mas conuiene de dar otras seña les manifiestas . por que cada vno viendo aquellas señales & quod verba et monitiones non sufficiunt ad dirigendum bellantes , \textbf{ sed oportet dare euidentia signa ; } ut quilibet solo intuitu sciat se tenere ordinate in acie , \\\hline
3.3.10 & para guiar los lidiadores . \textbf{ Mas conuiene de dar otras seña les manifiestas . por que cada vno viendo aquellas señales } se sepa tener ordenadamente en su az & quod verba et monitiones non sufficiunt ad dirigendum bellantes , \textbf{ sed oportet dare euidentia signa ; } ut quilibet solo intuitu sciat se tenere ordinate in acie , \\\hline
3.3.10 & Mas conuiene de dar otras seña les manifiestas . por que cada vno viendo aquellas señales \textbf{ se sepa tener ordenadamente en su az } e sepa que ha de fazer & sed oportet dare euidentia signa ; \textbf{ ut quilibet solo intuitu sciat se tenere ordinate in acie , } et cognoscat \\\hline
3.3.10 & se sepa tener ordenadamente en su az \textbf{ e sepa que ha de fazer } e desto puede paresçer & ut quilibet solo intuitu sciat se tenere ordinate in acie , \textbf{ et cognoscat | quid sit acturus . } Ex hoc autem patere potest , \\\hline
3.3.10 & e sepa que ha de fazer \textbf{ e desto puede paresçer } quales deuen ser los que lieuan los pendones e las señales . & quid sit acturus . \textbf{ Ex hoc autem patere potest , } quales debeant esse portantes insignia \\\hline
3.3.10 & Et pues que \textbf{ assi es con grant sabiduria es de escoger el alferez } assi que sea fuerte de cuerpo & Nam vexillo confracto totus exercitus est confusus . \textbf{ Cum magna igitur diligentia est vexillifer eligendus , } ut sit corpore fortis , \\\hline
3.3.10 & assi que non auian cabesça \textbf{ nin sabian a quien auian de tener mientes . } Et por ende sien la batalla la uida & quasi non habentes caput , \textbf{ et ignorantes ad quid deberent attendere : propter quod si in debellatione vita multorum hominum periculis mortis exponitur , } cum magna diligentia vexillifer est quaerendus . \\\hline
3.3.10 & e con grant diligençia deue ser escogido el alferes . \textbf{ de las cosas sobredichas puede paresçer } qual deua ser el cabdiello & cum magna diligentia vexillifer est quaerendus . \textbf{ Ex dictis etiam patere potest , } qualis esse debeat \\\hline
3.3.10 & e altos en el estado del cuerpo \textbf{ e que sepan lançar lanças e dardos . } e avn que sepan esgrimir las espadas & si debent boni bellatores existere debeant esse fortes viribus , \textbf{ proceri statura , scientes proiicere hastas } et tela : \\\hline
3.3.10 & e que sepan lançar lanças e dardos . \textbf{ e avn que sepan esgrimir las espadas } para ferir meior & proceri statura , scientes proiicere hastas \textbf{ et tela : } scire \\\hline
3.3.10 & e avn que sepan esgrimir las espadas \textbf{ para ferir meior } e rodear el escudo & proceri statura , scientes proiicere hastas \textbf{ et tela : } scire \\\hline
3.3.10 & para ferir meior \textbf{ e rodear el escudo } para encobrirse meior & et tela : \textbf{ scire } etiam debeant gladium vibrare ad percutiendum , portare scutum ad se protegendum : \\\hline
3.3.10 & e rodear el escudo \textbf{ para encobrirse meior } e avn que ayan los oios bien espiertos & et tela : \textbf{ scire } etiam debeant gladium vibrare ad percutiendum , portare scutum ad se protegendum : \\\hline
3.3.10 & e avn que ayan los oios bien espiertos \textbf{ e que sean ligeros e mesurados en beuer e gerrdados de vino } e avn que ayan vso de las armas . & scire \textbf{ etiam debeant gladium vibrare ad percutiendum , portare scutum ad se protegendum : } et cum debeant esse vigilantes , \\\hline
3.3.10 & grande en su estado \textbf{ e sabidor en lançar lanças e dardos } e sabio en lidiar & procer statura , \textbf{ sciens eiicere hastas | et iacula , } sciens dimicare gladio ad percutiendum , \\\hline
3.3.10 & e sabidor en lançar lanças e dardos \textbf{ e sabio en lidiar } e sepa esgrimir el espada & et iacula , \textbf{ sciens dimicare gladio ad percutiendum , } portare scutum ad se protegendum , \\\hline
3.3.10 & e sabio en lidiar \textbf{ e sepa esgrimir el espada } para ferir meior . Rodearse e cobrirse del escudo & et iacula , \textbf{ sciens dimicare gladio ad percutiendum , } portare scutum ad se protegendum , \\\hline
3.3.10 & e sepa esgrimir el espada \textbf{ para ferir meior . Rodearse e cobrirse del escudo } para se guardar e despierto e vigilante e ligero e mesurado & sciens dimicare gladio ad percutiendum , \textbf{ portare scutum ad se protegendum , } vigilans , agilis , \\\hline
3.3.10 & para ferir meior . Rodearse e cobrirse del escudo \textbf{ para se guardar e despierto e vigilante e ligero e mesurado } e que aya prueua de todas las armas & portare scutum ad se protegendum , \textbf{ vigilans , agilis , | sobrius , } habens omnium armorum experientiam ; \\\hline
3.3.10 & e que aya prueua de todas las armas \textbf{ e que sepa ensseñar los lidiadores } quel son acomendados & habens omnium armorum experientiam ; \textbf{ ut sciat erudire pugnantes sibi commissos , } et cogat eos ad bene debellandum , et ad arma tergendum . \\\hline
3.3.10 & e que los costringa \textbf{ para bien lidiar } e para alinpiar las armas & ut sciat erudire pugnantes sibi commissos , \textbf{ et cogat eos ad bene debellandum , et ad arma tergendum . } Nam ipse armorum nitor terrorem incutit hostibus , ut portans huiusmodi arma credatur \\\hline
3.3.10 & para bien lidiar \textbf{ e para alinpiar las armas } por que el resplandesçimiento de las armas pone grant espanto a los enemigos & et cogat eos ad bene debellandum , et ad arma tergendum . \textbf{ Nam ipse armorum nitor terrorem incutit hostibus , ut portans huiusmodi arma credatur } bonus esse bellator . Ipsa enim rubigo armorum in eo qui portat illa , \\\hline
3.3.10 & assi que el que traye tales armas es tenido por buen lidiador \textbf{ por que la ferrunbre e el oryn de las armas muestra pareza de lidiar } en aquel que las traye . & bonus esse bellator . Ipsa enim rubigo armorum in eo qui portat illa , \textbf{ arguit inertiam bellandi . } Si ergo talis debet esse \\\hline
3.3.10 & que sea ligero en el cuerpo \textbf{ por que pueda avn que sea armado sobir ligeramente en el cauallo } e sepa bien caualgar & ut possit \textbf{ etiam armatus agiliter equum conscendere : } scire fortiter equitare , \\\hline
3.3.10 & por que pueda avn que sea armado sobir ligeramente en el cauallo \textbf{ e sepa bien caualgar } e ferir fuertemiente con la lança & etiam armatus agiliter equum conscendere : \textbf{ scire fortiter equitare , } cum lancea percutere , eiicere iacula , \\\hline
3.3.10 & e sepa bien caualgar \textbf{ e ferir fuertemiente con la lança } e lançar lança e dardo & scire fortiter equitare , \textbf{ cum lancea percutere , eiicere iacula , } cum scuto se protegere , \\\hline
3.3.10 & e ferir fuertemiente con la lança \textbf{ e lançar lança e dardo } e sepa cobrirse del escudo & scire fortiter equitare , \textbf{ cum lancea percutere , eiicere iacula , } cum scuto se protegere , \\\hline
3.3.10 & e lançar lança e dardo \textbf{ e sepa cobrirse del escudo } et sepa ferir con la maca & cum lancea percutere , eiicere iacula , \textbf{ cum scuto se protegere , } cum claua \\\hline
3.3.10 & e sepa cobrirse del escudo \textbf{ et sepa ferir con la maca } e lidiar bien con el espada & cum scuto se protegere , \textbf{ cum claua } et ense dimicare , habere omnium armorum exercitium , \\\hline
3.3.10 & et sepa ferir con la maca \textbf{ e lidiar bien con el espada } e que aya uso en todas las armas & cum claua \textbf{ et ense dimicare , habere omnium armorum exercitium , } ut possit suos commilitones de pugna erudire , \\\hline
3.3.10 & e que aya uso en todas las armas \textbf{ por que pueda ensseñar todos los sus caualleros a la batalla } por que lidien fuertemente e quel alinpien las armas & et ense dimicare , habere omnium armorum exercitium , \textbf{ ut possit suos commilitones de pugna erudire , } ut fortiter pugnent , \\\hline
3.3.10 & por que lidien fuertemente e quel alinpien las armas \textbf{ e sepan fazer todas las otras cosas } que son menester para la batalla . & arma tergant , \textbf{ et alia faciant quae requiruntur ad bellum . } Mors est quid terribilissimum , \\\hline
3.3.11 & Et por ende alli do paresçen la muerte del pueblo \textbf{ e do los enemigos assechan a la muerte de los çibdadanos es de buscar toda cautela } por que la hueste sea guardada sin danno & Ubi ergo quaeritur mors populi , \textbf{ et ubi hostes insidiantur morti ciuium , | est omnis cautela adhibenda , } ut exercitus seruetur illaesus , \\\hline
3.3.11 & e por que sea guardada la uida de los çibdadanos . \textbf{ Et pues que assi es non abasta de penssar aquellas cosas } que son de cuydar en acometer la batalla publicamente & et ut vita ciuium conseruetur . \textbf{ Non ergo sufficit considerare ea quae sunt consideranda in pugna publica committenda nisi sciantur cautelae ad remouendum impedimenta viarum , } ne exercitus per insidias hostium periclitetur in via . Possumus autem , \\\hline
3.3.11 & Et pues que assi es non abasta de penssar aquellas cosas \textbf{ que son de cuydar en acometer la batalla publicamente } si non fueren sabidas las cautelas & et ut vita ciuium conseruetur . \textbf{ Non ergo sufficit considerare ea quae sunt consideranda in pugna publica committenda nisi sciantur cautelae ad remouendum impedimenta viarum , } ne exercitus per insidias hostium periclitetur in via . Possumus autem , \\\hline
3.3.11 & si non fueren sabidas las cautelas \textbf{ para tirar los enbargos de los enemigos } por que la hueste por assechos & ø \\\hline
3.3.11 & Mas nos podemos \textbf{ quanto pertenesçe a lo presente contar ocho cautelas . } las quales deue el caudiello de la batalla retener en ssu memoria & ne exercitus per insidias hostium periclitetur in via . Possumus autem , \textbf{ quantum ad praesens spectat , octo cautelas enumerare : } quas debet dux belli retinere memoriter , \\\hline
3.3.11 & quanto pertenesçe a lo presente contar ocho cautelas . \textbf{ las quales deue el caudiello de la batalla retener en ssu memoria } por que se salue la uida de los lidiadores & quantum ad praesens spectat , octo cautelas enumerare : \textbf{ quas debet dux belli retinere memoriter , } ut saluetur vita pugnatorum , \\\hline
3.3.11 & La primera es que los caminos de la tierra \textbf{ por do ha de yr la hueste deue tener el cabdiello escriptos . } assi que los valles de los logares & ut sciat itinera regionum , \textbf{ per quae exercitus proficisci debet : } et interualla locorum , \\\hline
3.3.11 & e los destaios e los montes e los rios \textbf{ que son en aquel camino todos los deuen tener escriptos . } Ca avn si podiesse ser & et montes , \textbf{ et flumina existentia in itinere illo debet habere conscripta . } Immo si viae illae , \\\hline
3.3.11 & Ca avn si podiesse ser \textbf{ que aquellas carreras e los passos e los rios el cabdiello de la batalla pudiesse auer pintados . } assi que por vista de los oios catasse en qual manera la hueste pudiesse andar . & Immo si viae illae , \textbf{ et passus , | et flumina dux exercitus haberet depicta , } quasi oculorum aspectu prospiceret qualiter exercitus deberet pergere , \\\hline
3.3.11 & que aquellas carreras e los passos e los rios el cabdiello de la batalla pudiesse auer pintados . \textbf{ assi que por vista de los oios catasse en qual manera la hueste pudiesse andar . } Mas seguramente podria guiar su hueste & et flumina dux exercitus haberet depicta , \textbf{ quasi oculorum aspectu prospiceret qualiter exercitus deberet pergere , } tutius posset suum exercitum ducere . Sic etiam marinarii faciunt , \\\hline
3.3.11 & assi que por vista de los oios catasse en qual manera la hueste pudiesse andar . \textbf{ Mas seguramente podria guiar su hueste } por que assi lo fazen los marineros . & quasi oculorum aspectu prospiceret qualiter exercitus deberet pergere , \textbf{ tutius posset suum exercitum ducere . Sic etiam marinarii faciunt , } qui videntes maris pericula , \\\hline
3.3.11 & las quales catando las los marineros luego entienden \textbf{ en qual manera deuen andar en la mar } e en qual logar son & qui marinarii intuentes , \textbf{ statim percipiunt qualiter debeant pergere , } et in quo loco existant , \\\hline
3.3.11 & e en qual logar son \textbf{ e de quales cosas se deuen guardar por la qual cosa commo } por los assechos de los enemigos la hueste sea puesta & et in quo loco existant , \textbf{ et a quibus debeant se cauere . } Quare propter insidias hostium exercitus tot quasi , \\\hline
3.3.11 & que los marineros en la mar . \textbf{ Et en ninguna manera la hueste non deue yr } por ninguna carrera en la qual puede rescebir daño de los asechos de los enemigos & vel etiam pluribus periculis exponatur in via quam nautae in mari , \textbf{ nullo modo } debet exercitus pergere per viam aliquam in qua pati possit insidias , \\\hline
3.3.11 & Et en ninguna manera la hueste non deue yr \textbf{ por ninguna carrera en la qual puede rescebir daño de los asechos de los enemigos } nin de las çeladas & nullo modo \textbf{ debet exercitus pergere per viam aliquam in qua pati possit insidias , } nisi qualitates viarum , montes , flumina , \\\hline
3.3.11 & que dicho es \textbf{ que deue auer las carreras } e las qualidades de los caminos escriptas e pintadas . & ut simul cum hoc quod habet vias \textbf{ et qualitates viarum conscriptas et depictas , } ducat dux belli conductores aliquos bene scientes vias illas , \\\hline
3.3.11 & e que las ayan prouado . \textbf{ Ca ver algunas cosas escriptas e pintadas non son } assi sabidas & et experti sint illas . \textbf{ Nam videre aliqua conscripta et depicta non sunt ita nota , } sicut si per seipsa sensibiliter videmus ipsa . Nam potior est cognitio \\\hline
3.3.11 & por el qual es sabida en pintura o en su semeiança . \textbf{ Enpero por que los guiadores non puedan fazer algunos engaños } deue el señor de la hueste poner en ellos buenas guardas & vel in alio simili . \textbf{ Ne tamen conductores moliantur fraudes aliquas , } debet circa eos dux belli bonas apponere custodias \\\hline
3.3.11 & Enpero por que los guiadores non puedan fazer algunos engaños \textbf{ deue el señor de la hueste poner en ellos buenas guardas } porque non puedan foyr . & Ne tamen conductores moliantur fraudes aliquas , \textbf{ debet circa eos dux belli bonas apponere custodias } ne possint fugere . Debet etiam eis minari mortem , \\\hline
3.3.11 & deue el señor de la hueste poner en ellos buenas guardas \textbf{ porque non puedan foyr . } Avn deuen los amenazar de muerte & debet circa eos dux belli bonas apponere custodias \textbf{ ne possint fugere . Debet etiam eis minari mortem , } si in aliquo fraudulenter se habeant , \\\hline
3.3.11 & porque non puedan foyr . \textbf{ Avn deuen los amenazar de muerte } si en alguna cosa se ouieren engañosamente & debet circa eos dux belli bonas apponere custodias \textbf{ ne possint fugere . Debet etiam eis minari mortem , } si in aliquo fraudulenter se habeant , \\\hline
3.3.11 & si en alguna cosa se ouieren engañosamente \textbf{ e deuenles prometer dones } si fueren fieles a su prinçipe . & si in aliquo fraudulenter se habeant , \textbf{ et promittere dona } si se fideliter gesserint . \\\hline
3.3.11 & de cuyo conseio faga todas aquellas cosas \textbf{ que ouiere de fazer . } Ca do puede contesçer tan grant periglo & quicquid viderit ipse dux belli esse fiendum . \textbf{ Nam ubi tantum currit periculum , } nullus debet inniti proprio capiti , \\\hline
3.3.11 & que ouiere de fazer . \textbf{ Ca do puede contesçer tan grant periglo } ninguno non deue esforçar se en su cabeça propria & quicquid viderit ipse dux belli esse fiendum . \textbf{ Nam ubi tantum currit periculum , } nullus debet inniti proprio capiti , \\\hline
3.3.11 & Ca do puede contesçer tan grant periglo \textbf{ ninguno non deue esforçar se en su cabeça propria } nin creer a ssi mesmo solo . & Nam ubi tantum currit periculum , \textbf{ nullus debet inniti proprio capiti , } nec credere sibi soli . \\\hline
3.3.11 & ninguno non deue esforçar se en su cabeça propria \textbf{ nin creer a ssi mesmo solo . } La quarta cautela es & nullus debet inniti proprio capiti , \textbf{ nec credere sibi soli . } Quarta cautela est , \\\hline
3.3.11 & que los caminos \textbf{ por do deue yr la hueste non deuen ser sabidos de otro . } si non del cabdiello e de los conseieros . & Quarta cautela est , \textbf{ ut itinera ignorentur ab hostib’ , } per quae debet exercitus proficisci . \\\hline
3.3.11 & que es librado por el conseio \textbf{ por quales caminos deue yr la hueste } e aquellas carreras touiere el cabdiello escriptas e pintadas & et citius fini debito mancipantur . \textbf{ Postquam igitur deliberatum est per quas vias debet exercitus pergere , } et vias illas Dux habet conscriptas et depictas , \\\hline
3.3.11 & que deue el señor de la hueste en cada conpaña \textbf{ e en cada vna az auer vnos caualleros muy fieles e muy estremados } que ayan cauallos muy ligeros & Quinta est , in quolibet munimine , \textbf{ et in qualibet acie habere aliquos equites fidelissimos | et strenuissimos , } habentes equos veloces et fortes ; \\\hline
3.3.11 & por que los enemigos sy yoguieren ascondidos en alguna parte \textbf{ no pueden fazer daño en la hueste . } Ca maguera quel conseio del cabdiello non sea sabido a ninguno . & illustrantes et discooperientes insidias , \textbf{ ne hostes aliqui latitantes ex aliqua parte molestent exercitum . } Nam \\\hline
3.3.11 & Ca maguera quel conseio del cabdiello non sea sabido a ninguno . \textbf{ Empero luego luegoque la hueste comiença a mouer } por algunos caminos pueda cada vno asmar & etsi nullis esset notum ducis consilium , \textbf{ eo tamen ipso quod per aliquas vias incipit exercitus iter arripere , } coniecturari quis potest per quas partes debeat proficisci . \\\hline
3.3.11 & Empero luego luegoque la hueste comiença a mouer \textbf{ por algunos caminos pueda cada vno asmar } por quales partes ha de yr . & etsi nullis esset notum ducis consilium , \textbf{ eo tamen ipso quod per aliquas vias incipit exercitus iter arripere , } coniecturari quis potest per quas partes debeat proficisci . \\\hline
3.3.11 & por algunos caminos pueda cada vno asmar \textbf{ por quales partes ha de yr . } Et por que es cosa prouada & eo tamen ipso quod per aliquas vias incipit exercitus iter arripere , \textbf{ coniecturari quis potest per quas partes debeat proficisci . } Et quia probabile est semper in talibus aliquos exploratores adesse cogitare debet dux belli quod et hoc posset , ad aures hostium peruenire \\\hline
3.3.11 & que sienpre en las tales cosas pueden ser algunos assechadores \textbf{ e descobridores deue cuydar el senor de la hueste } que avn este podria venir en las oreias de los enemigos . & coniecturari quis potest per quas partes debeat proficisci . \textbf{ Et quia probabile est semper in talibus aliquos exploratores adesse cogitare debet dux belli quod et hoc posset , ad aures hostium peruenire } Itaque cum pericula visa minus noceant , \\\hline
3.3.11 & e descobridores deue cuydar el senor de la hueste \textbf{ que avn este podria venir en las oreias de los enemigos . } Et por ende por que los periglos & coniecturari quis potest per quas partes debeat proficisci . \textbf{ Et quia probabile est semper in talibus aliquos exploratores adesse cogitare debet dux belli quod et hoc posset , ad aures hostium peruenire } Itaque cum pericula visa minus noceant , \\\hline
3.3.11 & que son ante vistos menos enpeesçen . \textbf{ por caualleros muy ligeros son de descobrir las çeladas } por que la hueste non aya de resçebir a desora en alguna parte algunos daños . & Itaque cum pericula visa minus noceant , \textbf{ per velocissimos equites sunt detegendae insidiae , } ne exercitus circa aliquam partem ex improuiso patiatur molestias . \\\hline
3.3.11 & por caualleros muy ligeros son de descobrir las çeladas \textbf{ por que la hueste non aya de resçebir a desora en alguna parte algunos daños . } La sexta cautela es & per velocissimos equites sunt detegendae insidiae , \textbf{ ne exercitus circa aliquam partem ex improuiso patiatur molestias . } Sexta est , \\\hline
3.3.11 & en la qual cuydaren \textbf{ que puede venir mayor perigso . } Et sy por auentura dubdan de periglo de cada parte & et magis bellicosi pedites apponantur , \textbf{ ex qua creditur maius periculum imminere : } quod si ex omni parte de periculo dubitatur , \\\hline
3.3.11 & Et sy por auentura dubdan de periglo de cada parte \textbf{ son de poner remedios . } La vij° . cautela es & quod si ex omni parte de periculo dubitatur , \textbf{ undique sunt remedia adhibenda . } Septima est , \\\hline
3.3.11 & Por ende en cada vna ora se deue \textbf{ assi auer la hueste } e assi estar aperçebida & Nam interrupta acie facilius debellatur . \textbf{ In qualibet enim hora sic exercitus se debet habere , } ut si et tunc hostes praesentes adessent , \\\hline
3.3.11 & que si los enemigos fuessen pressentes \textbf{ non les pudiessen fazer daño ninguno . } Et por ende dize & ut si et tunc hostes praesentes adessent , \textbf{ ei non possent efficere nocumentum . } Unde et prouerbialiter dicitur , \\\hline
3.3.11 & a qui es acomendada la uida de tantos omnes deue ser muy acuçioso e despierto \textbf{ por que los enemigos non puedan acometer los } assy commo a negligentes o adormidos . & et vigilans , \textbf{ ne hostes eum inuadere possent } quasi negligentem \\\hline
3.3.11 & que eran ante puestos a las obras de la batalla \textbf{ sienpre amonestar sus caualleros e sus peones } que sean prestos e apareiados a las armas & et alii , \textbf{ qui operibus bellicis praeponuntur , semper monere milites pedites , ut sint parati ad arma ; } ut si contingeret aliqua inuasio subita , \\\hline
3.3.11 & que alguno les acometiesse \textbf{ puedan defender se de los acometedores . } Ca assi diziendo & ut si contingeret aliqua inuasio subita , \textbf{ possent inuadentibus resistere . Sic enim dicendo , } dato quod accideret aliquis repentinus insultus , \\\hline
3.3.11 & Ca assi diziendo \textbf{ puesto que contesçiesse algun rebate a desora menos les podria enpeesçer } por que estauan aperçebidos . & possent inuadentibus resistere . Sic enim dicendo , \textbf{ dato quod accideret aliquis repentinus insultus , | esset quasi prouisus , } et minus praestaret nocumentum . \\\hline
3.3.11 & por que estauan aperçebidos . \textbf{ La . viij° . cautela es penssar } de quales ha mayor conplimiento la hueste de peones o de caualleros . & et minus praestaret nocumentum . \textbf{ Octaua cautela est , } considerare exercitum in quibus sit copiosior , \\\hline
3.3.11 & que ha conplimiento de caualleros o de peones \textbf{ podra escoger los caminos de los canpos } e carreras anchas & vel in peditibus , \textbf{ eligere poterit vias campestres et amplas , } vel montanas , syluestres , \\\hline
3.3.12 & e de quales tierras son los meiores lidiadores \textbf{ e de quales artes son de escoger los mayores lidiadores . } Et avn declaramos en qual manera en la hueste son de establesçer guarniçiones e castiellos & et ex quibus regionibus sunt meliores pugnantes \textbf{ et est | quibus artibus sunt meliores bellicosi : } declarauimus \\\hline
3.3.12 & e de quales artes son de escoger los mayores lidiadores . \textbf{ Et avn declaramos en qual manera en la hueste son de establesçer guarniçiones e castiellos } e quales cosas son de penssar & quibus artibus sunt meliores bellicosi : \textbf{ declarauimus | etiam qualiter in exercitu construendae sunt munitiones et castra , } et quae sunt consideranda si debeat publica pugna committi , \\\hline
3.3.12 & Et avn declaramos en qual manera en la hueste son de establesçer guarniçiones e castiellos \textbf{ e quales cosas son de penssar } si se deue la batalla acometer publicamente . & etiam qualiter in exercitu construendae sunt munitiones et castra , \textbf{ et quae sunt consideranda si debeat publica pugna committi , } et quibus cautelis abundare decet bellorum ducem \\\hline
3.3.12 & e quales cosas son de penssar \textbf{ si se deue la batalla acometer publicamente . } Et quales cautelas ha de auer el señor de la batalla & etiam qualiter in exercitu construendae sunt munitiones et castra , \textbf{ et quae sunt consideranda si debeat publica pugna committi , } et quibus cautelis abundare decet bellorum ducem \\\hline
3.3.12 & si se deue la batalla acometer publicamente . \textbf{ Et quales cautelas ha de auer el señor de la batalla } por que la su hueste non sea dañada en el camino . & et quae sunt consideranda si debeat publica pugna committi , \textbf{ et quibus cautelis abundare decet bellorum ducem } ne suus exercitus laedatur in via quantum ad campestrum bellum . \\\hline
3.3.12 & Et este quanto a la batalla del canpo \textbf{ e segunt que paresçe non nos finca de dezir ninguna cosa en esta materia } si non que mostremos & ne suus exercitus laedatur in via quantum ad campestrum bellum . \textbf{ Nihil ( ut videtur ) | ulterius dicere restat , } nisi \\\hline
3.3.12 & si non que mostremos \textbf{ en commo se deuen ordenar las azes } e ferir los contrarios & nisi \textbf{ ut doceamus ordinare acies , } percutere aduersarios , et inuadere hostes . Prius \\\hline
3.3.12 & en commo se deuen ordenar las azes \textbf{ e ferir los contrarios } e acometer los enemigos . & ut doceamus ordinare acies , \textbf{ percutere aduersarios , et inuadere hostes . Prius } tamen dicemus de ordine acierum . \\\hline
3.3.12 & e ferir los contrarios \textbf{ e acometer los enemigos . } Enpero primero diremos del ordenamiento de las azes . & ut doceamus ordinare acies , \textbf{ percutere aduersarios , et inuadere hostes . Prius } tamen dicemus de ordine acierum . \\\hline
3.3.12 & e non tomaren espaçio conuenible \textbf{ non podran bien lidiar . } Ca si mucho estudieren apretados enbargan los vnos a los otros & et occupent debitum spatium , \textbf{ bene pugnare non poterunt . } Nam si nimis sunt constricti , \\\hline
3.3.12 & Ca si mucho estudieren apretados enbargan los vnos a los otros \textbf{ que non puedan ferir . } mas si fueren muy ralos & impediuntur \textbf{ ne alios percutere possint . } Si vero nimis rari et interlucentes , \\\hline
3.3.12 & para que mas ligeramente los venzcan . \textbf{ mas guardar orden conuenible en la az } e que los caualleros e los peones guarden su az & ut facilius deuincantur . \textbf{ Seruare autem debitum ordinem in acie } ut equites \\\hline
3.3.12 & e que los caualleros e los peones guarden su az \textbf{ non se puede fazer } sin grant vso de las armas . & et pedites suam aciem seruent , \textbf{ non sine magno exercitio fieri potest . } Qui igitur in tempore aliquo vult bellare , \\\hline
3.3.12 & Pues que assi es aquel que quiere lidiaren algun tienpo deue \textbf{ por luengos tienpos acostunbrar los lidiadores } a guardar orden conuenible en la az & Qui igitur in tempore aliquo vult bellare , \textbf{ per diuturna tempora debet exercitare pugnatores ad seruandum debitum ordinem , et } ad faciendum ea quae requiruntur in bello . Modus autem , \\\hline
3.3.12 & por luengos tienpos acostunbrar los lidiadores \textbf{ a guardar orden conuenible en la az } e a fazer aquellas cosas & Qui igitur in tempore aliquo vult bellare , \textbf{ per diuturna tempora debet exercitare pugnatores ad seruandum debitum ordinem , et } ad faciendum ea quae requiruntur in bello . Modus autem , \\\hline
3.3.12 & a guardar orden conuenible en la az \textbf{ e a fazer aquellas cosas } que son menester en la batalla . & per diuturna tempora debet exercitare pugnatores ad seruandum debitum ordinem , et \textbf{ ad faciendum ea quae requiruntur in bello . Modus autem , } per quem pugnatores \\\hline
3.3.12 & Mas la manera \textbf{ por que los lidiadores aprenden guardar esta orden . } es que muchas vezes tan bien los caualleros & per quem pugnatores \textbf{ huiusmodi ordinem seruare discunt , | est , } ut frequenter tam equites quam pedites ducantur ad campos . \\\hline
3.3.12 & va a los peones primeramente deuen los caualleros \textbf{ e los peones ordenar en linea derecha } e en orden derecha & primo debet equites \textbf{ et pedites linealiter disponere ita } ut seriatim maneant , \\\hline
3.3.12 & que demanda el az de los caualleros o de los peones . \textbf{ Et despues desto deue mandar } que se doble el az & et aequaliter a se inuicem distent \textbf{ secundum distantiam quam requirit acies equestris vel pedestris . Postea praecipere debet } ut duplicent aciem ita quod medietas aciei statim separet se a medietate alia , \\\hline
3.3.12 & Et esto fecho \textbf{ luego deue el cabdiello de la batalla mandar } que fagan az quadrada . & et seriatim ordinet se ante aliam vel post ipsam . \textbf{ Quo facto statim debet praecipere dux belli , } ut aciem quadratam faciant , \\\hline
3.3.12 & e desende que establezcan vn triangulo \textbf{ que quiere dezir forma de tres linnas } e esto se faz ligeramente . & ø \\\hline
3.3.12 & que es figura de tres liñas . \textbf{ Et si quisieremos fablar mas claramente } por que non toman todos los omnes estas maneras de geometria podemos dezir & et partibus quadratis coniunctis simul faciunt trigonum . Vel , \textbf{ ut sit ad unum dicere , } qui non omnes hos modos geometricos capiunt , \\\hline
3.3.12 & Et si quisieremos fablar mas claramente \textbf{ por que non toman todos los omnes estas maneras de geometria podemos dezir } que despues que los lidiadores fueren traydos al canpo & ut sit ad unum dicere , \textbf{ qui non omnes hos modos geometricos capiunt , } ductis pugnatoribus ad campos siue equitibus siue peditibus imperare debet dux belli , \\\hline
3.3.12 & si quier peones . \textbf{ deue mandar el cabdiello de la batalla } que se ordenen los lidiadores segunt forma quadrada . & quod pugnatores ordinent \textbf{ se } secundum formam quadrangularem , \\\hline
3.3.12 & Et despues que se ordenen segunt forma redonda e almogotes . \textbf{ Et assi de las otras maneras deuen costunbrar los lidiadores } por que sepan parar el az & secundum rotundam : \textbf{ et sic deinceps debet assuefacere bellantes , } ut sciant construere aciem \\\hline
3.3.12 & Et assi de las otras maneras deuen costunbrar los lidiadores \textbf{ por que sepan parar el az } segun qua si quier forma o figura & et sic deinceps debet assuefacere bellantes , \textbf{ ut sciant construere aciem } secundum quamcunque formam . His visis sciendum quadrangularem formam aciei \\\hline
3.3.12 & que mas les cunple . \textbf{ Et vistas estas cosas conuiene de saber que entre todas las otras formas de la az la quadrada es mas sin prouecho . } Et por ende nunca es de formar el az sinplemente & secundum quamcunque formam . His visis sciendum quadrangularem formam aciei \textbf{ inter caeteras formas esse magis inutilem : } ideo secundum hanc formam nunquam formanda est acies simpliciter , \\\hline
3.3.12 & Et vistas estas cosas conuiene de saber que entre todas las otras formas de la az la quadrada es mas sin prouecho . \textbf{ Et por ende nunca es de formar el az sinplemente } segunt esta forma sacado & inter caeteras formas esse magis inutilem : \textbf{ ideo secundum hanc formam nunquam formanda est acies simpliciter , } sed in casu : \\\hline
3.3.12 & assi commo si el assentamiento del logar demandasse \textbf{ tal forma en este caso es de formar el az en forma quadrada . } Mas las formas de las azes & ut si situs talem formam requireret , \textbf{ in huiusmodi casu construenda est forma praedicta . } Formae autem acierum \\\hline
3.3.12 & que son prouechosas \textbf{ para lidiar son estas . } La . piramidalEt la tiiaral . & Formae autem acierum \textbf{ secundum se utiles ad bellandum , } sunt pyramidalis , rotunda , \\\hline
3.3.12 & Et la redonda . \textbf{ Ca los lidiadores o solamente se quieren defender } e sofrir colpes & Nam pugnantes \textbf{ vel solum volunt se defendere et sustinere ictus , } vel volunt alios inuadere . \\\hline
3.3.12 & Ca los lidiadores o solamente se quieren defender \textbf{ e sofrir colpes } o quieren acometer los otros . & Nam pugnantes \textbf{ vel solum volunt se defendere et sustinere ictus , } vel volunt alios inuadere . \\\hline
3.3.12 & e sofrir colpes \textbf{ o quieren acometer los otros . } Por ende si los lidiadores non se sienten de tan grand poder & vel solum volunt se defendere et sustinere ictus , \textbf{ vel volunt alios inuadere . } Si ergo bellantes non sentiunt se tantae potentiae \\\hline
3.3.12 & Por ende si los lidiadores non se sienten de tan grand poder \textbf{ por que los otros puedan vençer . } Mas cunpleles que se puedan defender . & Si ergo bellantes non sentiunt se tantae potentiae \textbf{ ut alios debellare possint , } sed sufficit eis \\\hline
3.3.12 & por que los otros puedan vençer . \textbf{ Mas cunpleles que se puedan defender . } Estonçe es de establesçer el az segunt forma redonda . & ut alios debellare possint , \textbf{ sed sufficit eis | ut se defendant : } tunc est construenda acies \\\hline
3.3.12 & Mas cunpleles que se puedan defender . \textbf{ Estonçe es de establesçer el az segunt forma redonda . } Et los lidiadores deuen se costreñir e apretarse en ssi & ut se defendant : \textbf{ tunc est construenda acies | secundum rotundam formam ; } et pugnantes debent se magis constringere et constipare , \\\hline
3.3.12 & Estonçe es de establesçer el az segunt forma redonda . \textbf{ Et los lidiadores deuen se costreñir e apretarse en ssi } assi que el az non pueda ser ronpida de los enemigos . & secundum rotundam formam ; \textbf{ et pugnantes debent se magis constringere et constipare , } ut acies non possit ab hostibus transcindi . Circa aciem autem in summitate , \\\hline
3.3.12 & contra los enemigos \textbf{ son de poner los meior armados } e mas prouados en las armas & et in exteriori parte constituendi sunt homines grauioris armaturae \textbf{ et melius armati , } qui absque minori grauamine possint ictus suscipere . \\\hline
3.3.12 & e mas prouados en las armas \textbf{ los quales puedan sofrir meior los colpes } e con menor agrauiamiento e con menor daño . & et melius armati , \textbf{ qui absque minori grauamine possint ictus suscipere . } Si vero pugnantes credunt se esse tantae potentiae , \\\hline
3.3.12 & Mas si los lidiadores cuydan ser de tanto poder \textbf{ que puedan acometer los enemigos estonçe } segunt el cuento dellos & Si vero pugnantes credunt se esse tantae potentiae , \textbf{ ut possint aduersarios inuadere : } tunc \\\hline
3.3.12 & o los enemigos son muchos o pocos \textbf{ si los enemigos son muy pocos es de establesçer el az segunt forma de tigeras } assi que el az este abierta & vel multi . \textbf{ Si hostes sunt valde pauci , construenda est acies | secundum formam forficularem , } ut acies sit aperta ad modum ferri equi \\\hline
3.3.12 & e los ençierren dentro . \textbf{ Mas si los enemigos son muchos es de establesçer el az segunt forma } que llaman cuño o segunt forma de pera e aguda . & quasi in medio capiat et concludat . \textbf{ Si vero hostes sunt multi , construenda est ipsa acies } secundum formam \\\hline
3.3.12 & que llaman cuño o segunt forma de pera e aguda . \textbf{ por que puedan fender e departir los enemigos mas ligeramente se vençen . } Et pues que ssi es el az establesçida & quam appellant conum , \textbf{ id est } secundum figuram pyramidalem et acutam , \\\hline
3.3.12 & en forma redonda es prouechosa \textbf{ para sofrir colpes . } Mas en forma de tigeras es prouechosa & secundum figuram pyramidalem et acutam , \textbf{ ut possit hostes scindere et diuidere . } Nam diuisis hostibus facilius debellantur . Acies ergo constructa in forma rotunda , \\\hline
3.3.12 & Mas en forma de tigeras es prouechosa \textbf{ para çercar } e ençerrar los enemigos & Nam diuisis hostibus facilius debellantur . Acies ergo constructa in forma rotunda , \textbf{ utilis est ad sustinendum . In forma vero forficulari , est utilis ad circum dandum et concludendum , } cum hostes sunt pauci . Sed in forma acuta \\\hline
3.3.12 & para çercar \textbf{ e ençerrar los enemigos } quando son pocos . & utilis est ad sustinendum . In forma vero forficulari , est utilis ad circum dandum et concludendum , \textbf{ cum hostes sunt pauci . Sed in forma acuta } et pyramidali , \\\hline
3.3.12 & Mas la forma aguda en manera de pera ens prouechosa \textbf{ para fender e departir los enemigos } quando son muchos . & et pyramidali , \textbf{ utilis est ad scindendum et diuidendum , } cum hostes sunt plures . \\\hline
3.3.12 & Et pues que assi es las maneras de las azes \textbf{ son de establescer } segunt la muchedunbre de los lidiadores . & Sciendum est ergo , \textbf{ quod numerus acierum constituendus est } secundum multitudinem pugnatorum : \\\hline
3.3.12 & que tiene muchos o pocos lidiadores \textbf{ assi puede establesçer muchas o pocas azes . Otrossi conuiene de saber } que sienpre ençima del as & aut pauciores pugnantes viderit se habere , poterit plures \textbf{ aut pauciores acies construere . | Sciendum etiam , } quod semper in cornu aciei \\\hline
3.3.12 & e en los logares \textbf{ do puede ser mayor periglo son de poner los meiores lidiadores } que meior puedan lidiar & et in locis \textbf{ ubi maius periculum est , | ne acies confundatur , apponendi sunt probiores pugnatores , } qui possint virilius dimicare . Est \\\hline
3.3.12 & do puede ser mayor periglo son de poner los meiores lidiadores \textbf{ que meior puedan lidiar } por que non pueda ser cofondida el az . & ne acies confundatur , apponendi sunt probiores pugnatores , \textbf{ qui possint virilius dimicare . Est } etiam aduertendum , \\\hline
3.3.12 & por que non pueda ser cofondida el az . \textbf{ Otrosi conuiene de tener mientes } que en cada vna de las azes & qui possint virilius dimicare . Est \textbf{ etiam aduertendum , } quod in qualibet acie praeter numerum pugnatorum constituentium aciem , reseruandi sunt aliqui strenui bellatores extra ipsam aciem qui possint \\\hline
3.3.12 & sin el cuento de los lidiadores \textbf{ que fazen el az son de guardar algunos buenos et fuertes lidiadores fuera del az } que puedan acorrer a aquella parte & etiam aduertendum , \textbf{ quod in qualibet acie praeter numerum pugnatorum constituentium aciem , reseruandi sunt aliqui strenui bellatores extra ipsam aciem qui possint } ad illam partem succurrere \\\hline
3.3.12 & que fazen el az son de guardar algunos buenos et fuertes lidiadores fuera del az \textbf{ que puedan acorrer a aquella parte } do vieren & quod in qualibet acie praeter numerum pugnatorum constituentium aciem , reseruandi sunt aliqui strenui bellatores extra ipsam aciem qui possint \textbf{ ad illam partem succurrere } ubi viderit magis aciem deficere . \\\hline
3.3.12 & que mas fallesçe el az . \textbf{ Et por ende estas tres cosas son de guardar en el ordenamiento de las azes . } Lo primero que el az sea bien ordenada & ubi viderit magis aciem deficere . \textbf{ Haec igitur tria obseruanda sunt in constitutione acierum . Primo , } ut acies bene ordinetur \\\hline
3.3.12 & Lo segundo que los mas fuertes lidiadores sean puestos en aquellas partes de la az \textbf{ en las quales mas ayna se puede ronper } e foradar el az . & ut probiores bellatores in illis partibus aciei apponantur , \textbf{ in quibus magis potest confundi et perforari acies . } Tertio , \\\hline
3.3.12 & en las quales mas ayna se puede ronper \textbf{ e foradar el az . } lo tercero que fuera de cada vna de las azes sean guardados algunos estremados caualleros e osados & in quibus magis potest confundi et perforari acies . \textbf{ Tertio , } ut extra quamlibet aciem reseruentur aliqui milites strenui \\\hline
3.3.12 & lo tercero que fuera de cada vna de las azes sean guardados algunos estremados caualleros e osados \textbf{ que puedan acorrer a aquella parte } que vieren & et audaces , \textbf{ qui possint succurrere ad partem illam , } erga quam viderint aciem titubare , et deficere . \\\hline
3.3.12 & que vieren \textbf{ que mas faz meester e mas ayna puede fallesçer . } m mostrado en qual manera son de establesçer & qui possint succurrere ad partem illam , \textbf{ erga quam viderint aciem titubare , et deficere . } Ostenso qualiter sunt acies ordinandae et construendae , \\\hline
3.3.13 & que mas faz meester e mas ayna puede fallesçer . \textbf{ m mostrado en qual manera son de establesçer } e de ordenar las azes & erga quam viderint aciem titubare , et deficere . \textbf{ Ostenso qualiter sunt acies ordinandae et construendae , } reliquum est ostendere , \\\hline
3.3.13 & m mostrado en qual manera son de establesçer \textbf{ e de ordenar las azes } fincanos & erga quam viderint aciem titubare , et deficere . \textbf{ Ostenso qualiter sunt acies ordinandae et construendae , } reliquum est ostendere , \\\hline
3.3.13 & fincanos \textbf{ de mostrar en qual manera los lidiadores deuen ferir } e si es meior de ferir cortando o ferir de punta o estocando . & reliquum est ostendere , \textbf{ qualiter pugnantes percutere debeant , } utrum eligibilius est percutere caesim \\\hline
3.3.13 & de mostrar en qual manera los lidiadores deuen ferir \textbf{ e si es meior de ferir cortando o ferir de punta o estocando . } Mas podemos mostrar por çinco razones & qualiter pugnantes percutere debeant , \textbf{ utrum eligibilius est percutere caesim | vel punctim . } Possumus autem quinque viis ostendere , \\\hline
3.3.13 & e si es meior de ferir cortando o ferir de punta o estocando . \textbf{ Mas podemos mostrar por çinco razones } que son de estrannar e de escarnesçer & vel punctim . \textbf{ Possumus autem quinque viis ostendere , } quod deridendi sunt percutientes caesim , \\\hline
3.3.13 & Mas podemos mostrar por çinco razones \textbf{ que son de estrannar e de escarnesçer } los que fieren cortando . & Possumus autem quinque viis ostendere , \textbf{ quod deridendi sunt percutientes caesim , } et eligibilius est percutere punctim . \\\hline
3.3.13 & los que fieren cortando . \textbf{ Et que mas de escoger es ferir de punta . } La primera razon se toma del defendimiento de las armas . & quod deridendi sunt percutientes caesim , \textbf{ et eligibilius est percutere punctim . } Prima sumitur ex prohibitione armorum . \\\hline
3.3.13 & Et dende viene que los que son prouados en las batallas . \textbf{ dizen que los lidiadores sienpre deuen auer las lorigas anchas . } assi que les aniellos de las lorigas se ayunten & Inde est \textbf{ quod bellorum experti dicunt pugnantes semper debere habere loricas amplas ita , } ut annuli loricarum se constringant : \\\hline
3.3.13 & dizen que los lidiadores sienpre deuen auer las lorigas anchas . \textbf{ assi que les aniellos de las lorigas se ayunten } e non esten estendidas las lorigas . & quod bellorum experti dicunt pugnantes semper debere habere loricas amplas ita , \textbf{ ut annuli loricarum se constringant : } quia quanto illi annuli magis sunt compacti , \\\hline
3.3.13 & por que quanto aquellos aniellos mas son ayuntados . \textbf{ tanto conuiene de cortar mas dellos } para que los colpes enpeescan . Bien assi los que fieren taiando conuiene & quia quanto illi annuli magis sunt compacti , \textbf{ tanto oportet plures ex eis frangere } ut vulnera noceant . \\\hline
3.3.13 & para que el colpe venga mas ayna a la carne . \textbf{ Et por ende mas de escoger es ferir de punta } que ferir taiando . & quam percutientes punctim ; \textbf{ ut ergo vulnus perueniat citius ad carnem , magis est eligibile percutere punctim , } quam caesim . Modica autem armorum incisio sufficit ad laedendum carnem percutiendo punctim , \\\hline
3.3.13 & Et por ende mas de escoger es ferir de punta \textbf{ que ferir taiando . } Ca el colpe mas ayna viene a la carne & ø \\\hline
3.3.13 & por que pequeno cortamiento de las armas abasta \textbf{ para ferir en la carne feriendo de punta } el qual non abastarie & ut ergo vulnus perueniat citius ad carnem , magis est eligibile percutere punctim , \textbf{ quam caesim . Modica autem armorum incisio sufficit ad laedendum carnem percutiendo punctim , } quae non sufficeret \\\hline
3.3.13 & La segunda razon \textbf{ para puar esto se toma del defendimiento de los huessos } ca si alguno avn que estudiesse desarmado en la ferida & Secunda via ad inuestigandum hoc idem , \textbf{ sumitur ex resistentia ossium . } Nam et si quis quasi inermis existeret , \\\hline
3.3.13 & o a los mienbros de vida \textbf{ conuernie de fazer muy grant llaga } e de cortar muchos huessos . & ad cor vel ad membra vitalia , \textbf{ oporteret magnam plagam facere } et multa ossa incidere : \\\hline
3.3.13 & conuernie de fazer muy grant llaga \textbf{ e de cortar muchos huessos . } Mas feriendo de punta pequeno colpe mata al omen . & oporteret magnam plagam facere \textbf{ et multa ossa incidere : } sed percutiendo punctim duae unciae sufficiunt \\\hline
3.3.13 & para que se fagan llaga mortal \textbf{ mas deuemos penssar } que qual si quier cosa & ad hoc ut fiat plaga mortalis , \textbf{ et sit lethale vulnus . Considerare quidem debemus , } quod quicquid est hosti nociuum , \\\hline
3.3.13 & quanto ella es tales a nos prouechosa . Et por ende en la hueste \textbf{ do queremos matar los enemigos } meior es ferir de punta . & ideo in exercitu \textbf{ ubi quaeritur mors aduersariorum , percutiendum est punctim , } quia sic feriendo citius infligitur plaga mortifera . \\\hline
3.3.13 & do queremos matar los enemigos \textbf{ meior es ferir de punta . } por que feriendo & ideo in exercitu \textbf{ ubi quaeritur mors aduersariorum , percutiendum est punctim , } quia sic feriendo citius infligitur plaga mortifera . \\\hline
3.3.13 & ca quanto el enemigo mas se prouee de los colpes \textbf{ que ha de resçebir . } mas se puede cobrir & Nam quanto hostis magis vulnera prouidet , \textbf{ magis potest se protegere , } et citius potest illa vitare : \\\hline
3.3.13 & que ha de resçebir . \textbf{ mas se puede cobrir } e mas ayna puede escular aquellos colpes . & Nam quanto hostis magis vulnera prouidet , \textbf{ magis potest se protegere , } et citius potest illa vitare : \\\hline
3.3.13 & Mas en feriendo cortando . \textbf{ por que conuiene de fazer grand mouimiento de los braços } ante que se de el colpe & In percutiendo autem caesim , \textbf{ quia oportet fieri magnum brachiorum motum prius quam infligatur plaga , aduersarius ex longinquo potest prouidere vulnus , } ideo magis sibi cauere potest \\\hline
3.3.13 & o el contrario \textbf{ de lueñe se puede guardar } que nol faga llaga . & quia oportet fieri magnum brachiorum motum prius quam infligatur plaga , aduersarius ex longinquo potest prouidere vulnus , \textbf{ ideo magis sibi cauere potest } et cooperire se ictibus . Ideo ait Vegetius , \\\hline
3.3.13 & que nol faga llaga . \textbf{ Et por ende puede se mas guardar e encobrirse de aquellos colpes . } Et por esso dize vegeçio & ideo magis sibi cauere potest \textbf{ et cooperire se ictibus . Ideo ait Vegetius , } quod punctim percutere aduersarium sauciat \\\hline
3.3.13 & Et por esso dize vegeçio \textbf{ que ferir de punta a su contrario } o a su enemigo mata & et cooperire se ictibus . Ideo ait Vegetius , \textbf{ quod punctim percutere aduersarium sauciat } antequam videat . \\\hline
3.3.13 & ante que lo vea . \textbf{ Et por esso los romanos vsaron prinçipalmente de esta materia de ferir . } Ca los romanos escarnesçien de todos los caualleros & antequam videat . \textbf{ Unde hoc genere percutiendi potissime usi sunt Romani . Deridebant enim Romani milites , omnes percutientes caesim , } quia \\\hline
3.3.13 & que ferien cortando \textbf{ por que ellos que rien sienpre ferir de punta . } La quarta razon se toma del canssamiento de los mienbros & quia \textbf{ et ipsi semper volebant percutere punctim . } Quarta via sumitur \\\hline
3.3.13 & por que entre todas las otras cosas \textbf{ que son las batalla mayormente es de penssar esto } que los lidiadores sin grand canssamiento de sus mienbros puedan ferir mucho a sus enemigos e a sus contrarios . & ex fatigatione membrorum . \textbf{ Inter cetera enim in bellis est hoc potissime attendendum : } ut pugnantes absque nimia fatigatione sui possint nimis aduersarios laedere . \\\hline
3.3.13 & que son las batalla mayormente es de penssar esto \textbf{ que los lidiadores sin grand canssamiento de sus mienbros puedan ferir mucho a sus enemigos e a sus contrarios . } Ca si los lidiadores canssaren mucho de guisa & Inter cetera enim in bellis est hoc potissime attendendum : \textbf{ ut pugnantes absque nimia fatigatione sui possint nimis aduersarios laedere . } Nam si bellantes nimis se fatigent , \\\hline
3.3.13 & Ca si los lidiadores canssaren mucho de guisa \textbf{ que non puedan sofrir aquel trabaio } de ligero dexaran el az & Nam si bellantes nimis se fatigent , \textbf{ non valentes laborem illum tolerare , } de facili dimittunt aciem , \\\hline
3.3.13 & de ligero dexaran el az \textbf{ e tornarsse han a foyr . } Por la qual cosa commo feriendo cortando & de facili dimittunt aciem , \textbf{ et conuertuntur in fugam . } Quare cum percutiendo caesim propter magnum motum brachiorum insurgat ibi magnus labor , \\\hline
3.3.13 & por el grand mouimiento de los braços leuantasse ende grant trabaio . \textbf{ Mas feriendo de punta el canssamiento es muy pequeno . Por ende es meior de ferir de punta que cortando } por que la ferida de taio & Quare cum percutiendo caesim propter magnum motum brachiorum insurgat ibi magnus labor , \textbf{ punctim uero feriendo modica fatigatio sufficiat , elegibilius est percutere punctim , } quam caesim . Caesa enim percussio \\\hline
3.3.13 & Ca el buen lidiador \textbf{ si puede deue ferir assu enemigo } en tal manera que sea sin daño dessi . & Nam bonus bellator , \textbf{ si potest , | sic debet aduersarium laedere , } ut tamen ipse non laedatur . \\\hline
3.3.13 & en tal manera que sea sin daño dessi . \textbf{ Et pues que assi es toda aquella manera de ferir } es & ut tamen ipse non laedatur . \textbf{ Omnis ergo ille modus percutiendi est magis eligendus , } secundum quem seriens minus discooperitur et detegitur ; \\\hline
3.3.13 & es \textbf{ mas de escoger } segunt la qual el que fiere se descubre menos . & Omnis ergo ille modus percutiendi est magis eligendus , \textbf{ secundum quem seriens minus discooperitur et detegitur ; } quia sic feriendo , \\\hline
3.3.13 & segunt la qual el que fiere se descubre menos . \textbf{ por que assi feriendo menor daño le puede contesçer . } Por la qual cosa commo feriendo de punta & secundum quem seriens minus discooperitur et detegitur ; \textbf{ quia sic feriendo , | minor laesio ei potest accidere . } Quare cum percutiendo punctim \\\hline
3.3.13 & avn que este el cuerpo cubierto \textbf{ puede resçebir grand daño el enemigo . } por ende es meior ferir de punta & Quare cum percutiendo punctim \textbf{ etiam tecto corpore possit nimis aduersarius laedi , } melius est percutere punctim \\\hline
3.3.13 & puede resçebir grand daño el enemigo . \textbf{ por ende es meior ferir de punta } que taiando . & etiam tecto corpore possit nimis aduersarius laedi , \textbf{ melius est percutere punctim } quam caesim . Percutiendo enim caesim oportet eleuare brachium dextrum : \\\hline
3.3.13 & que taiando . \textbf{ por que firiendo taiando conuiene de leuantar el braço derecho e diestro . } Et leuantando el braço derecho paresçe descubierto el costado derecho & melius est percutere punctim \textbf{ quam caesim . Percutiendo enim caesim oportet eleuare brachium dextrum : } quo eleuato dextrum latus nudatur \\\hline
3.3.13 & Et da manera al enemigo \textbf{ por quel pueda mas ligeramente ferir . } Ca mas ligeramente faze daño & et datur hosti via , \textbf{ ut possit nos laedere . } Nam leuius infertur laesio et nocumentum corpore nudato , quam tecto . \\\hline
3.3.14 & que fazen los enemigos ser fuertes \textbf{ para lidiar con sus enemigos } por que aquellas cosa son a ellos prouechosas . & et econuerso . Quaecunque igitur reddunt hostes fortiores \textbf{ ad resistendum bellantibus , } quia illa sunt eis proficua , \\\hline
3.3.14 & e los fazen ser mas flacos \textbf{ por que non puedan lidiar contra sus enemigos . } Mas quanto partenesçe a lo presente podemos contar siete cosas . & et reddunt eos debiliores \textbf{ ne possint impugnantibus resistere . } Quantum autem ad praesens spectat , \\\hline
3.3.14 & por que non puedan lidiar contra sus enemigos . \textbf{ Mas quanto partenesçe a lo presente podemos contar siete cosas . } por las quales los enemigos son mas fuertes contra sus enemigos . & ne possint impugnantibus resistere . \textbf{ Quantum autem ad praesens spectat , | possumus septem enumerare , } per quae hostes sunt fortiores contra impugnantes . \\\hline
3.3.14 & en el az commo deuen \textbf{ si los acometieren sus enemigos mas fuertes seran de vençerLo . } segundo que faze los enemigos mas fuertes & si inuadantur , \textbf{ difficilius euincuntur . } Secundum quod reddit hostes fortiores ad resistendum , \\\hline
3.3.14 & segundo que faze los enemigos mas fuertes \textbf{ para lidiar es el logar } do se assientan . & difficilius euincuntur . \textbf{ Secundum quod reddit hostes fortiores ad resistendum , } est locus . \\\hline
3.3.14 & do se assientan . \textbf{ Ca en vn logar los enemigos se pueden mas ligeramente defender } que en otro . & est locus . \textbf{ Nam in uno loco hostes facilius se tuentur , } quam in alio . \\\hline
3.3.14 & si contesçiere que los enemigos sean tomados en tal logar \textbf{ menos se pueden defender } e con mayor trabaio . & si contingat hostes in tali situ reperiri , \textbf{ difficilius se defendere poterunt : } quia oportet eos sparsim incedere . Quare sicut locus ineptus defensioni , \\\hline
3.3.14 & si alli fueren fallados e tomados los enemigos \textbf{ faze los ser mas flacos para lidiar . } Assi el logar conuenible & si in eo hostes inueniantur , \textbf{ reddit eos debiliores ad bellandum : } sic locus aptus facit eos potentiores ad resistendum . Tertium , \\\hline
3.3.14 & e bueno fazelos mas fuertes \textbf{ para se defender . } Lo terçero es el tienpo can en el tienpo & sic locus aptus facit eos potentiores ad resistendum . Tertium , \textbf{ est ipsum tempus . } Nam tempore in quo ventus est contra hostes , \\\hline
3.3.14 & e los rayos del sol les da en los oios . \textbf{ con mayor trabaio se pueden defender de sus enemigos . } Mas en el tienpo & et in quo solares radii opponuntur eorum oculis , \textbf{ difficilius possunt hostes resistere : } tempore vero in quo haec modo opposito se habent , \\\hline
3.3.14 & en que el viento e el poluo e el sol non les es contrario \textbf{ son mas apareiados para lidiar . } Lo quarto que fazen los enemigos mas esforçados e mas aperaiados & tempore vero in quo haec modo opposito se habent , \textbf{ hostes habiliores sunt ad pugnandum . Quartum quod reddit hostes magis animosos } et magis promptos ad renitendum , est praeuisio . Quia quanto praeuisi sunt \\\hline
3.3.14 & Lo quarto que fazen los enemigos mas esforçados e mas aperaiados \textbf{ e para lidiar es prouision . } Ca quando son proiusos e aperçebidos & hostes habiliores sunt ad pugnandum . Quartum quod reddit hostes magis animosos \textbf{ et magis promptos ad renitendum , est praeuisio . Quia quanto praeuisi sunt } et praesciunt pugnatorum aduentum , \\\hline
3.3.14 & Ca quando son proiusos e aperçebidos \textbf{ e saben que los enemigos han de venir contra ellos } meioͬ se guarnesçen & et magis promptos ad renitendum , est praeuisio . Quia quanto praeuisi sunt \textbf{ et praesciunt pugnatorum aduentum , } magis se muniunt , \\\hline
3.3.14 & Lo vj° es amistança e concordia entre ralmente \textbf{ mas de ligero se pueden vençer } e mucho mas si son departidos en los coraçones e en las uoluntades mas ayna seran vençidos & et concordia ipsorum . \textbf{ Nam si hostes diuisi corporaliter | facilius deuincuntur , } multo magis diuisi animo et voluntate debellantur celerius , \\\hline
3.3.14 & ma estan ayuntados corporalmente \textbf{ son mas poderosos para lidiar } e mucho mas si se aman & si hostes non sunt sparsi , sed sunt corporaliter coniuncti , \textbf{ potentiores sunt ad bellandum , } et multo magis si se diligunt \\\hline
3.3.14 & e mas prouechosos \textbf{ para se ayudar Lo vij . } que faze los enemigos mas poderosos & amor et unitas cordium eos viriliores reddit . Septimum , \textbf{ quod facit hostes potentiores ad renitendum , } est latentia propriarum conditionum existentium circa ipsos . \\\hline
3.3.14 & que faze los enemigos mas poderosos \textbf{ para se defender } es ascondimiento de las condiçiones propreas & quod facit hostes potentiores ad renitendum , \textbf{ est latentia propriarum conditionum existentium circa ipsos . } Nam \\\hline
3.3.14 & tanto meior se escoge el camino \textbf{ en qual manera los deuen acometer } Mas quanto los sus negoçios & tanto facilior eligitur via , \textbf{ qualiter debeant impugnari : } quanto vero eorum negocia sunt magis latentia , \\\hline
3.3.14 & e los sus fechos son mas ascondidos \textbf{ menos es sabida la manera de los acometer . } Et por ende contadas aquellas cosas & quanto vero eorum negocia sunt magis latentia , \textbf{ magis ignoratur impugnationis modus . } Enumeratis itaque quae reddunt hostes potentiores ad renitendum , \\\hline
3.3.14 & que fazen los enemigos mas poderosos \textbf{ para se defender } de ligero puede paresçer commo & ø \\\hline
3.3.14 & para se defender \textbf{ de ligero puede paresçer commo } e en qual manera los lidiadores deuen acometer sus enemigos . & Enumeratis itaque quae reddunt hostes potentiores ad renitendum , \textbf{ de facili patere potest , } quomodo et qualiter bellantes suos hostes inuadere debeant . \\\hline
3.3.14 & de ligero puede paresçer commo \textbf{ e en qual manera los lidiadores deuen acometer sus enemigos . } Ca commo en las siete maneras contadas sean los enemigos mas fuertes & de facili patere potest , \textbf{ quomodo et qualiter bellantes suos hostes inuadere debeant . } Nam cum septem modis enumeratis hostes fortiores existant ; \\\hline
3.3.14 & quando son las maneras contrarias \textbf{ de aquellas siete son de acometer e de ferir . } Et pues que assi es lo primero el señor de la batalla & cum modo opposito se habent , \textbf{ sunt inuadendi et debellandi . } Primo igitur dux belli per insidias vel propter aliquem alium modum , \\\hline
3.3.14 & e por çeladas \textbf{ o por alguna otra manera deue tener mientes con grant acuçia } quando los enemigos estan desparzidos & Primo igitur dux belli per insidias vel propter aliquem alium modum , \textbf{ debet diligenter aduertere , } quando hostes sunt dispersi : \\\hline
3.3.14 & quando los enemigos estan desparzidos \textbf{ e estonçe los deuen acometer } por que non han poder de se defender . & quando hostes sunt dispersi : \textbf{ et tunc debet eos inuadere , } quia non habebunt potentiam resistendi . Secundo debet diligenter explorare eorum itinera , \\\hline
3.3.14 & e estonçe los deuen acometer \textbf{ por que non han poder de se defender . } Lo segundo deue escudriñar con grand acuçia los caminos dellos & et tunc debet eos inuadere , \textbf{ quia non habebunt potentiam resistendi . Secundo debet diligenter explorare eorum itinera , } ut ad transitus fluuiorum , \\\hline
3.3.14 & por que non han poder de se defender . \textbf{ Lo segundo deue escudriñar con grand acuçia los caminos dellos } assi commo el passo de los rios & et tunc debet eos inuadere , \textbf{ quia non habebunt potentiam resistendi . Secundo debet diligenter explorare eorum itinera , } ut ad transitus fluuiorum , \\\hline
3.3.14 & e los uarrancos de los montes \textbf{ en que pueden caer } e las angosturas de las montañas & ø \\\hline
3.3.14 & mas ligeramente seran vençidos . \textbf{ Lo terçero deue catar el cabdiello al tienpo . } assi que quando el sol fiere en los oios de los enemigos & quia sic facilius deuincentur . \textbf{ Tertio debet aspicere ad ipsum tempus : } quando sol reuerberat ad oculos hostium , \\\hline
3.3.14 & e el poluo \textbf{ e el viento les diere en las caras estonçe los deuen acometer } por que la vista de los sus oios derramada e dañada & puluis et ventus repercutiunt ad eorum vultus : \textbf{ tunc debet eos inuadere , } quia oculis eorum disgregatis a sole , \\\hline
3.3.14 & e por el viento \textbf{ e por el poluo non pueden ver bien } en qual manera deuen lidiar . & et offensis per ventum \textbf{ et puluerem , | non bene videre poterunt , } qualiter debeant dimicare : \\\hline
3.3.14 & e por el poluo non pueden ver bien \textbf{ en qual manera deuen lidiar . } Por la qual cosa les conuiene de foyr . & non bene videre poterunt , \textbf{ qualiter debeant dimicare : } propter quod oportebit eos fugam eligere . \\\hline
3.3.14 & en qual manera deuen lidiar . \textbf{ Por la qual cosa les conuiene de foyr . } Lo quarto el señor de la hueste se deue tenprar & qualiter debeant dimicare : \textbf{ propter quod oportebit eos fugam eligere . } Quarto dux exercitus sic se temperare debet : \\\hline
3.3.14 & Por la qual cosa les conuiene de foyr . \textbf{ Lo quarto el señor de la hueste se deue tenprar } assi que en tal ora faga tomar la vianda a los caualleros & propter quod oportebit eos fugam eligere . \textbf{ Quarto dux exercitus sic se temperare debet : } ut tali hora faciat suos commilitones cibum capere , \\\hline
3.3.14 & Lo quarto el señor de la hueste se deue tenprar \textbf{ assi que en tal ora faga tomar la vianda a los caualleros } e folgar e dar çeuada a los cauallos & Quarto dux exercitus sic se temperare debet : \textbf{ ut tali hora faciat suos commilitones cibum capere , } et requiescere : \\\hline
3.3.14 & assi que en tal ora faga tomar la vianda a los caualleros \textbf{ e folgar e dar çeuada a los cauallos } por que pueda a desora acometer a los enemigos & ut tali hora faciat suos commilitones cibum capere , \textbf{ et requiescere : | et eorum equos pausare , } ut possint inuadere hostes ex improuiso ; \\\hline
3.3.14 & e folgar e dar çeuada a los cauallos \textbf{ por que pueda a desora acometer a los enemigos } assi que los acometa & et eorum equos pausare , \textbf{ ut possint inuadere hostes ex improuiso ; } ut eos inuadant quando cibum capiunt , \\\hline
3.3.14 & ally den ellos sin sospecha . \textbf{ Lo quinto deue el cabdiello escodriñar con grant acuçia } quando los enemigos fizieren grant iornada & non suspicantes eorum aduentum . \textbf{ Quinto debet diligenter explorare , } quando hostes magnam fecerunt dietam , sunt fatigati habent laxatos equos : \\\hline
3.3.14 & e touieren los cauallos canssados . \textbf{ Ca estonçe si los quisieren acometer } de ligero les faran boluer las espaldas a foyr . & quando hostes magnam fecerunt dietam , sunt fatigati habent laxatos equos : \textbf{ tunc enim , | si eos inuadere poterint , } de facili terga vertent . Sexto ( secundum Vegetium ) \\\hline
3.3.14 & Ca estonçe si los quisieren acometer \textbf{ de ligero les faran boluer las espaldas a foyr . } Lo . vj° . & si eos inuadere poterint , \textbf{ de facili terga vertent . Sexto ( secundum Vegetium ) } debet dux belli \\\hline
3.3.14 & Lo . vj° . \textbf{ segunt dize vegecio el señor de la batalla deue poner } por sio & de facili terga vertent . Sexto ( secundum Vegetium ) \textbf{ debet dux belli } inter suos hostes et inimicos , \\\hline
3.3.14 & por otros discordias entre los enemigos \textbf{ e boluer contiendas e lides o enemistades entre ellos } assi que non fien dessi mismos . & inter suos hostes et inimicos , \textbf{ vel per se , vel per alios mittere dissensiones , iurgia , commouere eos ad lites , | vel ad inimicitiam ; } ut de se inuicem non confidant . Hoc enim facto , \\\hline
3.3.14 & assi que non fien dessi mismos . \textbf{ Et esto fecho si los acometieren non auiendo fiuza enssi mismos de ligero se a foyr . } Mas esta cautela & si eos inuadat , \textbf{ non habentes fiduciam de se inuicem , | de facili conuertentur in fugam . } Sed haec cautela licet ponat eam Vegetius , \\\hline
3.3.14 & commo quier que la ponga vegeçio \textbf{ non es mucho de preçiar } por que es contraria a buenas costunbres . & Sed haec cautela licet ponat eam Vegetius , \textbf{ non multum est appretianda : } quia repugnaret bonis moribus . Septimo debet diligenter explorare conditiones hostium : \\\hline
3.3.14 & por que es contraria a buenas costunbres . \textbf{ lo . vij° . deue el cabdiello con grant acuçia escudriñar las condiçiones de los sus enemigos } en qual manera andan & non multum est appretianda : \textbf{ quia repugnaret bonis moribus . Septimo debet diligenter explorare conditiones hostium : } qualiter se gerant , \\\hline
3.3.14 & mas ligeramente sera fallada carrera e camino \textbf{ en qual manera puedan acometer los enemigos e vençerlos . } d dixiemos en vn capitulo sobredicho & de quo dux ille magis confidit , \textbf{ quos mores , habeat . Nam exploratis conditionibus singulis facilius inuenitur via , qualiter possit hostes inuadere , et debellare . } Diximus in quodam capitulo praecedenti , \\\hline
3.3.15 & d dixiemos en vn capitulo sobredicho \textbf{ que meior es de ferir de punta } que non taiando & Diximus in quodam capitulo praecedenti , \textbf{ percutiendum esse punctim non caesim : } in quo docuimus milites , \\\hline
3.3.15 & que non taiando \textbf{ en el qual mostramos a los caualleros e a los peones acometer la batalla . } Mas en este capitulo queremos dar espeçial enssenança a los peones & percutiendum esse punctim non caesim : \textbf{ in quo docuimus milites , | et etiam pedites . } In hoc autem capitulo specialem doctrinam volumus dare peditibus , \\\hline
3.3.15 & en el qual mostramos a los caualleros e a los peones acometer la batalla . \textbf{ Mas en este capitulo queremos dar espeçial enssenança a los peones } en qual manera deuen estar & et etiam pedites . \textbf{ In hoc autem capitulo specialem doctrinam volumus dare peditibus , } qualiter debeant stare cum volunt hostes percutere . \\\hline
3.3.15 & en qual manera deuen estar \textbf{ quando quisieren ferir los enemigos . } Et ay dos maneras de ferir los enemigos . & In hoc autem capitulo specialem doctrinam volumus dare peditibus , \textbf{ qualiter debeant stare cum volunt hostes percutere . } Percussionis autem hostium duplex est modus . \\\hline
3.3.15 & quando quisieren ferir los enemigos . \textbf{ Et ay dos maneras de ferir los enemigos . } La vna es de lueñe & qualiter debeant stare cum volunt hostes percutere . \textbf{ Percussionis autem hostium duplex est modus . } Unus a remotis , \\\hline
3.3.15 & Et en otra manera quando se fieren de çerca . \textbf{ Ca lançando dardos de lueñe deuen tener los pies esquierdos delante } e los derechos detras & et aliter cum ex propinquo se feriunt . \textbf{ Nam iaciendo iacula a remotis , debent habere ipsos pedes sinistros ante , } et dextros retro . \\\hline
3.3.15 & prinçipalmente enbia su uirtud a la parte derecha . \textbf{ Assi que la parte derecha en las animalias es mas fuerte en mouer } e mas apareiada a mouimiento . & quod est in animali principium motus , principalius influit in partem dextram ita , \textbf{ quod pars dextra in animalibus fortior est in mouendo , } et aptior ad motum : \\\hline
3.3.15 & Et por que la cosa que se mueue \textbf{ sienpre se afirma sobre alger cosa } que non se mueue . & et aptior ad motum : \textbf{ et quia mobile semper innititur alicui immobili , } ut si mouetur manus , innititur brachio stanti , \\\hline
3.3.15 & estonçe esta el omne bien apareiado \textbf{ para lançar dardos o piedras o otras cosas } que se pueden enbiar . & et latus dextrum elongatur , \textbf{ optime disponitur homo ad iaciendum iacula } et missilia : \\\hline
3.3.15 & para lançar dardos o piedras o otras cosas \textbf{ que se pueden enbiar . } Ca estonçe fuelga el omne en la parte esquierda & optime disponitur homo ad iaciendum iacula \textbf{ et missilia : } quia tunc quiescit in sinistra , \\\hline
3.3.15 & e mueuese en la derecha \textbf{ que ha de esgrimir el dardo . } el qual esgrimido mueue el ayre & quia tunc quiescit in sinistra , \textbf{ et mouetur in dextra vibrans ipsum iaculum ; } quo vibrato vehementius mouet aerem , \\\hline
3.3.15 & e faz mas fuerte colpe . \textbf{ Enpero maguer que podamos folgar tan bien sobre la parte derecha } commmo sobre la esquierda . Empero la parte derecha sienpre es mas apareiada para mouimiento & et fortius ferit . Licet enim tam \textbf{ secundum partem dextram quam secundum sinistram possumus quiescere } et moueri , \\\hline
3.3.15 & e la esquierda para folgura . \textbf{ Et por ende feriendo de lueñe deuemos folgar sobre el pie esquierdo puesto delante } e alongarnos con el derecho & dextra tamen est aptior ad mouendum , \textbf{ et sinistra ad quiescendum . Ideo percutientes a remotis debemus quiescere super sinistrum pedem ante missum , } et elongare nos cum dextro , \\\hline
3.3.15 & Et por ende feriendo de lueñe deuemos folgar sobre el pie esquierdo puesto delante \textbf{ e alongarnos con el derecho } por que podamos mas fuertemente esgremir & et sinistra ad quiescendum . Ideo percutientes a remotis debemus quiescere super sinistrum pedem ante missum , \textbf{ et elongare nos cum dextro , } ut possimus vehementius impellere , et vibrare iaculum . \\\hline
3.3.15 & e alongarnos con el derecho \textbf{ por que podamos mas fuertemente esgremir } e lançar el dardo . & et elongare nos cum dextro , \textbf{ ut possimus vehementius impellere , et vibrare iaculum . } Sed quando manu ad manum pugnatur gladio : \\\hline
3.3.15 & por que podamos mas fuertemente esgremir \textbf{ e lançar el dardo . } Mas quando venimos a las manos & et elongare nos cum dextro , \textbf{ ut possimus vehementius impellere , et vibrare iaculum . } Sed quando manu ad manum pugnatur gladio : \\\hline
3.3.15 & lidiando con las espadas o con los cuchiellos \textbf{ deuemos tener la manera contraria } de lo que dicho es & Sed quando manu ad manum pugnatur gladio : \textbf{ debemus e contrario nos habere , } ita quod pedem dextrum teneamus ante , et sinistrum post . \\\hline
3.3.15 & Ca por que el costado derecho es mas conuenible al mouimiento \textbf{ si el estudiere mas cerca del enemigo por razon del mouimiento podra foyr meior los colpes . } Ca mas fuertemente es ferido aquello que esta & Nam quia latus dextrum aptius est ad motum , \textbf{ si illud sit hosti propinquius , ratione motus poterit melius ictus effugere : | vehementius enim percutitur } quod stat , quam quod mouetur . Rursus , \\\hline
3.3.15 & Otrossi el costado derecho fuere \textbf{ mas çerca del enemigo meior le podra ferir . } Ca deuen los lidiadores & si dextrum latus sit hosti propinquius , \textbf{ melius poterit ipsum percutere . Debent enim bellatores , } cum ad manum pugnant , \\\hline
3.3.15 & Ca deuen los lidiadores \textbf{ quando vienen a las manos tener el pie esquierdo firme } e quando quieren ferir & melius poterit ipsum percutere . Debent enim bellatores , \textbf{ cum ad manum pugnant , | tenere pedem sinistrum immobiliter : } et cum volunt percutere , \\\hline
3.3.15 & quando vienen a las manos tener el pie esquierdo firme \textbf{ e quando quieren ferir } deuen se fazer adelante con el pie derecho & tenere pedem sinistrum immobiliter : \textbf{ et cum volunt percutere , } cum pede dextro debent se antefacere , \\\hline
3.3.15 & e quando quieren ferir \textbf{ deuen se fazer adelante con el pie derecho } e quando dieren los colpes deuense arredrar con el pie derecho . & et cum volunt percutere , \textbf{ cum pede dextro debent se antefacere , } et cum volunt ictus fugere , \\\hline
3.3.15 & deuen se fazer adelante con el pie derecho \textbf{ e quando dieren los colpes deuense arredrar con el pie derecho . } Et pues que assi es temiendo & cum pede dextro debent se antefacere , \textbf{ et cum volunt ictus fugere , | cum eodem pede debent se retrahere ; } sic itaque tenendo pedem sinistrum immobilem , \\\hline
3.3.15 & assi el pie esquierto firme \textbf{ e mouiendo se con el esquierdo pie podrian mas fuertemente ferir los enemigos e mas ligeramente foyr los colpes dellos . } Visto commo deuen estar los lidiadores & et cum dextro se mouendo , \textbf{ poterunt fortius hostes percutere , | et eorum ictus facilius fugere . } Viso quomodo debeant stare bellantes , \\\hline
3.3.15 & Visto commo deuen estar los lidiadores \textbf{ quando deuen ferir los enemigos . } finca nos de ver qual manera son los enemigos de ençerrar e de çercar . & Viso quomodo debeant stare bellantes , \textbf{ si debeant hostes percutere : } videre restat , \\\hline
3.3.15 & quando deuen ferir los enemigos . \textbf{ finca nos de ver qual manera son los enemigos de ençerrar e de çercar . } pues que assi es conuiene de saber & si debeant hostes percutere : \textbf{ videre restat , | quomodo sunt hostes includendi et circumdandi . } Sciendum igitur , \\\hline
3.3.15 & finca nos de ver qual manera son los enemigos de ençerrar e de çercar . \textbf{ pues que assi es conuiene de saber } que pocas vezes o nunca son los enemigos de ençerar & quomodo sunt hostes includendi et circumdandi . \textbf{ Sciendum igitur , } quod raro aut nunquam sic circumdandi sunt hostes in pugna publica , \\\hline
3.3.15 & pues que assi es conuiene de saber \textbf{ que pocas vezes o nunca son los enemigos de ençerar } e de çercar en la batalla publica . & Sciendum igitur , \textbf{ quod raro aut nunquam sic circumdandi sunt hostes in pugna publica , } quod non pateat aliquis aditus fugiendi : \\\hline
3.3.15 & que pocas vezes o nunca son los enemigos de ençerar \textbf{ e de çercar en la batalla publica . } assi que non les finque algun logar para foyr . & Sciendum igitur , \textbf{ quod raro aut nunquam sic circumdandi sunt hostes in pugna publica , } quod non pateat aliquis aditus fugiendi : \\\hline
3.3.15 & e de çercar en la batalla publica . \textbf{ assi que non les finque algun logar para foyr . } Ca estonçe con desesperamiento & quod raro aut nunquam sic circumdandi sunt hostes in pugna publica , \textbf{ quod non pateat aliquis aditus fugiendi : } quia desperantes quasi necessitate compulsi efficiuntur audaces , \\\hline
3.3.15 & Ca veyendo que les non finca \textbf{ si non la muerte pueden acometer muchas malas cosas contra aquellos que lidian contra ellos . } Et por ende es alabada la sentençia de çipion & videntes enim se necessario moriendos , \textbf{ possunt multa mala committere in eos } qui contra pugnant . Inde est quod laudatur Scipionis sententia , dicentis : \\\hline
3.3.15 & por la qual dizia \textbf{ que nunca eran de encerrar los enemigos } assi que les non fincasse logar para foyr . & qui contra pugnant . Inde est quod laudatur Scipionis sententia , dicentis : \textbf{ Nunquam sic esse claudendos hostes , } quod non pateat eis aditus fugiendi . \\\hline
3.3.15 & que nunca eran de encerrar los enemigos \textbf{ assi que les non fincasse logar para foyr . } Ca yendo los enemigos non ay ningun periglo & Nunquam sic esse claudendos hostes , \textbf{ quod non pateat eis aditus fugiendi . } Nam fugientibus hostibus nullum est periculum , \\\hline
3.3.15 & e en la fuyda mucho resçiben periglo sin enpesçimiento de aquellos que los persiguen . \textbf{ Mas quando se veen ençerrados han se de tornar } assi commo costrennidos de la muerte & et in fuga periclitantur multi absque nocumento persequentium ; \textbf{ sed cum se vident inclusos , } quasi coacti feriunt includentes . \\\hline
3.3.15 & assi commo costrennidos de la muerte \textbf{ a ferir } a aquellos que los ençierran . & sed cum se vident inclusos , \textbf{ quasi coacti feriunt includentes . } Cum ergo supra diximus , formandam aliquando esse aciem sub forma forficulari , \\\hline
3.3.15 & quando dixiemos dessuso \textbf{ que alguans vezes el az es de formar so forma de tiieras } e esto quando los enemigos son pocos & quasi coacti feriunt includentes . \textbf{ Cum ergo supra diximus , formandam aliquando esse aciem sub forma forficulari , } ut quando hostes pauci , \\\hline
3.3.15 & para que meior sean ençerrados e çerrados . \textbf{ esto non es de entender } assi que assi los deuamos çercar & ad hoc quod melius includantur et circundentur . \textbf{ Non sic intelligendum est , } quod ita debeant circumdari , \\\hline
3.3.15 & esto non es de entender \textbf{ assi que assi los deuamos çercar } que non les finque algun logar para foyr & Non sic intelligendum est , \textbf{ quod ita debeant circumdari , } quod nullus pateat aditus abscindenti , \\\hline
3.3.15 & assi que assi los deuamos çercar \textbf{ que non les finque algun logar para foyr } si non por auentura & quod ita debeant circumdari , \textbf{ quod nullus pateat aditus abscindenti , } nisi forte adeo essent pauci , \\\hline
3.3.15 & quando ellos fuessen tan pocos \textbf{ que commo quier que ellos quisiessen reuellar e defender } sse non les pudiessen fazer ningun enpeesçimiento . & nisi forte adeo essent pauci , \textbf{ quod quantumcunque debellare vellent nullum possent nocumentum efficere . } Ostenso itaque qualiter debeant stare pugnantes , \\\hline
3.3.15 & que commo quier que ellos quisiessen reuellar e defender \textbf{ sse non les pudiessen fazer ningun enpeesçimiento . } Pues que assi es mostrado & nisi forte adeo essent pauci , \textbf{ quod quantumcunque debellare vellent nullum possent nocumentum efficere . } Ostenso itaque qualiter debeant stare pugnantes , \\\hline
3.3.15 & en qual manera deuen estar los peones lidiadores \textbf{ si quisieren ferir los enemigos } e en qual manera los deuen çercar . & Ostenso itaque qualiter debeant stare pugnantes , \textbf{ si velint hostes percutere , } et qualiter eos circumstare . \\\hline
3.3.15 & si quisieren ferir los enemigos \textbf{ e en qual manera los deuen çercar . } finca nos agora lo terçero de mostrar & si velint hostes percutere , \textbf{ et qualiter eos circumstare . } Restat nunc tertio declarare , \\\hline
3.3.15 & e en qual manera los deuen çercar . \textbf{ finca nos agora lo terçero de mostrar } en qual manera se deuan arredrar de la batalla & et qualiter eos circumstare . \textbf{ Restat nunc tertio declarare , } qualiter sit declinandum a pugna , \\\hline
3.3.15 & finca nos agora lo terçero de mostrar \textbf{ en qual manera se deuan arredrar de la batalla } si non ouieren conseio de lidiar & Restat nunc tertio declarare , \textbf{ qualiter sit declinandum a pugna , } si non habeatur consilium , \\\hline
3.3.15 & en qual manera se deuan arredrar de la batalla \textbf{ si non ouieren conseio de lidiar } nin les semeiare bueno de acometer la batalla . & qualiter sit declinandum a pugna , \textbf{ si non habeatur consilium , } nec videatur bonam pugnam committere , \\\hline
3.3.15 & si non ouieren conseio de lidiar \textbf{ nin les semeiare bueno de acometer la batalla . } Et esto por que los enemigos son mas fuertes que ellos & si non habeatur consilium , \textbf{ nec videatur bonam pugnam committere , } eo quod hostes sint fortiores , \\\hline
3.3.15 & e non pueden estar contra ellos \textbf{ nin lidiar con ellos . } Enel qual caso & eo quod hostes sint fortiores , \textbf{ et non possumus illis resistere . } In quo ( quantum ad praesens spectat ) \\\hline
3.3.15 & Enel qual caso \textbf{ quanto pertenesçe a lo presente dos cautelas deue auer el cabdiello de la batalla . } La primera es de parte de la su hueste propria . & In quo ( quantum ad praesens spectat ) \textbf{ debet dux belli duplicem cautelam habere . Prima est , } quantum ad exercitum proprium . Nam et si dux consilium habeat non esse pugnandum , \\\hline
3.3.15 & La primera es de parte de la su hueste propria . \textbf{ Ca si el cabdiello ouiere conseio de non lidiar . } esto deue dezir en su poridat a muy pocos & debet dux belli duplicem cautelam habere . Prima est , \textbf{ quantum ad exercitum proprium . Nam et si dux consilium habeat non esse pugnandum , } debet hoc valde paucis patefacere , \\\hline
3.3.15 & Ca si el cabdiello ouiere conseio de non lidiar . \textbf{ esto deue dezir en su poridat a muy pocos } e non lo deue reuelar a toda la hueste & quantum ad exercitum proprium . Nam et si dux consilium habeat non esse pugnandum , \textbf{ debet hoc valde paucis patefacere , } et non debet illud toti exercitui pandere ; \\\hline
3.3.15 & esto deue dezir en su poridat a muy pocos \textbf{ e non lo deue reuelar a toda la hueste } por que non ayan de foyr malamente con temor . & debet hoc valde paucis patefacere , \textbf{ et non debet illud toti exercitui pandere ; } ne timentes , turpiter fugiant , \\\hline
3.3.15 & e non lo deue reuelar a toda la hueste \textbf{ por que non ayan de foyr malamente con temor . } e sean muertos fuyendo de sus enemigos & et non debet illud toti exercitui pandere ; \textbf{ ne timentes , turpiter fugiant , } et ab insequentibus hostibus occidantur . \\\hline
3.3.15 & Et pues que \textbf{ assi es en tal manera se deue auer el cabdiello } que non crea la hueste & et ab insequentibus hostibus occidantur . \textbf{ Taliter itaque dux se habere debet , } quod non credat exercitus quod velit fugere , \\\hline
3.3.15 & que non crea la hueste \textbf{ que quiere foyr } mas que quiere a otra parte echar çeladas & Taliter itaque dux se habere debet , \textbf{ quod non credat exercitus quod velit fugere , } sed quod velit alibi insidias parare , \\\hline
3.3.15 & que quiere foyr \textbf{ mas que quiere a otra parte echar çeladas } e que quiere mas brauamente lidiar contra los enemigos . & quod non credat exercitus quod velit fugere , \textbf{ sed quod velit alibi insidias parare , } et contra hostes velit acrius dimicare . \\\hline
3.3.15 & mas que quiere a otra parte echar çeladas \textbf{ e que quiere mas brauamente lidiar contra los enemigos . } Mas la segunda cautela es de tomar de parte de la hueste de los enemigos . & sed quod velit alibi insidias parare , \textbf{ et contra hostes velit acrius dimicare . } Secunda cautela adhibenda est ex parte exercitus hostium . \\\hline
3.3.15 & e que quiere mas brauamente lidiar contra los enemigos . \textbf{ Mas la segunda cautela es de tomar de parte de la hueste de los enemigos . } Ca assi deue escusar la batalla & et contra hostes velit acrius dimicare . \textbf{ Secunda cautela adhibenda est ex parte exercitus hostium . } Nam sic debet deducere bellum , \\\hline
3.3.15 & Mas la segunda cautela es de tomar de parte de la hueste de los enemigos . \textbf{ Ca assi deue escusar la batalla } pues non ha conseio de lidiar & Secunda cautela adhibenda est ex parte exercitus hostium . \textbf{ Nam sic debet deducere bellum , } ut hoc hostes lateat . \\\hline
3.3.15 & Ca assi deue escusar la batalla \textbf{ pues non ha conseio de lidiar } que esto non lo sepan los enemigos . & Nam sic debet deducere bellum , \textbf{ ut hoc hostes lateat . } Ideo multi tempore nocturno potius quam diuino hoc agunt : \\\hline
3.3.15 & contra los enemigos pusiessen se ante los peones \textbf{ por que los non pudiessen ver . } Por la qual cosa la batalla de los peones & quod milites stantes in acie ex opposito hostium , \textbf{ prohibent eos | ne pedites videre possint : } propter quod pedestris pugna latenter recedit , \\\hline
3.3.15 & encubiertamente se escusa yendo se los peones . \textbf{ Et ellos ydos los caualleros pueden meior despues escusar los colpes de los enemigos . } Avn conuiene de saber & propter quod pedestris pugna latenter recedit , \textbf{ qua recedente , equites postea melius possunt vitare hostium percussiones . } Est etiam aduertendum quod quando sic declinatur pugna , \\\hline
3.3.15 & Et ellos ydos los caualleros pueden meior despues escusar los colpes de los enemigos . \textbf{ Avn conuiene de saber } que quando se assi escusa la batalla nunca la az se deue departir . & qua recedente , equites postea melius possunt vitare hostium percussiones . \textbf{ Est etiam aduertendum quod quando sic declinatur pugna , } nunquam acies se debent diuidere : \\\hline
3.3.15 & Avn conuiene de saber \textbf{ que quando se assi escusa la batalla nunca la az se deue departir . } ca podrie contesçer & qua recedente , equites postea melius possunt vitare hostium percussiones . \textbf{ Est etiam aduertendum quod quando sic declinatur pugna , } nunquam acies se debent diuidere : \\\hline
3.3.15 & que quando se assi escusa la batalla nunca la az se deue departir . \textbf{ ca podrie contesçer } que los enemigos persiguirien & Est etiam aduertendum quod quando sic declinatur pugna , \textbf{ nunquam acies se debent diuidere : } quia si contingeret hostes insequi fugientes a bello , \\\hline
3.3.15 & que si se tornassen e lidiassen . \textbf{ Deue avn el cabdiello de la batalla catar } sy ay algun logar çerca & et bellarent . \textbf{ Debet | etiam dux belli inquirere , } utrum sit aliquis locus propinquus , \\\hline
3.3.15 & sy ay algun logar çerca \textbf{ a que se pueda acoger la hueste } si fuere menester & utrum sit aliquis locus propinquus , \textbf{ ad quem posset confugere exercitus , } si fugaretur ab hostibus . \\\hline
3.3.15 & si fuere menester \textbf{ que los sus contrarios los ayan a fazer foyr . } p paresçe que todas e las batallas se puenden adozir a quatro maneras . & ad quem posset confugere exercitus , \textbf{ si fugaretur ab hostibus . } Videntur omnia bella ad quatuor genera reduci , \\\hline
3.3.16 & que los sus contrarios los ayan a fazer foyr . \textbf{ p paresçe que todas e las batallas se puenden adozir a quatro maneras . } las quales son estas . & si fugaretur ab hostibus . \textbf{ Videntur omnia bella ad quatuor genera reduci , } videlicet ad campestre , \\\hline
3.3.16 & las quales son estas . \textbf{ Conuiene de saber . } Batalla canpal . Batalla de çerca . Batalla de defendimiento . & videlicet ad campestre , \textbf{ obsessiuum , } defensiuum , \\\hline
3.3.16 & tanto es mas canpal e mas periglosa . \textbf{ La segunda manera de batalla es manera de çercar . } e esto es quando los lidiadores son de tanto poder & et magis periculosa . \textbf{ Secundum genus pugnae dicitur obsessiuum , } quando bellatores sunt tantae potentiae \\\hline
3.3.16 & e de las çibdades \textbf{ e de los castiellos a lidiar al canpo . } Mas ellos acometen aquellas villas & quod non expectant , \textbf{ quod hostes de munitionibus exeuntes vadant bellare ad campum , } sed ipsi munitiones inuadunt \\\hline
3.3.16 & que enl canpo podrian estar \textbf{ nin se defender de los enemigos . } Et por ende se tienen çercados en sus logares & sic contingit aliquos esse adeo paucos et tam debiles , \textbf{ ut non putent in campo posse resistere impugnantibus . } Ideo se in munitionibus tenet clausos , \\\hline
3.3.16 & e cunpleles \textbf{ que puedan defender aquel los logares e aquellas fortalezas } si por auentura fueren çercados de los enemigos . & Ideo se in munitionibus tenet clausos , \textbf{ et sufficit eis quod possint munitiones defendere , } si contingat eas ab hostibus impugnari . Tale genus pugnae \\\hline
3.3.16 & es dicha defenssiua . \textbf{ La quarta manera de lidiar es naual de las naues e de la mar . } Ca assi commo contesçe de ser batalla en la tierra & quam inuasio . ideo tale genus pugnae merito dicitur defensiuum . \textbf{ Quartus autem pugnandi modus dicitur naualis : } quia sicut contingit esse pugnam in terra , \\\hline
3.3.16 & de la batalla de la tierra \textbf{ fincanos de dezir de las otras tres . } Conuieue de saber de la de çerca & postquam diximus de campestri , \textbf{ restat dicere de obsessiua , defensiua , } et nauali . Contingit enim aliquando Reges , \\\hline
3.3.16 & fincanos de dezir de las otras tres . \textbf{ Conuieue de saber de la de çerca } e de la de defendimiento & postquam diximus de campestri , \textbf{ restat dicere de obsessiua , defensiua , } et nauali . Contingit enim aliquando Reges , \\\hline
3.3.16 & que es de las naues \textbf{ Ca contesçe algunas uezes que los Reyes e los prinçipes lidian en todas estas maneras de lidiar . } Ca algunas uezes acometen batalla canpal & et nauali . Contingit enim aliquando Reges , \textbf{ et Principes pugnare omnibus his modis pugnandi . } Nam aliquando committunt campestre bellum . \\\hline
3.3.16 & que algunos otros çercan sus villas o sus castiellos . \textbf{ Por la qual cosa les conuiene de vsar de batalla defenssiua } para se defender . & Contingit \textbf{ etiam aliquando aliquos inuadere aliquas munitiones eorum ; } propter quod eos oportet uti pugna defensiua . Amplius in principatu et regno contingit esse portus et terras maritimas iuxta mare sitas : \\\hline
3.3.16 & Por la qual cosa les conuiene de vsar de batalla defenssiua \textbf{ para se defender . } Otrossi contesçe & Contingit \textbf{ etiam aliquando aliquos inuadere aliquas munitiones eorum ; } propter quod eos oportet uti pugna defensiua . Amplius in principatu et regno contingit esse portus et terras maritimas iuxta mare sitas : \\\hline
3.3.16 & conuiene a los Reyes \textbf{ e a los principes algunas vezes de ordenar } e de fazer batallas nauales e de naues . & expedit regibus \textbf{ et principibus aliquando ordinare bella naualia . } Dicto itaque de bello campestri , \\\hline
3.3.16 & e a los principes algunas vezes de ordenar \textbf{ e de fazer batallas nauales e de naues . } Et pues que assi es dicho de la lid canpal & expedit regibus \textbf{ et principibus aliquando ordinare bella naualia . } Dicto itaque de bello campestri , \\\hline
3.3.16 & Et pues que assi es dicho de la lid canpal \textbf{ e del canpo es de dezir } de las otras maneras de las batallas . & Dicto itaque de bello campestri , \textbf{ dicendum est de aliis generib’ bellorum . } Verum quia de campestri pugna diffusius diximus , \\\hline
3.3.16 & en toda lid \textbf{ e en toda batalla en qual manera cada vno se deue auer en ellas . } non conuiene çerca las otras maneras de las batallas & et de cautelis bellorum multa discutimus ; \textbf{ cum per iam dicta circa omne bellum possint cautelae haberi qualiter quis debeat se habere , } non oportet circa alia bellorum genera diutius immorari . \\\hline
3.3.16 & non conuiene çerca las otras maneras de las batallas \textbf{ de de tener nos mas luengamente . } Enpero diremos de la batalla osse ssiua & cum per iam dicta circa omne bellum possint cautelae haberi qualiter quis debeat se habere , \textbf{ non oportet circa alia bellorum genera diutius immorari . } Primo tamen dicemus de bello obsessiuo . \\\hline
3.3.16 & Enpero diremos de la batalla osse ssiua \textbf{ e que quiere dezir batalla de cercamiento . } Et por ende uisto quantas son las maneras de las batallas . & non oportet circa alia bellorum genera diutius immorari . \textbf{ Primo tamen dicemus de bello obsessiuo . } Viso ergo quot sunt bellorum genera , \\\hline
3.3.16 & Et por ende uisto quantas son las maneras de las batallas . \textbf{ Et dicho que despues de la lid canpal del canpo primero es de dezir de la lid } que se faz por çercar & Viso ergo quot sunt bellorum genera , \textbf{ et dicto quod post castrum campestre primo dicendum est de pugna obsessiua : } cum per huiusmodi pugnam contingat obtineri et deuinci munitiones \\\hline
3.3.16 & Et dicho que despues de la lid canpal del canpo primero es de dezir de la lid \textbf{ que se faz por çercar } por que por tal lid contesçe tomar & Viso ergo quot sunt bellorum genera , \textbf{ et dicto quod post castrum campestre primo dicendum est de pugna obsessiua : } cum per huiusmodi pugnam contingat obtineri et deuinci munitiones \\\hline
3.3.16 & que se faz por çercar \textbf{ por que por tal lid contesçe tomar } e vençer las villas e los castiellos e fortalezas . & et dicto quod post castrum campestre primo dicendum est de pugna obsessiua : \textbf{ cum per huiusmodi pugnam contingat obtineri et deuinci munitiones } et urbanitates : \\\hline
3.3.16 & por que por tal lid contesçe tomar \textbf{ e vençer las villas e los castiellos e fortalezas . } fincanos de dezir en quantas maneras tales fortalezas pueden ser vençidas . & cum per huiusmodi pugnam contingat obtineri et deuinci munitiones \textbf{ et urbanitates : } restat dicere quot modis talia deuinci possunt . \\\hline
3.3.16 & e vençer las villas e los castiellos e fortalezas . \textbf{ fincanos de dezir en quantas maneras tales fortalezas pueden ser vençidas . } Et conuiene de saber & et urbanitates : \textbf{ restat dicere quot modis talia deuinci possunt . } Est autem triplex modus obtinendi munitiones \\\hline
3.3.16 & fincanos de dezir en quantas maneras tales fortalezas pueden ser vençidas . \textbf{ Et conuiene de saber } que son tres maneras de ganar las fortalezas e los castiellos . & et urbanitates : \textbf{ restat dicere quot modis talia deuinci possunt . } Est autem triplex modus obtinendi munitiones \\\hline
3.3.16 & Et conuiene de saber \textbf{ que son tres maneras de ganar las fortalezas e los castiellos . } Conuiene saber . & restat dicere quot modis talia deuinci possunt . \textbf{ Est autem triplex modus obtinendi munitiones | et castra , } videlicet , \\\hline
3.3.16 & que son tres maneras de ganar las fortalezas e los castiellos . \textbf{ Conuiene saber . } por sed e por fanbre e por batalla . & et castra , \textbf{ videlicet , } per sitim , famem , \\\hline
3.3.16 & que los cercados non han agua . \textbf{ e por ende o les conuiene de peresçer } o de morir de sedo de dar las fortalezas . & et pugnam . Contingit enim aliquando obsessos carere aqua : ideo \textbf{ vel oportet eos siti perire , } vel munitiones reddere . \\\hline
3.3.16 & e por ende o les conuiene de peresçer \textbf{ o de morir de sedo de dar las fortalezas . } Por la qual cosa con grant acuçia deuen cuydar & vel oportet eos siti perire , \textbf{ vel munitiones reddere . } Quare diligenter excogitare debent obsidentes munitiones aliquas , \\\hline
3.3.16 & o de morir de sedo de dar las fortalezas . \textbf{ Por la qual cosa con grant acuçia deuen cuydar } los que cercan algunas fortalezas & vel munitiones reddere . \textbf{ Quare diligenter excogitare debent obsidentes munitiones aliquas , } utrum per aliqua ingenia , \\\hline
3.3.16 & los que cercan algunas fortalezas \textbf{ si por algunos engeñios o por alguna sotileza puedan tomar el agua de los cercados . } Ca muchas uegadas contesçe & utrum per aliqua ingenia , \textbf{ vel per aliquam industriam possint ab obsessis accipere aquam . Nam multotiens euenit , } aquam a remoto principio deriuari usque ad munitiones obsessas : \\\hline
3.3.16 & por do viene el agua a los cercados \textbf{ han de auer los cercados } por fuerça mengua de agua . & per quam pergit aqua ad obsessos , \textbf{ oportebit ipsos pati aquarum penuriam . Rursus , } aliquando munitiones sunt altae , \\\hline
3.3.16 & Otrossi algunas uegadas las fortalezas son altas \textbf{ e el agua non puede venir fasta ellas . } por la qual cosa si el agua fuere lueñe de la fortaleza & aliquando munitiones sunt altae , \textbf{ et aqua non peruenit usque ad eas : } quare si sit a munitionibus remota , \\\hline
3.3.16 & por la qual cosa si el agua fuere lueñe de la fortaleza \textbf{ los que cercan deuen auer grant acuçia } en commo defiendan el agua a los cercado & quare si sit a munitionibus remota , \textbf{ debent obsidentes adhibere omnem diligentiam , } quomodo possint obsessis prohibere aquam . Secundus modus impugnandi munitiones , est per famem . \\\hline
3.3.16 & o gela tiren . \textbf{ La segunda manera para ganar las fortalezas es por fanbre . } ca nos non podemos durar nin beuir sin comer . & debent obsidentes adhibere omnem diligentiam , \textbf{ quomodo possint obsessis prohibere aquam . Secundus modus impugnandi munitiones , est per famem . } Nam sine cibo durare non possumus ; ideo obsidentes , \\\hline
3.3.16 & La segunda manera para ganar las fortalezas es por fanbre . \textbf{ ca nos non podemos durar nin beuir sin comer . } Et por ende los que çercan & quomodo possint obsessis prohibere aquam . Secundus modus impugnandi munitiones , est per famem . \textbf{ Nam sine cibo durare non possumus ; ideo obsidentes , } ut munitiones obtineant , passus , vias \\\hline
3.3.16 & por que ganen mas fortalezas \textbf{ deuen guardar con grand acuçia todos los passos e los caminos e los logares } por do pueden venir viandas a los cercados & ut munitiones obtineant , passus , vias \textbf{ et omnia loca per quae possent obsessis victualia deferri , | diligenter custodire debent , } ne eis talia deferantur . \\\hline
3.3.16 & deuen guardar con grand acuçia todos los passos e los caminos e los logares \textbf{ por do pueden venir viandas a los cercados } por que tales cosas non puedan venir a ellos & diligenter custodire debent , \textbf{ ne eis talia deferantur . } In huiusmodi enim obsessionibus multotiens plus affligit fames \\\hline
3.3.16 & por do pueden venir viandas a los cercados \textbf{ por que tales cosas non puedan venir a ellos } de que se puedan mantener muchas contesçe en tales çercas & diligenter custodire debent , \textbf{ ne eis talia deferantur . } In huiusmodi enim obsessionibus multotiens plus affligit fames \\\hline
3.3.16 & por que tales cosas non puedan venir a ellos \textbf{ de que se puedan mantener muchas contesçe en tales çercas } que la fanbre mata & ne eis talia deferantur . \textbf{ In huiusmodi enim obsessionibus multotiens plus affligit fames } quam gladius . \\\hline
3.3.16 & Et por ende contesçe que muchas uegadas \textbf{ los que çercan queriendo mas ayna ganar las fortalezas } si contezca & quam gladius . \textbf{ Inde est quod multotiens obsidentes volentes citius opprimere munitiones , } si contingat eos capere aliquos de obsessis , \\\hline
3.3.16 & e mayor desfallesçimiento en viandas . \textbf{ La terçera manera de ganar las fortalezas es por batalla } assi commo quando van a los nivros & apud ipsos obsessos maiorem famem \textbf{ et inopiam inducant . Tertius modus obtinendi munitiones est per pugnam : } ut cum itur ad muros , \\\hline
3.3.16 & por batalla a los que son cercados . \textbf{ Mas commo e en quantas maneras conuiene de acometer tal batalla } mostrar lo hemos en el capitulo & et cum per pugnam dimicatur contra obsessos . \textbf{ Sed qualiter | et quot modis contingat pugnam committere , } in sequenti capitulo ostendetur . \\\hline
3.3.16 & Mas commo e en quantas maneras conuiene de acometer tal batalla \textbf{ mostrar lo hemos en el capitulo } que se sigue . & et quot modis contingat pugnam committere , \textbf{ in sequenti capitulo ostendetur . } Ostenso quot sunt genera bellorum , \\\hline
3.3.16 & Mostrado quantas son las maneras de las batallas \textbf{ e en quantas maneras son de vençer las fortalezas cerradas } finca de demostrar en que tienpo es meior de çercar las çibdades e las castiellos . & Ostenso quot sunt genera bellorum , \textbf{ et quot modis deuincendae sunt munitiones obsessae : } restat ostendere , \\\hline
3.3.16 & e en quantas maneras son de vençer las fortalezas cerradas \textbf{ finca de demostrar en que tienpo es meior de çercar las çibdades e las castiellos . } Et por ende conuiene de saber & et quot modis deuincendae sunt munitiones obsessae : \textbf{ restat ostendere , | quo tempore melius est obsidere ciuitates et castra . } Sciendum itaque quod tempore aestiuo \\\hline
3.3.16 & finca de demostrar en que tienpo es meior de çercar las çibdades e las castiellos . \textbf{ Et por ende conuiene de saber } que en el tienpo del uerano & quo tempore melius est obsidere ciuitates et castra . \textbf{ Sciendum itaque quod tempore aestiuo } antequam sint recollecta blada , vina , \\\hline
3.3.16 & nin el vino nin las otras cosas \textbf{ por que se ha de acorrer la çibdat o el logar de aquellos que estan cercados } en aquel tienpo & antequam sint recollecta blada , vina , \textbf{ et alia per quae subueniri potest inopiae obsessorum , } est melius obsessionem facere . \\\hline
3.3.16 & en aquel tienpo \textbf{ es meior de cercar las fortalezas . } Ca en aquel tienpo en toda manera & et alia per quae subueniri potest inopiae obsessorum , \textbf{ est melius obsessionem facere . } Illo enim tempore omni modo deuincendi , \\\hline
3.3.16 & Ca en aquel tienpo en toda manera \textbf{ los que se han e vençer } por çerca meior se vençen & Illo enim tempore omni modo deuincendi , \textbf{ melius deuincuntur obsessi . } Nam si per sitim sunt munitiones obtinendae , \\\hline
3.3.16 & por çerca meior se vençen \textbf{ Ca si por sed son de ganar las fortalezas } meior es de fazer la çerca en el tienpo del estiuo & melius deuincuntur obsessi . \textbf{ Nam si per sitim sunt munitiones obtinendae , } melius est facere obsessionem tempore aestiuo , \\\hline
3.3.16 & Ca si por sed son de ganar las fortalezas \textbf{ meior es de fazer la çerca en el tienpo del estiuo } por que entonçe & Nam si per sitim sunt munitiones obtinendae , \textbf{ melius est facere obsessionem tempore aestiuo , } eo \\\hline
3.3.16 & mas se dessecan las aguas nin las luuias del çielo non son de grant abondança \textbf{ por que se puedan mantener los cercados } por agua cogida en algibes . & nec si abundant pluuiae caelestes , \textbf{ ut possit per cisternas subuenire obsessis . Rursus si per famen est castrum , } vel ciuitas obsessa obtinenda , \\\hline
3.3.16 & por agua cogida en algibes . \textbf{ Otrossi si por fanbre se ha de tomar la çibdat cercada } o el castiello meior es de cercar la en el tienpo del estiuo & ut possit per cisternas subuenire obsessis . Rursus si per famen est castrum , \textbf{ vel ciuitas obsessa obtinenda , } melius est obsessionem facere aestiuo tempore , \\\hline
3.3.16 & Otrossi si por fanbre se ha de tomar la çibdat cercada \textbf{ o el castiello meior es de cercar la en el tienpo del estiuo } ante que ssean las miesses cogidas & vel ciuitas obsessa obtinenda , \textbf{ melius est obsessionem facere aestiuo tempore , } antequam messes \\\hline
3.3.16 & nin los vinos . \textbf{ Ca sienpre en tal tienpo suelen fallesçer los fructos del año passado . } Por la qual cosa & et vina sint recollecta : \textbf{ quia semper tali tempore consueuerunt deficere fructus anni praeteriti . } Quare si obsessi non possunt gaudere fructibus anni aduenientis , \\\hline
3.3.16 & Por la qual cosa \textbf{ si los que estan çercados non se pueden acorrer } de los fuctos de esse aneo & quia semper tali tempore consueuerunt deficere fructus anni praeteriti . \textbf{ Quare si obsessi non possunt gaudere fructibus anni aduenientis , } citius peribunt inopia . \\\hline
3.3.16 & Mas ayna pereres çran por fanbre e por mengua . \textbf{ Otrossi si por batalla o por lid son de ganar las fortalezas } meior es de acometerlas en este tienpo . & citius peribunt inopia . \textbf{ Amplius si post bellum et pugnam munitiones sunt obtinendae , | melius } est hoc agere aestiuo tempore . \\\hline
3.3.16 & Otrossi si por batalla o por lid son de ganar las fortalezas \textbf{ meior es de acometerlas en este tienpo . } Ca en el tienpo del yuierno & melius \textbf{ est hoc agere aestiuo tempore . } Nam tempore hyemali abundant pluuiae , \\\hline
3.3.16 & Por la qual cosa con muy grand graueza \textbf{ e con grand trabaio se pueden acometer } los que estan cercados . & replentur fossae aquis : \textbf{ quare difficilius impugnantur obsessi . Rursus incommoditates temporum magis molestant obsidentes } et existents in campis , \\\hline
3.3.16 & e estan en las casas \textbf{ Pues que assi es o las cercas son de fazer } en el tienpo del estiuo o łi por muchos tienpos han de durar las cercas & obsessiones fiendae sunt tempore aestiuo , \textbf{ vel si per multa tempora obsessiones durare debent , } saltem inchoandae sunt tempore aestiuo , \\\hline
3.3.16 & Pues que assi es o las cercas son de fazer \textbf{ en el tienpo del estiuo o łi por muchos tienpos han de durar las cercas } deuen se començar en el tienpo del estuo & vel si per multa tempora obsessiones durare debent , \textbf{ saltem inchoandae sunt tempore aestiuo , } priusquam blada , vina , \\\hline
3.3.16 & en el tienpo del estiuo o łi por muchos tienpos han de durar las cercas \textbf{ deuen se començar en el tienpo del estuo } ante que las miesses nin los vinos & vel si per multa tempora obsessiones durare debent , \textbf{ saltem inchoandae sunt tempore aestiuo , } priusquam blada , vina , \\\hline
3.3.16 & ante que las miesses nin los vinos \textbf{ nin los fructos de la tierra puedan coger } los que estan çercados . & priusquam blada , vina , \textbf{ et alios fructus terrae recolligere possint obsessi . }  \\\hline
3.3.17 & e non se guarnesçieren con grand acuçia \textbf{ pueden resçebir daño de los que estan çercados } Ca commo contezca & et non diligenter se muniant , \textbf{ ab obsessis molestari poterunt . } Nam cum contingat obsessiones per multa aliquando durare tempora , \\\hline
3.3.17 & Ca commo contezca \textbf{ que las çercas puedan durar algunas vezes } e por muchos tienpos non puede ser & ab obsessis molestari poterunt . \textbf{ Nam cum contingat obsessiones per multa aliquando durare tempora , } non est possibile obsidentes semper esse paratos aeque . Ideo nisi sint muniti , \\\hline
3.3.17 & nin de vna manera . \textbf{ Et por ende si non estudieren guarnesçidos puede les contesçer } que los que estan en los castiellos o en las çibdades cercadas & Nam cum contingat obsessiones per multa aliquando durare tempora , \textbf{ non est possibile obsidentes semper esse paratos aeque . Ideo nisi sint muniti , } contingit quod existentes in castris \\\hline
3.3.17 & quando los que çercan durmieren \textbf{ o comieren o estudieren de vagar } o fueren derramados & ( cum fuerint occupati obsidentes somno , \textbf{ vel ludo , | vel ocio , } aut aliqua necessitate dispersi ) repente prorumpunt in ipsos , et succedunt tentoria , \\\hline
3.3.17 & o fueren derramados \textbf{ por alguna neçessidat a desora pueden dar en ellos } e quemarles las tiendas & vel ocio , \textbf{ aut aliqua necessitate dispersi ) repente prorumpunt in ipsos , et succedunt tentoria , } destruunt obsidentium machinas , \\\hline
3.3.17 & por alguna neçessidat a desora pueden dar en ellos \textbf{ e quemarles las tiendas } e destruyr les los engeñios & aut aliqua necessitate dispersi ) repente prorumpunt in ipsos , et succedunt tentoria , \textbf{ destruunt obsidentium machinas , } et aliquando multi ex obsidentibus perdunt . Quare obsidentes \\\hline
3.3.17 & e quemarles las tiendas \textbf{ e destruyr les los engeñios } e a las vezes matar muchas de los cercadores & aut aliqua necessitate dispersi ) repente prorumpunt in ipsos , et succedunt tentoria , \textbf{ destruunt obsidentium machinas , } et aliquando multi ex obsidentibus perdunt . Quare obsidentes \\\hline
3.3.17 & e destruyr les los engeñios \textbf{ e a las vezes matar muchas de los cercadores } por la qual cosa los que çercan & destruunt obsidentium machinas , \textbf{ et aliquando multi ex obsidentibus perdunt . Quare obsidentes } ut tuti permaneant , \\\hline
3.3.17 & por la qual cosa los que çercan \textbf{ por que puedan estar seguros deuen fincar las tiendas } e el real alueñe de la çibdat & et aliquando multi ex obsidentibus perdunt . Quare obsidentes \textbf{ ut tuti permaneant , } longe a munitione obsessa saltem per ictum teli vel iaculi debent castrametari , et circa se facere fossas , \\\hline
3.3.17 & o del castiello cercado \textbf{ quanto podrie lançar la vallesta o el dardo } e fazer carcauas enderredor de ssi e finçar y grandes palos & ut tuti permaneant , \textbf{ longe a munitione obsessa saltem per ictum teli vel iaculi debent castrametari , et circa se facere fossas , } et figere ibi ligna , \\\hline
3.3.17 & quanto podrie lançar la vallesta o el dardo \textbf{ e fazer carcauas enderredor de ssi e finçar y grandes palos } e fazer algunas fortalezas & longe a munitione obsessa saltem per ictum teli vel iaculi debent castrametari , et circa se facere fossas , \textbf{ et figere ibi ligna , } et construere propugnacula : \\\hline
3.3.17 & e fazer carcauas enderredor de ssi e finçar y grandes palos \textbf{ e fazer algunas fortalezas } assi que si los que estan çercados a desora los quisieren acometer & et figere ibi ligna , \textbf{ et construere propugnacula : } ut si oppidani eos repente vellent inuadere , \\\hline
3.3.17 & e fazer algunas fortalezas \textbf{ assi que si los que estan çercados a desora los quisieren acometer } fallen enbargo & et construere propugnacula : \textbf{ ut si oppidani eos repente vellent inuadere , } resistentiam inuenirent . \\\hline
3.3.17 & fallen enbargo \textbf{ por que los non puedan enpesçer . } visto en qual manera se deuen guarnesçer los çercadores & ut si oppidani eos repente vellent inuadere , \textbf{ resistentiam inuenirent . } Viso quomodo se munire debent obsidentes , \\\hline
3.3.17 & por que los non puedan enpesçer . \textbf{ visto en qual manera se deuen guarnesçer los çercadores } por que non pueda resçebir daño de los cercados & resistentiam inuenirent . \textbf{ Viso quomodo se munire debent obsidentes , } ne ab oppidanis molestentur : \\\hline
3.3.17 & visto en qual manera se deuen guarnesçer los çercadores \textbf{ por que non pueda resçebir daño de los cercados } finca de demostrar & Viso quomodo se munire debent obsidentes , \textbf{ ne ab oppidanis molestentur : } restat ostendere quot modis impugnare debent obsessos . \\\hline
3.3.17 & por que non pueda resçebir daño de los cercados \textbf{ finca de demostrar } en quantas maneras se deuen acometer & ne ab oppidanis molestentur : \textbf{ restat ostendere quot modis impugnare debent obsessos . } Est autem unus modus impugnandi communis et publicus , \\\hline
3.3.17 & finca de demostrar \textbf{ en quantas maneras se deuen acometer } los que estan cercados & ne ab oppidanis molestentur : \textbf{ restat ostendere quot modis impugnare debent obsessos . } Est autem unus modus impugnandi communis et publicus , \\\hline
3.3.17 & los que estan cercados \textbf{ et ay vna manera comun e publica de acometer } e de lidiar & restat ostendere quot modis impugnare debent obsessos . \textbf{ Est autem unus modus impugnandi communis et publicus , } videlicet , \\\hline
3.3.17 & et ay vna manera comun e publica de acometer \textbf{ e de lidiar } contra los que estan cercados & Est autem unus modus impugnandi communis et publicus , \textbf{ videlicet , } per ballistas , arcus , \\\hline
3.3.17 & e ponen escaleras a los muros \textbf{ assi que si podieren sobir sean eguales dellos } para se dar con ellos & et arcubus : iaciunt contra ipsos lapides cum manibus vel cum fundis ; apponunt scalas ad muros , \textbf{ ut si possint ascendere ad partes illas . } Praeter tamen hos modos impugnationis apertos , \\\hline
3.3.17 & assi que si podieren sobir sean eguales dellos \textbf{ para se dar con ellos } e para entrar los . & et arcubus : iaciunt contra ipsos lapides cum manibus vel cum fundis ; apponunt scalas ad muros , \textbf{ ut si possint ascendere ad partes illas . } Praeter tamen hos modos impugnationis apertos , \\\hline
3.3.17 & para se dar con ellos \textbf{ e para entrar los . } Enpero sin estas maneras manifiestas de batalla & ut si possint ascendere ad partes illas . \textbf{ Praeter tamen hos modos impugnationis apertos , } est dare triplicem impugnationis modum non omnibus notum . \\\hline
3.3.17 & fechas \textbf{ por arte que se pueden llegar a los muros e a las cercas . } Pues que assi es de todas estas maneras de conbatir & Et tertius per aedificia impulsa usque \textbf{ ad muros munitionis obsessae . } De omnibus itaque his impugnationibus dicemus , \\\hline
3.3.17 & por arte que se pueden llegar a los muros e a las cercas . \textbf{ Pues que assi es de todas estas maneras de conbatir } diremos & ad muros munitionis obsessae . \textbf{ De omnibus itaque his impugnationibus dicemus , } sed primo de impugnatione per cuniculos . \\\hline
3.3.17 & de cada vna a su parte . \textbf{ Mas lo primero de aquella manera de acometer } que es por las cueuas conegeras . & De omnibus itaque his impugnationibus dicemus , \textbf{ sed primo de impugnatione per cuniculos . } Primo igitur per cuniculos , \\\hline
3.3.17 & Et pues que assi es lo primero \textbf{ por cueuas ssoterrañas se pueden vençer } e tomar las fortalezas . & Primo igitur per cuniculos , \textbf{ id est per vias subterraneas deuincuntur munitiones . } Debent enim obsidentes priuatim in aliquo loco terram fodere : \\\hline
3.3.17 & por cueuas ssoterrañas se pueden vençer \textbf{ e tomar las fortalezas . } Ca deuen los que çercan en grand poridat cauar la tierra en algun logar conenible & Primo igitur per cuniculos , \textbf{ id est per vias subterraneas deuincuntur munitiones . } Debent enim obsidentes priuatim in aliquo loco terram fodere : \\\hline
3.3.17 & e tomar las fortalezas . \textbf{ Ca deuen los que çercan en grand poridat cauar la tierra en algun logar conenible } ante el qual logar deuen poner algunan tienda o alguna choça & id est per vias subterraneas deuincuntur munitiones . \textbf{ Debent enim obsidentes priuatim in aliquo loco terram fodere : | ante quem locum , } tentorium \\\hline
3.3.17 & Ca deuen los que çercan en grand poridat cauar la tierra en algun logar conenible \textbf{ ante el qual logar deuen poner algunan tienda o alguna choça } por que non puedan veer los que estan cercados de comiençan a cauar . & ante quem locum , \textbf{ tentorium | vel aliquod aliud aedificium debent apponere , } ne obsessi videre possint \\\hline
3.3.17 & ante el qual logar deuen poner algunan tienda o alguna choça \textbf{ por que non puedan veer los que estan cercados de comiençan a cauar . } Ca cauando ally e faziendo cauas soterrañas & vel aliquod aliud aedificium debent apponere , \textbf{ ne obsessi videre possint | ubi incipiant fodere : } ibi enim faciendo vias subterraneas sicut faciunt fodientes argentum \\\hline
3.3.17 & assi commo cauan los que buscan la plata \textbf{ e los que buscan las maneras de los metales deuen yr } por aquellas carreras soterrañas faziendolas toda via mayores & ibi enim faciendo vias subterraneas sicut faciunt fodientes argentum \textbf{ et inuenientes venas metallorum , debent per vias illas , } faciendo eas profundiores , \\\hline
3.3.17 & que esta cercada . \textbf{ Et assi deuen yr fasta los muros de aquel logar . } Et si esto se puede fazer ligera cosa es de tomar aquella fortaleza o aquel logar & quam sint fossae munitionis deuincendae , pergere usque ad muros munitionis praedictae : \textbf{ quod } si hoc fieri potest , leue est munitionem capere . \\\hline
3.3.17 & Et assi deuen yr fasta los muros de aquel logar . \textbf{ Et si esto se puede fazer ligera cosa es de tomar aquella fortaleza o aquel logar } ca esto fecho primero deuen socauar los muros & quod \textbf{ si hoc fieri potest , leue est munitionem capere . } Nam hoc facto primo debent muros fodere , \\\hline
3.3.17 & Et si esto se puede fazer ligera cosa es de tomar aquella fortaleza o aquel logar \textbf{ ca esto fecho primero deuen socauar los muros } e so poner y maderos & si hoc fieri potest , leue est munitionem capere . \textbf{ Nam hoc facto primo debent muros fodere , } et supponere ibi ligna \\\hline
3.3.17 & ca esto fecho primero deuen socauar los muros \textbf{ e so poner y maderos } por que non puedan luego caer & Nam hoc facto primo debent muros fodere , \textbf{ et supponere ibi ligna } ne statim cadant . \\\hline
3.3.17 & e so poner y maderos \textbf{ por que non puedan luego caer } nin fazer daño & et supponere ibi ligna \textbf{ ne statim cadant . } Et cum omnes muros , \\\hline
3.3.17 & por que non puedan luego caer \textbf{ nin fazer daño } a los que cauan & ne statim cadant . \textbf{ Et cum omnes muros , } vel maximam partem murorum sic suffosserunt \\\hline
3.3.17 & si vieren los que çercan \textbf{ que cayendo los muros pueden luego tomar el logar } luego sin detenemiento ninguno deuen poner fuego en la madera & si viderint obsidentes \textbf{ quod per solum casum murorum possint munitionem obtinere , } statim debent apponere ignem in lignis sustinentibus muros \\\hline
3.3.17 & que cayendo los muros pueden luego tomar el logar \textbf{ luego sin detenemiento ninguno deuen poner fuego en la madera } que sotienen los muros & quod per solum casum murorum possint munitionem obtinere , \textbf{ statim debent apponere ignem in lignis sustinentibus muros } et facere omnes muros \\\hline
3.3.17 & que sotienen los muros \textbf{ e fazer } que todos los muros o grand parte dellos cayan en vno a desora . & statim debent apponere ignem in lignis sustinentibus muros \textbf{ et facere omnes muros } vel facere magnam eorum partem cadere , \\\hline
3.3.17 & que todos los muros o grand parte dellos cayan en vno a desora . \textbf{ Otrossi deuen fenchir las carcauas } assi que los que estan cercados a desora sean espantados & vel facere magnam eorum partem cadere , \textbf{ et replere fossas : } quo simul \\\hline
3.3.17 & por que mas ligeramente sea tomada la fortaleza . \textbf{ Mas avn conuiene aqui de saber } que las cueuas soterrañas deuen ser sotenidas de tablas & ut facilius deuincatur oppidum . Est \textbf{ etiam attendendum } quod viae subterraneae semper sunt muniendae tabulis \\\hline
3.3.17 & Et avn la tierra \textbf{ que sacan de aquellas cueuas es de asconder en tal manera por que non la vean los que estan cercados . } Otrossi quando se pone el fuego a la madera & ne cadat terra et suffocet fodientes : \textbf{ terra etiam quae extrahitur de dictis fossis est taliter abscondenda , | ne videatur ab obsessis . } Et rursus cum ignis apponitur ad ipsa ligna sustinentia murum , \\\hline
3.3.17 & el que pone el fuego \textbf{ e los que estan con el deuen se poner en saluo } por que non los maten los muros & apponens huiusmodi ignem \textbf{ et existentes | cum eo debent ad locum tutum fugere , } ne laedantur per murorum casum . \\\hline
3.3.17 & pues que assi es \textbf{ assi auemos de fazer en este conbatemiento } que es por cueuas conegeras & ne laedantur per murorum casum . \textbf{ Sic ergo agendum est in impugnatione per cuniculos , } cum ad munitionem obtinendam sufficit sola murorum ruina . \\\hline
3.3.17 & que es por cueuas conegeras \textbf{ e esto quanto para ganar la fortaleza } cunple la cayda de los muros . & Sic ergo agendum est in impugnatione per cuniculos , \textbf{ cum ad munitionem obtinendam sufficit sola murorum ruina . } Sed cum hoc creditur non sufficere , \\\hline
3.3.17 & cunple la cayda de los muros . \textbf{ Mas quando cuydan que non pueden entrar el logar } estando socauados los muros & cum ad munitionem obtinendam sufficit sola murorum ruina . \textbf{ Sed cum hoc creditur non sufficere , } muris existentibus subfossis \\\hline
3.3.17 & estando socauados los muros \textbf{ e sopuestos non deuen luego poner el fuego mas deuen yr so tierra mas adelante } a las mayores fortalezas & muris existentibus subfossis \textbf{ et subpunctatis , | nondum apponendus est ignis , } sed procedendum est \\\hline
3.3.17 & e a los mas fuertes adarues del castielloo de la çibdat çercada . \textbf{ Et por cueuas deuen venir } fasta que entiendan & sed procedendum est \textbf{ ad maiores munitiones et ad maiora moenia castri , vel ciuitatis obsessae , } et per similes vias subterraneas est similiter faciendum circa ea , \\\hline
3.3.17 & fasta que entiendan \textbf{ que poniendo fuego pueden caer las fortalezas } assi commo dicho es de los muros . & ad maiores munitiones et ad maiora moenia castri , vel ciuitatis obsessae , \textbf{ et per similes vias subterraneas est similiter faciendum circa ea , } quod factum est circa muros . \\\hline
3.3.17 & assi commo dicho es de los muros . \textbf{ Otrossi deuen yr so a tierra partiendose a muchas partes } por las cueuas soterrañas & quod factum est circa muros . \textbf{ Rursus procedendum est diuertendo vias subterraneas , } ut per eas possit haberi ingressus \\\hline
3.3.17 & por las cueuas soterrañas \textbf{ assi que por ellas puedan entrar a la çibdat o al castiello . } Et estas cosas todas deuense fazer muy encubiertamente & Rursus procedendum est diuertendo vias subterraneas , \textbf{ ut per eas possit haberi ingressus | ad ciuitatem et castrum : } quae omnia latenter fieri possunt absque eo quod sentiantur ab obsessis : \\\hline
3.3.17 & assi que por ellas puedan entrar a la çibdat o al castiello . \textbf{ Et estas cosas todas deuense fazer muy encubiertamente } por que non lo sepan & ad ciuitatem et castrum : \textbf{ quae omnia latenter fieri possunt absque eo quod sentiantur ab obsessis : } licet tamen sine difficultate \\\hline
3.3.17 & nin lo sientan los cercados . \textbf{ Et maguera que todas estas cosas non se puedan fazer sin muy grant graueza } e sin luengo tienpo . & quae omnia latenter fieri possunt absque eo quod sentiantur ab obsessis : \textbf{ licet tamen sine difficultate } et diuturnitate temporis , \\\hline
3.3.17 & e sin luengo tienpo . \textbf{ Empero e deuen las prouarlos omnes . } Pues que assi es estas cosas & et diuturnitate temporis , \textbf{ possint | haec omnia fini debito mancipari . } His itaque sic peractis in aliquo nocturno tempore , \\\hline
3.3.17 & o en otro tienpo conuenible \textbf{ a esta manera de conbatir } deue ser puesto el fuego & His itaque sic peractis in aliquo nocturno tempore , \textbf{ vel in aliquo alio congruo ad pugnandum per appositionem ignis fieri debet , } ut simul cadant muri \\\hline
3.3.17 & en la çibdato en el castiello çercado \textbf{ e assi podran ganar aquellas fortalezas . } m muchas uegadas contesçe & vel ciuitatem obsessam : \textbf{ et sic poterunt obtinere illam . } Contingit autem pluries , \\\hline
3.3.18 & assi que por cueuas conegeras \textbf{ o por cueuas soterranas nunca o con muy grand trabaio se pueden tomar . } Et avn acaesçe muchas vezes & ut per viculos \textbf{ et per subterraneas vias nunquam , | vel valde de difficili obtineri possint . Euenit } etiam pluries \\\hline
3.3.18 & Et avn acaesçe muchas vezes \textbf{ que si la fortaleza cercada se puede tomar } por cueuas soterrañas . & etiam pluries \textbf{ ut si munitio obsessa per vias subterraneas capi possit , } obsessi tamen prouidentes fossionem impediunt eam , \\\hline
3.3.18 & enpero los çercados veyendo \textbf{ que los pueden entrar } socauando los enbargan el cauar & ut si munitio obsessa per vias subterraneas capi possit , \textbf{ obsessi tamen prouidentes fossionem impediunt eam , } ne per ipsam fraudulenter \\\hline
3.3.18 & que los pueden entrar \textbf{ socauando los enbargan el cauar } assi que por tales cueuas non se pueda tomar la fortaleza & obsessi tamen prouidentes fossionem impediunt eam , \textbf{ ne per ipsam fraudulenter } et per insidias deuincantur . \\\hline
3.3.18 & socauando los enbargan el cauar \textbf{ assi que por tales cueuas non se pueda tomar la fortaleza } nin por engaño & obsessi tamen prouidentes fossionem impediunt eam , \textbf{ ne per ipsam fraudulenter } et per insidias deuincantur . \\\hline
3.3.18 & nin por engaño \textbf{ nin por encubierta nin assecha . Et esto commo se ha de fazer } mostrar lo hemos en los capitulos & ne per ipsam fraudulenter \textbf{ et per insidias deuincantur . } Quod quomodo fieri habeat , ostendemus in sequentibus capitulis , \\\hline
3.3.18 & nin por encubierta nin assecha . Et esto commo se ha de fazer \textbf{ mostrar lo hemos en los capitulos } que se siguen & et per insidias deuincantur . \textbf{ Quod quomodo fieri habeat , ostendemus in sequentibus capitulis , } ubi agetur de defensiua pugna . \\\hline
3.3.18 & Ca quando tractaremos \textbf{ en qual manera se deuen auer los que cercan } declararemos & ubi agetur de defensiua pugna . \textbf{ Cum enim tractabimus qualiter obsessi se defendere debeant , declarabitur qualiter obsessi } per cuniculos \\\hline
3.3.18 & declararemos \textbf{ en qual manera los cercados deuen ver } e guardarsse de las cueuas conegeras & ø \\\hline
3.3.18 & en qual manera los cercados deuen ver \textbf{ e guardarsse de las cueuas conegeras } e de las & Cum enim tractabimus qualiter obsessi se defendere debeant , declarabitur qualiter obsessi \textbf{ per cuniculos } et alia machinamenta obsidentium debeant prouidere . \\\hline
3.3.18 & e de las otras algarradas o engeñios \textbf{ que les pueden poner } los que cercan o encubiertas quales quier . & per cuniculos \textbf{ et alia machinamenta obsidentium debeant prouidere . } Quare si modus artis debet imitari naturam quae semper faciliori via res ad effectum producit : \\\hline
3.3.18 & Por la qual cosa \textbf{ si la manera del arte deue semeiar a la natura . } la qual natura sienpre aduze & et alia machinamenta obsidentium debeant prouidere . \textbf{ Quare si modus artis debet imitari naturam quae semper faciliori via res ad effectum producit : } cum per viculos non ita de facili munitiones impugnari possunt , \\\hline
3.3.18 & por el mas ligero camino que puede las cosas a su fin . \textbf{ Commo por cueuas conegeras non puedan tan de ligero tomar las fortalezas } commo se pueden tomar & Quare si modus artis debet imitari naturam quae semper faciliori via res ad effectum producit : \textbf{ cum per viculos non ita de facili munitiones impugnari possunt , } sicut per machinas lapidarias , \\\hline
3.3.18 & Commo por cueuas conegeras non puedan tan de ligero tomar las fortalezas \textbf{ commo se pueden tomar } por los engeñios & cum per viculos non ita de facili munitiones impugnari possunt , \textbf{ sicut per machinas lapidarias , } vel per aedificia propulsa usque ad moenia castri , \\\hline
3.3.18 & que lançan piedras o por castiellos \textbf{ que se pueden enpuxar } fasta las menas del castiello o de la çibdat cercada . & sicut per machinas lapidarias , \textbf{ vel per aedificia propulsa usque ad moenia castri , } vel ciuitatis obsessae , \\\hline
3.3.18 & fasta las menas del castiello o de la çibdat cercada . \textbf{ Conuiene de vsar de tales armadijas o de tales armamientos } por que puedan ganar el logar & vel ciuitatis obsessae , \textbf{ oportet talibus uti argumentis } ut habeatur intentum . \\\hline
3.3.18 & Conuiene de vsar de tales armadijas o de tales armamientos \textbf{ por que puedan ganar el logar } e auer & vel ciuitatis obsessae , \textbf{ oportet talibus uti argumentis } ut habeatur intentum . \\\hline
3.3.18 & por que puedan ganar el logar \textbf{ e auer } lo que entienden . & oportet talibus uti argumentis \textbf{ ut habeatur intentum . } Videndum est igitur , \\\hline
3.3.18 & lo que entienden . \textbf{ Et pues que assi es conuiene de ver e de saber } quantas son las maneras de los engennios & ut habeatur intentum . \textbf{ Videndum est igitur , } quot sunt genera machinarum lapidariarum , \\\hline
3.3.18 & e quantas son las maneras de los artifiçios \textbf{ por los quales se pueden ganar las fortalezas . } Mas los engeñios & et quot sunt modi aedificiorum \textbf{ per quae munitiones impugnantur . } Machinae autem lapidariae \\\hline
3.3.18 & Mas los engeñios \textbf{ que lançan las piedras puedense adozer a quatro maneras . } Ca en todo tal engeñio es de dar alguna cosa & Machinae autem lapidariae \textbf{ quasi ad quatuor genera reducuntur . } Nam in omni tali machina est dare aliquid trahens \\\hline
3.3.18 & que lançan las piedras puedense adozer a quatro maneras . \textbf{ Ca en todo tal engeñio es de dar alguna cosa } que traya & quasi ad quatuor genera reducuntur . \textbf{ Nam in omni tali machina est dare aliquid trahens } et eleuans virgam machinae , \\\hline
3.3.18 & o de algun otro cuerpo mas pesado . \textbf{ la qual manera de engeñio los vieios quisieron llamar trabuquete } Mas entre todos los engeñios este es el que mas derechamente arroia e alanca & vel aliquo alio graui corpore : \textbf{ quod genus machinae veteres Trabutium vocare voluerunt . } Inter ceteras autem machinas haec rectius proiicit , \\\hline
3.3.18 & e por ende sienpre en vna manera enbia la piedra . \textbf{ e con este engeñio pueden ferir tan açierto } commo si lançassen el aguia . & ideo semper eodem modo impellit , \textbf{ cum hac enim machina } quasi acus percuti posset . \\\hline
3.3.18 & commo si lançassen el aguia . \textbf{ Ca quando quieren lançar con el engenio a alguna señal } si lançan mucho a la mano & quasi acus percuti posset . \textbf{ Nam | cum aliquod signum percutiendum est per ipsam , } si nimis proiicit ad dextram , \\\hline
3.3.18 & si lançan mucho a la mano \textbf{ derechao a la esquierda deuen tornar el engemio a aquel logar } a que quieren lançar la piedra . & si nimis proiicit ad dextram , \textbf{ vel ad sinistram vertenda est ad locum erga quod iaciendus est lapis . } Si vero nimis alte proiicit , \\\hline
3.3.18 & derechao a la esquierda deuen tornar el engemio a aquel logar \textbf{ a que quieren lançar la piedra . } mas si lança muy alto & si nimis proiicit ad dextram , \textbf{ vel ad sinistram vertenda est ad locum erga quod iaciendus est lapis . } Si vero nimis alte proiicit , \\\hline
3.3.18 & mas si lança muy alto \textbf{ o es de alongar el engeñio de la señal } en que quiere feriro & Si vero nimis alte proiicit , \textbf{ vel elonganda est machina a signo , } vel in funda eius apponendus est lapis grauior , \\\hline
3.3.18 & en que quiere feriro \textbf{ es de poner en la fonda del piedra } mas pesada la qual non pueda tanto leuantar . Mas si el engeñio lançare muy baxo es de açercar el engeñio . & vel elonganda est machina a signo , \textbf{ vel in funda eius apponendus est lapis grauior , } quem non tantum eleuare poterit . \\\hline
3.3.18 & es de poner en la fonda del piedra \textbf{ mas pesada la qual non pueda tanto leuantar . Mas si el engeñio lançare muy baxo es de açercar el engeñio . } o es de aliuiar la piedra & vel in funda eius apponendus est lapis grauior , \textbf{ quem non tantum eleuare poterit . | Si vero nimis ime vel nimis basse , } appropinquanda est machina , \\\hline
3.3.18 & mas pesada la qual non pueda tanto leuantar . Mas si el engeñio lançare muy baxo es de açercar el engeñio . \textbf{ o es de aliuiar la piedra } que sea mas ligera . & appropinquanda est machina , \textbf{ vel alleuiandus est lapis . } Semper enim ponderandi sunt lapides ipsarum machinarum , \\\hline
3.3.18 & ca sienpre deuen ser pesadas las piedras de los engeñios \textbf{ si han de ferir determinadamente en alguna señal . } Et ay otra manera de engeñio & Semper enim ponderandi sunt lapides ipsarum machinarum , \textbf{ si determinate proiiciendum fit ad aliquod certum signum . } Aliud genus machinarum habet contrapondus mobiliter adhaerens circa flagellum , \\\hline
3.3.18 & Enpero non es menester tanto tienpo \textbf{ para armar este engeñio } commo en los otros tres sobredichos . & sicut praedicta tria genera machinarum : \textbf{ tamen non oportet tantum tempus apponere ad proportionandum huiusmodi machinam , } sicut in machinis praefatis , \\\hline
3.3.18 & assi es aquel que çerca algun castiello o alguna çibdat \textbf{ si la quiere tomar } por engeñios de piedras deue penssar con grand acuçia & aut ciuitatem aliquam , \textbf{ si vult eam impugnare per machinas lapidarias , } diligenter considerare debet , \\\hline
3.3.18 & si la quiere tomar \textbf{ por engeñios de piedras deue penssar con grand acuçia } si puede meior conbatir aquella fortaleza lançando derechamente o lançando & aut ciuitatem aliquam , \textbf{ si vult eam impugnare per machinas lapidarias , } diligenter considerare debet , \\\hline
3.3.18 & por engeñios de piedras deue penssar con grand acuçia \textbf{ si puede meior conbatir aquella fortaleza lançando derechamente o lançando } mas alueñe o en manera medianera entre estas dos & ø \\\hline
3.3.18 & mas alueñe o en manera medianera entre estas dos \textbf{ o deue avn penssar } si mas puede confonder & si vult eam impugnare per machinas lapidarias , \textbf{ diligenter considerare debet , } utrum magis possit munitionem illam impugnare proiiciendo rectius \\\hline
3.3.18 & o deue avn penssar \textbf{ si mas puede confonder } e dannarlos cercandos lançando amenudo & diligenter considerare debet , \textbf{ utrum magis possit munitionem illam impugnare proiiciendo rectius } vel longius , \\\hline
3.3.18 & si mas puede confonder \textbf{ e dannarlos cercandos lançando amenudo } e muchas vezes . & diligenter considerare debet , \textbf{ utrum magis possit munitionem illam impugnare proiiciendo rectius } vel longius , \\\hline
3.3.18 & que mas conuiene en todas aquellas maneras sobredichas de engeñios . \textbf{ o en todas aquellas maneras de lançar } que dichas son & vel medio modo \textbf{ inter utrunque } vel etiam magis posset obsessos offendere proiiciendo spissius et frequentius . \\\hline
3.3.18 & que dichas son \textbf{ o en algunas o en alguna dellas podra acomter el castiello } o la çibdat cercada . & inter utrunque \textbf{ vel etiam magis posset obsessos offendere proiiciendo spissius et frequentius . } Nam prout viderit expedire omnibus praedictis machinis , \\\hline
3.3.18 & en qual manera \textbf{ por los engenmos que lançan piedras se puede conbatir } e ganar toda fortaleza . & castrum , \textbf{ vel ciuitatem obsessam poterit impugnare . } Si enim plena notitia habeatur de machinis , \\\hline
3.3.18 & por los engenmos que lançan piedras se puede conbatir \textbf{ e ganar toda fortaleza . } Ca toda manera de engeñio & vel ciuitatem obsessam poterit impugnare . \textbf{ Si enim plena notitia habeatur de machinis , } de quibus mentionem fecimus , \\\hline
3.3.18 & o es en alguna manera de las que son sobredichas \textbf{ o tomar puede rayz o comienço de aquellas sobredichas . } Et avn conuiene de saber & Nam omne genus machinae lapidariae , \textbf{ vel est aliquod praedictorum , vel potest originem sumere ex praedictis . } Est etiam aduertendum quod die \\\hline
3.3.18 & o tomar puede rayz o comienço de aquellas sobredichas . \textbf{ Et avn conuiene de saber } que tan bien de noche & vel est aliquod praedictorum , vel potest originem sumere ex praedictis . \textbf{ Est etiam aduertendum quod die } et nocte per lapidarias machinas impugnari possunt munitiones obsessae . \\\hline
3.3.18 & que tan bien de noche \textbf{ commo de dia se pueden acometer las fortalezas cercadas } por los engeñios & Est etiam aduertendum quod die \textbf{ et nocte per lapidarias machinas impugnari possunt munitiones obsessae . } Tamen , \\\hline
3.3.18 & que lançan piedras . \textbf{ Empero por que vean en qual manera han de lançar las piedras de noche } que enbian con los engeñios & Tamen , \textbf{ ut videatur qualiter in nocte percutiunt lapides emissi a machinis , } semper cum lapide alligandus est ignis , \\\hline
3.3.18 & que enbian con los engeñios \textbf{ sienpre deuen atar algun fuego } o algun tizon ençendido con la piedra & ut videatur qualiter in nocte percutiunt lapides emissi a machinis , \textbf{ semper cum lapide alligandus est ignis , } vel ticio ignitus . \\\hline
3.3.18 & Ca por el tizon ençendido \textbf{ e atado a la piedra podra paresçer } en qual manera fiere el engeñio & vel ticio ignitus . \textbf{ Nam per ticionem ignitum lapidi alligatum apparere poterit qualiter machina proiicit , } et qualis siue quam ponderosus lapis est in funda machinae imponendus . \\\hline
3.3.18 & en qual manera fiere el engeñio \textbf{ e qual o quant pesada piedra se deue poner en la fonda del engeñio . } t tres maneras de conbatir las fortalezas cercadas fueron puestas dessuso de las quales la vna era por cueuas conegeras & Nam per ticionem ignitum lapidi alligatum apparere poterit qualiter machina proiicit , \textbf{ et qualis siue quam ponderosus lapis est in funda machinae imponendus . } Tangebatur autem supra tres modi impugnandi munitiones obsessas . \\\hline
3.3.19 & e qual o quant pesada piedra se deue poner en la fonda del engeñio . \textbf{ t tres maneras de conbatir las fortalezas cercadas fueron puestas dessuso de las quales la vna era por cueuas conegeras } la otra era por engeñios lançadores de piedras . & et qualis siue quam ponderosus lapis est in funda machinae imponendus . \textbf{ Tangebatur autem supra tres modi impugnandi munitiones obsessas . } Quorum unus erat per cuniculos . Alius per machinas lapidarias . Tertius vero , \\\hline
3.3.19 & o por los engeñios lançadores de piedras \textbf{ fincanos de dezir del acometimiento } que se puede fazer & et per lapidarias machinas ; \textbf{ restat dicere de impugnatione quam fieri contingit } per aedificia impulsa ad muros , \\\hline
3.3.19 & fincanos de dezir del acometimiento \textbf{ que se puede fazer } por los artifiçios de madera enpuxados & et per lapidarias machinas ; \textbf{ restat dicere de impugnatione quam fieri contingit } per aedificia impulsa ad muros , \\\hline
3.3.19 & e allegados a los muros del castiello o de la çibdat . \textbf{ Et estos artifiçios pueden se adozir a quatro maneras . } Conuiene de saber a carneros . & vel ad moenia castri , vel ciuitatis obsessae . \textbf{ Huiusmodi autem aedificia quasi ad quatuor genera reducuntur , } videlicet , ad arietes , vineas , turres , et musculos . \\\hline
3.3.19 & Et estos artifiçios pueden se adozir a quatro maneras . \textbf{ Conuiene de saber a carneros . } Et a vinnas . Et a torres . Et a muslos . & Huiusmodi autem aedificia quasi ad quatuor genera reducuntur , \textbf{ videlicet , ad arietes , vineas , turres , et musculos . } Vocatur enim Aries , testudo quaedam lignorum , quae , \\\hline
3.3.19 & la qual \textbf{ por que la non puedan quemar con fuego } cubrenla con cueros crudos & Vocatur enim Aries , testudo quaedam lignorum , quae , \textbf{ ne igne comburatur , } crudis coriis cooperitur . \\\hline
3.3.19 & que ponen y . ha muy fuerte \textbf{ et muy . dura fruente para ferir } e fazer grant colpe . & quia ratione ferri ibi appositi durissimam habet frontem ad percutiendum . \textbf{ Huiusmodi autem trabs funibus , } vel cathenis ferreis alligatur ad testudinem factam ex lignis , \\\hline
3.3.19 & et muy . dura fruente para ferir \textbf{ e fazer grant colpe . } Et esta viga a tal atanla con cuerdas e con cadenas de fierro a la bouada fecha de madera & Huiusmodi autem trabs funibus , \textbf{ vel cathenis ferreis alligatur ad testudinem factam ex lignis , } et ad modum arietis se subtrahit : \\\hline
3.3.19 & Ca quando con esta viga tal assi ferrada den muchos colpes en el muro \textbf{ assi que ya las piedras que estan en el muro se comiençan a mouer } en la cabeça de la viga fincan vn fierro a manera de foz & et disrumpit . Cum enim per huiusmodi trabem sic ferratam multis ictibus percussus est murus ita , \textbf{ quod iam lapides existentes in ipso incipiant commoueri : } in capite eius infligitur quoddam ferrem retortum ad modum falcis , \\\hline
3.3.19 & por que mas ayna sea foradado . \textbf{ Et uale este artifiçio para acometer alguna fortaleza . } puesto que non puedan llegar a los muros della . & ut citius perforetur . \textbf{ Valet autem huiusmodi aedificium | ad impugnandum munitionem aliquam , } dato \\\hline
3.3.19 & Et uale este artifiçio para acometer alguna fortaleza . \textbf{ puesto que non puedan llegar a los muros della . } Ca por que esta viga ha la cabeça & dato \textbf{ quod quis non possit pertingere usque | ad muros eius . } Nam quia huiusmodi trabs habens caput sic ferratum retrahitur \\\hline
3.3.19 & assi ferrada tiranla afuera \textbf{ e despues enpuxan la assi que puede de lueñe ferir en los muros de la fortaleza cercada } puesto que el techo so que estan los omnes & Nam quia huiusmodi trabs habens caput sic ferratum retrahitur \textbf{ et impingitur , poterit percuti murus ipsius munitionis obsessae , } dato \\\hline
3.3.19 & Mas ay otro artifiçio \textbf{ para acometer las fortalezas çercadas } que le llaman viñas & vel aliquo modo ex aliqua parte possint offendi . \textbf{ Aliud autem aedificium est ad impugnandum munitiones obsessas , } quod vocant Vineam . \\\hline
3.3.19 & por que las piedras \textbf{ que echaren dessuso non puedan quebrantar aquel artifiçio } ea vn cubrenle de cueros crudos & siue fit duplex tabulatum , \textbf{ ne lapides emissi possint tale aedificium frangere . Cooperitur } etiam crudis coriis , \\\hline
3.3.19 & ea vn cubrenle de cueros crudos \textbf{ por que non le puedan quemar . } Et solien fazer este artifiçio de ocho pies en ancho & etiam crudis coriis , \textbf{ ne ab igne possint offendi . Consueuit autem tale aedificium fieri in latitudine octo pedum , } et in longitudine sexdecim : \\\hline
3.3.19 & por que non le puedan quemar . \textbf{ Et solien fazer este artifiçio de ocho pies en ancho } e de seze en luengo . & etiam crudis coriis , \textbf{ ne ab igne possint offendi . Consueuit autem tale aedificium fieri in latitudine octo pedum , } et in longitudine sexdecim : \\\hline
3.3.19 & que los omnes puedan y bien estar . \textbf{ Et este artifiçio de quantidat tantas o avn de mayor es de guarnesçer muy bien de cada parte } e es de enpuxar & in altitudine vero tot pedum , \textbf{ quot homines ibi competenter possunt existere . Huiusmodi enim aedificium tantae quantitatis , } vel etiam maioris , \\\hline
3.3.19 & Et este artifiçio de quantidat tantas o avn de mayor es de guarnesçer muy bien de cada parte \textbf{ e es de enpuxar } fasta los muros de la fortaleza & quot homines ibi competenter possunt existere . Huiusmodi enim aedificium tantae quantitatis , \textbf{ vel etiam maioris , } est optime undique muniendum , et impellendum usque ad muros munitionis obsessae ; \\\hline
3.3.19 & quando tal es la fortaleza \textbf{ a cada que sasta los muros se pueden enpucar } tal artifiçio commo este . & cum talis est munitio obsessa , \textbf{ quod usque ad muros eius potest tale aedificium impelli . } Tertium genus aedificiorum sunt turres vel castra . \\\hline
3.3.19 & tal artifiçio commo este . \textbf{ la terçera manerar de artificio es torres o castiellos . } Ca si las fortalezas cercadas non se pueden tomar par los carneros & quod usque ad muros eius potest tale aedificium impelli . \textbf{ Tertium genus aedificiorum sunt turres vel castra . } Nam si nec per arietes , \\\hline
3.3.19 & la terçera manerar de artificio es torres o castiellos . \textbf{ Ca si las fortalezas cercadas non se pueden tomar par los carneros } nin por las viñas sobredichas deuen tomarla mesura & Tertium genus aedificiorum sunt turres vel castra . \textbf{ Nam si nec per arietes , } nec per vineas capi possunt munitiones obsessae , \\\hline
3.3.19 & Ca si las fortalezas cercadas non se pueden tomar par los carneros \textbf{ nin por las viñas sobredichas deuen tomarla mesura } e el alteza de los muros de aquella fortaleza & Nam si nec per arietes , \textbf{ nec per vineas capi possunt munitiones obsessae , } accipienda est mensura murorum munitionis illius , \\\hline
3.3.19 & e el alteza de los muros de aquella fortaleza \textbf{ e segunt aquella mesma o avn segunt mas asta medida son de fazer las torres } o los castiellos de madera & accipienda est mensura murorum munitionis illius , \textbf{ et | secundum huiusmodi mensuram , } vel etiam \\\hline
3.3.19 & e de qual es quier otras fortalezas \textbf{ que estan en los muros puedenlas entrar . } Ca assi se auran los que estan en los castiellos & et etiam curricularum \textbf{ et propugnaculorum existentium in ipsis : sic se habebunt existentes in castris ad existentes in munitionibus quodammodo , } sicut existentes in munitionibus ad eos , \\\hline
3.3.19 & a aquellos que estan en baxo o en la tierra \textbf{ e assy los puede entrar } Otrossi en estos castielos se ordenan puentes & ø \\\hline
3.3.19 & fasta los muros de la fortaleza cercada \textbf{ mas la alteza de los muros en dos maneras se puede tomar . } Lo primero por sonbra & per quos itur ad muros munitionis obsessae . \textbf{ Altitudo autem murorum dupliciter potest accipi . } Primo per umbram . \\\hline
3.3.19 & Lo primero por sonbra \textbf{ ca vn filo liuiano de çierta medida de palmos o de pies es de atar a la saeta } e deuen lançar la saeta & Primo per umbram . \textbf{ Nam leue filum , | cuius nota sit quantitas , ligandum est ad sagittam , } et proiiciendum usque ad muros munitionis \\\hline
3.3.19 & ca vn filo liuiano de çierta medida de palmos o de pies es de atar a la saeta \textbf{ e deuen lançar la saeta } fasta ençima del muro de la torreo de la çibdat & cuius nota sit quantitas , ligandum est ad sagittam , \textbf{ et proiiciendum usque ad muros munitionis } secundum quantitatem \\\hline
3.3.19 & fasta ençima del muro de la torreo de la çibdat \textbf{ e segunt la longura de aquel filo se puede tomar } la quantidat de la sonbra & et proiiciendum usque ad muros munitionis \textbf{ secundum quantitatem } cuius sciri poterit quantitas umbrae : \\\hline
3.3.19 & del muro \textbf{ es de alçar vn madero en alto } segunt la quantidat de aquella sonbra . & in qua accipitur umbrae quantitas , \textbf{ erigendum est aliquod lignum in altum , faciens tantam umbram ; } et \\\hline
3.3.19 & Ca algunas vezes esta cubierto de nuues . \textbf{ por ende daremos otra manera de tomar el alteza de cada fortaleza } o de qual quier muro tomesse vn madero o vna tabla & sed aliquando tegitur nubibus , \textbf{ dabimus alium modum accipiendi altitudinem cuiuslibet aedificii , } et quorumcumque murorum . \\\hline
3.3.19 & que el muro \textbf{ de que quiere tomar la alteza alleguesse } mas açerca del muro & et si visus eius proceditur magis alte quam sit aedificium \textbf{ cuius est altitudo sumenda , } trahat se magis prope aedificium illud , \\\hline
3.3.19 & que catando por ençima de la tabla vea la mas alta parte del muro \textbf{ ca puedese prouar } por geometria & Nam , \textbf{ ut probari potest geometrice , } quanta erit distantia a capite hominis sic iacentis usque \\\hline
3.3.19 & que assi yaze fasta el pie del muro \textbf{ Et avn puedensse mesurar las altezas de los muros } por las reglas & ad aedificium illud , \textbf{ tanta erit aedificii altitudo . | Possent } etiam mensurae talium altitudinum accipi per regulas traditas in Astrolabio , \\\hline
3.3.19 & mas desto non fazemos fuerça . \textbf{ ca cunple quanto a lo presente tantas cosas dezir desto } quanto cunple a esta materia . & Sed de hoc nobis non sit curae : \textbf{ sufficiat autem de talibus | ad praesens tanta dicere , } quanta sufficiunt ad propositum : \\\hline
3.3.19 & fasta los muros de la fortaleza cercada \textbf{ e pueden por estos muslos allegar los castiellos } fasta la fortaleza que tienen cercada . & ad moenia munitionis obsessae . \textbf{ Potest autem per huiusmodi musculos quasi continuari castra usque ad munitionem obsessam : } quod cum factum est , \\\hline
3.3.19 & Et esto quando assi fuere fecho en tres maneras \textbf{ puede acometer la fortaleza . } ca que el castiello assi fecho & quod cum factum est , \textbf{ tripliciter impugnanda est munitio . } Nam in castro sic aedificato ad munitionem impugnandam , \\\hline
3.3.19 & ca que el castiello assi fecho \textbf{ para conbatir la fortaleza } deuemos penssar tres cosas & tripliciter impugnanda est munitio . \textbf{ Nam in castro sic aedificato ad munitionem impugnandam , } est tria considerare , \\\hline
3.3.19 & para conbatir la fortaleza \textbf{ deuemos penssar tres cosas } Conuiene de saber la parte del castiello mas alta & Nam in castro sic aedificato ad munitionem impugnandam , \textbf{ est tria considerare , } videlicet partem superiorum excedentem muros ; \\\hline
3.3.19 & deuemos penssar tres cosas \textbf{ Conuiene de saber la parte del castiello mas alta } que los muros e los cadahalsos et las torrezilas de la fortaleza que quieren tomar . & est tria considerare , \textbf{ videlicet partem superiorum excedentem muros ; } et curriculas munitionis capiendae partem quasi mediam , ad quam applicantur pontes cadendi super illos muros : et partem infimam , ad quam applicantur musculi , a quibus sunt homines trahentes , vel impellentes castrum . \\\hline
3.3.19 & Conuiene de saber la parte del castiello mas alta \textbf{ que los muros e los cadahalsos et las torrezilas de la fortaleza que quieren tomar . } Et la parte medianera & videlicet partem superiorum excedentem muros ; \textbf{ et curriculas munitionis capiendae partem quasi mediam , ad quam applicantur pontes cadendi super illos muros : et partem infimam , ad quam applicantur musculi , a quibus sunt homines trahentes , vel impellentes castrum . } Cum ergo castrum illud appropinquauit \\\hline
3.3.19 & Et la parte medianera \textbf{ a la qual quieren echar la puente sobre los muros . } Et la parte mas postrimera & Cum ergo castrum illud appropinquauit \textbf{ quantum debuit ad muros munitionis obsessae , } illi qui sunt in parte superiori debent proiicere lapides , \\\hline
3.3.19 & e mas baxa \textbf{ a la qual se han de llegar los muslos } so los quales estan los omnes & et fugare eos , \textbf{ qui sunt in muris . } Qui vero sunt in parte intermedia , \\\hline
3.3.19 & quanto deuen a los muros de la fortaleza los cercados \textbf{ aquellos que estan en la parte mas alta deuen lançar piedras e fazer foyr } los que estan en los muros & Qui vero sunt in parte intermedia , \textbf{ debent pontes dimittere , } et inuadere muros . \\\hline
3.3.19 & los que estan en los muros \textbf{ mas los que estan en el soberado de medio deuen echar puentes e acometer por los adarues . } mas los que estan en la parte mas baxa & debent pontes dimittere , \textbf{ et inuadere muros . } Sed qui sunt in parte infima et sub musculis , \\\hline
3.3.19 & e so los muslos \textbf{ si pudieren llegar a los muros deuen cauar los } e foradar los & Sed qui sunt in parte infima et sub musculis , \textbf{ si possunt debent ad muros attendere , } et eos suffodere , \\\hline
3.3.19 & si pudieren llegar a los muros deuen cauar los \textbf{ e foradar los } por que puedan entrar en la fortaleza cercada & si possunt debent ad muros attendere , \textbf{ et eos suffodere , | ut etiam } et sic obsidentes intrare possint obsessam munitionem . Sunt etiam , ballistae arcus , \\\hline
3.3.19 & e foradar los \textbf{ por que puedan entrar en la fortaleza cercada } Avn son menester ballestas e arcos e engeñios de algarradas & ut etiam \textbf{ et sic obsidentes intrare possint obsessam munitionem . Sunt etiam , ballistae arcus , } machinae lapidariae , et omnia talia congreganda : \\\hline
3.3.19 & Avn son menester ballestas e arcos e engeñios de algarradas \textbf{ e todas las cosas tales son de allegar } por que quando todo fuere aperçebido . & et sic obsidentes intrare possint obsessam munitionem . Sunt etiam , ballistae arcus , \textbf{ machinae lapidariae , et omnia talia congreganda : } ut quando haec fienda sunt tunc munitionem percutiant . \\\hline
3.3.19 & por que quando todo fuere aperçebido . \textbf{ aquella ora deuen acometer reziamente la fortaleza . } Ca quanto en mas maneras se acomete la fortaleza & machinae lapidariae , et omnia talia congreganda : \textbf{ ut quando haec fienda sunt tunc munitionem percutiant . } Quanto enim pluribus modis simul munitio impugnatur , \\\hline
3.3.20 & que los que çercan las fortalezas e los castiellos \textbf{ en qual manera los deuen çercar } e los deuen conbatir . & et castra qualiter debeant \textbf{ ea obsidere , } et debellare . \\\hline
3.3.20 & en qual manera los deuen çercar \textbf{ e los deuen conbatir . } En esta parte queremos determinar de la batalla defenssiua & ea obsidere , \textbf{ et debellare . } In parte ista determinare volumus de bello defensiuo : \\\hline
3.3.20 & e los deuen conbatir . \textbf{ En esta parte queremos determinar de la batalla defenssiua } que es para se defender los çercados . & et debellare . \textbf{ In parte ista determinare volumus de bello defensiuo : } ut postquam docuimus obsidentes qualiter debeant inuadere obsessos , \\\hline
3.3.20 & En esta parte queremos determinar de la batalla defenssiua \textbf{ que es para se defender los çercados . } assi que despues que ensseñamos & et debellare . \textbf{ In parte ista determinare volumus de bello defensiuo : } ut postquam docuimus obsidentes qualiter debeant inuadere obsessos , \\\hline
3.3.20 & en qual manera \textbf{ los que çercan deuen acometer los cercados } queremos ensseñar & In parte ista determinare volumus de bello defensiuo : \textbf{ ut postquam docuimus obsidentes qualiter debeant inuadere obsessos , } volumus docere ipsos obsessos qualiter se debeant defendere ab obsidentibus . Primum autem quod maxime facit \\\hline
3.3.20 & los que çercan deuen acometer los cercados \textbf{ queremos ensseñar } en qual lomanera los cercados se deuen defender & In parte ista determinare volumus de bello defensiuo : \textbf{ ut postquam docuimus obsidentes qualiter debeant inuadere obsessos , } volumus docere ipsos obsessos qualiter se debeant defendere ab obsidentibus . Primum autem quod maxime facit \\\hline
3.3.20 & queremos ensseñar \textbf{ en qual lomanera los cercados se deuen defender } de los que çercan . & ut postquam docuimus obsidentes qualiter debeant inuadere obsessos , \textbf{ volumus docere ipsos obsessos qualiter se debeant defendere ab obsidentibus . Primum autem quod maxime facit } ne obsessa ciuitas deuincatur ab obsidentibus , \\\hline
3.3.20 & e lo que mas faze \textbf{ para que los çercados ligeramente puedan defender las fortalezas } e saber en qual manera son de construyr & et maxime facit \textbf{ ut obsessi faciliter possint defendere munitionem aliquam , } est scire , \\\hline
3.3.20 & para que los çercados ligeramente puedan defender las fortalezas \textbf{ e saber en qual manera son de construyr } e de fazer los castiellos et las cibdades & ut obsessi faciliter possint defendere munitionem aliquam , \textbf{ est scire , | qualiter aedificanda sunt castra , } et ciuitates , et munitiones ceterae , \\\hline
3.3.20 & e saber en qual manera son de construyr \textbf{ e de fazer los castiellos et las cibdades } e las otras fortalezas & qualiter aedificanda sunt castra , \textbf{ et ciuitates , et munitiones ceterae , } ne faciliter impugnentur . Sunt autem quinque in huiusmodi aedificatione consideranda : \\\hline
3.3.20 & e las otras fortalezas \textbf{ por que non se puedan conbatir ligeramente . } Et son çinco cosas de penssar & et ciuitates , et munitiones ceterae , \textbf{ ne faciliter impugnentur . Sunt autem quinque in huiusmodi aedificatione consideranda : } per quae munitiones fortiores existunt , \\\hline
3.3.20 & por que non se puedan conbatir ligeramente . \textbf{ Et son çinco cosas de penssar } en tal façion de los castiellos e de las fortalezas & et ciuitates , et munitiones ceterae , \textbf{ ne faciliter impugnentur . Sunt autem quinque in huiusmodi aedificatione consideranda : } per quae munitiones fortiores existunt , \\\hline
3.3.20 & por los quales las fortalezas son mas fuertes \textbf{ e mas graues de tomar . } Lo primero son las fortalezas mas fuertes & per quae munitiones fortiores existunt , \textbf{ et difficiliores ad capiendum . } Primo quidem fortificantur munitiones , \\\hline
3.3.20 & Lo primero son las fortalezas mas fuertes \textbf{ e son mas graues para las conbatir } por la natura del logar . Lo segundo por los rençones de los muros . & Primo quidem fortificantur munitiones , \textbf{ et sunt difficiliores ad bellandum ex natura loci . Secundo ex angularitate murorum . Tertio ex terratis . } Quarto ex propugnaculis . \\\hline
3.3.20 & mas altos atales \textbf{ que ninguno non pueda llegar a ellos } o si la mar los cercare & si editae sint in praeruptis rupibus , \textbf{ vel in locis eminentibus et inaccessibilibus : } aut si mare sit circa eas , \\\hline
3.3.20 & Et por ende en el comienço \textbf{ quando son de fundar } e de fazer las fortalezas & aut flumina circumdant ipsas . \textbf{ A principio igitur quando aedificandae sunt munitiones , } defendendae ab exteriori pugna , \\\hline
3.3.20 & quando son de fundar \textbf{ e de fazer las fortalezas } por que se puedan defender de los que la cercan de fuera & A principio igitur quando aedificandae sunt munitiones , \textbf{ defendendae ab exteriori pugna , } et ab obsidentibus , \\\hline
3.3.20 & e de fazer las fortalezas \textbf{ por que se puedan defender de los que la cercan de fuera } e de la batalla dellos & A principio igitur quando aedificandae sunt munitiones , \textbf{ defendendae ab exteriori pugna , } et ab obsidentibus , \\\hline
3.3.20 & . \textbf{ es de penssar la natura del logar . } assi que en tal logar sean fundados & et ab obsidentibus , \textbf{ consideranda est natura loci , } ut in tali loco aedificentur , \\\hline
3.3.20 & que por en aquel assentamiento sean mas fuertes . \textbf{ Et si non han uagar de fundar las fortalezas de nueuo } e algunos teniendo la yra de los señores & quod ex ipso situ fortiores existant . \textbf{ Vel si non vacat munitiones de nouo aedificare , } et aliqui timentes iram dominorum , \\\hline
3.3.20 & e la sanera del pueblo \textbf{ quieren se defender en algunan fortaleza } o si ouieren poderio es de fazer tal fortaleza & et aliqui timentes iram dominorum , \textbf{ aut Domini metuentes furorum populi , volunt se tueri in munitione aliqua : } si adsit facultas quaerenda est munitio talis , \\\hline
3.3.20 & quieren se defender en algunan fortaleza \textbf{ o si ouieren poderio es de fazer tal fortaleza } por que la natura del logarsea mas fuerte & aut Domini metuentes furorum populi , volunt se tueri in munitione aliqua : \textbf{ si adsit facultas quaerenda est munitio talis , } quae ex ipsa natura loci fortior existat , \\\hline
3.3.20 & por que la natura del logarsea mas fuerte \textbf{ e mas graue de conbatir . } Lo segundo las çibdades & quae ex ipsa natura loci fortior existat , \textbf{ et difficilior ad impugnandum . } Secundo urbes et munitiones sunt difficiliores \\\hline
3.3.20 & que los que çercan lleguen a los muros \textbf{ para entrar la fortaleza } los que estan cercados & si contingat obsidentes ad muros accedere \textbf{ ut munitionem deuincant , } obsessi facilius se tuentur ab illis , \\\hline
3.3.20 & mas ligeramente se defienden dellos \textbf{ e mas ligeramente les pueden fazer mal } e ferir & ø \\\hline
3.3.20 & e mas ligeramente les pueden fazer mal \textbf{ e ferir } a los que los çercan . & obsessi facilius se tuentur ab illis , \textbf{ et leuius offendunt obsidentes . } Nam propter angularitatem murorum \\\hline
3.3.20 & Ca por que la çerca es fecha a esquinas \textbf{ non solamente las pueden ferir de delante } mas avn en las espaldas & Nam propter angularitatem murorum \textbf{ non solum ex parte anteriori , } sed etiam a tergo , \\\hline
3.3.20 & mas avn en las espaldas \textbf{ e detras pueden ferir } a los que lleguan a la fortaleza o a los muros . & non solum ex parte anteriori , \textbf{ sed etiam a tergo , } et quasi ex parte posteriori percuriunt impugnantes munitionem illam . Fiendi itaque sunt muri angulares , \\\hline
3.3.20 & a los que lleguan a la fortaleza o a los muros . \textbf{ Et por ende son de fazer los muros de las fortalezas a esquinas } por que se pueda la fortaleza meior defender . & sed etiam a tergo , \textbf{ et quasi ex parte posteriori percuriunt impugnantes munitionem illam . Fiendi itaque sunt muri angulares , } ut munitio faciliter defendi possit . \\\hline
3.3.20 & Et por ende son de fazer los muros de las fortalezas a esquinas \textbf{ por que se pueda la fortaleza meior defender . } Lo tercero que faze la fortaleza mas fuerte & et quasi ex parte posteriori percuriunt impugnantes munitionem illam . Fiendi itaque sunt muri angulares , \textbf{ ut munitio faciliter defendi possit . } Tertium , quod reddit munitionem difficiliorem ad capiendum , \\\hline
3.3.20 & Lo tercero que faze la fortaleza mas fuerte \textbf{ para se non poder entrar } son terrados o torres albarranas & ut munitio faciliter defendi possit . \textbf{ Tertium , quod reddit munitionem difficiliorem ad capiendum , } dicuntur esse terrata , \\\hline
3.3.20 & Ca en la fortaleza \textbf{ que es de fazer } non solamente es de catar la bondat & dicuntur esse terrata , \textbf{ vel muri ex terra facti . } Nam in munitione fienda non solum est quaerenda bonitas situs , \\\hline
3.3.20 & que es de fazer \textbf{ non solamente es de catar la bondat } do esta assentada & vel muri ex terra facti . \textbf{ Nam in munitione fienda non solum est quaerenda bonitas situs , } et angularitas murorum , \\\hline
3.3.20 & Mas avn cerca aquella fortaleza \textbf{ son de fazer dos muros arredrados algun poco . } assi commo la cerca e la barbacana & et angularitas murorum , \textbf{ sed circa munitionem illam adificandi sunt duo muri aliqualiter distantes : } et intra spatium , \\\hline
3.3.20 & Et en el espaçio \textbf{ que esta entre estos dos muros es de poner la tierra que sacan de las carcauas . } las quales carcauas son de fazer enderredor de la fortaleza & ponenda est terra , \textbf{ quae fodienda est de fossis , } quae fiendae sunt circa munitionem illam , \\\hline
3.3.20 & que esta entre estos dos muros es de poner la tierra que sacan de las carcauas . \textbf{ las quales carcauas son de fazer enderredor de la fortaleza } e de los muros & quae fodienda est de fossis , \textbf{ quae fiendae sunt circa munitionem illam , } vel est aliunde terra apportanda , \\\hline
3.3.20 & o ssi alli non ouiesse tierra \textbf{ deue se traer de otra parte } e deuesse echar entre el espaçio & quae fiendae sunt circa munitionem illam , \textbf{ vel est aliunde terra apportanda , } et ponenda in illo spatio intermedio . Est \\\hline
3.3.20 & deue se traer de otra parte \textbf{ e deuesse echar entre el espaçio } que esta entre los dos muros . & vel est aliunde terra apportanda , \textbf{ et ponenda in illo spatio intermedio . Est } etiam huiusmodi terra \\\hline
3.3.20 & que esta entre los dos muros . \textbf{ Et esta tierra es anssi de tapiar } que amos los muros se fagan & et ponenda in illo spatio intermedio . Est \textbf{ etiam huiusmodi terra } inter tale spatium posita ita densanda , \\\hline
3.3.20 & assi commo vn muro . \textbf{ Ca pueden se fazer torres albarranas de la tierra } que sean bien feridas e bien tapiadas & et efficiatur quasi murus . \textbf{ Contingit | etiam turres ex terra facere , } si bene condensetur : \\\hline
3.3.20 & e fazen la fortaleza mas fuerte . \textbf{ Por la qual cosa mucho cunple fazer tales muros } e tales torres albarranas de tierra muy tapiada . & si bene condensetur : \textbf{ propter quod non est inconueniens construere huiusmodi muros ex terra depressata ; } valet quidem constitutio talium murorum \\\hline
3.3.20 & Et esto uale \textbf{ para defender la fortaleza } que se non pueda entrar & valet quidem constitutio talium murorum \textbf{ ad defendendam munitionem , } ne deuincatur \\\hline
3.3.20 & para defender la fortaleza \textbf{ que se non pueda entrar } nin vençer & ad defendendam munitionem , \textbf{ ne deuincatur } per machinas lapidarias . \\\hline
3.3.20 & que se non pueda entrar \textbf{ nin vençer } por las piedras de los engeñios . & ad defendendam munitionem , \textbf{ ne deuincatur } per machinas lapidarias . \\\hline
3.3.20 & en pero el muro que esta fecho de tierra muy tapiado \textbf{ puede resçebir los colpes de las piedras de los engeñios } sin grand danno fuyo . & ø \\\hline
3.3.20 & Lo quarto que faze las fortalezas mas fuertes son torres e menas e cadahalsos . \textbf{ Ca sienpre son de fazer en los muros torres e cadahalsos } por que se pueda meior defender la fortaleza . & et propugnacula . \textbf{ Nam in ipsis muris construendae sunt turres , | et propugnacula , } ut munitio leuius defendi possit . \\\hline
3.3.20 & Ca sienpre son de fazer en los muros torres e cadahalsos \textbf{ por que se pueda meior defender la fortaleza . } Et mayormente son de fazer las torres e los cadahallos & et propugnacula , \textbf{ ut munitio leuius defendi possit . } Maxime autem ante portam quamlibet ipsius munitionis , \\\hline
3.3.20 & por que se pueda meior defender la fortaleza . \textbf{ Et mayormente son de fazer las torres e los cadahallos } ante cada vna de las puertas de la fortaleza & ut munitio leuius defendi possit . \textbf{ Maxime autem ante portam quamlibet ipsius munitionis , } de qua timetur , \\\hline
3.3.20 & de la qual temen los enemigos \textbf{ que la çerquen podran llegar a ella . } Et ante destas puertas se deue poner la puerta de la trayçion & Maxime autem ante portam quamlibet ipsius munitionis , \textbf{ de qua timetur , } ne ad eam accendant obsidentes , \\\hline
3.3.20 & que la çerquen podran llegar a ella . \textbf{ Et ante destas puertas se deue poner la puerta de la trayçion } que esta colgada con cadenas de fierro & de qua timetur , \textbf{ ne ad eam accendant obsidentes , } fiendae sunt turres , \\\hline
3.3.20 & e ella toda cobierta de fierro \textbf{ por que non puedan entrar los enemigos nin puedan fazer daño con el fuego . } Ca si los que cercan quisieren llegar a quemar las puertas de la fortaleza . & ne ad eam accendant obsidentes , \textbf{ fiendae sunt turres , } et propugnacula : \\\hline
3.3.20 & por que non puedan entrar los enemigos nin puedan fazer daño con el fuego . \textbf{ Ca si los que cercan quisieren llegar a quemar las puertas de la fortaleza . } esta tal puerta & fiendae sunt turres , \textbf{ et propugnacula : } et ante huiusmodi portam ponenda est cataracta pendens annulis ferreis undique etiam ferrata , \\\hline
3.3.20 & deue ser el muro foradado \textbf{ de guisa que la puedan leuantar arriba } et baxarla & et ante huiusmodi portam ponenda est cataracta pendens annulis ferreis undique etiam ferrata , \textbf{ prohibens ingressum hostium , } et incendium ignis . \\\hline
3.3.20 & de guisa que la puedan leuantar arriba \textbf{ et baxarla } cada que quisieren . & prohibens ingressum hostium , \textbf{ et incendium ignis . } Nam si obsidentes vellent portas munitionis succendere , cataracta quae est ante portam prohibebit eos . Rursus supra cataractam debet esse murus perforatus recipiens ipsam , per quem locum poterunt proiici lapides , \\\hline
3.3.20 & cada que quisieren . \textbf{ Et por aquel logar pueden lançar piedras . } e echar agua & et incendium ignis . \textbf{ Nam si obsidentes vellent portas munitionis succendere , cataracta quae est ante portam prohibebit eos . Rursus supra cataractam debet esse murus perforatus recipiens ipsam , per quem locum poterunt proiici lapides , } emitti poterit aqua ad extinguendum ignem , \\\hline
3.3.20 & Et por aquel logar pueden lançar piedras . \textbf{ e echar agua } por matar el fuego & Nam si obsidentes vellent portas munitionis succendere , cataracta quae est ante portam prohibebit eos . Rursus supra cataractam debet esse murus perforatus recipiens ipsam , per quem locum poterunt proiici lapides , \textbf{ emitti poterit aqua ad extinguendum ignem , } si contingeret ipsum ab obsidentibus esse appositum . \\\hline
3.3.20 & e echar agua \textbf{ por matar el fuego } si contesçiesse que los enemigos pusiessen fuego a las puertas . & Nam si obsidentes vellent portas munitionis succendere , cataracta quae est ante portam prohibebit eos . Rursus supra cataractam debet esse murus perforatus recipiens ipsam , per quem locum poterunt proiici lapides , \textbf{ emitti poterit aqua ad extinguendum ignem , } si contingeret ipsum ab obsidentibus esse appositum . \\\hline
3.3.20 & Lo quinto que faze las fortalezas mas fuertes e peores \textbf{ para las entrar es quando las carcauas son muy anchas } e muy fondas & Quintum quod facit munitiones \textbf{ magis inacessibiles , } et fortiores : \\\hline
3.3.20 & las quales carcauas deuen ser lleñas de agua \textbf{ si sse podiere fazer Et pues que assi es en estas maneras sobredichas son las fortalezas mas fuertes } e peores de tomar . & et fortiores : \textbf{ est latitudo , } et profunditas fossarum : \\\hline
3.3.20 & si sse podiere fazer Et pues que assi es en estas maneras sobredichas son las fortalezas mas fuertes \textbf{ e peores de tomar . } Et por ende deuen catar en el comienço & est latitudo , \textbf{ et profunditas fossarum : } quae \\\hline
3.3.20 & e peores de tomar . \textbf{ Et por ende deuen catar en el comienço } aquellos que quieren fazer las fortalezas & et profunditas fossarum : \textbf{ quae } ( si adsit facultas ) replendae sunt aquis . \\\hline
3.3.20 & Et por ende deuen catar en el comienço \textbf{ aquellos que quieren fazer las fortalezas } por que las puedan defender de los enemigos & quae \textbf{ ( si adsit facultas ) replendae sunt aquis . } His ergo modis sunt munitiones difficiliores \\\hline
3.3.20 & aquellos que quieren fazer las fortalezas \textbf{ por que las puedan defender de los enemigos } que todas estas cosas & ( si adsit facultas ) replendae sunt aquis . \textbf{ His ergo modis sunt munitiones difficiliores | ad capiendum . } Ideo videndum est a principio ab his \\\hline
3.3.20 & o las mas dellas sean fechas en aquellas fortalezas \textbf{ por que se puedan meior defender . } N Non abasta dezir & ut in munitionibus illis omnia haec vel plura ex istis concurrant ad hoc , \textbf{ quod facilius defendantur . } Non sufficit scire , \\\hline
3.3.21 & por que se puedan meior defender . \textbf{ N Non abasta dezir } en qual manera son de fazer las fortalezas & quod facilius defendantur . \textbf{ Non sufficit scire , } quomodo aedificandae sunt munitiones , \\\hline
3.3.21 & N Non abasta dezir \textbf{ en qual manera son de fazer las fortalezas } e quales muros deuen auer & Non sufficit scire , \textbf{ quomodo aedificandae sunt munitiones , } et quales muros debent habere , \\\hline
3.3.21 & en qual manera son de fazer las fortalezas \textbf{ e quales muros deuen auer } e commo deuen ser assentadas & quomodo aedificandae sunt munitiones , \textbf{ et quales muros debent habere , } et quomodo debent esse sitae , \\\hline
3.3.21 & e commo deuen ser assentadas \textbf{ sil non sopieremos commo son de bastecer } por que non puedan de ligero ser tomadas . & et quomodo debent esse sitae , \textbf{ nisi sciatur quomodo sunt muniendae , } ut non de facili vinci possint . \\\hline
3.3.21 & que tres maneras ay \textbf{ para tomar las fortalezas . } Conuiene de saber . & Dicebatur enim supra , \textbf{ triplicem esse modum deuincendi munitiones : } videlicet per famem , \\\hline
3.3.21 & para tomar las fortalezas . \textbf{ Conuiene de saber . } Por fanbre e por sed e por batalla . & triplicem esse modum deuincendi munitiones : \textbf{ videlicet per famem , } sitim , et pugnam . Sic ergo muniendae sunt munitiones obsessae , \\\hline
3.3.21 & Por fanbre e por sed e por batalla . \textbf{ Et pues que assi es en tal manera son de basteçer las fortalezas } que temen de ser çercadas & videlicet per famem , \textbf{ sitim , et pugnam . Sic ergo muniendae sunt munitiones obsessae , } ne aliquo horum modorum possint deuinci . \\\hline
3.3.21 & e por que non sean tomadas nin vençidas por fanbre . \textbf{ tres cosas son de penssar e de proueer . } Lo primero que el trigo & Ne enim fame deuincantur , \textbf{ tria sunt attendenda , } videlicet ut frumenta , auena , ordeum , \\\hline
3.3.21 & Et aquellas cosas que pertenesçen a la uida de los omens \textbf{ todas son de traer a la fortaleza } que teme ser cercada & quae possunt deseruire ad victum , \textbf{ deportanda sint ad munitionem obsessam , } prius quam obsideatur ad extraneis : \\\hline
3.3.21 & ante que coian los fructos de los otros logares \textbf{ mas cercanos son de traer todas aquellas cosas } que son menester para bastesçimiento . & et si timetur de obsessione ante recollationem frugum , \textbf{ ex aliis locis propinquis sunt talia acquirenda , } ne munitio obsessa ob carentiam victus possit pati defectum . \\\hline
3.3.21 & por que la fortaleza \textbf{ quando fuere cercada non pueda auer mengua de vianda . Et aquellas cosas todas } que se non pueden traer a la fortaleza & ne munitio obsessa ob carentiam victus possit pati defectum . \textbf{ Quicquid autem non potest ad munitionem deferri } ( \\\hline
3.3.21 & quando fuere cercada non pueda auer mengua de vianda . Et aquellas cosas todas \textbf{ que se non pueden traer a la fortaleza } osi se pueden traer & ne munitio obsessa ob carentiam victus possit pati defectum . \textbf{ Quicquid autem non potest ad munitionem deferri } ( \\\hline
3.3.21 & que se non pueden traer a la fortaleza \textbf{ osi se pueden traer } e non son muy prouechosas al castiello o a la çibdat cercada & Quicquid autem non potest ad munitionem deferri \textbf{ ( } vel si deferretur non multum esset utile castro vel ciuitati obsessae ) totum est igni comburendum ; \\\hline
3.3.21 & e non son muy prouechosas al castiello o a la çibdat cercada \textbf{ todas son de quemar por fuego . } por que los cercadores quanda vinieren a cercar & ( \textbf{ vel si deferretur non multum esset utile castro vel ciuitati obsessae ) totum est igni comburendum ; } ne obsidentes superuenientes inde capiant emolumentum , et ex bonis propriis munitionis obsessae inpugnent ipsam . \\\hline
3.3.21 & todas son de quemar por fuego . \textbf{ por que los cercadores quanda vinieren a cercar } non se puedan aprouechar ellas & ø \\\hline
3.3.21 & por que los cercadores quanda vinieren a cercar \textbf{ non se puedan aprouechar ellas } nin puedan de los bienes proprios de la fortaleza cercada aprouecharse & vel si deferretur non multum esset utile castro vel ciuitati obsessae ) totum est igni comburendum ; \textbf{ ne obsidentes superuenientes inde capiant emolumentum , et ex bonis propriis munitionis obsessae inpugnent ipsam . } Si autem timeatur de diuturnitate temporis , \\\hline
3.3.21 & non se puedan aprouechar ellas \textbf{ nin puedan de los bienes proprios de la fortaleza cercada aprouecharse } para la conbatir . & vel si deferretur non multum esset utile castro vel ciuitati obsessae ) totum est igni comburendum ; \textbf{ ne obsidentes superuenientes inde capiant emolumentum , et ex bonis propriis munitionis obsessae inpugnent ipsam . } Si autem timeatur de diuturnitate temporis , \\\hline
3.3.21 & nin puedan de los bienes proprios de la fortaleza cercada aprouecharse \textbf{ para la conbatir . } Et si temen de ser çercados & ne obsidentes superuenientes inde capiant emolumentum , et ex bonis propriis munitionis obsessae inpugnent ipsam . \textbf{ Si autem timeatur de diuturnitate temporis , } ut quod per multa tempora debeat obsessio perdurare , \\\hline
3.3.21 & por mucho tienpo \textbf{ e que la çerca les a a durar luengamente . } entre todas las otras cosas & Si autem timeatur de diuturnitate temporis , \textbf{ ut quod per multa tempora debeat obsessio perdurare , } maxime munienda est ciuitas vel castrum obsessum milio : \\\hline
3.3.21 & entre todas las otras cosas \textbf{ de que deuen basteçer la fortaleza } e el castiello deuenla basteçer mayormente de mijo . Ca el mijo menos se podresçe & ut quod per multa tempora debeat obsessio perdurare , \textbf{ maxime munienda est ciuitas vel castrum obsessum milio : } nam milium \\\hline
3.3.21 & de que deuen basteçer la fortaleza \textbf{ e el castiello deuenla basteçer mayormente de mijo . Ca el mijo menos se podresçe } e mas dura & maxime munienda est ciuitas vel castrum obsessum milio : \textbf{ nam milium } inter cetera minus putrefit , \\\hline
3.3.21 & que tedos los otros granos . \textbf{ Avn basteçer de grand conplimiento de carnes saladas } e de mucha sal . & inter cetera minus putrefit , \textbf{ et plus durare perhibetur . Copia | etiam carnium salitarum non est praetermittenda . Salis } etiam multitudo multum est expediens munitioni obsessae , \\\hline
3.3.21 & que teme de ser cercada \textbf{ non solamente es de penssar } quanto a la vianda & vel ciuitatem aliquam obsidendam , \textbf{ quantum ad victum non solum attendendum est , } ut magna copia victualium deferatur \\\hline
3.3.21 & que teme de ser çercada . \textbf{ Mas avn han de tener mientes } que las viandas que traxieren & ø \\\hline
3.3.21 & quando la çibdat cercada es muy grande \textbf{ e non puede auer uianda de fuera en cada uarrio de la çibdat } deuen ser puestas las viandas en orrios e en alholis publicos . & si ciuitas obsessa esset magna , \textbf{ et non posset aliunde recuperare victum , } in qualibet contrata ciuitatis victualia reduci debent ad horrea publica , \\\hline
3.3.21 & Et si la fortaleza çercada es pequena \textbf{ non es graue cosa de fazer esto . } Ca non aprouecha nada traer muchas viandas & et parte , et temperate per viros prouidos dispensare . \textbf{ quod si munitio obsessa modici esset ambitus , hoc efficere non est difficile } quasi enim \\\hline
3.3.21 & non es graue cosa de fazer esto . \textbf{ Ca non aprouecha nada traer muchas viandas } si non fueren partidas con tenpramiento et escassamente . & quod si munitio obsessa modici esset ambitus , hoc efficere non est difficile \textbf{ quasi enim | nihil prodest multa praeparatio victualium , } nisi parce , et cum temperamento dispensetur . \\\hline
3.3.21 & si non fueren partidas con tenpramiento et escassamente . \textbf{ Lo terçero es de proueer en tales cosas } que las perssonas flacas e sin prouecho & nisi parce , et cum temperamento dispensetur . \textbf{ Tertio est in talibus attendendum , } ut personae debiles , et inutiles , \\\hline
3.3.21 & que non son prouechosas \textbf{ para defender la fortaleza sean enbiadas fuera a otra parte } si se puede fazer . & ut personae debiles , et inutiles , \textbf{ non valentes proficere ad defensionem munitionis obsessae , } si commode fieri potest , \\\hline
3.3.21 & para defender la fortaleza sean enbiadas fuera a otra parte \textbf{ si se puede fazer . } Ca tales perssonas gastan e comen aquello & non valentes proficere ad defensionem munitionis obsessae , \textbf{ si commode fieri potest , } sunt ad partes alias transmittendae : \\\hline
3.3.21 & que son dentro en la fortaleza cercada \textbf{ las que pueden bien escusar } son de matar & Rursus si timeatur de inopia victualium , \textbf{ bestiae quae sunt in munitione obsessa , } a quibus obsessi possunt commode abstinere , sunt occidendae , et comedendae , vel saliendae , \\\hline
3.3.21 & las que pueden bien escusar \textbf{ son de matar } para comer & Rursus si timeatur de inopia victualium , \textbf{ bestiae quae sunt in munitione obsessa , } a quibus obsessi possunt commode abstinere , sunt occidendae , et comedendae , vel saliendae , \\\hline
3.3.21 & son de matar \textbf{ para comer } o son de salar & bestiae quae sunt in munitione obsessa , \textbf{ a quibus obsessi possunt commode abstinere , sunt occidendae , et comedendae , vel saliendae , } si esui aptae sunt : \\\hline
3.3.21 & para comer \textbf{ o son de salar } si son conuenibles para comer . & bestiae quae sunt in munitione obsessa , \textbf{ a quibus obsessi possunt commode abstinere , sunt occidendae , et comedendae , vel saliendae , } si esui aptae sunt : \\\hline
3.3.21 & o son de salar \textbf{ si son conuenibles para comer . } Et avn en tal caso & a quibus obsessi possunt commode abstinere , sunt occidendae , et comedendae , vel saliendae , \textbf{ si esui aptae sunt : } immo in tali casu comedenda sunt multa , \\\hline
3.3.21 & Et avn en tal caso \textbf{ do temen de grand mengua muchas bestias son de comer que otramiente non son de comer } nin es vso de las comer . & si esui aptae sunt : \textbf{ immo in tali casu comedenda sunt multa , } quae ad esum vetat communis usus . \\\hline
3.3.21 & e se guarda \textbf{ que non sea tomada por fanbre de ligero puede paresçer } commo se deua auer & ne capiatur fame : \textbf{ de leui patere potest qualiter se debeant habere obsessi } ne deuincantur per sitim . \\\hline
3.3.21 & que non sea tomada por fanbre de ligero puede paresçer \textbf{ commo se deua auer } por que non sea tomada por sed . & ne capiatur fame : \textbf{ de leui patere potest qualiter se debeant habere obsessi } ne deuincantur per sitim . \\\hline
3.3.21 & por que non sea tomada por sed . \textbf{ Ca ante que sean cercadas deuen prouer se } que en tal fortaleza se ençierren & ne deuincantur per sitim . \textbf{ Nam antequam aliqui eos obsideant , | prouidere debent } quod ad talem munitionem pergant , \\\hline
3.3.21 & en que aya grand conplimiento de aguas \textbf{ e si y non ouieren fuentes deuen fazer pozos . } Et sy por auentura el logar es tan seco & in qua sit aquarum copia : \textbf{ quod si vero ibi non sint fontes , } fodendi sunt putet : \\\hline
3.3.21 & Et sy por auentura el logar es tan seco \textbf{ que non pueda y auer poço } pueden se ay fazer cisternas e algibes & quod si etiam locus sit siccus , \textbf{ ut ibi nec putei fieri possint : } fiendae sunt cisternae , \\\hline
3.3.21 & que non pueda y auer poço \textbf{ pueden se ay fazer cisternas e algibes } en que coisgan el agua & ut ibi nec putei fieri possint : \textbf{ fiendae sunt cisternae , } ut caelestium aquarum superabundantia suppleat aliarum aquarum defectum : \\\hline
3.3.21 & que viene del çielo \textbf{ por que puedan auer agua e el agua del çielo } cunpla la mengua de las otras aguas . & fiendae sunt cisternae , \textbf{ ut caelestium aquarum superabundantia suppleat aliarum aquarum defectum : } quod si munitio obsessa sit circa mare , \\\hline
3.3.21 & Et si la fortaleza cercada fuere çerca de la mar \textbf{ e non podieren auer } si non agua salada & quod si munitio obsessa sit circa mare , \textbf{ et non possit habere aquam nisi santam , } eo quod dulcem aquam habeat distantem , \\\hline
3.3.21 & por que el agua dulçe ba muy lueñe \textbf{ e non la pueden tomar } por los enemigos que la tienen cercada . & et non possit habere aquam nisi santam , \textbf{ eo quod dulcem aquam habeat distantem , } ad quam capiendam prohibent obsidentes : \\\hline
3.3.21 & por los enemigos que la tienen cercada . \textbf{ Estonçe pueden el agua salado fazer dulçe colando la } por la çera & ø \\\hline
3.3.21 & toda agua salada que passa por los foradillos menudos de la çera toda se torna dulçe . \textbf{ Et avn conuiene de traer } e de acarrear vinagre & ø \\\hline
3.3.21 & Et avn conuiene de traer \textbf{ e de acarrear vinagre } e vino en grand abondança a la fortaleza & totum in dulce conuertitur . Deferendum est \textbf{ etiam ad munitionem obsidendem in magna copia acerum , } et vinum , \\\hline
3.3.21 & que teme ser cercada \textbf{ por que beuiendo agua sola los lidiadores enflaquesçerse yan en tanto que non podrian defenderse de los enemigos . } Mostrado quales remedios se deuen tomar & et vinum , \textbf{ ne ex potu solius equae bellatores adeo debilitentur , | quod non possint viriliter resistere obsidentibus . } Ostenso quomodo sunt remedia adhibenda contra famem , \\\hline
3.3.21 & por que beuiendo agua sola los lidiadores enflaquesçerse yan en tanto que non podrian defenderse de los enemigos . \textbf{ Mostrado quales remedios se deuen tomar } contra la fanbre e contra la sed . & quod non possint viriliter resistere obsidentibus . \textbf{ Ostenso quomodo sunt remedia adhibenda contra famem , } et sitim per quae obsessa munitio deuinci consueuit : \\\hline
3.3.21 & contra la fanbre e contra la sed . \textbf{ Por las quales cosas las fortalezas cercadas se suelen tomar } e vençer finca de ver quales remedios son de poner & Ostenso quomodo sunt remedia adhibenda contra famem , \textbf{ et sitim per quae obsessa munitio deuinci consueuit : } restat videre , \\\hline
3.3.21 & Por las quales cosas las fortalezas cercadas se suelen tomar \textbf{ e vençer finca de ver quales remedios son de poner } por que la fortaleza cercada non se pueda vençer por batalla . & et sitim per quae obsessa munitio deuinci consueuit : \textbf{ restat videre , | quae sunt remedia adhibenda , } ne per pugnam obsessa munitio deuincatur . \\\hline
3.3.21 & e vençer finca de ver quales remedios son de poner \textbf{ por que la fortaleza cercada non se pueda vençer por batalla . } Et pues que assi es deuense traer a la çibdat o al castiello & quae sunt remedia adhibenda , \textbf{ ne per pugnam obsessa munitio deuincatur . } Debent ergo ad ciuitatem , \\\hline
3.3.21 & por que la fortaleza cercada non se pueda vençer por batalla . \textbf{ Et pues que assi es deuense traer a la çibdat o al castiello } que teme de ser cercado & ne per pugnam obsessa munitio deuincatur . \textbf{ Debent ergo ad ciuitatem , } vel ad castrum obsessum deportari in magna copia sulphur , pix , \\\hline
3.3.21 & e rasina en grand anbondança \textbf{ para quemar los engeñios . } Et conuiene avn de traer a la fortaleza & vel ad castrum obsessum deportari in magna copia sulphur , pix , \textbf{ oleum ad comburendum machinas hostium . Ferra autem } et ligna sunt ad munitionem obsessam in debita abundantia deportanda , \\\hline
3.3.21 & para quemar los engeñios . \textbf{ Et conuiene avn de traer a la fortaleza } que teme de ser cercada mucha llena & vel ad castrum obsessum deportari in magna copia sulphur , pix , \textbf{ oleum ad comburendum machinas hostium . Ferra autem } et ligna sunt ad munitionem obsessam in debita abundantia deportanda , \\\hline
3.3.21 & e mucho fierro e en grant abondaça por que les non fallezca \textbf{ assi que de la madera puedan fazer astas } para las saetas & et ligna sunt ad munitionem obsessam in debita abundantia deportanda , \textbf{ ut per ligna hastae sagittarum , } et telorum , \\\hline
3.3.21 & e para los dardos e para las lanças \textbf{ e avn que puedan fazer cadahalsos } los que fezieren menester en la fortaleza . & ut per ligna hastae sagittarum , \textbf{ et telorum , } et etiam aedificia necessaria munitioni fieri possint . Per ferra vero etiam reparari possint arma , et fieri tela ; \\\hline
3.3.21 & Et del fierro puedan las armas \textbf{ e fazer fierros de dardos et de saetas } e las otras cosas & et etiam aedificia necessaria munitioni fieri possint . Per ferra vero etiam reparari possint arma , et fieri tela ; \textbf{ et sagittae , et alia per quae impugnari valeant obsidentes . } Est etiam multitudo ferri perutilis ipsis obsessis ad destruendum aedificia , \\\hline
3.3.21 & que son menester \textbf{ para que se puedan defender de los enemigos } que los çercan & ø \\\hline
3.3.21 & a los que estan çertados \textbf{ para destroyr los hedifiçios } assi commo torres e gatas de madera e engeñios e algarradas de aquellas & et sagittae , et alia per quae impugnari valeant obsidentes . \textbf{ Est etiam multitudo ferri perutilis ipsis obsessis ad destruendum aedificia , } et machinas ipsorum obsidentium , \\\hline
3.3.21 & que se sigue . \textbf{ Et avn guyias e piedras muchas son de traer a la fortaleza en grand conplimiento } porque tales cosas son mas rezias et meiores para lançar . & ut in sequenti capitulo apparebit . \textbf{ Saxa etiam torrentium in magna copia sunt ad munitionem deportanda : } quia talia sunt solidiora , \\\hline
3.3.21 & Et avn guyias e piedras muchas son de traer a la fortaleza en grand conplimiento \textbf{ porque tales cosas son mas rezias et meiores para lançar . } Et destas piedras tales deuen finchir los muros e las torres de las fortalezas cercadas . & Saxa etiam torrentium in magna copia sunt ad munitionem deportanda : \textbf{ quia talia sunt solidiora , | et aptiora ad faciendum . } Ex eis ergo replendi sunt muri , \\\hline
3.3.21 & porque tales cosas son mas rezias et meiores para lançar . \textbf{ Et destas piedras tales deuen finchir los muros e las torres de las fortalezas cercadas . } Et avn es menester mucha cal fecha poluo & et aptiora ad faciendum . \textbf{ Ex eis ergo replendi sunt muri , } et turres munitionis obsessae . Calcem etiam puluerizatam deferendum est ad ipsam munitionem in magna abundantia , \\\hline
3.3.21 & Et avn es menester mucha cal fecha poluo \textbf{ e conuiene de la traer en grand abondança } donde quier que la podieren fazer a la fortaleza . & Ex eis ergo replendi sunt muri , \textbf{ et turres munitionis obsessae . Calcem etiam puluerizatam deferendum est ad ipsam munitionem in magna abundantia , } et ex ea replenda sunt multa vasa ; \\\hline
3.3.21 & e conuiene de la traer en grand abondança \textbf{ donde quier que la podieren fazer a la fortaleza . } Et conuiene de finchir della muchas vasigas de tierra assi commo tinaias e cantaros & Ex eis ergo replendi sunt muri , \textbf{ et turres munitionis obsessae . Calcem etiam puluerizatam deferendum est ad ipsam munitionem in magna abundantia , } et ex ea replenda sunt multa vasa ; \\\hline
3.3.21 & donde quier que la podieren fazer a la fortaleza . \textbf{ Et conuiene de finchir della muchas vasigas de tierra assi commo tinaias e cantaros } e otros belhezos quales quier . & et turres munitionis obsessae . Calcem etiam puluerizatam deferendum est ad ipsam munitionem in magna abundantia , \textbf{ et ex ea replenda sunt multa vasa ; } ut cum obsidentes appropinquant muris munitionis , iacenda sunt vasa illa , \\\hline
3.3.21 & quien les fiere \textbf{ nin a quien han de ferir . } Et avn es meester grand conplimiento de neruios & ut quasi caeci , \textbf{ et non videntes percuti possint . Neruorum etiam copia , } et funium utilis est munitioni obsessae , \\\hline
3.3.21 & que fazen menester \textbf{ E si por auentura fallesçieren los neruios en logar dellos pueden tomar las çerneias } e las colas dellos cauallos & et alia praeparanda , \textbf{ quod si nerui deficiant , } loco eorum adhiberi poterunt crines equi , \\\hline
3.3.21 & que quando a los romanos fallesçieron los neruios \textbf{ e non podien adobar los engeñios } para se defender de los enemigos . las mugeres de roma cortaron se los cabellos & quod cum Romanis neruorum copia defecisset , \textbf{ et non possent eorum machinas reparare ad resistendum bellatoribus mulieres Romanae abscissis crinibus eos suis maritis tradiderunt : } per \\\hline
3.3.21 & e non podien adobar los engeñios \textbf{ para se defender de los enemigos . las mugeres de roma cortaron se los cabellos } e dieron los a sus maridos & quod cum Romanis neruorum copia defecisset , \textbf{ et non possent eorum machinas reparare ad resistendum bellatoribus mulieres Romanae abscissis crinibus eos suis maritis tradiderunt : } per \\\hline
3.3.21 & que mas quisieron aquellas buenas mugeres muy castas beuir con sus maridos trasquiladas \textbf{ que non yr con sus enemigos con cabellos . } Avn son menester en las fortalezas cuernos de bestias & cum maritis conuiuere deformato capite , \textbf{ quam seruire hostibus integris crinibus . Sunt } etiam ad munitiones deportanda cornua bestiarum ad reformandum ballistas , \\\hline
3.3.21 & Avn son menester en las fortalezas cuernos de bestias \textbf{ para apareiar las ballestas e los arcos . } Et avn son menester cueros crudos & quam seruire hostibus integris crinibus . Sunt \textbf{ etiam ad munitiones deportanda cornua bestiarum ad reformandum ballistas , } et arcus : \\\hline
3.3.21 & Et avn son menester cueros crudos \textbf{ para cobrir los engeñios e las gatas e los otros artifiçios } por que los non puedan quamar los engeñios . & et arcus : \textbf{ et coria cruda ad tegendum machinas , | et alia aedificia , } ne ab aduersariis per incendia comburantur . \\\hline
3.3.21 & para cobrir los engeñios e las gatas e los otros artifiçios \textbf{ por que los non puedan quamar los engeñios . } Por estas cautelas & et alia aedificia , \textbf{ ne ab aduersariis per incendia comburantur . } His etiam cautelis , \\\hline
3.3.21 & e por aquellas cosas \textbf{ que son dichas se podran defender } los que estan çercados & His etiam cautelis , \textbf{ et per ea quae dicta sunt resistere poterunt obsessi ; } ne eorum munitiones per pugnam ab obsidentibus deuincantur . \\\hline
3.3.22 & nin vençidas por batalla de sus enemigos . \textbf{ C Contadas son de suso tres maneras espeçiales de acometer las fortalezas çercadas de las quales . } La vna era por cueuas conegeras o por carreras soterrañas . & ne eorum munitiones per pugnam ab obsidentibus deuincantur . \textbf{ Enumerabantur supra tres speciales modi impugnandi munitiones obsessas . } Quorum unus erat per cuniculos et vias subterraneas . \\\hline
3.3.22 & por las maneras sobredichas \textbf{ aquellos que çercan commo han de acometer } a los que estan cercados . & Quare si docuimus per praefatos modos inuadere obsidentes obsessos : \textbf{ reliquum est } ut declaremus \\\hline
3.3.22 & a los que estan cercados . \textbf{ fincanos de demostrar en qual manera los que estan çercados se pueden defender de aquellos que los çercan } e de las maneras en que los çercan . & ut declaremus \textbf{ quomodo obsessi a praedictis impugnationibus contra obsidentes se defendere valeant . } Primo ergo dicemus \\\hline
3.3.22 & assi es primeramente diremos de los remedios \textbf{ que se pueden poner } contra aquella manera de acometer & Primo ergo dicemus \textbf{ de remediis contra impugnationem per cuniculos . Possumus autem circa haec , duo remedia assignare . } Unum est per profunditatem fossarum repletarum aquis . \\\hline
3.3.22 & que se pueden poner \textbf{ contra aquella manera de acometer } que es por cueuas conegeras o por carreras soterrañas . & Primo ergo dicemus \textbf{ de remediis contra impugnationem per cuniculos . Possumus autem circa haec , duo remedia assignare . } Unum est per profunditatem fossarum repletarum aquis . \\\hline
3.3.22 & que es por cueuas conegeras o por carreras soterrañas . \textbf{ Et podemos contra esto poner dos remedios . } El vno es afondando mucho las carcauas & Unum est per profunditatem fossarum repletarum aquis . \textbf{ Nam } si circa munitionem obsessam sint profundae foueae aquis repletae , impediuntur obsidentes ; \\\hline
3.3.22 & los que çercan \textbf{ por que non puedan acometer los quiestan cercados } por carteras soterrañas . & si circa munitionem obsessam sint profundae foueae aquis repletae , impediuntur obsidentes ; \textbf{ ne obsessos impugnare possint per cuniculos , } et vias subterraneas . \\\hline
3.3.22 & por carteras soterrañas . \textbf{ Et puesto avn que las carcauas non se pueda finchir de agua } si fueren muy fondas e muy anchas conplidamente se enbargan las cueuas soterrañas & et vias subterraneas . \textbf{ Dato tamen quod fossae aquis repleri non possint , } si sint valde profundae , \\\hline
3.3.22 & si fueren muy fondas e muy anchas conplidamente se enbargan las cueuas soterrañas \textbf{ que non puedan passar por ellas } por que en esta manera de conbatemiento non se pueden conbatir las fortalezas cercadas & per ample , \textbf{ sufficienter impediunt subterraneas vias : } quia hoc genere impugnationis impugnari non possunt munitiones obsessae , \\\hline
3.3.22 & que non puedan passar por ellas \textbf{ por que en esta manera de conbatemiento non se pueden conbatir las fortalezas cercadas } si las dichas carreras soterrañas non fuessen muy mas fondas que las carcauas . & sufficienter impediunt subterraneas vias : \textbf{ quia hoc genere impugnationis impugnari non possunt munitiones obsessae , } nisi dictae viae subterraneae profundiores sint fossis . Munitio ergo defendenda \\\hline
3.3.22 & que se ha de defendero esta assentada sobre peña firme . \textbf{ Et estonçe non se puede acometer } por carreras soterrañas & nisi dictae viae subterraneae profundiores sint fossis . Munitio ergo defendenda \textbf{ vel est supra petram firmam : } et tunc propter duritiem lapidum \\\hline
3.3.22 & que se non puede cauaro esta la fortaleza assentada sobre peña blanda \textbf{ que se puede bien dolar e foradar } o esta assentada sobre tierra & non est \textbf{ facile per cuniculos debellare eam , } vel est supra petram \\\hline
3.3.22 & o esta assentada sobre tierra \textbf{ que se puede de ligero cauar e estonçe es de enfortaleçer el castiello } o la çibdat afondando mucho las carcauas & vel est supra petram \textbf{ de facili labilem , } aut supra terram , quae de facili fodi potest : et tunc per profundas foueas est fortificanda munitio , \\\hline
3.3.22 & o la çibdat afondando mucho las carcauas \textbf{ por que non puedan passar } por las cueuas conegeras a acometer la fortaleza . & aut supra terram , quae de facili fodi potest : et tunc per profundas foueas est fortificanda munitio , \textbf{ ne per cuniculos deuincatur . } Secundum remedium contra cuniculos \\\hline
3.3.22 & por que non puedan passar \textbf{ por las cueuas conegeras a acometer la fortaleza . } El segundo remedio contra las cueuas conegeras & aut supra terram , quae de facili fodi potest : et tunc per profundas foueas est fortificanda munitio , \textbf{ ne per cuniculos deuincatur . } Secundum remedium contra cuniculos \\\hline
3.3.22 & e contra las carreras soterrañas \textbf{ es de fazer vna fortaleza cercada } otra carrera & ø \\\hline
3.3.22 & por la qual cosa temen del cometemiento \textbf{ por las cueuas conegeras deuen penssar con grand acuçia los cercados } si pudieren veer que llegan la tierra de alguna parte & propter quod timetur de impugnatione per cuniculos ; \textbf{ diligenter considerare debent obsessi , } utrum ab aliqua parte videant terram deferri , \\\hline
3.3.22 & por las cueuas conegeras deuen penssar con grand acuçia los cercados \textbf{ si pudieren veer que llegan la tierra de alguna parte } o si por algunas señales pudieren conosçer & diligenter considerare debent obsessi , \textbf{ utrum ab aliqua parte videant terram deferri , } et utrum per aliqua signa cognoscere possint obsidentes inchoare cuniculos : \\\hline
3.3.22 & si pudieren veer que llegan la tierra de alguna parte \textbf{ o si por algunas señales pudieren conosçer } que los que cercan comiençan a fazer cueuas coneieras . & utrum ab aliqua parte videant terram deferri , \textbf{ et utrum per aliqua signa cognoscere possint obsidentes inchoare cuniculos : } quod cum perceperint , \\\hline
3.3.22 & o si por algunas señales pudieren conosçer \textbf{ que los que cercan comiençan a fazer cueuas coneieras . } Et quando esto entendieren luego & utrum ab aliqua parte videant terram deferri , \textbf{ et utrum per aliqua signa cognoscere possint obsidentes inchoare cuniculos : } quod cum perceperint , \\\hline
3.3.22 & Et quando esto entendieren luego \textbf{ sin detenimiento ninguno deuen fazer otras cueuas soterrañas } que respondan a aquellas cueuas coneieras & quod cum perceperint , \textbf{ statim debent viam aliam subterraneam facere correspondentem illis cuniculis , } ita tamen quod via illa pendeat contra obsidentes : et tunc per viam illam sic perforatam \\\hline
3.3.22 & que vengan derechamente contra ellas . \textbf{ Enpero assi lo deuen fazer } que aquellas carreras desçendan contra aquellas que fazen los que çercan . & ø \\\hline
3.3.22 & e otra fezieron los cercados \textbf{ se deue acometer la batalla continuadamente } por que los -\-> que cercan non puedan entrar por aquellas carreras ala fortaleza . & et partem obsessi ) \textbf{ debet esse bellum continuum , } ne obsidentes per viam illam munitionem ingrediantur . \\\hline
3.3.22 & se deue acometer la batalla continuadamente \textbf{ por que los -\-> que cercan non puedan entrar por aquellas carreras ala fortaleza . } Avn deuen los cercados çerca el comienço de las carreras soterrañas & debet esse bellum continuum , \textbf{ ne obsidentes per viam illam munitionem ingrediantur . } Debent \\\hline
3.3.22 & Avn deuen los cercados çerca el comienço de las carreras soterrañas \textbf{ auer tiñas lleñas de agua o de oriñas . } En quando lidian contra los que los çercan deuen fingir & etiam obsessi iuxta inchoationem viae subterraneae habere magnas tinnas plenas aquis \textbf{ vel etiam urinis : } et cum bellant contra obsidentes , \\\hline
3.3.22 & auer tiñas lleñas de agua o de oriñas . \textbf{ En quando lidian contra los que los çercan deuen fingir } que fuyen & vel etiam urinis : \textbf{ et cum bellant contra obsidentes , } debent se fingere fugere , \\\hline
3.3.22 & que fuyen \textbf{ e deuen salir de aquella cueua } la qual cosa fecho toda aquella agua o aquella orina & et cum bellant contra obsidentes , \textbf{ debent se fingere fugere , } et exire foueam illam \\\hline
3.3.22 & la qual cosa fecho toda aquella agua o aquella orina \textbf{ assy ayuntada deuen la echar sobre los que cercan } que estan las cueuas coneieras . & debent se fingere fugere , \textbf{ et exire foueam illam } quo facto totam aquam \\\hline
3.3.22 & si esto alguna vegada fue fecho \textbf{ non deuemos cuydar } que se non pueda fazer otra vegada . & quare \textbf{ si hoc aliquando factum fuit , non debemus reputare impossibile } ne iterum fieri possit . Viso quomodo resistendum sit debellationi factae per cuniculos : \\\hline
3.3.22 & non deuemos cuydar \textbf{ que se non pueda fazer otra vegada . } Visto en qual manera auemos de contrallar a la batalla fecha & si hoc aliquando factum fuit , non debemus reputare impossibile \textbf{ ne iterum fieri possit . Viso quomodo resistendum sit debellationi factae per cuniculos : } restat videre quomodo obsessi debeant obuiare impugnationi factae per lapidarias machinas . \\\hline
3.3.22 & que se non pueda fazer otra vegada . \textbf{ Visto en qual manera auemos de contrallar a la batalla fecha } por los engenios que lançan las piedras . & si hoc aliquando factum fuit , non debemus reputare impossibile \textbf{ ne iterum fieri possit . Viso quomodo resistendum sit debellationi factae per cuniculos : } restat videre quomodo obsessi debeant obuiare impugnationi factae per lapidarias machinas . \\\hline
3.3.22 & por los engenios que lançan las piedras . \textbf{ Et podemos dar contra los engeñios quatro maneras de acorro . } La primera es & restat videre quomodo obsessi debeant obuiare impugnationi factae per lapidarias machinas . \textbf{ Contra has autem quadrupliciter subuenitur . Primo , } quia aliquando subito ex munitione obsessa exiuit magna multitudo armatorum , \\\hline
3.3.22 & e aquellos que estan con el . \textbf{ Et Ante que pueda la hueste acorrer } a defenderle & et inuadunt machinam ; \textbf{ et prius quam exercitus possit succurrere ad defendendum eam , } succendunt ipsam . \\\hline
3.3.22 & Et Ante que pueda la hueste acorrer \textbf{ a defenderle } quemenle con fuego . & et inuadunt machinam ; \textbf{ et prius quam exercitus possit succurrere ad defendendum eam , } succendunt ipsam . \\\hline
3.3.22 & quemenle con fuego . \textbf{ Mas si non osan salir los que son cercados de aquella fortaleza } estonçe encubiertamente de noche deuen echar algunos & succendunt ipsam . \textbf{ Sed si munitionem ipsam obsessi exire non audeant : } tunc clam de nocte aliqui ligati funibus per muros emittuntur , \\\hline
3.3.22 & Mas si non osan salir los que son cercados de aquella fortaleza \textbf{ estonçe encubiertamente de noche deuen echar algunos } por los muros atados con cuerdas & Sed si munitionem ipsam obsessi exire non audeant : \textbf{ tunc clam de nocte aliqui ligati funibus per muros emittuntur , } qui absconse ignem portantes absque \\\hline
3.3.22 & Et esto fecho los de suso resçiban los por cuerdas a la fortaleza . \textbf{ La iij° manera de destroyr los engeñios } e las algarradas es fazer saetas & machinam incendunt : \textbf{ quo peracto trahuntur superius per funes ad munitionem illam . Est etiam et tertius modus destruendi machinas } faciendo sagittas \\\hline
3.3.22 & La iij° manera de destroyr los engeñios \textbf{ e las algarradas es fazer saetas } que llaman ruecas & quo peracto trahuntur superius per funes ad munitionem illam . Est etiam et tertius modus destruendi machinas \textbf{ faciendo sagittas } quas appellant telos . \\\hline
3.3.22 & engeñio muchas vegadas le quema . \textbf{ La quarta manera para destroyr los engeñios } que lançan las piedras es fazer otros engeñios de dentro & multotiens succendit ipsam . \textbf{ Quarto etiam modo resistitur machinis lapidariis , } faciendo alias machinas interius , \\\hline
3.3.22 & La quarta manera para destroyr los engeñios \textbf{ que lançan las piedras es fazer otros engeñios de dentro } que lançen a ellos & Quarto etiam modo resistitur machinis lapidariis , \textbf{ faciendo alias machinas interius , } percutiendo eas , et destruendo ipsas . \\\hline
3.3.22 & despues que es fecho el engeñio de los de dentro \textbf{ fazerle vna fonda con cadenas de fierro } o texidade fierro . & Inter caetera autem summum remedium est , \textbf{ postquam constituta est machina , interius facere ei fundam ex cathenulis ferreis , } vel testam ex ferro ; \\\hline
3.3.22 & o texidade fierro . \textbf{ Et cerca de aquel engeñio deuen fazer vna fragua } en que pongan vn grand pedaço de fierro & vel testam ex ferro ; \textbf{ et iuxta machinam illam construere fabricam in qua aliquod magnum ferrum bene ignatur , } quod bene ignitum apponatur super fundam ex ferro textam \\\hline
3.3.22 & o a otro qual se quier artifiçio de madera \textbf{ e quemarle ha . } Et contra esto non valen nada los cueros crudos & ad machinam aliam ; \textbf{ vel ad quodcunque aedificium lignorum . } Contra hoc enim coria cruda non valent , \\\hline
3.3.22 & Et contra esto non valen nada los cueros crudos \textbf{ nin la madera non se le puede defender } ca toda cosa fecha de madera se puede quemar en esta manera . & Contra hoc enim coria cruda non valent , \textbf{ ligna non habent resistentiam : } omne enim aedificium ligneum hoc modo comburi potest . Sunt autem \\\hline
3.3.22 & nin la madera non se le puede defender \textbf{ ca toda cosa fecha de madera se puede quemar en esta manera . } Mas avn ay otras muchas cautelas particulares & ligna non habent resistentiam : \textbf{ omne enim aedificium ligneum hoc modo comburi potest . Sunt autem } et multae aliae particulares cautelae , valentes ad defensionem contra lapidarias machinas : \\\hline
3.3.22 & que valen \textbf{ para defender se de las piedras de los engeñios } assi commo sarmientos o tierra cauada . & ø \\\hline
3.3.22 & mas por que estas cosas tales son muchas \textbf{ e non las puede omne conplidamente contar dexamoslas a iuyzio de omnes sabios . } Mostrado en qual manera nos podemos defender de las cueuas coneieras & et multae aliae particulares cautelae , valentes ad defensionem contra lapidarias machinas : \textbf{ sed quia talia complete sub narratione non cadunt , prudentis iudicio relinquantur . Ostenso quomodo resistendum sit cuniculis , et lapidariis machinis : } reliquum est declarare , \\\hline
3.3.22 & e non las puede omne conplidamente contar dexamoslas a iuyzio de omnes sabios . \textbf{ Mostrado en qual manera nos podemos defender de las cueuas coneieras } e de los engeñios & et multae aliae particulares cautelae , valentes ad defensionem contra lapidarias machinas : \textbf{ sed quia talia complete sub narratione non cadunt , prudentis iudicio relinquantur . Ostenso quomodo resistendum sit cuniculis , et lapidariis machinis : } reliquum est declarare , \\\hline
3.3.22 & que lançan las piedras . \textbf{ fincan nos de demostrar } en qual manera nos podamos defender de los otros artifiçios & sed quia talia complete sub narratione non cadunt , prudentis iudicio relinquantur . Ostenso quomodo resistendum sit cuniculis , et lapidariis machinis : \textbf{ reliquum est declarare , } quomodo obuiari debeat aedificiis aliis impulsis ad moenia munitionis obsessae . \\\hline
3.3.22 & fincan nos de demostrar \textbf{ en qual manera nos podamos defender de los otros artifiçios } que pueden ser enpuxados a los muros & reliquum est declarare , \textbf{ quomodo obuiari debeat aedificiis aliis impulsis ad moenia munitionis obsessae . } Ad hoc autem valeret \\\hline
3.3.22 & que dixiemos dessuso \textbf{ para destroyr los engeñios . } Ca assi commo se pueden destroyr los engeñios & Ad hoc autem valeret \textbf{ quaecumque diximus contra resistentiam machinarum . } Nam sicut destrui possunt lapidariae machinae per improuisum insultum obsessorum , \\\hline
3.3.22 & para destroyr los engeñios . \textbf{ Ca assi commo se pueden destroyr los engeñios } por arrebatado e ascondido acometemiento de los que estan çercados & quaecumque diximus contra resistentiam machinarum . \textbf{ Nam sicut destrui possunt lapidariae machinae per improuisum insultum obsessorum , } et per homines de nocte latenter emissos , \\\hline
3.3.22 & que lançen peda os de fierro ençedidos . \textbf{ Et assi todas maneras se pueden destroyr quemar los engeñios } e los artifiçios de madera . & vel per fundas ex ferro textas iacientes ignita ferra : sic \textbf{ omnibus his modis possunt huiusmodi aedificia lignea impugnari . Immo expertum est contra } singula huiusmodi aedificia maxime valere , \\\hline
3.3.22 & que mucho mas vale que otra cosa \textbf{ para quemar estos artifiçios de madera } si fueren lançadas pellas de fierro ençendidas & singula huiusmodi aedificia maxime valere , \textbf{ si per alias machinas , vel } aliquo alio modo , ferra ignita iaciantur in ipsa . \\\hline
3.3.22 & por engeñios o en otra manera . \textbf{ Enpero podemos dar } e mostrar a otros espeçiales remedios contra estos artifiçios de madera . & aliquo alio modo , ferra ignita iaciantur in ipsa . \textbf{ Possumus tamen specialia remedia contra huiusmodi aedificia assignare , } ut contra Arietem constituatur Lupus . \\\hline
3.3.22 & Enpero podemos dar \textbf{ e mostrar a otros espeçiales remedios contra estos artifiçios de madera . } assi que pongamos el lobo contra el carnero . & aliquo alio modo , ferra ignita iaciantur in ipsa . \textbf{ Possumus tamen specialia remedia contra huiusmodi aedificia assignare , } ut contra Arietem constituatur Lupus . \\\hline
3.3.22 & que la viga que ha la cabeça ferrada \textbf{ para ferir en los muros de la fortaleza } por la dureza de la cabeça es llamado carnero & ut contra Arietem constituatur Lupus . \textbf{ Dicebatur enim , trabem ferratam percutientem muros munitionis obsessae propter duritiem capitis vocari Arietem . } Contra hoc autem constituitur quoddam ferrum curuum dentatum dentibus fortissimis , \\\hline
3.3.22 & por la dureza de la cabeça es llamado carnero \textbf{ et contra este carnero se puede fazer vn fierro coruo dentado de dientes muy fuertes e muy agudos } e atado con fuertes cuerdas & Dicebatur enim , trabem ferratam percutientem muros munitionis obsessae propter duritiem capitis vocari Arietem . \textbf{ Contra hoc autem constituitur quoddam ferrum curuum dentatum dentibus fortissimis , | et acutis , } et ligatum funibus , \\\hline
3.3.22 & assi lo ternan enforcado \textbf{ que non pueda enpeesçer a los muros . } por ende los lidiadores antigos & vel ita suspenditur , \textbf{ ut muris nocere non possit . Unde et bellatores antiqui huiusmodi ferrum vocauerunt Lupum , } eo quod acutis dentibus arietem caperet . \\\hline
3.3.22 & Enpero contra esto daremos espeçial remedio . \textbf{ ca pueden se fazer cueuas coneieras de dentro } e carreras soterrañas & adhibetur tamen speciale remedium contra ipsa , \textbf{ quia fiunt cuniculi , } et viae subterraneae , \\\hline
3.3.22 & e carreras soterrañas \textbf{ e ascondidamente se puede cauar la tierra } por que puedan passar allende del castiello o de la villa çercada . & et viae subterraneae , \textbf{ et clam suffoditur terra unde debet transire castrum ; } qua suffossa , \\\hline
3.3.22 & e ascondidamente se puede cauar la tierra \textbf{ por que puedan passar allende del castiello o de la villa çercada . } la qual tierra cauada conuiene de apoyar bien el castiello o la çerca & et clam suffoditur terra unde debet transire castrum ; \textbf{ qua suffossa , } et castro demerso in ipsam propter magnitudinem ponderis , \\\hline
3.3.22 & por que puedan passar allende del castiello o de la villa çercada . \textbf{ la qual tierra cauada conuiene de apoyar bien el castiello o la çerca } por que se non funda & et clam suffoditur terra unde debet transire castrum ; \textbf{ qua suffossa , } et castro demerso in ipsam propter magnitudinem ponderis , \\\hline
3.3.22 & de aquellos muros deuen algunos castiellos de madera \textbf{ o si pueden deuen fazer muros de piedra } assi que si los que çercan entraren de dentro de la fortaleza sean retenidos & iuxta illos muros erigantur aedificia lignea , \textbf{ vel ( si sit possibile ) aedificentur muri lapidei : } ut si continget obsidentes intrare munitionem , \\\hline
3.3.22 & e ençerrados entre aquellos muros \textbf{ assi que se non puedan defender } por el ençerramiento de los muros . & retineantur clausi inter muros illos ; \textbf{ et non valentes se defendere propter murorum inclusionem , } lapidibus obruantur . \\\hline
3.3.22 & Et alli los mataran a piedras . \textbf{ Enpero deuen tener mientes acuciosamente en su fazienda . } ca algunas vezes los que çercan fazen se que fuyen . & lapidibus obruantur . \textbf{ Est tamen diligenter aduertendum , } quod aliquando obsidentes fingunt se fugere , \\\hline
3.3.22 & que tienen çercada \textbf{ e por ende non deuen luego sallir de la fortaleza en pos de los enemigos } nin dexarla desanparada & et versutias inuadunt munitionem obsessam . \textbf{ Ideo non statim post recessum hostium sunt munitiones dimittendae , } et est custodia negligenda . Immo inuestigandae sunt conditiones hostium : \\\hline
3.3.22 & e por ende non deuen luego sallir de la fortaleza en pos de los enemigos \textbf{ nin dexarla desanparada } maguer vean foyr los enemigos . & et versutias inuadunt munitionem obsessam . \textbf{ Ideo non statim post recessum hostium sunt munitiones dimittendae , } et est custodia negligenda . Immo inuestigandae sunt conditiones hostium : \\\hline
3.3.22 & nin dexarla desanparada \textbf{ maguer vean foyr los enemigos . } mas deuen poner en ella guarda & Ideo non statim post recessum hostium sunt munitiones dimittendae , \textbf{ et est custodia negligenda . Immo inuestigandae sunt conditiones hostium : } ut \\\hline
3.3.22 & maguer vean foyr los enemigos . \textbf{ mas deuen poner en ella guarda } e ante deuen escudriñar las condiçiones de los enemigos & Ideo non statim post recessum hostium sunt munitiones dimittendae , \textbf{ et est custodia negligenda . Immo inuestigandae sunt conditiones hostium : } ut \\\hline
3.3.22 & mas deuen poner en ella guarda \textbf{ e ante deuen escudriñar las condiçiones de los enemigos } si son ydos o si non o si estan en çelada . & et est custodia negligenda . Immo inuestigandae sunt conditiones hostium : \textbf{ ut } quod palam habere non potuerunt , per insidias \\\hline
3.3.22 & si son ydos o si non o si estan en çelada . \textbf{ assi que lo que non pueden auer manifiestamente ayan lo por escuchas e por arteras . } E En este postrimero capitulo & ut \textbf{ quod palam habere non potuerunt , per insidias | et astutias obtinere non possint . } In hoc ultimo capitulo tractare volumus aliqua de nauali bello : \\\hline
3.3.23 & E En este postrimero capitulo \textbf{ queremos tractar algunas cosas de la batalla de las naues . } enpero non conuiene de nos de tener çerca esto tanto . & et astutias obtinere non possint . \textbf{ In hoc ultimo capitulo tractare volumus aliqua de nauali bello : } non tamen oportet circa hoc tantum insistere , \\\hline
3.3.23 & queremos tractar algunas cosas de la batalla de las naues . \textbf{ enpero non conuiene de nos de tener çerca esto tanto . } por que muchas cosas que dichas son & In hoc ultimo capitulo tractare volumus aliqua de nauali bello : \textbf{ non tamen oportet circa hoc tantum insistere , } quia multa quae dicta sunt in aliis generibus bellorum , \\\hline
3.3.23 & por que muchas cosas que dichas son \textbf{ en las otras maneras de las batallas se podran traer } e ayuntar a esta lid de las naues . & non tamen oportet circa hoc tantum insistere , \textbf{ quia multa quae dicta sunt in aliis generibus bellorum , } applicari poterunt ad naualem pugnam . \\\hline
3.3.23 & en las otras maneras de las batallas se podran traer \textbf{ e ayuntar a esta lid de las naues . } Mas çerca esta manera de lidiar primeramente es de veer & quia multa quae dicta sunt in aliis generibus bellorum , \textbf{ applicari poterunt ad naualem pugnam . } Circa hoc autem pugnandi genus , \\\hline
3.3.23 & e ayuntar a esta lid de las naues . \textbf{ Mas çerca esta manera de lidiar primeramente es de veer } en qual manera es de fazer la naue . & applicari poterunt ad naualem pugnam . \textbf{ Circa hoc autem pugnandi genus , | primo videndum est , } qualiter fabricanda fit nauis : \\\hline
3.3.23 & Mas çerca esta manera de lidiar primeramente es de veer \textbf{ en qual manera es de fazer la naue . } ca la naue mal fecha & primo videndum est , \textbf{ qualiter fabricanda fit nauis : } nam nauis male fabricata , \\\hline
3.3.23 & por pequena batalla de los enemigos de ligero peresçe . \textbf{ Et pues que assi es conuiene de saber } que segunt que dize vegeçio & de facili perit . \textbf{ Sciendum ergo , } quod secundum Vegetium , \\\hline
3.3.23 & que segunt que dize vegeçio \textbf{ que los maderos que se deue fazer la naue non son de taiar } en qual si quier tienpo . & quod secundum Vegetium , \textbf{ ligna ex quibus construenda est nauis , } non sunt de quolibet tempore incidenda . \\\hline
3.3.23 & ca en el tienpo del março e del abril \textbf{ en que el humor comiença de abondar } e de cresçer en los arboles & ø \\\hline
3.3.23 & en que el humor comiença de abondar \textbf{ e de cresçer en los arboles } non es bueno de taiar los arboles & Nam tempore Martii et Aprilis , \textbf{ in quo humor incipit in arboribus abundare , non est bonum incidere arbores , } ex quibus fabricanda est nauis . \\\hline
3.3.23 & e de cresçer en los arboles \textbf{ non es bueno de taiar los arboles } de los quales deue ser fecha la naue . & Nam tempore Martii et Aprilis , \textbf{ in quo humor incipit in arboribus abundare , non est bonum incidere arbores , } ex quibus fabricanda est nauis . \\\hline
3.3.23 & por que los humores de los arboles se secan \textbf{ son de taiar los maderos } para esta façion de la naue . & in quo humor arborum desiccatur , \textbf{ ad huiusmodi fabricam incidenda sunt ligna . Rursus , } non statim incisis lignis est ex eis fabricanda nauis : \\\hline
3.3.23 & Otrossi taiados los maderos \textbf{ non es luego de fazer la naue dellos . } Mas primero los arboles deuen ser serrados e partidos por tablas & ad huiusmodi fabricam incidenda sunt ligna . Rursus , \textbf{ non statim incisis lignis est ex eis fabricanda nauis : } sed primo arbores sunt diuidendae per tabulas ; \\\hline
3.3.23 & Mas primero los arboles deuen ser serrados e partidos por tablas \textbf{ e por algun tienpo son de dexar } que esten assi & sed primo arbores sunt diuidendae per tabulas ; \textbf{ et per aliquod tempus dimittendae , } ut desiccari possint . \\\hline
3.3.23 & que esten assi \textbf{ por que se puedan secar . } ca si la naue se faze de madera verde & et per aliquod tempus dimittendae , \textbf{ ut desiccari possint . } Nam si ex lignis viridibus construatur nauis , \\\hline
3.3.23 & que estas aberturas . \textbf{ porque graue cosa es de tener mientes a estas dos cosas } en vno a la batalla de las naues & ø \\\hline
3.3.23 & e al periglo de las aberturas . \textbf{ por las quales la naua puede peresçer . } visto en qual manera es de taiar la madera . & ne puppis per rimas naufragium patiatur . \textbf{ Viso qualiter incidenda sunt ligna , } et quomodo reseruanda , \\\hline
3.3.23 & por las quales la naua puede peresçer . \textbf{ visto en qual manera es de taiar la madera . } e en qual manera es de guardar & ne puppis per rimas naufragium patiatur . \textbf{ Viso qualiter incidenda sunt ligna , } et quomodo reseruanda , \\\hline
3.3.23 & visto en qual manera es de taiar la madera . \textbf{ e en qual manera es de guardar } e de poner a secar & Viso qualiter incidenda sunt ligna , \textbf{ et quomodo reseruanda , } ut ex eis nauis debite valeat fabricari : \\\hline
3.3.23 & e en qual manera es de guardar \textbf{ e de poner a secar } por que della se pueda fazer la naue conueniblemente . & et quomodo reseruanda , \textbf{ ut ex eis nauis debite valeat fabricari : } restat videre ; \\\hline
3.3.23 & e de poner a secar \textbf{ por que della se pueda fazer la naue conueniblemente . } finca de veer & et quomodo reseruanda , \textbf{ ut ex eis nauis debite valeat fabricari : } restat videre ; \\\hline
3.3.23 & por que della se pueda fazer la naue conueniblemente . \textbf{ finca de veer } commo son de acometer las batallas & ut ex eis nauis debite valeat fabricari : \textbf{ restat videre ; } quomodo in naui bene fabricata committenda sunt bella . Habet autem nauale bellum quantum ad aliqua similem modum bellandi cum ipsa pugna terrestri . \\\hline
3.3.23 & finca de veer \textbf{ commo son de acometer las batallas } en la naue bien fecha e bien formada . & ut ex eis nauis debite valeat fabricari : \textbf{ restat videre ; } quomodo in naui bene fabricata committenda sunt bella . Habet autem nauale bellum quantum ad aliqua similem modum bellandi cum ipsa pugna terrestri . \\\hline
3.3.23 & ca la batalla de las naues \textbf{ a semeiança de lidiar en algunas cosas } con la batalla de la tierra . & ø \\\hline
3.3.23 & conuiene que los lidiadores sean bien armados \textbf{ e avn que se sepan bien cobrir } e guardar de los colpes e que sepan ferir a los enemigos . & Nam sicut terrestri pugna oportet pugnantes bene armatos esse , \textbf{ et bene se scire a persecutionibus protegere , } et hostibus vulnera infligere : \\\hline
3.3.23 & e avn que se sepan bien cobrir \textbf{ e guardar de los colpes e que sepan ferir a los enemigos . } Bien assi todas estas cosas fazen menester en la batalla de la naue . & et bene se scire a persecutionibus protegere , \textbf{ et hostibus vulnera infligere : } sic \\\hline
3.3.23 & e mueuen se muy poco . \textbf{ Et por ende meior pueden sofrir el peso de las armas . } Por la qual cosa las armaduras dellos deuen ser mas pesadas . & et quasi modicum se moueant , \textbf{ melius sustinere possunt armorum pondera : } quare eorum armatura grauior esse debet . \\\hline
3.3.23 & Por la qual cosa las armaduras dellos deuen ser mas pesadas . \textbf{ Enpero quanro pertenesçe a lo presente podemos contar diez cosas . } por las quales los lidiadores de la mar pueden acometer & quare eorum armatura grauior esse debet . \textbf{ Possumus tamen , } quantum ad praesens , \\\hline
3.3.23 & Enpero quanro pertenesçe a lo presente podemos contar diez cosas . \textbf{ por las quales los lidiadores de la mar pueden acometer } e vençer a sus enemigos . & quare eorum armatura grauior esse debet . \textbf{ Possumus tamen , } quantum ad praesens , \\\hline
3.3.23 & por las quales los lidiadores de la mar pueden acometer \textbf{ e vençer a sus enemigos . } Lo primero es fuego fuerte & Possumus tamen , \textbf{ quantum ad praesens , } decem enumerare , \\\hline
3.3.23 & en o fuego de alquitran \textbf{ ca conuine de auer en las naues mucha usija de tierra } assi commo cantaros & decem enumerare , \textbf{ per quae marini pugnatores hostes impugnare debent . Primum est ignis , } quem Incendiarium vocant . \\\hline
3.3.23 & e de rasina e de olio . \textbf{ las quales cosas todas son de enboluer con estopa . } Et estos belhezos tales assi llenos son de ençender & Expedit enim eis habere multa vasa plena pice , sulphure , rasina , oleo ; \textbf{ quae omnia sunt cum stupa conuoluenda . Haec enim vasa sic repleta sunt succendenda , } et proiicienda ad nauem hostium . Ex qua proiectione vas frangitur , et illud incendiarium comburitur \\\hline
3.3.23 & las quales cosas todas son de enboluer con estopa . \textbf{ Et estos belhezos tales assi llenos son de ençender } e de lançar a las naues de los enemigos . & Expedit enim eis habere multa vasa plena pice , sulphure , rasina , oleo ; \textbf{ quae omnia sunt cum stupa conuoluenda . Haec enim vasa sic repleta sunt succendenda , } et proiicienda ad nauem hostium . Ex qua proiectione vas frangitur , et illud incendiarium comburitur \\\hline
3.3.23 & Et estos belhezos tales assi llenos son de ençender \textbf{ e de lançar a las naues de los enemigos . } Et lançandolos assi en las naues quebrantan se los cantaros & quae omnia sunt cum stupa conuoluenda . Haec enim vasa sic repleta sunt succendenda , \textbf{ et proiicienda ad nauem hostium . Ex qua proiectione vas frangitur , et illud incendiarium comburitur } et succendit nauem . \\\hline
3.3.23 & e quema la naue . \textbf{ Et deuen echar muchos tales cantaros en la naue de los enemigos . } por que de muchas partes se pueda quemar la naue . & et succendit nauem . \textbf{ Sunt enim multa talia in naui proiicienda , } ut ex multis partibus possit nauis succendi ; \\\hline
3.3.23 & Et deuen echar muchos tales cantaros en la naue de los enemigos . \textbf{ por que de muchas partes se pueda quemar la naue . } Et entonçe deuen acometer muy fuerte batalla contra los enemigos & Sunt enim multa talia in naui proiicienda , \textbf{ ut ex multis partibus possit nauis succendi ; } et cum proiiciuntur talia , \\\hline
3.3.23 & por que de muchas partes se pueda quemar la naue . \textbf{ Et entonçe deuen acometer muy fuerte batalla contra los enemigos } por que se non puedan acorrer & ut ex multis partibus possit nauis succendi ; \textbf{ et cum proiiciuntur talia , | tunc est contra nautas committendum durum bellum , } ne possint currere ad extinguendum ignem . Secundo ad committendum marinum bellum multum valent insidiae . \\\hline
3.3.23 & Et entonçe deuen acometer muy fuerte batalla contra los enemigos \textbf{ por que se non puedan acorrer } para matar el fuego . & ø \\\hline
3.3.23 & por que se non puedan acorrer \textbf{ para matar el fuego . } Lo segundo para acometer batalla en la mar valen mucho las çeladas . & tunc est contra nautas committendum durum bellum , \textbf{ ne possint currere ad extinguendum ignem . Secundo ad committendum marinum bellum multum valent insidiae . } Nam sicut in terra ponuntur insidiae militum , \\\hline
3.3.23 & para matar el fuego . \textbf{ Lo segundo para acometer batalla en la mar valen mucho las çeladas . } ca assi commo en la tierra los caualleros ponen çeladas a los enemigos & tunc est contra nautas committendum durum bellum , \textbf{ ne possint currere ad extinguendum ignem . Secundo ad committendum marinum bellum multum valent insidiae . } Nam sicut in terra ponuntur insidiae militum , \\\hline
3.3.23 & ca assi commo en la tierra los caualleros ponen çeladas a los enemigos \textbf{ por que sin su apercebimiento los puedan acometer } e los espanten a desora & qui ex improuiso inuadentes hostes , \textbf{ eos terrent , } et de facili vincunt : sic in mari post aliquas insulas fiunt insidiae , \\\hline
3.3.23 & por que mas ligeramente los venzcan . \textbf{ lo terçero en la batalla de la mar conuiene de tener mientes } que los que lidian sobre mar & ut marini pugnatores ex improuiso irruentes in hostes , eos facilius vincant . Tertio est circa marinum bellum attendendum , \textbf{ ut semper pugnantes nauem } suam faciant \\\hline
3.3.23 & aquellos que se lleguan a la tierra . \textbf{ Lo quarto conuiene de colgar al maste de la naue vn madero luengo e ferrado de a mas partes . } para ferir tan bien en la naue commo en los marineros & qui detrahuntur ad terram . \textbf{ Quarto ad arborem nauis suspendendum est lignum quoddam longum ex utraque parte ferratum , } quod ad percutiendum tam nauem , \\\hline
3.3.23 & Lo quarto conuiene de colgar al maste de la naue vn madero luengo e ferrado de a mas partes . \textbf{ para ferir tan bien en la naue commo en los marineros } que sea tal commo el carnero & Quarto ad arborem nauis suspendendum est lignum quoddam longum ex utraque parte ferratum , \textbf{ quod ad percutiendum tam nauem , } quam nautas se habeat quasi aries , \\\hline
3.3.23 & que sea tal commo el carnero \textbf{ con el qual suelen quebrar los muros de la çibdat çercada . } Et deue este madero & quam nautas se habeat quasi aries , \textbf{ cum quo teruntur muri ciuitatis obsessae . } Debet autem sic ordinari lignum illud , \\\hline
3.3.23 & que con el atadura \textbf{ que el tiene se pueda alçar e baxar } ca esto echo siguiesse meior prouecho del ca pueden ferir tan bien en la naue commo en los que estan en ella & Debet autem sic ordinari lignum illud , \textbf{ ut ligamentum retinens ipsum possit deprimi , et eleuari : } quia hoc facto maior habetur commoditas , \\\hline
3.3.23 & que el tiene se pueda alçar e baxar \textbf{ ca esto echo siguiesse meior prouecho del ca pueden ferir tan bien en la naue commo en los que estan en ella } assi commo con garrocha & ut ligamentum retinens ipsum possit deprimi , et eleuari : \textbf{ quia hoc facto maior habetur commoditas , | ut cum ipso percuti possit tam nauis , } quam \\\hline
3.3.23 & assi commo con garrocha \textbf{ lo quinto en la batalla de la mar conuiene de auer grand conplimiento de saetas anchas . } con las quales se pueden ronper las uelas & etiam existentes in ipsa . \textbf{ Quinto in bello nauali | habenda est copia ampliarum sagittarum , } cum quibus scindenda sunt vela hostium . \\\hline
3.3.23 & lo quinto en la batalla de la mar conuiene de auer grand conplimiento de saetas anchas . \textbf{ con las quales se pueden ronper las uelas } e los treos de las naues de los enemigos . & habenda est copia ampliarum sagittarum , \textbf{ cum quibus scindenda sunt vela hostium . } Nam velis eorum perforatis , \\\hline
3.3.23 & ca foradadas las velas \textbf{ e los treos non pueden retener el viento . } Et assi non pueden los enemigos auer tanta fuerça & Nam velis eorum perforatis , \textbf{ et non valentibus retinere ventum ; } non tantum possunt ipsi hostes impetum habere pugnandi , \\\hline
3.3.23 & e los treos non pueden retener el viento . \textbf{ Et assi non pueden los enemigos auer tanta fuerça } para acometer & et non valentibus retinere ventum ; \textbf{ non tantum possunt ipsi hostes impetum habere pugnandi , } nec etiam possunt sic faciliter recedere , \\\hline
3.3.23 & Et assi non pueden los enemigos auer tanta fuerça \textbf{ para acometer } nin para lidiar & et non valentibus retinere ventum ; \textbf{ non tantum possunt ipsi hostes impetum habere pugnandi , } nec etiam possunt sic faciliter recedere , \\\hline
3.3.23 & para acometer \textbf{ nin para lidiar } nin a vn se pueden yr ligeramente & ø \\\hline
3.3.23 & nin para lidiar \textbf{ nin a vn se pueden yr ligeramente } ni pueden foyr & non tantum possunt ipsi hostes impetum habere pugnandi , \textbf{ nec etiam possunt sic faciliter recedere , } si volunt declinare a bello . Sexto consueuerunt nautae habere ferrum quoddam curuatum ad modum falcis bene incidens , \\\hline
3.3.23 & nin a vn se pueden yr ligeramente \textbf{ ni pueden foyr } si quisiere foyr de la batalla . & non tantum possunt ipsi hostes impetum habere pugnandi , \textbf{ nec etiam possunt sic faciliter recedere , } si volunt declinare a bello . Sexto consueuerunt nautae habere ferrum quoddam curuatum ad modum falcis bene incidens , \\\hline
3.3.23 & ni pueden foyr \textbf{ si quisiere foyr de la batalla . } Et lo sexto suelen los marineros auer vn fierro coruo bien agudo & nec etiam possunt sic faciliter recedere , \textbf{ si volunt declinare a bello . Sexto consueuerunt nautae habere ferrum quoddam curuatum ad modum falcis bene incidens , } quod applicatum ad funes retinentes vela ; \\\hline
3.3.23 & si quisiere foyr de la batalla . \textbf{ Et lo sexto suelen los marineros auer vn fierro coruo bien agudo } e bien taiante & nec etiam possunt sic faciliter recedere , \textbf{ si volunt declinare a bello . Sexto consueuerunt nautae habere ferrum quoddam curuatum ad modum falcis bene incidens , } quod applicatum ad funes retinentes vela ; \\\hline
3.3.23 & e derribadas non han poder los marineros \textbf{ de se defender de los enemigos . } Ca derribadas assi las velas la naue es mas perezosa & ne sic pugnare possint : \textbf{ quia per talem incisionem velorum redditur nauis pigrior , } et quodammodo inutilior ad pugnandum . \\\hline
3.3.23 & Ca derribadas assi las velas la naue es mas perezosa \textbf{ e non puede yr } por la mar & quia per talem incisionem velorum redditur nauis pigrior , \textbf{ et quodammodo inutilior ad pugnandum . } Septimo consueuerunt e iam nautae habere uncos ferreos fortes , \\\hline
3.3.23 & por la mar \textbf{ nin ha poder de lidiar . lo . vij° . } suelen avn los marineros auer coruos de fierro muy fuertes & et quodammodo inutilior ad pugnandum . \textbf{ Septimo consueuerunt e iam nautae habere uncos ferreos fortes , } ut cum vident se esse plures hostibus , \\\hline
3.3.23 & nin ha poder de lidiar . lo . vij° . \textbf{ suelen avn los marineros auer coruos de fierro muy fuertes } e quando veen & et quodammodo inutilior ad pugnandum . \textbf{ Septimo consueuerunt e iam nautae habere uncos ferreos fortes , } ut cum vident se esse plures hostibus , \\\hline
3.3.23 & con aquellos coruos prenden las naues dellos \textbf{ et non los dexan foyr . Lo . viij̇° . es de tomar esta cautela en la batalla dela naue } que conuiene que se fincan muchas cantaras de cal poluorizada . & cum illis uncis capiunt eorum naues , \textbf{ ut non permittant eos discedere . Octauo in nauali bello est haec cautela attendenda : } ut de calce alba puluerizata habeant multa vasa plena , \\\hline
3.3.23 & Esto tal ciega los oios de los enemigos \textbf{ assi que non podran ver } e assi commo çiegos no se podran defender . & et adeo offendit eos \textbf{ ut quasi caeci videre non possint : } quod in bello nauali est valde periculosum , \\\hline
3.3.23 & assi que non podran ver \textbf{ e assi commo çiegos no se podran defender . } la qual cosa es muy periglosa en la batalla de las naues & et adeo offendit eos \textbf{ ut quasi caeci videre non possint : } quod in bello nauali est valde periculosum , \\\hline
3.3.23 & assi se ciegan del poluo de la cal \textbf{ que non pueden ver de ligo } o morran a manos de sus enemigos & quare si oculi bellantium in tali pugna ex puluere calcis sic offenduntur , \textbf{ ut videre non possint ; } de facili vel perimuntur ab hostibus , \\\hline
3.3.23 & La . ix . \textbf{ cautela es auer muchos cantaros llenos dexabon muelle } que lançen de rezio en las naues de los enemigos . & vel submerguntur in aquis . \textbf{ Nona cautela est habere multa vasa plena ex molli sapone , } quae cum impetu proiicienda sunt ad naues hostium ; \\\hline
3.3.23 & Et esto sobre aquellos logares \textbf{ en que an de estar los enemigos para defender las naues . } Ca quebrantados aquellos cantaros en las naues & quae cum impetu proiicienda sunt ad naues hostium ; \textbf{ et hoc super loca illa , in quibus contingit hostes existere ad defendendum naues . } Nam vasis illis confractis in huiusmodi locis , loca illa per saponem liquidam redduntur adeo lubrica , \\\hline
3.3.23 & fazen se escorredizos por el xabon \textbf{ en tal manera que los enemigos non pueden y tener los pies } e caen en la mar & Nam vasis illis confractis in huiusmodi locis , loca illa per saponem liquidam redduntur adeo lubrica , \textbf{ quod hostes ibi ponentes pedes } statim labuntur in aquis . \\\hline
3.3.23 & e los marineros deuen se \textbf{ assi ordenar sabiamente contra las naues de los enemigos } que detras de la naue & qui diu sub aquis durare possunt : \textbf{ nautae igitur debent se serio ordinare contra nauem hostium , } et clam post tergum debent \\\hline
3.3.23 & e encubiertamente echen algunoo algunos en la mar \textbf{ que puedan y mucho estar e lieuen taladros para foradar } e llegunen se a la naue de los enemigos so el agua & aliquem emittere diu valentem durare sub aquis ; \textbf{ qui accepto penetrali sub aquis debet accedere ad hostilem nauem , } et eam in profundo perforare , \\\hline
3.3.23 & Et faziendo muchos forados \textbf{ los quales non podran los enemigos çercar } quando entrare mucha agua dentro en la naue sabullira a los enemigos & faciendo ibi plura foramina , \textbf{ quae foramina ab hostibus reperiri non poterunt , } cum per ipsa coeperit abundare aqua , qua abundante , et hostes , \\\hline
3.3.23 & Mas ay otras cosas \textbf{ que son de guardar en la batalla de las naues } assy que ayan y muy grand conplimiento de piedras & et nauem periclitabit . \textbf{ Sunt autem in bello nauali alia obseruanda , } ut sit ibi copia lapidum , \\\hline
3.3.23 & Mas estas otras tales cosas por que son muy particulares \textbf{ non las podemos contar } por menudo . & cum quibus hostes nimium offenduntur . \textbf{ Sed caetera talia quia nimis particularia sunt , } sub narratione non cadunt . Sufficiant ergo cautelae , \\\hline
3.3.23 & que auemos dadas cerca las batallas de las naues . \textbf{ Mostrado en qual manera es de taiar la madera } para fazer las naues & sub narratione non cadunt . Sufficiant ergo cautelae , \textbf{ quas tradidimus erga nauale bellum . Ostenso qualiter incidenda sunt ligna } ex quibus construenda est nauis , \\\hline
3.3.23 & Mostrado en qual manera es de taiar la madera \textbf{ para fazer las naues } e en qual manera auemos de lidiar en la mar . & quas tradidimus erga nauale bellum . Ostenso qualiter incidenda sunt ligna \textbf{ ex quibus construenda est nauis , } et quomodo bellandum est in nauali bello . \\\hline
3.3.23 & para fazer las naues \textbf{ e en qual manera auemos de lidiar en la mar . } finca nos de mostra concluyendo & ex quibus construenda est nauis , \textbf{ et quomodo bellandum est in nauali bello . } Reliquum est , \\\hline
3.3.23 & commo las de la tierra . \textbf{ Et para esto saber conuiene de notar } que segunt el philosofo non lidiamos & ad quid bella omnia ordinantur . \textbf{ Sciendum ergo quod } secundum philosophum non bellamus , \\\hline
3.3.23 & Et enpero las batallas sy derechamente las fezieren \textbf{ e las tomaren conueniblemente son de ordenar . } a paz e assossiego de los omnes . & Bella tamen si iuste gerantur , \textbf{ et debite fiant , | ordinanda sunt ad pacem , } et ad quietem hominum , \\\hline
3.3.23 & e al bien comun . \textbf{ ca assi se deuen auer las batallas } en la muchedunbre de los omnes & et ad commune bonum . \textbf{ Nam sic se debent habere bella in societate hominum , } sicut se habent potiones , \\\hline
3.3.23 & al otro non ay \textbf{ porque auer batalla ninguna . } Por la qual cosa & ø \\\hline
3.3.23 & por la quase enbarga la salut del cuerpo . \textbf{ assi por las batallas son los enemigos de taiar e de cortar . } por los quales se enbarga el bien comun & et potionem superfluitas humorum est eiicienda per quam turbatur sanitas corporis : \textbf{ sic per bella sunt hostes conculcandi , | et occidendi , } per quos impeditur commune bonum , \\\hline
3.3.23 & que nos ensseñemos e demos doctrina a todos los prinçipes \textbf{ para saber todas las maneras de lidiar } e toda manera . & et pacem ciuium . \textbf{ Nam si intendant commune bonum , | et pacem ciuium , } merebuntur pacem illam aeternam , \\\hline
3.3.23 & e toda manera . \textbf{ por que puedan vençer sus enemigos . }  & in qua est suprema requies : \textbf{ quam Deus ipse suis promisit fidelibus , | qui est benedictus in saecula saeculorum . Amen . FINIS }  \\\hline

\end{tabular}
