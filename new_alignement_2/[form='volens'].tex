\begin{tabular}{|p{1cm}|p{6.5cm}|p{6.5cm}|}

\hline
1.1.2 & nisi quis se det bonis actibus , et bonis operibus regulatis ordine rationis : \textbf{ volens tractare de regimine sui , } oportet ipsum notitiam tradere de omnibus his quae diuersificant mores et actiones . & sy non que se de el omne abunos fechos e abunas obras rregladas por orden de Razon \textbf{ el que quiere tractar del gouernaiento } e fablarde sy mesmo conujene de tractar e de dar conosçimiento de todas aquellas cosas \\\hline
1.1.5 & non laudamur nec vituperamur : unde in eodem 3 dicitur , \textbf{ quod nullus est beatus nisi volens . } Sed quae non agimus ex electione , non agimus volentes : & Ca asy lo dize el philosofo en el terçero libro de las ethicas \textbf{ que njguon non es bien auer turado | si non obra de voluntad } Mas aquellas cosas que nos non fazemos por electiuo non las fazemos de uoluntad ¶ \\\hline
1.1.5 & sed à fortuna . Unde Philosophus 1 Ethicor’ \textbf{ volens ostendere } necessariam esse praecognitionem finis , ait , quod cognitio finis & esto non es por uoluntad mas es por auentra \textbf{ a¶ Onde el philosofo quariendo mostrar en el primero libro delas ethicas } que es neçesario de connosçer ante la fin dize \\\hline
1.2.8 & Ratione igitur huiusmodi cognoscendi , qui est inditus hominibus , \textbf{ volens alios dirigere , } oportet quod sit intelligens , cognoscendo principia , & Et pues que assi es por razon desta manera de conos çer que es enxerida naturalmente alos omes . \textbf{ El que quiere alos otros guiar } conuiene le que sea entendido conosciendo los prinçipios e las razones . \\\hline
1.2.14 & Fortitudo enim ciuilis est , quando aliquis timens verecundiam , \textbf{ et volens honorem adipisci , } aggreditur aliquod terribile , unde ait Philosophus , & ¶La fortaleza çeuiles quando alguno temiendo uerguença \textbf{ e quariendo ganar honrra } acomete alguna cosa fuerte e espantable . Onde dize el philosofo \\\hline
1.2.16 & quod , cum quidam Dux exercitus diu ei seruiuisset , et fideliter , \textbf{ Rex ille volens complacere illi Duci , } praecepit quod duceretur ad ipsum . Dux autem ille assuetus rebus bellicis , & lo que auian de fazer Et acaesçio que vn prinçipe mucho su priuado que grant t p̃o le auia seruido e fiel mente . \textbf{ Et el Rey que tiendo fazer plazer a aquel } prinçipe mando qual pusiessen dentro ante si . Mas aquel prinçipe por que era acostunbrado delas batallas \\\hline
2.2.5 & etiam ab ipsa infantia . Unde et Philosophus 2 Meta’ \textbf{ volens probare } consuetudinem esse magnae efficaciae , ait , quantam vero vim habeat & Por ende tales cosas nos deuen anos dezir e demostrar en tienpo dela moçedat . Onde el philosofo en el segundo libro dela \textbf{ methafisica quariendo prouar } que la costunbre es de grand fuerça dize assi . Tu puedes ver \\\hline
2.2.9 & Debet enim esse memor , recolendo praeterita . \textbf{ Nam sicut volens rectificare virgam , } nunquam eam rectificare posset nisi cognosceret & enlo que ha de fazer ¶ Ca primero deue ser menbrado e acordado delas colas passadas . \textbf{ Ca assi commo aquel que quiere enderesçar la pierte } ga nunca la puede enderesçar si non conosçiere de qual parte esta tuerta . En essa misma manera aquel que quiere enderesçar los otros \\\hline
2.2.9 & nisi cognosceret ex qua parte esset obliquata : \textbf{ sic volens alios rectificare } nunquam eos congrue rectificare posset nisi haberet praeteritorum notitiam , & Ca assi commo aquel que quiere enderesçar la pierte ga nunca la puede enderesçar \textbf{ si non conosçiere de qual parte esta tuerta . En essa misma manera aquel que quiere enderesçar los otros } nunca los podria enderesçar si non ouiesse conosçimiento delas cosas passadas \\\hline
2.3.1 & et aliis instrumentis fabrilibus ; et spectat ad fabrum talia instrumenta cognoscere . \textbf{ Sic volens tradere notitiam de arte textoria , } debet determinare de pectinibus , et aliis organis illius artis , & Et al ferero pertenesçe de cognosçertales estrumentos . \textbf{ Et dessa misma manera | el que quiere dar conosçimiento del arte del texer } deue determinar de los peinnes e de los otros estrumentos de aquella arte \\\hline
2.3.1 & et spectat ad textorem talia instrumenta cognoscere . \textbf{ Quare volens tradere notitiam } de arte gubernationis domus , debet determinare de aedificiis , possessionibus , et numismatibus : & e de los otros estrumentos de aquella arte e pertenesçe al texedor de conosçer tales estrumentos . \textbf{ Por la qual cosa el que quisiere dar conosçimiento del arte del gouernamiento dela casa } deue determinar de los hedifiçios e delas posessiones e de los dineros \\\hline
3.1.10 & ut quod filii cognoscerent matres , et patres filias . \textbf{ Socrates volens hoc inconueniens vitare , } dixit , quod spectabat ad Principem ciuitatis & assi que los fijos baratarian e iazdrian con sus madres e los padres con sus fijas . \textbf{ Empero socrates quariendo escusar este mal dix̉o } que al prinçipe dela çibdat pertenesçia de auer cuydado e acuçia por que los fijos non yoguiessen con sus madres \\\hline
3.1.14 & in constituendo determinatum numerum bellatorum . Nam secundum Philosophum secundo Politicorum , \textbf{ volens ponere leges } vel facere ordinationem aliquam in ciuitate , ad tria debet respicere , & ca segunt dize el philosofo en el segundo libro delas politicas \textbf{ el que quiere poner leyes o fazer ordenaçion alguna en la çibdat a tres } co sas deue deuer mietes . ¶ Conuiene a saber alos omes Et al regno . \\\hline
3.1.14 & Quare cum ars et scientia non possit esse circa particularia signata , \textbf{ volens tradere artem de regimine ciuitatum , } non potest statuere determinatum numerum bellatorum : sed talia relinquenda sunt iudicio prudentis rectoris & por la qual cosa la arte e la sçiençia non pueden ser cerca las cosas particulares e senñaladas . \textbf{ El que quiere dar arte e sçiençia de gouernamiento dela çibdat } non puede escablesçer cuento determinado de los lidiadores \\\hline
3.1.15 & Quod autem ciuitatem diuidebat in agricolas , artifices , et bellatores , \textbf{ volens ciuitatem } ad minus mille continere bellatores . Forte per bellatores intendebat nobiles , & en menestrales e en labradores e en batalladores quariendo \textbf{ que alo menos la çibdat ouiesse mil ł batalladores } por auentura por los batalladores entendia nobles omes \\\hline
3.1.16 & ex his quae videbat in politiis aliis , statuit potissime curandum esse de possessionibus ciuium , \textbf{ volens eos aequatas possessiones habere . } Si considerentur dicta Philosophi 2 Politicorum , quantum ad praesens spectat , & Et por ende establesçio que much deuia ser tomado grant cuydado delas possessiones de los çibdadanos \textbf{ e quaria que los çibdadanos ouiessen las possessiones eguales } i fueren penssados los dichos del philosofo en el segundo libro delas politicas \\\hline
3.2.12 & quare ipse semper tristis existeret , et quare nunquam hylarem vultum ostenderet . \textbf{ Tyrannus ille volens reddere causam quaesiti , } eum expoliari fecit , et ligari : & por que sienpre andaua triste que nunca mostraua la cara alegte \textbf{ e aquel tirano quariendo dar razon desto fizo despoiar a su hͣmano } e fizola tar e fizol colgar una espada sobre su cabesça muy aguda de vn filo muy delgado \\\hline
3.2.21 & ut recte iudicet , sic debet se habere inter partes litigantes , \textbf{ sicut lingua volens discernere de saporibus , } vel sicut quilibet alius sensus volens discernere de proprii sensibilibus , & assi se deue auer entre las partes que contienden \textbf{ commo la lengua | quando quiere iudgar de los sabores } o si commo cada vno de los otros sesos quando quieren iudgar delas otras cosas \\\hline
3.2.21 & sic debet se habere inter partes litigantes , sicut lingua volens discernere de saporibus , \textbf{ vel sicut quilibet alius sensus volens discernere } de proprii sensibilibus , habere se debet & commo la lengua quando quiere iudgar de los sabores \textbf{ o si commo cada vno de los otros sesos | quando quieren iudgar delas otras cosas } que sienten propriamente ca la lengua se deue auer entre los sabores \\\hline
3.2.23 & quam iniuriae illatae . Ideo Philos’ 1 Rhet’ \textbf{ volens iudicantes } ad misericordiam adducere erga delinquentes in ipsos ; & que nos auie fech por ende el pho en el primero libro de la \textbf{ rectorica quariendo enduzir los iuezes a misericordia } contra los que yerran contra ellos dize que mas se deuen acordar de los biens \\\hline
3.2.32 & et animaduertendum est quod bonorum illorum sit potius . \textbf{ Narrat quidem Philosophus 3 Politic’ volens diffinire } quid sit ciuitas , sex bona ad quae ciuitas ordinatur . & Et deuemos tener mientes qual de aquellos bienes es el meior . \textbf{ Et cuenta el philosofo enel terçero libro delas politicas | quariendo de el arar } que cosa es la çibdat seys bienes alos quales es ordenada la çibdat . \\\hline

\end{tabular}
