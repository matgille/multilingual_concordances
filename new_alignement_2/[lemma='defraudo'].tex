\begin{tabular}{|p{1cm}|p{6.5cm}|p{6.5cm}|}

\hline
2.3.16 & quia sunt decipientes , \textbf{ et defraudantes iura legalia : aliqui vero male consequuntur ipsum , } sed non ex malitia voluntatis , & as otros ay \textbf{ que fazen mal el ofiçio | que les esacomne dado } mas esto non es por maliçia de uoluntad . \\\hline
2.3.16 & nec fraudent , decipiuntur \textbf{ tamen et defraudantur . } Conditio ergo ministrantium esse debet , & que ellos non engannen \textbf{ nin amenguen los derechs de los sennores . enpero son engannados e menguados en ssi por non lo entender ¶ Et pues que assi es las condiconnes de los seruientes deuen ser tales } que ellos sean fieles e sabios conuiene saber \\\hline
2.3.16 & prudentes vero quantum ad industriam intellectus , \textbf{ ne per insipientiam defraudentur . } Fidelitas autem cognosci habet per diuturnitatem : & quanto al acuçia del entendimiento \textbf{ por que non sean engannados | por non saber . } Mas la fiesdat se puede conosçer \\\hline
2.3.19 & nam si minister constituendus in aliquo officio vel in aliquo magistratu debet esse fidelis \textbf{ ne defraudet , } prudens & ca si el seruiente es de poner en algun ofiçio o en algun mahestradgo deue ser fiel \textbf{ por que non enganne e sabio } por qua non sea engannado . \\\hline
2.3.19 & prudens \textbf{ ne defraudetur : } quanto plus constat de eius fidelitate et prudentia , & por que non enganne e sabio \textbf{ por qua non sea engannado . } por que quanto mas es omne çierto de su fieldat \\\hline
3.2.19 & Rursus est attendendum , \textbf{ ne in suis prouentibus defraudetur : } expedit enim regium consilium pro viribus saluare iura Regis , & sin derech \textbf{ lo segundo ha de tener mient̃s el Rey de non ser engannado enlas sus rentas . | ca conuiene que el conseio del Rey sea bue no } para saluar \\\hline

\end{tabular}
