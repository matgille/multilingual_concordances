\begin{tabular}{|p{1cm}|p{6.5cm}|p{6.5cm}|}

\hline
1.1.11 & o ouiere buena proporçion de los huessos e de los neruios \textbf{ e si fuere fuerte en el cuerpo . } Mas si ouiere las potençias & vel habeat proportionem ossium et neruorum , \textbf{ et sit robustus corporaliter . } Sed si habeat aequatas potentias , \\\hline
1.1.11 & e al entendemiento \textbf{ Et fuere sano e fuerte en la uoluntad } Et si estas potençias del alma fueren conpuestas & ut quod inferiores potentiae subsint rationi ; \textbf{ et sit sanus , | et fortis mente : } et si huiusmodi potentias habeat \\\hline
1.2.2 & e por el mal \textbf{ en quanto han razon de cosa guaue e fuerte . } Ca commo el bien & irascibilis vero respicit bonum , \textbf{ et malum inquantum habent rationem difficilis , et ardui . } Nam cum bonum secundum se dicat prosequendum , \\\hline
1.2.5 & e iusta mente . \textbf{ fuerte mente . e tenprada mente . } Et por ende estas quatro uirtudes son dichas & si virtuosus esse debet , \textbf{ oportet quod fiat prudenter , iuste , fortiter , et temperate : } ideo hae quatuor uirtutes , \\\hline
1.2.10 & Ca commo quier que el iusto legal faga essas mismas obras \textbf{ que faze el fuerte e el tenprado . } Enpero non las faze segunt aquella enteçion & nam licet eadem opera agat Iustus legalis , \textbf{ quae agit fortis , et temperatus : } non tamen aget ea \\\hline
1.2.10 & Ca aquel que faze las obras fuertes \textbf{ en quanto se delecta en ellas es dicho fuerte . } Et el que faze las obras tenpradas & Nam qui agit opera fortia , \textbf{ quia delectatur in talibus , | fortis est , } et agens temperata , \\\hline
1.2.10 & En quanto todas aquellas obras son mandadas e ordenadas por la ley . \textbf{ Mas el fuerte e el tenprado } e el acabado & a lege praecepta . \textbf{ Fortis autem , et temperatus , } vel perfectus \\\hline
1.2.10 & por las quales es honrrado¶ \textbf{ Mas si el fuerte e el tenprado se deleytan en conplimiento de la ley } esto es & quibus ornatur . \textbf{ Si autem delectatur in impletione legis , } hoc est ex consequenti , \\\hline
1.2.10 & Et desta diferenços se sigue la segunda . \textbf{ Ca commo el fuerte e el tenprado se deleite } segunt & Ex ista autem differentia sequitur secunda : \textbf{ nam cum fortis , | et temperatus delectetur } in operibus talium virtutum , \\\hline
1.2.13 & mas es loco e landio . \textbf{ Et pues que assi es al fuerte pertenesçe temer las cosas } que ha de temer & non est Fortis , sed insanus . \textbf{ Spectat igitur ad fortem timere timenda , } et audere audenda . \\\hline
1.2.13 & sea çerca algun bien \textbf{ e cerca alguna cosa fuerte e graue } por que los periglos de las batallas son mas fuertes & circa bonum , \textbf{ et difficile ( quia difficiliora , } et terribiliora sunt pericula bellica , \\\hline
1.2.13 & Et ahun por que en los periglos delas batallas \textbf{ mas fuerte cosa es de repremer los temores } que de restenar las osadias . & et etiam quia in periculis bellicis \textbf{ difficilius est reprimere timores , } quam moderare audacias : \\\hline
1.2.13 & Otrosi por que en auiendo osadia \textbf{ non es tan fuerte nin tan graue cosa acometer la batalla } e la pellea commo sofrir & rursus quia in audendo \textbf{ non tam difficile est aggredi pugnam , } sicut tolerare , \\\hline
1.2.13 & Mas ahun en la mar e enlas enfermedades . \textbf{ aquel es fuerte que non es temeroso . } Et por ende al fuerte parte nesçe non temer qual si quier periglo que la razon o el entendimiento iudga sinplemente & Sed adhuc et in mari , \textbf{ et in aegritudinibus intimidus est , } qui est fortis . \\\hline
1.2.13 & aquel es fuerte que non es temeroso . \textbf{ Et por ende al fuerte parte nesçe non temer qual si quier periglo que la razon o el entendimiento iudga sinplemente } que non son de temer . & Sed adhuc et in mari , \textbf{ et in aegritudinibus intimidus est , } qui est fortis . \\\hline
1.2.13 & assi commo todos dizen comunal mente esto podemos prouar por tres razones ¶ \textbf{ La primera por que acometer parte nesçe al mas fuerte . } Mas sufrir parte nesçe al mas fiaco & triplici via venari potest . \textbf{ Primo , quia aggrediendum , | est fortioris : } sustinere autem , \\\hline
1.2.14 & e quariendo ganar honrra \textbf{ acomete alguna cosa fuerte e espantable . } Onde dize el philosofo & et volens honorem adipisci , \textbf{ aggreditur aliquod terribile , } unde ait Philosophus , \\\hline
1.2.14 & assi conmo dize el philosofo alli do pone tal \textbf{ enxienplo que ector era fuerte en esta manera } que temie ser denostado de polimas . & ( ut idem Philosophus ait ) \textbf{ Hector fortis erat , } qui timens increpationes Polydamantis , \\\hline
1.2.14 & acometie cosas espatables \textbf{ e estaua fuerte enla fazienda } por que dezie si el & qui timens increpationes Polydamantis , \textbf{ aggrediebatur terribilia . } Dicebat enim quod si fugeret , \\\hline
1.2.14 & para le dezir muchos denuestos . \textbf{ Et por ende temiendo qual denostaria su contrario era fuerte . } Et ahun pone otro & qui erat ex parte aduersa , \textbf{ primum sibi increpationes imponeret . } Sic etiam \\\hline
1.2.14 & e dize que di omnedes \textbf{ enesta manera de fortaleza era fuerte . } Ca dize que si non lidiase reziamente su contrario ector & Diomedes , \textbf{ hoc modo fortis erat . } Dicebat enim quod \\\hline
1.2.15 & nin aquel que es osado en todas las cosas \textbf{ otrosi non es fuerte . } Mas aquel es fuerte que teme & non est fortis , \textbf{ nec qui omnia audet : } sed qui timet timenda , \\\hline
1.2.15 & otrosi non es fuerte . \textbf{ Mas aquel es fuerte que teme } lo que deue temer & nec qui omnia audet : \textbf{ sed qui timet timenda , } et audet audenda . \\\hline
1.2.17 & Ca guardar omne lo suyo propio \textbf{ non es cosa fuerte por si . } Ca cada hun omne es naturalmente inclinado a amar asi mismo & Nam custodire propria \textbf{ secundum se non est difficile : } quia unusquisque naturaliter inclinatur \\\hline
1.2.22 & Et despues desto es çerca los otros periglas \textbf{ en tal manera que avn el fuerte se ha conueniblemente en los otros periglos . } En essa misma manera la magranimidat & ex consequenti autem erat circa pericula alia , \textbf{ quod fortis etiam in aliis periculis decenter se habeat . } Sic magnanimitas principaliter est circa honores , \\\hline
1.2.24 & si esto faze por que se delecta en tales obras \textbf{ este es dicho fuerte . } Mas si esto faze por que tal sobras son dignas de grant honrra & quia delectatur in talibus actibus , \textbf{ fortis est . } Si vero hoc agit , \\\hline
1.2.29 & Et desto pongamos enxenplo . \textbf{ Ca si alguno en tanto fuese fuerte e estremado sobre los otros en fortaleza en tal manera que los otros fuesen çiertos } que pudiessen lidiar contra çiento & non in hoc appareret verax , sed derisor . \textbf{ Ut si aliquis adeo esset fortis et strenuus , | quod constaret aliis } quod contra centum bellare posset : \\\hline
1.2.31 & acebadamente te prado \textbf{ empero non sera fuerte mas temeroso } nin sera largo nin liberal & Erit ergo ille perfecte temperatus , \textbf{ non tamen erit fortis , sed timidus : } nec erit largus , sed auarus . \\\hline
1.3.3 & prinçipalmente ha de ser çerca los bienes diuinales e comunes . \textbf{ Otrosi sera fuerte por que ante pone el bien comunal bien propreo } e avn non dubdara de poner la persona a muerte siuiere & esse circa diuina , et communia . \textbf{ Erit fortis ; quia cum bonum cumune proponat bono priuato , } non dubitabit etiam personam exponere , \\\hline
2.1.4 & e de cada dia \textbf{ mas non es cosa fuerte de veer } que la casa sea establesçida de muchas perssonas . & propter opera diurnalia et quotidiana . \textbf{ Quod autem oporteat domum | ex pluribus constare personis , videre non est difficile . } Nam cum domus \\\hline
2.2.4 & que de los fijos alos padres \textbf{ e es mas fuerte e mas afincado¶ } La segunda razon para puar esto mismo se toma dela c̀tidunbre de los fueros & quam econuerso : \textbf{ quare fortior et vehementior . } Secunda via ad inuestigandum hoc idem , \\\hline
2.2.17 & tal qual demanda el su ofiçio \textbf{ la qual cosa non puede ser sin fuerte trabaio de su cuerpo . } ¶ Et pues que assi es commo todos aquellos & quale requirit officium militare . \textbf{ Quod sine forti exercitatione corporis | esse non potest . } Cum ergo omnes volentes viuere vita politica , \\\hline
2.3.15 & Ca paresçe que cosa digna es \textbf{ que las partes que son mas çeranas dela fuerte mas abonden } e eche agua que las otras . & Dignum est enim \textbf{ ut partes propinquiores fonti plus profundantur aqua : } quare si omnis amor \\\hline
3.1.19 & e toda ferida \textbf{ quanto mas fuerte es tanto } mas destruyel dela sustançia . & quia talia ad mortem ordinantur , \textbf{ eo quod omnis passio magis facta } abiiciat a substantia . \\\hline
3.2.3 & ca quanto la uirtud es mas ayuntada \textbf{ e vna en ssi tanto es mas fuerte en ssi que si fuesse esꝑzida } assi commo se declara & Nam quanto virtus est magis unita , \textbf{ fortior est seipsa dispersa , } ut declarari habet in libro de Causis , \\\hline
3.2.3 & que es en muchs prinçipantes \textbf{ fuesse ayuntado en vn prinçipante e vn sennor mas fuerte seria . } Et aquel prinçipe por ma . & quae est in pluribus principantibus , \textbf{ congregaretur in uno Principe , | efficacior esset ; } et ille principans \\\hline
3.3.1 & podemos dezir \textbf{ que assi commo el fuerte pertenesçe prinçipalmente de saber bien en obras de batallas . } Et de si a esse mismo pertenesçe & quod sicut \textbf{ ad fortem principaliter spectat | bene se habere in opere bellico , } ex consequenti vero spectat \\\hline
3.3.4 & a periglos de muerte en la batalla \textbf{ nunca ninguno es fuerte de coraçon } nin buen lidiador & Nam cum tota operatio bellica exposita sit periculis mortis , \textbf{ nunquam quis est fortis animo } et bonus bellator , \\\hline
3.3.4 & que sin miedo en los periglos de la muerte . \textbf{ Ca pertenesçe al fuerte e al buen lidiador } assi commo dize el philosofo & nisi aliquo modo sit impauidus circa pericula mortis . \textbf{ Spectat enim ad fortem | et ad bonum bellatorem , } ut innuit Philosophus 3 Ethic’ \\\hline
3.3.8 & la carcaua deue ser muy ancha de nueue pies e alta de siete . \textbf{ Mas si la fuerça de los enemigos paresciere mas fuerte conuiene de fazer las carcauas mas anchas et mas fondas } si han uagar para las fazer & fossa debet esse lata pedes nouem , alta septem . \textbf{ Sed si aduersariorum vis acrior imminet , | contingit fossam ampliorem } et altiorem facere ita , \\\hline
3.3.10 & aquella parte es la meior para lidiar . \textbf{ s sienpre la uirtud ayuntada e ordenada es mas fuerte que quando esta desparzida e desordenada . } Mas contesçe algunas vezen & est pars potior ad bellandum . \textbf{ Semper virtus unita fortior est | seipsa dispersa et confusa . } Contingit autem aliquando commisso bello ordines \\\hline
3.3.10 & Et pues que assi es con grant sabiduria es de escoger el alferez \textbf{ assi que sea fuerte de cuerpo } e firme de coraçon & Cum magna igitur diligentia est vexillifer eligendus , \textbf{ ut sit corpore fortis , } animo constans , fidelis principi , \\\hline
3.3.10 & en la batalla \textbf{ deue ser fuerte en el cuerpo . } grande en su estado e sabidor en lançar lanças e dardos & qui in pugna pedicibus praeponitur \textbf{ esse fortis viribus , } procer statura , \\\hline
3.3.10 & en las armas ligero de cuerpo \textbf{ e fuerte en los mienbros } aquel que deue ser ante puesto a los caualleros . & et procer corpore , \textbf{ et fortis viribus } qui est equitibus praeponendus : \\\hline
3.3.15 & Assi que la parte derecha en las animalias \textbf{ es mas fuerte en mouer } e mas apareiada a mouimiento . & quod pars dextra \textbf{ in animalibus fortior est in mouendo , } et aptior ad motum : \\\hline
3.3.15 & el qual esgrimido mueue el ayre mas reziamente \textbf{ e faz mas fuerte colpe . } Enpero maguer que podamos folgar tan bien sobre la parte derecha & quo vibrato vehementius mouet aerem , \textbf{ et fortius ferit . } Licet enim \\\hline
3.3.19 & que ponen y . \textbf{ ha muy fuerte et muy . } dura fruente para ferir & ideo appellatur aries , \textbf{ quia ratione ferri ibi appositi durissimam habet } frontem ad percutiendum . \\\hline
3.3.19 & carnero se tira atras . \textbf{ Et despues da muy fuerte colpe en los muros de la fortaleza çercada } assi que los ronpen et los quebrantan . & et ad modum arietis se subtrahit : \textbf{ et postea fortiter muros munitionis obsessae percutit et disrumpit . } Cum enim per huiusmodi trabem sic ferratam \\\hline
3.3.20 & que sean bien feridas e bien tapiadas \textbf{ e fazen la fortaleza mas fuerte . } Por la qual cosa mucho cunple fazer tales muros & si bene condensetur : \textbf{ propter quod non est inconueniens } construere huiusmodi muros \\\hline
3.3.22 & Et esta saeta enuiada \textbf{ por muy fuerte ballesta al . engeñio muchas vegadas le quema . } La quarta manera para destroyr los engeñios & per ballistam fortem emissa \textbf{ usque ad machinam , | multotiens succendit ipsam . } Quarto etiam modo resistitur machinis lapidariis , \\\hline
3.3.23 & Et lançandolos assi en las naues quebrantan se los cantaros \textbf{ e aquel fuego fuerte ençiende } e quema la naue . & Ex qua proiectione vas frangitur , \textbf{ et illud incendiarium comburitur } et succendit nauem . \\\hline
3.3.23 & por que de muchas partes se pueda quemar la naue . \textbf{ Et entonçe deuen acometer muy fuerte batalla contra los enemigos } por que se non puedan acorrer & et cum proiiciuntur talia , \textbf{ tunc est contra nautas | committendum durum bellum , } ne possint currere ad extinguendum ignem . \\\hline

\end{tabular}
