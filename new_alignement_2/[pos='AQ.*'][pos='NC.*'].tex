\begin{tabular}{|p{1cm}|p{6.5cm}|p{6.5cm}|}

\hline
1.1.1 & E segund esto deuemos saber \textbf{ que en toda la moral philosophia la manera de fablar } segund el philosofo es figural e gruessa & Sciendum ergo , \textbf{ quod in toto morali negotio modus procedendi } secundum Philosophum est figuralis \\\hline
1.1.1 & que de omne sabio es en tanto demandar çertidunbre de cada cosa \textbf{ en quanto la naturaleza dessa mismͣ cosa lo demanda } Ca semeja la naturaleza Dela sçiençia moral del todo ser contraria ala sçiençia matematica & secundum unumquodque genus , \textbf{ inquantum natura rei recipit . Videtur enim natura } rei moralis omnino esse opposita negocio mathematico . \\\hline
1.1.1 & Por la qual rrazon dize a philosofo enel primero libro delas ethicas \textbf{ que semeie ante e egual pecado es } quel mathematico tiente de amonestar & Propter quod 1 Ethicorum scribitur , \textbf{ quod per peccatum est , mathematicum persuadentem acceptare , } et rhetoricum demonstrationes expetere . \\\hline
1.1.1 & Ca asi commo dize el phon enel Segundo libro de la ethicas las obras morales e De costunbres rreçibiemos \textbf{ non por grande contenplaçion } e de saber & Nam ( ut scribitur 2 Ethic’ ) opus morale suscipimus \textbf{ non contemplationis gratia , } neque ut sciamus , \\\hline
1.1.1 & Onde dize el philosopho enel primero libro delas ethicas \textbf{ que la moral sçiençia deue ser amada e deseada } si por ella e por lo qne fabla enella & procedendum est persuasiue et figuraliter . \textbf{ Unde 1 Ethicorum scribitur , morale negocium amabile de talibus et ex talibus dicentes , } et de iis quae sunt \\\hline
1.1.1 & si por ella e por lo qne fabla enella \textbf{ que conteçe munchans vezes } e non sienpre ouieremos demostrar la v̉dad de ella gruessamente & Unde 1 Ethicorum scribitur , morale negocium amabile de talibus et ex talibus dicentes , \textbf{ et de iis quae sunt } ut frequentius , grossae et figuraliter veritatem ostendere . Tertia via sumitur ex parte \\\hline
1.1.1 & e desta sçiençia \textbf{ mas por que pocos son los que han agudo entendimjento } asi commo dize el philosopho en el terçero libro de la rethorica & totus ergo populus auditor quodammodo est huius artis , \textbf{ sed pauci sunt vigentes acumine intellectus , } propter quod dicitur 3 Rhetoricorum , \\\hline
1.1.1 & e mas alongado el entendimjento onde se sigue \textbf{ que el oydor del a moral philosophia deue ser simple e grueso } asi commo demuestra el philosofo enel primero libro dela recthorica¶ & propter quod dicitur 3 Rhetoricorum , \textbf{ quod quanto maior est populus , remotior est intellectus . Auditor ergo moralis negocii est simplex et grossus , } ut ostendit in 1 Rhetoricorum . \\\hline
1.1.2 & asi commo el conosçimiento del entendimjento nasçe del conosçimjento Delos sesos \textbf{ Por ende buena cosa es de Recontar la orden de las cosas } que se han de dezir & ut dicitur 1 Posteriorum , \textbf{ bene se habet narrare ordinem dicendorum , } ut de ipsis quaedam praecognitio habeatur . \\\hline
1.1.2 & que dicho \textbf{ es seguuila ordennatanl delas sçinas especulatinas } ha verdad de todas las obras & Quod ergo \textbf{ secundum naturalem ordinem dictum est de speculabilibus , } veritatem habet \\\hline
1.1.2 & qunata para gouerna mj̊ de çibdado den rregno Conuiene \textbf{ Segund orden natural ala rreal magestad primeramente } que el Ruy sepa gouernar asy mesmo ¶ & tanta prudentia in regimine familiae , \textbf{ quanta in gubernatione ciuitatis et regni : ordine naturali decet regiam maiestatem } primo \\\hline
1.1.2 & Lo terçero \textbf{ que sepa gouernả su rregno e sus çibdades ¶ pues que asy es en el primo libro } en el qual tractaremos del gouerna mjeto del omne . & scire regere regnum , \textbf{ et ciuitatem . In primo autem libro in quo agetur de regimine sui , } sunt quatuor declaranda . \\\hline
1.1.2 & en el qual tractaremos del gouerna mjeto del omne . \textbf{ En sy mesmo son quatro cosas de declarar e de demostrͣ } Ca primamente demostrͣemos & et ciuitatem . In primo autem libro in quo agetur de regimine sui , \textbf{ sunt quatuor declaranda . } Nam Primo ostendetur in quo regia maiestas debeat suum finem , \\\hline
1.1.2 & asy mesmo non pue da ser \textbf{ sy non que se de el omne abunos fechos e abunas obras rregladas } por orden de Razon & esse non possit , \textbf{ nisi quis se det bonis actibus , } et bonis operibus regulatis ordine rationis : \\\hline
1.1.2 & aquello que dize el philosofo en el segundo libro delas ethicas \textbf{ que prouechosa cosa es en la scian moral } e en la sçiençia de costunbres escod̀nar & oportet ipsum notitiam tradere de omnibus his quae diuersificant mores \textbf{ et actiones . Inde est ergo , quod vult Philosophus 2 Ethic’ proficuum esse morali negocio , } scrutari ea quae sunt circa operationes , \\\hline
1.1.2 & commo las deue omne fazer \textbf{ Mas las nr̃as obras } quanto alo prisente parte nesçe de quatro gujsas & scrutari ea quae sunt circa operationes , \textbf{ quomodo faciendum sit eas . Operationes autem nostrae ex quatuor } ( quantum ad praesens spectat ) videntur oriri , et diuersificari ; \\\hline
1.1.2 & Mas las nr̃as obras \textbf{ quanto alo prisente parte nesçe de quatro gujsas } e de quatro maneras las veemos nasçer & scrutari ea quae sunt circa operationes , \textbf{ quomodo faciendum sit eas . Operationes autem nostrae ex quatuor } ( quantum ad praesens spectat ) videntur oriri , et diuersificari ; \\\hline
1.1.2 & quanto alo prisente parte nesçe de quatro gujsas \textbf{ e de quatro maneras las veemos nasçer } e departir ¶ primeramente de parte delas fines & quomodo faciendum sit eas . Operationes autem nostrae ex quatuor \textbf{ ( quantum ad praesens spectat ) videntur oriri , et diuersificari ; } videlicet , \\\hline
1.1.2 & Lo quarto de parte delas costunbres ¶ \textbf{ Ca commo la fin sea comjenço delas nr̃as obras } segunt que cada vno ordena & et moribus . \textbf{ Nam cum finis sit operationum nostrarum principium , } secundum quod quis sibi alium et alium finem praestituit , \\\hline
1.1.2 & pues que asy es para saber \textbf{ que deuemos obrar muy aprouechable cosa es de saber } que fin deuemos entender & et alia operatur . \textbf{ Ad sciendum ergo quae operari debemus , maxime proficuum esse videtur , } quem finem nobis praestituere debeamus . Rursus \\\hline
1.1.2 & asy commo se mostrͣa en su logar \textbf{ son natraalmente mal creyentes e auarientos } Mas los mançebos son naturalmente liƀales & quam habentes mores iuuenum . Senes enim \textbf{ ( ut suo loco ostendetur ) sunt naturaliter increduli , } et auari : iuuenes vero sunt naturaliter liberales , \\\hline
1.1.2 & pues que asy es paresçe \textbf{ que estas quatro cosas dichas han alguna conparaçion entre sy } por que departidas costunbres han de nasçer departidos pasiones & et auari : iuuenes vero sunt naturaliter liberales , \textbf{ et creditiui . Videntur autem haec quatuor habere aliquam analogiam adinuecem . } Nam ex aliis , et aliis motibus , \\\hline
1.1.2 & pues que asy es en el primero libro tractaremos \textbf{ destas quatro cosas } que dichͣ s son & ut finem sibi praestituant conformem suo habitui . \textbf{ In primo ergo libro de omnibus his quatuor tractabimus , } videlicet , de fine , \\\hline
1.1.2 & es \textbf{ mas prinçipal comjenço que ninguno de los otros } or que asy commo dicho es esta obra tomamos & quia finis respectu agendorum , est principalius principium , \textbf{ quam aliquod aliorum . } Quoniam ( ut dictum est ) \\\hline
1.1.3 & que nos prometiemos de tractar ligniamente ¶ \textbf{ et en el segundo capitulo fiziemos essa mjsma magestad doçible } e engannosa contando la orden de las cosas que aqui auemos de dezir ¶finca que en este terçero capitulo fagamos la Real magestad atenta & nos esse faciliter tractaturos : \textbf{ et in secundo reddidimus eam docilem , | narrando ordinem dicendorum : } restat \\\hline
1.1.3 & et en el segundo capitulo fiziemos essa mjsma magestad doçible \textbf{ e engannosa contando la orden de las cosas que aqui auemos de dezir ¶finca que en este terçero capitulo fagamos la Real magestad atenta } e acuçiosa declarandol & narrando ordinem dicendorum : \textbf{ restat | ut in hoc capitulo tertio reddamus eam attentam , } declarando quanta sit utilitas in dicendis . \\\hline
1.1.3 & mas por que comunalmente aborresçen los omes los sermones escd̀innadores \textbf{ e sotiles muchans vegadas } los & Nam \textbf{ quia communiter homines odiunt sermonem perscrutatum , } ut plurimum auditores sunt beniuoli proferentibus sermones faciles , \\\hline
1.1.3 & e del desseo del omne \textbf{ enpero en la philosopra moraldo tractamos delas buenas costunbres } En la qual la manera dela cosa demanda doctrina figanl e guaesa non es corrupçion del apetito ñj del desseo del omne & et grossam , \textbf{ non est corruptio appetitus , } sed magis est ordo rectus \\\hline
1.1.3 & Mas es orden derecha e deujda e atal arte ¶ \textbf{ pues que asy es en esta mjsma arte faremos el oydor begniuolo e uolunteroso } por ligera manera & sed magis est ordo rectus \textbf{ et debitus . } In huiusmodi ergo arte ex facilitate tradendi redditur auditor beniuolus : \\\hline
1.1.3 & pues que asy es en esta mjsma arte faremos el oydor begniuolo e uolunteroso \textbf{ por ligera manera } qual guardaremos & sed magis est ordo rectus \textbf{ et debitus . } In huiusmodi ergo arte ex facilitate tradendi redditur auditor beniuolus : \\\hline
1.1.3 & e le daremos \textbf{ mas por la orden de lans cosas } que son de dezer le faremos doçible e engeñoso & In huiusmodi ergo arte ex facilitate tradendi redditur auditor beniuolus : \textbf{ sed ex ordine dicendorum redditur docilis ; } nam quis maxime efficitur docilis idest habilis ad capiendum doctrinam , \\\hline
1.1.3 & sy las pusiern ordenadamente \textbf{ e por buena orden } Mas por el prouecho delas cosas que son de dezer es fecho el oydor atento & si ei dicenda quadam serie , \textbf{ et ordine proponantur , } ex utilitate autem dicendorum redditur auditor attentus , \\\hline
1.1.3 & commo deuen \textbf{ segnir se les an quatro cosas } en quanto ꝑtenesçe aeste presente arte & consequitur maiestas regia \textbf{ ( quantum ad praesens spectat ( quatuor , } quae quilibet maxime amare , \\\hline
1.1.3 & segnir se les an quatro cosas \textbf{ en quanto ꝑtenesçe aeste presente arte } Las quales cosas mucho deue cada vno amar & consequitur maiestas regia \textbf{ ( quantum ad praesens spectat ( quatuor , } quae quilibet maxime amare , \\\hline
1.1.3 & Lo primo es \textbf{ que ganaran muy grandes bienes ¶ } Lo segundo que ganaran asy mesmos¶ & et desiderare debet . \textbf{ Primo enim lucrabitur maxima bona . } Secundo lucrabitur seipsum . Tertio alios . \\\hline
1.1.3 & ca algunos dellos son muy pequanos¶ \textbf{ Et alguons bienes son medianeros ¶ } Et alguons son muy gerades ¶ & ø \\\hline
1.1.3 & e los poderios del alma \textbf{ Ca en estos bienes puede auer los malos parte } tan bien commo los buenos ¶ & huiusmodi sunt industria mentis , ingenium naturale , potentiae animae : \textbf{ his enim bonis etiam ipsi mali participant . } Bona vero maxima , \\\hline
1.1.3 & Et algunos bienes son honestos \textbf{ Mas los bienes honestos son bienes de grant aunataja } Ca enestos bienes honestos & quam etiam in Ethicis , \textbf{ quia quaedam sunt delectabilia , quaedam utilia , quaedam honesta . Bona autem honesta , sunt bona per excellentiam : } nam in his bonis \\\hline
1.1.3 & Ca los bienes sy son honestos \textbf{ e an en sy grant deletaçion } e ençierran en sy bondat de tondos los bienes prouechosos ¶ & si honesta sint , \textbf{ habent in se magnam delectationem , } et includunt bonitatem utilium bonorum . \\\hline
1.1.3 & Ca estas cosas guardadas podran auer los bienes honestos \textbf{ e los muy grandes bienes ¶ } Lo segundo es muy grant prouecho & quia , eis obseruatis , \textbf{ habebuntur bona maxima , } et honesta . Secundo est maxima utilitas in dicendis , \\\hline
1.1.3 & e los muy grandes bienes ¶ \textbf{ Lo segundo es muy grant prouecho } ca non solamente delas cosas & habebuntur bona maxima , \textbf{ et honesta . Secundo est maxima utilitas in dicendis , } quia ex eis non solum quis lucrabitur maxima bona , \\\hline
1.1.3 & mas avn ganara asy mesmo \textbf{ Ca la moral ph̃ia enla mayor partida solamente es } por que semos buenos & sed etiam lucrabitur seipsum . Est enim morale negocium \textbf{ ( ut plures tactum est ) } ut boni fiamus : \\\hline
1.1.3 & por que semos buenos \textbf{ ca el buen varona asy mesmo } Et el malo ha mengua de sy mesmo & ut boni fiamus : \textbf{ bonus autem vir seipsum habet , } malus autem seipso caret . Sic enim imaginari debemus , \\\hline
1.1.3 & asy es vno \textbf{ e en esse mismo omne los pode rios del alma } que son rrazonables & et ad unum regem , \textbf{ sic in uno et eodem homine potentiae } quae sunt rationales per participationem , \\\hline
1.1.3 & por partiçipaçion non obedesçieren ala rrazon o al entendemjento \textbf{ que es rrazonable por sy mesmo Ca segunt el philosofo en el nono delans ethicas El omne } mas es entendemiento e rrazon & et si rationale per participationem non obediat rationali per essentiam : \textbf{ homo enim | ( secundum Philosophum 9 Ethi’ ) } maxime est intellectus et ratio : \\\hline
1.1.3 & Et lo al por obro traydo e inclimado por posion ¶ \textbf{ Et por ende sy el mal omne non ha asy mesmo } mas el bueno ha asy mesmo & et aliud agit passione tractus \textbf{ et inclinatus quare si malus homo non habet seipsum , } sed bonus , \\\hline
1.1.3 & mas acabadamente ha a dios ¶ \textbf{ mas los malos omes non son ayuntados en ninguna manera } njn consigo njn con dios & et perfectius habet ipsum Deus . \textbf{ Mali autem homines minime sunt uniti : } totum enim regnum eorum est dispersum . Regnum enim cuiuslibet hominis , \\\hline
1.1.3 & mas es en sy mesmo departido \textbf{ e ha es sy mesmo discordia e departimjento } por la qual cosa es dessemejado e desacordado del primero prinçipio & tunc homo non est unitus , \textbf{ sed in seipso dissensionem habet , } propter quod difformis est à primo Principio . Sed pollentes virtutibus , \\\hline
1.1.4 & que se sigue \textbf{ feziemos la Real magestad begniuola et uolunterosa } para oyr e a prinder & ø \\\hline
1.1.4 & para oyr e a prinder \textbf{ por la liger eza dela manera de tractar e de fablar } Et ahun diemos la & quia respectu sequentis operis \textbf{ ex facilitate modi tradendi reddidimus regiam maiestatem beniuolam , } ex ordine dicendorum reddidimus eam docilem , \\\hline
1.1.4 & Segunt estos tres pensamjentos tomaron los philosofos las tres uidas sobredichas \textbf{ Ca quisieron alguons philosofos } que al omne conuiene la uida delectosa & siue cum substantiis separatis . \textbf{ Secundum has tres considerationes sumptae sunt a Philosophis praedictae tres vitae voluerunt enim } quod homini ut communicat cum brutis , \\\hline
1.1.4 & e con los angeles dizen qual conujene la ujda contenplatina e çelestial ¶ \textbf{ pues que asy es cada vno delons omes o biue } asy commo bestia o biue & competit ei vita contemplatiua . \textbf{ Quilibet ergo vel viuit } ut bestia , \\\hline
1.1.4 & diziendo que non es en elła . \textbf{ La qual cosa njegan los theologos en essa mjsma guisa } ca pusieron feliçidat & nisi duplicem felicitatem : \textbf{ nam in vita voluptuosa negauerunt esse felicitatem , quod et Theologi negant : } posuerunt enim felicitatem politicam , \\\hline
1.1.4 & non solamente asy commo omne \textbf{ mas asy commo aquel en que es algua cosa diujnal } e algua cosa mejor que omne & et quando est felix non solum ut homo , \textbf{ sed ut est in eo aliquid diuinum , } et aliquid melius homine . Perfectum igitur in agibilibus , \\\hline
1.1.4 & mas asy commo aquel en que es algua cosa diujnal \textbf{ e algua cosa mejor que omne } pues que asy es llaman al omne acabado en las obras & sed ut est in eo aliquid diuinum , \textbf{ et aliquid melius homine . Perfectum igitur in agibilibus , } vocabant felicem politice : \\\hline
1.1.4 & Et estonçe es cosa diujnal \textbf{ e semeiante a dios e medio dios . } Mas sy es omne & vel est homine melior , \textbf{ et tunc est quid diuinum et semideus . } Si autem est homo , \\\hline
1.1.4 & por que el omne es natural . \textbf{ naturalmente aina la conpanable çibdadano } e ordenado & quia homo \textbf{ ( ut ibi probatur ) est naturaliter animal sociale , ciuile , } et politicum , sequitur quod regatur secundum prudentiam , et viuat vita politica . \\\hline
1.1.4 & e ordenado \textbf{ asy commo prueua el philosofo en esse mjsmo libro siguese que el omne deue ser gouernado } segunt sabiduria e rrazon derecha & ø \\\hline
1.1.4 & que ha de fazer \textbf{ Et el acabado ente las sçiençias especulatians quanta es entre el que biue vida humanal e vida politica } que es vida ordenada . & Tanta est ergo differentia inter prudentem in agibilibus , \textbf{ et perfectum in speculabilibus , } quanta est \\\hline
1.1.4 & por departimjento delos negoçios \textbf{ que acaesçen traban se en muchͣ̃s cosas } e en muchas maneras & inter viuentem vita humana et politica , \textbf{ et viuentem uita contemplatiua } et angelica . Dediti enim operabilibus , propter diuersitatem negociorum emergentium , turbantur erga plurima , et ut plurimum isti sentiunt passiones carnis : \\\hline
1.1.4 & e en muchas maneras \textbf{ e avn en la mayor partida sienten muchas pasiones } e muchos moujmjentos de la carne & et viuentem uita contemplatiua \textbf{ et angelica . Dediti enim operabilibus , propter diuersitatem negociorum emergentium , turbantur erga plurima , et ut plurimum isti sentiunt passiones carnis : } dediti vero speculabilibus , \\\hline
1.1.4 & digamos \textbf{ que conujene ala rreal magestad } e a todo rrey saber & et vitam voluptuosam fugere , \textbf{ ne sit homine peior : } nam tales \\\hline
1.1.4 & tantomas conviene dela auer los rreys e los prinçipes \textbf{ por quanto han de dar mayor cuenta ant̃la siella del primer alcałł } que es dios Et quanto de mayores conpannas han de dar Razon e cuenta & tanto magis decet habere reges et principes , \textbf{ quanto apud tribunal summi Iudicis reddituri sunt de pluribus rationem . } Quod maxime expedit regiae maiestati \\\hline
1.1.4 & por quanto han de dar mayor cuenta ant̃la siella del primer alcałł \textbf{ que es dios Et quanto de mayores conpannas han de dar Razon e cuenta } as conuiene de notar e de saber acuçiosamente & quanto apud tribunal summi Iudicis reddituri sunt de pluribus rationem . \textbf{ Quod maxime expedit regiae maiestati } Est autem diligenter notandum , \\\hline
1.1.5 & que asy commo la materia \textbf{ por sus conueientes e ordenadas t̃ns muta connes viene a rresçebir su forma } e su perfecçion & quod sicut materia per debitas transmutationes \textbf{ consequitur } suam perfectionem \\\hline
1.1.5 & que lo fiza deletablemente \textbf{ e de buena uoluntad . } Ca sy non feziese bien & quod agat delectabiliter . \textbf{ Si enim non ageret bene } sed male , \\\hline
1.1.5 & Ca sy non feziese bien \textbf{ mas mal non podria alcançar buena fin } mas avria el contrario & Si enim non ageret bene \textbf{ sed male , | non consequeretur finem , } sed contrarium finis : \\\hline
1.1.5 & mas avria el contrario \textbf{ e mala fin ¶ } Ca segunt que dize el philosofo en el segundo libro delos fisicos & sed contrarium finis : \textbf{ nam finis } ut dicitur 2 Physic’ \\\hline
1.1.5 & la fin non dize solamente cosa postrimera \textbf{ mas dize muy buena cosa } por la qual cosa dize el philosofo en el primero libro delas ethicas & quid ultimum , \textbf{ sed etiam quid optimum . Propter 2 Meta’ innuit Philosophus , } quod finis et bonum idem : \\\hline
1.1.5 & los que mal fazen dignos son \textbf{ que ayan mala postremeria . } Ca el mal es contrario del bien & Quare si male agentes digni sunt ut mala consequantur , \textbf{ quia malum contrariatur bono , } per consequens contrariatur fini , \\\hline
1.1.5 & Por la qual rrazon los malos dignos son \textbf{ non que alcançen buena fin } mas el contrario e mala fin . & digni sunt , \textbf{ non ut consequantur finem , } sed contrarium finis . Immo non solum male agentes non consequuntur finem , \\\hline
1.1.5 & non que alcançen buena fin \textbf{ mas el contrario e mala fin . } Mas non solamente los que mal fazen non alcançan buena fin & non ut consequantur finem , \textbf{ sed contrarium finis . Immo non solum male agentes non consequuntur finem , } sed potentes bene agere , \\\hline
1.1.5 & mas el contrario e mala fin . \textbf{ Mas non solamente los que mal fazen non alcançan buena fin } mas avn los que pueden bien fazer & non ut consequantur finem , \textbf{ sed contrarium finis . Immo non solum male agentes non consequuntur finem , } sed potentes bene agere , \\\hline
1.1.5 & non les es deuido corona \textbf{ njn les es deuido buena fin } njn buean andança ¶ & non debetur eis corona , \textbf{ nec debetur eis finis , } vel felicitas . \\\hline
1.1.5 & pues que asi es conviene bien fazer de fecho \textbf{ por que por las nr̃as obras merescamos de auer buena fino buena ventura } segunt deuemos los omes fas̉ & actu bene agere , \textbf{ ut per opera nostra mereamur consequi finem , | vel felicitatem . } Secundo requiritur \\\hline
1.1.5 & njn por esto non le es deujda buean fin \textbf{ njn buena ventura . } Ca las cosas que se non fazen & ex hoc non est laudandus , \textbf{ nec debetur ei ex hoc finis vel felicitas : } nam quae ex electione non fiunt , \\\hline
1.1.5 & pues que asy es fablando propiamente de tales obras qua non son de uoluntad fechas \textbf{ segunt que tales son non se nos sigue buena fin } njn buena ventura de los . ¶ & ( \textbf{ secundum quod huiusmodi sunt ) | non consequimur beatitudinem , } et felicitatem . \\\hline
1.1.5 & segunt que tales son non se nos sigue buena fin \textbf{ njn buena ventura de los . ¶ } Mas commo nos alcançamos la buena ventura & non consequimur beatitudinem , \textbf{ et felicitatem . } Immo cum ex operibus virtuosis felicitatem consequamur \\\hline
1.1.5 & njn buena ventura de los . ¶ \textbf{ Mas commo nos alcançamos la buena ventura } por obras uirtuosas & et felicitatem . \textbf{ Immo cum ex operibus virtuosis felicitatem consequamur } ( quia virtus est habitus electiuus in medietate consistens , \\\hline
1.1.5 & que muestra a omne escoger . \textbf{ Et esta sabiduria esta en medio delas buenas obras } Ca asy lo praeua el philosofo en el segundo libro delas ethicas & Immo cum ex operibus virtuosis felicitatem consequamur \textbf{ ( quia virtus est habitus electiuus in medietate consistens , } ut dicitur 2 Ethic . ) oportet operationes , \\\hline
1.1.5 & por las quales nos alcançamos la fin \textbf{ que salgan denr̃a elecçio e denr̃a uoluntad ¶ } Lo terçero conuiene & per quas finem consequimur , \textbf{ ex electione procedere . } Tertio agere oportet delectabiliter : \\\hline
1.1.5 & Onde dize el philosofo en el segundo libro delas ethicas \textbf{ que non cunple solamente fazer buenas obras } mas fazerlas bien njn cunple de obrar obras iustas & Unde Philosophus 2 Ethic’ vult , \textbf{ quod non sufficit agere bona , sed bene : nec sufficit operari iusta , } sed iuste . Contigit enim aliquos prauos facere aliqua de genere bonorum , \\\hline
1.1.5 & por que contesçe \textbf{ que algunos malos fazen algunas buenas obras } Enpero por que las non . fazen bien & quod non sufficit agere bona , sed bene : nec sufficit operari iusta , \textbf{ sed iuste . Contigit enim aliquos prauos facere aliqua de genere bonorum , } tamen quia non faciunt ea bene et delectabiliter , \\\hline
1.1.5 & njn delectosamente non les conuiene \textbf{ que por aquellas obras alcançen buena fin } nin bue an uentura ¶ & tamen quia non faciunt ea bene et delectabiliter , \textbf{ non oportet per huiusmodi opera eos consequi finem vel felicitatem . } Cum ergo ista tria contingunt , \\\hline
1.1.5 & e delectosamente estonçe contesçe \textbf{ que nos alcançemos conplidamente buena fin . } Mas estas cosas contesçen & et delectabiliter , \textbf{ maxime contingit nos sic finem consequi haec autem maxime contingunt , } cognito fine . \\\hline
1.1.5 & pues que asy es conuiene en toda guisa \textbf{ que nos que conoscamos primero alanr̃a fin } por que la podamos alcançar & cognito fine . \textbf{ Expedit ergo } ( ut finem consequamur ) finem praecognoscere . Sic enim imaginari debemus , \\\hline
1.1.5 & asy en los fines los segundos fines non mueuen \textbf{ sy non en uirtud dela postrimera fin } por la qual cosa & quod sicut est in causis efficientibus , \textbf{ quod agentia secunda mouent in virtute agentis primi : sic fines secundarii mouent in virtute finis vltimi : } propter quod sicut si non esset agens primum , \\\hline
1.1.5 & e fin postrima \textbf{ que mouiese lanr̃a uoluntad } ninguon otro bien non la podria mouer & ut bonum ultimatum quod voluntatem moueret , \textbf{ nullum aliud bonum voluntatem mouere posset ; } quicquid enim vult voluntas , \\\hline
1.1.5 & en orden a alguna cosa \textbf{ que quiere finalmente e postrimera mente ¶ } pues que asy es non conosçiendo alguna cosa & vult in ordine ad aliquod quod vult finaliter \textbf{ et ultimate . } Non apprehenso ergo aliquo sub ratione finis , \\\hline
1.1.5 & que todas las nuestras obras toman nasçençia dela fin \textbf{ Ca non podria ser ninguna buena obra } sy fuese ordenada a mala fin & quia ex fine opera nostra speciem summunt : \textbf{ non enim esse posset quod esset bonum opus , } si ordinaretur in malum finem : \\\hline
1.1.5 & Ca non podria ser ninguna buena obra \textbf{ sy fuese ordenada a mala fin } donde se sigue & non enim esse posset quod esset bonum opus , \textbf{ si ordinaretur in malum finem : } praecognitio ergo finis requiritur , \\\hline
1.1.5 & ante la fin dize \textbf{ que para lanr̃auida grant acresçentamiento faze connosçer ante la fin } Ca por esta Razon alcançaremos ante la fin & quod cognitio finis \textbf{ ad vitam nostram magnum habet incrementum : } consequemur enim ex hoc magis ipsum finem ; \\\hline
1.1.5 & Ca pensada la bien andança \textbf{ e la buena uentraa } que omne ha por las buean s obras las obras guaues e fuertes de fazer se fazen muy delectables e plazenteras ¶ & considerata beatitudine , \textbf{ et felicitate , } quam ex ipsis consequimur . Cuilibet ergo homini , \\\hline
1.1.5 & que a otro ninguno . \textbf{ por que pueda fazer buenas obras e comunes } que son en alguna manera obras diuina les . & cognoscere suam felicitatem , \textbf{ ut opera communia , } quae sunt quodammodo diuina , \\\hline
1.1.5 & que los faga bien e delectosamente \textbf{ e de buena uoluntad } ¶ & delectabiliter , \textbf{ et ex electione . } Secundo patet hoc idem \\\hline
1.1.6 & nin segunt Razon assaz paresçe \textbf{ que en las tales delectaçiones non auemos nos de poner lanr̃a feliçidat nin lanr̃a bien andança } Mas que estas plazenterias & et secundum rationem , \textbf{ constat in talibus non esse felicitatem ponendam . Quod autem huiusmodi voluptates , } non sint bonum perfectum , et sufficiens , de leui patet . \\\hline
1.1.6 & que maguera que alguas delecta connes sean conuenibles e honestas \textbf{ por que las obras uirtuosas fazen al omnen bueno e uirtuoso e de grant delectaçion . } Enpero njnguna delectaçion non es feliçadat & quod licet sint delectationes aliquae licitae , et honestae , \textbf{ eo quod ipsa opera virtutum Homini bono , et virtuoso magnam delectationem faciant : } nulla tamen delectatio est essentialiter ipsa felicitas , licet possit esse aliquid felicitatem consequens , \\\hline
1.1.6 & pues que asi es \textbf{ quanto cada vn omne esta en mas alto grado } tanto por la mala uida es mas abaxado . & et desiderantes sic viuere sunt vitam pecudum eligentes : \textbf{ quanto ergo quis in altiori gradu existit , } tanto per hanc vitam magis deprimit . \\\hline
1.1.6 & quanto cada vn omne esta en mas alto grado \textbf{ tanto por la mala uida es mas abaxado . } Ca dicho n auemos ya desuso & quanto ergo quis in altiori gradu existit , \textbf{ tanto per hanc vitam magis deprimit . } Dictum est enim quod decet Principem esse supra Hominem , \\\hline
1.1.6 & asi que cada vno \textbf{ quanto es mayor prinçipe tanto mas deue sobrepuiar los otros } en dignidat deuida e en grandeza de bondat ¶ & ut quanto quis maior Princeps existit , \textbf{ tanto alios magis excellere debet in dignitate vitae , } et magnitudine bonitatis . \\\hline
1.1.6 & en dignidat deuida e en grandeza de bondat ¶ \textbf{ pues que asi es el que esta en tan alto grado } non deue escoger uida de bestia . & et magnitudine bonitatis . \textbf{ In tanto ergo gradu existens , } indignum est , \\\hline
1.1.6 & que non pueden usar de razon nin de entendimiento¶ \textbf{ pues que asi es commo seg̃t ese mismo philosofo } non deua ninguno de ligeros menospreciado & et ebrii uti ratione non possunt . \textbf{ Cum ergo | secundum eundem Philosophum non facile sit contemptibilis } qui sobrius , \\\hline
1.1.6 & e enbriago es de menospreçiar \textbf{ por ende ¶Conuienea la real magestad de searedrar de tales delectaçonnes desmesuradas e carnales } por que non sea menospreçiado de su pueblo¶ & nec qui uigil , \textbf{ sed qui dormiens , decet Regiam maiestatem tales delectationes immoderatas fugere , } ne contemptibilis uideatur . Tertio decet Principem talia detestari , ne principari efficiatur indignus , \\\hline
1.1.6 & Ca el que sigue las delectaçonnes carnales \textbf{ puesto que sea uieio entp̃o e en hedat } por que es moço en costunbres & Sequens enim delectationes sensibiles , \textbf{ dato quod sit Senex tempore , } quia est Puer moribus , \\\hline
1.1.7 & Ca segunt que el dize algunas riquezas son natraales \textbf{ e alguons son artifiçiales ¶Riquezas natraales son dichas aquellas } que uienen n atraalmente de cosas natraales & nam secundum ipsum quaedam sunt diuitiae naturales , \textbf{ quaedam sunt artificiales . Diuitiae naturales , | quaedam sunt artificiales . Diuitiae naturales dicuntur esse , } quae naturaliter ex rebus naturalibus producuntur , \\\hline
1.1.7 & asy commo son el oro e la plata \textbf{ e generalmente tonda moneda } que luego por sy non cunplen & cuiusmodi sunt aurum , et argentum , \textbf{ et uniuersaliter omne numisma , } quae immediate indigentias corporales non supplent , \\\hline
1.1.7 & por que non auemos de estableçer \textbf{ nin de poner lanr̃a feliçidat ni lanr̃abine andança en las riquezas } artifiçiales & eo ipso quod ordinantur ad diuitias naturales , \textbf{ rationem felicitatis habere non possunt . | Secundo in talibus non est ponenda felicitas , } quia non habent \\\hline
1.1.7 & si non por el ordenamiento de los omes ¶ \textbf{ Onde el philosofo enl primero libro delans politicas dize } que trismudados los usadores & nisi ex ordinatione Hominum . \textbf{ Unde 1 Politicorum dicitur , transmutatis utentibus , } idest transmutata dispositione utentium , siue transmutato ordine , et dispositione hominum , huiusmodi diuitiae nullam habent dignitatem , \\\hline
1.1.7 & Asi commo cuentay de un omne a que dezian meda el qual ero muy codiçioso de auerors \textbf{ E asi commo dla fabliella gano de dios } que quanto el tanxiesse que todo se le tornasse oro & qui cum nimis esset auidus auri \textbf{ ( ut fabulose dicitur ) | impetrauit a Deo , } ut quicquid tangeret , \\\hline
1.1.7 & Onde se sigue \textbf{ que este auia . grand Raqueza } e grand cunplimiento de oro & quin conuertetur in aurum . \textbf{ Erat ergo ei magna copia auri , } cum tamen fame periret . Quod esse non posset , \\\hline
1.1.7 & que este auia . grand Raqueza \textbf{ e grand cunplimiento de oro } Enpero muria de fanbre & quin conuertetur in aurum . \textbf{ Erat ergo ei magna copia auri , } cum tamen fame periret . Quod esse non posset , \\\hline
1.1.7 & Lo segundo que por que estas Riquezas son riquezas \textbf{ por ordenamiento e estableçemiento delons omes } e non en otra gnisa¶ & ipsi indigentiae corporali per se non sufficiunt , \textbf{ in eis non est ponenda felicitas . } Quod autem in naturalibus diuitiis , \\\hline
1.1.7 & e su bien andança en las riquezas corporales . \textbf{ Mas mayor mente es de denostar la Real magestad } e el Rey o el prinçipe & quae sunt bona animae . Cuilibet ergo Homini detestabile est ponere suam felicitatem in diuitiis , \textbf{ sed maxime detestabile est regiae maiestati . } Nam si Rex aut Princeps ponat suam felicitatem in diuitiis , \\\hline
1.1.7 & Ca si el rey o el prinçipe pone su feliçadato su bien andança en las riquezas corporales . \textbf{ tres males muy grandes se le siguen dende ¶ El pmero mal es que pierde muy grandes bienes ¶ } El segundo es que se faze & Nam si Rex aut Princeps ponat suam felicitatem in diuitiis , \textbf{ tria maxima mala inde consequuntur . Primo , } quia amittit maxima bona . Secundo , \\\hline
1.1.7 & nunca pue de ser magnifico nin granado \textbf{ el qual magnifico ha de fazer grandes espensas } para ser granado & nunquam potest esse magnificus , \textbf{ cuius est facere magnos sumptus : } nec etiam potest esse Magnanimus , \\\hline
1.1.7 & nin ahun puede ser mager fico \textbf{ nin de grant coraçon . } Ca temiendo deꝑder los des & cuius est facere magnos sumptus : \textbf{ nec etiam potest esse Magnanimus , } quia metuens pecuniam perdere , \\\hline
1.1.7 & Ca temiendo deꝑder los des \textbf{ e las riquezas nunca acometra grandes cosas } Et la razon es esta & quia metuens pecuniam perdere , \textbf{ nihil magnum attentabit . Immo } ( cum ille sit Magnanimus , cui nihil corporale est magnum , \\\hline
1.1.7 & Et la razon es esta \textbf{ que aquel es magnamimo e de grant coraçon } al qual ninguna cosa corporal non le es grande & nihil magnum attentabit . Immo \textbf{ ( cum ille sit Magnanimus , cui nihil corporale est magnum , } ut vult Philosophus 4 Ethicorum cap’ \\\hline
1.1.7 & que pone su bien andança en las riquezas . \textbf{ las riquezas son grant cosa e grant bien . } Et por ende non puede ser ninguon tal magnanimo & et in opinione ponentis suam felicitatem in diuitiis , \textbf{ diuitiae sunt quid magnum , } impossibile est talem esse Magnanimum . \\\hline
1.1.7 & Et por ende non puede ser ninguon tal magnanimo \textbf{ nin de alto coraçon . } pues que asi es ssy la magnifiçençia & diuitiae sunt quid magnum , \textbf{ impossibile est talem esse Magnanimum . } Si ergo Magnificentia , \\\hline
1.1.7 & e la grandeza de coraçon son muy grandesbienes \textbf{ los quales bienes deue auer la Real magestad } mucho conuiene al Rei & Si ergo Magnificentia , \textbf{ et Magnanimitas sunt maxima bona , } et maxime decet regiam maiestatem \\\hline
1.1.7 & mucho conuiene al Rei \textbf{ e ala Real magestad de ser conpuesta } e ennobleçida de tales uirtudes & et Magnanimitas sunt maxima bona , \textbf{ et maxime decet regiam maiestatem } esse ornatam talibus virtutibus , \\\hline
1.1.7 & Ca por esto se fare tirano \textbf{ ca ay grant diferençia entre el Rey e tirano } assy commo demostrͣemos enl terçero libro & quia hoc facto Tyrannus efficitur . Est enim differentia \textbf{ inter Regem et Tyrannum , } ut patebit in 3 lib’ cum determinabitur de regimine Regni . \\\hline
1.1.7 & e al bien comun¶ \textbf{ Mas si alguas vezes para mientes } por el su bien propio & et bonum commune : \textbf{ si autem intendit bonum proprium , } hoc est ex consequenti . \\\hline
1.1.7 & Por la qual cosa si es muy contra razon \textbf{ que el Rey dexe muy grandes bienes . Et si es contra razon otrosi } que sea robador del pueblo & Quare si detestabile est , \textbf{ Regem admittere maxima bona , } esse Tyrannum , et depraedatorem detestabile quoque est suam felicitatem in diuitiis ponere . \\\hline
1.1.8 & e la bien andança enlas honrras . \textbf{ por que por la mayor ꝑͣte todos les çibdadanos dessean honrra et de ser honrrados . } Mas en las honrras son tres cosas de cuydar & Forte multi viuentes vita politica credunt felicitatem ponendam esse in honoribus , \textbf{ eo quod ut plurimum Ciues maxime honorari desiderant . } Sunt autem in honoribus tria attendenda , \\\hline
1.1.8 & si conplida mente quiere demostrar le que significa conuiene \textbf{ que sea conosçida cosa e magnifiesta . } Mas las cosas que son de dentro del alma & si plene manifestare vult ipsum signatum , \textbf{ oportet quod sit quid notum et manifestum : } intrinseca autem non sunt nobis nota , \\\hline
1.1.8 & Et muy mas sin razones \textbf{ que la real magestad } ponga la su bien andança en las honrras . & si ab hominibus honoratur . \textbf{ Maxime tamen hoc est indecens regiae maiestati : } quod etiam triplici via venari potest . \\\hline
1.1.8 & Lo segundo se muestra \textbf{ assi que muy desconueible cosa es al Rey poner su bien andança en las honrras } Ca por esso seria prisuptuoso e sob̃uio & ø \\\hline
1.1.8 & que dizien torcato \textbf{ que era muy cudiçioso de grant honrra } contra el imperio de su padre . & de filio cuiusdam Romani Principis nomine Torquati , \textbf{ qui nimii honoris auidus , } contra Imperium Patris , \\\hline
1.1.8 & segunt las dignidades delas personas ¶ \textbf{ assi que de mayores bienes alos que son dignos e sabios } que non alos que non son dignos nin alos iuglares ¶ & secundum dignitatem personarum , \textbf{ ut plura bona decet dignis , | et sapientibus , quam indignis , } et Histrionibus . \\\hline
1.1.8 & qual es acomnedado de gouernar \textbf{ Ca non fara fuerça de poner el pueblo a grandes peligros presuptuosamente e arrebatadamente . } Et sera malo en partir sus aueres . & quia eas non distribuet aequaliter \textbf{ secundum personarum dignitatem . } Quod non decet regiam maiestatem , \\\hline
1.1.9 & t deuedes saber \textbf{ que entre estas dos cosas eglesia e fama ha grand diferençia } e departimiento destas otras dos & vel in \textbf{ Differunt autem gloria , } et fama ab honore , et laude : \\\hline
1.1.9 & Ca la alabaça propriamente non es \textbf{ si non por sseñales uocales e de palabras } Mas la honrra a de ser quales por se quier sseñales & quia est uniuersalior ea : \textbf{ nam laus proprie non est nisi per signa vocalia ; } sed honor esse habet per quaecunque signa exteriora : \\\hline
1.1.9 & Et esso mesmo es la fama \textbf{ Ca la fama es vn claro conosçimiento con loor . } Enpero si quisieremos fazer depart ineto entre la fama e la eglesia diremos que la fama nasçe de la eglesia & haec etiam est fama , \textbf{ quia fama est quaedam clara cum laude notitia . | Si tamen vellemus aliquo modo } distinguere \\\hline
1.1.9 & Calagh̃ia nasçe dela honrra \textbf{ Ca quando alguno es en grant honrra } e en grant aparado & quia gloria oritur ex honore : \textbf{ nam ex hoc quod aliquis est in magno honore , } et in magno apparatu , \\\hline
1.1.9 & e en fama e en eglesia \textbf{ por que ella es de grant anchura } e se estiende a muchas partes & posset forte alicui videri felicitatem ponendam esse in fama et gloria , \textbf{ eo quod ipse possit esse magnae latitudinis , } cum ad diuersas partes diuulgare possit : \\\hline
1.1.9 & Et otrosi \textbf{ por que es de grant durança det pon } por que dura por mucho stp̃os & cum ad diuersas partes diuulgare possit : \textbf{ et magnae diuturnitatis , } cum per multa tempora contingat ipsam indelebilem esse . \\\hline
1.1.9 & la fama non es bondat nr̃a \textbf{ nin nasçe della lanr̃a bondat . } Mas solamente es alguna señal & et quid causatum ab eis : fama nec est bonitas nostra , \textbf{ nec ab ea dependet bonitas nostra , } sed solum est quoddam apparens signum nostrae bonitatis . In ea igitur felicitas nostra esse non potest . Secundo hoc stare non potest , \\\hline
1.1.9 & Mas solamente es alguna señal \textbf{ que paresçe denr̃a bondat } e por ende non puede ser en ella lanr̃a feliçidat & nec ab ea dependet bonitas nostra , \textbf{ sed solum est quoddam apparens signum nostrae bonitatis . In ea igitur felicitas nostra esse non potest . Secundo hoc stare non potest , } quia fama , et gloria \\\hline
1.1.9 & que estan dentro en el coraço \textbf{ mas si iudgamos algᷤ cosa buean e uirtuosa } esto es & non enim videmus ipsas cogitationes mentis , non ipsas virtutes , et malitias in animo existentes : \textbf{ sed si iudicamus aliquem bonum , } et virtuosum , \\\hline
1.1.9 & que alguno sea en fama e en eglesia entre los omes abasta \textbf{ que paresca bueno e muestre algua bondat de fuera } por la qual cosa commo al Rey conuenga ser todo diuinal & et gloria , \textbf{ sufficere videtur , | quod exterius bona praetendat . } Quare cum Regem deceat esse totum diuinum , \\\hline
1.1.9 & que paresçen de fuera \textbf{ Pues que assi es lanr̃a bondat desçende derechamente del conosçimiento de dios } assi commo obra de su obrador & et directe solum exteriora comprehendit : \textbf{ bonitas ergo nostra per se dependet a notitia Dei , } tanquam effectus a sua causa . \\\hline
1.1.9 & e assi commo cosa fechan de su fazedor \textbf{ ¶Otro si el conosçimientode dios no puede ser engannado en lanr̃a bondat . } Ca lascian de dios non puede resçebir enganno . & tanquam effectus a sua causa . \textbf{ Rursus circa bonitatem nostram notitia } Dei non fallit , \\\hline
1.1.9 & que dios mas claramente vee \textbf{ e entiende lanr̃a bondat } e lanr̃a maliçia de dentro del alma & et malitiam nostram intimam Deus clare videt , \textbf{ et percipit . } Non igitur esse potest , \\\hline
1.1.9 & que nos mesmos . ¶ pues que assi es non puede ninguno ser ante dios en eglesia e en fama \textbf{ nin puede ser ante e en claro conosçimiento con alabança } si non fuere bueno en uerdat e de fecho & apud Deum sit in gloria , \textbf{ et in fama , vel sit apud ipsum in clara notitia cum laude , } nisi sit bonus , \\\hline
1.1.9 & que es ante el . \textbf{ Enpo en ninguna manera non es de poner la bienandança de lons omes } en la fama de los omes & et de gloria apud ipsum Deum , \textbf{ nequaquam tamen in fama , } et in gloria hominum felicitas est ponenda . \\\hline
1.1.9 & mas por tanto dixo esto el philosofo \textbf{ por que conuiene alos Reys alos prinçipes de ser manificos e grandes e magnanimos e de grandes coraçones . } Ca commo quier los de altos coraçones entienden prinçipalmente de tomar honrra & sed hoc pro tanto dictum est , \textbf{ qua decet Reges , et Principes esse magnificos , | et magnanimos . } Magnanimi autem licet non intendant principaliter honorem , \\\hline
1.1.9 & por que conuiene alos Reys alos prinçipes de ser manificos e grandes e magnanimos e de grandes coraçones . \textbf{ Ca commo quier los de altos coraçones entienden prinçipalmente de tomar honrra } mas de alcançar algun bien & et magnanimos . \textbf{ Magnanimi autem licet non intendant principaliter honorem , } sed bonum : \\\hline
1.1.9 & que les fazenn los omes \textbf{ por que los omes non les pueden dar mayor cosa que honrra } que non han meior cosa & honor tamen eos consequitur , \textbf{ et decet eos acceptare honorem sibi exhibitum , } non habentibus Hominibus aliquid maius , \\\hline
1.1.9 & por que los omes non les pueden dar mayor cosa que honrra \textbf{ que non han meior cosa } que les puedan dar ¶ & et decet eos acceptare honorem sibi exhibitum , \textbf{ non habentibus Hominibus aliquid maius , } quod eis tribuant , \\\hline
1.1.9 & e non propiamente la honrra en sy . \textbf{ Ca si el prinçipe non resçibiese esta buena uoluntad delas } que les fazen honrramas & et non proprie honor datus . \textbf{ Quod si tamen Principes non acceptarent hanc affectionem dantium , } sed requirerent a gente sibi commissa alia exteriora bona , \\\hline
1.1.9 & asi commo oro o plata o otras riquezas \textbf{ non serie buen prinçipe } mas serie tirano ¶ & vel argentum , \textbf{ vel diuitias alias , } Tyranni essent : \\\hline
1.1.10 & La quarta razon es \textbf{ que por este prinçipado los çibdadanos son ordenados alos menores bienes } e non alos mayores & Quarta autem , \textbf{ ex eo quod per huiusmodi principatum ciues ordinantur ad minora bona . } Quinta vero , \\\hline
1.1.10 & ¶La quinta daquello \textbf{ por que tal sennorio en la mayor parte faze gerat danno¶ } La primera razon se puede & Quinta vero , \textbf{ ex eo quod tale dominium | ut plurimum infert nocumentum . } Prima via sic patet . \\\hline
1.1.10 & por que es natraalmente franco \textbf{ por libre aluidrio } sienpre quiere ser libre & sic homo , \textbf{ qui est naturaliter liber arbitrio , } tunc naturaliter dominatur hominibus , \\\hline
1.1.10 & e non subiecto Estonçe \textbf{ a omne natraalmente es señor delons omes } quando ha señorio sobrellos liberalmente & qui est naturaliter liber arbitrio , \textbf{ tunc naturaliter dominatur hominibus , } quando eis libere , \\\hline
1.1.10 & assi commo non puede seer \textbf{ que en alguna cosa sea grand blancura } si aquella cosa non fuere muy blanca . & et sit maxime bonus : \textbf{ sicut impossibile est in aliquo esse intensam albedinem , } nisi ille fit intense albus . \\\hline
1.1.10 & e prouamos \textbf{ que muchos muy malos tiranos ouieron muy grant poderio çiuil } assi commo dize el philosofo en las politicas & Nam Dionysius Syracusanus , \textbf{ siue Sicilianus , } ut recitat Philosophus in politicis , \\\hline
1.1.10 & assi commo dize el philosofo en las politicas \textbf{ que dionisio siracusano o dionisios çiçiliano ouo muy grant poderio çiuil empero fue muy mal tirano ¶ Masnero e cesar } que fueron prinçipes Ro manos & ut recitat Philosophus in politicis , \textbf{ maxime abundauit in ciuili potentia , | et tamen pessimus Tyrannus erat . Nero autem , } et Heliogabalus , \\\hline
1.1.10 & que dionisio siracusano o dionisios çiçiliano ouo muy grant poderio çiuil empero fue muy mal tirano ¶ Masnero e cesar \textbf{ que fueron prinçipes Ro manos } e abondaron mucho en poderio çiuil . & et Heliogabalus , \textbf{ qui fuerunt Romani Principes , } maxime abundauerunt in ciuili potentia ; \\\hline
1.1.10 & Empero viuian muy mal \textbf{ por que fueron de tan grand luxͣia } que en todas maneras paresçiençia mugeriles . & pessime tamen viuebant . \textbf{ Nam tantae fuerunt luxuriae , } ut omnino viderentur muliebres ; tantae fuerunt crudelitatis , \\\hline
1.1.10 & que en todas maneras paresçiençia mugeriles . \textbf{ Et fueron de tan grant crueldat } que non paresçia en ellos ningunan cosablanda & Nam tantae fuerunt luxuriae , \textbf{ ut omnino viderentur muliebres ; tantae fuerunt crudelitatis , } ut non videretur in eis esse aliquid molle , \\\hline
1.1.10 & mas de grand cruel dat ¶ \textbf{ Et por ende el philosofo dize en el se partimo libro delas politicas } que cosa de escarnio es cuydar & nec clementia aliqua . \textbf{ Unde Philosophus 7 Politicorum ait , } quod ridiculum est \\\hline
1.1.10 & Et por ende dize el philosofo \textbf{ en el se partimo libro delas polticas } que el prinçipado & non est optimus , \textbf{ neque dignus . Ideo 7 Politicorum dicitur , } quod principatus liberorum , \\\hline
1.1.10 & alos bienes mayores \textbf{ e a mayores uertudes } asi commo la uistiçi & Non ergo ordinabit ciues ad bona maiora , \textbf{ ut ad iustitiam , } sed solum ad minora , \\\hline
1.1.10 & por que non conuiene al prinçipe poner la su bien andança en el poderio çiuiles \textbf{ por que este sennorio faze grant danno en las mas cosas . } Ca commo la feliçidat & Quinto hoc non decet ipsum , \textbf{ quia huiusmodi principatus infert | ut plurimum nocumentum . } Nam cum felicitas sit finis omnium operatorum , quilibet totam vitam suam , \\\hline
1.1.10 & Enpero non sabra beuir bien entp̃o de paz . \textbf{ Ca commo en la mayor parte non aya estudiando } si non en vsos de armas e de batallas . & tempore tamen pacis nesciet bene viuere . Nam , \textbf{ cum ut plurimum studuerit , } nisi in exercitiis bellicis , \\\hline
1.1.10 & mas sera uiçioso e pecador . \textbf{ Et contesçer le ha muy grant daño segunt su alma . } por la qual cosa dize el philosofo en el septimo libro delas politicas & nesciet viuere , \textbf{ sed fiet vitiosus , et incurret nocumentum } secundum animam . Propter quod Philosophus 7 Politicorum vituperans Lacedaemones , \\\hline
1.1.10 & Et dize \textbf{ assi que torpe cosa es } que nos quando lidiamos & ait , \textbf{ turpe esse , } cum bellamus , \\\hline
1.1.10 & nin en cosa que non es digna nin buena . \textbf{ Et por la qual los omes son ordenados a menores bienes } posponiendo los mayores . & et est quid indignum , \textbf{ et per quod ordinantur ciues ad minora bona , } et ut plurimum infert nocumentum : \\\hline
1.1.11 & que le son estos bienes corporales paresçen le mayores de quanto son \textbf{ Mas despues que los ha auido paresçen meno res de quanto el cuydaua . } Mas las uertudes del alma & creduntur esse maiora , \textbf{ quam sint : eis autem adeptis , | apparent non esse tanta , } quanta credebantur . Virtutes autem , \\\hline
1.1.11 & Mas las uertudes del alma \textbf{ e los bienes intellectuales han contraria manera . } Ca quando los omes los han paresçen les mayores de quanto cuydaua & quanta credebantur . Virtutes autem , \textbf{ et bona intellectualia modum conuersum tenent : } nam eis adeptis , \\\hline
1.1.11 & Et la fermosura es mesuramiento conuenible de los mienbros . \textbf{ Et la fortaleza conuenible proporçion de los huessos e delons nieruos . } Por la qual cosa commo los humores & sanitas est debita adaequatio humorum . Pulchritudo est debita commensuratio membrorum . \textbf{ Robur debita proportio ossium et neruorum . } Quare cum humores , \\\hline
1.1.11 & que la materia . \textbf{ Et pues que assi es aquellas cosas son grandes bienes } que son de dentro & quam materia . \textbf{ Illa ergo sunt bona maxima interiora nostra , } quae se tenent ex parte animae : \\\hline
1.1.11 & o si ouiere los mienbros conformados vno a otro ¶ \textbf{ Et si fuere . fermoso o ouiere fermosa disposiçion de los mienbros } o ouiere buena proporçion de los huessos e de los neruios & vel quod habeat conformia membra , \textbf{ et sit pulcher : } vel habeat proportionem ossium et neruorum , \\\hline
1.1.11 & Et si fuere . fermoso o ouiere fermosa disposiçion de los mienbros \textbf{ o ouiere buena proporçion de los huessos e de los neruios } e si fuere fuerte en el cuerpo . & et sit pulcher : \textbf{ vel habeat proportionem ossium et neruorum , } et sit robustus corporaliter . \\\hline
1.1.11 & Et fuere sano e fuerte en la uoluntad \textbf{ Et si estas potençias del alma fueren conpuestas e enfermosadas e honrradas de uirtudes e de buenas obras } asi que el omne . sea fermoso en el alma . & et sit sanus , et fortis mente : \textbf{ et si huiusmodi potentias habeat ornatas virtutibus , | et bonis operibus } et sit pulcher in anima , \\\hline
1.1.11 & do dize que dios non es bien auentraado \textbf{ por ninguons bienes } que sean de fuera & ut idem ibidem innuit : \textbf{ nam Deus non est beatus per aliqua exteriora bona , } sed per ea quae sunt in seipso . \\\hline
1.1.11 & ¶por la qual cosa se tuelle la proporçion de los neruios \textbf{ e delons huessos } en los quales esta la fortaleza corporal . & quia laxantur membra , debilitantur nerui . Propter quod tollitur proportio illa neruorum , \textbf{ et ossium , } in quibus habet esse fortitudo corporalis . Sic etiam tollitur pulchritudo , \\\hline
1.1.11 & en los quales esta la fortaleza corporal . \textbf{ Et en essa misma gnisa ahun se tira la fermosura deł cuepo . } Ca perdida la sanidat fazen se los mienbros magros e flacos & et ossium , \textbf{ in quibus habet esse fortitudo corporalis . Sic etiam tollitur pulchritudo , } quia amissa sanitate , \\\hline
1.1.11 & o por engendrar fijos . \textbf{ Ca por fallesçimiento de los fijos muchos regnos ouieron grand departimiento } e grandes escandalos . & siue propter procreationem prolis : \textbf{ nam ex defectu filiorum multa regna passa sunt diuisionem , et scandala : } decedentibus enim Principibus absque liberis , \\\hline
1.1.11 & Ca por fallesçimiento de los fijos muchos regnos ouieron grand departimiento \textbf{ e grandes escandalos . } Ca muriendo los Reys sin fijos legitimos muchos se le una tan para seer señores & siue propter procreationem prolis : \textbf{ nam ex defectu filiorum multa regna passa sunt diuisionem , et scandala : } decedentibus enim Principibus absque liberis , \\\hline
1.1.11 & Ca muriendo los Reys sin fijos legitimos muchos se le una tan para seer señores \textbf{ e fazen grand discordia en el pueblo ¶ } Otrossi deuen los prinçipes auer riquezas sufiçientes & plures insurgunt , \textbf{ ut dominentur , | et faciunt dissensionem in Populo . } Debet enim Princeps possidere sufficientes diuitias , \\\hline
1.1.11 & la qual cosa non se puede fazer sin riquezas ¶ \textbf{ Ahun en essa misma manera es el Rey digno de honrra } e deue seer honrrado & beneficiare personas dignas : \textbf{ quod sine diuitiis fieri non potest . Sic etiam , } ne vilipendatur maiestas regia , \\\hline
1.1.11 & e deue seer honrrado \textbf{ por que non sea menospreçiada la Real magestad . } Et por ende le conuiene de auer poderio çeuil . & quod sine diuitiis fieri non potest . Sic etiam , \textbf{ ne vilipendatur maiestas regia , } est Rex dignus honore , et expedit ei habere ciuilem potentiam : \\\hline
1.1.11 & Et por ende le conuiene de auer poderio çeuil . \textbf{ Ca por el menospreçiamientodel prinçipe muchͣs vezes contesçe que alguons fazen } e obran malas cosas & est Rex dignus honore , et expedit ei habere ciuilem potentiam : \textbf{ nam propter paruipensionem Principis , } ut plurimum aliqui operantur mala , et offendunt alios , \\\hline
1.1.11 & Ca por el menospreçiamientodel prinçipe muchͣs vezes contesçe que alguons fazen \textbf{ e obran malas cosas } e fazen muchos tuertos a otros & nam propter paruipensionem Principis , \textbf{ ut plurimum aliqui operantur mala , et offendunt alios , } quod Regno non expedit . Sic \\\hline
1.1.11 & la qual cosa non conuiene al Regno . \textbf{ Et ahun en essa misma guła deue auer cuydado el prinçipe de auer buena fama . } Ca por esso se enduzen los sus subditos & etiam \textbf{ debet esse curae ipsi Principi de debita fama , quia propter hoc inducuntur subditi ad virtutem . } Nam ( ut probatum est ) \\\hline
1.1.11 & para fazer bien . \textbf{ Ahun en essa misma manera la sanidat e la fermosura } e la fortaleza conuiene bien al prinçipe non por que en ellos sea propiamente la feliçidat e la bien andança & subditi suscipiunt materiam benefaciendi . Sic etiam , \textbf{ sanitas , pulchritudo , } et fortitudo competunt Principi , \\\hline
1.1.12 & Et la uirtud acabada en la uidaçon tenplatiua es sapiençia o methaphisica \textbf{ segund esse mismo philosofo ¶ } Et pues que assi es qual si quier omne & siue Metaphysica , \textbf{ secundum ipsum , } quicunque scit alios bene regulare \\\hline
1.1.12 & segunt pradençia \textbf{ que es derecha sabiduria de todas las obras } que han de fazer & ø \\\hline
1.1.12 & Enpero por que paresca lo primero \textbf{ en qual manera conuenga ala Real magestad } de poner la primera feliçidat en las obras de pradençia . ¶ Et la segunda commo le conuiene de poner er la su bien andança solamente en dios . & ut appareat , \textbf{ quomodo deceat regiam maiestatem | ponere } suam felicitatem in actu prudentiae , \\\hline
1.1.12 & e delan razon es bien comun e entelligible \textbf{ Conuiene ala Real magestad } en quanto es omne & bonum uniuersale , et intelligibile , \textbf{ decet Regiam maiestatem } eo ipso quod homo est , \\\hline
1.1.12 & Ca es bien de todos los bienes \textbf{ por que en el es fallada tonda bondat . } Et otrosi es bien muy entelligible & quia est bonum omnis boni , \textbf{ in eo enim omnis bonitas reperitur : | est } etiam bonum maxime intelligibile , \\\hline
1.1.12 & mas ahun es ministro \textbf{ e ofiçial de dios en espreçial manera . } Ca aquel que he alguna cosa por partiçipaçion & non solum quia homo est , \textbf{ sed etiam quia speciali modo est Dei minister . } Nam illud quod habet aliquid per participationem , \\\hline
1.1.12 & e de gouernar prinçipalmente e acabadamente . \textbf{ Conuiene que qual se quier prinçipeo Rey } que ha de gouernar sea instrumento de dios & et perfecte solus Deus , \textbf{ oportet quod quicunque principatur , | siue regnat , } sit diuinum organum , \\\hline
1.1.12 & Conuiene al Rey \textbf{ que es ofiçial de dios poner la su bien andança en dios que es prinçipal señor } e del solo deue esperar gualardon e merçed & decet Regem , \textbf{ qui est Dei minister , suam felicitatem ponere in ipso Deo , } et suum praemium expectare ab ipso . Tertio hoc decet Regem , ex eo , \\\hline
1.1.12 & en aquel que es bien mas comun \textbf{ que es solo dios . } Pues que assi es el Rey lo vno & quod est maxime , et commune bonum . \textbf{ Huiusmodi autem est ( ut dicebatur ) Deus ipse : } Rex ergo tum quia est Dei minister , \\\hline
1.1.12 & Et esta es obra de caridat e de amor de dios \textbf{ Ca el amor e la caridat ha muy grant fuerça para nos ayuntar con dios ¶ } Et por ende dionisio & siue charitatis . \textbf{ Nam amor , | et dilectio maxime vim unitiuam , } et coniunctiuam habent . \\\hline
1.1.12 & commo aquel commo humanal o angelical o diuinal \textbf{ ha muy grant uirtud de ayuntar } al que ama & siue humanum , \textbf{ siue naturalem , } siue animalem , \\\hline
1.1.12 & por la qual nos somos ayuntados con dios \textbf{ sin ningun medio enlła } deuemos poner la feliçidat e la bien andança & Propter quod in ipso actu charitatis , \textbf{ per quam immediatius coniungimur ipsi Deo , } magis est ponenda felicitas , \\\hline
1.1.13 & mas alcança dela semeiança de dios \textbf{ e mas se conforma con el mayor gualardon resçibra del . } Mas el estado del Rey demanda que sea mas acordable et mas acordable & quanto ergo quis magis gerit imaginem eius , \textbf{ et plus se conformat ei , | maius meritum ab ipso suscipiet : } Principis autem status requirit , \\\hline
1.1.13 & e guardan se delo fazer \textbf{ Enpero si a mayor estado fuese leunata dos aurian razon } para fazer muchͣs males . & praeseruant se a malo : \textbf{ quod si tamen ad statum dignitatis assumerentur , } multas transgressiones efficerent . Propter quod Ethic’ 5 scribitur , \\\hline
1.1.13 & Enpero si a mayor estado fuese leunata dos aurian razon \textbf{ para fazer muchͣs males . } Et por esso dize aristotiles en el quinto libro delas ethicas & quod si tamen ad statum dignitatis assumerentur , \textbf{ multas transgressiones efficerent . Propter quod Ethic’ 5 scribitur , } quod principatus virum ostendit . Tunc enim apparet qualis homo sit , cum in principatu existens , in quo potest bene et male facere , \\\hline
1.1.13 & si non trispassaren los mandamientos de dios \textbf{ conmolos podiessen trispassar son de mayor meresçimiento . } Et dezimos que los reys trabaian en el bien comun . & si non transgrediantur , \textbf{ cum possint transgredi , | maioris meriti esse videntur . Dicimus autem } ( vacantes communi bono ) \\\hline
1.1.13 & Ca si non trabaiassen en el bien comun \textbf{ non les conuernia de ser de mayor meresçimiento } por que ellos pueden passar los mandamientos & quia si bono communi non vacarent , \textbf{ non oportet eos esse maioris meriti ex hoc quod transgredi possent , } quia non considerato communi bono , \\\hline
1.1.13 & mas penssando el bien comun \textbf{ por que se ponen a peligro han mayor meresçiminto . } Ca puesto que non passen la ley & quia non considerato communi bono , \textbf{ si quis periculo se exponat , } dato quod non transgrediatur , \\\hline
1.1.13 & ¶ La terçera razon \textbf{ por que es grande el meresçimiento de lons Reyes } e de los prinçipes & quia indiscrete agit , suum meritum non augmentatur . Tertio , \textbf{ magnum est meritum Principis , } considerato actu , \\\hline
1.1.13 & tanto es meior \textbf{ e da mayor meresçimiento . } Mas mucho paresçe segunt natera & et ordinem rationis , \textbf{ tanto est magis bonus , et magis meritorius . Valde autem } secundum naturam esse videtur , \\\hline
1.1.13 & que les son acomne dadas \textbf{ por las sus buenas obras } e por el su buen gouierno & Reges ergo si bene regant gentem sibi commissam , \textbf{ ex operibus eorum consequenter mercedem magnam : } quia pro bono gentis . Et pro bono totali , \\\hline
1.1.13 & por las sus buenas obras \textbf{ e por el su buen gouierno } alcançara grant merçed e grant gualardon de dios . & ex operibus eorum consequenter mercedem magnam : \textbf{ quia pro bono gentis . Et pro bono totali , } totaliter se exponunt . \\\hline
1.1.13 & e por el su buen gouierno \textbf{ alcançara grant merçed e grant gualardon de dios . } Ca por el bien dela gente & ex operibus eorum consequenter mercedem magnam : \textbf{ quia pro bono gentis . Et pro bono totali , } totaliter se exponunt . \\\hline
1.1.13 & Ca por el bien dela gente \textbf{ e por el bien comun se ponen a grandes trabaios e a grandes periglos¶ } La quarta razon por que la merçed de lons Reyes es muy grande & quia pro bono gentis . Et pro bono totali , \textbf{ totaliter se exponunt . } Quarto magna est merces Regum ratione virtutis , \\\hline
1.1.13 & e por el bien comun se ponen a grandes trabaios e a grandes periglos¶ \textbf{ La quarta razon por que la merçed de lons Reyes es muy grande } e es por razon delas uirtu desque han . & totaliter se exponunt . \textbf{ Quarto magna est merces Regum ratione virtutis , } per qua quis meretur huiusmodi meritum : \\\hline
1.1.13 & Por las quales cada vno meresçe su gualardon . \textbf{ Ca menor uirtud cunple atondo omne } para gouernar & per qua quis meretur huiusmodi meritum : \textbf{ nam minor virtus requiritur ad regendum seipsum , } quam ad regendum familiam , \\\hline
1.1.13 & mas ahun a todo el regno . \textbf{ Et pues que assi es commo grant uirtud deua auer grant merçed } e gm̃t gualardon gerad sera el meresçimiento & ad quem spectat regere non solum se , et suam familiam , sed etiam totum regnum . \textbf{ Cum ergo magnae virtuti debeatur magna merces , } magnum erit meritum bene regentium regnum suum . Quinto , \\\hline
1.1.13 & La quinta razonn \textbf{ por que es grand el meresçimien todelons reys } si es esta & magnum erit meritum bene regentium regnum suum . Quinto , \textbf{ magnum erit praemium ipsorum Regum , } si consideretur materia , \\\hline
1.1.13 & en que obra el rey \textbf{ la qual materia es la gente e la muchedunbre delons pueblos esta muestra } que el su gualardon es muy grande¶ & circa quam operatur Rex , \textbf{ quae est gens , et multitudo , | indicat eius praemium esse magnum . Primae partis primi libri de regimine Principum finis . } Ubi tractauit , \\\hline
1.2.1 & nin en poderio çiuil \textbf{ nin en nuguaso tris cosas corporales nin tenporales } Mas assi commo prouamos conplidamente de suso deuen husar de todas estas cosas & nec in ciuili potentia , \textbf{ nec in aliquibus talibus , } sed omnibus his \\\hline
1.2.1 & e la bien andança de los Reyes \textbf{ e de los prinçipes es de poner en solo dios . } Ca deuen ellos ordenar la su uida & et iuste regant . \textbf{ Principaliter ergo Regum felicitas ponenda est in ipso Deo , } et ex cognitione et dilectione eius studium suum , \\\hline
1.2.1 & e amandol \textbf{ assi conmo oficiales uerdaderos suyos enderesçen } e guyen el pueblo & et diligentes Deum , \textbf{ tanquam veri ministri eius } secundum ordinem rationis dirigant Populum sibi commissum . \\\hline
1.2.1 & ca alguons destos poderios del alma son naturales \textbf{ e algunos son poderios senssitiuos conosçedores . } Et algunos desseadores & et Principes tales virtutes habere . Potentiae autem animae sic distingui possunt , \textbf{ quia potentiae animae quaedam sunt naturales , quaedam cognitiuae sensitiuae , quaedam appetitiuae , } et quaedam intellectiuae . Naturales potentiae sunt illae , in quibus communicamus cum vegetabilibus , \\\hline
1.2.1 & Et algunos desseadores \textbf{ e algunos intellectiuos e entendedores ¶ Naturales poderios son aquellos } enlos quales partiçipamos con los arboles e con las plantas & quia potentiae animae quaedam sunt naturales , quaedam cognitiuae sensitiuae , quaedam appetitiuae , \textbf{ et quaedam intellectiuae . Naturales potentiae sunt illae , in quibus communicamus cum vegetabilibus , } et plantis , \\\hline
1.2.1 & para bien obrar o mal assi commo contesçe \textbf{ que por las malas disposiconnes se determinan los poderios del alma } para mal fazer . & vel male agant , \textbf{ ut per habitus vitiosos determinatur potentia ad agendum male , } per virtuosos ad agendum bene : \\\hline
1.2.1 & para mal fazer . \textbf{ Et por las buenas disposiconnes e uirtudes se apareian a bien obrar . } ¶ Et pues que assi es commo la natura sea determimada a vna cosa & ut per habitus vitiosos determinatur potentia ad agendum male , \textbf{ per virtuosos ad agendum bene : } cum ergo natura sit determinata ad unum , \\\hline
1.2.1 & assi non es alabado \textbf{ por que vee aguda mente o oye sotil mente . saluo ende } si por auentura esto fuese por algun açidente & sic non laudatur ex eo quod acute videt , \textbf{ vel subtiliter audit , } nisi forte hoc esset per accidens , \\\hline
1.2.1 & nin por que mal moliese . \textbf{ mas por que por su mala constunbre perdio la uista } e la uirtud del estomago & vel quia non clare videt , \textbf{ quia clare videre , | et bene digerere } ( per se loquendo ) non est in potestate hominis : \\\hline
1.2.2 & Enpero puede se contar con las uirtudes morales \textbf{ por que la pradençia non es sinon en los buenos omes . } Assi commo las uirtudes morales non son & cum virtutibus moralibus : \textbf{ nam Prudentia non est nisi in hominibus bonis , } sicut nec virtutes morales : praui enim homines \\\hline
1.2.2 & si non en los buenos \textbf{ Ca los malos omes maguera sean sabidores e faldridos e engannosos } enpero non pueden ser pradentes ni sabios & sicut nec virtutes morales : praui enim homines \textbf{ et si possunt esse scientes , astuti , et versipelles ; } prudentes tamen esse non possunt , \\\hline
1.2.2 & Pues que assi es dexando aparte las uirtudes intellectuales \textbf{ e las scians especulatias } que son en el entendimiento especulatino . & Dimissis ergo virtutibus huiusmodi intellectualibus , \textbf{ ut dimissis scientiis speculatiuis , } de prudentia , \\\hline
1.2.2 & e esta scina non \textbf{ por que nos faga sabidores descina } mas porque nos faga buenos . & praesens opus , \textbf{ non ut sciamus , } sed ut boni fiamus : \\\hline
1.2.2 & Vnos enssituio \textbf{ por el qual alcançan su propia folgura e su propia delectaçion } assi commo es el appetito desseador . & Tribuit ergo eis duplicem appetitum sensitiuum , \textbf{ unum per quem prosequuntur propriam quietem , | et propriam delectationem , } ut concupiscibilem : \\\hline
1.2.2 & e por la frialdat se arrefieren de sus contrarios \textbf{ por que no sean corrunpidos de su propia natura } e ayan de desanparar los logares propios & per frigiditatem vero resistunt contrariis , \textbf{ ne corrumpantur a propria natura , } et deserant propria loca , \\\hline
1.2.2 & e ayan de desanparar los logares propios \textbf{ e la su propia folgura Vien } assi las ainalias e las bestias & et deserant propria loca , \textbf{ et propriam quietem . } Sic animalia per concupiscibilem fugiunt mala tristia , et prosequuntur \\\hline
1.2.2 & por el qual se enssanan acometen les sus contrarios \textbf{ que los podan enbargar de sus propias delectaçiones ¶ v̉bigera . } assi commo el leon & Per irascibilem vero aggrediuntur contraria , \textbf{ quae possent ea ab huiusmodi delectationibus prohibere . } Ut Leo per concupiscibilem pergit , \\\hline
1.2.2 & Mas el appetito del entendimiento \textbf{ que es dich̃o uoluntad non es departido en ningunas partes . } Ca commo el entendimiento sea mas general que el seso . & et concupiscibilis . Appetitus autem intellectiuus , qui dicitur voluntas , \textbf{ remanet indiuisus . | Nam } cum intellectus uniuersaliori modo respiciat suum obiectum quam sensus , \\\hline
1.2.2 & ¶ El appetito del entendimiento \textbf{ que se sigue a esse mismo entendimiento . } Mas generalmente ua al bien & appetitus intellectiuus , \textbf{ qui sequitur intellectum uniuersaliori modo , } fertur in bonum , \\\hline
1.2.2 & quanto es tomada so manera mas general \textbf{ tanto essa misma cosa se estieñde amas cosas } maguera el appetito & sub uniuersaliori modo accipitur , \textbf{ tanto unum | et idem existens ad plura se extendit , } licet appetitus sensitiuus sit alius , \\\hline
1.2.2 & assi podemos dezir \textbf{ que segund estos quatro poderios del alma } en los quales ha de seer la uirtud podemos tomar & dicere possumus \textbf{ quod } secundum has quatuor potentias animae in quibus habet esse virtus , sumptae sunt quatuor Virtutes Cardinales ; \\\hline
1.2.2 & en los quales ha de seer la uirtud podemos tomar \textbf{ e entender quatro uirtudes cardinales e generales delas quales ¶ } La vna es la pradençia e la otra la iustiçia & quod \textbf{ secundum has quatuor potentias animae in quibus habet esse virtus , sumptae sunt quatuor Virtutes Cardinales ; } videlicet , Prudentia , Iustitia , \\\hline
1.2.3 & La sexta es magnifiçençia \textbf{ que es uirtud para fazer grandes cosas ¶ } La septima es manssedunbre ¶ & Magnanimitatem , Largitatem , Magnificentiam , \textbf{ Mansuetudinem , } Veritatem , Affabilitatem , \\\hline
1.2.3 & La octaua uerdat ¶ \textbf{ La nona famil yaridat } ¶La . x̊ . . eutropolia & Mansuetudinem , \textbf{ Veritatem , Affabilitatem , } et Eutrapeliam , \\\hline
1.2.3 & que es uirtud \textbf{ que faze omne desçender a seer buen conpannon de todos ¶ } pues que assi es contando la iustiçia e la pradençia con estas dichas diez son & et Eutrapeliam , \textbf{ quam bene vertibilitatem , } vel societatem appellare possumus . Igitur computata Iustitia , et Prudentia duodecim sunt virtutes morales ; \\\hline
1.2.3 & Mas el cuento destas uirtudes puede se assi tomar . \textbf{ Ca las uirtudes son alguas meatades } e non han de seer & quas tangit Philosophos circa finem 2 Ethicor’ . Possunt alio modo sic accipi hae virtutes . \textbf{ Nam virtutes sunt medietates quaedam , } et non habent fieri , \\\hline
1.2.3 & Mas en quanto son en nos las passions regladas \textbf{ e orde nadas podemos tomar } e entender las otras diez uirtudes morales & ø \\\hline
1.2.3 & que son en nos dentro en el alma¶ \textbf{ Mas en la terçera manera se puede tomar esta misma diuision } e departimiento de las uirtudes . & passiones in nobis rectificantur , \textbf{ et moderantur . Tertio autem , | haec eadem diuisio sumi potest . } Nam omnis virtus moralis tendit in bonum rationis . \\\hline
1.2.3 & Mas la pradençia va a bien de razon \textbf{ por que acaba essa misma razon . } Ca la pradençia es en el entedimiento & vel quia huiusmodi bonum perficit , \textbf{ vel quia ipsum efficit , } vel quia ipsum custodit , et conseruat . Prudentia autem tendit in bonum rationis , \\\hline
1.2.3 & que nasçen del mal presente . \textbf{ Assi commo la fortaleza es çerca delas passiones quan asçen del mal futuro } que ha de venir . & quae oriuntur ex malo , \textbf{ ut fortitudo est circa passiones ortas ex malo futuro : } mansuetudo circa passiones ortas ex malo praesenti . \\\hline
1.2.3 & Et la manssedunbre es çerca delas passiones \textbf{ que nasçendel mal presente . } Mas si fueren tomadas las uirtudes cerca delas passiones & ut fortitudo est circa passiones ortas ex malo futuro : \textbf{ mansuetudo circa passiones ortas ex malo praesenti . } Si autem sumantur virtutes circa passiones , \\\hline
1.2.3 & asi commo bie de comer o de beuer \textbf{ e de o tristales cosas ¶ } Ay otro bien prouechoso & ut cibi , et potus , \textbf{ et talia : } quoddam vero utile , \\\hline
1.2.3 & Ca magnifiçençia es uirtud \textbf{ que faze grandes cosas . } Pues que assi es la liberalidat & quae est circa magnos sumptus : \textbf{ idem est enim magnificentia quod magnificens : } erit igitur liberalitas in concupiscibili , \\\hline
1.2.3 & que faze al coraçon alto \textbf{ e ha de seer çerca grandes honrras } Mas el amor de honrra sera en el appetito cobdiçiador ¶ & sic dicitur Magnanimitas , \textbf{ quae est circa honores . Erit autem honoris amatiua in concupiscibili , } Magnanimitas vero in irascibili . \\\hline
1.2.3 & Mas si estas uirtu des fueren tomadas çerca delas passiones \textbf{ que nasçen de lons bienes } en quanto los auemos comunes con los otros & Si vero huiusmodi virtutes sumantur circa passiones , \textbf{ quae oriuntur ex bonis , } ut communicamus cum aliis , sic ( ut dicitur secundo Ethicorum ) sumuntur tres virtutes . \\\hline
1.2.3 & La otra es eutropolia \textbf{ que quiere dezir buena conuerssaçion } o buena manera de beuir . & opera autem , et verba , \textbf{ ut communicamus } cum aliis deseruiunt nobis ad veritatem , vitam , et ludum . \\\hline
1.2.3 & que quiere dezir buena conuerssaçion \textbf{ o buena manera de beuir . } Mas la uerdat assi conma aqui fablamos de uerdat non en quanto es uirtud & opera autem , et verba , \textbf{ ut communicamus } cum aliis deseruiunt nobis ad veritatem , vitam , et ludum . \\\hline
1.2.3 & mas en quanto se toma aqui por uerdat deuida \textbf{ que nos ypocrisia nin vana eglesia } assi commo dezimos & cum aliis deseruiunt nobis ad veritatem , vitam , et ludum . \textbf{ Erit ergo triplex virtus ; } videlicet , Veritas , Affabilitas , \\\hline
1.2.3 & Mas entropolia \textbf{ que quiere dezir buena conpanma } o buena manera de beuir en conpanna es & non est discolus , sed est affabilis , et curialis . Eutrapelia vero siue bona versio , \textbf{ est , } quando aliquis sic se habet in ludis , \\\hline
1.2.3 & que quiere dezir buena conpanma \textbf{ o buena manera de beuir en conpanna es } quando alguno se sabe bien auer en los trebeios & non est discolus , sed est affabilis , et curialis . Eutrapelia vero siue bona versio , \textbf{ est , } quando aliquis sic se habet in ludis , \\\hline
1.2.3 & sino que non quiera de ninguna cosa fablar nin trebeiar . \textbf{ Mas sea buen conpanon } e sepa bien beuir con los omes & quod de nullo velit ludere : \textbf{ sed sit Eutrapelus } et bene se vertens , \\\hline
1.2.3 & Et la su uirtud es eutropolia \textbf{ que ̀ere dezir buena conpanma . } Mas estas tres uirtudes ya dichas & et bene se vertens , \textbf{ ut se habeat circa ludos prout expedit . } Omnes autem hae tres virtutes , \\\hline
1.2.3 & Pues que assi es paresçe \textbf{ que commo sean quatro poderios del alma } que pueden resçebir las uirtudes & quia non sunt circa aliquid arduum , sunt in concupiscibili . Patet ergo quod cum quatuor potentiae animae sint susceptibiles virtutum \textbf{ de quibus loquimur , } duodecim sunt huiusmodi virtutes , \\\hline
1.2.3 & Et en el appetito enssannadores la fortaleza \textbf{ para acometer grandes cosas . } Et la massedunbre & In irascibili est Fortitudo , \textbf{ Mansuetudo , } Magnanimitas , \\\hline
1.2.3 & para seer mansso e mesurado . \textbf{ Et magnanimidat por auer alto coraçon } Et la magnificençia & Mansuetudo , \textbf{ Magnanimitas , } et Magnificentia , \\\hline
1.2.3 & Et la magnificençia \textbf{ para fazer grandezas e grandes cosas . Las quales uirtudes se pueden tomar assi . } Ca la fortaleza e la manssedunbre son çerca delas passiones del coraçon & et Magnificentia , \textbf{ quae sic accipiuntur : } quia Fortitudo , \\\hline
1.2.3 & Ca la magnifiçençia es çerca de los bienes grandes e prouechosos \textbf{ assi commo en fazer grandes espenssas . } En la qual cosa se muestra omne por magnifico e granado . & quia Magnificentia est circa magna bona utilia , \textbf{ ut circa magnos sumptus : } Magnanimitas vero circa magna bona honesta , \\\hline
1.2.3 & que es asteza de coraçon \textbf{ es çerca delos grandes bienes e honestos } assi commo çerca de gran deshonrras . & ut circa magnos sumptus : \textbf{ Magnanimitas vero circa magna bona honesta , } ut circa magnos honores . In concupiscibili autem sunt sex virtutes , \\\hline
1.2.3 & es çerca delos grandes bienes e honestos \textbf{ assi commo çerca de gran deshonrras . } Ca el que quiere alcançar grandes honrras es magnanimo et de alto coraçon ¶ & ut circa magnos sumptus : \textbf{ Magnanimitas vero circa magna bona honesta , } ut circa magnos honores . In concupiscibili autem sunt sex virtutes , \\\hline
1.2.3 & assi commo çerca de gran deshonrras . \textbf{ Ca el que quiere alcançar grandes honrras es magnanimo et de alto coraçon ¶ } Mas en el appetito cobdiçiador son seys uirtudes & Magnanimitas vero circa magna bona honesta , \textbf{ ut circa magnos honores . In concupiscibili autem sunt sex virtutes , } videlicet , Temperantia , Liberalitas , \\\hline
1.2.3 & Ca el que quiere alcançar grandes honrras es magnanimo et de alto coraçon ¶ \textbf{ Mas en el appetito cobdiçiador son seys uirtudes } las quales son estas . & Magnanimitas vero circa magna bona honesta , \textbf{ ut circa magnos honores . In concupiscibili autem sunt sex virtutes , } videlicet , Temperantia , Liberalitas , \\\hline
1.2.3 & tenprança ¶ Lib̃alidat . \textbf{ O franqueza ¶ amorde honrra . } v̉dat de uida ¶a fabilidato bien fablança . & videlicet , Temperantia , Liberalitas , \textbf{ Honoris amatiua , Veritas , } Affabilitas , \\\hline
1.2.3 & que es buen a conpania . \textbf{ Las quales seys uirtudes se pueden assi tomar . } Ca las tres dellas & et Eutrapelia : \textbf{ quae sic accipiuntur , } quia tres harum , \\\hline
1.2.3 & Ca las tres dellas \textbf{ assi conmola tenprança e la liberalidat } e el amor de honrra se toman & quia tres harum , \textbf{ ut Temperantia , | Liberalitas , } et Honoris amatiua , sumuntur \\\hline
1.2.3 & Ca los bienes del omne en si son tres . \textbf{ Ca alguons bienes son delectables } e en estos ha de seer la tenprança ¶ & Nam bona hominis in se tria sunt : \textbf{ nam quaedam sunt delectabilis , } circa quae est Temperantia ; quaedam utilia , \\\hline
1.2.3 & en el omne en conparaçion \textbf{ de los otros entrs maneras se pueden entender . } O en quanto siruenanos & quaedam honesta , \textbf{ circa quae est honoris amatiua . Sic etiam bona in ordine ad alium tripliciter possunt considerari : } vel ut deseruiunt nobis ad manifestationem , \\\hline
1.2.3 & e assi es la uerdat \textbf{ ¶O los bienes nos siruen a buena uida e a buena manera } Et assi es afabilidat o familiaridat & et sic est veritas : \textbf{ vel ad vitam , } et sic est affabilitas : \\\hline
1.2.3 & por de buen solas \textbf{ e de buen trebeio . } Et assi es la uirtud & vel ad ludum , \textbf{ et sic est Eutrapelia . Ostensum est ergo , } quot sunt huiusmodi virtutes : \\\hline
1.2.3 & que llaman eutropolia \textbf{ que es buena conpannia } assi como dicho es ¶Pues que assi es mostrado auemos & et sic est Eutrapelia . Ostensum est ergo , \textbf{ quot sunt huiusmodi virtutes : } et quomodo distinguuntur . Quod bonarum dispositionum , quaedam sunt virtutes , quaedam supra virtutes , quaedam sunt ancillantes virtutibus , \\\hline
1.2.4 & Et assi auemos la fin de este capitulo . \textbf{ Neste capitulo } que agora dixiemos departimos & et \textbf{ Distinximus in praecedenti capitulo , } quod duodecim sunt Virtutes : \\\hline
1.2.4 & que agora dixiemos departimos \textbf{ en commo eran doze uirtudes delas quales . } La vna es en el entendimiento & Distinximus in praecedenti capitulo , \textbf{ quod duodecim sunt Virtutes : } quarum una est in intellectu , ut Prudentia : \\\hline
1.2.4 & que es grandeza de coraçon ¶ \textbf{ Et las seys uirtudesson en el appetito cobdiçiador } assi conmola tenprança & ut Temperantia , Liberalitas , \textbf{ Honoris amatiua , Veritas , } Affabilitas , et Eutrapelia , \\\hline
1.2.4 & Et las seys uirtudesson en el appetito cobdiçiador \textbf{ assi conmola tenprança } e las otras & Honoris amatiua , Veritas , \textbf{ Affabilitas , et Eutrapelia , } quas etiam in praecedenti capitulo enumerauimus . \\\hline
1.2.4 & assi es \textbf{ por que non cuydasse alguno que non auia otras buenas disposiconnes } si non estas uirtudes & quas etiam in praecedenti capitulo enumerauimus . \textbf{ Ne ergo aliquis crederet non esse aliquas alias bonas dispositiones praeter uirtutes enumeratas : decreuimus ostendere , } quod bonarum dispositionum \\\hline
1.2.4 & propusiemos de mostrar \textbf{ que fablando delans buenas disposiconnes del alma } delas quales fablaron los philosofos . & quod bonarum dispositionum \textbf{ ( loquendo de bonis dispositionibus , de quibus locuti sunt Philosophi , } quia de aliis ad praesens non intendimus tractatum constituere ) quaedam sunt uirtutes , \\\hline
1.2.4 & Ca delas otras non entendemos aqui fablar . \textbf{ dezimos que delas buenas disposiconnes algunas son uirtudes } e algunas siruientes alas uirtudes . & ( loquendo de bonis dispositionibus , de quibus locuti sunt Philosophi , \textbf{ quia de aliis ad praesens non intendimus tractatum constituere ) quaedam sunt uirtutes , } quaedam ancillantes uirtuti , \\\hline
1.2.4 & Et algunas sobre las uirtudes \textbf{ Ca largamente tomando la uirtud todas estas buenas disposiconnes se pueden llamar uirtudes . } Enpero algunas destas buenas disposiconnes siruen alas uirtudes . & Large enim accipiendo uirtutem , \textbf{ omnes huiusmodi bonae dispositiones , | uirtutes appellari possunt . } Nihilominus tamen quaedam bonae dispositiones ancillantur uirtuti , \\\hline
1.2.4 & Ca largamente tomando la uirtud todas estas buenas disposiconnes se pueden llamar uirtudes . \textbf{ Enpero algunas destas buenas disposiconnes siruen alas uirtudes . } assi commo bien consseiar & uirtutes appellari possunt . \textbf{ Nihilominus tamen quaedam bonae dispositiones ancillantur uirtuti , } ut bene consiliari , \\\hline
1.2.4 & e bie iudgar es apto e ydoneo para seer sabio . \textbf{ Mas algunas delas buenas disposiconnes } non son uirtudes conplidas & aptus est \textbf{ ut sit prudens . Quaedam uero bonae dispositiones non sunt completa virtus , } sed sunt dispositiones ad virtutem , \\\hline
1.2.4 & Mas continente es dicho aquel que es tentado e passionado \textbf{ e ha fuertes passiones } e tentaçiones en si . & qui passionatur \textbf{ et habet passiones fortes , } tamen continet se , \\\hline
1.2.4 & conmo quier que bien faga . \textbf{ Enpero non lo faze delectable mente } njn plazeterosa mente con mo faze el uirtuoso ¶ & ø \\\hline
1.2.4 & Enpero non lo faze delectable mente \textbf{ njn plazeterosa mente con mo faze el uirtuoso ¶ } Et pues que assi es la continençia e la perseuerançia & ø \\\hline
1.2.4 & Et pues que assi es la continençia e la perseuerançia \textbf{ en esta guisa tomadas maguera que sean algunas buenas disposiconnes } empero non son uirtudes . & quia propter passiones fortes licet bene agat , \textbf{ non tamen est ei delectabile bene agere . Continentia ergo , et Perseuerantia sic accepta , quamuis sint quaedam bona dispositio , } non tamen est virtus , \\\hline
1.2.4 & Mas son disposiconnes e apareiamientos para las uirtudes . \textbf{ Mas ay otras buenas disposiconnes } que son sobre las uirtudes & sed dispositio ad virtutem . \textbf{ Sunt etiam quaedam bonae dispositiones , } quae sunt supra virtutem , \\\hline
1.2.4 & e son buenos sobre la manera comunal . \textbf{ de lons omes . } Por la qual cosa tales omes pueden ser dichos sobre uirtuosos & sic aliqui sunt quasi diuini , \textbf{ et sunt boni supra modum propter quod tales , } superuirtuosi dici possunt . \\\hline
1.2.4 & que son mas uirtuosos \textbf{ que son los omes comunal mente . } Mas esta uirtud diuinal & superuirtuosi dici possunt . \textbf{ Huiusmodi autem uirtutem diuinam , } quae est quodammodo super virtus , maxime habere debent Reges \\\hline
1.2.4 & e çercanos a dios . \textbf{ ¶ Lo quarto algunas buenas disposiconnes son uirtudes } asi commo son la pradençia e la iustiçia & ( ut dictum est ) semidii esse debent . \textbf{ Quarto quaedam bonae dispositiones sunt ipsae uirtutes , } cuiusmodi sunt Prudentia , \\\hline
1.2.4 & de que fiziemos mençion en el capitu lo ant̃ dich̃o ¶ \textbf{ pues que assi es de todas estas quatro disposiconnes } diremos & de quibus in praecedenti capitulo fecimus mentionem . \textbf{ De omnibus ergo his quatuor suo loco dicemus . Determinabimus ergo de uirtutibus , } ostendentes , \\\hline
1.2.5 & mas non son todas aquellas doze yguales \textbf{ nin de ygunal perfecçion } nin son igualmente prinçipales . & Enumerauimus supra duodecim virtutes : \textbf{ sed non omnes illae duodecim sunt aequalis perfectionis , } nec sunt aeque principales . Consueuit enim apud Sanctos , \\\hline
1.2.5 & e menos prinçipales \textbf{ e ayuntables alas prinçipales . } Mas las cardinales son quatro delas quales . & quia quaedam sunt Cardinales et principales , \textbf{ quaedam vero annexae . Cardinales autem sunt quatuor , } videlicet , Prudentia , \\\hline
1.2.5 & e non prinçipales son las otras ocho delas quales fezimos mençion de suso ¶ \textbf{ Mas que estas quatro uirtudes sean prinçipales } e cardinales podemos lo mostrar & et Temperantia . Annexae autem et non principales sunt aliae octo , \textbf{ de quibus supra fecimus mentionem . Has autem quatuor virtutes esse Cardinales et principales , triplici via inuestigare possumus . Prima via sumitur ex parte materiae , circa quam versantur . Secunda ex parte subiecti , } in quo existunt . \\\hline
1.2.5 & Por la qual cosa somos endereçados \textbf{ en razonando delans obras } en essa mis ma guas a auemos de dar uirtud & sic ut est dare virtutem , \textbf{ per quam dirigimur in ratiocinando de agibilibus : } sic est dare virtutem , \\\hline
1.2.5 & e resçebidores de passiones en el alma derechamente \textbf{ e non derecha mente . } Conuiene nos de dar uirtudes algunas & Amplius quia contingit nos passionari recte \textbf{ et non recte , } oportet dare virtutes aliquas , \\\hline
1.2.5 & por las quales somos muy prestos e inclinados \textbf{ para fazer aquells males que cobdiçiamos } Et algunas passiones son & ut passiones concupiscibiles , \textbf{ quia proni sumus } ad agendum illa : quaedam vero retrahunt nos a bono , \\\hline
1.2.5 & en la qual obran son dichas estas quatro cardinales e prinçipales \textbf{ Ca son cerca tal materia cerca la qual prinçipal mente se trabaia toda la uida humanal . } La segunda razon para mostrar que estas uirtudes quatro son cardinales & et principales ; \textbf{ quia sunt circa materiam illam , circa quam principaliter versatur humana vita . Secunda via ad inuestigandum has esse virtutes cardinales et principales , } sumi potest ex parte subiecti , \\\hline
1.2.5 & Ca dicho es de suso \textbf{ que estas quatro uirtunds } de que aqui fablamos son en los poderios del alma . & in quo existunt . Dictum est enim supra , \textbf{ virtutes , } de quibus loquimur , \\\hline
1.2.5 & Et pues que assi es commo en el entendimiento pratico \textbf{ la mas prinçipal uirtud son la pradençia . } Et en la uoluntad la prinçipal uirtud sea la iustiçia fablando delas uirtudes & in concupiscibili : \textbf{ cum ergo in intellectu practico principalior virtus sit prudentia , } in voluntate \\\hline
1.2.5 & la mas prinçipal uirtud son la pradençia . \textbf{ Et en la uoluntad la prinçipal uirtud sea la iustiçia fablando delas uirtudes } que nos ganamos & ø \\\hline
1.2.5 & Et en el appetito enssannador \textbf{ la mas prinçipal uirtud sea la fortaleza . } Et en el apetito cobdiçian dor la mas prinçipal sea virtud la tenprança ¶ & ( loquendo de virtutibus acquisitis ) principalior sit Iustitia , \textbf{ in irascibili vero principalior sit Fortitudo , } et in concupiscibili Temperantia : ideo haec quatuor virtutes , scilicet Prudentia , Iustitia , Fortitudo , et Temperantia , principales et cardinales esse dicuntur . \\\hline
1.2.5 & conuiene que se faga sabiamente \textbf{ e iusta mente . fuerte mente . } e tenprada mente . & oportet quod fiat prudenter , \textbf{ iuste , | fortiter , } et temperate : \\\hline
1.2.5 & lo magnanimidat e gndeza de coraçon es vn conponimiento de todas las otras uirtudes . \textbf{ Et ahun en essa misma manera la magnifiçençia } que es uirtud para fazer grandes cosas ha alguna prinçipalidat & quia secundum Philosophum 4 Ethicorum , \textbf{ Magnanimitas uidetur esse ornatus quidam omnium aliarum uirtutum . Sic etiam et Magnificentia quandam principalitatem habet propter magnitudinem sumptuum , circa } quem uersatur . Habent ergo aliae uirtutes quandam principalitatem , \\\hline
1.2.5 & Et ahun en essa misma manera la magnifiçençia \textbf{ que es uirtud para fazer grandes cosas ha alguna prinçipalidat } por la grandeza delas cosas espenssas & ø \\\hline
1.2.5 & Et pues que assi es . \textbf{ si estas quatro uirtudes son prinçipales } e caddinales en conparaçonn delas otras . & in prosequendo de eis singulariter plenius ostendetur . \textbf{ Si ergo hae quatuor sunt principales , } et cardinales respectu aliarum , \\\hline
1.2.6 & en la qual ha de obrar \textbf{ ¶L quarto alascina ¶ } Et lo quinto ala arte delas quales dos cosas se departe ¶ & ø \\\hline
1.2.6 & que la pradençia es perfeçion del entendimiento \textbf{ que as buena calidat endereçadora } e regladora del alma . & ut comparatur ad virtutes morales , sic diffiniri potest , \textbf{ quod est perfectio intellectus , } siue quod est bona qualitas mentis , directiua in finem virtutum moralium . \\\hline
1.2.6 & Et pues que assi es en el nuestro entendimiento \textbf{ deuen ser tres uirtudes ¶ Vna } por la qual bien busquemos & et iudicata . \textbf{ In intellectu ergo nostro debent esse tres virtutes . } Una per quam bene inueniamus \\\hline
1.2.6 & seg̃t las reglas vniuerssales . \textbf{ las quales reglas generales son buenas leyes e buenas costunbres . } Et otras cosas tales & secundum uniuersales maximas particularia facta concernens . \textbf{ Huiusmodi autem uniuersales regulae sunt bonae leges , } debitae consuetudines , \\\hline
1.2.7 & que por la pradençia somos enderesçados \textbf{ e guiados derechamente a buena fin } a la qual nos inclinan las uirtudes morales . & et ostenso quod per prudentiam \textbf{ recte dirigimur in bonum finem , } in quem inclinant virtutes morales : \\\hline
1.2.7 & que assi e alos otros pueden catar \textbf{ e pueer buenas cosas . } Pues que assi es la pradençia & qui sibi et aliis possunt bona speculari \textbf{ et prouidere . } Prudentia ergo est quidam oculus , \\\hline
1.2.7 & Ca los mercandores su mando e razoñado cuentan algunas vezes vn dinero de cobre de plomo \textbf{ e ponen le en logar de minl libras . } Pues que assi es el dinero que non vale vn hueuo representa ualor de grant preçio . & vel plumbeus positus in computo mercatorum , \textbf{ Mercatores enim cum ratiocinando computant , } aliquando \\\hline
1.2.7 & e ponen le en logar de minl libras . \textbf{ Pues que assi es el dinero que non vale vn hueuo representa ualor de grant preçio . } Et aquel dinero & Mercatores enim cum ratiocinando computant , \textbf{ aliquando } unum denarium aeneum vel plumbeum ponunt loco mille librarum : \\\hline
1.2.7 & e husare de dignidat de Rey \textbf{ commo el sea de poco ualor esta en logar de grant preçio . } Pues que assi es & quod ergo non valet unum ouum , repraesentat valorem magni precij . Denarius ergo ille magis est signum valoris , quam valeat . Sic si vir prudentia careat , et regia dignitate fungatur , \textbf{ cum ipse parui valoris sic , | est loco magni precii : } magis ergo est signum regis , \\\hline
1.2.7 & que por la sabiduria somos guiandos \textbf{ e endereçados a buena fin } ala qual nos inclinan las uirtudes morales . & Dictum enim est , \textbf{ quod per prudentiam dirigimur in bonum finem , } in quem inclinant virtutes morales . Est enim prudentis , prouidere bona sibi et aliis , \\\hline
1.2.7 & ala qual nos inclinan las uirtudes morales . \textbf{ Ca de omne sabio es proueer buenas cosas . } assi e alos otros & quod per prudentiam dirigimur in bonum finem , \textbf{ in quem inclinant virtutes morales . Est enim prudentis , prouidere bona sibi et aliis , } et dirigere se et alios in optimum finem . \\\hline
1.2.7 & assi e alos otros \textbf{ e de guiar assi e alos otros a buena fin ¶ } pues si alguno non ouiere sabiduria & in quem inclinant virtutes morales . Est enim prudentis , prouidere bona sibi et aliis , \textbf{ et dirigere se et alios in optimum finem . } Si ergo aliquis prudentia careat , per quam dirigimur in optima bona \\\hline
1.2.7 & pues si alguno non ouiere sabiduria \textbf{ por la qual guie asi e alos otros a muy grandes bienes } seg̃t uerdat de ligero se torna tirano . & et dirigere se et alios in optimum finem . \textbf{ Si ergo aliquis prudentia careat , per quam dirigimur in optima bona } secundum veritatem , \\\hline
1.2.7 & por la qual guie asi e alos otros a muy grandes bienes \textbf{ seg̃t uerdat de ligero se torna tirano . } Ca las riquezas e los bienes de fuera & Si ergo aliquis prudentia careat , per quam dirigimur in optima bona \textbf{ secundum veritatem , | de facili efficietur tyrannus : } quia diuitias et bona exteriora , \\\hline
1.2.7 & Ca las riquezas e los bienes de fuera \textbf{ que son grandes bienes segunt paresçen } e non seg̃t uerdat cuydara que son grandes bienes uerdaderamente . & quia diuitias et bona exteriora , \textbf{ quae sunt bona optima secundum apparentiam , } credet esse optima simpliciter . \\\hline
1.2.7 & que son grandes bienes segunt paresçen \textbf{ e non seg̃t uerdat cuydara que son grandes bienes uerdaderamente . } ante por que non son sabios non conosçen & quae sunt bona optima secundum apparentiam , \textbf{ credet esse optima simpliciter . } Immo quia imprudentes non cognoscunt nisi sensibilia bona , \\\hline
1.2.7 & por que sin ella non puede ser señor \textbf{ nin enssennorear natural mente . } Ca assi commo dize el philosofo en el primero libro delas politicas & et Principes habere prudentiam , \textbf{ quia sine ea non possunt naturaliter dominari . } Nam ( ut declarari habet 1 Polit’ ) \\\hline
1.2.7 & Et sabe gouernar assi \textbf{ e guiar alos otros a buena fin . } Ca esta uirtud de sabiduria non sola mente la alaban los dichos de los philosofos & et nouit se \textbf{ et alios in debitum finem dirigere . Hanc enim veritatem non solum approbant physica dicta , } sed etiam confirmant singula regimina naturalia . Videmus enim naturaliter homines dominari bestiis , \\\hline
1.2.7 & e guiar alos otros a buena fin . \textbf{ Ca esta uirtud de sabiduria non sola mente la alaban los dichos de los philosofos } mas ahun alabanla & et nouit se \textbf{ et alios in debitum finem dirigere . Hanc enim veritatem non solum approbant physica dicta , } sed etiam confirmant singula regimina naturalia . Videmus enim naturaliter homines dominari bestiis , \\\hline
1.2.7 & en el primero libro delas politicas \textbf{ la hmuger ha poco de sabiduria e flaco consseio . Ca comunal . } mente las mugers fallesçen dela sabiduria delas omes . & Sic etiam viri dominantur foeminis , \textbf{ quia ( ut declarari habet 1 Politic’ ) foemina habet consilium inualidum . } Communiter enim foeminae deficiunt a virorum prudentia : \\\hline
1.2.7 & e pocas vezes . \textbf{ Et por ende por la mayor parte la muger deue ser naturalmente subietta al ome } por que naturalmente fallesçe dela sabiduria del omne & et in paucioribus ut plurimum . \textbf{ Ergo foemina viro naturaliter debet esse subiecta , } eo quod naturaliter deficiat a viri prudentia . \\\hline
1.2.7 & por que naturalmente fallesçe dela sabiduria del omne \textbf{ ¶ahun en esta misma gusa las moços } e los mançebos conuiene & Ergo foemina viro naturaliter debet esse subiecta , \textbf{ eo quod naturaliter deficiat a viri prudentia . } Hoc etiam modo iuuenes naturaliter decet antiquioribus esse subiectos , \\\hline
1.2.8 & La quarta razon \textbf{ ¶La quina sotileza . o agudeza ¶ La sexta docterna e enssenança¶ } La septima experiençia e prueua¶ & ø \\\hline
1.2.8 & por esso es alguon dicho sabio \textbf{ porque es suficiente para enderesçar assi e alos otros e de guiar assi e alos otros a alguons bienes } o a algunas buenas fines ¶ & ex hoc aliquis dicitur esse prudens , \textbf{ quia est sufficiens dirigere se , | et alios in aliqua bona , } siue in aliquos bonos fines . \\\hline
1.2.8 & porque es suficiente para enderesçar assi e alos otros e de guiar assi e alos otros a alguons bienes \textbf{ o a algunas buenas fines ¶ } Pues que assi es quatro cosas nos conuieney de penssar . & et alios in aliqua bona , \textbf{ siue in aliquos bonos fines . } Quatuor ergo est ibi considerare , \\\hline
1.2.8 & o a algunas buenas fines ¶ \textbf{ Pues que assi es quatro cosas nos conuieney de penssar . } Conuiene de saber los bienes & siue in aliquos bonos fines . \textbf{ Quatuor ergo est ibi considerare , } videlicet , bona , \\\hline
1.2.8 & Et por razon dela manera \textbf{ segunt laquel guia . } Conuiene al Rey de ser entendido e razonable ¶ & oportet Regem esse memorem , et prouidum : \textbf{ propter modum } secundum quem dirigit , oportet ipsum esse intelligentem , et rationabilem : ratione propriae personae \\\hline
1.2.8 & Conuiene al Rey de ser entendido e razonable ¶ \textbf{ por razon de la su propia persona } que ha de guiar los otros . & propter modum \textbf{ secundum quem dirigit , oportet ipsum esse intelligentem , et rationabilem : ratione propriae personae } quae alios est dirigens , oportet quod sit solers , et docilis : \\\hline
1.2.8 & Ca si el Rey ha a guiar la su gente \textbf{ e la su conpanna a alguons bienes . } Conuiene que aya memoria de las cosas passadas . & congruit quod sit expertus et cautus . \textbf{ Si enim Rex debet gentem aliquam ad bonum dirigere , } oportet quod habeat memoriam praeteritorum , \\\hline
1.2.8 & Ca esto ninguno non lo pie de fazer . \textbf{ Mas conuiene al Rey de auer memoria delans cosas passadas } por que pue da & quia nulli agenti hoc est possibile , \textbf{ sed decet Regem habere praeteritorum memoriam , } ut possit ex praeteritis cognoscere , \\\hline
1.2.8 & que acaesçe n o pueden acaesçer \textbf{ por la mayor partida las cosas que han de venir } son semeiantes alos cosas que son passadas & Nam ( ut scribitur secundo Rhetoricorum ) in contingentibus agibilibus , \textbf{ ut plurimum futura sunt praeteritis similia . Secundo decet ipsum habere prouidentiam futurorum : } quia homines prouidentes futura bona , \\\hline
1.2.8 & sabiendo las leys e las costunbres buenas \textbf{ e las otrans cosas } que pueden ser prinçipios & et consuetudines bonas , \textbf{ et alia quae possunt esse Principia , } et regulae agendorum . Oportet autem quod sit rationalis , \\\hline
1.2.8 & aquello que dize el philosofo del magnanimo \textbf{ e del que ha grant coraçon en el deçimo libro delas ethicas } do dize que non conuiene al magnanimo menospreçiara & aliorum consiliis acquiescendo . Possumus enim dicere de Rege , \textbf{ quod dicitur de Magnanimo 4 Ethicorum , } quod non decet ipsum fugere commouentem . \\\hline
1.2.8 & para tomar doctrina de los otros tom̃ado conseio de buenos \textbf{ assi de Ricos omes commo de uieios commo de sabios commo de los otras } que aman el regno ¶ & et consiliis baronum , \textbf{ seniorum , | sapientum , } et diligentium regnum . \\\hline
1.2.8 & pues que assi esparesce \textbf{ ya que por razon dela su propia persona } que es gouernandor de los otros . & et diligentium regnum . \textbf{ Patet ergo quod ratione propriae personae quae est alios dirigens , } oportet Regem esse solertem , et docilem . \\\hline
1.2.8 & assi en las sçiençias praticas \textbf{ que son para obrar muchas vezes algunas malas cosas se mezclan con las buenas . } Por la qual cosa cuydan los omes & sed apparent vera : \textbf{ sic in agibilibus mala multotiens admiscentur bonis , } propter quod creduntur bona , \\\hline
1.2.9 & non se deuen dar auanidades \textbf{ mas deuen espender la mayor parte de su uida } en cuydar quales son las cosas & non debent vanitatibus intendere : \textbf{ sed maiorem partem vitae suae debent expendere in cogitando quae possunt esse regno proficua . } Quod non sic intelligendum est , \\\hline
1.2.9 & Esto non se entiende \textbf{ assi que ellos non de una auer alguas vezes algunos solazes corporales e honestos . } Mas deuen usar dellos tenpradamente & Quod non sic intelligendum est , \textbf{ ut nullas recreationes corporales habere debeant , } sed debent eis adeo moderate uti , \\\hline
1.2.9 & e los prinçipes se pueden fazer sabios \textbf{ por quatro maneras ¶ } La primera es esta & ut non impediantur in regimine regni sui . \textbf{ Seipsos ergo poterunt prudentes facere , } ut naturaliter regnum regant : \\\hline
1.2.9 & Et para esto deuen leer las coronicas \textbf{ e los fechos antigos de los buenos Reyes } por que ayan memoria de los fechos que passaron & ø \\\hline
1.2.9 & que deuen los reyes muy acuçiosamente catar las bueans cosas \textbf{ e los bueons fechos } que son de venir que pueden ser prouechosos al su regno . & Secundo debent diligenter intueri futura bona , \textbf{ quae possunt esse proficua regno : } et mala , quae possunt esse nociua . \\\hline
1.2.9 & que son de venir que pueden ser prouechosos al su regno . \textbf{ Et otrosi deuen penssar en los malos fecho } que pue den ser dannosos al su regno & quae possunt esse proficua regno : \textbf{ et mala , quae possunt esse nociua . } Nam ex hoc habebunt prouidentiam futurorum , \\\hline
1.2.9 & que pue den ser dannosos al su regno \textbf{ por que por esta manera aur̃a sabiduria delas cosas } que han de venir & et mala , quae possunt esse nociua . \textbf{ Nam ex hoc habebunt prouidentiam futurorum , } ut possint mala expeditius vitare , \\\hline
1.2.9 & e traer a su memoria las buenas costun bres \textbf{ e las buenas leyes . } Ca las bueans costunbres & et bona facilius adipisci . Tertio debent saepe recogitare bonas consuetudines , \textbf{ et bonas leges : } nam talia sunt maxime principia agibilium , \\\hline
1.2.9 & e las buenas leyes . \textbf{ Ca las bueans costunbres } e las buenas leyes son principalmente comienços & et bonas leges : \textbf{ nam talia sunt maxime principia agibilium , } sicut intellectus principiorum est . \\\hline
1.2.9 & Ca las bueans costunbres \textbf{ e las buenas leyes son principalmente comienços } e razones para bien obrar et para bien gouernar . Ca el entendimiento de los prinçipes & et bonas leges : \textbf{ nam talia sunt maxime principia agibilium , } sicut intellectus principiorum est . \\\hline
1.2.9 & lo que deue fazer \textbf{ quanto mas buean s leyes e bueans costunbres tiene en su memoria } para gouernar & Tanto ergo Rex magis intelligens est circa agibilia , \textbf{ quanto plures bonas leges , | et bonas consuetudines in mente habet : } ex quibus scire potest , \\\hline
1.2.9 & en qual manera ahun \textbf{ por estas leys buenas e buenas costunbres puede bien gouernar su regno } tomando delas razones conuenibles conclusiones & Quarto saepe saepius excogitare debet , \textbf{ quomodo per huiusmodi bonas leges , et consuetudines debite regnum regat , eliciendo ex eis debitas conclusiones agibilium . } Non enim sufficit esse intelligentem , \\\hline
1.2.9 & por estas leys buenas e buenas costunbres puede bien gouernar su regno \textbf{ tomando delas razones conuenibles conclusiones } para todas las cosas & Quarto saepe saepius excogitare debet , \textbf{ quomodo per huiusmodi bonas leges , et consuetudines debite regnum regat , eliciendo ex eis debitas conclusiones agibilium . } Non enim sufficit esse intelligentem , \\\hline
1.2.9 & delas quales fablamos \textbf{ ya en el capitulo soƀ dicho podran fazer assi mismos sabios } Mas por que la malicia es corronpadera dela razon & de quibus in praecedenti capitulo fecimus mentionem , \textbf{ poterunt seipsos prudentes facere . } Verum quia malitia est corruptiua principii . \\\hline
1.2.9 & con esto \textbf{ que deuen ser acordables prouisores engennosos e doctrinables } e auer las otras cosas & et Principes volunt esse prudentes , \textbf{ cum hoc quod debent esse memores , prouidi , solertes , et dociles , et alia , } quae superius diximus , oportet ipsos esse bonos , et non habere voluntatem deprauatam : \\\hline
1.2.10 & e enn iustiçia ygual . \textbf{ Mas la iustiçia legales uirtud general } e es en algua manera toda uirtud . & et quodammodo omnis virtus . \textbf{ Iustitia vero aequalis , } est quid speciale , \\\hline
1.2.10 & Mas la iustiçia legales uirtud general \textbf{ e es en algua manera toda uirtud . } Ca todas las uirtudes se ençierran enlla . & Iustitia vero aequalis , \textbf{ est quid speciale , } et est quaedam particularis virtus . \\\hline
1.2.10 & Ca todas las uirtudes se ençierran enlla . \textbf{ Mas la uirtud iguales uirtud espeçial } Et es vna uirtud singular . & est quid speciale , \textbf{ et est quaedam particularis virtus . } Nam ex hoc est quis iustus legalis , \\\hline
1.2.10 & Mas assi commo dize el philosofo \textbf{ en el primero libro dela grand ph̃ia moral . } La ley manda fazer las obras de todas las uirtudes . & quia adimplet praecepta legis . \textbf{ Sed ( ut dicitur primo Magnorum Moralium ) } lex praecipit actus omnium virtutum . \\\hline
1.2.10 & que deuen ser fechas segunt uirtudes ¶ \textbf{ Onde en esse mismo libro dize el philosofo } que la iustiçialegal es uirtud acabada . & secundum virtutes : \textbf{ unde ibidem dicitur , } quod Iustitia legalis est perfecta virtus . Sic etiam Ethicorum 5 scribitur , \\\hline
1.2.10 & que la iustiçialegal es uirtud acabada . \textbf{ Et en essa misma manera ahun dize el philosofo en el quinto libro delas ethicas } que la ley manda & unde ibidem dicitur , \textbf{ quod Iustitia legalis est perfecta virtus . Sic etiam Ethicorum 5 scribitur , } quod lex praecipit non derelinquere aciem , \\\hline
1.2.10 & sienpre la çibdat es meior \textbf{ que es de meiores conpannas . } Et en quanto en mas maneras son buenos los çibdadanos & siue in ordine ad alios , \textbf{ semper ciuitas melior est quae ex melioribus ciuibus constat : } et quanto pluribus modis sunt boni Ciues , \\\hline
1.2.10 & Mas esta iustiçia legal departese de cada vna dela sotras uirtudes en dos cosas . \textbf{ Ca commo quier que el iusto legal faga essas mismas obras } que faze el fuerte e el tenprado . & Differt autem huiusmodi Iustitia a qualibet virtute in duobus : \textbf{ nam licet eadem opera agat Iustus legalis , } quae agit fortis , \\\hline
1.2.10 & Et por ende la iustiçia legal \textbf{ commo quier que faga aquellas mismas obras } que faze la tenperança e la fortaleza & cui lex imponitur . \textbf{ Iustitia ergo legalis licet faciat illa eadem opera , } quae facit Temperantia , \\\hline
1.2.10 & Pues que assi es paresçe commo el iusto se gales \textbf{ en alguna manera uirtuoso en tonda uirtud } Et non se determina a ninguna manera espeçial & ø \\\hline
1.2.10 & si esta iustiçia sp̃al es dicha igual por que entiende a egualdat . \textbf{ Como los çibdadanos pue dan estos biens de fuera partiçipar en dos maneras desigualmente } siguese que dos maneras ay desta iustiçia particular . & Si igitur haec Iustitia specialis aequalis dicitur , \textbf{ et aequalitati intendit : | cum bona exteriora dupliciter ciues inaequaliter participare possint , } dupliciter erit huiusmodi particularis Iustitia . Accidit autem aliquos participare bona inaequaliter in commutationibus , \\\hline
1.2.11 & podemos lo prouar en dos maneras \textbf{ ¶la primera se toma de parte dessa misma iustiçia general¶ } La segunda se toma de parte del regno & duplici via inuestigare possumus . \textbf{ Prima sumitur | ex parte ipsius Iustitiae generale . Secunda , } ex parte regni , \\\hline
1.2.11 & pues que assi es \textbf{ quando los çibdadanos ennigua cosa non guardan las leyes } nin toma ninguna parte dela iustiçia legal . & legalis Iniustitia est integra , et perfecta malitia . \textbf{ In nullo ergo obseruare leges , } et ciues non participare in aliquo legalem Iustitiam , \\\hline
1.2.11 & Ca por esta razon es entre algunos iustiçia mudadora \textbf{ por que el vno abonda en algua cosa } en la qual fallesçe el otro . & Ex hoc enim inter aliquos commutatiua Iustitia ; \textbf{ quia unus abundat in uno , } in quo alter deficit : \\\hline
1.2.11 & que tiene aprouecho del otro \textbf{ Et esta misma manera ahun fallamos en los mienbros de vn cuerpo . } Ca el oio vee & et econuerso . \textbf{ Hunc autem modo reperimus in membris eiusdem corporis . Oculus enim pollet acumine visus : } deficit tamen a potentia gressiua , \\\hline
1.2.11 & assi conmo el cuerpo natural no podria estar \textbf{ si en el non fuesse guardada vna iustiçia partidora engsa } que los mienbros partiessen los vnos con los otros & non subsisteret , \textbf{ nisi in eo reseruaretur quaedam distributiua Iustitia , } ut quod cor membris debite influere : \\\hline
1.2.11 & es lo que dixo el philosofo \textbf{ en el primero libro dela grant ph̃ia moral } en el capitulo dela iustiçia & Bene ergo dictum est , \textbf{ quod dicitur 1 Magnorum moralium , } cap’ de iustitia , \\\hline
1.2.11 & Pues que assi es el que non guarda la iustiçia \textbf{ grant tuerto faze al regno e al rey . } Et si la iustiçia es tanto bien del rey & qui Iustitiam non obseruet . \textbf{ Si igitur Iustitia est tantum bonum Regis , et regni , } summo opere Rex studere debet , \\\hline
1.2.12 & magnifiestamente seria fecho tuerto al Rey e al regno \textbf{ su gregnos poro quier que assaz es ya prouado } por el capitulo sobredicho & ø \\\hline
1.2.12 & e los regnos non pueden durar \textbf{ Empero por que el coraçon del noble ome sienprees cobdiçioso de oyr nueuas razones . } Por ende aduremos agora nueuas maneras e razones & et regna durare non possint . \textbf{ Tamen quia est animus hominis generosus , | et semper auidus nouas rationes audire : } adducemus nouos modos , \\\hline
1.2.12 & Empero por que el coraçon del noble ome sienprees cobdiçioso de oyr nueuas razones . \textbf{ Por ende aduremos agora nueuas maneras e razones } por las quales podremos mostrar & et semper auidus nouas rationes audire : \textbf{ adducemus nouos modos , } quibus ostendi poterit , \\\hline
1.2.12 & de ser iustos et de guardar iustiçia . \textbf{ Mas esta uerdat podemos prouar en quatro maneras } segunt quatro cosas & quod maxime decet Reges , \textbf{ et Principes esse iustos . Possumus autem hanc veritatem quadruplici via venari , } secundum quatuor quae tanguntur de Iustitia in 5 Ethicorum . Prima via sumitur ex parte personae regiae . Secunda ex parte ipsius Iustitiae . \\\hline
1.2.12 & Mas esta uerdat podemos prouar en quatro maneras \textbf{ segunt quatro cosas } que tanne el philosofo dela iustiçia en el quimo libro delas ethicas ¶ & ø \\\hline
1.2.12 & tanto el Rey o el prinçipe deue sobrepuiar la ley . \textbf{ Ca deue el prinçipe o el Rey ser de tan grant iustiçia } e de tan grant egualdat por que pueda enderesçar & tantum Rex siue Princeps debet superare legem . \textbf{ Debet | etiam Rex esse tantae Iustitiae , } et tantae aequitatis , \\\hline
1.2.12 & Ca deue el prinçipe o el Rey ser de tan grant iustiçia \textbf{ e de tan grant egualdat por que pueda enderesçar } e egualar las leyes . & etiam Rex esse tantae Iustitiae , \textbf{ et tantae aequitatis , } ut possit ipsas leges dirigere : \\\hline
1.2.12 & e delan derechura ningunan cosa non sera reglada por ella . \textbf{ Ca çierta cosa es } que por la regla se reglan & Si enim regula ab aequalitate deficiat , \textbf{ nihil regulatum erit : } cum omnia per regulam regulentur . \\\hline
1.2.12 & e se egualan todas las cosas . \textbf{ Et en essa misma gnisa } si los Reyes non son iustos & cum omnia per regulam regulentur . \textbf{ Sic si Reges sint iniusti , } disponunt regnum , \\\hline
1.2.12 & en orden a otro . \textbf{ Mas estonçe lanr̃a bondat resplandeçe mucho } quando se estiende alos otros . & ( ut dictum est ) hominem in ordine ad alium . Tunc autem maxime clarescit bonitas nostra , quando usque ad alios se extendit . Unde 5 Ethic’ dicitur , \textbf{ quod praeclarissima virtutum videtur esse Iustitia : } et neque Hesperus , \\\hline
1.2.12 & si conuiene alos Reyes \textbf{ e alos prinçipes de auer muy claras uirtudes } paresçe de parte dela iustiçia & Si \textbf{ ergo decet Reges et Principes habere clarissimas virtutes ex parte ipsius Iustitiae , quae est quaedam clarissima virtus , } probari potest , \\\hline
1.2.12 & paresçe de parte dela iustiçia \textbf{ que es muy clara uirtud } que se puede prouar & ø \\\hline
1.2.12 & e quando la su sçiençia se estiende alos otros . \textbf{ Et por ende dize el philosofo en el primers libro dela methaphisica } que señal manifiesta es de omne sabio & quando potest alios docere , \textbf{ et quando scientia sua ad alios se extendit . Ideo scribitur 1 Metaphys’ quod signum omnino scientis , } est posse docere . \\\hline
1.2.12 & mas manifiestas \textbf{ quanto cada vno es puesto en mayor prinçipado e en mayor dignidat } por que las sus obras se estienden amas estonçe paresçe meior cada vno quales . & Si ergo nobis exteriora magis nota sunt , \textbf{ quanto aliquis in maiori principatu constituitur ; } quia opera sua ad plura se extendunt , \\\hline
1.2.12 & que se deuen fazer en el regno . \textbf{ lo otro por quela iustiçia es muy clara uirtud . } Et otrosi savn & ø \\\hline
1.2.12 & iustiçia \textbf{ se toma de parte dela grant maliçia } que nasçe en el regno & et Principes ex magnitudine malitiae , \textbf{ quae ex iniustitia consurgit . } Nam ( ut dicitur 5 Ethicor’ ) \\\hline
1.2.12 & para guardar la iustiçia \textbf{ e para escusar la mi ustiçia e el mal quanto por la mengua dela su iustiçia se puede seguir mayor mal } Et puede venir mayor deño a muchos . & quanto ex eorum Iustitia potest \textbf{ consequi maius malum , } et potest inferri pluribus nocumentum . \\\hline
1.2.12 & e para escusar la mi ustiçia e el mal quanto por la mengua dela su iustiçia se puede seguir mayor mal \textbf{ Et puede venir mayor deño a muchos . } Mas avn conuiene mas de declarar commo los Reyes & consequi maius malum , \textbf{ et potest inferri pluribus nocumentum . } Esset autem ulterius declarandum , \\\hline
1.2.13 & que auemos de fazer ¶ \textbf{ pues que assi es commo los omes alguas vezes puedan } e les contezca de se auer derechamente & ø \\\hline
1.2.13 & que algunos remen algunas cosas \textbf{ que han de temer e alas uegadas temen alguas cosas } que non han de temer . & et non recte , \textbf{ oportet dare virtutem aliquam circa timores , } et audacias . Accidit enim aliquos timere timenda , \\\hline
1.2.13 & Por la qual cosa dize el philosofo \textbf{ en el primero libro dela grant phia moral } que si alguno fuere tan sin temor e tan osado & ut etiam Deum non timeant , quod non est Fortitudinis , \textbf{ sed insaniae . Propter quod 1 Magnorum moralium scribitur , quod si quis valde impauidus existat , } quod Deum non timeat , \\\hline
1.2.13 & que los otros periglos . \textbf{ Et ahun por que en los periglos delas batallas mas fuerte cosa es de repremer los temores } que de restenar las osadias . & ( quia difficiliora , \textbf{ et terribiliora sunt pericula bellica , } quam alia : \\\hline
1.2.13 & Otrosi por que en auiendo osadia non es tan fuerte \textbf{ nin tan graue cosa acometer la batalla } e la pellea commo sofrir & et terribiliora sunt pericula bellica , \textbf{ quam alia : } et etiam quia in periculis bellicis difficilius est reprimere timores , \\\hline
1.2.13 & por que ymaginamos \textbf{ que tales periglos fazen grant dolor ¶ } Lo segundo sofrimos & et imaginamur \textbf{ ea maxime dolorem inferre , | difficiliora sunt ad sustinendum , } quam alia . Secundo , \\\hline
1.2.13 & por que son mas manifiestos \textbf{ nin por que fazen mayor dolor mas ahun } por que ymaginamos & non solum , \textbf{ quia sunt magis manifesta , et doloris illatiua , } sed quia imaginamur \\\hline
1.2.13 & Et pues que assi es commo sea \textbf{ mas guaue cosa de endurar } e de sufrir aquellos periglos & sicut pericula belli . \textbf{ Cum ergo difficilius sit durare , et sustinere pericula illa quae per fugam vitare possumus , } quam ad quae sustinenda necessitamur , \\\hline
1.2.13 & por que por tal foyr non los podemos escusar . \textbf{ Mas guaue cosa es de sofrir los periglos delas batallas } que los otros ¶lo terçero & eo quod per fugam ea vitare non possumus ; \textbf{ difficilius sustinentur pericula bellica , } quam alia . \\\hline
1.2.13 & que prinçipalmente es dicho fuerte \textbf{ aquel que non es temeroso cerca la buena muerte } Et mayormente cerca aquella muerte & quod principaliter dicetur utique fortis , \textbf{ qui impauidus est circa bonam mortem , } et maxime circa eam quae est \\\hline
1.2.13 & que non es temeroso . \textbf{ Et por ende al fuerte parte nesçe non temer } qual si quier periglo & ø \\\hline
1.2.13 & que refrenan las osadias . \textbf{ Ca los periglos espantables en la mayor parte son tristes } Et naturalmente cada vno fuye dela tristoza & quam circa moderationem audaciarum . \textbf{ Nam pericula terribilia } ut plurimum tristia sunt . Tristia autem naturaliter quilibet fugit , sicut naturaliter sequitur delectabilia . \\\hline
1.2.13 & assi commo naturalmente sigue las cosas delectables . \textbf{ Et pues que assi es commo nos natural mente fuyamos dela tristeza } graue cosa es de repmir los temores & ut plurimum tristia sunt . Tristia autem naturaliter quilibet fugit , sicut naturaliter sequitur delectabilia . \textbf{ Cum ergo naturaliter tristia fugiamus , } difficile est reprimere timores , \\\hline
1.2.13 & Et pues que assi es commo nos natural mente fuyamos dela tristeza \textbf{ graue cosa es de repmir los temores } por los quales fuyamos dela tristeza . & Cum ergo naturaliter tristia fugiamus , \textbf{ difficile est reprimere timores , } per quos tristia fugimus . \\\hline
1.2.13 & assi commo amas fuertes . \textbf{ Et por ende mas guaue cosa es de esforçarse contra los mas fuertes } que contra los mas flacos . & sicut ad fortiores : \textbf{ Difficilius est autem inniti contra fortiores , } quam contra debiliores . \\\hline
1.2.13 & e assi conmo presente . \textbf{ Et por ende mas guaue cosa es de esforçar se el omne } e auer se fuertemente contra los males presentes & et ut praesens . \textbf{ Difficilius autem est inniti , } et habere se fortiter contra mala praesentia , \\\hline
1.2.13 & que contra los males que han de venir \textbf{ ¶ lo terçero esto es mas guaue cosa } por que acometer puede se fazer adesora & quam contra mala futura . \textbf{ Tertio hoc est difficilius , } quia aggredi potest fieri subito : \\\hline
1.2.13 & mas sofrir requiere mas luengotron . \textbf{ Et por ende mas guaue cosa es auerse ome fuertemente } e firmemente en sufriendo las batallas & quia aggredi potest fieri subito : \textbf{ sed sustinere requirit diuturnitatem , et tempus . Difficilius est autem habere se fortiter , et constanter in sustinendo bella , } quod requirit durabilitatem et tempus , \\\hline
1.2.13 & que en acometiendo \textbf{ lo que se puede fazera desora . } Onde el philosofo en el terçero libro de las ethicas & quam in aggrediendo , \textbf{ quod subito fieri potest . } Unde Philosophus 3 Ethicorum cap’ de Fortitudine ait , \\\hline
1.2.13 & e la otra fallesçe e mengua \textbf{ assi commo la largueza es contraria alga stamiento } que sobrepuia en espendiendo . & et altera deficit : \textbf{ ut largitas opponitur prodigalitati , } quae superabundat in expendendo : \\\hline
1.2.13 & assi commo \textbf{ mas guaue cosa es de repremir los temores } que refrenar las osadias . & quam alii . \textbf{ Sed quia difficilius est reprimere timores , } quam moderare audacias : \\\hline
1.2.14 & La quarta os fortaleza rrauiosa \textbf{ e de grant sanya } ¶Laquina es fortaleza acostunsda¶ & Secunda seruilis . Tertia , militaris . Quarta , \textbf{ furiosa . Quinta , } consuetudinalis . Sexta , bestialis . Et septima , \\\hline
1.2.14 & Lisertima es fortaleza uirtuosa e de uirtud \textbf{ ¶La fortaleza çeuiles } quando alguno temiendo uerguença & consuetudinalis . Sexta , bestialis . Et septima , \textbf{ est Fortitudo virtuosa . Fortitudo enim ciuilis est , } quando aliquis timens verecundiam , \\\hline
1.2.14 & Et por ende temiendo qual denostaria su contrario era fuerte . \textbf{ Et ahun pone otro enxienplo el philosofo en esse mismo logar } e dize & qui erat ex parte aduersa , \textbf{ primum sibi increpationes imponeret . Sic etiam ( ut idem Philosophus recitat ) Diomedes , } hoc modo fortis erat . \\\hline
1.2.14 & que algunos quando son entrr̃as agenas \textbf{ do non son conosçidos acometen alguas torpedades } las quales non quarrian acometer nin tentar entre los sus çibdadanos en ningunan manera & et quaerit honores . Videmus enim aliquos , \textbf{ quum sunt in ignotis partibus , committere aliqua turpia , quae inter ciues et notos nullatenus attentarent . } Secunda Fortitudo dicitur seruilis , \\\hline
1.2.14 & que han en las armas tornan se a fuyr¶ \textbf{ La quarta fortaleza es rauiosa et de grant saña . } Ca alguons por la sanna & in fugam conuertuntur . \textbf{ Quarto Fortitudo dicitur furiosa . } Nam aliqui propter furorem , \\\hline
1.2.14 & Ca quando esta sin temor \textbf{ por razon dela sana passada la sanna en comienca de temer } Et assi non sufre la batalla mas encomienca luego de fuyr¶ La quinta fortaleza es acostunbrada e ganada por costunbre . & quia qui fit increpidus propter furorem , \textbf{ tranquillato furore , | incipit timere , } et non sustinet bellum , \\\hline
1.2.14 & Ca alguons \textbf{ por que se acaesçieron en muchans batallas } enlas quales por auentura ouieron buean dicha & sed arripit fugam . Quinta Fortitudo est consuetudinalis . Aliqui enim , \textbf{ quia in pluribus bellis fuerunt , } in quibus forte bene accidit eis , \\\hline
1.2.15 & euedes saber \textbf{ que la tenprança entre las quatro uirtudes cardinales tiene el postrimer grado . } Calapdençia e la iustiçia son mas prinçipales & quod ex tali bello consequi possit . \textbf{ Temperantia inter virtutes cardinales ultimum gradum tenet . } Prudentia enim et Iustitia principaliores sunt virtutibus moralibus ; \\\hline
1.2.15 & en los quales se muestra la fortaleza . \textbf{ Etrosi en essa misma manera entre las o triscosas } que nos inclinan a aquello que vieda la razon & circa quae est Fortitudo . Sic etiam \textbf{ inter caetera allicentia nos ad id quod ratio vetat , sunt venerea , } et delectabilia \\\hline
1.2.15 & finca . nos de dezir dela tenperança \textbf{ que entre las uirtudes prinçipales tiene el postrimero grado . } Et pues que assi es deuedes saber & quae est principalior Temperantia . Restat dicere de ipsa Temperantia , quae \textbf{ inter virtutes principales } ultimum gradum tenet . Sciendum ergo , \\\hline
1.2.15 & por dos maneras o por dos razones \textbf{ ¶La primera es } que quanto mas nos ayuntamos a las cosas delectables & et odoratum , \textbf{ quod dupliciter patet . Primo , } quia quantomagis unimur delectationibus , \\\hline
1.2.15 & tanto mas desseosamente \textbf{ e con mayor feruor nos delectamos enellas . } Et por que mas nos ayuntamos alas cosas delectables del gusto e del tan nimiento & tanto ardentius , \textbf{ et feruentius delectamur . Magis autem unimur delectationibus gustus , et tactus , quam delectationibus aliorum sensuum . Possumus enim videre , audire , et odorare distantia : } sed non possumus gustare , \\\hline
1.2.15 & por ellos nos delectamos mas desseosamente \textbf{ e con mayor feruor ¶ } La segunda razon & Ideo in huiusmodi sensibilibus arditius , \textbf{ et feruentius delectamur . } Secundo hoc idem patet : \\\hline
1.2.15 & Et las cosas que delectan el tannimiento \textbf{ assi commo los matrimo ion sson ordenados aguarda } e a continuaçion del humanal linage . & sed delectabilia tactus , \textbf{ ut matrimonia , } ordinantur ad conseruationem speciei . Ideo forte natura tantam delectationem posuit in delectationibus gustus , \\\hline
1.2.15 & assi commo los matrimo ion sson ordenados aguarda \textbf{ e a continuaçion del humanal linage . } Et por ende por auentura la natura puso tanta delectaconn en las cosas & sed delectabilia tactus , \textbf{ ut matrimonia , } ordinantur ad conseruationem speciei . Ideo forte natura tantam delectationem posuit in delectationibus gustus , \\\hline
1.2.15 & por que non ꝑesçiesse la ꝑsona \textbf{ e por que fuesse saluado el humanal linage . } Por la qual razon & ordinantur ad conseruationem speciei . Ideo forte natura tantam delectationem posuit in delectationibus gustus , \textbf{ et tactus , } ne periret indiuiduum , \\\hline
1.2.15 & prinçipalmente es de poner la tenpranca çerca aquellas delectaçonnes \textbf{ delas quales nos arredramos con mayor g̃ueza } pues que assi es la tenpranca & et \textbf{ ut saluaretur species . Quare si virtus est circa bonum et delectabile , ponenda est principaliter Temperantia circa delectationes illas , } a quibus est difficilius abstinere . \\\hline
1.2.15 & en las quales partiçipan todas las aialias . \textbf{ Mas laso trisa inalias se delectan engostar } e en el taner por si . & circa delectabilia illa , \textbf{ in quibus reliqua animalia communicant . } Reliqua autem animalia in gustu , \\\hline
1.2.15 & magozan se con la fartraa dellas . \textbf{ Et si alguas vezes se gozan con el olor delas liebres } esto es cuydando & sed cibatione . \textbf{ Si autem gaudent odore eorum , } hoc est , \\\hline
1.2.15 & por que los omes en estas delecta conn \textbf{ es se delecta con mayor desseo . } la qual cosa paresçe & et gustus : \textbf{ quia in illis homines ardentius delectantur ; } quod ( ut videtur ) rationabiliter accidit . \\\hline
1.2.15 & la qual cosa paresçe \textbf{ e acaesçe con grant razon } por que las delectaconnes delas viandas & quia in illis homines ardentius delectantur ; \textbf{ quod ( ut videtur ) rationabiliter accidit . } Nam delectationes nutrimentales quae fiunt per gustum , ordinantur ad conseruationem propriae personae : \\\hline
1.2.15 & a engendramiento de los fijos \textbf{ e a continua çonn de la genera conn humanal . } ¶ Pues que assi es & filiorum , \textbf{ et ad conseruationem speciei . } Si ergo \\\hline
1.2.15 & Pues que assi es \textbf{ si contra la mas guaue cosa deuemos } mas lidiarmas prinçipalmente deuemos trabaiar & sicut in matrimonio , quod ordinatur ad bonum alterius , \textbf{ et ad conseruationem speciei . Si ergo contra difficilius magis bellandum est , } principalius insistendum est , \\\hline
1.2.15 & que era muy goloso de puchas \textbf{ e tomaua muy grant delectaçion en ellas } g̃rago a dios quel feziese la garganta & cum esset pultiuorax , \textbf{ orauit , } ut guttur eius longius quam gruis fieret . \\\hline
1.2.15 & mas auemos aser osados e temerosos . \textbf{ assi en essa misma manera la tenpranca } mas conuiene con el non sentimiento & quam timidi : \textbf{ sic Temperantia } plus conuenit \\\hline
1.2.15 & que sean desagnisadas \textbf{ Et por ende declaradas son aquellas quatro cosas que propusiemos en el comienço de este capitulo } ¶ & ut a delectationibus sensibilibus caueamus . Melius est enim aliquas delectationes etiam licitas vitare , \textbf{ quam aliquas illicitas insequi . Declarata igitur sunt illa quatuor , } quae in principio Capituli proponebantur . Primo enim ostendebatur , \\\hline
1.2.15 & lo que fezieron los bieios de troya contra elena \textbf{ que era vna fermosa muger } e dixieron echemos la de nos que quieredezir tanto commo non la catemos nin la veamos . & secundum Philosophum Ethicorum \textbf{ 2 hoc pati , quod senes Troiae patiebantur , ad Helenam dicentes : } Abiiciamus eam , id est , \\\hline
1.2.16 & rueua el philosofo en el terçero libro delas ethicas \textbf{ por quatro razones } que mas de deno star es el omne & non respiciamus in ipsam . \textbf{ Probat Philosophus 3 Ethic’ quatuor rationibus , } detestabilius esse intemperatum , \\\hline
1.2.16 & e ganar tenpranca \textbf{ que el temeroso fortaleza . } Et assi en dos maneras se puede prouar & ø \\\hline
1.2.16 & que el temeroso la fortaleza \textbf{ Ca assi commo mostramos en el capitulo sobredicho tenpranca } ganamos retrayendonos & quam fortitudo . \textbf{ Nam | ( ut in praecedenti capitulo dicebatur ) temperantiam acquirimus , } abstinendo , \\\hline
1.2.16 & por que aquella uirtud dela fortaleza es mas guaue \textbf{ e se gana con mayor periglo . } Otrosi mas ligeramente se puede ganar la tenpranca & quia virtus illa est difficilis , \textbf{ et cum maiori periculo acquiritur . } Rursus facilius acquiri potest temperantia , \\\hline
1.2.16 & assi acostunbrar alasobras dela fortaleza \textbf{ commo ala sobrde tenpranca } por la qual cosa & Non ergo sic possumus assuefieri ad opera fortitudinis , \textbf{ sicut ad opera temperantiae : } quare exprobrabilius est nos esse intemperatos , \\\hline
1.2.16 & si fuer deste prado e segnidor de passiones . \textbf{ Enpero podemos aduzir nueuas razones } para prouar & et insecutorem passionum . Possumus \textbf{ tamen nouas rationes adducere , ostendentes , } quam detestabile sit , Regem intemperatum esse . Tangit enim Philosophus 3 Ethic’ \\\hline
1.2.16 & por que assi commo es dicho de suso la tenpranca ha de ser cora las cosas delectables del tannemiento e del gusto \textbf{ segunt las quales cosas es cosa comun anos } e alas bestias de nos delectar . & et intemperantia fieri habent circa delectabilia tactus , \textbf{ et gustus , | secundum quae , } delectari commune est nobis , \\\hline
1.2.16 & non biuen por razon e por entendimiento \textbf{ mas por pasion e par delectaçion . Et por ende veemos } que los moços & quia usum rationis non habent , \textbf{ non viuunt ratione sed passione . Ideo maxime videmus eos sequi delectabilia , } et esse insecutores passionum . \\\hline
1.2.16 & e muy digna de honrra . \textbf{ Much̃o desconuenible cosa es } que el Rey sea destenprado & et honore dignam , \textbf{ maxime indecens est eam esse intemperatam . Exemplum autem huius habemus in Rege Sardanapallo , } qui cum esset totus muliebris , et deditus intemperantiae ( ut recitat Iustinus Historicus , \\\hline
1.2.16 & non salia fuera dela camara \textbf{ a auer fabla con los Ricos omes } e cauałłos de su regno . & non exibat extra , \textbf{ ut haberet colloquia cum baronibus regni sui ; } sed omnes collocutiones eius erant in cameris ad mulieres : \\\hline
1.2.16 & mas todas sus fablas eran en las camaras con las mugieres \textbf{ e por escͥptos enbiaua todas sus razones alos ricos omes e alos prinçipe ᷤ en que les mandaua } lo que auian de fazer & sed omnes collocutiones eius erant in cameris ad mulieres : \textbf{ et per literas mittebat Baronibus } et Ducibus , \\\hline
1.2.16 & que vn prinçipe mucho su priuado \textbf{ que grant t p̃o le auia seruido e fiel mente . } Et el Rey que tiendo fazer plazer & quid vellet eos facere . Accidit autem , \textbf{ quod , } cum quidam Dux exercitus diu ei seruiuisset , et fideliter , Rex ille volens complacere illi Duci , \\\hline
1.2.17 & por ende \textbf{ despues que dixiemos destas quatro uirtudes } e mostramos commo los Reyes & Fortitudo , et Temperantia . \textbf{ Postquam ergo diximus de his quatuor virtutibus , } et ostendimus quomodo Reges et Principes illis virtutibus decet esse ornatos . \\\hline
1.2.17 & fincanos \textbf{ de dezir delas otras och̃o uirtudes . } Onde conuiene saber & et ostendimus quomodo Reges et Principes illis virtutibus decet esse ornatos . \textbf{ Reliquum est pertransire ad virtutes alias . } Virtutes autem aliae vel respiciunt exteriora bona , \\\hline
1.2.17 & Mas la magnificençia es tal uirtud \textbf{ que cata alas grandes despenssas . } Et esto en qual manera se deue entender adelante lo mostrͣemos¶ & ( quae alio modo largitas nuncupatur ) dicitur respicere sumptus mediocres : \textbf{ magnificentia vero dicitur respicere magnos sumptus ; } quod quomodo sit intelligendum , \\\hline
1.2.17 & e de lons dineros tres cosasson menester \textbf{ ¶La primera es que non tome el auer nin los dineros } donde non los deue tomar & Ad rectum autem usum pecuniae tria requiruntur . \textbf{ Primo quod non accipiat eam unde non debet . Secundo quod accipiat unde debet . } Tertio quod expendat \\\hline
1.2.17 & en que ha de ser la franqueza \textbf{ mas non ha de ser çerca estas tres cosas egualmente e prinçipal mente . } Ca primeramente en espender & sunt illa tria circa quae videtur esse liberalitas . \textbf{ Non autem est circa haec tria aeque principaliter | et primo . } Nam circa expendere et circa debitos sumptus facere , \\\hline
1.2.17 & Et por ende la franqueza es en non tomar on de non deue tomar . \textbf{ Enꝑnon es cerca esto prinçipal mente . } Ca los que toman onde non deuen & circa non accipere unde non debet : \textbf{ non tamen est circa haec principaliter . Nam accipientes unde non debent , } magis sunt iniusti , quam illiberales , \\\hline
1.2.17 & Lo segundo esto mismo se praeua \textbf{ assi por que ala uirtud mas prinçipal parte nesçe de fazer mayor bien . } Et mayor bien es en bien fazer & Secundo hoc idem patet , \textbf{ quia ad virtutem principalius spectat facere maius bonum . Maius autem bonum est benefacere , } quam bene pati , \\\hline
1.2.17 & Et pues que assi es \textbf{ si meior cosa es bien fazer } que non mal fazer & non operatur turpia . \textbf{ Si ergo melius est benefacere , } quam non malefacere , \\\hline
1.2.17 & o que bien sofrir \textbf{ meior cosa es bien espender } que non tomar las cosas agenas & vel quam bene pati : \textbf{ melius est bene expendere , } quam aliena non surripere vel quam propria custodire . \\\hline
1.2.17 & por que en aquello esta \textbf{ mas la uirtud delo qual se leunata mayor loor e mayor honrra . } Et mayor loor e mayor honrra se le unata & Tertio hoc idem patet : \textbf{ quia circa illud magis consistit virtus , circa quod consurgit maior laus . Maior autem laus consurgit in bene expendendo , } et aliis benefaciendo , quam in custodiendo propria , \\\hline
1.2.17 & mas la uirtud delo qual se leunata mayor loor e mayor honrra . \textbf{ Et mayor loor e mayor honrra se le unata } en bien espendiendo & Tertio hoc idem patet : \textbf{ quia circa illud magis consistit virtus , circa quod consurgit maior laus . Maior autem laus consurgit in bene expendendo , } et aliis benefaciendo , quam in custodiendo propria , \\\hline
1.2.17 & Ca cada hun omne es naturalmente inclinado a amar asi mismo \textbf{ e aguardar los sus biens propos } Mas dar los sus biens propios ha alguna guaueza por si . & ut se diligat , \textbf{ et ut sua bona custodiat . Dare autem propria bona , } secundum se difficultatem habet : \\\hline
1.2.17 & a nos bien commo es cosa guauedar lo propio \textbf{ a que auemos mayor amor } por que nos es çercano & ad quae non ita afficimur , \textbf{ quia non sunt nobis coniuncta : } sicut est difficile dare propria , \\\hline
1.2.17 & Mas es muy amado \textbf{ mas si faze conuenibles espensas de lo suyo propio } e parte e da de lo suyo grandes dones alos buenos & Sed maxime diligitur \textbf{ si ex propriis redditibus debitos sumptus faciat , } et bonis \\\hline
1.2.17 & mas si faze conuenibles espensas de lo suyo propio \textbf{ e parte e da de lo suyo grandes dones alos buenos } e a aquellos que le meresçen . & si ex propriis redditibus debitos sumptus faciat , \textbf{ et bonis } et dignis magna dona tribuat . Quare in bene expendendo , \\\hline
1.2.17 & assi que seamos mas osados que temerosos . \textbf{ Et en essa misma guisa la franqueza } mas contradize ala auariçia & ø \\\hline
1.2.18 & que el que menos da sea mas franco algunas vezes \textbf{ si delas sus propias possesiones tomare e diere . } Ca la franqueza resçibe & nihil prohibet minora dantem , \textbf{ vel liberaliorem esse , } si a minoribus recipiat . \\\hline
1.2.18 & ¶La primera razon por que es esta \textbf{ Ca meior cosa es enfermar el omne de enfermedat } de que puede guaresçer & quod si possent esse prodigi , melius esset eos esse prodigos quam auaros . Primo enim , \textbf{ quia melius est infirmari } morbo curabili , quam incurabili . Prodigalitas autem morbus est curabilis , vel ab aetate , vel ab egestate . \\\hline
1.2.18 & que quando eran mançebos \textbf{ e assi los mas dellos guaresçendesta enfermedat } quando vienen ala vegez & ut plurimum curatur eorum prodigalitas , \textbf{ et desinunt esse prodigi . } Curari \\\hline
1.2.18 & que la auariçia . \textbf{ Ca el franco non resçibe de buenamente mas de buena mente El gastador ha estas dos cosas } e el auariento non ha ninguno dellas . & quam auaritia : \textbf{ nam liberalis non libenter recipit , | et libenter dat : } quorum utrunque facit prodigus : \\\hline
1.2.18 & que non ser ricos . \textbf{ Et en essa misma manera dan } non por la razon & quia magis deceret eos esse pauperes , quam diuites . Sic etiam dant \textbf{ cuius gratia non oportet . Non enim dant boni gratia , } sed magis dant ut laudentur , \\\hline
1.2.18 & Ca non lo dan por razon de bien \textbf{ mas dan lo por vana eglesia } o por que sean loados & sed magis dant ut laudentur , \textbf{ et propter inanem gloriam } vel propter aliquam aliam causam . \\\hline
1.2.19 & que es largueza e franqueza ¶ \textbf{ La otra cata alas grandes espenssas } la qual laman magnifiçençia & quam liberalitatem vocant . \textbf{ Aliam , | quae respicit sumptus magnos , } quam magnificentiam nominant . \\\hline
1.2.19 & que non suficientemente se departe la magnificençia dela liberalidat \textbf{ por aquesto que la magnifiçençia es cerca delas grandes espenssas . } Et la liberalidat çerca las medianas e mesuradas ¶ & non videtur sufficienter distingui magnificentia a liberalitate per hoc , \textbf{ quia haec est circa magnos sumptus , } illa vero circa mediocres . \\\hline
1.2.19 & e guaue de fazer . \textbf{ por la qual cosa commo en las mayores espenssas sea fallada . } mas espeçial razon de bondat e de guaueza & quod sit circa bonum , \textbf{ et difficile . quare cum in maioribus sumptibus reperiatur specialis ratio bonitatis et difficultatis , } quae non reperitur in mediocribus sumptibus , \\\hline
1.2.19 & por la qual cosa commo en las mayores espenssas sea fallada . \textbf{ mas espeçial razon de bondat e de guaueza } que en las espenssas medianas e mesuradas . & quod sit circa bonum , \textbf{ et difficile . quare cum in maioribus sumptibus reperiatur specialis ratio bonitatis et difficultatis , } quae non reperitur in mediocribus sumptibus , \\\hline
1.2.19 & e apartada dela liberalidat \textbf{ que es çerca delas medianas mesuradas espenssas . } E podemos dezir de otra guisa & esse virtutem aliam a liberalitate , \textbf{ quae est circa mediocres . } Vel possumus dicere , \\\hline
1.2.19 & que la liberalidat non ha de ser en muchedunbre de donadios ¶ \textbf{ Pues que assi es el liberal egual a los dones } e las passiones alas sus riquezas . & Dicebatur enim supra non esse liberalitatem in multitudine datorum . \textbf{ Liberalis ergo adaequat dationes , } et sumptus facultatibus . \\\hline
1.2.19 & Ende el magnifico nonbre toma de aquello que faze es nonbrada la magnificençia \textbf{ ca es dichon magnifico aquel que faze grandes cosas } Et por ende & et ab ipso opere denominatur magnificentia . Dicitur enim magnificus , \textbf{ quasi magna faciens . Inde est ergo , } quod quia in quocunque statu homo sit , \\\hline
1.2.19 & que ome aya \textbf{ si quier muchas si quier medianas guaue cosa es bien husar dellas } e bien egualar las espenssas alas riquezas . & siue multas , \textbf{ siue mediocres : | difficile est bene } uti illis et bene aequare sumptus facultatibus . Ideo liberalitas se extendit ad mediocres sumptus . Sumptus enim , \\\hline
1.2.19 & e son conuenibles alos pobres . \textbf{ Assi las grandes espenssas son mesuradas } e conuenibles alos ricos & Nam sicut parui sumptus sunt mediocres \textbf{ et proportionati pauperibus : sic magni , sunt mediocres , et proportionati diuitibus . } Cum ergo liberalitas non respiciat sumptus \\\hline
1.2.19 & por ende la libalidat non es . \textbf{ dicha ser çerca las grandes espenssas e medianas mas estan solamente çerca las medianas ¶ } Pues que assi es la liberalidat & sed ut proportionantur facultatibus , \textbf{ non dicitur esse circa magnos et mediocres sumptus , | sed circa mediocres tantum . } Liberalitas igitur , \\\hline
1.2.19 & Ca ahun que las riquezas sean medianas \textbf{ guaue cosa es dese auer bien los omes en ellas . } Mas la magnificençia & se extendit ad quascunque facultates : \textbf{ quia etiam in mediocribus facultatibus difficile est bene se habere in illis . Magnificentia vero , } respiciens sumptus \\\hline
1.2.19 & nin tiene oio quales sean las obras . \textbf{ Ca non es guaue cosa de fazer conuenibles espenssas } en quales se quier obras . & non respicit quaecunque opera : \textbf{ quia non est difficile facere decentes sumptus in quibuscunque operibus , } sed in magnis . Ideo magnificentia a magnis operibus sumpsit nomen . In magnis autem operibus contingit aliquos deficere , \\\hline
1.2.19 & en quales se quier obras . \textbf{ mas es guaue cosa en fazerconuenibles espenssasen las grandesobras . } Et por ende la magnificençia toma nonbre de grandes fechos e de grandesobras . & quia non est difficile facere decentes sumptus in quibuscunque operibus , \textbf{ sed in magnis . Ideo magnificentia a magnis operibus sumpsit nomen . In magnis autem operibus contingit aliquos deficere , } quia non intendunt quomodo magna opera faciant , \\\hline
1.2.19 & mas es guaue cosa en fazerconuenibles espenssasen las grandesobras . \textbf{ Et por ende la magnificençia toma nonbre de grandes fechos e de grandesobras . } Mas deuedes saber & quia non est difficile facere decentes sumptus in quibuscunque operibus , \textbf{ sed in magnis . Ideo magnificentia a magnis operibus sumpsit nomen . In magnis autem operibus contingit aliquos deficere , } quia non intendunt quomodo magna opera faciant , \\\hline
1.2.19 & Mas deuedes saber \textbf{ que enlas grandes obras contesçe a algimos de fallesçer . } Ca non entienden commo han de fazer las grandes obras & sed in magnis . Ideo magnificentia a magnis operibus sumpsit nomen . In magnis autem operibus contingit aliquos deficere , \textbf{ quia non intendunt quomodo magna opera faciant , } sed quomodo parum expendant . \\\hline
1.2.19 & que enlas grandes obras contesçe a algimos de fallesçer . \textbf{ Ca non entienden commo han de fazer las grandes obras } mas tienen mientes a poco espender . & sed in magnis . Ideo magnificentia a magnis operibus sumpsit nomen . In magnis autem operibus contingit aliquos deficere , \textbf{ quia non intendunt quomodo magna opera faciant , } sed quomodo parum expendant . \\\hline
1.2.19 & Mas ay otros ahun \textbf{ que en las grandes obras espienden } mas de quanto deuen & Et tales vocantur paruifici . \textbf{ Aliqui vero etiam in magnis operibus plus expendunt quam opera illa requirant , } et tales vocantur consumptores . \\\hline
1.2.19 & Mas ay otros \textbf{ que en las grandes obras fazen conuenibles espenssas } e tales son uirtuosos & et fornaces , \textbf{ quia tales sicut fornax omnia consumunt . Quidam vero in magnis operibus faciunt decentes sumptus : } et tales sunt virtuosi , \\\hline
1.2.19 & Et pues que assi es la magnificençia es medianera entre la paruifiçençia \textbf{ e los grandes gastamientos } assi commo la liberalidat es medianera entre la auariçia e el gastamiento . & et vocantur magnifici . Erit igitur magnificentia media inter paruificentiam , \textbf{ et consumptionem : } sicut liberalitas est media \\\hline
1.2.19 & et tienpra los gastamientos \textbf{ assi la magnificençia repreme las paruifiçençias e las pequenas espenssas } e tienpra los consumimientos e destruymientos . & et moderans pro digalitates : \textbf{ sic magnificentia est reprimens paruificentias , } et moderans consumptiones . \\\hline
1.2.19 & e da dones conuenibles a las riquezas \textbf{ assi la magnificençia faze espenssas conuenibles alas grandes obras ¶ } visto que cosa es la magnificençia fincanos de veer & et dationes proportionatas facultatibus : \textbf{ sic magnificentia est faciens sumptus decentes magnis operibus . } Viso \\\hline
1.2.19 & Lo quarto assi mismo . \textbf{ Ca el magnefico se deue auer conueniblemente çerca estas quatro cosas . } Mas enpero non deue entender egualmente nin prinçipalmente cerca estas quatro cosas . & Nam principaliter et primo , \textbf{ homo debet esse magnificus circa diuina , constituendo | ( si facultates tribuant ) } templa magnifica , \\\hline
1.2.19 & Ca el magnefico se deue auer conueniblemente çerca estas quatro cosas . \textbf{ Mas enpero non deue entender egualmente nin prinçipalmente cerca estas quatro cosas . } Ca primero e prinçipalmente deue seer el omne magnifico çerca las cosas dauinołs & ( si facultates tribuant ) \textbf{ templa magnifica , } sacrificia honorabilia , \\\hline
1.2.19 & Ca primero e prinçipalmente deue seer el omne magnifico çerca las cosas dauinołs \textbf{ Et si cunplieren las sus riquezas deue fazer grandes eglesias e sac̀fiçios honrrados et apareiamientos dignos } assi commo uestimentas e calices e otros apareiamientostales . & templa magnifica , \textbf{ sacrificia honorabilia , } praeparationes dignas . Ideo dicitur 4 Ethicorum , \\\hline
1.2.19 & en el quarto libro delas ethicas \textbf{ que el magnifico deue fazer honrradas espenssas en aquellas cosas } que parte nesçen a dios¶ & praeparationes dignas . Ideo dicitur 4 Ethicorum , \textbf{ quod honorabiles sumptus , | quos debet facere magnificus , } sunt circa Deum . \\\hline
1.2.19 & Ca en esto paresçe mayormente la magnificençia \textbf{ quando alguno faze grandes bienes } a aquellos que son mas dignos ¶ & Nam in hoc potissime apparet magnificentia , \textbf{ quando quis magna facit iis } qui sunt magis digni . \\\hline
1.2.19 & Lo quarto el magnifico se deue auer conueinblemente cerca la su perssona propia \textbf{ por que deue cada vno granadamenᷤte se auer cerca las grandes obras en conparaçion dela su perssona } mas las grandes obras pueden ser dichas & Quarto debet se decenter habere magnificus circa personam propriam : \textbf{ debet enim quis magnifice se habere circa magna opera respectu personae propriae . } Magna autem opera possunt dici illa , \\\hline
1.2.19 & por que deue cada vno granadamenᷤte se auer cerca las grandes obras en conparaçion dela su perssona \textbf{ mas las grandes obras pueden ser dichas } aquellas que duran por toda la uida del omne . & debet enim quis magnifice se habere circa magna opera respectu personae propriae . \textbf{ Magna autem opera possunt dici illa , } vel quae durant \\\hline
1.2.19 & aquellas que duran por toda la uida del omne . \textbf{ assi commo son grandes casas e grandes fortalezas o a qual las que se fazen pocas vezes en toda la uida del omne } assi commo son los casamientos e las caualłias . & per totam vitam , \textbf{ cuiusmodi sunt domus , | et aedificia . } Vel quae fiunt raro in tota vita , \\\hline
1.2.19 & nin tan firmes en si . \textbf{ En essa misma manera conuiene al magnifico de fazer muy honrradamente } las sus bodas e las sus cauallerias & et apparentes . \textbf{ Sic etiam decet magnificum , nuptias , } et militias , \\\hline
1.2.19 & Ca ha de ser çerca espenssas conuenibles \textbf{ en la grandes obras . } Empero prinçipalmente et primero es cerca las grandes espenssas & Patet ergo circa quid est magnificentia : \textbf{ quia est circa sumptus decentes magnis operibus . } Principaliter tamen est circa magnos sumptus impensos erga diuina , \\\hline
1.2.19 & en la grandes obras . \textbf{ Empero prinçipalmente et primero es cerca las grandes espenssas } pues tas en las colas que parte nesçen a dios & quia est circa sumptus decentes magnis operibus . \textbf{ Principaliter tamen est circa magnos sumptus impensos erga diuina , } et erga totam communitatem . Ex consequenti vero est erga magnos sumptus impensos circa personas dignas , et circa seipsum . Ostenso quid est magnificentia , \\\hline
1.2.19 & Assi la magnificençia es mas contraria ala pariuuficençia \textbf{ que faze pequanas cosas } que al destruyemiento & quam prodigalitati : \textbf{ sic magnificentia plus contrariatur paruificentiae , } quam consumptioni . \\\hline
1.2.19 & que ala parui ficençia \textbf{ en tal manera que enlas grandes obras las espenssas } sobra pugen mayormente en despendiendo & declinando ad consumptionem , \textbf{ ut etiam in magnis operibus potius superabundent sumptus , } quam deficiant . \\\hline
1.2.20 & l philosofo en el quarto libro delas ethicas \textbf{ pone seys propiedades del parufico } que es omne menguado en despender & quam deficiant . \textbf{ Tangit autem Philosophus 4 Ethicor’ sex proprietates ipsius paruifici : } quae si inessent regibus , \\\hline
1.2.20 & por el nonbre mismo \textbf{ por que el parufico es ome que faze pequanas cosas e menguadas ¶ } La segunda propiendat es & quod ex ipso nomine patet . \textbf{ Esse enim paruificum , | est facere parua et defectiua . } Secunda proprietas est , \\\hline
1.2.20 & La segunda propiendat es \textbf{ que si contezta que el paruifico aya de despender grandes espenssas } pierde muy grant & Secunda proprietas est , \textbf{ quia si contingat paruificum , | et magna expendere pro paruo , } perdit magnum bonum . Unde et prouerbialiter dicitur , Viles homines nuptias , \\\hline
1.2.20 & pierde muy grant \textbf{ bien por pequana cosa . Onde dize el prouerbio } que los uiles omes pierden las bodas e el conbite & et magna expendere pro paruo , \textbf{ perdit magnum bonum . Unde et prouerbialiter dicitur , Viles homines nuptias , } et conuiuium perdunt pro denariato piperis . \\\hline
1.2.20 & bien por pequana cosa . Onde dize el prouerbio \textbf{ que los uiles omes pierden las bodas e el conbite } por vna dinarada de pimienta . & perdit magnum bonum . Unde et prouerbialiter dicitur , Viles homines nuptias , \textbf{ et conuiuium perdunt pro denariato piperis . } Dum enim circa magnum conuiuium non faciunt decentes sumptus , volentes parcere modicae expensae , totum conuiuium indecens redditur . Tertia proprietas est , \\\hline
1.2.20 & por vna dinarada de pimienta . \textbf{ Ca quando non fazen espenssas conuenibles en el grant conbite } esto non es & et conuiuium perdunt pro denariato piperis . \textbf{ Dum enim circa magnum conuiuium non faciunt decentes sumptus , volentes parcere modicae expensae , totum conuiuium indecens redditur . Tertia proprietas est , } quod quaecunque facit paruificus , \\\hline
1.2.20 & Bien alłi puesto que el paruifico \textbf{ e al escasso sea dado de fazer grandes espenssas } sienpre tarda e fuye & et subterfugit quantum potest : \textbf{ sic dato quod paruificum oporteat expensas facere , } tamen illos sumptus tardat , \\\hline
1.2.20 & que el paruifico non entiende \textbf{ en qual manera faga granada obra } e en qual manera faga sus dones granados e conuenibles & quantum potest . Quarta est , \textbf{ quia paruificus non intendit qualiter faciat magnum opus , } ut qualiter faciat debitas largitiones , \\\hline
1.2.20 & que la obra . \textbf{ por ende mayor acuçia pone en commo esperienda poco } que en commo faga grand obra¶ & ut appretietur \textbf{ plus pecunia quam opus . } Quinta proprietas eius est , \\\hline
1.2.20 & La sexta propiendat es que quando el parufico faze alguna cosa aparesçe ael \textbf{ que sienpre faze mayores cosas } que deue & quia cum paruificus nihil faciat , \textbf{ videtur tamen ei quod semper agat maiora , } quod debeat . Dictum est enim , \\\hline
1.2.20 & mas que deue dar . \textbf{ assi en essa misma manera non puede el parufico fazer despenssas tan pequanas } en qual si quier obra que faga & quam debeat : \textbf{ non potest paruificus ita modicum sumptum facere erga quodcunque opus , } quin semper videatur ei quod agat maiora , \\\hline
1.2.20 & en qual si quier obra que faga \textbf{ que non le paresca sienpre a el que faze mayores espenssas } que deua . & non potest paruificus ita modicum sumptum facere erga quodcunque opus , \textbf{ quin semper videatur ei quod agat maiora , } quam debeat . \\\hline
1.2.20 & que deua . \textbf{ Por la qual cosa si cosa muy denostada es en la real magestad fallesçer } e menguar en todas las cosas & quam debeat . \textbf{ Quare si detestabile est regiam maiestatem } circa omnia deficere , \\\hline
1.2.20 & e menguar en todas las cosas \textbf{ e perder muy grandes bienes } por muy pequana cosa . & circa omnia deficere , \textbf{ magna bona pro modico perdere , } semper tardare , \\\hline
1.2.20 & e perder muy grandes bienes \textbf{ por muy pequana cosa . } Et tardar sienpre en las cosas & circa omnia deficere , \textbf{ magna bona pro modico perdere , } semper tardare , \\\hline
1.2.20 & e nunca entender \textbf{ en commo faga grandes obras de uirtud } mas penssar & et nihil prompte facere , \textbf{ et nunquam intendere quomodo faciat magna opera virtutum , } sed quomodo modicum expendat , \\\hline
1.2.20 & Et quando non faze ningunan cosa cree el \textbf{ que faze grandescosas e grandes obras . } Et por que todas estas cosas ponen grand mengua en la Real magestad & sumptum cum tristitia et dolore ; et cum nihil facit , \textbf{ credere se magna operari , } quia omnia haec valde derogant regiae maiestati , \\\hline
1.2.20 & que faze grandescosas e grandes obras . \textbf{ Et por que todas estas cosas ponen grand mengua en la Real magestad } en todas maneras es de denostar & credere se magna operari , \textbf{ quia omnia haec valde derogant regiae maiestati , } omnino detestabile est Regem esse paruificum . \\\hline
1.2.20 & que el Rey sea periufico \textbf{ mas que conuengaal Rey de ser magnifico } e de fazer grandes espenssas conplidamente es prouado & omnino detestabile est Regem esse paruificum . \textbf{ Quod autem deceat ipsum esse magnificum , } sufficienter probant superiora dicta : \\\hline
1.2.20 & mas que conuengaal Rey de ser magnifico \textbf{ e de fazer grandes espenssas conplidamente es prouado } por lo que dicho es de suso & Quod autem deceat ipsum esse magnificum , \textbf{ sufficienter probant superiora dicta : } in quibus ostendimus circa quae magnificentia habet esse . Dicebatur enim magnificentiam principaliter esse circa opera diuina , \\\hline
1.2.20 & e leaꝑlona honrrada \textbf{ e de grand reuerençia e persona publica } e a el pertenesca de partir los bienes del regno mucho le conuiene a el de ser magnifico . & Cum ergo Rex sit caput regni , \textbf{ et sit persona honorabilis , reuerenda , et publica , } et ad ipsum pertineat distribuere bona regni , maxime decet ipsum esse magnificum . Nam quia est caput regni , \\\hline
1.2.20 & e cerca los apareiamientos diuinales \textbf{ e dela santa eglesia . } Mas por el que es persona publica e comuna & maxime spectat ad ipsum magnifice se habere circa templa sacra , \textbf{ et erga praeparationes diuinorum . } Quia vero est persona publica , \\\hline
1.2.20 & que son dignas e meresçen de auer aquellos bienes ¶ \textbf{ Otrosi por que assi commo dicho es . La persona del Rey deue ser de grand reuerençia } e digna de grand honrra parte nesçe mucho al Rey & omnino decet eum magnifice se habere erga personas dignas , \textbf{ quibus digne competunt illa bona . Amplius quia ( ut dictum est ) regia persona debet esse reuerenda et honore digna , } spectat ad Regem magnifice se habere erga personam propriam , \\\hline
1.2.20 & Otrosi por que assi commo dicho es . La persona del Rey deue ser de grand reuerençia \textbf{ e digna de grand honrra parte nesçe mucho al Rey } de se auer granadamente & quibus digne competunt illa bona . Amplius quia ( ut dictum est ) regia persona debet esse reuerenda et honore digna , \textbf{ spectat ad Regem magnifice se habere erga personam propriam , } et erga personas sibi coniunctas , \\\hline
1.2.21 & euedes saber \textbf{ que el philosofo enl quarto libro delas ethicas capitulo dela magnificençia } pone seys propiedadesdel magnifico & exercendo militias admirabiles . \textbf{ Philosophus 4 Ethicorum capitulo De magnificentia , } tangit sex proprietates magnifici , \\\hline
1.2.21 & que el philosofo enl quarto libro delas ethicas capitulo dela magnificençia \textbf{ pone seys propiedadesdel magnifico } las quales conuiene alos Reyes e alos prinçipes auer ¶ & Philosophus 4 Ethicorum capitulo De magnificentia , \textbf{ tangit sex proprietates magnifici , } quas habere decet Reges et Principes . Prima proprietas est , \\\hline
1.2.21 & Ca dixiemos de suso \textbf{ que conuenia al magnifico de fazer conuenientes espenssas en las grandes obras . } Mas conosçer en qual es grandes obras & quia magnificus assimilatur scienti . \textbf{ Dicebatur enim spectare ad magnificum in magnis operibus facere decentes sumptus . } Cognoscere autem quibus magnis operibus \\\hline
1.2.21 & que conuenia al magnifico de fazer conuenientes espenssas en las grandes obras . \textbf{ Mas conosçer en qual es grandes obras } quales espenssas son conuenibles . & Dicebatur enim spectare ad magnificum in magnis operibus facere decentes sumptus . \textbf{ Cognoscere autem quibus magnis operibus } qui sumptus sint conuenientes , \\\hline
1.2.21 & esto non puende ser \textbf{ si non en aquel que ouiere grand sabaduria e grand entendimiento¶ } La segunda propiedat del magnifico es fazer grandes espenssas & esse non potest , \textbf{ nisi quis polleat scientia , | et intellectu . } Secunda proprietas magnifici , est facere magnos sumptus , \\\hline
1.2.21 & si non en aquel que ouiere grand sabaduria e grand entendimiento¶ \textbf{ La segunda propiedat del magnifico es fazer grandes espenssas } non por que se muestre nin por vanagłia & et intellectu . \textbf{ Secunda proprietas magnifici , est facere magnos sumptus , } non ut ostendat seipsum , \\\hline
1.2.21 & espeçialmente esto es propio ala magnifiçençia . \textbf{ Ca commo en las grandes obras } sienpre deua tener omne nayentes a buena fin & sed gratia boni . Specialiter tamen hoc dicitur esse proprium magnificentiae . \textbf{ Nam cum in operibus magnificis , } ut cum aliquis magnifice se habet erga cultum diuinum , \\\hline
1.2.21 & Ca commo en las grandes obras \textbf{ sienpre deua tener omne nayentes a buena fin } assi commo quando alguno granadamente sea en la honrra de dios & Nam cum in operibus magnificis , \textbf{ ut cum aliquis magnifice se habet erga cultum diuinum , } et erga rempublicam , \\\hline
1.2.21 & e mayormente aquel es alabado entre los omnes por estas cosas . \textbf{ Enpero guaue cosa es en tales cosas } non demandar loor delas gentes & et maxime quis ab hominibus laudatur , \textbf{ difficile est in talibus non quaerere humanam laudem . } Et quia virtus est circa bonum \\\hline
1.2.21 & Por ende mucho parte nesçe al magnifico \textbf{ en las sus muy grandes obras } e en las sus parti connsenteder finalmente el bien & et difficile , \textbf{ ideo maxime spectat ad magnificum in suis magnificis operibus , et distributionibus intendere finaliter bonum , } et non fauorem , \\\hline
1.2.21 & e en las sus parti connsenteder finalmente el bien \textbf{ e non honrra nin vana eglesia de los omes } ¶ & ideo maxime spectat ad magnificum in suis magnificis operibus , et distributionibus intendere finaliter bonum , \textbf{ et non fauorem , | et gloriam hominum . } Tertia proprietas magnifici , est facere sumptus et distributiones delectabiliter \\\hline
1.2.21 & mas pariufico \textbf{ que quiere dezir de pequena fazienda . } Et por ende dize el philosofo . & non est magnificus , \textbf{ sed paruificus . } Ideo dicitur 4 Ethic’ \\\hline
1.2.21 & en el quarto libro delas etris \textbf{ que la grant auariçia de tomar cuenta faze al omne de poca fazienda ¶ } La quarta propiedat es & Ideo dicitur 4 Ethic’ \textbf{ quod diligentia ratiocinii est paruifica . Quarta , } est magis intendere qualiter faciat opus optimum , \\\hline
1.2.21 & es alguna eglesia en honrra de dios \textbf{ o ouiesse de dar e de partir alguons dones a personas dignis } mas deue entender & ut si debet magnificus aliquod templum construere in honorem diuinum , \textbf{ vel aliqua dona conferre personis dignis , } magis intendet quomodo templum illud sit admirabile \\\hline
1.2.21 & que el magnifico es conplidamente liberal \textbf{ por que essa misma cosa es ser magnifico } que ser conplidamente libal e franço . & quod magnificus est excellenter liberalis . \textbf{ Idem est enim esse magnificum , } quod esse abundanter liberalem . \\\hline
1.2.21 & Ca commo aquel sea magnifico \textbf{ que faze conuenibles espenssas enlas grandes obras } si faz conuenibles espenssas & Cum enim ille sit magnificus , \textbf{ qui in magnis operibus facit decentes sumptus : } si facere decentes sumptus est esse liberalem , \\\hline
1.2.21 & que faze conuenibles espenssas enlas grandes obras \textbf{ si faz conuenibles espenssas } faz omne ser liberal fazer muy grandes & qui in magnis operibus facit decentes sumptus : \textbf{ si facere decentes sumptus est esse liberalem , } facere maximos decentes sumptus , \\\hline
1.2.21 & faz omne ser liberal fazer muy grandes \textbf{ e muy conuenibles espenssas } lo que faze el magnifico es ser mucho mas liberal ¶ & si facere decentes sumptus est esse liberalem , \textbf{ facere maximos decentes sumptus , } quos facit magnificus , est esse maxime liberalem . Sexta proprietas magnifici , \\\hline
1.2.21 & que de egual es penssa faga obra mas granada \textbf{ que otro ninguno . Nas los paruficos e de pequana fazienda } por que sienpre entienden & quos facit magnificus , est esse maxime liberalem . Sexta proprietas magnifici , \textbf{ est aequali sumptu facere opus magis magnificum . Paruifici enim , } quia semper intendunt , \\\hline
1.2.21 & assi commo dich̃ones dessuso \textbf{ por muy pequana cosa pierden lo mucho . } Et pues que assi es el magnifico aqui parte nesçe de non auer cuydado de contar & ut supra dicebatur , \textbf{ pro modico multum perdunt . Magnificus ergo , } cui non est curae de ratiocinio ab aequali sumptu , \\\hline
1.2.21 & lo que despiende de despenssa \textbf{ egual faze mas granada obra } e mas mognifica & cui non est curae de ratiocinio ab aequali sumptu , \textbf{ facit opus magis magnificum , } quia non parcit decentibus sumptibus . \\\hline
1.2.21 & avn que sean pequeñas \textbf{ e de pequans valor . } Por ende la obra & si modice sint , \textbf{ et parui valoris , ideo opus , } ubi multum expendit , indecenter facit . Quare ab aequale sumptu , \\\hline
1.2.21 & Por la qual cosa dela espenssa \textbf{ egual o alguas vegadas } dela mas pequeña el liberal o el magnifico fazemas conueniblemente su obra & ubi multum expendit , indecenter facit . Quare ab aequale sumptu , \textbf{ vel etiam aliquando a minori , } liberalis , \\\hline
1.2.21 & e conosçedores quales despenssas a quales obras conuienen . \textbf{ Et aellos otrosi mucho mas pertenesçe de fazer grandes donaconnes } e lobre puiantes de espenssas & qui sumptus quibus operibus deceant . \textbf{ Ad eos autem maxime spectat facere magnas largitiones , } et excellentes sumptus boni gratia \\\hline
1.2.21 & que otro ninguno \textbf{ e avn essa misma manera pertenesçe a ellos fazer despenssas muy delectablemente e sin detenimiento . } Ca assi commo dicho es de suso & qui regnum \textbf{ debet in bonum dirigere . Sic etiam ad eos spectat delectabiliter , | et prompte sumptus facere . } Nam \\\hline
1.2.21 & por que la natura es contenta \textbf{ e pagada de pequenas cosas } las riquezas son oçiosas & ( \textbf{ ut supra dicebatur ) quia modicis natura contenta est , } ociosae sunt diuitiae , \\\hline
1.2.21 & e en riquezas \textbf{ tantomas les conuiene aellos de fazer mayores particonnes e mayores dones } e mas espender delectable ment en sin detenimiento & et diuitiis , \textbf{ tanto magis decet eos ampliores retributiones facere , } et magis delectabiliter , \\\hline
1.2.21 & tantomas les conuiene aellos de fazer mayores particonnes e mayores dones \textbf{ e mas espender delectable ment en sin detenimiento } Et otrosi avn conuiene alos Reyes & tanto magis decet eos ampliores retributiones facere , \textbf{ et magis delectabiliter , | et prompte expendere . Decet } etiam Reges , \\\hline
1.2.21 & que non pue de cada vno ser magnifico \textbf{ por que non puede cada vno fazer grandes espenssas } Mas assi commo alli dize el philosofo tales son los nobles e los głiosos . & tales oportet esse nobiles \textbf{ et gloriosos . } Quare quanto est nobilior aliis , \\\hline
1.2.22 & Por enl de assi commo çerca los bienes aprouechosos son dos uirtudes . \textbf{ La vna que cata alas grandes espenssas } assi commo es la magnificençia . & cuiusmodi sunt honores . \textbf{ Sicut igitur circa ipsa bona utilia est duplex virtus una respiciens magnos sumptus , } ut magnificentia , \\\hline
1.2.22 & asy commo la liberalidat . \textbf{ En essa misma manera çerca los bienes honestos son dos uirtudes ¶ } La vna que cata las grandes horras & ut Liberalitas : \textbf{ sic circa ipsa bona honesta est duplex virtus , } una quae respicit magnos honores , \\\hline
1.2.22 & En essa misma manera çerca los bienes honestos son dos uirtudes ¶ \textbf{ La vna que cata las grandes horras } assi commo la magnanimidat ¶ & sic circa ipsa bona honesta est duplex virtus , \textbf{ una quae respicit magnos honores , } ut magnanimitas , \\\hline
1.2.22 & que comunalmente se puede dezir uirtud amadora de honrra \textbf{ Mas en tres maneras se puede cada vno auer en las grandes honrras } Ca alguon sson & et alia quae respicit mediocres , \textbf{ quae communi nomine dici potest honoris amatiua . In magnis autem honoribus tripliciter quis se habere potest . } Nam quidam in talibus deficiunt , \\\hline
1.2.22 & assi commo los pusillani mes \textbf{ que son de flaco coraçon . } Et otros son que sobrepuian en las honrras & Nam quidam in talibus deficiunt , \textbf{ ut pusillanimes . Quidam vero superabundant , } ut praesumptuosi . Quidam autem se habent , \\\hline
1.2.22 & que pressumendes si mas que deuen . \textbf{ Mas otros son que se han cerca las grandes honrras assic̃omo conuiene . } Et estos son dichos magnanimos & ut praesumptuosi . Quidam autem se habent , \textbf{ ut decet , } ut magnanimi . \\\hline
1.2.22 & Et estos son dichos magnanimos \textbf{ que quiere dezir omes de grand coraçon ca nos ueemos algunos } que dessi son aptos e apareiados & ut magnanimi . \textbf{ Videmus enim aliquos de se aptos ad magna , } potentes magna et ardua exercere : \\\hline
1.2.22 & que dessi son aptos e apareiados \textbf{ para fazer grandes cosas . } Et poderosos para vsar de cosas grandes e altas . & Videmus enim aliquos de se aptos ad magna , \textbf{ potentes magna et ardua exercere : } quadam tamen pusillanimitate ducti , \\\hline
1.2.22 & assi commo faze el presuptuoso . \textbf{ Et desto paresçe manifiesta miente } que cosa es la magranimidat & ut praesumptuosus . \textbf{ Quare manifeste patet , } quid sit magnanimitas . \\\hline
1.2.22 & e tienpra los gastamientos en espender . \textbf{ En essa misma manera la magnanimidat } por que es medianera entre la pusillanmidat e la presunpçion es dicha uirtud & et moderans prodigalitates : \textbf{ sic magnanimitas , } quia est media inter pusillanimitatem , et praesumptionem , est virtus quaedam reprimens pusillanimitates , et moderans praesumptiones . Viso \\\hline
1.2.22 & e tienpra las presunpçiones \textbf{ que son sobrepuiamientos en cometer las grandes cosas } que non pueden acabar . & quid est magnanimitas , \textbf{ de leui patet , circa } quae habet esse . Videtur autem Philosophus 4 Ethicor’ velle , \\\hline
1.2.22 & Ca las riquezas e los prinçipados \textbf{ e los otros bienes de fuera non son tan grandes bienes commo la honrra } Por que los omes las mas vezes ordenan estas cosas & non sunt tantum bonum , \textbf{ sicut honor : } quia ut plurimum homines haec ordinant , \\\hline
1.2.22 & que avn el fuerte se ha conueniblemente en los otros periglos . \textbf{ En essa misma manera la magranimidat prinçipalmente es cerca las honrras } assi commo çerca grandes bienes . & etiam in aliis periculis decenter se habeat . \textbf{ Sic magnanimitas principaliter est circa honores , } tanquam circa maxima bona exteriora : \\\hline
1.2.22 & En essa misma manera la magranimidat prinçipalmente es cerca las honrras \textbf{ assi commo çerca grandes bienes . } Mas despues desto es çerca las riquezas e los prinçipados e çerca los otros bienes de fuera en tal manera & Sic magnanimitas principaliter est circa honores , \textbf{ tanquam circa maxima bona exteriora : } ex consequenti autem est circa diuitias , \\\hline
1.2.22 & Ca si el magnanimo es honrrado \textbf{ o sil contesçiere ael buena ventura non se leunata } por ende en vana gloria . & Si enim honoratur , \textbf{ vel si magna fortunia ei occurrant , | non extollitur . } Si autem inhonoratur , \\\hline
1.2.22 & o sil contesçiere ael buena ventura non se leunata \textbf{ por ende en vana gloria . } Otrossi si es desonrrado & non extollitur . \textbf{ Si autem inhonoratur , } et magna infortunia veniant super ipsum , \\\hline
1.2.22 & Otrossi si es desonrrado \textbf{ e vinieren sobre el grandes desauenturas } non cahe por ende . & Si autem inhonoratur , \textbf{ et magna infortunia veniant super ipsum , } non deiicitur . \\\hline
1.2.22 & que conueniblemente se deue auer en qual si quier estado . \textbf{ Mas al pusill animo } e de flaco coraçon pertenesçe non saber sofrir buenas uenturas . & In quolibet enim statu nouit magnanimus se decenter habere . \textbf{ Ad pusillanimem enim pertinet nescire fortunas ferre . } Ideo Andron’ Perip’ ait : \\\hline
1.2.22 & Mas al pusill animo \textbf{ e de flaco coraçon pertenesçe non saber sofrir buenas uenturas . } Por ende dize andronico el sabio philosofo & In quolibet enim statu nouit magnanimus se decenter habere . \textbf{ Ad pusillanimem enim pertinet nescire fortunas ferre . } Ideo Andron’ Perip’ ait : \\\hline
1.2.22 & e de flaco coraçon pertenesçe non saber sofrir buenas uenturas . \textbf{ Por ende dize andronico el sabio philosofo } que las obras dela pusilla nimidat son tales & Ad pusillanimem enim pertinet nescire fortunas ferre . \textbf{ Ideo Andron’ Perip’ ait : } Opera pusillanimitatis esse , \\\hline
1.2.22 & nin desonrras \textbf{ nin buena uenturͣa nin mala . } Mas despues que es honrrado ynchasse con honrra & neque inhonorationem , \textbf{ neque bonam fortunam , | neque infortunium possit ferre : } sed honoratum quidem intumescere , \\\hline
1.2.22 & Mas despues que es honrrado ynchasse con honrra \textbf{ e en poto bien auenturado se leunata en vana eglesia } ¶Mostrado que cosa es la magnan midat & sed honoratum quidem intumescere , \textbf{ et parum bene fortunatum extolli . Ostenso } quid est magnanimitas , et circa quae habet esse . Restat ostendere , quomodo possumus nosipsos magnanimos facere . \\\hline
1.2.22 & Ca dicho es de suso \textbf{ que el pusillanimo non sabe sofrir buenas uenfas } mas de muy pequana buena uentura se leunata en vana gloria & siue quaecunque alia huiusmodi bona . \textbf{ Dictum est enim pusillanimem nescire fortunas ferre , } sed ex modico fortunio extolli , \\\hline
1.2.22 & que el pusillanimo non sabe sofrir buenas uenfas \textbf{ mas de muy pequana buena uentura se leunata en vana gloria } e se enssoberuesçe . & Dictum est enim pusillanimem nescire fortunas ferre , \textbf{ sed ex modico fortunio extolli , } ex infortunio vero deprimi . Magnanimus quidem non sic , \\\hline
1.2.22 & mas el magranimo non taze ali . \textbf{ Ca assi commo dicho es de suso el magnanimo sabe sosrir buenas venturas e sabe conueniblemente se auer en todo estado } Mas la razon & sed ( ut dicebatur ) nouit fortunas ferre , \textbf{ et in quolibet statu scit se decenter habere . Causa autem , } quare quis nescit fortunas ferre , est , \\\hline
1.2.22 & Mas la razon \textbf{ por que algunos non sabe sofrir buenas uenturas } es por que preçia mucho los bienes de fuera . & et in quolibet statu scit se decenter habere . Causa autem , \textbf{ quare quis nescit fortunas ferre , est , } quia nimis appretiatur exteriora bona . Ideo quando aliquid \\\hline
1.2.22 & por que non cuydaremos \textbf{ que tales bienes son los mayores bienes . } Et si contesçiere quenos uiniere alguna desauentura & non extollemur , \textbf{ cum non reputamus talia esse simpliciter maxima bona . } Et si contingat nos infortunari circa ea , decenter tolerabimus , \\\hline
1.2.23 & nin esponer su cuerpo a periglos \textbf{ por pequanans cosas } mas por las grandes & est non esse amatorem periculorum , \textbf{ neque se exponere pro paruis periculis , } sed pro magnis , \\\hline
1.2.23 & e por tales delas quales se puede leunatar \textbf{ e muy grand prouecho } Et quando assi el mangnanimo se pusiere alos periglos & sed pro magnis , \textbf{ ut pro iis ex quibus potest consurgere magna utilitas . } Cum autem sic se periculis exponit , \\\hline
1.2.23 & Ca assi commo dize el philosofo en el quarto libro delas ethicas pertenesce mucho almagnanimo ser mucho partidor e dador de gualardones ¶ \textbf{ La tercera propiedat es que pertenesçe al magnanimo ser de pocasobras . } por que dicho es de suso & ut dicitur 4 Ethicor’ . \textbf{ Tertio spectat ad ipsum esse paucorum operatiuum . } Dictum est enim magnanimum esse circa magna , \\\hline
1.2.23 & por que dicho es de suso \textbf{ que el magnanimo ha de seer çerca grandes cosas } assi commo cerca aquellas & Tertio spectat ad ipsum esse paucorum operatiuum . \textbf{ Dictum est enim magnanimum esse circa magna , } ut circa ea , \\\hline
1.2.23 & Et tales cosas commo estas non contesçen muchas vezes . \textbf{ Et por ende conuiene al magnanimo ser de pocas obras ¶ } La quarta propiedat es que conuiene al maguanimo ser magnifiesto & talia autem non multotiens occurrunt , \textbf{ ideo decet magnanimum esse paucorum operatiuum . } Quarto decet magnanimum esse apertum , \\\hline
1.2.23 & assi que sea uerdadero \textbf{ e que sea manifiesto aborreçedor e manifiesto amador } Et mas deue auer cuydado del audat & ut sit veridicus , \textbf{ et | sit manifestus oditor , et amator , } et magis curare de veritate , \\\hline
1.2.23 & entroͤ \textbf{ los bienes de fuera non faze grant fuerça dellas . } Ca assi commo dicho es muy poco preçia los bienes de fuera¶ & Nam cum talia inter exteriora bona computentur , \textbf{ ipse non multum curat de eis , } quia ( ut dicebatur ) \\\hline
1.2.23 & tal manera deue ser firme \textbf{ e de grant coraçon } que por el bien de dios & ita debet esse constans , \textbf{ et magna nimus , } ut pro bono diuino et communi , paratus sit vitam exponere . \\\hline
1.2.23 & de ser partidore ser muy gualardonadores \textbf{ por que i quanto mas en alto grado estan que los otros } e quanto mas han delas riquezas & et Principes esse plurimum retributiuos : \textbf{ quia quanto sunt in altiori gradu | quam alii , } et quanto plus affluunt diuitiis quam alii : \\\hline
1.2.23 & nin conuiene aellos de seer obradores de todas las cosas \textbf{ mas por que puedan mas liberalmente entender a desenbargar los grandes negoçios que son pocos deuena comne dar los otros negoçios pequa nons } que son muchos alos otros ¶ & nec decet eos omnium esse operatiuos ; \textbf{ sed ut possit liberius intendere expeditioni negociorum magnorum quae sunt pauca , } debet alia minora negocia aliis committere , quae sunt multa . \\\hline
1.2.23 & La qual regla non se deue torcer nin falssar \textbf{ Et ahun conuiene les de seer manifiestos aborresçedores e amadores } por que manifiesta miente aborrescan los males et los pecados & et falsificari non debet . \textbf{ Decet etiam eos esse manifestos oditores , | et amatores , } ut manifeste odiant vitia , \\\hline
1.2.23 & Et ahun conuiene les de seer manifiestos aborresçedores e amadores \textbf{ por que manifiesta miente aborrescan los males et los pecados } e persiguna los malos & et amatores , \textbf{ ut manifeste odiant vitia , } persequantur malos , \\\hline
1.2.23 & que sean alabados de los omes . \textbf{ Por que los que son pu estos en grandes dignidades han muchos li sogeros } que les fablan las cosas & ut laudentur ab hominibus . \textbf{ Nam positi in dignitatibus multos habent adulatores , } et loquentes eis placentia . \\\hline
1.2.23 & de quales plaze . \textbf{ Mas si creyeren atales lisongeros conuienea ellos } de non obrar segunt las leyes & et loquentes eis placentia . \textbf{ Si autem talibus adulatoribus credant , | contingit eos non agere } secundum legem et rationem , \\\hline
1.2.24 & Et por ende lo amos aquellos que non curan de su honrra . \textbf{ ¶ Pues que assi es auer cuydado ome de su propia honrra } en vna manera es de loar & et rursus quia vituperamus ambitiosos laudamus non curantes \textbf{ de honore suo . | Curare igitur de proprio honore , } uno modo est laudabile , \\\hline
1.2.24 & non por que seamos cobdiçiosos della \textbf{ nin por que pongamos nr̃a fin } e nra bien andança en honrras . & non quod simus ambitiosi , \textbf{ nec quod finem nostrum ponamus in honoribus , } sed quod agamus opera honore digna . Opera autem honore digna dupliciter considerari possunt . \\\hline
1.2.24 & Mas las obras dignas de honrra se puede entender en dos maneras . \textbf{ O en quanto son proporçionadas anos . } E en quanto son dignas de grant honrra . & sed quod agamus opera honore digna . Opera autem honore digna dupliciter considerari possunt . \textbf{ Vel ut sunt proportionata nobis . } Vel ut sunt digna magno honore . \\\hline
1.2.24 & O en quanto son proporçionadas anos . \textbf{ E en quanto son dignas de grant honrra . } Onde assi cerca las despenssas son dos uirtudes conuiene saber La liƀalidat Et franqza & Vel ut sunt proportionata nobis . \textbf{ Vel ut sunt digna magno honore . } Sicut enim circa sumptus sunt duae virtutes , \\\hline
1.2.24 & e en quanto son ordenadas a grandesobras . \textbf{ En essa misma manera cerca las honrras deuen seer dos uirtudes ¶ } Lauona & et ut ordinantur \textbf{ ad magna opera . Sic circa honores est duplex virtus . } Una quae respicit honores mediocres , \\\hline
1.2.24 & La otra es \textbf{ que cata alas grandes honrras } assi commo la magnanimidat & et haec a Philosopho dicitur honoris amatiua . Alia est , \textbf{ quae respicit magnos honores , } ut magnanimitas . Eadem ergo opera possunt esse aliarum virtutum , \\\hline
1.2.24 & que es grandeza de coraçon \textbf{ Et pues que assi es essas mismas obras pueden ser delas otras uirtudes } e dela magnanimidat . & quae respicit magnos honores , \textbf{ ut magnanimitas . Eadem ergo opera possunt esse aliarum virtutum , } et magnanimitatis . \\\hline
1.2.24 & Mas si esto faze \textbf{ por que tal sobras son dignas de grant honrra } assi es dichomagranimo . & Si vero hoc agit , \textbf{ quia talia opera sunt magno honore digna , } magnanimus est . Sic etiam \\\hline
1.2.24 & assi es dichomagranimo . \textbf{ Avn en essa misma manera } si feziere obras de castidat & quia talia opera sunt magno honore digna , \textbf{ magnanimus est . Sic etiam } si agat opera castitatis , \\\hline
1.2.24 & por que se delecte enllas es dicho casto e tenprado . \textbf{ Mas si estas cosas feziere en quanto son dignas de grant honrra es dicho magranimo . } Et por ende la prop̃a materia dela magranimidat & quia delectatur in eis , castus et temperatus est . \textbf{ Sed si hoc agat , } quia sunt magno honore digna , magnanimus est . Ideo propria materia magnanimitatis non sunt pericula bellica , \\\hline
1.2.24 & Mas si estas cosas feziere en quanto son dignas de grant honrra es dicho magranimo . \textbf{ Et por ende la prop̃a materia dela magranimidat } non son los periglos delas batallas & Sed si hoc agat , \textbf{ quia sunt magno honore digna , magnanimus est . Ideo propria materia magnanimitatis non sunt pericula bellica , } sed talia sunt materia fortitudinis , \\\hline
1.2.24 & non son los periglos delas batallas \textbf{ Cata łs̃ cosas son materia dela fortaleza } nin otrosi las cosas delectables & quia sunt magno honore digna , magnanimus est . Ideo propria materia magnanimitatis non sunt pericula bellica , \textbf{ sed talia sunt materia fortitudinis , } nec delectabilia \\\hline
1.2.24 & por que çerca tales cosas es la tenprança . \textbf{ Mas propia materia dela magranimidat es dichan honrra } por que quales se quier cosas & secundum sensum , quia circa talia versatur temperantia . \textbf{ Sed propria materia magnanimitatis dicitur esse honor , } quia quaecunque agit magnanimus , \\\hline
1.2.24 & que faga el magranimo todas las faze \textbf{ en quanto son dignas de grant honrra . } Et por ende la magnanimidat & quia quaecunque agit magnanimus , \textbf{ agit ea prout sunt magno honore digna . Magnanimitas ergo } ( ut dicitur quarto Ethicorum ) \\\hline
1.2.24 & Ca assi commo la magnificençia es vn honrramiento dela libalidat \textbf{ por que el magnifico faze obras de li beralidat en mas alta manera que la libalidat . } assi en essa misma manera la magranimidat es vn honrramiento de tondas las uirtudes . & Nam sicut magnificentia est quidam ornatus liberalitatis , \textbf{ qua magnificus opera liberalitatis facit excellentiori modo : } sic magnanimitas est quidam ornatus omnium virtutum . \\\hline
1.2.24 & por que el magnifico faze obras de li beralidat en mas alta manera que la libalidat . \textbf{ assi en essa misma manera la magranimidat es vn honrramiento de tondas las uirtudes . } Por que todas las obras delas uirtudes son dignas de honrra & qua magnificus opera liberalitatis facit excellentiori modo : \textbf{ sic magnanimitas est quidam ornatus omnium virtutum . } Nam opera omnium virtutum sunt honore digna . \\\hline
1.2.24 & de que entendemos agora aqui trartarque es dicha amadora de honrra . \textbf{ Ca assi commo essas mismas obras pueden seer delas otras uirtudes } e dela magranimidat & quae dicitur honoris amatiua . \textbf{ Nam sicut eadem opera possunt esse aliarum virtutum , } et magnanimitatis : \\\hline
1.2.24 & e dela magranimidat \textbf{ assi essas mismas obras pueden seer delas otras uirtudes } e desta uirtud & et magnanimitatis : \textbf{ sic eadem esse possunt aliarum virtutum , } et honoris amatiuae . Opera enim aliarum virtutum , \\\hline
1.2.24 & Por que las obras delas otras uirtudes \textbf{ en quanto son dignas de grant honrra parte nesçen ala magranimidat . } Mas en quanto son proporçionadas a nos & et honoris amatiuae . Opera enim aliarum virtutum , \textbf{ ut sunt magno honore digna , | pertinent ad magnanimitatem . } Sed ut sunt proportionata nobis , \\\hline
1.2.24 & e alos prinçipes de seer magnificos e liberales \textbf{ assi en essa misma manera les conuiene de seer magranimos } e amado res de honrra . & Sicut ergo decet Reges et Principes esse magnificos , \textbf{ et liberales : } sic decet eos esse magnanimos , et honoris amatiuos . Reges enim et Principes decet honores \\\hline
1.2.24 & assi commo la fermosura cortoral se ha ala apostura grande de todo el cuerpo . \textbf{ Ca los pequa nons omes } si han los mienbros bien proporçionados & Videtur enim honoris amatiua se habere ad magnanimitatem , \textbf{ sicut formositas corporis se habet ad pulchritudinem . Parui enim si habent membra bene proportionata , } et conformia , \\\hline
1.2.24 & Enpero non son dichos muy apuestos \textbf{ por que la grant apostura non es synon en el grant cuerpo . } Et en essa misma manera los que fazen obras dignas de honrra medianera & non tamen pulchri , \textbf{ quia pulchritudo non est nisi in magno corpore . Sic facientes opera mediocri honore digna , } dicuntur honoris amatiui : \\\hline
1.2.24 & por que la grant apostura non es synon en el grant cuerpo . \textbf{ Et en essa misma manera los que fazen obras dignas de honrra medianera } son dichos amadores de henrra & non tamen pulchri , \textbf{ quia pulchritudo non est nisi in magno corpore . Sic facientes opera mediocri honore digna , } dicuntur honoris amatiui : \\\hline
1.2.24 & son dichos amadores de henrra \textbf{ Mas propreamente quando fazen obras dignas de grant honrra estonçe son dichos magranimos } Et pues que assi es . . & dicuntur honoris amatiui : \textbf{ tamen tunc proprie sunt magnanimi , | quando agunt opera magno honore digna . } Ut igitur Reges \\\hline
1.2.25 & Mas si assi dixieremos nascenden de asgunas dudas \textbf{ por que todo magran imo es amador de honrra } assi commo todo magnifico es liberal . & quaedam dubietates insurgunt . \textbf{ Nam omnis magnanimus est honoris amatiuus , } sicut omnis magnificus est liberalis : \\\hline
1.2.25 & Et pues que assi es \textbf{ si vna misma cosa departidamente tomada nos tira } e nos allega a aquello que la razon manda o uieda . & nec est eadem virtus principaliter moderans passiones , et impellens nos ab eo quod ratio vetat . Inde est ergo \textbf{ quod si unum } et idem aliter \\\hline
1.2.25 & onde nos veemos \textbf{ que la magranimidat ha de seer çerca las grandes honrras . } Mas la grand honrra es en alguna manera grant bien . & et aliter acceptum nos retrahit et impellit , \textbf{ oportebit circa illud dare duas virtutes , } unam impellentem , \\\hline
1.2.25 & que la magranimidat ha de seer çerca las grandes honrras . \textbf{ Mas la grand honrra es en alguna manera grant bien . } Et el grand bien & oportebit circa illud dare duas virtutes , \textbf{ unam impellentem , } et aliam retrahentem . Videmus autem quod magnanimitas est circa magnos honores . Magnus autem honor est quodammodo magnum bonum . Magnum autem bonum , \\\hline
1.2.25 & Mas la grand honrra es en alguna manera grant bien . \textbf{ Et el grand bien } en quanto ha razon de grande nos tira & oportebit circa illud dare duas virtutes , \textbf{ unam impellentem , } et aliam retrahentem . Videmus autem quod magnanimitas est circa magnos honores . Magnus autem honor est quodammodo magnum bonum . Magnum autem bonum , \\\hline
1.2.25 & Et pues que assi es cerca las grandeshonrras \textbf{ e cerca los grandes bienes contesçe de pecar en dos maneras ¶ } Lo primero & ut tendamus in ipsum . Ergo circa magnos honores , \textbf{ et circa magna bona dupliciter contingit peccare . Primo , } si ultra quam ratio dictet prosequamur ea inquantum bona sunt . Secundo , \\\hline
1.2.25 & e por fuerçala magnanimidat nos esfuerca \textbf{ en que uayamos a grandes bienes } assi que non nos tiremos & si ultra quam ratio dictet prosequatur honores , \textbf{ de necessitate cum magnanimitas impellit nos in aliqua magna bona , } ne trahamur ratione difficultatis , \\\hline
1.2.25 & Ca en otra manera el magnani mo seria vicioso e pecaria . \textbf{ Et pues que assi es yr a grandes honrras } e a grandes biens puede ser por dos uirtudes . & ne ultra quam ratio dictet tendat in ea ratione bonitatis , \textbf{ aliter enim esset vitiosus . Tendere igitur in magnos honores , et ad magna bona , esse potest a magnanimitate , } et ab humilitate : \\\hline
1.2.25 & Et pues que assi es yr a grandes honrras \textbf{ e a grandes biens puede ser por dos uirtudes . } ¶ por la magnanimidat e por la humildat . & aliter enim esset vitiosus . Tendere igitur in magnos honores , et ad magna bona , esse potest a magnanimitate , \textbf{ et ab humilitate : } non tamen secundum rationem eandem . \\\hline
1.2.25 & que el magranimo despreçia los otros non \textbf{ por que faga tuerto en mala manera alos otros } mas . por que es de tan alto coraçon . & et ex consequenti retrahit . Ideo magnanimus dicitur alios despicere , \textbf{ non quod aliis vitiose iniurietur , } sed quia est tanti cordis , \\\hline
1.2.25 & por que faga tuerto en mala manera alos otros \textbf{ mas . por que es de tan alto coraçon . } Et assi se fuerça por la uirtud & non quod aliis vitiose iniurietur , \textbf{ sed quia est tanti cordis , } et sic impellitur a virtute , \\\hline
1.2.25 & por razon dela guaueza \textbf{ que non pueda alcançar obras dignas de grand sonrra . } Mas la humildat prinçipalmente tienpra la esꝑanca & ne aliquis ratione difficultatis desperet , \textbf{ ne tendat in opera magno honore digna . Humilitas vero principaliter moderat ipsam spem , } ne aliquis nimis sperans \\\hline
1.2.25 & Mas la humildat prinçipalmente tienpra la esꝑanca \textbf{ por que alguno auiendo grand esperança del bien non vaya en pos grandes honrras } mas de quanto la razon manda & ne tendat in opera magno honore digna . Humilitas vero principaliter moderat ipsam spem , \textbf{ ne aliquis nimis sperans | de ipso bono , ultra rationem prosequatur magnos honores . } Erit igitur omnis magnanimus humilis modo \\\hline
1.2.25 & que dicha es . \textbf{ Et la humildat en alguna manera es essa misma cosaque la uirtud } que es dicha amadora de honrra medianera & quo dictum est . \textbf{ Et humilitas quodammodo est idem , } quod mediocris honoris amatiua : \\\hline
1.2.25 & por que aquel es dicho humildoso \textbf{ que tienpra la esperança de ganar grandes honrras } e va a ellas medianeramente . & quia ille dicitur humilis , \textbf{ qui moderans spem ipsam ad adipiscendum honores magnos , } mediocriter tendit in honores illos . \\\hline
1.2.25 & e va a ellas medianeramente . \textbf{ Mas si la humildat es essa misma cosa sinplemente } que amar las honrras medianeras . & mediocriter tendit in honores illos . \textbf{ Utrum autem humilitas } sit idem simpliciter quod diligere mediocres honores , \\\hline
1.2.25 & que amar las honrras medianeras . \textbf{ O si es essa misma cosa sinplemente con aquella uirtud } la qual el philosofo aparta dela magnanimidat & Utrum autem humilitas \textbf{ sit idem simpliciter quod diligere mediocres honores , } vel utrum sit idem simpliciter cum virtute illa quam Philosophus distinguens a magnanimitate appellat eam honoris amatiuam ? \\\hline
1.2.25 & e llama la amadora de honrra . \textbf{ Esto non es de esta presente arte } o desta presente manera & vel utrum sit idem simpliciter cum virtute illa quam Philosophus distinguens a magnanimitate appellat eam honoris amatiuam ? \textbf{ Non est praesentis speculationis . } Sed si in Moralibus contingeret nos ulteriora componere , \\\hline
1.2.25 & Esto non es de esta presente arte \textbf{ o desta presente manera } en que fablamos . & vel utrum sit idem simpliciter cum virtute illa quam Philosophus distinguens a magnanimitate appellat eam honoris amatiuam ? \textbf{ Non est praesentis speculationis . } Sed si in Moralibus contingeret nos ulteriora componere , \\\hline
1.2.26 & e deꝑtimiento dela magranimidat . \textbf{ Ca commo quier que vna e essa misma uirtud es aquella quereꝑme las suꝑ habundançias } e tienpra los fallesçimientos . & humilitatem a magnanimitate differre . \textbf{ Nam licet una et eadem virtus reprimet superabundantias , } et moderet defectus : \\\hline
1.2.26 & que assi commo cerca la magnanimidat puede contesçer de sobrepuiar \textbf{ e defallesçer en essa misma manera puede contesçer cerca la humildat . } Ca assi commo paresce delas cosas ya dichas & Sciendum igitur quod sicut circa magnanimitatem conuenit abundare \textbf{ et deficere : | sic et circa humilitatem esse habet . } Nam ( ut patet ex dictis ) magnanimitas est virtus \\\hline
1.2.26 & por que non seamos retenidos \textbf{ qua non siguamos las obras dignas de grant honrra } por razon dela guaueza dellas . Et por ende por que en esto algunos sobrepuian & ne ratione difficultatis retrahamur , \textbf{ ut non prosequamur opera magno honore digna . } Quia igitur in hoc aliqui superabundant , ut praesumptuosi , \\\hline
1.2.26 & assi commo los pusillammos \textbf{ que son de flaco coraçon } Por ende la magranimidat es medianera entre las presupconnes & quidam vero deficiunt , \textbf{ ut pusillanimes : } ideo magnanimitas est media inter praesumptiones , et pusillanimitates . \\\hline
1.2.26 & Por ende la magranimidat es medianera entre las presupconnes \textbf{ e las pus illammmidades e flaquezas de coraçon en essa misma manera } assi commo ya dicho es paresçe & ut pusillanimes : \textbf{ ideo magnanimitas est media inter praesumptiones , et pusillanimitates . } Sic etiam ex dictis patet , \\\hline
1.2.26 & que nos tira \textbf{ por que non siguamos las grandes honrras } por razon dela bondat e dela plazenctia & quod humilitas est virtus nos retrahens , \textbf{ ne ratione bonitatis et delectabilitatis , } quae est in rebus honorificis , \\\hline
1.2.26 & Et por ende en esto algunos lobrepuian assi conmo los sob̃nos \textbf{ que siguen las excellençias e las grandes honrras } mas que deuen . & In hoc ergo aliqui superabundant , \textbf{ ut superbi , prosequentes excellentias , et honores , ultra quam debeant . Aliqui vero deficiunt , } qui se deiiciunt ultra quam debeant , \\\hline
1.2.26 & Ca en quariendo omne obrar obras \textbf{ que son dignas de grant honrra } non solamente pueden pecar & ex consequenti vero contrariatur deiectioni . \textbf{ Inquirendo enim opera honore digna , } non solum contingit peccare per superbiam , \\\hline
1.2.26 & que se alaban \textbf{ por que se vestian de villes pannos } mas que el su estado demandaua creyendo & quia ultra quam eorum status requireret , \textbf{ vilius induebantur : } credentes ex hoc in quendam honorem , \\\hline
1.2.26 & mas que el su estado demandaua creyendo \textbf{ por esto que se lle una tarian en grant honrra e en grant excellençia } Onde dize el philosofo en esse logar & vilius induebantur : \textbf{ credentes ex hoc in quendam honorem , | et in quandam excellentiam consurgere . } Unde ibidem scribitur , \\\hline
1.2.26 & Onde dize el philosofo en esse logar \textbf{ que la sobrepuianca e el grant fallesçemiento son cosas } de que se preçian alguons omes & Unde ibidem scribitur , \textbf{ quod superabundantia , } et valde defectus , \\\hline
1.2.26 & que la sobrepuianca e el grant fallesçemiento son cosas \textbf{ de que se preçian alguons omes } e non son uirtudes ¶ & Unde ibidem scribitur , \textbf{ quod superabundantia , } et valde defectus , \\\hline
1.2.26 & Et otrosi que non pongan la su bien andança en sobrepuiança de honrra lo que fazen los sobuios \textbf{ por que deuen fazer los Reyes bueans obras } e dignas de honrra & quod tamen suam felicitatem non ponant in excellentia et honore , \textbf{ quod faciunt superbi . Debent enim agere bona opera } et honore digna boni gratia , \\\hline
1.2.26 & mas que deue \textbf{ por la mayor parte va a mayores cosas } de quanto puede el su poder . & ut plurimum tendit \textbf{ ad ea quae proprias vires excellunt . } Ideo decet homines esse humiles , \\\hline
1.2.26 & assi \textbf{ e alos otros a grandes periglos } por que non pueden conplir las cosas que comiençan & et ultra rationem excellentiam querentes , \textbf{ se et alios exponunt periculis , } non valentes adimplere quod inchoant . \\\hline
1.2.27 & Ca assi commo la fortaleza es medianera entre los miedos e las osadias \textbf{ assi en essa misma manera es la manssedunbre medianera entre la sanna } por la qual desseamos ser vengados . & quia sicut fortitudo est media inter timores et audacias , \textbf{ sic mansuetudo est media inter iram , per quam cupimus vindictam , et irascibilitatem , per quam condonamus mala nobis illata . Ille vero mitis est , qui nec de omnibus cupit vindictam , } nec in tantum deficit a punitione \\\hline
1.2.27 & assi commo la fortaleza repreme los miedos \textbf{ e tienpra las osadias en essa misma manera la manssedunbre repreme las sannas } e tienpralas non sannas & Quare sicut fortitudo reprimit timores , \textbf{ et moderat audacias , | sic mansuetudo reprimit iras , } et moderat irascibilitates . \\\hline
1.2.27 & que es nunca se enssanar \textbf{ por lo que ha razon de se ensannar . Ca natural cosa es anos } que por los males e por las jniurias & ex consequenti autem intendit moderare passiones oppositas irae . \textbf{ Nam naturale est nobis } ut ex malis nobis illatis appetamus punitionem , \\\hline
1.2.27 & para querer ser vengados \textbf{ e dar pena a aquellos que nos fazen alguons males . } Mas avn en alguna manera natural cosa esa nos de dessear & non solum naturaliter inclinamur , \textbf{ ut velimus puniri inferentes nobis aliqua mala , } sed etiam quodammodo naturale est nobis appetere punitionem ultra condignum . \\\hline
1.2.27 & e dar pena a aquellos que nos fazen alguons males . \textbf{ Mas avn en alguna manera natural cosa esa nos de dessear } de ser vengados dellos & ut velimus puniri inferentes nobis aliqua mala , \textbf{ sed etiam quodammodo naturale est nobis appetere punitionem ultra condignum . } Nam quia malum nobis illatum videtur nobis maius esse , \\\hline
1.2.27 & que aquellos quanos fazen mal \textbf{ e iniuria ayan mayor pena } que deuen & quam sit ; \textbf{ iniuriatores nostros plus puniri volumus , } quam puniendi sint . Difficile est ergo valde reprimere iras , \\\hline
1.2.27 & por el mal que nos fazen . \textbf{ Et por que muy guaue cosa es de repremir las sannas } e de non dessear uengança delas iniurias & iniuriatores nostros plus puniri volumus , \textbf{ quam puniendi sint . Difficile est ergo valde reprimere iras , } et non appetere punitiones iniuriarum ultra quam dictet ratio . Plures ergo peccant in appetendo plus : \\\hline
1.2.27 & mas que la razon e el entendimiento muestra \textbf{ por que muchos pecan en desseando mayor vengança } e pocos pecan en desseando menor vengança . & et non appetere punitiones iniuriarum ultra quam dictet ratio . Plures ergo peccant in appetendo plus : \textbf{ pauci vero delinquunt in appetendo minus . } Propter quod si virtus est circa bonum \\\hline
1.2.27 & por que muchos pecan en desseando mayor vengança \textbf{ e pocos pecan en desseando menor vengança . } Por la qual cosa & et non appetere punitiones iniuriarum ultra quam dictet ratio . Plures ergo peccant in appetendo plus : \textbf{ pauci vero delinquunt in appetendo minus . } Propter quod si virtus est circa bonum \\\hline
1.2.27 & o por otros humores non iudga derechamente de los sabores . \textbf{ assi en essa misma manera } si el appetito es enconado e desordenado . & vel per alios humores , \textbf{ non recte iudicamus de saporibus : } sic infecto appetitu per immoderatam iram , \\\hline
1.2.27 & por que por la saña non sea tristornado nin torçido . \textbf{ Et avn en essa misma manera non es cosa conuenible al Rey } de nunca se enssanar & ne per iram peruertatur \textbf{ et obliquatur . Sic etiam , } si nullo modo esset irascibilis , \\\hline
1.2.28 & conuiene a ellos de ser manssos \textbf{ egund que demanda este presente negoçio } suficientemente dixiemos de las uirtudes & decet eos mansuetos esse . \textbf{ Ut postulat praesens negocium , } sufficienter diximus de virtutibus respicientibus bona \\\hline
1.2.28 & e ordenar las nr̃as palauras \textbf{ e las nuestras obras a buena conuerssaçion e conuenible . } lo segundo las nr̃as palauras & et ordinare opera , \textbf{ et verba nostra ad debitam conuersationem . Secundo , verba , } et opera nostra deseruiunt nobis ad veritatem : \\\hline
1.2.28 & lo segundo las nr̃as palauras \textbf{ e las nr̃as obras siruennos ala uerdat } por que por ellas somos iudgados quales somos . & et verba nostra ad debitam conuersationem . Secundo , verba , \textbf{ et opera nostra deseruiunt nobis ad veritatem : } quia per ea iudicamur quales sumus . \\\hline
1.2.28 & que dessi nin de sus palauras \textbf{ nin de sus fechos muestre mayores cosas } que son & non est aliud nisi quod homo non sit iactator , \textbf{ quod de se verbis aut factis maiora ostendat quam sint , } nec sit irrisor et despector aliorum , \\\hline
1.2.28 & e çerca las obras \textbf{ en las quales partiçipamos con los otros han de ser tres uirtudes } conuiene saber ¶amistança & circa verba et opera , \textbf{ in quibus cum aliis communicamus , | habet esse triplex virtus , } videlicet , amicabilitas , \\\hline
1.2.28 & Et otrosi alegera conuenible la qual el philosofo llama heutropeliam \textbf{ que quiere dezir buena conpanina . } Et pues que assi es & quae apertio nuncupatur : et debita iocunditas , \textbf{ quam eutrapeliam vocat . Communicando igitur cum aliis , } si bene conuersari volumus , \\\hline
1.2.28 & Ca ha de seer çerca las palauras \textbf{ en quanto son ordenadas a buena conuerssaçion en la uida del omne . } Ca si el omne es naturalmente animalia aconpanable & quia est circa opera , et uerba , \textbf{ ut ordinantur ad debitam conuersationem in uita . } Si enim homo est naturaliter animal sociale , \\\hline
1.2.28 & e el entendemiento muestre \textbf{ que segt el departimiento delons omes ha omne de conuerssar departidamente con ellos } commo quier que todos los omes & Quare cum recta ratio dictet , \textbf{ quod } secundum diuersitatem personarum diuersimode sit conuersandum : licet omnes homines uolentes viuere politice debeant esse amicabiles \\\hline
1.2.28 & Empero non deuen todos en vna manera seramigables e bien fablantes . \textbf{ por que la grant familiaridat pare e faze despreçiamiento . } Et pues que assi es los Reyes e los p̃nçipes & non tamen omnes eodem modo amicabiles debent esse . \textbf{ Nam quia nimia familiaritas contemptum parit , } Reges et Principes , \\\hline
1.2.28 & para el enfermo \textbf{ el qual seria pequano para el sanno assi en essa misma manera en la conuerssacion de los omes alguna familiaridat es contada al Rey a uirtud } e es dicho por ende amigable & qui esset modicus sano : \textbf{ sic in conuersatione hominum . | Aliqua enim familiaritas reputatur regi ad virtutem , } et dicitur \\\hline
1.2.28 & e es dicho por ende amigable \textbf{ que si alguna otra perssona comun non mostrasse mayor familiaridat } de quanta muestra el Rey seer le ye contado a menos preçio & ex hoc amicabilis esse : \textbf{ quia si aliqua una communis persona non plus de familiaritate participaret , } reputaretur ei ad vitium , \\\hline
1.2.29 & assi como es mostrado dessuso \textbf{ assi en essa misma manera cerca la uerdat contesçe de sobrepuiar } e defallesçer & Sicut circa conuersationem in vita contingit superabundare , et deficere : \textbf{ sic circa veritatem contingit superabundare , } et deficere . \\\hline
1.2.29 & por sobrepuiança mostrando de ssi mismos \textbf{ por palauras o por fechos mayores cosas } que sean en ellos & de se verbis \textbf{ aut factis maiora quam sint , } et tales vocantur iactatores . Aliqui vero ab hac veritate declinant \\\hline
1.2.29 & e segerendo de ssi mismos algunas cosas villes \textbf{ que en ellos non sono negando de ssi mismos alguas cosas } que en ellos son . & per defectum , fingentes de se aliqua vilia \textbf{ quae in ipsis non sunt , } quos Philosophus vocat irones , \\\hline
1.2.29 & que es en ssi . Enpero non conuiene \textbf{ que dessi mismo daga toda la bondat } que conosçe & quam sibi inesse non credit . \textbf{ Non tamen oportet quod de se dicat totam bonitatem , } quam sibi inesse cognoscit : \\\hline
1.2.29 & que declinar alo menos \textbf{ e dezir dessi menores cosas que sean es obra de sabio . } Pues que assi es parte nesce al uerdadero & quod declinare ad minus , \textbf{ et dicere de se minora quam sint , est opus prudentis . Spectat igitur ad veracem nullo modo dicere de se maiora , } quam sint , \\\hline
1.2.29 & non dezir \textbf{ dessi mayores cosas } que sean en el & ø \\\hline
1.2.29 & commo quier que en ellos nen sean \textbf{ e prometen alos amigos conosçidos grandes bienes e grandes ayudas . } delas quales cosas muy poco o ninguna cosa non cunplen por obra . & cum tamen illis careant : \textbf{ et promittunt amicis et notis magna bona et magna auxilia , } de quibus modicum , \\\hline
1.2.29 & mas deuemos de elinar en tales cosas alo menos . . \textbf{ deziendo cada vno dessi menores cosas que sean en el } que sobrepiuaren & ( \textbf{ ut dictum est ) in talibus magis est declinandum in minus dicendo de se minora } quam sint , \\\hline
1.2.29 & mas afirmando \textbf{ e deziendo de ssi mayores cosas } que son en el . & quam sint , \textbf{ quam sit excedendum in plus asserendo de se maiora . } Possumus autem duplicem causam assignare \\\hline
1.2.29 & que pertenesçe al sabio de declinar alo menos . \textbf{ Ca muy grand pradençia } e grant sabiduria es conosçer & quod prudentis est declinare in minus . \textbf{ Nam magnae prudentiae est , } cognoscere seipsum , \\\hline
1.2.29 & Ca muy grand pradençia \textbf{ e grant sabiduria es conosçer } assi mismo . omne e saber & Nam magnae prudentiae est , \textbf{ cognoscere seipsum , } et sciri quod propria bona semper aestimantur maiora quam sint . \\\hline
1.2.29 & alo menos diziendo \textbf{ de ssi menores cosas } e mas viles de quanto son . & Nam declinantes notabiliter in minus , \textbf{ et dicentes } de se notabiliter minora et viliora , quam sint , \\\hline
1.2.29 & dessi mas manifiestos e uerdaderos non mostrando \textbf{ nin alabando dessi mayores cosas } que son & non ostendendo , \textbf{ vel iactando de se maiora quam sint , } vel promittendo aliis maiora quam faciant . \\\hline
1.2.29 & que son \textbf{ nin prometiendo alos otros mayores cosas } que faran . & vel iactando de se maiora quam sint , \textbf{ vel promittendo aliis maiora quam faciant . } Immo tanto magis decet Reges et Principes cauere iactantiam , \\\hline
1.2.30 & si es liberal e honesto \textbf{ e tenprado hase de ordenara buena fin . } por que es en alguna manera necessario ala uida del omne . & honestus , \textbf{ et modestus , | ordinari habet in bonum finem : } quia est quodammodo necessarius in vita . \\\hline
1.2.30 & Et assi el suenno es cosa necessaria en la uida . \textbf{ En essa misma manera } por que en estudiando e trabaiando en los negoçios del regno & et est necessarius somnus in vita . \textbf{ Sic quia studendo , vel negociis regni insistendo , } vel alia faciendo , \\\hline
1.2.30 & por que en estudiando e trabaiando en los negoçios del regno \textbf{ e faziendo otras cosas muchas trabaiamos continuada mente . } Por ende fueron falladas algunas delecta connes de uiegos que son entrepuestas alas uegadas alos nuestros cuydados & Sic quia studendo , vel negociis regni insistendo , \textbf{ vel alia faciendo , | continue laboramus , } aliquae delectationes \\\hline
1.2.30 & en qual manera podian tomar de aquella prea alguna cosa \textbf{ en essa misma manera } los que quieren fazer de todo en todo riso & Sicut enim aues illae non curabant qualitercunque possent aliquid \textbf{ de illa praeda capere : } sic volentes \\\hline
1.2.30 & Empero non es çerca estas cosas \textbf{ egual menre nin prinçipal mente . } Ca uirtudes prinçipalmente çerca aquellas cosas & quam circa repressionem superfluitatum in ludo : \textbf{ non est tamen circa haec aeque principaliter , } quia virtus semper est principalius circa difficilius . Reprimere autem superfluitates ludorum est difficilius , quam moderare defectus . Habet enim ipse ludus quandam delectationem annexam , propter quam magis inclinamur , \\\hline
1.2.30 & mas repremir las superfluydadesde los iegos \textbf{ es muy mas guaue cosa } que tenprar los fallesçimientos dellos . & ø \\\hline
1.2.30 & mucho mas esto conuiene alos Reyes e alos prinçipes en tanto vsar tenpradamente delas delecta connes delos iuegos \textbf{ que si esto feziesen algunas ottas personas comunes paresçeria } que serian montesinos e siluestres . & et Principes adeo moderate uti iocosis delectationibus , \textbf{ quod si hoc facerent personae communes , } viderentur esse durae et agrestes . \\\hline
1.2.30 & por las cosas de suso dichas \textbf{ la delectaçion de los iuegos non es baldia nin en vano por que es ordenada a buena finca es entrepuesta alos nuestros cuydados . } Et por ende rescebimos recrea conn . & ( ut patet ex habitis ) \textbf{ ex hoc iocosa delectatio non est otiosa , et ordinatur in bonum finem , | quia interposita nostris curis , } quandam recreationem accipimus , \\\hline
1.2.30 & mas cobdiconsamente nos leunatamos \textbf{ e nos trabaiamos en las buenas obras } delas quales deuemos auer cuydado . & cuius ardentius insurgimus , \textbf{ ut insistamus bonis operibus , quorum curam habere debemus . } Quare \\\hline
1.2.30 & nin honesto nos parta \textbf{ e nos tire delas buenas obras } e de los cuydados conuenibles & Quare \textbf{ cum iocus immoderatus , vel inhonestus distrahat nos a bonis operibus , } et a debitis curis : \\\hline
1.2.30 & e de los cuydados conuenibles \textbf{ en tanto mayor pecado es alos Reyes } e alos prinçipes de vsar destenp̃damente & et a debitis curis : \textbf{ tanto detestabilius est Reges , et Principes immoderate , } vel inhoneste uti delectationibus ludorum , \\\hline
1.2.31 & que pueda seer acabadamente sin las otras uirtudes todas \textbf{ Et en esta misma manera } avn todos los que tractaron delas uirtudes sentieron esto & Immo nunquam est aliqua una virtus , \textbf{ quae sine aliis virtutibus omnibus perfecte possit haberi . Sic etiam tractatores veritatis senserunt dicentes virtutes connexas esse . } Ad sensum tamen videtur apparere contrarium . \\\hline
1.2.31 & empero que non pueden seer magnificos \textbf{ por que non pue den fazer grandes cosas } por que non han nin pueden fazer grandes espenssas . & qui tamen non possunt esse magnifici : \textbf{ quia nequeunt | magna facere , } eo quod non habeant magnos sumptus . \\\hline
1.2.31 & por que non pue den fazer grandes cosas \textbf{ por que non han nin pueden fazer grandes espenssas . } Et pues que assi es deuedes saber & magna facere , \textbf{ eo quod non habeant magnos sumptus . } Sciendum igitur , Philosophum circa finem 6 Ethicor’ \\\hline
1.2.31 & Et pues que assi es deuedes saber \textbf{ quel philosofo çerca la fin del sexto libro delas ethicas prueua manifiestamente } que todas las uirtudes son conexas & eo quod non habeant magnos sumptus . \textbf{ Sciendum igitur , Philosophum circa finem 6 Ethicor’ } manifeste probare virtutes connexas esse . \\\hline
1.2.31 & nin han las otras uirtudes morales . \textbf{ En essa misma manera avn ueemos algunos } que en el tp̃o dela su moçedat & nec liberales , nec habent virtutet morales alias . Sic \textbf{ etiam ex ipsa pueritia videmus aliquos mox inclinari ad opera largitatis , } qui non sunt casti : \\\hline
1.2.31 & Estas tales por fuerca han de ser conexas e ayuntadas \textbf{ e dezimos en çercana disposiçion } por que los pobres & vel in quadam valde propinqua dispositione : \textbf{ de necessitate habent esse connexae . Diximus autem in propinqua dispositione , } quia pauperes propter carentiam \\\hline
1.2.31 & assi commo en las riquesas \textbf{ luego farian grandes cosas . } Et pues que assi es deuemos declarar & ut sint magnifici : \textbf{ quia si bonis exterioribus abundarent , statim magnifica facerent . } Declarandum est ergo ad plenam intelligentiam dictorum , \\\hline
1.2.31 & Lo primero si establesçieremos \textbf{ e ordenaremos a nos mala fin } assi commo fazen los que pecan & Dupliciter ergo in talibus peccare contingit . Primo , \textbf{ si proponamus nobis malum finem , } ut vitiosi faciunt . Auari enim proponunt sibi , \\\hline
1.2.31 & assi en logar de fin obras de luxͣia . \textbf{ Et en essa misma manera deuemos entender en todos los otros que pecan¶ } Lo segundo contesçe de pecar en tales cosas & ut finem , studere auaritiae . Intemperati vero , venerea , \textbf{ et sic de aliis . Secundo in talibus peccare contingit , } si non debite tendamus in bonum finem . Volunt enim aliqui esse distributores bonorum , \\\hline
1.2.31 & Lo segundo contesçe de pecar en tales cosas \textbf{ si non entendieremos conueniblemente en buena fin . } por que algunos quieren ser partidores de los bienes & et sic de aliis . Secundo in talibus peccare contingit , \textbf{ si non debite tendamus in bonum finem . Volunt enim aliqui esse distributores bonorum , } et proponunt sibi , \\\hline
1.2.31 & quier que establezcan \textbf{ assi buena fin } empero non van derechamente a aquella fin . & Hi ergo licet proponant sibi bonum finem , \textbf{ non tamen recte tendunt in finem illum . } Non ergo peccant in termino , \\\hline
1.2.31 & Otrosi auemos mester la pradençia e la sabiduria \textbf{ por que por ella derecha mente razonemos } e fablemos de aquellas cosas & etiam prudentia , \textbf{ quia per ipsam recte ratiocinamur } de iis quae sunt ad finem . \\\hline
1.2.31 & si mucho es llena de flema \textbf{ dulçe paresçen le todas las cosas dulçes . En essa misma manera quales somos segunt nr̃a uoluntad } e segunt nuestro apetitotal fin proponemos & si vero sit multum infecta phlegmate dulci , \textbf{ videtur participare quandam dulcedinem . Sic quales sumus } secundum voluntatem \\\hline
1.2.31 & dulçe paresçen le todas las cosas dulçes . En essa misma manera quales somos segunt nr̃a uoluntad \textbf{ e segunt nuestro apetitotal fin proponemos } e ordenamos a nos & videtur participare quandam dulcedinem . Sic quales sumus \textbf{ secundum voluntatem } et appetitum , talem finem nobis imponimus : \\\hline
1.2.31 & que son ordenadas ala fin . \textbf{ Ca nos proponiendonos e ordenando nos a buena fin } por las uirtudes morales . & quae sunt ad finem . \textbf{ Nam proponentes nobis bonum finem per virtutes morales , } per prudentiam bene ratiocinamur de iis quae sunt ad finem , \\\hline
1.2.31 & Mas commo non puede seer camino e carrera acabada \textbf{ si non fuere ordenada a buena fin } e a buen termino nunca puede ser auida pradençia & Prudentia vero rectificat viam . \textbf{ Sed cum non sit perfecta via nisi ordinetur in bonum finem et terminum , } nunquam habetur perfecta prudentia , \\\hline
1.2.31 & si non fuere ordenada a buena fin \textbf{ e a buen termino nunca puede ser auida pradençia } e sabiduria acabada & Sed cum non sit perfecta via nisi ordinetur in bonum finem et terminum , \textbf{ nunquam habetur perfecta prudentia , } nisi sit \\\hline
1.2.31 & si non fuere ayuntada alas uirtudes morales \textbf{ por las quales proponemos e ordenamos a nos a buena fin e a buen termino . } En essa misma manera avn por que nunca conuenible mente & coniuncta virtutibus moralibus , \textbf{ per quas proponamus nobis bonum terminum , | et bonum finem . Sic etiam , } quia nunquam debite , \\\hline
1.2.31 & por las quales proponemos e ordenamos a nos a buena fin e a buen termino . \textbf{ En essa misma manera avn por que nunca conuenible mente } nin acabadamente es auida la buean fin & et bonum finem . Sic etiam , \textbf{ quia nunquam debite , } et perfecte habetur bonus finis , \\\hline
1.2.31 & sinos non fueremos a aquella fin \textbf{ por buena e derecha carrera nunca nos dabadamente } podemos auer ningunan uirtud moral & et perfecte habetur bonus finis , \textbf{ nisi tendamus in ipsum per bonam viam , } nunquam perfecte habebimus aliquam virtutem moralem , \\\hline
1.2.31 & Ca commo la uirtud moral sea habito e disposiçion firme de alma \textbf{ e buena escogedora } e acaba a aquel que la ha & nisi sit prudens . \textbf{ Nam cum virtus moralis sit habitus bonus , et electiuus , et perficiat habentem , } et opus suum bonum reddat : \\\hline
1.2.31 & Por ende commo havien escoger \textbf{ e a buena obra fazer } non abasta de entender buena fin & et opus suum bonum reddat : \textbf{ cum ad bene eligere , et ad bonum opus , } sufficiat proponere bonum finem , \\\hline
1.2.31 & e a buena obra fazer \textbf{ non abasta de entender buena fin } si non fuere a aquella fin & cum ad bene eligere , et ad bonum opus , \textbf{ sufficiat proponere bonum finem , } nisi per bonam viam eatur in finem illum , \\\hline
1.2.31 & si non fuere a aquella fin \textbf{ por buen camino o por buenan carrera . } la uirtud moral & sufficiat proponere bonum finem , \textbf{ nisi per bonam viam eatur in finem illum , } virtus moralis \\\hline
1.2.31 & la uirtud moral \textbf{ por la qual proponemos a nos buena fin non puede ser } sin la pradençia e sabiduria . & virtus moralis \textbf{ per quam nobis proponimus bonum finem , } non potest esse sine prudentia per quam recte tendimus in finem illum . \\\hline
1.2.31 & por la qual derechamente entendemos nos yr a aquella fin . \textbf{ En essa misma manera avn la pradençia non puede ser sin uirtud moral . mas deuedes sabra } e notar & non potest esse sine prudentia per quam recte tendimus in finem illum . \textbf{ Sic et prudentia esse non potest sine virtute morali . } Differt enim prudentia , \\\hline
1.2.31 & si non fallare bueans carreras \textbf{ e buenos caminos ordenados abuean fin . } Et pues que assi es non es la pradençia & per quas consequantur venerea et turpia , quae sibi proponunt ut fines . Prudens tamen nullus dicitur , \textbf{ nisi ad bonum finem inueniat bonas vias . } Non est ergo prudentia , \\\hline
1.2.31 & o la muerte escogeria de obrar las cosas de luxia . \textbf{ En essa misma manera avn si alguno fuesse auariento } por que pusiesse la su fin en auer riquezas e dineros & eligeret venerea operari . \textbf{ Sic etiam si esset auarus , } quia finem suum poneret in habendo pecuniam , \\\hline
1.2.31 & Por la qual cosa si conuiene alos Reyes e alos prinçipes de ser \textbf{ assi commo medios dioses } e auer las uirtudesacabadas . & sed complete et perfecte nullatenus fieri potest . \textbf{ Quare sic decet Reges , et Principes esse quasi semideos , } et habere virtutes perfectas : \\\hline
1.2.31 & avn que non obre cosas magnificas e grandes \textbf{ por que non ha onde faza grandes despenssas } Mas el rey nunca puede ser acabado & si non operetur magnifica , \textbf{ et magnos sumptus , | quia non habet } unde faciat magnos sumptus . Rex autem nunquam perfectus esse potest , \\\hline
1.2.32 & e si non resplandeçiere por todas las uirtudes \textbf{ e entendieremos e penssaremos con grant estudio } las palauras del philosofo & ø \\\hline
1.2.32 & e quitro quados de buenos \textbf{ por que los malos omes son en quatro departimientos . } Ca algunos son mueles e blandos & et quatuor bonorum . \textbf{ Mali enim homines sunt in quadruplici gradu : } quidam sunt molles , \\\hline
1.2.32 & Mas algsay que son bestiales ¶ \textbf{ En essa misma manera avn los buenos son en quatro maneras . } Ca algunos son perseuerantes en el bien . & quidam intemperati , \textbf{ quidam bestiales . Sic etiam et boni sunt in quadruplici gradu : } quia quidam sunt perseuerantes , \\\hline
1.2.32 & que muelles son dichos aquellos \textbf{ que por pequeña passion o por pequeña tentaçion caen . } Ca el uocabulo mesmo del muelle lo muestra assi . & quidam vero diuini . Molles autem dicuntur illi , \textbf{ qui modica passione , | vel parua tentatione ruunt , } quod ipsum nomen designat . \\\hline
1.2.32 & ca en estos tales son dichͣs muelles . \textbf{ En otro guado de malos sodichͣs } e puestos los non continentes & vel cum tentantur , cadunt . \textbf{ In alio gradu dicuntur esse incontinentes . } Hi autem differunt a primis . \\\hline
1.2.32 & que se non contienen \textbf{ assi commo lo muestrael uocablo mismo . } Ca contener se este nerse contra alguna cosa . & ø \\\hline
1.2.32 & Ca contener se este nerse contra alguna cosa . \textbf{ Et por enerde el non contener se es acometer algua batalla } e en aquella batalla non se poder tener & sed in sustinendo deficiunt . Continere enim , \textbf{ ut ipsum nomen designat , hoc est , contra aliud se tenere . Incontinere ergo est aggredi pugnam , et in pugna non posse se tenere , } sed deficere . \\\hline
1.2.32 & assemeia los alos paraliticos \textbf{ los quales quieran yr ala diestra parte . } Enpero por la dissoluçion & ø \\\hline
1.2.32 & del que non puede bien gouernar el cuerpo \textbf{ van ala simestro En essa misma manera los muelles } e los non continentes proponen de bien fazer & qui eligentes ire in dextram , propter dissolutionem corporis , et non valentes corpus regere , \textbf{ vadunt in sinistram . } Sic molles et incontinentes proponunt benefacere , \\\hline
1.2.32 & e en muy pequanna tentaçion caen . \textbf{ Mas los non continentes en mayor batalla } e en mayor tentaçonn caen . & Quia molles modica pugna ruunt : \textbf{ incontinentes vero maiori pugna succumbunt . Intemperati ergo peiores sunt , } quam incontinentes , \\\hline
1.2.32 & Mas los non continentes en mayor batalla \textbf{ e en mayor tentaçonn caen . } Mas los otros que son malos en el terçero guado son destenprados & incontinentes vero maiori pugna succumbunt . Intemperati ergo peiores sunt , \textbf{ quam incontinentes , } et molles : \\\hline
1.2.32 & e firmados en el mal \textbf{ que les es cosa delectable de mal fazer¶ } Mas en el quarto guado de los malos son los omes bestiales . & quia sunt adeo habituati in malo , \textbf{ quod delectabile est eis malafacere . In quarto gradu autem sunt bestiales . } Hi autem peiores sunt , \\\hline
1.2.32 & que commo fuesse vna mugier prenada \textbf{ e non pudiesse parir desto conçibio tan grant dolor } que se torno en bestialidat & multas huiusmodi bestialitates . Dicit enim quod cum quaedam praegnans esset , \textbf{ et non potuisset parere , | ex hoc tantum dolorem concepit , } et sic in bestialitatem conuersa est , \\\hline
1.2.32 & que ninguon dende adelante non pariesse ¶ \textbf{ Er essa misma manera } assi commo dize el philosofo & ut nulla de caetero parturiret . \textbf{ Sic etiam ( ut ait ) circa quandam Insulam maris } ut circa Pontum existebant \\\hline
1.2.32 & que fiziesse conbite del¶ \textbf{ En essa misma manera avn çerca aquella tierra que llama una falatim mora una vnas gentes bestiales } assi commo el philosofo dize & etiam multae de Phalaride bestialitates narrantur . Quaedam \textbf{ etiam aliae gentes bestiales } ( ut ait ) sacrificabant matres proprias , \\\hline
1.2.32 & ¶ dicho de los quatroguados delons malos \textbf{ finca de dezir delos quatro linages de los buenos } Ca assi conmo algunos muelles son malos & qui ultra modum hominum male agunt . Dicto de quatuor generibus malorum , \textbf{ restat dicere de quatuor generibus bonorum . } Nam sicut quidam mali sunt molles , \\\hline
1.2.32 & los quales caen \textbf{ por pequeña passion . } assi algunos son dichos buenos & Nam sicut quidam mali sunt molles , \textbf{ qui modica passione ruunt : } sic quidam boni dicuntur esse perseuerantes . \\\hline
1.2.32 & que son perseuerantes en bien . \textbf{ Et estos tales son aquellos que estan et non caen por pequana passion } e por pequana tentaçion . & sic quidam boni dicuntur esse perseuerantes . \textbf{ Huiusmodi autem sunt , | qui stant , } et non modica tentatione cadunt . Loquendo ergo de Perseuerantia , \\\hline
1.2.32 & Et estos tales son aquellos que estan et non caen por pequana passion \textbf{ e por pequana tentaçion . } Mas fablando dela persseuerança & qui stant , \textbf{ et non modica tentatione cadunt . Loquendo ergo de Perseuerantia , } ut Philosophus loquitur , \\\hline
1.2.32 & non es otra cosa \textbf{ si non vna buena disposiconn del alma } que es contraria alla molleza de los omes . & ut Philosophus loquitur , \textbf{ nihil est aliud , quam quaedam bona dispositio opposita mollicier : ergo in primo gradu dicuntur esse perseuerantes . In secundo vero sunt continentes . } Continere enim est plus , \\\hline
1.2.32 & la qual cosafazen los persseuerates \textbf{ o los que se tienen contra las fuertes passiones } la qual cosa faz en los continentes . Mas son assi castigados en el appetito e en el desseo & quod faciunt perseuerantes : \textbf{ vel contra fortes passiones se tenent , quod faciunt continentes : } sed sunt ita castigati in appetitu , \\\hline
1.2.32 & e los continentesalos non continentes . \textbf{ en essa misma man era los tenprados son contrarios alos destenprados . } Ca assi commo es cosa delectable alos destenprados de mal fazer & et continentes incontinentibus : \textbf{ sic temperati opponuntur intemperatis . } Nam sicut delectabile est intemperatis mala facere , \\\hline
1.2.32 & e fuera de magnera de los otros omes \textbf{ assi ay algers omes } que son diuinales & et sunt mali ultra modum hominum : sic aliqui sunt quasi diuini , \textbf{ et sunt boni ultra modum humanum . } Unde \\\hline
1.2.32 & nin destenprados nin bestiales . \textbf{ Mas conuiene aellos de ser en el mas alto grado de los buenos } por que aquel que dessea de prinçipar e enssenorear alos otros . & nec incontinentes , \textbf{ nec intemperati , nec bestiales , sed oportet eos esse in summo gradu bonorum : } qui enim aliis dominari , \\\hline
1.2.33 & Esto sea dexado en el iuyzio del sabio \textbf{ e parten alguons philosofos } tan bien macrobio commo plotino quatro guados de uirtudes . & ø \\\hline
1.2.33 & que era ganada de los omes \textbf{ por vso de buenas obras } ¶Et pues que assi es siguiendo el camino & Omnem ergo virtutem , \textbf{ quam ponebant , } dicebant esse acquisitam . Sectando ergo Philosophorum viam , \\\hline
1.2.33 & Conuiene a saber que segunt que cada vno es mas altamente bueno \textbf{ ha mas alto grado de uirtudes . } Et pues que assi es assi conmo el philosofo pone quatro linages de buenos commo paresçe & ita quod \textbf{ secundum quod aliquis est excellentior bonus , excellentiorem etiam gradum virtutum habet . } Sicut ergo Philosophus innuit quatuor genera bonorum , \\\hline
1.2.33 & ha mas alto grado de uirtudes . \textbf{ Et pues que assi es assi conmo el philosofo pone quatro linages de buenos commo paresçe } en el capitulo sobredichodo dixo & secundum quod aliquis est excellentior bonus , excellentiorem etiam gradum virtutum habet . \textbf{ Sicut ergo Philosophus innuit quatuor genera bonorum , } ut patet ex capitulo praecedenti , \\\hline
1.2.33 & en el capitulo sobredichodo dixo \textbf{ que algunos eran perseuerates e alås continentes } e algunos tenprados e algunos diuinales . & ut patet ex capitulo praecedenti , \textbf{ qui ait aliquos esse perseuerantes , | aliquos continentes , } aliquos temperatos , \\\hline
1.2.33 & e algunos tenprados e algunos diuinales . \textbf{ Et en essa misma manera podemos departir quatro ordenes de uirtudes } assi que a cada vn linage de los bueons demos su orden de uirtudes . & aliquos continentes , \textbf{ aliquos temperatos , } aliquos vero diuinos , \\\hline
1.2.33 & que los persseuerantes . \textbf{ En essa misma manera las uirtudes exenplares son mas altas } que las uirtudes del coraçon pragado . & continentes , \textbf{ purgatorias : temperati vero habent virtutes purgati animi : } sed diuini habent virtutes exemplares . Propter quod sicut diuini meliores sunt temperatis , temperati continentibus , continentes perseuerantibus : sic virtutes exemplares excellunt virtutes purgati animi : virtutes vero purgati animi excellunt purgatorias : \\\hline
1.2.33 & que son exenplares \textbf{ serie muy mala cosa deñobrar ninguna cosa torpe en ellas } por la qual cosa bien dicho es & Sed in quartis , scilicet exemplaribus , \textbf{ nefas est turpe aliquod nominari . } Quare bene dictum est , \\\hline
1.2.33 & que amollesçen e ordenan el coraçon abien fazer \textbf{ e lo reduzen a medio parte nesçen alos perseuerantes } por que estas tales uirtudes son muy pequanas & idest disponunt animum ad benefaciendum , \textbf{ et reducunt ipsum ad medium , } competunt perseuerantibus . Sunt autem huiusmodi virtutes minime inter virtutes alias : \\\hline
1.2.33 & Et en algunan manera le es delectable de bien fater . \textbf{ Et por ende con grant razon es dicho el tal auer las uirtudes del coraçon pragado } las quales le fazen escaeçer e oluidar las passiones & et quodammodo delectabile est ei benefacere . \textbf{ Merito ergo talis dicitur habere virtutes purgati animi facientes ipsum obliuisci passiones illas crebras , } quia \\\hline
1.2.33 & veyendo la uida \textbf{ e la grant perfeçion del prinçipe e del señor . } por la qual cosanon solamente es mucho de denostar & et cognoscat defectum suum , \textbf{ videns vitam et perfectionem principantis . } Quare apud Reges et Principes non solum detestabile esse \\\hline
1.2.33 & segunt las quales uirtudes \textbf{ assi commo dize plotino aquel philosofo grand pecado es de nonbrar cosas torpes . } Et commo ninguno non pueda ser de tan grand bondat & Bene ergo eis competunt exemplares virtutes , \textbf{ quibus secundum Plotinum nefas est turpia nominari : } sed cum tantae bonitatis nullus esse possit absque Dei gratia , et eius auxilio : \\\hline
1.2.33 & assi commo dize plotino aquel philosofo grand pecado es de nonbrar cosas torpes . \textbf{ Et commo ninguno non pueda ser de tan grand bondat } sin la gera de dios & quibus secundum Plotinum nefas est turpia nominari : \textbf{ sed cum tantae bonitatis nullus esse possit absque Dei gratia , et eius auxilio : } quanto Reges , \\\hline
1.2.33 & tanto mas cobdiçiosamente \textbf{ e con mayor desseo deuen demandar la gera de dios . } ¶ Et pues que & tanto ardentius decet \textbf{ eos diuinam gratiam postulare . } In hoc ergo eliditur Philosophorum elatio , \\\hline
1.2.33 & que dixieron \textbf{ que por prinçipios puros naturales podriemos escusar todos los males } e podriemos ganar bondat acabada . & In hoc ergo eliditur Philosophorum elatio , \textbf{ violentium quod ex puris naturalibus possemus | omnia mala vitare , } et perfectam bonitatem acquirere . \\\hline
1.2.34 & puede ser dicha alguna uirtud . \textbf{ Enpero delas bueans disposiconnes algunas } mas son anexas e siruientes alas uirtudes & omnis bona dispositio mentis possit dici quaedam virtus : \textbf{ attamen bonarum dispositionum quaedam magis sunt annexae , } et adminiculantes virtutibus et disponentes ad virtutem ; \\\hline
1.2.34 & que uirtudes \textbf{ ¶En essa misma manera avn algunas disposiconnes bueans } mas deuen ser dichͣa sobre uirtudes que uirtudes . & quam sint virtutes . \textbf{ Sic etiam quaedam dispositiones bonae magis debent dici supra virtutes quam virtus . } Sed quomodo hoc esset , \\\hline
1.2.34 & Enpero tomando la uirtud \textbf{ assi commo della fablamos aqui en algua manera fallesçen demzon de uirtud } por que assi commo es dich̃o en el segundo libro delas ethicas & eas virtutes appellat . Accipiendo virtutem tamen , \textbf{ ut hic de virtute loquimur , | aliquo modo deficiunt a ratione virtutis . } Nam ( ut dicitur 2 Ethicorum ) \\\hline
1.2.34 & mas de parte dela obra es mayor et meior \textbf{ e ha mayor razon de uirtud } que non aquella que conseia & et iudicata : prudentia , quae immediatius se tenet ex parte operis , est potior , \textbf{ et magis habet rationem virtutis , quam consiliatiua , et iudicatiua , } illae ergo magis sunt adminiculantes \\\hline
1.2.34 & Et por ende bien dicho es \textbf{ que delas bueans disposiconnes del alma algunas son anexas } e siruientes alas uirtudes & Bene ergo dictum est , \textbf{ quod bonarum dispositionum quaedam sunt adminiculantes , } et annexae virtutibus : \\\hline
1.2.34 & Et en qual manera es condicion segniente ala uirtud \textbf{ non es deste presente negoçio } nin parte nesçe anos de tractar dello & vel est quaedam conditio sequens virtutem . Declarare ergo quomodo perseuerantia est dispositio ad virtutem , \textbf{ et quomodo est conditio sequens virtutem , non est praesentis speculationis . } Sufficit autem ad praesens scire , \\\hline
1.2.34 & non es deste presente negoçio \textbf{ nin parte nesçe anos de tractar dello } mas quanto pertenesçe a lo presente abasta de saber & et quomodo est conditio sequens virtutem , non est praesentis speculationis . \textbf{ Sufficit autem ad praesens scire , } quod loquendo de perseuerantia \\\hline
1.2.34 & Mas la continençia e la perseueranca son disposiconnes ala uirtud \textbf{ e la uirtudero ica es ssobre uirtud } Por ende bien dichones & et perseuerantia sint dispositiones ad virtutem : \textbf{ heroica sit supra virtutem : } benedictum est , \\\hline
1.2.34 & Por ende bien dichones \textbf{ que delas bueans dispoliçiones del alma algunas son uirtudes } e algunas son anexas e siruientes alas uirtudes & benedictum est , \textbf{ quod bonarum dispositionum quaedam sunt virtutes , } quaedam annexae virtutibus , \\\hline
1.2.34 & conuiene aellos de conosçer estos linages \textbf{ e estas maneras destas lueans disposiciones . } Es enbargadas ya las dos partes desta obra & TERTIA PARS Primi Libri de regimine Principum : \textbf{ in qua tractatur , | quas passiones Reges et Principes debeant sequi . } Expeditis duabus partibus huius operis , \\\hline
1.3.1 & assi commo dixiemos de suso \textbf{ que eran doze uirtudes } assi podemos dezinr & ideo de his primo tractabimus . Accipiendo autem numerum passionum , \textbf{ sicut dicebamus esse duodecim virtutes , sic dicere possumus quod sunt duodecim passiones : } videlicet , amor , odium , desiderium , abominatio , delectatio , tristitia , spes , desperatio , timor , audacia , ira , et mansuetudo . Computabatur enim supra mansuetudo inter virtutes : \\\hline
1.3.1 & doze conuiene saber amor \textbf{ e mal querençia e desseo . } e aborrençia er delectacion . & ø \\\hline
1.3.1 & e concupiçible que quiere dezir desseador . \textbf{ Et por ende las sobredichas passiones } assi se departen . & et concupiscibilem . \textbf{ Praedictae ergo passiones sic distinguuntur , } quia primae sex videlicet , \\\hline
1.3.1 & El amor ¶ \textbf{ Et la mal querençia ¶ } Et el dessea¶ Et la aborrençia ¶ & amor , \textbf{ odium , desiderium , } abominatio , delectatio , \\\hline
1.3.1 & Et la tristeza pertenesçen al appetito desseador . \textbf{ Mas las otras seys passiones pertenesçen al appetito enssannador } Mas el cuento delas passiones & pertinent ad concupiscibilem ; \textbf{ reliquae vero sex ad irascibilem spectant . Numerus autem passionum concupiscibilis sic accipi potest : } quia omnis passio \\\hline
1.3.1 & por el appetito cobdiçiador . \textbf{ Ca el bien conosçido primera mente nos plaze ¶ } Lo segundo ymos a ello¶ & Nam erga bonum per concupiscibilem tripliciter passionamur . \textbf{ Nam bonum apprehensum } primo nobis placet , secundo tendimus in ipsum : tertio adepto quietamur in eo . \\\hline
1.3.1 & e la delectaçion son tomados en conparaçion de algun bien \textbf{ Mas la mal querençia } e la aborrençia e la tristeza son tomadas en conparaçion de algun mal & Amor ergo , desiderium , \textbf{ et delectatio sumuntur respectu boni . Odium vero , } abominatio et tristitia , \\\hline
1.3.1 & Ca en quanto el mal non nos plaze \textbf{ es mal querençia . } Mas en quanto fuymos deles en nos anborrençia . & sumuntur respectu mali . \textbf{ Nam prout malum nobis displicet , est odium : } prout vero ipsum fugimus , \\\hline
1.3.1 & ¶ Et pues que assi es conmo los nuestros mouimientos del alma \textbf{ et las nr̃as afectiones e passiones non se puedan departir } en mas maneras & Cum ergo non possint pluribus modis variari nostri motus \textbf{ et nostrae affectiones , } in uniuerso duodecim erunt passiones : \\\hline
1.3.1 & en mas maneras \textbf{ que dichas son seran por todas doze passiones las seys } que parte nesçen al appetito cobdiçiador . & et nostrae affectiones , \textbf{ in uniuerso duodecim erunt passiones : } sex pertinentes \\\hline
1.3.2 & orque niguno non puede bien gor̉inar assi mismo \textbf{ si non sopiere quals passiones son de fuyr } e quales son de leguir & Quia nullus bene seipsum regere potest , \textbf{ nisi sciat quae passiones sunt fugiendae , } et quae prosequendae : \\\hline
1.3.2 & por que por todas estas cosas sera \textbf{ mas conosçida anos la naturaleza delas passiones } la qual catada & et quomodo una passio reducitur ad aliam : \textbf{ quia ex omnibus his magis innotescit nobis natura ipsarum passionum , } qua inspecta cognoscere possumus quae sunt laudabiles , \\\hline
1.3.2 & e ayuntamiento podemos dezir \textbf{ que las primeras passiones son amor e mal querençia . } Et en el segundo guado son el desseo et el aborrençia . & dicere possumus primas passiones esse , \textbf{ amor , et odium . In secundo vero gradu sunt desiderium , et abominatio . In tertio vero , spes , } et desperatio . In quarto autem , timor , et audacia . In quinto autem , ira , et mansuetudo . Ultimae autem passiones sunt delectatio , et tristitia . Delectatio enim et tristitia consequuntur ad omnes alias passiones . \\\hline
1.3.2 & Et en el quanto son la sana et la manssedunbre . \textbf{ Mas las postrimeras passiones son delectaçion e tristeza } por que la delectacion e la tristeza se sigue & et desperatio . In quarto autem , timor , et audacia . In quinto autem , ira , et mansuetudo . Ultimae autem passiones sunt delectatio , et tristitia . Delectatio enim et tristitia consequuntur ad omnes alias passiones . \textbf{ Nam quacunque passiones quis passionetur : } vel delectatur , \\\hline
1.3.2 & e puestasen pos todas las otras . \textbf{ Mas el amor e la mal querençia son sinplemente las passiones primeras . } Ca assi commo la delectaçion e la tristeza siguen a todas las o tris passiones & ad omnes alias passiones consequuntur , \textbf{ quia omnes aliae passiones uel tarminantur ad delectationem , } uel ad tristitiam : \\\hline
1.3.2 & Mas el amor e la mal querençia son sinplemente las passiones primeras . \textbf{ Ca assi commo la delectaçion e la tristeza siguen a todas las o tris passiones } por que todas las o tris passiones se determinana del estaçion o a tristeza . & ad omnes alias passiones consequuntur , \textbf{ quia omnes aliae passiones uel tarminantur ad delectationem , } uel ad tristitiam : \\\hline
1.3.2 & Ca assi commo la delectaçion e la tristeza siguen a todas las o tris passiones \textbf{ por que todas las o tris passiones se determinana del estaçion o a tristeza . } En essa misma manera dezimos & quia omnes aliae passiones uel tarminantur ad delectationem , \textbf{ uel ad tristitiam : } sic amor , \\\hline
1.3.2 & por que todas las o tris passiones se determinana del estaçion o a tristeza . \textbf{ En essa misma manera dezimos } que el amor e la mal querençia son primero que las otras passiones & uel ad tristitiam : \textbf{ sic amor , } et odium alias passiones praecedunt . Quare omnes aliae passiones sumunt originem uel ex amore , \\\hline
1.3.2 & En essa misma manera dezimos \textbf{ que el amor e la mal querençia son primero que las otras passiones } por que todas las otras passiones toman rayz o nasçen & sic amor , \textbf{ et odium alias passiones praecedunt . Quare omnes aliae passiones sumunt originem uel ex amore , } uel ex odio . quicunque enim passionatur aliqua passione , \\\hline
1.3.2 & por que todas las otras passiones toman rayz o nasçen \textbf{ de amoro de mal querençia . } Ca qual si quier que sea passionado de alguna passion & et odium alias passiones praecedunt . Quare omnes aliae passiones sumunt originem uel ex amore , \textbf{ uel ex odio . quicunque enim passionatur aliqua passione , } uel sic passionatur , \\\hline
1.3.2 & es passionado della \textbf{ por que ama o por que aborresçe . Mas despues del amor o dela mal querençia han de ser el desseo e la aborrençia . } Ca el desseo sin ningun medio se ayunta luego al amor . & uel sic passionatur , \textbf{ quia amat , uel quia odit . Post amorem autem , et odium , | esse habent desiderium , } et abominatio . \\\hline
1.3.2 & O si la ouieremos desseamos la de guardar en auiendo la . \textbf{ Mas la aborrençia sin ningun medio se ayunta ala mal querençia } por que luego commo queremos mal a alguna persona . & uel si ipsum habemus , \textbf{ desideramus conseruari in habendo ipsum . Abominatio uero immediate innititur odio : } quia statim cum odimus , \\\hline
1.3.2 & Por la qual cosa \textbf{ assi commo el amor e la mal querençia son las primeras passiones } en essa misma manera el desseo e la aborrençia son las segundas & aliquid abominamur illud . \textbf{ Quare sicut amor , | et odium sunt passiones primae : } sic desiderium , \\\hline
1.3.2 & assi commo el amor e la mal querençia son las primeras passiones \textbf{ en essa misma manera el desseo e la aborrençia son las segundas } Maen el tercero logar son de poner la esperançar la desesꝑança . & et odium sunt passiones primae : \textbf{ sic desiderium , } et abominatio sunt passiones secundae . In tertio autem loco ponendae sunt spes , et desperatio . Nam spes , \\\hline
1.3.2 & por si deuen se assi ordenar dizienda \textbf{ que el amor es primero que la mal quetençia ¶ } Et el desseo primo que la aborrençia¶ & sic ordinari debent : \textbf{ quia amor est prior odio : } desiderium abominatione : \\\hline
1.3.2 & Et la delectaçion primero que la tristeza . \textbf{ Et dezimos que el amor es primero que la mal querençia } por que sienpre la passion & desiderium abominatione : \textbf{ spes desperatione : timor audacia : ira mansuetudine : delectatio tristitia . Amor enim est prior odio : } quia semper passio sumpta respectu boni \\\hline
1.3.2 & e la primera passion . \textbf{ Ca la mal querençia tomarays } e nasçençia del amor . & et prima passio . \textbf{ Nam ipsum odium ex amore sumit originem : } nisi amaremus aliquid , \\\hline
1.3.2 & por esso es primero que la aborrençia \textbf{ que se ayunta ala mal querençia . } Mas la esperança es primero que la desesperança . & quae innititur odio . \textbf{ Spes uero praecedit desperationem . } Nam si primum uolitum est bonum , \\\hline
1.3.2 & por ende la passion que va al bien es primera \textbf{ que la passion que fallesce del bien En essa misma guisa } por que fuyr del mal ha razon de bien & ideo passio quae tendit in bonum est prior passione \textbf{ quae deficit ab ipso : } sic quia refugere malum habet rationem boni , \\\hline
1.3.2 & mas parte nesçe alos Reyes e alos prinçipes \textbf{ en quanto por las passiones dellos mayor mal puede venir } o mayer bien & Quod scire tanto magis decet Reges et Principes , \textbf{ quanto per passiones ipsorum maius valet induci malum , } et potest bonum excellentius impediri . \\\hline
1.3.3 & en el nuestro gouernamiento \textbf{ e en lanr̃a uida } por ende escoła neçesaria de mostrar & Passiones autem quia diuersificant regnum et vitam nostram , \textbf{ ideo necessarium est ostendere } quomodo nos habere debeamus ad illas . \\\hline
1.3.3 & por qual orden determinariemos dellas . \textbf{ Por la qual cosa commo el amor e la mal querençia sean las primeras passiones } primero deuemos ver & ut sciremus quo ordine determinaremus de illis . \textbf{ Quare cum amor , | et odium sint passiones primae , } prius videndum est , \\\hline
1.3.3 & en qual manera conuiene alos Reyes \textbf{ et alos prinçipes de ser amadores e de ser mal queredores . } Et por que entendamos la fuerca del amor & quomodo deceat Reges et Principes esse amatiuos , \textbf{ et oditiuos . } Et ut ostendamus nomen amoris , \\\hline
1.3.3 & que en si mesmo \textbf{ por que el bien diuinales guardador del nuestro bien . } Ante si el nuestro bien fuese destroydo & Nam in bono diuino magis habet esse bonitas uniuscuiusque , \textbf{ quam etiam in seipso . Bonum enim diuinum est conseruatiuum boni nostri : } immo \\\hline
1.3.3 & assi mismo fazer bueno o guardar \textbf{ assymismo en bondat . La razon natural muestra } que mas deue amar el omne el bien diuinal & vel se in bonitate conseruare , dictat \textbf{ naturalis ratio } ut magit diligat Deum quam seipsum : \\\hline
1.3.3 & por que non sean los mienbros llagados \textbf{ enlos quales esta la salud comun de todo el cuerpo prinçipal mente . } Et esto por inclinaçion natural pone el braço a periglo & ex naturali enim instinctu cum quis vult percuti , \textbf{ ne vulnerentur membra a quibus principaliter dependet salus corporis , } et ne totum corpus pereat , \\\hline
1.3.3 & por que todo el cuerpo non ꝑesca . \textbf{ Et en essa misma manera } avn si cataremos al tp̃o & brachium periculo se exponit . \textbf{ Sic etiam antiquitus } si perspeximus ciuitatem aliquam dominari \\\hline
1.3.3 & por tres maneras \textbf{ segunt que la real magestad } quanto parte nesçe alo presente puede ser conparada a tres cosas . & et Principes , quod triplici via declarare possumus . \textbf{ Regi enim dignitas } ( quantum ad praesens ) \\\hline
1.3.3 & del tirano \textbf{ ala qual es contraria la real magestad . } Et alas uirtudes & ad tria comparari potest scilicet ad tyrannidem , \textbf{ cui contrariatur : } ad virtutes , \\\hline
1.3.3 & Et alas uirtudes \textbf{ por las quales deue ser honrrada la real magestad } e alos males e pecados & cui contrariatur : \textbf{ ad virtutes , | quibus debet ornari : } et ad vitia , \\\hline
1.3.3 & e alos males e pecados \textbf{ de que deue fuyr la real magestad } ¶la primera razon se praeua assi . & et ad vitia , \textbf{ quae debet fugere . Prima via sic patet : } nam \\\hline
1.3.3 & Ca assi commo es dicho de suso \textbf{ e assi commo el philosofo lo praeua en las politicas difetençia e deꝑtimiento es entre el Rey e el tyra non } por que el Rey prinçipalmente entiende el bien comun de todos . & ( ut superius dicebatur , \textbf{ et ut Philosophus in Polit’ probat ) | differentia est } inter Regem , \\\hline
1.3.3 & que el bien propre o \textbf{ Mas avn por que en espeçial manera el rey } e cada vn prinçipe es ofiçial de dios & ut bonum commune praeponat priuato bono . \textbf{ Immo quia speciali modo Rex } et quilibet principans est minister Dei \\\hline
1.3.3 & en que deue dar sçiençia alos otros . \textbf{ En essa misma manera es mas de denostar el Rey en fallesçer en . } las uirtudes que los subditos & quia magister est in statu in quo ipso debet scientiam aliis tradere : \textbf{ sic detestabilius est in Rege carere virtutibus , } quam in subditis , \\\hline
1.3.3 & que resplandezca por pradençia e por sabiduria . \textbf{ quanto mayor pradençia e mayor sabiduria es meester } para guardar el bien comun & ut prudentia polleat , \textbf{ quanto maior prudentia requiritur ad custodiendum bonum commune , } quam proprium . \\\hline
1.3.3 & por que los bienes comunes son muy mas altos \textbf{ e mas dignos de grand honrra } en la qual honrra entiende el magnanimo . & quia bona communia maxime sunt ardua \textbf{ et magno honore digna , } in quae magnanimus tendit . Erit magnificus ; \\\hline
1.3.3 & e fazen e vsan de toda iniustiçia . \textbf{ Onde ualerio maximo cuenta de dionsio seziliano } que commo fuesse tyrano & omnem iniustitiam exercent . \textbf{ Unde et Valerius Maximus de Dionysio Ciciliano recitat , } qui cum esset tyrannus , \\\hline
1.3.3 & que commo fuesse tyrano \textbf{ e amador de propio prouecho despoblaua } e despoiaua las çibdades & qui cum esset tyrannus , \textbf{ erat amator proprii commodi , depopulabat urbes , expoliabat ciuitates , } et depraedabat sacra . Viso quomodo Reges et Principes \\\hline
1.3.3 & Ca nal e comun de ligero puede paresçer \textbf{ en qual manera se de una auer los Reyes ala mal querençia . } por que el amor es el primero mouimiento & de facili patere potest , \textbf{ quomodo se habere debeant ad odium . } Nam amor est primus motus \\\hline
1.3.3 & que tuelle la uida ¶ \textbf{ Et pues que assi es el temor e la mal querençia } e breue mençe toda pasion & qui eam tollit . \textbf{ Timor ergo | et odium , } et breuiter \\\hline
1.3.3 & o todos los mouimiento del coraçon toman rayz e nasçençia del amor \textbf{ Et pues que assi es la prinçipal entençion de cada vno deue ser } que cosa ha de amar & omnis passio siue omnis motus animi ex amore sumit originem ; \textbf{ potissimum ergo in intentione cuiuslibet esse debet } quid amandum . Ostenso ergo quomodo Reges et Principes quodam speciali modo \\\hline
1.3.3 & Et generalmente todos males e todos pecados \textbf{ por la qual cosa commo de razon dela mal querençia sea matar } e nunca se fartar & et uniuersaliter omnia vitia . \textbf{ Quare cum de ratione odii sit exterminare , } et nunquam satiari nisi exterminet , \\\hline
1.3.4 & que los otros . \textbf{ icho del amor e dela mal querençia } que son las primeras passiones finca de dezir & et communis . \textbf{ Dicto de amore , | et odio , } quae sunt passiones primae : \\\hline
1.3.4 & que ha diferençia e departimiento entre el desseo e el amor \textbf{ e entre la aborençia e la mal querençia . } Ca los fechͣs e las obras morales son semeiables en alguna manera alas cosas naturales . & Differt autem desiderium ab amore : \textbf{ et abominatio differt ab odio . } Nam gesta moralia quodammodo rebus naturalibus sunt similia . \\\hline
1.3.4 & e el fuego va açima a su logar . \textbf{ En essa misma manera } por el amor va cada vno al bien & ut per grauitatem vel per leuitatem quandam tendunt in loca propria : \textbf{ sic per amorem tendit quis in bonum sibi proportionatum } et conueniens . In grauibus ergo \\\hline
1.3.4 & Lo tercero la stacion et la folgera por la qual fuelgan en aquel lugar . \textbf{ En essa misma manera en las obras morales } assi commo todos dizen comunalmente deuemos penssar estas tres cosas . & per quam quiescunt in dicto loco . \textbf{ Sic et in gestis moralibus , } ut communiter ponitur , \\\hline
1.3.4 & assi commo dessuso dixiemos . \textbf{ assi de uemos entender del mal en conparaçion de la mal querençia } e dela aborrençia e dela tristeza & ut supra tangebatur , \textbf{ intelligendum est de malo respectu odii , } abominationis , \\\hline
1.3.4 & Et en quanto es ya ganado delectamos nos enel . \textbf{ En essa misma manera el mal } en quanto es mal & ut est desideratum , \textbf{ in ipsum tendimus : } ut est adeptum , \\\hline
1.3.4 & enpero deue resçebir medida e manera del amor . \textbf{ En essa misma manera commo } quier que la aborrençia non sea esso mismo & et si contingat ipsum adipisci , \textbf{ dolemus et tristamur . Desiderium ergo licet non sit idem quod amor , mensuram tamen et modum debet suscipere ex amore . Sic } et abominatio licet non sit idem quod odium , \\\hline
1.3.4 & quier que la aborrençia non sea esso mismo \textbf{ que la mal querençia . } Empero deue resçebir medida e mesura dela mal querençia . & dolemus et tristamur . Desiderium ergo licet non sit idem quod amor , mensuram tamen et modum debet suscipere ex amore . Sic \textbf{ et abominatio licet non sit idem quod odium , } tamen ex odio debet mensuram suscipere . \\\hline
1.3.4 & que la mal querençia . \textbf{ Empero deue resçebir medida e mesura dela mal querençia . } Ca assi commo en las cosas naturales beemos & et abominatio licet non sit idem quod odium , \textbf{ tamen ex odio debet mensuram suscipere . } Sic enim in naturalibus videmus , \\\hline
1.3.4 & Et por ende el fisico non entiende \textbf{ quanto mayor melezina o sanguaa puede dar } ca luego matarie al entermo . & potionem autem et phlebotomiam non intendit \textbf{ quanto maiorem potest , } quia tunc exterminaret infirmum : \\\hline
1.3.4 & si el desseo deue tomar mesura del amor \textbf{ Conuiene que los Reyes e los prinçipes desse en prinçipalmente el buen estado del regno } assi que todos quantos son en el regno se ayan bien alas cosas diuinales & si desiderium debet mensuram sumere ex amore , \textbf{ principaliter Reges et Principes debent desiderare bonum statum regni : } ut quod qui in regno sunt , \\\hline
1.3.4 & commo estas delas quales prinçipalmente \textbf{ e por si et essençialmente nasçe el buen estado del regno . } Mas las otras cosas deuen los reyes dessear & et caetera talia , \textbf{ a quibus per se et essentialiter dependet bonus status regni . } Alia autem desiderare debent , \\\hline
1.3.4 & por las cosas desiguales e malas . \textbf{ Et fazer o tris cosas tales delas quales nasçe } e cuelga la salud del regno ¶ & et caetera talia bona in tantum desiderare debent , \textbf{ inquantum per ea possunt cohercere malos , punire iniusta , et facere talia , } a quibus regni salus dependere videtur . Viso , quae , \\\hline
1.3.4 & en quanto son ordenadas a estas . \textbf{ Et essa misma manera primeramente } e prinçipalmente deuen dessear el bien diuinal & ut ordinantur ad ista : \textbf{ sic desiderare debent primo } et principaliter bonum diuinum et commune , \\\hline
1.3.4 & e comunal . \textbf{ mas las o tris cosas deuen dessear } en quanto son ordenadas a estas ¶ & et principaliter bonum diuinum et commune , \textbf{ alia autem sunt desideranda ut ordinantur ad ista : } de leui patere potest quae \\\hline
1.3.4 & de auer cuydado del regno e del bien comun . \textbf{ Mas quales cosas son aquellas que guardan el regno en buen estado } e en qual manera el Rey se deue auer a su regno & et communi . \textbf{ Quae sunt autem illa quae regnum conseruant in bono statu , } et quomodo Rex se debeat habere ad ipsum regnum , \\\hline
1.3.5 & por que los magninimos \textbf{ por grant graueza } que sea en la obra non desesperan . & quia humiles cognoscentes defectum proprium , non sperant ultra quam debeant : magnanimitas vero reprimat desperationem , \textbf{ quia magnanimi propter difficultatem operis non desperant : } si Reges et Principes fuerint humiles \\\hline
1.3.5 & por la humildat que ha en ellos . \textbf{ Mas nos podemos mostrar en quatro maneras } que conuiene alos Reyes e alos prinçipes & et non sperabunt non speranda propter humilitatem . Possumus autem quadrupliciter ostendere , \textbf{ quod deces Reges et Principes decenter se habere circa spem , } et sperare speranda , \\\hline
1.3.5 & e non acometiessen ninguna cosa \textbf{ serian de flacos coraçones } e non tractarian conueniblemente los negoçios & et nihil aggrederentur , \textbf{ essent pusillanimes , } et non debite pertractarent negocia regni . Sunt autem in spe , \\\hline
1.3.5 & assi commo dizen todos \textbf{ comunalmente son quatro cosas de penssar } por las quales podemos mostrar que conuiene alos Reyes & et non debite pertractarent negocia regni . Sunt autem in spe , \textbf{ ut communiter ponitur , quatuor consideranda , } propter quae arguere possumus , quod decet Reges et Principes esse bene sperantes . \\\hline
1.3.5 & ¶ \textbf{ Et estas quatro cosas conuiene saber El bien . } Et el bien alto e guaue Et el futuro que ha de ser . & quia circa impossibile nullus sperat , sed desperat . \textbf{ Haec autem quatuor , | videlicet , } bonum , arduum , \\\hline
1.3.5 & Et parte nesçe a ellos de es par algun bien . \textbf{ Otrosi por que la prinçipal entençion del fazedor delas leyes e del rey } assi commo mostramos de suso deue ser el bien diuinal e comunal & spectat ad eos sperare bonum . \textbf{ Rursus quia principale intentum a legislatore et a lege , } ut supra ostendimus , \\\hline
1.3.5 & tantomas cosas le pueden auenir e contesçer . \textbf{ Por ende ha menester de seer de mayor prouidençia } e de mas sano consseio & tanto plura possunt ei contingere , \textbf{ ideo maiori indiget prouidentia } et saniori consilio : \\\hline
1.3.5 & Por ende ha menester de seer de mayor prouidençia \textbf{ e de mas sano consseio } Et pues & ideo maiori indiget prouidentia \textbf{ et saniori consilio : } cum ergo prouidentia \\\hline
1.3.5 & e aquellos que non son nobles \textbf{ si non son magnanimos e de grant coraçon } e si se tiran de algunos bienes altos e grandes meresçen perdon por que el poderio ciuil & ut possibilia . Nam pauperes , impotentes et ignobiles \textbf{ si non sunt magnanimi , } et subtrahunt se ab aliquibus bonis arduis , \\\hline
1.3.5 & e el poderio çiuil e abondança de riquezas non se pueden escusar \textbf{ que non sean de flacos coraçones } si non creyeren & videntur mereri indulgentiam , \textbf{ quia ciuilis potentia , diuitiae , } et nobilitas non adminiculantur eis , \\\hline
1.3.5 & si non creyeren \textbf{ que ellos pueden alcançar tan grandes bienes } e tan dignos de grand honrra . & et nobilitas non adminiculantur eis , \textbf{ ut possint prosequi talia bona : } Reges autem et Principes , \\\hline
1.3.5 & que ellos pueden alcançar tan grandes bienes \textbf{ e tan dignos de grand honrra . } Por la qual cosa commo los Reyes & et nobilitas non adminiculantur eis , \textbf{ ut possint prosequi talia bona : } Reges autem et Principes , \\\hline
1.3.5 & Ca conuiene alos Reyes de cuydar \textbf{ e escodinar con grand diligençia } que cosa es aquello que deuen esparar & restat videre quomodo se habere debeant in non sperando non speranda . \textbf{ Decet enim eos cum magna diligentia inuestigare , } quid sperent , \\\hline
1.3.5 & nin alcançar ¶ \textbf{ En elsa misma manera avn podemos dezir } que los beddos mas esperan & prorumpunt \textbf{ ut attentent aliqua quae consumate non possunt . Sic etiam dicere possemus , } quod ebriosi plus sperant quam debent : \\\hline
1.3.5 & e todo el regno a peligros deuen los Reyes \textbf{ e los prinçipes con grand diligençia } e con conseio grande e prolongado cuydar & diuturno consilio \textbf{ et magna diligentia excogitare debent Reges et Principes } quid aggrediantur , \\\hline
1.3.6 & assi commo dize el philosofo en el primero libro de los fisicos . \textbf{ por ende en este primero libro conuiene de tractar delas costunbres de lons Reyes } uniuersalmente e generalmente e en semeiança . & ut dicitur 1 Physicorum , \textbf{ deo in hoc primo de moribus Regum oportet pertransire uniuersaliter typo : } quia in secundo , \\\hline
1.3.6 & e en los palaçios mayormente los Reyes e los prinçipes \textbf{ por grand muchedunbre delos negoçios } e de los fechos & quia experti in curiis , et maxime Reges \textbf{ et Principes propter eorum negociorum frequentiam , } satis particularia gesta nouerunt . \\\hline
1.3.6 & por la qual cosasi en este negoçio moral o en este libro \textbf{ que tracta delas constunbres nos dieremos a ellos alguas reglas generales } con la praeua e con la esperiençia & satis particularia gesta nouerunt . \textbf{ Quare si aliqua uniuersalia in morali negocio eis tradantur , } suffragante experientia quam habent de moralibus gestis , \\\hline
1.3.6 & e dichͣs estas cosas de suso digamos \textbf{ que assi commo general mente } e en figera & His ergo praelibatis dicamus \textbf{ quod sicut uniuersaliter et typo } instruximus Reges , \\\hline
1.3.6 & que ha de venir . \textbf{ En essa misma manera seg̃t essa misma sciençia los podemos ensseñar } en qual manera se de una auer çerca la osadia & sic \textbf{ secundum eandem methodum eos instruere possumus , } quomodo se habere debeant contra timorem , \\\hline
1.3.6 & por que tales cosas \textbf{ commo estas paresçen ser contrarias ala Real magestad . } Ca los Reyes e los prinçipes suelen auer muchos & quae respiciunt futurum malum . Videtur autem forte aliquibus Reges , \textbf{ et Principes in nullo debere esse timidos , quia talia regiae maiestati derogare dicuntur . Multos autem sic incitantes Reges habere consueuerunt , } persuadentes eis \\\hline
1.3.6 & que assi es commo todo vn regno non pueda ser conueniblemente gouernado \textbf{ sin grand conseio } conuiene alos Reyes & quod dubitat . \textbf{ Cum ergo unum totum regnum absque magno consilio debite gubernari non possit , } expedit Principibus , et Regibus ut consiliatiui reddantur , habere aliquem moderatum timorem . Secundo hoc idem inuestigare possumus \\\hline
1.3.6 & Mas el temortenprado \textbf{ non solamente faze alos Reyes tomadores de conseio mas faze avn que fagan las obras mas acuçiosa mente . } Ca si nos viniere algun temor tenprado & Moderatus autem timor non solum consiliatiuos facit , \textbf{ sed etiam agit ut opera diligentius operemur . } Nam si moderatus adsit timor , \\\hline
1.3.6 & t te paresçe \textbf{ que el temor destenprado ha quatro cosas } que del todo ponen mengua en el gouernamiento del regno . & quod decet Reges , et Principes moderatum habere timorem . \textbf{ Attamen immoderate timere nullo modo decet eos . Immoderatus enim timor quatuor habere videtur , } quae omnino derogant regno . \\\hline
1.3.6 & luego fuyen alos castiellos e alas torres . \textbf{ En essa misma manera } quando alguno teme la calentura natural & Cum enim homines existentes in campis timent , \textbf{ statim confugiunt ad castrum , } vel ad arcem : \\\hline
1.3.7 & or que la saña e la yra paresçe \textbf{ que ha grant ayuntamiento } e grand vezindat con la mal querençia . & Quia ira maximam affinitatem videtur habere cum odio , \textbf{ Prius quam ostendamus , } quomodo Reges , \\\hline
1.3.7 & que ha grant ayuntamiento \textbf{ e grand vezindat con la mal querençia . } primeronte que mostremos en qual manera los reyes e los prinçipes se deuen auer cerca la sanna & Quia ira maximam affinitatem videtur habere cum odio , \textbf{ Prius quam ostendamus , } quomodo Reges , \\\hline
1.3.7 & e çerca la mansedunbre deuemos \textbf{ veeren qual manera la sanera se departe dela mal querençia . } Et qual cosa es mas desconuenible & et Principes se habere debeant circa iram , et mansuetudinem . \textbf{ Videndum est quomodo ira differat ab odio : } et quod est detestabilius \\\hline
1.3.7 & Et qual cosa es mas desconuenible \textbf{ e peor o la mal querençia o la saña desordenada ¶ } Mas la prinçipal diferençia e departimiento entre la yra e la malquerençia . & et quod est detestabilius \textbf{ an odium , } an ira inordinata . Est autem una principalis differentia inter iram , \\\hline
1.3.7 & e peor o la mal querençia o la saña desordenada ¶ \textbf{ Mas la prinçipal diferençia e departimiento entre la yra e la malquerençia . } es que la mal querençia es appetito & an odium , \textbf{ an ira inordinata . Est autem una principalis differentia inter iram , | et odium : } quia odium est appetitus mali simpliciter , \\\hline
1.3.7 & Mas la prinçipal diferençia e departimiento entre la yra e la malquerençia . \textbf{ es que la mal querençia es appetito } e desseo de mal sinplemente & et odium : \textbf{ quia odium est appetitus mali simpliciter , } et absolute . Nam odium opponitur amori : amare autem \\\hline
1.3.7 & e desseo de mal sinplemente \textbf{ e sueltamente Ca la mal querençia es contraria al amor } porque assi commo dize el philosofo en el segundo libro dela rectorica . & quia odium est appetitus mali simpliciter , \textbf{ et absolute . Nam odium opponitur amori : amare autem } ( \\\hline
1.3.7 & porque assi commo dize el philosofo en el segundo libro dela rectorica . \textbf{ amares essa misma cosa } que querera alguno algun bien & ( \textbf{ ut dicitur 2 Rhetor’ ) est idem quod velle alicui bonum } secundum se . Sic odire aliquem est velle malum ei simpliciter , \\\hline
1.3.7 & que querera alguno algun bien \textbf{ segunt si En essa misma manera querer mala alguna cosa esquerer } que luenga algun mal siplemente e suelta mente . & ut dicitur 2 Rhetor’ ) est idem quod velle alicui bonum \textbf{ secundum se . Sic odire aliquem est velle malum ei simpliciter , } et absolute . Ira autem non sit : \\\hline
1.3.7 & e declarar que la saña es appetito de pena ordenada ala uengança \textbf{ Et desta diferençia prinçipalmente la san na e la mal querençia le toman ocho disterençias } entre ellas & quod est appetitus poenae in vindictam . \textbf{ Ex hac autem differentia principali inter iram et odium , | sumuntur octo differentiae , } quas assignat Philos’ \\\hline
1.3.7 & o de aquellas cosas que parte nesçen assi mismo . \textbf{ Mas la mal quetençia puede ser delas cosas } que parte nesçen & quia est ira ex iis quae sunt ad seipsum , \textbf{ vel ex pertinentibus ad ipsum . Odium autem esse potest de pertinentibus ad ipsum , } et ad alium . \\\hline
1.3.7 & La segunda diferençia es que la sanna sienpre es en vna cosa singular . \textbf{ mas la mal querençia puede ser en comun } por que alguno puede mal queter todo ladron o todo retrahendor de mal comunal mente . & quia ira semper est in singulari : \textbf{ odire potest esse in communi . Odire autem potest aliquis communiter omnem furem , } et detractorem : \\\hline
1.3.7 & mas la mal querençia puede ser en comun \textbf{ por que alguno puede mal queter todo ladron o todo retrahendor de mal comunal mente . } Mas enssannar se non puede & quia ira semper est in singulari : \textbf{ odire potest esse in communi . Odire autem potest aliquis communiter omnem furem , } et detractorem : \\\hline
1.3.7 & ¶ La terçera differençia es \textbf{ que la mal querençia es tal cosa que se non farta . } Mas la sanna es tal cosa & Tertia differentia est , \textbf{ quia odium est | quid insatiabile : } sed ita satiatur \\\hline
1.3.7 & que se farta . \textbf{ Por que si la mal querençia es apetito de mal sinplemente } aquel que nos mal queremos & sed ita satiatur \textbf{ nam si odium est appetitus mali simpliciter , } ille quem odimus , non posset tantum habere de malo , quin vellemus quod haberet plus . \\\hline
1.3.7 & que se farta \textbf{ Por que quando alguno tanto mal sufre } que paresçe al sannudo & est quid satiabile : \textbf{ quia est aliquis tantum passus est , } quod videatur irato ultionem decentem factam esse , satiatur ira , et quiescit iratus . \\\hline
1.3.7 & Ca el sannudo quiere dar dolor e tsteza \textbf{ mas el mal quariente quiere fazer danno e enpeçemiento¶ } La quinta diferençia es & Vult enim iratus inferre dolorem , et tristitiam : \textbf{ sed odiens vult inferre damnum , } et nocumentum . Quinta differentia est , \\\hline
1.3.7 & Mas por çierto el que quiere mal a otro non cura desto . \textbf{ Ca commo la mal querençia sea algun mal segunt si } e sueltamente abasta el mal quariente & Sed odienti quidem nihil differt : \textbf{ nam cum odium sit mali | secundum se et absolute , } sufficit odienti , \\\hline
1.3.7 & Ca commo la mal querençia sea algun mal segunt si \textbf{ e sueltamente abasta el mal quariente } que el otro padezra mal & secundum se et absolute , \textbf{ sufficit odienti , } quod alter patiatur malum , \\\hline
1.3.7 & que fasta que sea fecha aquella uegaça esta sienpre commo en tristeza continuadamente . \textbf{ Mas la mal querençia puede ser sintsteza } por que la mal querençia puede ser contra alguna cosa en comun . & quod donec sit facta ultio , \textbf{ quasi continue est in tristia . Sed odium sine tristitia esse potest : } nam odium esse valet ad aliquid in communi . Odire enim possumus uniuersaliter omnes fures : \\\hline
1.3.7 & Mas la mal querençia puede ser sintsteza \textbf{ por que la mal querençia puede ser contra alguna cosa en comun . } Ca nos podemos natanlmente querer mal a todos los ladrones . & quasi continue est in tristia . Sed odium sine tristitia esse potest : \textbf{ nam odium esse valet ad aliquid in communi . Odire enim possumus uniuersaliter omnes fures : } non tamen oportet , \\\hline
1.3.7 & pero non conuiene de temer \textbf{ quetsteza se aconpanne a esta mal querençia . } ¶ La septima diferençia es & non tamen oportet , \textbf{ quod tristitia committetur huiusmodi odium . } Septima differentia est : \\\hline
1.3.7 & que sea ayuntada a misericordia . \textbf{ Mas la mal querençia non } Porque commo la sanna se pueda fartar & quia irae videtur esse annexa misericordia , \textbf{ non autem odio . } Nam cum ira satietur , \\\hline
1.3.7 & contra quien ha sana el sañudo apiadasse del \textbf{ Mas la mal querençia de ninguno non se apiada } por que es cosa que se non farta . & si multa mala inferantur alteri , \textbf{ iratus miseretur ei sed odium pro nullo miserebitur , } cum sit quid insatiabile . Octaua differentia est : \\\hline
1.3.7 & fasta que sea fecha uengança conuenible . \textbf{ Mas la mal querençia mata } e quiere que non sea aquel aqui quier mal . & donec fiat condigna ultio . \textbf{ Sed odium exterminat , } et vult non esse : \\\hline
1.3.7 & e quiere que non sea aquel aqui quier mal . \textbf{ Ca non le abasta al mal quariente } que el otro padezca mal & et vult non esse : \textbf{ non enim sufficit odienti , quod alter contra patiatur , } sed vult eum interimi \\\hline
1.3.7 & e non sea . \textbf{ Et pues que assi es commo las condiconnes dela mal querençia } sean mucho peores & et non esse . \textbf{ Cum ergo conditiones odii sint multo peiores , } quam conditiones irae , \\\hline
1.3.7 & sean mucho peores \textbf{ que las condiconnes dela saña . Mas nos deuemos guardar dela mal querençia } que dela sanna ante segunt que dize sat̃ agostin la saña passar se en mal querençia & Cum ergo conditiones odii sint multo peiores , \textbf{ quam conditiones irae , | magis cauendum est odium quam ira . } Immo iram transire in odium \\\hline
1.3.7 & que las condiconnes dela saña . Mas nos deuemos guardar dela mal querençia \textbf{ que dela sanna ante segunt que dize sat̃ agostin la saña passar se en mal querençia } esto es de vna paia fazer ugalagar ¶ & magis cauendum est odium quam ira . \textbf{ Immo iram transire in odium | secundum Augustinum , } hoc est , \\\hline
1.3.7 & esto es de vna paia fazer ugalagar ¶ \textbf{ Et pues que assi es la mal querençia es de esquiuar en toda manera alos Reyes e alos principes } por que podrien fazer mucho danno & trabem facere de festuca . \textbf{ Est ergo huiusmodi odium cauendum a quolibet . | Magis tamen cauendum est Regibus , et Principibus : } quia inferre possunt pluribus nocumentum . Sic igitur sentiendum est de odio et ira : \\\hline
1.3.7 & e mucho mal a muchos . \textbf{ Et por ende en esta manera deuemos sentir dela mal querençia } e dela sana & Magis tamen cauendum est Regibus , et Principibus : \textbf{ quia inferre possunt pluribus nocumentum . Sic igitur sentiendum est de odio et ira : } quia odium detestabilius est , quam ira . Nihilominus tamen ira si inordinata fit , detestabilis est . \\\hline
1.3.7 & e dela sana \textbf{ que la mal querençia es cosa mas desconuenible quela saña . } Empero quando la sana fuere desordenada sera cosa muy desconuenible . & quia inferre possunt pluribus nocumentum . Sic igitur sentiendum est de odio et ira : \textbf{ quia odium detestabilius est , quam ira . Nihilominus tamen ira si inordinata fit , detestabilis est . } Ut ergo appareat , quomodo Reges , \\\hline
1.3.7 & e cerca la mansedunbre conuiene de saber \textbf{ que la sanna algunans vezes va ante la razon e ante el entendimiento . } Et estonçe es desorde nada & et tunc est inordinata et cauenda , \textbf{ aliquando sequitur ordinem rationis , } et tunc potest esse inordinata , \\\hline
1.3.7 & en qual manera se auia de conplir aquel mandamiento . \textbf{ En essa misma manera avn los canes } luego que oy en el sueno de aquel que viene & quia non perfecte perceperunt quomodo exequendum sit mandatum illud . \textbf{ Sic etiam | et canes } statim cum audiunt sonitum venientis , latrant , \\\hline
1.3.8 & que las delecta connes \textbf{ e las tristezas tienen el postrimero grado } en la orden delas passiogones & delectationes , \textbf{ et tristitias tenere ultimum gradum in ordine passionum : } quia ad eas , tanquam ad passiones ultimas , \\\hline
1.3.8 & por el philosofo en el quarto libro delas ethicas \textbf{ En essa misma manera el que pone } que toda delectaçiones de esquiuar & ut patet per Philos 4 Metaphy’ ) \textbf{ sic ponens omnem delectationem esse fugiendam , } ponit aliquam delectationem esse prosequendam . \\\hline
1.3.8 & e pone la fabla . \textbf{ En essa misma manera } por que ninguno non puede foyr toda delectaçion & quare negando loquelam , \textbf{ concedit loquelam . Sic quia nullus omnem delectationem fugit , } nisi delectabile sit ei omnem delectationem fugere ; \\\hline
1.3.8 & nin es toda delectaçion sinplemente mala . \textbf{ Mas alg̃ delectaçion es buena } e esta bien . & nec omnis simpliciter mala . \textbf{ Sed aliqua delectatio est bona existenter , } aliqua apparenter , aliqua simpliciter , \\\hline
1.3.8 & assi commo los ssanos . \textbf{ En essa misma manera algunos han el appetito corrupto } assi commo los uiçiosos e los malos & ut sani . \textbf{ Sic aliqui habent appetitum infectum , } ut vitiosi , \\\hline
1.3.8 & e aquellos que han las lenguas bien ordenadas \textbf{ e en buena disposicion . } Et essa misma manera non son de dezir uerdaderamente cosas delectables & quae videntur dulcia infirmis , et habentibus gustum infectum : sed quae videntur dulcia sanis , \textbf{ et habentibus linguam bene dispositam . } Sic non sunt dicenda vere delectabilia , \\\hline
1.3.8 & e en buena disposicion . \textbf{ Et essa misma manera non son de dezir uerdaderamente cosas delectables } aquellas que son delectables alos uiçiosos e alos malos & et habentibus linguam bene dispositam . \textbf{ Sic non sunt dicenda vere delectabilia , } quae sunt delectabilia vitiosis , \\\hline
1.3.8 & e segunt alguna parte . \textbf{ Et pues que assi es en essa misma manera alguas delecta connes son bueans uerdaderamente e sinplemente } e alguas segunt el paresçer & secundum quid . \textbf{ Sic erunt delectationes bonae existenter et simpliciter , } aliquae vero apparenter \\\hline
1.3.8 & Por ende commo algunas cosas conuengan alas bestias \textbf{ e algunos alos omes algunans delectaçiones seran conuenientes alas bestias } e algunas seran conuenientes alos omes . & Rursus quia delectatio contingit ex coniunctione conuenientis cum conuenienti : \textbf{ cum ergo alia conueniant bestiis , alia hominibus : } aliquae delectationes sunt conuenientes bestiis , aliquae vero hominibus . Delectationes autem intelligibiles \\\hline
1.3.8 & Mas aquellas cosas que son delectables alos omes razonables \textbf{ e de buen entendimiento } e alos omes uirtuosos . & et hominibus vitiosis : \textbf{ sed quae sunt delectabilia rationabilibus , } et hominibus virtuosis . Omnis igitur delectatio bona est , \\\hline
1.3.8 & non gozan de ssi mismos . \textbf{ Et por ende grand remedio es anos } que en nos mismos fallemos delectaçion . & de seipsis non gaudent . \textbf{ Magnum ergo remedium , } et ut in nobis ipsis delectationem inueniamus , \\\hline
1.3.8 & El segundo remedio es la consolaçion de los amigos . \textbf{ assi commo en esse mismo nono de las ethicas dize el philosofo } por qui paresçe & Secundum remedium est consolatio amicorum , \textbf{ ut in eodem 9 Ethic’ traditur , } videtur enim tristitia esse quoddam pondus aggrauans animam . \\\hline
1.3.8 & menos nos aguauiamos del peso . \textbf{ en essa misma manera quando ueemos muchedunbre de amigos } que se duelen connusco & Sicut ergo in pondere corporali cum multi iuuant nos \textbf{ ad portandum illud , minus grauamur : sic cum videmus multitudinem amicorum condolore nobis , } alleuiamur a dolore illo , \\\hline
1.3.8 & mas por que ueemos aellos doler \textbf{ se engendrase en nos vna firme fantasia } que son nros amigos . & non quia ipsi dolent de dolore nostro minuitur dolor noster , \textbf{ sed quia videmus eos dolere , adgeneratur nobis quaedam firma fantasia quod sint amici : } et quia delectabile est habere amicos , delectamur : et delectando , \\\hline
1.3.8 & se engendrase en nos vna firme fantasia \textbf{ que son nros amigos . } Et por que es cosa delectable auer amigos delectamos nos & sed quia videmus eos dolere , adgeneratur nobis quaedam firma fantasia quod sint amici : \textbf{ et quia delectabile est habere amicos , delectamur : et delectando , } minuitur dolor noster : \\\hline
1.3.8 & Ca por esta tal consideraçion conosçemos \textbf{ que tales bienes commo estos son muy pequa nons bienes . } Et por ende aquellos perdudos non nos dolemos dellos sinon por auentura por accidente alguno & maximum remedium est consideratio veritatis . \textbf{ Nam per huiusmodi considerationem cognoscimus talia esse modica bona : ideo eis amissis non dolebimus , } nisi \\\hline
1.3.8 & sinon delas cosas torpes \textbf{ e delans obras uiçiosas e malas . } Mas si el dolor fuere & nisi de turpibus , \textbf{ et non de operibus virtuosis . } Si autem propter alia dolor , \\\hline
1.3.9 & de sobrepuiar los otros en obras uirtuosas \textbf{ ssi commo fueron contadas dessuso doze uirtudes } delas quales las quatro eran prinçipales & quanto decentius est eos excellere in operibus virtuosis . \textbf{ Sicut enumerabantur superius duodecim virtutes , } quarum quatuor erant principales , \\\hline
1.3.9 & e ayuntadas alas prinçipalsassi \textbf{ entre estas doze passiones } que son cotadas de ssuso & et octo quasi annexae : \textbf{ sic inter has duodecim passiones enumeratas , } sequendo praedecessorum doctrinam , \\\hline
1.3.9 & e es ya ganado . \textbf{ Mas quando es en conparaçion de algun mal comiença en la mal querençia } e vayendo para la foyr & quando bonum illud est praesens , et adeptum . \textbf{ Respectu vero mali incipit ab odio , et procedit in fugam , vel abominationem , et terminatur in timorem , } si malum illud sit futurum : \\\hline
1.3.9 & e la tristeza son avn passiones prinçipales \textbf{ por que son ordenadas aellas las o tris passiones . que son tomadas en conparaçion de mal } assi commo ala esperança & et tristitia sunt passiones principales : \textbf{ quia ad eas ordinantur passiones sumptae respectu mali : } sicut ad spem et gaudium ordinatur passiones sumptae respectu boni . \\\hline
1.3.9 & e del presente es el gozo . \textbf{ Et del mal futuro } que es de venir es el temor . & de praesenti est gaudium : \textbf{ de malo futuro est timor , } de praesenti vero est tristitia . \\\hline
1.3.9 & quando ya es ganada terca la qual es el dolor e la tristeza . \textbf{ Conuiene que la delectaçion e la tristeza sean prinçipales passiones } en conparaçion del appetito cobdiçiador . & circa quod est dolor \textbf{ et tristitia oportet delectationem | et tristitiam esse principales passiones respectu concupiscibilis . } Spes autem et timor sunt principales passiones respectu irascibilis . \\\hline
1.3.9 & en conparacion del appetito enssannador . \textbf{ Mas commo las nr̃as obras ayan de ser departidas } por estas passiones . & spes et timor sunt principales passiones respectu irascibilis . \textbf{ Sed cum ex passionibus diuersificari habeant opera nostra , decet nos diligenter intendere , in quibus delectemur , et tristemur , } et quae speremus , \\\hline
1.3.10 & es en los capitulos ya dichos de suso . \textbf{ es suso contamos doze passiones . } Conuiene sabra . Amor . Et reçia Delectacion . Et tristeza . & patefactum est per Capitula supra dicta . \textbf{ Enumerabantur supra duodecim passiones , } videlicet , \\\hline
1.3.10 & es suso contamos doze passiones . \textbf{ Conuiene sabra . Amor . Et reçia Delectacion . Et tristeza . } Esperança mal querençia . Desseo . & Enumerabantur supra duodecim passiones , \textbf{ videlicet , | amor , odium , } desiderium , abominatio , delectatio , \\\hline
1.3.10 & Conuiene sabra . Amor . Et reçia Delectacion . Et tristeza . \textbf{ Esperança mal querençia . Desseo . } Et abor E desesparaça . de mor . & amor , odium , \textbf{ desiderium , abominatio , delectatio , } tristitia , spes , desperatio , timor , audacia , ira , et mansuetudo . \\\hline
1.3.10 & que el philosofo en el segundo libro de la rectorica \textbf{ cuenta otras seys passiones . } Conuiene saber Relo . & Sed praeter omnes has passiones Philosop’ \textbf{ 2 Rhetor’ sex alias passiones enumerare videtur , } videlicet , zelum , gratiam , nemesin \\\hline
1.3.10 & gera Njemesim \textbf{ que quiere dezir tanto commo indignacion dela buena andança de los malos . } Misericordia e jnuidia . & videlicet , zelum , gratiam , nemesin \textbf{ ( quod idem est quod indignatio de prosperitatibus malorum ) misericordiam , inuidiam , et erubescentiam siue verecundia . } Sed omnes hae passiones reducuntur ad aliquas passiones praedictarum : \\\hline
1.3.10 & uirtudessi non quisiesse auer conpania en aquellas uirtudes . \textbf{ Et por ende este tal zelo e amor en conparaçion delons bienes honrrables es difinido } e declarado & ø \\\hline
1.3.10 & Et pues que assi es el zelo \textbf{ que es amor grande es aducho al amor . En essa misma manera lagera es aducha al amor . } Ca quando alguno ama . & sed \textbf{ quia non insunt sibi : zelus ergo reducitur ad amorem . Sic etiam gratia reducitur ad amorem : } quia ex amore efficitur aliquis alteri gratiosus . Gratia enim \\\hline
1.3.10 & Onde la uerguença \textbf{ que es b̃me iura en la cara } suele se nonbrar uerguença paresçida en el rostre̊ . & vel inglorificationis . \textbf{ unde verecundia erubescentia nominari consueuit , } quia verecundantes communiter erubescunt , \\\hline
1.3.10 & e mayormente si cree \textbf{ que alguno sufre aquel mala tuerto } assi es miscderia . & si credit ipsum indigne \textbf{ pati illud malum , } sic est misericordia . Nam \\\hline
1.3.10 & Et mayormente la inuidia es entre los que se semeian \textbf{ assi conmolos olleros han inuidia alos otros olleros } e los ferreros alos otros ferreros & Et potissime est inuidia circa similes , \textbf{ ut figuli inuident figulis , } et fabri fabris , et sic de aliis . \\\hline
1.3.10 & si todas estas passiones han de partir \textbf{ todasnr̃as obras conuiene a nos delas cognosçer todas . } Et tanto mas esta conuiene alos Reyes e alos prinçipes & Si ergo omnes hae passiones diuersificare habent omnes operationes nostras , \textbf{ decet nos omnes eas cognoscere ; } et tanto magis hoc decet Reges et Principes , \\\hline
1.3.11 & empero es passion de loar a vn \textbf{ en essa misma manera la gera enemessis paresçen ser passiones de loar } Mas otras algunas ay & licet non sit virtus est tamen laudabilis passio . \textbf{ Sic etiam gratia , } et nemesis laudabiles passiones esse videntur . Quaedam autem sunt vituperabiles : \\\hline
1.3.11 & assi commo la inuidia \textbf{ e ahun la mal querençia } que es cosa de denostar & ut inuidia , \textbf{ et odium , } vituperabile est odium , \\\hline
1.3.11 & que es cosa de denostar \textbf{ si non fuer mal querençia de los pecados . } Et o tris passiones ay & et odium , \textbf{ vituperabile est odium , | nisi sit vitiorum . } Aliae autem passiones videntur se habere ad utrunque , \\\hline
1.3.11 & si non fuer mal querençia de los pecados . \textbf{ Et o tris passiones ay } que se han a amas las partes & nisi sit vitiorum . \textbf{ Aliae autem passiones videntur se habere ad utrunque , } quia possunt esse laudabiles , \\\hline
1.3.11 & por que pueden ser de loar \textbf{ o pueden ser de deno star Empero deuedes parar mientes con grant acuçia } que sienpre en las costunbres las meytades son de loar & quia possunt esse laudabiles , \textbf{ et vituperabiles . Est enim diligenter aduertendum , } quod semper in moribus laudantur media , et vituperantur extrema . Passiones ergo illae , \\\hline
1.3.11 & e es de loar . \textbf{ En essa misma manera avn la miscderia es mediana entre la crueldat e la molleza . } Ca aquel que non ha con passion & ut non debet , dicitur verecundus : \textbf{ et est laudabilis . Sic etiam misericordia media est inter crudelitatem , et molliciem . } Nam qui nulli compatitur , est crudelis , \\\hline
1.3.11 & Et aquel que ha piedat de todas las cosas es dicho muelle e mugeril \textbf{ Mas aquel que ha piedat de los que sufren algun mala tuerto tiene el medio } e es de loar & qui omnibus compatitur , est mollis , et muliebris ; \textbf{ qui vero compatitur indigne patientibus , | tenet medium , } et laudatur , \\\hline
1.3.11 & e es dicho miscderioso . \textbf{ En essa misma manera avn nemessis o desdennamiento es medianera entre la imudia e la plazenteria } por que el inuidioso de todas las bien auenturanças se duelle . & et dicitur misericors . \textbf{ Sic et nemesis media est inter inuidiam , | et placiditatem . } Nam inuidus de omnibus prosperitatibus dolet , \\\hline
1.3.11 & e delas bien auen traanças de los malos ¶ \textbf{ En essa misma manera avn la gran en } quanto es de loar es medianera entre lo sobeio e lo menguado & et dolet de prosperitatibus malorum . \textbf{ Sic etiam gratia ut est quid laudabile , } media est inter superfluum , \\\hline
1.3.11 & que se duele de toda bien andança . \textbf{ En essa misma manera ahun la mal querençia } en quanto es de denostar tiene el estremo . Mas las passiones & Nam inuidia extremum tenet , \textbf{ quae de omni prosperitate dolet . Sic etiam odium ut est vituperabile , } extremum tenet . \\\hline
1.3.11 & e mis cordiosos . \textbf{ Ca ellos deuen ser muy conuenibles partidores de los bienes e delons males si quier de benefiçios si quier de penas . } Ca los benefiçios estonçe son derechamente partidos & et misericordes . \textbf{ Ipsi enim maxime esse debent debiti distributores malorum et bonorum , | siue benefactorum et poenarum . } Beneficia autem tunc recte distribuuntur , \\\hline
1.3.11 & quando se mueuen amiscderia \textbf{ sobre aquellos que padesçen mala tuerto } e non sobre los que sufren mal a su culpa . & mala autem poenae tunc debite infliguntur , \textbf{ quando semper indigne patientibus ad misericordiam commouentur . Verecundia autem , } et nemesis , licet videantur esse laudabiles passiones , \\\hline
1.3.11 & en que caya uerguença . \textbf{ En essa misma manera ahun es dicho } y que los estudiosos non deuen ser uirgon cosos . & in quibus est verecundia . \textbf{ Sic etiam ibidem dicitur , } quod studiosi non est verecundari , \\\hline
1.3.11 & o delas bien andanças de lons malos \textbf{ porque los malos non pueden auer grandes bienes } assi commo son las uirtudes . & qui nimis indignatur de prosperitatibus malorum . \textbf{ Nam mali non possunt possidere maxima bona , } cuiusmodi sunt virtutes : \\\hline
1.3.11 & que es passion de denostar \textbf{ Et avn deuen foyr dela mal querençia } si non fuesse de pecados o de males . & quae est vituperabilis passio , \textbf{ penitus fugere debent : | et etiam odium , } nisi esset vitiorum : \\\hline
1.3.11 & e son de deraygar por toda su fuerça \textbf{ Mas en las o tris passiones } que pueden ser de loar & nam vitia sunt odienda , \textbf{ et sunt pro viribus extirpanda . In aliis autem passionibus , } quae possunt esse laudabiles , \\\hline
1.3.11 & e dela miscd̃ia \textbf{ e delanso tris passiones } segunt que conuiene a cada cosa . & infra diffusius tractabitur . \textbf{ Agetur enim in tertio de timore , amore , misericordia , et de aliis , } ut rei cognoscentia postulabit . Ibi enim ostendemus , quomodo Reges et Principes se habere debeant , ut a populis timeantur , et amentur : \\\hline
1.3.11 & Cay mostraremos en qual manera los Reyes se deuen auer \textbf{ por que sean temidos de lons pueblos e amados . } Et en qual manera deuen ser mibicordiosos . & Agetur enim in tertio de timore , amore , misericordia , et de aliis , \textbf{ ut rei cognoscentia postulabit . Ibi enim ostendemus , quomodo Reges et Principes se habere debeant , ut a populis timeantur , et amentur : } et quomodo debent esse misericordes : \\\hline
1.3.11 & Et en qual manera zerotipos e çelolos . \textbf{ Ca muchͣs colas } que aqui son dichos & et quomodo debent esse misericordes : \textbf{ et quomodo zelatiur . Nam multa , } quae hic uniuersaliter sunt tradita , \\\hline
1.4.1 & de quales utudes deuen ser honrrados \textbf{ e quels passiones deuen segnir ¶ } finça de dezir dela quarta parte . & et quibus virtutibus debeant esse ornati : \textbf{ et quas passiones debent sequi . Restat exequi de parte quarta , } videlicet quos , \\\hline
1.4.1 & e otras costunbres han los uieios . Mas avn por la uentura se pueden departir las costunbres . \textbf{ Ca o tris costunbres han aquellos que estan en estado de buena uentura . } Et o trishan aquellos que non estan en aquella uentraa & quia alios mores habent iuuenes , \textbf{ alios senes , } alios illi qui sunt in statu fortunae . \\\hline
1.4.1 & e las riquezas fazen las costunbres muy departidas . \textbf{ Ca por la mayor parte otras costunbres han los nobles } que non los que non son nobles . & ut nobilitas , potentia , et diuitiae , non modicum mores diuersificant . \textbf{ Nam ut plurimum alios mores habent nobiles , } quam ignobiles : \\\hline
1.4.1 & en el segundo de la rectorica \textbf{ tanne seys costunbres de loar } e seys de denostar ¶ & ø \\\hline
1.4.1 & esperança¶ \textbf{ Lo terçero por que son de grand coraçon } ¶ Lo quarto por que non son de mala uoluntad¶ & quia animosi , \textbf{ et bonae spei . Tertio , | quia sunt magnanimi . } Quarto , quia non sunt maligni moris . Quinto , \\\hline
1.4.1 & Lo terçero por que son de grand coraçon \textbf{ ¶ Lo quarto por que non son de mala uoluntad¶ } Lo quinto por que son de ligero piadosos e misericordiosos ¶ & quia sunt magnanimi . \textbf{ Quarto , quia non sunt maligni moris . Quinto , } quia sunt de facili miseratiui . Sexto , et ultimo , \\\hline
1.4.1 & por su trabaio propreo . \textbf{ Ca cada vno con mayor acuçia guarda las sus riquezas } e el su auer & non acquisiuerunt proprio labore . \textbf{ Nam quilibet cum maiori diligentia retinet facultates suas , } quando propter indigentiam passus est aliqua mala , \\\hline
1.4.1 & por su sabiduria propia o por su trabaio proprao . \textbf{ Ca aquella cosa que es ganada con trabaio con mayor acuçia es guardada e retenida . } Et avn en tanto se delecta cada vno en su obra propria & vel quando facultates illas acquisiuit propria industria , \textbf{ et proprio labore . } Nam quod cum labore acquiritur , \\\hline
1.4.1 & mas caramente la guarda . \textbf{ ¶ Lo segundo los mançebos son de buena esperança } la qual cosa les contesçe & diligentius custoditur , immo adeo quilibet delectatur in proprio opere , \textbf{ quod quicquid sua industria acquirit , charius possidet . Secundo iuuenes sunt bonae spei , } quod triplici ratione contingit , \\\hline
1.4.1 & e de buean esꝑança . \textbf{ ¶ Lo primero por que pocas cosas han prouado e non han sofrido en muchͣs cosas repoyo . } Et por ende creen que ganaran todas las cosas & et bonae spei primo ; \textbf{ quia paucorum experti , | non in multis sunt passi repulsam , } ideo credunt omnia obtinere . \\\hline
1.4.1 & Et por ende creen que ganaran todas las cosas \textbf{ e avn son de buena esꝑança } por que en ellos abonda mucho calentura natural . & ideo credunt omnia obtinere . \textbf{ Sunt etiam bonae spei , } quia in eis multum abundat calor : \\\hline
1.4.1 & que es en los mançebos \textbf{ por esso son los mançebos de buena esꝑança } e de buen coraçon & ø \\\hline
1.4.1 & por esso son los mançebos de buena esꝑança \textbf{ e de buen coraçon } assi que se atreuen a todas las cosas . & corde ergo et aliis membris inflammatis ex calore existente in ipsis iuuenibus , fiunt iuuenes bonae spei , \textbf{ et animosi } ut omnia audeant . \\\hline
1.4.1 & que han de fazer \textbf{ por que esperan que han de fazer grandes cosas . } Por la qual cosa contesce & quae facturi sunt . \textbf{ Sperant enim se magna facere , } quare contingit eos animosos esse , \\\hline
1.4.1 & Por la qual cosa contesce \textbf{ aellos de ser de grand coraçon e de grant esperança } ¶Lo terçero contesçe a ellos de ser de grand coraçon & quare contingit eos animosos esse , \textbf{ et bonae spei . } Tertio contingit eos esse magnanimos , \\\hline
1.4.1 & aellos de ser de grand coraçon e de grant esperança \textbf{ ¶Lo terçero contesçe a ellos de ser de grand coraçon } por la razon que ya es dicha e mostrada de ssuso . & et bonae spei . \textbf{ Tertio contingit eos esse magnanimos , } cuius causa ex praecedentibus assignatur . Nam ex hoc est quis magnanimus , \\\hline
1.4.1 & por la razon que ya es dicha e mostrada de ssuso . \textbf{ Ca por esso es dicho alguno magnanimo e de grand coraçon } por que se tiene por digno para g̃ndescolas & Tertio contingit eos esse magnanimos , \textbf{ cuius causa ex praecedentibus assignatur . Nam ex hoc est quis magnanimus , } quia dignificat se magnis , et ingerit se ad faciendum magna . Iuuenes ergo , cum sint liberales , et cum sint animosi \\\hline
1.4.1 & por que se tiene por digno para g̃ndescolas \textbf{ e entremetesse de fazer grandes cosas . } Et pues que assi es commo los mancebos non ayan ninguna cosa & cuius causa ex praecedentibus assignatur . Nam ex hoc est quis magnanimus , \textbf{ quia dignificat se magnis , et ingerit se ad faciendum magna . Iuuenes ergo , cum sint liberales , et cum sint animosi } et bonae spei , \\\hline
1.4.1 & que non sean magranimos \textbf{ por ende son animolos e de grand esperança . } Et avn podemos aesto adozir otra razon espeçial . & non habent \textbf{ unde retrahantur quin sint magnanimi . } Posset \\\hline
1.4.1 & Et por ende en alguna manera son magnanimos \textbf{ e de grandes coraçones } delos quales mag̃nimos la proprea materia es la honrra & quia sunt percalidi , et cupiunt excellere , maxime desiderant gloriam , et honorem : \textbf{ et per consequens aliquo modo sunt magnanimi , } cuius propria materia videtur esse honor . \\\hline
1.4.1 & por que non creen que los otros sean malos . \textbf{ mas por la mayor parte creen } que todos los omes son buenos . & quia non credunt alios esse malos , \textbf{ sed ut plurimum credunt omnes homines esse bonos . } Cuius ratio est , \\\hline
1.4.1 & assi creen que los otros todos son moçentes e sinples ¶ \textbf{ Lo quanto los mançebos son de ligero misericordiosos . } Ca assi commo dicho es dessuso destose leunata mayormente la mibicordia & quod pueri sua innocentia alios mensurant . Sicut enim ipsi sunt innocentes , \textbf{ sic credunt alios innocentes esse . Quinto iuuenes sunt de facili miseratiui : } quia ( ut supra dicebatur ) \\\hline
1.4.1 & Ca assi commo dicho es dessuso destose leunata mayormente la mibicordia \textbf{ si creyeremos que los otros sufren mala tuerto e sin meresçimiento . } Por la qual cosa & ex hac misericordia maxime consurgit , \textbf{ si credamus alios indigne pati . } Quare si iuuenes sua innocentia alios mensurant , \\\hline
1.4.1 & Por ende los mançebos mucho dessean aquellas cosas \textbf{ que traen a estado de honrra . Et por el contrario muchon temen aquellas cosas } que trahen a denuesto e a desonira . & iuuenes multum affectant ea quae importare uidentur honoris statum , \textbf{ et per locum ab oppositis , | multum timent quae important ignominiam et inhonorationem : } et quia erubescentia est timor inglorificationis , \\\hline
1.4.1 & e non en otro o es de loar \textbf{ por alguna condicion non es de loar sinple mente . } Ca ueemos que ser sanudo es de loar en el can . & uidemus enim quod esse furibundum , \textbf{ est laudabile in cane , } non tamen est laudabile in homine . Sic , \\\hline
1.4.1 & Empero non es de loar en el omne \textbf{ En essa misma manera } maguer ser uergonçoso sea de loar en los mançebos & est laudabile in cane , \textbf{ non tamen est laudabile in homine . Sic , } licet uerecundari sit laudabile in iuuenibus , \\\hline
1.4.1 & por que los Reyes e los prinçipes alos quales conuiene de ser \textbf{ assi commo medios dioses } non solamente non les conuiene de fazer cosas torpes & quos decet esse \textbf{ quasi semideos , } non solum quod turpia committant , \\\hline
1.4.1 & por que las malas palauras \textbf{ e torpes corronpen las buenas costunbres . } Et pues que assi es non son ellos en estado & sed abominabile eis esse debet quod audiant turpia nominari : \textbf{ quia corrumpunt bonos mores colloquia praua . } Non ergo sunt in statu quod debeant uerecundari . \\\hline
1.4.1 & por que conuiene aellos de ser liberales \textbf{ e de buena esperança . } Et otrosi de non ser maliciosos de uoluntad & quia decet eos esse liberales , \textbf{ bonae spei , non maligni moris , magnanimos , } et miseratiuos . Sextum autem , videlicet , esse verecundos , non decet simpliciter competere Regibus et Principibus . Decet enim Reges et Principes esse liberales : \\\hline
1.4.1 & Otrossi conuiene alos Reyes \textbf{ e alos prinçipes de ser magranimos e de grand coraçon } Ca assi commo es dicho dessuso en el capitulo dela magranimidat & Rursus decet eos esse magnanimos : \textbf{ quia } ( ut dicebatur in quodam capitulo de magnanimitate ) maxime magnanimitas competit Regibus et Principibus , \\\hline
1.4.1 & por que mucho conuiene a ellos \textbf{ de obrar grandes cosas } e entender cerca las cosas altas ¶ & ( ut dicebatur in quodam capitulo de magnanimitate ) maxime magnanimitas competit Regibus et Principibus , \textbf{ quia eos maxime magna decet operari , } et in ardua tendere . Sic etiam congruum est eos non esse maligni moris , \\\hline
1.4.1 & e entender cerca las cosas altas ¶ \textbf{ An en essa misma manera conuiene aellos } de non ser maliçiosos de uoluntad & quia eos maxime magna decet operari , \textbf{ et in ardua tendere . Sic etiam congruum est eos non esse maligni moris , } ut non de quibuslibet habeant opinionem malam . \\\hline
1.4.1 & por que non ayan de qual \textbf{ si quier de los suys mala opimion . } Ca si los fechos de los subditos lienpre le interpetrassen en mala parte contesçrie & et in ardua tendere . Sic etiam congruum est eos non esse maligni moris , \textbf{ ut non de quibuslibet habeant opinionem malam . } Nam si acta subditorum semper interpretarentur in malam partem , \\\hline
1.4.1 & si quier de los suys mala opimion . \textbf{ Ca si los fechos de los subditos lienpre le interpetrassen en mala parte contesçrie } que los Reyes serien tiranos & ut non de quibuslibet habeant opinionem malam . \textbf{ Nam si acta subditorum semper interpretarentur in malam partem , } contingeret eos esse tyrannos , \\\hline
1.4.1 & losomes las mas cosas fazen e obran mal \textbf{ Et por ende la flaqueza delons omes demanda perdon } por las cosas mal fechos o malobradas . & ut plurimum mala faciunt , \textbf{ ipsa ergo humana fragilitas veniam postulat pro delictis : } quare Reges et Principes , \\\hline
1.4.1 & Et por ende la flaqueza delons omes demanda perdon \textbf{ por las cosas mal fechos o malobradas . } Por la qual cosa conuiene alos Reyes & ut plurimum mala faciunt , \textbf{ ipsa ergo humana fragilitas veniam postulat pro delictis : } quare Reges et Principes , \\\hline
1.4.1 & por si \textbf{ mas por la mala obra } si la fezieren & qua infliguntur poenae , dicet miseratiuos esse . Esse autem verecundos \textbf{ non decet eos simpliciter , } ut superius dicebatur . \\\hline
1.4.2 & assi commo es dicho de ssuso . ¶ \textbf{ ssi commo dessuso contamos seys costunbres de los mançebos } que sonb de loar . & ut superius dicebatur . \textbf{ Sicut supra enumerauimus } ipsorum iuuenum sex mores laudabiles : \\\hline
1.4.2 & que sonb de loar . \textbf{ Assi podemos contar seys costunbres } que son de denostar & ipsorum iuuenum sex mores laudabiles : \textbf{ sic enumerare possumus sex vituperabiles : } quas etiam tangit Philosophus 2 Rhetoricorum . \\\hline
1.4.2 & por dos razones ¶ \textbf{ La primera razon es que por que los mancebos son muy calient s̃ } e el cuerpo escalentado faze appetito & duplici de causa contingit . \textbf{ Nam | cum iuuenes sint percalidi , } et corpore calefacto fiat venereorum appetitus , \\\hline
1.4.2 & e desseo de luxia . \textbf{ Por ende la natural disposiçion del cueꝑpo mueue } e abiua los mançebos a cobdiçia de luyia . & et corpore calefacto fiat venereorum appetitus , \textbf{ naturalis dispositio corporis incitat iuuenes ad concupiscentias venereorum . } Rursus hoc idem conuenit , \\\hline
1.4.2 & por que non son prouados en las cosas mundanales \textbf{ e non han grant entendemiento nin grant sabiduriamas se gouiernan } por razon que por passion & quia sunt inexperti , \textbf{ et non vigent intellectu | et prudentia , } magis regnatur passione quam ratione : \\\hline
1.4.2 & por que el alma sigue la conplession del cuerpo . \textbf{ Et assi commo los humores los cuerpos de los mançebos son en grant mouimiento } assi ellos han las uoluntades muy mouibles & Nam anima sequitur complexiones corporis . \textbf{ Sicut ergo in corporibus iuuenum humores sunt in magno motu : sic ipsi habent voluntates } et concupiscentias valde vertibiles . Ideo dicitur 2 Rhetoricorum , \\\hline
1.4.2 & e por la su sinpleza mesuran alos otros . \textbf{ Et pues que assi es commo natural cosa sea } que qual quier omne de ligero cree a aquel que cuyda que es bueno . & sed sua innocentia alios mensurant . \textbf{ Cum ergo naturale sit , } quod quis de facili credat ei , \\\hline
1.4.2 & por que non son docternados \textbf{ por muchͣs razones } e son estrannos en las cosas & Nam ex multis sermonibus indocti existentes , \textbf{ ad pauca respicientes , } iudicant facile . \\\hline
1.4.2 & Et pues que assi es \textbf{ por que los mançebos non son prouados nin han esperiença de muchͣs cosas } luego & cito iudicat . Iuuenes ergo , \textbf{ quia non sunt multorum experti , } statim cum eis aliquod proponitur negocium , \\\hline
1.4.2 & e les proponen algun negoçio \textbf{ non pueden catar a muchͣs cosas } por que non son sabios de muchos negoçios & statim cum eis aliquod proponitur negocium , \textbf{ non valentes ad multa respicere , } eo quod sint multorum ignari , \\\hline
1.4.2 & assi commo ante ellos es propuesto . \textbf{ Ca si muchͣs cosas conosçiessen } e penssase las condiconnes de lons omes & ut eis proponitur . \textbf{ Si enim multa cognoscerent , conditiones hominum considerare scirent , } et non statim assentirent corde omnibus iis quae eis dicuntur , \\\hline
1.4.2 & Ca si muchͣs cosas conosçiessen \textbf{ e penssase las condiconnes de lons omes } non consenten luego en el coraçon & ut eis proponitur . \textbf{ Si enim multa cognoscerent , conditiones hominum considerare scirent , } et non statim assentirent corde omnibus iis quae eis dicuntur , \\\hline
1.4.2 & que a ellos son dichͣs \textbf{ mas catarian con grand acuçia } si aquellas cosas deuen ser creydas & et non statim assentirent corde omnibus iis quae eis dicuntur , \textbf{ sed diligenter inspicerent utrum illa essent credenda . } Quarto sunt contumeliosi . \\\hline
1.4.2 & por que son de sana mucho agunda \textbf{ e han muy guandes cobdiçias . Ca las sannas e las cobdiçias sienpre son muy grandes } si non fueren atenpdas & quia sunt acutae irae , \textbf{ habent vehementes concupiscentias . | Nam irae et concupiscentiae semper vehementes sunt , } nisi per rationem moderentur . Iuuenes igitur , \\\hline
1.4.2 & que son de denostar \textbf{ de ligero pue de omne ueren qual manera los Reyes et los prinçipes se de una auer atales costunbres . } Ca si tales costunbres son de denostar en los mançebos mucho & qui sunt mores iuuenum vituperabiles ; de facili videri potest , \textbf{ quomodo Reges et Principes | ad huiusmodi mores debeant se habere . } Nam si talia sunt vituperabilia in iuuenibus , \\\hline
1.4.2 & mas son de deno star \textbf{ en los que son ya criados e de mayor hedat . } Et ahun mucho mas son de deno star en los Reyes e en los prinçipes & Nam si talia sunt vituperabilia in iuuenibus , \textbf{ multo magis sunt vituperabilia in adultis : } et adhuc multo maxime vituperabilia sunt in Regibus et Principibus , \\\hline
1.4.2 & Ca commo ellos ayan muchos lisongeros \textbf{ e muchos les estenruyendo alas oreias deuen penssar con grand acuçia } commo les fabla cada vno & Nam cum multos habeant adulatores , \textbf{ et plurimi sint in eorum auribus susurrantes , | cum maxima diligentia cogitare debent , } qui sunt \\\hline
1.4.2 & por que non fagan assi mismos seer despreçiados . Ca si la mentira faze seer los omes ser menospreçiados \textbf{ quanto mas es cosa desconuenible ala Real magestad de ser despreçiada } tanto con mayor cautella deue estudiar & si enim mendacium reddit homines contemptibiles , \textbf{ quanto magis indecens est regiam maiestatem contemptibilem esse , } tanto maiori cautela studere debent , \\\hline
1.4.2 & quanto mas es cosa desconuenible ala Real magestad de ser despreçiada \textbf{ tanto con mayor cautella deue estudiar } que se alleguen ala uerdat & quanto magis indecens est regiam maiestatem contemptibilem esse , \textbf{ tanto maiori cautela studere debent , } ut inhaereant ueritati . \\\hline
1.4.3 & entre las otras costunbres \textbf{ que tanne de los uieios cuenta seys costunbres } que son de denostar & inter alios mores \textbf{ quos tangit de senibus , } enumerat sex uituperabiles mores . \\\hline
1.4.3 & que son temerosos \textbf{ e de flaco coraçon ¶ } La quarta que son escassos & Tertio sunt timidi \textbf{ et pusillanimes . } Quarto sunt illiberales . Quinto sunt difficilis spei , \\\hline
1.4.3 & ¶ \textbf{ laquina que son de ꝑequana esperança o de mala esperança ¶ } La sexta es que non toman uerguença & Quarto sunt illiberales . Quinto sunt difficilis spei , \textbf{ uel sunt malae spei . Sexto non uerecundantur , sed sunt inerubescitiui . } Sunt enim primo senes increduli , \\\hline
1.4.3 & Lo primero dize el philosofo \textbf{ que los uieios son mal creyentes } e esto les contesçe & uel sunt malae spei . Sexto non uerecundantur , sed sunt inerubescitiui . \textbf{ Sunt enim primo senes increduli , } quod ex experientia contingit . \\\hline
1.4.3 & e conosçen los omes \textbf{ que mienten en muchͣs cosas non les creen de ligero } nin dan fe alos sy dichos . & Nam quia in multis experti sunt cognoscunt homines in multis mentiri ; \textbf{ non de facili fit eis fides , } sed credunt omnes alios esse deceptores . \\\hline
1.4.3 & que por que los uieios visquieron muchos años \textbf{ e en muchͣs cosas fueron engannados . } por ende por esta prueualuenga les contesçe & quod senes multis annis vixerunt , \textbf{ et in pluribus decepti sunt : } ergo propter hanc experientiam contingit ipsos incredulos esse . Secundo senes sunt suspitiosi . quaecunque enim vident , \\\hline
1.4.3 & ca todo lo que veen \textbf{ por la mayor parte sospecha que es mal } e tuerten lo ala peorparte . & ergo propter hanc experientiam contingit ipsos incredulos esse . Secundo senes sunt suspitiosi . quaecunque enim vident , \textbf{ ut plurimum suspicantur malum , } et referunt ea in deteriorem partem . Videntur enim senes econtrario disponi iuuenibus . Pueri enim , \\\hline
1.4.3 & e tuerten lo ala peorparte . \textbf{ Ca los uieios e los mançebos han contrarias condiconnes . } Ca por que los mocos son sinples e sin manziella & ut plurimum suspicantur malum , \textbf{ et referunt ea in deteriorem partem . Videntur enim senes econtrario disponi iuuenibus . Pueri enim , } quia non multa mala fecerunt \\\hline
1.4.3 & et inoçençia iudgan todos los otros . \textbf{ Et todas las cosas retuerçen ala meior parte } por que cuydan que todos son bueons & sua innocentia alios mensurant , \textbf{ et omnia referunt in meliorem partem : } credunt enim omnes bonos esse . \\\hline
1.4.3 & Ca por que ius quieron muchos a nons \textbf{ e fallesçieron en muchͣs cosas iudgan los fechos de los otros } segunt aquellas cosas que ellos mismos fezieron . & quia multis annis vixerunt , \textbf{ et in multis peccauerunt , mensurant facta aliorum } secundum ea quae gesserunt in seipsis : \\\hline
1.4.3 & por la qual cosa \textbf{ por la mayor parte creen } que los otros son malos & secundum ea quae gesserunt in seipsis : \textbf{ propter quod ut plurimum credunt alios malos esse , } et in peiorem partem referunt eorum opera . \\\hline
1.4.3 & que los otros son malos \textbf{ e retuerçen las obras de los otrosa la peor parte . Et por ende dize el philosofo en el segundo libro de la rectorica } que por que los uieios uisquieron much s̃ annos non puede ser & propter quod ut plurimum credunt alios malos esse , \textbf{ et in peiorem partem referunt eorum opera . | Unde dicitur 2 Rhetoricorum ; } quod quia senes vixerunt multis annis , \\\hline
1.4.3 & e retuerçen las obras de los otrosa la peor parte . Et por ende dize el philosofo en el segundo libro de la rectorica \textbf{ que por que los uieios uisquieron much s̃ annos non puede ser } que non ayan pecado en muchͣs cosas & Unde dicitur 2 Rhetoricorum ; \textbf{ quod quia senes vixerunt multis annis , } esse non potest quin in multis peccauerint . Ideo sunt male suspitiosi , \\\hline
1.4.3 & que por que los uieios uisquieron much s̃ annos non puede ser \textbf{ que non ayan pecado en muchͣs cosas } e por ende son sospechosos de mal & quod quia senes vixerunt multis annis , \textbf{ esse non potest quin in multis peccauerint . Ideo sunt male suspitiosi , } et omnia in deterius aestimant . Tertio senes sunt pusillanimes , \\\hline
1.4.3 & e por ende son sospechosos de mal \textbf{ e todas las cosas iudgan ala peor parte ¶ } Lo terçero los vieios son de flaco coraçon e temerosos & esse non potest quin in multis peccauerint . Ideo sunt male suspitiosi , \textbf{ et omnia in deterius aestimant . Tertio senes sunt pusillanimes , } et tumidi . Pusillanimes enim sunt , \\\hline
1.4.3 & e todas las cosas iudgan ala peor parte ¶ \textbf{ Lo terçero los vieios son de flaco coraçon e temerosos } e son de flaco coraçon & esse non potest quin in multis peccauerint . Ideo sunt male suspitiosi , \textbf{ et omnia in deterius aestimant . Tertio senes sunt pusillanimes , } et tumidi . Pusillanimes enim sunt , \\\hline
1.4.3 & Lo terçero los vieios son de flaco coraçon e temerosos \textbf{ e son de flaco coraçon } por que les va fallesçiendo la uida . & et omnia in deterius aestimant . Tertio senes sunt pusillanimes , \textbf{ et tumidi . Pusillanimes enim sunt , } quia sunt humiliati a vita : \\\hline
1.4.3 & assi commo fallesçen en ellos los humores e la uida \textbf{ assi fallesçe en ellos el coraçon e son de flaco coraçon } e ahun los uieios son temerosos . & et vitam : \textbf{ sic deficit in eis cor , | et sunt pusillanimes . Sunt } etiam timidi : \\\hline
1.4.3 & La primera por desfallesçimiento deuida¶ \textbf{ La segunda por luenga praeuadetp̃o ¶ } La terçera por que non buien en esperançamas en memoria . & Primo ex defectu vitae . \textbf{ Secundo ex experientia temporis . } Tertio eo quod non viuunt spe , \\\hline
1.4.3 & Ca el alma \textbf{ por la mayor parte sigue las conplissiones del cuerpo . } Et por ende & Tertio eo quod non viuunt spe , \textbf{ sed memoria . Anima enim ut plurimum sequitur complexiones corporis . } Sicut ergo senes in propriis corporibus deficiunt in humoribus , et in vita : \\\hline
1.4.3 & a nons de creer \textbf{ es que ellos sufrieron muchͣs menguas . } Et por ende temiendo & illiberales fiunt . \textbf{ Contingit } etiam eos illiberales esse , \\\hline
1.4.3 & nin las dan de ligero ¶ \textbf{ lo quinto son de mala esperança } por que non es ꝑan de ninguna cosa bien & et acquisita non de facili tribuunt . \textbf{ Quinto sunt malae spei : } nihil enim bene sperant , \\\hline
1.4.3 & Ca por que los vieios son escassos \textbf{ mayor cuydado han del pro } que dela honestad . & ø \\\hline
1.4.3 & que los uieios son frios \textbf{ e el frio tondas las cosas estrinne } e aprieta e costͥmedolas & Dictum est enim senes esse frigidos . \textbf{ Frigidus enim omnia constipat , } et constringit : \\\hline
1.4.3 & en qual manera los Reyes e los prinçipes se de una auer atales costunbres . \textbf{ Ca cierta cosa es } que commo quier & de leui patere potest quomodo Reges et Principes ad huiusmodi se debeant habere . \textbf{ Nam constat } quod licet Reges et Principes non debeant esse in omnibus de facili creditiui , \\\hline
1.4.3 & Lo segundo non conuiene alos Reyes de ser sospechosos \textbf{ assi que to das las cosas retuercan ala peor parte } Ca desto se les leunataria a ellos ser crueles e non misi cordiosos & et ordinem rationis . Secundo non decet eos esse suspitiosos , \textbf{ ut omnia referant in deteriorem partem : } quia ex hoc contingeret eos esse seueros \\\hline
1.4.3 & Lo terçero non conuiene a ellos \textbf{ de ser tem̃osos e de flacos coraçones } mas conuiene les de ser fuertes e de grandes coraçones & et incurrerent maliuolentiam subditorum . \textbf{ Tertio non decet eos esse timidos } et pusillanimes , \\\hline
1.4.3 & de ser tem̃osos e de flacos coraçones \textbf{ mas conuiene les de ser fuertes e de grandes coraçones } Por que conmo los negoçios e los fech̃d & Tertio non decet eos esse timidos \textbf{ et pusillanimes , | immo fortes et magnanimos : } quia cum negocia respicientia totum regnum , \\\hline
1.4.3 & son grandes e altos \textbf{ por ende conuiene les aellos de ser fuertes e de grandes coraçones ¶ } Lo quarto non conuiene a ellos de ser escassos & sint magna et ardua , \textbf{ oportet eos esse fortes | et magnanimos . } Quarto detestabile est ipsos esse illiberales . Nam supra cum de virtutibus tractabatur , \\\hline
1.4.3 & mas ahun les conuiene de sern magnificos \textbf{ e granados fazie do grandes cosas ¶ } Lo quinto con uiene a ellos de ser de buena elperança . & faciendo mediocres sumptus : \textbf{ sed etiam congruit eos esse magnificos , magnifica faciendo . Quinto oportet eos esse bonae spei : } quia si in omnibus se crederent deficere , \\\hline
1.4.3 & e granados fazie do grandes cosas ¶ \textbf{ Lo quinto con uiene a ellos de ser de buena elperança . } Ca si en todas las cosas creyessen & faciendo mediocres sumptus : \textbf{ sed etiam congruit eos esse magnificos , magnifica faciendo . Quinto oportet eos esse bonae spei : } quia si in omnibus se crederent deficere , \\\hline
1.4.4 & que non son de loar fincanos de poner las costunbres dellos qson de loar \textbf{ Mas paresçe que el philosofo en el segundo libro dela rectorica pone quatro costunbres de los uieios } que pueden ser de loar ¶ & restat enumerare mores ipsorum laudabiles . \textbf{ Videtur autem Philosophus 2 Rhetoricorum , | circa senes tangere quatuor mores , } qui possunt esse laudabiles . \\\hline
1.4.4 & La segunda es que son mibicordiosos ¶ \textbf{ la terçera es que las cosas dubdosas non las afirman con grand afincamiento ¶ } La quarta es & Primo enim senes habent concupiscentias remissas , \textbf{ et temperatas . Secundo sunt miseratiuis . Tertio dubia non pertinaciter asserunt . } Quarto nihil agunt valde . Concupiscentiae enim senum \\\hline
1.4.4 & ca nin mas que deuen \textbf{ ¶La primera paresçe assi ca las cobdiçias de los uieios } e mayormente çerca & ø \\\hline
1.4.4 & por que los uieios visquieron muchos años \textbf{ e vieron que fueron muchͣs uezes engannados } non osan afirmar ningunan cosa afincandamente & ( ut ait Philosoph’ 2 Rheto’ ) quia senes vixerunt multis annis , et viderunt \textbf{ quod saepe sunt decepti : } non audent pertinaciter aliquid asserere , \\\hline
1.4.4 & por que han las passiones e las cobdiçias abaxadas \textbf{ por la mayor parte fazen todas las cosas tenpradamente ¶ } Visto quales son las costunbres de los mançebos & et concupiscentias remissas , \textbf{ ut plurimum agunt omnia moderate . Viso , } qui sunt mores iuuenum , \\\hline
1.4.4 & assi commo los uieios \textbf{ nin son assi de grand coraçon e valientes commo los mançebos } nin son assi temerosos & ø \\\hline
1.4.4 & nin son assi temerosos \textbf{ e de flaco coraçon commo los vieios . } Mas tienen el medio entre los mançebos e los uieios . & et praeuolantes , ut iuuenes , \textbf{ nec sunt sic timidi , et pusillanimes ut senes : } sed tenentes medium inter utrosque , \\\hline
1.4.4 & e osados do lo han de ser . \textbf{ ¶ En essa misma manera ahun } por que non son del todo sin praeua & et audaces , \textbf{ ubi est audendum . Sic etiam , } quia nec sunt omnino inexperti , \\\hline
1.4.4 & por que las non han prouado \textbf{ nin del todo descreen commo los uieios por que fueron en muchͣs cosas engannados } mas han se medianamente & nec omnino discredunt \textbf{ quod faciunt senes eo quod sint in multis decepti : } sed habent se medio modo . Ideo dicitur 2 Rhetoricorum , quod qui sunt in statu , nec sunt omnibus credentes , \\\hline
1.4.4 & en el segundo sibro de la rectorica \textbf{ son esforçados contenprànça } e tenprados con uirtud . & sed sunt viriles cum temperantia , \textbf{ et temperati cum virilitate . } Ut ergo sit \\\hline
1.4.4 & que son de denostar en ellos todas son arredradas delas medianeras . \textbf{ Ca assi commo dicho es de suso muchͣs uezes } sienpre las cosas estremas son de denostar & Et quicquid vituperabilitatis est in eis totum remouetur ab illis . \textbf{ Nam } ( ut supra pluries dicebatur ) semper extrema sunt vituperabilia , et medium est laudabile . \\\hline
1.4.4 & e las medianerasson de loar . \textbf{ Et por ende si en los vieios o en los mançebos es alg̃ cosa de loar } esto es & ( ut supra pluries dicebatur ) semper extrema sunt vituperabilia , et medium est laudabile . \textbf{ Si ergo in senibus , | vel in iuuenibus est aliquid laudabile , } hoc est , \\\hline
1.4.4 & Et pues que \textbf{ assi es en tal manera deuemos fablar delas costunbres delons omes . } Empero non se deue entender & ab iis \textbf{ qui sunt in statu . | Sic ergo loquendum est de moribus hominum . } Non tamen intelligenda sunt \\\hline
1.4.4 & assi que los uieios non puedan ser francos \textbf{ e de grand coraçon } e que los mançebos non puo dan ser tenpdos e firmes . & talia omnino necessitatem habentia , \textbf{ ut quod senes non possint esse liberales et magnanimi , } et quod iuuenes non possint esse temperati et stabiles : \\\hline
1.4.4 & e regla de todos los otros . \textbf{ Et pues que assi es alguas costunbres delons uieios } e de los mançebosson bueans e de segnir & et regula aliorum . Senum ergo , \textbf{ et iuuenum aliqui mores sunt imitandi , } aliqui fugiendi . \\\hline
1.4.4 & e inclina conn natural \textbf{ e segnir bueans costunbrs e de loar . } Vien assi ahun aquellos que son en estado medianero maguera & possunt tamen contra illam pronitatem facere consequi laudabiles mores . \textbf{ Sic et illi } qui sunt in statu , \\\hline
1.4.4 & e de denostar . \textbf{ Por la qual cosa sinoble cosa es } e muy digna de enssenorear & vituperabiles mores . \textbf{ Quare si dignum est dominari rationi , } et intellectui , \\\hline
1.4.5 & que quatro son las costunbres bueans \textbf{ e de lapña esboar delos nobles omes ¶ } que son de grand coraçon ¶ & dicere possumus , \textbf{ ipsorum nobilium } esse quatuor mores laudabiles . \\\hline
1.4.5 & e de lapña esboar delos nobles omes ¶ \textbf{ que son de grand coraçon ¶ } La segunda es que son magnificos e de grand fazienda¶ & ipsorum nobilium \textbf{ esse quatuor mores laudabiles . } Primo enim sunt magnanimi . Secundo magnifici . Tertio dociles et industres . \\\hline
1.4.5 & que son de grand coraçon ¶ \textbf{ La segunda es que son magnificos e de grand fazienda¶ } La terçera & esse quatuor mores laudabiles . \textbf{ Primo enim sunt magnanimi . Secundo magnifici . Tertio dociles et industres . } Quarto sunt politici et affabiles . Sunt enim nobiles magnanimi . \\\hline
1.4.5 & La quarta que son bien acostunbrados e amigables ¶ \textbf{ Pues que assi es lo primero nobles son de grand coraçon } por que la nobleza & Primo enim sunt magnanimi . Secundo magnifici . Tertio dociles et industres . \textbf{ Quarto sunt politici et affabiles . Sunt enim nobiles magnanimi . } Nam nobilitas ( ut dicitur 2 Rhetoricorum ) \\\hline
1.4.5 & assi common dize el philosofo en el segundo libro delan rectorica \textbf{ si antigua miente desçendieron de aquel linage muchos granados omes et muchos nobles . } Et pues que assi es la uirtud del linage & ( ut vult Philos’ 2 Rhet’ ) \textbf{ si ab antiquo ex illo genere processerunt multi praesides , } et multi insignes . Virtus ergo generis , \\\hline
1.4.5 & o de alguna sangre escogida \textbf{ enla qual antigua miente fueron muchos prinçipes e muchos nobles . } Et por ende assi nos conuiene de sentir desta nobleza Enpero por que segunt la comun opinion de los omes todas las cosas son mesuradas por dineros e por riquezas & vel ex aliqua prosapia , \textbf{ in qua etiam ab antiquo fuere multi principantes , | et multi insignes , } sic ergo sentiendum est de nobilitate . \\\hline
1.4.5 & enla qual antigua miente fueron muchos prinçipes e muchos nobles . \textbf{ Et por ende assi nos conuiene de sentir desta nobleza Enpero por que segunt la comun opinion de los omes todas las cosas son mesuradas por dineros e por riquezas } e las riquezas paresçen ser preçio & sic ergo sentiendum est de nobilitate . \textbf{ Verum quia | secundum communem opinionem hominum omnia mensurantur numismate , } et diuitiae videntur esse pretium rei cuiuslibet , \\\hline
1.4.5 & si non riquezas mucho antiguas . \textbf{ Et por ende por que los nobles de antiguo tienpo fueron prinçipes } e en su linage fueres mucho nobles e ricos leunatase el coraçon de los nobles & nihil est aliud quam antiquatae diuitiae ; \textbf{ quia ergo nobiles ex antiquo fuerunt praesides , et in suo genere fuerunt multi insignes et diuites , } eleuatur cor nobilium ex exemplo parentum , \\\hline
1.4.5 & por enxienplo delos parientes \textbf{ que vayan a grandes cosas } e sean de grant coraçon . & eleuatur cor nobilium ex exemplo parentum , \textbf{ ut tendant in magna , } et sint magnanimi . \\\hline
1.4.5 & que vayan a grandes cosas \textbf{ e sean de grant coraçon . } Ca natural cosa es & ut tendant in magna , \textbf{ et sint magnanimi . } Naturale est enim , \\\hline
1.4.5 & e sean de grant coraçon . \textbf{ Ca natural cosa es } que sienpre la fechura quiera semeiar a su fazedor & et sint magnanimi . \textbf{ Naturale est enim , } quod semper effectus vult assimilari causae : \\\hline
1.4.5 & que sienpre la fechura quiera semeiar a su fazedor \textbf{ por que los fijos son fechuras de los padron natural cosa es } que los fuos semeien alos paradres . & quod semper effectus vult assimilari causae : \textbf{ cum filii sint quidam effectus parentum , } naturale est filios imitari parentes . \\\hline
1.4.5 & que en el su linage fueron muchos nobles \textbf{ e que entendien a grandes cosas } por que lemeien a sus parientes dessean grandes cosas & quod in eorum genere fuerunt multi insignes , \textbf{ et tendentes in ardua , } ut imitentur parentes , \\\hline
1.4.5 & e que entendien a grandes cosas \textbf{ por que lemeien a sus parientes dessean grandes cosas } e acahesçe les ser de grandes coraçones & et tendentes in ardua , \textbf{ ut imitentur parentes , } affectant magna , \\\hline
1.4.5 & por que lemeien a sus parientes dessean grandes cosas \textbf{ e acahesçe les ser de grandes coraçones } ¶Lo segundo son los nobles non solamente magnanimos e de grand coraçon . & ut imitentur parentes , \textbf{ affectant magna , } et contingit eos esse magnanimos . Secundo nobiles non solum sunt magnanimi \\\hline
1.4.5 & e acahesçe les ser de grandes coraçones \textbf{ ¶Lo segundo son los nobles non solamente magnanimos e de grand coraçon . } Mas ahun si han poder son magnificos & affectant magna , \textbf{ et contingit eos esse magnanimos . Secundo nobiles non solum sunt magnanimi } et magni cordis , \\\hline
1.4.5 & Mas ahun si han poder son magnificos \textbf{ e de grand fazienda . } Ca si en las otrascondiconnes son eguales & et contingit eos esse magnanimos . Secundo nobiles non solum sunt magnanimi \textbf{ et magni cordis , } sed \\\hline
1.4.5 & Ca assi commo dich̃es la nobleza \textbf{ segunt la comun opinion de los omes } non es otra cosa sinon riquezas de antiguedat & nobilitas idem est \textbf{ secundum communem acceptionem quod antiquatae diuitiae , } vel quod virtus \\\hline
1.4.5 & Mas el linage de algunos prinçipalmente es tenido por hanrrado \textbf{ si de antigo tienpo abondo en riquezas . } Et pues que assi es comm sienpre ayamos de dar comienço & et honorabilitas generis . Genus autem alicuius maxime reputatur honorabile , \textbf{ si ab antiquo affluebat diuitiis . } Cum ergo semper sit dare initium , \\\hline
1.4.5 & commo la nobleza sienpre incline el coraçon de los nobles \textbf{ para fazer grandes cosas } siguese que los nobles han de ser magnificos & magis antiquatae diuitiae in filiis quam in parentibus : \textbf{ quare cum nobilitas semper inclinet animum nobilium ut faciant magna , } sequitur nobiles esse magnificos , \\\hline
1.4.5 & Et por ende non solamente son magnificos \textbf{ mas avn esfuercan se de acometer mayores fechos } que los padres . & quia quodammodo sunt nobiliores illis . Ideo nobiles non solum sunt magnifici , \textbf{ sed etiam nituntur maiora facere quam parentes . } Unde Philos’ 4 Eth’ ait , \\\hline
1.4.5 & que conuiene de ser los nobles magranimos \textbf{ e de grandes coraçones e magnificos } e de grandes fechos e głiosos e much̃ honrrados & Unde Philos’ 4 Eth’ ait , \textbf{ quod magnanimos et magnificos decet esse nobiles et gloriosos . Vult enim ibidem , } quod nobiles ex sua nobilitate incitantur , \\\hline
1.4.5 & e de grandes coraçones e magnificos \textbf{ e de grandes fechos e głiosos e much̃ honrrados } por que dize alli el philosofo & ø \\\hline
1.4.5 & ¶la primera razon le prueua alsi . \textbf{ Ca por que los nobles son cados con grand acuçia } e con grand cura & et ex custodia corporis . Alia vero ex conuersatione , et societate quadam aliorum . \textbf{ Cum enim nobiles cum magna diligentia nutriantur , } et cum magna cura proprium corpus custodiant : \\\hline
1.4.5 & Ca por que los nobles son cados con grand acuçia \textbf{ e con grand cura } e con grand guarda delos sus cuerpos & Cum enim nobiles cum magna diligentia nutriantur , \textbf{ et cum magna cura proprium corpus custodiant : } rationabile est , \\\hline
1.4.5 & e con grand cura \textbf{ e con grand guarda delos sus cuerpos } con razon es que ellos ayan los cuerpos bien ordenados & Cum enim nobiles cum magna diligentia nutriantur , \textbf{ et cum magna cura proprium corpus custodiant : } rationabile est , \\\hline
1.4.5 & Ca commo los nobles non sean reprehendidos destos lisongos \textbf{ mas los sus malos fechos sean alabados dellos disponen los } e fazen los que non conoscan assi mismos & Cum enim nobiles non reprehenduntur , \textbf{ sed ab adulatoribus etiam eorum mala facta commendantur , disponuntur , } ut seipsos ignorent , \\\hline
1.4.5 & e si temieren de fazer cosas reprehenssibles e si cuydaren sotilmente todo lo que han de fazer \textbf{ ¶La quatta condicion de los nobles es que son corteses e amigables . } Ca porque en la mayor parte en las cortes delos nobles es acostunbrado de auer grandes & si timentes reprehensibilia facere , diligenter considerent \textbf{ quid agendum . Quarto nobiles contingit esse politicos , et affabiles . } Nam quia ut plurimum in curiis nobilium consueuit esse magna societas , \\\hline
1.4.5 & ¶La quatta condicion de los nobles es que son corteses e amigables . \textbf{ Ca porque en la mayor parte en las cortes delos nobles es acostunbrado de auer grandes } con p̃anas acahesçeles de ser corteses e aconpanables & quid agendum . Quarto nobiles contingit esse politicos , et affabiles . \textbf{ Nam quia ut plurimum in curiis nobilium consueuit esse magna societas , } conuenit eos esse politicos et sociales , \\\hline
1.4.5 & con p̃anas acahesçeles de ser corteses e aconpanables \textbf{ por que en la mayor parte visquieron en conpanina de buenos . } Ca assi commo los rusticos e los aldeanos & conuenit eos esse politicos et sociales , \textbf{ quia ut plurimum in societate vixerunt . } Sicut enim rustici , \\\hline
1.4.5 & La primera es \textbf{ que los nebles son desseadores de grand honrra ¶ } La segunda son despreçiadores de aquellos & quos dicit competere ipsius nobilibus . Primus est , \textbf{ quia nobiles sunt nimis honoris appetitiui . Secundus , sunt progenitorum despectores . } Naturale est enim , \\\hline
1.4.5 & que los engendran . \textbf{ Ca natural cosa es } que cada vno quiere ayuntar & quia nobiles sunt nimis honoris appetitiui . Secundus , sunt progenitorum despectores . \textbf{ Naturale est enim , } quod quilibet vult accumulare \\\hline
1.4.5 & e despreçiadores de aquellos que los engendraron . \textbf{ Ca en la mayor parte los nobles } por mayores se tienen & Secundo sunt elati et despectatores progenitorum : \textbf{ ut plurimum enim nobiles , } maiores se reputant , \\\hline
1.4.5 & e ser muy cobdiçiosos de honrra \textbf{ paresçe de ser malas costunbres } por que non deuemos dessear las honrras en lli . & et nimis esse honoris cupidi , \textbf{ videtur esse mali moris . } Non enim debemus appetere ipsos honores in se , \\\hline
1.4.5 & la qual cosa fazen los uirtuosos \textbf{ e los de grand coraçon } Et pues que assi es commo los Reyes & sed debemus appetere opera honore digna , \textbf{ quod faciunt virtuosi et magnanimi . } Reges ergo et Principes , \\\hline
1.4.5 & si non fueren bueons e uirtuosos \textbf{ conuiene les aellos de segnir las bueans costunbres de los nobles } por que sean de grand coraçon e de grand fazienda & et virtuosi , \textbf{ decet eos sequi bonos mores nobilium , } ut sint magnanimi et magnifici , \\\hline
1.4.5 & conuiene les aellos de segnir las bueans costunbres de los nobles \textbf{ por que sean de grand coraçon e de grand fazienda } e muy sabios e bien razonados . & decet eos sequi bonos mores nobilium , \textbf{ ut sint magnanimi et magnifici , } prudentes et affabiles : \\\hline
1.4.6 & uenta el philosofo en el segundo libro de la rectorica \textbf{ que son çinço malas costunbres de los ricos ¶ } la primera es que los ricos son sobra uos ¶ & alios despicientes . \textbf{ Narrat autem Philosophus 2 Rhetoricor’ quinque malos mores ipsorum diuitum . } Diuites enim primos sunt elati . Secundo contumeliosi . \\\hline
1.4.6 & que non pueden sofrir \textbf{ nin gͦs trabaios . } Por ende luego que son passionados & et intemperati ; quod eis contingit ex deliciis viuendi . Assueti enim sunt viuere adeo delicate , \textbf{ quod non possunt aliquas molestias sufferre . } Ideo statim compassionantur , \\\hline
1.4.6 & Ca assi commo dize el philosofo \textbf{ en el segundo libro dela rectorica grant abiuamiento han } para ser tales . & ( \textbf{ ut innuit Philosoph’ 2 Rhetor’ ) maximum incitamentum habent , ut sint tales . } Nam \\\hline
1.4.6 & si non los bienes senssibles cuydan \textbf{ que este es el mayor bien } e el mas preçiado & a vulgaribus , \textbf{ qui non cognoscunt nisi bona sensibilia , reputatur bonum excellens , } et maximum ; \\\hline
1.4.6 & que las aya conplidamente es diguno de ser sennor e prinçipe . \textbf{ Et pues que assi es todas estas costunbres malas parte nesçen a los ricos } por tanto que son engannados en las riquezas & dignus sit principari . \textbf{ Omnes ergo hi mali mores diuitibus competunt , } eo quod decipiantur circa diuitias , \\\hline
1.4.6 & Mas si las riquezas fueren ordenadas a alabança o a pelea o a destenprança o a otrasma las obras estonçe \textbf{ mas fazen al omne mal andante } que bien andante . & vel ad contumeliam , \textbf{ vel ad intemperantiam , | vel ad alia opera vitiosa : } tunc magis reddunt hominem infelicem , quam felicem . Reputatur ergo quis dignus principari , \\\hline
1.4.6 & e alos prinçipesarredrar se de tales costunbrs finca de ueer \textbf{ quales son las bueans costunbres de los ricos . } Et el philosofo en el segundo libro de la rectorica & videre restat , \textbf{ qui sunt boni mores eorum . Philosophus autem 2 Rhetoricorum , } solum unum bonum morem videtur diuitibus attribuere , \\\hline
1.4.6 & e esto es lo que dize el philosofo en el segundo libro de la rectorica \textbf{ que vna buena costunbre se sigue alosricos que se han bien certa las cosas diuinales } creyendo en alguna manera que por las fadas & Hoc est ergo quod dicitur 2 Rhetorico . \textbf{ Unus optimus mos assequuntur diuites : | quia bene se habent circa diuina , } tredentes aliqualiter per fata , idest , \\\hline
1.4.6 & e estas riquezas . \textbf{ Et por ende este dicho tan sotil del philosofo deuemos le estudiar con grand acuçia . } Ca las riquezas & tredentes aliqualiter per fata , idest , \textbf{ per ordinationem diuinam habere huiusmodi bona . Hoc autem dictum Philosophicum diligenter est considerandum . } Nam diuitias , \\\hline
1.4.6 & nin ala sabiduria proprea delos omes . \textbf{ Et si los ricos bien parasen mientesa esto farien grandes cosas çerca las cosas diuinales . } Et por ende non creyrien que ellos dan grandes dones a dios & quam in propriam industriam : \textbf{ quod si hoc bene attenderent diuites faciendo magnifica circa diuina , } non crederent \\\hline
1.4.6 & Et si los ricos bien parasen mientesa esto farien grandes cosas çerca las cosas diuinales . \textbf{ Et por ende non creyrien que ellos dan grandes dones a dios } mas cuydarian quel dan aquello & quod si hoc bene attenderent diuites faciendo magnifica circa diuina , \textbf{ non crederent | se Deo dona largiri , } sed magis cogitaret quod ei reddunt \\\hline
1.4.7 & Et pero non son ricos . \textbf{ Avn en essa misma manera contesçe } que algunos son ricos & quia processerunt ex aliquo nobili genere , \textbf{ qui tamen non sunt diuites . Sic etiam contingit aliquos esse diuites } qui non sunt nobiles , \\\hline
1.4.7 & e ser toderoso¶ \textbf{ Visto quales son las costunbres delons nobłs } e de los ricos & et esse non potentem . Viso ergo \textbf{ qui sunt mores nobilium , } et qui diuitum restat videre , \\\hline
1.4.7 & assi conmo dize el philosofo en el segundo libro de la rectorica \textbf{ en toda manera han meiores costunbres } que los ricos . & qui sunt mores potentum . Potentes autem \textbf{ ( ut vult Philosophus 2 Rhetor’ ) omnino habent meliores mores , } quam diuites . Narrat autem Philosophus tria , \\\hline
1.4.7 & por que les conuiene de entender \textbf{ e auer cuydados de muchͣs cosas retrahen se } e tiran se & et fiant intemperati . Potentes vero et principantes , \textbf{ quia oportet eos intendere exterioribus curis , } retrahuntur , \\\hline
1.4.7 & que si los poderosos fazen tuerto a alguons \textbf{ non ge lo fazen en pequanas cosas } mas en grandes . Ca los poderosos estando en gerad sennorio & si potentes iniuriantur , \textbf{ non iniuriantur in paruis , } sed in magnis . Potentes enim existentes in Principatu , \\\hline
1.4.7 & mas en grandes . Ca los poderosos estando en gerad sennorio \textbf{ por que son en logar digno de grand honrra } non entienden & sed in magnis . Potentes enim existentes in Principatu , \textbf{ quia sunt in loco magno honore digno , } non tendunt nisi in magna et in ardua . \\\hline
1.4.7 & non entienden \textbf{ si non en grandes cosas e altas . } Et por ende si contezca & quia sunt in loco magno honore digno , \textbf{ non tendunt nisi in magna et in ardua . } Ideo si contingat eos aliis iniuriari , \\\hline
1.4.7 & por que non curan de fazer pequano tuerto \textbf{ nin pequeno danno . } Mas o en ninguna cosa non faran danno alos otros & non iniuriabuntur in paruis , \textbf{ sed in magnis . Non enim curabunt facere paruam offensam , } sed vel in nullo damnificabunt alios , \\\hline
1.4.7 & Mas o en ninguna cosa non faran danno alos otros \textbf{ o les faran grand danno . } Et pues que assi es menos son peleadores los podero łos & sed vel in nullo damnificabunt alios , \textbf{ vel inferent magnum damnum . } Minus igitur sunt contumeliosi potentes , \\\hline
1.4.7 & nin denuesto a ninguno en las pequennas cosas \textbf{ mas solamente son peleosos en las grandes cosas . } Et pues que assi es las costunbres de los ricos & quia quamlibet contumeliam inferre non curant , \textbf{ sed solummodo contumeliosi in magnis . } Mores diuitum omnino peiores , \\\hline
1.4.7 & si non fueren a ellas ayuntado el poderio \textbf{ e la nobleza en la mayor parte } fazen al omne mas desauentraado que auenturado . & vel potentum . Ideo diuitiae , \textbf{ si eas non concomitentur potentatus et nobilitas , } ut plurimum magis reddunt hominem infelicem quam felicem . \\\hline
1.4.7 & que non es noble \textbf{ nin es en algun poderio en la mayor parte } es sienpre destenprado & Qui enim sic diues est , \textbf{ quod nec est nobilis , nec est in aliquo potentatu , } ut plurimum est intemperatus , \\\hline
1.4.7 & que es noble \textbf{ e de antigo tienpo los sus auuelos fueron ricos meior sabe sofrir las riquezas } e por ellas non se leu nata en so ƀͣuia & quod tamen est nobilis , \textbf{ et ab antiquo sui progenitores diuites extiterunt , | melius nouit diuitias supportare , } et propter eas non tantum extollitur . \\\hline
1.4.7 & e de antigo tienpo los sus auuelos fueron ricos meior sabe sofrir las riquezas \textbf{ e por ellas non se leu nata en so ƀͣuia } por que los bienes senssibles paresçen del todo ser contrarios alas sciençias e alas uirtudes . & melius nouit diuitias supportare , \textbf{ et propter eas non tantum extollitur . } Videntur enim bona sensibilia omnino esse contraria scientiis , \\\hline
1.4.7 & que aquellas cosas son mayores \textbf{ e meiores bienes } que el cuydaua . & ø \\\hline
1.4.7 & Por la qual cosa \textbf{ los que son enrriqueçidos de luengo tienpo } mas son acostunbrados en las rriquezas & minus reputantur . \textbf{ Quare cum ditati ab antiquo , } magis assueti sint in diuitiis , \\\hline
1.4.7 & pues que assi es los Reyes e los prinçipes \textbf{ por que por la mayor partida abonda en estos tres bienes de fuera } que son rriquezas nobleza e poderio . & non ita possunt vacare venereis . Reges ergo et Principes , \textbf{ quia ut plurimum his tribus exterioribus affluunt , } videlicet , diuitiis , nobilitate , potentia : \\\hline
1.4.7 & ca por la nobleza \textbf{ e por que de antigo tienpo abondaron } en rriquezas & decet eos esse eruditos , et temperatos . \textbf{ Nam per nobilitatem eo quod ab antiquo abundauerunt diuitiis , } sciunt magis fortunas ferre , \\\hline
1.4.7 & e por su poderio \textbf{ conuiene les de auer grandes cuydados } e de se tyrar de obras lux̉iosas & Rursus quia pollent principatu et potentia , \textbf{ et oportet eos diuersis curis intendere , } retrahuntur a venereis , \\\hline
1.4.7 & quales ya dixiemos . \textbf{ non que todos sean tales general mente } nin que por fuerça de una ser tales & quales diximus : \textbf{ non quod omnes uniuersaliter tales sint , } et quod necesse sit eos tales esse : sed ut plurimum diuitum , nobilium , \\\hline
1.4.7 & nin que por fuerça de una ser tales \textbf{ mas por que en la mayor parte tales son las costunbres de los ricos e de los nobles } e de los otros quales nos dixiemos de suso . & non quod omnes uniuersaliter tales sint , \textbf{ et quod necesse sit eos tales esse : sed ut plurimum diuitum , nobilium , } et aliorum sunt tales mores , quales superius dicebamus . \\\hline
1.4.7 & que ellos ayan tales costunbres enpero mucho son enclinados \textbf{ e han grand disposiçion } para segnir las costunbres sobredichͣs . & tamen multum inclinantur , \textbf{ et magnam pronitatem habent , } ut sequantur praedictos mores . \\\hline
1.4.7 & Ahun en essa mis ma guasa non se deuen enssannar los nobłs \textbf{ nin los ricos li dellos contamos algunas malas costunbres } ca non conuiene & vel diuites indignari debent , \textbf{ si ipsorum narrauimus aliquos malos mores : } quia non oportet omnes esse tales , \\\hline
1.4.7 & Et tanto mas conuiene esto alos Reyes e alos prinçipes \textbf{ quanto estan en mas alto grado . } Ca ellos segunt diches de suso deuen ser exienplo & et Principes , \textbf{ quanto in altiori gradu existunt . } Ipsi enim ( ut superius dicebatur ) debent esse exemplar , \\\hline
1.4.7 & qual si quier cosa \textbf{ que sea ded en estar loar en las buenas costunbres de los otros } omes todo deue ser fallado en ellos & et forma viuendi ; \textbf{ quicquid ergo laudabilitatis est in moribus singulorum , totum debet in ipsis peramplius et perfectius reperiri . Primi libri de regimine Principis finis , in quo traditum fuit , } quomodo Princeps seipsum regere debeat . D . AEGIDII ROMANI Ordinis Fratrum Eremitarum S . Augustini , \\\hline
2.1.1 & Et quales costunbres deuen guardar . \textbf{ Ca por estas quatro cosas conplidamente se puede auer } en qual manera deua cada vno gouernar assi mesmo & et quos mores debeant imitari . \textbf{ Per haec enim quatuor , | sufficienter habetur , } qualiter quilibet debeat seipsum regere , \\\hline
2.1.1 & Et por ende conuiene de saber \textbf{ que el omne sobre todas las otras ainalias ha menester quatro cosas } por las quales podemos prouar en quatro maneras & Sciendum igitur , \textbf{ quod homo ultra alia animalia quatuor indigere videtur ex quibus quadruplici via venari possumus , ipsum esse communicatiuum } et sociale . Prima via sumitur ex victu , \\\hline
2.1.1 & que el omne sobre todas las otras ainalias ha menester quatro cosas \textbf{ por las quales podemos prouar en quatro maneras } que el omne es natra alnen te comunal a todos e conpannero ¶ & Sciendum igitur , \textbf{ quod homo ultra alia animalia quatuor indigere videtur ex quibus quadruplici via venari possumus , ipsum esse communicatiuum } et sociale . Prima via sumitur ex victu , \\\hline
2.1.1 & Mas que la conpannia mucho faga a conplimien to deuida de omne puede se demostrar \textbf{ por aquellas quatro cosas } que contamos desuso & Quod autem societas maxime faciat ad sufficientiam vitae humanae , \textbf{ patere potest ex his quatuor supra enumeratis : } quibus homo , \\\hline
2.1.1 & que omne ha mester . \textbf{ Ca el omne entre todas las aian lias ha meior conplission . } Et por ende entre todas las ainalias ha mester meior uianda e meior apareiada . & quo indiget . \textbf{ Nam homo | inter caetera animalia habet meliorem tactum , } et meliorem complexionem . Ideo \\\hline
2.1.1 & Ca el omne entre todas las aian lias ha meior conplission . \textbf{ Et por ende entre todas las ainalias ha mester meior uianda e meior apareiada . } Ca la natura a todas las otrasaian las da conplidamente su uianda . & inter caetera animalia habet meliorem tactum , \textbf{ et meliorem complexionem . Ideo | inter omnia animalia indiget cibo diligenter , } et artificialiter praeparato . Natura enim animalibus aliis quasi sufficienter ministrat cibum : \\\hline
2.1.1 & Ca la natura a todas las otrasaian las da conplidamente su uianda . \textbf{ assi commo alas aina lias } que non bauen de robo quales son oueias e bueyes & inter omnia animalia indiget cibo diligenter , \textbf{ et artificialiter praeparato . Natura enim animalibus aliis quasi sufficienter ministrat cibum : } ut animalibus non viuent bus ex rapina , \\\hline
2.1.1 & a todas las otras aian lias \textbf{ enpero non es sufiçiente uianda al omne } si non fuere apareiado e apurado . & et si esset sufficiens cibus animalibus aliis : \textbf{ homini autem non est sufficiens cibus , } nisi praeparetur et depuretur ; ideo molitur , \\\hline
2.1.1 & por que se auianda conuenible al omne ¶ \textbf{ Et por que para todas estas colas non cunple bien vna sola perssona por ende } por que el omne aya cunplimiento en ssi mismo en la uida & ut sit hominum congruus cibus . \textbf{ Et quia ad haec omnia una sola persona non bene sufficit , } ideo \\\hline
2.1.1 & Ca las bestias e las aues veemos \textbf{ que han natural uestido } assi commo las bestias han la lana & Nam sicut natura sufficienter aliis animalibus prouidere videtur in victu : sic videtur quod eis sufficienter prouideat in vestitu . Bestiae enim , et aues , \textbf{ quasi naturale indumentum , habere videntur lanam , vel pennas . Homini autem non sufficienter prouidet natura in vestitu : } cum enim homo sit nobilioris complexionis quam animalia alia , \\\hline
2.1.1 & Mas la natura non prouee al omne tan conplidamente en uestidura \textbf{ por que el omne es de mas nobł conplission } que las otras aianlias & cum enim homo sit nobilioris complexionis quam animalia alia , \textbf{ a frigiditate } et ab intemperie temporis magis habet offendi , \\\hline
2.1.1 & para estas cosas \textbf{ siguese que el o omne ha natural inclinamiento } para ser conp̃anero & sequitur \textbf{ quod homo naturalem impetum habeat } ut sit animal sociale ; \\\hline
2.1.1 & para nuestro defendemiento . \textbf{ Par la qual cosa si natural cosa es al omne de dessear conseruaçion } e guarda de su uida & fabricare valemus . \textbf{ Quare si naturale est homini desiderare conseruationem vitae , } cum homo solitarius non sufficiat sibi ad habendum congruum victum \\\hline
2.1.1 & por los quales se pueda defender de los enemigos . \textbf{ por ende natural cosa es ael } que dessee beuir en conpanna & et ad fabricandum sibi arma et organa , per quae a contrariis defendatur : \textbf{ naturale est ei , ut desideret viuere in communitate , } et ut sit animal sociale . \\\hline
2.1.1 & avn que nunca aya visto otras arannas texer \textbf{ en essa misma manera } avn las golondrinas fazen su nido conueniblemente & ut aranea ex instinctu naturae debitam telam faceret , \textbf{ si nunquam vidisset araneas alias texuisse . Sic etiam et hirundines debite facerent nidum , } si nunquam vidissent alias nidificasse . \\\hline
2.1.1 & si non biuieremos en vno con los otros omes . \textbf{ Por ende natural cosa es al omne de beuir con los otros omes } e de ser aia la conpannable¶ & cum aliis conuiuamus : \textbf{ naturale est homini simul conuiuere cum aliis , } et esse animal sociale . \\\hline
2.1.2 & Et pues que assy es deuedes saber \textbf{ que si nos cuydamos con grant acuçia los dichos del philosofo en las politicas paresçra } que son quatro las comuindades Conuiene a saber . & quia per hoc manifeste ostenditur necessariam esse communitatem domesticam : \textbf{ cum omnis alia communitas communitatem illam praesupponat . Aduertendum ergo quod si dicta Politica diligenter consideremus , } apparebit quadruplicem esse communitatem ; videlicet , domus , \\\hline
2.1.2 & que pueden fazer çibdat e regno . \textbf{ Et por ende la comunidat dela casa sea alas trͣs comunidades } en tal manera que todas las otras comunidades ençierren & et regnum . \textbf{ Hoc ergo modo communitas domus se habet ad communitates alias : } quia omnes aliae ipsam praesupponunt : \\\hline
2.1.2 & por el philosofo en el primero libro delas politicas \textbf{ ca uatural nasçemiento } e comienço delas çibdades en esta manera . & ut patet per Philosophum 1 Politicorum , \textbf{ hoc modo existit , } quia primo facta fuit una aliqua domus : \\\hline
2.1.2 & las quales llaman algunos nietos e fijos e fijos de fijos . \textbf{ Et pues que assi es natural fazimiento e comienço del uarrio } es por vezindat delas casas & et pueros puerorum . \textbf{ Naturalis ergo origo vici , } est ex conuicinia domorum , \\\hline
2.1.3 & por figera e por exienplo \textbf{ que conuiene alos omes de auer conueibles moradas } segunt el su poder e la su riqueza . & quia spectat ad ipsum uniuersaliter et typo ostendere , \textbf{ quod decet homines habere habitationes decentes } secundum suam possibilem facultatem ; \\\hline
2.1.3 & Mas la comunidat del regno es fin de todas las otras comuindades sobredichas . \textbf{ Por la qual cosa conmola cosa non acabada sea primero que la acabada por generacion e por tienpo } e la cosa acabada sea primero & sed communitas regni est finis omnium praedictorum . \textbf{ Quare cum imperfectum in via generationis } et imperfectionis praecedat perfectum , \\\hline
2.1.3 & que la çibdat se faze de muchos uarrios \textbf{ assi comm̃el uarrio se faze de muchas casas . } Et por ende deue se entender & cum ipsemet dicat ciuitatem procedere ex multiplicatione vici , \textbf{ sicut et vicus procedit | ex multiplicatione domorum , } intelligendum est ergo hoc de prioritate perfectionis , \\\hline
2.1.3 & e dela çibdat ante ponen el gouernamiento dela casa \textbf{ Et ela conosçimiento del gouernamento dela casa ¶ Por que nunca ninguno puede ser buen gouernador del regno o dela çibdat } si non sopiere bien gouernar & sic regimen regni et ciuitatis praesupponit notitiam regiminis domus , \textbf{ et personae propriae : | nunquam enim quis debitus rector regni vel ciuitatis efficitur , } nisi se et suam familiam sciat debite gubernare . \\\hline
2.1.3 & e ala çibdat \textbf{ assi commo es declarado en este presente capitulo . } Ca por esta aur̃a muy grant manera para saber et buscar el gouernamiento dela çibdat e del regno . & ut se habet ad regnum et ciuitatem , \textbf{ ut est in praesenti capitulo declaratum : } nam per hoc magnam viam habebunt ad inuestigandum regimen ciuitatis \\\hline
2.1.3 & assi commo es declarado en este presente capitulo . \textbf{ Ca por esta aur̃a muy grant manera para saber et buscar el gouernamiento dela çibdat e del regno . } Enpero deuedes saber con grant acuçia & ut est in praesenti capitulo declaratum : \textbf{ nam per hoc magnam viam habebunt ad inuestigandum regimen ciuitatis } et regni . Est tamen diligenter notandum quod licet quodam speciali \\\hline
2.1.3 & Ca por esta aur̃a muy grant manera para saber et buscar el gouernamiento dela çibdat e del regno . \textbf{ Enpero deuedes saber con grant acuçia } que commo quier que en alguna manera espeçial & ut est in praesenti capitulo declaratum : \textbf{ nam per hoc magnam viam habebunt ad inuestigandum regimen ciuitatis } et regni . Est tamen diligenter notandum quod licet quodam speciali \\\hline
2.1.3 & que commo quier que en alguna manera espeçial \textbf{ e alta part enesça alos Reyes } e alos prinçipes de entender & nam per hoc magnam viam habebunt ad inuestigandum regimen ciuitatis \textbf{ et regni . Est tamen diligenter notandum quod licet quodam speciali } et excellenti modo \\\hline
2.1.4 & que son neçessarias para toda la uida . \textbf{ Otrossi . por que contesçe que las çibdades han entressi guerras prouechosa cosa es a vna çibdat } para que pueda lidiar con otra çibdat & ad totam vitam . \textbf{ Rursus autem , | quia contingit ciuitates habere guerras , } utile est uni ciuitati ad expugnandam ciuitatem \\\hline
2.1.4 & que es amistança de muchos castiellos \textbf{ e de muchͣs çibdades so vn prinçipe o so vn Rey . } Et pues que assi es esta es la ordende todas estas cosas & et ciuitatis , inuenta fuit communitas regni et principatus , \textbf{ quae est confoederatio plurium castrorum et ciuitatum existentium sub uno principe siue sub uno Rege . Erit ergo hic ordo , quod domus est communitas } secundum naturam constituta in omnem diem . Vicus autem est communitas constituta in opera non diurnalia . \\\hline
2.1.4 & por las cosas sobredichͣs la casa es vna comunidat \textbf{ e vna conpannia de muchͣs ꝑssonas . } Et commo non sea propreamente comunidat nin conpannia de vno & ( ut patet ex habitis ) \textbf{ sit communitas quaedam et societas personarum : } cum non sit proprie communitas nec societas ad seipsum , \\\hline
2.1.4 & assi commo si queremos saluar la comuidat dela casa conuiene \textbf{ que ella sea establesçida de muchͣs perssonas } mas assi commo adelanţe paresçra & si in domo communitatem saluare volumus , \textbf{ oportet eam ex pluribus constare personis ; } immo ( \\\hline
2.1.4 & Et por ende paresçe \textbf{ quela casa es establesçida de muchͣs perssonas } e ahun qual es & sed in domo oportet dare plures communitates : \textbf{ quod sine pluralitate personarum esse non potest . Patet ergo quod domus ex pluribus constat personis . Patet } etiam qualis sit , \\\hline
2.1.4 & Mas quantoalo presente abasta de dezir \textbf{ en tantodel regno e dela çibdat } en quanto el conosçimiento dellos siruen el en alguna manera al conosçimiento dela casa . & in tertio libro plenius ostendetur . \textbf{ Ad praesens autem sufficiat in tantum tangere de regno et ciuitate , } in quantum eorum notitia aliquo modo deseruit ad cognoscendum domum , \\\hline
2.1.4 & Et tanto mas esto parte nesçe alos Reyes e alos prinçipes \textbf{ quanto por mal gouernamiento de su casa proprea } mas se puede leuna tar piuyzio ala çibdat e al regno & tanto tamen magis hoc spectat ad Reges et Principes , \textbf{ quanto ex incuria propriae domus magis potest insurgere praeiudicium ciuitati et regno , } quam ex incuria aliorum . \\\hline
2.1.4 & mas se puede leuna tar piuyzio ala çibdat e al regno \textbf{ por mal gouernamiento de los otros . } l philosofo en el primero libro delas politicas & quanto ex incuria propriae domus magis potest insurgere praeiudicium ciuitati et regno , \textbf{ quam ex incuria aliorum . } Philosophus 1 Politic’ vult , \\\hline
2.1.5 & e la mantenençia delas cosas es cosanatal . \textbf{ Ca en vano seria algua cosa engendrada naturalmente } si non fuesse conseruada & quid naturale est . \textbf{ Nam frustra esset aliquid naturaliter generatum , } si non posset in esse conseruari . \\\hline
2.1.5 & si non fuere ante ella la generaçion dellas . \textbf{ avn en essa misma manera la generaçion non deue ser apartada dela conseruaçion . } pues assi commo dicho es de ssuso & nisi ipsam earum generatio antecedat . \textbf{ Sic etiam generatio a conseruatione separari non debet : } quia ( ut dicebatur ) frustra esset aliquid generatum , \\\hline
2.1.5 & Ca aquel propreamente es sennor \textbf{ que ha meior entendemiento . } Mas aqueles propraamente sieruo segunt dize el philosofo en el primero libro delas politicas & Nam ille proprie est dominus , \textbf{ qui viget intellectu : } ille vero est proprie seruus \\\hline
2.1.5 & que fallesçe en el entondemiento \textbf{ e ha mayor fortalleza en el cuerpo . } Pues que assi es & qui deficiens intellectu , \textbf{ pollet fortitudine corporali Seruus ergo , } quia seipsum nescit dirigere , \\\hline
2.1.5 & si quando anda non fue regua ado \textbf{ por alguno otro de ligero estropeçara en alguna cosa } qual fara danno & nisi ( cum pergit ) \textbf{ dirigatur ab aliquo , | de leui obuiat alicui offensiuo : } et quanto fortius pergit , \\\hline
2.1.5 & por que mas ayna se puede ferir \textbf{ en essa misma manera el sieruo } por quenatraalmente ha mayor fuerca en el cuerpo & tanto magis indiget tali dirigente , \textbf{ quia magis offendi potest . Sic quia naturaliter seruus pollet viribus , } et deficit scientia , \\\hline
2.1.5 & en essa misma manera el sieruo \textbf{ por quenatraalmente ha mayor fuerca en el cuerpo } e fallesçe enla sabiduria e enl entendemiento & ø \\\hline
2.1.5 & que sea seruido del sieruo . \textbf{ Por que naturalmente los sennores han mayor sabiduria } e mayor entendemiento & ut ei seruiatur a seruo : \textbf{ nam naturaliter domini vigent prudentia , et intellectu . } Cum ergo vigentes intellectu , \\\hline
2.1.5 & Por que naturalmente los sennores han mayor sabiduria \textbf{ e mayor entendemiento } por la qual cosa commo los que han mayor entendemiento & ut ei seruiatur a seruo : \textbf{ nam naturaliter domini vigent prudentia , et intellectu . } Cum ergo vigentes intellectu , \\\hline
2.1.5 & e mayor entendemiento \textbf{ por la qual cosa commo los que han mayor entendemiento } o son mas sotiles en el alma & nam naturaliter domini vigent prudentia , et intellectu . \textbf{ Cum ergo vigentes intellectu , } et existentes apti mente , \\\hline
2.1.5 & o son mas sotiles en el alma \textbf{ e por la mayor parte ayan las carnes blandas } e falleztan en las fuerças corporales & et existentes apti mente , \textbf{ ut plurimum habeant molles carnes , } et deficiant corporalibus viribus \\\hline
2.1.5 & es men ester la comunidat del uaron e dela mugni \textbf{ para la generaçion en essa misma manera es y . } menester la comunidat del sennor e del sieruo & quia sicut ad constitutionem domus requiritur communitas viri \textbf{ et uxoris propter generationem , } sic requiritur ibi communitas domini \\\hline
2.1.5 & que la casa primera \textbf{ por razon destas dos comini dades ha menester quatro linages de ꝑssonas . } Assi que la primera perssona sera el uaron . & uel duo genera personarum , \textbf{ quod domus prima ratione duarum communitatum requirat quatuor genera personarum , } ita quod prima persona erit uir , \\\hline
2.1.5 & Pues que assi es tres linages de perssonas \textbf{ o alguna cosa en logar dellas establesçen la prima casa } e por tantodezimos & Tria ergo genera personarum , \textbf{ uel aliquid loco eorum constituunt domum primam . Dicimus autem aliquid loco eorum , } quia non semper seruus concurrit ad constitutionem domus , \\\hline
2.1.5 & por que los omes pobres \textbf{ assi commo esse mismo philosofo dio a entender } que en logar de sieruo han el buey & quia pauperes homines \textbf{ ( ut idem Philosophus innuit ) loco serui habent bouem , } vel habent \\\hline
2.1.6 & Ca ueemos en las cosas naturales \textbf{ que luego que son engendradas las cosas dla natura } luego es acuçiosa dela salud & scilicet patris et filii . \textbf{ Videmus enim in naturalibus rebus quod statim quum generatae sunt , } natura est solicita de eorum salute : \\\hline
2.1.6 & que estosson del fazimiento dela primera casa \textbf{ ca sin ellas non puede ser la primera casa conuenible mente . } Mas la terçera comunidat & esse de ratione domus primae , \textbf{ quia sine eis domus congrue esse non potest : } sed tertiam etiam communitatem , \\\hline
2.1.6 & Ca la perfection e el cunplimiento deuemos la tomar dela natura \textbf{ e dela forma de essa misma cosa . } por la qual toda cosa es en su ser conplido & sed quod possit sibi simile producere : perfectio enim consideranda est ex natura et ex forma \textbf{ rei , per quam aliquid est in actu , } et potest agere ; \\\hline
2.1.6 & e es seneriego e sin fijos en ninguna manera non puede ser bien andante . \textbf{ Ca tal commo este en toda manera es dicho mal andante } por que non ha la bien andança çiuil con todas las cosas & ignobilis , \textbf{ et solitarius , } et sine filiis , \\\hline
2.1.6 & Et destas cosas puede paresçer \textbf{ que en la casa acabada deuen ser tres gouernamientos . } Ca nunca podemos dar comunindat ninguna bien ordenada & Ex his autem patere potest , \textbf{ quod oportet in domo perfecta esse tria regimina . } Nam nunquam est dare communitatem aliquam bene ordinatam , \\\hline
2.1.6 & commo en la casa acabada \textbf{ ian de ser tres comuindades } e tres gouernamientos de ligero puede paresçer & secundum quod dominus praeest seruis . Viso , \textbf{ in domo perfecta esse communitates tres , } et tria regimina : \\\hline
2.1.6 & e tres gouernamientos de ligero puede paresçer \textbf{ que conuiene que sean y . quatro linages de perssonas } Empero podrie paresçer a alguno por auentura que deuen y ser seys linages de perssonas alssi que la primera perssona deue ser ꝑ el uaron¶ La segunda dela muger ¶ & de leui patere potest , \textbf{ quod ibi oportet esse quatuor genera personarum . Videretur } tamen forte alicui ibi debere esse sex genera personarum , \\\hline
2.1.6 & que conuiene que sean y . quatro linages de perssonas \textbf{ Empero podrie paresçer a alguno por auentura que deuen y ser seys linages de perssonas alssi que la primera perssona deue ser ꝑ el uaron¶ La segunda dela muger ¶ } Et la terçera el padre . & quod ibi oportet esse quatuor genera personarum . Videretur \textbf{ tamen forte alicui ibi debere esse sex genera personarum , | ita quod prima persona sit ibi vir , secunda uxor , } tertia pater , quarta filius , \\\hline
2.1.6 & Mas esto se suelue \textbf{ que el uaron e el padre e el sennor nonbran vna perssona sola mente . } Ca aquel mismo que es uaron dela mugeres padre de los fijos & Sed vir , pater , \textbf{ et dominus unam tantum personam nominant . | Nam ille idem , } qui est vir uxoris , \\\hline
2.1.6 & Et por ende de ligero puede paresçer quantas partes deuen auer este segundo libro \textbf{ en que trảctaremos eł gouertiamiento dela casa } Ca commo en la casa acabada sean tres gouernamientos . & ø \\\hline
2.1.6 & e alos prinçipes de conosçer bien estos tres gouernamientos \textbf{ por que estos gouernamientos catados con grant acuçia auran grant ayuda } por que sepan bien gouernar sus regnos et sus çibdades & Haec autem tria regimina bene cognoscere maxime decet Reges et Principes ; \textbf{ quia eis diligenter inspectis , | magnum adminiculum habebunt , } ut bene sciant regere regnum , et ciuitatem . \\\hline
2.1.7 & en el qual tractaremos del gouernamiento dela casa \textbf{ segunt que enla casa han de ser tres gouernamientos . } Conuiene a saber que el vno es conuigal . & in quo agitur de regimine domus , \textbf{ secundum quod in ipsa domo tria contingit esse regimina ; } videlicet , \\\hline
2.1.7 & Ca primero diremos \textbf{ qual es este ayun tamiento } e quales mugieres deuen tomar & hunc tenebimus ordinem : quia primo dicemus , \textbf{ quale sit ipsum coniugium , } et quales uxores , ciues , \\\hline
2.1.7 & Ca prouado es en el primero deste segundo libro \textbf{ que el omne es naturalmente aina l aconpannable e comun incatiuo } que quiere dezir ꝑtiçipante con otro & Probabatur enim in primo capitulo huius secundi libri , \textbf{ hominem esse naturaliter animal sociale et communicatiuum . Communitas autem in vita humana } ( \\\hline
2.1.7 & assi commo dicho es dessuso \textbf{ aduze se aquetro linages o a quatro maneras ¶ } La vna escomunidat de casa¶ & ut supra tangebatur ) \textbf{ ad quadruplex genus reducitur : } quia quaedam est communitas domus , \\\hline
2.1.7 & Otrossi es cosa ordenada ala generaçion \textbf{ que conuiene mucho al mantene miento del humanal linage } e al bien comun . & etiam ad generationem , \textbf{ quae maxime expedit conseruationi speciei , } et bono communi . Vicus autem , ciuitas , \\\hline
2.1.7 & nin son ordenados a la generaçion \textbf{ e al mantenemiento del humanal linage } assi commo es ordenada la comunidat dela casa & et bono communi . Vicus autem , ciuitas , \textbf{ et regnum non sic immediate ordinantur ad nutritionem propter bonum personae propriae , et ad generationem propter conseruationem speciei , } sicut communitas domus : \\\hline
2.1.7 & por que la casa es por ende meior \textbf{ si es en buen uarrio o en buena çibdat o en buen regno ¶ } Et pues que assi es si el omne & ipsa enim domus est inde melior , \textbf{ sic est in bono vico , | vel in bona ciuitate , } aut in bono regno . \\\hline
2.1.7 & que de çibdat \textbf{ commo la primera comunidat dela casa sea ayuntamientode uaron } e de muger siguese de parte desta conñçion humanal & Si ergo homo magis est naturaliter animal domesticum , quam politicum : \textbf{ cum prima communitas ipsius domus sit coniunctio viri et uxoris , sequitur ex parte ipsius communitatis humanae , } quod homo magis sit animal coniugale quod politicum ; et quod magis sit \\\hline
2.1.7 & Ca aquella cosaparesçe ser mayormente natural \textbf{ ala qual el omne ha natural inclinaçion } e natra al apetito & Nam illud maxime videtur naturale , \textbf{ ad quod homo habet naturalem impetum : } quare cum homo \\\hline
2.1.7 & do prueua que el casamiento conuiene alos omes segunt natura \textbf{ por que natural cola es al omne } e atondas las aianlias auer natural inclinaçion & secundum naturam , \textbf{ quia naturale est homini , } et omnibus animalibus , \\\hline
2.1.7 & por que natural cola es al omne \textbf{ e atondas las aianlias auer natural inclinaçion } e appetito para engendrar cosa semeiable & quia naturale est homini , \textbf{ et omnibus animalibus , } habere naturalem impetum ad producendum sibi simile . \\\hline
2.1.7 & assi commo dize el philosofo en el viij ̊ libro delas ethicas . \textbf{ Ca ordenar assi los bienes propreos al bien comun fazen avn abastamientode uida . } Por la qual cosa si natural cosa es al omne de auer inclinaçion & ut dicitur 8 Ethic’ . \textbf{ Nam sic propria ordinare ad bonum commune , | facit ad quandam sufficientiam vitae . } Quare si naturale est homini , \\\hline
2.1.7 & Ca ordenar assi los bienes propreos al bien comun fazen avn abastamientode uida . \textbf{ Por la qual cosa si natural cosa es al omne de auer inclinaçion } e appetito al abastamiento dela uida natural cosa es a el de querer ser a i al conuigable & facit ad quandam sufficientiam vitae . \textbf{ Quare si naturale est homini , } habere impetum ad sufficientiam vitae : \\\hline
2.1.7 & Por la qual cosa si natural cosa es al omne de auer inclinaçion \textbf{ e appetito al abastamiento dela uida natural cosa es a el de querer ser a i al conuigable } e ayuntable a su muger & Quare si naturale est homini , \textbf{ habere impetum ad sufficientiam vitae : | naturale est ei , } quod velit esse animal coniugale . \\\hline
2.1.7 & La qual fornicaçion \textbf{ e general mente todo vso de luxuria non conueinble tanto } mas conuiene alos Reyes & quam videlicet fornicationem , \textbf{ et uniuersaliter omnem venereorum usum illicitum , } tanto magis decet fugere Reges et Principes , \\\hline
2.1.7 & alo que ya dicho es de ligero se puede soluer . \textbf{ Ca si natural cosa es al omne } de ser aianlia conuigable & de leui refellitur . \textbf{ Nam si naturale est homini esse animal coniugale , } quicunque renuit coniugem ducere , non viuit \\\hline
2.1.7 & este tal o es bestia o es assi commo dios \textbf{ En essa misma manera podemos dezir del mater moion } que aquel que non quiere beuir conuigalmente e con su muger & uel est deus : \textbf{ sic et de coniugio dicere possumus . Nam nolens coniugaliter uiuere , } uel hoc est , \\\hline
2.1.7 & Et pues que assi es los que non quieren casar \textbf{ e se dana mayores bienes } que son los bienes del casamiento & Non nubentes ergo , \textbf{ si dent se potioribus bonis quam sint bona coniugii , } licet non uiuant \\\hline
2.1.8 & Ca commo entre el uaron e la muger sea amistança natural \textbf{ assi commo se prueua en el viij̊ libro delas ethicas } commo non sea natural amistan & et econuerso . Cum enim inter virum et uxorem sit amicitia naturalis , \textbf{ ut probatur 8 Ethicorum , } cum non fit naturalis amicitia \\\hline
2.1.8 & commo non sea natural amistan \textbf{ ca entre algunos si non guardaren assi mesmos fe conuenible } para el casamiento & cum non fit naturalis amicitia \textbf{ inter aliquos nisi obseruent sibi debitam fidem ; } ad hoc quod coniugium sit \\\hline
2.1.8 & assi commo dela razon del bien propreo es que ayunte e desayunte el vno del otro . \textbf{ Et esta razon pone el philosofo en el viii̊ libro delas ethicas } do dize que por que el bien comunal contiene & sicut de ratione proprii , \textbf{ est quod diuidat et distinguat . Hanc autem rationem tangit Philosophus 8 Ethic’ dicens , } quod quia commune continet et coniungit , \\\hline
2.1.8 & Mas coͣtra odo amor aya alguna fuerça \textbf{ para ayuncar los omes el actes çentamiento del amor } por la genneraçion de los fijos acresçiental a uoluntad dellos & augmentatur eorum amicitia naturalis . \textbf{ Sed cum omnis amor vim quandam unitiuam dicat , } augmentato amore propter prolem genitam , \\\hline
2.1.8 & por la qual cosa \textbf{ si el amor e el acuçia de los fijos faza mayor ayuntamiento del casamiento } quanto mayor es de tomar la cura & inferre nocumenti ipsi regno , quam incuria cuiuscunque alterius . \textbf{ Quare si amor et diligentia circa prolem facit | maiorem unionem coniugum : } quanto maior cura et diligentia adhibenda est circa prolem regiam quam circa alias , \\\hline
2.1.9 & e algunas sectas non los iudgan contra razon \textbf{ que vn omne aya muchͣs mugers } mas lo que dizela razon derecha es & Apud aliquas sectas non reputatur contra dictamen rationis , \textbf{ quod unius et eiusdem viri simul plures existant uxores . } Sed quod recta ratio dictat \\\hline
2.1.9 & La primera se prueua assi . \textbf{ Ca assi commo la muchedunbre de los maniares trahe al omne a grant inchimiento } assi la muchedunbre de las mugieres trahe al omne a grand cobdiçia de luyuria . & Prima via sic patet . \textbf{ Nam | sicut pluralitas ciborum prouocat } ad nimiam repletionem , \\\hline
2.1.9 & Ca assi commo la muchedunbre de los maniares trahe al omne a grant inchimiento \textbf{ assi la muchedunbre de las mugieres trahe al omne a grand cobdiçia de luyuria . } Por la qual cosa commo estas tales cobdiçias de luxias & sicut pluralitas ciborum prouocat \textbf{ ad nimiam repletionem , | sic pluralitas mulierum prouocat ad nimiam concupiscentiam venereorum : } Quare cum huiusmodi concupiscentiae \\\hline
2.1.9 & non es cosa conuenible a ellos \textbf{ de auer muchͣs mugieres . } Enpero tanto esto es mas desconuenible alos Reyes e alos prinçipes & et ab operibus ciuilibus , \textbf{ indecens est eos plures habere coniuges . } Tamen tanto hoc indecens est magis Regibus , \\\hline
2.1.9 & por que non sean enbargados \textbf{ por el grand huso de luxa a del cuydado } que deuen tomar en el gouernar aiento del regno non les conuiene de auer muchͣs mugiers¶ & quam aliqui aliorum . \textbf{ Ne ergo per nimiam operam venereorum nimis retrahantur ab huiusmodi cura , indecens est eos plures habere uxores . } Secunda via sumitur ex parte ipsius uxoris . \\\hline
2.1.9 & Ca assi commo de parte del uaron \textbf{ es cosa desconuenible de auer muchͣs mugiers } por que por el guaadhuso dela luxia el marido non sea enbargado en el cuydado & Secunda via sumitur ex parte ipsius uxoris . \textbf{ Nam sicut ex parte viri indecens est uxorum pluralitas , } ne propter nimiam operam venereorum vir a cura debita retrahatur : \\\hline
2.1.9 & quel conuiene de auer . \textbf{ En essa misma manera esto es desconueinble de parte dela muger } por que la muger non sea desamada & ne propter nimiam operam venereorum vir a cura debita retrahatur : \textbf{ si hoc indecens est parte uxoris , } ne uxor a suo coniuge non debite diligatur . \\\hline
2.1.9 & Ca entre la mugni \textbf{ e el uaron deue ser grand amor } ca assi commo dize el philosofo en el octauo libro & ne uxor a suo coniuge non debite diligatur . \textbf{ Nam inter uxorem et virum debet esse amor magnus , } quia inter eos \\\hline
2.1.9 & delas ethicas entre ellos es amistança muy grande e muy natural . \textbf{ Mas commo el grand amor non pueda ser departido amuchͣs partes } assi conmo dize el philosofo en elix̊ . delas ethicas cosa desconuenible es & et naturalis . \textbf{ Sed cum excellens amor non possit esse ad plures , } ut vult Philosophus 9 Ethicor’ , indecens est quoscunque ciues plures habere uxores : \\\hline
2.1.9 & Mas commo el grand amor non pueda ser departido amuchͣs partes \textbf{ assi conmo dize el philosofo en elix̊ . delas ethicas cosa desconuenible es } a quales si quier çibdadanos & Sed cum excellens amor non possit esse ad plures , \textbf{ ut vult Philosophus 9 Ethicor’ , indecens est quoscunque ciues plures habere uxores : } quia eas non tanta amicitia diligerent , \\\hline
2.1.9 & a quales si quier çibdadanos \textbf{ e a quales se quier uatones de auer muchͣs mugieres } por que non las amarien de tan grand amor quanto deue ser entre los maridos e las mugers . & ut vult Philosophus 9 Ethicor’ , indecens est quoscunque ciues plures habere uxores : \textbf{ quia eas non tanta amicitia diligerent , } quanta \\\hline
2.1.9 & e a quales se quier uatones de auer muchͣs mugieres \textbf{ por que non las amarien de tan grand amor quanto deue ser entre los maridos e las mugers . } Et esto mayormente es desconuenible alos Reyes e alos prinçipes & ut vult Philosophus 9 Ethicor’ , indecens est quoscunque ciues plures habere uxores : \textbf{ quia eas non tanta amicitia diligerent , } quanta \\\hline
2.1.9 & por aquellas cosas que veemos en las otrasaianlias . \textbf{ Ca veemos que assi commo comunalmente dizen los otros doctores que en algunas ainalias vna sola fenbra abasta } para cança de muchos fijos . & per ea quae in aliis animalibus conspicimus . Videmus autem \textbf{ ( ut communiter alii doctores tradunt ) } quod in aliquibus animalibus sola foemella sufficit ad nutritionem filiorum , \\\hline
2.1.9 & si mientra dura el parto si el mas so biua en vno con la fenbra \textbf{ Mas en aquellas aina lias } en las quales vna fenbra sola non abasta & utrum durante partu masculus conuiuat foeminae . \textbf{ Sed in illis , } in quibus sola foemina non sufficit ad praestandum filiis debitum nutrimentum , \\\hline
2.1.9 & en las quales vna fenbra sola non abasta \textbf{ para dar conuenible nudͣmiento alos fijos } vn mas lo non se ayunta sinon a vna fenbra & Sed in illis , \textbf{ in quibus sola foemina non sufficit ad praestandum filiis debitum nutrimentum , } unus masculus non adhaeret \\\hline
2.1.9 & Ca nos deuemos iudgar las cosas naturales \textbf{ segunt que son en la mayor parte } assi commo dezimos & Ea enim naturalia iudicare debemus \textbf{ quae sunt | ut in pluribus , } ut naturale est homini quod sit dexter , \\\hline
2.1.9 & assi commo dezimos \textbf{ que natural cosa es al omne } de ser diestro & ut in pluribus , \textbf{ ut naturale est homini quod sit dexter , } licet contingat aliquos esse sinistros . \\\hline
2.1.9 & commo quier que contezca a algunos de ser esquierdos en essa manera \textbf{ por que en la mayor parte vna sola fenbra non puede sofrir las cargas del matermonio } nin abonda & licet contingat aliquos esse sinistros . \textbf{ Sic quia ut in pluribus sola foemina non potest portare onera matrimonii , } nec sufficit ad praestandum filiis omnia necessaria \\\hline
2.1.9 & Por ende commo quier que por auentura algunas muger s abondan en rriquezas abastarian \textbf{ para dar conuenible nudermiento alos fijos . } Empero por que non es iudgar la cosa natural & quia facultatibus abundant sufficerent \textbf{ ad praestandum filiis debitum nutrimentum , } quia tamen naturale non est iudicandum illud quod est in paucioribus , \\\hline
2.1.9 & que sienpre se ayunte vn \textbf{ mas lo a vna sola fenbra por matrimoio . } Ca non es assi de los omes & Sed filii quam diu viuunt indigent ope parentum , decens est \textbf{ ut semper per coniugium humanum unus masculus uni soli foeminae copuletur . } Non enim sic est de hominibus sicut de animalibus aliis , \\\hline
2.1.9 & que alos pollos delas aues cresçieren pennolas conuenibles \textbf{ e vinieren acres çentamiento conueni e e ble } por si mesmos pueden buscar su iuda conuenible & Postquam ergo pulli auium apposuerunt debitas pennas , \textbf{ et peruenerunt ad debitum incrementum : } per seipsos possunt sibi debitum cibum quaerere , \\\hline
2.1.9 & e segunt ordenna traal . \textbf{ Conuiene que todos los çibdadanos sean pagados cada vno de vna sola mugier . } Et tanto esto mas pertenesçe a los Reyes & secundum modum , \textbf{ et ordinem naturalem , decet omnes ciues una sola uxore esse contentos . } Et tanto magis hoc decet Reges et Principes , \\\hline
2.1.10 & assi commo entre los moros . \textbf{ Et por auentraa commo entre alguna sts̃ naçiones baruaras non se tenido } por desconuenible & ut apud Sarracenos , \textbf{ et apud forte aliquas alias barbaras nationes non reputetur in congruum unum virum } etiam simul habere plures uxores ; \\\hline
2.1.10 & por desconuenible \textbf{ que vn uaron aya muchͣs mugieres en vno . } Empero entre ningunas gentes & et apud forte aliquas alias barbaras nationes non reputetur in congruum unum virum \textbf{ etiam simul habere plures uxores ; } apud nullas tamen gentes \\\hline
2.1.10 & que vna muger sea casada en vno con muchos maridos \textbf{ que vn uaroncase con muchͣs mugers . ante han en la ley de dios } assi commo en el testamento vieio & unam foeminam simul pluribus nubere viris , \textbf{ quam unum virum pluribus foeminis . Immo , | et in lege diuina , } ut in veteri Testamento , \\\hline
2.1.10 & cosa de denostares \textbf{ que vn uaron aya en vno muchͣ̃s mugers . } Empero mas de denostares & Secundum enim commune dictamen rationis detestabile est unum virum simul plures habere uxores : \textbf{ detestabilius tamen esset , } si una foemina per coniugium simul pluribus copularetur viris . Coniugium enim ad quatuor comparari potest , \\\hline
2.1.10 & Empero mas de denostares \textbf{ que vna muger aya muchos maridos en vno . Ca el casamiento puede ser conparado a quatro cosas } delas quales podemos tomar quatro razones & detestabilius tamen esset , \textbf{ si una foemina per coniugium simul pluribus copularetur viris . Coniugium enim ad quatuor comparari potest , } ex quibus sumi possunt quatuor rationes , \\\hline
2.1.10 & que vna muger aya muchos maridos en vno . Ca el casamiento puede ser conparado a quatro cosas \textbf{ delas quales podemos tomar quatro razones } Por las quales podemos prouar & si una foemina per coniugium simul pluribus copularetur viris . Coniugium enim ad quatuor comparari potest , \textbf{ ex quibus sumi possunt quatuor rationes , } per quas inuestigare possumus , \\\hline
2.1.10 & Ca nos veemos \textbf{ que muchͣs guerras e muchͣs discordias se amanssan } por que se tractan casamientos entre las partes ¶ Lo terçero del casamiento nasçen los fijos & et amicitia : \textbf{ multas enim guerras , | et multas discordias sedari videmus } eo quod inter partes contrahuntur coniugia . Tertio , \\\hline
2.1.10 & assi commo el casamiento es ordenado a generaçion de los fijos \textbf{ assi es ordenado anudermiento conuenible dellos . } Et por ende non es cosa conueniente & ad procreationem filiorum . Quarto , \textbf{ sicut ordinatur coniugium ad filiorum procreationem : } sic ordinatur ad eorum debitam nutritionem . Inconueniens est ergo unam foeminam , coniugem esse plurium virorum , \\\hline
2.1.10 & que vna fenbra sea muger de muchos uarones . \textbf{ Ca por esto se enbargarian le todas las quatro cosas sobredichͣs } e por esto se tolleria la orden natural & sic ordinatur ad eorum debitam nutritionem . Inconueniens est ergo unam foeminam , coniugem esse plurium virorum , \textbf{ quia per hoc omnia praedicta quatuor impediuntur . } Tolletur enim ex hoc naturalis ordo ; \\\hline
2.1.10 & e por esto se tolleria la concordia e la paz e nasçria dende discordia e enemistança \textbf{ nin seria y conuenible generaçion de los fijos } nin les seria dado alos fijos conuenible nudermiento & sed magis dissensio et inimicitia ; \textbf{ non erit debita ibi procreatio filiorum , } non tribuetur filiis debitum nutrimentum . \\\hline
2.1.10 & nin seria y conuenible generaçion de los fijos \textbf{ nin les seria dado alos fijos conuenible nudermiento } Et & non erit debita ibi procreatio filiorum , \textbf{ non tribuetur filiis debitum nutrimentum . } Quod autem ex hoc tollatur naturalis ordo , \\\hline
2.1.10 & e la mugni deue ser subiecta e obediente al uaron ¶ \textbf{ Otrossi segunt la orden natural en essas mismas obras } ninguon non puede ser subiecto & Rursus \textbf{ secundum ordinem naturalem in eisdem operibus nullus aeque per se duobus } vel pluribus potest esse subiectus : \\\hline
2.1.10 & Enpero mas de denostar es \textbf{ que vna muger case en vno con mugons varones . } ¶ Et pues que assi es conuiene & secundum naturalem ordinem esse non potest . Quare et si detestabile est plures foeminas coniuges esse unius viri , \textbf{ unam tamen nubere simul viris pluribus detestabilius esse debet . } Decet ergo coniuges omnium ciuium uno viro esse contentas : \\\hline
2.1.10 & por la qual cosa conuiene alas mugieres de todos los çibdadanos \textbf{ de ser pagadas de vn solo uaron . } Empero tanto o mas conuiene esto alas mugers de los Reyes e de los prinçipes & Quare decet coniuges omnium ciuium uno uiro esse contentas : \textbf{ tanto } tamen hoc magis decet coniuges Regum et Principum , \\\hline
2.1.10 & que las otras mugers . \textbf{ Et por ende departe dela generaçion de lons fijos es cosadesconuenible } en qua vna fenbra aya muchos maridos & unde et meretrices conspicimus esse magis steriles , \textbf{ quam alias mulieres . Igitur ex parte procreationis filiorum omnino indecens est unam foeminam plures habere uiros . } Nam etsi unus masculus potest plures foecundare foeminas : \\\hline
2.1.10 & quanto en tal casamiento es mas periglosa la manneria de los fijos ¶ \textbf{ Et pues que assi es cada vnas mugers deuen tener mientes con grand acuçia } que las que son ayuntadas a sus maridos & quanto in tali coniugio periculosior est sterilitas filiorum . \textbf{ Diligenter ergo aduertere debent singulae mulieres , } quae suis viris per coniugium copulantur , \\\hline
2.1.10 & Mas si vna fenbra casare con muchos uarones los padres non podrian ser çiertos de sus fijos . \textbf{ Et por ende non aurian tan grand cuydado } nin tan grand acuçia & patres de suis filiis certi esse non poterunt , \textbf{ quare non adhibebunt illam diligentiam } quam debent \\\hline
2.1.10 & Et por ende non aurian tan grand cuydado \textbf{ nin tan grand acuçia } commo deurien en el nudermiento conuenible de sus fijos & quare non adhibebunt illam diligentiam \textbf{ quam debent } ut suis filiis debite in nutrimento \\\hline
2.1.10 & Por la qual cosa sy conuiene a todos los çibdadanos ser çier tos de lus fios \textbf{ por que los puedan proueer con grand acuçia en las hedades e en el nudrimiento . } Conuiene alas mugers de todos los çibdadanos & de suis filiis , \textbf{ ut eis diligenter prouideant in haereditate et in nutrimento : } decet coniuges omnium ciuium uno viro esse contentas ; \\\hline
2.1.10 & Conuiene alas mugers de todos los çibdadanos \textbf{ de ser pagadas de vn solo uaron . } Et esto tanto conuiene mas alas mugers de los Reyes e de los prinçipes & ut eis diligenter prouideant in haereditate et in nutrimento : \textbf{ decet coniuges omnium ciuium uno viro esse contentas ; } tanto tamen hoc magis decet coniuges Regum , \\\hline
2.1.11 & nin con parientes ayuntados \textbf{ por grand parentesco . Esto podemos prouar } por tres razones ¶ & et quod cum parentibus et consanguineis nimia consanguineitate coniunctis non sit ineundum coniugium , \textbf{ triplici via venari possumus . } Prima sumitur ex debita reuerentia , \\\hline
2.1.11 & por tres razones ¶ \textbf{ La primera se toma dela grand reuerençia } e muy conuenible & triplici via venari possumus . \textbf{ Prima sumitur ex debita reuerentia , } quae est parentibus et consanguineis exhibenda . Secunda , \\\hline
2.1.11 & de casar con su padre nin alos fijos con su madre \textbf{ por la grand reuerençia } que les deuen & ø \\\hline
2.1.11 & e les son tenudos de fazer . \textbf{ Avn essa misma manera non les conuiene de casar con los parientes } que les son muy çercanos & quam sibi inuicem debent . \textbf{ Sic etiam non licet eis contrahere cum consanguineis aliis , } si sint eis nimia consanguineitate coniuncti , \\\hline
2.1.11 & tanto mas parte nesçe alos Reyes e alos prinçipes . \textbf{ Ca quanto son en mayor estado } e en mas alto grado & et Principes , \textbf{ quia quanto sunt in maiori statu et in altiori gradu , } tanto magis egent affinitate \\\hline
2.1.11 & Ca quanto son en mayor estado \textbf{ e en mas alto grado } tanto mas han menester cunnaderia & et Principes , \textbf{ quia quanto sunt in maiori statu et in altiori gradu , } tanto magis egent affinitate \\\hline
2.1.11 & e los prinçipes \textbf{ quanto estan en mas alto estado } tanto han menester mayores & ne quatiatur a vento . Reges ergo et Principes \textbf{ quanto in altiori gradu existunt , } tanto indigent habere maiores , \\\hline
2.1.11 & si grandes fueren çieguen la razon e el entendimiento \textbf{ assi commo dixiemos muchͣs uezes de suso . } Conuienea todos aquellos que quieren vsar de razon e de entendemiento & si nimiae sint , rationem percutiant , \textbf{ ut supra pluries diximus ; } expedit quibuslibet volentibus vigere ratione et intellectu , non nimiam operam dare venereis . \\\hline
2.1.11 & Conuienea todos aquellos que quieren vsar de razon e de entendemiento \textbf{ que non den grand obra adelectaçiones dela catue . Et pues que } assi es commo ayan natural amor las perssonas & ut supra pluries diximus ; \textbf{ expedit quibuslibet volentibus vigere ratione et intellectu , non nimiam operam dare venereis . } Cum ergo ad personas nimia affinitate coniunctas habeatur naturalis amor , \\\hline
2.1.11 & que non den grand obra adelectaçiones dela catue . Et pues que \textbf{ assi es commo ayan natural amor las perssonas } que son ayuntadas en grand parentesco & expedit quibuslibet volentibus vigere ratione et intellectu , non nimiam operam dare venereis . \textbf{ Cum ergo ad personas nimia affinitate coniunctas habeatur naturalis amor , } si supra amorem illum superaddatur amicitia coniugalis , \\\hline
2.1.11 & assi es commo ayan natural amor las perssonas \textbf{ que son ayuntadas en grand parentesco } si sobre aquel amor se añdiesse otra amistança & expedit quibuslibet volentibus vigere ratione et intellectu , non nimiam operam dare venereis . \textbf{ Cum ergo ad personas nimia affinitate coniunctas habeatur naturalis amor , } si supra amorem illum superaddatur amicitia coniugalis , \\\hline
2.1.11 & quanto ellos mas deuen ser conplidos de razon e de entendemiento \textbf{ e quanto mayor periglo se podeleunatar al regno } si los Reyes e los prinçipes non entendiessen con grand acuçia & quanto ipsi plus vigere debent prudentia et intellectu : \textbf{ et quanto maius periculum potest regno consurgere , } si Reges , et Principes circa salutem regni \\\hline
2.1.11 & e quanto mayor periglo se podeleunatar al regno \textbf{ si los Reyes e los prinçipes non entendiessen con grand acuçia } en la salud & et quanto maius periculum potest regno consurgere , \textbf{ si Reges , et Principes circa salutem regni } et circa ciuilia opera non diligenter intendant . \\\hline
2.1.12 & que algunos de los bienes son bienes del alma \textbf{ assi commo las uirtudes e las buenans costunbres } Et alguons son bienes del cuerpo . & Bonorum autem quaedam sunt bona animae , \textbf{ ut virtututes , | et boni mores : } quaedam vero sunt bona corporis , \\\hline
2.1.12 & e las otras cosas atales \textbf{ Et alguons bienes } que son dichos bienes de fuera & et cetera talia : \textbf{ quaedam autem dicuntur exteriora bona : } quae ( quantum ad praesens spectat ) in triplici genere habent esse . \\\hline
2.1.12 & entre todos los bienes de fuera . \textbf{ Et desi deuen enten dera muchedunbre de rriquezas } assi commo a cosa que se sigue . & quasi primo et per se : \textbf{ sed pluralitas diuitiarum est intendenda quasi ex consequenti . } Decet enim eos talem uxorem acceptare , \\\hline
2.1.12 & que sea noble por linage \textbf{ e por la qual gane muchos nobles amigos e poderosos } mas que por aquella muger ganen abondança de rriquezas . & quae sit nobilis genere , \textbf{ et per quam acquirant multos amicos nobiles | et potentes : } sed quod per uxorem illam acquiratur diuitiarum copia , \\\hline
2.1.12 & que ala muchedunbre delas rrianzas . \textbf{ Enpero a todas estas tres cosas deuen entender en algua manera . Ca el mater moion es ordenado } assi commo paresçe en las cosas sobredichͣs aconpannia conuenible & et ad sufficientiam vitae . \textbf{ Prout ergo coniugium ordinatur } ad debitam societatem , \\\hline
2.1.12 & Enpero a todas estas tres cosas deuen entender en algua manera . Ca el mater moion es ordenado \textbf{ assi commo paresçe en las cosas sobredichͣs aconpannia conuenible } e abien de paz e abastamiento dela uida . & Prout ergo coniugium ordinatur \textbf{ ad debitam societatem , } apud Reges , et Principes in sui , \\\hline
2.1.12 & que el casamiento deue ser segunt natura \textbf{ por que el omne naturalmente es aian la conpannable ama termoino } ca la primera con pama natural & secundum naturam , \textbf{ eo quod homo naturaliter esse animal sociale : prima autem naturalis societas ( ut patet per Philosophum in Polit’ ) } est maris \\\hline
2.1.12 & Mas esto non \textbf{ si asi el casamiento non fuesen ordenado a algua conpanna conuenible e natural . } ¶ Et pues que assi es commo deuidamente & viri , et uxoris . Hoc autem non esset , \textbf{ nisi coniugium ordinaretur in quandam societatem debitam | et naturalem . } Cum ergo debite et congrue nobili societur : \\\hline
2.1.12 & e da mandar mugers \textbf{ que sean de noble linage¶ } Lo segundo & et congruam , \textbf{ debent } sibi uxores quaerere quae sint ex nobili genere . Secundo propter esse pacificum quaerenda est amicorum multitudo . \\\hline
2.1.12 & que por desygualdat de los humores se leunata enfermedat e lid en el cuerpo del omne . \textbf{ En essa misma manera entre los omes } por las iniunas e desigualdades & quod propter inaequalitatem humorum consurgit infirmitas et pugna : \textbf{ sic et inter homines propter iniurias , } et inaequalitates quas inter se exercent , \\\hline
2.1.12 & por las cosas sobredichͣstanto \textbf{ mas loh̃a mester los Reyes e los prinçipes } quanto el estado dellos es ma salto & Hoc autem ( ut patet ex habitis ) \textbf{ tanto magis egent Reges , | et Principes , } quanto eorum status , \\\hline
2.1.12 & con que se puedan defender \textbf{ en quanto pueden ser conbatidos de mayores periglos } Et pues que assi es paresçe & quia altior magis indiget sustentamentis , \textbf{ et pluribus potest concuti infortuniis . } Patet ergo quomodo in coniuge regum , \\\hline
2.1.12 & en todo casamiento \textbf{ deue ser esquiuada la grand desigualeza del maridor dela muger . } Ca la desigualeza en sobrepuiança si quier sea segunt nobleza & ut inter eos sit pax \textbf{ et digna societas ; in omni coniugio nimia imparitas videtur esse vitanda . Nam imparitas in excessu , } siue sit \\\hline
2.1.12 & si quier sea segunt hedat \textbf{ En la mayor parte es razon de uaraia } Et avn es razon que los casados non se guarden fialdat & siue secundum aetatem , \textbf{ ut plurimum est causa litigii , } vel est causa \\\hline
2.1.12 & Mas vno se esforçara de enssennorear al otro \textbf{ mas que demanda la ley del mater moino . En essa misma manera avn si el uieio casare con la moça } por que las moças non se gozan dela conpania de los nieios & sed unus alteri , \textbf{ ultra quam leges coniugii requirant , dominari conabitur . Sic etiam si nimis senes iuuenculae nubat , } quia iuuenes societate senum non gaudent , \\\hline
2.1.13 & Et avn mostratemos entre quales perssonas deue ser el casamiento \textbf{ ca non de ue ser entre los que lon ayuitados } por grand parentesco & etiam \textbf{ inter quas personas debet esse coniugium , } quia non inter nimia propinquitate coniunctos . Ulterius autem declarauimus , \\\hline
2.1.13 & ca non de ue ser entre los que lon ayuitados \textbf{ por grand parentesco } e avn adelante declararemos quales bienes de fuera son de demandar enla muger . & etiam \textbf{ inter quas personas debet esse coniugium , } quia non inter nimia propinquitate coniunctos . Ulterius autem declarauimus , \\\hline
2.1.13 & en la mugier fermosura e grandeza . \textbf{ Mas quanto los bienes del alma mayor ment en paresçe } que deue ser demandada en la fenbra tenprança & et magnitudo : \textbf{ sed quantum ad bona animae , | maxime videtur } esse quaerendum in foemina quod sit temperata , \\\hline
2.1.13 & Por que los fijos enla quantidat del cuerpo \textbf{ en la mayor parte sallen ala madre } por que toda la sustançia del cuerpo en alguna manera la han dela madre . & ad bonum prolis . \textbf{ Nam filii in quantitate corporis ut plurimum matrizant , } quia totam corpulentam substantiam quodammodo habent a matre . \\\hline
2.1.13 & Et pues que assi es \textbf{ assi commo enlas o trisaian lias en la mayor parte de grand linage sallen grandes animalias } assi en los omes & Sicut ergo in aliis animalibus , \textbf{ ut plurimum ex magno genere magna procedunt : } sic et in hominibus , \\\hline
2.1.13 & si el padre e la madre fueren grandes los fijos \textbf{ en la mayor parte nasçen grandes de cuerpos . } Et pues que assi es conuiene a todos los çibdadanos & si parentes magni existant , \textbf{ filii utplurimum magni nascuntur . } Decet omnes ciues propter bonum prolis , \\\hline
2.1.13 & mas esto conuiene alos Reyes e alos prinçipes \textbf{ quanto ellos deuen auer mayor cuydado de sus fijos propreos } por que dellos cuelga el bien comun & tanto tamen magis hoc decet Reges et Principes , \textbf{ quanto ipsi circa proprios filios , } eo quod ex eis dependeat bonum commune \\\hline
2.1.13 & por que esto fazal bien de la generaçion dellos fijos . \textbf{ Ca assi commo en la mayor parte de los grandes nasçen quandes . } Assi de los fermosos nasçen los fermosos . Por la qual cosa si conuiene a todos los çibdadanos & ø \\\hline
2.1.13 & mas alos Reyes e alos prinçipes \textbf{ quanto la destenprança delas mugers dellos puede fazer mayor danno e enpeçemiento } que la destenprança delas mugers de los otros . & tanto tamen hoc decet Reges et Principes , \textbf{ quanto intemperantia coniugum ipsorum | plus nocumenti inferre potest , } quam intemperantia coniugum aliorum . \\\hline
2.1.13 & que las muger ssean tenpradas . \textbf{ Et avn les conuiene aellas de amar fazer buenas obras . } Ca quando alguna persona esta de uagar mas ligeramente es inclinada a aquellas cosas & Decet ergo coniuges temperatas esse . \textbf{ Decet eas etiam amare operositatem : } quia cum aliqua persona ociosa existat , \\\hline
2.1.13 & e conuenibles la uoluntad del entiende çerca otras cosas \textbf{ e el misino se pone en malos cuydados e en malos . penssamientos . } Et pues que assi es pare sçe delas colas ya dichas & et licitis exercitiis , \textbf{ eius mens vagatur circa alia et occupatur cogitationibus turpibus . Patet ergo ex iam dictis , } quale debet esse coniugium , \\\hline
2.1.13 & e mayormente los Reyes e los prinçipes se deuen auer en tomar sus mugers . \textbf{ Ca ante que las tomne deuen primero catar con grand diligençia } en qual manera son honrradas e conpuestas de los biens de fuera . & et maxime Reges et Principes se habere debent in ducendis uxoribus . \textbf{ Nam priusquam eas ducant , | diligenter debent primo inquirere , } qualiter sint ornatae exterioribus bonis : \\\hline
2.1.14 & Ca segunt los philosofos toda la manera del gouernamiento del mundo es fallada en vn omne . \textbf{ Et por ende los pp̃os llaman al omne menor mundo . } Ca assi conmo todo el mundo es gouernado e gado por vn prinçipe & secundum Philosophos ) modus regiminis uniuersi \textbf{ reseruatur in uno homine : unde et ab eis homo appellatur minor mundus . } Nam sicut totum uniuersum dirigitur uno Principe , \\\hline
2.1.14 & Ca enssennorear realmente es senoreat en toda manera \textbf{ e seg̃t aluedrio . } mas enllennore ar çiuilmente es enssennorear & Nam praeesse regaliter est praeesse totaliter et secundum arbitrium : \textbf{ praeesse vero politice est praeesse non totaliter nec simpliciter , } sed secundum quasdam conuentiones \\\hline
2.1.14 & Et por ende el sennorio del padre es dicho \textbf{ mas segina natura } que el sennorio matrimoinal . & Dicitur dominium paternale esse plus \textbf{ secundum naturam , } quam coniugale : \\\hline
2.1.14 & quanto mas ellos deuen guardar aquellas cosas \textbf{ que la orden e la razon natural muestra . } a dixiemos de suso & quanto ipsi plus obseruare debent quae dictat ordo \textbf{ et ratio naturalis . } Dicebatur superius in domo esse tria distincta regimina : \\\hline
2.1.15 & Ca toda la natura es mouida \textbf{ e gouernada de dios e de los angeles . Ende el sh̃o en el libro dela buena uentura dize } que los mouimientos naturales & et regitur a Deo , \textbf{ et a substantiis separatis : | unde et Philosophus in libro De bona fortuna ait , } Impetus naturales \\\hline
2.1.15 & Mas esto contesçia \textbf{ assi commo el philosofo dize en esso mismo logar } por que entre los barbaros non era ninguno natrealmente sennor & Sed hoc ideo contingebat \textbf{ ( ut recitatur ibidem ) } quia inter Barbaros nullus est naturaliter principans , \\\hline
2.1.15 & cosa muy desconuenble es a ellos de vsar delas muger \textbf{ sassi conmode sieruos . } Et tanto esto es mas desconueniente alos Reyes e alos prinçipes & et ordinem naturalem ; \textbf{ indecens est eos uti uxoribus tanquam seruis . } Tanto tamen hoc magis indecens est \\\hline
2.1.16 & genera les cerca las costunbres non son de despreçiar . \textbf{ Ca el non saber delas cosas genera les nos faria muchͣs uezes } que non sopiessemos las particulares . & quia \textbf{ ignorantia uniuersalium saepe facit particularia ignorare : } ipsis \\\hline
2.1.16 & en qual hedat deua ser vsado el casamiento . \textbf{ Ca el philosofo tanne en el . vi̊ libro delas politicas quatro razones } por que praeua & in qua aetate sit utendum coniugio . \textbf{ Tangit enim Philosophus 7 Polit’ | quatuor rationes probantes } quod in aetate nimis iuuenili non est utendum coniugio . \\\hline
2.1.16 & por que praeua \textbf{ que enla hedat de grand moçedat non deuamos vsar del casamiento ¶ } La primera razon se toma de parte del dannamiento delos fijos ¶ & quatuor rationes probantes \textbf{ quod in aetate nimis iuuenili non est utendum coniugio . } Prima ratio sumitur ex electione filiorum . Secunda , \\\hline
2.1.16 & quanto al cuerpo . \textbf{ Ca en la mayor parte son muy flacos de cuerpo } e non son acabados & laeduntur inde filii quantum ad corpus , \textbf{ quia ut plurimum sunt nimis debiles corpore et imperfecti : } et etiam laeduntur \\\hline
2.1.16 & ca los que assi nasçen \textbf{ quanto por la mayor parte fallesçen en razon e entendemiento } por que assi commo veemos en las otras cosas naturales & quia sic nascentes \textbf{ ut plurimum deficiunt ratione et intellectu . } Sic enim videmus in aliis rebus naturalibus , \\\hline
2.1.16 & Ca assi commo para escalentar es menester calentura \textbf{ si aquella calentura non es calentura acabada siguese que non es caliente acabada mente . En essa misma nanera avn pero que algunan cosa sea escalençada es mester } que sea apareiada & ut si ad calefactionem requiritur calidum , \textbf{ si illud sit imperfecte calidum , | sequitur quod imperfecte calefaciat . Sic } etiam quia \\\hline
2.1.16 & por la qual cosa quanto al cuerpo \textbf{ por ayuntamiento dela hedat muy de moços seleunata danno alos fijos . } En essa misma manera & sequitur quod producantur imperfecti et debiles corpore ; \textbf{ quare quantum ad corpora ex coniunctione nimis iuuenili , } consurgit laesio filiorum : sic ex tali coniunctione laeduntur filii non solum quantum ad corpus , \\\hline
2.1.16 & por ayuntamiento dela hedat muy de moços seleunata danno alos fijos . \textbf{ En essa misma manera } por tal ayuntamiento sallen los fijos menguados & sequitur quod producantur imperfecti et debiles corpore ; \textbf{ quare quantum ad corpora ex coniunctione nimis iuuenili , } consurgit laesio filiorum : sic ex tali coniunctione laeduntur filii non solum quantum ad corpus , \\\hline
2.1.16 & ¶ Et pues que assi es conuiene a todos los çibdadanos \textbf{ de non vsar de casamiento en hedat de grand moçedat . } Et esto tantomas conuiene alos Reyes e alos prinçipes & Decet ergo omnes ciues \textbf{ non uti coniugio in aetate nimis iuuenili ; } hoc tamen tanto magis decet Reges et Principes , \\\hline
2.1.16 & por que los fijos dellos sean fermosos \textbf{ e de grandes cuerpos } e muy nobles e acabados en el alma e enel entendimiento . & quanto ipsi plus debent esse soliciti , \textbf{ ut eorum filii sint formosi et elegantes corpore , } et industres mente . \\\hline
2.1.16 & para prouar esto mesmo se toma dela destenprança delas mugers . \textbf{ Ca si en la hedat de grand moçedat las mugers se ayuntaren a sus maridos } non solamente los fijos resçiben ende danno & sumitur ex intemperantia mulierum . \textbf{ Nam si in aetate valde iuuenili uxores suis viris copulentur , } non solum filii inde laeduntur , \\\hline
2.1.16 & mas avn ellas mismas se fazen destenpradas e orguollosas \textbf{ por que quando alguna perssona mucho dessea alguna cosa con grant ardor } cobdiçia aquella misma costrdeli en grand moçedat fuer muy acostunbrada & et lasciuae : \textbf{ quia quaelibet persona nimis desiderat aliquid , } et nimio ardore concupiscit ipsum , \\\hline
2.1.16 & por que quando alguna perssona mucho dessea alguna cosa con grant ardor \textbf{ cobdiçia aquella misma costrdeli en grand moçedat fuer muy acostunbrada } a aquella cosa sera muy destenprada en ella . & quia quaelibet persona nimis desiderat aliquid , \textbf{ et nimio ardore concupiscit ipsum , } si ex nimia iuuentute sit assueta ad illud . \\\hline
2.1.16 & Ca los uarones resçiben danno \textbf{ si en grand moçedat vsaren de casamiento . } Onde en las politicas dize el philosofo & quia ipsi viri laeduntur , \textbf{ si in nimia iuuentute utantur coniugio . } Unde in Politicis , \\\hline
2.1.16 & si estos males que pueden acahesçer \textbf{ por la grand mançebia } e por la grand moçedat del casamiento & Quare \textbf{ si haec mala , quae ex nimia iuuentute coniugum accidere possunt } tam coniugibus quam eorum filiis , \\\hline
2.1.16 & por la grand mançebia \textbf{ e por la grand moçedat del casamiento } tan bien alas mugers & si haec mala , quae ex nimia iuuentute coniugum accidere possunt \textbf{ tam coniugibus quam eorum filiis , } magis vitanda sunt in Princibus \\\hline
2.1.16 & que en los otros \textbf{ Et por ende mayormente conuiene aellos de non vsar de casamiento en grand moçedat . } Mas si fuere demandado quanto tienpo ha menester & eos \textbf{ uti coniugio in nimia iuuentute . } Sed si quaeratur \\\hline
2.1.16 & ante que case . \textbf{ Mas en el uaron ha menester mayor tienpo . } Ca si por todo elt podel cresçer es muy enpesçible a los alos & huiusmodi tempus in coniuge debere esse decem et octo annorum . \textbf{ In viro vero plus temporis requiritur . } Nam si per totum tempus augmenti nociuum est masculis uti coniugio , \\\hline
2.1.17 & que los omes non deue dar obra al casamiento \textbf{ en la he perdat de grand mançebia demanda } en quet pon deuen dar mas obra ala generaçion delos fijos . & non esse dandam \textbf{ operam coniugio in aetate nimis iuuenili : } inquirit quo tempore magis insistendum est procreationi filiorum , \\\hline
2.1.17 & que en el tp̃o frio \textbf{ en que vienta el cierço meior muelle la uianda } por la calentura natural se ençierran de dentro por el frio qual çerca de fuera & Unde quilibet in seipso experitur , \textbf{ quod tempore frigido flante borea melius digerit , } quia calor eius interius propter frigus circunstans non exalat , \\\hline
2.1.17 & mesmo se toma del danno delos uarones \textbf{ ca los uarones mayor danno resçibe } si vsan del ayuntamiento delas mugieres & Secunda via ad inuestigandum hoc idem , \textbf{ sumitur } ex laesione filiorum . \\\hline
2.1.17 & el abrego aduze muchedunbre de luuias \textbf{ e el estando puro meiora se la conplision de aquellos } que estan en el e fazen se meiores las generaçiones . & Nam secundum Philosophum in Meteoris , Auster pluuiarum multitudinis adductiuus . \textbf{ Aere autem existente puro melioratur complexio existentium } in eo , \\\hline
2.1.18 & quanto mas les conuiene aellos de auer los fijos grandes e esforcados de cuerpo \textbf{ euedes saber que las costunbres delas mugers en la mayor parte son } assi commo las costunbres delos moços e de los mançebos . & tanto tamen hoc magis decet Reges et Principes , quanto decet eos elegantiores habere filios . \textbf{ Mulierum autem mores } ut plurimum \\\hline
2.1.18 & e non conplido . \textbf{ Et en essa misma manera avn los moços non son tan noblesçidos } por razon commo les guandes . & et quasi vir incompletus . \textbf{ Sic etiam et pueri non sic ratione vigent } ut adulti : \\\hline
2.1.18 & que tales son las costunbres delas mugers \textbf{ commo las costunbres delons moços . } Enpero por que en este segundo libro el gouernamiento de los casados demanda tractado espeçial & sed supposuimus coniecturandum esse de huiusmodi moribus ex moribus puerorum . \textbf{ Verumtamen quia in hoc secundo libro de regimine coniugum specialem requirit tractatum , } ut sciamus \\\hline
2.1.18 & Mas lo primero que es de loar en ellas \textbf{ es que en la mayor parte acaesçe ala mugieres de ser uergonçosas } la qual cosa contesçe & et quae vituperabilia in ipsis foeminis . Est autem primo laudabile in eis , \textbf{ quia } ut plurimum contingit foeminas verecundas esse . Quod duplici de causa contingit . Primo , \\\hline
2.1.18 & mas dessean ser llamados sabios \textbf{ e que ayan nonbre de ser grandes cłigos } que los que son uerdaderamente sabios & ut vocentur scientes , \textbf{ et ut habeant nomen quod sint excellentes clerici , } quam vere scientes . \\\hline
2.1.18 & o non han la sçiençia acabadamente \textbf{ por que non veen en si mismos sçiençia } donde se puedan gozar & ideo non tantum curant de apparentia . \textbf{ Sed qui imperfecte cognoscunt , } quia non uident in seipsis scientiam unde gaudere possint ; quod non habent in rei ueritate , uolunt habere in hominum opinione . \\\hline
2.1.18 & en si mesmos fallan \textbf{ donde se puedan gozar Por la qual cosa non han grant cuydado } de se gozar dela opinion de los omes . & in seipsis inueniunt \textbf{ unde gaudere possint : } propter quod non multum curant gaudere de hominum opinione . \\\hline
2.1.18 & assi commo la grandeza e la fermosura \textbf{ e las otras cosas tales son bienes menguadᷤ Et pues que assi es las mugers en la mayor parte o partiçipan los bienes menguados } assi commo son los bienes del cuerpo & et caetera talia , \textbf{ imperfecta bona sunt . | Mulieres ergo ut plurimum uel participant bona imperfecta , } ut aliqua bona corporis : \\\hline
2.1.18 & o temen perder alabança \textbf{ la qual dessean con muy grand apetito ¶ pueᷤ } que assi es desta cobdiçia de alabança se prueua & et de amissione laudis , \textbf{ mulieres communiter sunt uerecundae , | quia timent inglorificari et amittere laudem quam nimia affectione desiderant . } Ex ipsa igitur cupiditate laudis probatur mulieres uerecundas esse . \\\hline
2.1.18 & Enpero que quier que sea destas razones mucho es de alabar en ellas ser uergon cosas \textbf{ ca por la uerguença dexan de fazer muchs cosas torpes } que non dexarien de fazer & laudabile est in ipsis esse uerecundas : \textbf{ quia propter uerecundiam multa turpia dimittunt } quae non dimitterent , \\\hline
2.1.18 & ca los mocos e los vieios e las mugers \textbf{ por la mayor parte son misicordiosos . } Enpero esto non paresçe que les contesca por vna razon & Pueri enim et senes et foeminae \textbf{ ut plurimum sunt misericordes . } Non tamen ex eadem causa hoc uidetur contingere : \\\hline
2.1.18 & por su sinpleza \textbf{ et creen que los otros son sinples ynoçentes } e sin culpa & quia pueri ( ut superius dicebatur ) sunt miseratiui , \textbf{ quia sua innocentia alios mensurantes credunt omnes innocentes esse , } et putant eos indigne pati : \\\hline
2.1.18 & Et quando son crueles son muy crueles Et quando son desuergonçedas son muy sin uerguença . \textbf{ Ca del pues que las mugers toman osadia acometen tan torpes cosas } que apenas auria omes en el mundo tan desuergonçados & et cum sunt inuerecundae , sunt nimis inuerecundae . Postquam enim mulieres audaciam capiunt , \textbf{ tanta turpia perpetrant , } quod vix inuenirentur viri adeo inuerecundi \\\hline
2.1.18 & que tres cosas son de denostar en ellas ¶ \textbf{ Lo primero que por la mayor parte son destep̃das e seguidoras delas passiones ¶ } Lo segundo por que son parleras e varaiadoras ¶ & restat narrare quae sunt vituperabilia in eis . Possumus autem narrare tria in mulieribus vituperabilia . Primo , \textbf{ quia ut plurimum sunt intemperatae , | et passionum insecutrices . Secundo , } quia sunt garrulae \\\hline
2.1.18 & quando pueden \textbf{ por la mayor parte son destenpdas } e son seguidoras delas passiones & cum possunt , \textbf{ ut plurimum sunt intemperatae , } et passionum insecutrices . \\\hline
2.1.18 & la qual cosa les contesçe \textbf{ por essa misma razon . Ca el freno delas mugers } por la mayor parte non es razon & et litigiosae : \textbf{ quod ex eadem causa contingit . | Nam fraenum mulierum } ut plurimum non est ratio , \\\hline
2.1.18 & por essa misma razon . Ca el freno delas mugers \textbf{ por la mayor parte non es razon } ca çentesçe les por la mayor parte & Nam fraenum mulierum \textbf{ ut plurimum non est ratio , } quia communiter a ratione deficiunt : \\\hline
2.1.18 & por la mayor parte non es razon \textbf{ ca çentesçe les por la mayor parte } que fallesçen de razon & ut plurimum non est ratio , \textbf{ quia communiter a ratione deficiunt : } sed magis est passio ut verecundia : \\\hline
2.1.18 & assi conmo dicho es desuso \textbf{ en la mayor parte sigunan las conplisiones del cuerto } assy commo las mugers & ( ut superius dicebatur ) \textbf{ ut plurimum sequatur complexiones corporis : } sicut mulieres habent corpus molle et instabile , \\\hline
2.1.19 & e otro el paternal \textbf{ Et otro el sul o de señora sieruo } mas por que non abasta de tractar & aliud enim est regimen coniugales a paternali , \textbf{ et etiam a seruili . } Sed \\\hline
2.1.19 & que los ensseñasse \textbf{ que tomassen espeçial batalla } e espeçial esfuerço cerca aquellas palauras & instruere , \textbf{ ut specialem pugnam et specialem conatum acciperent circa ea verba quae deterius proferre possent . } Unde et aliquos Philosophos legimus sic fecisse , \\\hline
2.1.19 & que tomassen espeçial batalla \textbf{ e espeçial esfuerço cerca aquellas palauras } que peor pueden pronunçiar . & instruere , \textbf{ ut specialem pugnam et specialem conatum acciperent circa ea verba quae deterius proferre possent . } Unde et aliquos Philosophos legimus sic fecisse , \\\hline
2.1.19 & assi los quales commo ouiessen las lenguas enbargadas \textbf{ tomaron especial esfuerço çerca aquellas letras } que peor pronunçiauan & qui cum essent impeditae linguae , \textbf{ accipientes specialem conatum circa illas literas quas deterius proferebant , facti sunt eloquentes . } Hoc ergo modo \\\hline
2.1.19 & e assi fueron fechos bien fablantes . \textbf{ Et pues que assi es en essa misma manera deuemos fazer çerca las obras } por que quando alguno vee & accipientes specialem conatum circa illas literas quas deterius proferebant , facti sunt eloquentes . \textbf{ Hoc ergo modo | et circa opera se habet . } Cum enim quis se vel alium videt circa aliqua deficere , \\\hline
2.1.19 & assi o a otro gouernar \textbf{ derechamente deue tomar espeçial esfuerco çerca aquellas cosas } en que puede caer & vel alium vult recte regere , \textbf{ specialem conatum assumere debet circa ea in quibus esse contingit facilior casus . } Quare cum mulieres \\\hline
2.1.19 & en que puede caer \textbf{ e fallesçer mas ligera mente . } por la qual cosa & specialem conatum assumere debet circa ea in quibus esse contingit facilior casus . \textbf{ Quare cum mulieres } ( ut in praecedenti capitulo dicebatur ) communiter sint intemperatae , \\\hline
2.1.19 & quanto de los fijos non legitimos dellas \textbf{ podria nasçer mayor contienda e mayor discordia } que lo segundo couiene a el de los otros . & quanto ex earum illegitima prole potest maior lis \textbf{ et discordia , } vel dissensio oriri . Secundo decet eas esse pudicas \\\hline
2.1.19 & en el capitulo delas constitucon nes antigas entre las mugers \textbf{ romana sera grand denuesto beuer el vino . } Ende dize & de Institutis antiquis ) \textbf{ quodammodo nefas erat bibere vinum . } Unde ait , \\\hline
2.1.19 & Mas los que abondan en nobleza e en rianza e en poderio çiuil \textbf{ conuiene les de bulcar buenas mugers } e antiguas de buen testimoino prouadas & et ciuili potentia , \textbf{ decet inquirere matronas aliquas boni testimonii per diuturna tempora prudentia } et bonis moribus approbatas , \\\hline
2.1.19 & conuiene les de bulcar buenas mugers \textbf{ e antiguas de buen testimoino prouadas } por luengos tp̃os en sabiduria e en bueans costunbres & et ciuili potentia , \textbf{ decet inquirere matronas aliquas boni testimonii per diuturna tempora prudentia } et bonis moribus approbatas , \\\hline
2.1.19 & e antiguas de buen testimoino prouadas \textbf{ por luengos tp̃os en sabiduria e en bueans costunbres } que muestren la mugni e la iudgan & decet inquirere matronas aliquas boni testimonii per diuturna tempora prudentia \textbf{ et bonis moribus approbatas , } instruentes coniugem , \\\hline
2.1.19 & en el pramer libro delas politicas \textbf{ grand conponimiento es delas mugers el silençio . } Ca sy las mugersse han conueniblemente en su fabla e guardan silençio commo conuiene & sit , ut debite sit taciturna . Nam , \textbf{ ut scribitur 1 Polit’ ornamentum mulieris est taciturnitas . } Si enim mulieres debite se habeant , \\\hline
2.1.19 & por esto paresçen mas honrradas e mas apuestas \textbf{ e then a sus maridos a mayor amor . } Et por ende les conuiene de ser calladas & ex hoc magis ornatae apparent , \textbf{ et ad maiorem amorem viros inducunt : } decet ergo eas esse taciturnas . Sic etiam decet esse stabiles : \\\hline
2.1.19 & Et por ende les conuiene de ser calladas \textbf{ e en essa misma manera avn les conuiene de ser estables e firmes } que quanto la mug̃res mas firme e mas estable & et ad maiorem amorem viros inducunt : \textbf{ decet ergo eas esse taciturnas . Sic etiam decet esse stabiles : } quia quanto uxor est magis constans , \\\hline
2.1.19 & que quanto la mug̃res mas firme e mas estable \textbf{ tanto mayor firmeza faze en su marido } para quel guarde fialdat . & et firma , \textbf{ tanto maior credulitas adgeneratur viro , } ut ei debitam fidem seruet . \\\hline
2.1.19 & por que resplandescan \textbf{ por las seys bondades sobredichͣs . } Conuiene a saber & Tali ergo regimine regendae sunt coniuges , \textbf{ ut polleant praedictis sex bonitatibus , } videlicet , \\\hline
2.1.19 & por si mismos \textbf{ o por madronas de buen testimo } non o fallando otras cautellas & vel per seipsos , \textbf{ vel per matronas boni testimonii , } vel per cautelas alias adhibendo . \\\hline
2.1.19 & quanto por el gouernamiento desconuenible dellos \textbf{ puede acaesçer mayor periglo } cerca el gouernamiento del regno & et Principes , \textbf{ quanto ex eorum indebito regimine potest circa regnum maius periculum imminere . } Non sufficit scire , \\\hline
2.1.20 & quanto parte nesçe alo present \textbf{ en que deuemos cuydar con grand acuçia } en las quales conuiene alos uarones de se auer & ( quantum ad praesens spectat ) \textbf{ diligenter consideranda , } in quibus viros circa proprias coniuges decet debite se habere . \\\hline
2.1.20 & dio la qual cosa non puede ser \textbf{ sin grand flaqueza de su cuerpo . Ende el meollo e la uista e los otros mienbros } nobles se ensiaqueçen & et ultra quam sufficiat ad restaurandum : \textbf{ quod sine debilitatione proprii corporis esse non poterit . Unde cerebrum , } et visus , et alia membra nobilia debilitantur \\\hline
2.1.20 & el martiello enflaqueçen todas las obras del ferrero . \textbf{ En essa misma manera enfiaqueçido el meollo } e todos los otros & debilitato martello debilitantur actiones fabriles : \textbf{ sic debilitato cerebro } et aliis membris nobilibus impeditur anima a rationis usu , \\\hline
2.1.20 & Et si el marido enssencare ala muger \textbf{ por conuenibles castigos . } Mas declarar quales son las señales conueinbles dela mistança & si ei ostendat debita signa amicitiae , \textbf{ et si eas per debitas monitiones instruat . Declarare autem quae sunt signa amicitiae debita , } et quae sunt monitiones congruae , \\\hline
2.1.20 & esto non puede ser \textbf{ si non fuere catado con grand acuçia el departimiento delos estados . } Et si non fueren penssadas las condiconnes delas perssonas . & fieri non potest \textbf{ nisi inspecta diuersitate statuum , } et consideratis conditionibus personarum : \\\hline
2.1.20 & Et si non fueren penssadas las condiconnes delas perssonas . \textbf{ ca los maridos deuen catar con grand acuçia } si las muger sson soƀuias o si son homildosas & et consideratis conditionibus personarum : \textbf{ debent enim viri diligenter aduertere , } utrum uxores sint superbae , \\\hline
2.1.20 & e se enloçanesçen \textbf{ si les fuere mostrada grand amistança } que avn quieren enssenorear a sus maridos . & superbae enim adeo fiunt elatae , \textbf{ si eis multa amicitia ostendatur , } ut velint \\\hline
2.1.20 & assi commo conuiene \textbf{ por conuenibłs castigos . } onuiene alos Reyes e alos prinçipes & et eas \textbf{ ( ut expedit ) per debitas monitiones instruere . } Decet Reges , et Principes , \\\hline
2.1.21 & e a fechos uirtuosos . \textbf{ Conuiene avn que todos los uarones tengan mientes con grand acuçia } en todas aquellas cosas & et ad opera virtuosa : \textbf{ expedit quoslibet viros in iis , } in quibus ut plurimum delinquunt foeminae , \\\hline
2.1.21 & que pueden fallesçer \textbf{ e errar las sus mugers en la mayor parte } e que cosas les son conuenibles & expedit quoslibet viros in iis , \textbf{ in quibus ut plurimum delinquunt foeminae , } diligenter aduertere quae sunt ibi licita , \\\hline
2.1.21 & Et pues que assi es commo las mugieres \textbf{ por la mayor parte desseen desseruistas fermosas } mayormente pecan en el conponimiento de los cuerpos . & Cum ergo mulieres \textbf{ ut plurimum appetant videri pulchrae , } potissime delinquunt circa ornatum corporis ; \\\hline
2.1.21 & deuen retrenar conueiblemente \textbf{ e castigaras Ꝯ mugeres } quanto al conpongmiento e honrramiento de sus cuerpos . & secundum medium sit infelix , \textbf{ quantum ad ornatum } et etiam quantum ad omnia alia , \\\hline
2.1.21 & Lo segundo por desfallesçimi ento . \textbf{ Mas alli commo parelçe en el sobrepiuamiento conuiene de ser tres uirtudes } las quales tanne andronico peri patetico en el libro & Primo ex superabundantia . Secundo ex defectu . Superabundantia vero \textbf{ ( ut videtur ) oportet ibi triplicem virtutem concurrere , } quam tangit \\\hline
2.1.21 & quando non se conponen \textbf{ nin se afeytan por vana eglesia } mas esto fazen por fazer plazer a sus maridos e por los tirar de forncacion e de luxuria . & et simplicitas . Tunc enim mulieres circa ornatum corporis sunt humiles , \textbf{ quando non propter vanam gloriam se ornant , } sed agunt \\\hline
2.1.21 & Et pues que assi es dado \textbf{ que la muger eł uaron fuesse humillosa } e non se posie con sse & Adhuc etiam uxorem Principis , \textbf{ vel etiam Regis decet magis ornatam esse . } Dato igitur quod uxor alicuius viri esset humilis non ornans se propter vanam gloriam , posset delinquere in ornatum , \\\hline
2.1.21 & e non se posie con sse \textbf{ por uana eglesia podria pecar en el conponimiento del cuerpo } si non fuese tenprada . & vel etiam Regis decet magis ornatam esse . \textbf{ Dato igitur quod uxor alicuius viri esset humilis non ornans se propter vanam gloriam , posset delinquere in ornatum , } si non esset moderata , \\\hline
2.1.21 & enel conponimiento de su cuerpo \textbf{ por que non demanden con grand acuçia mayores conponimientos del su cuerpo } de quanto les conuiene . & Tertio decet foeminas circa ornatum corporis esse simplices , \textbf{ ut non nimia solicitudine ornamenta requirant . } Nam et si foemina non propter vanam gloriam se ornaret , \\\hline
2.1.21 & Ca si la muger se conpone \textbf{ non por vana eglesia } nin dessea conponimiento & ut non nimia solicitudine ornamenta requirant . \textbf{ Nam et si foemina non propter vanam gloriam se ornaret , } nec ultra suum statum ornamenta appeteret : \\\hline
2.1.21 & ca assi conmo en los coyos e en los enfermos \textbf{ que estan ante las puertas de las eglesias enla mayor parte } contesçe & Nam sicut in ipsis claudis , \textbf{ et in infirmis existentibus apud ianuas ecclesiarum , } ut in plurimum accidit quod infirmior magis gloriatur , \\\hline
2.1.21 & para que resçibra mas helemosinas \textbf{ que los otros . En essa misma man era alguas uezes } aquel que es mas vil en su uestidura faze & et sperat se plures eleemosynas accepturum : \textbf{ sic aliquando qui vilior est in habitu , } magis superbus efficitur , \\\hline
2.1.21 & ssi por uileza de sus uestiduras \textbf{ o por fallesçimiento dellas demandan vana eglesia . } Onde el philosofo en el quarto libro delas ethicas denuesta a aquellos pueblos & si ex vilitate habitus , \textbf{ et ex defectu vestium gloriam quaerant : } unde Philos’ 4 Ethicor’ vituperat Laconios , \\\hline
2.1.21 & quanto al conponimiento del cuerpo \textbf{ assi que en aquellas seys cosas } que dixiemos sean las mugers enssennadas ᷤ & Hoc ergo modo regendae sunt coniuges quantum ad ornatum corporis , \textbf{ ut circa illa sex quae tetigimus , } diligenter instruantur \\\hline
2.1.21 & que dixiemos sean las mugers enssennadas ᷤ \textbf{ e amonestadas con grand acuçia . } Et pues que assi es lo primero deuen ser amonestadas & ut circa illa sex quae tetigimus , \textbf{ diligenter instruantur } et moueantur . \\\hline
2.1.21 & por que se non conpongan mucho \textbf{ por vana eglesia . } ¶ Lo terçero que sean mesuradas & ut sint humiles , \textbf{ ne propter vanam gloriam nimium se ornent . Tertio , } ut sint moderatae , \\\hline
2.1.21 & Lo quarto que sean sinples \textbf{ e que se non trabaien con grand cuydado çerca los conponimientos de sus cuerpos . } ¶ Lo quinto que non sean negligentes & ut sint simplices , \textbf{ ne circa ornamenta nimia solicitudine vexentur . Quinto , } ne sint negligentes , \\\hline
2.1.22 & por que son muy çelosos de sus mugrs . \textbf{ Mas que el grand çelo en los omnes non sea de alabar } esto podemos prouar & quia circa uxores proprias sunt nimis zelotypi . \textbf{ Sed quod nimis zelus | non sit laudabilis , } triplici via ostendere possumus . \\\hline
2.1.22 & Ca quando alguno es muy çeloso de su muger \textbf{ por el grand çelo } que ha della sospecha todas las cosas & Nam cum quis erga suam coniugem est nimis zelotypus , \textbf{ ex nimio zelo } quem erga illam gerit , \\\hline
2.1.22 & ala peor ꝑ te . \textbf{ Et muchͣs uezes las mugers } que biuen bien e fazen & omnia suspicatur in peius . \textbf{ Multotiens quidem uxores bene viuentes , } et debite se gerentes increpantur a viris , \\\hline
2.1.22 & que los maridos son muy atormentados en si mismos \textbf{ por grand trabaçion ¶ } La segunda se toma desto & non esse laudandos . Primum est , \textbf{ quia viri in seipsis nimia turbatione vexantur . } Secundum , \\\hline
2.1.22 & La terçera por que deste çelo \textbf{ por la mayor partida se leuata en la casa muy grand uaraia e grand trabaçion } ¶La primera razen paresçe & quia ex hoc ipsae uxores incitantur ad malum . Tertium vero , \textbf{ quia ut plurimum ex tali zelo consurgit in domo litigium et perturbatio . } Prima via sic patet . \\\hline
2.1.22 & ¶La primera razen paresçe \textbf{ assi que si los maridos con grand çelo se mueuen contra sus muger } ssienpre son en sospecha mala & Nam \textbf{ si viri in nimio zelo mouentur circa uxores proprias , } quasi semper sunt in suspitione , \\\hline
2.1.22 & e dende se sigue \textbf{ que sienpreson en grand angostura de su çoraçon . } Por la qual cosa commo el vn cuydado enbargue el otro . & quasi semper sunt in suspitione , \textbf{ et per consequens semper sunt in anxietate cordis : } quare cum una cura impediat aliam , \\\hline
2.1.22 & Et tanto mas esto conuiene alos Reyes e alos prinçipes \textbf{ quanto mayor mal e mayor preiizio se puede leunatar al regno } si los Reyes fueren en grand angostura de su coraçon & et tanto magis hoc decet Reges et Principes , \textbf{ quanto maius praeiudicium potest insurgere regno , } si Reges sint in anxietate cordis , \\\hline
2.1.22 & quanto mayor mal e mayor preiizio se puede leunatar al regno \textbf{ si los Reyes fueren en grand angostura de su coraçon } e si fueren enbargados en la cura conueinble del regno ¶ & quanto maius praeiudicium potest insurgere regno , \textbf{ si Reges sint in anxietate cordis , } et retrahantur a debita cura regni . \\\hline
2.1.22 & quando sus maridos son muy çelosos dellas . \textbf{ Ca comunal cosa es sienpre } que la cosa uedada acresçienta la cobdiçia e el appetito . & ex eo quod uxores incitantur ad malum , \textbf{ si contingat suos viros esse nimis zelotypos . Commune est enim quod semper prohibitio auget concupiscentiam . Nam , } ut dicitur 2 Rhetoricorum concupiscentia est eius quod abest . Ideo probatur ibi , \\\hline
2.1.22 & Et por ende prueua alli \textbf{ que los vieios mas dessean beuir en el postrimer dia } que en todos los otros dias passados . & ut dicitur 2 Rhetoricorum concupiscentia est eius quod abest . Ideo probatur ibi , \textbf{ quod senes magis desiderant uiuere ultima die , quam praecedentibus diebus , } eo quod tunc magis deficit eis uita . \\\hline
2.1.22 & Por ende non lo pueden sofrir las mugersen paciençia . \textbf{ por la qual cosa leunatase en aquella casa muchͣs uezes varaias e contiendas . } ¶ Et pues que assi es non conuiene alos maridos ser muy çelosos de sus mugrs & non possunt patienter sufferre : \textbf{ propter quod in domo illa | ut plurimum oriuntur lites } et iurgia . \\\hline
2.1.23 & en el primero delas politicas es flaco . \textbf{ Ca assi comm̃el moço ha consseio menguado } por que fallesçedel conplimiento de uaron & ut dicitur 1 Politicorum est inualidum : \textbf{ nam sicut puer habet consilium imperfectum , } quia deficit a perfectione viri : \\\hline
2.1.23 & por que es mas flaco e menos poderoso \textbf{ que el conseio delons uarones̃ } mas ayna viene a su conplimiento . & Consilium ergo mulieres , \textbf{ quia est debilius et inualidus quam consilium virile , } citius venit ad suum complementum . Ceteris ergo paribus \\\hline
2.1.23 & e mas pequano tienpo poner en las cosas uiles \textbf{ delas quales deue auer menor cuydado . } Et por ende dize el prouerbio & et modicum tempus apponere in rebus vilioribus \textbf{ de quibus est minus curandum : } unde et prouerbialiter dicitur , \\\hline
2.1.23 & e ayna la aduze a su conplimiento . \textbf{ ¶ En essa misma manera la muger } quanto al cuerto mas ayna ha la sustançia acabada & ø \\\hline
2.1.23 & por que es mas uil \textbf{ e la natura ha menor cuydado del } mas ayna viene a su conplimiento & eo quod sit vilius , \textbf{ et natura minus de ipso curet , } citius venit ad suum complementum , \\\hline
2.1.23 & que el cuerpo del uaron . \textbf{ E a essa misma manera el consseio dela muger } mas ayna es en su perfecçion & quam virile . \textbf{ Sic consilium muliebre citius est in perfectione } et complemento , \\\hline
2.1.24 & or tres razones podemos prouarque las mugeres comunalmente \textbf{ e por la mayor parte non puede guardar los secretos ¶ } La primera razon se toma del fallesçimiento dela razon . & quod mulieres communiter , \textbf{ et ut plurimum secreta retinere non possunt . Prima via sumitur ex defectu rationis . } Secunda ex mollicie cordis . \\\hline
2.1.24 & que los uarones \textbf{ quanto mas fallesçe enlłas razon } que en los omes ¶ la segunda razon & ø \\\hline
2.1.24 & e del mouimientod el coraçon . \textbf{ Por que las mugers en la mayor parte son muelles } e mas de ligero son mouibles & sumitur ex molicie cordis . \textbf{ Nam quia mulieres } ut plurimum sunt molles et ductibiles , \\\hline
2.1.24 & que comunalmente \textbf{ e en la mayor parte las mugieres descubien las poridades de sus maridos } alas otras mugers & Inde est igitur quod communiter \textbf{ et ut plurimum mulieres aliis mulieribus , } quas credunt amicas esse , \\\hline
2.1.24 & que son amadas de sus maridos . \textbf{ por ende desseando alguna uana gloria } e alguna alabança de ligero descubren los secretos de sus maridos & si possint se laudari quod a suis maritis diligantur , \textbf{ appetentes quandam inanem gloriam , } et quandam laudem , \\\hline
2.2.1 & assi commo de aquello \textbf{ de que deuemos auer mayor cuydado . } Et en determinando del gouernamiento de los fijos primeramente queremos mostrar & ideo decreuimus prius determinare de regimine filiali quam de regimine seruili , \textbf{ tanquam de eo circa quod esse debet amplior cura . In determinando quidem de regimine filiorum , } primo ostendere uolumus , \\\hline
2.2.1 & por tres razones \textbf{ que conuienen a todos los padres de auer grand cuydado de sus fijos ¶ } La primera se toma & ut suos filios bene regant . Possumus autem triplici via venari , \textbf{ quod decet huiusmodi solicitudinem habere parentes . Prima via sumitur } ex eo quod patres sunt causa filiorum , \\\hline
2.2.1 & que en el logar de yuso . \textbf{ Et en essa misma manera avn si la naturada alas ainalias ser . } Et por que la aina lia non puede beuir sin comer & eo quod in loco superiori magis habet in esse conseruari quam in inferiori : \textbf{ sic etiam si natura dat animalibus esse , } quia animal sine cibo conseruari non potest , \\\hline
2.2.1 & Et en essa misma manera avn si la naturada alas ainalias ser . \textbf{ Et por que la aina lia non puede beuir sin comer } luego es cuydados a de dar & sic etiam si natura dat animalibus esse , \textbf{ quia animal sine cibo conseruari non potest , } natura est solicita dare animalibus ora \\\hline
2.2.1 & e mantienen los en su ser . \textbf{ Por la qual cosasi natural cosa es } que los cuerpos de suso ensseñoreen & et conseruare . \textbf{ Quare si naturale est , } ut superiora et praeeminentia in inferiora influant , \\\hline
2.2.1 & que es sennor sobre todas las cosas \textbf{ ha grand cuydado } e grand prouidençia de todo el mundo . ¶ Et pues que assi es & et quomodo eos regulet et conseruet . Unde et ipse Deus , \textbf{ qui singulis rebus praeest , habet solicitudinem , et prouidentiam totius Uniuersi . } Patres ergo \\\hline
2.2.1 & ha grand cuydado \textbf{ e grand prouidençia de todo el mundo . ¶ Et pues que assi es } por que los padres enssenore a naturalmente alos fijos & et quomodo eos regulet et conseruet . Unde et ipse Deus , \textbf{ qui singulis rebus praeest , habet solicitudinem , et prouidentiam totius Uniuersi . } Patres ergo \\\hline
2.2.1 & por que de razon de amores \textbf{ que aquel que ama aya grand cuydado dela cosa que ama . } Ca cada vno ha cuydado de su amor . & De ratione enim amoris , \textbf{ est ut solicitet amantem circa rem amatam , } quilibet enim solicitatur circa dilectum : \\\hline
2.2.2 & mas ha cuydado de sus fijos \textbf{ Ca natural cosa es que cada vno ame sus obras } assi commo dize el philosofo en las ethicas & magis habet solicitudinem circa filios : \textbf{ naturale est enim quemlibet diligere sua opera , } ut Philosophus in Ethicorum unde et patres naturaliter diligunt filios , et poetae sua poemata tanquam proprium opus . \\\hline
2.2.2 & e mas conosçe su obra \textbf{ tanto con mayor cuydado } e con mayor amor se deue mouer a ella . & et magis cognoscit proprium opus , \textbf{ tanto maiori solicitudine et dilectione mouetur circa illud . } Patres ergo \\\hline
2.2.2 & tanto con mayor cuydado \textbf{ e con mayor amor se deue mouer a ella . } Et por ende los padres & et magis cognoscit proprium opus , \textbf{ tanto maiori solicitudine et dilectione mouetur circa illud . } Patres ergo \\\hline
2.2.2 & Et por ende los padres \textbf{ tanto mayor cuydado deuen auer de los fijos } quanto mas sabios son & Patres ergo \textbf{ tanto magis debent solicitari circa filios , } quanto predentiores sunt , \\\hline
2.2.2 & quanto mas sabios son \textbf{ e quanto mayor entendimiento ha en ellos . } Mas conmo sea dicho & quanto predentiores sunt , \textbf{ et quanto maiori intelligentia vigent . } Sed cum habitum sit quod Reges , \\\hline
2.2.2 & Et tanto mas conuiene alos Reyes \textbf{ e alos prinçipes de auer mayor cuydado de sus fijos } que alos otros & oportet quod polleant prudentia et intellectu : \textbf{ tanto decet Reges et Principes magis solicitari circa proprios filios quam ceteri , } quanto in eis magis vigere \\\hline
2.2.2 & e de los prinçipes \textbf{ de auer mayor bondat e mayor nobleza que los otros . } Ca segunt el philosofo enlas politicas . & Decet enim filios Regum et Principum maiori bonitate \textbf{ pollere quam alios : } quia secundum Philosophum in Politic’ \\\hline
2.2.2 & Ca segunt el philosofo enlas politicas . \textbf{ segunt que algunos son en mayorestado e en mas alta dignidat } assi deuen ser meiores & quia secundum Philosophum in Politic’ \textbf{ secundum quod aliqui sunt in maiori statu et in altiori dignitate , } sic debent meliores esse , \\\hline
2.2.2 & e ser mas acabados en sçiençia e en uirtudes . \textbf{ Ca conuenible cosa es } que aquel que quiere gouernar los otros sea tan sabio & sic debent meliores esse , \textbf{ et esse magis perfecti scientia et virtutibus . Congruum enim est } qui alios regere cupit , \\\hline
2.2.2 & mucho les conuiene de ser sabios e buenos . \textbf{ Mas commo los fijos bengan a mayor bondat e a mayor sabiduria } si los padres ouieron cuydado dellos & et bonos . \textbf{ Et cum filii perueniunt ad maiorem bonitatem | et prudentiam , } si patres circa eos sint soliciti , \\\hline
2.2.2 & quanto los sus fijos deuen auer mayor sabidina \textbf{ e mayor bondat ¶ } La terçera razon & quanto filii eorum pollere debent maiori prudentia \textbf{ et ampliori bonitate . } Tertia via ad hoc ostendendum sumitur ex utilitate regni . Bonitas enim regni dependet ex bonitate eorum \\\hline
2.2.2 & que desçende dela bondat de aquellos que son en el regno . \textbf{ Et mayor miente desçende dela bondat de aquellos que son prinçipes en el regno . } Ca bien commo la sanidat na tra᷑al del cuerpo desçende dela sanidat de todos los mienbros e mayormente dela samdat del coraçon e de los mienbros prinçipales & qui sunt in regno , \textbf{ et maxime dependet ex bonitate principantium in ipso . } Nam sicut sanitas corporis naturalis dependet ex sanitate omnium membrorum , \\\hline
2.2.2 & Et mayor miente desçende dela bondat de aquellos que son prinçipes en el regno . \textbf{ Ca bien commo la sanidat na tra᷑al del cuerpo desçende dela sanidat de todos los mienbros e mayormente dela samdat del coraçon e de los mienbros prinçipales } por que el coraçon e los mienbros prinçipales han de dar uirtud alos otros mienbros & et maxime dependet ex bonitate principantium in ipso . \textbf{ Nam sicut sanitas corporis naturalis dependet ex sanitate omnium membrorum , | et maxime ex sanitate cordis } et membrorum principalium , \\\hline
2.2.2 & Ca bien commo la sanidat na tra᷑al del cuerpo desçende dela sanidat de todos los mienbros e mayormente dela samdat del coraçon e de los mienbros prinçipales \textbf{ por que el coraçon e los mienbros prinçipales han de dar uirtud alos otros mienbros } e enderescarlos e garlos . & et membrorum principalium , \textbf{ eo quod cor et principalia membra habent influere in alia et rectificare ipsa : } sic bonitas regni dependet ex bonitate omnium ciuium ; \\\hline
2.2.2 & e son sennors en el regno ¶ \textbf{ pues que assi es prouechosa cosa es a todo el regno } de auer bueon sçibdadanos . & et dominantur in regno . \textbf{ Utile est ergo toti regno habere bonos ciues , } sed utilius est habere bonos principantes , \\\hline
2.2.2 & de auer bueon sçibdadanos . \textbf{ Mas mas prouechosa cosa es de auer bueons prinçipes } por que alos prinçipes parte nesçe de gouernar & Utile est ergo toti regno habere bonos ciues , \textbf{ sed utilius est habere bonos principantes , } eo quod principantis sit alios regere et gubernare : \\\hline
2.2.2 & que ayan sabiduria e bondat \textbf{ quanto mayor prouecho seleunata al regno dela bodat de los fijos delos Reyes } que deuen auer el prinçipado e el senorio en el regno & et bonitate ; \textbf{ quanto maior utilitas consurgit ipsi regno ex bonitate filiorum Regum , } qui debent habere principatum et dominium in regno ; \\\hline
2.2.3 & e por que en pos el tractado del gouernamiento dela muger \textbf{ auemos atrattardel gouernamiento paternal . } Conuienne de uer onde toma comienço el gouernamiento paternal . & Quia post tractatum de regimine coniugis , \textbf{ determinandum est de regimine paternali : } videndum est , \\\hline
2.2.3 & o los gouierna \textbf{ segunt ciertas leyes } e segunt çiertos abenemientos . & vel regit eos \textbf{ secundum certas leges , } et \\\hline
2.2.3 & segunt ciertas leyes \textbf{ e segunt çiertos abenemientos . } Et tal gouer namiento & secundum certas leges , \textbf{ et | secundum certa pacta . } Et tale regimen \\\hline
2.2.3 & Mas estos tres gouernamientos seguñ los quales veemos algunos regnar en las çibdades \textbf{ e en las villas son semeiantes trs gouernamientos } que son fallados en vna casa . & secundum quod videmus aliquos regnare in ciuitatibus et castris , \textbf{ assimilantur tria regimina reperta in una domo . } Nam regnum coniugale assimilatur regimini politico : \\\hline
2.2.3 & de esceger su gouernador . \textbf{ Mas non es en poderio de los fijos de escoger assi mismos padres } mas por natra al nasçençia los fijos desçenden de los padres . & nisi sit in potestate subiecti eligere sibi rectorem : \textbf{ non est autem in potestate filiorum eligere sibi patrem , } si ex naturali origine filii procederent a parentibus . \\\hline
2.2.3 & mas al Real . \textbf{ Onde el philosofo en el primo libro delas politicas dize } que el uaron deue ensennorear ala mug̃ & huiusmodi regimen non assimilatur regimini politico , \textbf{ sed regali . Unde et Philosophus 1 Politicorum ait , } virum praeesse mulieri , \\\hline
2.2.3 & nin de vna guasa . \textbf{ Mas ala muger deue enssennorear çibdadana niete } e alos fues real mente & tamen eodem modo principandi : \textbf{ sed mulieri quidem politice , natis autem regaliter . Tertium autem regimen quod est in domo , } vel regimen \\\hline
2.2.3 & Mas ala muger deue enssennorear çibdadana niete \textbf{ e alos fues real mente } Mas el tercero gouernamiento & tamen eodem modo principandi : \textbf{ sed mulieri quidem politice , natis autem regaliter . Tertium autem regimen quod est in domo , } vel regimen \\\hline
2.2.3 & el qual es en la casa \textbf{ e assi commo es el goñ namiento } por que se gouierna toda la conpanna de la casa . & vel regimen \textbf{ quo regitur familia caetera , } assimilatur regimini dominatiuo . \\\hline
2.2.3 & Enpero por que mas mani fiestamente paresca \textbf{ lo que dezimos todemos prouar pardas rasones } quel gouernamiento qł padre toma comie y de amor¶ & ut manifestius appareat quod dicitur , \textbf{ possumus duplici via venare , } paternale regimen trahere originem ex amore . Prima via sumitur ex ordine naturali . \\\hline
2.2.3 & lo que dezimos todemos prouar pardas rasones \textbf{ quel gouernamiento qł padre toma comie y de amor¶ } La primera razon se torna de la orden natural & possumus duplici via venare , \textbf{ paternale regimen trahere originem ex amore . Prima via sumitur ex ordine naturali . } Secunda \\\hline
2.2.3 & e para cerar sus fijos \textbf{ assi en ellos es natural appetito } para los amar & sic est in eis \textbf{ naturalis impetus ad eos diligendum , } et per consequens ad eos gubernandum et regendum , \\\hline
2.2.3 & e para los gouernar \textbf{ e para auer grant cuydado dellos ¶ } Et pues que assi es esta es natra al orde & et per consequens ad eos gubernandum et regendum , \textbf{ et ad habendum solicitudinem circa ipsos . } Est ergo hic naturalis ordo , \\\hline
2.2.3 & Et pues que assi es esta es natra al orde \textbf{ que el gouernamiento e el cuydado del paradreal fijo toma comienço de amor ¶la segunda razon } para prouar esto mismo se toma dela perfecçion del padre . & quod regnum , \textbf{ et solicitudo paterna sumat originem ex amore . | Secunda via ad inuestigandum hoc idem , } sumitur ex ipsa perfectione patris . \\\hline
2.2.4 & ssi commo es dicho en el capitulo sobredich̃ . \textbf{ El gouernamiento patrinal toma comienço del amor } Et pues que assi es deuemos uer & Dicebatur in praecedenti capitulo , \textbf{ paternale regimen sumere originem ex amore . } Videndum est igitur quantus sit amor patrum ad filios , \\\hline
2.2.4 & Enpero los fijos luego que nasçen non comiençan a amar los padres \textbf{ por que luego non son de tan grand conosçimiento } que pueden conoscer & non tamen filii statim incipiunt amare parentes , \textbf{ quia statim non sunt tantae cognitionis } ut possint discernere \\\hline
2.2.4 & Empero en el comienço dela su nasçençia del moço \textbf{ non es el de tan grand conosçimiento } que pueda conosçer & In principio tamen natiuitatis eius \textbf{ non est illius cognitionis } ut possit cognoscere a qua matre procedit , \\\hline
2.2.4 & tanto este amor es mas guande \textbf{ quanto en los padres es mayor çertidunbre de los fuos . } Et por aquesta razon se puede mostrar & tanto huiusmodi amor est validior , \textbf{ quanto apud parentes est maior certitudo de prole . } Ex hac autem ratione ostendi potest , \\\hline
2.2.4 & Empero los fijos non se mueuen \textbf{ assi con tan grant feruor al amor deles padres . } Por que aquello que es dellos padres non parte nesçe & et quod \textbf{ secundum aliquem modum non pertineat ad illos . Filii tamen non sic vehementer mouentur ad dilectionem parentum : } quia quod est parentum , \\\hline
2.2.4 & Ca cada vno deue ser cuy dados \textbf{ o de aquellas cosas que ama con grant amor . } Mas alos fiios parte nesçe de obedesçer alos padres & quod ad parentes spectat solicitari circa regimen filiorum , \textbf{ quia quilibet solicitus esse debet circa ea quae vehementi amore diligit . } Ad filios vero pertinet obedire parentibus : \\\hline
2.2.4 & assi commo alguna parte \textbf{ e conmo algua cosa } que salle dellos & Nam ut dicitur 8 Ethicorum parentes diligunt filios \textbf{ ut existentes aliquid ipsorum : } filii autem diligunt parentes , \\\hline
2.2.4 & e dela cosamas baxa ala cosa mas alta . \textbf{ Et natraal cosa es } que las cosas mas altas & et a superiori ad inferius : \textbf{ sed amor filiorum ad eos procedit ab effectu ad causam , } et ab inferiori ad superius . Naturale est autem quod superiora influant in inferiora , et conseruent ea : non autem econuerso . Ideo parentes naturaliter afficiuntur ad filios , \\\hline
2.2.4 & e que los mantengan . \textbf{ Et non es natural cosa que las de diyuso } enbien su uirtud alas de suso & ø \\\hline
2.2.4 & por que enbien su uirtud enlleros \textbf{ e que les alleguen muchs bienes } assi commo son possesiones e riquezas e dineros & ut possessiones \textbf{ et numismata , } per quae sufficiant sibi ad vitam , \\\hline
2.2.4 & e non las de suso \textbf{ nin las altas alas de yuso nin alas baxas . } Et por ende los filos mas honrran natraalmente & et rapiunt bona parentum . Inferiora vero reuerentur , \textbf{ et sunt subiecta superioribus , } non econuerso . Ideo filii naturaliter magis honorant , et reuerentur parentes , quam econuerso . \\\hline
2.2.4 & que los fijos alos padres \textbf{ commo amara alguno sea essa misma cosa } que querer bien & cum diligere aliquem , \textbf{ idem sit quod velle ei bonum , } distinguendum est de ipso bono . \\\hline
2.2.4 & nin sofrir los deuuestos que fazen los otros a sus padres . \textbf{ Mas los padres non toman tan grand indignaçion } nin ta grand pesar de los denuestos delos fujes & quam econuerso . Viso , \textbf{ qualis est amor } inter patrem \\\hline
2.2.4 & Mas los padres non toman tan grand indignaçion \textbf{ nin ta grand pesar de los denuestos delos fujes } commo los fijos de los padres . & quam econuerso . Viso , \textbf{ qualis est amor } inter patrem \\\hline
2.2.5 & e aquellas cosas que son de fe non se pueden prouar \textbf{ por razon prouechosa cosa es que en aquella hedat sean enssennadas las cosas } que son de fe & et ea quae sunt fidei ratione comprehendi non possunt : \textbf{ utile est ut in illa aetate proponantur ea quae sunt fidei , } in qua ratio non quaeritur dictorum , \\\hline
2.2.5 & que las dixo \textbf{ e por sinple creençia } non por sotileza de razon . & ideo eis simpliciter est credendum . Acquiescendum est autem iis quae sunt fidei ex auctoritate diuina , \textbf{ et ex simplici credulitate , } non ex perspicacia rationis . \\\hline
2.2.5 & que la sabiduria de dios \textbf{ e la su auctoridat sobrepiua toda sotileza de engennio humanal . por la qual cosa mas prouechosa cosa es de creer } sinplemente la auctoridat de dios & quia nulli dubium esse debet diuinam prudentiam \textbf{ et eius auctoritatem , | omnem perspicaciam humani generis superare . } Quare utilius auctoritati diuinae simpliciter creditur , \\\hline
2.2.5 & si aquellas cosas que son de fe son de creer suplemente . \textbf{ Conueinble cosa es } que aquellas cosas tales & et demonstrationibus hominum . \textbf{ Si ergo ea quae sunt fidei simpliciter sunt credenda , } conuenienter talia in illa aetate proponuntur , \\\hline
2.2.5 & do dize el philosofo \textbf{ que la natraa es sobre la costunbre . Ca la cosa que contesçe muchͣs uezes es cercana a aquello que es sienpre . } Et por ende la costunbre es cercana ala natura . & ubi dicitur , \textbf{ quod natura est consuetudo saepe . | Saepe autem propinquum est ei quod est Semper : } quare consuetudo est propinqua naturae . \\\hline
2.2.5 & tanto mas se torna qen natura \textbf{ e tanto con mayor feruor } e con mayor acuçianos llegamos a ella . & tanto magis utitur in naturam , \textbf{ et tanto feruentius adhaeremus illi . } Cum ergo magis simus assuefacti ad ea , \\\hline
2.2.5 & si las leyes de los gentiles que contienen en si muchͣs fabliellas \textbf{ enlas quales leyes son muchͣs cosas falssas e de escarneçer } e estas son allegadas al coraçon & Si ergo leges Gentilium continentes multas fabulas \textbf{ et apologos idest multa fabulatoria et derisoria , | plus possunt propter consuetudinem , } et sunt sic applicabiles animo , \\\hline
2.2.5 & et tractar lotilmente aquellas colas \textbf{ que son dela fe esto pertenesçe alos cłigos doctors } que han de enformar los otros en la fe . & et subtiliter ea quae sunt fidei pertractare , \textbf{ spectat ad clericos doctores instruentes alios in ipsa fide , } quam subtilem perscrutationem laici , \\\hline
2.2.5 & e menos los moços . \textbf{ Abasta que tałs cosas sotiles } que parte nesçen ala fe sean dichͣs a los legos & spectat ad clericos doctores instruentes alios in ipsa fide , \textbf{ quam subtilem perscrutationem laici , } et maxime pueriscire non possunt : \\\hline
2.2.5 & e que adam primero padre peco \textbf{ e que el humanal linage fue ensuziado } por el pecadodt & qui est pater et filius \textbf{ et spiritus sanctus . Quod Adam primo parente nostro peccante , et humano genere per peccatum eius infecto , } Dei filius , \\\hline
2.2.5 & en la bien auentraada santamͣ \textbf{ e nasçio della . Et que esse mismo fijo de dios padesçio } e fue muerto e soterrado & assumpsit carnem in beata Virgine , \textbf{ et est natus ex ipsa . | Quod ipse Dei filius propter peccata nostra fuit passus , } mortuus , \\\hline
2.2.5 & e por la deuoçion dela fe dellos se puede seguir mayor bien en toda la x̉andat . \textbf{ Et por la mengua dellos podria venir mayor periglo alos xanos } N quanto el alma es mas noble que el cuerpo & quanto ex feruore fidei ipsorum potest maius bonum consequi religio christiana , \textbf{ et ex eorum tepiditate potest ei maius periculum imminere . } Quanto anima est nobilior corpore , \\\hline
2.2.6 & por que puedan fazer asus fijos Ricos \textbf{ e acorrer los quanto ala menguadel cuerpo } mucho mas deuen ser acuçiosos & et circa numismata , \textbf{ ut possint subuenire filiis quantum ad indigentiam corporalem : } multo magis solicitari debent , \\\hline
2.2.6 & por que ellos sean acabados en el alma e en uirtudes \textbf{ e por que sean enformados en bueans costunbres . Et por que esto es vn grant bien non lo deuen dexar peresçer } por negligençia en ningunan manera . & et ut virtutibus et bonis moribus imbuantur . \textbf{ Et quia hoc , | tantum existit bonum , non } debet per negligentiam praeteriri : \\\hline
2.2.6 & por que dexen la locania \textbf{ e siguna bueans costunbres } Et nos podemos esto mostrar & instruendi sunt pueri , \textbf{ ut relinquentes lasciuiam sequantur bonos mores . Possumus autem quadruplici via venari , } quod ab ipsa puerilitate \\\hline
2.2.6 & Et nos podemos esto mostrar \textbf{ por quatro razones } que los moços son de enssennar en su ninnes en bueans costunbres ¶ & ut relinquentes lasciuiam sequantur bonos mores . Possumus autem quadruplici via venari , \textbf{ quod ab ipsa puerilitate } instruendi sunt pueri ad bonos mores . \\\hline
2.2.6 & por quatro razones \textbf{ que los moços son de enssennar en su ninnes en bueans costunbres ¶ } La primera se toma dela naturaleza dela delectacion & quod ab ipsa puerilitate \textbf{ instruendi sunt pueri ad bonos mores . } Prima via sumitur \\\hline
2.2.6 & Ca segunt dize el philosofo en las ethicas \textbf{ en tanto es natural anos de nos delectar enla ninnes } assi que los moços luego se delectan & secundum Philosophum in Ethic’ \textbf{ adeo connaturale est nobis delectari , } quod ab ipsa infantia delectari incipimus : \\\hline
2.2.6 & e contradezir ala cobdiçia \textbf{ enla nr̃a moçedat . } Et paues que assi es dela naturaleza e dela delectaçion paresçe & Si ergo sic ab ipsa infantia nobiscum crescit concupiscentia delectabilium , \textbf{ ab ipsa infantia est tali concupiscentiae resistendum : } ex ipsa ergo connaturalitate delectationis , statim cum pueri sunt sermonum capaces , \\\hline
2.2.6 & para tomar razon \textbf{ estonçe son de enssennar en bueans costunbres } e deuen les ser fechos amonestamientos conuenibles . & ø \\\hline
2.2.6 & para prouar esto mismo se toma del fallesçimiento dela razon \textbf{ ca estonce son algunos de amonestar a buenas costunbres } quando mas son mouidos ala loçania & sumitur ex rationis defectu . \textbf{ Nam tunc aliqui sunt magis mouendi ad bonos mores , } quando magis incitantur \\\hline
2.2.6 & Et pues que assi es en la hedat dela moçedat \textbf{ son los moços de enformar en bueans costunbres } por que entonçe fallesçen & ut sequatur passiones . \textbf{ In iuuenili ergo aetate sunt pueri instruendi ad bonos mores : } quia tunc magis ab usu rationis deficiunt , \\\hline
2.2.6 & por que non sea inclinado a aquella cosa . \textbf{ Ca los omes en la mayor parte } segunt las costunbres son & ne inclinetur ad illud : \textbf{ ut plurimum enim homines } secundum mores sunt \\\hline
2.2.6 & Onde el philosofo çerca la fin del segundo libro delas ethicas \textbf{ en esta manera nos muestra ser endereçados a bueans costunbres } en la qual manera se enderesça la piertega tuerta . & unde \textbf{ et Philosophus circa finem 2 Ethic’ hoc modo docet nos dirigere ad bonos mores , } quo dirigitur virga tortuosa . \\\hline
2.2.6 & en la qual manera se enderesça la piertega tuerta . \textbf{ Ca aquel que quiere endereçar la pierte ga tuerta } inclina la mucho ala parte contraria & quo dirigitur virga tortuosa . \textbf{ Volens enim virgam tortuosam rectificare , } inclinat eam ad partem contrariam valde , \\\hline
2.2.6 & la qual \textbf{ assi inclinada torna al medio e aser derecha . En essa misma manera } por qua nos auemos tortura & inclinat eam ad partem contrariam valde , \textbf{ quae sic inclinata redit | ad medium et ad rectitudinem . } Sic nos , \\\hline
2.2.6 & por que la podamos fazer uenir al medio . \textbf{ En essa misma manera nos en fuyendo delas cosas delectabłs deuemos tris passar } allende del medio onde deuemos foyr & ut possit ad medium redire : \textbf{ sic } et nos in fugiendo delectabilia , debemus ultra medium nos facere , \\\hline
2.2.6 & allende del medio onde deuemos foyr \textbf{ e esquiuar muchͣs delectaçonnes } que son conueibles & sic \textbf{ et nos in fugiendo delectabilia , debemus ultra medium nos facere , } idest debemus multas delectationes \\\hline
2.2.6 & si nos auemos tanta inclinaçion a mal conuiene nos de acostunbrar nos \textbf{ por luengost pons al contrario e al bien } por que podamos escusar & Si ergo tantam pronitatem habemus ad malum , \textbf{ et oportet nos sic per diuturna tempora assuescere ad contrarium , | ut ad bonum ; } ut facilius hanc pronitatem vitare possimus , \\\hline
2.2.6 & por ende deuemos comneçar luego enla moçedat \textbf{ por que dexando las locanias siguamos bueans costunbres . } Et en esto non deuemos poner alongamiento . & est ab ipsa infantia inchoandum , \textbf{ ut relinquentes lasciuias sequamur bonos mores , } nec est ulterius differendum . \\\hline
2.2.6 & e de castigar \textbf{ assi que por bueons amonestamientos } e por bueons castigos sean tirados delas loçanias . & statim ab ipsa infantia sunt monendi et corrigendi ; \textbf{ ut per monitiones et correctiones debitas a lasciuiis retrahantur . } Decet ergo omnes ciues solicitari erga filios , \\\hline
2.2.6 & e por bueons castigos sean tirados delas loçanias . \textbf{ Et pues que assi es conuieneque todos los çibdadanos ayan grand cuydado de sus fijos } assi que luego en su moçedat sean acuçiosos & ut per monitiones et correctiones debitas a lasciuiis retrahantur . \textbf{ Decet ergo omnes ciues solicitari erga filios , } ut ab ipsa infantia \\\hline
2.2.6 & assi que luego en su moçedat sean acuçiosos \textbf{ en en poter los en bueans costunbres . } Enpero esto & instruentur \textbf{ ad bonos mores . } Tanto tamen hoc magis decet Reges et Principes , \\\hline
2.2.6 & quanto la bondat de sus fijos es mas prouechosa al regno . \textbf{ Et quanto dela maliçia dellos vernie mayor periglo a todo el regno } omo quier que conuenga a todos los omes de saber letras & quanto bonitas filiorum est utilior ipsi regno , \textbf{ et quanto ex eorum malitia potest in regno maius periculum imminere . } Licet deceret omnes homines cognoscere literas ; \\\hline
2.2.7 & do los legunaies son departidos del lenguaie de su padre e de su made avn \textbf{ que esten luengo tienpo en aquellas tierrasapenas } o nunca pueden fablar derechamente aquella lengua . & ubi idiomata differunt a materno , \textbf{ etiam si per multa tempora in partibus illis existat , } vix aut nunquam potest recte loqui linguam illam ; \\\hline
2.2.7 & alas quales somos acostunbrados en nr̃que \textbf{ nin uero ennr̃a moçedat . } por que cada vno es mas cuydadoso & ad quae sumus ab ipsa infantia assueti : \textbf{ quia quilibet est magis intentus , } et feruens circa ea quae sibi magis sunt placita ; \\\hline
2.2.7 & e para rescebit castigo . \textbf{ Ca commo quier que vn omne sea de meior engennio que otro . } Enpero generalmente todos los omes son mal ordenados & ad capiendam disciplinam . \textbf{ Licet enim unus homo sit melioris ingenii quam alius , } uniuersaliter \\\hline
2.2.7 & Onde el philosofo en el primero libro del alma dize \textbf{ que el alma mayor tienpo pone enł non saber } que en la sabiduria . & ø \\\hline
2.2.7 & que en la sabiduria . \textbf{ Ca por muy granttp̃o el omne trabaia eñł estudio } ante que pueda venir a perfectiuo e conosçimiento dela sçiençia . & et a posterioribus . Unde et Philosophus in primo de anima vult , \textbf{ quod anima plus temporis apponat in ignorantia , quam in scientia . Per multum enim temporis quis insudat studio , } antequam peruenire possit ad perfectionem scientiae . \\\hline
2.2.7 & e las artes e las sçiençias son guaues \textbf{ e de luengo tienpo } e los omes comunalmente son mal ordenados & ø \\\hline
2.2.7 & deuenlos luego poner en su moçedat alas letros e alas sçiençias liberales . \textbf{ Ca assi conmodicho es de suso ninguno non es dich̃ sennor naturalmente } si non fuere enoblesçido & eos tradere literalibus disciplinis . \textbf{ Nam ( ut superius dicebatur ) nullus est naturaliter dominus , } nisi vigeat prudentia et intellectu . \\\hline
2.2.7 & por que puedan enssennorear mas sabiamente \textbf{ e mas natural mente . } Mas podriemos para prouar esto mismo & quanto decet eos intelligentiores \textbf{ et prudentiores esse , } ut possint naturaliter dominari . Posset autem ad hoc idem alia ratio adduci . \\\hline
2.2.8 & por sabiduria e por entendimiento . ¶ \textbf{ a auctoridat antigua prueua } e muestra & ø \\\hline
2.2.8 & que fuero entre los antiguos \textbf{ e estas son fmatica¶ Logica ¶ Rectox } ca¶musica ¶arismetica ¶ geometria ¶ Et astrologia . & antiqua auctoritas protestatur . \textbf{ Huiusmodi autem sunt , Grammatica ; Dialectica , | Rhetorica , Musica , } Arithmetica , \\\hline
2.2.8 & que por argumentos conuenibles \textbf{ e por razones derechas i anifestamos nr̃a uoluntad e nr̃a entençion . } Et por ende conuiene de fallar algua sçiençia & ut per debita argumenta , \textbf{ et per debitas rationes manifestemus propositum . } Oportuit ergo inuenire aliquam scientiam docentem modum , \\\hline
2.2.8 & e por razones derechas i anifestamos nr̃a uoluntad e nr̃a entençion . \textbf{ Et por ende conuiene de fallar algua sçiençia } que nos mostrasse & et per debitas rationes manifestemus propositum . \textbf{ Oportuit ergo inuenire aliquam scientiam docentem modum , } quo formanda sunt argumenta , \\\hline
2.2.8 & e en razonar ¶ \textbf{ La tercera sçiençia liberales dicha rectorica . } Mas la rectorica & ne erretur in arguendo . \textbf{ Tertia scientia liberalis dicitur esse Rhetorica . Est autem Rhetorica , } ut innuit Philosophus in Rhetoricis suis , \\\hline
2.2.8 & en la rectorica es \textbf{ assi commo vna gruessa logica . } Ca assi commo son de fazer razones sotiles en las sçiençias especulatinas & ut innuit Philosophus in Rhetoricis suis , \textbf{ quasi quaedam grossa dialectica . } Nam sicut fiendae sunt rationes subtiles in scientiis naturalibus \\\hline
2.2.8 & que nos mostrasse manera gruessa e figural \textbf{ para argumentar gruessa mente . } Et esta es neçessaria alos fijos de los nobles e de los libres & ø \\\hline
2.2.8 & e mayormente alos fijos de los Reyes e de los prinçipes . \textbf{ Ca a estos buenos parte nesçe de beuir entre las gentes } e de enssennorear al pueblo el qual pueblo non puede entender & et maxime Regum , et Principum : \textbf{ quia horum est conuersari } inter gentes et dominari populo , \\\hline
2.2.8 & e de enssennorear al pueblo el qual pueblo non puede entender \textbf{ si non gruesas razones e exenplarias ¶ } La quarta sçiençia libal es dichͣ musica . & qui non potest percipere \textbf{ nisi rationes grossas et figurales . } Quarta scientia liberalis dicitur esse Musica . Haec \\\hline
2.2.8 & si non gruesas razones e exenplarias ¶ \textbf{ La quarta sçiençia libal es dichͣ musica . } Et esta segunt que dize el philosofo & nisi rationes grossas et figurales . \textbf{ Quarta scientia liberalis dicitur esse Musica . Haec } secundum Philos’ \\\hline
2.2.8 & si les son otorgadas algunas cosas delectabłs conuiene que les otorguen cosas delectabło sin daño . \textbf{ por ende segunt que dize este mismo philosofo la musica es conueinble ala naturaleza de los mançebos por que les muestra delectaçicen sin danno ¶ } La segunda razon & dignum est quod ordinentur ad delectationes innocuas : \textbf{ quare ( secundum eundem Philosophum ) musica est consentanea naturae iuuenum , | quia habent innocuas delectationes . } Secunda ratio ad hoc idem esse potest , \\\hline
2.2.8 & que son conueinbles e honestos . \textbf{ Mas el philosofo tanne muchͣs razones en las politicas } por las quales se podrie mostrar & quae sunt licitae et honestae . \textbf{ Tangit enim Philosophus multas rationes in Politicis , } per quas ostendi posset , \\\hline
2.2.8 & e de los mouimientos dellas \textbf{ non puede ser sabida acabada mente . ¶ la septi mal çiençia } libal es dicho astrolozia çerca & et de cursibus eorum , \textbf{ perfecte sciri non potest . } Septima scientia liberalis dicitur esse astronomia , \\\hline
2.2.8 & libal es dicho astrolozia çerca \textbf{ la qual en antigo tienpo } por auentura trabaiaun a los fijos de los nobles & Septima scientia liberalis dicitur esse astronomia , \textbf{ circa quam forte antiquitus filii nobilium ideo insudabant , } quia gentiles \\\hline
2.2.8 & qua non estas . \textbf{ Ca la natural ph̃ia } que muestra conosçer las naturas delas cosas muy meior es & et has septem artes nimium extollebant . Veruntamen plures sunt aliae scientiae longe nobiliores istis . \textbf{ Nam Naturalis Philosophia docens cognoscere naturas rerum , } longe melior est , \\\hline
2.2.8 & mucho son aprouechables \textbf{ e neçessarias alos fijos de los bueons omes } e de los nobles & quae est de regimine ciuitatis et regni , valde sunt utiles \textbf{ et necessariae filiis liberorum et nobilium . Immo } ( ut in prosequendo patebit ) \\\hline
2.2.8 & que es dela linna visual e del iuso . \textbf{ La qual sçiençia es sola geometera . } Et la sçiençia dela fisica es sola ph̃ia natural . & quae est de visu , \textbf{ est sub Geometria . Medicina vero est } sub naturali Philosophia . \\\hline
2.2.8 & La qual sçiençia es sola geometera . \textbf{ Et la sçiençia dela fisica es sola ph̃ia natural . } Et las leyes e los derechos & est sub Geometria . Medicina vero est \textbf{ sub naturali Philosophia . } Leges et iura , \\\hline
2.2.8 & Et las leyes e los derechos \textbf{ que son de las obras de los omes sonsola politica } quees del gouernamiento delas çibdades . & Leges et iura , \textbf{ quae sunt de actibus hominum , | sub politica , } quae est de regimine ciuitatum . \\\hline
2.2.8 & que todos los legistas son \textbf{ assi commo vnos nesçios politicos . } Ca assi commo los legos e los omes del pueblo & ( ut alibi nos dixisse meminimus ) omnes legistae sunt \textbf{ quasi quidam idiotae politici . } Nam sicut laici et vulgares , \\\hline
2.2.8 & Por ende aquellos que non quieren argumentar artifiçialmente por sçiençia de logica estos son llamados del philosofo logicos \textbf{ nesçios en essa misma manera los legistas } por que aquellas cosas delas quales es la sçiençia politica & appellantur a Philosopho idiotae dialectici : \textbf{ sic Legistae , } quia ea de quibus est politica , \\\hline
2.2.8 & Ca commo les conuenga a ellos de ser \textbf{ assi commo medios dioses e de entender conueinblemente } e sin ninguna negligençia en los negoçios del regno . & Nam cum oporteat eos esse quasi semideos , \textbf{ et debite et absque negligentia negotium regni intendere , } non vacat eis subtiliter perscrutari scientias : \\\hline
2.2.8 & Et estas sçiençias son las sçiençias i trorales . \textbf{ Et pues que assi es en tanto les conuiene aellos de sablas o tris sçiençias } en quanto siruen ala ph̃ia moral . & ø \\\hline
2.2.8 & conuiene les alguacosado saber dela musica \textbf{ en quanto ella sirue alas bueans costunbres . } Et pues que & secundum Philosophum in politicis eos scire decet , \textbf{ inquantum deseruit ad bonos mores . } Sic ergo morale negocium scire expedit ab iis \\\hline
2.2.9 & assi el moço ha menester maestro e ayo . \textbf{ Ca assi commo la razon sienpre entiende a muy bueans cosas } assi commo es dicho en esse mismo terçero libro delas ethicas . & et paedagogo . \textbf{ Quare sicut ratio semper deprecatur ad optima , } ut in eodem tertio Ethicorum dicitur : \\\hline
2.2.9 & assi commo es dicho en esse mismo terçero libro delas ethicas . \textbf{ Assi avn el maestro de los moços deue sienpre enduzir e acuçiar los mocos a muy buenas costunbres . } Et pues que assi es & ut in eodem tertio Ethicorum dicitur : \textbf{ sic et magister puerorum semper debet eos | ad optima instigare . } Si igitur in doctrina puerorum finis intentus est , \\\hline
2.2.9 & que entiende el maestro en enssennar los moços \textbf{ es enduzir los a muy buenas costunbrs e asçiençia } por que dela fin se deue tomar razon de todas las otras cosas & Si igitur in doctrina puerorum finis intentus est , \textbf{ instigatio ad optima ; } quia ex fine accipienda est ratio omnium aliorum , \\\hline
2.2.9 & que pueda endozir los moços \textbf{ quel son en comnedados a muy grandes bienes . } Mas dos son las muy bueans cosas & talis quaerendus est magister , \textbf{ qui possit iuuenes sibi commissos ad optima inducere . Optimum autem } et bonum , ad quod inducendi sunt iuuenes , \\\hline
2.2.9 & quel son en comnedados a muy grandes bienes . \textbf{ Mas dos son las muy bueans cosas } e los grandes bienes & qui possit iuuenes sibi commissos ad optima inducere . Optimum autem \textbf{ et bonum , ad quod inducendi sunt iuuenes , } est duplex : \\\hline
2.2.9 & Mas dos son las muy bueans cosas \textbf{ e los grandes bienes } a que deuen ser endozidos los moços . & et bonum , ad quod inducendi sunt iuuenes , \textbf{ est duplex : } scientia scilicet , et mores . \\\hline
2.2.9 & a que deuen ser endozidos los moços . \textbf{ Conuiene a saber . Sçiençia . Et bueans costun bres . } Mas a buenas costunbres es adozido cada vno en dos ianeras . & est duplex : \textbf{ scientia scilicet , et mores . } Ad bonos autem mores inducitur dupliciter quis , \\\hline
2.2.9 & Conuiene a saber . Sçiençia . Et bueans costun bres . \textbf{ Mas a buenas costunbres es adozido cada vno en dos ianeras . } Couiene a saber . & scientia scilicet , et mores . \textbf{ Ad bonos autem mores inducitur dupliciter quis , } exemplo per bonitatem vitae , \\\hline
2.2.9 & que sea fallador dessi e entendedor de los otros \textbf{ e buen iudgador tan bien delas cosas entendidas } commo delas falladas . & qui sit inuentiuus ex se , intelligens alios , \textbf{ et bene iudicatiuus } tam de inuentis quam de intellectis . \\\hline
2.2.9 & que ha de fazer \textbf{ por que enssenne los moços en bueans costunbres . } Mas esta sabiduria & ut doceat in scientia : restat videre , \textbf{ qualiter debet esse prudens in agibilibus ut instruat in bonis moribus . } Ad huiusmodi autem prudentiam describendam , \\\hline
2.2.9 & que dixiemos enel primero libro dela sabiduria . \textbf{ Enpero cunple nos de contar las quatro cosas de aquellas ocho . } ¶ Et pues que assi es el maestro & quae in primo libro de prudentia tetigimus , \textbf{ sufficiat } tamen ad praesens quatuor \\\hline
2.2.9 & ga nunca la puede enderesçar \textbf{ si non conosçiere de qual parte esta tuerta . En essa misma manera aquel que quiere enderesçar los otros } nunca los podria enderesçar & nunquam eam rectificare posset \textbf{ nisi cognosceret | ex qua parte esset obliquata : } sic volens alios rectificare nunquam eos congrue rectificare posset \\\hline
2.2.9 & sin ningun mezclamiento de cosas falssas . \textbf{ En essa misma manera el que quiere enderesçar } e enformar los moços deue ser sabio & ø \\\hline
2.2.9 & aponiendo les los bienes \textbf{ sin mesclamiento de ninguons males ¶ } Lo quarto este doctor atal deue ser acatado e prouado por esperiençia . Ca aquel que prueua las cosas es & ut proponat suis auditoribus vera sine admixtione falsorum . Sic qui vult iuuenes dirigere debet esse , \textbf{ cautus proponens eis bona sine admixtione malorum . } Quarto huiusmodi doctor debet esse circumspectus , \\\hline
2.2.9 & que los pueda endozir \textbf{ por conueinbles castigos abien . } Los quales moços non solamente son endozidos & ut eos per debitos sermones , \textbf{ et per debitas monitiones inducat ad bonum . | Verum } quia ad bonum quis inducitur non solum monitionibus \\\hline
2.2.9 & Los quales moços non solamente son endozidos \textbf{ por buenos amonestamientos } e por buenos castigos mas avn por buean sobras & Verum \textbf{ quia ad bonum quis inducitur non solum monitionibus } vel correctionibus , \\\hline
2.2.9 & por buenos amonestamientos \textbf{ e por buenos castigos mas avn por buean sobras } e por bueons exienplos & quia ad bonum quis inducitur non solum monitionibus \textbf{ vel correctionibus , } sed etiam operibus et exemplis ; \\\hline
2.2.9 & e por buenos castigos mas avn por buean sobras \textbf{ e por bueons exienplos } e por ende conuiene & vel correctionibus , \textbf{ sed etiam operibus et exemplis ; } requiritur quod huiusmodi doctor sit in vita bonus \\\hline
2.2.9 & Como quier que el doctor de los moços les gponga \textbf{ por buenas palauras } lo que han de fazer . & et lasciuiam ; \textbf{ quantumcunque puerorum doctor eis verba bona proponeret , } si tamen opere contraria faceret , iuuenes illi exemplo inducti de facili ad illicita declinarent . \\\hline
2.2.9 & que dize el philosofo en el primero libro delas politicas \textbf{ mayor cuydado deuemos auer sienpre delas cosas que han alma } que delas que non han alma & quia \textbf{ secundum Philosophum 1 Politicorum , semper de animatis amplior cura habenda } quam de inanimatis , \\\hline
2.2.9 & que delas otras cosas . \textbf{ por cuydadolos e tener mientes con grand acuçia } qual maestro deuen poner en gouernamiento de sus fijos & et de filiis quam de aliis ; \textbf{ valde deberent esse soliciti , | et cum magna diligentia attendere , } qualem magistrum proponerent in regimine filiorum . \\\hline
2.2.10 & ¶ Lo segundo porque de ligero fablan cosas falłas \textbf{ ¶ Lo tercero por que en la mayor parte fablan cosas locas e non penssadas ante . } ¶ Lo primero fabla de ligero palauras orgullosas & quia \textbf{ ( } ut superius in primo libro dicebatur ) iuuenes sunt insecutores passionum , \\\hline
2.2.10 & ¶ Lo tercero por que en la mayor parte fablan cosas locas e non penssadas ante . \textbf{ ¶ Lo primero fabla de ligero palauras orgullosas } por que asi commo es dicho de suso en el primer libro & quia \textbf{ ( } ut superius in primo libro dicebatur ) iuuenes sunt insecutores passionum , \\\hline
2.2.10 & Et por ende es bien dicho \textbf{ que las palabras malas corronpen las bueans costunbres . } Otrossi deuen ser costrinudos & propter quod bene dictum est , \textbf{ quod corrumpunt bonos mores colloquia praua . Cohibendi ergo } et corrigendi sunt iuuenes , \\\hline
2.2.10 & Ca assi commo dixiemos dessuso los moços \textbf{ e los mançebos son de ligero mintrolos . } Et por ende commo la costunbre sea & ( ut superius diximus ) iuuenes sunt de facili mentitiui : \textbf{ cum ergo consuetudo sit } quasi altera natura , \\\hline
2.2.10 & que assi commo non les conuiene de fablar cosas torpes \textbf{ Et la razon desto pone el philosofo en łvij̊ libro delas ethicas } do dize & sic indecens est eos turpia videre . \textbf{ Ratio autem eius assignatur a Philosopho 7 Polit’ } ubi ait , \\\hline
2.2.10 & con mayor a miraçion \textbf{ e con mayor marauilla las veemos . } Por la qual cosamas somos acuçiosos cerca aquellas & quia quae primo aspiciuntur , \textbf{ cum maiori admiratione videntur , } quare magis sumus attenti circa illa , \\\hline
2.2.10 & co¶la segunda \textbf{ quanto ala manera dt veret conuiene de saber } quanto alas cosas iusibło sas & ø \\\hline
2.2.10 & por que los mançebos todas las cosas les paresçen \textbf{ assi commo si fuessen nueuas con mayor feruor } e mayor acuçiavan a aquellas cosas que veen . & quasi omnia sunt noua , \textbf{ maiori ardore feruntur } in illa quae vident . \\\hline
2.2.10 & por uision de cosas torpes \textbf{ de inclinar los amayores males . } ¶ Lo segundo deuemos dar cautella alos moços & Nam quia satis illa aetas de se prouocatur ad lasciuiam \textbf{ et ad passiones insequendas , non oportet ipsam per visionem turpium ad ulteriorem prouocare . } Secundo adhibenda est cautela in iuuenibus , \\\hline
2.2.10 & quanto ala manera de ver \textbf{ assi que alçen las palpebras de los oios con grand madureza } e que non echen los oios a cada parte con locura . & Secundo adhibenda est cautela in iuuenibus , \textbf{ ut instruantur quod palpebras oculorum cum maturitate eleuent , } ut non habeant oculos vagabundos . Inclinatur enim aetas illa ( eo quod omnia respiciat tanquam noua ) \\\hline
2.2.10 & quento ala fabla \textbf{ e quanto ala iusta finca de demostrar } en qual manera son de enssennar & et visionem : \textbf{ restat ostendere , } quomodo sunt instruendi , \\\hline
2.2.10 & Et quanto pertenesçe alo presente dos cautellas son de tomar \textbf{ ¶La primera quanto alas cosas que oyen ¶ La segunda } quanto alas personas que oyen . & etiam duplex cautela est adhibenda . Primo , \textbf{ quantum ad res auditas . Secundo , } quantum ad eos quos audit . In rebus autem auditis obseruatur cautela quantum ad iuuenes , \\\hline
2.2.10 & si les fuer defendido de oyr cosas torpes . \textbf{ Ca segunt el philosofo en el vi̊ libro delas politicas } do fabla desta materia es deue dar alos mançebos & ab auditione turpium . \textbf{ Nam secundum philosophum vii Polit’ } ubi de hac materia loquitur , \\\hline
2.2.10 & e desconuenible de oyr cosas torpes \textbf{ assi les conuienea ellos de oyr a bueons omes e honestos } e son de refrenar & et pulchra , et indecens audire turpia : \textbf{ sic decet eos audire viros bonos | et honestos , } et cohibendi sunt \\\hline
2.2.11 & a dixiemos de suso en comm̃ los moços \textbf{ luego en su moçedat deuia ser enssennados en buenas costunbres . } Onde por que dixiemos de suso & Diximus superius , iuuenes ab infantia \textbf{ instruendos esse in bonis moribus : } verum quia ( ut supra dicebatur ) \\\hline
2.2.11 & en qual manera los moços deuen ser enssennados \textbf{ e enformados en bueans costunbres . } Por la qual cosa despues que dixiemos & oportet specialiter tradere , \textbf{ quomodo iuuenes sunt in bonis moribus instruendi . } Quare postquam diximus , \\\hline
2.2.11 & Mas conuiene saber \textbf{ que cerça el comer pueden los omes errar en seys maneras . } ¶ Lo primero si comieren con grand garganteria ¶ Lo segundo si comieren much¶ & et qualiter se debeant habere iuuenes circa ipsum . Circa cibum autem contingit sex modis peccare , \textbf{ vel delinquere . } Primo si sumatur ardenter . Secundo , si nimis . Tertio , \\\hline
2.2.11 & que cerça el comer pueden los omes errar en seys maneras . \textbf{ ¶ Lo primero si comieren con grand garganteria ¶ Lo segundo si comieren much¶ } Lo terçero si comieren suzia & vel delinquere . \textbf{ Primo si sumatur ardenter . Secundo , si nimis . Tertio , } si turpiter . Quarto , \\\hline
2.2.11 & Lo quanto si comieren muy delicadamente \textbf{ ¶Lo vj̊ si comieren uianda apareiada con grand estudio } ¶lo primero digo que los omes pecan çerca el comer & si nimis laute vel delicate . Sexto , \textbf{ si nimis studiose . Delinquitur enim primo circa cibum , } si sumatur ardenter . \\\hline
2.2.11 & ¶lo primero digo que los omes pecan çerca el comer \textbf{ si lo tomaren con grand garganteria o con grand glotonia } por que esto non solamente enpeesçe al alma & si nimis studiose . Delinquitur enim primo circa cibum , \textbf{ si sumatur ardenter . } Nam hoc non solum nocet animae , \\\hline
2.2.11 & por que se fazen golosos aquellos que toman el comer muy cobdiçiosamente \textbf{ e con grand garganteria } e son destenprados . & et auide sumentes cibum fiunt gulosi et intemperati ; \textbf{ sed } etiam nocet corpori . \\\hline
2.2.11 & Mas avn enpeesçe al cuerpo . \textbf{ Ca la uianda tomada con grand garganteria non se } mas ca bien & etiam nocet corpori . \textbf{ Nam cibus nimia auiditate sumptus non bene masticatur , } et per consequens minus semper digeritur . Ordinauit enim natura animalibus dentes , ut per eos cibus debite tritus , \\\hline
2.2.11 & que mas ligeramente se conuerteria en la sustançia et en el nudrimiento del cuerpo . \textbf{ assi conmola senna taiada muy menuda } mas ayna se quema & et per consequens facilius conuerteretur in nutrimentum : \textbf{ sicut ligna minute trita facilius igniuntur , } et conuertuntur in ignem . \\\hline
2.2.11 & e se conuierte en fuego . \textbf{ Mas esta orden natural en la mayor parte } non la guardan & et conuertuntur in ignem . \textbf{ Hunc autem ordinem naturalem , } ut plurimum non obseruant sumentes cibum auide . \\\hline
2.2.11 & los que toman la uianda muy cuydadosamente \textbf{ e con grand golosia . } Ca assi commo da a entender el philosofo & ø \\\hline
2.2.11 & Ca assi commo da a entender el philosofo \textbf{ en el terçero libro delas ethicas pequana delectaçiones } quando la lengua alcança la uianda & ut plurimum non obseruant sumentes cibum auide . \textbf{ Nam ( ut innuit Philosophus 3 Ethicorum ) modica delectatio est , } cum cibus attingit linguam : \\\hline
2.2.11 & quando la lengua alcança la uianda \textbf{ mas mayor delectaçiones } quando la uianda llega ala garganta . & cum cibus attingit linguam : \textbf{ sed maior est , } cum attingit guttur . Gulosi ergo , \\\hline
2.2.11 & quando la uianda llega ala garganta . \textbf{ Et por ende los golosos que con grand cobdiçia e con grand delectaçion toman sauianda non se delectan mucho } por que este la uianda luengo tienpo en la boca & cum attingit guttur . Gulosi ergo , \textbf{ qui nimis auide et cum magna delectatione cibum sumunt , | non multum delectantur , } quod cibus \\\hline
2.2.11 & Et por ende los golosos que con grand cobdiçia e con grand delectaçion toman sauianda non se delectan mucho \textbf{ por que este la uianda luengo tienpo en la boca } mas cobdiçian & non multum delectantur , \textbf{ quod cibus | diu in ore existat , sed cupiunt } quod cito perueniat ad guttur . \\\hline
2.2.11 & que luego vaya ala garganta . \textbf{ Et por ende estos tales en la mayor parte } non mas can la uianda & quod cito perueniat ad guttur . \textbf{ Ideo tales ut plurimum cibum non masticant , } sed eum immasticatum trasglutiunt . \\\hline
2.2.11 & Por la qual cosa \textbf{ si en tan grand quantia se tomaque la calentura natural non pueda enssennorar } sobrella non se puede bien moler nin cozer . & Si enim cibus digeri debeat , oportet ipsum esse proportionatum calori naturali . \textbf{ Quare si in tanta quantitate sumatur , quod calor naturalis ei dominari non possit , non bene digeritur , } et per consequens non causat debitum nutrimentum . \\\hline
2.2.11 & iij . pecan si toma la uianda torpemente \textbf{ e suzia mente . } Ca son muchos que non saben gouernar assi mismos . & Tertio delinquitur , \textbf{ si sumatur turpiter . Sunt enim plurimi seipsos pascere nescientes , } quod vix aut nunquam comedere possunt , \\\hline
2.2.11 & en el resçibimiento del maniar \textbf{ non solamente deuemos esquiuar la golosina e la grand cobdiçia } mas avn deuemose squiuar de comer la vianda torpemente & signum sit cuiusdam gulositatis , \textbf{ vel inordinationis mentis in sumptione cibi , non solum cauendus est ardor et nimietas , } sed \\\hline
2.2.11 & quando alguno se acostunbra a tomar la uianda en algua ora desordenada \textbf{ por la mayor parte dessea dela tomar en aquella misma ora . } Et pues que assi es & sumere cibum in aliqua hora , \textbf{ ut plurimum appetit sumptionem eius in eadem hora . } Si ergo inordinate \\\hline
2.2.11 & si del ordenadamente toma la uianda \textbf{ e vsare comer ante de tien po en la mayor parte } assi que ante dela digestiuo & Si ergo inordinate \textbf{ et praeter consuetam horam utatur quis cibum sumere , } ut plurimum ante digestionem primi cibi sumitur secundus cibus , laeditur ergo inde corpus . Hora ergo debita et determinata , non solum propter bonitatem animae , \\\hline
2.2.11 & mas que la condiçionde la su persona demanda \textbf{ e mas que conuiene al su estado demanda viandas delicadas peca enllo . } Por que esto viene & ultra quam conditio personae exigat , \textbf{ et ultra quam eius status requirat , | delicata cibaria quaerat , delinquit : } quia hoc ex aliqua intemperantia , \\\hline
2.2.11 & Lo sexto peca el omne çerca la vianda \textbf{ si demandare viandas apareiadas con grant estudio } por que avn en las viles viandas cada vno se puede mostrar & vel ex aliquo vitio prouenit . Sexto delinquitur , \textbf{ si quaerantur cibaria nimis studiose parata . Nam } etiam in vilibus cibariis potest \\\hline
2.2.11 & si demandare viandas apareiadas con grant estudio \textbf{ por que avn en las viles viandas cada vno se puede mostrar } por muy goloso & si quaerantur cibaria nimis studiose parata . Nam \textbf{ etiam in vilibus cibariis potest } quis ostendere se nimis gulosum , \\\hline
2.2.11 & et non quieren comer \textbf{ por que biuna por que ponen grand estudio e grant cuydado çerca los apareiamientos delas viandas . } Et pues que assi es estas cosas vistas de ligero pueden paresçer & non comedere ut viuant , \textbf{ cum nimium studium | et nimiam } curam apponant circa praeparamenta ciborum . \\\hline
2.2.12 & Ca do mayor es el periglo \textbf{ alli deue omne poner mayor remedio } Et por ende en la hedat de los moços deuemos guardar & maxime est prona ad intemperantiam , \textbf{ quare cum semper sit adhibenda cautela , } ubi maius periculum imminet , \\\hline
2.2.12 & que abiua el omne asa lux̉ia . \textbf{ Ca escalentado el cuerpo faze se enł omne mayor inclinaçion alas obras de luxia . } Onde el vino tomado destenpradamente faze en el omne grand calentraa & quia venerea prouocat . \textbf{ Cum enim corpore calefacto maior fiat incitatio ad actus venereos , } vinum \\\hline
2.2.12 & Ca escalentado el cuerpo faze se enł omne mayor inclinaçion alas obras de luxia . \textbf{ Onde el vino tomado destenpradamente faze en el omne grand calentraa } e abiualo a destenprança de lux̉ia . & Cum enim corpore calefacto maior fiat incitatio ad actus venereos , \textbf{ vinum | quod maxime calorem efficit immoderate sumptum , } incitat ad incontinentiam nimiam . \\\hline
2.2.12 & e ençiende la sangre \textbf{ por ende faze el omne de mayor coraçon } e mas sanudo la qual cosa fechͣ & quod propter sui caliditatem inflammat sanguinem , \textbf{ reddit hominem animosum et irascibilem : } quo facto \\\hline
2.2.12 & e mayormente alos Reyes \textbf{ e alos prinçipes de auer grand cuydado çerca el gouernamientode los fijos } por que en tal manera sean gouernados en sumo oçedat & decet omnes patres \textbf{ et maxime reges et Principes solicitari circa regimen filiorum , } ut ab ipsa infantia sic regantur , \\\hline
2.2.12 & e alos prinçipes de auer grand cuydado çerca el gouernamientode los fijos \textbf{ por que en tal manera sean gouernados en sumo oçedat } por que sean guardados e mesurados ¶ & et maxime reges et Principes solicitari circa regimen filiorum , \textbf{ ut ab ipsa infantia sic regantur , } quod sint abstinentes \\\hline
2.2.12 & que los mançebos deuen ser enssennados \textbf{ por que non sean golosos finca de dezer } en qual manera deuen ler enssennados & iuuenes ipsos instruendos esse , \textbf{ ne sint gulosi : | restat dicere , } quomodo instruendi sunt , \\\hline
2.2.12 & por que non puedan ser malos nin viçiosos . \textbf{ En essa misma manera los mançebos } que non quieren guardar sus cuerpos deuen los en dozir & quia decet patrem sic solicitari erga filios , \textbf{ ut sint virtuosi , } iuuenes continere nolentes , inducendi sunt \\\hline
2.2.12 & e enbatganse en el acresçentamiento de los cuerpos de los moços e de los nançebos \textbf{ assi commo dize el philosofo en esse mismo libro dela o politicas } Et pues que assi es en esta misma manera deuemos vsar del casamiento & et impeditur eorum augmentum , \textbf{ ut vult Philosophus in eisdem Poli’ . } Sic ergo utendum est coniugio , \\\hline
2.2.12 & assi commo dize el philosofo en esse mismo libro dela o politicas \textbf{ Et pues que assi es en esta misma manera deuemos vsar del casamiento } si lanr̃afuerça del appetito desseador non fuere muy corrupta . & ut vult Philosophus in eisdem Poli’ . \textbf{ Sic ergo utendum est coniugio , } si nostra vis concupiscibilis \\\hline
2.2.13 & assi commo prueua el philosofo \textbf{ en el viij̊ libro delas politicas } es neçessario enla vida humanal & et circa vestitum debeant se habere . Ludus autem , \textbf{ ut probat Philosophus 8 Poli’ est necessarius in vita quod ( quantum ad praesens spectat ) duplici via declarari potest . Primo , } ex vitatione illicitae solicitudinis . Secundo , \\\hline
2.2.13 & por que non fallezca el omne \textbf{ por los grandes trabaios de alcançar su fin } conuienel de entreponer alguons trebeios & expedit aliquos ludos \textbf{ et aliquas deductiones interponere suis curis , } ut ex hoc aliquam requiem recipientes , \\\hline
2.2.13 & por los grandes trabaios de alcançar su fin \textbf{ conuienel de entreponer alguons trebeios } e algunos solazes en sus cuydados . & et aliquas deductiones interponere suis curis , \textbf{ ut ex hoc aliquam requiem recipientes , } magis possint laborare in consecutione finis . Unde et Philosophus 8 Politicorum ait , \\\hline
2.2.13 & que alcançe aquella fin \textbf{ por ende conuienele de entroponer alguons trebeios } e algunas delectaconnes & antequam consequatur illum , \textbf{ ideo oportet interponere aliquos ludos , } et aliquas delectationes , \\\hline
2.2.13 & en vano tiene la boca abierta . \textbf{ Avn en esta misma manera el omne non fabla } por los pies & cum quis vult audire alium , \textbf{ retinet os apertum . Sic etiam homo non loquitur pedibus , } nec manibus , \\\hline
2.2.13 & para esto son desenssennandos en los gestos . \textbf{ En essa misma manera son desenssennados segunt los gestos } aquellos que quando que eren fablar estienden los pies & tenent ora aperta : \textbf{ sic sunt indisciplinati | secundum gestus , } qui cum volunt loqui , \\\hline
2.2.13 & que non siruen en ninguna cosa ala fabla . \textbf{ En esta misma manera son de enssennar los moços } que ayan tales gestos & vel erigunt humeros , vel faciunt alia , \textbf{ quae ad locutionem nihil deseruiunt . Sic ergo disciplinandi sunt iuuenes , } ut habeant tales gestus , et ut sic utantur motibus membrorum , \\\hline
2.2.13 & que entienden fazer . \textbf{ Ca fazer alguons mouimientos de los mienbros } que non siruen ala obra que entienden fazer . & ut deseruiant ad opera quae intendunt . \textbf{ Nam agere aliquos motus membrorum non deseruientes operi intento , } vel procedit ex insipientia mentis , \\\hline
2.2.13 & por que el bien honesto puede ser dich̃o honrrable . \textbf{ Ca el bien honesto es essa misma cosa } que estado de honrra . & Bonum enim honestum , bonum honorabile dici potest : \textbf{ nam honestum idem est , } quod honoris status . \\\hline
2.2.13 & e de ligero saltan en locania . \textbf{ ¶ Lo segundo la grand blandura delas vestiduras faze al omne temeroso } por que las armas del fierro han en ssi alguna dureza & sed molles et de facili in lasciuiam prorumpunt . Secundo , \textbf{ nimia mollicies vestium reddit hominem timidum . } Nam cum arma ferrea in se quandam duriciem habeant , \\\hline
2.2.14 & e en qual conpannia deuen beuir . \textbf{ Mas quanto pertenesçe alo presente quatro cosas paresçe } que son enlos mançebos & nisi sciuerint cum quibus sociis debeant conuersari . \textbf{ Quanto autem ad praesens spectat , | quatuor videntur inesse ipsis iuuenibus , } ex quibus quatuor rationes sumi possunt , \\\hline
2.2.14 & que son enlos mançebos \textbf{ delas quales se pueden tomar quatro razones } para prouar & quatuor videntur inesse ipsis iuuenibus , \textbf{ ex quibus quatuor rationes sumi possunt , } quod maxime in iuuenili aetate fugienda sit puerorum societas . Iuuenes enim primo sunt molles , \\\hline
2.2.14 & para prouar \textbf{ que en la hedat dela mançebia deue ser mucho escusada la mala conpannia } e deuen foyr della . & ex quibus quatuor rationes sumi possunt , \textbf{ quod maxime in iuuenili aetate fugienda sit puerorum societas . Iuuenes enim primo sunt molles , } et ductiles . Secundo sunt passionum \\\hline
2.2.14 & para mostrar \textbf{ que conuiene alos mançebos de foyr la mala conpannia se toma } desto que los mançebos son muy muelles e muy tristor nabłs & Prima ergo via ad ostendendum maxime competere iuuenibus , \textbf{ fugere societatem prauam , sumitur ex eo quod iuuenes sunt nimis molles } et ductiles . \\\hline
2.2.14 & desto que los mançebos son muy muelles e muy tristor nabłs \textbf{ Ca el alma en la mayor parte sigue la conplission del cuerpo . } Ca por que el nuestro conosçimiento comiença en el seso & et ductiles . \textbf{ Anima enim ut plurimum sequitur complexiones corporis : } nam quia nostra cognitio incipit a sensu , \\\hline
2.2.14 & e las colas senllibls lon anos manitieltas mas . \textbf{ Por ende en la mayor parte los omes } siguen el appetito senssitiuo de los sesos & nam quia nostra cognitio incipit a sensu , \textbf{ et sensibilia sunt nobis magis nota ; } ideo ut plurimum homines sequuntur appetitum sensitiuum . Appetitus autem sensitiuus est virtus organica siue corporalis . \\\hline
2.2.14 & Et pues que assi es mucho son de castigar en aquella hedat \textbf{ que non se alleguen a mala conpannia . } ¶ La segunda razon & quam in alia . \textbf{ Maxime ergo tunc prohibendi sunt a societate praua . } Secunda via ad inuestigandum hoc idem , \\\hline
2.2.14 & quando determinamos delas costunbres de los moços mayormente es de esquiuar en la hedat dela mançebia \textbf{ de se llegara mala conpannia . } ¶ La terçera razon & ut patuit cum determinauimus de moribus iuuenum ; \textbf{ maxime in iuuenili aetate cauendum est a societate praua . } Tertia via ad probandum hoc idem , \\\hline
2.2.14 & mucho son de costrennir \textbf{ e arredrar los mançebos dela mala conpannia } por que estonçe se enforman mayormente en las costunbres de los conpannones & ø \\\hline
2.2.15 & e dende adelante \textbf{ Mas el pho tanne enł septimo libro delas politicas seys cosas } que se deuen guardar çerca los moços enla primera hedat . & ad decimumquartum annum . Postea a decimoquarto et deinceps . \textbf{ Tangit autem Philosophus 7 Polit’ sex circa ipsos pueros , } quae seruanda sunt in aetate primitiua . Primum est , \\\hline
2.2.15 & por que de ligero enferman los moços \textbf{ e se fazen de mala disposicion enel cuerpo } si en el tienpo & ø \\\hline
2.2.15 & que el vso \textbf{ que toman en el frio faze buena disposiconn en los moços } por la calentura que es en ellos & unde idem Philosophus ait , \textbf{ quod exercitium ad frigora facit bonum habitudinem in pueris propter caliditatem existentem in ipsis . Secundo exercitium ad frigora paruis pueris utile est ad bellicas actiones . } Nam frigus membra consolidat \\\hline
2.2.15 & si encomiento dela moçedat fueren vsados en alguna manera alos frios \textbf{ Onde esse mismo philosofo dize } que es costunbre entre algunas nasçiones barbaras & si a pueritia sint aliqualiter exercitati ad frigora : \textbf{ unde idem Philosophus ait , } quod apud aliquas Barbaras nationes consuetudo est in fluminibus frigidis balneare filios , \\\hline
2.2.15 & que quando dezimos \textbf{ que los moços pequanos son de acostunbrara esto o aquello deue se entender tenpradamente } e por grados & quod cum dicimus pueros paruos assuescendos esse ad hoc vel ad illud , \textbf{ intelligendum est moderate } et gradatim , \\\hline
2.2.15 & por que segunt el philosofo el mouimiento tenprado \textbf{ en los mocos faze seys bienes ¶ } Lo primero faze el cuerpo mas sano & et temperatos motus . Nam \textbf{ secundum Philosophum , motus temperatus in pueris quatuor bona facit . Primo , } quia reddit corpora magis sana : \\\hline
2.2.15 & Ca si non fueren acostunbrados a algunos mouimientos \textbf{ e fazen se pesados e ꝑezosos e mal dispuyestos ¶ } Lo terçero faze el mouimiento & nam nisi ad aliquales motus assuescant , \textbf{ fiunt graues , pigri , | et inertes . } Tertio facit ad augmentum . \\\hline
2.2.15 & e faze abuean disposiçion del cuerpo \textbf{ Et por ende se sigue que sea aprouechoso alacresçentamiento del cuerpo . } Ca commo el acresçentamiento se faga del nudrimiento aquellas cosas & et facit ad bonam dispositionem corporis , \textbf{ sequitur quod sit quoddam proficuum ad augmentum . } Nam \\\hline
2.2.15 & que fazen el cuerpo bien dispuesto \textbf{ e quel fazen que se cere bien son aprouechabłsal acresçentamiento del ¶ } Lo quarto el mouimiento tenprado firma los mienbros & et quod bene nutriatur , \textbf{ et alatur , | sunt proficua ad augmentum . } Quarto moderatus motus membra consolidat : \\\hline
2.2.15 & Et pues que assi es los moços \textbf{ por que han los mienbros muy tiernos deuen los acostunbrar a algers mouimientos pequanos et tenpdos } por que los mienbros dellos sean mas firmes . & et fiunt fortiora . Pueri ergo \textbf{ quia nimis habent tenera membra , | ad aliquos motus modicos } et temperatos sunt assuescendi , \\\hline
2.2.15 & en tanto lo alaba el philosofo \textbf{ que diz que luego enł comienço de su nasçençia deuen fazer alguons instrumentos } en que se mueun a los moços & et ad non defluere propter teneritudinem : moderatum enim motum in pueris adeo laudat Philosophus , \textbf{ ut ab ipso primordio natiuitatis dicat , | fienda esse aliqua instrumenta , } in quibus pueri vertantur , \\\hline
2.2.15 & e fazen se los cuerpos mas ligeros \textbf{ Otrossi avn deuen rezar alos moços alguas estorias } despues que comiençan at entender las significaçiones delas palabras . & et redduntur corpora agiliora . Sunt \textbf{ etiam pueris recitandae aliquae fabulae , | vel aliquae historiae , } postquam incipiunt percipere significationes verborum . \\\hline
2.2.15 & por que los moços non pueden sostir ninguna cosa triste . \textbf{ Por ende es bien de los acostunbrara algs trebeios tenprados } e a alguas delectaçiones honestas & Vel etiam aliqui cantus honesti sunt eis cantandi . \textbf{ Nam ipsi nihil tristes sustinere possunt : ideo bonum est , eos assuescere ad aliquos moderatos ludos , } et ad honestas aliquas et innocuas delectationes . \\\hline
2.2.15 & que non lloren \textbf{ por esse mismo defendimiento se faze } que retengan en ssi el spun e el eneldo . & Nam cum pueri a ploratu cohibentur , \textbf{ ex ipsa prohibitione fit , } ut retineant spiritum et anhelitum . \\\hline
2.2.16 & segunt el departimiento delas ꝑssonas . \textbf{ Ca algunos son mas fuertes ent cuerpo entdeçimo año } que otros en el seyto deçimo . & secundum diuersitatem personarum . \textbf{ Nam aliqui sunt robustiores corpore in duodecim annis , } quam alii in sedecim . Ideo \\\hline
2.2.16 & deuen se acostun brar poco a poco \textbf{ e de grado en grado amas altos trabaios } e amas fuertes mouimientos . & Sed cum impleuerunt septennium usque ad annum decimum quartum , \textbf{ debent gradatim assuescere ad ulteriores labores , } et ad fortiora exercitia . \\\hline
2.2.16 & e de grado en grado amas altos trabaios \textbf{ e amas fuertes mouimientos . } Ca el trebeio dela pella & debent gradatim assuescere ad ulteriores labores , \textbf{ et ad fortiora exercitia . } Ludus enim pilae \\\hline
2.2.16 & nin otras obras fuertes . \textbf{ Et por ende el pho enł viij̊ libro delas politicas dize } que fasta la hedat dela pubeçençia & quod nimis sit tenera , non sunt assumenda opera militaria nec opera ardua . \textbf{ Unde Philosophus 8 Polit’ ait , } quod usque ad pubescentiam , \\\hline
2.2.16 & que el philosofo enl vij̊ . \textbf{ libro delas politicas dizeque muy mala cosa es de non enssennar } e de non enduzir los mocos a uirtudes & quod Philosophus 5 Polit’ ait , \textbf{ quod pessimum est non instruere pueros ad virtutem , et ad obseruantiam legum utilium . Inquirit enim Philosophus 8 Polit’ } utrum prius curandum sit de pueris , \\\hline
2.2.16 & e aguardar las leyes bueans e aprouechosas . \textbf{ Ca el philosofo enłvij̊ libͤ delas politicas } demanda & ø \\\hline
2.2.16 & por que ayan appetito e desseo conuenible o por que ayan entendimiento acabado . \textbf{ Mas el ph̃ prueua } que pmeramente deuemos auer cuydado de la ordenaçion dela uoluntad & vel ut habeant perfectum intellectum . \textbf{ Probat autem prius esse curandum } de ordinatione voluntatis , \\\hline
2.2.16 & por la qual deuen ser tenpradas . \textbf{ las cobdiçias delos mançebos es en poner speçial cautela çerca aquellas cosas } en que ellos se acostunbraron comunalmente de fallesçer . & quo moderandae sunt concupiscentiae iuuenum , \textbf{ est | ut specialis cautela adhibeatur circa illa , } circa quae maximae consueuerunt deficere . \\\hline
2.2.16 & que fazen fazen las mucho \textbf{ mas que deuen asi que quando aman am̃a much̃ . Etrͣndo comiençan de trebeiar trebeian much̃ . } Et assi que todas las cosas que fazen fazen la ssienpre con sobrepuiança . & et omnia faciunt valde , \textbf{ ita quod cum amant nimis amant , | cum incipiunt ludere nimis ludunt , } et in caeteris aliis semper excessum faciunt , \\\hline
2.2.16 & e cuydar çerca la ordeuaçion dela uoluntad \textbf{ por que por la mayor parte en todo el segundo se tenatio } los moços & ideo principalius est insistendum circa ordinationem appetitus ; \textbf{ quasi enim per totum } secundum septennium pueri non addiscunt \\\hline
2.2.16 & non aprenden \textbf{ si non palabras por que non son de tan grand entendimiento } que puedan penssar en las cosas . & secundum septennium pueri non addiscunt \textbf{ nisi verba : | nondum enim sunt tanti intellectus , } ut de ipsis rebus considerare possint : \\\hline
2.2.17 & ni emos dessuso \textbf{ que trs cosas deuemos entender c̃ca los fijos . } Conuiene a saber & de quibus fecimus mentionem . \textbf{ Dicebatur supra circa filios tria intendenda esse , } videlicet quomodo habeant bene dispositum corpus , \\\hline
2.2.17 & commo quier que non acabada mente \textbf{ por ende en aqł tienpo } non solamente auemos de auer cuydado & et aliquo modo \textbf{ ( licet imperfecte ) incipiunt participare rationis usum , } ideo in illo tempore non solum curandum est quomodo habeant perfectum corpus , \\\hline
2.2.17 & non solamente la guamatica que paresçe \textbf{ que es assi comm̃ sciençia dela signiłicaçion delas palabras o la logica } que es assi commo vna manera de saber & non solum grammaticam \textbf{ quae videtur esse scientia verborum , | vel dialecticam } quae est quidam modus sciendi , \\\hline
2.2.17 & Et pues que assi es enel primer se tenario \textbf{ despues del resçebemientodel bautismo } e de los sacͣmentos deuen entender prinçipalmente çerca vna cosa & In primo ergo septennio post receptionem baptismatis \textbf{ et sacramentorum Ecclesiae , } intendendum est principaliter \\\hline
2.2.17 & e de los sacͣmentos deuen entender prinçipalmente çerca vna cosa \textbf{ assi commo çerca buena disposiconn del cuerpo . } Mas enł segundo se tenario & quasi circa unum , \textbf{ ut circa bonam dispositionem corporis . } Sed in secundo septennio , \\\hline
2.2.17 & Et çerca el alunbramiento del entendimiento . \textbf{ Mas la buena disposiçonn del cuerpo } aquellos que quieren beuir uida politica e de çibdat es esta & et circa illuminationem intellectus . \textbf{ Bona autem dispositio corporis , } ut volentibus viuere vita politica , \\\hline
2.2.17 & Et destonçe adelante deuen tomar trabaios mas fuertes \textbf{ Et esto dize elpho enł viij̊ libro delas politicas } que fasta los . xiiij años los moços deuen ser acostunbrados a trabaios ligeros & ex tunc autem assumendi sunt labores fortiores . \textbf{ Nam et Philosophus 8 Poli’ | ait , } quod usque ad quartumdecimum annum pueri assuescendi sunt ad labores leues : \\\hline
2.2.17 & quando lo ha tal qual demanda el su ofiçio \textbf{ la qual cosa non puede ser sin fuerte trabaio de su cuerpo . } ¶ Et pues que assi es commo todos aquellos que quieren beuir uida çiuil & quale requirit officium militare . \textbf{ Quod sine forti exercitatione corporis esse non potest . } Cum ergo omnes volentes viuere vita politica , \\\hline
2.2.17 & ¶ Et pues que assi es commo todos aquellos que quieren beuir uida çiuil \textbf{ conuiene les de sofrir alguas uegadas fuertes trabaios } por defendemiento dela tierra . & oporteat \textbf{ aliquando sustinere fortes labores pro defensione reipublicae : } omnes volentes viuere tali vita a quartodecimo anno ultra sic debent assuefieri ad aliqua officia robusta , \\\hline
2.2.17 & por defendemiento dela tierra . \textbf{ Et por ende desde los xiiij̊ annos adelante } deuen se acostunbrar a fuertes trabaios & aliquando sustinere fortes labores pro defensione reipublicae : \textbf{ omnes volentes viuere tali vita a quartodecimo anno ultra sic debent assuefieri ad aliqua officia robusta , } quod si aduenerit tempus \\\hline
2.2.17 & Et por ende desde los xiiij̊ annos adelante \textbf{ deuen se acostunbrar a fuertes trabaios } assi que si viniere tienpo & aliquando sustinere fortes labores pro defensione reipublicae : \textbf{ omnes volentes viuere tali vita a quartodecimo anno ultra sic debent assuefieri ad aliqua officia robusta , } quod si aduenerit tempus \\\hline
2.2.17 & por que ayan el appetito bien ordenado . \textbf{ Ca paresçe que los mançebos de xiiii̊ años arriba } quanto ala ordenaçion del appetito espeçialmente fallesçen en dos cosas . & quomodo solicitari debeant circa eos , \textbf{ ut habeant ordinatum appetitum . Videntur autem iuuenes a decimoquarto anno } intra , quantum ad inordinationes appetitus \\\hline
2.2.17 & mas cobdiçiosamente de segnir las obras de luxia Et \textbf{ pues que assi es do paresçe mayor periglo } alli deuemos poner mayor cautella e mayor remedio . & quia tunc incipiunt ardentius circa venerea incitari . \textbf{ Quia ergo semper cautela est adhibenda } ubi maius periculum imminet : \\\hline
2.2.17 & pues que assi es do paresçe mayor periglo \textbf{ alli deuemos poner mayor cautella e mayor remedio . } por ende del xiiij & Quia ergo semper cautela est adhibenda \textbf{ ubi maius periculum imminet : } a quartodecimo anno ultra specialiter monendi sunt iuuenes , \\\hline
2.2.17 & Mas el pho pone en el vii̊ . \textbf{ libro delas politicas tres razono breues } por que conuiene alos fijos de ser lubiectos e obedientes a lus padres e alos vieios . & ut non sint elati , \textbf{ sed sint subiecti et obedientes suis patribus et senioribus . Tangit autem Philosophus 8 Poli’ tres breues rationes , quare decet filios esse subiectos , et obedire senioribus , } et patribus . Prima ratio est : \\\hline
2.2.17 & a aquel \textbf{ que sabe que nol manda algua cosa } si non por el su prouecho ¶ & et patres imperant iuuenibus propter eorum bonum : \textbf{ quilibet autem obedire debet ei quem scit non percipere aliqua nisi ad bonum eius . } Secunda est ; \\\hline
2.2.17 & si primero non aprendiere de ser subiecto \textbf{ por que ninguon non es buen maestro } si primero non fuere buen disçipulo . & nisi prius didicerit esse subiectus . \textbf{ Nullus enim est bonus magister , } nisi prius extiterit bonus discipulus : \\\hline
2.2.17 & por que ninguon non es buen maestro \textbf{ si primero non fuere buen disçipulo . } Et pues que assi es & Nullus enim est bonus magister , \textbf{ nisi prius extiterit bonus discipulus : } ut ergo ipsi valeant bene principari , \\\hline
2.2.17 & commo non conuiene . \textbf{ Ca los mançebos de los xiiij̊ anos adelante } non solamente son de enduzir & ut velint patribus et senioribus obedire . Secundo rectificandus est , \textbf{ ne venerea illicita prosequantur . Iuuenes a decimoquarto anno ultra non solum inducendi sunt , } ut sint abstinentes \\\hline
2.2.17 & Et por ende en la hedat muy de moço \textbf{ que dura fasta los . xiiii̊ años tałs ̃ sçiençias altas non son de proponer } nin de enssennar a ellos & et insecutor passionum non est sufficiens auditor . In aetate ergo nimis puerili , quae durat usque \textbf{ ad decimumquartum annum , tales scientiae non sunt proponendae illis : } sed a decimoquarto anno ultra \\\hline
2.2.18 & e de los prinçipes atales \textbf{ tra lasios corporales } merenos son doma e devlar & Non tamen omnes iuuenes \textbf{ ad huiusmodi labores sunt equaliter exercitandi : } nam filii Regum et Principum ad huiusmodi labores corporales minus sunt exercitandi quam alii , et adhuc primogeniti \\\hline
2.2.18 & que deue regnar \textbf{ conuiene de tomar menores trabaios } ca segunt el pho en łviij̊ libro delas politicas los trabaios corporales & nam filii Regum et Principum ad huiusmodi labores corporales minus sunt exercitandi quam alii , et adhuc primogeniti \textbf{ qui regere debent decet minores labores assumere . } Nam secundum Philosophum 8 Politicorum labor corporalis , \\\hline
2.2.18 & conuiene de tomar menores trabaios \textbf{ ca segunt el pho en łviij̊ libro delas politicas los trabaios corporales } e el cuydar del entendimiento & qui regere debent decet minores labores assumere . \textbf{ Nam secundum Philosophum 8 Politicorum labor corporalis , } et consideratio per intellectum , \\\hline
2.2.18 & por los quales se faz la carne dura enbargan la sotileza del alma e del entendimiento . \textbf{ por ende bien dicho es lo que dize el pho enlvii̊ libro delos fisicos } que el alma en seyendo e enfolgado se faze sabia & ex quibus redditur caro dura , \textbf{ impediunt subtilitatem mentis . Bene ergo dictum est quod dicitur 7 Physicorum } quod anima in sedendo \\\hline
2.2.18 & si non fuere bien ayuntada e bien ordenada es muy pequana mas ayuntada e orde nada \textbf{ por la sabidia del rey puede obrar grandes cosas . } Et pues que assy es commo quier que los Reyes e los prinçipeᷤ non de una de todo dexar el vso delas armas & modica possunt : \textbf{ unita vero | et ordinata per reges maxima operari valent . } Licet ergo Reges et Principes non omnino ignorare debeant armorum usum , \\\hline
2.2.18 & e pueden se llegar ala sabiduria \textbf{ si con grand acuçia estudiaren enlas sçiençias morales } por que puedan conosçer las costunbres de los omes & quam fortitudini corporali . Vacabunt autem prudentiae , \textbf{ si diligenter insistant circa morales scientias , } ut possint mores hominum \\\hline
2.2.18 & estudiando enlas sçiençias morales \textbf{ cuydando mucho a menudo enlas bueans costunbres del regno } e oyendo mucħa menudo los fechos & vacando moralibus scientiis , \textbf{ recogitando frequenter bonas consuetudines regni , audiendo saepius acta praedecessorum bene regentium regnum . Hoc ergo modo , } videlicet , vacando actibus prudentiae , \\\hline
2.2.18 & e bien gouernaron el regno . Et pues que assi es dando se los reyes a sabiduria \textbf{ e abueans costunbres pueden ellos } e sus herederos & a Regibus et Principibus , \textbf{ et a suis haeredibus magis est vitanda inertia , } quam per laborem , \\\hline
2.2.19 & ommo por el vso del casamiento \textbf{ non sola mente nazcanfuos } mas pueden nasçer fijos e fijas spues & et exercitium corporalem . \textbf{ Cum ex usu coniugii non solum oriantur filii et mares , } sed oriri possunt filiae \\\hline
2.2.19 & ca assi commo conuiene alas madres \textbf{ de ser continentes e castas e guardadas e mesuradas en essa misma manera conuiene alas fijas de ser tales Et pues que assi es estas cosas } e otras muchas & Nam sicut decet coniuges esse continentes , pudicas , abstinentes , et sobrias : sic decet \textbf{ et filias . Haec ergo , } et multa alia , \\\hline
2.2.19 & Et pues que assi es \textbf{ por que tan grand inclinaçion auemos alas delecta con nes senssibles } en la mayor parte pecamos en tales cosas si nos fuere dada manera de pecar & quod communiter homines non reputant nisi sensibilia bona . \textbf{ Quia ergo tantum habemus impetum ad delectationes sensibiles , } ut plurimum deliquimus in talibus ; \\\hline
2.2.19 & por que tan grand inclinaçion auemos alas delecta con nes senssibles \textbf{ en la mayor parte pecamos en tales cosas si nos fuere dada manera de pecar } Onde el philosofo en la rectorica dize & Quia ergo tantum habemus impetum ad delectationes sensibiles , \textbf{ ut plurimum deliquimus in talibus ; | si adsit nobis commoditas delinquendi , } unde et Philosophus in Rheto’ vult quod homines \\\hline
2.2.19 & Onde el philosofo en la rectorica dize \textbf{ que los omes en la mayor parte fazen mal quando pueden . } Et pues que assi es muy grant cautela es de poter & unde et Philosophus in Rheto’ vult quod homines \textbf{ ut plurimum male faciant , | cum possunt . } Maxima ergo cautela ad conseruandam puritatem et innocentiam , \\\hline
2.2.19 & que los omes en la mayor parte fazen mal quando pueden . \textbf{ Et pues que assi es muy grant cautela es de poter } para guardar la sinpieza & cum possunt . \textbf{ Maxima ergo cautela ad conseruandam puritatem et innocentiam , } est vitare commoditates malefaciendi , \\\hline
2.2.19 & e por ende si en los omes \textbf{ en los quales es la razon e el entendimiento mayor es grant peligro } de non escusar las azinas de los pecados much mas es esto de escusar en las mugers . & Si ergo in viris , \textbf{ in quibus est ratio praestantior , } est magnum periculum non vitare commoditates delictorum : multo magis hoc est in foeminis , \\\hline
2.2.19 & que non sigua las cosas desconuenibles . \textbf{ Et pues que assi es muy grant freno delas fenbras } e mayormente delas moças es la uerguença & propter quam quis prohibetur , \textbf{ ne prosequatur illicita maximum fraenum foeminarum , } et potissime puellarum , \\\hline
2.2.19 & mas si fueren alongadas de vsar con los omes \textbf{ assi conmo cosas montes } mas fuyen del tannimiento e del allegamiento delos omes . Et esto que ueemos en las otras aian lias & si vero a conuersationibus hominum sint remotae , \textbf{ quasi syluestria tactum } et etiam approximationem hominum fugiunt . \\\hline
2.2.19 & e mucho mas alos nobles \textbf{ e mayoͬmente alos Reyes e alos prinçipes de auer tanto mayor cuydado çerca las sus fiias propraas } que conueniblemente anden & et potissime Reges \textbf{ et Principes | tanto maiorem curam circa proprias filias adhibere debent , } ne indebite circuant \\\hline
2.2.19 & e dela desuergonança delas sus fijas \textbf{ puede contesçer mayor mal e mayor periglo . } ssi commo es dich de suso el philosofo enl primero libro de la rectorica & quanto ex impudicitia \textbf{ et lasciuia suarum filiarum potest maius malum vel periculum imminere . } Ut superius dicebatur , Philosophus 1 Rheto’ commendat in foeminis amorem operositatis . \\\hline
2.2.20 & o cerca el gouernamiento dela casa o cerca otros vsos conuenibles \textbf{ en essa misma manera avn las mugers } por que non buian en vagar deuen amar & vel circa ea quae spectant ad gubernationem regni , \textbf{ vel circa regimen domus , vel circa aliqua alia exercitia licita . Sic et mulieres , } ne ociose viuant , debent operositatem amare : et decet eas exercitati circa aliqua opera licita et honesta . \\\hline
2.2.20 & por que non buian en vagar deuen amar \textbf{ de ser azendosassienpre en algua cosa conueinble } e conuiene les de vsar en algunas obras & vel circa regimen domus , vel circa aliqua alia exercitia licita . Sic et mulieres , \textbf{ ne ociose viuant , debent operositatem amare : et decet eas exercitati circa aliqua opera licita et honesta . } Quia igitur omnes delectantur in propriis operibus , \\\hline
2.2.20 & e non saben beuir \textbf{ si non toman delecta çonnen alg̃s obras senssibło } mas quales deuen ser las obras cerca & quanto magis a ratione deficientes nesciunt viuere \textbf{ nisi ex aliquibus exercitiis sensibilibus delectationem sumant . Qualia autem debent esse opera , } circa quae mulieres insudare decet , \\\hline
2.2.20 & e mayormente las moças \textbf{ por la mayor parte deuen estar sienpre en casa } e non se deuen entremeter de vagar nin cuydar çerca & Nam quia mulieres et maxime puellae \textbf{ ut plurimum domi stant , } et non vacant ciuilibus operibus , \\\hline
2.2.20 & o alguna buean disposiconn dela uoluntad de dentro . \textbf{ Et pues que assi es deuen auer los padres grant acuçia } e auer grant cuydado & vel aliqua bona dispositio interioris mentis . \textbf{ Debet ergo diligentia } et cura adhiberi , \\\hline
2.2.20 & Et pues que assi es deuen auer los padres grant acuçia \textbf{ e auer grant cuydado } por que las muger ssean bueans e uirtuosas por la qual cosa el philosofo enł primero libro de la rectorica denuesta los gniegos & Debet ergo diligentia \textbf{ et cura adhiberi , } ut mulieres sint bonae et virtuosae , \\\hline
2.2.20 & en qual manera las sus muger ssean uirtuosas e bueans . \textbf{ Pues que assi es deuemos auer grand cuydado } por que las mugers non biuna en uagar & quomodo earum foeminae essent virtuosae \textbf{ et bonae . Adhibenda est ergo solicitudo , } ne foeminae ociose viuant , \\\hline
2.2.20 & por que las mugers non biuna en uagar \textbf{ mas que se trabaien çerca alguas cosas conueibles e honestas } por que dende salga fructo e prouecho & ne foeminae ociose viuant , \textbf{ sed exercitent se circa aliqua opera licita , et honesta , } ut ex hoc resultet fructus , \\\hline
2.2.20 & Enpero si la muger \textbf{ que ha de ser ensiennada fuesse entn el alto grado } e que non fuesse costunbre cła tierra & satis videntur opera competentia foeminis . \textbf{ Quod si tamen foemina instruenda in tam alto gradu esset , } quod non esset dignum vel non esset consuetum \\\hline
2.2.20 & que ha de ser ensiennada fuesse entn el alto grado \textbf{ e que non fuesse costunbre cła tierra } que se trabaiasse en tales obras & Quod si tamen foemina instruenda in tam alto gradu esset , \textbf{ quod non esset dignum vel non esset consuetum | secundum morem patriae , } ut se circa talia exercitaret : \\\hline
2.2.21 & e allende \textbf{ nin les conuiene de beuir ociosas finca } que agora lo terçero mostremos & quod non decet puellas esse vagabundas , \textbf{ nec decet eas viuere otiose : } restat ut nunc tertio ostendamus , \\\hline
2.2.21 & e dessea de auer \textbf{ assi commo dize el philosofo en el ij̊ libro de la rectorica } quanto alguna cosa que se puede auer & nam cum desiderium sit eius quod abest , \textbf{ ut vult Philosophus 2 Retor’ quanto aliquid quod est possibile haberi , } magis videtur arduum et inacessibile ; tanto magis videtur abesse \\\hline
2.2.21 & por la qual cosa la muger \textbf{ que es parlera en algua manera se faze muy familiar } e muy conpanera en algua mauera se torna despreçiada & eo ipso quod est loquax , \textbf{ quodammodo se nimis familiarem exhibet , } et quodammodo se contemptibilem reddit ; \\\hline
2.2.21 & e non parleras seran mas afincadamente amadas \textbf{ e en mayor amor de sus maridos . } ¶ La segunda razon & ab eis , \textbf{ si sint debite taciturnae , feruentius diligentur . } Secunda via ad inuestigandum hoc idem , \\\hline
2.2.21 & que sienpre deuemos poner \textbf{ y mayor cautelao se acostunbro } de se leunatar mayor fallesçimiento o mayor pecado . & Dicebatur enim supra , \textbf{ ibi semper maiorem cautelam adhibendam esse } ubi consueuit maior defectus consurgere . \\\hline
2.2.21 & y mayor cautelao se acostunbro \textbf{ de se leunatar mayor fallesçimiento o mayor pecado . } Et pues que assi es commo & ibi semper maiorem cautelam adhibendam esse \textbf{ ubi consueuit maior defectus consurgere . } Cum ergo ex hoc quis loquatur prudenter , et caute , \\\hline
2.2.21 & por ende cerca las mugers \textbf{ e mayormente çera las moças deuemos auer grant cuydado } por que non fablen sin sabiduria . & circa foeminas , \textbf{ et maxime circa puellas sollicitandum est , } ne incaute loquantur . \\\hline
2.2.21 & que ninguno non diga \textbf{ palabrasi primero non la examinare con grand renssamiento ¶ } La terçera razon & ut foeminae \textbf{ etiam a puellari aetate discant cautos proferre sermones , decet eas non esse loquaces : | sed oportet ipsas esse debite taciturnitas , ut possint omnem sermonem prolatum diligenter excutere . } Tertia via ad inuestigandum hoc idem , \\\hline
2.2.21 & para prouar esto se tomadesto \textbf{ que las mugrͣ̃s non sean prestas avaraias e apeleas } ca commo las muger se mayormente las mocas fallezcan de vso de razon & Tertia via ad inuestigandum hoc idem , \textbf{ sumitur , | ne sint pronae ad iurgia et ad lites : } nam cum foeminae , \\\hline
2.2.21 & en manera que les conuiene \textbf{ e si non examinaten con grand cordura las palabras } que han de dezer & nisi sint modo debito taciturnae , \textbf{ et nisi sermones dicendos diligenter examinent : } sicut propter rationis defectum de facili loqui possunt pertinentia ad simplicitatem , et imprudentiam , \\\hline
2.2.21 & que pertenesçen a sinplicidat e a neçedat e palabras sin sabiduria \textbf{ assi de ligero pue den fablar cosas } que pertenesçen a peleas e abaraias & sicut propter rationis defectum de facili loqui possunt pertinentia ad simplicitatem , et imprudentiam , \textbf{ sic de facili loqui possunt pertinentes ad lites , } et ad iurgia . Decet ergo ipsas per debitam taciturnitatem adeo examinare dicenda , \\\hline
2.2.21 & por que fallesçe de vso de razon e de entendimiento \textbf{ e por la mayor parte biuen } mas por passion que por razon . & quod ab usu rationis deficientes \textbf{ ut plurimum plus viuunt passione quam ratione . TERTIA PARS Secundi Libri de regimine Principum : | in qua tractatur , } quo regimine a Regibus et Principibus regendi sunt ministri , \\\hline
2.3.1 & e de los sirmientes . \textbf{ Mas que conuenga de cuydar todas estas cosas al sabio padre familias } que es padre dela conpanna & et ministrorum . \textbf{ Quod autem omnia haec considerare deceat prudentem patremfamilias , } vel doctum gubernatorem familiae : \\\hline
2.3.1 & que es padre dela conpanna \textbf{ o al sabio gouernador dela conpanna . } Et que estas materias son iuntas en vno & Quod autem omnia haec considerare deceat prudentem patremfamilias , \textbf{ vel doctum gubernatorem familiae : } et quod hae materiae sint connexae , \\\hline
2.3.1 & que es senor dela casa aquien parte nesçe gouernar la casa \textbf{ conueinblemente pueda auer grant cuydado çerca aquellas cosas } que fazen para beuir bien & cuius est domum gubernare debite solicitetur \textbf{ circa ea quae faciunt ad bene viuere , } et quae requiruntur ad sufficientiam vitae : \\\hline
2.3.1 & por el philosofo \textbf{ en esse mismo logar se toma dela semeiança desta arte } que es del gounamiento dela casa alas otrasartes . & ( ut patet per Philosophum ibidem ) sumitur \textbf{ ex similitudine huius artis , } quae est de gubernatione domus ad artes alias . \\\hline
2.3.1 & por los quales acaban sus obras . \textbf{ En essa misma manera el arte del gouernamiento dela casa demanda sus estrumentos } por los quales pueda conplir sus obras . & ut ars fabrilis , \textbf{ et textoria , habent sua organa , per quae perficiunt actiones suas : sic et gubernatio domus requirit sua organa , } per quae opera sua complere possit . \\\hline
2.3.1 & e de los otros estrumentos del ferrero . \textbf{ Et al ferero pertenesçe de cognosçertales estrumentos . } Et dessa misma manera el que quiere dar conosçimiento del arte del texer & et aliis instrumentis fabrilibus ; \textbf{ et spectat ad fabrum talia instrumenta cognoscere . } Sic volens tradere notitiam de arte textoria , \\\hline
2.3.1 & Et al ferero pertenesçe de cognosçertales estrumentos . \textbf{ Et dessa misma manera el que quiere dar conosçimiento del arte del texer } deue determinar de los peinnes e de los otros estrumentos de aquella arte & et spectat ad fabrum talia instrumenta cognoscere . \textbf{ Sic volens tradere notitiam de arte textoria , } debet determinare de pectinibus , \\\hline
2.3.1 & assi conmo \textbf{ por prop̃os estrumentos pueda alcançar estas cosas } que fazen al abastamiento dela uida . & quia talia sunt organa huius artis . Spectat ergo ad gubernationem domus talia cognoscere : \textbf{ quia per haec tanquam propria organa consequi poterit , } quae faciunt ad sufficientiam vitae : \\\hline
2.3.1 & que fazen al abastamiento dela uida . \textbf{ Ca el arte del gouernamiento dela casaha prop̃os estrumentos } assi commo las otrasartes mechancas han sus propreos estrumentos & quae faciunt ad sufficientiam vitae : \textbf{ habet enim ars gubernationis domus propria organa , | sicut } et caeterae artes moechanicae habent propria instrumenta . \\\hline
2.3.1 & mas los estrumentos delas artes mecanicas son factiuos e obradores \textbf{ por que por ellos finca algua cosa fecha de fuera } assi commo el arca . & secundum Philosophum primo Poli’ \textbf{ quia organa gubernationis sunt actiua , organa vero moechanicorum sunt factiua . } Nam in moechanicis regulamur per artem , \\\hline
2.3.1 & mas el gouernamiento dela casa la regla es sabiduria \textbf{ que es derecha razon de todas las cosas } que ha de fazer . & Nam in moechanicis regulamur per artem , \textbf{ sed in gubernatione domus regula est prudentia . } Sicut ergo ars differt a prudentia , sic organa moechanicorum differunt ab organis gubernationis . \\\hline
2.3.1 & ¶avn ay departimiento entre esta arte e la otra \textbf{ por que el arte mecanica es derecha razon delas cosas } que ha de fazer de fuera . & ø \\\hline
2.3.1 & Et por tal arte sale algcosa fechͣ en la materia de fuera . \textbf{ Mas la sabiduria es derecha razon delas cosas } que son de fazer & quia ars est recta ratio factibilium et per artem resultat aliquid factum in materia extra : \textbf{ sed prudentia est recta ratio agibilium , } et per eam non proprie resultat aliquid factum extra : \\\hline
2.3.1 & Et pues que assi es el arte del gouernamiento dela casa maguera . \textbf{ largamente fablando puede ser dichͣ arte } mas propreamente fablando deue ser dicha sabiduria . & quae circa agibilia habet esse . \textbf{ Ars ergo gubernationionis domus licet largo modo possit dici ars , } proprie tamen prudentia dici debet . Praemissis \\\hline
2.3.1 & que fazen a gouernamiento e a conplimiento dela uida dela uida dela casa . \textbf{ Mas por que aquellas mismas razones sen podia prouar } que parte nesçe al gouernamiento dela casa & et ad sufficientiam vitae . \textbf{ Per illas autem easdem rationes probari posset , } quod spectat ad gubernatorem domus scire debite se habere circa ministros \\\hline
2.3.1 & e çerca los sieruos \textbf{ por que los siruientes e los sieruos son tals comm̃estrumentos } que siruen al gouernamiento dela casa . & et seruos : \textbf{ nam ministri | et serui sunt } etiam quaedam organa deseruientia gubernationi domus , \\\hline
2.3.2 & e dize \textbf{ assi que algs estrumentos son anumados } assi commo los sieruos . & Philosophus primo Politicorum omnia organa gubernationis domus ad bimembrem distinctionem reducit . \textbf{ Ait enim quaedam esse animata , } ut serui : \\\hline
2.3.2 & segunt el ordenamiento del mundo \textbf{ que las cosas muy altas sra ningun medio mueuna e manden alas cosas muy baxas } mas ley del mundo conueinble es & secundum ordinem uniuersi , \textbf{ ut suprema immediate administrent infima : } sed lex Uniuersi est , \\\hline
2.3.2 & o que sean porteroso \textbf{ que vsen de o tristales cosas . } mas cosaco nueible es & ø \\\hline
2.3.3 & ¶ La primera radzon paresçe assi \textbf{ ca segunt el philosofo en el quarto libro delas ethicas enł capitulo dela magnifiçençia } que much mas conuiene alos Reyes & nam \textbf{ secundum Philosophum 4 Ethicorum capitulo de Magnificentia , maxime gloriosos } et nobiles decet esse magnificos : \\\hline
2.3.3 & ca si los otros nobles \textbf{ que han possessio nes tenp̃das pueden ser libales e frquecos } por que la libalidat puede ser avn en pequanas despenssas muchmas los Reyes e los prinçipes & potissime decet esse magnificos . \textbf{ Alii enim moderatas possessiones habentes , | possunt esse liberales , } eo quod liberalitas in paruis sumptibus esse possit : \\\hline
2.3.3 & que han possessio nes tenp̃das pueden ser libales e frquecos \textbf{ por que la libalidat puede ser avn en pequanas despenssas muchmas los Reyes e los prinçipes } que han muy grandes possessiones deuen ser magnificos & possunt esse liberales , \textbf{ eo quod liberalitas in paruis sumptibus esse possit : | sed Reges et Principes multitudine possessionum pollentes , } debent esse magnifici , \\\hline
2.3.3 & por que la libalidat puede ser avn en pequanas despenssas muchmas los Reyes e los prinçipes \textbf{ que han muy grandes possessiones deuen ser magnificos } e fazedores de grandes cosas & sed Reges et Principes multitudine possessionum pollentes , \textbf{ debent esse magnifici , } eo quod magnificentia circa magnos sumptus esse contingat . \\\hline
2.3.3 & que han muy grandes possessiones deuen ser magnificos \textbf{ e fazedores de grandes cosas } por quela magnifiçençia ha de ser çerca grandes despenssas & sed Reges et Principes multitudine possessionum pollentes , \textbf{ debent esse magnifici , } eo quod magnificentia circa magnos sumptus esse contingat . \\\hline
2.3.3 & e fazedores de grandes cosas \textbf{ por quela magnifiçençia ha de ser çerca grandes despenssas } mas assi commo es dicho & debent esse magnifici , \textbf{ eo quod magnificentia circa magnos sumptus esse contingat . } Sed magnifici , \\\hline
2.3.3 & en esse mismo quarto libro delas ethicas \textbf{ al magnifico parte nesçe de apareiar morada conuenible } mas non es conuenible morada al magnifico & Sed magnifici , \textbf{ ut dicitur in eodem 4 Ethicor’ est praepare habitationem decentem : } non est autem decens habitatio , \\\hline
2.3.3 & al magnifico parte nesçe de apareiar morada conuenible \textbf{ mas non es conuenible morada al magnifico } si non es fecha & ut dicitur in eodem 4 Ethicor’ est praepare habitationem decentem : \textbf{ non est autem decens habitatio , } nisi magnifico opere , et mirabili industria sit constructa . Ex parte ergo ipsius magnificentiae regiae patet , quod Reges et Principes , quantum ad industriam operis , \\\hline
2.3.3 & si non es fecha \textbf{ por muy marauillosa maestria . Et pues que assi es paresçe } por essa misma magnificençia real & non est autem decens habitatio , \textbf{ nisi magnifico opere , et mirabili industria sit constructa . Ex parte ergo ipsius magnificentiae regiae patet , quod Reges et Principes , quantum ad industriam operis , } decet habere habitationes mirabiles . \\\hline
2.3.3 & por muy marauillosa maestria . Et pues que assi es paresçe \textbf{ por essa misma magnificençia real } que alos Reyes e alos prinçipes & non est autem decens habitatio , \textbf{ nisi magnifico opere , et mirabili industria sit constructa . Ex parte ergo ipsius magnificentiae regiae patet , quod Reges et Principes , quantum ad industriam operis , } decet habere habitationes mirabiles . \\\hline
2.3.3 & para prouar esto mismo se toma de parte del pueblo \textbf{ e esta razon tanne el philosofo enł vi̊ libro delas politicas } do dize & sumitur ex parte ipsius populi : \textbf{ et hanc tangit Philosophus 6 Politicorum , } ubi ait , \\\hline
2.3.3 & do dize \textbf{ que alos Reyes e alos prinçipes parte nesçe de fazer tan grandes cosas } e de estroyr e de fazer tales moradas & ubi ait , \textbf{ quod Principes decet sic magnifica facere , } et talia aedificia construere , \\\hline
2.3.3 & assi commo espantado \textbf{ por la grand marauilla } que veran & quod populus ea videns , \textbf{ quasi sit mente suspensus propter vehementem admirationem . } Nam populus minus insurgit contra principem , \\\hline
2.3.3 & maguer non se de una fazer aparesçençia \textbf{ nin a grandezaauana eglesia } enpero conuiene alos Reyes & et tam magnificum . Magnitudo enim aedificiorum \textbf{ licet non sit fienda ad ostentationem et inanem gloriam : } decet tamen Reges et Principes , \\\hline
2.3.3 & que non corre \textbf{ por que las aguas estantias en la mayor parte se fazen gruessas } e se podresçen en essa misma manera el ayre & quod sicut aqua currens sanior est quam stans , \textbf{ eo quod aquae stantes | ut plurimum ingrossantur } et putrescunt : \\\hline
2.3.3 & por que las aguas estantias en la mayor parte se fazen gruessas \textbf{ e se podresçen en essa misma manera el ayre } que es ençerrado en los valls & ut plurimum ingrossantur \textbf{ et putrescunt : } sic aer reclusus in vallibus , \\\hline
2.3.3 & de los que moran en ella \textbf{ si han sano color e fermoso } e si son bien sanos delas cabeças & si eis \textbf{ sit color sanus et pulcher , } si sit ipsis firma sinceritas capitis , \\\hline
2.3.4 & por que el agua segunt el philosofo es muy comun a todos \textbf{ e en much ͣs cosas sirue ala neçessidat dela iuida } e por ende mucho es de cuydar & Aqua enim ( secundum Philosophum ) est valde communis , \textbf{ et in multis deseruit ad necessaria vitae . Ideo valde considerandum est , } ut sic aedificium situetur , \\\hline
2.3.4 & nin por corronpemiento della non ayan de caer en enfermedades \textbf{ e por ende tanne paladio en el libro de la agnicultura seys cosas } que dize & ne habitatores eius \textbf{ ob infectionem aquae infirmitatem contrahant . Tangit autem Palladius in libro De agricultura sex } quae ait esse consideranda in cognitione aquae salubritatis . Primum est : \\\hline
2.3.4 & por que en ellas es el agua estantiaqua non corre \textbf{ e por ende por la mayor parte non han el agua sana¶ } Lo segundo deuemos cuydar & eo quod eis sit aqua quodammodo stans , \textbf{ ut plurimum habent aquam non salubrem . } Secundo considerandum est , \\\hline
2.3.4 & e passan \textbf{ por las venas sola tierra } por la qual cosa si en logar del engendramiento delas aguas o las carreras & et aqua in locis subterraneis generatur , \textbf{ et per venas subterraneas transit ; } quare si locus generationis aquarum , \\\hline
2.3.4 & do non son los metalles puros \textbf{ assi commo pax ̉esçe eñł agua e passa por la piedra sutre } e algunas uezes con los metalles son ayuntadas alguas podredunbres & ø \\\hline
2.3.4 & lo quarto es que el agua non sea corronpida \textbf{ por mal olor nin por mal sabor } porque el mal olor o el mal sabor delas aguas demuestran en la mayor parte & aquae infectionem demonstrat . Quartum est , \textbf{ ne aliquo odore , vel sapore vitientur } nam prauus odor , \\\hline
2.3.4 & por mal olor nin por mal sabor \textbf{ porque el mal olor o el mal sabor delas aguas demuestran en la mayor parte } que aquellas aguaas son engendradas en logares corruptoso que passan por algunos logares non sanos & ne aliquo odore , vel sapore vitientur \textbf{ nam prauus odor , | vel sapor aquarum , } ut plurimum designat illas aquas generari , vel transire per aliqua loca infecta , \\\hline
2.3.4 & que aya abondamiento de agua sana \textbf{ e que aya aquellas seys condiconnes de que ya dixiemos . } Enpero si tanta fuere la neçesidat & quod sit ei copia aquae salubris habentis conditiones illas , \textbf{ de quibus fecimus mentionem . } Quod si tamen aedificandi necessitas urgeat , \\\hline
2.3.4 & mas en estas çisternas o en estos algibes deuemos poner peçes de rio \textbf{ por que por el nadamiento de los peces el agua estante semeie en lignieza } al agua que corre ¶ Visto en qual manera son de fazer las moradas & secundum eundem ) aqua caelestis et pluuialis ad bibendum \textbf{ quasi omnibus amefertur , sunt autem in cisterna illa pisces fluuiales apponendi , } ut horum natatu aqua stans agilitatem currentis imitetur . Viso , qualiter est aedificium construendum quantum ad salubritatem aquae : \\\hline
2.3.4 & La segunda que en el estiuo non sean apremiadas \textbf{ por grant calentura¶ } Lo primero puede contesçer & et dispositio terrarum . Quantum ad conditionem caelestem duo sunt attendenda . Primo ut hyeme debita claritate illustretur . Secundo , \textbf{ ne in aestate immoderato calore opprimatur , } quod fieri contingit , si aedificium \\\hline
2.3.4 & en que los omes muy ligeramente enferman son de fazer alguas camaras contrarias al uiento set enteronal e del çierço \textbf{ por que sea enllas guardada la uida con mayor salud . } ¶ Lo terçero & in quo homines facilius infirmantur , aedificandae sunt aliquae camerae oppositae vento septentrionali , \textbf{ ut in eis salubrior custodiatur vita . Tertio } quantum ad ordinem uniuersi \\\hline
2.3.4 & e de los muradales e del estiercol . \textbf{ ¶ Avn en essa misma manera se pueden departir o triscondiconnes particulares delas moradas . } mas por que tales cosas son much̉ particular & et fluminibus : et longe a stabulis , fimo , \textbf{ et sterquiliniis . Sic etiam aliae particulares conditiones aedificiorum distingui possent . } Sed \\\hline
2.3.4 & ¶ Avn en essa misma manera se pueden departir o triscondiconnes particulares delas moradas . \textbf{ mas por que tales cosas son much̉ particular } es esto finque en la sotileza e en el engennio de los fazedores delas moradas & et sterquiliniis . Sic etiam aliae particulares conditiones aedificiorum distingui possent . \textbf{ Sed } quia talia nimis particularia sunt , aedificatorum industriae relinquantur . \\\hline
2.3.5 & maspodemos prouar \textbf{ por tres razons la possession delas cosas es natural en algua manera al omne } ¶La primera se toma dela neçessidat dela uida . & restat dicere de possessionibus . Possumus autem triplici via venari , \textbf{ quod rerum possessio est quodammodo naturalis . } Prima sumitur ex necessitate vitae . Secunda , \\\hline
2.3.5 & por ende han señorio natural sobrellas . \textbf{ por la qual cosa natural cosa es al ome } que enssennore & habet naturale dominium super ipsa : \textbf{ quare naturale est homini } quod dominetur istis sensibilibus , \\\hline
2.3.5 & e si ellas fuyen del omne seruiçio del omne el omne batalla contra ellas \textbf{ assi commo contra aquellas de quena tra ᷑almente deue ser sennor . } Et pues que assi es & homo contra ipsas bellat , \textbf{ tanquam contra ea quibus naturaliter dominabitur . } Sicut ergo homo naturaliter dominatur bestiis , \\\hline
2.3.5 & assi commo el omne naturalmente enssennorea alas bestias \textbf{ avn en essa misma manera naturalmente enssennorea a todas las otras cosas de fuera . } la qual cosa non podrie ser & Sicut ergo homo naturaliter dominatur bestiis , \textbf{ sic et naturaliter dominatur aliis exterioribus rebus ; } quod non esset , \\\hline
2.3.5 & mas naturalmente apareia el nudmiento conuenible a ellas \textbf{ mas conueible cosa es } que les non fallesca & nam si in prima generatione natura non deficit animalibus , \textbf{ sed naturaliter praeparat eis debitum nutrimentum ; congruum est } ut non deficiat eis iam perfectis . \\\hline
2.3.5 & que estas cosas de que tomamosnudmiento son dadas a nos por nata . \textbf{ Et pues que assi es natal cosa es a nos de auer las cosas de fuera } e por ende el sennorio delas cosas de fuera es en algua manera natural al omne . & datae sunt nobis a natura . \textbf{ Naturale est ergo nobis habere res exteriores . } Habere ergo dominium rerum exteriorum est \\\hline
2.3.5 & Et pues que assi es natal cosa es a nos de auer las cosas de fuera \textbf{ e por ende el sennorio delas cosas de fuera es en algua manera natural al omne . } por que la natan engendro & Naturale est ergo nobis habere res exteriores . \textbf{ Habere ergo dominium rerum exteriorum est | quodammodo homini naturale : } quia natura produxit huiusmodi sensibilia propter hominem . Sumus enim quodammodo nos finis omnium , \\\hline
2.3.5 & mas escogen para si uida sobre omen \textbf{ e son dichos meiores que omes . Et pues que assi esnatal cosa es al ome } en quanto es omne & sed eligunt sibi vitam supra hominem , \textbf{ et dicendi sunt hominibus meliores . Naturale est homini ergo , } ut homo est , \\\hline
2.3.5 & segunt dize el philosofo enl primero libro delas politicas de auer possession \textbf{ e sennorio de algers cosas de fuera } para conplimiento dela uida . & ø \\\hline
2.3.5 & e que resçebiesse ende nudermiento conuenible \textbf{ sin el qual non puede durar lanr̃a uida ¶ } ne opinion de socrates e de platon & et ut susciperet inde debitum nutrimentum , \textbf{ sine quo vita nostra durare non potest . } Fuit opinio Socratis et Platonis , \\\hline
2.3.6 & por su uoluntad \textbf{ que por esto serie grand ayuntamiento } e grand amorio en la çibdat & sed quilibet ad quamlibet pro sua voluptate accederet , \textbf{ esset suprema unitas , } et suprema dilectio in ciuitate . \\\hline
2.3.6 & que por esto serie grand ayuntamiento \textbf{ e grand amorio en la çibdat } por que estonçe todos los omes a marien a todas las mugers & esset suprema unitas , \textbf{ et suprema dilectio in ciuitate . } Tunc enim omnes viri diligerent omnes foeminas tanquam proprias , \\\hline
2.3.6 & por que estonçe todos los omes a marien a todas las mugers \textbf{ assi commo sus prop̃as mugers . } Et avn en essa misma manera todos los omes amarien a todos los moços & Tunc enim omnes viri diligerent omnes foeminas tanquam proprias , \textbf{ sic } etiam omnes homines diligerent omnes pueros tanquam filios proprios , \\\hline
2.3.6 & assi commo sus prop̃as mugers . \textbf{ Et avn en essa misma manera todos los omes amarien a todos los moços } assi commo a sus fijos propraos & sic \textbf{ etiam omnes homines diligerent omnes pueros tanquam filios proprios , } eo quod nesciret pater quis puer filius suus esset , \\\hline
2.3.6 & en quento esto faze al gouernamiento dela casaca \textbf{ si los omes en la mayor parte non ouiessen los desseos corronpidos } e en la mayor parte los omes non fuessen inclinados amal . & prout facit ad regimen et gubernationem domus . \textbf{ Si enim homines | ut plurimum non haberent appetitum corruptum , } et ut in pluribus non \\\hline
2.3.6 & si los omes en la mayor parte non ouiessen los desseos corronpidos \textbf{ e en la mayor parte los omes non fuessen inclinados amal . } estonçe conuernie ala çibdat & ut plurimum non haberent appetitum corruptum , \textbf{ et ut in pluribus non } essent proui ad malum , expediret ciuitati possessiones communes esse , eo quod bonum quanto communius , \\\hline
2.3.6 & por que el bien \textbf{ quanto mas comun estato es mas digno e meior } ca todo bien aducho en comun & essent proui ad malum , expediret ciuitati possessiones communes esse , eo quod bonum quanto communius , \textbf{ tanto diuinius : } omne enim bonum in commune reductum \\\hline
2.3.6 & assi ca el ome entanto ama asi mismo \textbf{ que sienpre ha mayor cuydado del su bien propreo } que del bien ageno caueemos & nam homo adeo diligit seipsum , \textbf{ quod semper magis solicitus est de bono proprio quam de alio . } Videmus enim quod sic est facile possessiones consumere \\\hline
2.3.6 & e se desgastarien \textbf{ e en tanto seria guaue cosa de trabaiar cada vno } por su ganançia & et deuastare , \textbf{ et adeo est difficile laborare } et proprio lucro substantiam \\\hline
2.3.6 & que todas las cosas son comunes en la çibdat \textbf{ que ninguon non quarrie trabaiar por ellas ca agora en la çibdat lon muchs pobres } avn que non contradigamos & et possessiones acquirere , \textbf{ quod in ciuitate contingit multos egere et esse pauperes , } non obstante quod ciues possunt gaudere possessionibus propriis , \\\hline
2.3.6 & assi commo han cuydado de labrar las propreas \textbf{ por ende contesçeria en la mayor parte } que aquella çibdat & sicut sunt soliciti ad proprias , \textbf{ ut plurimum contingeret ciuitatem } illam sic ordinatam venire ad inopiam , \\\hline
2.3.6 & que aquella çibdat \textbf{ assi orde nada uerme a grant pobreza } por que los çibdadanos non podrien abondar & ut plurimum contingeret ciuitatem \textbf{ illam sic ordinatam venire ad inopiam , } ut ciues non possent \\\hline
2.3.6 & assi commo dicho es \textbf{ pro prouechosa cosa es ala çibdat } que los çibdadanos ayan possessionspropreas & In rebus ergo sic se habentibus , \textbf{ utile est ciuitati ciues habere possessiones proprias , } ne propter ignauiam circa communia , domus ciuium patiantur inopiam . \\\hline
2.3.6 & por tirar la contienda de entre los omnes \textbf{ ca por la mayor parte se le una tan contiendas e uaraias } entre los que han cosas comunes algunas . & Secunda via ad inuestigandum hoc idem , sumitur ex remotione litigii : \textbf{ ut plurimum enim consurgunt lites } et bella inter participantes aliquid commune : videmus enim ipsos fratres \\\hline
2.3.6 & entre los que han cosas comunes algunas . \textbf{ caueemos que los hͣrmaros fijos de vn padre entre los quales seg̃tel philosofo enł . } viij̊ libro delas ethicas es amistança natural & et bella inter participantes aliquid commune : videmus enim ipsos fratres \textbf{ ex eodem patre natos , } inter quos secundum Philosophum 8 Ethicorum est amicitia naturalis , \\\hline
2.3.6 & viij̊ libro delas ethicas es amistança natural \textbf{ en la mayor parte veemos los contender } e uaraiar sobre la heredat & inter quos secundum Philosophum 8 Ethicorum est amicitia naturalis , \textbf{ ut plurimum bellare ad inuicem , } eo quod sit eis communis haereditas : \\\hline
2.3.6 & Et pues que assi es mugua mas serie discordia entre los çibdadanos \textbf{ entre los quales non es tan grand amistaça } commo entre los hmanos & inter ipsos ciues , \textbf{ inter quos non est tanta amicitia , } si essent eis possessiones communes . \\\hline
2.3.6 & commo contesçe \textbf{ quando alguna cosa es mandada a muchs siruientes que la fagan . } Ca quando esto se manda & sicut accidit , \textbf{ cum aliquid committitur pluribus ministris : } cum enim hoc fit , \\\hline
2.3.7 & que fazen fructo sean labrados \textbf{ el pho enł primero libro delas politicas muestra } que por departidos usos de usar delas cosas & et alia fructifera excolantur . \textbf{ Philosophus primo Politicorum ostendit , } quod ex alio et alio usu exteriorum rerum , \\\hline
2.3.7 & e en las bestias \textbf{ que non usan todas de essas mismas cosas } nin han todas vna manera de beuir & Sicut enim videmus in animalibus et bestiis , \textbf{ quod non omnes utuntur eisdem rebus , } nec omnes habent eundem modum viuendi ; \\\hline
2.3.7 & Et avn los omes vsan en deꝑtidas maneras delas cosas de fuera \textbf{ e non han vna manera de beuir . Et el philosofo enł primero libro delas politicas departe quatro vidas sinples } o quatro maneras de beuir & Homines etiam diuersimode utuntur exterioribus rebus , \textbf{ et non eodem modo viuendi viuunt . | Distinguit enim Philosophus primo Politicorum quatuor vitas simplices , } vel quatuor modos viuendi , \\\hline
2.3.7 & e non han vna manera de beuir . Et el philosofo enł primero libro delas politicas departe quatro vidas sinples \textbf{ o quatro maneras de beuir } delas quales ayuntadas se le una tan departidas maneras de beuir & Distinguit enim Philosophus primo Politicorum quatuor vitas simplices , \textbf{ vel quatuor modos viuendi , } ex quibus conbinatis resultant alii modi viuendi , \\\hline
2.3.7 & delas quales ayuntadas se le una tan departidas maneras de beuir \textbf{ e departidas uidas . Ca la uida es en quatro maneras } conuiene sabra ¶Vida de pasto . & ex quibus conbinatis resultant alii modi viuendi , \textbf{ vel aliae vitae . Est enim vita quadruplex , } videlicet pascualis , venatiua , \\\hline
2.3.7 & ca sienpre las cosas que non son acabadas son ordenadas alas mas acabadas \textbf{ assi commo el agua e ła terrra } que son cosas sin alma son ordenadas al nudermiento de los arboles & semper enim imperfecta ordinantur ad perfectiora ; \textbf{ ut aqua , et terra , } quae sunt inanimata , \\\hline
2.3.7 & Mas los arboles e las plantas \textbf{ assi comm̃o cosas que non sienten son ordenadas al nu dermiento delas aianlias } que sienten ¶ & et plantae , tanquam insensibilia , \textbf{ ordinantur ad nutrimentum animalium sensibilium : } omnia autem haec tam inanimata quam vegetabilia , \\\hline
2.3.7 & que la natura dio anos tales cosas \textbf{ e las ordeno a vso e añro sennorio . } ¶ Et pues que assi es cosa conuenible & quod scribitur 1 Polit’ \textbf{ quod natura dedit nobis talia , ordinauit enim ea ad usum } et dominium nostrum ; licitum est ergo sumere nutrimentum ex agris , et animalibus domesticis quare vita pascualis est licita . \\\hline
2.3.7 & et por ende la uida de los pastos es conuenible a nos . \textbf{ avn en essa misma manera la uida de caçar } e de pescar des si non son desconueibles & et dominium nostrum ; licitum est ergo sumere nutrimentum ex agris , et animalibus domesticis quare vita pascualis est licita . \textbf{ Sic etiam venatiua } et piscatiua \\\hline
2.3.7 & Et el omne ha contra tales aina lias batalla derecha \textbf{ por la qual cosa el philosofo enl ꝑ̀mo libro delas politicas dize } que la uida de caçar & sed etiam bestiis syluestribus , et piscibus ; \textbf{ habet contra talia iustum bellum propter quod Philosophus 1 Politic’ vult venatiuam } et piscatiuam esse vitas licitas . \\\hline
2.3.7 & e la uida de pescar son uidas conuenibles \textbf{ por que dize que es derecha batalla de los omes alas bestias } e alas o trisaian lias & et piscatiuam esse vitas licitas . \textbf{ Ait enim , | quod hominum ad bestias } et ad alia animalia est iustum bellum ; \\\hline
2.3.7 & esto es por algun açidente \textbf{ en quanto aquellos en algua manera fazen o fizieron algua cosa desagnisada contra ellos . } ¶ Et pues que assi es el ome peca en faziendo mal al omne & hoc quasi per accidens , \textbf{ inquantum illi aliquo modo forefaciunt | vel forefecerunt in ipsos . } Delinquit ergo homo offendendo hominem : \\\hline
2.3.7 & por sabiduria e por entendimiento han batalla derecha contra los rusticos \textbf{ e contralos aldeanos } si refusaz̧en de ser subiectos aellos & qui magis uigent prudentia et intellectu , \textbf{ iustum habent bellum contra rusticos , } si recusent subiici illis . \\\hline
2.3.7 & los quales son dignos de ser subiectos \textbf{ segunt la sentençia del philosofo es derecha batalla contra ellos . } Et esta misma manera las cosas assi entendidas paresçe & quos dignum est esse subiectos , \textbf{ secundum sententiam Philosophi sic intellectam , uidetur esse licita praedatiua uita ; } ut quod licitum esset non solum hos depraedari \\\hline
2.3.7 & segunt la sentençia del philosofo es derecha batalla contra ellos . \textbf{ Et esta misma manera las cosas assi entendidas paresçe } que la uida del robar es conuenible & quos dignum est esse subiectos , \textbf{ secundum sententiam Philosophi sic intellectam , uidetur esse licita praedatiua uita ; } ut quod licitum esset non solum hos depraedari \\\hline
2.3.7 & e quantas son las maneras de beuir \textbf{ e quales de aquellass maneras son conuenibles } e quales desconuenibles & uel quot sunt modi uiuendi , \textbf{ et qui illorum sunt liciti , } et qui illiciti : \\\hline
2.3.8 & o enformaçion o estimaçion dela fin \textbf{ ¶La primera se partueua } assi ca el phodizer que la cobdiçia delas riquezas es sin fin e sin mesura & Secunda ex falsa aestimatione finis . \textbf{ Prima via sic patet . | nam ( } ut ipse ait ) \\\hline
2.3.8 & e non sin mesura \textbf{ por que bien beuires beuir uirtuosa mente . } ¶ Et pues que assi es los omes & quicunque autem propter ipsum bene viuere diuitias volunt , fruitiones corporales non infinitas quaerunt : bene enim viuere , \textbf{ est viuere virtuose . } Homines ergo quia non curant viuere virtuose , \\\hline
2.3.8 & La segunda razon \textbf{ para prouar esto mismo se toma dela falsa estimaçion } e informaçion dela fin . & quod volunt , \textbf{ diuitiis } et possessionibus non satiantur . \\\hline
2.3.8 & quela fin del fisico es sanar \textbf{ e en esta fin nunca tiene el fisico mesuta } ca nunca podrie el fisico tanta salud aduzir al enfermo & ø \\\hline
2.3.8 & Et pues que assi es los omes comunalmente \textbf{ por que han falsa estimaçion dela fin } e por que cuydan & et mensuram sanitatis . \textbf{ Communiter ergo homines quia falsam aestimationem habent de fine , } et putant ipsum finem in diuitiis esse ponendum , \\\hline
2.3.8 & que ha la yconomica con las o trisartes ¶ \textbf{ La prima razon se declara assi . Ca ueemos que lan atraa non es cuydados } a çerca el nudermiento sin mesura & quam habet ad artes alias . \textbf{ Prima via sic patet . } Videmus enim naturam non solicitari circa nutrimentum infinitum , \\\hline
2.3.8 & assi commo paresçe \textbf{ por lo que dicho es estod uiene de falsa estimaçionde } la fin o de desordenamiento dela uoluntad¶ & et diuitiis \textbf{ ( ut patet per habita ) uel procedit | ex falsa aestimatione finis , } uel ex inordinatione uoluptatis . \\\hline
2.3.8 & que en los otros \textbf{ quanto mas conuiene aellos de auer mayor ordenamiento dela uoluntad } e meior estimacion dela finca & quam in aliis , \textbf{ quanto decet habere ordinatiorem uoluptatem , } et meliorem aestimationem finis : \\\hline
2.3.8 & quanto mas conuiene aellos de auer mayor ordenamiento dela uoluntad \textbf{ e meior estimacion dela finca } assi commo dixiemos enł primero libro & quanto decet habere ordinatiorem uoluptatem , \textbf{ et meliorem aestimationem finis : } nam sicut in primo libro dicebatur , \\\hline
2.3.8 & assi commo dixiemos enł primero libro \textbf{ mas de denostares enl Rey de non auer uerdadera estimaçonn dela fin } que enl pueblo & nam sicut in primo libro dicebatur , \textbf{ detestabilius est in Rege non habere ueram aestimationem de fine quam in populo , } eo quod populus a Rege dirigitur , \\\hline
2.3.9 & todas las muda connes son aduchos a tres linages \textbf{ de los quales la vna es muda connde alguas cosas } a otras cosas & quasi ad tria genera reducuntur . \textbf{ Quarum una est commutatio rerum ad res : } ut frumenti ad uinum , \\\hline
2.3.9 & que es comuidat dela casa es cosa prouada \textbf{ que non es menest obra de muda conn ninguna } Et por ende por las otras comuidades fue puesta la mudaçion delas cosas . & ø \\\hline
2.3.9 & Et pues que assi es \textbf{ por eltas comiundades sobredichͣs fueron puestas aquellas tres maneras de mudaçonnes } ca si ala comuidat de vn barrio o de vna çibdatur llgua manera abastasse la mudaçion delas cosas alas cosas enpero ala comunidat & etiam diuersarum prouinciarum , et regnorum ; \textbf{ propter has ergo communitates introductae sunt illae tres species commutationum . | Nam et si ad communitatem vici , } vel ciuitatis , \\\hline
2.3.9 & e de departidas \textbf{ prouinçias conuiene de poner non sola mente mudaçion delas cosas alas cosas } o delas cosas alos dineros & in commutatione tamen , quae est diuersorum regnorum , et prouinciarum , \textbf{ oportuit introduci non solum commutationem rerum ad res , vel rerum ad denarios ; } sed \\\hline
2.3.9 & ca en el tienpo antiguo los oens assi commo da a entender el philosofo \textbf{ en el prim̃o libro delas politicas } buuendo en su sinpliçidat muda una vnas cosas en otras cosas & etiam denariorum ad denarios . Antiquitus enim homines \textbf{ ( ut satis innuit Philosophus primo Politicorum ) } in simplicitate viuentes \\\hline
2.3.9 & assi commo dize el philosofo \textbf{ avn es guarda entre muchͣs naçiones barbaras e estrannas } que non han uso de dineros & ( ut ipse ait ) \textbf{ adhuc reseruatur | apud multas Barbaras nationes , } quae non habentes denariorum usum , \\\hline
2.3.9 & en todo vn regno o en toda vna prouinçia \textbf{ si aquella prouinçia fuesse de muy grant anchura } ca el vino e el trigo . & sed in toto uno regno , \textbf{ vel in tota una prouincia , si prouincia illa esset magnae latitudinis , | commode obseruari non potest . } Nam vinum , frumentum , \\\hline
2.3.9 & que auemos menester para la uida \textbf{ por que son de grand peso } non las poderemos leuar & et talia quibus indigemus ad vitam , \textbf{ cum sint magni ponderis , | commode } ad partes longinquas portari non possunt . \\\hline
2.3.9 & non las poderemos leuar \textbf{ conueniblemente a luengas tierras . } Et pues que assi es conuiene de fablar alguna cosa & commode \textbf{ ad partes longinquas portari non possunt . } Oportuit ergo inuenire aliquid quod esset portabile , \\\hline
2.3.9 & e en toda uendida pesar \textbf{ si enfͤlos metales } por que los uendedores e los conpradores fuessen tirados de tal trabaio & Sed quia difficile erat in omni emptione vel venditione , \textbf{ semper ponderare metalla , } ut ementes et vendentes \\\hline
2.3.9 & por ende conniene \textbf{ que los que mora una en apartadas prouinçias ouiessen de fallar } sin la mudaçion delas cosas alas cosas . & praeter commutationem rerum ad res , \textbf{ et rerum ad numismata , } oportuit inuenire commutationem numismatum ad numismata . \\\hline
2.3.9 & et qual fue la neçessidat \textbf{ para fallar los des e por ende conuiene al sabio padre familias } e al sabio gouernador dela casa & et quae fuit necessitas inuenire denarios . \textbf{ Decet ergo prudentem patremfamilias , } et doctum gubernatorem cognoscere , \\\hline
2.3.9 & para fallar los des e por ende conuiene al sabio padre familias \textbf{ e al sabio gouernador dela casa } saber & Decet ergo prudentem patremfamilias , \textbf{ et doctum gubernatorem cognoscere , } quomodo ortae sunt tales commutationes , \\\hline
2.3.10 & Et el philosofo en las politicas \textbf{ pone quatro maneras de dineros conuiene saber . Natural . } Et canssoria de canbio . & numismata et pecuniam , restat dicere , quot sunt species pecuniatiuae . \textbf{ Distinguit autem Philos’ in Poli’ quatuor species pecuniatiuae : } videlicet naturalem , \\\hline
2.3.10 & por razon que ha comienco delas cosas naturales . \textbf{ ¶ la segunda manera de los dineros es dichͣ camiadora . } Et esta segunt el philosofo enel primero libro delas politicas & quasi naturalis diceretur , \textbf{ quia a rebus naturalibus inciperet . Secunda species pecuniatiuae dicitur esse campsoria : haec enim } ( \\\hline
2.3.10 & mas uale en su regno que en otro \textbf{ Et por ende alguas uezes contesçe } que algunos por auentura han alguas monedas & plus valet in propria regione . \textbf{ Accidit ergo forte aliquos habere aliqua numismata , } quae non multum appretiabantur in regione sua , \\\hline
2.3.10 & Et por ende alguas uezes contesçe \textbf{ que algunos por auentura han alguas monedas } que non son muy preçiadas en su regno & plus valet in propria regione . \textbf{ Accidit ergo forte aliquos habere aliqua numismata , } quae non multum appretiabantur in regione sua , \\\hline
2.3.10 & e en quales partes se espienden \textbf{ por que despues el arte camiadora artifiçial niente fuesse fechͣ por razon de ganer ardiños } e Mas esta arte pecumatiua de dineros non deue ser dicha natural & quae numismata in quibus partibus expenduntur , \textbf{ ut postea campsoria artificialiter effecta , | esset causa lucrandi pecuniam . } Haec enim pecuniatiua , \\\hline
2.3.10 & e el dinero es comienço e fin \textbf{ por que esta e arte comiencaen erl dinero } que omne da & Sed in ea ( secundum Philosophum primo Politicorum ) denarius est elementum et terminus , \textbf{ idest principium et finis . Incipit enim haec ars a denario , } quem dat ; \\\hline
2.3.10 & e en cada vn dinero es puesta senal publica . \textbf{ assi alguas uezes } por algun menester & Nam sicut massa metalli in denarios diuiditur , et imprimitur ibi signum publicum ; \textbf{ sic aliquando aliqua necessitate interueniente , } ut propter vasa fienda , \\\hline
2.3.10 & e por esta razon acaesçe por auentura \textbf{ que de tantos dineros en cuento se faze massa de mayor peso . } Et desta auentraa tomo comienço esta arte & Accidit ergo forte ex totidem denariis numero , \textbf{ confici massam maioris ponderis : } ex quo casu ars sumpsit originem , \\\hline
2.3.10 & Et tornando los en massavalen \textbf{ despues tres o quatro meaias ¶ } La terçera manera del arte pecuniatiua & ø \\\hline
2.3.10 & por nonbre comunal llamamos usura \textbf{ por que niguas cosas nunca cresçen en ssi mismas saluo } por parto o por generacion . & quam nos communi nomine appellamus usuram : \textbf{ nunquam enim aliqua crescunt in se ipsis , | nisi per partum , } vel per generationem ; \\\hline
2.3.10 & e mo parto de diueros . \textbf{ ¶ Mas destas quatro maneras segunt } que dize el p̃h̃o en las politicas la primera & quasi denariorum partus . \textbf{ Harum autem quatuor species } secundum Philosophum in Polit’ sola prima , \\\hline
2.3.10 & que es delas meaias . La usura es de denostar sienpre e en todas cosas \textbf{ assi commo parezcra enł capitulo } que se sigiͤ & et in omnibus detestanda , \textbf{ ut in sequenti capitulo apparebit . Usuras enim nemo exercere debet : campsoria autem , } et obolostatica \\\hline
2.3.10 & e algunos otros . \textbf{ Enpero alos Reyes e alos prinçipes los quales deuen ser medios dioses } non los conuiene de usar dellas & vel aliquibus aliis permittuntur regibus tamen \textbf{ et principibus | quod decet esse quasi semideos , } exercere non congruit . \\\hline
2.3.11 & e es robado el uso dela cosa \textbf{ por que vna cosa se vendedos uegadaso se vende el uso } y que non es suyo dela cosa . & quod rapina usus . In usura enim usus rapitur et usurpatur , \textbf{ uel idem uenditur bis , | uel uenditur ibi usus } qui non est suus . \\\hline
2.3.11 & de l esuso della \textbf{ mientra la sustançia fuere sienpre el uso diła parte nesçe } a aquel cuya es la sustançia . & et quia cuius est substantia , \textbf{ eius est usus : } quandiu substantia est ipsius , \\\hline
2.3.11 & para paresçer con ellas . \textbf{ La qual cosa fazen los rảcadores muchͣs uezes } por que parescan ricos & sed ad apparendum : \textbf{ quod forte multotiens mercatores faciunt , } qui ut appareant diuites , \\\hline
2.3.11 & Mas en las otras cosas \textbf{ por la mayor parte acaesçe el contrario . } Ca enluso propreio no se acomete usura & In rebus autem aliis \textbf{ ut plurimum contingit econtrario , } ut in usu proprio non committitur usura , \\\hline
2.3.12 & assi commo en çinco maneras ¶ \textbf{ La vna es dichͣ possessoria e de possessions ¶ } La segunda mercatiua e de mercadurias ¶ & quasi quinque viis . \textbf{ Quarum una dicitur possessoria . } Secunda mercatiua . \\\hline
2.3.12 & Por la primera manera de possessiones se ganan los aueres \textbf{ quando alguno abonda en muchͣs posessiones } e del fructo dellas resçibe muchs dineros . & Quarta experimentalis . Quinta artificia . Via autem possessoria acquiritur pecunia , \textbf{ quando quis possessionibus abundans , } ex fructibus earum pecuniam acquirit . Decet enim \\\hline
2.3.12 & quando alguno abonda en muchͣs posessiones \textbf{ e del fructo dellas resçibe muchs dineros . } Ca conuiene segunt el philosofo al mayordomo & quando quis possessionibus abundans , \textbf{ ex fructibus earum pecuniam acquirit . Decet enim } ( secundum Philosophum ) oeconomicum \\\hline
2.3.12 & e sabio derca las possessiones \textbf{ sabien do quales son de mayor fructo } e de quales puede meior acorrer ala mengua dela cala . & et dispensatorem domus esse expertum circa possessiones , \textbf{ sciendo quae sunt magis fructiferae , } et ex quibus potest melius subueniri indigentiae corporali domesticae siue gubernationi domus . Hoc autem fieri contingit , \\\hline
2.3.12 & Mas esto en qual manera se puede saber \textbf{ e en qual manera cada vno se deua auer çerca las possessiones . Et en qual manera las aues e las aialas de quatro pies se pueden guardar . } Et qual tr̃ra vale para labrar & Haec autem quomodo sciri possint , \textbf{ et qualiter circa possessiones quis se habere debeat , } ut qualiter aues , \\\hline
2.3.12 & para ganar las riquezas \textbf{ es dichͣ mercaduria } assi commo quando alguno por la mar o por la tierra lieuna algunas nicadurias o esta con aquellos que lieun a las mercadurias . & et qualis cura circa arbores sit gerenda , disposuimus silentio pertransire , \textbf{ eo quod alii de talibus sufficienter tradidisse videntur . Palladius enim multa huiusmodi enarrauit . Secunda via utilis ad pecuniam acquirendam , dicitur esse mercatiua , cum quis per mare aut per terram defert mercationes aliquas , } vel assistit deferentibus mercationes ipsas . Diuiditur autem ( secundum Philosophum ) mercatoria in tres partes , \\\hline
2.3.12 & Et por ende quando alguno conosçelos fechs particulares de algunos omes \textbf{ por los quales fechos ganaron alguas riquezas } Esta prueua tal es dicho ganançiosa & cum ergo quis nouit particularia facta aliquorum , \textbf{ quibus pecuniam sunt lucrati , dicitur } scire lucratiuam experimentalem . Recitat enim Philosophus duo particularia gesta , quibus fuit pecunia acquisita Primum est , \\\hline
2.3.12 & Ca el philosofo cuenta dos fechos particulares \textbf{ de los quales se ganaron grandes aueres . } ¶ El primero es & quibus pecuniam sunt lucrati , dicitur \textbf{ scire lucratiuam experimentalem . Recitat enim Philosophus duo particularia gesta , quibus fuit pecunia acquisita Primum est , } quod fecit Thales Milesius unus de septem sapientibus , \\\hline
2.3.12 & Mas por que mostrasse \textbf{ que ligera cosa seria alos philosofos de se enrriquesçer signind } cuydado ouiessen delas riquezas . & sed ut ostenderet \textbf{ quod facile esset Philosophis ditari , } si circa talia curam gererent ; \\\hline
2.3.12 & que auie de uenir \textbf{ que auie de ser grant cunplimiento de oliuas e de olio . } Et el por ende conpro todo el olio & vidit per astronomiam , \textbf{ futuram esse magnam copiam oliuarum : } et ab omnibus incolis regionis illius emit tantum oleum , \\\hline
2.3.12 & Et por estarazon demando dineros prestados \textbf{ et dio bueons fiadores } por todo el olio que auie de venir . & ø \\\hline
2.3.12 & Et lo otro \textbf{ por que auie el grand cunplimiento de olio pusol preçio } qual queso & nisi ipse : \textbf{ tum quia etiam erat magna copia olei , } lucratus est pecuniam multam , \\\hline
2.3.12 & qual queso \textbf{ e gano muy grand auer . } Et en esto mostro & tum quia etiam erat magna copia olei , \textbf{ lucratus est pecuniam multam , } et ostendit \\\hline
2.3.12 & ¶ El segundo fecho particular \textbf{ que cuenta esse mismo philosofo es de vn ceçiliano } que conpro todo es fierro delas ferrerias . & Secundum particulare gestum , \textbf{ quod recitat idem Phil’ est de quodam Siculo , } qui emit totum ferrum nundinarum : \\\hline
2.3.12 & e otros semeiantes \textbf{ por los quales algunos ganaron grandes algos } por que sil contesçiesse tal opportunidat de semeiantes fechos & oportet haec et similia particularia gesta , \textbf{ per quae aliqui pecuniam sunt lucrati , | habere in memoria : } ut si occurreret oportunitas , \\\hline
2.3.12 & sil fuessen conuenibles \textbf{ por ellos ganarian grand auer . } ¶ La quarta manera es dicha artifiçial & dum tamen illa sint licita , \textbf{ pecuniam lucraretur . } Quinta via dicitur esse artifica , \\\hline
2.3.12 & esto es \textbf{ quando alguno por su arte faze alguas obras } por que gana dineros . Ca commo quier la fin dela arte dela caualłia sea uictoria & Quinta via dicitur esse artifica , \textbf{ quando quis per artem suam aliqua exerceret , } propter quae pecuniam lucratur . \\\hline
2.3.12 & e alos prinçipes en ganar riquezas \textbf{ avn en esta misma manera es aprouechable la manera de possessiones } non solamente en las possessiones & utilis est Regibus et Principibus in acquisitione pecuniae . \textbf{ Sic etiam utilis est via possessionalis non solum in possessionibus immobilibus , } cuiusmodi sunt agri , et vineae , \\\hline
2.3.12 & segunt que los omes dizen \textbf{ que era de tan grand sabiduria del sieglo } que auie greyes maguer que fuesse señor de tierra muy abondosa & Vidimus enim Federicum Imperatorem , \textbf{ qui tantae sapientiae secularis praedicabatur , } habuisse massaritias multas . Non obstante enim quod terrae fertilissimae dominabatur , \\\hline
2.3.12 & que auie greyes maguer que fuesse señor de tierra muy abondosa \textbf{ en la qual auya muchͣs uiandas } e de grand mercado . & qui tantae sapientiae secularis praedicabatur , \textbf{ habuisse massaritias multas . Non obstante enim quod terrae fertilissimae dominabatur , } ubi victualia modici precii existebant ; nihilominus quasi tamen semper ex propriis alimenta carnium volebat assumere , \\\hline
2.3.12 & en la qual auya muchͣs uiandas \textbf{ e de grand mercado . } Enpero con todo esto & ø \\\hline
2.3.12 & que cunplen alas menguas dela uida \textbf{ non solamente delas animalias de quatro pies . } Mas avn delas aues que buelan & Quare decet Reges , et Principes habere homines industres tam super cultura agrorum et vinearum , quam etiam super armentis bonum et super multitudinem ouium , \textbf{ et aliorum animalium deseruientium ad indigentiam vitae , non solum quadrupedium sed etiam volucrum : } sicut alicubi consuetudo est habere multitudinem columbarum \\\hline
2.3.12 & e de otras aues delas quales son tomados gouernamientos para la casa ¶ \textbf{ Avn en essa misma manera } segunt el philosofo dize & ex quibus domestica alimenta sumuntur . \textbf{ Sic etiam , } ut Philosophus in Polit’ ait , \\\hline
2.3.12 & para resçebir tal labrança . \textbf{ Otrossi ueemos que delas abeias en algunas trras se cogen muy grand fructo } con pequanans despenssas en los logares & si partes illae aptae essent ad talem cultum : \textbf{ ex apibus enim in partibus conuenientibus colligitur multus fructus cum paruis expensis . } Ut ergo sit ad unum dicere , \\\hline
2.3.12 & Otrossi ueemos que delas abeias en algunas trras se cogen muy grand fructo \textbf{ con pequanans despenssas en los logares } que son conueinbles a ellas . Et pueᷤ que assi es & si partes illae aptae essent ad talem cultum : \textbf{ ex apibus enim in partibus conuenientibus colligitur multus fructus cum paruis expensis . } Ut ergo sit ad unum dicere , \\\hline
2.3.13 & que en esta terçera parte deste segundo libro auemos de tractar \textbf{ e de dezir de quatro cosas . } Conuiene a saber Delos hedifiçios & Dicebatur supra in hac tertia parte huius secundi libri , \textbf{ de quatuor esse dicendum : } videlicet de aedificiis , \\\hline
2.3.13 & que alg ssean subietos naturalmente a algunos otros \textbf{ la qual cosa praeua el philosofo en el primero libro delas politicas por quatro razones . } ¶ La primera razon se toma dela semeiança & et quod naturaliter expedit aliquibus aliis esse subiectos : \textbf{ quod probat | Philosophus primo Polit’ quadruplici via , } sumpta ex quadruplici similitudine . Prima via sumitur ex similitudine reperta in rebus inanimatis . \\\hline
2.3.13 & La terçera de departidos linages de ainalias ¶ \textbf{ La quarta del departimiento dela mena la muger en el humanal linage } ¶ & Secunda ex partibus eiusdem animalis . \textbf{ Tertia ex diuersis speciebus animalium . Quarta ex diuersitate sexuum in specie humana . Prima via sic patet . } Nam numquam aliqua multa \\\hline
2.3.13 & segunt la qual serie iudgada toda aquella concordança delas uozes delas otras \textbf{ avn en essa misma manera } si muchos elementos vienen a costruçion & oportet ibi dare aliquam vocem praedominantem , \textbf{ secundum quam tota harmonia diiudicatur . } Sic etiam si plura elementa concurrunt ad constitutionem eiusdem corporis mixti , \\\hline
2.3.13 & que algunos fuessen sennores \textbf{ e alguons sieruos . } Et pues que assi es algunos son natraalmente sieruos e alguas naturalmente senores ¶ & nisi naturale esset aliquos principari \textbf{ et aliquos seruire . } Sunt ergo aliqui naturaliter domini , \\\hline
2.3.13 & assi como es diche de suso naturalmente enssennorea \textbf{ alas bestiascaveemos muchͣs bestias domadas de casa } assi commo los canes e los cauallos & ( \textbf{ ut supra dicebatur ) naturaliter dominatur bestiis . Videmus enim multas bestias domesticas , } ut canes , \\\hline
2.3.13 & assi commo los canes e los cauallos \textbf{ que en muchͣs cosas han salud } por la sabiduria de los omes & ut canes , \textbf{ et equos in multis consequi salutem } propter prudentiam hominum , \\\hline
2.3.13 & la qual non podrian auer \textbf{ por su propia acuçia . } Et pues que assi es conuieneleᷤ & propter prudentiam hominum , \textbf{ quam ex propria industria consequi non possent . } Expedit ergo eis , \\\hline
2.3.13 & nin gniar assi mismos . \textbf{ Por ende assi commo natural cosa es alas bestias de seruir alos omes } assi cosa naturales & sicut naturale est bestias seruire hominibus , \textbf{ sic naturale est ignorantes subiici prudentibus expedit enim eis sic esse subiectos , } ut per eorum industriam dirigantur \\\hline
2.3.13 & Pot la qual cosa la piudunbre es en alguna manera cosa natural \textbf{ Et natural mente conuiene ala conpannia human } al quaalgers sean sieruos e alguas sennors assi comms es dicho en el comienço del capitulo & Quare seruitus est aliquo modo quid naturale , \textbf{ et naturaliter expedit societati humanae aliquos seruire , } et aliquos principari , \\\hline
2.3.14 & ssi commo sin el derecho natraal \textbf{ por el bien comun connino de dar } e de fazer alg̃s leyes pointiuas legunt & ut in principio capituli dicebatur . \textbf{ Sicut praeter ius naturale propter commune } bonum oportuit dare leges aliquas positiuas , \\\hline
2.3.14 & por el bien comun connino de dar \textbf{ e de fazer alg̃s leyes pointiuas legunt } las quales se gouernassen los regnos e las çibdades & Sicut praeter ius naturale propter commune \textbf{ bonum oportuit dare leges aliquas positiuas , } secundum quas regentur regna et ciuitates : \\\hline
2.3.14 & las quales se gouernassen los regnos e las çibdades \textbf{ assi paresçio alos estables çedores delas leyes } que lin la hudunbre natural & secundum quas regentur regna et ciuitates : \textbf{ sic visum fuit conditoribus legum , } quod praeter seruitutem naturalem , \\\hline
2.3.14 & segunt que dize el philosofo en las politicas \textbf{ aya algua auentaia sobre el su sieruo . } Et esta auentaia puede ser en dos maneras & Prima congruitas sic patet : \textbf{ oportet enim dominans ( ut dicitur in Politic’ ) habere aliquem excessum respectu serui . } Huiusmodi autem excessus dupliciter esse potest , \\\hline
2.3.14 & Mas el auentaia \textbf{ que es segunt los bienes del alma puede ser dichͣa uentaia sinple mente . } Mas el auentaia & Excessus autem \textbf{ secundum bona animae , | est quasi excessus simpliciter , } eo quod illa bona simpliciter dici possunt . Excessus vero \\\hline
2.3.14 & si pararemos mientes alos establesçedores delas leyes . \textbf{ Ca commo los establesçedores sołas leyes sean omes alos quales } mas son conosçidos los bienes del cuerpo & si considerentur legum conditores . \textbf{ Nam cum legum latores sint homines , } quibus magis sunt nota bona corporis \\\hline
2.3.14 & e los bienes de fuera \textbf{ que los bienes del alma e los biens dedem̊ } por que la ley diesse iuyzio fuerte & quibus magis sunt nota bona corporis \textbf{ et exteriora , quam animae et interiora : } ut lex daret iudicium de aliquo certo , visum fuit legum latoribus , ut superantes in bello congrue dominarentur aliis superatis , \\\hline
2.3.14 & que en el vençido . Onde el philosofo dize \textbf{ que mas e iusta cosa es sentir } que el señorio deue ser & Unde et Philosophus ait , \textbf{ Iustius esse diffiniri dominium } secundum bona animae , \\\hline
2.3.14 & que segunt los biens del cuerpo Mas assi commo el dize \textbf{ non es semeiante cosa fazer paresçer la fermosura del alma } e la fermosura del cuerpo ¶ & sed ( ut ait ) \textbf{ non similiter esse facile , } videre pulchritudinem animae , et corporis . \\\hline
2.3.14 & La terçera razon se toma dela salud de los vençidos . \textbf{ Ca por esta ley muchͣs vezes los vençidos enla batalla son saluos } por que los otros oms vençedores serique & Tertia congruitas sumitur ex salute debellatorum : \textbf{ nam propter hanc legem multotiens superati in bello saluantur : } homines enim alios debellantes proniores essent ad homicidium , \\\hline
2.3.14 & mas enclinados a matar e a fazer honuçidio \textbf{ si soperiessen que nigunt pro non aurian de tal uençimiento . } Mas quando pienssan & homines enim alios debellantes proniores essent ad homicidium , \textbf{ si scirent se ex eis nullam utilitatem consecuturos ; } sed cum cogitant eos acquirere in seruos , \\\hline
2.3.14 & que es pan de auer dellos . \textbf{ ¶ Oude sieruo segunt vna echunologia qͥe redez irgidado } por que tales son guardados enla batalla delos vençedores e non los matan & Unde seruus \textbf{ secundum unam etymologiam dicitur a seruando : } quia tales a victoribus reseruantur in bello ; \\\hline
2.3.15 & segunt el sobrepinamiento de los quales bienes conuiene a algunos \textbf{ de ser senno res dellos natural mente . } Mas aquellos que non son poderosos conuiene & Deficiunt enim in bonis animae , \textbf{ secundum quorum excessum contingit aliquos naturaliter dominari . } Impotentes vero contingit esse ministros ex lege : \\\hline
2.3.15 & non guardamos sienpre la orden natural . \textbf{ Et por ende en la mayor parte los prinçipados } e los sennorios son malos & et non semper reseruamus ordinem naturalem , \textbf{ ut plurimum principatus sunt peruersi : } nam ignoratis priuatim bonis animae , \\\hline
2.3.15 & Ottossi por que contesçe algunas uegadas \textbf{ que mucho son nasçidos de noble linage } que en todo tienpo de su uida non fazen ningua batalla iusta & et intellectu . Rursus , \textbf{ quia contingit } aliquando plures \\\hline
2.3.15 & La primera razon se prueua assi . \textbf{ Ca sienpre alos mas digunos son de dar los mayores benefiçios . } Et por ende commo el seruiente uirtuoso & ad principantem . Prima via sic patet . \textbf{ Nam dignioribus semper sunt ampliora beneficia tribuenda : cum ergo virtuosus seruiens ex amore honesti , } et ex dilectione boni , \\\hline
2.3.15 & mas por que es catiuado en la batalla . \textbf{ Otrosi el sobredicho seruidor } que sirue & ø \\\hline
2.3.15 & que los otrs . \textbf{ por la qual cosa digna cosa es } que ellos reçiban & quam alii . \textbf{ Quare dignum est ipsos } plus de influentia recipere , \\\hline
2.3.16 & do dize \textbf{ que alguas vezes peor siruen los muchs seruidores que los pocos . } Ca quando vn seruiçio es a comnedado a muchs muchͣs vezes & quod aliquando deterius seruiunt \textbf{ multi ministrantes quam pauci . } Nam cum multis idem ministerium committitur ; \\\hline
2.3.16 & que alguas vezes peor siruen los muchs seruidores que los pocos . \textbf{ Ca quando vn seruiçio es a comnedado a muchs muchͣs vezes } aquel seruiçio es mal fech o perdido & multi ministrantes quam pauci . \textbf{ Nam cum multis idem ministerium committitur ; } saepe illud negligitur : \\\hline
2.3.16 & Ca quando vn seruiçio es a comnedado a muchs muchͣs vezes \textbf{ aquel seruiçio es mal fech o perdido } Ca muchͣs uezes cada vno de aquellos seruientes menospreçia aquel seruiçio cuydando & Nam cum multis idem ministerium committitur ; \textbf{ saepe illud negligitur : } nam saepe quilibet ministrantium huiusmodi ministerium negligit , \\\hline
2.3.16 & aquel seruiçio es mal fech o perdido \textbf{ Ca muchͣs uezes cada vno de aquellos seruientes menospreçia aquel seruiçio cuydando } que el otro lo fara . & saepe illud negligitur : \textbf{ nam saepe quilibet ministrantium huiusmodi ministerium negligit , } credens quod alius exequatur illud : \\\hline
2.3.16 & por quien todas las cosas son ordenadas . \textbf{ En essa misma manera cada vna muchedunbre } si bien ordenada es conuiene & ut in unum Deum , a \textbf{ quo omnia ordinantur . Sic quaelibet multitudo , } si debet esse ordinata , oportet reduci in unum aliquem , \\\hline
2.3.16 & que son dichͣs en el quarto libro delas politicas . \textbf{ Ca deuemoos assi ymaginar que comm̃o se ha la guand çibdat ala pequeña . } Assi se ha la grand casa ala pequana . & ex iis quae dicuntur 4 Polit’ . \textbf{ Debemus enim sic imaginari quod sicut se habet magna ciuitas paruam , } sic se habet magna domus ad paruam . \\\hline
2.3.16 & Ca deuemoos assi ymaginar que comm̃o se ha la guand çibdat ala pequeña . \textbf{ Assi se ha la grand casa ala pequana . } Ca enla grand çibdat & Debemus enim sic imaginari quod sicut se habet magna ciuitas paruam , \textbf{ sic se habet magna domus ad paruam . } In magna enim ciuitate non sunt congreganda officia et principatus , \\\hline
2.3.16 & Assi se ha la grand casa ala pequana . \textbf{ Ca enla grand çibdat } assi commo el dize & Debemus enim sic imaginari quod sicut se habet magna ciuitas paruam , \textbf{ sic se habet magna domus ad paruam . } In magna enim ciuitate non sunt congreganda officia et principatus , \\\hline
2.3.16 & assi que a vno sean acomne dados departidos ofiçios o departidos mahestradgos . \textbf{ Ca en la grand çibdat los officios e los mahestradgos } tan grand cura hananexa e ayuntada & vel diuersi magistratus : \textbf{ quia in ciuitate magna officia et principatus } tantam curam habent annexam , \\\hline
2.3.16 & Ca en la grand çibdat los officios e los mahestradgos \textbf{ tan grand cura hananexa e ayuntada } que vna perssona non puede conplir & quia in ciuitate magna officia et principatus \textbf{ tantam curam habent annexam , } ut eadem persona sufficere non posset \\\hline
2.3.16 & que vna perssona non puede conplir \textbf{ para segnir ligeramente muchs ofiçios . } Mas en la pequanan çibdat & ut eadem persona sufficere non posset \textbf{ ad faciliter exequendum officia multa . } Sed in parua ciuitate vel in parua villa , \\\hline
2.3.16 & Mas en la pequanan çibdat \textbf{ o avn enla pequana uilla } do non pueden muchs auer los ofiçios & ad faciliter exequendum officia multa . \textbf{ Sed in parua ciuitate vel in parua villa , } ubi propter habitantium paucitatem non multi possunt praesidere in officiis et ubi officia commissa non magnam curam habent annexam , \\\hline
2.3.16 & Et do los ofiçios acomnedados \textbf{ non han grand cura anexa } pueden se muchos ofiçios & ø \\\hline
2.3.16 & do es muchedunbre de seruientes \textbf{ e do son muy guatdes ofiçios } e han grand cura anexa son en toda manera los ofiçios de partir & Sed in domibus Regum et Principum , \textbf{ ubi est multitudo ministrantium , et ubi officia maximam curam habent , } sunt omnino officia particulanda , \\\hline
2.3.16 & e do son muy guatdes ofiçios \textbf{ e han grand cura anexa son en toda manera los ofiçios de partir } e de dar & ubi est multitudo ministrantium , et ubi officia maximam curam habent , \textbf{ sunt omnino officia particulanda , } et distinguenda , \\\hline
2.3.16 & a much s ofiçiales . \textbf{ Et non son muchs ofiçios de a comne dar a vno } por que non sea enbarguada la ligereza dela enecuçion & et distinguenda , \textbf{ et non sunt plura committenda eidem , } ne impediatur facilitas exequendi . \\\hline
2.3.16 & que sean fieles \textbf{ quanto a derechͣ uoluntad } por que non engannen & et prudentes : \textbf{ fideles quidem quantum ad rectitudinem voluntatis , } ne fraudent : \\\hline
2.3.16 & Mas la fiesdat se puede conosçer \textbf{ por luengo tienpo } por que nos non podemos ver el coraçon del omne & ne per insipientiam defraudentur . \textbf{ Fidelitas autem cognosci habet per diuturnitatem : } ipsum enim cor hominis videre non possumus ; \\\hline
2.3.16 & por que nos non podemos ver el coraçon del omne \textbf{ mas si por luengos tienpos en departidos offiçios } qual fueren acomne dados & ipsum enim cor hominis videre non possumus ; \textbf{ sed si per diuturna tempora , et in diuersis officiis commissis fideliter se gessit , } fidelis est iudicandus . Prudentia vero cognosci habet per ea quae diximus in primo libro : \\\hline
2.3.16 & assi commo si alguon fuere bien acordable \textbf{ e buen proueedor } e bien acatado e bien aguardado delas cosas & fidelis est iudicandus . Prudentia vero cognosci habet per ea quae diximus in primo libro : \textbf{ ut si aliquis sit memor , prouidus , cautus , et circumspectus , et alia quae ibi diximus , } prudens est reputandus : \\\hline
2.3.17 & e sil diere ordenadamente \textbf{ e conuenble mente las colas nesçessarias } e por que prouision conueible delas uestiduras & si suam familiam debito modo gubernet , \textbf{ et si ei debite et ordinate necessaria tribuat : } et quia debita prouisio maxime videtur facere ad honoris statum , \\\hline
2.3.17 & quanto pertenesçe alo presente \textbf{ que cinco cosas son de penssar en esto . Conuiene a saber la magnificençia et gran dia del Rey . } La ordenança e semeiança de los siruientes . & Ad cuius euidentiam sciendum quod circa hoc ( quantum ad praesens spectat ) quinque sunt attendenda , \textbf{ videlicet regis magnificentia , } uniformitas ministrantium , conditio personarum , consuetudo patriae , \\\hline
2.3.17 & e en uestiduras conuenibles \textbf{ ca commo quier que non por uana eglesia } nin por aparesçençia uanas & et in debitis indumentis . \textbf{ Nam licet non ad inanem gloriam , } nec ad ostentationem talia sint fienda : \\\hline
2.3.17 & por que non conuiene \textbf{ que todos sean uestidos de eguales uestiduras } caenta grandescasas non solamente son legos & esse unius Principis ministros . Tertio circa prouisionem indumentorum consideranda est conditio personarum . \textbf{ Nam non omnes decet habere aequalia indumenta . } In tantis enim domibus non solum sunt laici , \\\hline
2.3.17 & assi es \textbf{ assi commo vna casa del muy alto prinçipe } que es dios & et aliter esse ordinatos . Sic enim videmus in ordine Uniuersi , quod cum totum uniuersum sit \textbf{ quasi una domus summi Principis , } Dei scilicet , \\\hline
2.3.17 & que es todo el mundo non gozan todas las cosas \textbf{ de egual apareiamiento o de egual fermosura } mas penssada la condiçonn de las cosas alguas cosas resplandesçen & in hac domo non omnia gaudent aequali apparatu , \textbf{ nec aequali pulchritudine , } sed considerata conditione rerum aliqua pollent maiori pulchritudine , \\\hline
2.3.17 & de egual apareiamiento o de egual fermosura \textbf{ mas penssada la condiçonn de las cosas alguas cosas resplandesçen } por mayor fermosura & nec aequali pulchritudine , \textbf{ sed considerata conditione rerum aliqua pollent maiori pulchritudine , } aliqua minori , \\\hline
2.3.17 & mas penssada la condiçonn de las cosas alguas cosas resplandesçen \textbf{ por mayor fermosura } e alguas por menor & nec aequali pulchritudine , \textbf{ sed considerata conditione rerum aliqua pollent maiori pulchritudine , } aliqua minori , \\\hline
2.3.17 & e la iustiçia del mundo \textbf{ enla qual cosa es magnifestada la muy marauillosa sabiduria de aquel que la fizo que es dios . } Por la qual cosa enlas casas de los Reyes e de los prinçipes & et iustitia Uniuersi , \textbf{ in quo declaratur conditoris mirabilis sapientia . } Quare in domibus Regum et Principum , \\\hline
2.3.18 & Vna segunt opinion de los omes \textbf{ assi conmola nobleza del linage ¶ Et otra segunt uerdat } assi commo es la nobleza de bueans costunbres . & unam secundum opinionem , \textbf{ ut nobilitatem generis : | et aliam } secundum veritatem , \\\hline
2.3.18 & assi conmola nobleza del linage ¶ Et otra segunt uerdat \textbf{ assi commo es la nobleza de bueans costunbres . } Ca la nobleza deue ser raygada en sobrepinança & et aliam \textbf{ secundum veritatem , } ut nobilitatem morum . Nobilitas autem , in excessu alicuius boni fundari videtur : \\\hline
2.3.18 & Et pues que assi es \textbf{ assi commo son dos linages de grandes bienes . } Ca algunas cosas son grandes bieño & nam nunquam diceretur unus nobilior alio , \textbf{ nisi in aliquo excederent illum . Sicut ergo sunt duo genera bonorum magnorum , } quia quaedam sunt magna bona \\\hline
2.3.18 & assi commo son dos linages de grandes bienes . \textbf{ Ca algunas cosas son grandes bieño } segunt epinion de los omes & nisi in aliquo excederent illum . Sicut ergo sunt duo genera bonorum magnorum , \textbf{ quia quaedam sunt magna bona } secundum opinionem , \\\hline
2.3.18 & e los bienes de fuera \textbf{ e otras cosas son grandes bienes } segunt uerdat & ut corporalia et extrinseca : \textbf{ quaedam vero sunt magna } secundum veritatem , \\\hline
2.3.18 & assi han de ser dos noblezas . \textbf{ Vna que se funda en sobrepuiamiento de grandes bienes segunt opinion de los omes . } Et otra que se funda en grandes bienes segunt uerdat & sic duplex nobilitas habet esse , \textbf{ una quae fundatur in excessu magnorum bonorum | secundum opinionem , } alia vero quae fundantur in magnis bonis \\\hline
2.3.18 & Vna que se funda en sobrepuiamiento de grandes bienes segunt opinion de los omes . \textbf{ Et otra que se funda en grandes bienes segunt uerdat } e esta es nobleza uerdadera & secundum opinionem , \textbf{ alia vero quae fundantur in magnis bonis } secundum existentiam et veritatem . \\\hline
2.3.18 & assi commo son honrra de linage riquezas e poderio çiuil \textbf{ e por ende el que viene de honrrado linage } o de ricos parientes o de poderosos & et ciuilis potentia . \textbf{ Qui ergo processit ex genere horabili , ut ex diuitibus , vel ex potentibus , } et hoc ex antiquo , \\\hline
2.3.18 & e por ende el que viene de honrrado linage \textbf{ o de ricos parientes o de poderosos } e esto de tienpo antiguo & et ciuilis potentia . \textbf{ Qui ergo processit ex genere horabili , ut ex diuitibus , vel ex potentibus , } et hoc ex antiquo , \\\hline
2.3.18 & e por ende se sigue \textbf{ que son nobles seg̃t reputaçion del pueblo } e segunt opiuon de los omes . & et per consequens est nobilis \textbf{ secundum reputationem populi } et secundum opinionem . \\\hline
2.3.18 & e segunt opiuon de los omes . \textbf{ Enpero uerdadera nobleza es } segunt auna taia de uertudes & et secundum opinionem . \textbf{ Vera tamen nobilitas est } secundum excessum virtutem \\\hline
2.3.18 & segunt auna taia de uertudes \textbf{ e segunt bondat de buenas costunbres } ca por que la fama & secundum excessum virtutem \textbf{ et bonitatem morum . | Tamen } quia nunquam fama totaliter perditur , \\\hline
2.3.18 & assi commo dize el philosofo en el viij̊ . libro delas politicas . \textbf{ Et esta tal opinion del pue blo ha de ssi alguna prueua } por que vemos enla mayor parte que los nobles de linage son de meiors costunbres & ut videtur velle Philosophus 7 Ethicorum , \textbf{ huiusmodi vulgaris opinio alicui probabilitati innititur . Videmus enim ut plurimum quod nobiles genere sunt nobiliorum morum quam alii : } nam sicut ex homine nascitur homo , et ex bestiis bestia : \\\hline
2.3.18 & Et esta tal opinion del pue blo ha de ssi alguna prueua \textbf{ por que vemos enla mayor parte que los nobles de linage son de meiors costunbres } que los otros . & ut videtur velle Philosophus 7 Ethicorum , \textbf{ huiusmodi vulgaris opinio alicui probabilitati innititur . Videmus enim ut plurimum quod nobiles genere sunt nobiliorum morum quam alii : } nam sicut ex homine nascitur homo , et ex bestiis bestia : \\\hline
2.3.18 & e de bestia nasçe bestia . \textbf{ Assi en la mayor parte de bueons nasce bueno } e de sabio suasçe sabio & nam sicut ex homine nascitur homo , et ex bestiis bestia : \textbf{ sic ut plurimum ex bonis nascitur bonus , } et ex prudentibus prudens , \\\hline
2.3.18 & e por que han prouado muchas cosas \textbf{ en la mayor parte son mas sabios que los otros . } Et otrossi & quia cum pluribus conuersantur , \textbf{ quasi experti ut plurimum sunt prudentiores aliis : } et quia multi oculi in ipsos respiciunt , \\\hline
2.3.18 & que assi es \textbf{ por que los nobles omes } segunt linageson en tal estado & quam alii . \textbf{ Quia ergo nobiles homines } secundum genus sunt in statu , \\\hline
2.3.18 & en las politicas la natura quiere sienpte fazer alguna cosa . \textbf{ Enpero muchͣs uezes non la puede fazer } mas fallesçe por algun enbargo & qui de hoc facere multotiens , \textbf{ tamen non potest , } sed deficit . \\\hline
2.3.18 & por que alguons nobles de linage de su naian se \textbf{ e desdizense de uerdadera nobleza } e son menos sabios & sed deficit . \textbf{ Quidam enim nobiles genere degenerant a naturae nobilitate , } et sunt peruersiores aliis \\\hline
2.3.18 & e desto paresçe donde vino la curialidat e la cortesia \textbf{ ca propreamente fablando non es dichͣ corte } si non casa de grandes e de nobles & Ex hoc ergo curialitas venisse videtur . \textbf{ Nam curia proprie non dicitur } nisi domus nobilium \\\hline
2.3.18 & que sean dichs curiales e corteses \textbf{ los que han nobles costunbres } por la qual cosa la curialidat e la cortesia es dicho alguna nobleza de costunbres & secundum mores , inde sumptum est , \textbf{ ut dicantur esse curiales habentes mores nobiles : } propter quod curialitas morum quaedam nobilitas dici potest . Est igitur curialitas quodammodo omnis virtus per comparationem ad nobilitatem morum , \\\hline
2.3.18 & e assi de todas las otras uirtudes . \textbf{ et essa misma manera conuiene que la nobleza delas costunbres sea toda uirtud por la qual cosa la curialidat e la cortesia } en alguna manera es dicho toda uirtud & et sic de aliis . \textbf{ Sic nobilitas docet omnem virtutem : } propter quod curialitas est quodammodo omnis virtus . Dicuntur enim curiales , \\\hline
2.3.18 & la qual cosa es obra de fraqueza \textbf{ en essa misma manera son dichos algunos curiales } si conueiblemente se ouieren en fazer guaades despessas & quod est opus liberalitatis . \textbf{ Sic et liberales dicuntur , } si decenter se habeant in magnis sumptibus , \\\hline
2.3.18 & si conueiblemente se ouieren en fazer guaades despessas \textbf{ la qual cosa es obra de magnifiçençia . En essa misma manera avn los comedores son dichos curiales } los que non comne con grand desseo & si decenter se habeant in magnis sumptibus , \textbf{ quod est opus magnificentiae . Sic et curiales dicuntur comestores , } qui non nimis ardenter vel turpiter comedunt , \\\hline
2.3.18 & la qual cosa es obra de magnifiçençia . En essa misma manera avn los comedores son dichos curiales \textbf{ los que non comne con grand desseo } nin torpemente la qual cosa es obra de tenprança . & quod est opus magnificentiae . Sic et curiales dicuntur comestores , \textbf{ qui non nimis ardenter vel turpiter comedunt , } quod est opus temperantiae . Curiales etiam dicuntur homines se habere erga suos ciues , \\\hline
2.3.18 & si fuer bien fablante alos otros \textbf{ e los catare con alegre cara . } la qual cosa es obra de bien fablança e de amistança . & si sit aliis affabilis , \textbf{ et hylari vultu alios recipiat ; } quod est opus affabilitatis . \\\hline
2.3.18 & avn dezer \textbf{ en algua manera la curialidat es toda uirtud } Et pues que assi es vna misma obra puede ser de uirtud spunal & Ut ergo sit ad unum dicere , \textbf{ quodammodo curialitas est omnis virtus . } Idem ergo opus esse potest a virtute speciali , \\\hline
2.3.18 & en algua manera la curialidat es toda uirtud \textbf{ Et pues que assi es vna misma obra puede ser de uirtud spunal } e de iustiçia legal & quodammodo curialitas est omnis virtus . \textbf{ Idem ergo opus esse potest a virtute speciali , } et a iustitia legali , \\\hline
2.3.18 & este es dichcurial . \textbf{ Aon en essa misma manera } si algbiuiere et morare con los otros alegremente & curialis est . \textbf{ Sic etiam si hylariter et affabiliter } quis cum aliis conuersetur , \\\hline
2.3.18 & por que les plaze de despender \textbf{ nin por que se delecten en dar la gual cosa faze el omne liberal } nin otrossi non lo faze & ut bona sua aliis largientes , non agentes hoc quia eis placeat expendere ; \textbf{ nec quod delectentur in dando , | quod facit liberalis ; } nec quod ex hoc velint implere \\\hline
2.3.18 & Mas por que el quiere retener las costunbres dela corte \textbf{ e de los no nobles omes } alos que les conuienne de ser dadores e cobidadores & quod facit iustus legalis : \textbf{ sed quia volunt retinere mores curiae et nobilium , } quos decet datiuos esse ; \\\hline
2.3.18 & assi conuienea los Reyes e alos prinçipes \textbf{ por que son en muy grand grado de nobleza auer buenas costunbres } e de ser curiales e nobles & habere mores nobiles et curiales , \textbf{ ministros , } quos in bonis decet suos dominos imitari , \\\hline
2.3.19 & e de su sabiduria tantol parte nesçen \textbf{ e le son de acomendar mayores ofiçios } ca ninguno non pue de auer praena conplidamente dela bondat e dela fieldat de otro & quanto plus constat de eius fidelitate et prudentia , \textbf{ tanto sunt ei maiora officia committenda . nullus autem ad plenum potest experiri de beniuolentia } et fidelitate alterius nisi per diuturnitatem temporis , \\\hline
2.3.19 & fasta que sea despendidos muchos moyos de sal . \textbf{ En essa misma manera avn dela sabiduria de alguno non pue de ser el omne conplidamente çierto } si non le uieremos fazer las cosas sabiamente & priusquam simul multos modios salis consumant . \textbf{ Sic etiam et de prudentia alicuius plene non constat , } nisi per diuturnum tempus viderimus ipsum prudenter egisse . \\\hline
2.3.19 & si non le uieremos fazer las cosas sabiamente \textbf{ por luengo tienpo . } ¶ Et pues que & Sic etiam et de prudentia alicuius plene non constat , \textbf{ nisi per diuturnum tempus viderimus ipsum prudenter egisse . } Hoc ergo modo sunt ministris officia committenda , \\\hline
2.3.19 & que primeramente deuen ser puestos en pequanos mahestradgos \textbf{ e en pequa nons ofiçios } por que dela fieldat & Hoc ergo modo sunt ministris officia committenda , \textbf{ quia primo praeponendi sunt in paruis magistratibus , } ut de eorum fidelitate \\\hline
2.3.19 & que ellos bien se ayan en aquellos ofiçios menores \textbf{ que les puedan acomne dar otros mayores ofiçios . } Enpero deuedes tener mientes con grand diligençia & et prudentia experimentum habeatur : \textbf{ quod si contingat eos bene se habere in illis , poterunt eis ulteriora committi . Est tamen diligenter considerandum , } quod quia mores nuper ditatorum , \\\hline
2.3.19 & que les puedan acomne dar otros mayores ofiçios . \textbf{ Enpero deuedes tener mientes con grand diligençia } que por que las costunbres & quod si contingat eos bene se habere in illis , poterunt eis ulteriora committi . Est tamen diligenter considerandum , \textbf{ quod quia mores nuper ditatorum , } et de nouo ascendentium ad altum statum , \\\hline
2.3.19 & de los que en el otro dia se fezieron ricos \textbf{ e los que de nueno suben en alto estado } assi como dize el philosofo enł segundo libro de la rectoriça & quod quia mores nuper ditatorum , \textbf{ et de nouo ascendentium ad altum statum , } ut dicitur 2 Rhetor’ \\\hline
2.3.19 & assi como dize el philosofo enł segundo libro de la rectoriça \textbf{ por la mayor parte son peores } que las costunbres de los & ut dicitur 2 Rhetor’ \textbf{ ut plurimum sunt peiores moribus aliorum : } ut si praepositi officiorum \\\hline
2.3.19 & e prueua dellos \textbf{ por luengo stp̃os ante } que ellos suban atan al cogrado . & et per diuturna tempora est habenda de ipsis experientia , \textbf{ prius quam ad aliud altum ascendant . } Viso quomodo ministris sunt officia committenda , \\\hline
2.3.19 & que les son acomnedados \textbf{ ca non conuiene a grandes sennores } nin a Reyes & et solicitudinem de ministris , \textbf{ non decet magnos dominos , nec Reges et Principes . } Nam ( ut dicitur primo Polit’ ) \\\hline
2.3.19 & que quales quier señores \textbf{ que han tan grand poder } por que esto puedan escusar el su procurador & quod quibusdam est potestas , \textbf{ ut hoc vitent , } procurator accipit hunc honorem : \\\hline
2.3.19 & e ellos deuen beuir çiuilmente \textbf{ e darse a grandes cosas o a sabiduria . } ¶ Et pues que assi es alos Reyes e alos prinçipes & ipsi vero ciuiliter viuunt , \textbf{ aut philosophantur . } Reges ergo et Principes \\\hline
2.3.19 & ¶ Et pues que assi es alos Reyes e alos prinçipes \textbf{ alos quales couiene de auer altos coraçones } conuiene les de obrar pocas cosas e grandes & Reges ergo et Principes \textbf{ quos decet esse magnanimos decet operari pauca et magna , } ut decet ipsos solicitari \\\hline
2.3.19 & en el quarto libro delas ethicas do dize que conuiene alos magnanimos \textbf{ e de altos coraçones } de se auer tenpradamente alos hunul lodos & Quod aliquomodo tradit Philosophus 4 Ethi’ \textbf{ ubi vult , } quod ad humiles decet magnanimos se habere moderate , \\\hline
2.3.19 & e de altos coraçones \textbf{ de se auer tenpradamente alos hunul lodos } mas a aquellos que son en grandes dignidades los magnanimos se deuen mostrar grandes . & ubi vult , \textbf{ quod ad humiles decet magnanimos se habere moderate , } sed ad eos \\\hline
2.3.19 & de se auer tenpradamente alos hunul lodos \textbf{ mas a aquellos que son en grandes dignidades los magnanimos se deuen mostrar grandes . } ¶ Et pues que assi es los Reyes e los prinçipes & quod ad humiles decet magnanimos se habere moderate , \textbf{ sed ad eos | qui sunt in dignitatibus decet magnanimos ostendere se magnos . } Reges ergo et Principes , \\\hline
2.3.19 & ¶ Et pues que assi es los Reyes e los prinçipes \textbf{ alos quales conuiene de ser magn animos } deuen se mostrar tonprados & Reges ergo et Principes , \textbf{ quos decet esse magnanimos ad proprios ministros , | qui respectu eorum sunt inferiores et humiles , } debent se ostendere moderatos : \\\hline
2.3.19 & deuen se mostrar tonprados \textbf{ a los sus seruientes propreos } los quales en conparacion dellos son humillosos e baxos & debent se ostendere moderatos : \textbf{ quia erga eos velle se habere in nimia excellentia } ( \\\hline
2.3.19 & los quales en conparacion dellos son humillosos e baxos \textbf{ por que contra ellos non se deuen mostrar en grand alteza } assi commo dize el philosofo llanamente & quia erga eos velle se habere in nimia excellentia \textbf{ ( } ut plane tradit Philosophos ) \\\hline
2.3.19 & por que esto non es uirtuoso \textbf{ mas es cosa de grand carga . } Mas que en tales cosas de uemeros tener el medio e manera e ser tenprados puedese tomar de aquellas cosas & non virtuosum , \textbf{ sed onerosum . } Quid sit autem tenere in talibus medium , \\\hline
2.3.19 & por que en toda manera parezca cruel \textbf{ e de grand carga } por que en todas estas cosas assi commo es dicho en las ethicas & nec debet se sic excellentem ostendere , \textbf{ ut omnino appareat austerus } et onerosus . In omnibus enim ( ut traditur in Ethi’ ) medium laudatur , \\\hline
2.3.19 & Mas fablando elpho destas cosas \textbf{ cerca la fin del prim̃o libro delas politicas dize } que la fenbra & et ab usu rationis deficiat . \textbf{ De his autem loquens Philosophus circa finem primi Politicorum ait , } quod foemina quidem habet consilium inualidum , \\\hline
2.3.19 & Mas el sieruo en ninguna manera non ha conseio \textbf{ ca essa misma manera avn a los sieruos } por ley non deuen ser descubiertos los conseios & puer autem habet sed imperfectum , \textbf{ seruus vero omnino habet nihil consiliatiuum . Sic etiam nec seruis ex lege communiter sunt communicanda consilia : } quia tales \\\hline
2.3.19 & por ley non deuen ser descubiertos los conseios \textbf{ por que tales por la mayor parte } mas siruen por teraor que por amor . & seruus vero omnino habet nihil consiliatiuum . Sic etiam nec seruis ex lege communiter sunt communicanda consilia : \textbf{ quia tales } ut plurimum magis seruiunt ex rimore , \\\hline
2.3.19 & e el gualardon de los siruientes \textbf{ mas deuen les partir mayores e menores bñfiçios } segunt que les paresçiere & mercedes ministrorum retinere non debent : \textbf{ sed debent ipsis maiora | et minora beneficia tribuere , } prout apparebit ipsos minus vel amplius meruisse . \\\hline
2.3.20 & desto que tal cosa commo esta contradize ala orden natural . \textbf{ ¶ La segunda de aquello que contradize ala bondat delas buenas costunbres . } ¶ La primera razon paresçe & etiam omnium ciuium abundare eloquiis . Prima via sumitur ex eo quod hoc repugnat ordini naturali . \textbf{ Secunda ex eo quod contradicit bonitati morum . Prima via sic patet . Nam , } ut dicitur primo Poli’ \\\hline
2.3.20 & por que non sea confusion en las obras \textbf{ e por que el vno non enbargue al otro contra natural ordenes } quando por aquel estrumento entendemos fazen vna de aquellas cosas & ne sit in operibus confusio , \textbf{ et ne unum impediat aliud , | contra naturalem ordinem est } cum per illud organum intendimus circa unum illorum operum , \\\hline
2.3.20 & por la qual cosa segunt el philosofo en el terçero libro del alma \textbf{ conmola lengua sea dada } assi commo estrumento & secundum Philosophum in tertio de Anima , \textbf{ lingua congruat in duo opera naturae , } ut in gustum , \\\hline
2.3.20 & por que avn cosa conuenible es aellos de auer uirtudes \textbf{ e bueans costunbres . } Mas alos que son assentados en las mesas conuiene de escusar muchedunbre de palabras & quia congruum est etiam et ipsos participare virtutes \textbf{ et bonos mores . } Sed si recumbentes , \\\hline
2.3.20 & sienpre se leyessen algunas cosas proprouechosas \textbf{ assi que quando ellos comne oyessen alguas palabras de bueans costunbres } e de bueons castigas asto serie mas conueinble & et Principum aliqua utilia legerentur , \textbf{ ut simul , } cum fauces recumbentium cibum sumunt , earum aures doctrinam perciperent ; esset omnino decens et congruum . \\\hline
2.3.20 & en que diemos arte \textbf{ segunt la manera delanr̃a sçiençia del gouernamiento dela casa } es fortando nos en la ayuda de aquel & ea silentio pertransire , imponentes finem huic secundo Libro , \textbf{ in quo de regimine domestico | secundum modulum nostrae scientiae artem tradidimus , } immo eius auxilio , \\\hline
3.1.1 & ssica commo las nuestras obras nos ordenamos \textbf{ a algun bien alguas uezes a inclinaçion } e amouemiento naturala aquel bien . & Nam cum opera nostra ordinamus ad aliquod bonum , \textbf{ aliquando ad bonum illud habemus impetum a natura , } aliquando \\\hline
3.1.1 & por corrupcion dela natura \textbf{ et aquel bien alguas uezes paresçe nos bien } e non es bien mas aquel bien & aliquando \textbf{ quasi ex corruptione naturae . Bonum autem illud , } ad quod omnes homines habent impetum ex natura , \\\hline
3.1.1 & que establescen la çibdat \textbf{ por que han natural inclinaçion han establesçimiento della } por ende la çibdat non solamente es establesçida & Igitur per respectum ad homines ciuitatem constituentes , \textbf{ eo quod habent naturalem impetum | ad constitutionem eius , } ciuitas non solum constituta est gratia eius \\\hline
3.1.1 & si la comuidat dela casa es ordenada a bien \textbf{ e avn a muchs bienes } assi commo es prouado de suso & Quare si communitas domestica ordinatur ad bonum \textbf{ et etiam ad multa bona , } ut supra in secundo libro diffusius probabatur : \\\hline
3.1.1 & que es llamada \textbf{ por nonbre comunal çibdat . } Enpero conuiene de saber & haec autem est communitas politica , \textbf{ quae communi nomine vocatur ciuitas . Aduertendum tamen , } communitatem ciuitatis esse principalissimam non simpliciter \\\hline
3.1.2 & e otra es el ser uirtuosamente . \textbf{ En essa misma manera otra cosa es beuir conplidamente } e otra cosa es beuir uirtuosamente & aliud virtuose esse : \textbf{ sic aliud est viuere , | aliud sufficienter viuere , } et aliud virtuose viuere . \\\hline
3.1.2 & que aquel ser conplidamente ¶ \textbf{ Et pues que assi es mas larga cosa es el ser } que el ser conplidamente & Si enim alicui rei deficiat aliqua perfectio competens suae speciei , \textbf{ licet possit habere illa res esse aliquod , } ut imperfectum esse : \\\hline
3.1.2 & que el ser conplidamente \textbf{ En essa misma manera avn cosa } mas general es el ser conplidamente & ut imperfectum esse : \textbf{ tamen sufficiens esse habere non dicitur : } latius est ergo esse , \\\hline
3.1.2 & e el beuir conplidamente \textbf{ e el benir uirtuosa mente . } Ca en qual si quier manera & distinguere possumus viuere , sufficienter viuere , \textbf{ et virtuose viuere . Qualitercumque } etiam homo habeat esse , viuit : non tamen dicitur sufficienter viuere , \\\hline
3.1.2 & e otra cosa es beuir conplidamente . \textbf{ En essa misma manera avn otra cosa es beuir conplidamente } e otra cosa es beuir uirtuosamente . & aliud est ergo viuere aliud sufficienter viuere . \textbf{ Sic etiam aliud est uiuere sufficienter , | et viuere virtuose ; } multi enim habent sufficientiam \\\hline
3.1.2 & e otra cosa es beuir uirtuosamente . \textbf{ ¶ Et pues que assi es el establesçimiento dela çibdat es razon de muchos e muy grandes bienes } porque por ella alcançan los omes ser acabados & ad vitam , \textbf{ qui propter corruptionem appetitus virtuose viuere negligunt . | Multorum igitur et maximorum bonorum causa est constitutio ciuitatis , } quia per eam homines consequuntur omnia tria praedicta bona . \\\hline
3.1.2 & que cunplena la uida mas avn \textbf{ por que bi una bien segunt ley e uirtuosa mente . } Et por ende dize el pho en el primero libro delas politicas & et constituentis ciuitatem \textbf{ non solum debet esse , } ut ciues in ciuitate habeant sufficientia ad vitam , \\\hline
3.1.2 & que fue fecha la çibdat \textbf{ non solamente por grande beuir } mas por grande beuir bien & non solum debet esse , \textbf{ ut ciues in ciuitate habeant sufficientia ad vitam , } sed ut viuant bene \\\hline
3.1.2 & non solamente por grande beuir \textbf{ mas por grande beuir bien } e segunt ley & ut ciues in ciuitate habeant sufficientia ad vitam , \textbf{ sed ut viuant bene } secundum leges \\\hline
3.1.2 & Et veyendo \textbf{ que por grant cuydado } que ouiessen non podrian abastar assi mesmos en la uida & fuit ipsum viuere , \textbf{ et habere sufficientia in vita . Videntes autem } quod solitarie non poterant sibi in vita sufficere , constituerunt ciuitatem , \\\hline
3.1.2 & e para beuir bien \textbf{ e segunt ley e uirtuosa mente . } ¶ Et pues que assi es & ut ad uiuere \textbf{ secundum legem et uirtuose . } Si ergo tot bona consequimur ex constitutione ciuitatis , bene dictum est quod scribitur primo Politicorum , \\\hline
3.1.2 & que el primo \textbf{ que establesçio la çibdat fue razon de muy grandes biens . } udaron alguons & quod qui primus ciuitatem instituit , \textbf{ extitit causa maximorum bonorum . } Dubitant nonnulli , \\\hline
3.1.3 & e si el omne es natraalmente aian l . politicas . \textbf{ e çiuilca aque łłas cosas } que son segunt natura paresçe & et An homo sit naturaliter animal politicum \textbf{ et ciuile ? | Nam ea quae sunt } secundum naturam videntur esse semper et ubique : \\\hline
3.1.3 & ca non es esto \textbf{ assi natural al omne conmoes cosa natural al fuego de escalentar } e ala piedra de desçender ayuso & Non enim hoc est sic homini naturale , \textbf{ sicut est naturale igni calefacere , } et lapidi deorsum tendere : \\\hline
3.1.3 & assi naturalmente aianl çiuil \textbf{ mas es dicho qual conuiene naturalmente de seraianl ciuil } por que ha alguna inclinaçion & homo ergo non sic naturaliter est animal ciuile , \textbf{ sed dicitur ei hoc naturaliter conuenire , } quia habet quendam impetum \\\hline
3.1.3 & assi commo dezimos \textbf{ que maguera natural cosa sea al omne } de ser diestro enpero much sson fallados simestris & ex casu vel ex aliquo impedimento siue ex aliqua causa impediri possunt , \textbf{ ut licet naturale sit homini esse dextrum , } multi \\\hline
3.1.3 & por algun enbargo o por algua otra cosa bien \textbf{ assi maguera natural cosa sea al omne de beuir çiuil mente . } Enpero muchos son fallados canpesinos e montanneses & vel ex aliqua causa reperiuntur abdextri : \textbf{ sic licet naturale sit homini viuere ciuiliter , } reperiuntur tamen multi campestre viuentes . Tangit autem Philosophus 1 Polit’ tria , \\\hline
3.1.3 & La primera es uentura¶ \textbf{ La segunda es grant maldat ¶ } La terçera es grant bondat ¶ & secundum nimia prauitas , \textbf{ et tertium nimia bonitas . } Contingit enim primo aliquos esse non ciuiles ex fortuna : \\\hline
3.1.3 & La segunda es grant maldat ¶ \textbf{ La terçera es grant bondat ¶ } La primera paresçe & et tertium nimia bonitas . \textbf{ Contingit enim primo aliquos esse non ciuiles ex fortuna : } ut quia nimium pauperes existentes , \\\hline
3.1.3 & ¶ La segunda razon \textbf{ por que el omne algunas uezes non biue çiuilmente es grant maldat . } ca algunos omes & et agros excolere . \textbf{ Secundum propter quod homo aliquando redditur non ciuilis est nimia prauitas : } aliqui enim sic habent appetitum corruptum \\\hline
3.1.3 & ¶la terçera razon \textbf{ por que algunos non bin ençiuilmente es por grant bondat } que han ca la uida çiuil & Aliqui autem sunt tantae perfectionis , \textbf{ quibus non sufficit viuere ut homo , } sed renuentes coniugium et ciuilitatem , \\\hline
3.1.4 & la qual forma por sobrepuiança es cosa natraal \textbf{ e es essa misma natura . } mas ueemos que la comuidat dela casa es cosa natural & quae per antonomasiam est quid naturale , \textbf{ et est ipsa natura . Videmus autem quod communitas domus est } quid naturale , \\\hline
3.1.4 & siguese \textbf{ que el uarriosa cosa natural } e por ende la çibdat & propter quod si tale crementum est naturale , \textbf{ vicus ipse quid naturale erit . } Ciuitas ergo , \\\hline
3.1.4 & por que por la palabra tioma el omne costunbres e disçiplina . \textbf{ En essa misma manera aqui deꝑte dela palabra podemos demostrar } que el omne es natraalmente aianl politicas e ciuil & et disciplinam . \textbf{ Hoc autem ex parte sermonis ostendere possumus hominem esse naturaliter animal politicum et ciuile , } ex eo quod vox humana , \\\hline
3.1.4 & por que la boz del omne \textbf{ que es dichͣ palabra es mas significatiua } que la bos delas bestias . & ex eo quod vox humana , \textbf{ quae dicitur sermo , | est aliter significatiua , } quam vox brutorum . \\\hline
3.1.4 & para prouaͬes pose toma de parte dela inclinacion natural \textbf{ ca todas las ainalias han natural inclinaçon } para guarda aquellas cosas & sumitur ex parte impetus naturalis . \textbf{ Nam omnia animalia habent naturalem impetum ad conseruandum ea quae sunt eis a natura tributa : } quare si natura dedit homini viuere , \\\hline
3.1.4 & que les son dadas dela natura . \textbf{ por la qual cosa si la natura dio al omne el beuir diol natural inclinaçion } para fazer aquellas cosas & Nam omnia animalia habent naturalem impetum ad conseruandum ea quae sunt eis a natura tributa : \textbf{ quare si natura dedit homini viuere , | dedit ei naturalem impetum } ad faciendum ea per quae possit sibi in vita sufficere . \\\hline
3.1.4 & e por ende en todos los omes es inclinaçion natural \textbf{ para beuir politicas miente en çibdat } e para fazer çibdat & quae ad vitam sufficiunt . Inerit ergo hominibus impetus naturalis \textbf{ ad viuendum politice , } et ad constituendum ciuitatem . \\\hline
3.1.5 & bien assi contesçe \textbf{ que alguas çibdades abondan en uino } e fallesçen entgo en el qual otras çibdades abondan & quam textorum : \textbf{ sic quia contingit ciuitates aliquas abundare in vino , } et deficere in frumento , \\\hline
3.1.5 & por que todas las çibdades non abondan en todas las cosas \textbf{ prouechosa cosa es alas çibdades de ser ay uirtadas so vn regno } por que meior se puedan acorrer las vnas alas otras & non omnes ciuitates abundant in eisdem , \textbf{ utile est eis congregari } sub uno regno , \\\hline
3.1.5 & prinçipesi sopieren \textbf{ que el ha grant poderio en la çibdat } e que han grant señorio en muchͣs cibdades & Quare cum peruersi in ciuitate aliqua non audeant insurgere contra principem , \textbf{ si sciant ipsum magnam habere ciuilem potentiam , } et dominare in ciuitatibus multis , \\\hline
3.1.5 & que el ha grant poderio en la çibdat \textbf{ e que han grant señorio en muchͣs cibdades } si fuere çierto del prinçipe & si sciant ipsum magnam habere ciuilem potentiam , \textbf{ et dominare in ciuitatibus multis , } si constet de principe quod iuste regat \\\hline
3.1.5 & los que prinçipes \textbf{ quesiessen tiranizar e ser malos en todo quanto menor poderio ellos ouieren tanto } mas es pro dela çibdat ¶ & quod si tamen Princeps tyrannizare vellet , \textbf{ quanto minorem haberet potentiam , } tanto magis esset expediens ciuitati . \\\hline
3.1.5 & ¶ Et pues que assy es commo el regno sea \textbf{ assi commo vna amistança de muchͣs çibdades } por que ellas son ayuntadas & cum ergo regnum sit \textbf{ quasi quaedam confederatio plurium ciuitatum , } eo quod uniantur sub uno rege , \\\hline
3.1.5 & si contesca que vna çibdat sea conbatida de los enemigos \textbf{ por mas ligero defendemiento } e mas seguro cosa aprouechosa fue & si contingat eam ab extraneis impugnari , \textbf{ propter faciliorem defensionem } et tuitionem utile fuit \\\hline
3.1.5 & e mas seguro cosa aprouechosa fue \textbf{ que de muchͣs comuidades publicas } se ssziese vna comuidat de regno & et tuitionem utile fuit \textbf{ ex pluribus communitatibus politicis constituere communitatem unam regni . } Generationis ciuitatis \\\hline
3.1.6 & assi commo sy muchͣs çibdades \textbf{ o muchs castiellos se ayuntassen en vno por amistança } e concordassen & ut si multae ciuitates \textbf{ et castra simul confoederarentur } et concordarent , \\\hline
3.1.6 & e del regno es natural en dos maneras \textbf{ ¶La primera es natural } por que es establesçida por generaçion de fijos & sed primus est naturalior secundo . Prima enim constitutio ciuitatis \textbf{ et regni est dupliciter naturalis . Primo enim naturalis est , } quia constituitur ex generatione \\\hline
3.1.6 & al \textbf{ non solamente por que los omes han natural inclinacion atal establesçimiento } mas avn por que tal establesçimiento nasçede la generaçion de los fijnos & non solum \textbf{ quia homines habent naturalem impetum ad talem constitutionem , } sed \\\hline
3.1.6 & commo quier que non sea tanna tal commo la primera enpero es natural \textbf{ por que los omes han natal inclinaçion para beuir } por que concuerdan de establesçer çibdat & ut prima attamen naturalis est , \textbf{ quia homines propter viuere habent naturalem impetum , } ut concordent ciuitatem constituere , \\\hline
3.1.6 & que cunplen para la uida . \textbf{ Avn en essa misma manera han natural inclinaçion } para establesçer prinçipado e regno & ut concordent ciuitatem constituere , \textbf{ in qua reperiuntur sufficientia ad vitam . Sic etiam habent naturalem impetum , } ut constituant principatum \\\hline
3.1.6 & que les quieren mal fazer et esta tal inclinaçion es natural \textbf{ que assi commo los omes han naturͣal inclinaçion } por que biuna & et magis resistere hostibus volentibus impugnare ipsos . Est enim huius impetus naturalis : \textbf{ nam sicut homines naturalem habent impetum ut viuant , } et ut sufficiant sibi in vita , \\\hline
3.1.6 & por la conpannia çiuil . \textbf{ En essa misma manera han inclinacion natraal } para que biuna en paz & quod fit per ciuilem societatem : \textbf{ sic naturalem habent impetum ut pacifice } viuant \\\hline
3.1.6 & Reyes han muchͣs guerras \textbf{ e muchͣs discordias entressi mas que las que estan so vn Rey . } Mas contada las dos maneras del fazemiento dela çibdat & videmus enim ciuitates non existentes sub uno rege plures guerras \textbf{ et discordias habere ad inuicem . Enumeratis autem duobus modis generationis ciuitatis et regni , } quorum quilibet dici potest naturalis : \\\hline
3.1.6 & e establesçer ende çibdat . \textbf{ En essa misma manera avn el Regno por fuerça } e por tirania se podia establesçer & et constituere inde ciuitatem . \textbf{ Sic etiam et regnum per uiolentiam } et per tyrannidem constitui posset , \\\hline
3.1.6 & ¶ Et pues que assi es conuiene nos de mostrar \textbf{ qual es la meior manera de gouernamiento de çibdat e de regno } e en qual manera se puede bien gouernar la çibdat & Ostendendum est ergo qualiter possit bene regi \textbf{ ciuitas } siue regnum tempore pacis , \\\hline
3.1.6 & e en qual manera deuemos lidiar \textbf{ contra los enemigos entp̃o de guerra . } Et pero assi commo paresçe & siue regnum tempore pacis , \textbf{ et qualiter impugnandi sint hostes tempore belli . } Verum \\\hline
3.1.6 & por que por esto seamos en algun manera endozidos \textbf{ a sabrque cosa deuemos esquiuar enł gouernamiento del rogno e dela çibdat . } Et pues que assy es todo este re terçero & ex hoc aliqualiter manducemur ad sciendum \textbf{ quid vitandum , et quid imitandum sit in regimine regni et ciuitatis . } Totum ergo hunc librum tertium diuidemus in tres partes . \\\hline
3.1.6 & Ca primero contaremos en quales cołas paresçe \textbf{ que non sintieron bien los antiguos philosofos } en este gouernamiento del regno e de la çibdat . Et por ende primeramente contaremos los gouernamientos establesçidos de los philosofos antiguos & Totum ergo hunc librum tertium diuidemus in tres partes . \textbf{ Nam primo narrabuntur opiniones Philosophorum circa regimen ciuile , } et reprobabuntur ea in quibus visi sunt non bene sentire circa huiusmodi regimen . \\\hline
3.1.6 & Lo segundo mostraremos \textbf{ qual es la muy buean politica o çibdat o muy vuen regno } e de quales cautelas deuen usar los prinçipes e los . Reyes & et regna . Secundo ostendetur , \textbf{ quae sit optima politia siue optimum regnum , } et quibus cautelis uti debeant principantes , \\\hline
3.1.7 & e de quales cautelas deuen vsar los lidiadores \textbf{ as socrates commo ouiesse phophado luengo tienpo çerca las naturas delas cosas } ueyendo muy grant guaueza cerca la sciençia natural & et quibus cautelis uti debeant bellantes . \textbf{ Socrates autem quandiu philosophatus esset circa naturas rerum , } videns circa naturalem scientiam magnam difficultatem esse , \\\hline
3.1.7 & as socrates commo ouiesse phophado luengo tienpo çerca las naturas delas cosas \textbf{ ueyendo muy grant guaueza cerca la sciençia natural } assi commo cuenta el pho & Socrates autem quandiu philosophatus esset circa naturas rerum , \textbf{ videns circa naturalem scientiam magnam difficultatem esse , } ut narrat Philosophus in Metaphysica sua , \\\hline
3.1.7 & assi commo cuenta el pho \textbf{ enla su mecha phisica conuirtiosse alascina moral . } al qual socrates siguio platon su disçipulo en muchͣs cosas & videns circa naturalem scientiam magnam difficultatem esse , \textbf{ ut narrat Philosophus in Metaphysica sua , | conuertit se ad Moralia , } quem Plato suus discipulus in multis secutus est , \\\hline
3.1.7 & enla su mecha phisica conuirtiosse alascina moral . \textbf{ al qual socrates siguio platon su disçipulo en muchͣs cosas } por la qual cosa el philosofo aristotiles llamo a platon el segundo socrates . & conuertit se ad Moralia , \textbf{ quem Plato suus discipulus in multis secutus est , | propter quod Philoso’ } Platonem ipsum \\\hline
3.1.7 & mas fuessen comunes los fijos seria comunes \textbf{ e por que es muy grant amor de los padres alos fijos . } Por ende en aquella çibdat seria muy grant amor & essent communes filii : \textbf{ et quia patrum ad filios est maxima dilectio , } in ciuitate illa esset maximus amor , \\\hline
3.1.7 & e por que es muy grant amor de los padres alos fijos . \textbf{ Por ende en aquella çibdat seria muy grant amor } por que los mas antiguos amarian todos los moços & et quia patrum ad filios est maxima dilectio , \textbf{ in ciuitate illa esset maximus amor , } eo quod antiquiores diligerent omnes iuniores \\\hline
3.1.7 & e de los fijos \textbf{ por que fuesse muy grant vnidat } e muy grant ayuntamiento enla çibdat . & Posuerunt etiam huiusmodi communitatem uxorum et filiorum \textbf{ ut esset maxima unitas } et maxima coniunctio in ciuitate . \\\hline
3.1.7 & por que fuesse muy grant vnidat \textbf{ e muy grant ayuntamiento enla çibdat . } Ca commo sea muy grant vnidat & ut esset maxima unitas \textbf{ et maxima coniunctio in ciuitate . } Nam cum sit maxima unitas , \\\hline
3.1.7 & e muy grant ayuntamiento enla çibdat . \textbf{ Ca commo sea muy grant vnidat } e grant ayuntamiento de los padres alos fijos & et maxima coniunctio in ciuitate . \textbf{ Nam cum sit maxima unitas , } et maxima coniunctio patrum ad filios , \\\hline
3.1.7 & Ca commo sea muy grant vnidat \textbf{ e grant ayuntamiento de los padres alos fijos } los mas antiguos cuydarian & Nam cum sit maxima unitas , \textbf{ et maxima coniunctio patrum ad filios , } antiquiores reputarent \\\hline
3.1.7 & los mas antiguos cuydarian \textbf{ que auian muy grant vnidat con los moços } e esso mismo los mocos cuydarian & et maxima coniunctio patrum ad filios , \textbf{ antiquiores reputarent } se habere maximam unitatem cum iunioribus , \\\hline
3.1.7 & e esso mismo los mocos cuydarian \textbf{ que auian grant vnidat } e grant amor con los antiguos & antiquiores reputarent \textbf{ se habere maximam unitatem cum iunioribus , } et econuerso , \\\hline
3.1.7 & que auian grant vnidat \textbf{ e grant amor con los antiguos } por que los antiguos creerian & antiquiores reputarent \textbf{ se habere maximam unitatem cum iunioribus , } et econuerso , \\\hline
3.1.7 & que biuen de rapina \textbf{ mas fuert s̃ paresçen las fenbras } que los mallos calicuydaremos & et in auibus viuentibus \textbf{ ex rapina ferociores esse videntur foeminae quam mares : } nam si consideramus aues ipsas viuentes ex raptu , \\\hline
3.1.7 & si en las otrasaia las vemos esto \textbf{ que non sola mente batallan los mas los mas avn las fenbras . } por ende paresçe & ø \\\hline
3.1.7 & ca la vena del oro es \textbf{ assi comm̃los prinçipantes de mayor prinçipado } e semeia & vena auri \textbf{ ut principantes maiori principatu , } vel existentes in maioribus principatibus ; \\\hline
3.1.7 & e semeia \textbf{ alos que estan en mayores senno rios } e la uena dela plata es & ut principantes maiori principatu , \textbf{ vel existentes in maioribus principatibus ; } et vena argenti ut existentes in minoribus , \\\hline
3.1.7 & e la uena dela plata es \textbf{ assi commo de aquellos que estan en menores poderios } e por ende non deuen ser conuertidos en venas de fierro & vel existentes in maioribus principatibus ; \textbf{ et vena argenti ut existentes in minoribus , } non debent conuerti in venam ferri , \\\hline
3.1.7 & assi que sean subditos \textbf{ e sean despuestos de sů maestradgos . } Mas lo quinto & ut quod fiant subditi , \textbf{ et deponantur a magistratibus suis . Quintum autem } quod dicti Philosophi senserunt statuendum circa ciuitatem , \\\hline
3.1.7 & Mas lo quinto \textbf{ que los dichs phos sintieron } e establesçienro çerca la çibdat es & et deponantur a magistratibus suis . Quintum autem \textbf{ quod dicti Philosophi senserunt statuendum circa ciuitatem , } est , \\\hline
3.1.8 & on conuiene de demandar en todas las cosas \textbf{ grant egualdat cosas fuessen eguales } ya non ser ianto & Maximam unitatem et aequalitatem non oportet quaerere in omnibus rebus . \textbf{ Nam si omnia essent aequalia , } iam non essent omnia . \\\hline
3.1.8 & ya non ser ianto \textbf{ e grant vnidat } ca si todas las das las cosas & iam non essent omnia . \textbf{ Ad hoc enim quod sunt omnia quae requiruntur } ad uniuersum , \\\hline
3.1.8 & y deꝑ tidas espeçies e departidas semeianças \textbf{ ca en muchos espeçies e semeianças delas cosas se salua mayor perfectiuo que en vna tan sola mente . } En essa misma manera avn en la çibdat & oportet ibi dare species diuersas ; \textbf{ ut in pluribus speciebus entium | reseruetur maior perfectio , } quam in una tantum . Sic etiam in ciuitate \\\hline
3.1.8 & ca en muchos espeçies e semeianças delas cosas se salua mayor perfectiuo que en vna tan sola mente . \textbf{ En essa misma manera avn en la çibdat } para que aya ser acabada conuiene de dar ay algun departimiento & reseruetur maior perfectio , \textbf{ quam in una tantum . Sic etiam in ciuitate } ad hoc quod habeat esse perfectum , oportet dare diuersitatem aliquam , nec oportet ibi esse omnimodam conformitatem \\\hline
3.1.8 & assi commo soctates e platon dixieron \textbf{ ca el philosofo pone etł segundo libro delas politicas seys razones } que prue una que non conuiene ala çibdat & et aequalitatem , \textbf{ ut Socrates et Plato credebant . Tangit autem Philosophus 2 Polit’ | quasi sex rationes , } probantes \\\hline
3.1.8 & que de vn regno \textbf{ ca la casa se faze de muchͣs personas } e eluarrio de muchͣ̃s casas & et ciuitatis quam regni . \textbf{ Constat enim domus ex pluribus personis , } vicus ex pluribus domibus , \\\hline
3.1.8 & ca la casa se faze de muchͣs personas \textbf{ e eluarrio de muchͣ̃s casas } e la çibdat de much suarrios & Constat enim domus ex pluribus personis , \textbf{ vicus ex pluribus domibus , } ciuitas ex pluribus vicis , \\\hline
3.1.8 & e la çibdat de much suarrios \textbf{ e el regno de muchͣs çibdades ¶ Et pues que assi es si la casa en tanto se feziesse vna } assi que todas las perssonas & ciuitas ex pluribus vicis , \textbf{ et regnum ex pluribus ciuitatibus . | Si ergo tantum uniretur domus , } quod omnes habitantes in ipsa fieret una persona , \\\hline
3.1.8 & mas serie vn omne singular . \textbf{ En essa misma manera } si el uarrio en tanto se feziesse vno & iam non remaneret domus , \textbf{ sed fieret homo unus aliquis singularis : } sic si tantum uniretur vicus , \\\hline
3.1.8 & que se feziesse vna casa ya non fincaria uarrio \textbf{ e en essa misma manera } si la çibdat fuesse en tanto vna que se feziesse vnuamno & quod fieret una domus , \textbf{ iam non remaneret vicus : eodem etiam modo } si tantum uniretur ciuitas , \\\hline
3.1.8 & por que la çibdat demanda algun departimiento \textbf{ e si se estendiere a mayor vnidat paresçra el ser dela çibdat . } Et pues que assi es dezer & requirit enim ciuitas aliquam diuersitatem , \textbf{ ultra quam si descendatur , perit esse eius . } Dicere ergo in ciuitate \\\hline
3.1.8 & Et pues que assi es dezer \textbf{ que enla çibdat o en el regno deua ser tan grant vnidat } commo dizian socrates e platones dezer & Dicere ergo in ciuitate \textbf{ vel in regno esse debere omnem unitatem , } est dicere ciuitatem non esse ciuitatem , \\\hline
3.1.8 & para prouar esto mismo se toma \textbf{ por conpara çiconala hueste o ala inpugnacion de los enemi gosca } alli commo dize el pho enł segundo libro delas politicas & Secunda via ad inuestigandum hoc idem , \textbf{ sumitur per comparationem ad exercitum , | vel ad pugnationem . } Nam ut dicitur 2 Polit’ \\\hline
3.1.8 & commo quier que sean ellos todos vna cosa en espeçie \textbf{ ca este defendemiento es por razon de mayor ayuda } que han los vnos de los otros & quamuis sit idem specie : \textbf{ est enim compugnatio auxilii gratia . } Sicut enim plures homines magis trahunt nauem , \\\hline
3.1.8 & ca assi comm̃ vemos \textbf{ que muchs omes tiran } mas la naue & est enim compugnatio auxilii gratia . \textbf{ Sicut enim plures homines magis trahunt nauem , } quantumcunque illi plures sint eiusdem ritus \\\hline
3.1.8 & que pocos commo quier \textbf{ que aquellos muchs omes sean muchs } e de vna costunbre & Sicut enim plures homines magis trahunt nauem , \textbf{ quantumcunque illi plures sint eiusdem ritus } et sint similes \\\hline
3.1.8 & e sean semeiables e ayuntables en aquel fech̃o . \textbf{ En essa misma manera much soens } mas lidian & et conformes , \textbf{ sic plures homines magis bellant . } Ciuitas autem non sic , \\\hline
3.1.8 & que assi es \textbf{ por que nos auemos me estermuchͣs cosas departidas para abastamiento dela uida conuiene que enla çibdat sea algun departimiento . } La tercera razon & Quia ergo diuersis indigemus ad vitam , \textbf{ oportet in ciuitate diuersitatem esse . } Tertia via declarans et manifestans vias praedictas , \\\hline
3.1.8 & y departidos mienbros que fagan estas obras departidas . \textbf{ En essa misma manera } por que & ø \\\hline
3.1.8 & Et pues que \textbf{ assi es en esta misma manera } assi commo si non fuesse departimiento en los mienbros del & ø \\\hline
3.1.8 & nin andar . \textbf{ En essa misma manera si fuesse grant ayuntamiento } e vnidat en la çibdat & nec ambulare : \textbf{ sic si esset maxima conformitas in ciuitate , } ut quod omnes essent textores vel coriarii , \\\hline
3.1.8 & la çibdat non seria acabada \textbf{ por que para abastameriento dela uida non sola mente auemos mester calçaduras e vestiduras } mas auenmos mester viandas e casas e otras cosas & ciuitas imperfecta esset , \textbf{ quia ad sufficientiam vitae | non solum indigemus calciamentis , } sed etiam indigemus victualibus , \\\hline
3.1.8 & a algun prinçipe o algun sennor \textbf{ e commo en la çibdat conuenga de dar alguons ofiçioso alguons maestradgos o algunas alcaldias } la qual cosa non seria & ut cum in ciuitate oporteat \textbf{ dare aliquos magistratus , | et aliquas praeposituras , } quod non esset , \\\hline
3.1.8 & para abastamiento deuida \textbf{ son meester muchͣs cosas departidas } por ende conuiene enla çibdat de auer en ssi algun departimiento & sed ut dictum est ad sufficientiam vitae requiruntur diuersa , ideo oportet ciuitatem habere aliquam diuersitatem in se , \textbf{ et diuersos habere vicos , } ut expediens \\\hline
3.1.8 & que es muchedunbre de bozes nunca es bien proporçionada si non fueren y todas las bozes eguales \textbf{ mas ala derecha consonançia delas bozes conuiene de dar } y departimiento de los tonos & ut nunquam melodia ; \textbf{ quae est multitudo vocum , } est bene proportionata , \\\hline
3.1.8 & assi commo la pintura non es bien ordenada \textbf{ si y non fuer departimiento de colores en essa misma manera } avn la çibdat non sera bien ordenada & est bene proportionata , \textbf{ si sint omnes voces aequales : } sed ad rectam consonantiam oportet ibi dare diuersitatem tonorum . Sic pictura nunquam est bene ordinata , \\\hline
3.1.9 & Et pues que assi es luengamente son los comienços de terctar \textbf{ e por luengos sermones son de escodrinnar } por que çerca ellos non contezca yerro . & Diu ergo sunt principia pertractanda , \textbf{ et longis sermonibus sunt excutienda , } ne circa ipsa contingat error . \\\hline
3.1.9 & que todos los moços eran sus fijos propreos \textbf{ e por ende en la çibdat seria muy grant amor } Et pues que assi es nos podemos mostrar & ciues omnes pueros esse filios suos , \textbf{ et sic esset in ciuitate maximus amor . Possumus ergo triplici uia ostendere , } quod hoc non expedit ciuitati . Primo , \\\hline
3.1.9 & que de todos los çibdadanos se fagan vn cuerporealmente \textbf{ assi non puede ser que vn mismo maniar } que cria el cuerpo de vn çibda & quod ex omnibus ciuibus fiat unum corpus realiter ; \textbf{ ita esse non potest , } quod ille idem cibus qui nutrit corpus unius ciuis , \\\hline
3.1.9 & e deue ser tal \textbf{ que segunt ella losoens puedan beuir comunal mente . } Onde el pho dize en el terçero libro delas politicas & et toti populo debet esse talis , \textbf{ secundum quam homines communiter possint uiuere . } Unde et Philosophus ait in Politicis ciues non esse applicandos legibus , \\\hline
3.1.9 & que biuien comunalmente \textbf{ sin muchͣs uaraias } e muchas contiendas . & quia inter gentes communiter uiuentes absque multis litigiis obseruari non posset . Dato tamen communitatem possessionum posse obseruare absque litigiis , communitatem tamen foeminarum , propter quam uolebat Socrates filios esse communes , \textbf{ non est possibile obseruare absque litigiis . } Secunda uia sic patet : \\\hline
3.1.9 & por la qual quarie socrates que los fijos fuessen comunes esto non podria ser \textbf{ nin se podria guardar sin muy grandes uaraias ¶ la segunda razon paresçe } assi ca si todas las cosas fuessen assi comunes & non est possibile obseruare absque litigiis . \textbf{ Secunda uia sic patet : } nam si omnia sic essent communia , \\\hline
3.1.9 & assi commo nieto \textbf{ e otro a otro assi conmo sobrino } e el pho llama primos alos & alius tanquam nepotem , \textbf{ alius tanquam fratruelem } ( appellat autem Philosophus fratrueles \\\hline
3.1.9 & o ha algun parentesco con el \textbf{ e este es mayor amor que aquel } que ponia socͣtesca pequano parentesco & secundum aliquam aliam consanguineitatem , \textbf{ quam fuerit dilectio } quam ponebat Socrates . \\\hline
3.1.9 & e este es mayor amor que aquel \textbf{ que ponia socͣtesca pequano parentesco } si fuere conosçido & quam fuerit dilectio \textbf{ quam ponebat Socrates . | Modica enim consanguineitas , } si sit nota et certa , \\\hline
3.1.9 & Mas conuiene alos Reyes \textbf{ e alos prençipes̃ de tener mientes a esto con grant acuçia } por que sepan ordenar la çibdat & Decet autem hoc Reges , \textbf{ et Principes diligenter aduertere , } ut sciant sic ciuitatem ordinare , \\\hline
3.1.10 & si las mugers e los fuos fueren puestos de ser comunes \textbf{ e quanto parte nesçe a lo presente podemos contar c̃co males } que se sigune de tal comuidat & si uxores et filii ponantur esse communes . \textbf{ Quantum autem ad praesens spectat , | enumerare possumus quinque mala , } quae Philosophus tangit ibidem sequentia ex tali communitate . Primum est , \\\hline
3.1.10 & que se sigune de tal comuidat \textbf{ los quales el pho pone en esse mismo logar . } El primero esiuria & enumerare possumus quinque mala , \textbf{ quae Philosophus tangit ibidem sequentia ex tali communitate . Primum est , } iniuria consanguineorum . \\\hline
3.1.10 & e el tuerto de los parientes ¶ \textbf{ El segundo es abiltamiento de los malos omes . } El terçero es non auer cuydado delos fijos ¶ & iniuria consanguineorum . \textbf{ Secundum , vilificatio nobilium . Tertium , } iniuria filiorum . Quartum , intemperantia venereorum . Quintum , \\\hline
3.1.10 & por la qual cosa commo la comuidat delas mugieres \textbf{ que ordeno socrates tire la cercidunbre de los fijos e el conosçimiento del parentesco non es de ella bartal comunidat } ca della se sigue & inter seipsos iniurias \textbf{ et contumelias inferant . | Quare cum communitas uxorum , } quam Socrates ordinauit , \\\hline
3.1.10 & ca por esta tal comiundat delas mugers \textbf{ e delos fijos siguese abiltamiento de los nobles omes } e enxalçamiento de los labradores & quia ex hoc consequitur aliquos de facili propter ignorantiam iniurari consanguineis suis . \textbf{ Secundum malum sic ostenditur . Nam supposita communitate uxorum et filiorum sequeretur vilificatio nobilium , } et exaltatio agricolarum , \\\hline
3.1.10 & e enxalçamiento de los labradores \textbf{ e delas uiles perssonas . } poues que assi es non puede ser de razon & et exaltatio agricolarum , \textbf{ et personarum vilium . } Nam esse non potest uxores \\\hline
3.1.10 & e de los mayores dela çibdat e essa misma sea puesta de los fijos de los labradores \textbf{ e de las uiles perssonas } por la qual cosa los fijos de los nobles seria abatidos & quae geritur de filiis agricolarum \textbf{ et personarum vilium . Quare filii nobilium deprimuntur , } et vilium exaltabantur . \\\hline
3.1.10 & los quales cuydaria \textbf{ que eran suᷤ fijos propreos } e por que aquellos nonl serian conosçidos çiertamente por ende cuydaua son crates & vel trium puerorum , \textbf{ quos crederent esse proprios filios : } et quia illi non essent eis certitudinaliter noti , opinabatur Socrates \\\hline
3.1.10 & Mas esto reprahende el philosofo enel segundo libro delas politicas \textbf{ ca por dos o por tres o por pocos mocos querera mar grant muchedunbre de moços } assi conmo a fijos propreos & Sed ut arguit Philosophus 2 Politic’ propter duos \textbf{ vel tres vel propter paucos pueros velle magnam multitudinem diligere puerorum tanquam proprios filios , hoc est ponere parum de melle in multa aqua . } Sicut ergo parum mellis totum unum fluuium \\\hline
3.1.10 & assi conmo a fijos propreos \textbf{ esto es poner poco de miel en muchͣ agua . } Et pues que assi es & vel tres vel propter paucos pueros velle magnam multitudinem diligere puerorum tanquam proprios filios , hoc est ponere parum de melle in multa aqua . \textbf{ Sicut ergo parum mellis totum unum fluuium } non posset facere dulcem , \\\hline
3.1.10 & Et pues que assi es \textbf{ assi commo poca miel puesta en vn grant rio non puede fazer todo el rio dulçe } assi amor de dos o de tro fiios non puede faz & Sicut ergo parum mellis totum unum fluuium \textbf{ non posset facere dulcem , } sic amor duorum \\\hline
3.1.10 & assi amor de dos o de tro fiios non puede faz \textbf{ que sea amada grant muchedunbre et sin cuenta de mocos } de que son en vna çibdat & sic amor duorum \textbf{ vel trium filiorum innumerabilem multitudinem puerorum } existentium in ciuitate una , \\\hline
3.1.10 & e muy golola \textbf{ por grant muchedunbre de uiandas } es cosa guaue de fazer al omne astinençia . & et temperate se habere erga illam . \textbf{ Sicut ergo prouocata gula per multitudinem ciborum difficile est esse abstinentes , } sic prouocaris venereis \\\hline
3.1.10 & El quinto mal se puede assy manifestar \textbf{ ca quando los padres e las madres non ouiessen conosçimiento de sus propios fijos e de sus fijas de ligero auria } y mal uso delas madres & et quasi impossibile est esse temperatum . Quintum autem malum sic manifestari potest . \textbf{ Nam non habentibus parentibus cognitionem | de propriis filiis et filiabus , } de leui fieret abusus parentum \\\hline
3.1.10 & ca quando los padres e las madres non ouiessen conosçimiento de sus propios fijos e de sus fijas de ligero auria \textbf{ y mal uso delas madres } e de las parientas & de propriis filiis et filiabus , \textbf{ de leui fieret abusus parentum } et consanguineorum , \\\hline
3.1.10 & e los padres con sus fijas . \textbf{ Empero socrates quariendo escusar este mal dix̉o } que al prinçipe dela çibdat pertenesçia de auer cuydado e acuçia & et patres filias . \textbf{ Socrates volens hoc inconueniens vitare , dixit , } quod spectabat ad Principem ciuitatis habere curam et diligentiam , \\\hline
3.1.11 & que los çibdadanos se aman much \textbf{ e que biuen sin contienda entre los quales se guarda tan grant comunidat . } mas assi commo dize el philosofo en el segundo libro delas politicas & et existimat ciues se maxime diligere , \textbf{ et absque litigio viuere , | inter quos tanta communitas obseruatur . } Sed ut dicitur secundo Politicorum in actibus particularibus oportet ad experientiam recurrere : experti enim sumus \\\hline
3.1.11 & que los que han algunas cosas comunesentte \textbf{ ssi han mayores contiendas que si cada vno ouiesse sus cosas proprias } e esto podemos prouar & quod habentes aliqua communia , \textbf{ plura litigia inter se habent , } quam si quilibet propria possideret . Possumus autem triplici via inuestigare , \\\hline
3.1.11 & que son pocoscreen firmemente \textbf{ que son ayuntados en tan grant parentesco } e han muchas contiendas & qui pauci sunt \textbf{ et firmiter credunt se esse tanta consanguinitate coniunctos , } multa habent litigia , \\\hline
3.1.11 & si todas las cosas fuessen comunes nasçeri \textbf{ que muchͣs contiendas } por que ellos son muchs & si omnia eis essent communia , \textbf{ multa orirentur litigia , } cum ipsi multi sint et diuersarum voluntatum , \\\hline
3.1.11 & e de departidas uoluntades \textbf{ e non ha entre ellos tan grant ayuntamiento de parentesco commo entre los hr̃manos de vn vientre } por que si se fiziesse & nec sit \textbf{ inter eos tanta coniunctio consanguinitatis , sicut | inter uterinos fratres , } nam \\\hline
3.1.11 & que eran ayuntados \textbf{ por mayor parentesco . } Enpero segunt uerdat entre ellos & secundum rei veritatem non esset \textbf{ tanta consanguinitas , } sicut inter fratres uterinos . \\\hline
3.1.11 & Et pues que assi es \textbf{ si çierto parentesco non tirala uaraia de pocos } que ban heredat en comun much mas el parentesço & ut in praehabitis probabatur . \textbf{ Si ergo certa consanguinitas non tollit dissensionem paucorum habentium haereditatem communem , } multo magis infra huiusmodi dissensionem \\\hline
3.1.11 & por que han de beuir en vno \textbf{ que por la mayor parte han contiendas e uaraias por la qual cosa dize el philosofo en el segundo libro de las politicas } que de los siruientes & ostenditur \textbf{ ut plurimum homines habere lites | et iurgia propter quod Philosophus ait 2 Polit’ } quod ab ipsis famulis , \\\hline
3.1.11 & que de los siruientes \textbf{ de los que nos auemos meester suiçio de cada dia resçebunos grandes ofenssas } e nos enssannamos contra ellos muchͣs uezes & quod ab ipsis famulis , \textbf{ quibus plurimum indigemus propter ancillares administrationes , } maxime offendimur , et indignamur erga illos , \\\hline
3.1.11 & de los que nos auemos meester suiçio de cada dia resçebunos grandes ofenssas \textbf{ e nos enssannamos contra ellos muchͣs uezes } por que nos conuiene de fablar muchͣs uezes con ellos & quibus plurimum indigemus propter ancillares administrationes , \textbf{ maxime offendimur , et indignamur erga illos , } quia oportet nos habere ad illos multa colloquia , \\\hline
3.1.11 & e nos enssannamos contra ellos muchͣs uezes \textbf{ por que nos conuiene de fablar muchͣs uezes con ellos } e de beuir conellos non los podiendo escusar & maxime offendimur , et indignamur erga illos , \textbf{ quia oportet nos habere ad illos multa colloquia , } et diu conuersari cum illis . \\\hline
3.1.11 & por que non nazcan entre los çibdadanos uarias e contiendas \textbf{ qua non sean las possessio nes } assi comunes commo establesçio socrates mas lamzon delpho & et iurgia , \textbf{ non sic esse possessiones communes , } ut Socrates statuebat . Via autem Philosophi , \\\hline
3.1.11 & ca dize que las possessiones e las cosas de los çibdadanos deuen ser propreas \textbf{ e comunes propraas } quanto al sennorio . & est expeditior civitati : \textbf{ ait enim possessiones et res civium debere esse proprias , et communes . Proprias quidem quantum ad dominum , communes vero propter virtutem liberalitatis . } Diligenter igitur inspecta humana conditione , \\\hline
3.1.11 & en quanto el pueblo deue guardar la ley \textbf{ e las buenos ordenamientos } conuiene alos çibdadanos de auer las cosas & prout communiter populus observatiuus est legum \textbf{ et laudabilium ordinationum , } expedit cuilibet habere res et possessiones proprias \\\hline
3.1.11 & quanto al sennorio . \textbf{ ca cada vn sennor de sus bienes propreos aura mayor acuçia de aquellos bienes } que si fuessen comunes & quantum ad dominum : \textbf{ nam quilibet dominans bonis propriis adhibebit debitam diligentiam circa illa . } Expedit autem talia esse communia \\\hline
3.1.12 & por tres razones \textbf{ segunt trs cosas } que son meester para la batalla ca los omes lidiadores conuiene & sed viros , \textbf{ triplici via venari possumus , } secundum tria quae requiruntur ad bellum . Homines enim bellatores decet esse mente cautos \\\hline
3.1.12 & e fuertes e ualientes de cuerpo \textbf{ por que es neçessaria cautella e sabiduria en las batallas } ca alguas & et prouidos : \textbf{ corde viriles et animosos : } corpore robustos et fortes . Cautela enim et prouidentia est necessaria in bellis . Nam aliquando propter cautelas adhibitas plus superantur hostes ex sagacitate \\\hline
3.1.12 & Et por ende non son de poner en las \textbf{ batallasca grant cautela e grant sabiduria } es meester en las batallas & non sunt ordinandae ad opera bellica . \textbf{ Nam in bellis magna cautela et industria est adhibenda , } quia secundum Vegetium in De re militari , \\\hline
3.1.12 & segunt dizeuegeçio en el libro del negoçio dela caualleria \textbf{ ca si las otras cosas mal fechͣs se pueden cobrar . } Enpero las auenturas delas batallas non han remedio ninguno . & quia secundum Vegetium in De re militari , \textbf{ si alia male acta recuperari possunt , } casus \\\hline
3.1.12 & para prouar estomesmo se toma dela fortaleza de uirtud \textbf{ e del buen coraçon } que es meester enlos lidiadores & sumitur ex virilitate \textbf{ et animositate , } quae requiritur in bellantibus . \\\hline
3.1.12 & que es meester enlos lidiadores \textbf{ ca assi commo dize el pho en el terçero libro delas ethicas fin e acabamiento de todas las cosas espantables } es la muerte & quae requiritur in bellantibus . \textbf{ Nam ut dicitur 3 Ethic’ finis | et terminus omnium terribilium , } est mors : \\\hline
3.1.12 & es la muerte \textbf{ e por ende las obras dela batalla demandan s omes sin miedo e de grant coraçon } por que los lidiadores se ponen a peligro de muerte & est mors : \textbf{ opera ergo bellica requirunt hominem inpauidum et animosum , } eo quod bellantes opponant se periculis mortis : \\\hline
3.1.12 & por que el frio a restrennir e apretar \textbf{ mas los aionsos e de grant coraçon e los esforçados han se de estendera muchos cosas . } Et por ende la calentura faze al omne sera ionso & ( ut dicebatur supra ) frigiditas viam timori praeparat ; frigidi enim est costringere \textbf{ et retrahere ; animosi vero et virilis est ad alia se extendere , } calor enim reddit habentem animosum et virilem , \\\hline
3.1.12 & al que la ha temeroso \textbf{ e de flaco coraçon . } Et por ende las mugers & frigiditas vero timidum et pusillanimum . \textbf{ Quare mulieres , } quae sunt pauidae \\\hline
3.1.12 & que son frias son temerosas \textbf{ e de flaco coraçon } e non se deuen enbiar alas batallas & quae sunt pauidae \textbf{ et pusillanimes , } ad opera bellica destinari non debent . \\\hline
3.1.12 & e non se deuen enbiar alas batallas \textbf{ ca meior cosa es de echar los temerosos dela batalla } que auerlos en su conpannia & ad opera bellica destinari non debent . \textbf{ In bellis enim melius est pauidos expellere , } quam eos in societate habere \\\hline
3.1.12 & ca commo todos los omes teman la muerte los esforçados \textbf{ e de grandes coraçones temen } quando veen & quam eos in societate habere \textbf{ nam } cum humanum sit timere mortem , viriles etiam et animosi trepidant videntes timidos trepidare : \\\hline
3.1.12 & conuiene de echar dela batalla \textbf{ e dela fazienda alos de flaco coraçon } de los quales es çierto & ø \\\hline
3.1.12 & ca commo los lidiadores ayan de sofrir el peso delas armas \textbf{ e ayan de dar grandes colpes } conuieneles de auer fuertes honbros e fuertes rennes & Tertia via sumitur ex parte fortitudinis corporalis . \textbf{ Nam cum bellantes oporteat diu sustinere armorum pondera , et dare magnos ictus , } expedit eos habere magnos humeros et renes ad sustinendum armorum grauedinem , \\\hline
3.1.12 & e ayan de dar grandes colpes \textbf{ conuieneles de auer fuertes honbros e fuertes rennes } para sofrir la pesadura delas armas & Nam cum bellantes oporteat diu sustinere armorum pondera , et dare magnos ictus , \textbf{ expedit eos habere magnos humeros et renes ad sustinendum armorum grauedinem , } et habere fortia brachia ad faciendum percussiones fortes : \\\hline
3.1.12 & para sofrir la pesadura delas armas \textbf{ e conuiene les de auer fuertes braços } para fazer fuertes colpes . & expedit eos habere magnos humeros et renes ad sustinendum armorum grauedinem , \textbf{ et habere fortia brachia ad faciendum percussiones fortes : } mulieres igitur \\\hline
3.1.12 & e conuiene les de auer fuertes braços \textbf{ para fazer fuertes colpes . } Et por que las mugers esto non pueden auer & expedit eos habere magnos humeros et renes ad sustinendum armorum grauedinem , \textbf{ et habere fortia brachia ad faciendum percussiones fortes : } mulieres igitur \\\hline
3.1.13 & por cada vn sennorio o por cada vn maestradgo o por cada vn ofiçio \textbf{ mas podemos mostrar por tons razones } que non es conuenible ala çibdat & siue pro qualibet praepositura . \textbf{ Possumus autem triplici via ostendere , } quod non sit expediens ciuitati semper praeponere eosdem in eisdem magistratibus . Prima via sumitur \\\hline
3.1.13 & que sienpre sean puestos vnos maestros \textbf{ e vnos efiçiales en essos mismos ofiçios ¶ } La primera razon se toma de parte dela magestad real & Possumus autem triplici via ostendere , \textbf{ quod non sit expediens ciuitati semper praeponere eosdem in eisdem magistratibus . Prima via sumitur } ex parte regiae maiestatis , \\\hline
3.1.13 & que esle un atada en alguna dignidat o en algun maestradgo o en algun poderio \textbf{ por la qual razon commo venga ala real magestad } e generalmente a qual quier que ha de dar & postquam ad aliquam praeposituram vel ad aliquem magistratum assumitur . \textbf{ Quare cum deceat regia maiestatem } et uniuersaliter omnem ciuem , \\\hline
3.1.13 & e partiendo los a departidas \textbf{ ꝑsonas faze a buen estado e paçifico dela çibdat e de los çibdadanos } assi commo dize elpho en el segundo libro delas politicas & Mutare autem aliquando magistratus et principatus , et distribuere eos diuersis personis , \textbf{ ut innuit Philosophus 2 Polit’ videtur facere ad quietum } et pacificum statum ciuium . \\\hline
3.1.13 & quando dizeque socrates \textbf{ sienpre fazie vnos mismos prinçipes } que era razon de discordia e de vanderia en la çibdat & Hanc autem tertiam rationem improbantem ordinationem Socraticam tangit Philosophus 2 Poli’ \textbf{ cum ait . Socrates semper facit eosdem Principes , } quod est seditionis causa \\\hline
3.1.14 & por ende bienes de contar sus o pimones \textbf{ non por razon de vana gloria } mas por que esto demanda esta arte presente & ideo bene se habet eorum opiniones tractare , \textbf{ non ostentationis causa , } sed quia hoc requirit praesens methodus , \\\hline
3.1.14 & nin es prouechoso \textbf{ que sienpre vnos ofiçialon sean puestos en essos mismos ofiçios } assi commo paresçia & ut Socrates statuebat ; nec esse decens , mulieres ordinari ad opera bellica ; nec esse utile , \textbf{ eosdem semper in eisdem magistratibus praefici , } ut Socrates statuisse videbatur . \\\hline
3.1.14 & que los lidiadores deuian ser apartados de los otros çibdadanos \textbf{ Lo segundo establesçie grant muchedunbre de lidiadores } ¶ & et distinctos ab aliis ciuibus . \textbf{ Secundo statuebat magnam multitudinem bellatorum . } Tertio ponebat eos in quodam determinato numero . \\\hline
3.1.14 & La segunda razon se prueua \textbf{ assy ca establesçer tan grant muchedunbre de lidiadores en cada vna delas çibdades } assi que sean çinco miłl omin lesto serie muy graue & Secunda via sic patet , \textbf{ nam constituere tantam multitudinem bellatorum in qualibet ciuitate } ut quinque milia , \\\hline
3.1.14 & assy ca establesçer tan grant muchedunbre de lidiadores en cada vna delas çibdades \textbf{ assi que sean çinco miłl omin lesto serie muy graue } e de grant carga alos çibdadanos & nam constituere tantam multitudinem bellatorum in qualibet ciuitate \textbf{ ut quinque milia , } vel etiam mille , \\\hline
3.1.14 & assi que sean çinco miłl omin lesto serie muy graue \textbf{ e de grant carga alos çibdadanos } por que grant carga serie & ut quinque milia , \textbf{ vel etiam mille , } esse valde difficile \\\hline
3.1.14 & e de grant carga alos çibdadanos \textbf{ por que grant carga serie } e graue cosa serie alos çibdadanos de vna çibdat & vel etiam mille , \textbf{ esse valde difficile } et onerosum ipsis ciuibus . Onerosum enim et difficile esset ciuibus unius ciuitatis sustentare mille viros in stipendiis communibus , \\\hline
3.1.14 & por que grant carga serie \textbf{ e graue cosa serie alos çibdadanos de vna çibdat } mantener mill caualleros de las rentas comunes de vna çibdat & esse valde difficile \textbf{ et onerosum ipsis ciuibus . Onerosum enim et difficile esset ciuibus unius ciuitatis sustentare mille viros in stipendiis communibus , } quorum nullum esset aliud officium , nisi bellare , \\\hline
3.1.14 & e graue cosa serie alos çibdadanos de vna çibdat \textbf{ mantener mill caualleros de las rentas comunes de vna çibdat } los quales caualleros non ouiessen otro ofiçio ninguno si non lidiar & esse valde difficile \textbf{ et onerosum ipsis ciuibus . Onerosum enim et difficile esset ciuibus unius ciuitatis sustentare mille viros in stipendiis communibus , } quorum nullum esset aliud officium , nisi bellare , \\\hline
3.1.14 & quando fuesse me este \textbf{ Et muy mayor carga } e peor de sofrir serie & quorum nullum esset aliud officium , nisi bellare , \textbf{ cum adesset oportunitas : } et onerosius et quasi omnino importabile esset sustentare sic quinque milia : \\\hline
3.1.14 & quantas quisiesse a ssu uoluntad \textbf{ por que pudiesse de las rentas comunes abondar atanta muchedunbre } por la qual cosa el philosofo en el libro delas politicas reprehende a socrates deste tal gouernamiento de çibdat & oporteret enim ciuitatem illam habere possessiones quasi ad votum , \textbf{ ut posset ex communibus sumptibus tantam multitudinem pascere . } Propter quod Philosophus 2 Politicorum reprehendens Socratem de huiusmodi ordine ciuitatis , \\\hline
3.1.14 & en la qual por auentura \textbf{ por la grant muchedunbre dela tierra desierta } ay grant espaçio de tierras & quod oportet ciuitatem illam sic institutam esse in regione Babilonica , \textbf{ ubi forte propter magnitudinem desertorum est magnum spatium terrarum , } ex quo posset magna multitudo pasci . \\\hline
3.1.14 & por la grant muchedunbre dela tierra desierta \textbf{ ay grant espaçio de tierras } de que se podria gouernar grant muchedunbre de gente & quod oportet ciuitatem illam sic institutam esse in regione Babilonica , \textbf{ ubi forte propter magnitudinem desertorum est magnum spatium terrarum , } ex quo posset magna multitudo pasci . \\\hline
3.1.14 & ay grant espaçio de tierras \textbf{ de que se podria gouernar grant muchedunbre de gente } ca guaue cosa es & ubi forte propter magnitudinem desertorum est magnum spatium terrarum , \textbf{ ex quo posset magna multitudo pasci . } Difficile est enim , \\\hline
3.1.14 & de que se podria gouernar grant muchedunbre de gente \textbf{ ca guaue cosa es } assi commo dize el philosofo gouernar atanta muchedunbre de lidiadores sin çibdadanos e sin mugers e sin siruientes e sin fijos & ex quo posset magna multitudo pasci . \textbf{ Difficile est enim , } ut Philosophus innuit , \\\hline
3.1.14 & ca guaue cosa es \textbf{ assi commo dize el philosofo gouernar atanta muchedunbre de lidiadores sin çibdadanos e sin mugers e sin siruientes e sin fijos } Lo terçero erraua socrates en el ordenamiento dela çibdat & Difficile est enim , \textbf{ ut Philosophus innuit , | praeter ipsos ciues } et praeter mulieres , \\\hline
3.1.14 & conuiene a saber . a los çibdadanos \textbf{ ca si los çibdadanos fuessen de flacos coraçones } e sin prouech para lidiar & ø \\\hline
3.1.14 & e sin prouech para lidiar \textbf{ e para se defender aurian meester mayor conpanna delidiadores } que los pudiessen ayudar & ad tria deberet respicere . \textbf{ Primo ad ciues : } nam si ciues ipsi essent pusillanimes et inutiles ad bellum , indigeret maiori copia bellatorum sic se habentium . Secundo inspiciendum esset ad regionem : \\\hline
3.1.14 & Lo segundo deuen tener mientes al regno \textbf{ ca quanto la çibdat esta en mayor regnado } e vsa de mayor espaçio de tierras & nam si ciues ipsi essent pusillanimes et inutiles ad bellum , indigeret maiori copia bellatorum sic se habentium . Secundo inspiciendum esset ad regionem : \textbf{ nam quanto ciuitas illa maiori regione } et maiori terrarum spatio potiretur , \\\hline
3.1.14 & ca quanto la çibdat esta en mayor regnado \textbf{ e vsa de mayor espaçio de tierras } tanto mayor cuento de lidiadores pueden mantener ¶ & nam quanto ciuitas illa maiori regione \textbf{ et maiori terrarum spatio potiretur , } tanto sustentare posset maiorem numerum bellantium . \\\hline
3.1.14 & e vsa de mayor espaçio de tierras \textbf{ tanto mayor cuento de lidiadores pueden mantener ¶ } Lo terçero deuen tener mientes alos logares & et maiori terrarum spatio potiretur , \textbf{ tanto sustentare posset maiorem numerum bellantium . } Tertio aspiciendum esset \\\hline
3.1.14 & quel son uezinos assi commo si aquella çibdat ouiesse çerca \textbf{ de ssi uezinos amigos o enemigos flacos de coraçon o homillosos } Ca departidas las condiçonnes de los uezinos & ut utrum ciuitas illa haberet \textbf{ circa se vicinos amicos | vel inimicos , } pusillanimes vel viriles . \\\hline
3.1.15 & assy commo assi mesmo . \textbf{ En essa misma manera deue amar los fijos e las mugers e las possessiones de los otros çibdadanos } assi commo las suyas propreas . & sicut seipsum : \textbf{ sic debent diligere uxores , | filios , } et possessiones aliorum , \\\hline
3.1.15 & quanto ala comunidat de los çibdadanos \textbf{ En essa misma manera podemos saluar el su dicho } del & quantum ad communitatem ciuium : \textbf{ sic etiam saluare possumus dictum eius } quantum ad unitatem ciuitatis . \\\hline
3.1.15 & que son demandadas para conplimiento dela uida \textbf{ assi que non fuessen en la çibdat muchͣs casas e departidas } assi commo demanda la mengua delas cosas & nec de unitate eorum quae requiruntur ad sufficientiam vitae , \textbf{ ut quod non essent in ciuitate plura } et diuersa prout requirit indigentia vitae . \\\hline
3.1.15 & assi es poniendo la entençio de socrates dela comunidat delas cosas \textbf{ e dela vnidat de los çibdadanos uerdat es } lo que el cuydaua e ymaginaua & et de unitate ciuium , \textbf{ verum est quod ipse opinabatur , } quod in ciuitate esset maxima pax , et non orirentur ibi litigia . \\\hline
3.1.15 & lo que el cuydaua e ymaginaua \textbf{ que en la çibdat seria grant paz } e non nasçerian y contiendas nin uaraias & verum est quod ipse opinabatur , \textbf{ quod in ciuitate esset maxima pax , et non orirentur ibi litigia . } Nam proprius amoris effectus , \\\hline
3.1.15 & que las mugers deuian batallar \textbf{ ca muchͣs uegadas } contesçio esto çerca las partes de ytalia & quod et mulieres bellare oporteret . \textbf{ Multotiens autem circa partes Italiae hoc contigit , } quod viris deserentibus ciuitatem , \\\hline
3.1.15 & por auentura \textbf{ por los batalladores entendia nobles omes } alos quales non parte nesçia de obrar ningunan otra cosa con sus manos & volens ciuitatem ad minus mille continere bellatores . \textbf{ Forte per bellatores intendebat nobiles , } quorum non est manibus operari . \\\hline
3.1.15 & si a lo de menos non ouiesse en ella minłłomes nobles los quales llamaua \textbf{ por grant excellençia batalladores } a socrates e sin platon fue otro philosofo & et eorum maxime est vacare circa armorum industriam . Volebat ergo Socrates politiam aliquam non debere nominari ciuitatem , nisi saltem contineret mille nobiles , \textbf{ quos per quandam excellentiam vocabat bellatores . } Praeter Socratem \\\hline
3.1.16 & ca creya que estonçe seria la çibdat muy bien ordenada \textbf{ si ninguno de los çibdadanos non ouiesse mayores rentas o mayores possessiones } que el otro & Volebat enim tunc esse ciuitatem optime ordinatam , \textbf{ si nullus ciuium haberet plures redditus , | vel maiores possessiones , } quam alter : \\\hline
3.1.16 & que se dan en los casamientos \textbf{ ca establesçiendo que los pobrescasen con las ricas e en casamiento resçiban grandes arras } e ellos que non den nada de lo suyo . & cum diuitibus : \textbf{ et in contrahendo accipiant dotes , } et non dent pauperes ergo accipiendo magnas dotes a diuitibus poterunt aequari eis in possessionibus . \\\hline
3.1.16 & por las possessiones \textbf{ por que comunalmente son muy codiçiosos para auer grandes rentas e grandes possessiones } por la qual cosa nasçen entre ellos muchͣs contiendas e muchͣs uaraias . & Nam ciues valde litigant pro possessionibus : \textbf{ sunt enim communiter nimis cupidi ad habendum redditus , | et possessiones ; } unde et propter hoc multa litigia oriuntur . \\\hline
3.1.16 & por que comunalmente son muy codiçiosos para auer grandes rentas e grandes possessiones \textbf{ por la qual cosa nasçen entre ellos muchͣs contiendas e muchͣs uaraias . } Et por ende si los çibdadanos ouiessen possessiones eguales & et possessiones ; \textbf{ unde et propter hoc multa litigia oriuntur . } Quare si ciues haberent possessiones aequales , \\\hline
3.1.16 & que delas arras fuesse ygualado en las possessiones con los otros çibdadanos . \textbf{ En essa misma manera el que perdiesse el pleito non perderia mucho } por que tomado las arras & quia oporteret ipsum per dationem dotium aequari in possessionibus aliis ciuibus . Sic etiam perdens placitum , \textbf{ non multum amitteret : } quia accipiendo dotes , aequaretur aliis in diuitiis , \\\hline
3.1.16 & que el veya enlas otras çibdades \textbf{ ca veya en muchͣs çibdades bien ordenadas } que auyan grant cura los fazedores dela ley delas possessiones delos çibdadanos . & ex his quae videbat in ciuitatibus aliis : \textbf{ nam in multis politiis bene ordinatis } magna cura fuit legislatoribus \\\hline
3.1.16 & ca veya en muchͣs çibdades bien ordenadas \textbf{ que auyan grant cura los fazedores dela ley delas possessiones delos çibdadanos . } Et por ende felleas por auentura fue mouido & nam in multis politiis bene ordinatis \textbf{ magna cura fuit legislatoribus | de possessionibus ciuium : } ideo Phalas forte motus \\\hline
3.1.16 & por estas cosas \textbf{ que veya en las otrå sçibdades . } Et por ende establesçio & ideo Phalas forte motus \textbf{ ex his quae videbat in politiis aliis , } statuit potissime curandum esse de possessionibus ciuium , \\\hline
3.1.16 & Et por ende establesçio \textbf{ que much deuia ser tomado grant cuydado delas possessiones de los çibdadanos } e quaria que los çibdadanos ouiessen las possessiones eguales & ex his quae videbat in politiis aliis , \textbf{ statuit potissime curandum esse de possessionibus ciuium , } volens eos aequatas possessiones habere . \\\hline
3.1.17 & que todos los çibdadanos ayan ygual cuento de fijos \textbf{ por la qual cosa se muestra manifiesta miente de parte dela generaçion de los fijos } quela ley puesta por felleas non es conuenible & statuere in ciuitate omnes ciues habere aequalem numerum filiorum . \textbf{ Propter quod ex parte procreationis prolis manifeste ostenditur praedictam } legem non esse congruentem ; \\\hline
3.1.17 & que faze la ley \textbf{ ca pertenesçe al prinçipe de auer grant cuydado } que los çibdadanos non sean bulliçiosos ni turbadores dela paz & quas legislator summo studio cauere debet . \textbf{ Spectat enim ad principem , | omnem } curam habere \\\hline
3.1.17 & enlas possessiones \textbf{ tomando grandes dones } e arras los pobres de los ricos e non las dando & sed si in ciuitate statuitur pauperes aequari diuitibus in possessionibus , \textbf{ accipiendo magnas dotes ab eis , } et non dando dotes illis , contingit multotiens diuites fieri pauperes , \\\hline
3.1.17 & Lo primero quando los pobres se fazen ricos \textbf{ non saben sofrir la su buena uentura } assi commo el philosofo muestra llanamente en el segundo libro de la rectoriça & et iurgia in ciuitate . \textbf{ Primo quia pauperes cum ditantur nesciunt fortunas ferre , } ut plane ostendit Philosop’ 2 Rhet’ . Iniuriabuntur ergo aliis . Filii enim pauperum inflati , \\\hline
3.1.17 & ca assi commo dize el philosofo \textbf{ en el segundo libro delas politicas meesteres ala pazer dela çibdat } que los fijos de los ricos non sean sobuios & sed etiam ex parte filiorum diuitum : \textbf{ quia ut dicitur 2 Polit’ } opus est ad pacem ciuitatis filios diuitum non esse insolentes . \\\hline
3.1.17 & ca porque los fijnos de los ricos \textbf{ por la mayor parte son magnanimos } e de grant coraçon & Quare cum filii diuitum \textbf{ ut plurimum sint magnanimi } et magni cordis \\\hline
3.1.17 & por la mayor parte son magnanimos \textbf{ e de grant coraçon } si uieren que son despreçiados & ut plurimum sint magnanimi \textbf{ et magni cordis } si videant \\\hline
3.1.17 & e por ende non es bien dicho \textbf{ que a buen gouernamiento dela çibdat } cunple de ser las possessiones egualadas & decet enim ipsos esse liberales et temperatos : \textbf{ non ergo bene dictum est quod ad bonum regimen } ciuitatis \\\hline
3.1.17 & cunple de ser las possessiones egualadas \textbf{ si algua cosa non fue determinada dela cantidat de aquellas possessiones } por que podrian los çibdadanos auer tan pocas possessiones & sufficit ciues habere possessiones aequatas , \textbf{ nisi aliquid determinetur de quantitate possessionum illarum : } possent enim ciues adeo modicas possessiones habere , \\\hline
3.1.18 & muchͣs cosas çerca las possessiones de los çibdadanos . \textbf{ En essa misma manera avn en aquella tierra } que dizen lotros & et doctus , \textbf{ multas statuit possessiones ciuium . Sic etiam apud Locros , } ut recitat Philosophus 2 Politic’ \\\hline
3.1.18 & no non conuerma a ninguon de uender sus possessiones \textbf{ si non podiessen mostrar conplidamente qual contesçiera alguna grant ocasion o algua mala uentura } por que son muchͣs cosas de determinar & nulli licebat possessiones vendere , \textbf{ nisi posset sufficienter ei ostendere aliquod magnum infortunium accidisse . Sunt enim multa determinanda in legibus circa possessiones , } sed non expedit hoc lege statui circa ipsas , \\\hline
3.1.18 & si non podiessen mostrar conplidamente qual contesçiera alguna grant ocasion o algua mala uentura \textbf{ por que son muchͣs cosas de determinar } en las leyes çerca las possession s & nulli licebat possessiones vendere , \textbf{ nisi posset sufficienter ei ostendere aliquod magnum infortunium accidisse . Sunt enim multa determinanda in legibus circa possessiones , } sed non expedit hoc lege statui circa ipsas , \\\hline
3.1.18 & mas quanto alo presente cunple de saber \textbf{ que la prinçipal entençion del que faze la ley non deue ser } en mesurarlas possessiones de fuera & Ad praesens autem scire sufficiat , \textbf{ principalem intentionem legislatoris non debere esse circa possessiones exteriores mensurandas . } Quod triplici via venari possumus . \\\hline
3.1.18 & e las possessiones de fuera \textbf{ por la mayor parte uaraian } por la sustançia de fuera enpero lastex pleonas honrradas & eo quod minus egent exteriori substantia , \textbf{ ut plurimum litigant propter exteriorum substantiam , personae } tamen nobiles et gratiosae magis turbantur \\\hline
3.1.18 & e tanto mas prinçipalmente deue entender en partir las honrras \textbf{ quanto las peleas entre las perssonas honrradas son de mayor periglo } que entre las otras personas & Et tanto principalius debet hoc intendere circa honores , \textbf{ quanto litigia } inter personas honorabiles sunt magis detestanda , quam inter personas alias . Secunda via ad inuestigandum hoc idem sumitur \\\hline
3.1.18 & para prouar esto mismo se toma de parte delas delecta connes \textbf{ las quales los omes siguen en la mayor parte } por que los omes non solamente fazen tuertos e inuirias & ex parte delectationum , \textbf{ quas homines ut plurimum insequuntur . Non enim homines solum iniustificant , } et iniurias inferunt in res exteriores propter auaritiam : \\\hline
3.1.18 & lo que suyo es \textbf{ Mas conuiene les de ordenar muchͣs cosas } para repremir los desseos desordenados & ut quod quilibet quod suum est possideat : \textbf{ sed multa ordinare decet circa possessiones reprimendas , } ne ciues sint intemperati , \\\hline
3.1.18 & e por ende fazen tuertos alos otros \textbf{ non por conplir grant mengua } assi commo por tirar fanbre o por tirar frio dessi . & Volunt enim homines gaudere delectationibus absque tristitiis , \textbf{ ideo iniuriantur aliis in se non solum propter supplendam indigentiam , } ut propter tollendam famem , \\\hline
3.1.19 & que aquellas cosas que el establesçio cerca este tal gouernamiento \textbf{ que pueden ser reduzidas a seys cosas . } Ca primero establesçio algunas cosas & ad regimen ciuium . \textbf{ Videntur autem quasi ad sex reduci quae statuebat circa huiusmodi regimen . Primo enim statuit quaedam pertinentia ad multitudinem et distinctionem ciuium . } Secundo determinat de distinctione possessionum . \\\hline
3.1.19 & Lo quinto dela manera de iudgar ¶ Lo sesto e lo postrimo establesçio algunas leyes \textbf{ que tannian alguons linages de personas dezimos } que y podo mio establesciendo su poliçia & Quarto de distinctione iudicantium . \textbf{ Quinto de modo iudicandi . Sexto et ultimo statuit quasdam leges tangentes diuersa genera personarum . Hippodamus autem statuens suam politiam , } primo intromisit se de multitudine \\\hline
3.1.19 & e dizia \textbf{ que sia muy buena quantidat de çibdadanos } si fuessen fasta diez minl uarones & et distinctione ciuium , dicens \textbf{ quod optima quantitas ciuium est circa decem millia virorum . } Hanc autem quantitatem distinxit in tres partes , \\\hline
3.1.19 & por la calentura natural \textbf{ que consume el humido radical } Lo segundo ha mester casa & videlicet victu , potu , \textbf{ et cibo propter calorem naturalem consumentem huiusmodi radicale . } Secundo indiget domo , et vestitu , \\\hline
3.1.19 & Enpero dizie \textbf{ que algunte rrectorio deuie ser comun } del qual deuian beuir los lidiadores & sed solum tribuebat eis arma . \textbf{ Dicebat autem debere esse aliquod territorium commune , de quo bellatores viuerent } quasi de communi aerario . \\\hline
3.1.19 & en el segundo libro delas politicas \textbf{ deuiese de vieios sabios escogidos } assi que si los pleitos fueren mal iudgados en la audiençia ordinaria & ut narrat Philosophus 2 Politicorum , \textbf{ debere esse ex senibus electis : } ut si causae male iudicatae fuerunt in praetorio ordinario , \\\hline
3.1.19 & aqueͣllos uieios e sabios \textbf{ e de buen testimo mio iudgas son } e enderesçassen aquellos pleytos & ut si causae male iudicatae fuerunt in praetorio ordinario , \textbf{ senes illi electi } et boni testimonii rectificant ipsas . Quinto intromisit se de modo iudicandi ; \\\hline
3.1.19 & qunato alos sabios establesçio \textbf{ que qual quier que fallasse algua cosa } que fuesse buena e conueinble ala çibdat & Quantum ad sapientes statuit , \textbf{ quod quicunque inueniret aliquid , } quod esset expediens ciuitati , \\\hline
3.1.19 & que fuesse buena e conueinble ala çibdat \textbf{ qual feziessen grant honrra } por ello¶ & quod esset expediens ciuitati , \textbf{ quod retribueretur ei debitus honor . } Quantum ad bellatores ordinauit , \\\hline
3.1.19 & que non auian poder \textbf{ nin podian por si mismas guardar su derechca parte nesçe al Rey e al prinçipeque deue ser guardador dela iustiçia de auer cuydado espeçial delas cosas comunes } e delos pelegninos & de peregrinis , et de orphanis . Appellabat autem orphanos uniuersaliter omnes personas impotentes , \textbf{ non valentes per se ipsas sua iura conquirere . Spectat enim ad Regem et Principem , } qui debet esse custos iusti , \\\hline
3.1.20 & por que non pueden defender su derecho \textbf{ uchos bienes se nos siguen delas opiniones de los phos antigos } por que ellos en sus dichos dixieron muchsbieño e uerdaderas cosas . & cum non possint defendere iura sua . \textbf{ Multa bona consequimur | ex opinionibus antiquorum Philosophorum , } eo quod ipsi in suis dictis multa bona \\\hline
3.1.20 & uchos bienes se nos siguen delas opiniones de los phos antigos \textbf{ por que ellos en sus dichos dixieron muchsbieño e uerdaderas cosas . } Enpero puesto que non dixiessen algua cosa uerdadera & ex opinionibus antiquorum Philosophorum , \textbf{ eo quod ipsi in suis dictis multa bona | et vera dixerunt . } Dato tamen quod nihil veri dixissent , \\\hline
3.1.20 & que ellos dizen algunas uezes abuian el nuestro entendimiento \textbf{ ca los malos dichos de los otros muchͣs ueze ᷤ despiertan el entendimiento } por que iudgue derechamente e por ende contamos la opinion de ipodomio & excitant enim talia aliquando intellectum : \textbf{ mala quidem aliorum dicta multotiens intellectum excitant , } ut recte iudicet . Hippodami ergo opinionem recitauimus , \\\hline
3.1.20 & por que iudgue derechamente e por ende contamos la opinion de ipodomio \textbf{ por que el en la su poliçia manifesto muchͣs bueanssmans . } Empo algunas cosas establesçio non conuenible mente . & ut recte iudicet . Hippodami ergo opinionem recitauimus , \textbf{ quia in sua politia multas bonas sententias promulgauit : aliqua } tamen incongrue statuit . Possumus autem \\\hline
3.1.20 & por que el en la su poliçia manifesto muchͣs bueanssmans . \textbf{ Empo algunas cosas establesçio non conuenible mente . } Et por ende podemos & ut recte iudicet . Hippodami ergo opinionem recitauimus , \textbf{ quia in sua politia multas bonas sententias promulgauit : aliqua } tamen incongrue statuit . Possumus autem \\\hline
3.1.20 & ca el establesçimiento del departimiento de los çibdadanos non puede estar \textbf{ con el establesçimiento dela elecçion oł prinçipe } por que si la çibdat & Nam statutum de distinctione ciuium stare non poterat \textbf{ cum statuto de electione principis . } Si enim ciuitas \\\hline
3.1.20 & Et segunt esto conuiene \textbf{ que los lidiadores ouiessen mayor poderio que los menestrales nin los labradores todos en vno . } Et por que los mas poderosos sienpre quieren ser senors en las çibdades & et iurgia facta in ciuitate remouere , et patriam ab hostibus defendere : \textbf{ oportebat bellatores habere maiorem potentiam , | quam agricolae , } et artifices simul sumpti : \\\hline
3.1.20 & Lo primero \textbf{ que desta manera de iudgar se sigue mayor discordia } e mayor desaguisado de los uiezes & hoc autem duplici via improbari potest . Primo , \textbf{ quia ex isto modo iudicandi sequitur maior peruersio } et maior degeneratio iudicum . \\\hline
3.1.20 & que desta manera de iudgar se sigue mayor discordia \textbf{ e mayor desaguisado de los uiezes } por que mas ayna se trastornan los uezes & quia ex isto modo iudicandi sequitur maior peruersio \textbf{ et maior degeneratio iudicum . } Nam citius peruertuntur iudices , \\\hline
3.1.20 & Onde el phoda a entender \textbf{ que en algunos buenos ordenamientos de çibdat el contra no es establesçido delo } que ordeno ypodomio & citius degenerabunt hoc modo quam aliter . Unde et Philosophus innuit , \textbf{ quod in quibusdam bonis politiis econtrario statuitur quam ordinauerit Hippodamus : } ubi ordinantur iudices posse loqui sibi inuicem publice , \\\hline
3.1.20 & e non puedan auer conseio vno con otro en ascondido . \textbf{ Otrossi fallesçe la dichͣ manera } por que en alguons iuyzios conuiene de auer fablas los iiezes vno con otro & non tamen posse ad inuicem habere consilium in priuato . \textbf{ Rursus deficit dictus modus , } quia in aliquibus iudiciis oportet collationem habere ad inuicem ; \\\hline
3.1.20 & Otrossi fallesçe la dichͣ manera \textbf{ por que en alguons iuyzios conuiene de auer fablas los iiezes vno con otro } assi que si los uiezes descordassen en publico & Rursus deficit dictus modus , \textbf{ quia in aliquibus iudiciis oportet collationem habere ad inuicem ; } ut si iudices discordarent , \\\hline
3.1.20 & para la çibdat deue \textbf{ por ende resçebir grant honrra } por ende se esforcarian los sabios de fallar nueuas leyes & deberet \textbf{ ex hoc debitum honorem accipere ; conarentur sapientes ad inueniendum nouas leges , } et ad ostendendum nouas leges inuentas esse proficuas ciuitati : \\\hline
3.1.20 & por ende resçebir grant honrra \textbf{ por ende se esforcarian los sabios de fallar nueuas leyes } para mostrar que las leys nueuas & deberet \textbf{ ex hoc debitum honorem accipere ; conarentur sapientes ad inueniendum nouas leges , } et ad ostendendum nouas leges inuentas esse proficuas ciuitati : \\\hline
3.1.20 & la qual cosa seria muy dannosa e muy peligrosa ala çibdat \textbf{ por que las leyes han grant fuerça } por alongamiento de tp̃o & quod est valde in ciuitate periculosum , \textbf{ quia leges assidue magnam efficaciam habent ex diuturnitate temporis . } Quare si assidue innouentur leges , \\\hline
3.2.1 & es pues que con el ayuda de dios cunpliemos la primera parte deste terçero libro \textbf{ ante pomiendo alguons preanbulos al nuestro proponimiento } e rezando opiniones de departidos philosofos & primam partem huius tertii libri , \textbf{ praemittendo quaedam praeambula ad propositum , } et recitando opinionem diuersorum Philosophorum instituentium politiam , \\\hline
3.2.1 & Mas el philosofo en el terçero libro delas politicas \textbf{ tanne quatro cosas } que son de penssar & quae et quot consideranda sunt in tali regimine . \textbf{ Videtur autem Philos’ 3 Polit’ tangere , } quatuor quae consideranda sunt in regimine ciuitatis . \\\hline
3.2.1 & que son de penssar \textbf{ enł gouernamiento del regno et dela çibdat } e estas son ¶ El prinçipe Et el conseio . & Videtur autem Philos’ 3 Polit’ tangere , \textbf{ quatuor quae consideranda sunt in regimine ciuitatis . } Haec autem sunt princeps , \\\hline
3.2.1 & por dos razones \textbf{ que destas quatro cosas dichͣ sauemos de tractar enł gouernamiento } por el qual es de gouernar la çibdat en el tp̃o dela guerra ¶ & consilium , praetorium , \textbf{ et populus . Possumus autem ex ipsis legibus quibus regenda est ciuitas tempore pacis duplici via inuestigare , } quod de praedictis quatuor considerandum est in regimine quo regenda est ciuitas tempore pacis . Prima via sumitur ex his quae requiruntur ad legem : \\\hline
3.2.1 & assi ca para esto que por las lepes sea auido gouernamiento derech \textbf{ quanto ꝑtenesçe alo presente quatro cosas son menester . } lo primero que las leyes sean bien falladas & Nam ad hoc quod per leges recta gubernatio habeatur , \textbf{ quantum ad praesens spectat , | quatuor requiruntur . } Primo ut leges per sapientiam sint bene inuentae . \\\hline
3.2.1 & Mas guardar bien las leyes \textbf{ por el poderio çiuil esto parte nesçe alos prinçipes onde el philosofo enłqnto libbo delas ethicas dize } que el prinçipe es guardador dela iustiçia & in legibus sint debite obseruata . Bene autem custodire leges \textbf{ per ciuilem potentiam spectat ad principem . | Unde } et Philosophus 5 Ethicor’ ait , quod princeps esse debet custos iusti , \\\hline
3.2.1 & que son establesçidas \textbf{ ca los establesçimientos delas leyes pueden ser dichͣs leyes } mas fallar bien las leyes pertenesçe al conseio & idest iustarum legum et statutorum : \textbf{ nam statuta ipsa legalia leges quaedam dici possunt . } Bene quidem inuenire leges per sapientiam spectat ad consilium : \\\hline
3.2.1 & que la çibdat sea gouernada entp̃o dela paz \textbf{ por las leyes conuiene de fazer tractado destas quatro cosas } sobredichͣs en este gouernamiento ¶ & per leges bene gubernetur ciuitas , \textbf{ oportet in huiusmodi regimine de praedictis quatuor considerationem facere . } Secunda via ad inuestigandum hoc idem sumitur ex fine qui in legibus intenditur : \\\hline
3.2.1 & por las leyes \textbf{ para fazer buenas obras } por que en las leyes son mandadas las cosas & quae potest respicere totum populum : populus enim ad bene agendum , \textbf{ per leges maxime inducendus est , } eo quod ipsis legibus praecipiuntur laudabilia , \\\hline
3.2.1 & e fuya lo malo e lo feo e lo de denostar . \textbf{ Pues que assi es de todas estas quatro cosas } que dichas son conuiene desaƀ del prinçipado del conseio del alcalłia del pueble . & et laudabile et fugiat vituperabile et turpe . \textbf{ De omnibus ergo his quatuor , videlicet de principatu , consilio , } praetorio , et populo , \\\hline
3.2.2 & el terçero libro delas politicas \textbf{ departe el philosofo seys linaies de prinçipados } de los quales tro son buenos & primo tamen dicemus de ipso principatu . \textbf{ Tertio Politicorum distinguit Philosophus sex modos principantium , } quorum tres sunt boni , \\\hline
3.2.2 & que quiere dezer sennorio de buenos \textbf{ e la poliçia que quiere dezer pueblo bien enssenoreante son bueons prinçipados . } La thirama que quiere dezer sennorio malo & Nam regnum aristocratia , \textbf{ et politia sunt principatus boni : } tyrannides , \\\hline
3.2.2 & e la obligaçia que quiere dezer sennorio duro . \textbf{ Et la democraçia que quiere dez maldat del pueblo } enssennoreante son malos prinçipados . & tyrannides , \textbf{ oligarchia , } et democratia sunt mali . \\\hline
3.2.2 & Et la democraçia que quiere dez maldat del pueblo \textbf{ enssennoreante son malos prinçipados . } Et alli muestra el pho departir el buen prinçipado del malo . & oligarchia , \textbf{ et democratia sunt mali . } Docet enim idem ibidem discernere bonum principatum a malo . \\\hline
3.2.2 & enssennoreante son malos prinçipados . \textbf{ Et alli muestra el pho departir el buen prinçipado del malo . } ca si en algun señorio o prinçipado es entendido el bien comun & et democratia sunt mali . \textbf{ Docet enim idem ibidem discernere bonum principatum a malo . } Nam si in aliquo dominio \\\hline
3.2.2 & quando por el bien propreo enssennorea vno \textbf{ e este es dich tiranno . Enpero contesçe alguas uezes } que alguna çibdat non es gouernada por vn sennor & ut cum propter bonum priuatum principatur Tyrannus . \textbf{ Contingit | tamen aliquando ciuitatem } aliquam non regi uno aliquo dominante , \\\hline
3.2.2 & por algers pocos uarones \textbf{ e de buen testimo mio eran escogidos doze uarones prouados e buenos los quales llaman } que los doze omes buenos & regebatur totus Romanus populus quibusdam paucis viris : \textbf{ eligebantur enim duodecim viri approbati et boni testimonii , } quos vocabant duodecim bonos homines ; \\\hline
3.2.2 & e de buen testimo mio eran escogidos doze uarones prouados e buenos los quales llaman \textbf{ que los doze omes buenos } assi commo agora dizen en françialos doze pares & eligebantur enim duodecim viri approbati et boni testimonii , \textbf{ quos vocabant duodecim bonos homines ; } et horum gubernatione tota ciuitas regebatur : \\\hline
3.2.2 & que los doze omes buenos \textbf{ assi commo agora dizen en françialos doze pares } e por el gouernamiento destos se gouernaua toda la çibdat . & eligebantur enim duodecim viri approbati et boni testimonii , \textbf{ quos vocabant duodecim bonos homines ; } et horum gubernatione tota ciuitas regebatur : \\\hline
3.2.2 & Mas de otro dizian \textbf{ que le fezierentrro de los buenos omes } Et por ende si se gouernare la çibdat & ø \\\hline
3.2.2 & e tal prinçipado es dicħa ristrocaçia \textbf{ que quiere dezer prinçipado de buenos omes e uir̉tuosos } e dende vienen & et tunc talis principatus dicitur Aristocratia , \textbf{ quod idem est quod principatus bonorum et virtuosorum . Inde autem venit } ut maiores in populo , \\\hline
3.2.2 & que son en ella \textbf{ e prinçipalmente quanto al grant prinçipado que enssennorea a todos los otros } ca la poliçia esta mayormente en el ordenamiento del grant prinçipado & qui sunt in ea , \textbf{ et principaliter | quantum ad maximum principatum } qui dominantur omnibus aliis . Politia enim consistit maxime in ordine summi principatus , \\\hline
3.2.2 & e prinçipalmente quanto al grant prinçipado que enssennorea a todos los otros \textbf{ ca la poliçia esta mayormente en el ordenamiento del grant prinçipado } que es en la çibdat & quantum ad maximum principatum \textbf{ qui dominantur omnibus aliis . Politia enim consistit maxime in ordine summi principatus , } qui est in ciuitate . \\\hline
3.2.2 & que es en la çibdat \textbf{ Pues que assi es todo ordenamiento de çibdat puede ser dicħ poliçia . } Enpero el prinçipado del pueblo & qui est in ciuitate . \textbf{ Omnis ergo ordinatio , | ciuitatis Politia dici potest . } Principatus tamen populi \\\hline
3.2.3 & ir que entendemos mostrar \textbf{ qual es la muy buena poliçia } e que cosa es el muy buen prinçipado despues & et qui peruersi . \textbf{ Quia intendimus ostendere quae sit optima politia , } et quis sit optimus principatus . \\\hline
3.2.3 & qual es la muy buena poliçia \textbf{ e que cosa es el muy buen prinçipado despues } que deptimos las maneras de los prinçipados e de los señorios & Quia intendimus ostendere quae sit optima politia , \textbf{ et quis sit optimus principatus . } Postquam distinximus modus principandi , \\\hline
3.2.3 & e mostramos quales dellos son buenos \textbf{ e quales son malos finca de demostrar entre los prinçipados derechs e buenos } qualdlleros es el meior & qui illorum sunt recti , \textbf{ et qui peruersi : | restat ostendere } inter principatus rectos quis sit melior . \\\hline
3.2.3 & qualdlleros es el meior \textbf{ e quatro razones son } que prue una señalladamente & inter principatus rectos quis sit melior . \textbf{ Assignantur autem communiter quatuor viae } quod regnum est optimus principatus , \\\hline
3.2.3 & que prue una señalladamente \textbf{ que el regno es muy buen prinçipado } e que meior cosa es & Assignantur autem communiter quatuor viae \textbf{ quod regnum est optimus principatus , } et quod melius est ciuitatem aliquam , \\\hline
3.2.3 & que el regno es muy buen prinçipado \textbf{ e que meior cosa es } que alguna çibdat o alguna prouiçia sea gouernada & quod regnum est optimus principatus , \textbf{ et quod melius est ciuitatem aliquam , } siue prouinciam regi uno , \\\hline
3.2.3 & antes por ende siguese \textbf{ que si fuere vno lo lo el que enssennoreaua aura mayor paz en la çibdat } ca non se puede turbar la paz tan de ligero & si plures principes sint quam unum : \textbf{ ergo si solus unus principaretur | inter eos , } non sic de facili turbari posset pax ipsorum ciuium . \\\hline
3.2.3 & si todas las fuerças \textbf{ que son en muchs tiradores fuessen ayuntadas en vn tirador } por que aquel mas ayuntadamente tiraria & Immo si omnes vires , \textbf{ quae sunt in pluribus trahentibus , } congregarentur in uno , \\\hline
3.2.3 & Et aquel prinçipe \textbf{ por ma . yor cunplimiento de poderio meior podria gouernar la çibdat } que muchos ¶ & quae est in pluribus principantibus , congregaretur in uno Principe , efficacior esset ; \textbf{ et ille principans propter abundantiorem potentiam melius posset politiam gubernare . } Tertia via sumitur ex his quae videmus in natura . \\\hline
3.2.3 & mas contie ne al cuerpo \textbf{ que non el cuerpo al alma . En essa misma manera avn vn cuerpo çelestial } assi commo es el primero çielo es aquello & quod anima magis continet corpus , \textbf{ quam econuerso . Sic etiam unum caeleste corpus primum mobile , } est illud per \\\hline
3.2.4 & que se puede prouar \textbf{ que meinor cosa es } que la çibdat o la prounçia se gouienne & per quas probari videtur , \textbf{ quod melius sit ciuitatem } aut prouinciam regi pluribus , \\\hline
3.2.4 & que por vno ca paresçe \textbf{ que en el prinçipe son tr̃o cosas neçessarias } para que bien gouierne su pueblo & quam uno . \textbf{ Videntur enim in Principe tria esse necessaria , } ut bene regat populum sibi commissum . Primo enim debet habere perspicacem rationem . Secundo rectam intentionem . \\\hline
3.2.4 & Lo terçero firmeza acabada . \textbf{ Et destas trs cosas se pueden tomar tres razons } por las quales podemos prouar & Tertio perfectam stabilitatem . \textbf{ Ex his autem sumi possunt tres viae , } ex quibus venari possumus , \\\hline
3.2.4 & por las quales podemos prouar \textbf{ que buena cosa es } que muchos lean los prinçipes & ex quibus venari possumus , \textbf{ quod bonum sit principari plures , } et non unum tantum . Prima via sic patet . \\\hline
3.2.4 & e non vno solo ¶ La primera razon paresçe \textbf{ assi ca muchͣs oios } mas veen que vno et muchͣs manos & quod bonum sit principari plures , \textbf{ et non unum tantum . Prima via sic patet . } Nam plures oculi plus vident quam unus , et plures manus \\\hline
3.2.4 & assi ca muchͣs oios \textbf{ mas veen que vno et muchͣs manos } mas pueden & et non unum tantum . Prima via sic patet . \textbf{ Nam plures oculi plus vident quam unus , et plures manus } plus possunt quam una , et plures intellectus superant \\\hline
3.2.4 & que vna \textbf{ e muchs entendimientos mas entienden que vno } e mas saben que vno & Nam plures oculi plus vident quam unus , et plures manus \textbf{ plus possunt quam una , et plures intellectus superant } unum in cognoscendo : \\\hline
3.2.4 & assi commo vn omne de muchos oios \textbf{ e de muchͣs manos . } Por la qual cola meior sera este tal prinçipado & quod plures homines sic principantes \textbf{ quasi constituunt unum hominem multorum oculorum et multarum manuum . } Quare melior erit huius principatus , \\\hline
3.2.4 & Por la qual cola meior sera este tal prinçipado \textbf{ ca el omne assi conpuesto de muchs oios e la muchedunbre } que assi ensseñorea sera mas abiuada & Quare melior erit huius principatus , \textbf{ quia homo sic constitutus } et multitudo sic principans , efficacio erit in principando . \\\hline
3.2.4 & que deue ser en el prinçipe \textbf{ por que estonçe ha derecha entençion } si non entiende abien propo & ex recta intentione quae requiritur in principante . \textbf{ Tunc enim principans rectam habet intentionem , } si non intendat bonum proprium sed commune : quanto igitur minus intenditur commune bonum , \\\hline
3.2.4 & o apartasse much del bien comun . \textbf{ Et pues que assi es meior cosa es de prinçipar } muchs que vno¶ & quasi omnino a communi bono . \textbf{ Peius est igitur principari unum , } quam plures . \\\hline
3.2.4 & que el pho pone en el terçero libro delas politicas dudado \textbf{ e poniendo muchͣs razones } para esto & dubitando , \textbf{ assignans rationes multas , } quod melius sit dominari multitudinem : \\\hline
3.2.4 & Puesto que amos los ssennorios sean derechs \textbf{ ca el dize muchͣs uezeᷤ en esse mismo libro delas politicas } que el regno es prinçipado muy digno & dum tamen utrunque sit rectum , \textbf{ cum ipse pluries dicat in eisdem politicis , } regnum esse dignissimum principatum : \\\hline
3.2.4 & que es dicho regno es muy bueno . \textbf{ Mas entre los malos prinçipados el prinçipado de vno } que por nonbre comun es dich tirannia es muy mal prinçipado & est optimus : \textbf{ inter peruersos vero principatus , | principatus , } unius , \\\hline
3.2.4 & Mas entre los malos prinçipados el prinçipado de vno \textbf{ que por nonbre comun es dich tirannia es muy mal prinçipado } mas desto diremos ayuso mas conplidamente & principatus , \textbf{ unius , | qui communi nomine tyrannis nuncupatur , est pessimus . } Sed de hoc infra dicetur : \\\hline
3.2.4 & e el prinçipado real es muy bueno \textbf{ assi por que al mayor bien es contrario el mayor mal } por ende el prinçipado thiranico es muy malo & ostendetur enim quod sicut monarchia regia est optima ; \textbf{ ita quia maiori bono maius malum opponitur , monarchia tyrannica est pessima . Dominari autem plures dominio recto , } non est dignius , \\\hline
3.2.4 & por señorio derech \textbf{ non es mas digna cosa nin meior } que si enssennoreas se vno . & quam dominari unum ; \textbf{ cum nunquam plures } recte dominari possint , \\\hline
3.2.4 & por que ayan muchs oios \textbf{ e deue tomar muchs buenos e uirtuosos omes } por que ayan muchos pies e muhͣs manos & ut habeat multos oculos \textbf{ et multos bonos | et virtuosos , } ut habeat multos pedes \\\hline
3.2.4 & e deue tomar muchs buenos e uirtuosos omes \textbf{ por que ayan muchos pies e muhͣs manos } e en esta manera se fara el prinçipe vn omne de muchs oios e de muchͣs manos e de muchs pies . & et virtuosos , \textbf{ ut habeat multos pedes } et multas manus : \\\hline
3.2.4 & por que ayan muchos pies e muhͣs manos \textbf{ e en esta manera se fara el prinçipe vn omne de muchs oios e de muchͣs manos e de muchs pies . } Et pues que assi es non se puede dezer & ut habeat multos pedes \textbf{ et multas manus : } et sic fiet unus homo multorum oculorum , multarum manuum , \\\hline
3.2.4 & que non conogca \textbf{ e non sepa muchͣs cosas . } por que quanto parte nesçe al gouernamientod el regno & et sic fiet unus homo multorum oculorum , multarum manuum , \textbf{ et multorum pedum . Non ergo dici poterit talem unum monarchiam non cognoscere multa ; } quia quantum spectat ad regimen regni , quicquid omnes illi sapientes cognoscunt , \\\hline
3.2.4 & e todos los sabios \textbf{ e todos los buenos omes } que assi ouo aconpannado fuessen tristornados e corronpidos & et omnes sapientes , \textbf{ et bonos quos sibi associauit , } contingeret esse peruersos : \\\hline
3.2.5 & que seria meior en toda manera \textbf{ e mas digna cosa } qua el senñorio real & Videretur forte alicui omnino esse melius \textbf{ et dignius dominationem regiam } et principatum \\\hline
3.2.5 & fallatemos el contrario \textbf{ por muchͣs razones ¶ } Lo vno & melius est Principem praestituendum esse per electionem , \textbf{ quam per haereditatem . } Tamen quia ut plurimum homines habent corruptum appetitum , \\\hline
3.2.5 & Lo vno \textbf{ por que en la mayor parte los omes en la elecçion han corrupta la uoluntad Et lo al } por que p̃ssados los fechs & quam per haereditatem . \textbf{ Tamen quia ut plurimum homines habent corruptum appetitum , } consideratis gestis et conditionibus hominum , \\\hline
3.2.5 & que penssadas las condiciones de los omes \textbf{ meior cosa es } que el regno sea por heredat & quam per electionem . Possumus autem triplici via ostendere , \textbf{ quod consideratis conditionibus hominum melius est , } huiusmodi regimen per haereditatem ire . Prima via sumitur ex parte ipsius regis regentis populum . Secunda ex parte filii , \\\hline
3.2.5 & e por ende natra al cosa es \textbf{ que tato el Rey aya mayor cuydado enel bien de su regno } quanto cree & naturale est igitur \textbf{ tanto regem magis solicitari circa bonum regni , } quanto credit ipsum regnum magis esse bonum suum \\\hline
3.2.5 & por que toda la esperança del padre esta enlos fijos \textbf{ e con muy grant ardor los padres se mueuen al amor delos fijos . } por ende el Rey en toda cosa & Immo quia tota spes patris requiescit in filiis , \textbf{ et nimio ardore mouentur patres erga dilectionem filiorum : } ideo omni cura qua poterit mouebitur ad procurandum bonum statum regni , \\\hline
3.2.5 & por ende el Rey en toda cosa \textbf{ que pudiere mouerse a procurar todo buen estado del regno } si penssare & et nimio ardore mouentur patres erga dilectionem filiorum : \textbf{ ideo omni cura qua poterit mouebitur ad procurandum bonum statum regni , } si cogitet ipsum prouenire ad dominium filiorum . Philosophus tamen 3 Politic’ \\\hline
3.2.5 & por que antiguamente los Reyes \textbf{ por la mayor parte se torna una tiranos } e non dura una prolongadamente & quia antiquitus \textbf{ ut plurimum reges conuertebantur in tyrannos } et non diu durabant , \\\hline
3.2.5 & pues que assi es departe del Rey \textbf{ por que mayor cuydado aya del bien del regno podemos prouar } que el gouernamiento real & vix aut nunquam regnabunt filii filiorum . Ex parte ergo regis \textbf{ ut magis solicitetur circa bonum regni , } arguere possumus , \\\hline
3.2.5 & ca assi commo las costunbres de los que en el otro dia fueron enrrequeçidos \textbf{ por la mayor parte son peores } que las costuͣbron de aquellos que fueron ricos antiguamente & Nam sicut mores nuper ditatorum \textbf{ ut plurimum peiores sunt moribus eorum , } qui fuerunt diuites ab antiquo : \\\hline
3.2.5 & assi las costubres de aquellos que en el otro dia fueron poderosos . \textbf{ de nueuo leunatados en poderio . } por este poderio çiuil peotes son en costunbres & sic mores nuper potentum \textbf{ et de nouo eleuatorum per adeptionem ciuilis potentiae , } peiore , \\\hline
3.2.5 & de nueuo leunatados en poderio . \textbf{ por este poderio çiuil peotes son en costunbres } que los otros & sic mores nuper potentum \textbf{ et de nouo eleuatorum per adeptionem ciuilis potentiae , } peiore , \\\hline
3.2.5 & ca estos tales \textbf{ por la mayor parte thiranzan } e fazense thiranos & est quasi quaedam ineruditio regiae dignitatis tales quidem \textbf{ ut plurimum tyrannizant , } et inflati corde et inerudite regnant . \\\hline
3.2.5 & assi commo a cosanatal \textbf{ ca commo toda cosa uoluntaria sea de menor carga } e menos graue & ut voluntarie obediant : \textbf{ quare cum omne voluntarium sit minus onerosum et difficile , } ut libentius \\\hline
3.2.5 & al mandamiento del Rey \textbf{ conuiene que la Real dignidat vaya por hedamiento } Pues que assi es determinar la casa & et facilius obediat populus mandatis regis , \textbf{ expedit regiae dignitati per haereditatem succedere . Determinare igitur domum et prosapiam , } ex qua assumendus est rex , sedat litigia , tollit tyrannidem , \\\hline
3.2.5 & Lo primero tira las uaraias et las discordias \textbf{ que nasçen muchͣs uezes entre los prinçipes } por las electonnes & quia multotiens \textbf{ inter eligentes dissensiones oriuntur } propter electionem Principis : \\\hline
3.2.5 & do el sennorio viene por hedamiento \textbf{ ca si la dignidat Real passa alos fijos } por hedamiento conuiene alos pueblos & si non determinetur quae persona in illa prosapia debeat principari . Talem autem determinare personam , difficultatem non habet : \textbf{ nam si dignitas regia per haereditatem transferatur ad posteros , oportet eam transferre in filios , quia secundum lineam consanguinitatis filii parentibus maxime sunt coniuncti : } oportet autem talem dignitatem magis transferre ad masculos quam ad foeminas , \\\hline
3.2.5 & que el regno pertenesca al primo genito \textbf{ por que el padre con mayor acuçia aya cuydado del bien del regno } sabiendo que el regno parte nesçe al su fiio mas amado & expedit statuere regnum succedere primogenito : \textbf{ ut pater ampliori solicitudine curet de bono regni , } sciens ipsum peruenire ad filium plus dilectum . \\\hline
3.2.5 & so regla derechͣ \textbf{ nin so çierta ley } nin so çierta cuenta . & quia facta humana et gesta particularia omnino sub recta regula , \textbf{ et sub certa narratione non cadunt : } sufficit ergo in talibus pertransire probabiliter , \\\hline
3.2.5 & nin so çierta ley \textbf{ nin so çierta cuenta . } Pues que assi es cunple en tales cosas yr & quia facta humana et gesta particularia omnino sub recta regula , \textbf{ et sub certa narratione non cadunt : } sufficit ergo in talibus pertransire probabiliter , \\\hline
3.2.5 & por aquello que prouamos \textbf{ e establesçer leyes por que en la mayor parte ayan uerdat e sean uerdaderas . } Mas aqual lo que dessuso fue dich conuiene saber & sufficit ergo in talibus pertransire probabiliter , \textbf{ et leges statuere quae ut in pluribus continent veritatem . } Quod vero superius tangebatur , \\\hline
3.2.5 & Enpero aquellas cosas son mas de escusar \textbf{ que pueden auer mayores peligros } ca veeraos & quae ex aliqua parte non exponantur periculis . Illa tamen magis cauenda sunt , \textbf{ quae pluribus periculis exponuntur . } Videmus autem per experientiam plura mala oriri in ciuitatibus et regnis , \\\hline
3.2.5 & que muchs males nasçen enlas çibdades e en los regnos do non ay algun sennor natural ca contesçe \textbf{ que alguons regnos algunas uezes non han gouernador en algunt p̃o . } Et quando han gouernador muchͣs uezes t hiranzan . & nam contingit \textbf{ ea aliquando diu carere gubernatore , } et cum gubernatorem habent \\\hline
3.2.5 & Et quando han gouernador muchͣs uezes t hiranzan . \textbf{ Onde muchs males auemos visto en tales gouernamientos } que non podemos contar & ut plurimum tyrannizant : \textbf{ plura enim huiusmodi mala in talibus regiminibus vidimus , } quae enumerare \\\hline
3.2.5 & a quien parte nesçe auer cuydado del regno \textbf{ este fallesçimiento se puede conplir por sabios et por buenos omes } lo quales deue el Rey ayuntar assi & et si aliquis defectus esset in filio regio , ad quem deberet regia cura peruenire , suppleri poterit per sapientes \textbf{ et bonos , } quos tanquam manus et oculos debet sibi Rex in societatem coniungere : \\\hline
3.2.5 & si non fueren muy acuçiosos \textbf{ e non ouieren muy grant cuydado } que los sus fijos de su ninnez sean bie doctrinados & valde tamen reprehensibiles sunt Reges et Principes , \textbf{ si non nimia cura solicitentur } quod filii ab ipsa infantia \\\hline
3.2.5 & e bien acostunbrados \textbf{ e bien enformados en todas buenas costunbres } por que el bien de todo el regno esta en esto . & et disciplina \textbf{ et bonis moribus sint imbuti ; | cum bonum commune } et totius regni in hoc consistat . \\\hline
3.2.5 & Por ende por que el bien comun non sea puesto a peligro de todos los fios \textbf{ deuen auer los padres grant cuydado } pho en el quanto libro delas poluenta tres cosas & ne periclitetur bonum commune , \textbf{ de omnibus filiis Regis cura diligens est habenda . } Philosophus 5 Politic’ narrat tria , \\\hline
3.2.6 & en las quales auataia soƀre los otros . \textbf{ ca dize que antiel Rey deue sobrepiuar } e auerguamente los Reyes eran establesçidos enlos sennorios & Philosophus 5 Politic’ narrat tria , \textbf{ in quibus Rex alios debet excedere . Dicit enim , } quod antiquitus Reges a triplici excessu constituebantur . \\\hline
3.2.6 & por ende aquellos que vee ser liberales \textbf{ e bien fechores mueuense con grant ardor alos amar } e dessean de los auer por sennores & quos videt esse liberales et beneficos , \textbf{ nimis ardenter mouetur in eorum amorem , } et optat eos habere in dominos . Inde est quod antiquitus plures sic praeficiebantur in Reges . \\\hline
3.2.6 & por que si alguno era atal que feziera bien al pueblo aquella gente inclinada a el \textbf{ por grant amor tomauas lo } por su Rey & et optat eos habere in dominos . Inde est quod antiquitus plures sic praeficiebantur in Reges . \textbf{ Nam si aliquis fuerat primo beneficus , } gens illa tracta ad amorem eius praeficiebat ipsum in Regem . Secundo potest aliquis praefici in Regem ab excessu virtuosarum actionum : \\\hline
3.2.6 & ca por que es cosa prouada \textbf{ que los nobles e los poderosos toman mayor uerguença de obrar cosas torpes e feas } que los otros & et dignitatis . \textbf{ Nam quia probabile est nobiles et potentes , } magis verecundari operari turpia quam alios : \\\hline
3.2.6 & e por que tales \textbf{ por la mayor parte o por que son tales de fecho } por que los tienen los omes & et quia tales \textbf{ ut plurimum vel sunt } vel creduntur esse magis apti ad principandum , \\\hline
3.2.6 & e dela obra uirtuosa \textbf{ e del grant poderio pueden ser razon } porque alguno sea tomado derechamente por Rey . & Quare si excessus beneficii actionis virtuosae \textbf{ et potentiae possunt esse causa ut rite aliquis constituatur in regem , } decens est tales excessus in ipsa monarchia perfectius reperiri . Decet enim ipsum regem volentem \\\hline
3.2.6 & porque alguno sea tomado derechamente por Rey . \textbf{ conuenible cosa es } que tases aun ataias & ø \\\hline
3.2.6 & ¶ lo primero que sea amado del pueblo . \textbf{ L segundo que con grant acuçia procure el bien } comun¶ & ( quantum ad praesens spectat ) ad tria solicitari . Primo , \textbf{ ut ametur a populo . Secundo , ut cura peruigili procuret commune bonum . } Tertio , \\\hline
3.2.6 & que el rey aya aquellas tres aun ataias \textbf{ e aquellas tres condiçiones buenas sobredichͣs . ca si abondare en bien fazer } seria muy amado del pueblo & Quare expedit regem habere praedictos tres excessus . \textbf{ Nam si abundet } in beneficiis tribuendis , \\\hline
3.2.6 & en obras uirtuosas procurara el bien comun . \textbf{ ca si la uirtud parte nesçede se estender a mayor bien } e en mayor bien dela gente & procurabit commune bonum : \textbf{ quia si virtutis est , } tendere in bonum , \\\hline
3.2.6 & ca si la uirtud parte nesçede se estender a mayor bien \textbf{ e en mayor bien dela gente } es el bien comun & quia si virtutis est , \textbf{ tendere in bonum , } eius erit magis tendere in maius bonum , \\\hline
3.2.6 & mas procurara el bien comun \textbf{ que el su bien pro ̉o priuado . } Lo terçero conuiene & et commune \textbf{ quod est diuinius quam aliquod bonum singulare , magis procurabit vir bonus } et virtuosus , quam bonum aliquod proprium \\\hline
3.2.6 & Mas el philosofo en el quinto libro delas pol . \textbf{ tanne quatro departimientos } que son entre el thirano e el Rey . & quomodo differat a Tyranno . \textbf{ Tangit autem Philosophus 5 Politicorum quatuor differentias } inter tyrannum et regem . Prima est , \\\hline
3.2.6 & Mas e Rey faze todo el contrario \textbf{ por que vee que ha muy grant cuydado del regno } e del bien comun fia muchͣ de aquellos que son en el su regno . & sed Rex econuerso \textbf{ eo quod videat se maximam curam habere de bono regni et communi , } maxime confidit de his qui sunt in regno . Ideo facit se custodiri a propriis , \\\hline
3.2.6 & e del bien comun fia muchͣ de aquellos que son en el su regno . \textbf{ Et por ende se faze guardar de los sus çibdadanos propreos } e non de los estrannos . & eo quod videat se maximam curam habere de bono regni et communi , \textbf{ maxime confidit de his qui sunt in regno . Ideo facit se custodiri a propriis , } non ab extraneis . \\\hline
3.2.7 & e non de los estrannos . \textbf{ or quatro razones podemos prouar } que la thirama es muy mal prinçipado & non ab extraneis . \textbf{ Quadruplici via venari possumus , } tyrannidem esse pessimum principatum . \\\hline
3.2.7 & or quatro razones podemos prouar \textbf{ que la thirama es muy mal prinçipado } ¶La primera se toma & Quadruplici via venari possumus , \textbf{ tyrannidem esse pessimum principatum . } Prima sumitur ex eo quod tale dominium maxime recedit \\\hline
3.2.7 & la quarta \textbf{ por razon que tal prinçipado ha de enbargar muy grandeᷤ bienes delos çibdadanos ¶ } La primera se declara & ex eo quod est efficacissimum ad nocendum . Quarta , \textbf{ ex eo quod impedire habet maxime bona ipsorum ciuium . Prima via sic patet . } Quia si dominatur Rex , \\\hline
3.2.7 & por que en tal prinçipado es entendido el bien de muchs . \textbf{ En essa misma manera avn si enssenoreare muchs } por que son ricos o por que son tenidos por uirtuosos & eo quod in tali principatu intendatur bonum multorum . \textbf{ Sic etiam si dominentur plures , } quia diuites , \\\hline
3.2.7 & por que non es derech \textbf{ si non fuere en alguͣ manera cosa diuinal } e por que en el prinçipalmente se deue entender el bien comun & nisi sit quodammodo \textbf{ quid diuinum , } et quia in eo principaliter \\\hline
3.2.7 & que es mas diuinal que ningun bien singular . \textbf{ por ende tanto la tirania es peor prinçipado } quanto por ella & quod est diuinius quam aliquod singulare ; \textbf{ tanto tyrannis est principatus peior , } quanto in eo plus receditur ab intentione communis boni . Hanc autem rationem tangit Philosophus circa principium 4 Politicorum \\\hline
3.2.7 & do dize que assi conmo el regno es muy buena \textbf{ et muy digna poliçia . } assi la tirania es muy mala & ubi ait , \textbf{ quod sicut Regnum est optima et dignissima politia , } sic tyrannis est pessima : \\\hline
3.2.7 & Mas el omne \textbf{ por que ha libre aluedrio } e ha razon estonçe es naturalmente gouernado & ut sunt apti nati regi : \textbf{ homo autem quia libero arbitrio et ratione participat , } tunc naturaliter regitur \\\hline
3.2.7 & Et esta razon tanne avn el philosofo en el quarto libro delas politicas \textbf{ do dize que la tirania es muy mal prinçipado } por que ninguno de los omes francos e libres non sufre & Hanc autem rationem tangit Philosophus in eodem 4 Politicorum \textbf{ ubi ait , | tyrannidem esse pessimum principatum , } quia nullus liberorum voluntarie sustinet principatum talem . \\\hline
3.2.7 & e el su prinçipado es muy bueno . \textbf{ ca por la uirtud ayuntada en el puede fazer muchͣs bueans cosas . } Et si por auentura el prinçipe ha la entençion tuerta estonçe es tirano & tunc est Rex et est optimus principatus : \textbf{ quia propter unitatem virtutes potest multa bona efficere : } si vero monarchia habet intentionem peruersam , \\\hline
3.2.7 & ca por el su poderio muy grande \textbf{ que es ayuntado en vno puede fazer muchs males } e esta razon tanne el philosofo en el quinto libro delas politicas & et est pessimus , \textbf{ quia propter suam unitam potentiam potest multa mala efficere . } Hanc autem rationem tangit Philosophus quinto Politicorum \\\hline
3.2.7 & do dize \textbf{ que la tirnia es la postrimera obligarçia } que quiere dezer muy mala obligacion & ubi ait , \textbf{ tyrannidem esse oligarchiam extremam idest pessimam : } quia est maxime nociua subditis . \\\hline
3.2.7 & que la tirnia es la postrimera obligarçia \textbf{ que quiere dezer muy mala obligacion } por que es muy enpesçedera alos subditos ¶ & ubi ait , \textbf{ tyrannidem esse oligarchiam extremam idest pessimam : } quia est maxime nociua subditis . \\\hline
3.2.7 & La quarta razon se toma \textbf{ por que por tal sennorio son enbargados grandes bienes de los çibdadanos . } ca el tirano non solamente procura los males de aquellos & Quarta via sumitur \textbf{ ex eo quod per tale dominium impediuntur maxima bona ipsorum ciuium . } Tyrannus enim non solum procurat mala eorum \\\hline
3.2.7 & que ellos se \textbf{ que de grandes coraçones e uirtuosos } e avn non quiere que los çibdadanos sean sabios e entendidos . & rursus nolunt eos esse magnanimos \textbf{ et virtuosos : } nec etiam volunt ipsos esse sapientes \\\hline
3.2.7 & Et cunpla agora de saber \textbf{ que la tirania es muy mal prinçipado } por las razones sobredichͣs . & Sufficiat autem ad praesens scire , \textbf{ tyrannidem esse pessimum principatum propter rationes tactas . } Quod autem reges summo opere cauere debeant , \\\hline
3.2.7 & quanto peor es el su sennorio \textbf{ por la qual cosa el Rey deue poner muy grant acuçia } por non ser tiranno & videre non est difficile . Nam tanto peior est Princeps , \textbf{ quanto peiori dominio principatur : quare si omnem diligentiam adhibere debet Rex } ne sit pessimus , \\\hline
3.2.7 & por non ser tiranno \textbf{ e por non ser mal prinçipe } e much deue escusar & quanto peiori dominio principatur : quare si omnem diligentiam adhibere debet Rex \textbf{ ne sit pessimus , } summe curare debet \\\hline
3.2.7 & de non enssennorear con señorio de tirania \textbf{ que es muy mal prinçipado } el Rey o el prinçipe quiere gouernar conueniblemente la gente & per tyrannidem \textbf{ qui est pessimus principatus . } Si Rex aut Princeps gentem sibi commissam vult debite gubernare , \\\hline
3.2.8 & e si dessea saber \textbf{ qual es el su ofiçio deue penssar con grant acuçia en las cosas naturales } ca si toda la natura es gouernada & et scire desiderat \textbf{ quod sit eius officium : | diligenter considerare debet in naturalibus rebus . } Nam si natura tota administratur per ipsum Deum , \\\hline
3.2.8 & cado es la sabiduria \textbf{ e la fuente delas esc̀ yturas . } conuiene que de ally tome todo el pueblo algun enssennamiento & Nam ubi viget sapientia \textbf{ et fons scripturarum , } oportet \\\hline
3.2.8 & Lo segundo para alcançar la fin \textbf{ que entiende siruen buenos apareiamientos del alma e uirtudes } ca non cunple conosçer la fin & sed tyrannus . \textbf{ Secundo ad consequendum finem intentum deseruiunt boni habitus et virtutes . } Non enim sufficit finem cognoscere , \\\hline
3.2.8 & que quiera e pueda alcançar aquella fin . \textbf{ Et por ende parte nesçe al gouernador del regno de otdenar sus subditosa uirtudes } e a buenas costunbres ¶ & finem illum : \textbf{ spectat igitur ad rectorem regni ordinare suos subditos ad virtutes . Tertio ad consequendum } finem \\\hline
3.2.8 & Et por ende parte nesçe al gouernador del regno de otdenar sus subditosa uirtudes \textbf{ e a buenas costunbres ¶ } Lo terçero para alcançar la fin & finem illum : \textbf{ spectat igitur ad rectorem regni ordinare suos subditos ad virtutes . Tertio ad consequendum } finem \\\hline
3.2.8 & los quales enbargos son tres de los quales . ¶ El vno toma nasçençia dela natura \textbf{ El otro dela maldat delos çibdadanos ¶ El terçero dela mal querençia de los enenigos . } lo primero se prueua assi . & quorum unum quasi sumit originem ex natura , \textbf{ aliud vero ex peruersitate ciuium , | tertium quidem ex maleuolentia hostium . } Primum sic patet . \\\hline
3.2.8 & segunt su establesçimientero otro aya hedat en sucession \textbf{ pues que as susi es much se puede turbar el buen estado e la paz dela çibdat } e la fin & secundum suam institutionem alius in haereditatem succedat . \textbf{ Valde ergo turbari potest tranquillus status | et pax ciuitatis } et finis intentus in politica , \\\hline
3.2.8 & que enbarga \textbf{ much la buena uida çiuil } es ordenar bien & et Principes \textbf{ non solicitentur qualiter posteriores succedant in haereditatem priorum : remouere igitur unum maxime prohibentium bonam vitam politicam , } est bene ordinare quomodo haereditates decedentium perueniant ad posteros . Secundo , \\\hline
3.2.8 & et los malos sean castigados . \textbf{ ¶ El terçero enbargo del buen gouierno } toma nasçençia dela mal querençia de los enemigos & ut exterminentur malefici , \textbf{ et corrigantur delinquentes . Tertium huiusmodi impeditiuum sumit originem ex malitia hostium : } quasi enim nihil esset vitare interiora discrimina , \\\hline
3.2.8 & ¶ El terçero enbargo del buen gouierno \textbf{ toma nasçençia dela mal querençia de los enemigos } por que non seria nada escusar los males de dentro del alma & ut exterminentur malefici , \textbf{ et corrigantur delinquentes . Tertium huiusmodi impeditiuum sumit originem ex malitia hostium : } quasi enim nihil esset vitare interiora discrimina , \\\hline
3.2.8 & ¶ Ca lo primero las cosas \textbf{ que menguna a buen gouernamiento son de ennader } assi que si viessen & Haec etiam tria sunt . \textbf{ Nam primo commissa sunt supplenda : } ut si viderint aliquid deesse ad bonum regimen ciuitatis , illud est supplendum : \\\hline
3.2.8 & que menguaua alguna cosa \textbf{ para buen gouernamiento dela çibdat } aquello deuen cunplir & Nam primo commissa sunt supplenda : \textbf{ ut si viderint aliquid deesse ad bonum regimen ciuitatis , illud est supplendum : } hoc autem fieri potest \\\hline
3.2.8 & Lo segundo los bueons ordenamientos \textbf{ e los bueons establesçimientos son de guardar ¶ } Lo terçero los que bien obran & de quo infra dicetur . Secundo , bonae ordinationes , \textbf{ et bona statuta debent esse obseruanda . Tertio , bene operantes } et maxime facientes \\\hline
3.2.8 & entre aquellos son los omes muy fuertes \textbf{ entre los quales los fuertes omes son muy honrrados . } Bien assi entre aquellos son los sabios & apud illos sunt fortes , \textbf{ apud quos maxime honorantur : } apud illos sunt sapientes et boni , \\\hline
3.2.8 & e los buenos entre los quales \textbf{ los que lon labios son gualardonados e honrrados . } Et pues que assi es auer acuçia çerca las cosas & apud illos sunt sapientes et boni , \textbf{ apud } quos tales remunerantur , \\\hline
3.2.9 & parte nesçe al ofiçio del Rey \textbf{ asnos podemos contar dies cosas } que deue obrar el uerdadero rey & et honorantur . Solicitari igitur circa praedicta nouem , ad Regis officium pertinere videtur . \textbf{ Narrare autem possumus decem quae debet operari bonus Rex , } et quae Tyrannus se facere simulat . \\\hline
3.2.9 & asnos podemos contar dies cosas \textbf{ que deue obrar el uerdadero rey } Et estas diez cosas el tirano se enfinze delas fazer & et honorantur . Solicitari igitur circa praedicta nouem , ad Regis officium pertinere videtur . \textbf{ Narrare autem possumus decem quae debet operari bonus Rex , } et quae Tyrannus se facere simulat . \\\hline
3.2.9 & e non las faze \textbf{ Et conmo quier que aquellas diez cosas en alguna manera general mente sean contenidas en las cosas sobredichos . } Enpero por que muchͣs uegadas en la sçiençia moral . & et quae Tyrannus se facere simulat . \textbf{ Illa enim decem licet aliquo modo in uniuersali contineantur in dictis , | tamen quia ( ut pluries dictum est ) } circa morale negocium uniuersales sermones proficiunt minus , \\\hline
3.2.9 & Et la primera \textbf{ que parte nesçe de fazer aludadero Reyes } que mucha ya cuydado de procurar e acresçentar los bienes comunes & ideo bene se habet illa decem narrare per singula . \textbf{ Est autem primum quod spectat ad verum Regem facere , } ut maxime procuret bona communia , et regni redditus studeat expendere in bonum commune , \\\hline
3.2.9 & Mas deue paresçer perssona pesada \textbf{ e de muy grant reuerençia . } la qual cosa non se pue de fazer conueniblemente sin uirtud . & nec decet se nimis familiarem exhibere , \textbf{ sed apparere debet persona grauius } et reuerenda , quod congrue sine virtute fieri non potest : ideo verus Rex vere virtuosus existit : \\\hline
3.2.9 & la qual cosa non se pue de fazer conueniblemente sin uirtud . \textbf{ Et por ende el uerdadero Rey deue ser uirtuoso } Mas el tirano non es tal & sed apparere debet persona grauius \textbf{ et reuerenda , quod congrue sine virtute fieri non potest : ideo verus Rex vere virtuosus existit : } tyrannus autem non est , \\\hline
3.2.9 & Mas el tirano non es tal \textbf{ mas quiere pare sçertal . lo quarto parte nesçe a uerdadero Rey } non despreciar a ninguon de los subditos & tyrannus autem non est , \textbf{ sed esse se simulat . | Quarto spectat ad Regem , } nullum subditorum contemnere , \\\hline
3.2.9 & e alos prinçipes \textbf{ non solamente de auer buenos familiares } e de amar los nobles & ø \\\hline
3.2.9 & e de amar los nobles \textbf{ e los ricos omes } e todos los otros omes . & et diligere nobiles , \textbf{ et barones , } et alios \\\hline
3.2.9 & e todos los otros omes . \textbf{ por los quales se puede guardar el buen estado del regno . } Mas avn assi commo dize el philosofo & et alios \textbf{ per quos bonus status regni conseruari potest , } sed etiam \\\hline
3.2.9 & en el terçero libro delas politicas \textbf{ deue enduziras Ꝯmugres propraas } por que sean familiares e bien querençiosas alas mugers de los sobredichos omes buenos del regno & sed etiam \textbf{ ut ait Philosophus in Polit’ inducere debent uxores proprias } ut sint familiares et beniuolae uxoribus praedictorum : \\\hline
3.2.9 & deue enduziras Ꝯmugres propraas \textbf{ por que sean familiares e bien querençiosas alas mugers de los sobredichos omes buenos del regno } por que las muger smuch inclinan a sus maridos a sus uoluntades propreas & ut ait Philosophus in Polit’ inducere debent uxores proprias \textbf{ ut sint familiares et beniuolae uxoribus praedictorum : } nam mulieres valde inclinant viros ad voluntates proprias , \\\hline
3.2.9 & e de los otros \textbf{ por los quales se deue guardar el buen estado vieren } que son menospreçiadas dela muger del prinçipe o del Rey enduzirian a sus maridos & Quare si uxores nobilium \textbf{ et aliorum per quos bonus status regni conseruari habet , } viderent se contemni ab uxore Regis aut Principis , inducerent viros \\\hline
3.2.9 & que assi es . \textbf{ assi se deue auer buen Rey e buen gouernador deregas e de çibdat } Mas el tirano non se ha assi & sic ergo gerere se debet \textbf{ bonus rector regni aut ciuitatis . } Tyrannus autem non sic se habet , \\\hline
3.2.9 & Et por ende aquellas cosas \textbf{ que ha el uerdadero rey essas mismas muestra } que ha el tirano & propter quod nec ipse nec personae sibi coniunctae aliis sunt beniuolae . \textbf{ Ea igitur quae habet verus Rex , licet simulate illa videatur habere Tyrannus , } inquantum tamen huiusmodi est \\\hline
3.2.9 & e non las hauer dada mente . \textbf{ Lo seeto conuiene audadero Rey de ser muy mesura } do tanbien en las uiandas & secundum veritatem caret illis . \textbf{ Sexto decet verum Regem esse moderatum in cibis , et venereis , } ne a subditis habeatur in contemptu : \\\hline
3.2.9 & assi commo auer dineros o auer delecta connes . \textbf{ Et por que es muy grant delectaçion sensible enlas uiandas e enlas luyias . } los tiranos sin fre no vsan delas delecta connes & et delectabile , \textbf{ quia maxima delectatio sensibilis est in cibis | et venereis , } tyranni absque fraeno fruuntur voluptatibus illis . \\\hline
3.2.9 & si los uieren estar de cada dia \textbf{ en grandes comeres } e en grandes beueres e en grandes conuides . & si singulis diebus aspiciantur esse in commessationibus , \textbf{ et ebrietatibus , } et in conuiuiis affluentibus : \\\hline
3.2.9 & en grandes comeres \textbf{ e en grandes beueres e en grandes conuides . } Enpero non se deuen assi auer . & et ebrietatibus , \textbf{ et in conuiuiis affluentibus : | non } tamen sic se habere deberent . Dato enim quod temperamento carerent \\\hline
3.2.9 & Et es de deuostar la glotonia e la destenprança e la auariçia¶ \textbf{ Lo septimo conuiene al uerdadero Rey de conponer } e guarnesçer las çibdades e los castiellos & laudatur enim sobrietas et temperantia , \textbf{ vituperatur autem auaritia et gulositas . Septimo decet verum Regem ornare } et munire ciuitates \\\hline
3.2.9 & assi que tengan los sabios \textbf{ que non serien tan honrradosente los sus çibdadanos propreos } si entre ellos estudiessen . & etiam extraneos adeo honorare , \textbf{ ut putetur non sic honoratos esse a ciuibus propriis , } si inter ipsos existerent , \\\hline
3.2.9 & por que por auentura non la tenie commo deuie sue denostado de su muger diziendo \textbf{ que grand uerguença le era } que dexaua menor regno a sus fijos & increpatus ab uxore dicente , \textbf{ quod verecundari deberet , } quia minus regnum dimitteret filiis , \\\hline
3.2.9 & que grand uerguença le era \textbf{ que dexaua menor regno a sus fijos } que resçibiera de su padre . & quod verecundari deberet , \textbf{ quia minus regnum dimitteret filiis , } quam accepisset a patre , \\\hline
3.2.9 & Et el Rey respondiol \textbf{ que si les dexaua menor regno } en quantidat enpero dexauales mayor regno & respondit Rex ille , \textbf{ quod si dimittebat eis minus in quantitate regnum , dimittebat } tamen maius et diuturnius tempore . Tyranni autem hoc non faciunt ; \\\hline
3.2.9 & que si les dexaua menor regno \textbf{ en quantidat enpero dexauales mayor regno } e mas duradero & respondit Rex ille , \textbf{ quod si dimittebat eis minus in quantitate regnum , dimittebat } tamen maius et diuturnius tempore . Tyranni autem hoc non faciunt ; \\\hline
3.2.9 & la prouidençia de dios aqui todas las cosas son manifiestas \textbf{ e el su poderio aqui non puede ser ninguna cosa contraria legnia } assi conmo cunple a su salut . & quod si Rex habeat amicum Deum diuina prouidentia \textbf{ cui omnia sunt nota , et eius potentia cui nihil potest resistere , } continget eum \\\hline
3.2.9 & Et fazel ser sienpre bien auentraado en todos sus fechos Et por ende \textbf{ por la sanidat del Rey dios muchas uezes faze muchs bienes } a aquellos que son en el su regno . & Immo propter sanctitatem regis , \textbf{ multotiens Deus multa bona confert existentibus in ipso regno . } Ultimo autem diximus \\\hline
3.2.9 & Et esto dezimos a postremas de todo notablemente \textbf{ que conuiene ala Real maiestad } de se auer bien çerca aquellas cosas & Ultimo autem diximus \textbf{ quod decet regiam maiestatem bene se habere circa diuina : } quia hoc debet esse finis \\\hline
3.2.10 & de se mantener en su sennorio . \textbf{ La primera cautela del tirano es matar los grandes omes e los poderosos . } Ca commo el tirano non ame & tyrannus se in suo dominio praeseruare . \textbf{ Prima cautela tyrannica , | est excellentes perimere . } Cum enim tyrannus non diligat \\\hline
3.2.10 & e con poconna o con otra falssedat . \textbf{ la qual cosaes muy mala señal de tirama . } Mas el uerdadero Rey faze todo el contrario & et nimia sibi consanguinitate coniunctos venenant , \textbf{ et perimunt : quod signum est tyrannidis pessimae . } Verus autem Rex econuerso intendens commune bonum , \\\hline
3.2.10 & la qual cosaes muy mala señal de tirama . \textbf{ Mas el uerdadero Rey faze todo el contrario } entendiendo en el bien comun & et perimunt : quod signum est tyrannidis pessimae . \textbf{ Verus autem Rex econuerso intendens commune bonum , } et cognoscens se diligi \\\hline
3.2.10 & e much menos los sus parientes propreos . \textbf{ por los quales el buen estado del regno se puede guardar mas saluales } e mantienelos en su honrra . & et multo magis cognatos proprios , \textbf{ per quos bonus status regni conseruari potest , } non perimit , \\\hline
3.2.10 & en quanto puede destruye los sabios . \textbf{ Mas el uerdadero Rey faze todo el confͣrio . } Ca sabeque lo que el faze faze lo con razon de techͣ & sapientes pro posse destruit . \textbf{ Verus autem Rex econtrario sciens se secundum rectam rationem agere , } sapiente , saluat , promouet , \\\hline
3.2.10 & por razon que los sabios conosciendo \textbf{ e sabiendo las sus buenas obras mueuen } e enduzen el pueblo al amor del Rey . & et honorat , \textbf{ eo quod ipsi cognoscentes bona opera ipsius , } populum commouent ad amorem eius . Tertia , \\\hline
3.2.10 & Ca el tirano \textbf{ segunt qsu mala condiçion non solamente destruye los sabios } mas avn defiende & secundum quod huiusmodi est ) \textbf{ non solum sapientes destruit , } sed etiam truncat viam , \\\hline
3.2.10 & e por la sçiençia \textbf{ a as el uerdadero Rey faze el contrario } que promueue el estudio & ne efficiantur aliqui sapientes : semper enim timet per sapientiam reprehendi . \textbf{ Verus autem Rex econtrario studium promouet , } et conseruat , \\\hline
3.2.10 & que por el bien comun \textbf{ e el buen estado del regno } en el qual el entiende prinçipalmente ha de ser meiorado ¶ & bonum commune , \textbf{ et bonus status regni , } quem principaliter intendit , \\\hline
3.2.10 & que les faze . \textbf{ Mas eludadero Rey faze todo el contrario } ca consiente todas las conpannias & contra ipsum insurgant . \textbf{ Verus autem Rex econtrario permittit sodalitates ciuium , } et vult ciues sibi inuicem esse notos , \\\hline
3.2.10 & si conosçiesse \textbf{ que auia buen Rey e uerdadero } e que amaua el bien comun & quia tunc magis unanimiter diligunt bonum Regis . Omnino enim esset peruersus populus , \textbf{ si cognosceret se habere verum Regem , } et diligere commune bonum , \\\hline
3.2.10 & La quinta cautela del tirano \textbf{ es auer muchs assechadores } e escodrinnar & ø \\\hline
3.2.10 & que non lon amados del pueblo . \textbf{ por que en muchͣs cosas le aguauian quieren auer muchs assechadores } por que si vieren & Cum enim tyranni sciant se non diligi a populo , \textbf{ eo quod in multis offendant ipsum , | volunt habere exploratores multos , } ut si viderent aliquos ex populo machinari aliquid contra eos , \\\hline
3.2.10 & Enpero sienpre temen que y entre ellos assechadores e acusadores . \textbf{ Mas el uerdadero Rey non ha cuydado de auer tales assechadores entre los çibdadanos } nin entre los que son en el regno . & et si inter eos sic congregatos nullus exploratorum existeret , \textbf{ semper tamen timerent ibi exploratores esse . Huiusmodi autem exploratores verus Rex habere non curat ad ciues , } et ad eos qui sunt in regno : \\\hline
3.2.10 & e los ricos de los çibdadanos . \textbf{ Entre tanto non puede de ligero contradezer al su mal poderio } por que estonçe cada vna delas partes ha miedo dela otra & et diuites a diuitibus : \textbf{ tamdiu non potest aeque de facili eius potentiae resisti : } nam tunc quaelibet partium timens alteram , \\\hline
3.2.10 & e ninguna dellas non se leunata contra el tyrano . \textbf{ Mas el uerdadero Rey faze el contrario . } ca non procura turbaçion delos & neutra insurgit contra tyrannum . \textbf{ Verus autem Rex econtrario non procurat turbationem existentium in regno , } sed pacem et concordiam : \\\hline
3.2.10 & que son en el regno mas paz e concordia \textbf{ e en otra manera non sia uerdadero Rey } ca non entendrie en el bien comun . & sed pacem et concordiam : \textbf{ aliter enim non esset verus Rex , } quia non intenderet commune bonum . Septima , \\\hline
3.2.10 & por temor dellos . \textbf{ Mas el uerdadero Rey . que entiende en el bien de los subditos } non los atormenta & ut non vacet eis aliquid machinari contra ipsos , nec oporteat ipsos habere aliquam custodiam propter illos . \textbf{ Verus autem Rex quia intendit bonum subditorum } non affligit \\\hline
3.2.10 & por razon que ellos en tal manera sean ocupados en las guercas \textbf{ que non les vague de seleunatar contra el tirano . Mas el uerdadero rey non entiende de atormentar los subditos } mouiendo les & quatenus semper circa bellorum onera intenti , \textbf{ non vacet eis aliquid machinari contra tyrannum . | Verus autem Rex non intendit affligere subditos , } suscitando \\\hline
3.2.10 & o por algun guerra derechurera . \textbf{ La nouena caute la del tirano es poner grant guarda en el su cuerpo } non por aquellos que son del regno & vel pro aliquo alio iusto bello . Nona , \textbf{ est custodiam corporis exercere } non per eos \\\hline
3.2.10 & mas por los estrannos . \textbf{ Mas el uerdadero Rey faze el contrario } assi commo & sed per extraneos . \textbf{ Verus autem Rex econuerso se habet , } ut supra plenius dicebatur . \\\hline
3.2.10 & que con vn unado atormente al otro assi que con vn clauo atenaçe el otro . \textbf{ Mas el buen Rey faze todo el contrario } non procura departimientos & cum una parte affligit aliam \textbf{ ut clauum clauo retundat . Rex autem econtrario non procurat diuisiones } et partes in regno , \\\hline
3.2.11 & que auemos contado en este capitulo \textbf{ sobredicho trae el pho a quatro cautelas } ca los tyranos quat cosas entienden & quas in praecedenti capitulo numerauimus , \textbf{ philosophus reducit ad quatuor : } tyranni enim quatuor intendunt . Primo , \\\hline
3.2.11 & sobredicho trae el pho a quatro cautelas \textbf{ ca los tyranos quat cosas entienden } ¶La primera es que los subditos sean nesçios e sin sabiduria . & philosophus reducit ad quatuor : \textbf{ tyranni enim quatuor intendunt . Primo , } ut subditi sint inscii . Secundo , \\\hline
3.2.11 & ca los tyranos quat cosas entienden \textbf{ ¶La primera es que los subditos sean nesçios e sin sabiduria . } la segunda que sean de flacos coraçones ¶Lat̃çera & philosophus reducit ad quatuor : \textbf{ tyranni enim quatuor intendunt . Primo , } ut subditi sint inscii . Secundo , \\\hline
3.2.11 & ¶La primera es que los subditos sean nesçios e sin sabiduria . \textbf{ la segunda que sean de flacos coraçones ¶Lat̃çera } que non fiendessi mismos . & tyranni enim quatuor intendunt . Primo , \textbf{ ut subditi sint inscii . Secundo , | ut sint pusillanimes . Tertio , } ut de se inuicem non confidant . \\\hline
3.2.11 & que non pue dan le una tarse contra ellos . \textbf{ A estas quatro cosas se traen todas } las que dichos son . & ut contra eos non possint insurgere . \textbf{ Ad haec autem quatuor , } praedicta omnia reducuntur : \\\hline
3.2.11 & para que alguno seleunate o faga algun ayuntamiento \textbf{ contra el tyrano de alguna destas quatro coosas puede sallir ¶ } La primera sale le de sabiduria e de grant entendimiento . & praedicta omnia reducuntur : \textbf{ nam quantum ad praesens spectat , ad hoc quod aliquis inuadat vel machinetur aliquid contra tyrannum ex aliquo praedictorum quatuor videtur procedere . Primo enim potest hoc accidere ex magnanimitate , } ut quia insurgens est tanti cordis \\\hline
3.2.11 & contra el tyrano de alguna destas quatro coosas puede sallir ¶ \textbf{ La primera sale le de sabiduria e de grant entendimiento . } por que cuyda que podra fallar tantas carreras e tantas maneras & nam quantum ad praesens spectat , ad hoc quod aliquis inuadat vel machinetur aliquid contra tyrannum ex aliquo praedictorum quatuor videtur procedere . Primo enim potest hoc accidere ex magnanimitate , \textbf{ ut quia insurgens est tanti cordis } ut nihil reputet magnum . \\\hline
3.2.11 & La segunda paresçe \textbf{ que salle de omne de grant coraçon . } por que el que se leunata contra el tyra no es de tan grant coraçon & ut nihil reputet magnum . \textbf{ Secundo ex industria et sagacitate , } ut quia credit se tot adinuenire vias \\\hline
3.2.11 & que salle de omne de grant coraçon . \textbf{ por que el que se leunata contra el tyra no es de tan grant coraçon } que non tiene por grant cosa de leunatarse contrael tirano . & Secundo ex industria et sagacitate , \textbf{ ut quia credit se tot adinuenire vias } et versutias posse , \\\hline
3.2.11 & por que el que se leunata contra el tyra no es de tan grant coraçon \textbf{ que non tiene por grant cosa de leunatarse contrael tirano . } ¶ La . iij es que esto puede ser por fiança & ut quia credit se tot adinuenire vias \textbf{ et versutias posse , | ut valeat tyrannum perimere . } Tertio potest contingere ex potentia , \\\hline
3.2.11 & por el poderio çiuil a cometera al tirano . \textbf{ La quarta puede acaesçer por grant uagar } assi que si los çibdadanos fueren muy ricos & propter ciuilem potentiam tyrannum inuadit . \textbf{ Quarto hoc poterit accidere ex nimio ocio , } ut si ciues nimis sint opulenti , \\\hline
3.2.11 & assi que si los çibdadanos fueren muy ricos \textbf{ e estidieren en grant uagar } que non ayan de obrar nada . & ut si ciues nimis sint opulenti , \textbf{ et vacent nimio ocio } quia mens humana nescit ociosa esse , \\\hline
3.2.11 & por que los sus subditos non sean osados \textbf{ nin de grandes coraçones . } Otrossi procuran de destroyr los sabios & Contra haec ergo quatuor procurant tyranni perimere excellentes , \textbf{ ne sui subditi sunt magnanimi : } destruere sapientes , \\\hline
3.2.11 & ca este tal seria \textbf{ assi commo medio dios } si ninguna cosanon ouiesse de tyrania & qui sit omnino Rex quin in aliquo tyrannizet : \textbf{ esset enim quasi semideus , } si nihil de tyrannide participaret . \\\hline
3.2.11 & e tanto meior es el sennorio \textbf{ quanto mas se allega a buen gouernamiento de Rey } e mas se arriedra de manera del tirano . & et tanto est melius dominium , \textbf{ quanto plus accedit ad regnum , } et est longius a tyranno . Voluimus quidem utriusque opera enarrare , et utrasque cautelas describere : \\\hline
3.2.11 & mostrando que las obras del tirano son muy malas \textbf{ e desto paresçe manifiesta miente } que la tirama es much de escusar alos Reyes & magis enim apparet opera regia esse optima , ostendendo tyrannica esse pessima . \textbf{ Ex hoc autem manifeste patet , } tyrannidem maxime esse fugiendam a regibus : \\\hline
3.2.12 & a tanto los sesos de los omes son enclinados a mal \textbf{ que mucho es prouechoso en la carrera de buenans costunbres mostrar } por muchͣs maneras & Adeo sensus hominum sunt ad malum proni , \textbf{ ut valde utile sit in via morum multis viis ostendere , } et multis rationibus probare malum , \\\hline
3.2.12 & que mucho es prouechoso en la carrera de buenans costunbres mostrar \textbf{ por muchͣs maneras } e prouar por muchͣs razones & Adeo sensus hominum sunt ad malum proni , \textbf{ ut valde utile sit in via morum multis viis ostendere , } et multis rationibus probare malum , \\\hline
3.2.12 & por muchͣs maneras \textbf{ e prouar por muchͣs razones } que el mal e el pecado de si es cosa muy uil & ut valde utile sit in via morum multis viis ostendere , \textbf{ et multis rationibus probare malum , } et vitium de se esse quid vile et fugiendum , \\\hline
3.2.12 & que temos prouar \textbf{ que los Reyes con grand acuçia se deuen guardar } que non se tornen tiranos & eo quod opera tyranni sunt pessima : \textbf{ probare volumus Reges summa diligentia cauere debere } ne conuertantur in tyrannos , \\\hline
3.2.12 & por que qual si quier cosa de corrupcion e de maldat \textbf{ que es en los otros malos prinçipados } todo es ayuntado en la tyrama & quia quicquid corruptionis \textbf{ et iniquitatis est in aliis peruersis principatibus , } totum in tyrannidem congregatur . Sunt enim \\\hline
3.2.12 & Et el sennorio malo del pueblo son tres señorios muy malos . \textbf{ Enpero la tyrania es peor sennorio } que ninguno de los otros . & quod idem est quod quasi peruersio et corruptio populi . Tyrannis vero corruptus principatus diuitum , \textbf{ et iniquum dominium populi , sunt regimina peruersa . Tyrannis } tamen est peruersior principatus : \\\hline
3.2.12 & en el quanto libro delas politicas . \textbf{ qual quier cosa de maldat es en el mal sennorio de los ricos } e en el mal señorio del pueblo todo es ayuntado enla tirania & quia ut probat Philosophus 5 Polit’ \textbf{ quicquid peruersitatis est aliquo principatu diuitum , } et in peruerso dominio populi , \\\hline
3.2.12 & qual quier cosa de maldat es en el mal sennorio de los ricos \textbf{ e en el mal señorio del pueblo todo es ayuntado enla tirania } la qual cosa se prueua & quicquid peruersitatis est aliquo principatu diuitum , \textbf{ et in peruerso dominio populi , | totum congregatur in tyrannidem . } Quod sic patet : \\\hline
3.2.12 & Conuiene a saber riquezas de dineros . \textbf{ Vv electa connes corporales . } Et guarda de su cuerro . & quantum , ad praesens spectat , tria intendunt , \textbf{ videlicet pecuniam , corporales delicias , } et custodiam corporis . \\\hline
3.2.12 & nin aplazentias corporales \textbf{ si non aguauiando el pueblo en muchͣs cosas . } ca tal commo este & Nam nullus spreto communi bono intendit \textbf{ ad pecuniam et voluptates corporis , nisi in multis offendat populum : } nam talis ut pecuniam habeat de praedatur ciues , \\\hline
3.2.12 & e dela ira del pueblo . \textbf{ e por ende ha muy grant acuçia dela guarda del su cuerpo . } Todos estos males & sed semper dubitans de furia populi , \textbf{ maxime solicitatur | circa custodiam corporis . } Omnia autem huiusmodi mala reperta in peruerso principatu diuitum , congregantur in tyrannide . \\\hline
3.2.12 & Todos estos males \textbf{ que son en el mal señorio de los ricas son fallados en la tirania e el tirano } assi commo diches de suso non entienden enel bien comun & circa custodiam corporis . \textbf{ Omnia autem huiusmodi mala reperta in peruerso principatu diuitum , congregantur in tyrannide . } Nam tyrannus \\\hline
3.2.12 & assi commo dicho es en bien de honrra \textbf{ mas en delecta conn escarnales . Lo terçero el tirano ha grant acuçia en poner guarda en su cuerpo } assi con mo paresçe por enxientlo . & sed pecuniam . Rursus ut supra dicebatur non intendit bonum honorificum , \textbf{ sed delectabile . Tertio tyrannus maxime delectatur circa custodiam corporis , } eo quod videat se plurimos offendisse . \\\hline
3.2.12 & e fizo poñuallesteros con ballestas armadas contrael . \textbf{ Et estonçe commo aquel su hͣrmano tomasse grant espanto e grant temor } e ouiesse miedo de ser ferido del cuchiello & ø \\\hline
3.2.12 & que lo non podia fazer \textbf{ por muchs peligros qual esta una aprestados . } Et dixo el tirano & timens ab acuto gladio vulnerari , et a ballistis perfodi , ait Tyrannus , Gaude frater , \textbf{ hylarem ostende vultum . Respondente illo quod non posset propter imminentia pericula : } inquit Tyrannus quod nec ipse gaudere poterat , \\\hline
3.2.12 & que nin el esso mesmo non se podia gozar \textbf{ por que estauen mayor peligro } por que tanços males el auia fecho & inquit Tyrannus quod nec ipse gaudere poterat , \textbf{ eo quod in maiori periculo existebat , } quia in tot forefecerat \\\hline
3.2.12 & ca los tiranios maguera pierdan la uida perdirable \textbf{ avn en esta uida tenporal apenas ha vn buen dia } por que sienpre se veen estar en grandes peligros . & cum hoc quod perdunt aeternam vitam , \textbf{ in hac temporali vita vix unam diem bonam habent , } eo quod semper videant sibi pericula imminere . Rursus , \\\hline
3.2.12 & avn en esta uida tenporal apenas ha vn buen dia \textbf{ por que sienpre se veen estar en grandes peligros . } Otrossi non han tantas riquezas & in hac temporali vita vix unam diem bonam habent , \textbf{ eo quod semper videant sibi pericula imminere . Rursus , } pecuniam non tantam habent tyranni , quantam veri reges : \\\hline
3.2.12 & nin tantos dineros los titannas \textbf{ quantas han los uerdaderos Reyes ¶ } Lo vno por que les conuiene a ellos responder muchͣs cosas superfluas . & pecuniam non tantam habent tyranni , quantam veri reges : \textbf{ tum } quia oportet eos multa expendere superuacue , tum etiam quia veris regibus plus donatur \\\hline
3.2.12 & Lo vno por que les conuiene a ellos responder muchͣs cosas superfluas . \textbf{ lo otro por que alos uerdaderos Reyes mas cosas son dadas } e ofresçidas por amor & tum \textbf{ quia oportet eos multa expendere superuacue , tum etiam quia veris regibus plus donatur } ex amore \\\hline
3.2.12 & que alos tiranos \textbf{ e mas han ellos de buena parte } que los tiranos de mala e de robo del pueblo & quam tyrannis proueniat \textbf{ ex praedatione populi . } Experti enim sumus tyrannizantes \\\hline
3.2.12 & esto es muy derstable \textbf{ e por ende el tirano es pri uado de grant delectaçion } quando bee & et credere se odiosum esse multitudini , \textbf{ est maxime tristabile . Priuatur ergo tyrannus a maxima delectatione , } cum videat se esse populis odiosum . Viso tyrannidem cauendam esse , \\\hline
3.2.12 & por que es mala \textbf{ e por que los malos prinçipados } e los malos señorios de los rricos son ayuntados en ella , & cum videat se esse populis odiosum . Viso tyrannidem cauendam esse , \textbf{ quia iniqui principatus diuitum congregantur in ea : restat } videre esse eam cauendam , \\\hline
3.2.12 & e por que los malos prinçipados \textbf{ e los malos señorios de los rricos son ayuntados en ella , } finca de ver que es de foyr e de aborresçer avn & cum videat se esse populis odiosum . Viso tyrannidem cauendam esse , \textbf{ quia iniqui principatus diuitum congregantur in ea : restat } videre esse eam cauendam , \\\hline
3.2.12 & finca de ver que es de foyr e de aborresçer avn \textbf{ por que en ella son ayuntados los males del mal priͥnçipado del pueblo } por que quando el pueblo enseñorea malamente non entiende guaedar & videre esse eam cauendam , \textbf{ eo quod etiam in ipsa congregantur mala peruersi principatus populi . } Cum enim populus principatur peruerse , \\\hline
3.2.13 & e non entender al bien comun \textbf{ commo quier que por munchons rrazones mostramos } por lo que dicho ess & et non intendere commune bonum ; \textbf{ licet pluribus viis ostenderimus per iam dicta detestabile } et periculosum esse regiae maiestati tyrannizare , \\\hline
3.2.13 & por lo que dicho ess \textbf{ que cosa peligrosa e aborresçible deue ser ala rreal maiestad tiranzar } e non gouernar derechamente el pueblo & licet pluribus viis ostenderimus per iam dicta detestabile \textbf{ et periculosum esse regiae maiestati tyrannizare , } et non recte gubernare populum : \\\hline
3.2.13 & mas queremos declarar en este capitulon \textbf{ que por munchans rrazones los sb̃ditos asechan alos tiranos } por que es muy pelignis ala vida del tirano & Volumus autem declarare in hoc capitulo \textbf{ quod quia multis de causis subditi insidiantur tyrannis , } et quia valde est periculosa vita tyrannica \\\hline
3.2.13 & en arredrarse del gouernamjento derecho \textbf{ por ende los rreyes deuen estudiar con grand ciudado e con grand acnçia } que non tira njz en dexado el gouna mj̊ derecho & et quia valde est periculosa vita tyrannica \textbf{ et deuiatio a recto regimine , reges cura peruigili studere debent , } ne delinquentes rectum gubernaculum , tyrannizent . \\\hline
3.2.13 & por ende los rreyes deuen estudiar con grand ciudado e con grand acnçia \textbf{ que non tira njz en dexado el gouna mj̊ derecho } Ca cuenta el philosofo & et deuiatio a recto regimine , reges cura peruigili studere debent , \textbf{ ne delinquentes rectum gubernaculum , tyrannizent . } Narrat autem Philosophus 5 Polit’ sex causas , \\\hline
3.2.13 & La primera es por temorça munchons \textbf{ que son de flaco coraçon } quando temen muncto & Prima est propter timorem . \textbf{ Nam multi pusillanimes existentes , } cum nimis timent , \\\hline
3.2.13 & que quien muncho faze foyr al temeroso \textbf{ por fuerça lo costrange desee oscido en essa misma manera } avn en las otras ainalias & Unde et prouerbialiter dicitur , \textbf{ quod nimis fugans timidum , vi compellit esse audacem . Sic etiam et alia animalia } quasi communiter timent hominem , \\\hline
3.2.13 & que rresçibrien del omne tarde ominca daño le cometeren \textbf{ e pues que asi es las bestias enla mayor parte cos cringidas } por temor assecha al omne & et nisi crederent se laedi ab eo , raro \textbf{ aut nunquam inuaderent ipsum : bestiae ergo } ut plurimum timore compulsae insidiantur homini et inuadunt eum : \\\hline
3.2.13 & e acometenlo \textbf{ e por ende en esta misma manera munchas vezes los subditos asechanal tirano } e matan lo temjendo & ut plurimum timore compulsae insidiantur homini et inuadunt eum : \textbf{ hoc ergo modo multotiens subditi insidiantur tyranno , } et perimunt ipsum tyrannum , \\\hline
3.2.13 & es por por tuerto \textbf{ que rresçibien dehcanatanl cosa } e ᷤalos omes desear uengança del mal que rresçibe & Secunda causa propter quam subditi tyranno insidiantur , \textbf{ est iniuria quam passi sunt ab ipso : } naturale est enim desiderare vindictam , \\\hline
3.2.13 & que rresçibien dehcanatanl cosa \textbf{ e ᷤalos omes desear uengança del mal que rresçibe } Por la qual cosa omero aquel poeta dezia & est iniuria quam passi sunt ab ipso : \textbf{ naturale est enim desiderare vindictam , } propter quod Homerus dicebat , \\\hline
3.2.13 & e por ende \textbf{ e ssi por que es cosa delectable de querer los omes natanl mente vengança munchos asechan al tirano } e por muy gerad saña & quia est appetitus poenae in vindictam : \textbf{ quia ergo sic est delectabile vindictam exposcere , } multi insidiantur tyranno , \\\hline
3.2.13 & e algunas vegadas Lo matan \textbf{ ca por la mayor parte los tiranos fazen algunas cosas } por que se fazen despreçiados de los pueblos & et aliquando perimunt ipsum : \textbf{ quia ut plurimum tyranni faciunt ea per quae se contemptibiles reddunt . } Nam cum non quaerant bonum commune , \\\hline
3.2.13 & mas quieren delectaçiones del su cuerpo \textbf{ por ende enla mayor parte non son mesurados } mas son golosos & sed delectationes corporis , \textbf{ ut plurimum non sunt sobrii sed gulosi , } vel non sunt casti , \\\hline
3.2.13 & ca el rrer̃sur danna palo despçiado \textbf{ el bien tomun todo se dapla lururia } e vn prinçipe despciandol & ubi habemus de Sardinapalo rege , \textbf{ qui spreto communi bono totum se dedit venereis . Quidam vero dux contemnens eum , } eo quod vitam pecudum elegisset , inuasit , \\\hline
3.2.13 & por ganar honrra o auer ganançia , \textbf{ Ca commo la honrra e la gloria deste mundo sean muy grandes bienes entre los bienes } que paresçen alos omes de fuera munchos & aut propter lucrum adipiscendum . \textbf{ Nam cum honor et gloria | inter bona exteriora sint bonum maximum , } multi videntes tyrannum non quaerere nisi honorem \\\hline
3.2.13 & que veen enel tirano acometen ler matanle , \textbf{ Ca essa misma manera avri } por que munchos cuda & volentes adipisci honorem \textbf{ et gloriam quam conspiciunt in tyranno , inuadunt eum , } et perimunt ipsum . Sic etiam quia multi reputant pecuniam esse maximum bonum , \\\hline
3.2.13 & que el tirano non entiende \textbf{ si non a grant ganançia propria } e allegar grand auer acor̃tele & et perimunt ipsum . Sic etiam quia multi reputant pecuniam esse maximum bonum , \textbf{ videntes tyrarannum non intendere nisi ad lucrum proprium , } et ad congregandam pecuniam inuadunt ipsum , \\\hline
3.2.13 & si non a grant ganançia propria \textbf{ e allegar grand auer acor̃tele } por quel tomne los tesoos & videntes tyrarannum non intendere nisi ad lucrum proprium , \textbf{ et ad congregandam pecuniam inuadunt ipsum , } et accipiunt thesauros eius . \\\hline
3.2.13 & Oude Dize el philosofo enel quinto libro delas politicas \textbf{ que algunos veyendo las grandes ganançias } e las grandes honrras & Unde dicitur 5 Polit’ \textbf{ quod quidam tyrannos inuadunt , | videntes lucra magna , } et honores magnos existentes in ipsis . \\\hline
3.2.13 & que algunos veyendo las grandes ganançias \textbf{ e las grandes honrras } que son enlos tiranos acometenlos & videntes lucra magna , \textbf{ et honores magnos existentes in ipsis . } Quinto fiunt insidiae tyrannis ab aliquibus , \\\hline
3.2.13 & sse ponen asechons los tiranos \textbf{ e por que non sean puestos asechos a la rreal magestad . } E por que el rrey non dubde sienprede si & His ergo de causis fiunt insidiae contra tyrannum . \textbf{ Ne ergo fiant insidiae contra regiam maiestatem , } et ne Rex semper dubitet \\\hline
3.2.13 & ca si non fisjere tueᷤto alos sbditos \textbf{ en si fuere bueno rmesurador non escogiere vida bestial } e menespretiable si honrrare los nobles los otros & Nam si non iniuriatur subditis , \textbf{ si est continens | et sobrius , et non eligit vitam bestialem et contemptibilem , } si honorat insignes , \\\hline
3.2.13 & non asi commo tirano \textbf{ mas asi commo uerdadero rrey adura } a todos & et in omnibus se habet non ut tyrannus , \textbf{ sed ut verus rex : } inducit omnes existentes in regno ad amorem eius , \\\hline
3.2.14 & por que non le asech \textbf{ aguera por munchos caplons sobredh̃os ayamos en dos Poslons rreyes } e los prinçipes & et tollit ab eis omnem materiam et causam quare insidientur ipsi . \textbf{ Licet per multa capitula praecedentia induxerimus Reges et Principes } ut non tyrannizent , \\\hline
3.2.14 & por que se no fagantiranos . \textbf{ ca la tiranja e el mal señorio en munchas } mas maneras se corronpe & ne efficiantur tyranni , \textbf{ eo quod tyrannis } et principatus peruersus pluribus modis corrumpitur , \\\hline
3.2.14 & commo esta non ha contrariedat nin departimiento . \textbf{ mas cotesçe que en muchͣs maneras desuia ome dela sennal } quando lança & nec contrarietas : \textbf{ sed contingit multipliciter deuiare ab ipso . } Quare in deuiationibus contrarietas esse potest , \\\hline
3.2.14 & e lançar ala mano derechͣ e ala mano desquierda son cosas contrarias . \textbf{ Et en essa misma manera en este proponimiento } e prinçipado derech non es contra no al prinçipado derech . & ut si multum opponitur pauco , proiectio ultra signum contrariatur proiectioni citra : et proiectio in dextrum proiectioni in sinistrum . \textbf{ Sic etiam in proposito , } principatus rectus non contrariatur recto principatui : \\\hline
3.2.14 & e prinçipado derech non es contra no al prinçipado derech . \textbf{ Mas el mal prinçipado } e el mal senorio es contrario al malo . & principatus rectus non contrariatur recto principatui : \textbf{ sed peruersus peruerso . } Una ergo tyrannis potest contrariari alii , \\\hline
3.2.14 & Mas el mal prinçipado \textbf{ e el mal senorio es contrario al malo . } Et por ende vna tirama puede ser contraria a otra & principatus rectus non contrariatur recto principatui : \textbf{ sed peruersus peruerso . } Una ergo tyrannis potest contrariari alii , \\\hline
3.2.14 & assi cotio la tirania del pueblo \textbf{ escontrana ala tirama del mal prinçipado } Et vn prinçipado tiranico es contrario a otro prinçipado tiranico e malo & et corrumpere ipsam ; \textbf{ ut tyrannis populi contrariatur tyrannidi monarchiae : } et una monarchia tyrannica contrariatur alii . \\\hline
3.2.14 & Mas avn el bien es contrario al mal \textbf{ ca non solamente el tirano cont dize al tirano . } mas avn el Rey si vee & sed etiam bonum malo contrariatur : \textbf{ non solum enim tyrannus tyranno obuiat , } sed etiam et Rex \\\hline
3.2.14 & que cunple cotra dize al tirano . \textbf{ Et pues atantos peligros es puesto el tirano } e en tantas manerasse ha de desfazer el su prinçipado & sed etiam et Rex \textbf{ ( si viderit expedire ) tyranno se opponit . Tot ergo discriminibus oppositus est tyrannus , } et tot modis habet dissolui eius principatus . Regium autem dominium non tot periculis exponitur , \\\hline
3.2.14 & non es bue no de tiranizar \textbf{ mas el regno e el sennorio bueno non es puesto atantos peligros } nin en tantas maneras se puede desatar commo la tirana . & ( si viderit expedire ) tyranno se opponit . Tot ergo discriminibus oppositus est tyrannus , \textbf{ et tot modis habet dissolui eius principatus . Regium autem dominium non tot periculis exponitur , } nec tot modis habet dissolui . \\\hline
3.2.14 & por todo su poder a destroyr \textbf{ e aprimiir el uerdadero Rey . } Enpero ningun uerdadero rey non se pone contra otro uerdadero Rey & Nam licet tyrannus satagat pro viribus verum regem opprimere , \textbf{ nullus tamen verus Rex vero Regi se opponit : } nam nullus bonus et virtuosus insequitur alios bonos et virtuosos , \\\hline
3.2.14 & e aprimiir el uerdadero Rey . \textbf{ Enpero ningun uerdadero rey non se pone contra otro uerdadero Rey } ca ningun bueno e uirtuoso non perssigue a otros buenos e uirtuosos . & Nam licet tyrannus satagat pro viribus verum regem opprimere , \textbf{ nullus tamen verus Rex vero Regi se opponit : } nam nullus bonus et virtuosus insequitur alios bonos et virtuosos , \\\hline
3.2.14 & que alguon bueno e uirtuoso se pusiesse contra otro bueno e uirtuoso dexaria de ser bueno e uirtuoso . \textbf{ Et pues que assi es conuiene ala Real magestad } de escusar con grant estudio & et virtuosis se opponeret , deficeret esse bonus et virtuosus . \textbf{ Decet ergo regiam maiestatem summo studio cauere tyrannidem , } ne praedictis periculis exponatur . \\\hline
3.2.14 & Et pues que assi es conuiene ala Real magestad \textbf{ de escusar con grant estudio } e con grant acuçia la tirama & et virtuosis se opponeret , deficeret esse bonus et virtuosus . \textbf{ Decet ergo regiam maiestatem summo studio cauere tyrannidem , } ne praedictis periculis exponatur . \\\hline
3.2.14 & de escusar con grant estudio \textbf{ e con grant acuçia la tirama } por que non se pongan alos peligros sobredichos & ø \\\hline
3.2.15 & La primera es que non consienta en su regno muchos pequanos males \textbf{ ca muchs pequannos males egualan se a vn grant mal } assi commo dize el philosofo en las politicas bien & non permittere in suo regno transgressiones modicas . \textbf{ Nam multae modicae transgressiones } ( ut ait Philos’ ) aequantur uni magnae , \\\hline
3.2.15 & assi commo dize el philosofo en las politicas bien \textbf{ assi commo muchas pequanans despenssas se ygualan a vna despenssa grade } por que los pequa non s males & ( ut ait Philos’ ) aequantur uni magnae , \textbf{ sicut multae paruae expensae aequiualent uni magno sumptui : } paruae enim transgressiones \\\hline
3.2.15 & por que los pequa non s males \textbf{ si fueren muchs apareian a omne a grandes males . } ca el que menospreçia lo poco ayna cae en lo much & paruae enim transgressiones \textbf{ si multae sint , | disponunt ad transgressiones magnas . } Nam \\\hline
3.2.15 & que son en el regno \textbf{ poniendo los en alguons prinçipados } e honrrando los & qui sunt in regno , \textbf{ introducendo eos ad aliquos principatus , } honorando eos , \\\hline
3.2.15 & Estos tales \textbf{ segunt el philosofo en el vii̊ libro delas politicas semeian al fierro } que mientra es en vso continuado esta luzio e claro . & quibus non fuit curae de virtutibus aliis nisi de fortitudine , \textbf{ secundum Philosophum 7 Politicorum assimilantur ferro , } quod dum est in continuo exercitio claritatem habet , \\\hline
3.2.15 & que mientra es en vso continuado esta luzio e claro . \textbf{ mas sil uiengo tienpo estudiere } que non se usare tomarahurin . & secundum Philosophum 7 Politicorum assimilantur ferro , \textbf{ quod dum est in continuo exercitio claritatem habet , } sed si diu remanet inofficiosum , rubiginem contrahit . Sic et tales bellantes fiunt obedientes Principi : \\\hline
3.2.15 & Otrossi si alguno comneçare de nueuo a regnar \textbf{ por que contra tal prinçipe nueuo de ligero seleuna tan los cibdadanos } por que se non le unaten contra el & si quis super aliquos de nouo principari coepit , \textbf{ quia contra talem principatum | facilius insurgitur ; } ne ciues insurgant in Principem , \\\hline
3.2.15 & mas si el regnado fuere en estado de antiguedat \textbf{ Et aquell sennor fuere natural de antiguo tienpo } assi que non sea memoria delos omes & sed si regnum diu in statu perstiterit , \textbf{ et dominus ille sit naturalis , } ita quod quasi non sit in memoria hominum ex quo ille \\\hline
3.2.15 & para se destroyr La quinta cosa \textbf{ que guardan la poliçia es catar con grant acuçia } en qual manera se han & ø \\\hline
3.2.15 & en qual manera se han \textbf{ aquellos que honrro la real maiestad } e aquellos que promouio a alguna honrra o alguna dignidat & quomodo se habeant , \textbf{ quos maiestas regia honorauit , et promouit ad aliquam praeposituram , } vel ad \\\hline
3.2.15 & que mucho salua la poliçia \textbf{ es que el Rey piensse con grant acuçia } a quales dio los maestradgos e las dignidades . & Quare maxime saluatiuum politiae est , \textbf{ regiam maiestatem considerare diligenter quos praeficit in aliquibus magistratibus : } et si bene se habuerint , \\\hline
3.2.15 & que guarda la poliçia . \textbf{ es non dara ninguer muy grant señorio } ca los grandes sennorios & Sextum est , \textbf{ nulli valde magnum dominium conferre . } Nam magna dominia ex nimia bonitate \\\hline
3.2.15 & es non dara ninguer muy grant señorio \textbf{ ca los grandes sennorios } por la grant bondat dela uentura & nulli valde magnum dominium conferre . \textbf{ Nam magna dominia ex nimia bonitate } et fortitudine \\\hline
3.2.15 & ca los grandes sennorios \textbf{ por la grant bondat dela uentura } por la mayor parte corronpen las uoluntades de los oens & nulli valde magnum dominium conferre . \textbf{ Nam magna dominia ex nimia bonitate } et fortitudine \\\hline
3.2.15 & por la grant bondat dela uentura \textbf{ por la mayor parte corronpen las uoluntades de los oens } por que se fagan traspassadores dela iustiçia & Nam magna dominia ex nimia bonitate \textbf{ et fortitudine | ut plurimum corrumpunt mentes hominum , } ut fiant transgressores iustitiae . Est autem haec cautela maxime utilis ad homines , \\\hline
3.2.15 & de los quales el Rey non ha auido çierta \textbf{ e luenga prueua } e por ende muches de guardar & de quibus Rex certam et diuturnam experientiam non accepit . \textbf{ Ideo potissime obseruandum est , } ne repente constituatur aliquis in maximo principatu . \\\hline
3.2.15 & e por ende muches de guardar \textbf{ que adesora non sea ninguno puesto en muy grant senorio . } La vi jncosa & Ideo potissime obseruandum est , \textbf{ ne repente constituatur aliquis in maximo principatu . } Septimum saluans regnum et politiam , \\\hline
3.2.15 & e la poliçia es \textbf{ que el Rey e el prinçipe aya grant amor al bien del regno } e al bien dela çibdat & Septimum saluans regnum et politiam , \textbf{ est Regem siue principantem habere dilectionem | et amorem ad bonum regni , } et ad politiam , \\\hline
3.2.15 & assi commo dize el philosofo en el segundo libre de la rectorica \textbf{ que alli es grant salud } do son los muchs consseios & ut dicitur 2 Rhet’ ibi enim est magna salus , \textbf{ ubi } et multa consilia . \\\hline
3.2.15 & que alli es grant salud \textbf{ do son los muchs consseios } ca si el Rey amare el bien del regno saluat se ha el regno & ubi \textbf{ et multa consilia . | Quare } si Rex \\\hline
3.2.15 & ca si el Rey amare el bien del regno saluat se ha el regno \textbf{ ca temiendo que cotezccan alguas cosas contrarias en el regno aura tomar muchs consseios } en qual manera pueda promouer el bien del regno & bonum regni diligat , saluabitur regnum ; \textbf{ quia timens ne in regno aduersa contingant , } adhibebit multa consilia qualiter possit bona regni promouere , \\\hline
3.2.15 & e la poliçia es auer poderio çiuilca \textbf{ assi commo dize el philosofo en el libro delas grandes costunbres . } la iustiçia guarda las cortesias et las buenas maneras delas çibdades . & est habere ciuilem potentiam . \textbf{ Nam ( ut dicitur in Magnis moralibus ) } iustitia urbanitates conseruat . \\\hline
3.2.15 & assi commo dize el philosofo en el libro delas grandes costunbres . \textbf{ la iustiçia guarda las cortesias et las buenas maneras delas çibdades . } Mas la iustiçia non se puede guardar en el regno & Nam ( ut dicitur in Magnis moralibus ) \textbf{ iustitia urbanitates conseruat . } Sed iustitia in regno conseruari non potest , \\\hline
3.2.15 & auer much sassechadores \textbf{ e muchs pesquiridores } que escodrinen e sepan los fechos de los çibdadanos & aut Princeps si vult seruare iustitiam \textbf{ et vult punire transgressores iusti , habere multos exploratores , } et multos inquisitores inuestigantes facta ciuium , et inquirentes unde ciues accipiunt \\\hline
3.2.15 & que traspassan la iustiçia . \textbf{ La ixͣ cosa } que much salua el regno & et praeseruare regnum a maleficis , \textbf{ et transgressoribus iusti . Nonum maxime saluans regnum , } est esse regem bonum et virtuosum . \\\hline
3.2.15 & en el quinto libro delas politicas \textbf{ mayor uirtud es menester enla guatda dela çibdat } e del regno & Nam ut dicitur 5 Politicorum , \textbf{ maior virtus requiritur in custodia ciuitatis et regni , } quam in duce exercitus . \\\hline
3.2.15 & ca quando alguno ha prouado \textbf{ por luengo tienpo los negoçios del regno } de ligero puede penssar & nam \textbf{ cum quis diu expertus est regni negocia , } de leui arbitrari poteritque bonum statum regni corrumpunt , et saluant . Decet ergo Regem frequenter meditari et habere memoriam praeteritorum \\\hline
3.2.15 & de ligero puede penssar \textbf{ qual cosa corronpe el buen estado del regno } e qual cosa lo salua . & ø \\\hline
3.2.15 & Et pues que assi es conuiene al Rey de penssar mucha menudo \textbf{ e muchͣs uezes delas cosas que passaron . } Et conuiene le de auer memoria de los fecho passados & cum quis diu expertus est regni negocia , \textbf{ de leui arbitrari poteritque bonum statum regni corrumpunt , et saluant . Decet ergo Regem frequenter meditari et habere memoriam praeteritorum } quae contigerunt in regno , \\\hline
3.2.15 & e commo en los tp̃os passados \textbf{ fue meior guardado el buen estado del regno } por que sepa entender & et quomodo temporibus retroactis melius conseruatus fuerit \textbf{ bonus status regni , } ut sciat cognoscere qualiter principari debeat , \\\hline
3.2.16 & rcho es de suso \textbf{ que quatro cosas son de penssar } en el gouernamiento dela çibdat . & recte regum gubernare non poterit . \textbf{ Dicebatur supra , quatuor consideranda esse in regimine ciuitatis . } videlicet Principem , \\\hline
3.2.16 & e quales tuertos . \textbf{ Et declaramos en commo el regno era muy buen prinçipado } e la tirania muy malo . & et qui illorum peruersi , \textbf{ et declarauimus regnum esse optimum principatum , } et tyrannidem pessimum ; manifestauimus item quod sit Regis officium , \\\hline
3.2.16 & para que derechamente gouierne el pueblo qual es acomendado . \textbf{ e prouamos por nuchͣs razones } que conuenia al Rey de ser acuçioso & probauimus \textbf{ etiam multis viis decere Regem vigilem curam assumere , } ne conuertatur in tyrannum : \\\hline
3.2.16 & que conuenia al Rey de ser acuçioso \textbf{ e de tomar grant cuydado en ssi para se non tomarentirano . } Et quanto parte nesçe a este negoçio presente & etiam multis viis decere Regem vigilem curam assumere , \textbf{ ne conuertatur in tyrannum : | et tandem , } quantum spectat ad praesens negocium , \\\hline
3.2.16 & que por çierto alguno tomara consseio non de aquellas cosas \textbf{ de que el omne non sabio e el omne loco toma consseio . } Mas de aquellas cosas & cum dicat Philosophus 3 Ethic’ consiliabitur utique aliquis non pro quibus consiliatur insipiens , \textbf{ et insanus , } sed pro quibus sapiens , \\\hline
3.2.16 & de que el sabio \textbf{ e el que ha buen entendimiento toma consseio } por ende primero deuemos ver quales cosas son aconsseiables & sed pro quibus sapiens , \textbf{ et intellectum habens . } Ideo primo videndum est quae sunt consiliatiua , \\\hline
3.2.16 & de que non deuemos tomar conseios . \textbf{ Mas quanto alo presente nos podemos poner seys cosas } que non caen so consseio . & ne tanquam ignorantes consiliari velimus de quibus non sunt consilia adhibenda . Possumus autem tangere sex , \textbf{ quantum ad praesens spectat , } quae sub consilio non cadunt . Primo enim quaecunque sunt immutabilia consilium nostrum subterfugiunt . Nam ideo consiliamur , ut regulemur in actionibus nostris , \\\hline
3.2.16 & que se non pueden escusar \textbf{ e que non se pueden mudar non caen ssonro conseio } e por ende dize elpho & quae ergo vitari non possunt , \textbf{ et quae mutationi non subiacent , | sub consilio non cadunt . } Ideo dicitur 3 Ethi’ \\\hline
3.2.16 & que han a fazer calentura o frio \textbf{ segunt departidost pons pueden caer en nuestro consseio } non por si & secundum diuersa tempora calorem \textbf{ et frigiditatem , } non per se , \\\hline
3.2.16 & L tercero non cae so consseio \textbf{ aquellas cosas que se fazen muchͣs uezos } li le fazen naturalmente & quo tempore quae opera sunt fienda . \textbf{ Tertio non sunt consiliabilia etiam quae fiunt frequenter , } si fiunt a natura . Ideo de imbribus quae semper fiunt tempore hyemali , \\\hline
3.2.16 & e en otras quales se quier cosas \textbf{ que sinen alas nr̃as obras . } Et por ende bien diches & ideo de eis \textbf{ non consiliamur , nisi ut deseruiunt operibus nostris . Hoc ergo modo est consilium circa ipsa sicut et circa cursus syderum , et circa quaecunque alia deseruientia humanis actibus . } Bene ergo dicitur in Ethic’ \\\hline
3.2.16 & que son auentura \textbf{ assi comda e tallar thesoros . } Lo v̊a vnño son todas las obras de los omes conseiables & quae sunt a fortuna , \textbf{ puta de thesauri inuentione . Quinto non sunt consiliabilia } etiam omnia humana opera : \\\hline
3.2.16 & sobre ello \textbf{ avn en essa misma manera } nin los françeses non toman consseio & quod qualiter utique Scythae optime conuersentur , \textbf{ nullus Lacedaemoniorum consiliatur : } nec etiam nulli Gallici consiliantur qualiter optime viuant Indii . Consiliabilia ergo non sunt immutabilia , \\\hline
3.2.16 & que se puden fazer \textbf{ por el ¶Lo vi̊ non caen sosico consseio todas aqllas cosas } que por non seiseden ser fechas & de iis operabilibus , \textbf{ quae fieri possunt per ipsos . Sexto non sunt consiliabilia omnia quae per nos fieri possunt . } Nam quaecunque finaliter per opera nostra adipisci intendimus , \\\hline
3.2.16 & por las quales se pueda sanar mas ligeramente e meior . \textbf{ En essa misma manera el gouernador de la çibdat } e del regno non toma consseio & per quas facilius \textbf{ et melius sanetur . Sic rector ciuitatis et regni non consiliatur } utrum ciues inter se pacem debeant habere , \\\hline
3.2.16 & Et si conuiene \textbf{ que el regno se en buen estado . } mas esto sopone & utrum ciues inter se pacem debeant habere , \textbf{ et utrum regnum oporteat esse in bono statu : } sed haec accipit tanquam certa et nota , \\\hline
3.2.17 & o do consseio es vna questiuo \textbf{ e na demanda . } Ca assi commo dize el philosofo en el sesto libro delas ethicas & sed ut ea quae sunt ad finem . \textbf{ Est autem omne consilium quaedam quaestio , } quia ( ut dicitur 6 Ethicorum ) consilians siue bene siue male consiliatur , \\\hline
3.2.17 & si quier demande conseio \textbf{ bien si quier mal demanda alguna cosa } et razon a demandando alguna cosa . & quaerit aliquid , et ratiocinatur . \textbf{ Propter quod quicunque consiliatur , } quaerit , \\\hline
3.2.17 & e qual manera deuemos tener en los conseios \textbf{ Mas quanto pertenesçe alo presente seys cosasson de guardar } para que sepamos en qual manera auemos de tomar conseio . & restat videre qualiter est consiliandum , \textbf{ et quem modum in consiliis habere debemus . Sunt autem ( quantum ad praesens spectat ) sex obseruanda , } ut sciamus qualiter sit consiliandum . Primum est , \\\hline
3.2.17 & que quanto alguna cosa es mas deteranada \textbf{ e mas çierta tanto menor conseio ha menester } Ante & quia quanto aliquid est magis determinatum , \textbf{ tanto minori eget consilio : } immo si omnino determinatum esset , \\\hline
3.2.17 & e determinadas carreras \textbf{ para se fazer tanto mayor tienpo ha menester omne } para tomar consseio dello . & et quanto minus habet certas et determinatas vias , \textbf{ tanto per plus tempus est consiliandum , } ut de illis viis facilior \\\hline
3.2.17 & as si es auemos de tener manera en los conseios . \textbf{ por que de grandes cosas tomemos consseio . } la tercera cosa es & ut quae sunt apta nata efficere paruum bonum , \textbf{ vel prohibere modicum malum , } non sunt consiliabilia . Est ergo modus attendendus in consiliis , \\\hline
3.2.17 & enpero non es sabio aquel que se esfuerça en su cabeça sola e menospreçia de oyr las suinas de los otros \textbf{ ca de grant sabiduria es en los consseios tener esta manera } que con los otros ayamos acuerdo delo que auemos de fazer & et renuit aliorum audire sententias . \textbf{ Magnae enim prudentiae est in consiliis hunc habere modum : } ut cum aliis conferamus quid agendum , \\\hline
3.2.17 & ca los conseios \textbf{ assi conmo dicho es deuen ser de grandes cosas } e enlas tales cosas ninguno non deue creer & Nam consilia \textbf{ ( ut dictum est ) | esse debent de rebus magnis . } In talibus autem nullus \\\hline
3.2.17 & Et por ende dize el pho en el terçero libro delas ethicas \textbf{ que nos tomamos consłeios en las grandes cosas } desfunzando de nos mismos & quod plura cognoscere possunt multi , \textbf{ quam unus . Ideo dicitur 3 } Ethicorum consiliatores assumimus in magna discernentes , nobis ipsis velut non sufficientibus dignoscere . \\\hline
3.2.17 & assi commo si non fuessemos sufiçientes para lo conosçer \textbf{ Otrossi esto mismo paresçe por otta cosa } por que el conseio es delas obras particulares & Ethicorum consiliatores assumimus in magna discernentes , nobis ipsis velut non sufficientibus dignoscere . \textbf{ Rursus hoc idem patet } ex eo quod consilium est circa agibilia particularia , \\\hline
3.2.17 & que vno solo conuiene de llamar otros \textbf{ para los negoçios . por que por el conseio dellos pueda ser escogida la meior carrera } mas quales consseieros de una ser llamados adelante paresçra & quam unus solus : decet ad huiusmodi negocia alios aduocare , \textbf{ ut per eorum consilium possit eligi via melior quales autem esse debeant consiliarii aduocandi , } in prosequendo patebit . Quarto est in consiliis attendendum , \\\hline
3.2.17 & que cosseio sea dicha conssilendo \textbf{ que quiere tanto dezir commo cosa que se deue callar entre muchs camuches } eston de guardar en los consseios & quod consilium dictum sit a Con et Sileo ut illud dicatur esse Consilium , \textbf{ quod simul aliqui plures silent | et tacent . } Nam maxime est hoc in consiliis attendendum , \\\hline
3.2.17 & que se dixieren en los consseios . \textbf{ ca esto fue lo que enssalço la comunidat de Roma fieldat de buenos consseieros } alos quales todas las cosas & quae ibi sunt tradita . \textbf{ Hoc enim fuit , | quod apud Romanam Rempublicam exaltauit fidelitas consiliantium : } quibus quicquid per consilium in eorum auribus dicebatur , \\\hline
3.2.17 & alabando alas consseieros de roma dize \textbf{ que de grant fe } e muy alto era conssisto no secreto dela comunidat de roma & commendans Romanos consiliatores , \textbf{ ait , } quod fidum et altum erat secretum consistorium reipublicae , silentique salubritate munitum : \\\hline
3.2.17 & e muy alto era conssisto no secreto dela comunidat de roma \textbf{ alos quel guardananca era guaruido de grant fialdat . } en el qual conssistorio & ait , \textbf{ quod fidum et altum erat secretum consistorium reipublicae , silentique salubritate munitum : } cuius limen intrantes abiecta priuata dilectione ita dilectionem publicam inducebant , \\\hline
3.2.17 & Et por ende ponen todo su prinçipado \textbf{ e todo su regno en grant peligro } e dende es avn que vn sabio & et placita promulgantes , \textbf{ exponunt periculo totum principatum } vel totum regnum . Inde est ergo \\\hline
3.2.17 & ¶ La otra que non sean plazenteros \textbf{ assi que parezcan lisongeros auiendo mayor cuydado de fablar cosas plazenteras que uerdaderas . } En essa misma manera abn segunt dize el pho en el terçero libro de la rectorica & et propalatores consiliorum , \textbf{ nec essent placentes , } ut quod essent adulatores , \\\hline
3.2.17 & assi que parezcan lisongeros auiendo mayor cuydado de fablar cosas plazenteras que uerdaderas . \textbf{ En essa misma manera abn segunt dize el pho en el terçero libro de la rectorica } que vn poeta & et propalatores consiliorum , \textbf{ nec essent placentes , } ut quod essent adulatores , \\\hline
3.2.17 & o non ¶ \textbf{ pues que assi es muy bien es de escodrinnar con grant acuçia todo negoçio alto e noble } si es prouechoso delo fazer . & an expediat illud fieri . \textbf{ Bene ergo se habet diligenter quodlibet negocium discutere arduum , } an utile sit illud facere : \\\hline
3.2.18 & mas conuiene de fazerl cosas conseiadas mucho ayna . \textbf{ mar ala Real magestado das aquellas cosas } que deue auer aquel & ø \\\hline
3.2.18 & todas aquellas cosas conuiene \textbf{ que aya el buen conseiero de fech . } ca por esto cada vno razon a bien en los consseios & et bene creditiuus apparenter , \textbf{ expedit ut habeat bonos consiliarios existenter . } Nam ex hoc aliquis persuadet \\\hline
3.2.18 & por que cuydan los omes \textbf{ que es buen consseiero } para dar razon de su consseio . & in consiliis et creditur dictis eius , \textbf{ quia existimatur bonus consiliarius esse ad persuadendum . } Sed ad hoc quod aliquis sit bene creditiuus , \\\hline
3.2.18 & si cuydan los omes \textbf{ del que es buen consseiero . } mas para que el sea buen cosseiero & et redditur ei aliquis creditiuus , \textbf{ si existimet illum bonum consiliatorem esse . } Sed ad hoc quod bonus consiliator existat , \\\hline
3.2.18 & que ha el que bien razona en paresçençia \textbf{ todas deue auer en fech el buen consseiero . } por la qual cosa & habet apparenter , \textbf{ bonus consiliator existenter habere debet . } Quare si scire volumus quales consiliarios habere deceat regiam maiestatem , \\\hline
3.2.18 & si queremos saber \textbf{ quales consseieros deue auer la real magestad e quales e quantas cosas son menester } en los consseios conuiene de saber & bonus consiliator existenter habere debet . \textbf{ Quare si scire volumus quales consiliarios habere deceat regiam maiestatem , } et quae et quot sunt in consiliis requirenda : \\\hline
3.2.18 & que es bueno . \textbf{ ca los buenos omes } avn que non sepan dar razones & vel credatur bonus . \textbf{ Nam bonis hominibus etiam si nullas rationes assignare sciant , creditur eis , } et faciunt fidem auditoribus : \\\hline
3.2.18 & ca assi commo los omes comunalmente \textbf{ e por la mayor parte se engannan en ssi mesmos } e creen que valen mas de quanto valen & Nam sicut homines communiter \textbf{ et ut plurimum decipiuntur circa seipsos , } et credant se plus valere \\\hline
3.2.18 & los que amamos \textbf{ e los que aborresçemos non los iudgamos egual mente . } ca cosa comunales & et odientes \textbf{ non pariter iudicamus . Commune est enim } ut amicorum facta \\\hline
3.2.18 & que los fechs de los amigos \textbf{ por la mayor parte } que los recontemos en vien . & ut amicorum facta \textbf{ ut plurimum deferamus in bonum , } et inimicorum in malum . Passionamur enim circa amicos sicut circa nos ipsos , \\\hline
3.2.18 & Ca fazerse el omne digno de creer \textbf{ e buen amonestador e razonador por si . } nasçe de aqual las cosas & nam reddere se credibilem \textbf{ et bene persuadere per se , } est ex ipsis rebus , \\\hline
3.2.18 & que sea sabio o que sea tenido por sabio . \textbf{ Et pues que assi es todo buen amonestador o razonador } o todo aquel & vel quod credatur esse prudens . \textbf{ Omnis ergo bene persuadens , } vel omnis ille cuius dictis creditur et adhibetur fides , \\\hline
3.2.18 & que parezca tal . Et por ende commo sea dicho \textbf{ que el que es buen amonestador e razonador } e aquel a que los omes creen deue auer en el & apparenter saltem . \textbf{ Itaque cum dictum sit quod qui bene persuadens , } et ille cui fides adhibetur , \\\hline
3.2.18 & e paresçer todas aquellas cosas \textbf{ que ha todo buen conseiero en ssi de fecho . } Et por ende assaz parelçe & debet habere apparenter , \textbf{ oportet quod bonus consiliator habeat existenter : } satis apparet quales consiliatores deceat quaerere regiam maiestatem ; \\\hline
3.2.18 & por razon dessi \textbf{ ca alos buenos pesa todo mal } e toda cosa de denostar . & Ut si boni sint , non mentiantur ratione sui , \textbf{ quia bonis displicet omne malum , } et omne detestabile : \\\hline
3.2.18 & assi commo dize el philosofo \textbf{ en el quato libro delas etihͣses por si mala cosa } e es de denostar ¶ & mendacium autem ut dicitur 4 Ethic’ \textbf{ per se est malum } et detestabile . \\\hline
3.2.18 & e de dar les bueons \textbf{ e uerdaderos conseios } por todo su poder . & ad quem loquuntur \textbf{ et cui consilium praebent : } quia amicorum est amicis vera et bona consulere . Tertio consiliarii debent esse sapientes , \\\hline
3.2.18 & Lo segundo amistança¶ Lo terçero sabiduria . \textbf{ ca para que bien e derecha miente conseien } assi commo paresçe & videlicet , bonitas , amicitia , et sapientia . Nam , \textbf{ ut recte consulant , } sicut patet ex dictis , \\\hline
3.2.19 & e ame la comunidat et ꝓmueua los q̃ son eñl su regno \textbf{ e los hõ rre . } la qual cosa non podria ser & probabatur enim supra , Regem debere esse talem , \textbf{ quod esset bonus virtuosus } et politiam diligeret , existentes in regno promoueret et honoraret : quod esse non posset , si bona eorum quae sunt in regno usurparet iniuste . \\\hline
3.2.19 & e a defendimiento del regno \textbf{ e a buen estado del Rey e del pueblo } Et pues que assi es conuiene & ut ad defensionem regni , \textbf{ et ad bonum statum eius : } decet ergo consiliarios scire introitus et prouentus regni , \\\hline
3.2.19 & quals deuen o quales se pueden fazer \textbf{ ca non es pequana cosa de auer consseio er la mantenençia } e enlas uiandas & et debitae ordinationes fieri possint : \textbf{ non enim modicum consiliandum est circa alimentum , } ut quaelibet ciuitas habeat sufficientia ad vitam , \\\hline
3.2.19 & por que se non le una ten discordias nin se fagan malefiçios entre los çibdadanos . \textbf{ por ende deuemos cuydar con grant acuçia } quales delos çibdadanos son tenidos por bueons & ne insurgant seditiones et malitia \textbf{ inter ciues . Ideo attendendum est diligenter } qui ciuium reputantur boni , \\\hline
3.2.19 & e quales difamados \textbf{ por que çerca los difamados sea puesta mayor guarda . } Et si por auentraa fueren conphendidos & qui ciuium reputantur boni , \textbf{ et qui infames , et circa infames maior custodia adhibeatur : } et si deprehendantur male egisse , \\\hline
3.2.19 & Avn son de penssar los logares \textbf{ en los quales se suelen fazer malos malefiçios } ca assi commo alguons delos omes fazen mayores tuertos & quia Reges et Principes non debent pati maleficos viuere . Sunt \textbf{ etiam consideranda loca in quibus consueuerunt magis maleficia perpetrari : } nam sicut quidam hominum magis iniustificant \\\hline
3.2.19 & en los quales se suelen fazer malos malefiçios \textbf{ ca assi commo alguons delos omes fazen mayores tuertos } que los otros & etiam consideranda loca in quibus consueuerunt magis maleficia perpetrari : \textbf{ nam sicut quidam hominum magis iniustificant } quam alii sic sunt quaedam loca magis apta ad iniustificandum quam alia : \\\hline
3.2.19 & mas sospethosos que los otros . \textbf{ por que los mal fechores se acostunbraron } de esconder se y mas que en los otros . & ut in ciuitate contingit esse vicos aliquos magis esse suspectos quam alios : \textbf{ quia iniustificantes ibidem possunt magis latere , } et effugere punientes : \\\hline
3.2.19 & Et alli fuyen de la iustiçia \textbf{ avn en essa misma manera fuera dela çibdat } son alguons logares montanneses & et effugere punientes : \textbf{ sic } etiam extra ciuitatem loca aliqua sunt nemorosa \\\hline
3.2.19 & avn en essa misma manera fuera dela çibdat \textbf{ son alguons logares montanneses } e son brios mas apareiados & sic \textbf{ etiam extra ciuitatem loca aliqua sunt nemorosa } et umbrosa magis apta ad iniustificandum , \\\hline
3.2.19 & e por ende deue ser tomado consseio \textbf{ por que en tales sea puesta mayor guarda . } Otrossi çerca la guarda dela çibdat & debet ergo adhiberi consilium , \textbf{ ut circa talia maior custodia praebeatur . Rursus circa custodiam ciuitatis } et regni non solum sunt adhibenda consilia propter ipsos ciues vel propter eos \\\hline
3.2.19 & e del regno non solamente son de tomar consseios \textbf{ por essos mismos çibdadanos } o por aquellos & ut circa talia maior custodia praebeatur . Rursus circa custodiam ciuitatis \textbf{ et regni non solum sunt adhibenda consilia propter ipsos ciues vel propter eos } qui sunt in regno , \\\hline
3.2.19 & nin de dar alos estrannos \textbf{ mas son con grant acuçia de guardat e de guamesçer . } Lo quarto deue ser tomadon consseio dela paz e dela guerra & non sunt extraneis committenda vel tribuenda , \textbf{ sed sunt diligenter custodienda } et munienda . Quarto habet esse consilium circa pacem et bellum , \\\hline
3.2.19 & mas dela fin \textbf{ e del su contrario ninguno de sano entendimiento non deue tomar consseio . . } ca el consseio non es de tomar & non pax ciuium est aliquid finaliter intentum , bellum autem est eius oppositum : \textbf{ de fine autem et de eius opposito nullus sanae mentis consiliatur : } nam non est consilium \\\hline
3.2.19 & por que fazer tuerto alos otros \textbf{ e apremiar sos sin derecho es mala cosa por si } e es muchͣ de escusar & quia iniustificari in alios , \textbf{ et eos indebite opprimere , | per se est malum , } et fugiendum . Deinde , \\\hline
3.2.19 & alos quales non podemos contradezer \textbf{ nos fagan alguna fuerca o algun tuerto grant sabiduria } es non nos leunatar contra ellos & ad quos resistere non valemus , \textbf{ in nos forefaciant , | prudentiae est , } non insurgere in ipsos , \\\hline
3.2.19 & segunt el qual alguno enssennorea se ha de saluar o corronper . \textbf{ por que escogiendo la meior manera de prinçipar o de enssennorear ponga leyes muy derechos . } segunt las quales aquel prinçipado se ha de saluar . & secundum quem dominatur habet saluari et corrumpi : \textbf{ ut eligens optimum modum principandi , | ferat leges iustissimas , } secundum quas saluari habet principatus ille . \\\hline
3.2.19 & Enpero entendemos en los capitulos \textbf{ que se siguen requarir muchͣs cosas delas leyes . } por que en vno de muchͣs cosas & intendimus in capitulis sequentibus \textbf{ de legibus multa inquirere , } ut simul ex dictis et dicendis melius veritas patere possit . \\\hline
3.2.19 & que se siguen requarir muchͣs cosas delas leyes . \textbf{ por que en vno de muchͣs cosas } que son ya dichas & intendimus in capitulis sequentibus \textbf{ de legibus multa inquirere , } ut simul ex dictis et dicendis melius veritas patere possit . \\\hline
3.2.20 & por las leyes \textbf{ e las menores cosas } que pueden ser deuen ser puestas en aluedrio de los uiezes . & ø \\\hline
3.2.20 & Et esto podemos demostrar \textbf{ por quatro razones delas quales las tres tanne el philosofo en el primero libro de la rectorica } e la quat catanne en el sexto libro delas politicas & Quod quadruplici via inuestigare possumus , \textbf{ quarum tres tanguntur 1 Rhet’ quarta vero tangitur 1 Polit’ . } Prima via sic patet . \\\hline
3.2.20 & que son puestas en vna çibdat se pueden reglar muchas çibdadeᷤ \textbf{ Et en aquella misma çibdat } en que se fazen las leyes pueden ser mas uiezes & per leges tamen conditas in una ciuitate regularis possunt ciuitates multae . Immo illa eadem ciuitate in \textbf{ qua leges conduntur contingit plures esse iudices , } quam legum conditores . \\\hline
3.2.20 & Enpero los iuezes que iudgan seg̃t las leyes mueren se \textbf{ e corronpensse . Et algs uezes contesçe } que sin cornupçion e sin muerte los iuezes son tirados de sus oficlvii i̊ çios & secundum illas moriuntur et corrumpuntur . \textbf{ Contingit } etiam absque corruptione \\\hline
3.2.20 & e corronpensse . Et algs uezes contesçe \textbf{ que sin cornupçion e sin muerte los iuezes son tirados de sus oficlvii i̊ çios } e son prouestos otros en sir logar & Contingit \textbf{ etiam absque corruptione | et morte iudices a suo officio remoueri , } et alio in suum locum succedere . Igitur saltem per successionem ipsorum oportet in eadem ciuitate multos esse iudices : \\\hline
3.2.20 & por la qual cosa si los fazedores son pocos en conparaçion de los iuezes \textbf{ porque mas ligera cosa es de fallar pocos sabios que muchs . } por que todas las cosas sean ordenadas sabiamente & Quare si legum conditores respectu iudicum sunt pauci , \textbf{ quia facilius est inuenire paucos sapientes , } quam multos , ut omnia sapienter disponantur , \\\hline
3.2.20 & por mucho tienpo \textbf{ e con grant diligençia fueren escodrinnados } que si luego man ama no fuesse dadas m̃a difinitiua . & quam si oporteat \textbf{ statim iudicatiuam sententiam proferre . Itaque cum legum conditores } multo tempore et magno consilio deliberare possint quales debeant leges fieri : \\\hline
3.2.20 & por ende commo los fazedores delas leyes en muchtp̃o \textbf{ e con grant conseio puedan et de una determinar quales leyes de una poner . } Mas los iuezes & statim iudicatiuam sententiam proferre . Itaque cum legum conditores \textbf{ multo tempore et magno consilio deliberare possint quales debeant leges fieri : } iudices vero propter instantiam partium , \\\hline
3.2.20 & se \textbf{ por amor o por mal querençia . } Mas los iuezes non fazen & non peruertuntur in iudicando amore , \textbf{ vel odio inclinati . } Iudices autem non sic . \\\hline
3.2.20 & por la qual cosa \textbf{ por que el iuyzio de los iuezes non sea torçido buena cosa es } quanto puede ser & non aequaliter iudicant : \textbf{ quare ne iudicium iudicum obliquatur , bonum est } ( quantum possibile est ) omnia lege determinare , \\\hline
3.2.20 & por penssamiento prolongado \textbf{ e por luengo tp̃o . } Mas los iuysios se dan adesora en el tp̃o & Deinde quia legislatores fiunt ex consideratis ex multo tempore , \textbf{ iudicia autem ex suborto et subito . Tertium vero , } quod est maximum omnium : \\\hline
3.2.21 & ca muchos de los que contienden en iuyzio \textbf{ sabiendo que tienen mal pleito non cuentan } lo que es fecho & in iudicio prohibeantur : \textbf{ multi enim litigantium cognoscentes se habere malam causam , } non narrant \\\hline
3.2.21 & en las cosas que sienten egualmente \textbf{ por que el seso es medianera ygualdat entre las cosas que sienten . } Et mientra que se non corronpen & vel inter propria sensibilia . \textbf{ Est enim sensus media proportio sensibilium , } et quamdiu non inficitur \\\hline
3.2.21 & assi commo regla tuerta \textbf{ e iudgar mal e desigual mente . } Et por que esto fazen las palabras desiguales & et amicitiam , ab altera vero recedat per iram \textbf{ et odium , quasi regula tortuosa peruerse iudicabit , } et quia hoc faciunt sermones passionales , \\\hline
3.2.21 & e entre los contendedores en el pleito . \textbf{ Et porque la cosa medianera deue tomar alguͣ cosa de cada vno de los estremos . } Et el iuez & inter legislatorem et litigantes , \textbf{ et quia medium aliquid | debet accipere ab utroque extremorum , } iudex tanquam medius aliquid accepit ab utrisque . \\\hline
3.2.21 & Et el iuez \textbf{ assi commo medianero toma algunan cosa de cada vno destos . } ca del ponedor dela ley toma & debet accipere ab utroque extremorum , \textbf{ iudex tanquam medius aliquid accepit ab utrisque . } Nam a legislatore accipit \\\hline
3.2.21 & por las palabras . \textbf{ Et por ende iudgan ygual mente . } Et pues que assi es conssentir tales palabras & eo quod passionati , \textbf{ ut amantes , et odientes , et gaudentes , et tristantes non pariter iudicamus , permittere passionales sermones in iudicio , } est peruertere ordinem iudicandi : \\\hline
3.2.21 & Ca las partes mouiendo el iuez \textbf{ assi fazen paresçer alguͣ cosa derechͣo non de rethica . } la qual cosa es ofiçio del ponedor dela ley . & quia partes passionando iudicem , \textbf{ ei faciunt apparere aliquid iustum vel iniustum , } quod non est officium partium , \\\hline
3.2.21 & Et en esta manera inclinar al iuez a malenconia \textbf{ e a mal querençia dela parte contraria } e a bien querençia de ssi mismo . & vel narrare bona quae ipsi iudici contulerunt , \textbf{ et hoc modo prouocare iudicem ad maliuolentiam partis aduersae , } et ad beniuolentiam sui , \\\hline
3.2.22 & por la qual razon non son de consentir tales palabras en iuyzio . \textbf{ demos contar quatro cosas } que conuiene de auer alos iuezes & non sunt talia permittenda . \textbf{ Possumus autem quatuor enumerare , } quae oportet habere iudices , \\\hline
3.2.22 & que conuiene de auer alos iuezes \textbf{ para que den uerdaderos iuisios } e para que iudguen derechamente & quae oportet habere iudices , \textbf{ ut vera iudicia proferant , } et ut recte iudicent . Primum , \\\hline
3.2.22 & Ca en todo pleito \textbf{ quanto parte nesçe alo presente quatro cosas podemos peussar . } Conuiene de saber las partes que contienden . & In omni enim litigio \textbf{ ( quantum ad praesens spectat ) quatuor est considerare , } videlicet partes litigantes , negocium \\\hline
3.2.22 & por lo que dich̉es el iues deue ser \textbf{ assi commo regla derecha medianera entre amas las partes } por la qual cosa & nam \textbf{ ut patet ex habitis iudex debet esse quasi regula recta media inter utrasque partes : } quare si huiusmodi regula a medio deuiat , \\\hline
3.2.22 & por amor \textbf{ et se auiedra dela otra por abortençia o por mal querençia . } conuiene que el uiez judgue mal e desigual mente . & et ad unam partem declinat per amorem , \textbf{ ab alia vero recedit per odium , } oportet ipsum iudicare inique : \\\hline
3.2.22 & et se auiedra dela otra por abortençia o por mal querençia . \textbf{ conuiene que el uiez judgue mal e desigual mente . } Ca entonçe el uuzio non salle de zelo de iustiçia & ab alia vero recedit per odium , \textbf{ oportet ipsum iudicare inique : } quia tunc iudicium non procedit ex zelo iustitiae , \\\hline
3.2.22 & Ca entonçe el uuzio non salle de zelo de iustiçia \textbf{ mas salle de amor o de mal querençia delas partes . } Lo quarto si los uiezes non se ouieren conueniblemente alos negoçios & quia tunc iudicium non procedit ex zelo iustitiae , \textbf{ sed ex amore vel odio partium . } Quarto \\\hline
3.2.22 & Et todas estas cosas sobredichͣs son menester \textbf{ para iudgar derechamente e conuenible mente . } Et de aqui paresçe quales iuezes e quales examinadores de los pleitos deue tomar el Rey . & itaque quales iudices \textbf{ et quales discussores causarum quaerere deceat regiam maiestatem : } nam decet eos tales quaerere \\\hline
3.2.22 & assi que non se inclinen a ninguna de las partes \textbf{ por amor o por mal querençia delas partes . } mas que por amor de iustiçia dens m̃as uerdaderas . & vel \textbf{ ut non ex amore vel odio partium , } sed ex dilectione iustitiae sententias proferant : \\\hline
3.2.22 & por amor o por mal querençia delas partes . \textbf{ mas que por amor de iustiçia dens m̃as uerdaderas . } Otrossi que ayan prueua de los fechs & ut non ex amore vel odio partium , \textbf{ sed ex dilectione iustitiae sententias proferant : } habeant experientiam agibilium , \\\hline
3.2.23 & uanto pertenesçe alo presente podemos contar diez cosas de la rectorica . \textbf{ Las quales dies cosas } con que tanne el philosofo en el primero libro uiene que tenga el iuez sienpre mientes & ut cognoscentes particularia acta melius discutiant causas . \textbf{ Quantum ad praesens spectat decem numerare possumus , } quae videtur tangere Philos’ 1 Rhet’ \\\hline
3.2.23 & Lo segundo deue tener mientes al ponedor dela ley ¶ \textbf{ Lo terçero al piadoso entendimiento delas leyes ¶ } Lo quarto ala entençion del que obra . & et ut sit clemens potius quam seuerus . Primum est ipsa natura humana , \textbf{ secundum legislator , | tertium pius intellectus legum , } quartum operantis intentio , \\\hline
3.2.23 & si se puede castigar el que peça¶ Lo x̊ . \textbf{ deue parar mientes el uiezala humildat } e ala subiectiuo del que peca . & octauum patientia accusati , \textbf{ nonum corrigibilitas peccantis , } decimum subiectio delinquentis . Primo enim ipsa natura humana clamat pro clementia delinquentis . \\\hline
3.2.23 & e ala subiectiuo del que peca . \textbf{ lo primero la naturͣa humanal llama } e ruega & nonum corrigibilitas peccantis , \textbf{ decimum subiectio delinquentis . Primo enim ipsa natura humana clamat pro clementia delinquentis . } Nam cum natura humana de se sit debilis , \\\hline
3.2.23 & Lo terçero que inclina al iuez a piedat \textbf{ e es piadoso entendimiento delas leyes . } por que las leyes & quam ad leges . \textbf{ Tertium inclinans ad pietatem est pius intellectus legum . } Leges enim ad terrendum delinquentes \\\hline
3.2.23 & por que las leyes \textbf{ para espantar los que yerran contienen en ssi alguna grant asꝑeza . } por la qual cosa si las palabras delas leyes & Tertium inclinans ad pietatem est pius intellectus legum . \textbf{ Leges enim ad terrendum delinquentes } quandam ampliorem seueritatem continent : \\\hline
3.2.23 & por que espant en los pecadores contienen \textbf{ en ssi mayoraspeza e mayor dureza de quanta deue . } Conuiene que por el entendimiento piadoso sea atenprada la guaueza dela pena & quare si legum verba \textbf{ ut terreant peccantes sunt amplioris seueritatis contentiua , decet } ut per pium intellectum moderetur supplicii magnitudo , \\\hline
3.2.23 & ca commo quier que la obra \textbf{ de que alguno es acusado sea de mala manera . } enpero por auentura pudo ser & Nam licet \textbf{ de quo incusatur aliquis sit de genere malorum , } ipse forte non habuit prauam intentionem , \\\hline
3.2.23 & enpero por auentura pudo ser \textbf{ que el non ouo mala entençion } o si la ouo mala non la ouo & de quo incusatur aliquis sit de genere malorum , \textbf{ ipse forte non habuit prauam intentionem , } vel si habuit prauam , \\\hline
3.2.23 & commo muestre la obra \textbf{ e por que las cosas dubdosas son de iudgar ala meior parte } por ende si el iues en alguna manera puede entender & ut opus ostendit : \textbf{ et quia dubia iudicanda sunt in meliorem partem , } si aliquo modo potest percipere iudex peccantem non peccasse ex electione , \\\hline
3.2.23 & ca por auentra a aquel \textbf{ que agora peca fizo ante muchas bueans obras } Et por ende eliez non deue & sed ad electionem . Quintum inducens ad misericordiam , est multitudo bonorum operum . \textbf{ Nam sorte ille qui nunc deliquit multa bona opera prius fecit : } debet ergo iudex non ita respicere ad partem ut ad hoc particulare negocium in quo delinquunt , \\\hline
3.2.23 & que fizo \textbf{ mas a todas las buenas obras } que auia fecho ¶ Lo sexto que inclina ali es a piedat es alongamiento detpo passado & sed ad totum . \textbf{ Sextum est diuturnitas temporis retroacti . } Nam contingit \\\hline
3.2.23 & por que contesçe \textbf{ que alas uezes alguno en poco tp̃o faze muchas buenas obras . } Et por ende dos cosas deuen endozir al Rey o al prinçipe & Nam contingit \textbf{ etiam in pauco tempore facere multa bona opera : } duo ergo debent inducere Regem aut quemcunque alium dominum \\\hline
3.2.23 & por que much tp̃olo siruio . \textbf{ e estas dos cosas maguer que por la mayor parte se aconpanen en vno } ca el que mucho t pon sirmo & vel quia multo tempore seruiuit sibi . \textbf{ Haec enim duo licet } ut plurimum se committentur , quia qui multo tempore seruiuit \\\hline
3.2.23 & ca el que mucho t pon sirmo \textbf{ por la mayor parte much sseruiçios fizo . } Et el que muchs seruiçios fizo por la mayor parte mucho t p̃o siruo & Haec enim duo licet \textbf{ ut plurimum se committentur , quia qui multo tempore seruiuit } ut plurimum multa seruitia fecit , \\\hline
3.2.23 & por la mayor parte much sseruiçios fizo . \textbf{ Et el que muchs seruiçios fizo por la mayor parte mucho t p̃o siruo } Et enpero contesçe & ut plurimum se committentur , quia qui multo tempore seruiuit \textbf{ ut plurimum multa seruitia fecit , } et econuerso ; \\\hline
3.2.23 & Por la qual cosa si contezca \textbf{ que algun subdito peca agora en algua parte detp̃o } el que fizo bien en todo elt & quod respicit multitudinem operum . \textbf{ Quare si contingat aliquem subditorum nunc in aliqua parte temporis delinquere : } qui toto tempore se bene habuit praecedenti , \\\hline
3.2.23 & Et por ende dado que alguno errasse \textbf{ o pecasse contra nos del qual resçibiemos ya en lost pons passados } muchs bienes deuemos nos mouer contra el & sed obliuiscebatur illas . \textbf{ Dato ergo aliquem in nos delinquere , } a quo temporibus retroactis multa bona suscepimus , \\\hline
3.2.23 & e atalon deuemos mucho perdonar . \textbf{ e tales son detractar muy benigna mente . } Et por ende dize el philosofo en el primero libro dela rectorica & talibus ergo est valde indulgendum , \textbf{ et tales sunt valde benigne tractandi . Ideo dicitur 1 Rhet’ } quod iudex \\\hline
3.2.23 & que contra los que son homillosos \textbf{ deue quedar la sanna esto se muestra avn por las bestias crueles obravis } que non muerden a aquellos que yazen homillosamente ante ellos & quod autem ad humiliantes cesset ira \textbf{ etiam canes manifestant non mordentes eos } qui resident . \\\hline
3.2.23 & alos quales conuiene de resplandesçer \textbf{ por mayor bondat . } Et pues que assi es conuiene a ellos & multo magis decet Reges et Principes , \textbf{ quibus congruit ampliori bonitate pollere . } Decet \\\hline
3.2.23 & sin ella la paz del regno \textbf{ e el buen estado de los çibdadanos non podria estar } nin ser guardado . & quia sine ea pax regni \textbf{ et bonus status ciuium } non potest consistere ; \\\hline
3.2.24 & e el quinto departimiento del derech̃ . \textbf{ diziendo que en quatro maneras se departe el derech . } Conuiene a saber ende recħ natural & et dare quintam distinctionem iuris , \textbf{ dicendo quod quadruplex est ius , } videlicet naturale , animalium , gentium , \\\hline
3.2.24 & de ser tales \textbf{ o por que auemos natural apetito o natural inclinaçion } a ellos aas los derechos positiuos & quae sunt adaequata et proportionata ex natura sua , vel dicuntur iusta naturaliter quae dictat esse talia ratio naturalis , \textbf{ vel ad quae habemus naturalem impetum et inclinationem . } Iusta vero positiua dicuntur , \\\hline
3.2.24 & Mas la razon \textbf{ por que al derech natural conuinio anneder derecho positiuo es esta } por que muchas cosas son derechas naturalmente & postquam autem est editum incipit habere ligandi efficaciam . Ratio autem , \textbf{ quare iuri naturali oportuit superaddere positiuum , est : quia multa sunt sic iusta naturaliter , } sicut est naturale homini loqui : \\\hline
3.2.24 & por que muchas cosas son derechas naturalmente \textbf{ assi commo natural cosa es al ome de fablar } ca auemos natural apetito e natural inclinacion para fablar & quare iuri naturali oportuit superaddere positiuum , est : quia multa sunt sic iusta naturaliter , \textbf{ sicut est naturale homini loqui : } habemus enim naturalem impetum \\\hline
3.2.24 & assi commo natural cosa es al ome de fablar \textbf{ ca auemos natural apetito e natural inclinacion para fablar } e para manifestar a otri & sicut est naturale homini loqui : \textbf{ habemus enim naturalem impetum | et naturalem inclinationem ut loquamur , } et ut per sermonem manifestemus \\\hline
3.2.24 & que las palabras e las smones son a uoluntad . \textbf{ Et esse mismo pho dize en el primero delas politicas } que la palabra non es dada por natura . & voces et sermones dicit esse ad placitum , \textbf{ qui primo Politicorum ait , sermonem nobis esse datum a natura . Sicut ergo loqui est naturale , sic autem loqui vel sic , est positiuum et ad placitum . Sic , } fures punire , \\\hline
3.2.24 & e o tristales cosas son de derechnatural \textbf{ por que la razon natural muestra que se deuen fazer . } Et auemos natural inclinaçion & de iure naturali , \textbf{ quia haec esse fienda dictat ratio naturalis , } et habemus naturalem impetum \\\hline
3.2.24 & por que la razon natural muestra que se deuen fazer . \textbf{ Et auemos natural inclinaçion } que estas cosas se fagan . & quia haec esse fienda dictat ratio naturalis , \textbf{ et habemus naturalem impetum } ut haec fiant . Surgunt enim ista ex ipsa natura rei , \\\hline
3.2.24 & ca estas cosas sele una tan dela natura dela cosa . \textbf{ Ca natural cosa es } que espongamos a periglo el bien de vna parte & ø \\\hline
3.2.24 & por que non pesçiesse todo el cuerpo . \textbf{ En essa misma manera el ladron e el malfechor } e qual si quier de los malos & ne pereat totum corpus : sic , \textbf{ quia fur et maleficus , } et quilibet malefactor turbat pacem ciuium insidiatur communi bono , \\\hline
3.2.24 & nin sea enbargado . \textbf{ Et pues que assi es natural cosa es } que tales cosas sean castigadas & et ne impediatur commune bonum : \textbf{ naturale est ergo talia punire . } Sed ea punire sic , \\\hline
3.2.24 & nin de vna manera \textbf{ nin por vna misma pe na . } Et pues que assi es en aquel logar ose termina el derecho natural & sed non apud omnes eadem maleficia corriguntur eisdem poenis . \textbf{ Ubi ergo terminatur ius naturale , } ibi incipit oriri ius positiuum : \\\hline
3.2.24 & e los mas fechores sean castigados \textbf{ e ayan pena el derech positiuo prisu pone esto va adelante } determinando & et maleficos esse puniendos , \textbf{ hoc praesupponens | ius positiuum procedit ulterius , } determinans qua poena sint talia punienda . Hoc viso \\\hline
3.2.24 & e aquello que es del derecho natural \textbf{ sinplemente es dicho seer escpto en los nros coraçones } Ca las gentes & quia sic se offerunt intellectui nostro naturales leges , \textbf{ quod est de iure naturali simpliciter dicitur esse scriptum in cordibus nostris . } Nam gentes quae legem non habent , \\\hline
3.2.25 & si aquellas cosas son del derech natural . \textbf{ a que auemos natal inclinaçion esta inclinaçion natural } del apeti too ligue lanr̃a natura & secundum rationem communem acceptus conuenit cum illis . \textbf{ Si igitur ea sunt de iure naturali , ad quae habemus naturalem impetum et inclinationem : } huiusmodi naturalis impetus \\\hline
3.2.25 & a que auemos natal inclinaçion esta inclinaçion natural \textbf{ del apeti too ligue lanr̃a natura } en quanto somos omes & Si igitur ea sunt de iure naturali , ad quae habemus naturalem impetum et inclinationem : \textbf{ huiusmodi naturalis impetus | vel sequitur naturam nostram , } ut sumus homines , \\\hline
3.2.25 & do estas cosas son puestas \textbf{ dize el derecho natural enssenna a todas las ainalias . } Ca este derechotal & ubi haec sunt tradita , dicitur , \textbf{ quod ius naturale , | est quod natura omnia animalia docuit . } Huiusmodi autem ius \\\hline
3.2.25 & Mas el derecho delas gentes es dich̃aquel \textbf{ que non es comun alas o trisaian las } mas es comuna todo el humanal linage Et pues que assi es deste derecho ssallen & et ut ipsos nutriant , et foueant . Ius vero gentium dicitur , \textbf{ quod non est commune animalibus aliis : sed commune est omni humano generi . } Ex hoc ergo iure pene omnes contractus sunt introducti , \\\hline
3.2.25 & que non es comun alas o trisaian las \textbf{ mas es comuna todo el humanal linage Et pues que assi es deste derecho ssallen } por la mayor parte todos los contractos & et ut ipsos nutriant , et foueant . Ius vero gentium dicitur , \textbf{ quod non est commune animalibus aliis : sed commune est omni humano generi . } Ex hoc ergo iure pene omnes contractus sunt introducti , \\\hline
3.2.25 & mas es comuna todo el humanal linage Et pues que assi es deste derecho ssallen \textbf{ por la mayor parte todos los contractos } que son entre los omes & quod non est commune animalibus aliis : sed commune est omni humano generi . \textbf{ Ex hoc ergo iure pene omnes contractus sunt introducti , } ut emptio , \\\hline
3.2.25 & aquello que ha mester para la uida . \textbf{ Et por ende a este mismo derecho parte nesçe el enprestar } e el guardar & sine quibus societas humana non bene sufficit sibi ad vitam . \textbf{ Inde est ergo quod mutuum } et depositio , \\\hline
3.2.25 & por que las reglas del derecho \textbf{ quanto mas se allegan a algua materia espeçial tanto mas trahe consigo alguons desfallesçimientos } e en muchos casos non son de guardar & tanto plures defectus contrahunt , \textbf{ et in pluribus casibus } non sunt obseruandae , \\\hline
3.2.25 & e en muchos casos non son de guardar \textbf{ e resçiben mayor mudamiento . } Et pues que & non sunt obseruandae , \textbf{ et maiorem mutationem suscipiunt : } merito igitur huiusmodi ius , \\\hline
3.2.25 & Et pues que \textbf{ assi es con grant razon el derecho delas aianlias } es dicho ser derecho natural & et maiorem mutationem suscipiunt : \textbf{ merito igitur huiusmodi ius , } naturale dicitur , \\\hline
3.2.25 & assi natural commo es aquel derecho \textbf{ que sigue la inclinaçion del anr̃a natura } en quanto non solamente auemos conueniençia con las otras aian las . & non est ita naturale , \textbf{ sicut ius illud quod sequitur inclinationem naturae nostrae : } prout non solum communicamus cum animalibus aliis , \\\hline
3.2.25 & assi es esta sera la orden entre estos de ti xu rechos \textbf{ que el derecho que ligue lanr̃a natura } en quanto desseamos ser & Erit igitur hic ordo , \textbf{ quod ius consequens naturam nostram } prout appetimus esse et bonum , est naturale respectu iuris animalium , \\\hline
3.2.25 & lo que ayamos a ello inclinaçion natural . \textbf{ Mas segunt que la intlinaçion sigue lanr̃a natura } en quanto es humanal & secundum quod inclinatio sequitur naturam nostram : \textbf{ Nam si inclinatio illa sequitur naturam nostram } ut humana est , \\\hline
3.2.26 & non es ley \textbf{ mas es coronpimientode ley . } Ca ninguna cosa non es establesçida derechͣmente del omne . & non est lex , \textbf{ sed corruptio legis ; } nihil enim ab homine statuitur iuste , \\\hline
3.2.26 & segunt el departimiento delas comunidades . \textbf{ Et pues que assi es el que quisiere poner leyes con grand acuçia } deue tener mientes & secundum diuersitatem politiarum . \textbf{ Volens ergo leges ferre , | diligenter debet attendere , } qualis sit populus , \\\hline
3.2.26 & assi commo dize al philosofo \textbf{ en el x̊ libro delas ethicas } han uirtud & ø \\\hline
3.2.27 & e de enderesçar todos los otros al bien comun . \textbf{ Et pues que assi es cada vna perssena singular } que es parte de alguna muchedunbre deue ser guardadora delas leyes . & habet dirigere et ordinare alios in commune bonum . \textbf{ Quaelibet ergo persona particularis , } quae est pars multitudinis alicuius , \\\hline
3.2.27 & Mas estas muniçonnes \textbf{ e estas condiçiones tales non son dich̃ͣs leyes } por que non han ningun poderio para costrennir . & sed huiusmodi monitiones \textbf{ et persuasiones non dicuntur leges , quia nihil habent coactiuum . } Extendendo autem nomen legis , \\\hline
3.2.27 & e alas otras perssonas \textbf{ que son en su casa pueden ser dichͣs leyes . } mas esto non es & et aliis existentibus in domo , \textbf{ leges nominari possent : } sed hoc non est \\\hline
3.2.28 & Ca las leyes \textbf{ assi commo dicho es de suso son reglas delas nr̃as obras . } Ca assi commo la fisica quiere reglar & permittere , prohibere , praemiare , et punire . Sunt enim leges \textbf{ ( ut supra dicebatur ) quaedam regulae actionum nostrarum : } sicut enim medicina per dietam , \\\hline
3.2.28 & que le contienen en aquella sçiençia \textbf{ por que los çibdadanos biuna derechamente } e saayan conmose deuen auer . & et per alia quae ibi docentur vult aequare \textbf{ et regulare actiones humanas , | ut ciues iuste viuant , } et debite se habeant . \\\hline
3.2.28 & que nin es buena nin mala . \textbf{ Enpero si alguns le leunatare con mala entençion } para poner la en el oio a su conpannon & de se est opus indifferens : \textbf{ si quis tamen mala intentione eleuaret illam , } ut quia vellet ponere in oculum socii , \\\hline
3.2.28 & para poner la en el oio a su conpannon \textbf{ es mala obra } e de deno star . & ut quia vellet ponere in oculum socii , \textbf{ esset opus prauum et vituperabile : } si vero eleuando eam vellet purgare domum vel facere aliquod aliud opus pium , \\\hline
3.2.28 & o para fazer alguna otra obra buean \textbf{ por la buena entençion de aquel } que la faze esta tal obra & si vero eleuando eam vellet purgare domum vel facere aliquod aliud opus pium , \textbf{ propter bonam intentionem operantis , } quod de se est \\\hline
3.2.28 & alas leyes conuiene de saber . \textbf{ Mandar quanto alas buenas obras . } vedar quanto alas malas & videlicet praecipere , \textbf{ quantum ad opera bona : } prohibere quantum ad mala : permittere quantum ad indifferentia . Aduertendum \\\hline
3.2.28 & Mas avn aquellas cosas \textbf{ que non contienen en ssi grant culpa } e han poca maliçia . & aut nunquam posset \textbf{ aliquem populum regere . Ideo non solum permittenda sunt indifferentia , } sed etiam quae modicam malitiam habent annexam permitti possint a legislatore . Viso quae attribuenda sunt legibus respectu operum fiendorum : \\\hline
3.2.28 & e esto es conssentir \textbf{ las que non es grant fuerça en dexar las passar . } Et por ende estas cosas & vel quasi indifferentium , \textbf{ ut permittere . } His itaque sic pertractatis , \\\hline
3.2.28 & e delas çibdadeᷤ \textbf{ assi que con grant cuydado e con grant estudio deuen trabaiar quales leyes } e quales establesçimientos pongan a sus çibdadanos . & circa regimen regni , \textbf{ et ciuitatis cura peruigili insudare quas leges , } et quae instituta imponant ciuibus , \\\hline
3.2.28 & e quales establesçimientos pongan a sus çibdadanos . \textbf{ asi que con grant acuçia por si } e por sus consseieros examun en quales bueans obras son demandar & et quae instituta imponant ciuibus , \textbf{ et diligenter per se et suos consiliarios discutiant quae bona sunt praecipienda } et praemianda , \\\hline
3.2.29 & o quales son de conssentir \textbf{ L philosofo en el terçero delas politicas demanda } si es meior de ser gouernado el Regno o la çibdat & et quae dissimulanda , et permittenda . \textbf{ Philosophus 3 Politicorum inquirit , } utrum regnum aut ciuitas sit melius Regi optimo Rege , \\\hline
3.2.29 & si es meior de ser gouernado el Regno o la çibdat \textbf{ por muy buen Rey o por muy buena ley . } Mas para esto prouar aduze dos razones & utrum regnum aut ciuitas sit melius Regi optimo Rege , \textbf{ aut optima lege . } Adducit autem rationes duas , \\\hline
3.2.29 & que meior es de ser gouernado el regno \textbf{ por muy buena ley } que por muy buen Rey ¶ & Adducit autem rationes duas , \textbf{ quod melius sit politiam regni Regi optima lege , } quam optimo Rege . \\\hline
3.2.29 & por muy buena ley \textbf{ que por muy buen Rey ¶ } La primera razon se toma & quod melius sit politiam regni Regi optima lege , \textbf{ quam optimo Rege . } Prima sumitur \\\hline
3.2.29 & La segunda se toma de aquello \textbf{ que mas ligera cosa es es } de se corronper el rey & esse quasi organum et instrumentum legis . \textbf{ Secunda ex eo quod facilius est corrumpi Regem quam legem . } Prima via sic patet . \\\hline
3.2.29 & en el quinto libro delas ethicas \textbf{ el prinçipe deue ser guardador de derecho o de derecha ley . } Et pues que assi es el prin çipe li en ssennoreare conueinblemente es & Nam ut dicitur 5 Ethicorum Princeps debet esse custos iusti , \textbf{ idest iustae legis : } est ergo Princeps , \\\hline
3.2.29 & Et pues que assi es el prin çipe li en ssennoreare conueinblemente es \textbf{ assi commo vn organo e instrumento de derecha ley } assi que aquello quela ley manda & si debite principetur , \textbf{ quasi quoddam organum iustae legis , } ut quod iuste lex fieri praecipit , Rex per ciuilem potentiam obseruari facit . \\\hline
3.2.29 & que el instrumento dela ley mas es de escoger es que la çibdat e el regno sea gouernado \textbf{ por muy buena ley } que por muy buen Rey & Regi optima lege eligibilius est , \textbf{ quam Regi optimo Rege . } Hoc est ergo quod ait Philosophus 3 Politicorum , \\\hline
3.2.29 & por muy buena ley \textbf{ que por muy buen Rey } Et esto es lo que dize el philosofo en el tercero delas politicas & Regi optima lege eligibilius est , \textbf{ quam Regi optimo Rege . } Hoc est ergo quod ait Philosophus 3 Politicorum , \\\hline
3.2.29 & para mostrar esto mismo se toma daquello \textbf{ que mas ligera cosa es dese tris tornar } e de corconper el rey & Secunda via ad inuestigandum hoc idem , \textbf{ sumitur } ex eo quod facilius est peruerti Regem , \\\hline
3.2.29 & por ende dize elpho en el terçero libro delas ethicas \textbf{ que algunas uegadas los muy bueons omes la sanna } e la cobdiçia finalmente los mata e los tristorna . & Ideo dicitur 3 Ethic’ \textbf{ quod aliquando optimos viros furor } et concupiscentia tandem interimit \\\hline
3.2.29 & e la cobdiçia finalmente los mata e los tristorna . \textbf{ Ca el muy buen omne matase por cobdiçia } e si non se mata & et concupiscentia tandem interimit \textbf{ et peruertit . Interimitur optimus homo per concupiscentiam , } et si non quantum ad esse naturae , \\\hline
3.2.29 & e de cobdiçiar las cosas malas \textbf{ si se non mata quantoal ser sinple mente . } Enpero matasse & et concupiscere peruersa , \textbf{ et si non interimitur } quantum ad esse simpliciter , interimitur tamen quantum ad esse optimum , \\\hline
3.2.29 & que por Rey . \textbf{ Mas que esto non sea sinplemente de otorgar muestralo esse mismo pho en esse libro terçero } que assi commo el dize la ley dize generalmente & quam Rege . \textbf{ Sed quod hoc non sit simpliciter fatendum , } ostendit ibi Philosophus in eodem 3 . \\\hline
3.2.29 & que assi commo el dize la ley dize generalmente \textbf{ aquello que non es general mente } Por que conuiene que las leyes humanales & Sed quod hoc non sit simpliciter fatendum , \textbf{ ostendit ibi Philosophus in eodem 3 . } Nam ( ut ait ) lex uniuersaliter dicit quod non est uniuersaliter : oportet enim humanas leges quantumcunque sint exquisitae in aliquo casu deficere : melius est igitur regnum Regi Rege , \\\hline
3.2.29 & assi commo la razon derecha o el entendimiento manda . \textbf{ En essa misma manera la ley positiua nunca liga } nin obliga derechamente & ut recta ratio dictat : \textbf{ sic lex positiua nunquam recte ligat , } nisi innitatur auctoritati legis aut alterius principantis . \\\hline
3.2.29 & que el regno o la çibdat sea gouernado \textbf{ por muy buen Rey o por muy buena ley . } si fablaremos dela leyna turͣal paresçe & utrum melius sit regnum aut ciuitatem Regi optimo Rege , \textbf{ aut optima lege . } Si loquamur de lege naturali , \\\hline
3.2.29 & que el rey \textbf{ por que ninguon non es derecho rey } si non en quanto se esfuerça enla ley natural . & quam sit ipse Rex : \textbf{ eo quod nullus sit rectus Rex nisi in quantum innititur illi legi . } Propterea bene dictum est quod innuit Philosophus tertio Politicorum \\\hline
3.2.29 & en el tercero delas politicas \textbf{ que en el derecho gouernamiento non deue enssennorear la bestia } nin el ome bestia la mas dios e el entendimiento . & Propterea bene dictum est quod innuit Philosophus tertio Politicorum \textbf{ quod in recto regimine principari non debet bestia , } sed Deus et intellectus . \\\hline
3.2.29 & quando alguno en gouernando los otros non se parte \textbf{ nin se arriedra de derecha razon } nin dela ley natural & Tunc vero principatur Deus , \textbf{ quando quis in regendo alios non deuiat a ratione recta , } et a lege naturali , \\\hline
3.2.29 & es que el regno e la cobdiçia sea gouernada \textbf{ por muy buen Rey } e por muy buena ley & ø \\\hline
3.2.29 & por muy buen Rey \textbf{ e por muy buena ley } e mayormente en aquellos casos & ø \\\hline
3.2.29 & Et manda generalmente guardar aquello \textbf{ que non es de guardar general mente . } Et pues que assi es & et dicit uniuersaliter \textbf{ quod non est uniuersaliter obseruandum . } Secundum hoc ergo concludebat ratio in oppositum facta , \\\hline
3.2.29 & que meior es de ser gouernado el regno \textbf{ por buen Rey } que por buena ley & ø \\\hline
3.2.29 & por buen Rey \textbf{ que por buena ley } que por la ley non puede determinar todas los casos particulates . & Secundum hoc ergo concludebat ratio in oppositum facta , \textbf{ quod melius est Regi Rege , quam lege eo quod lex particularia determinare non potest . } Ideo expedit Regem aut alium principantem per rationem rectam , \\\hline
3.2.29 & Et de algunas de clemençia e de piedat . \textbf{ Ca mientra que la regla dela len finca derecha } e egual non se encorua ala vna parte los iuyzios me didos e mesurados & aliqua vero ex clementia . \textbf{ Quamdiu regula Regis manet recta } et aequalis , \\\hline
3.2.30 & Por ende conuiene \textbf{ que dessemeie alguons pecados } e que les de passada & quia communiter populus non potest attingere punctalem formam viuendi , \textbf{ ideo oportet aliqua peccata dissimulare } et non punire lege humana , \\\hline
3.2.30 & Et por ende conuiene que sin la ley humanal fuesse dada otra ley diuinal \textbf{ e e un agłica l . } por que ningun mal non fincasse sin pena & quae lege humana puniri non possunt . \textbf{ Oportuit igitur praeter legem humanam dari aliquam legem , } ut nullum malum remaneret impunitum , \\\hline
3.2.30 & quando uieñe del apetito \textbf{ et ple desseo del coraçon . } Mas si fuere penssa & si sint mali , \textbf{ cum procedant ex interiori appetitu . } Sed si consideretur \\\hline
3.2.30 & as avn non defiende todos los pecados de fuera del coraçon . \textbf{ Ca en las leyes humanales algunas uezes se consienten los menores males } por que se escusen los mayores . & sed etiam non prohibet omnia exteriora delicta . \textbf{ Nam legibus humanis aliquando dissimulantur minora mala } ut vitentur maiora : \\\hline
3.2.30 & en tanto son los iuyzios de los o omes departidos \textbf{ que de vnas mismas cosas entre departidas gentes son departidas leyes . } Ca segunt el iuyzio de algunos alguna cosa es derecha . & et de agibilibus humanis adeo sunt diuersa iudicia hominum , \textbf{ ut de eisdem } apud diuersas gentes diuersi sint leges , \\\hline
3.2.30 & la ley natural e la humanal \textbf{ que nos ayudan a alcançar este bien . el qual non podemos natural mente } alcançar non cunplen & lex naturalis \textbf{ et humana iuuantes nos ad consecutionem illius boni } quod possumus naturaliter adipisci , \\\hline
3.2.30 & alos quales parte nesçe ser \textbf{ assi commo medios dioses } e de auer entendimiento sin cobdiçia & Decet ergo reges et principes , \textbf{ quos competit esse quasi semideos , et esse intellectum sine concupiscentia , } et esse formam viuendi , et regulam agibilium , \\\hline
3.2.31 & alas çibdades de renouar las leyes dela tierra \textbf{ e de enduzir nueuas costunbres } por que ypodomio ordenara & utrum sit expediens ciuitatibus innouare patrias leges , \textbf{ et inducere nouas consuetudines . } Ordinauerat enim Hippodamus \\\hline
3.2.31 & que paresçiessen ser mas aprouechosas e meiores . \textbf{ Mas el philosofo enel quarto libro delas politicas pone quatro razones } por las quales para pesçe & etiam quod occurrant leges aliquae quae videantur esse magis proficuae et meliores . \textbf{ Adducit autem Phil’ in Polit’ quatuor vias , } per quas videtur ostendi , \\\hline
3.2.31 & que assi deuie ser en las leyes \textbf{ que si alguons fallassen meiores leyes } que las que fallaron los primeros padres & et innouata : \textbf{ quare sic erit in legibus quod si occurrant meliores leges } quam sint traditae a prioribus patribus , \\\hline
3.2.31 & e nesçia . \textbf{ Ca nesçia cosa era establesçer tales leyes } por las quales los çibdadanos pudiessen vender so mugers . & omnino enim erat barbaricum , \textbf{ statuere leges , } ut ciues possent uxores suas vendere . Sic etiam contingit leges aliquas esse stultas , \\\hline
3.2.31 & e algun pariente uiniesse acometiendo contra algun çibdadano . \textbf{ Et aquel presentes alguons çibdadanos fuyesse del } maguer el non fuesse el matador & et aliquis consanguineus mortui inuaderet aliquem ciuem , \textbf{ et ille praesentibus aliquibus fugeret ab eo reputabatur fugiens reus homicidii : } dicebat enim legislator quod non fugeret , \\\hline
3.2.31 & por que si quier alguno fuere culpado \textbf{ o si quier non si temiere de ser ferido natitral cosa es } que fuya de aquel que lo persigue . & quia siue aliquis sit culpabilis siue non , \textbf{ si timeat vulnerari , | naturale est } ut fugiat persequentem : \\\hline
3.2.31 & que son en las obras de todos los omes \textbf{ Et por ende si los postrimeros sabios } por la esperiençia delas obras particulares alguna cosa fallar en meior non es cosa sin razon de tirar las leyes dela tierra antiguas & potuerunt eos latere aliquae particulares circumstantiae circa agibilia hominum . \textbf{ Si igitur posterioribus propter experientiam agibilium particularium occurrit aliquid melius , } inconueniens est non remouere leges paternas \\\hline
3.2.31 & por la esperiençia delas obras particulares alguna cosa fallar en meior non es cosa sin razon de tirar las leyes dela tierra antiguas \textbf{ por las meiores leyes falladas nueuamente por ellos . } Et por ende paresçe & Si igitur posterioribus propter experientiam agibilium particularium occurrit aliquid melius , \textbf{ inconueniens est non remouere leges paternas } et antiquas propter meliores leges nouiter inuentas . Videntur \\\hline
3.2.31 & Et por ende paresçe \textbf{ que estas razones sobredichas prueuna que cada que acahesçiere algua cosa meior las leyes dela tierra son de mudar . } Mas afirmar esto sinplemente es muy perigloso ala çibdat e altegno . & et antiquas propter meliores leges nouiter inuentas . Videntur \textbf{ itaque hae rationes probare quod quotiescunque occurrit aliquid melius , | sunt leges paternae immutandae . } Sed hoc simpliciter afferre est valde periculosum ciuitati et regno . \\\hline
3.2.31 & Mas afirmar esto sinplemente es muy perigloso ala çibdat e altegno . \textbf{ Ca acostunbrar se los omes afaznueuas leyes } assi commo dize elpho en el segundo libro delas politicas & Sed hoc simpliciter afferre est valde periculosum ciuitati et regno . \textbf{ Nam assuescere inducere nouas leges } ( ut innuit Philosophus 2 Pol’ ) est assuescere non obedire legibus . \\\hline
3.2.31 & es acostunbrar sea non obedesçer alas leyes \textbf{ Ca las leyes grant fuerça toman dela costunbre . } Et esto por que con grant guauezafaze cada vno contra aquello que es guardado por luengos tienpos & ( ut innuit Philosophus 2 Pol’ ) est assuescere non obedire legibus . \textbf{ Nam leges magnam efficaciam habent ex consuetudine : } de difficili enim quis facit contra aliquid , \\\hline
3.2.31 & Ca las leyes grant fuerça toman dela costunbre . \textbf{ Et esto por que con grant guauezafaze cada vno contra aquello que es guardado por luengos tienpos } mas acostunbrar se los omes & Nam leges magnam efficaciam habent ex consuetudine : \textbf{ de difficili enim quis facit contra aliquid , } quod est per diuturna tempora obseruatum . Assuescere autem non obedire legibus , est assuescere non obedire Regibus et Principibus , \\\hline
3.2.31 & de non obedesçer alas leyes \textbf{ e alos reyes muestra lo el philosofo en el primero libro delaL rectorica } do dize & Quantum autem malum sequitur non obedire Regibus \textbf{ et legibus , | ostendit Philosophus 1 Rhetor’ } qui ait , \\\hline
3.2.31 & Ca los fisicos entienden enla sanidat del cuerpo . \textbf{ por que quieren adozir el cuerpo a sanidat . Mas los uerdaderos ordenadores delas leyes } e los uerdaderos Reyes sinplemente entienden en el bien del alma & Nam medici intendunt bonum corporis : \textbf{ volunt enim corpora inducere ad sanitatem . | Sed veri legislatores } et veri Reges principaliter intendunt bonum animae ; \\\hline
3.2.31 & por que quieren adozir el cuerpo a sanidat . Mas los uerdaderos ordenadores delas leyes \textbf{ e los uerdaderos Reyes sinplemente entienden en el bien del alma } por que entienden de adozir los çibdadanos a uirtud & Sed veri legislatores \textbf{ et veri Reges principaliter intendunt bonum animae ; } quia intendunt ciues inducere ad virtutem . \\\hline
3.2.31 & assi commo paresçe en la institutado dize \textbf{ que las leyes humanales contrarias son al derecho natraal . } Ca de comienço todos los omes nascian forros e libres . & Secundum hunc modum loquendi loquuntur Iuristae , ut patet ex Institutis de iure naturali , \textbf{ ubi dicitur quod leges humanae contrariae sunt iuri naturali ; } quia iure naturali ab initio homines liberi nascebantur . Seruitus ergo est contra naturam , \\\hline
3.2.31 & por alongamiento del tienpo \textbf{ por el qual la ley ha muy grant fuerça ¶ } Et pues que assi es non es lemei ante delas artes e delas leyes & tollendo consuetudinem et diuturnitatem temporis , \textbf{ per quam lex habet efficaciam magnam . Non est ergo simile de artibus et de legibus : } quia artes \\\hline
3.2.31 & Mas non es assi de las leyes \textbf{ ca las leyes han grant fuerça } por prolongamiento de tienpo et por vso e por costunbre . & et scientiae totam efficaciam habent ex ratione , \textbf{ sed leges non sic : Immo magnam efficaciam habent ex diuturnitate et assuefactione . } Decet ergo reges et principes obseruare bonas consuetudines principatus et regni , \\\hline
3.2.31 & Et por ende conuiene alos Reyes \textbf{ e alos prinçipes de guardar las bueans costunbres del prinçipado e del regno } e non renouar las leyes dela tierra & sed leges non sic : Immo magnam efficaciam habent ex diuturnitate et assuefactione . \textbf{ Decet ergo reges et principes obseruare bonas consuetudines principatus et regni , } et non innouare patrias leges , \\\hline
3.2.32 & assi commo dixiemos \textbf{ de suso quatro cosas eran de tranctar . } Conuiene saber & ( ut superius dicebatur ) \textbf{ erant quatuor pertractanda , } videlicet qualis debet esse Rex siue Princeps , \\\hline
3.2.32 & que commo quier que la çibdat en alguna manera sea cosa natural \textbf{ por que auemos natural inclinaçion } e desseo para establesçer & quod cum ciuitas sit aliquo modo \textbf{ quid naturale , eo quod naturalem habemus impetum ad ciuitatem constituendam : } non tamen efficitur , \\\hline
3.2.32 & Et cuenta el philosofo enel terçero libro delas politicas \textbf{ quariendo de el arar que cosa es la çibdat seys bienes } alos quales es ordenada la çibdat . & et animaduertendum est quod bonorum illorum sit potius . Narrat quidem Philosophus 3 Politic’ \textbf{ volens diffinire | quid sit ciuitas , } sex bona ad quae ciuitas ordinatur . \\\hline
3.2.32 & sin conpannia . \textbf{ Ca maguer que alguno ouiesse much oro e muchͣ plata } e abondasse muchen viandas & nam nullius sine socio iocunda est possessio . \textbf{ Si quis enim magna multitudine argenti et auri polleret , } et omnibus victualibus abundaret , \\\hline
3.2.32 & non solamente por beuir en alegera e en consolaçion \textbf{ mas avn por el beuir sola mente . } Ca los omes que estan en vna çibdat siruen & et delectabiliter conuersari , \textbf{ sed propter ipsum viuere . } Nam homines in eadem ciuitate existentes deseruiunt sibi ad vitam , \\\hline
3.2.32 & segunt el derech delas gentes . \textbf{ Et era propreo al humanal linage } por que ningun omne non ha conplidamente todas las cosas & secundum ius gentium , \textbf{ et erat proprium humano generi : } quia enim nullus homo habet omnia sufficientia ad vitam , \\\hline
3.2.32 & que por temor de pena muchs dexan de fazermal \textbf{ e acostunbran se a fazer buenas obras . } la qual cosa faziendo los omes ordenansse & quod timore poenae multi desinunt malefacere , \textbf{ et assuescunt ad operationes bonas : } quod faciendo , \\\hline
3.2.32 & Et pues que assi es commo las uirtudes \textbf{ e las obras uirtuosas sean muy grandes bienes } maguera que por todas las cosas sobredichas sea fecho la çibdat en alguna manera . & Quoniam igitur virtutes \textbf{ et opera virtuosa sunt maxima bonorum , } licet propter omnia praedicta bona sit aliquo modo ciuitas constituta , potissime \\\hline
3.2.32 & Enpero prinçipalmente fue ella establesçida \textbf{ por que biuiessen los omes escogidamente e uirtuosa mente . } Por la qual cosa si dela mayor fin & tamen constituta est propter viuere eligibiliter \textbf{ et victuose . } Quare si a maiori fine , \\\hline
3.2.32 & por que biuiessen los omes escogidamente e uirtuosa mente . \textbf{ Por la qual cosa si dela mayor fin } e del mayor bien & et victuose . \textbf{ Quare si a maiori fine , } et a maiori bono \\\hline
3.2.32 & Por la qual cosa si dela mayor fin \textbf{ e del mayor bien } que es entendido en cada cosa . & Quare si a maiori fine , \textbf{ et a maiori bono } quod intenditur in re accipienda est eius notitia , benedictum est \\\hline
3.2.32 & que cosa es el regno . \textbf{ Ca el regno eñade sobre la çibdat muchedunbre de nobles omes } e de alto linage . & quid est regnum . \textbf{ Nam regnum supra ciuitatem videtur addere multitudinem nobilium et ingenuorum . Est enim ciuitas pars regni ; } et in regno est maior multitudo , \\\hline
3.2.32 & Ca el regno eñade sobre la çibdat muchedunbre de nobles omes \textbf{ e de alto linage . } por que la çibdat es parte del regno . & quid est regnum . \textbf{ Nam regnum supra ciuitatem videtur addere multitudinem nobilium et ingenuorum . Est enim ciuitas pars regni ; } et in regno est maior multitudo , \\\hline
3.2.32 & por que la çibdat es parte del regno . \textbf{ e en el tegno es mayor muchedunbre } e son mas nobles e mas altos omes & Nam regnum supra ciuitatem videtur addere multitudinem nobilium et ingenuorum . Est enim ciuitas pars regni ; \textbf{ et in regno est maior multitudo , } et sunt plures nobiles et ingenui , \\\hline
3.2.32 & e en el tegno es mayor muchedunbre \textbf{ e son mas nobles e mas altos omes } que en vna çibdat . & et in regno est maior multitudo , \textbf{ et sunt plures nobiles et ingenui , } quam in ciuitate una . Potest ergo sic diffiniri regnum , quod est multitudo magna , \\\hline
3.2.32 & e demostrar diziendo \textbf{ que el regno es grand muchedunbre } en la qual son muchs nobles e de alto linaze & ø \\\hline
3.2.32 & que viuen segunt uirtud \textbf{ e son ordenados so vn muy buen uaron } assi commo so vn Rey . & secundum virtutem , \textbf{ ordinati sub uno viro optimo , } ut sub rege . Viuere enim secundum virtutem , \\\hline
3.2.32 & Ca deue el Rey \textbf{ si es uerdadero e derecho Rey querer esse mismo bien en vn çibdadano } e en toda la çibdat & Debet enim Rex , \textbf{ si sit verus et rectus idem intendere in uno ciue , } et in tota ciuitate , et in regno toto . \\\hline
3.2.32 & e por que toda la çibdat sea uirtuosa \textbf{ e por que todo el regno biua bien e uirtuosa mente . } Mas si cada vn çibdadano deue ser uirtuoso & ut tota ciuitas virtuosa existat , \textbf{ et ut totum regnum bene et virtuose viuat . } Sed si quilibet ciuis debet virtuose se habere ; \\\hline
3.2.32 & e en pero de rio sea muy bueno \textbf{ e sea assi commo medio dios . } Et por ende el regno es dicho ser muchedunbre & et ingenuos esse magis bonos et virtuosos quam ciues alios : propter quod regem ipsum tanquam omnibus excellentiorem decet esse optimum , et \textbf{ quasi semideum . Inde est igitur quod regnum dicitur esse multitudo , } in qua sunt multi nobiles \\\hline
3.2.32 & e sean honrrados \textbf{ so vn muy buen uaron } assi commo son vn Rey . & secundum virtutem , \textbf{ et ordinari sub uno viro optimo , } ut sub Rege . \\\hline
3.2.32 & que es en el regno e enla çibdat . \textbf{ Ca si la çibdat e el regno son ordenados prinçipalmente a buena uida uertuosa . } los moradores del regno e el pueblo & qualis debeat esse populus existens in ciuitate et regno . \textbf{ Nam si ciuitas et regnum principaliter ordinatur } ad vitam bonam et virtuosam , \\\hline
3.2.32 & Et por ende assi conmo dize el philosofo en el terçero libro delas politicas \textbf{ mas es çibdadano aquel que abonda en buenas obras e uertuosas } que si abondasse en riquezas o en nobleza de linage & habitatores regni et populum existentem in ciuitate et regno , oportet esse talem , \textbf{ quod viuat bene et virtuose . Inde est ergo quod ait Philosop’ 3 Politicorum quod magis est ciuis abundans in bonis operibus virtuosis , } quam si abundaret in diuitiis , \\\hline
3.2.33 & e el regno \textbf{ si y fuere pueblo establesçido de muchͣs perssonas medianeras } que nin sean muy ricos nin muy pobres . & et regnum , \textbf{ si ibi sit populus ex multis personis mediis constitutus . Tangit autem Philosophus 4 Politicorum , quatuor , } ex quibus sumi possunt quatuor viae , ostendentes meliorem esse politiam , \\\hline
3.2.33 & que nin sean muy ricos nin muy pobres . \textbf{ Et pone el pho en el quarto libro delas politicas quatro cosas } delas quales se pueden tomar quatro razonnes & ø \\\hline
3.2.33 & Et pone el pho en el quarto libro delas politicas quatro cosas \textbf{ delas quales se pueden tomar quatro razonnes } que muestran que meior es la poliçia & si ibi sit populus ex multis personis mediis constitutus . Tangit autem Philosophus 4 Politicorum , quatuor , \textbf{ ex quibus sumi possunt quatuor viae , ostendentes meliorem esse politiam , } vel melius esse regnum et ciuitatem , \\\hline
3.2.33 & mas con razon . \textbf{ La segunda de aquello que entre ellos ha de ser mayor amor ¶ } La terçera razon por que es y mayor egualdat . & ex eo quod talis populus magis rationabiliter viuit . \textbf{ Secunda ex eo quod inter ipsos habet esse maior dilectio . } Tertia , \\\hline
3.2.33 & La segunda de aquello que entre ellos ha de ser mayor amor ¶ \textbf{ La terçera razon por que es y mayor egualdat . } La quarta por que es y menor enuidia e menor menos preçio ¶ La primera razon se declara assi . & Secunda ex eo quod inter ipsos habet esse maior dilectio . \textbf{ Tertia , | quia ibi maior aequalitas . Quarta , } quia est ibi minor inuidia , \\\hline
3.2.33 & La terçera razon por que es y mayor egualdat . \textbf{ La quarta por que es y menor enuidia e menor menos preçio ¶ La primera razon se declara assi . } Ca si en el pueblo fueren muchs muy ricos & quia ibi maior aequalitas . Quarta , \textbf{ quia est ibi minor inuidia , | et minor contemptus . } Prima via sic patet . \\\hline
3.2.33 & e muchs muy pobres \textbf{ e pocas perssonas mediana s . apenas o nunca biuran acordadamente } nin con razon & et multi valde pauperes , \textbf{ et paucae personae mediae , } vix aut nunquam rationabiliter viuet . \\\hline
3.2.33 & con razon alos muy pobres \textbf{ mas por pequana ocasion } que ayan contra ellos manifiesta miente les enpeesçen . & Nam multum diuites ad multum pauperes nesciunt rationabiliter se habere , \textbf{ sed modica occasione sumpta eis manifeste nocent . Sic etiam multum pauperes ad nimium diuites nesciunt } se rationabiliter gereres insidiantur enim eis quomodo possint astute \\\hline
3.2.33 & mas por pequana ocasion \textbf{ que ayan contra ellos manifiesta miente les enpeesçen . } Assi avn los muy pobres non se saben auer & Nam multum diuites ad multum pauperes nesciunt rationabiliter se habere , \textbf{ sed modica occasione sumpta eis manifeste nocent . Sic etiam multum pauperes ad nimium diuites nesciunt } se rationabiliter gereres insidiantur enim eis quomodo possint astute \\\hline
3.2.33 & Ca si la vna parte dela çibdat fuere muy noble \textbf{ e muy poderosa et̃ muy rica . } la otra parte contraria desta fuere muy menguada & Nam si una pars ciuitatis sit superingenua , \textbf{ superpotens , | et superdiues : } alia vero huic contraria sit superegena , superdebilis \\\hline
3.2.33 & non se saben omillar . \textbf{ Mas los otros que sobrepuian en grand mengua } e en grand pobreza non saben enssennorear & nesciunt subiici . \textbf{ Qui autem } secundum excessum sunt indigentes et pauperes , \\\hline
3.2.33 & Mas los otros que sobrepuian en grand mengua \textbf{ e en grand pobreza non saben enssennorear } nin ser señores . & Qui autem \textbf{ secundum excessum sunt indigentes et pauperes , } nesciunt principari . \\\hline
3.2.33 & Ca assi commo dize el philosofo enł quarto libro delas politicas . \textbf{ los pobres han grant enuidia alos ricos } e los ricos desprecian mucho alos pobres & Nam pauperes ( ut dicitur 4 Polit’ ) \textbf{ maxime inuident diuitibus , } et maxime diuites contemnunt eos : \\\hline
3.2.33 & mas ninguna conpannia non es bien durable \textbf{ do ay muchͣ enuidia e grand menospreçio . } Por la qual cosa el pueblo es muy bueno & nulla societas est bene durabilis , \textbf{ ubi est multa inuidia | et contemptus . } Quare optimus est populus ex multis personis mediis constitutus , \\\hline
3.2.33 & Por la qual cosa el pueblo es muy bueno \textbf{ si fuere establesçido de muchͣs perssonas medianeras } ca entonçe de ligero biuran con razon & et contemptus . \textbf{ Quare optimus est populus ex multis personis mediis constitutus , } quia tunc de facili rationabiliter viuent , \\\hline
3.2.33 & por que en el su regno sean muchͣs perssonas medianeras \textbf{ por que los vnos non vengan atan grant pobreza } nin los otros atan grand riqueza & ut in regno suo abundent multae personae mediae ; \textbf{ ut ne aliis ad nimiam paupertatem deuenientibus efficiantur reliqui nimis diuites : } quod fieri poterit , \\\hline
3.2.33 & por que los vnos non vengan atan grant pobreza \textbf{ nin los otros atan grand riqueza } que entre ellos non aya egualdat . & ut in regno suo abundent multae personae mediae ; \textbf{ ut ne aliis ad nimiam paupertatem deuenientibus efficiantur reliqui nimis diuites : } quod fieri poterit , \\\hline
3.2.33 & e delas tierras \textbf{ e delas o tris possessiones } podria ser guardada algunan egualdat entre los çibdadan & nec quibuslibet indifferenter liceat quascunque possessiones emere , adhibita enim debita diligentia circa emptionem , et venditionem agrorum \textbf{ et terrarum , poterit aliqualis aequalitas reseruari inter ciues . } Consequitur autem populus \\\hline
3.2.34 & podria ser guardada algunan egualdat entre los çibdadan \textbf{ uanto alo presente parte nesçe el pueblo alcança tres bienes } si obedesçiere alos Reyes e alos prinçipes con grand acuçia . & et terrarum , poterit aliqualis aequalitas reseruari inter ciues . \textbf{ Consequitur autem populus | ( quantum ad praesens spectat ) tria , } si cum magna diligentia obediat regibus , \\\hline
3.2.34 & uanto alo presente parte nesçe el pueblo alcança tres bienes \textbf{ si obedesçiere alos Reyes e alos prinçipes con grand acuçia . } Et si guardare las leyes de los Reyes & ( quantum ad praesens spectat ) tria , \textbf{ si cum magna diligentia obediat regibus , | et principibus , } et obseruet leges regias . \\\hline
3.2.34 & e guardar las leyes . \textbf{ Ca lo primero desto alçança el pueblo uirtudes e grandes bienes } ¶ & quantum sit utile et expediens populo obedire Regibus et Principibus , et obseruare leges . \textbf{ Primo enim | ex hoc consequitur populus virtutes , } et maxima bona . \\\hline
3.2.34 & es enduzer los çibdadanos o uirtud . \textbf{ Ca en la derecha poliçia } assi commo dize el philosofo çerca el comienço del quarto libro delas polticas & intentio legislatoris est inducere ciues ad virtutem . \textbf{ In recta enim Politia } ( ut vult Philosophus ) \\\hline
3.2.34 & assi commo dize el philosofo çerca el comienço del quarto libro delas polticas \textbf{ vna misma es la uirtud del buen çibdadano } e del buen uaron . & circa principium 4 Polit’ \textbf{ eadem est virtus boni ciuis , } et boni viri : \\\hline
3.2.34 & vna misma es la uirtud del buen çibdadano \textbf{ e del buen uaron . } Et por esso mismo o por essas mismas costunbres & eadem est virtus boni ciuis , \textbf{ et boni viri : } et per idem siue per eosdem mores , \\\hline
3.2.34 & e del buen uaron . \textbf{ Et por esso mismo o por essas mismas costunbres } que es alguno buen çibdadano es buen omne . & et boni viri : \textbf{ et per idem siue per eosdem mores , } est aliquis bonus ciuis , \\\hline
3.2.34 & Et por esso mismo o por essas mismas costunbres \textbf{ que es alguno buen çibdadano es buen omne . } Mas alguno es buen çibdadano & et per idem siue per eosdem mores , \textbf{ est aliquis bonus ciuis , } et bonus homo . Est quidem aliquis bonus ciuis , \\\hline
3.2.34 & que es alguno buen çibdadano es buen omne . \textbf{ Mas alguno es buen çibdadano } si ben obedesçiere al prinçipe & est aliquis bonus ciuis , \textbf{ et bonus homo . Est quidem aliquis bonus ciuis , } si bene obediat principanti , \\\hline
3.2.34 & que sea en el gouernamiento derech \textbf{ que el buen çibdada no sea buen omne . } Et aquel que bien obedesçe al Rey es buen uaron & et opus bonum , oportet in recto regimine , \textbf{ quod bonus ciuis } sit bonus homo ; et qui bene subiicitur Regi , \\\hline
3.2.34 & que el buen çibdada no sea buen omne . \textbf{ Et aquel que bien obedesçe al Rey es buen uaron } casi el Rey non entendiesse & quod bonus ciuis \textbf{ sit bonus homo ; et qui bene subiicitur Regi , } sit bonus vir . \\\hline
3.2.34 & por que sea bueno e uirtuoso . \textbf{ por que las uirtudes son muy grandes bienes . } Et avn con grant diligençia deue estudiar cada vn çibdadano & et virtuosus , \textbf{ eo quod virtutes sunt maxima bona , } cum summa diligentia studere debet , \\\hline
3.2.34 & por que las uirtudes son muy grandes bienes . \textbf{ Et avn con grant diligençia deue estudiar cada vn çibdadano } por que obedezca al Rey & eo quod virtutes sunt maxima bona , \textbf{ cum summa diligentia studere debet , } ut Regi obediatur , \\\hline
3.2.34 & mas se allega ala natura bestial \textbf{ tanto mas es natural miente sieruo } e de natura seruil . & ad naturam bestialem , \textbf{ tanto est magis naturaliter seruus . Esse quidem sceleratum et affectatorem belli , } et turbatorem pacis , \\\hline
3.2.34 & e guardaren las leyes . \textbf{ Mas quanta salut se leu nata en el regno dela obediençia del Rey } conplidamente se muestra si fueren penssadas las palabras del philosofo & et obseruent leges . \textbf{ Quanta autem salus surgat in regno ex obedientia Regis , } sufficienter ostenditur , \\\hline
3.2.34 & Por la qual cosa \textbf{ assi commo es muy mala cosa al cuerpo desmanparar el alma } e non se gouernar por ella . & et vita regni . \textbf{ Quare sicut pessimum est corpori delinquere animam , } et non regi per eam , \\\hline
3.2.34 & e non se gouernar por ella . \textbf{ assi es muy mala cosa } que el regno desanpare las leyes e los mandamientos reales & et non regi per eam , \textbf{ sic pessimum est regno deserere leges regias } et praecepta legalia , et non regi per Regem . \\\hline
3.2.34 & que la egualdat de los humores o la sanidat de los cuerpos \textbf{ Et por ende con muy grant acuçia deue estudiar el pueblo } e todos los moradores del regno en obedesçer alos Reyes & qui sunt in regno potior est quam aequalitas humorum , \textbf{ vel quam sanitas corporum . Summo ergo opere studere } debet populus , et omnes habitatores regni circa obedientiam regiam , \\\hline
3.2.34 & e guardar las leyes . \textbf{ Commo desto se leunate tan grant bien } quanto es la paz e el assessiego de los que son en el regno & et obseruationem legum : \textbf{ cum ex hoc consurgat tantum bonum , } quantum bona est pax et tranquillitas existentium in regno . \\\hline
3.2.34 & e non vienen bien los tenporales . \textbf{ Et por fuerça los moradores del regno vienen a grant pobreza . } Et por ende si fuere penssado & fiunt depraedationes , \textbf{ oriuntur sterilitates deducuntur habitatores regni | ad inopiam . } Si ergo consideretur \\\hline
3.2.35 & Et pues que assi es \textbf{ quando alguno en alguna destas dichas maneras faze tuerto a otro manifiesta mente . } aquel a quien es fech tu elerto se en tristez & vel ea quae aliquo modo ordinantur ad ipsum . \textbf{ Cum ergo quis aliquo dictorum modorum manifeste forefacit in alium , } ille tristatus ex appetitu punitionis , \\\hline
3.2.35 & Conuiene a saber la honrra \textbf{ e la obediençia son tiradas al Rey con grand derech se mueue a sanna . } Et por ende dize el philosofo en el segundo libro de la rectorica & Cum enim haec duo , \textbf{ honor videlicet , et obedientia subtrahuntur a Rege , } merito prouocatur ad iram . Ideo dicitur 2 Rhet’ \\\hline
3.2.35 & que deuien . \textbf{ En essa misma manera avn } assi commo ally dize el philosofo & Nam si nos non despicerent , impenderent nobis honorem dignum . \textbf{ Sic etiam ibidem dicitur , } quod et ad minores irascimur , \\\hline
3.2.35 & Et deuedes saber \textbf{ que al Rey parte nesçen quatro maneras de perssonas . } Conuiene a saber . El padir . Et la madre . & et qui pertinent ad ipsum . \textbf{ Ad Regem autem pertinere videntur quatuor genera personarum } videlicet parentes \\\hline
3.2.35 & en el capitulo dela sanna \textbf{ que nos nos enssanamos contra aquellos que fazen tuerto anros parientes } e a nuestros fijos & efficitur iniuria aliqua . Hoc est ergo quod dicitur 2 Rhet’ cap’ de ira , \textbf{ quod irascimur forefacientibus in parentes , } filios , \\\hline
3.2.35 & e a nuestras mugers e a nuestros subditos . \textbf{ Ca torpe cosa es alos Reyes } e avn alos otros & filios , \textbf{ uxores , } et subiectos \\\hline
3.2.35 & nin contra ningunos derechs del regno . \textbf{ Por que muy mala cosa es } segunt dize el philosofo en el quinto libro & nec in aliquas personas ei subiectas , \textbf{ nec in aliqua iura regni . Pessimum est enim ( ut dicitur 7 Pol’ ) } non instruere pueros ad virtutem , \\\hline
3.2.36 & La primera que sean bien fechores e liberales e francos ¶ \textbf{ La segunda que sean fuertes e de grant coraçon . la terçera que sean eguales e derechureros¶ } La primera se prueua assi . & et liberales . Secundo fortes \textbf{ et magnanimi . Tertio aequales et iusti . Primum autem sic patet . Nam vulgus non percipit } nisi sensibilia bona , \\\hline
3.2.36 & e aueres la segunda cosa \textbf{ para que los Reyes sean ama dos del pueblo es que deuon ser fuertes e de grandes coraçones } esponiendo se assi & et honorat beneficos in pecunia , et liberales . \textbf{ Secundo ut Reges amentur in populo , | debent esse fortes et magnanimi , } ponentes \\\hline
3.2.36 & Ca el pueblo muchama los fuertes \textbf{ e alos de grandes coraçones } que se ponen a quales se quier uenturas & Nam populus valde diligit fortes \textbf{ et magnanimos , } exponentes se pro bonis communibus : \\\hline
3.2.36 & Et por ende amamos los fuertes \textbf{ e los de grandes coraçones ¶ } La terçera cosa & qui possunt nobis benefacere nos saluando \textbf{ et liberando , ideo diligimus fortes } et cordatos . Tertio , ut Reges diligantur a populo , \\\hline
3.2.36 & es quales conuiene de ser derechureros e eguales . \textbf{ Ca el pueblo mayormente se le una taria a mal querençia del Rey } si viesse que el non guardaua nistiçia . & decet eos esse iustos , et aequales . \textbf{ Nam maxime prouocatur populus ad odium Regis , } si viderit ipsum non obseruare iustitiam : \\\hline
3.2.36 & si mostraren grandescrueldades \textbf{ e dieren grandes penas a aquellos que turban de mala manera el regno o la çibdat . } Et por ende dize el philosofo en el segundo libro de la rectorica & et politiam per turbant , \textbf{ inexquisitas crudelitates exerceant . } Ideo dicitur 2 Rhetor’ \\\hline
3.2.36 & por que sean mas temidos \textbf{ e por que con mayor acuçia se guarde la iustiçia } que mayor penaden & ut decet Reges magis timeantur , \textbf{ et ut virilius obseruent iustitiam , } magis punire , \\\hline
3.2.36 & e por que con mayor acuçia se guarde la iustiçia \textbf{ que mayor penaden } e mas cruelmente se ayan contra los amigos & et ut virilius obseruent iustitiam , \textbf{ magis punire , } et seuerius se gerere contra amicos , \\\hline
3.2.36 & que de ser temidos . \textbf{ Ca dicho es dessuso que la prinçipal entençion del Rey } e de cada vn prinçipe deue ser & de leui patere potest quod licet utrunque sit necessarium , \textbf{ amari } tamen debent \\\hline
3.3.1 & quando sabe gouernar assi mesmo . \textbf{ Et esta es menor sabiduria } que es la sabiduria yconomica & scit regere et gubernare : \textbf{ et haec est minor prudentia , quam oeconomica et regnatiua . } Nam minus est , \\\hline
3.3.1 & por que sabe bien consseiar \textbf{ e bien guiar a buena fin . } Et pues que assi es do son falladas departidas razones de bien & quia scit bene consiliari , \textbf{ et bene dirigere ad bonum finem : } ubi ergo reperitur alia \\\hline
3.3.1 & e ha de despenssar los bienes de la casa . \textbf{ Et enel teçero libro ensseñamos al Rey o al prinçipe } en quanto es cabeça del regno o del prinçipado & et ut habet dispensare bona domestica . \textbf{ In tertio vero eruditur Rex aut Princeps ut est caput regni aut principatus , } et ut habet ferre leges \\\hline
3.3.1 & mas enl çibdadano es meester sabiduria alguna . \textbf{ por la qual aya buena opinion de aquellas cosas quel son mandadas por el Rey . } Mas esta sabiduria se departe & sed solum agitatur a fabro , \textbf{ sed in ciue requiritur prudentia aliqua per quam habeat bonam opinionem de iis quae imperantur a Rege . } Differt autem haec prudentia a prudentia particulari , \\\hline
3.3.1 & Mas esta sabiduria se departe \textbf{ de la sabidurina particular } la qual pusiemos en la primera manera de sabiduria . & sed in ciue requiritur prudentia aliqua per quam habeat bonam opinionem de iis quae imperantur a Rege . \textbf{ Differt autem haec prudentia a prudentia particulari , } quam collocauimus in prima specie . \\\hline
3.3.1 & que el es ensseñado en aquella sçiençia de la qual ha de ser maestro . \textbf{ En essa misma manera ninguno non deue ser tomado } para dignidat de caualleria & nisi constet ipsum esse doctum in arte illa : \textbf{ sic nullus assumendus est ad dignitatem militarem , } nisi constet ipsum diligere bonum regni \\\hline
3.3.2 & si queremos saber \textbf{ en quales regnos o en quales tierras son meiores lidiadores . } Conuiene de tener mientes en estas dos cosas sobredichas . & et prudentia erga bella . \textbf{ Quare si scire volumus in quibus regionibus meliores sunt bellatores , } oportet attendere circa praedicta duo . In partibus igitur nimis propinquis soli , \\\hline
3.3.2 & que son muy cercanas al sol \textbf{ por la grant calentura } que resçiben del sol son secas & ø \\\hline
3.3.2 & que resçiben del sol son secas \textbf{ e han mayor sabiduria } e han menos de sangre . & ubi dicitur . Nationes quae vicinae sunt soli , \textbf{ nimio calore siccare ; amplius quidem sapiunt , sed modicum abundant in sanguine , } et propterea non habent constantiam pugnandi neque fiduciam , \\\hline
3.3.2 & assi que non teman las feridas . \textbf{ Enpero por la grant abondança de la sangre son sañudos e arrebatados } e non son sabios en la batalla . & ut vulnera non metuant , \textbf{ tamen propter sanguinis abundantiam | et impetum , sunt quasi furibundi } et imprudentes . Ideo non omnino sunt utiles operibus bellicis , \\\hline
3.3.2 & Enpero deuemos tener mientes en estos tales ensseñamientos \textbf{ que se deuen entender en la mayor parte . } Ca en todas las partes son sabidores & nec omnino soli propinqua eligendi sunt bellantes , \textbf{ ut tam prudentia quam animositate participent . Aduertendum tamen circa talia , documenta accipienda esse ut in pluribus . Nam in omnibus partibus sunt aliqui industres , } et aliqui animosi : \\\hline
3.3.2 & e algunos fuertes de coraçon . \textbf{ Enpero en la mayor parte } aquellos que son mas çerca del sol & ut tam prudentia quam animositate participent . Aduertendum tamen circa talia , documenta accipienda esse ut in pluribus . Nam in omnibus partibus sunt aliqui industres , \textbf{ et aliqui animosi : } ut plurimum tamen soli propinqui animositate deficiunt , \\\hline
3.3.2 & que las otras gentes . \textbf{ Visto de quales partes son los meiores lidiadores } finca de ver & ut magis animositate participent . Viso \textbf{ ex quibus partibus meliores sunt bellatores : } videre restat , \\\hline
3.3.2 & e deuan ser animosos \textbf{ e de grant coraçon } para acometer & non debeant horrere sanguinis effusionem , \textbf{ debeant esse animosi ad inuadendum , } et etiam potentes ad tolerandum labores : \\\hline
3.3.2 & por que por la su arte han los braços acostunbrados e apareiados para ferir . \textbf{ Avn en essa misma manera son aprouechables los carniceros } por que non aborresçen el derramamiento de la sangre & et assueta ad percutiendum . Sic \textbf{ etiam utiles sunt Macellarii : } quia non horrent sanguinis effusionem , cum assueti sint ad occisionem animalium , \\\hline
3.3.2 & son de resçebir a las obras de la batalla \textbf{ por que non pueden sin grant osadia acometer los puercos monteses } e les otros fuertes venados . & etiam aprorum admittendi sunt ad huiusmodi opera : \textbf{ quia non sine magna audacia contingit aliquos inuadere apros . } Sunt ergo tales animosi \\\hline
3.3.2 & por que non pueden sin grant osadia acometer los puercos monteses \textbf{ e les otros fuertes venados . } Et por ende tales son de grant coraçon & etiam aprorum admittendi sunt ad huiusmodi opera : \textbf{ quia non sine magna audacia contingit aliquos inuadere apros . } Sunt ergo tales animosi \\\hline
3.3.2 & e les otros fuertes venados . \textbf{ Et por ende tales son de grant coraçon } e estremados para la batalla & quia non sine magna audacia contingit aliquos inuadere apros . \textbf{ Sunt ergo tales animosi } et strenui ad bellandum . \\\hline
3.3.2 & e estremados para la batalla \textbf{ ante por auentura non es menor peligro lidiar con el puerco montes } que lidiar con el enemigo . & et strenui ad bellandum . \textbf{ Imo forte non minus periculosum est bellare cum apro , } quam pugnare cum hoste . \\\hline
3.3.2 & para las obras de la batalla \textbf{ por que tales son acostunbrados a grandes trabaios . } Et por ende destas tales artes son de escoger los lidiadores & Rursus venatores ceruorum non sunt repudiandi ab actibus bellis : \textbf{ eo quod tales assueti sunt ad labores nimios . } Ex his ergo artibus propter ea quae diximus eligendi sunt bellatores . Barbitonsores autem et sutores , \\\hline
3.3.2 & a tal es de la nauaia a la espada . \textbf{ Avn en essa misma manera los apotecarios e los paxareros e los bretadores } que toman los paxaros con el brete e los pescadores no son do escoger & et rasorii ad clauam ? \textbf{ Sic etiam Apothecarii , Aucupes , et Piscatores non sunt eligendi ad huiusmodi opera : } eo quod non habeant artem conformem operibus bellicosis . Potest ergo contingere \\\hline
3.3.2 & Enpero puede contesçer \textbf{ que en cada vna destas artes son algunos buenos lidiadores e atreuidos } e ay otros temerosos e de flacos coraçones . & eo quod non habeant artem conformem operibus bellicosis . Potest ergo contingere \textbf{ quod in qualibet arte sint aliqui bellicosi et audaces ; aliqui vero timidi } et pusillanimes . \\\hline
3.3.2 & que en cada vna destas artes son algunos buenos lidiadores e atreuidos \textbf{ e ay otros temerosos e de flacos coraçones . } Mas quanto es de la manera de la arte & quod in qualibet arte sint aliqui bellicosi et audaces ; aliqui vero timidi \textbf{ et pusillanimes . } Sed \\\hline
3.3.3 & que los mançebos fuessen usados e acostubrados del . xiiij° . \textbf{ año adelante a fuertes trabaios } assi conmo a los trabaios de la caualleria & ø \\\hline
3.3.3 & e nos delectamos en ellas . \textbf{ Et si quisiere el ponedor de la ley fazer los çibdadanos buenos lidiadores } e fazer los apareiados & et delectamur in illis : \textbf{ si vult legislator ciues bellatores facere , } et reddere ipsos aptos ad pugnandum , potius debet praeuenire tempus quam praetermittere . \\\hline
3.3.3 & Ca assi commo dize \textbf{ vegeçio meior cosa es } que el mançebo usado cuyde & et reddere ipsos aptos ad pugnandum , potius debet praeuenire tempus quam praetermittere . \textbf{ Nam } ut ait Vegetius , melius est ut iuuenis exercitatus causetur aetatem nondum aduenisse pugnandi gratia , quam doleat praeteriisse . Est \\\hline
3.3.3 & por que non es pequena \textbf{ nin ligera arte auer sabiduria de las armas . } Ca si quier sea cauallero si quier peon el que ha de lidiar paresçe & ab ipsa iuuentute assuescere ad artem bellandi : \textbf{ quia non parua nec leuis ars esse videtur armorum industria . } Nam siue equitem siue peditem oportet esse bellantem , \\\hline
3.3.3 & commo de los peones son de catar muchas cautelas \textbf{ por que çiertamente grand locura es aquella hora } o aquel tienpo aprender de lidiar & Tam enim in pedestri quam etiam in equestri pugna , \textbf{ sunt multae adhibendae cautelae . Fatuum est quidem non prius , } sed tunc velle addiscere bellare , quando imminet necessitas pugnandi , \\\hline
3.3.3 & finca de ver \textbf{ por quales señales se han de conosçer los buenos lidiadores . } Et para esto conuiene de saber & videre restat , \textbf{ ex quibus signis cognosci habeant homines bellicosi . } Sciendum igitur viros audaces et cordatos utiliores esse ad bellum , \\\hline
3.3.3 & Et para esto conuiene de saber \textbf{ que los omnes osados eatreuidos } e de grandes coraçones & ø \\\hline
3.3.3 & que los omnes osados eatreuidos \textbf{ e de grandes coraçones } son mas prouechosos para la batalla & ex quibus signis cognosci habeant homines bellicosi . \textbf{ Sciendum igitur viros audaces et cordatos utiliores esse ad bellum , } quam timidos . Rursus , \\\hline
3.3.3 & son mas prouechosos para la batalla \textbf{ que los temerosos e de flacos coraçones . } Otrossi los omnes fuertes de cuerpo & Sciendum igitur viros audaces et cordatos utiliores esse ad bellum , \textbf{ quam timidos . Rursus , } homines fortes et duros corpore , \\\hline
3.3.3 & e algunas temerosas los omnes \textbf{ que son semeiantes masa las animalias lidiadoras } paresçe ser mas prouechosos para la batalla . & ø \\\hline
3.3.3 & que todas las otras animalas \textbf{ por que han grandes braços e anchos pechos . } Et pues que assi es & ø \\\hline
3.3.3 & Et pues que assi es tales son de escoger los lidiadores \textbf{ que por la mayor parte sean apareiados } para las obras de la batalla . & Tales ergo quaerendi sunt bellatores , \textbf{ quia ut plurimum contingit eos esse aptos ad actiones bellicas . } Quantum ad praesens spectat , \\\hline
3.3.4 & para la batalla \textbf{ Lo primero conuiene que los omnes lidiadores puedan sofrir grandes pesos . } lo segundo que puedan sofrir grandes trabaios & aut Princeps eligere . \textbf{ Primo enim oportet pugnatiuos homines posse sustinere magnitudinem ponderis . Secundo posse sufferre } quasi assiduos membrorum motus , \\\hline
3.3.4 & Lo primero conuiene que los omnes lidiadores puedan sofrir grandes pesos . \textbf{ lo segundo que puedan sofrir grandes trabaios } e continuados mouimientos de los mienbros & aut Princeps eligere . \textbf{ Primo enim oportet pugnatiuos homines posse sustinere magnitudinem ponderis . Secundo posse sufferre } quasi assiduos membrorum motus , \\\hline
3.3.4 & lo segundo que puedan sofrir grandes trabaios \textbf{ e continuados mouimientos de los mienbros } Lo terçero que puedan sofrir escasseza de vianda e fanbre e sed . & Primo enim oportet pugnatiuos homines posse sustinere magnitudinem ponderis . Secundo posse sufferre \textbf{ quasi assiduos membrorum motus , } et labores magnos . \\\hline
3.3.4 & nin aborrezcan de derramar su sangreLo septimo conuiene \textbf{ que ayan buena disposiçion e buena sabidura } para defender assi & Quinto \textbf{ quasi non appretiare corporalem vitam . Sexto non horrere sanguinis effusionem . Septimo habere aptitudinem , } et industriam ad protegendum se \\\hline
3.3.4 & que es meester a los lidiadores \textbf{ es que puedan sofrir grandes pesos . } Ca los desarmados de qual quier parte & et feriendum alios . Octauo verecundari et erubescere eligere turpem fugam . \textbf{ Est enim primo necessarium bellantibus posse sustinere ponderis magnitudinem . } Nam inermes a quacunque parte foriantur , succumbunt : \\\hline
3.3.4 & Lo segundo conuiene a los lidiadores \textbf{ que puedan sofrir continuados mouimientos de los mienbros . } Ca si alguno estudiere quedo en la batalla & inutilis est ad bellum . \textbf{ Secundo bellantibus expedis posse sufferre quasi assiduos membrorum motus . } Nam si quis in bello non continue se ducat , \\\hline
3.3.4 & el enemigo non errara el golpe en el ferir \textbf{ por la qual cosa sienpre estara en peligro de resçebir mayores golpes . } ca cosa prouada es & aduersarius non fallitur in percutiendo , \textbf{ quare semper exponitur ad sustinendum fortiores ictus : } expertum est enim quod homine continue se ducente et mouente , \\\hline
3.3.4 & apenas o nunca le puede alcançar ninguna ferida . \textbf{ Mas por la mayor parte sienpre fuye los colpes . } Ca assi commo la señal del uallestero & vix aut nunquam ad plenum aliqua percussio potest ipsum attingere , \textbf{ sed semper vulnera subterfugit . } Nam sicut si signum se moueret \\\hline
3.3.4 & Lo terçero los omnes lidiadores \textbf{ non deue auer cuydado de escassa uianda . } Ca muy graue cosa & ø \\\hline
3.3.4 & non deue auer cuydado de escassa uianda . \textbf{ Ca muy graue cosa } que el peso de las armas aquallas & ut diu tolerare possint assiduum membrorum motum . Tertio homines pugnatiuos decet non curare de parcitate victus . \textbf{ Nam graue est , ultra armorum pondera } et eorum quae requiruntur \\\hline
3.3.4 & mas avn puesto que e ningun agrauamiento pudiessen auer los lidiadores \textbf{ gunand cunplimiento de viandas } avn meestet les es en la batalla de coner poco e de fazer abstineçia & immo \textbf{ et si adesset pugnantibus ciborum ubertas , } adhuc esset eis necessaria abstinentia , \\\hline
3.3.4 & nunca ninguno es fuerte de coraçon \textbf{ nin buen lidiador } si en alguna manera non fuere sin temer & nunquam quis est fortis animo \textbf{ et bonus bellator , } nisi aliquo modo \\\hline
3.3.4 & que sin miedo en los periglos de la muerte . \textbf{ Ca pertenesçe al fuerte e al buen lidiador } assi commo dize el philosofo en el terçero libro de las Ethicas & sit impauidus circa pericula mortis . Spectat enim ad fortem \textbf{ et ad bonum bellatorem , } ut innuit Philosophus 3 Ethic’ \\\hline
3.3.4 & por que si alguno ouiere el coraçon muele \textbf{ e fuere assi commo mugeril aborresçra esparzer la sangre } e non osara fazer llagas a los enemigos . & de facili eligit turpem fugam . Sexto pugnantes non debent horrere sanguinis effusionem . \textbf{ Nam si quis cor molle habens , muliebris existens , horreat effundere sanguinem ; non audebit hostibus plagas infligere , et per consequens bene bellare non potest . } Septimo decet eos habere aptitudinem , \\\hline
3.3.4 & Mas entre todas las cosas \textbf{ que fazen al omne buen lidiador } es dessear de ser honrado por batalla & apud quos honorantur fortes . \textbf{ Inter caetera autem quae reddunt hominem bellicosum , } est diligere honorari expugna , \\\hline
3.3.5 & que los lidiadores deuen auer finca de demandar \textbf{ quales son los meiores lidiadores . } o los de las çibdades e los nobles . & Numeratis iis quae habere debent bellatores viri : \textbf{ restat inquirere , | qui sunt meliores bellantes , } an urbani an nobiles , \\\hline
3.3.5 & que los omnes lidiadores sean tales \textbf{ que puedan sofrir grandes pesos } e grandes trabaios continuados en los sus cuerpos & Dicebatur enim , viros pugnatiuos tales esse debere , \textbf{ qui possent sustinere magnitudinem ponderis , } continuum laborem membrorum , \\\hline
3.3.5 & que puedan sofrir grandes pesos \textbf{ e grandes trabaios continuados en los sus cuerpos } e que puedan sofrir escasseça de viandas & qui possent sustinere magnitudinem ponderis , \textbf{ continuum laborem membrorum , } parcitatem victus , \\\hline
3.3.5 & que son criados viçiosamente . \textbf{ por que los aldeanos son acostunbrados a leuar grandes cargas e grandes pesos . Et por ende non se agrauiarian de grant carga de amas en las batallas } aquellos que continuadamente entienpo de la paz son acostunbrados a mayores cargas . & Sunt enim rustici assueti \textbf{ ad magnitudinem ponderum | non enim in bellis grauabit eos armorum sarcina , } qui assidue tempore pacis assueti sunt \\\hline
3.3.5 & por que los aldeanos son acostunbrados a leuar grandes cargas e grandes pesos . Et por ende non se agrauiarian de grant carga de amas en las batallas \textbf{ aquellos que continuadamente entienpo de la paz son acostunbrados a mayores cargas . } Et avn non canssan en correr & non enim in bellis grauabit eos armorum sarcina , \textbf{ qui assidue tempore pacis assueti sunt | ad maiora pondera . } Nec etiam eos fatigabit cursus \\\hline
3.3.5 & nin vsar los mouimientos de los otros mienbros del cuerpo . \textbf{ Ca a estos e a mayores trabaios son vsados de cadal dia . } Otrossi estos non resçiben gunand agrauiamento de la escasseza de la vianda & vel aliorum membrorum motus , \textbf{ qui ad hos et ad maiores sunt continue assueti . Rursus , } eos \\\hline
3.3.5 & Ca a estos e a mayores trabaios son vsados de cadal dia . \textbf{ Otrossi estos non resçiben gunand agrauiamento de la escasseza de la vianda } a los quales abasta el agua en la sed & qui ad hos et ad maiores sunt continue assueti . Rursus , \textbf{ eos | non affliget parcitas victus , } quibus potus aquae satisfaciebat in siti , \\\hline
3.3.5 & Otrossi los aldeanos non se agranian en mal yazer nin en mal estar \textbf{ ca non temen de la grant calentura del sol } nin han cuydado de sonbra & ex incommoditate iacendi vel standi , \textbf{ qui solis ardorem non timent , } umbras non curant , balneorum solatia nesciunt , \\\hline
3.3.5 & por las quales se puede prouar \textbf{ que los nobles e los cibdadanos son meiores lidiadores . } Ca entre todas las cosas & inter cetera , per quae quis redditur bonus pugnatiuus , \textbf{ est } ( ut dicebatur ) \\\hline
3.3.5 & por las quales tenemos a alguno \textbf{ por buen lidiador es querer ser honrrado por la batalla } e tomar uerguença de foyr torpemente & ( ut dicebatur ) \textbf{ velle honorari ex pugna , } et erubescere turpem fugam . \\\hline
3.3.5 & Lo que fizo a ector atreuido en las armas \textbf{ e buen lidiador . } Ca dizia ector & ( ut ait Philosophus 3 Ethic’ ) \textbf{ quod Hectorem fecit audacem . } Dicebat enim Hector , \\\hline
3.3.5 & que si el fuyesse de la batalla \textbf{ que polimidias otro cauallero le denostaria muy mal . Avn en essa misma manera diomedes fue fecho atreuido cauallero . } Ca dizie que si boluiesse las espaldas en la batalla & Si ex pugna fugiam , \textbf{ Polydamas mihi redargutiones ponet . | Sic etiam et Diomedes hoc modo effectus fuit strenuus , } quia dicebat , \\\hline
3.3.5 & Por la qual cosa commo querer auer honrra de la batalla \textbf{ e tomar uergueña de torpe fecho } mas partenezca a los nobles e a los fijosdalgo & quia dicebat , \textbf{ Si in bello terga vertam , Hector cum concionabitur } inter Troianos , dicet , \\\hline
3.3.5 & que los villanos e los aldeanos \textbf{ por que mayor uergueña toman de foyr } que ellos . & Quare cum velle honorari \textbf{ et erubescere de aliquo turpi facto , magis conueniat nobilibus quam rusticis , ii meliores esse videntur ad pugnam , } eo quod verecundentur fugere . \\\hline
3.3.5 & que la fortaleza del pueblo . \textbf{ Por la qual cosa commo los nobles omnes sean } mas sotiles comunalmente & quam corporis fortitudo . \textbf{ Quare cum communiter nobiles homines industriores sint rusticis , } sequitur hos meliores esse pugnantes . Videntur enim haec duo maxima esse ad obtinendam victoriam , \\\hline
3.3.5 & que los aldeanos siguese \textbf{ que son meiores lidiadores . } Ca paresçe que estas dos cosas son prinçipales & ø \\\hline
3.3.5 & mas son de escoger aldeanos que nobles \textbf{ por que y mucho vale uso de traer grandes cargas } e de sofrir grandes trabaios . & itaque certamine magis eligendi sunt rurales , quam nobiles : \textbf{ eo quod illis maxime valet assuefactio ad portationem ponderum , } et tolerantiam laborum . In equestri vero magis eligendi sunt ipsi nobiles : \\\hline
3.3.5 & por que y mucho vale uso de traer grandes cargas \textbf{ e de sofrir grandes trabaios . } Mas en la batalla de los de cauallo & itaque certamine magis eligendi sunt rurales , quam nobiles : \textbf{ eo quod illis maxime valet assuefactio ad portationem ponderum , } et tolerantiam laborum . In equestri vero magis eligendi sunt ipsi nobiles : \\\hline
3.3.6 & si aquellos pocos romanos non fueren primeramente muy vsados de las armas \textbf{ e si non ouieran grant sabiduria de lidiar . } Et non es cosa desconuenible & nisi plus illis fuissent exercitati in armis , \textbf{ et magis habuissent bellandi industriam . } Non est enim inconueniens virum prudentem \\\hline
3.3.6 & Ca el vso en cada vn negocio \textbf{ e en cada fecho de grant osadia } por que non ayan miedo los omnes de fazer & ø \\\hline
3.3.6 & Ca si el az si quier de peones \textbf{ si quier de caual leros non andudiere ordonadamentedos males se siguen dende . } Ca non guardando la orden & Nam \textbf{ si acies siue peditum siue militum non ordinate incedat , } duo mala inde consequuntur . Nam non seruato debito ordine , \\\hline
3.3.6 & Et si muy desordenado andudiere deuen le echar de la az \textbf{ assi commo a aquel non es prouechoso lidiador . } Lo segundo son de vsar los lidiadores & vel si nimis delinquat , \textbf{ ipsum omnino repellat ab acie tanquam inutilem bellatorem . } Secundo exercitandi sunt bellatores \\\hline
3.3.6 & Lo primero para assechar e ascuchar el estado de los enemigos . \textbf{ ca buena cosa es en la fazienda auer algunos mas ligeros que los otros } que non puedan ser alcançados do ligero de los enemigos & Primo ad explorandum inimicorum facta . \textbf{ Nam bonum est in exercitu aliquos exiliores praecurrere , } qui de facili non possint ab ipsis hostibus comprehendit , \\\hline
3.3.6 & Lo segundo esto es prouechoso \textbf{ para ganar meior lugar en la batalla . por que el logar mucho ayuda a la batalla . } Et por ende si los lidiadores fueren usados a correr & Secundo hoc est utile ad obtinendum meliorem locum . \textbf{ Nam et locus multum facit ad pugnam . } Ideo si bellatores exercitati sunt ad cursum , \\\hline
3.3.6 & Et por ende si los lidiadores fueren usados a correr \textbf{ mas ligeramente ganaran meior logar parar lidiar } Lo terçero esto es aprouechoso & Ideo si bellatores exercitati sunt ad cursum , \textbf{ facilius obtinebunt aptiorem locum ad pugnandum . Est } etiam hoc utile ad prosequendum hostes fugientes . \\\hline
3.3.6 & Lo segundo para espantar los enemigos . \textbf{ Lo tercero para fazer mayores llagas . } Ca contesçe algunas vezes de fallar algunas carcauas e arroyos e açequias & Quod etiam ad tria est utile . Primo ad remouendum impedimenta . \textbf{ Secundo ad terrendum aduersarios . Tertio ad infligendum maiores plagas . } Contingit enim aliquando inuenire fossas \\\hline
3.3.6 & que sin salto non lo pueden saltar nin passar \textbf{ por la qual cosa prouechosa cosa es el saltar } para tirar estos enbargos . & quae sine saltu in via transire non possunt : \textbf{ quare utile est ad remouenda impedimenta , } ut equites sic sint docti , \\\hline
3.3.6 & si contesçiere que ellos esten de pie \textbf{ si quieren ser buenos lidiadores } conuiene que en su mançebia sean usados a saltar & et etiam milites , \textbf{ si contingat eos pedestres esse , si volunt boni bellatores existere , } sic ab ipsa iuuentute exercitandi sunt ad saliendum , \\\hline
3.3.6 & por razon del mouimiento vale \textbf{ para fazer mayores llagas . } e et podemos sin aquellas tres cosas aque dixiemos & ipse saltus ratione motus facit \textbf{ ut plaga amplior infligatur . } Possumus autem praeter tria praedicta , \\\hline
3.3.7 & aque se deuen vsar los lidiadores . \textbf{ Lo primero se deuen vsar aleuar grandes pesos . } Lo segundo a acometer & ad quae exercitari debent homines bellicosi . \textbf{ Primo enim exercitandi sunt ad portandum pondera . } Secundo ad inuadendum \\\hline
3.3.7 & por el arte del esgrima \textbf{ mas desto faremos espeçial capitulo . } lo primero dezimos & et ensibus . \textbf{ Sed de hoc speciale capitulum faciemus . } Primo enim sunt bellatores exercitandi \\\hline
3.3.7 & lo primero dezimos \textbf{ que son de vsar los lidiadores a leuar grandes pesos } en tal manera que se acostubren a leuar mayor peso & Sed de hoc speciale capitulum faciemus . \textbf{ Primo enim sunt bellatores exercitandi } ad portandum pondera , \\\hline
3.3.7 & que son de vsar los lidiadores a leuar grandes pesos \textbf{ en tal manera que se acostubren a leuar mayor peso } que el peso de las armas & Primo enim sunt bellatores exercitandi \textbf{ ad portandum pondera , | ut plus ponderis portare assuescant } etiam quam sit armorum sarcina . \\\hline
3.3.7 & Et por ende \textbf{ quando alguno es acostunbrado a leuar mayor carga } que la carga de las armas semeial & quasi natura quaedam . \textbf{ Cum ergo quis assuetus ad portandum maius pondus , } videtur sibi \\\hline
3.3.7 & que son de leuar en las batallas . \textbf{ Et por ende prouechosa cosa es de se acostunbrar los lidiadores a leuar grandes pesos . } Lo segundo se deuen usar los lidiadores a acometer & sed etiam plura alia sunt ferenda in bello : \textbf{ ideo | etiam ad maiora pondera } non est inutile assuescere bellatores . \\\hline
3.3.7 & Et los moços \textbf{ que querian acostunbrara fazer los buenos lidiadores . } vsauan los a ferir en aquellos palos & et iuuenes \textbf{ quos volebant } facere optimos bellatores exercitabant ad palos illos ita , \\\hline
3.3.7 & ca esgrimiendo el dardo \textbf{ por el mayor mouimineto } que el ome faze en el ayre & et postea fortiter impellendum : vibrato enim telo propter maiorem motum \textbf{ quem efficit in aere , longius pergit } et amplius vulnus infligit . \\\hline
3.3.7 & mas aluene fiere \textbf{ e mayor colpe faze . } Lo quarto son de vsarlos lidiadores & quem efficit in aere , longius pergit \textbf{ et amplius vulnus infligit . } Quarto exercitandi sunt bellantes ad iaciendum sagittas , \\\hline
3.3.7 & para ferirlos . \textbf{ prouechosa cosa es lançar las saetas } mas puesto que los lidiadores se puedan ayuntar con los enemigos & Nam quia contingit quod ipsos hostes non possumus immediate attingere , \textbf{ utile est eos sagittis impugnare : } immo dato quod pugnantes se cum hostibus possint coniungere , \\\hline
3.3.7 & ante que se apunte con ellos \textbf{ prouechosa cosa es de los espantar con los arcos e con las ballestas . } Ca leemos de çipion africano & antequam coniungantur proficuum est eos arcubus \textbf{ et ballistis terrere . } Legitur enim de Africano Scipione , \\\hline
3.3.7 & Et los mançebos vsauan en el yuierno \textbf{ a sobir en ellos solos techos . } Et en el uerano en el canpo primera miente & in hyeme exercitabantur sub tecto : \textbf{ aestate vero in campo ; } et primo equos illos ascendebant inermes , \\\hline
3.3.7 & despues \textbf{ que por vna grant parte del dia eran usados en las armas } si tienpo era conuenible para nadar & quod iuuenes futuri bellatores \textbf{ postquam per magnam partem dici exercitati essent ad arma , } si tempus erat natationi congruum , \\\hline
3.3.7 & e algunos a todos . \textbf{ Et esto en qual manera sea non ha menester grant estudio . } Ca non se puede asconder a ome sabio . & Quod quomodo sit , \textbf{ non magna consideratione eget , } et solertem mentem latere non potest . \\\hline
3.3.8 & P Paresçe que los negoçios de la batalla entre los otros son mas periglosos . \textbf{ Et por ende es de poner grant acuçia en ellos . } Ca en tales cosas tan periglosas non puede auer omne tantas cautelas & inter caetera periculosiora esse videntur , \textbf{ ideo in eis est magna diligentia adhibenda . } In talibus igitur non potest quis superabundare cautelis . \\\hline
3.3.8 & que mas non sean menester . \textbf{ Ca en la batalla sienpre auemos de tomar mayor acuçia } de que demandan las batallas . & In talibus igitur non potest quis superabundare cautelis . \textbf{ In pugna enim omnino est eligendum , } maiorem diligentiam habuisse quam bella commissa requirerent , \\\hline
3.3.8 & en algun logar \textbf{ o quiere y fazer mayor tardança } si aquel logar en algun caso o a auentura puedan a desora los enemigos venir & postquam exercitus suam dietam compleuit , \textbf{ alicubi vult pernoctare , vel ulteriorem moram contrahere , } si ad locum illum in aliquo casu , \\\hline
3.3.8 & e manden a cada vno qual cosa deua fazer . \textbf{ Mostrado que prouechosa cosa es de fazer los castiellos . } avn en qual manera los enemigos presentes son de fazer los castiellos & quod ipsum oporteat facere . \textbf{ Ostenso utile esse castra construere , } et qualiter \\\hline
3.3.8 & Mas cerca del assentamiento \textbf{ quanto pertenesçe a lo persente son de penssar quatro cosas . } la primera que sea y abondamiento de agua & Circa situm \textbf{ ( quantum ad praesens spectat ) sunt quatuor attendenda . Primo } ut sit ibi copia aquae , \\\hline
3.3.8 & del qual son de leuantar las guarniçiones \textbf{ assi que non sea tomado mayor espaçio } que demanda tal muchedunbre de lidiadores & circa quod sunt munitiones erigendae : \textbf{ ut non accipiatur de spatio ultra quam requirat huiusmodi multitudo , } nec etiam accipietur tam modicus , \\\hline
3.3.8 & que demanda tal muchedunbre de lidiadores \textbf{ nin avn sea tomado tan pequeno espaçio por que la hueste este a mayor estrechura } que deue . & ut non accipiatur de spatio ultra quam requirat huiusmodi multitudo , \textbf{ nec etiam accipietur tam modicus , | ut ultra quam debeat , } oporteat exercitum constringi et constipari . \\\hline
3.3.8 & Et por ende en tal caso deuen se fazer los castiellos \textbf{ en figura de medio cerco o quedrados o de tres rencones } o segunt alguna otra figura & nisi loci situs impediat . \textbf{ Nam contingit aliquando situm illum non pati talem formam . In tali ergo casu construenda sunt castra semicircularia , triangularia , quadrata , } vel aliquam formam aliam \\\hline
3.3.8 & ally son de catar \textbf{ e de escoger mas fuertes guarniçiones } e son de fazer mas anchas carcauas . & aut ibi debet \textbf{ per modicum tempus existere , } non oportet tantas munitiones expetere . Modum autem , \\\hline
3.3.8 & e de escoger mas fuertes guarniçiones \textbf{ e son de fazer mas anchas carcauas . } mas solamente quieren y estar vna noche o por poco tienpo non conuiene de fazer tantas guarniçiones . & per modicum tempus existere , \textbf{ non oportet tantas munitiones expetere . Modum autem , } et quantitatem fossarum tradit Vegetius dicens , \\\hline
3.3.8 & diziendo \textbf{ que si paresciere grant fuerça de los enemigos . } la carcaua deue ser muy ancha de nueue pies & et quantitatem fossarum tradit Vegetius dicens , \textbf{ quod si non immineat magna vis hostium , } fossa debet esse lata pedes nouem , alta septem . \\\hline
3.3.8 & si han uagar para las fazer \textbf{ assi que sea la carcaua ancha de doze pies } e alta de nueue . & contingit fossam ampliorem et altiorem facere ita , \textbf{ ut sit lata pedes duodecim , } et alta nouem . Est tamen aduertendum quod si fossa sit alta pedum nouem , propter terram eiectam supra fossam crescit \\\hline
3.3.8 & Enpero conuiene de saber \textbf{ que si la carcaua fuere fonda de nueue pies echando la tierra a la parte de la hueste fazese la carcaua mas alta de quatro pies } assi que toda la carcaua sera alta de treze pies . & et alta nouem . Est tamen aduertendum quod si fossa sit alta pedum nouem , propter terram eiectam supra fossam crescit \textbf{ quasi pedes quatuor : } ita quod tota fossa alta erit quasi pedes tresdecim : \\\hline
3.3.8 & que si la carcaua fuere fonda de nueue pies echando la tierra a la parte de la hueste fazese la carcaua mas alta de quatro pies \textbf{ assi que toda la carcaua sera alta de treze pies . } Ca deue la carcaua ser delante de parte de los enemigos baxa & quasi pedes quatuor : \textbf{ ita quod tota fossa alta erit quasi pedes tresdecim : } debet enim esse fossa in ante ex parte hostium , \\\hline
3.3.8 & Et en aquella tierra \textbf{ que echan faza dentro deuen fincar grandes palos e grandes maderos } e otras guarniçiones & et terra proiicienda est ad partem intra , \textbf{ ubi est exercitus collocandus . In terra autem illa figendi sunt stipites , et ligna , } et munitiones aliae ; \\\hline
3.3.9 & de los fechos de las batallas \textbf{ es de poner muy grant cautella } e de tomar grand sabiduria . & Ut patet per habita , \textbf{ circa negocia bellica est cautela maxima adhibenda . } Nam quia bellorum casus irremediabiles sunt , \\\hline
3.3.9 & es de poner muy grant cautella \textbf{ e de tomar grand sabiduria . } por que los acaesçimientos de las batallas son su remedio . & circa negocia bellica est cautela maxima adhibenda . \textbf{ Nam quia bellorum casus irremediabiles sunt , } diligenter videnda sunt , \\\hline
3.3.9 & por que los acaesçimientos de las batallas son su remedio . \textbf{ Por ende con grant acuçia deuen ser penssadas todas aquellas cosas } que son menester en la batalla & Nam quia bellorum casus irremediabiles sunt , \textbf{ diligenter videnda sunt , | quaecunque circa bella consideranda existunt , } prius quam pugna publica committatur : \\\hline
3.3.9 & ante que la batalla publica se acometa . \textbf{ Ca meior cosa es non acometer la batalla } que se exponer sin prouision conuenible a auentura e a acaesçimiento . & prius quam pugna publica committatur : \textbf{ melius est enim pugnam non committere , } quam absque debita praeuisione fortunae \\\hline
3.3.9 & que lidian \textbf{ quanto pertenesçe a lo presente son seys cosas de penssar . } Assi commo avn de parte de las ayudas & Ex parte autem virorum pugnantium , \textbf{ quantum ad praesens spectat sex sunt attendenda : } sicut etiam ex parte auxiliorum \\\hline
3.3.9 & para la batalla \textbf{ se pueden contar otras seys cosas } las quales avn son de penssar . & et adminiculantium ad bellum , \textbf{ sex alia enumerari possunt , } quae etiam sunt attendenda . In uniuerso igitur rex , \\\hline
3.3.9 & e entendido deue penssar \textbf{ doze cosas las seys de parte de los omnes } que han de lidiar . & sobrius , prudens , \textbf{ et industris , } duodecim debet considerare : \\\hline
3.3.9 & ante que venga acometer la batalla publicamente . \textbf{ porque son seys cosas de parte de los enemigos lidiadores } que fazen ganar victoria . & et sex ex parte amminiculantium , \textbf{ prius quam eligat publicam pugnam committere . Sunt autem sex ex parte hominum bellatorum , } quae faciunt ad obtinendam victoriam . \\\hline
3.3.9 & la quantia en la batalla es prouechosa \textbf{ assi commo el mayor peso de la ualança trae al menor . } Lo segundo de parte de los que lidian es de penssar & nam ut dicitur 2 Polit’ quantitas in compugnatione est utilis , \textbf{ sicut maius pondus magis trahit . Secundo , } ex parte bellatorum attendenda est exercitatio . \\\hline
3.3.9 & e mas prouadamente \textbf{ e sin mayor trabaio } e pena faze las obras & ø \\\hline
3.3.9 & Lo quarto es de penssar la fortaleza e la dureza del coraçon . \textbf{ Ca grant deferençia es entre la dureza del fierro } e la blandeza del paño del sirgo & Quarto consideranda est fortitudo et durities corporis . \textbf{ Multum enim interest | inter duritiem ferri } et mollitiem panni serici , \\\hline
3.3.9 & e la aspereza de la batalla \textbf{ Ca penssada la batalla en general todos quieren ser buenos lidiadores } mas despues que vienen a la prueua de los fechos particulares & et inter suauitatem ludi et asperitatam pugnae . Considerato enim bello in uniuersali , \textbf{ omnes volunt esse boni bellatores , } sed \\\hline
3.3.9 & e quanto atormentan las llagas de los enemigos \textbf{ por la mayor partida } si fuere el lidiador duro de carne e rezio de cuerpo & et quantus sit labor pugnae , et quantum affligunt vulnera hostium : \textbf{ ut plurimum est durus carne et robustus corpore , } si propter talia non retrahitur a bellando . \\\hline
3.3.9 & para entender e para saber \textbf{ mas en la mayor partida non son apareiados para lidiar . } Por que tales menos sufren el peso de las armas & ( ut supra tangebatur ) sunt aptiores ad intelligendum , \textbf{ sed ut plurimum sunt inepti ad pugnandum : } nam tales difficilius sustinent armorum pondus , \\\hline
3.3.9 & Ca los mas osados \textbf{ e de mayores coraçons } por la mayor parte alcançan en la batalla la victoria . & quia audaciores \textbf{ et magis cordati } ut plurimum in pugna victoriam obtinent . \\\hline
3.3.9 & e de mayores coraçons \textbf{ por la mayor parte alcançan en la batalla la victoria . } Et pues que assi es el rey o el prinçipeo el señor de la hueste & et magis cordati \textbf{ ut plurimum in pugna victoriam obtinent . } Rex igitur \\\hline
3.3.9 & ante que publicamente comiençen a lidiar \textbf{ deue penssar seys cosas de parte de los omnes } que han de lidiar . & priusquam publice dimicet , \textbf{ ex parte hominum bellatorum septem considerare debet . Primo , } ex qua parte sunt plures bellatores . Secundo , \\\hline
3.3.9 & e en viso segunt \textbf{ que viere la su hueste ha conplimiento en estas seys condiçiones } et fallesçe en ellas podra acometer la vatalla & et vigilans \textbf{ prout viderit suum exercitum in his conditionibus abundare , } aut deficere : \\\hline
3.3.9 & e lidiar publicamente o manifiestamente o por assechos \textbf{ e por çeladas e ascondidamente . Contadas las seys condiçiones } que son de penssar ante & et bellare publice et aperte , \textbf{ vel per insidias et latenter . | Enumeratis septem conditionibus , } quae considerandae sunt prius \\\hline
3.3.9 & que acometan publicamente de parte de los ommes lidiadores \textbf{ finca nos de contar otras seys condiçiones } que son tomadas de parte de aquellas cosas & quam committatur bellum publicum ex parte hominum bellatorum : \textbf{ reliquum est enumerare sex alia , } quae sumuntur ex parte amminiculantium \\\hline
3.3.9 & de qual parte son mas caualleros e meiores . \textbf{ Lo segundo de qual parte son meiores balleros } e mas armados & et meliores . \textbf{ Secundo ex qua parte sunt meliores sagittarii , } plures armati , \\\hline
3.3.9 & e mas armados \textbf{ e de qual parte han meiores armas . } Lo terçero de qual parte ay mayor conplimiento de viandas . & plures armati , \textbf{ et habentes meliora arma . Tertio } ubi plus victualia abundant : \\\hline
3.3.9 & e de qual parte han meiores armas . \textbf{ Lo terçero de qual parte ay mayor conplimiento de viandas . } Ca algunas vegadas sin ferida & et habentes meliora arma . Tertio \textbf{ ubi plus victualia abundant : } nam aliquando absque vulnere et absque bello aduersarii cedunt deficientes in victualibus , \\\hline
3.3.9 & Lo vi° es de penssar \textbf{ quales esperan mayores ayudas . } ca si los enemigos esperan mayores ayudas & Sexto est attendendum , \textbf{ qui plures auxiliatores expectant . | Nam } si hostes plura expectant auxilia , vel non est bellandum , \\\hline
3.3.9 & quales esperan mayores ayudas . \textbf{ ca si los enemigos esperan mayores ayudas } o non conuiene de lidiaro & Nam \textbf{ si hostes plura expectant auxilia , vel non est bellandum , } vel acceleranda est pugna . \\\hline
3.3.9 & conuiene de apressurar la batalla . \textbf{ Mas si ellos esperan mayores ayudas deuen alongar la batalla . } Et pues que & vel acceleranda est pugna . \textbf{ Si autem ipsi plures auxiliatores expectant , } est compugnatio differenda . \\\hline
3.3.9 & assi es todas estas cosas vistas \textbf{ con grant sabiduria } el cabdello sabio de la hueste & est compugnatio differenda . \textbf{ His itaque igitur omnibus diligenter inspectis , } prudens dux exercitus sufficienter aduertere potest , \\\hline
3.3.9 & e por auentura nunca contezçra que todas estas condiçones puedan ser de la vna parte . \textbf{ Enpero do mas e meiores condiçiones fueren falladas } aquella parte es la meior para lidiar . & forte enim nunquam contingeret omnes conditiones praefatas concurrere ex una parte : \textbf{ ubi temen plures | et meliores conditiones concurrunt , } est pars potior ad bellandum . \\\hline
3.3.10 & para aquella \textbf{ señalPor ende prouechosa cosa fue } e es en las batallas de leuar pendones e sobreseñales & ut si contingeret aliquem bellatorem deuiare a propria acie , \textbf{ de facili rediret ad illam ; utile ergo fuit in bellis insignia et vexilla deferre , } ne confunderetur exercitus . \\\hline
3.3.10 & que en la hueste establesçiessen cabdiellos e çenturiones \textbf{ que son señores de çient caualleros e deanes } que son señores de diez caualleros & Rursus constituere expediebat duces , \textbf{ centuriones , decanos , } et alios praepositos belli . \\\hline
3.3.10 & assi commo todos los mienbros del cuerpo se ayudan vno a otro . \textbf{ en essa misma manera todos los lidiadores } et todas las partes de la hueste se defienden vno a otro . & quare sicut omnia membra corporis se inuicem iuuant , \textbf{ sic omnes bellatores } et omnes partes eiusdem exercitus se inuicem defendunt . \\\hline
3.3.10 & Et so este cabdiello eran los çenturiones \textbf{ que eran señores de çient caualleros . } Et so el centurion eran los deanes & qui toti exercitui erat praepositus . \textbf{ Sub hoc autem duce erant centuriones . } Sub centurione vero decani . \\\hline
3.3.10 & Et so el centurion eran los deanes \textbf{ e el dean es dicho cabdiello de dies caualleros } assi commo el centurion de çiento . & Sub centurione vero decani . \textbf{ Dicitur enim decanus a decem . } Sic centurio a centum . Habebat enim centurio \\\hline
3.3.10 & de los quales deanes auia cada vno so si diez lidiadores . \textbf{ Et en el yelmo del çenturion eran escerptas letras algunas } o alguna señal manifiesta . & sub se decem viros pugnatiuos habebat . \textbf{ In galea enim centurionis scriptae erant literae aliquae , } vel signum aliquod euidens ; \\\hline
3.3.10 & o a qual dean acometer . \textbf{ Avn en essa misma manera en el yelmo de cada dean } era alguna señal proprea puesta & quem sequi debebant . Sic \textbf{ etiam in galea cuiuslibet | decanorum } signum aliquod erat impressum , per quod decem bellatores viri , \\\hline
3.3.10 & Et pues que \textbf{ assi es con grant sabiduria es de escoger el alferez } assi que sea fuerte de cuerpo & Nam vexillo confracto totus exercitus est confusus . \textbf{ Cum magna igitur diligentia est vexillifer eligendus , } ut sit corpore fortis , \\\hline
3.3.10 & que son menester a bueno e a estremado lidiador \textbf{ por que avn contesçio en el nr̄o tienpo } que todo el pueblo de la çibdat fue vençido de pocos lidiadores . & et alia singula habeat , \textbf{ quae requiruntur ad probum } et strenuum bellatorem . Contigit enim nostris temporibus totum populum ciuitatis cuiusdam deuictum esse a bellatoribus paucis , \\\hline
3.3.10 & e el estado de muchos omnes es puesta \textbf{ a periglos de muerte con grant acuçia } e con grant diligençia deue ser escogido el alferes . & et ignorantes ad quid deberent attendere : propter quod si in debellatione vita multorum hominum periculis mortis exponitur , \textbf{ cum magna diligentia vexillifer est quaerendus . } Ex dictis etiam patere potest , \\\hline
3.3.10 & a periglos de muerte con grant acuçia \textbf{ e con grant diligençia deue ser escogido el alferes . } de las cosas sobredichas puede paresçer & et ignorantes ad quid deberent attendere : propter quod si in debellatione vita multorum hominum periculis mortis exponitur , \textbf{ cum magna diligentia vexillifer est quaerendus . } Ex dictis etiam patere potest , \\\hline
3.3.10 & por la qual cosa commo los peones \textbf{ si quisieren ser buenos lidiadores } deuan ser fuertes & Quare cum pedites , \textbf{ si debent boni bellatores existere debeant esse fortes viribus , } proceri statura , scientes proiicere hastas \\\hline
3.3.10 & e avn que ayan vso de las armas . \textbf{ Et si todas estas cosas deuen ser falladas en los buenos lidiadores } mucho & etiam debeant gladium vibrare ad percutiendum , portare scutum ad se protegendum : \textbf{ et cum debeant esse vigilantes , } agiles , sobrii , habentes armorum experientiam : \\\hline
3.3.10 & e para alinpiar las armas \textbf{ por que el resplandesçimiento de las armas pone grant espanto a los enemigos } assi que el que traye tales armas es tenido por buen lidiador & et cogat eos ad bene debellandum , et ad arma tergendum . \textbf{ Nam ipse armorum nitor terrorem incutit hostibus , ut portans huiusmodi arma credatur } bonus esse bellator . Ipsa enim rubigo armorum in eo qui portat illa , \\\hline
3.3.10 & por que el resplandesçimiento de las armas pone grant espanto a los enemigos \textbf{ assi que el que traye tales armas es tenido por buen lidiador } por que la ferrunbre e el oryn de las armas muestra pareza de lidiar & Nam ipse armorum nitor terrorem incutit hostibus , ut portans huiusmodi arma credatur \textbf{ bonus esse bellator . Ipsa enim rubigo armorum in eo qui portat illa , } arguit inertiam bellandi . \\\hline
3.3.10 & aquel que deue ser ante puesto a los caualleros . \textbf{ Por que en la batalla de los caualleros se faze mayor pelea que en la batalla de los peones . } Et pues que assi es conuiene & qui est equitibus praeponendus : \textbf{ quia in bello equestri maior conflictus efficitur , | quam in pedestri pugna . } Oportet igitur praepositum et ducem militaris belli esse habilem corpore , \\\hline
3.3.11 & e las qualidades de las carreras \textbf{ e los fuertes passos de los caminos } e los departimientos de las carreras & et qualitates viarum , \textbf{ compendia et diuerticula , } et montes , \\\hline
3.3.11 & por los assechos de los enemigos la hueste sea puesta \textbf{ a tantos e a mayores periglos en el camino } que los marineros en la mar . & Quare propter insidias hostium exercitus tot quasi , \textbf{ vel etiam pluribus periculis exponatur in via quam nautae in mari , } nullo modo \\\hline
3.3.11 & que ouiere de fazer . \textbf{ Ca do puede contesçer tan grant periglo } ninguno non deue esforçar se en su cabeça propria & quicquid viderit ipse dux belli esse fiendum . \textbf{ Nam ubi tantum currit periculum , } nullus debet inniti proprio capiti , \\\hline
3.3.11 & e aquellas carreras touiere el cabdiello escriptas e pintadas \textbf{ e ouiere algunos omes guiadores fieles } quanto esto menos fuere publico & et vias illas Dux habet conscriptas et depictas , \textbf{ et habentur conductores aliqui fideles , } quanto hoc minus est publicum \\\hline
3.3.11 & que sienpre en aquella parte de la hueste \textbf{ deuen ser puestos los meiores caualleros } e los meiores peones & Sexta est , \textbf{ ut semper ex illa parte exercitus probiores milites } et magis bellicosi pedites apponantur , \\\hline
3.3.11 & deuen ser puestos los meiores caualleros \textbf{ e los meiores peones } en la qual cuydaren & ut semper ex illa parte exercitus probiores milites \textbf{ et magis bellicosi pedites apponantur , } ex qua creditur maius periculum imminere : \\\hline
3.3.11 & en la qual cuydaren \textbf{ que puede venir mayor perigso . } Et sy por auentura dubdan de periglo de cada parte & et magis bellicosi pedites apponantur , \textbf{ ex qua creditur maius periculum imminere : } quod si ex omni parte de periculo dubitatur , \\\hline
3.3.11 & Otrossi deue ser el señor de la hueste \textbf{ e el que es señor de çient caualleros } e de diez deanes & quasi negligentem \textbf{ et dormientem . Debet etiam dux exercitus centuriones , } et decani , \\\hline
3.3.11 & La . viij° . cautela es penssar \textbf{ de quales ha mayor conplimiento la hueste de peones o de caualleros . } ca los caualleros meior se defienden en los canpos . & Octaua cautela est , \textbf{ considerare exercitum in quibus sit copiosior , | utrum magis abundet peditibus , vel equitibus . } Nam equites melius se defendunt in campis . \\\hline
3.3.12 & so qual arte se contiene la obra de la batalla \textbf{ e de quales tierras son los meiores lidiadores } e de quales artes son de escoger los mayores lidiadores . & Postquam diximus sub quo continetur operatio bellica , \textbf{ et ex quibus regionibus sunt meliores pugnantes } et est \\\hline
3.3.12 & e de quales tierras son los meiores lidiadores \textbf{ e de quales artes son de escoger los mayores lidiadores . } Et avn declaramos en qual manera en la hueste son de establesçer guarniçiones e castiellos & et ex quibus regionibus sunt meliores pugnantes \textbf{ et est | quibus artibus sunt meliores bellicosi : } declarauimus \\\hline
3.3.12 & Enpero primero diremos del ordenamiento de las azes . \textbf{ Ca si fuere guardado el buen ordenamiento en la az mucho vale para la batalla . } Ca assi commo dize vegeçio & tamen dicemus de ordine acierum . \textbf{ Si enim ordo seruetur in ipsa acie , non modicum valet ad pugnam . } Nam , \\\hline
3.3.12 & non se puede fazer \textbf{ sin grant vso de las armas . } Pues que assi es aquel que quiere lidiaren algun tienpo deue & et pedites suam aciem seruent , \textbf{ non sine magno exercitio fieri potest . } Qui igitur in tempore aliquo vult bellare , \\\hline
3.3.12 & Pues que assi es aquel que quiere lidiaren algun tienpo deue \textbf{ por luengos tienpos acostunbrar los lidiadores } a guardar orden conuenible en la az & Qui igitur in tempore aliquo vult bellare , \textbf{ per diuturna tempora debet exercitare pugnatores ad seruandum debitum ordinem , et } ad faciendum ea quae requiruntur in bello . Modus autem , \\\hline
3.3.12 & o quieren acometer los otros . \textbf{ Por ende si los lidiadores non se sienten de tan grand poder } por que los otros puedan vençer . & vel volunt alios inuadere . \textbf{ Si ergo bellantes non sentiunt se tantae potentiae } ut alios debellare possint , \\\hline
3.3.12 & los quales puedan sofrir meior los colpes \textbf{ e con menor agrauiamiento e con menor daño . } Mas si los lidiadores cuydan ser de tanto poder & et melius armati , \textbf{ qui absque minori grauamine possint ictus suscipere . } Si vero pugnantes credunt se esse tantae potentiae , \\\hline
3.3.12 & e en los logares \textbf{ do puede ser mayor periglo son de poner los meiores lidiadores } que meior puedan lidiar & et in locis \textbf{ ubi maius periculum est , | ne acies confundatur , apponendi sunt probiores pugnatores , } qui possint virilius dimicare . Est \\\hline
3.3.12 & sin el cuento de los lidiadores \textbf{ que fazen el az son de guardar algunos buenos et fuertes lidiadores fuera del az } que puedan acorrer a aquella parte & etiam aduertendum , \textbf{ quod in qualibet acie praeter numerum pugnatorum constituentium aciem , reseruandi sunt aliqui strenui bellatores extra ipsam aciem qui possint } ad illam partem succurrere \\\hline
3.3.12 & asi commo demanda la manera de la batalla . \textbf{ Lo segundo que los mas fuertes lidiadores sean puestos en aquellas partes de la az } en las quales mas ayna se puede ronper & ut requirit bellum committendum . Secundo , \textbf{ ut probiores bellatores in illis partibus aciei apponantur , } in quibus magis potest confundi et perforari acies . \\\hline
3.3.13 & Ca el colpe mas ayna viene a la carne \textbf{ por que pequeno cortamiento de las armas abasta } para ferir en la carne feriendo de punta & ut ergo vulnus perueniat citius ad carnem , magis est eligibile percutere punctim , \textbf{ quam caesim . Modica autem armorum incisio sufficit ad laedendum carnem percutiendo punctim , } quae non sufficeret \\\hline
3.3.13 & o a los mienbros de vida \textbf{ conuernie de fazer muy grant llaga } e de cortar muchos huessos . & ad cor vel ad membra vitalia , \textbf{ oporteret magnam plagam facere } et multa ossa incidere : \\\hline
3.3.13 & e de cortar muchos huessos . \textbf{ Mas feriendo de punta pequeno colpe mata al omen . } ca dos onças de sangre abastan & et multa ossa incidere : \textbf{ sed percutiendo punctim duae unciae sufficiunt } ad hoc ut fiat plaga mortalis , \\\hline
3.3.13 & Mas en feriendo cortando . \textbf{ por que conuiene de fazer grand mouimiento de los braços } ante que se de el colpe & In percutiendo autem caesim , \textbf{ quia oportet fieri magnum brachiorum motum prius quam infligatur plaga , aduersarius ex longinquo potest prouidere vulnus , } ideo magis sibi cauere potest \\\hline
3.3.13 & que son las batalla mayormente es de penssar esto \textbf{ que los lidiadores sin grand canssamiento de sus mienbros puedan ferir mucho a sus enemigos e a sus contrarios . } Ca si los lidiadores canssaren mucho de guisa & Inter cetera enim in bellis est hoc potissime attendendum : \textbf{ ut pugnantes absque nimia fatigatione sui possint nimis aduersarios laedere . } Nam si bellantes nimis se fatigent , \\\hline
3.3.13 & Por la qual cosa commo feriendo cortando \textbf{ por el grand mouimiento de los braços leuantasse ende grant trabaio . } Mas feriendo de punta el canssamiento es muy pequeno . Por ende es meior de ferir de punta que cortando & et conuertuntur in fugam . \textbf{ Quare cum percutiendo caesim propter magnum motum brachiorum insurgat ibi magnus labor , } punctim uero feriendo modica fatigatio sufficiat , elegibilius est percutere punctim , \\\hline
3.3.13 & por que la ferida de taio \textbf{ por grant fuerça } que venga tarde mata . & quam caesim . Caesa enim percussio \textbf{ quouis impetu veniat raro occidit . } Sed puncta modico impetu inflicta , facit lethale vulnus . \\\hline
3.3.13 & que venga tarde mata . \textbf{ Mas la ferida de punta fecha con muy pequeña fuerca faze llaga mortal . } La quinta razon se toma & quouis impetu veniat raro occidit . \textbf{ Sed puncta modico impetu inflicta , facit lethale vulnus . } Quinta via sumitur \\\hline
3.3.13 & del descrubimiento del que fiere . \textbf{ Ca el buen lidiador } si puede deue ferir assu enemigo & ex detectione percutientis . \textbf{ Nam bonus bellator , } si potest , \\\hline
3.3.14 & menos se pueden defender \textbf{ e con mayor trabaio . } Ca conuieneles que anden esparzidos . & si contingat hostes in tali situ reperiri , \textbf{ difficilius se defendere poterunt : } quia oportet eos sparsim incedere . Quare sicut locus ineptus defensioni , \\\hline
3.3.14 & e los rayos del sol les da en los oios . \textbf{ con mayor trabaio se pueden defender de sus enemigos . } Mas en el tienpo & et in quo solares radii opponuntur eorum oculis , \textbf{ difficilius possunt hostes resistere : } tempore vero in quo haec modo opposito se habent , \\\hline
3.3.14 & e mucho mas si son departidos en los coraçones e en las uoluntades mas ayna seran vençidos \textbf{ ca mayor departimiento es el de los coracones } que el de los cuerpos . & multo magis diuisi animo et voluntate debellantur celerius , \textbf{ quia maior est diuisio animorum quam corporum . Et e contrario , } si hostes non sunt sparsi , sed sunt corporaliter coniuncti , \\\hline
3.3.14 & e por çeladas \textbf{ o por alguna otra manera deue tener mientes con grant acuçia } quando los enemigos estan desparzidos & Primo igitur dux belli per insidias vel propter aliquem alium modum , \textbf{ debet diligenter aduertere , } quando hostes sunt dispersi : \\\hline
3.3.14 & por que non han poder de se defender . \textbf{ Lo segundo deue escudriñar con grand acuçia los caminos dellos } assi commo el passo de los rios & et tunc debet eos inuadere , \textbf{ quia non habebunt potentiam resistendi . Secundo debet diligenter explorare eorum itinera , } ut ad transitus fluuiorum , \\\hline
3.3.14 & ally den ellos sin sospecha . \textbf{ Lo quinto deue el cabdiello escodriñar con grant acuçia } quando los enemigos fizieren grant iornada & non suspicantes eorum aduentum . \textbf{ Quinto debet diligenter explorare , } quando hostes magnam fecerunt dietam , sunt fatigati habent laxatos equos : \\\hline
3.3.14 & Lo quinto deue el cabdiello escodriñar con grant acuçia \textbf{ quando los enemigos fizieren grant iornada } e touieren los cauallos canssados . & Quinto debet diligenter explorare , \textbf{ quando hostes magnam fecerunt dietam , sunt fatigati habent laxatos equos : } tunc enim , \\\hline
3.3.14 & non es mucho de preçiar \textbf{ por que es contraria a buenas costunbres . } lo . vij° . deue el cabdiello con grant acuçia escudriñar las condiçiones de los sus enemigos & non multum est appretianda : \textbf{ quia repugnaret bonis moribus . Septimo debet diligenter explorare conditiones hostium : } qualiter se gerant , \\\hline
3.3.14 & por que es contraria a buenas costunbres . \textbf{ lo . vij° . deue el cabdiello con grant acuçia escudriñar las condiçiones de los sus enemigos } en qual manera andan & non multum est appretianda : \textbf{ quia repugnaret bonis moribus . Septimo debet diligenter explorare conditiones hostium : } qualiter se gerant , \\\hline
3.3.15 & en el qual mostramos a los caualleros e a los peones acometer la batalla . \textbf{ Mas en este capitulo queremos dar espeçial enssenança a los peones } en qual manera deuen estar & et etiam pedites . \textbf{ In hoc autem capitulo specialem doctrinam volumus dare peditibus , } qualiter debeant stare cum volunt hostes percutere . \\\hline
3.3.15 & mas reziamente \textbf{ e faz mas fuerte colpe . } Enpero maguer que podamos folgar tan bien sobre la parte derecha & quo vibrato vehementius mouet aerem , \textbf{ et fortius ferit . Licet enim tam } secundum partem dextram quam secundum sinistram possumus quiescere \\\hline
3.3.15 & assi el pie esquierto firme \textbf{ e mouiendo se con el esquierdo pie podrian mas fuertemente ferir los enemigos e mas ligeramente foyr los colpes dellos . } Visto commo deuen estar los lidiadores & et cum dextro se mouendo , \textbf{ poterunt fortius hostes percutere , | et eorum ictus facilius fugere . } Viso quomodo debeant stare bellantes , \\\hline
3.3.15 & quando dixiemos dessuso \textbf{ que alguans vezes el az es de formar so forma de tiieras } e esto quando los enemigos son pocos & quasi coacti feriunt includentes . \textbf{ Cum ergo supra diximus , formandam aliquando esse aciem sub forma forficulari , } ut quando hostes pauci , \\\hline
3.3.16 & que los sus contrarios los ayan a fazer foyr . \textbf{ p paresçe que todas e las batallas se puenden adozir a quatro maneras . } las quales son estas . & si fugaretur ab hostibus . \textbf{ Videntur omnia bella ad quatuor genera reduci , } videlicet ad campestre , \\\hline
3.3.16 & Ca assi commo contesçe \textbf{ que algunos lidiadores son en tan grand muchedunbre } e de tan grant poder & quod defensiuum vocari potest . \textbf{ Nam sicut contingit pugnantes aliquos in tanta multitudine esse , } et tantam habere potentiam , \\\hline
3.3.16 & que algunos lidiadores son en tan grand muchedunbre \textbf{ e de tan grant poder } que no esperan que salgan los enemigos al canpo . & Nam sicut contingit pugnantes aliquos in tanta multitudine esse , \textbf{ et tantam habere potentiam , } ut non expectant hostes exire ad campum , \\\hline
3.3.16 & que sean aquellas aguas son dichas batallas nauales e de naues . \textbf{ Por la qual cosa commo sean quatro maneras da batallas } despues & quam terrestres . Huiusmodi autem pugna in aquis facta cuiuscunque conditionis aquae illae existant , nauales dicuntur . \textbf{ Quare cum sint quatuor genera pugnarum , } postquam diximus de campestri , \\\hline
3.3.16 & Otrossi contesçe \textbf{ que en el prinçipado e en el regno ay puertos e tierras marinas assentadas çerca de la mar . } Por la qual cosa & etiam aliquando aliquos inuadere aliquas munitiones eorum ; \textbf{ propter quod eos oportet uti pugna defensiua . Amplius in principatu et regno contingit esse portus et terras maritimas iuxta mare sitas : } propter quod ne portus destruantur , \\\hline
3.3.16 & o de morir de sedo de dar las fortalezas . \textbf{ Por la qual cosa con grant acuçia deuen cuydar } los que cercan algunas fortalezas & vel munitiones reddere . \textbf{ Quare diligenter excogitare debent obsidentes munitiones aliquas , } utrum per aliqua ingenia , \\\hline
3.3.16 & por la qual cosa si el agua fuere lueñe de la fortaleza \textbf{ los que cercan deuen auer grant acuçia } en commo defiendan el agua a los cercado & quare si sit a munitionibus remota , \textbf{ debent obsidentes adhibere omnem diligentiam , } quomodo possint obsessis prohibere aquam . Secundus modus impugnandi munitiones , est per famem . \\\hline
3.3.16 & por que ganen mas fortalezas \textbf{ deuen guardar con grand acuçia todos los passos e los caminos e los logares } por do pueden venir viandas a los cercados & ut munitiones obtineant , passus , vias \textbf{ et omnia loca per quae possent obsessis victualia deferri , | diligenter custodire debent , } ne eis talia deferantur . \\\hline
3.3.16 & e despues enbianlos a las fortalezas cercadas \textbf{ por que y con los otros comedores cercados fagan mayor fanbre } e mayor desfallesçimiento en viandas . & et postea illos remittunt ad munitiones obsessas , \textbf{ ut ibi una cum aliis comedentes } apud ipsos obsessos maiorem famem \\\hline
3.3.16 & por que y con los otros comedores cercados fagan mayor fanbre \textbf{ e mayor desfallesçimiento en viandas . } La terçera manera de ganar las fortalezas es por batalla & ut ibi una cum aliis comedentes \textbf{ apud ipsos obsessos maiorem famem } et inopiam inducant . Tertius modus obtinendi munitiones est per pugnam : \\\hline
3.3.16 & por que entonçe \textbf{ mas se dessecan las aguas nin las luuias del çielo non son de grant abondança } por que se puedan mantener los cercados & eo \textbf{ quod tunc magis desiccantur aquae , | nec si abundant pluuiae caelestes , } ut possit per cisternas subuenire obsessis . Rursus si per famen est castrum , \\\hline
3.3.16 & Ca en el tienpo del yuierno \textbf{ ay grand anbondança de luuias } e finchense las carcauas de aguas . & est hoc agere aestiuo tempore . \textbf{ Nam tempore hyemali abundant pluuiae , } replentur fossae aquis : \\\hline
3.3.16 & e finchense las carcauas de aguas . \textbf{ Por la qual cosa con muy grand graueza } e con grand trabaio se pueden acometer & replentur fossae aquis : \textbf{ quare difficilius impugnantur obsessi . Rursus incommoditates temporum magis molestant obsidentes } et existents in campis , \\\hline
3.3.16 & Por la qual cosa con muy grand graueza \textbf{ e con grand trabaio se pueden acometer } los que estan cercados . & replentur fossae aquis : \textbf{ quare difficilius impugnantur obsessi . Rursus incommoditates temporum magis molestant obsidentes } et existents in campis , \\\hline
3.3.16 & los que estan cercados . \textbf{ Otrossi los malos tienpos } mas atormentan a los que çercan & et existents in campis , \textbf{ quam obsessos manentes in domibus . Vel igitur } obsessiones fiendae sunt tempore aestiuo , \\\hline
3.3.17 & s si los que çercan fueren negligentes \textbf{ e non se guarnesçieren con grand acuçia } pueden resçebir daño de los que estan çercados & Si obsidentes negligentes fuerint , \textbf{ et non diligenter se muniant , } ab obsessis molestari poterunt . \\\hline
3.3.17 & quanto podrie lançar la vallesta o el dardo \textbf{ e fazer carcauas enderredor de ssi e finçar y grandes palos } e fazer algunas fortalezas & longe a munitione obsessa saltem per ictum teli vel iaculi debent castrametari , et circa se facere fossas , \textbf{ et figere ibi ligna , } et construere propugnacula : \\\hline
3.3.17 & Et la otra es por algarradas e por engeñios \textbf{ que lançan grandes piedras e muy pesadas } Et la terçera es por castiellos e por gatas & Quorum unus est per cuniculos . \textbf{ Alius est per machinas proiicientes lapides magnos et graues . } Et tertius per aedificia impulsa usque \\\hline
3.3.17 & e tomar las fortalezas . \textbf{ Ca deuen los que çercan en grand poridat cauar la tierra en algun logar conenible } ante el qual logar deuen poner algunan tienda o alguna choça & id est per vias subterraneas deuincuntur munitiones . \textbf{ Debent enim obsidentes priuatim in aliquo loco terram fodere : | ante quem locum , } tentorium \\\hline
3.3.17 & Et assi deuen yr fasta los muros de aquel logar . \textbf{ Et si esto se puede fazer ligera cosa es de tomar aquella fortaleza o aquel logar } ca esto fecho primero deuen socauar los muros & quod \textbf{ si hoc fieri potest , leue est munitionem capere . } Nam hoc facto primo debent muros fodere , \\\hline
3.3.17 & a los que cauan \textbf{ e quando todos los muros o grant parte dellos } assi ouieren socauados & Et cum omnes muros , \textbf{ vel maximam partem murorum sic suffosserunt } et subpunctauerunt , \\\hline
3.3.17 & e fazer \textbf{ que todos los muros o grand parte dellos cayan en vno a desora . } Otrossi deuen fenchir las carcauas & et facere omnes muros \textbf{ vel facere magnam eorum partem cadere , } et replere fossas : \\\hline
3.3.17 & e sopuestos non deuen luego poner el fuego mas deuen yr so tierra mas adelante \textbf{ a las mayores fortalezas } e a los mas fuertes adarues del castielloo de la çibdat çercada . & nondum apponendus est ignis , \textbf{ sed procedendum est } ad maiores munitiones et ad maiora moenia castri , vel ciuitatis obsessae , \\\hline
3.3.17 & a las mayores fortalezas \textbf{ e a los mas fuertes adarues del castielloo de la çibdat çercada . } Et por cueuas deuen venir & sed procedendum est \textbf{ ad maiores munitiones et ad maiora moenia castri , vel ciuitatis obsessae , } et per similes vias subterraneas est similiter faciendum circa ea , \\\hline
3.3.17 & nin lo sientan los cercados . \textbf{ Et maguera que todas estas cosas non se puedan fazer sin muy grant graueza } e sin luengo tienpo . & quae omnia latenter fieri possunt absque eo quod sentiantur ab obsessis : \textbf{ licet tamen sine difficultate } et diuturnitate temporis , \\\hline
3.3.17 & Et maguera que todas estas cosas non se puedan fazer sin muy grant graueza \textbf{ e sin luengo tienpo . } Empero e deuen las prouarlos omnes . & licet tamen sine difficultate \textbf{ et diuturnitate temporis , } possint \\\hline
3.3.18 & assi que por cueuas conegeras \textbf{ o por cueuas soterranas nunca o con muy grand trabaio se pueden tomar . } Et avn acaesçe muchas vezes & ut per viculos \textbf{ et per subterraneas vias nunquam , | vel valde de difficili obtineri possint . Euenit } etiam pluries \\\hline
3.3.18 & la qual natura sienpre aduze \textbf{ por el mas ligero camino que puede las cosas a su fin . } Commo por cueuas conegeras non puedan tan de ligero tomar las fortalezas & Quare si modus artis debet imitari naturam quae semper faciliori via res ad effectum producit : \textbf{ cum per viculos non ita de facili munitiones impugnari possunt , } sicut per machinas lapidarias , \\\hline
3.3.18 & Mas los engeñios \textbf{ que lançan las piedras puedense adozer a quatro maneras . } Ca en todo tal engeñio es de dar alguna cosa & Machinae autem lapidariae \textbf{ quasi ad quatuor genera reducuntur . } Nam in omni tali machina est dare aliquid trahens \\\hline
3.3.18 & e tornasse çerca del pertegal . \textbf{ Et esta manera de engeñio llaman los lidiadores romanos bifan } Mas este engeñio ha departimiento del trabuquete . & vertens se circa huiusmodi virgam . \textbf{ Et hoc genus machinae Romani pugnatores appellauerunt Biffam . } Differt autem haec a Trabutio . \\\hline
3.3.18 & si la quiere tomar \textbf{ por engeñios de piedras deue penssar con grand acuçia } si puede meior conbatir aquella fortaleza lançando derechamente o lançando & aut ciuitatem aliquam , \textbf{ si vult eam impugnare per machinas lapidarias , } diligenter considerare debet , \\\hline
3.3.18 & en qual manera fiere el engeñio \textbf{ e qual o quant pesada piedra se deue poner en la fonda del engeñio . } t tres maneras de conbatir las fortalezas cercadas fueron puestas dessuso de las quales la vna era por cueuas conegeras & Nam per ticionem ignitum lapidi alligatum apparere poterit qualiter machina proiicit , \textbf{ et qualis siue quam ponderosus lapis est in funda machinae imponendus . } Tangebatur autem supra tres modi impugnandi munitiones obsessas . \\\hline
3.3.19 & e allegados a los muros del castiello o de la çibdat . \textbf{ Et estos artifiçios pueden se adozir a quatro maneras . } Conuiene de saber a carneros . & vel ad moenia castri , vel ciuitatis obsessae . \textbf{ Huiusmodi autem aedificia quasi ad quatuor genera reducuntur , } videlicet , ad arietes , vineas , turres , et musculos . \\\hline
3.3.19 & e fuertemente atada e texida \textbf{ por que la non quebranten con piedras ponen de dentro vna grand viga } que ha la cabeça enuestida de fierro . & et fortiter contexta , \textbf{ ne lapidibus obruatur intrinsecus ponitur trabs , } cuius caput ferro vestitur , \\\hline
3.3.19 & et muy . dura fruente para ferir \textbf{ e fazer grant colpe . } Et esta viga a tal atanla con cuerdas e con cadenas de fierro a la bouada fecha de madera & Huiusmodi autem trabs funibus , \textbf{ vel cathenis ferreis alligatur ad testudinem factam ex lignis , } et ad modum arietis se subtrahit : \\\hline
3.3.19 & e a manera de carnero se tira atras . \textbf{ Et despues da muy fuerte colpe en los muros de la fortaleza çercada } assi que los ronpen et los quebrantan . & et ad modum arietis se subtrahit : \textbf{ et postea fortiter muros munitionis obsessae percutit } et disrumpit . Cum enim per huiusmodi trabem sic ferratam multis ictibus percussus est murus ita , \\\hline
3.3.19 & Lo primero por sonbra \textbf{ ca vn filo liuiano de çierta medida de palmos o de pies es de atar a la saeta } e deuen lançar la saeta & Primo per umbram . \textbf{ Nam leue filum , | cuius nota sit quantitas , ligandum est ad sagittam , } et proiiciendum usque ad muros munitionis \\\hline
3.3.19 & segunt la quantidat de aquella sonbra . \textbf{ Et este madero fara tan grand sonbra en aquel alto commo es el muro } e segund la alteza del madero se da el alteza del muro . & erigendum est aliquod lignum in altum , faciens tantam umbram ; \textbf{ et } secundum altitudinem illius ligni erit altitudo murorum . Verum quia non semper sol splendet \\\hline
3.3.19 & mas con la tabla en tal manera \textbf{ que catando por ençima de la tabla vea la mas alta parte del muro } ca puedese prouar & elonget se ab aedificio praedicto , \textbf{ donec per summitatem tabulae punctaliter viduat summitatem eius . | Nam , } ut probari potest geometrice , \\\hline
3.3.20 & son terrados o torres albarranas \textbf{ o muros çiegos fechos de tierra . } Ca en la fortaleza & Tertium , quod reddit munitionem difficiliorem ad capiendum , \textbf{ dicuntur esse terrata , } vel muri ex terra facti . \\\hline
3.3.20 & puede resçebir los colpes de las piedras de los engeñios \textbf{ sin grand danno fuyo . } Ca quando la piedra del engennio firiere en el muro de tierra . & murus constitutus ex terra quasi absque laesione susciperet ictus machinarum : \textbf{ quia cum lapis eiectus a machina perueniret ad huiusmodi murum , } propter mollitiem eius cederet terra , \\\hline
3.3.21 & todas son de quemar por fuego . \textbf{ por que los cercadores quanda vinieren a cercar } non se puedan aprouechar ellas & ø \\\hline
3.3.21 & que tedos los otros granos . \textbf{ Avn basteçer de grand conplimiento de carnes saladas } e de mucha sal . & inter cetera minus putrefit , \textbf{ et plus durare perhibetur . Copia | etiam carnium salitarum non est praetermittenda . Salis } etiam multitudo multum est expediens munitioni obsessae , \\\hline
3.3.21 & quanto a la vianda \textbf{ que sea y traydo grand conplimiento de viandas a la fortaleza } que teme de ser çercada . & quantum ad victum non solum attendendum est , \textbf{ ut magna copia victualium deferatur } ad munitionem obsessam , \\\hline
3.3.21 & por las conpañas \textbf{ e por sabios despensseros . } Onde si pudiere ser & ad munitionem obsessam , \textbf{ sed etiam ut victualia delata per temperatos erogatores per familias dispensentur . } Unde ( si fieri posset ) \\\hline
3.3.21 & e partidas tenpradamente e escassamente \textbf{ por muy sabios despensseros . } Et si la fortaleza çercada es pequena & in qualibet contrata ciuitatis victualia reduci debent ad horrea publica , \textbf{ et parte , et temperate per viros prouidos dispensare . } quod si munitio obsessa modici esset ambitus , hoc efficere non est difficile \\\hline
3.3.21 & Et si la fortaleza çercada es pequena \textbf{ non es graue cosa de fazer esto . } Ca non aprouecha nada traer muchas viandas & et parte , et temperate per viros prouidos dispensare . \textbf{ quod si munitio obsessa modici esset ambitus , hoc efficere non est difficile } quasi enim \\\hline
3.3.21 & Et avn en tal caso \textbf{ do temen de grand mengua muchas bestias son de comer que otramiente non son de comer } nin es vso de las comer . & si esui aptae sunt : \textbf{ immo in tali casu comedenda sunt multa , } quae ad esum vetat communis usus . \\\hline
3.3.21 & que en tal fortaleza se ençierren \textbf{ en que aya grand conplimiento de aguas } e si y non ouieren fuentes deuen fazer pozos . & quod ad talem munitionem pergant , \textbf{ in qua sit aquarum copia : } quod si vero ibi non sint fontes , \\\hline
3.3.21 & e de acarrear vinagre \textbf{ e vino en grand abondança a la fortaleza } que teme ser cercada & totum in dulce conuertitur . Deferendum est \textbf{ etiam ad munitionem obsidendem in magna copia acerum , } et vinum , \\\hline
3.3.21 & piedra sufre pez e oleo \textbf{ e rasina en grand anbondança } para quemar los engeñios . & Debent ergo ad ciuitatem , \textbf{ vel ad castrum obsessum deportari in magna copia sulphur , pix , } oleum ad comburendum machinas hostium . Ferra autem \\\hline
3.3.21 & que teme de ser cercada mucha llena \textbf{ e mucho fierro e en grant abondaça por que les non fallezca } assi que de la madera puedan fazer astas & oleum ad comburendum machinas hostium . Ferra autem \textbf{ et ligna sunt ad munitionem obsessam in debita abundantia deportanda , } ut per ligna hastae sagittarum , \\\hline
3.3.21 & que se sigue . \textbf{ Et avn guyias e piedras muchas son de traer a la fortaleza en grand conplimiento } porque tales cosas son mas rezias et meiores para lançar . & ut in sequenti capitulo apparebit . \textbf{ Saxa etiam torrentium in magna copia sunt ad munitionem deportanda : } quia talia sunt solidiora , \\\hline
3.3.21 & Et avn es menester mucha cal fecha poluo \textbf{ e conuiene de la traer en grand abondança } donde quier que la podieren fazer a la fortaleza . & Ex eis ergo replendi sunt muri , \textbf{ et turres munitionis obsessae . Calcem etiam puluerizatam deferendum est ad ipsam munitionem in magna abundantia , } et ex ea replenda sunt multa vasa ; \\\hline
3.3.21 & nin a quien han de ferir . \textbf{ Et avn es meester grand conplimiento de neruios } e de sogas de cañamo en la fortaleza & ut quasi caeci , \textbf{ et non videntes percuti possint . Neruorum etiam copia , } et funium utilis est munitioni obsessae , \\\hline
3.3.21 & Ca dize vegeçio \textbf{ que mas quisieron aquellas buenas mugeres muy castas beuir con sus maridos trasquiladas } que non yr con sus enemigos con cabellos . & ut ait Vegetius ) illae pudicissimae foeminae \textbf{ cum maritis conuiuere deformato capite , } quam seruire hostibus integris crinibus . Sunt \\\hline
3.3.22 & por la qual cosa temen del cometemiento \textbf{ por las cueuas conegeras deuen penssar con grand acuçia los cercados } si pudieren veer que llegan la tierra de alguna parte & propter quod timetur de impugnatione per cuniculos ; \textbf{ diligenter considerare debent obsessi , } utrum ab aliqua parte videant terram deferri , \\\hline
3.3.22 & que estan las cueuas coneieras . \textbf{ Ca muchos de aquellos que çercaron en el nuestro tienpo resçibieron grand peligro desto . } Por la qual cosa & quo facto totam aquam \textbf{ aut urinam congregatam effundere debent supra obsidentes existentes in cuniculis . Temporibus enim nostris multi obsidentium sic periclitati sunt : } quare \\\hline
3.3.22 & por los engenios que lançan las piedras . \textbf{ Et podemos dar contra los engeñios quatro maneras de acorro . } La primera es & restat videre quomodo obsessi debeant obuiare impugnationi factae per lapidarias machinas . \textbf{ Contra has autem quadrupliciter subuenitur . Primo , } quia aliquando subito ex munitione obsessa exiuit magna multitudo armatorum , \\\hline
3.3.22 & La primera es \textbf{ que algunas vegadas salga a desora de la fortaleza cercada grand muchedunbre de omnes armados } e acometan al engeñio & Contra has autem quadrupliciter subuenitur . Primo , \textbf{ quia aliquando subito ex munitione obsessa exiuit magna multitudo armatorum , } et inuadunt machinam ; \\\hline
3.3.22 & Et esto fecho los de suso resçiban los por cuerdas a la fortaleza . \textbf{ La iij° manera de destroyr los engeñios } e las algarradas es fazer saetas & machinam incendunt : \textbf{ quo peracto trahuntur superius per funes ad munitionem illam . Est etiam et tertius modus destruendi machinas } faciendo sagittas \\\hline
3.3.22 & que llaman ruecas \textbf{ e esta saeta es en medio hueca . } assi commo si fuesse cauada . & quas appellant telos . \textbf{ Est autem sagitta illa in medio quasi quaedam cauea , } in qua ponitur ignis fortis factus ex oleo , \\\hline
3.3.22 & e de pez e de rasina . \textbf{ El qual fuego buelto en estopa llamaronle los lidiadores antigos ençendemiento . } Et esta saeta enuiada & et pice , et resina : \textbf{ quem ignem cum stupa conuolutum bellatores antiqui Incendiarium vocauerunt . } Huiusmodi autem sagitta per ballistam fortem emissa usque ad machinam , \\\hline
3.3.22 & Et esta saeta enuiada \textbf{ por muy fuerte ballesta al . } engeñio muchas vegadas le quema . & quem ignem cum stupa conuolutum bellatores antiqui Incendiarium vocauerunt . \textbf{ Huiusmodi autem sagitta per ballistam fortem emissa usque ad machinam , } multotiens succendit ipsam . \\\hline
3.3.22 & Et cerca de aquel engeñio deuen fazer vna fragua \textbf{ en que pongan vn grand pedaço de fierro } e escalientenle & vel testam ex ferro ; \textbf{ et iuxta machinam illam construere fabricam in qua aliquod magnum ferrum bene ignatur , } quod bene ignitum apponatur super fundam ex ferro textam \\\hline
3.3.22 & Enpero podemos dar \textbf{ e mostrar a otros espeçiales remedios contra estos artifiçios de madera . } assi que pongamos el lobo contra el carnero . & aliquo alio modo , ferra ignita iaciantur in ipsa . \textbf{ Possumus tamen specialia remedia contra huiusmodi aedificia assignare , } ut contra Arietem constituatur Lupus . \\\hline
3.3.22 & et contra este carnero se puede fazer vn fierro coruo dentado de dientes muy fuertes e muy agudos \textbf{ e atado con fuertes cuerdas } con el qual prenden & et acutis , \textbf{ et ligatum funibus , } cum quo capitur caput arietis , \\\hline
3.3.22 & Mas contra los castiellos de madera valen mucho los pedaços de fierro ençendidos . \textbf{ Enpero contra esto daremos espeçial remedio . } ca pueden se fazer cueuas coneieras de dentro & Contra castra vero multum valent ferra ignita : \textbf{ adhibetur tamen speciale remedium contra ipsa , } quia fiunt cuniculi , \\\hline
3.3.22 & por que se non funda \textbf{ por la grand pesadura del o della . } Mas contra las vinnas & qua suffossa , \textbf{ et castro demerso in ipsam propter magnitudinem ponderis , } oportet castrum iterum construi , \\\hline
3.3.23 & assi que lo que non pueden auer manifiestamente ayan lo por escuchas e por arteras . \textbf{ E En este postrimero capitulo } queremos tractar algunas cosas de la batalla de las naues . & et astutias obtinere non possint . \textbf{ In hoc ultimo capitulo tractare volumus aliqua de nauali bello : } non tamen oportet circa hoc tantum insistere , \\\hline
3.3.23 & ca la naue mal fecha \textbf{ por pequena batalla de los enemigos de ligero peresçe . } Et pues que assi es conuiene de saber & nam nauis male fabricata , \textbf{ ex modica impugnatione hostium | de facili perit . } Sciendum ergo , \\\hline
3.3.23 & que estas aberturas . \textbf{ porque graue cosa es de tener mientes a estas dos cosas } en vno a la batalla de las naues & ø \\\hline
3.3.23 & por que de muchas partes se pueda quemar la naue . \textbf{ Et entonçe deuen acometer muy fuerte batalla contra los enemigos } por que se non puedan acorrer & ut ex multis partibus possit nauis succendi ; \textbf{ et cum proiiciuntur talia , | tunc est contra nautas committendum durum bellum , } ne possint currere ad extinguendum ignem . Secundo ad committendum marinum bellum multum valent insidiae . \\\hline
3.3.23 & que el tiene se pueda alçar e baxar \textbf{ ca esto echo siguiesse meior prouecho del ca pueden ferir tan bien en la naue commo en los que estan en ella } assi commo con garrocha & ut ligamentum retinens ipsum possit deprimi , et eleuari : \textbf{ quia hoc facto maior habetur commoditas , | ut cum ipso percuti possit tam nauis , } quam \\\hline
3.3.23 & assi commo con garrocha \textbf{ lo quinto en la batalla de la mar conuiene de auer grand conplimiento de saetas anchas . } con las quales se pueden ronper las uelas & etiam existentes in ipsa . \textbf{ Quinto in bello nauali | habenda est copia ampliarum sagittarum , } cum quibus scindenda sunt vela hostium . \\\hline
3.3.23 & La . ix . \textbf{ cautela es auer muchos cantaros llenos dexabon muelle } que lançen de rezio en las naues de los enemigos . & vel submerguntur in aquis . \textbf{ Nona cautela est habere multa vasa plena ex molli sapone , } quae cum impetu proiicienda sunt ad naues hostium ; \\\hline
3.3.23 & quando entrare mucha agua dentro en la naue sabullira a los enemigos \textbf{ e a la naue todo en vno sola mar . } Mas ay otras cosas & cum per ipsa coeperit abundare aqua , qua abundante , et hostes , \textbf{ et nauem periclitabit . } Sunt autem in bello nauali alia obseruanda , \\\hline
3.3.23 & que son de guardar en la batalla de las naues \textbf{ assy que ayan y muy grand conplimiento de piedras } e e avn varras de fierro agudas & Sunt autem in bello nauali alia obseruanda , \textbf{ ut sit ibi copia lapidum , } et etiam ferrorum acutorum , \\\hline

\end{tabular}
