\begin{tabular}{|p{1cm}|p{6.5cm}|p{6.5cm}|}

\hline
3.2.4 & et virtuosos , ut habeat multos pedes et multas manus : et sic fiet unus homo multorum oculorum , multarum manuum , \textbf{ et multorum pedum . Non ergo dici poterit talem unum monarchiam non cognoscere multa ; } quia quantum spectat ad regimen regni , quicquid omnes illi sapientes cognoscunt , totum ipse Rex cognoscere dicitur . Nec etiam dici poterit ipsum de leui posse corrumpi et peruerti : nam si Rex recte dominari desiderat , & Et pues que assi es non se puede dezer que vn tal monarchia o tal prinçipe assi fech̃ de muchos que non conogca \textbf{ e non sepa muchͣs cosas . | por que quanto parte nesçe al gouernamientod el regno } todo aquello que aquellos sabios conosçen e saben todo lo deue conosçer e saber el Rey . \\\hline
3.2.14 & et tyrannizat in ipsum , eum perimens vel expellens . Totus ergo populus efficitur quasi unus tyrannus contra Principem , \textbf{ et tyrannis populi corrumpit tyrannidem monarchiam . Sic } etiam una tyrannis monarchia corrumpit aliam : quia multotiens unus monarcha tyrannus insurgit in alium , ut obtineat principatum eius . Debent ergo cauere Reges et Principes & matandol o echandol del prinçipado . Et por ende todo el pueblo es fech assi commo vn tirano contra el prinçipe e cotra su tirama . \textbf{ la tirama del pueblo corronpe | e destruye la tirana del prinçipe } bien assi avn vna tirama de sennorio corronpe a otra por que muchͣs vezes vn prinçipe tirano se leunata contra otro prinçipe tirano por que gane el su prinçipado . Por ende mucho deuen escusar los Reyes e los prinçipes \\\hline

\end{tabular}
