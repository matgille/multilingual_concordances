\begin{tabular}{|p{1cm}|p{6.5cm}|p{6.5cm}|}

\hline
1.2.10 & Rursus contingit inaequalitas in distributionibus , \textbf{ quia aliquando aliqui plus laborantes pro Republica , } minus accipiunt : & e en los partires . \textbf{ Ca alas vezes algunos trabaian mas por la comunidat } e resçiben menos ante esta iustiçia \\\hline
1.2.27 & vel propter amorem Reipublicae \textbf{ quia sine ea Respublica durare non posset . } Quare si quis in tantum esset mitis , & o por amor de la comunidat \textbf{ por que sin ella la comunidat non podrie durar . | por la qual cosa si el Rey o el prinçipe } o otro qual si quieren tanto quisiesse ser mansso \\\hline
2.2.20 & et non vacant ciuilibus operibus , \textbf{ nec regiminibus reipublicae ; } si mens humana & çerca \textbf{ quales si quier obras | nin c̃ca los gouernamientos dela comunidat nin dela çibdat } Si la uoluntad del omne \\\hline
3.2.17 & ait , quod fidum et altum erat \textbf{ secretum consistorium reipublicae , } silentique salubritate munitum : & e muy alto era conssisto \textbf{ no secreto dela comunidat de roma } alos quel guardananca era guaruido de grant fialdat . \\\hline

\end{tabular}
