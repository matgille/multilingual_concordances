\begin{tabular}{|p{1cm}|p{6.5cm}|p{6.5cm}|}

\hline
1.1.1 & suis subditis imperare , \textbf{ oportet doctrinam hanc extendere usque ad populum , } ut sciat qualiter debeat & E si por qual manera Deuen mandar a los sus Subditos \textbf{ conujene esta doctrina | e esta sçiençia estender la fasta el pueblo } por que Sepa commo ha de obedesçer a sus prinçipes \\\hline
1.1.1 & oportet modum procedendi in hoc opere , \textbf{ esse grossum et figuralem . } Cum omnis doctrina & Conuie ne \textbf{ que la manera que deuemos tener enesta obra sea gruesa e figural e exenplar } a asi commo dize el philosopho \\\hline
1.1.2 & bene se habet \textbf{ narrare ordinem dicendorum , } ut de ipsis quaedam praecognitio habeatur . & asi commo el conosçimiento del entendimjento nasçe del conosçimjento Delos sesos \textbf{ Por ende buena cosa es de Recontar la orden de las cosas } que se han de dezir \\\hline
1.1.2 & ordine naturali decet regiam maiestatem \textbf{ primo scire se ipsum regere , } secundo scire suam familiam gubernare , & primeramente que el Ruy sepa gouernar asy mesmo ¶ \textbf{ Lo segundo que sepa gouernar su conpanna¶ | Lo terçero que sepa } gouernả su rregno \\\hline
1.1.2 & primo scire se ipsum regere , \textbf{ secundo scire suam familiam gubernare , } tertio scire regere regnum , et ciuitatem . In primo autem libro in quo agetur de regimine sui , & Lo terçero que sepa \textbf{ gouernả su rregno | e sus çibdades ¶ } pues que asy es en el primo libro \\\hline
1.1.2 & proficuum esse morali negocio , \textbf{ scrutari ea quae sunt circa operationes , } quomodo faciendum sit eas . & que prouechosa cosa es en la scian moral \textbf{ e en la sçiençia de costunbres escod̀nar aquellas cosas | que son cerca delas obras } commo las deue omne fazer \\\hline
1.1.3 & ostendendo , quae dicenda sunt , \textbf{ nos esse faciliter tractaturos : } et in secundo reddidimus eam docilem , & que son de dezer en este libro \textbf{ que nos prometiemos de tractar ligniamente ¶ } et en el segundo capitulo fiziemos \\\hline
1.1.3 & qui est rationalis per essentiam : \textbf{ sicut ergo rex non dicitur habere regnum , } nec dux dicitur habere ciuitatem , & que es rrazonable \textbf{ por sy mesmo peren asy es asy commo el rrey non puede auer el Regno } njn el caudiello non puede auer la çibdat \\\hline
1.1.3 & sicut ergo rex non dicitur habere regnum , \textbf{ nec dux dicitur habere ciuitatem , } si in regno vel ciuitate sunt aliqui , & por sy mesmo peren asy es asy commo el rrey non puede auer el Regno \textbf{ njn el caudiello non puede auer la çibdat | sy en el rregno } o en la çibdat ouiere discordia \\\hline
1.1.4 & ut homo est : \textbf{ sed speculari et cognoscere veritatem , } competit ei & en quanto es omne . \textbf{ Mas estudiar e conosçer la uerdat conviene al omne en quanto es en el entedimjento especłatino } e escodrinador que es alg̃cos e diujnal . \\\hline
1.1.5 & expedit volenti \textbf{ consequi suum finem , } vel suam felicitatem , & Conviene a todo omne \textbf{ que quiera alcançar e auer su fin } e la su bien andança de auer \\\hline
1.1.5 & duplici via venari possumus , \textbf{ quod expedit regi suum finem cognoscere . } Prima est , & e cobrar prouar \textbf{ que conujene al rrey | en toda manera de conosçer la su fin ¶ } La primera rrazon es en quanto el rrey \\\hline
1.1.5 & ut per opera nostra mereamur \textbf{ consequi finem , vel felicitatem . } Secundo requiritur & pues que asi es conviene bien fazer de fecho \textbf{ por que por las nr̃as obras merescamos de auer buena fino buena ventura } segunt deuemos los omes fas̉ bien \\\hline
1.1.5 & Unde Philosophus 2 Ethic’ vult , \textbf{ quod non sufficit agere bona , } sed bene : nec sufficit operari iusta , & en el segundo libro delas ethicas \textbf{ que non cunple solamente fazer buenas obras } mas fazerlas bien njn cunple de obrar obras iustas \\\hline
1.1.5 & volens ostendere \textbf{ necessariam esse praecognitionem finis , ait , } quod cognitio finis & a¶ Onde el philosofo quariendo mostrar en el primero libro delas ethicas \textbf{ que es neçesario de connosçer ante la fin | dize } que para lanr̃auida grant acresçentamiento \\\hline
1.1.5 & quia in operibus suis debet \textbf{ intendere bonum gentis et commune , } quod est magis expediens et diuinius , & por que entondas sus obras \textbf{ deue entender al bien dela gente | e al bien comun } que es mas conuenible \\\hline
1.1.6 & In huiusmodi autem voluptatibus sensibilibus \textbf{ non esse felicitatem ponendam , } triplici via venari possumus . & que en estas delecta çonnes sensibles de los sesos . \textbf{ non es de poner la feliçidat e la bien andança . } Ca quanto parte nesçe alo prèsente la bien andança ençierra en si tres cosas \\\hline
1.1.6 & constat in talibus \textbf{ non esse felicitatem ponendam . } Quod autem huiusmodi voluptates , & que en las tales delectaçiones \textbf{ non auemos nos de poner lanr̃a feliçidat | nin lanr̃a bien andança } Mas que estas plazenterias \\\hline
1.1.6 & nulla tamen delectatio est essentialiter ipsa felicitas , \textbf{ licet possit esse aliquid felicitatem consequens , } sed hoc declarare non est praesentis negocii . & por si mesma \textbf{ maguera que se pueda conseguir | ala feliçidat e ala bien andança . } Mas declarar esto non parte nesçe a esta arte presente . \\\hline
1.1.6 & licet possit esse aliquid felicitatem consequens , \textbf{ sed hoc declarare non est praesentis negocii . } Forte tamen de hoc aliquid infra dicetur . & ala feliçidat e ala bien andança . \textbf{ Mas declarar esto non parte nesçe a esta arte presente . } Enpero que por auentra a adelante diremos alguna cosa desto . \\\hline
1.1.7 & in artificialibus diuitiis \textbf{ felicitatem non esse ponendam . } Primo , quia artificiales diuitiae & por las quales nos pondemos prouar \textbf{ que la feliçidat e la bien andança non es de poner en les riquezas artifiçiales ¶ } La primera razon es por que las riquezas artifiçiales son orderandas alas riquezas natraales ¶la segunda \\\hline
1.1.7 & Dicebatur enim supra , \textbf{ felicitatem esse illud bonum , } ad quod alia bona ordinantur , & Ca dichones de suso \textbf{ que la feliçidat | e la bien andança es de poner en aquel bien } aque todos los otros bienes son ordenados \\\hline
1.1.7 & nunquam potest esse magnificus , \textbf{ cuius est facere magnos sumptus : } nec etiam potest esse Magnanimus , & nin granado \textbf{ el qual magnifico ha de fazer grandes espensas para ser granado } nin ahun puede ser mager fico \\\hline
1.1.7 & principaliter intendit reseruare sibi , \textbf{ et congregare pecuniam . } Non ergo est Rex , & e en los aueres \textbf{ prinçipalmente entiende de thesaurizar e fazer thesoro e llegar muchos dineros } Et por ende se sigue \\\hline
1.1.7 & quo potest , \textbf{ consequi finem suum . } Ponens igitur suam felicitatem in diuitiis , & que pudiere \textbf{ por que pueda alcançar aquella fin | e aquel bien ¶Donde se sigue } que el prinçipe \\\hline
1.1.7 & omni via qua potest , \textbf{ velle consequi suum finem . } Est igitur Rex Tyrannus , & por ninguna manera \textbf{ que non pueda querer seguir la su fin | ante se trabaia dela alcançar quanto puede ¶ } Pues que assi es el Rei es tirano \\\hline
1.1.8 & maxime indecens est \textbf{ ipsum ponere felicitatem in honoribus , } ne sit fictus , et superficialis . & muy mas desconueinble cosaes ael \textbf{ que otro ninguon de poner su bienandança en las honrras } por que non paresca infinto e superfiçial ¶ \\\hline
1.1.8 & Secundo indecens est Regi , \textbf{ ponere suam felicitatem in honoribus , } quia ex hoc efficietur periclitator Populi , et praesumptuosus : & assi que muy desconueible cosa es al Rey \textbf{ poner su bien andança en las honrras | Ca por esso seria prisuptuoso } e sob̃uio \\\hline
1.1.9 & Quare cum Regem deceat \textbf{ esse totum diuinum , } et quasi semideum , & por la qual cosa commo al Rey conuenga ser todo diuinal e semeiante a dios \textbf{ si non es cosa conuenible | de poner la feliçidat } e la bien andança \\\hline
1.1.9 & mercedem tribuendam esse Regibus , \textbf{ et hunc esse honorem et gloriam . } Non est intelligendus textus Philosophi , & gualardon que los Reis deuian \textbf{ auer era en eglesia e en honrra segunt el philosofo dezie . } El testo del philosofo non se deue \\\hline
1.1.9 & honor tamen eos consequitur , \textbf{ et decet eos acceptare honorem sibi exhibitum , } non habentibus Hominibus aliquid maius , & enpero la honrra les parte nesçe a ellos . \textbf{ Et conuiene les alos Reys de resçebir la honrra | que les fazenn los omes } por que los omes non les pueden dar mayor cosa que honrra \\\hline
1.1.10 & Vegetius in libro De re militari , \textbf{ super omnia commendare videtur bellorum industriam . } Hoc enim ( secundum ipsum ) est & que fizo dela caualleria \textbf{ que sobre todas las cosas es de alabar la maestria | e la sabiduria delas batallas . } Et esta es vna cosa segunt que el dize \\\hline
1.1.10 & et summo opere studuerunt , \textbf{ quomodo possent sibi subiicere nationes . } Propter quod & Et sobre todas las cosas estudiaron \textbf{ commo pudiessen subiugar todas las naçiones } e todas las gentes . \\\hline
1.1.10 & Nam per ciuilem potentiam \textbf{ velle sibi subiicere nationes , } hoc est , velle dominari per violentiam . & La primera razon se puede assi declarar \textbf{ Ca querer subiugar las naconnes | e las gentes } por poderio çiuil esto esquerer \\\hline
1.1.10 & velle sibi subiicere nationes , \textbf{ hoc est , velle dominari per violentiam . } Violentia autem perpetuitatem nescit . & e las gentes \textbf{ por poderio çiuil esto esquerer | enssen onrear por fuerça } e non pornatraa \\\hline
1.1.10 & et ad ea , \textbf{ per quae sibi possit subiicere nationes . } Inducet ergo ciues & Et aquellas cosas \textbf{ por que pueda subiugar | assi las naçiones e los pueblos . } Et por ende non induzir a los çibdadanos \\\hline
1.1.11 & Debet enim Princeps \textbf{ possidere sufficientes diuitias , } ut possit regnum defendere , & e fazen grand discordia en el pueblo ¶ \textbf{ Otrossi deuen los prinçipes auer riquezas sufiçientes } por que puedan defender los regnos \\\hline
1.1.11 & ut possit regnum defendere , \textbf{ et exercere operationes virtutum : } decet enim Regem esse magnificum , & por que puedan defender los regnos \textbf{ e fazer obras de uertudes . } E conuiene al Rey de seer magnifico e largo \\\hline
1.1.11 & est Rex dignus honore , \textbf{ et expedit ei habere ciuilem potentiam : } nam propter paruipensionem Principis , & por que non sea menospreçiada la Real magestad . \textbf{ Et por ende le conuiene de auer poderio çeuil . | Ca por el } menospreçiamientodel prinçipe muchͣs vezes contesçe que alguons fazen e obran malas cosas \\\hline
1.1.11 & sed quia possunt \textbf{ esse organa ad felicitatem . } Talia ergo diligenda sunt , & mas por que son instru mentos \textbf{ para ganar la feliçidat e la bien andança . } Et pues que assi es estas cosas tales son de amar \\\hline
1.1.12 & Voluit autem felicitatem \textbf{ non esse ponendam in viribus , } siue in potentiis animae , & que la feliçidat \textbf{ e la bien andança non se deue poner en las fuerças corporales } nin en las potençias del alma senssetuias \\\hline
1.1.12 & quomodo deceat regiam maiestatem \textbf{ ponere suam felicitatem } in actu prudentiae , & en qual manera conuenga ala Real magestad \textbf{ de poner la primera feliçidat } en las obras de pradençia . \\\hline
1.1.12 & et rationem participat , \textbf{ ponere suam felicitatem } in bono maxime uniuersali , & e ha razon e entendimiento \textbf{ de poner la su bien andança en bien muy comun } e muy entelligible \\\hline
1.1.12 & si Princeps est felix diligendo Deum , \textbf{ debet credere se esse felicem operando } quae Deus vult . & Si el prinçipe es bien auenturado \textbf{ amando a dios deue creer | que es bien auenturado si obra } segunt que dios quiere e manda . \\\hline
1.1.13 & aliqualiter felicitas sit ponenda . \textbf{ Magnum autem esse praemium Regis , } et magnam eius esse felicitatem , & segunt dicho es \textbf{ or çinco razones podemos prouar | quant grant es el gualardon de los reyes } e quant grande es la su feliçidat \\\hline
1.1.13 & magna ergo debet esse virtus Regis , \textbf{ ad quem spectat regere non solum se , } et suam familiam , & pues que assi es grande deue ser la uirtud del Rey \textbf{ a quien parte nesçe de gouernar | non solamente assi mesmo } e asu conpanna \\\hline
1.2.1 & ( ut supra plenius probabatur ) \textbf{ debent uti tanquam organis ad felicitatem . } Suam autem felicitatem ponere debent & Mas assi commo prouamos conplidamente de suso deuen husar de todas estas cosas \textbf{ assi commo de instrumentos | para ganar la feliçidat e la bien andança . } Mas la su bien andança deuen poner en obras de pradençia e de sabiduria \\\hline
1.2.2 & nam arduitas , et difficultas potissime sunt repugnantia , et prohibentia , \textbf{ ne possimus consequi bonum , } et vitare malum . & e nos retienen \textbf{ por que non podamos seguir el bien } e esquiuar el mal . \\\hline
1.2.2 & secundum modum sibi conuenientem , \textbf{ prout bene possint delectari per concupiscibilem , } data est eis irascibilis , & Pues que assi es en quanto las . \textbf{ aian las | segund su manera conuenible se pueden delectar conplidamente } por el appetito \\\hline
1.2.3 & per virtutes enim debemus \textbf{ habere rationes rectas , } passiones moderatas , & e son obras que fazemos de fuera Et en estas nos conuiene \textbf{ e poner meatad e egualdat . | Ca por las uirtudes deuemos auer las razones derechas } e las pasiones ordenadas e tenprados . \\\hline
1.2.3 & Dictum est enim virtutes \textbf{ illas esse circa passiones : } quas per se habent moderare , & e la sufiçiençia se puede tomar en esta manera . Ca dicho es ya que aquellas uirtudes han de seer çerca delas passiones \textbf{ las quales passiones han de mesurar e de ygualar por si ¶ } Pues que assi es en \\\hline
1.2.3 & videlicet , Veritas , Affabilitas , et Eutrapelia , \textbf{ quae potest dici bona versio . } Est autem Veritas & La otra es eutropolia \textbf{ que quiere dezir buena conuerssaçion o buena manera de beuir . | Mas la uerdat assi conma aqui fablamos de uerdat } non en quanto es uirtud \\\hline
1.2.4 & ostendentes , \textbf{ quomodo Reges et Principes debent habere uirtutes . } Determinabimus etiam de adminiculantibus uirtuti , & de cada vna en su logar . Ca determinaremos delas uirtudes mostrando . \textbf{ en qual manera los Reyes e los prinçipes . | han de auer uirtudes e seer uirtuosos . } Et ahun determinaremos delas otras \\\hline
1.2.5 & ratiocinari recte et non recte , \textbf{ oportet dare virtutem aliquam , } quae sit recta ratio , & Ca commo contesca de razonar derechamente \textbf{ e non derechamente conuiene de dar alguna uirtud } que sea razon derecha . \\\hline
1.2.5 & Rursus cum contingat operari recte et non recte , \textbf{ sic ut est dare virtutem , } per quam dirigimur & e non derechamente \textbf{ assi commo auemos a dar uirtud . } Por la qual cosa somos endereçados en razonando delans obras en essa mis ma guas a auemos de dar uirtud \\\hline
1.2.5 & Amplius quia contingit nos passionari recte et non recte , \textbf{ oportet dare virtutes aliquas , } per quas modificentur in ipsis passionibus . & e non derecha mente . \textbf{ Conuiene nos de dar uirtudes algunas } por las quales seamos tenprados e reglados en aquellas passiones ¶ \\\hline
1.2.5 & circa passiones oportet \textbf{ dare virtutem aliquam , } ne passiones nos impellant & assi commo son las passiones dela saña . \textbf{ Conuiene dar alguna uirtud en las passiones } por la qual las passiones non nos pueden mouer \\\hline
1.2.5 & ad id quod ratio vetat : \textbf{ et oportet dare virtutem aliam , } ne passiones retrahant nos ab eo , & nin inclinar a aquelo que uieda la razon ¶ \textbf{ Et otrosi nos conuiene de dar otra uirtud } por la qual las passiones non nos pueden arredrar \\\hline
1.2.6 & quomodo possumus \textbf{ consequi talem finem , } quod fit per prudentiam . & si non sopiere \textbf{ en qual manera el puede alcançar tal fin . } e esto ha de saber \\\hline
1.2.7 & nomen enim regum a regendo sumptum est : \textbf{ regere autem alios , } et dirigere ipsos in finem debitum , & Ca el nonbre del Rey es tomado de gouernamiento . \textbf{ Mas gouernar alos otros } e guiar los en su fin conuenible . \\\hline
1.2.7 & regere autem alios , \textbf{ et dirigere ipsos in finem debitum , } sit per prudentiam . & Mas gouernar alos otros \textbf{ e guiar los en su fin conuenible . } Esto ha de ser por la pradençia \\\hline
1.2.7 & Qui ergo hoc oculo caret , \textbf{ non sufficienter videre potest ipsum bonum , } nec ipsum debitum finem , & que catamos el bien e la fin conuenible . \textbf{ Et el que non ha este oio non puede conplidamente ueer el bien } nin la su fin conuenible \\\hline
1.2.7 & non solum nomine sed re , \textbf{ decet ipsum habere prudentiam . } Secundo hoc decet eum , & non solamente segunt el nonbre \textbf{ mas segunt el fech̃o | conuiene le de auer sabiduria . } La segunda manera por que conuiene al Rey de ser sabio \\\hline
1.2.7 & Est enim prudentis , \textbf{ prouidere bona sibi et aliis , } et dirigere se et alios in optimum finem . & ala qual nos inclinan las uirtudes morales . \textbf{ Ca de omne sabio es proueer buenas cosas . } assi e alos otros e de guiar \\\hline
1.2.7 & prouidere bona sibi et aliis , \textbf{ et dirigere se et alios in optimum finem . } Si ergo aliquis prudentia careat , & Ca de omne sabio es proueer buenas cosas . \textbf{ assi e alos otros e de guiar | assi e alos otros a buena fin ¶ } pues si alguno non ouiere sabiduria \\\hline
1.2.7 & Tertio decet Reges , \textbf{ et Principes habere prudentiam , } quia sine ea non possunt naturaliter dominari . & si non commo podra sacardes e algo del su pueblo . \textbf{ La terçera manera por que conuiene al Rey de auer sabiduria es } por que sin ella non puede ser señor \\\hline
1.2.8 & et omnes partes eius . \textbf{ Consueuerunt autem assignari octo partes prudentiae , } videlicet , memoria , prouidentia , intellectus , ratio , solertia , docilitas , experientia , et cautio . & Et connuiene le de auer todas las partidas de la sabiduria . \textbf{ Mas suele le sennalar e departir ocho partes dela praderçia e dela sabiduria¶ | La primera es memoria ¶ } La segunda prouidençia ¶ \\\hline
1.2.8 & ex hoc aliquis dicitur esse prudens , \textbf{ quia est sufficiens dirigere se , } et alios in aliqua bona , & por esso es alguon dicho sabio \textbf{ porque es suficiente para enderesçar assi e alos otros e de guiar assi e alos otros a alguons bienes o a algunas buenas fines ¶ } Pues que assi es quatro cosas nos \\\hline
1.2.8 & et prouidentiam futurorum . \textbf{ Debet enim habere praeteritorum memoriam , } non quod possit praeterita immutare , & e que han de venir \textbf{ Ca deue el Rey auer memoria e remenbrança delas cosas passadas } non por que las pueda mudar . \\\hline
1.2.8 & quia nulli agenti hoc est possibile , \textbf{ sed decet Regem habere praeteritorum memoriam , } ut possit ex praeteritis cognoscere , & Ca esto ninguno non lo pie de fazer . \textbf{ Mas conuiene al Rey de auer memoria delans cosas passadas | por que pue da } por las cosas passadas conosçer e tomar \\\hline
1.2.8 & ut possit ex praeteritis cognoscere , \textbf{ quid euenire debeat in futurum . } Nam ( ut scribitur secundo Rhetoricorum ) & e ꝑcebimiento delas cosas \textbf{ que han de venir | ¶ } Ca assi commo dize el philosofo \\\hline
1.2.8 & ut plurimum futura sunt praeteritis similia . \textbf{ Secundo decet ipsum habere prouidentiam futurorum : } quia homines prouidentes futura bona , & que son passadas \textbf{ ¶lo segundo conuiene al Rey de auer | prouisionde las cosas } que han de venir . \\\hline
1.2.8 & ut ex actis praeteritis sciat \textbf{ quid agere debeat in futurum . } Ratione vero modi & por que delas cosas passadas \textbf{ sepa lo que ha de fazer en lo que ha de venir . } Mas por razon dela manera \\\hline
1.2.8 & oportet quod sit industris , et solers , \textbf{ ut sciat ex se inuenire bona gentis sibi commissae . } Verum quia nullus homo sufficit & por que sepa \textbf{ por si buscar e fallar aquellos bienes | que conuiene a su pueblo e asu gente ¶ } Mas porque ningun omne non puede conplidamente penssar aquellas cosas \\\hline
1.2.8 & Non enim decet Regem \textbf{ in omnibus sequi caput suum , } nec inniti semper solertiae propriae : & aquel que bien le conseia \textbf{ por la qual cosa non le conuiene al Rey de seguir en todas cosas su cabeça } nin atener se sienpre al su engennio propio . \\\hline
1.2.9 & fiunt magis prudentes in agibilibus . \textbf{ Secundo debent diligenter intueri futura bona , } quae possunt esse proficua regno : & que han de fazer ¶ \textbf{ La segunda manera es esta | que deuen los reyes muy acuçiosamente catar las bueans cosas } e los bueons fechos \\\hline
1.2.9 & Tertio debent saepe \textbf{ recogitare bonas consuetudines , } et bonas leges : & La terçera manera es que los Reyes et los prinçipes deuen penssar muchas uezes \textbf{ e traer a su memoria | las buenas costun bres } e las buenas leyes . \\\hline
1.2.9 & eliciendo ex eis debitas conclusiones agibilium . \textbf{ Non enim sufficit esse intelligentem , } habendo cognitionem legum , & para todas las cosas \textbf{ que ha de fazer . | Ca non abasta seer entendido } sabiendo las leyes e las costunbres \\\hline
1.2.10 & lex praecipit actus omnium virtutum . \textbf{ Praecipit enim lex operari fortia et temperata , } et uniuersaliter omnia & La ley manda fazer las obras de todas las uirtudes . \textbf{ Ca manda la ley obrar obras fuertes e obras tenpradas . } Et generalmente todas las obras \\\hline
1.2.10 & Sic etiam Ethicorum 5 scribitur , \textbf{ quod lex praecipit non derelinquere aciem , } neque fugere , & en el quinto libro delas ethicas \textbf{ que la ley manda non del enparar elaz en la fazienda } nin foyr dela fazienda \\\hline
1.2.10 & neque fugere , \textbf{ neque obiicere arma , } quod spectat ad fortitudinem . & nin foyr dela fazienda \textbf{ nin echar las armas dessi altp̃o del mester } las quales cosas pertenesçen ala fortaleza ¶ \\\hline
1.2.10 & Esse igitur Iustum secundum legem , \textbf{ et implere legalem Iustitiam , } est sequi omne bonum , & e ordenan toda manera de bondat Et por ende seer el omne iusto segunt la ley \textbf{ e conplir la iustiçia legales } segnir todo bien \\\hline
1.2.10 & et implere legalem Iustitiam , \textbf{ est sequi omne bonum , } et fugere omne vitium , & e conplir la iustiçia legales \textbf{ segnir todo bien } e esquiuar todo mal . \\\hline
1.2.10 & et fugere omne vitium , \textbf{ et habere quodammodo omnem virtutem , } propter quod legalis Iustitia dicta est & e esquiuar todo mal . \textbf{ Et es auer en alguna manera toda uirtud } ¶Et por ende la iustiçia legales \\\hline
1.2.10 & sed quia ea lex praecipit , \textbf{ et vult implere legem , } iustus legalis est . & mas en quanto las manda fazer la ley \textbf{ e el quiere conplir la ley es dicho iusto legal . } Et pues que assi es el iusto legal \\\hline
1.2.10 & Sic etiam dicitur \textbf{ unicuique tribuere quod suum est : } quia aequum est , & e lo que es igual \textbf{ Et assi es dicha dar a cada vno | lo que es suyo . } Ca cosa igual es \\\hline
1.2.11 & Habere enim huiusmodi Iustitiam , \textbf{ est implere legem . } Si ergo lex iubet omne bonum , & sobredicho la iustiçia legales en alguna meranera toda uirtud \textbf{ Ca auer esta iustiçia es conplir la ley ¶ } Pues que assi es si la ley manda \\\hline
1.2.11 & et prohibet omne malum : \textbf{ implere legem , } est esse perfecte virtuosum . & que le saga todo bien \textbf{ e uieda todo mal cunplir la les es seer omne uertuoso acabadamente . } Et por ende dize el philosofo en el primero cabło dela \\\hline
1.2.11 & implere legem , \textbf{ est esse perfecte virtuosum . } Ideo primo Magnorum moralium , & que le saga todo bien \textbf{ e uieda todo mal cunplir la les es seer omne uertuoso acabadamente . } Et por ende dize el philosofo en el primero cabło dela \\\hline
1.2.11 & cuius ciues integre essent mali , \textbf{ et in nullo vellent implere legem , } nec vellent in aliquo participare legalem Iustitiam . & si los çibdadanos fuessen enteramente malos . \textbf{ en ninguna cosa non quisi es en cunplir la ley } ni quisiesen tomar ninguna parte dela ley nin dela iustiçia . \\\hline
1.2.11 & et in nullo vellent implere legem , \textbf{ nec vellent in aliquo participare legalem Iustitiam . } Ex parte igitur ipsius legalis Iustitiae , & en ninguna cosa non quisi es en cunplir la ley \textbf{ ni quisiesen tomar ninguna parte dela ley nin dela iustiçia . | el regno no los podrie sos rir } ni la su çibdat non podrie mucho durar . \\\hline
1.2.11 & et in membris eiusdem corporis possumus \textbf{ aliquo modo contemplari Iustitiam . } Unius enim , & que la iustiçia es de vno assi mismo \textbf{ e en mienbros de vn cuerpo podemos entender la iustiçia en alguno manera . } Ca los mienbros de vn cuerpo mismo han ordenamiento entre si mismos \\\hline
1.2.12 & quod maxime decet Reges , \textbf{ et Principes esse iustos . } Possumus autem hanc veritatem & e alos prinçipes \textbf{ de ser iustos | et de guardar iustiçia . } Mas esta uerdat podemos prouar \\\hline
1.2.12 & Quantum ergo animatum inanimatum superat , \textbf{ tantum Rex siue Princeps debet superare legem . } Debet etiam Rex esse tantae Iustitiae , & que ha alma sobrepiua ala que non ha alma . \textbf{ tanto el Rey o el prinçipe deue sobrepuiar la ley . } Ca deue el prinçipe o el Rey ser de tan grant iustiçia \\\hline
1.2.12 & maxime decet \textbf{ ipsum seruare Iustitiam . } Secundo possumus inuestigare hoc idem & que conuiene mucho al Rey \textbf{ de guardar la iustiçia¶ } La segunda manera por que podemos prouar \\\hline
1.2.12 & Si ergo decet Reges et Principes \textbf{ habere clarissimas virtutes } ex parte ipsius Iustitiae , & Et pues que assi es si conuiene alos Reyes \textbf{ e alos prinçipes de auer | muy claras } uirtudes paresçe de parte dela iustiçia \\\hline
1.2.12 & quae est quaedam clarissima virtus , probari potest , \textbf{ quod decet eos obseruare Iustitiam . } Tertio hoc probari potest & que se puede prouar \textbf{ que conuiene alos Reyes | de guardar la iustiçia . } lo terçero esso mismo se puede prouar \\\hline
1.2.12 & sicut subditi , \textbf{ qui quodammodo solum habent regere seipsos , } se habent ad Principem , & assi conmo los subditos \textbf{ que en alguna manera solamente han de gouernar assi mismos . } han se a su prinçipe \\\hline
1.2.12 & quanto ex eorum Iustitia potest \textbf{ consequi maius malum , } et potest inferri pluribus nocumentum . & e para escusar la mi ustiçia e el mal quanto por la mengua dela su iustiçia se puede seguir mayor mal \textbf{ Et puede venir mayor deño a muchos . } Mas avn conuiene mas de declarar commo los Reyes \\\hline
1.2.13 & et bene agere , \textbf{ oportet dare virtutem aliquam , } per quam regulentur in agendo . & e pecar en obrando . \textbf{ Conuiene de dar e de ponetur algua uirtud } por la qual seamos reglados en las obras \\\hline
1.2.13 & et in aegritudinibus , \textbf{ et in aliis circa quae conuenit esse pericula . } Rursus in periculis bellorum homines diuersimode se habent . & e en las enfermedades \textbf{ e en los otros negoçios | en los quales pueden conteçer periglos . } Otrosi en los periglos delas faziendas \\\hline
1.2.13 & et etiam quia in periculis bellicis \textbf{ difficilius est reprimere timores , } quam moderare audacias : & Et ahun por que en los periglos delas batallas \textbf{ mas fuerte cosa es de repremer los temores } que de restenar las osadias . \\\hline
1.2.13 & rursus quia in audendo \textbf{ non tam difficile est aggredi pugnam , } sicut tolerare , & Otrosi por que en auiendo osadia \textbf{ non es tan fuerte nin tan graue cosa acometer la batalla } e la pellea commo sofrir \\\hline
1.2.13 & quod per fugam ea de facili vitare non possumus . \textbf{ Non enim sic per fugam vitare possumus aegritudines : } quia cum aegritudo sit aliquid in nobis existens , & que por foyr podemos ligeramente escapar dellos . \textbf{ Ca nos non podemos | assi por foyr escapar las enfermedades } por que la enfermedat es alguna cosa \\\hline
1.2.13 & Cum ergo naturaliter tristia fugiamus , \textbf{ difficile est reprimere timores , } per quos tristia fugimus . & Et pues que assi es commo nos natural mente fuyamos dela tristeza \textbf{ graue cosa es de repmir los temores } por los quales fuyamos dela tristeza . \\\hline
1.2.13 & Propter quod difficilius est \textbf{ sustinere pugnam , } quam aggredi pugnantes . & Et por çierto mas \textbf{ guauecosa es de sefrir lo batalla } que de acometer los lidiadores ¶ \\\hline
1.2.13 & Difficilius autem est inniti , \textbf{ et habere se fortiter } contra mala praesentia , & guaue cosa es de esforçar se el omne \textbf{ e auer se fuertemente contra los males presentes } que contra los males \\\hline
1.2.13 & quia aggredi potest fieri subito : \textbf{ sed sustinere requirit diuturnitatem , et tempus . } Difficilius est autem habere se fortiter , & ¶ lo terçero esto es mas guaue cosa por que acometer puede se fazer \textbf{ adesora mas sofrir requiere mas luengotron . } Et por ende mas guaue cosa es auerse ome fuertemente \\\hline
1.2.13 & sed sustinere requirit diuturnitatem , et tempus . \textbf{ Difficilius est autem habere se fortiter , } et constanter in sustinendo bella , & adesora mas sofrir requiere mas luengotron . \textbf{ Et por ende mas guaue cosa es auerse ome fuertemente } e firmemente en sufriendo las batallas \\\hline
1.2.13 & ( et subdit ) \textbf{ Fortitudinem esse in sustinendo tristia . } Declaratum est igitur , & Et adelante dize \textbf{ que la fortaleza es en sofrir las cosas tristes . } Et por ende ya declarado es cerca quales cosas ha de seer la fortaleza . \\\hline
1.2.13 & restat ergo declarandum , \textbf{ quomodo possumus facere nos ipsos fortes . } Notandum ergo , & Pues que assi es fincanos de declarar \textbf{ en qual manera podemos fazer anos mismos fuertes } Pues que assi es deuen dos notar e entender que commo quier que la uirtud sea contraria . \\\hline
1.2.13 & Sed quia difficilius est \textbf{ reprimere timores , } quam moderare audacias : & assi commo mas \textbf{ guaue cosa es de repremir los temores } que refrenar las osadias . \\\hline
1.2.13 & Quia igitur non possumus punctualiter \textbf{ attingere medium inter audaciam , } et timorem : & Et mas contradize el temor ala fortaleza que la osadia \textbf{ ¶Pues que assi es porque non podemos en punto alcançar el medio entre la osadia e el temor . } por ende auemos de inclinar nos mas ala osadia \\\hline
1.2.13 & Tertio declaratum fuit , \textbf{ quomodo possumus facere nos ipsos fortes : } quia maxime hoc faciemus , declinando magis ad audaciam , & ¶ Lo terçero ya declaramos \textbf{ en qual manera podemos fazer a nos mismos fuertes . } Ca mayormente nos podemos fazer fuertes \\\hline
1.2.15 & quam delectationibus aliorum sensuum . \textbf{ Possumus enim videre , audire , et odorare distantia : } sed non possumus gustare , & que en los otros . \textbf{ Ca podemos veer e oyr | e oler cosas que estan arredradas de nos . } mas non podemos gostar nin tanner \\\hline
1.2.15 & His visis de leui patet , \textbf{ quomodo nosipsos facere possumus temperatos . } Nam Temperantia , & Et estas cosas vistas \textbf{ que dichas son de ligero paresçe commo nos mismos nos podemos fazer tenprados . } Ca la tenpranca e la fortaleza se ha \\\hline
1.2.15 & sic Temperantia plus conuenit cum insensibilitate . \textbf{ Si ergo volumus nosipsos facere temperatos , } ad illam partem declinandum est , & que con la senssiblidat de los sesos . \textbf{ Et por ende si nos quisieremos fazer a nos mismos tenprados deuemos } declinara aquella parte \\\hline
1.2.15 & Quarto vero declaratum fuit , \textbf{ quomodo possumus nosipsos facere temperatos : } quia hoc maxime faciemus & ¶Lo quarto declaramos \textbf{ en qual manera podemos fazer a nos mismos tenprados . } Ca esto podemos fazer mayormente \\\hline
1.2.16 & tum etiam quia facilius est \textbf{ ei facere bonum , } et acquirere temperantiam , & Lo vno por que peta mas de uoluntad \textbf{ ¶Lo otro por que mas ligeramente puede bien fazer e ganar tenprança que fortaleza . } Mas que el \\\hline
1.2.16 & est delectabile : \textbf{ fugere autem et timere , est tristabile . } Magis quis voluntarie agit & plazenterias desfenpradas es cosa delectable \textbf{ Mas fuyr e temer es cosatste . } Et mas de uoluntad faze cada vno \\\hline
1.2.16 & nec voluntarie et deliberate agit quod agit . \textbf{ Tolerabilius est igitur peccare per timorem , } quam per intemperantiam : & e esta fuera de ssi non faz aquello que faze por uoluntad ñcon delibramiento . \textbf{ Et por ende mas de foyr | e de escusares de pecar } por temor o por miedo \\\hline
1.2.16 & sed aggredi terribilia , \textbf{ et experiri bellum , sine periculo non potest . } Valde est ergo increpandus carens tempesantia , & mas acometer las cosas espantables \textbf{ e puar las batallas non se puede fazer sin periglo . } Et pues que assi es mucho es de denostar el \\\hline
1.2.16 & patet quod est exprobrabilius \textbf{ ipsum esse intemperatum , } et insecutorem passionum . & e non fuer firme en el coraçon es de deno star por ello . \textbf{ Et es mas de denostar si fuer deste prado } e segnidor de passiones . \\\hline
1.2.16 & cuius est aliis dominari , \textbf{ esse bestialem et seruilem : } indecens est ipsum esse intemperatum . & que ha de \textbf{ enssennorear alos otros de ser bestial e sieruo } e non es cosa conuenible \\\hline
1.2.16 & esse vitium intemperantiae assimilat puero : \textbf{ quia sicut puer debet regi per paedagogum , } sic vis concupiscibilis est regenda , & destenpranca al moço . \textbf{ Ca assi commo el moço se deue gouernar | por su ayo o por su maestro } assi el apetito cobdiciador \\\hline
1.2.16 & esse puerum moribus , \textbf{ et non sequi rationem , sed passionem : } indecens est ipsum esse intemperatum . & de ser el Rey moço en costunbres \textbf{ e de non segnir razon | e en entendimiento mas passiones } e delectaçiones \\\hline
1.2.17 & contra rectam regulam rationis , \textbf{ oportet dare virtutem aliquam mediam } inter auaritiam , et prodigalitatem : & escontra regla derecha de razon e de entendimiento . \textbf{ Conuiene de dar alguna uirtud medianera } entre la auariçia e el gastamiento . \\\hline
1.2.17 & non debet proprios redditus inaniter dispergere . \textbf{ Ergo non usurpare redditus alienos , } habere debitam curam de propriis , & nin espender vanamente \textbf{ nin deue tomar las rentas agenas por fuerca } mas deue auer cuydado de su fazienda \\\hline
1.2.17 & Ergo non usurpare redditus alienos , \textbf{ habere debitam curam de propriis , } et ex eis debitos sumptus facere : & nin deue tomar las rentas agenas por fuerca \textbf{ mas deue auer cuydado de su fazienda } e delas sus rentas propias e fazer dellas sus espenssas quales conuiene ¶ \\\hline
1.2.17 & Circa autem proprios redditus custodire , \textbf{ et circa non accipere alienos , } est ex consequenti . & Mas despues desto es en guardar las tentas propias . \textbf{ Et despues es en non tomar nin forcar los bienes agenos . } Ca aquel que vsurpa e toma los bienes prouechosos agenos malamente commo non deue . \\\hline
1.2.17 & ex consequenti autem est \textbf{ circa custodire proprios redditus , } et circa non usurpare alienos . & que mas prinçipalmente es la franqueza en espender e en fazer bien alos otros . \textbf{ Et despues desto es en guardar las sus rentas propreas } e non vsurpar nin tomar las agenas . \\\hline
1.2.17 & circa custodire proprios redditus , \textbf{ et circa non usurpare alienos . } Probat enim Philosophus & Et despues desto es en guardar las sus rentas propreas \textbf{ e non vsurpar nin tomar las agenas . } Ca el philosofo praeua en el quarto libro delas ethicas \\\hline
1.2.17 & est expendere eam \textbf{ et tribuere eam aliis . } Custodire autem proprios redditus , & Ca husar del auer es en espender lo \textbf{ e partir lo alos otros } mas guardar el omne \\\hline
1.2.17 & et tribuere eam aliis . \textbf{ Custodire autem proprios redditus , } non est uti pecunia , & e partir lo alos otros \textbf{ mas guardar el omne } lo suyo non es husar del auerante es mas ganar lo e allegar lo . \\\hline
1.2.17 & esse magis circa expendere \textbf{ et circa tribuere pecuniam aliis , } quam circa proprios redditus custodire . & que la franqueza es mas en espender \textbf{ e partir el auer alos otros que en guardar las rentas propias } ¶ \\\hline
1.2.17 & quia ad virtutem principalius spectat \textbf{ facere maius bonum . } Maius autem bonum est benefacere , & Lo segundo esto mismo se praeua assi por que ala uirtud \textbf{ mas prinçipal parte nesçe de fazer mayor bien . } Et mayor bien es en bien fazer \\\hline
1.2.18 & omnino detestabile est \textbf{ Regem esse auarum , } et quod melius esset & de que non pueda guaresçer \textbf{ por ende mucho de denostar es el Rey } sy fuere auariento ¶ Et pues que assi es paresçe \\\hline
1.2.18 & tanto detestabilius est \textbf{ ipsum esse auarum , } quam prodigum : & en toda manera de ser uirtuoso tanto \textbf{ mas de denostar es el Rey si fuer auariento } que si fuere gastador \\\hline
1.2.18 & Omnino ergo detestabile est , \textbf{ Regem esse auarum . } Viso quod quasi impossibile est & Et el gastadora muchos aprouecha dando . \textbf{ Et por ende muy de depostar es el Rey si fuer auariento ¶ visto } que los Reyes non pueden ser gastadores \\\hline
1.2.18 & quae continet . \textbf{ Cum ergo tanto deceat fontem habere os largius , } quanto ex eo plures participare debent : & Ca ha . manera daua so ancho e largo e da conplidamente lo que tiene \textbf{ ¶pues que assi es conmo tanto conuenga ala fuente auer la boca | mas ancha } quanto della deuen \\\hline
1.2.18 & Spectat autem ad liberalem primo \textbf{ respicere quantitatem dati , } ut non det minus , & Mas par tenesce al libal e alstan ço de catar tres cosas ¶ \textbf{ La primera deue catar la quantidat | delo que da } por que non de menos o mas \\\hline
1.2.18 & quia magnitudo expensarum vix potest \textbf{ excedere multitudinem reddituum . } Imo si contingat liberalem & por que la grandeza delas espenssas \textbf{ apenas puede sobrepuiar ala muchedunbre de las sus rentas . Por ende si contesçe algunas uegadas al liberal de dar } mas de quanto deue legunt \\\hline
1.2.19 & Sumptus enim , \textbf{ vel possunt considerari secundum se , } vel ut proportionantur facultatibus . & Por ende la liłalidat se estiende alas espenssas mesuradas \textbf{ ca las espenssas o se pueden penssar | segunt } si o se pueden penssar \\\hline
1.2.19 & ista decenter se habere debet : \textbf{ non tamen aeque principaliter intendere debet circa omnia ista . } Nam principaliter et primo , & conueniblemente çerca estas quatro cosas . \textbf{ Mas enpero non deue entender egualmente nin prinçipalmente cerca estas quatro cosas . } Ca primero e prinçipalmente deue seer el omne magnifico \\\hline
1.2.20 & quod remouere a se pecuniam , \textbf{ sit abscindere membra a proprio corpore . } Ideo sicut dato & Ca paresçe leal paruifico \textbf{ que tirar el auer de ssi | estaiarle los mienbros de su cuerpo . } Por ende assi commo si fuesse menester \\\hline
1.2.20 & non potest paruificus \textbf{ ita modicum sumptum facere erga quodcunque opus , } quin semper videatur ei & assi en essa misma manera non puede el \textbf{ parufico fazer despenssas tan pequanas } en qual si quier obra que faga que non le paresca sienpre a el \\\hline
1.2.20 & Quod autem deceat \textbf{ ipsum esse magnificum , } sufficienter probant superiora dicta : & que el Rey sea periufico mas que conuengaal Rey de ser magnifico \textbf{ e de fazer grandes espenssas } conplidamente es prouado \\\hline
1.2.20 & distribuere bona regni , \textbf{ omnino decet eum magnifice se habere erga personas dignas , } quibus digne competunt illa bona . & prinçipalmente partir los bienes del regno \textbf{ en todas maneras le conuiene ael de se auer grande | e honrradamente a aquellas personas } que son dignas \\\hline
1.2.20 & regia persona debet esse reuerenda et honore digna , \textbf{ spectat ad Regem magnifice se habere erga personam propriam , } et erga personas sibi coniunctas , & La persona del Rey deue ser de grand reuerençia \textbf{ e digna de grand honrra parte nesçe mucho al Rey de se auer granadamente } e honrradamente çerca dela su persona propia \\\hline
1.2.21 & Secunda proprietas magnifici , \textbf{ est facere magnos sumptus , } non ut ostendat seipsum , & entendimiento¶ \textbf{ La segunda propiedat del magnifico es fazer grandes espenssas } non por que se muestre \\\hline
1.2.21 & Est enim hoc commune cuilibet virtuti , \textbf{ agere non propter fauorem , } vel propter gloriam hominum , & nin por vanagłia mas por razon de algun bien \textbf{ ca esto es comun a cada vna delas uirtudes obrar non } por honrra o por vanagłoia de los omes \\\hline
1.2.21 & difficile est in talibus \textbf{ non quaerere humanam laudem . } Et quia virtus est & Enpero guaue cosa es en tales cosas \textbf{ non demandar loor delas gentes } Et por quela uirtud es cerca bien e cerca la cosaguaue . \\\hline
1.2.21 & in suis magnificis operibus , \textbf{ et distributionibus intendere finaliter bonum , } et non fauorem , et gloriam hominum . & en las sus muy grandes obras \textbf{ e en las sus parti | connsenteder finalmente el bien } e non honrra \\\hline
1.2.21 & quam quot et quanta numismata oporteat \textbf{ ipsum consumere propter huiusmodi opera . } Quinta proprietas est , & e en qual manera aquellos dones sean grandes e conuenibles que entender e cuydar quantos des \textbf{ e quanto auer le conuiene ael de despender en estas obras ¶ } La quinta propiedates \\\hline
1.2.21 & si facere decentes sumptus est \textbf{ esse liberalem , } facere maximos decentes sumptus , & que faze conuenibles espenssas enlas grandes obras \textbf{ si faz conuenibles espenssas faz omne ser liberal fazer muy grandes } e muy conuenibles espenssas \\\hline
1.2.21 & Ad eos autem maxime spectat \textbf{ facere magnas largitiones , } et excellentes sumptus boni gratia & e conosçedores quales despenssas a quales obras conuienen . \textbf{ Et aellos otrosi mucho mas pertenesçe de fazer grandes donaconnes } e lobre puiantes de espenssas \\\hline
1.2.21 & Oportet etiam eos esse excellenter liberales , \textbf{ et semper facere magnifica opera . } Omnes igitur proprietates magnifici per amplius , & e alos prinçipes de ser liberales muy altamente \textbf{ e de fazer sienpre obras muy grandes e magnificas . } Et pues que assi es todas las propiedades del magnifico \\\hline
1.2.21 & quia non quilibet potest \textbf{ facere magnos sumptus . } Sed , ut ibidem dicitur , & que non pue de cada vno ser magnifico \textbf{ por que non puede cada vno fazer grandes espenssas } Mas assi commo alli dize el philosofo tales son los nobles e los głiosos . \\\hline
1.2.21 & tanto decet ipsum pollere magnificentia , \textbf{ et habere proprietates magnifici . } Bonorum exteriorum & mas le conuiene ael de resplandesçer \textbf{ por magnificençia e auer propiedades de magnifico | e de ome muy guanade } ssi commo dicho es de suso algunos de los bienes de fuera son aprouechosos \\\hline
1.2.22 & Ad pusillanimem enim pertinet \textbf{ nescire fortunas ferre . } Ideo Andron’ Perip’ ait : & Mas al pusill animo \textbf{ e de flaco coraçon pertenesçe non saber sofrir buenas uenturas . } Por ende dize andronico el sabio philosofo \\\hline
1.2.22 & quae nos ad magnanimitatem trahunt , \textbf{ est parua pretiari exteriora bona , } quaecunque sint illa , & que inclinan anos a magnanimidat \textbf{ es poco preçiar todos los bienes } de fuera quales se quier que sean . \\\hline
1.2.22 & siue quaecunque alia huiusmodi bona . \textbf{ Dictum est enim pusillanimem nescire fortunas ferre , } sed ex modico fortunio extolli , & si quier quales si quier otros tales bienes \textbf{ Ca dicho es de suso | que el pusillanimo non sabe sofrir buenas uenfas } mas de muy \\\hline
1.2.23 & Prima proprietas magnanimi , \textbf{ est bene se habere circa pericula . } Bene autem se habere circa ea , & La primera propriedat del magnanimo es \textbf{ que se deue bien auer çerca los periglos } Mas auer se bien çerca los periglos \\\hline
1.2.23 & Propter quod , quia esse plurimum retributiuum , \textbf{ est agere opera virtutum , } conuenit magnanimo esse plurimum retributiuum , & e dador de los galardones \textbf{ es fazer obras de uirtudes . } Ca assi commo dize el philosofo en el quarto libro delas ethicas pertenesce mucho \\\hline
1.2.23 & ut circa ea , \textbf{ ex quibus consurgere possunt magni honores ; } talia autem non multotiens occurrunt , & assi commo cerca aquellas \textbf{ de que se pueden le unatar grandeshonrras . } Et tales cosas commo estas non contesçen muchas vezes . \\\hline
1.2.24 & et honoris amatiuos . Reges enim et Principes decet honores diligere modo quo dictum est ; \textbf{ videlicet , ut diligant et cupiant facere opera , } quae sint honore digna . & e alos prinçipes amar las honrras \textbf{ en la manera que dich̃ones de suso . | Conuiene saber que amen e cobdicien fazer lobras } que sean dignas de honrra . \\\hline
1.2.25 & non appellat magnanimum , sed temperatum . \textbf{ Cum igitur habere temperantiam in honoribus , } sit idem , & mas llamale tenprado . \textbf{ Et por ende auer algun tenpramiente en las honrras } es esso mismo \\\hline
1.2.25 & sit idem , \textbf{ quod habere humilitatem : } virtus illa , & es esso mismo \textbf{ que auer humildat . } Et aquella uirtud o razon de tenprança \\\hline
1.2.25 & si unum et idem aliter et aliter acceptum nos retrahit et impellit , \textbf{ oportebit circa illud dare duas virtutes , } unam impellentem , & e nos allega a aquello que la razon manda o uieda . \textbf{ Conuiene de dar en aquella cosa dos uirtudes ¶ La vna que nos allegue . } Et la otra qua nos arriedre dello . \\\hline
1.2.25 & Utrum autem humilitas sit idem simpliciter \textbf{ quod diligere mediocres honores , } vel utrum sit idem simpliciter & Mas si la humildat es essa misma cosa \textbf{ sinplemente que amar las honrras medianeras . } O si es essa misma cosa \\\hline
1.2.26 & haec duo eidem virtuti competere possunt . \textbf{ Spectat igitur ad magnanimitatem reprimere desperationem , } ne desperemus de bonis arduis , & e prinçipalmente pueden parte nesçer a vna uirtud \textbf{ e por ende pertenesçe ala magranimidat repremir la desparaçion } por que non desesꝑemos de los bienes muy altos . \\\hline
1.2.26 & circa quae habet esse . \textbf{ Intendit enim humilis reprimere superbias , } et moderare deiectiones . & çerca quales cosas ha de seer . \textbf{ Ca el humildoso entiende repremir las soƀͣiuas } e tenprar los despreçiamientos e los decaemientos . \\\hline
1.2.26 & circa haec aeque principaliter . \textbf{ Nam humilitas principaliter intendit reprimere superbias , } ex consequenti vero moderare deiectiones . & Enpo non es cerca desto egualmente nin prinçipalmente \textbf{ por que la humildat prinçipalmente entiende repremir las soƀͣmas . } mas despues desto entiende tenprar los despreçiamientos \\\hline
1.2.26 & Inquirendo enim opera honore digna , \textbf{ non solum contingit peccare per superbiam , } sed etiam per deiectionem . & que son dignas de grant honrra \textbf{ non solamente pueden pecar | por sobrepuiamiento } mas ahun puede pecar \\\hline
1.2.26 & Debent enim Reges \textbf{ sic quaerere opera honore digna , } non ultra quam ratio dictet , & Por que conuiene alos Reyes \textbf{ et alos prinçipes | assi de madar las obras dignas de honrra } que non sean mas que la razon \\\hline
1.2.26 & quod faciunt superbi . \textbf{ Debent enim agere bona opera } et honore digna boni gratia , & en sobrepuiança de honrra lo que fazen los sobuios \textbf{ por que deuen fazer los Reyes bueans obras e dignas de honrra | non por alabança } e por sobrauia \\\hline
1.2.27 & Mansuetudo enim principaliter \textbf{ et primo intendit reprimere iras , } ex consequenti autem intendit moderare passiones oppositas irae . & e prinçipalmente entiende repremir las sañas \textbf{ mas despues desto entiende tenprar las passiones contrarias dela sana | que es nunca se enssanar } por lo que ha razon de se ensannar . Ca natural cosa es anos \\\hline
1.2.27 & sed etiam quodammodo naturale est nobis \textbf{ appetere punitionem ultra condignum . } Nam quia malum nobis illatum videtur nobis maius esse , & a aquellos que nos fazen alguons males . \textbf{ Mas avn en alguna manera natural cosa esa nos de dessear de ser vengados dellos mas que deuemos . } Ca por que el mal que ellos nos fazen paresçe a nos \\\hline
1.2.27 & quam puniendi sint . \textbf{ Difficile est ergo valde reprimere iras , } et non appetere punitiones iniuriarum & por el mal que nos fazen . \textbf{ Et por que muy | guaue cosa es de repremir las sannas } e de non dessear uengança delas iniurias \\\hline
1.2.27 & Difficile est ergo valde reprimere iras , \textbf{ et non appetere punitiones iniuriarum } ultra quam dictet ratio . Plures ergo peccant in appetendo plus : & guaue cosa es de repremir las sannas \textbf{ e de non dessear uengança delas iniurias } mas que la razon e el entendimiento muestra \\\hline
1.2.27 & mansuetudo nominat temperamentum irae . \textbf{ Quod autem deceat Reges et Principes esse mansuetos , } ostendere non est difficile . & La manssedunbre nonbra tenpramiento de sana . \textbf{ mas mostrar que conuiene alos Reyes | e alos prinçipes de ser manssos } esto non es cosa guaue mas ligera . \\\hline
1.2.28 & nisi recte conuersari cum hominibus , \textbf{ et ordinare opera , } et verba nostra & si non derechamente beuir con todos \textbf{ e ordenar las nr̃as palauras } e las nuestras obras a buena conuerssaçion e conuenible . \\\hline
1.2.28 & quia nec quis se debet \textbf{ tantum aliis ostendere socialem , } ut videatur placidus , & Mas cada vno destos fallesçen en cada vna destas razo nes \textbf{ por que ninguno non le deue en tanto mostrar conpanero alos otros } por que sea visto plazentero e falaguero . \\\hline
1.2.28 & in quibus communicat cum aliis , \textbf{ dare uirtutem aliquam , } per quam debite conseruetur . & en las quales el omne partiçipa con los otros \textbf{ de dar alguna uirtud } por la qual conueniblemente sepa conuerssar e beuir con los otros . \\\hline
1.2.28 & ait , quod decet Reges et Principes \textbf{ apparere personas reuerendas , } ne contemptibiles habeantur . & e alos prinçipes de paresçer \textbf{ perssonas reuerendas a quien deuen fazer reuerençia } por qua non sean auidos en despreçiamiento . \\\hline
1.2.29 & nisi non esse apertum , \textbf{ et non ostendere se talem , } qualis est . & por que vn mentires non ser el omne manifiesto \textbf{ nin se mostrar tal qual es . } Et pues que assi es desta uirtud \\\hline
1.2.29 & idest irrisores , et despectores . \textbf{ Oportet ergo dare aliquam virtutem mediam , } per quam moderentur diminuta , & que quiere dezir escarnidores e despreçiadores dessi mismos . \textbf{ Et pues que assi es conuiene de dar alguna uirtud medianera } por la qual sean tenpradas las cosas menguadas \\\hline
1.2.29 & Sciendum ergo quod licet \textbf{ affirmare in se esse quod non est , } vel negare quod est , & Et pues que assi es conuiene saber \textbf{ que maguera firmar cada vno de ser en ssi aquello que non es o negar } aquello que es en ssi sea mentira \\\hline
1.2.29 & sit mentiri : \textbf{ tamen non dicere totum quod est , } absque mendacio fieri potest . & aquello que es en ssi sea mentira \textbf{ Empero non dezir todo aquello que es en el puede ser sin mentira . } Pues que assi es el que quiere ser uerdadero \\\hline
1.2.29 & quod declinare ad minus , \textbf{ et dicere de se minora quam sint , } est opus prudentis . & dize que declinar alo menos \textbf{ e dezir dessi menores cosas } que sean es obra de sabio . \\\hline
1.2.29 & est opus prudentis . \textbf{ Spectat igitur ad veracem nullo modo dicere de se maiora , } quam sint , & que sean es obra de sabio . \textbf{ Pues que assi es parte nesce al uerdadero | non dezir dessi mayores cosas } que sean en el \\\hline
1.2.29 & moderare huiusmodi derisiones , \textbf{ et reprimere iactantias . } Principalius tamen spectat & de tenprar estos tales escarnesçimientos \textbf{ e de repremir los alabamientos . } Enpero mas prinçipalmente parte nesçe ala uerdat \\\hline
1.2.30 & circa ipsos iocos \textbf{ dare virtutem aliquam , } per quam debite nos habeamus ad ludos . & conuiene erca tales iuegos \textbf{ e cerca tales delecta connes deuiegos dar alguna uirtud . } por la qual conueniblemente nos ayamos alos iuegos e alos trabaios . \\\hline
1.2.30 & qualitercunque possent aliquid de illa praeda capere : \textbf{ sic volentes omnino facere risum , } et prouocare alios ad cachinnum , & de aquella prea alguna cosa en essa misma manera \textbf{ los que quieren fazer de todo en todo riso } e enduzir alos otros a escarnio \\\hline
1.2.30 & sic volentes omnino facere risum , \textbf{ et prouocare alios ad cachinnum , } non curant & los que quieren fazer de todo en todo riso \textbf{ e enduzir alos otros a escarnio } non curan en qual se quier manera puedan tomar los dichos o los fechos de los otros \\\hline
1.2.30 & uti delectationibus ludorum , \textbf{ quanto detestabilius est eos esse pueriles . } Amplius ( ut patet ex habitis ) & tenpradamente delas delecta connes de los iuegos \textbf{ en quanto mas de denostar es aellos de paresçer moços . } Otrossi assi commo paresçe \\\hline
1.2.31 & dici debent : quia ad perfectam virtutem spectat \textbf{ non solum proponere bonum finem , } sed etiam debite tendere in illum finem . & por que pertenesçe ala uirtud acabada \textbf{ non solamente establesçer fin conuenible } mas ahun deuen yr conueniblemente a aquella fin . \\\hline
1.2.31 & non solum proponere bonum finem , \textbf{ sed etiam debite tendere in illum finem . } Indigemus ergo virtutibus moralibus , & non solamente establesçer fin conuenible \textbf{ mas ahun deuen yr conueniblemente a aquella fin . } Et pues que assi es auemos meester las uirtudes morales \\\hline
1.2.31 & et ad bonum opus , \textbf{ sufficiat proponere bonum finem , } nisi per bonam viam eatur in finem illum , & e a buena obra fazer \textbf{ non abasta de entender buena fin } si non fuere a aquella fin \\\hline
1.2.31 & et denidici , \textbf{ si sciant excogitare vias , } per quas consequantur venerea et turpia , & e de moticos \textbf{ si sopieren cuydar las carreras e los caminos | por los quales pueden alcançar las cosas delectables } segunt la carne \\\hline
1.2.31 & habeat omnes virtutes morales . \textbf{ Potest enim quis habere perfecte temperantiam , } et habere prudentiam , & aya todas las uirtudes morales \textbf{ por que puede alguno auer acabadamente la tenpnca } e auer la pradençia \\\hline
1.2.31 & Potest enim quis habere perfecte temperantiam , \textbf{ et habere prudentiam , } ut deseruit temperantiae : & por que puede alguno auer acabadamente la tenpnca \textbf{ e auer la pradençia } e la sabiduria en quanto sirue ala tenprança . \\\hline
1.2.31 & non potest autem aliquis \textbf{ habere aliquam virtutem , } nisi habeat omnes virtutes . & e si non ouiere las otras uirtudes \textbf{ nin puede ninguno acabadamente auer alguna uirtud } si non ouiere todas las uirtudes . \\\hline
1.2.31 & ei displicerent venerea : \textbf{ tamen si posset lucrari pecuniam , } quam intenderet ut finem , & que por auentra asi non plogeres en ael las cosas de luxuria . \textbf{ Enpero si pudiesse ganar el auer | e los dineros la qual cosa entendie } assi commo su fin \\\hline
1.2.31 & et Principes esse quasi semideos , \textbf{ et habere virtutes perfectas : } decet eos habere omnes virtutes , & Por la qual cosa si conuiene alos Reyes e alos prinçipes de ser \textbf{ assi commo medios dioses | e auer las uirtudesacabadas . } Conuiene a ellos de auer todas las uirtudes \\\hline
1.2.32 & contra aliud se tenere . \textbf{ Incontinere ergo est aggredi pugnam , } et in pugna non posse se tenere , & Ca contener se este nerse contra alguna cosa . \textbf{ Et por enerde el non contener se es acometer algua batalla } e en aquella batalla non se poder tener mas fallesçer en el ła¶ \\\hline
1.2.32 & Incontinere ergo est aggredi pugnam , \textbf{ et in pugna non posse se tenere , } sed deficere . & Et por enerde el non contener se es acometer algua batalla \textbf{ e en aquella batalla non se poder tener mas fallesçer en el ła¶ } En el terçero guado de malos son los destenprados . \\\hline
1.2.32 & Tales autem Philosophus assimilat paralyticis , \textbf{ qui eligentes ire in dextram , } propter dissolutionem corporis , et non valentes corpus regere , & assemeia los alos paraliticos \textbf{ los quales quieran yr ala diestra parte . | Enpero por la dissoluçion del } que non puede bien gouernar el cuerpo van ala simestro En essa misma manera los muelles \\\hline
1.2.32 & Sic molles et incontinentes proponunt benefacere , \textbf{ et eligunt ire in dextram : } tamen quia habent potentias animae dissolutas , & e los non continentes proponen de bien fazer \textbf{ e escogen de yr ala diestra . } Empero por que han los poderios del alma dessoluidos \\\hline
1.2.33 & quod sicut est \textbf{ dare diuersos gradus bonorum , } sic est dare diuersa virtutum genera , & podemos dezir \textbf{ que assi commo contesçe de dar guados de ptidos de bueons } assi conuiene de dar den parti dos linages de uirtudes . \\\hline
1.2.33 & dare diuersos gradus bonorum , \textbf{ sic est dare diuersa virtutum genera , } ita quod secundum quod aliquis est excellentior bonus , & que assi commo contesçe de dar guados de ptidos de bueons \textbf{ assi conuiene de dar den parti dos linages de uirtudes . | Conuiene a saber } que segunt que cada vno es mas altamente bueno ha mas alto grado de uirtudes . \\\hline
1.2.33 & aliquos vero diuinos , \textbf{ sic possumus distinguere quatuor ordines virtutum , } ita quod cuilibet generi bonorum & e algunos tenprados e algunos diuinales . \textbf{ Et en essa misma manera podemos departir quatro ordenes de uirtudes } assi que a cada vn linage de los bueons demos su orden de uirtudes . \\\hline
1.2.33 & Modus autem vincendi eas est auferre , \textbf{ et remouere se de eis . } Ideo continentibus dicuntur & Mas la manera para vençer estas passiones \textbf{ es tirar se e partir se dellas . } Et por ende es dich̃o \\\hline
1.2.34 & Sed visis praehabitis , \textbf{ ostendere quomodo haec sic se habent , } non est difficile . & Mas iustas las cosas dichas de suso non es cosa \textbf{ guaue demostrar en qual manera estas cosas se han assi . } Ca enbolia que es uirtud para conseiar \\\hline
1.3.1 & sicut dicebamus esse duodecim virtutes , \textbf{ sic dicere possumus quod sunt duodecim passiones : } videlicet , amor , odium , desiderium , abominatio , delectatio , tristitia , spes , desperatio , timor , audacia , ira , et mansuetudo . & que eran doze uirtudes \textbf{ assi podemos dezinr que las passiones son doze | conuiene saber amor e mal querençia e desseo . } e aborrençia er delectacion . \\\hline
1.3.1 & passiones concupiscibiles : \textbf{ restat videre , quomodo sumendae sunt passiones irascibiles . } Differunt autem hae passiones ab illis , & en qual manera se toman las passiones del appetito desseador \textbf{ fincanos deuer en qual manera se han de tomar las passiones del appetito enssannador } Mas estas passiones han diferençia e departimiento de aquellas otras . \\\hline
1.3.1 & Cum ergo non possint \textbf{ pluribus modis variari nostri motus et nostrae affectiones , } in uniuerso duodecim erunt passiones : & ¶ Et pues que assi es conmo los nuestros mouimientos del alma \textbf{ et las nr̃as afectiones e passiones | non se puedan departir en mas maneras } que dichas son seran \\\hline
1.3.2 & Accipiendo ergo huiusmodi ordinem secundum combinationem , \textbf{ dicere possumus primas passiones esse , amor , et odium . } In secundo vero gradu sunt desiderium , et abominatio . & tal segunt conbinaçion \textbf{ e ayuntamiento podemos dezir | que las primeras passiones son amor e mal querençia . } Et en el segundo guado son el desseo et el aborrençia . \\\hline
1.3.2 & uel si ipsum habemus , \textbf{ desideramus conseruari in habendo ipsum . } Abominatio uero immediate innititur odio : & O si la ouieremos \textbf{ desseamos la de guardar en auiendo la . } Mas la aborrençia sin ningun medio se ayunta ala mal querençia \\\hline
1.3.2 & praecedit alias passiones : \textbf{ et quia tendere in bonum est } magis coniungi bono & que todas las otraspassiones . \textbf{ Et por que yr al bien nos ayunta mas al bien } que fallesçer del bien \\\hline
1.3.2 & quae deficit ab ipso : \textbf{ sic quia refugere malum habet rationem boni , } ideo timor per quem refugimus malum , & que fallesce del bien \textbf{ En essa misma guisa | por que fuyr del mal ha razon de bien } por ende el temor \\\hline
1.3.2 & Utrum autem secundum aliquem alium modum mansuetudo praecedat iram , \textbf{ inuestigare non est praesentis negocii . } Delectatio autem , & Mas si en alguna manera la mansedunbrees primero que la saña \textbf{ esto non lo auemos de escrudinar aqui ¶ } Otrosi la delectaçion \\\hline
1.3.3 & ideo necessarium est ostendere \textbf{ quomodo nos habere debeamus ad illas . } Oportebat ergo enumerare omnes passiones , & por ende escoła neçesaria de mostrar \textbf{ en qual manera nos deuemos auer a aquellas passiones } Et por ende conuena de contar tondas las passiones \\\hline
1.3.3 & quomodo nos habere debeamus ad illas . \textbf{ Oportebat ergo enumerare omnes passiones , } ut sciremus numerum passionum , & en qual manera nos deuemos auer a aquellas passiones \textbf{ Et por ende conuena de contar tondas las passiones } por que sopiessemos el cuento dellas delas \\\hline
1.3.3 & de quibus determinare debemus . \textbf{ Oportebat etiam ostendere ordinem earum , } ut sciremus quo ordine determinaremus de illis . & quales auemos de determinar e de dezir . \textbf{ ¶ Otrosi conuenia avn demostrar la orden dellas } por que sopiessemos \\\hline
1.3.3 & Dilectatio enim quam habebant Romani \textbf{ ad Rempublicam fecit Romam esse principantem et monarcham . } Hoc ergo modo quoslibet homines decet esse amatiuos , & Ca el amor que auian los romanos al bien comun \textbf{ e publicofizo a Roma ser sennora | e auer sennorio en todo el mundo . } Pues que assi es que esto conuiene a todos los omes de ser amadores \\\hline
1.3.3 & tales enim sunt tyranni , \textbf{ volentes explere voluptatem propriam , } et quaerentes excellentiam singularem : & Et tales commo estos son los tiranos \textbf{ que quieren conplir su uoluntad proprea } e demandan grandia singular de su persona \\\hline
1.3.3 & quodam speciali modo prae aliis debent \textbf{ diligere bonum diuinum et commune , } et quodam speciali modo & que los Reyes et los prinçipes \textbf{ por alguna manera especial sobre todos los otros deuen amar el bien diuinal } e el bien comunal en alguna manera espeçial sobre todos los otros deuen aborresçer todas aquellas cosas \\\hline
1.3.3 & decet Reges \textbf{ et Principes amare Iustitiam , } et odit vitia , & Conuiene alos Reyes \textbf{ e alos prinçipes | assi saber amar iustiçia } e aborresçer todos los pecados \\\hline
1.3.3 & aliter vitia extirpari , \textbf{ nec potest aliter durare commune bonum , } nisi exterminando maleficos homines , & si por auentra a non pueden en otra manera destroyr los males \textbf{ nin puede en otra manera durar el bien comun } si non destruiendo \\\hline
1.3.4 & dicere restat , \textbf{ quomodo Reges et Principes se habere debeant ad desiderium , } et abominationem , & que son las primeras passiones finca de dezir \textbf{ en qual manera los Reyes et los pnçipes se deue auer al desseo } e ala aborrençia \\\hline
1.3.4 & per quam conformatur loco sursum vel deorsum . \textbf{ Secundo est ibi considerare motum , } per quem tendunt in talem locum . & al su logar de yuso o de suso ¶ \textbf{ Lo segundo es hy de penssar el mouimiento } por el qual van a aquel lugar ¶ \\\hline
1.3.4 & et mensuram ipsius finis : \textbf{ ut medicus intendit inducere sanitatem , } quantum potest , & e la mesura de aquella fin \textbf{ assi commo el fisico entiende enduzir | quanto puede sanidat en el enfermo } la mayor e meior que pudiere \\\hline
1.3.4 & inquantum per ea possunt \textbf{ cohercere malos , } punire iniusta , & e todos los otros dales bienes \textbf{ commo estos deuen los Reyes dessear en tanto en quanto por ellos pueden apremiar los malos } e dar las penas \\\hline
1.3.4 & punire iniusta , \textbf{ et facere talia , } a quibus regni salus dependere videtur . & por las cosas desiguales e malas . \textbf{ Et fazer o tris cosas tales delas quales nasçe e cuelga la salud del regno } ¶ \\\hline
1.3.4 & tanto magis decet Reges et Principes , quanto magis eos decet \textbf{ habere curam de bono regni et communi . } Quae sunt autem illa & quanto mas conuiene a ellos \textbf{ de auer cuydado del regno e del bien comun . } Mas quales cosas son aquellas que guardan el regno en buen estado \\\hline
1.3.4 & quae regnum conseruant in bono statu , \textbf{ et quomodo Rex se debeat habere ad ipsum regnum , } in tertio libro diffusius ostendetur . & Mas quales cosas son aquellas que guardan el regno en buen estado \textbf{ e en qual manera el Rey se deue auer a su regno } en el terçero libro lo mostraremos mas conplidamente \\\hline
1.3.5 & non solum spectat ad Reges \textbf{ et Principes tendere in bonum , } sed etiam decet eos tendere in bonum arduum . & Por ende non solamente parte nesçe alos Reyes \textbf{ e alos prinçipes de entender en el bien } Mas avn les conuiene de entender \\\hline
1.3.5 & et Principes tendere in bonum , \textbf{ sed etiam decet eos tendere in bonum arduum . } Amplius quanto maior est communitas , & e alos prinçipes de entender en el bien \textbf{ Mas avn les conuiene de entender | en bien alto e grande e guaue de fazer De mas desto } quanto mayor es la comunidat \\\hline
1.3.5 & Decet ergo Reges et Principes \textbf{ considerare bona non solum } ut sunt ardua , & Et pues que assi es conuiene alos Reyes \textbf{ e alos prinçipes de penssar los bienes | non solamente en quanto son altos e grandes mas avn les conuiene de los penssar } en quanto son bienes \\\hline
1.3.5 & sed ut sunt futura . \textbf{ Congruit etiam eos considerare talia , } ut possibilia . & que han de venir \textbf{ Et avn les conuiene de penssar tales bienes } en quanto pueden ser . \\\hline
1.3.5 & et nobilitas non adminiculantur eis , \textbf{ ut possint prosequi talia bona : } Reges autem et Principes , & non les siruen aellos \textbf{ por que puedan alcançar tales bienes . } Mas los Reyes e los prinçipes \\\hline
1.3.5 & et non credant eis esse possibile \textbf{ prosequi bona ardua } et magno honore digna . & si non creyeren \textbf{ que ellos pueden alcançar tan grandes bienes } e tan dignos de grand honrra . \\\hline
1.3.5 & tendere debeant in bona ardua , \textbf{ et debeant prouidere bona futura possibilia ipsi regno : } decet eos esse bene sperantes per magnanimitatem , & de una entender alos bienes altos e grandes \textbf{ e de una proueer los biens | que han de venir e los bienes que pueden acahesçer a su regno . } Por ende conuiene a ellos de serbine esparautes \\\hline
1.3.5 & Sperare enim ultra quam sit sperandum , \textbf{ et aggredi opus ultra vires suas , } videtur ex imprudentia procedere , & que deue omne esparar \textbf{ e acometer alguna obra | mas que la su fuerca demanda paresçe } que esto uiene mas de mengua de sabiduria \\\hline
1.3.5 & decet Reges et Principes \textbf{ non aggredi aliquid ultra vires , } et non sperare aliqua non speranda . & Conuiene alos Reyes e alos prin çipes \textbf{ de non acometer ninguna cosa | mas que la su fuerça demanda . } Otrossi les conuiene de non esparar alguas cosas \\\hline
1.3.6 & circa spem et desperationem \textbf{ quae respiciunt futurum bonum ; } sic secundum eandem methodum & e cerca la desesperanca \textbf{ que catan al bien | que ha de venir . } En essa misma manera seg̃t essa misma sciençia los podemos ensseñar \\\hline
1.3.6 & diligentius agimus opera , \textbf{ per quae fugere credimus timorem illum . } Ostensum est ergo , & acuciosamente fazemos las obras \textbf{ por las quales queremos foyr de aquel temor } Et por ende mostrado es que los Reyes e los prinçipes deuen auer temor tenprado . \\\hline
1.3.6 & quod decet Reges , \textbf{ et Principes moderatum habere timorem . } Attamen immoderate timere & por las quales queremos foyr de aquel temor \textbf{ Et por ende mostrado es que los Reyes e los prinçipes deuen auer temor tenprado . } Enpero temer destenpradamente en ninguna manera \\\hline
1.3.6 & Nerui ergo fiunt frigefacti , \textbf{ et non valentes sustinere membra , } quare accidit ei tremor . & Et por ende quando los neruios son enfriados \textbf{ e non pueden sofrir los mienbros del cuerpo acahesçeles } e viene les luego el tremer ¶ \\\hline
1.3.6 & si Rex sit inoperatiuus , \textbf{ et imperare non valeat propter immoderatum timorem , } toti regno praeiudicium gignitur : & si el rey fuere tal que no nobre \textbf{ e non pueda mandar | por el temor destenprado } e sin razon \\\hline
1.3.6 & Regem immoderato timore timere . \textbf{ Viso quomodo Reges se habere debeant ad timorem , } quia difficilius est reprimere timorem , & e sin razon . \textbf{ ¶ visto en qual manera los Reyes se deuen auer al temor } por que cosa mas guaue es de repremir el temor que tenprar la osadia \\\hline
1.3.6 & Viso quomodo Reges se habere debeant ad timorem , \textbf{ quia difficilius est reprimere timorem , } quam moderare audaciam , & ¶ visto en qual manera los Reyes se deuen auer al temor \textbf{ por que cosa mas guaue es de repremir el temor que tenprar la osadia } assi commo fue dicho de suso \\\hline
1.3.6 & ad audacias decet \textbf{ enim eos non habere audaciam immoderatam , } sed moderatam . & en qual manera se deuen auer los Reyes ala osadia . \textbf{ Ca conuiene aellos de non auer osadia destenprada } e sin razon mas tenprada \\\hline
1.3.7 & est idem quod velle alicui bonum secundum se . \textbf{ Sic odire aliquem est velle malum ei simpliciter , } et absolute . Ira autem non sit : & que querera alguno algun bien segunt si \textbf{ En essa misma manera querer mala alguna cosa esquerer | que luenga algun mal siplemente e suelta mente . } mas la saña non es assi . \\\hline
1.3.7 & per aliquem hominem specialem : \textbf{ licet odire possumus fures uniuersaliter ; } non tamen irascimur , & por algun omne espeçial . \textbf{ Et por ende podemos querer mal generalmente alos ladrones } enpero non nos enssannamos si non a alguna perssona singular . \\\hline
1.3.7 & Vult enim iratus \textbf{ inferre dolorem , et tristitiam : } sed odiens vult & Mas el que quiere mal a alguno tenssea dele enpeçer . \textbf{ Ca el sannudo quiere dar dolor e tsteza } mas el mal quariente quiere fazer danno e enpeçemiento¶ \\\hline
1.3.7 & sed odiens vult \textbf{ inferre damnum , et nocumentum . } Quinta differentia est , & Ca el sannudo quiere dar dolor e tsteza \textbf{ mas el mal quariente quiere fazer danno e enpeçemiento¶ } La quinta diferençia es \\\hline
1.3.7 & magis cauendum est odium quam ira . \textbf{ Immo iram transire in odium secundum Augustinum , } hoc est , trabem facere de festuca . & que dela sanna ante segunt que dize \textbf{ sat̃ agostin la saña passar se en mal querençia } esto es de vna paia fazer ugalagar ¶ \\\hline
1.3.7 & a Regibus , et Princibus , \textbf{ quia eos maxime decet sequi imperium rationis . } Cauenda est ergo ira inordinata , & e alos prinçipes \textbf{ por que mucho mas conuiene aellos | de segnir el iuyzio dela razon e del entendimiento . } Et pues que assi es paresçe \\\hline
1.3.8 & Alii autem econtrario , \textbf{ dicebant omnem delectationem esse fugiendam . } Sed hi omnem delectationem condemnantes , & Mas otros dizian todo el contrario desto \textbf{ diziendo | que toda delectaçion era mala de foyr e de esquiuar } Mas todos estos que despreçia un a todas las delectaçiones \\\hline
1.3.8 & ( ut patet per Philos 4 Metaphy’ ) \textbf{ sic ponens omnem delectationem esse fugiendam , } ponit aliquam delectationem esse prosequendam . & en el quarto libro delas ethicas \textbf{ En essa misma manera el que pone que toda delectaçiones de esquiuar e de foyr pone que alguna delectaciones de segnir . } Ca assy commo la fabla non puede ser negada sinon por la fabla . \\\hline
1.3.8 & sic ponens omnem delectationem esse fugiendam , \textbf{ ponit aliquam delectationem esse prosequendam . } Nam cum loquela non possit negari , & en el quarto libro delas ethicas \textbf{ En essa misma manera el que pone que toda delectaçiones de esquiuar e de foyr pone que alguna delectaciones de segnir . } Ca assy commo la fabla non puede ser negada sinon por la fabla . \\\hline
1.3.8 & Quanto ergo detestabilius est Reges , \textbf{ et Principes eligere vitam pecudum , } tanto detestabilius est eos & Et por ende quanto mas de deno stares alos Reyes \textbf{ e alos prinçipes de escoger uida de bestias . } Tanto mas de deno stares a ellos segnir las delecta connes bestiales \\\hline
1.3.8 & tanto detestabilius est eos \textbf{ sequi bestiales delectationes . } Patet igitur quomodo Reges , & e alos prinçipes de escoger uida de bestias . \textbf{ Tanto mas de deno stares a ellos segnir las delecta connes bestiales } ¶ Et pues que assi es paresçe \\\hline
1.3.8 & et in operibus virtuosis , \textbf{ expeditiori modo et magis perfecte efficere poterunt huiusmodi opera . Nam quanto quis vehementiori modo delectatur } in actibus virtuosis , & e en las obras uirtuosas mas desenbargadamente \textbf{ e mas acabadamente podrian fazer estas tales obras . | Ca quando alguno mas fuertemente se delecta en las obras uirtuosas } tanto \\\hline
1.3.8 & Viso , quomodo Reges , \textbf{ et Principes se habere debeant ad delectationes : } videre restat , & ¶ Visto en qual manera los Reyes \textbf{ e los prinçipes se deuen auer alas delectaçiones } finça deuer en qual maneras \\\hline
1.3.8 & videre restat , \textbf{ quomodo se habere debent ad tristitias . } Tristitia autem nunquam est assumenda , & e los prinçipes se deuen auer alas delectaçiones \textbf{ finça deuer en qual maneras } e de una auer alas tristezas . \\\hline
1.3.8 & vel per cognitionem veritatis . \textbf{ Consueuit etiam ad hoc dari quartum subsidium , } videlicet , remedia corporalia , & o por conosçimiento de uerdat . \textbf{ Mas avn suel en dar otro remedio quarto a esto . } Conuien e saber remedios corporales . \\\hline
1.3.9 & vel ut est iam praesens . \textbf{ Secundum hoc ergo sumi habent hae quatuor passiones ; } quia de bono futuro est spes , & que es de venir o en quanto es presente . \textbf{ Et por ende segunt esto se han de tomar estas quetro passions } por que del bien futuro es la esperança \\\hline
1.3.10 & quod aliqui dicuntur Zelotypi de persona aliqua , \textbf{ si noluerint in ea habere aliquod consortium . } Intensus ergo amor corporalium videtur esse amor priuatus , et reprehensibilis , & Et por ende dende sallio el vso que algunos son dichos çelosos de alguna \textbf{ personasi non quieren auer alguna conparia en ella ¶ } Et pues que assi es el amor grande de las colas corporales parelçe \\\hline
1.3.10 & nec proprie virtutes diligeret , \textbf{ si nollet in eis habere consortium . } Huiusmodi ergo zelus respectu bonorum honorabilium & nin amaria propiamente las \textbf{ uirtudessi non quisiesse auer conpania en aquellas uirtudes . } Et por ende este tal zelo \\\hline
1.3.10 & timet : \textbf{ sed ex eo , quod credit se amittere gloriam et honorem , } quae sunt bona exteriora , & que es bien de dentro teme mas \textbf{ por que alguno teme perder la honrra | e la eglesia } que son bienes de fuera ha uerguença . \\\hline
1.3.10 & quod quis se credit \textbf{ amittere interiora bona . } Sed cum quis verecundatur , & que perdera los bienes de dentro . \textbf{ Et por ende la sangre core al coraçon | para esforçar los mienbros de dentro . } Mas quando alguno ha uerguença \\\hline
1.3.10 & Si ergo omnes hae passiones \textbf{ diuersificare habent omnes operationes nostras , } decet nos omnes eas cognoscere ; & Et pues que assi es si todas estas passiones han de partir \textbf{ todasnr̃as obras conuiene a nos delas cognosçer todas . } Et tanto mas esta conuiene alos Reyes \\\hline
1.3.10 & et tanto magis hoc decet Reges et Principes , \textbf{ quanto habere debent operationes maxime excellentes . } Praedictarum passionum & e alos prinçipes \textbf{ quantomas deuen auer las obras mas altas e mas nobles . } lgunas delas passiones sobredichas paresçen ser de loar \\\hline
1.3.11 & nisi ex suppositione : \textbf{ nam si contingeret eos operari turpia , } verecundari deberent . Nemesis etiam non multum videtur esse laudabilis , & si non por alguna condiçion . \textbf{ Ca si les contesçiesse a ellos de obrar algunas cosas torpes | e malas deuen se enuergonçar . } Ahun la nemessis non paresçe mucho \\\hline
1.3.11 & Nam mali non possunt \textbf{ possidere maxima bona , } cuiusmodi sunt virtutes : & o delas bien andanças de lons malos \textbf{ porque los malos non pueden auer grandes bienes } assi commo son las uirtudes . \\\hline
1.3.11 & cuiusmodi sunt virtutes : \textbf{ sed forte possidere possunt bona media , } vel bona minima , & assi commo son las uirtudes . \textbf{ Mas por auentra a pueden auer bienes medianeros o bienes muy pequanos } los quales son bienes de fuera . \\\hline
1.4.1 & quomodo Reges et Principes \textbf{ se debeant habere ad illos . } Nam non quicquid est laudabile in hoc , & puede paresçer en qual manera los Reyes \textbf{ e los prinçipesse de una auera aquellas costunbres . | Ca algunas costunbres son de loar en los mançebos } que non son de loar en los uieios nin en los Rleyes . \\\hline
1.4.1 & est laudabile simpliciter : \textbf{ uidemus enim quod esse furibundum , } est laudabile in cane , & Mas qual si quier cosa \textbf{ que sea de loar en vno | e non en otro o es de loar } por alguna condicion non es de loar sinple mente . \\\hline
1.4.1 & non tamen est laudabile in homine . \textbf{ Sic , licet uerecundari sit laudabile in iuuenibus , } quia ratione aetatis se continere non possunt & Empero non es de loar en el omne \textbf{ En essa misma manera maguer ser | uergonçoso sea de loar en los mançebos } por razon delan hedat \\\hline
1.4.1 & Sextum autem , \textbf{ videlicet , esse verecundos , } non decet simpliciter competere Regibus et Principibus . Decet enim Reges et Principes esse liberales : & e de ser mibicordiosos . \textbf{ Mas la sexta condicion conuiene a saber ser uergoncosos . } Esta non conuiene nin pertenesçe \\\hline
1.4.1 & si multitudinem diuitiarum qua pollent , \textbf{ non multiplicarent in debitos et pios usus , } ut supra in tractatu de liberalitate sufficienter tetigimus . & que ellos han \textbf{ e por las quales los preçian en las non espender en vsos o en obras conuenibles e piadosas } assi commo dessuso dixiemos \\\hline
1.4.2 & esse passionum insecutores , \textbf{ et venereorum habere concupiscentias vehementes : } quia in eis maxime dominari habet ratio , et intellectus . & Et por ende cosa desconuenible es alos Reyes de ser segnidores delas passiones \textbf{ e de auer cobdiçias afincadas de lux̉ia } por que en ellos mucho mas se deue apoderar la razon e el \\\hline
1.4.2 & Tertio indecens est \textbf{ eos esse nimis creditiuos . } Nam cum multos habeant adulatores , & Lo terçero cosa desconuenible es alos Reyes \textbf{ e alos prinçipes de çreer de ligero . } Ca commo ellos ayan muchos lisongeros \\\hline
1.4.2 & Quarto indecens est \textbf{ eos esse iniuriatores et contumeliosos . Nam poenas inferre debent , } non iniuriam , & Lo quarto non es cosa conueniente aellos de ser tortizeros \textbf{ e deno stadores | por que ellos deuen dar penas } e non deuen dar iniurias \\\hline
1.4.3 & qui sunt mores senum , \textbf{ et quomodo Reges et Principes se debeant habere ad mores illos . } Senum autem quidam mores sunt laudabiles , & que son de denostar en los vieios \textbf{ e en qual manera los Reyes | e los prinçipes se deuan auer a aquellas costunbres . } Et deuedessaber que delas costunbres de los uieios \\\hline
1.4.3 & reddit ipsa grauiora , \textbf{ et facit ea appetere inferiorem locum . } Videmus enim quod elementa frigida & e apretandolas torna las colas mas pesadas \textbf{ e faz las dessear el logar mas bayo . } Ca nos veemos \\\hline
1.4.3 & quia frigidi non est \textbf{ appetere locum superiorem , sed inferiorem . } Viso qui sunt mores senum vituperabiles ; & nin de ser tenidos en muchͣ \textbf{ por que la cosa fria non ha de querer logar alto mas baxo . } visto quales son las costunbres de los uieios \\\hline
1.4.3 & Non decet tamen eos verecundari : \textbf{ quia indecens est ipsos operari turpia , } ex quibus verecundia consurgit . & e assi non les conuiene aellos de auer uerguenna \textbf{ por que non les conuiene de obrar cosas torꝑes } delas quales se le unata la uerguença . \\\hline
1.4.4 & Positis moribus senum vituperabilibus , \textbf{ restat enumerare mores ipsorum laudabiles . } Videtur autem Philosophus 2 Rhetoricorum , & que non son de loar \textbf{ fincanos de poner las costunbres dellos qson de loar } Mas paresçe que el philosofo en el segundo libro dela rectorica pone quatro costunbres de los uieios \\\hline
1.4.4 & viuere ratione quam passione , \textbf{ decet eos habere concupiscentias temperatas : } quia ( ut supra dicebatur ) & por razon que por passion dela carne \textbf{ conuiene les aellos | de auer las cobdiçias tenpdas . } Ca assy commo es dicho desuso las cobdiçias \\\hline
1.4.4 & ne per hoc iudicentur leues et indiscreti . \textbf{ Quarto in suis actionibus debent habere moderationem et temperamentum : } quia ( ut dictum est ) & e de poco saber ¶ lo quarto los Reyes \textbf{ e los prinçipes deuen auer | en las sus obras mesura e tenpramiento } por que assi commo dicho es ellos \\\hline
1.4.4 & qui aliis dominantur , \textbf{ sequi mores laudabiles } secundum dictamen , & e por entendemiento Conuiene alos Reyes e alos prinçipes \textbf{ que son senores de los otros segnir costunbres bueans e de loar } segunt \\\hline
1.4.5 & dicere possumus , \textbf{ ipsorum nobilium esse quatuor mores laudabiles . } Primo enim sunt magnanimi . & quanto parte nesçe alo prasente podemos dezir \textbf{ que quatro son las costunbres bueans | e de lapña esboar delos nobles omes ¶ } que son de grand coraçon ¶ \\\hline
1.4.5 & si ab antiquo affluebat diuitiis . \textbf{ Cum ergo semper sit dare initium , } in quo genitores alicuius ditari inceperunt : & si de antigo tienpo abondo en riquezas . \textbf{ Et pues que assi es comm sienpre ayamos de dar comienço } en que los padres de alguons comne caron de se enrriqueçer \\\hline
1.4.5 & ut vult Philos’ 2 de Anima : \textbf{ contingit nobiles habere mentem aptam , } et esse dociles et industres , & en el segundo del alma contesçe \textbf{ alos nobles de auer el alma mas apareiada e de ser ellos mas enssennados e mas engennosos que los otros } por que en ellos es la buean \\\hline
1.4.5 & contingit nobiles habere mentem aptam , \textbf{ et esse dociles et industres , } quia in eis viget carnis mollicies , & en el segundo del alma contesçe \textbf{ alos nobles de auer el alma mas apareiada e de ser ellos mas enssennados e mas engennosos que los otros } por que en ellos es la buean \\\hline
1.4.5 & ex diligenti consideratione suorum agibilium \textbf{ esse dociles , et industres . } Ex hoc autem apparere potest , & en todas sus obras \textbf{ que deuen fazer . } Et desto puede paresçer \\\hline
1.4.5 & esse magna societas , \textbf{ conuenit eos esse politicos et sociales , } quia ut plurimum in societate vixerunt . & Ca porque en la mayor parte en las cortes delos nobles \textbf{ es acostunbrado de auer grandes con p̃anas acahesçeles de ser corteses e aconpanables } por que en la mayor parte visquieron en conpanina de buenos . \\\hline
1.4.5 & Esse autem elatum , \textbf{ et despicere suos progenitores , } et nimis esse honoris cupidi , & por que sienpre es mas antigua¶ \textbf{ Mas ser sobrauios e despreçiar los sus engendradores } e ser muy cobdiçiosos de honrra \\\hline
1.4.5 & Non enim debemus \textbf{ appetere ipsos honores in se , } quia hoc faciunt elati et superbi : & paresçe de ser malas costunbres \textbf{ por que non deuemos dessear las honrras en lli . } Ca esto fazen los orgullolos et los sob̃uios . \\\hline
1.4.5 & quia hoc faciunt elati et superbi : \textbf{ sed debemus appetere opera honore digna , } quod faciunt virtuosi et magnanimi . & Ca esto fazen los orgullolos et los sob̃uios . \textbf{ Mas deuemos dessear las obras | que son dignas de honrra } la qual cosa fazen los uirtuosos \\\hline
1.4.5 & nisi sint boni et virtuosi , \textbf{ decet eos sequi bonos mores nobilium , } ut sint magnanimi et magnifici , prudentes et affabiles : & e uirtuosos conuiene les aellos \textbf{ de segnir las bueans costunbres de los nobles } por que sean de grand coraçon e de grand fazienda \\\hline
1.4.5 & ut sint magnanimi et magnifici , prudentes et affabiles : \textbf{ et fugere malos mores , } ut non sint elati , & e muy sabios e bien razonados . \textbf{ Otrossi les conuiene de foyr malas costunbres } por que non senas obrauios e deipreçiadores de los otros . \\\hline
1.4.6 & Utrum esset melius , \textbf{ fieri diuitem , } quam sapientem . & Et ella respondio \textbf{ que ma veye yr los sabios alas puertas de los ricos } que los ricos alas puertas de los sabios . \\\hline
1.4.6 & et quod decet Reges , \textbf{ et Principes fugere tales mores : } videre restat , & e que conuiene alos Reyes e alos \textbf{ prinçipesarredrar se de tales costunbrs finca de ueer } quales son las bueans costunbres de los ricos . \\\hline
1.4.6 & tanto magis decet Reges , et Principes , \textbf{ quanto summo Deo iudici de pluribus debent reddere rationem . } Nobilitas , diuitiae , et ciuilis potentia , & e alos prinçipes \textbf{ quanto ellos han de dar cuenta de mas cosas a dios | que asuiez de todas las cosas . } ra nobleza e la riqueza e el poderio çiuil \\\hline
1.4.7 & et esse diuitem . \textbf{ Differt etiam esse nobilem , } et esse diuitem , & entre ser noble e ser rico \textbf{ e ahun disferençia ay entre ler noble e rico } e entre ser poderoso . \\\hline
1.4.7 & Differt etiam esse nobilem , \textbf{ et esse diuitem , } ab esse potentem . & entre ser noble e ser rico \textbf{ e ahun disferençia ay entre ler noble e rico } e entre ser poderoso . \\\hline
1.4.7 & verecundatur omnino declinare a medio , \textbf{ et non agere opera virtuosa . } Ipse ergo principatus & en que esta la uirtud \textbf{ e de non fazer obras uirtuosas } e pues que assi es el prinçipado \\\hline
1.4.7 & et magnam pronitatem habent , \textbf{ ut sequantur praedictos mores . } Iuuenes ergo et senes non indignentur , & e han grand disposiçion \textbf{ para segnir las costunbres sobredichͣs . } Et por ende non se deuen enssonnar los mançebos \\\hline
1.4.7 & quin possint omnes malos mores vitare , \textbf{ et sequi ordinem rationis . } Sic etiam nec nobiles , & que non puedan ellos esquiuar todas estas malas costunbres \textbf{ e segnir orden de razon | e de entendemiento } e auer las bueans . \\\hline
1.4.7 & quia non oportet omnes esse tales , \textbf{ sed sufficit reperiri illud in pluribus : } pronitatem enim quandam , et non necessitatem , & ca non conuiene que todos seantales . \textbf{ Mas abasta que aquellas costunbres sean falladas en muchos por que non | entendiemosponer en ellos } por estas costunbres neçessidat ninguna mas alguna disposicion e inclinaçion para auer las \\\hline
1.4.7 & decet omnes homines \textbf{ sequi mores laudabiles , } et fugere vituperabiles . & Conuiene a todos los omes \textbf{ de segnir las costunbres | que son de loar } e desse arredrar delas \\\hline
2.1.1 & quod Reges debite seipsos regant , \textbf{ nisi regere sciant domum , ciuitatem , et regnum . } In hoc ergo secundo libro determinabitur de regimine domus . & e los prinçipes gouiernen assi mismos conueniblemente \textbf{ sinon sopieren gouernar su casa | e la çibdat e el regno . } Et por ende en este segundo libro \\\hline
2.1.1 & si de domo determinare volumus , \textbf{ videndum est quomodo se habeat homo adesse communicatiuum , et sociale . } Sciendum igitur , & conuiene nos de ver \textbf{ commo se deue auer el omne | para ser comunal con todos e conpanenro . } Et por ende conuiene de saber \\\hline
2.1.1 & fabricare valemus . \textbf{ Quare si naturale est homini desiderare conseruationem vitae , } cum homo solitarius non sufficiat sibi & para nuestro defendemiento . \textbf{ Par la qual cosa si natural cosa es al omne de dessear conseruaçion e guarda de su uida } commo el omne \\\hline
2.1.2 & si sit recte ordinata , \textbf{ continere debet expedientia in tota vita , } ut in tertio libro plenius ostendetur . & Et pues que assi es si la çibdat es derechamente ordenada \textbf{ deue auer en ssi todas las cosas | que son neçessarias . } a toda la uida humanal \\\hline
2.1.2 & si dicta Politica diligenter consideremus , \textbf{ apparebit quadruplicem esse communitatem ; } videlicet , domus , vici , ciuitatis , et regni . & en las politicas paresçra \textbf{ que son quatro las comuindades Conuiene a saber . | Comuidat dela casa } Et comunidat de uarrio . \\\hline
2.1.2 & et non valentibus habitare in una domo , \textbf{ compulsi sunt facere domos plures , } et constituere vicum . & por que non podien todos morar en vna casa \textbf{ por fuerça ouieron de fazer muchas casas } e fizieron vn uarrio . \\\hline
2.1.3 & Agens enim primo et principaliter intendit finem . \textbf{ Verum quia non potest habere finem , } nisi per ea , & assi commo el carpento el arca \textbf{ que faz de los maderos ¶as | por que non puede alcançar la fin } sin aquellas cosas \\\hline
2.1.3 & spectat enim non solum ad Principem siue ad legislatorem , \textbf{ sed etiam ad quemlibet ciuem prius intendere bonum ciuitatis et regni , } quam etiam bonum propriae domus , & Por ende non solamente pertenesçe al prinçipe o al fazedor delas leyes \textbf{ mas ahun a cada vno de los çibdadanos enparar mientes | primero abalbien dela çibdat e del regno } que al bien dela su casa proprea \\\hline
2.1.3 & scire gubernare domestica , \textbf{ et regere familiam siue domum , } non solum inquantum esse debent viri sociales et politici , & de saber gouernar las cosas dela casa \textbf{ e gouernar la conpanna dela casa } non solamente en quanto deuen ser uarones aconpannables e bien acostunbrados . \\\hline
2.1.3 & non solum inquantum esse debent viri sociales et politici , \textbf{ quia sic scire gubernationem domus pertinet ad omnes ciues : } sed spectat specialiter & non solamente en quanto deuen ser uarones aconpannables e bien acostunbrados . \textbf{ Ca en esta manera saber gouernar la casa | par tenesçe a todos los çibradadanos . } Mas espeçialmente esto parte nesçe a los Reyes e alos prinçipes . \\\hline
2.1.3 & Quare si specialiter spectat ad Reges et Principes \textbf{ regere regnum et ciuitates , } specialiter spectat ad eos , & Por la qual cosasi espeçialmente pertenesçe alos Reyes \textbf{ e alos prinçipes de gouernar el regno } e la çibdat espeçialmente pertenesçe a ellos \\\hline
2.1.3 & ad Reges et Principes spectat \textbf{ intendere bonum regni et principatus : } attamen huiusmodi bonum intendere & e alos prinçipes de entender \textbf{ e cuydar çerca el bien del regno | e del prinçipado } empero pertenesçe entender este bien \\\hline
2.1.3 & intendere bonum regni : \textbf{ spectat ad unumquemque ciuem scire regere domum suam , } non solum inquantum huiusmodi regimen est bonum proprium , & entenderal bien del regno . \textbf{ Por ende pertenesçe a cada vn | çibdadano saber gouernar su casa } non solamente en quanto este gouernamiento es bien propreo suyo \\\hline
2.1.4 & non sufficiebat communitas domestica , \textbf{ sed oportuit dare communitatem vici , } ita quod cum vicus constet & non cunplie la comunidat de vna casa \textbf{ mas conuiene de dar comunidat de varrio . } Por que commo el uarrio sea fech̃ de muchas casas \\\hline
2.1.4 & sed in domo oportet \textbf{ dare plures communitates : } quod sine pluralitate personarum & non solamente la casa es vna comiundat \textbf{ mas en la casa conuiene de dar muchͣs comunidades } la qual cosa non puede ser sin muchͣs perssonas . \\\hline
2.1.4 & tam necessaria in vita ciuili , \textbf{ spectat ad quemlibet ciuem scire debite regere suam domum : } tanto tamen magis hoc spectat ad Reges et Principes , & sea tan neçessaria \textbf{ enla uida çiuil pertenesçe a cada vn çibdada | no de laber gouernar conueniblemente lucasa . } Et tanto mas esto parte nesçe alos Reyes \\\hline
2.1.5 & qui non solum non possunt \textbf{ habere ministrum rationalem , } sed etiam habere non possunt ministrum animatum : & Et avn algunos son tan pobres \textbf{ que non solamente pueden auer seruidor razonable } mas avn non pueden auer huidor con alma . \\\hline
2.1.5 & habere ministrum rationalem , \textbf{ sed etiam habere non possunt ministrum animatum : } sed loco eius habent aliquid inanimatum ; & que non solamente pueden auer seruidor razonable \textbf{ mas avn non pueden auer huidor con alma . | assi commo bestia anas en logar de tal seruidor } ponen alguna cosa \\\hline
2.1.6 & magnum adminiculum habebunt , \textbf{ ut bene sciant regere regnum , et ciuitatem . } Dicebatur in praecedenti capitulo , & por que estos gouernamientos catados con grant acuçia auran grant ayuda \textbf{ por que sepan bien gouernar sus regnos et sus çibdades } icho es en el capitulo sobredicho \\\hline
2.1.7 & primum oportet \textbf{ congregare marem , et foeminam . } Est autem hic ordo rationabilis . & en la comunidat dela casa \textbf{ primeramente conuiene de ayuntar el uaron con la mugni } e esta orden es muy con razon . \\\hline
2.1.7 & Deinde ostendemus , \textbf{ qualiter viri suas uxores regere debeant } et ad quas virtutes , & e mayormente los Reyes e los prinçipes . \textbf{ Despues mostraremosen qual manera los uarones deuen gouernar sus mugers } e a quales uirtudes \\\hline
2.1.7 & primo declarandum occurrit , \textbf{ coniugium esse aliquid secundum naturam , } et quod homo naturaliter est animal coniugale . & Mas en demostrando quales el ayuntamiento del uaron e dela muger \textbf{ pmeramente nos conuiene de declarar en qual manera el mater moino es alguna cosa segunt natura . } Et que el omne naturalmente \\\hline
2.1.7 & Probabatur enim in primo capitulo huius secundi libri , \textbf{ hominem esse naturaliter animal sociale et communicatiuum . } Communitas autem in vita humana & que el omne es naturalmente \textbf{ aina l aconpannable e comun incatiuo | que quiere dezir ꝑtiçipante con otro } Mas la comunidat en la uida humanal \\\hline
2.1.7 & et omnibus animalibus , \textbf{ habere naturalem impetum } ad producendum sibi simile . & e atondas las aianlias \textbf{ auer natural inclinaçion e appetito } para engendrar cosa semeiable \\\hline
2.1.8 & quod decet coniugia indiuisibilia esse . \textbf{ Ad quod ostendendum adducere possumus duas vias , } quas philosophi tetigerunt . & que los casamientos sean sin departimiento ninguno \textbf{ e que non le puedan partir . | Et para esto mostrar podemos dezir dos razones } las quales posieron los philosofos \\\hline
2.1.8 & si ab amicitia eius discedat : \textbf{ si inter virum et uxorem debitam fidem , } vel fidelem amicitiam saluare volumus , & por amistança \textbf{ si se departe della . Si entre el marido e la mugier queremos saluar fe conuenible } e amistan ça leal \\\hline
2.1.8 & tanto magis hoc decet reges et principes , \textbf{ quanto magis in eis relucere debet fidelitas , et ceterae bonitates . } Secunda via ad inuestigandum & tantomas parte nesçe alos Reyes e alos prinçipes \textbf{ quanto mas deue en ellos reluzir la fialdat | e todas las otras bondades ¶ } La segunda razon para prouar esto \\\hline
2.1.8 & prae omnibus aliis \textbf{ debent diligentiorem habere curam . } Incuria enim regiae prolis & que alos otros \textbf{ ca non auer cuydado de los fijos del Rey } mas puede fazer danno a todo el regno \\\hline
2.1.9 & nimis vacare venereis , \textbf{ et retrahere se ab actibus prudentiae , } et ab operibus ciuilibus , & mucho alos deleytes de lux̉ia \textbf{ e arredrar se delas obras dela sabiduria } e delas obras ciuiles \\\hline
2.1.9 & ab huiusmodi cura , \textbf{ indecens est eos plures habere uxores . } Secunda via sumitur & que deuen tomar en el gouernar aiento del regno \textbf{ non les conuiene de auer muchͣs | mugiers¶ } La segunda razon se toma de parte dela muger . \\\hline
2.1.9 & non potest \textbf{ portare onera matrimonii , } nec sufficit ad praestandum filiis omnia necessaria & por que en la mayor parte vna sola fenbra \textbf{ non puede sofrir las cargas del matermonio } nin abonda para dar todas las cosas neçessarias alos fijos \\\hline
2.1.9 & Quare si nolentes adhaerere coniugio , \textbf{ decens est eos adhaerere secundum modum , } et ordinem naturalem , & por casamiento es cosa conuenible \textbf{ que se ellos ayunten | segunt manera conuenible } e segunt ordenna traal . \\\hline
2.1.10 & Coniugium enim ad quatuor comparari potest , \textbf{ ex quibus sumi possunt quatuor rationes , } per quas inuestigare possumus , & que vna muger aya muchos maridos en vno . Ca el casamiento puede ser conparado a quatro cosas \textbf{ delas | quales podemos tomar quatro razones } Por las quales podemos prouar \\\hline
2.1.10 & per quas inuestigare possumus , \textbf{ omnino detestabile esse unam foeminam nuptam } esse pluribus viris . & Por las quales podemos prouar \textbf{ que es de denostar en todo en todo que vna muger sea casada con muchos uarones } Ca en el mater moino \\\hline
2.1.10 & Nam etsi unus masculus potest \textbf{ plures foecundare foeminas : } una tamen foemina non sic foecundari potest & cosadesconuenible en qua vna fenbra aya muchos maridos \textbf{ ca commo quier que vn mas lo puede enprenniar muchͣs fenbras . } En pero vna muger non puede \\\hline
2.1.10 & et in haereditate prouideant . \textbf{ Detestabile est ergo unum virum plures habere uxores : } sed detestabilius est unam uxorem plures habere viros , & nin en proueer los dela hedat ¶ \textbf{ Et pues que assi es cosa de denostar | que vn ome aya muchas mugers . } Et mucho mas de denostares \\\hline
2.1.11 & Prima via sic patet . \textbf{ Nam cum ex naturali ordine debeamus parentibus debitam subiectionem , } et consanguineis debitam reuerentiam , & La primera razon se declara assi . \textbf{ Ca commo por la orden natural deuamos auer | subiectiuo al padre e ala madre } e reuerençia conueible alos parientes \\\hline
2.1.12 & quae sint ex nobili genere . \textbf{ Secundo propter esse pacificum quaerenda est amicorum multitudo . } Nam pax inter homines se habet & que sean de noble linage¶ \textbf{ Lo segundo por el bien dela paz es de querer en el mater moino la muchedunbe de los amigos . } Ca la paz se ha entre los omes \\\hline
2.1.12 & et Principes in suis coniugibus \textbf{ debent quaerere exteriora bona : } de leui patet & e los prinçipes \textbf{ deuen demandar con sus mugers los bienes de fuera } que pertenesçen a honrramiento del cuerpo . \\\hline
2.1.13 & et quod non amet esse ociosa , \textbf{ sed diligat facere opera non seruilia . } Quae autem sunt opera non seruilia & e que non ame ser uagarosa . \textbf{ mas que ame fazer obras non seruiles | nin de sieruo . } Mas quales son estas obras \\\hline
2.1.13 & quam seruando fornicationem vitant ) \textbf{ et bonum prolis magis directe pertinere videntur ad coniugium , } quam ea quae in praecedenti capitulo diximus . & La qual fialdat guardando escusan la fornicaçion \textbf{ e avn el bien dela generaçion de los fijos . | Mas paresçe parte nesçer derechamente al casamiento } que aquellas cosas \\\hline
2.1.13 & ut filii polleant magnitudine corporali , \textbf{ quaerere in suis uxoribus magnitudinem corporis : } tanto tamen magis hoc decet Reges et Principes , & e por que los fijos dellos \textbf{ resplandezcan por grandeza de cuepo de demandar en las sus mugers grandeza de cuerpo . } Enpero tanto mas esto conuiene alos Reyes \\\hline
2.1.13 & decet eos \textbf{ in suis uxoribus quaerere magnitudinem , } et pulchritudinem corporalem : & por fiios grandes e fermosos . \textbf{ Conuiene a ellos de demandar en las sus mugieres grandeza e fermosura corporal . } Ca paresçe que la fermosura dela muger \\\hline
2.1.13 & Dicebatur autem supra , \textbf{ quod agere secundum rationem , } et insequi passiones , & a cuyo contrario son las mugers mas inclinadas . \textbf{ Et dicho es de ssuso que obrar segunt razon e seguir las passiones } lonco las contrarias \\\hline
2.1.13 & Decet ergo coniuges temperatas esse . \textbf{ Decet eas etiam amare operositatem : } quia cum aliqua persona ociosa existat , & que las muger ssean tenpradas . \textbf{ Et avn les conuiene aellas de amar fazer buenas obras . } Ca quando alguna persona esta de uagar mas ligeramente es inclinada a aquellas cosas \\\hline
2.1.13 & et quomodo per tale coniugium \textbf{ consequi possint ciuilem potentiam , } et multitudinem amicorum . & Et en qual manera por tal casamiento \textbf{ pueden auer poderio çiuil } e muchedunbre de amigos . \\\hline
2.1.14 & Quia non sufficit \textbf{ scire quale coniugium , } et qualiter quis se habere debeat & que sean conuenibles e honestas . \textbf{ or que non abasta saber qual es el casamiento } e en qual manera se deue cada vno auer en tomar sumus \\\hline
2.1.14 & et sermones quidam , \textbf{ quomodo vir habere se debeat circa ipsam . } Dicitur ergo tale regimen politicum : & e algunas palauras \textbf{ en qual manera el marido se deua auer çerca la muger . } Et por ende es dicho tal gouernamiento politico e çiuil por que es semeiado a aquel gouernamiento \\\hline
2.1.15 & ita quod unus gladius deseruiebat pluribus officiis : \textbf{ utputa pauperes non valentes plura habere instrumenta , } faciebant aliquod instrumentum fabricari , & assi que vn cuchiello sirue a muchos ofiçios . \textbf{ Conuiene saber que por que los pobres non podian auer } muchos instrumentos fazian fazer vn instrͤde \\\hline
2.1.15 & faciebant aliquod instrumentum fabricari , \textbf{ quo possent ad plura uti officia . } Natura autem non sic agit , & muchos instrumentos fazian fazer vn instrͤde \textbf{ que podiesen vsar en muchos ofiçies } mas la natura non faze \\\hline
2.1.15 & patet aliud esse regimen coniugale quam seruile : \textbf{ et non esse utendum uxoribus tanquam seruis . } Secunda via ad inuestigandum hoc idem , & Et pues que assi es de parte de la orden natural paresçe que otra cosa es el gouernamiento del marido ala mug̃r \textbf{ que del señor al sieruo . | Et paresçe que non deuen vsar los omes delas mugers } assi commo de sieruas ¶ \\\hline
2.1.16 & esse debitum tempus \textbf{ dare operam copulae coniugali : } verum quia vis generatiua est & terçerosetenario de los uarones es tienpo conueinble \textbf{ para dar obra al | ayuntamientod el casamiento . } Empero por que la fu erça de engendtar es muy corrupta \\\hline
2.1.17 & postquam probauit per rationes plurimas , \textbf{ non esse dandam operam coniugio in aetate nimis iuuenili : } inquirit quo tempore & por muchos razones \textbf{ que los omes non deue dar obra al casamiento } en la he perdat de grand mançebia demanda en quet pon deuen dar mas obra ala generaçion delos fijos . \\\hline
2.1.17 & quo flant venti boreales , \textbf{ melius est dare operam coniugio , } quam calido tempore quo flant australes . & en que vientan los vientos del çierco \textbf{ es meior de dar obra al casamiento . } que en el tp̃o caliente \\\hline
2.1.17 & propter roborationem caloris materni uteri , \textbf{ magis possunt conseruare suos foetus , } et eos perfectiores faciunt . & calentraa del uientre de la \textbf{ madremas pueden guardar las ceraturas } e fazer las mas fuertes \\\hline
2.1.17 & tanto tamen hoc magis decet Reges et Principes , \textbf{ quanto decet eos elegantiores habere filios . } Mulierum autem mores & e alos prinçipes \textbf{ quanto mas les conuiene aellos de auer los fijos grandes e esforcados de cuerpo } euedes saber \\\hline
2.1.18 & ut superius dicebatur . \textbf{ Ex diuersis ergo causis probare possumus mulieres uerecundas esse : } quicquid tamen sit de eius causis , & assi commo dicho es de suso . \textbf{ Et pues que assi es por muchos razones podemos prouar | que las muger sson uergonçosas . } Enpero que quier que sea destas razones \\\hline
2.1.18 & quicquid tamen sit de eius causis , \textbf{ laudabile est in ipsis esse uerecundas : } quia propter uerecundiam multa turpia dimittunt & Enpero que quier que sea destas razones \textbf{ mucho es de alabar en ellas ser uergon cosas } ca por la uerguença dexan de fazer muchs cosas torpes \\\hline
2.1.19 & dicere aliqua de regimine coniugum . \textbf{ Sciendum ergo unam esse communem regulam } ad omne regimen . & por qual gouernamiento se han de gouernar las mugers . Por ende deuemos dezir en espeçial algunas cosas del gouernamiento del casamiento . \textbf{ Et pues que assi es deuedes saber | que es vna regla comunal } para todo gouernamiento \\\hline
2.1.19 & sic suo modo est in ipsis operibus . \textbf{ Videmus enim aliquos habere linguas disertas , } aliquos vero balbutientes esse : & assi en su manera es en las obras propreas . \textbf{ Ca veemos alguon sauer las lenguas escorrechas | Et ueemos algunos ser tartamudos } e los que son tartamudos non son de vna \\\hline
2.1.19 & ( ut recitat Valerius Maximus libro II capitulo de Institutis antiquis ) \textbf{ quodammodo nefas erat bibere vinum . } Unde ait , & constitucon nes antigas entre las mugers romana \textbf{ sera grand denuesto beuer el vino . } Ende dize que el vso del uino en el tp̃o trispassado \\\hline
2.1.19 & debent per seipsos suas instruere coniuges , \textbf{ et debitas cautelas adhibere , } ut polleant bonitatibus supradictis . & por si mesmos \textbf{ e dar les castigos conuenibles } por que puedan resplandesçer en las bondades sobredichͣ̃s . \\\hline
2.1.20 & et si eas per debitas monitiones instruat . \textbf{ Declarare autem quae sunt signa amicitiae debita , } et quae sunt monitiones congruae , & por conuenibles castigos . \textbf{ Mas declarar quales son las señales conueinbles dela mistança } e quales son las moniçonnes e castigos conuenibles \\\hline
2.1.21 & non ornans se propter vanam gloriam , \textbf{ posset delinquere in ornatum , } si non esset moderata , & e non se posie con sse \textbf{ por uana eglesia podria pecar | en el conponimiento del cuerpo } si non fuese tenprada . \\\hline
2.1.22 & ostendentes nimis \textbf{ zelotypos non esse laudandos . } Primum est , quia viri in seipsis nimia turbatione vexantur . & para prouar \textbf{ que los muy çelosos non son de loar ¶ } La primera se toma de esto \\\hline
2.1.22 & circa suas coniuges nullam habere custodiam \textbf{ et nullum habere zelum , } sed consideratis conditionibus personarum , & nin les conuiene avn \textbf{ de non auer algun çelo dellas } Mas penssadas las condiconnes delas perssonas \\\hline
2.1.22 & circa propriam coniugem \textbf{ debet debitam curam , } et debitam diligentiam adhibere . & e catadas las costunbres dela tr̃ra \textbf{ ca da vno deue auer cura } e cuydado conuenible de su muger \\\hline
2.1.22 & debet debitam curam , \textbf{ et debitam diligentiam adhibere . } Sic enim decet uirum quemlibet & ca da vno deue auer cura \textbf{ e cuydado conuenible de su muger | e deue auer acuçia conuenible de su casa . } Ca assi conuiene a cada vn marido de auer çelo ordenado de su mugni \\\hline
2.1.22 & Sic enim decet uirum quemlibet \textbf{ erga suam coniugem ornatum habere zelum , } ut sit inter eos amicitia naturalis delectabilis , et honesta . & e deue auer acuçia conuenible de su casa . \textbf{ Ca assi conuiene a cada vn marido de auer çelo ordenado de su mugni } por que sea entre ellos amistança natural delectable e honesta \\\hline
2.1.24 & ut operemur illud . \textbf{ Quare cum ponere aliquid in praecepto , } sit prohibere , & para obrar aquella cosa . \textbf{ Por la qual razon commo poner alguna cosa } en poridat se a uedar \\\hline
2.1.24 & ab usu rationis deficiunt , \textbf{ nec possunt sic refraenare concupiscentias , } sunt magis propalatiuae secretorum , & por que fallesçen de vso de razon \textbf{ non pueden | assi refrenar sus cobdiçias e sus appetitos } e por ende son mas reueladoras delas poridades que los uatones . \\\hline
2.1.24 & esse constantes , \textbf{ et vincere huiusmodi impetus et inclinationes . } Nam licet sit difficile & si quisieren ser constantes e firmes \textbf{ e vençer estos appetitos natraales e estas iclinaçiones . } Ca conmoquier que sea cosa \\\hline
2.1.24 & Nam licet sit difficile \textbf{ superare incitamenta concupiscentiarum , } et sit hoc magis difficile in foeminis & Ca conmoquier que sea cosa \textbf{ guaue sobrepuiar | e vençer los entendimientos delas cobdiçias } et maguer esto sea mas guaue en las mugers \\\hline
2.2.1 & non enim sufficit patrifamilias , \textbf{ scire coniugem regere , } nisi nouerit filios debite gubernare . & en la qual ¶ diremos del gouernamiento del padre alos fijos . \textbf{ Ende non abasta al padre dela casa saber gouernar a su muger } si non sopiere gouernar conueniblemente asus fijos . \\\hline
2.2.1 & post determinationem de regimine nuptiali , \textbf{ determinandum esse de regimine seruorum . Verum quia , } ut dicitur primo Politicorum , & por ende por auentura parescria a alguno que luego despues que dixiemos del gouernamiento del casamiento \textbf{ deuiemos determinar del gouernamiento de los sieruos . } Enpero assi commo dize el philosofo en el primero delas politicas en el gouernamiento dela \\\hline
2.2.1 & dare animalibus ora et alia organa , \textbf{ per quae possunt sumere cibum et nutrimentum . } Quare si patres sunt causa filiorum , & e todos los organos e instrumentos \textbf{ por los quales puedan tomar la uianda qual les conuiene . } Por la qual cosa sy los padres son comienço \\\hline
2.2.1 & a patribus esse habent , \textbf{ decet patres habere curam filiorum , } et solicitari erga eos , & e razon de los fijos \textbf{ e los fijos naturalmente han el ser de los padres . Conuiene alos padres de auer cuydado de los fijos } e ser cuydadosos dellos \\\hline
2.2.2 & tanto maiori solicitudine et dilectione mouetur circa illud . \textbf{ Patres ergo tanto magis debent solicitari circa filios , } quanto predentiores sunt , & e con mayor amor se deue mouer a ella . \textbf{ Et por ende los padres | tanto mayor cuydado deuen auer de los fijos } quanto mas sabios son \\\hline
2.2.2 & Decet enim filios Regum et Principum \textbf{ maiori bonitate pollere quam alios : } quia secundum Philosophum in Politic’ & e de los prinçipes \textbf{ de auer mayor bondat | e mayor nobleza que los otros . } Ca segunt el philosofo enlas politicas . \\\hline
2.2.2 & et dominantur in regno . \textbf{ Utile est ergo toti regno habere bonos ciues , } sed utilius est habere bonos principantes , & e son sennors en el regno ¶ \textbf{ pues que assi es prouechosa cosa es a todo el regno | de auer bueon sçibdadanos . } Mas mas prouechosa cosa es de auer bueons prinçipes \\\hline
2.2.2 & Utile est ergo toti regno habere bonos ciues , \textbf{ sed utilius est habere bonos principantes , } eo quod principantis sit alios regere et gubernare : & de auer bueon sçibdadanos . \textbf{ Mas mas prouechosa cosa es de auer bueons prinçipes } por que alos prinçipes parte nesçe de gouernar e de garalo sots . \\\hline
2.2.2 & ex bonitate filiorum Regum , \textbf{ qui debent habere principatum et dominium in regno ; } quam ex bonitate et prudentia aliorum . & quanto mayor prouecho seleunata al regno dela bodat de los fijos delos Reyes \textbf{ que deuen auer el prinçipado | e el senorio en el regno } que dela bondat e dela sabiduria de los otros \\\hline
2.2.3 & Nam pacta et conuentiones non interueniunt inter subditum et praeeminentem , \textbf{ nisi sit in potestate subiecti eligere sibi rectorem : } non est autem in potestate filiorum eligere sibi patrem , & que caen entre el subdito e el sennar \textbf{ si non fuere en poderio del subdito | de esceger su gouernador . } Mas non es en poderio de los fijos de escoger \\\hline
2.2.3 & nisi sit in potestate subiecti eligere sibi rectorem : \textbf{ non est autem in potestate filiorum eligere sibi patrem , } si ex naturali origine filii procederent a parentibus . & de esceger su gouernador . \textbf{ Mas non es en poderio de los fijos de escoger | assi mismos padres } mas por natra al nasçençia \\\hline
2.2.3 & et propter bonum ipsorum : \textbf{ cum amare aliquod , } idem sit quod velle ei bonum , & enssennorear alos fiios realmente \textbf{ e por el bien dollos commo amar a alguno sea esso mismo } que querer qual bien . \\\hline
2.2.3 & praeesse aliquibus dominatiue , \textbf{ non intendere bonum ipsorum , } sed proprium : & Et esto es segunt el philosofo enssennarear a algunos seruilmente \textbf{ e non çibdadanamente non entender | enssennorear el bien de los sieruos } mas por el luyo propreo . \\\hline
2.2.3 & patet quod filiis debet \textbf{ praeesse pater propter bonum filiorum . } Non ergo regendi sunt filii eodem regimine , & que el padre deue \textbf{ enssennorear alos fiios | por el bien de los fijos . } Et por ende non son de gouernar los fuos \\\hline
2.2.4 & ut nobis innotescat , \textbf{ quomodo patres debeant regere filios , } et filii patribus obedire . & por que nos conosca mos \textbf{ en qual manera de una los padres gouernar alos fijos } e los fujos obedesçer alos padres¶ Et \\\hline
2.2.4 & quod ab ea comprehendi non potest . \textbf{ In toto enim est assignare aliquid , } quod multum distat a parte : & por que non puie des e conphendido della . \textbf{ Por que en el todo non se puede | sennalar alguna cosa } que es alongada mucho dela parte . \\\hline
2.2.4 & ut congregant eis bona : \textbf{ et congregare aliis bona , } et solicitari circa eorum vitam , & para allegar les los bienes \textbf{ que les faz menestra . | Commo allegar les los bienes } e ser cuydadosos \\\hline
2.2.4 & in honore et reuerentia : \textbf{ cum honorari et reuereri alium sit } quodammodo subiici illi ; & Commo honrrar \textbf{ e auer reuerençia a otro sea en alguna manera ser subiecto a el . } Por ende assi commo por el amor que han los padres alos fijos los deuen gouernar ben \\\hline
2.2.5 & quam parentes tenent . \textbf{ Si enim in aliis legibus parentes statim sunt soliciti erudire proprios filios } in iis quae sunt fidei suae , & que tienen el padre e la madre . \textbf{ Ca si e las otras leyes los padres son acuçiosos de enssennar sus fijos en aquellas cosas } que son de su fe \\\hline
2.2.6 & retrahantur a lasciuiis . \textbf{ Quare cum rationis sit concupiscentias refraenare et lasciuias , } quanto aliquis magis a ratione deficit , & por que de la razon \textbf{ e del entendimiento | es de refrenar los desseos e las locanias . } Et por ende quanto alguon mas fallesçe en razon \\\hline
2.2.7 & vix aut nunquam potest \textbf{ recte loqui linguam illam ; } et ab incolis illius terrae semper cognoscitur & que esten luengo tienpo en aquellas \textbf{ tierrasapenas o nunca pueden fablar derechamente aquella lengua . } Mas luego son conosçidos de los moradores de aqual la tierra \\\hline
2.2.7 & esse completum et perfectum , \textbf{ per quod perfecte exprimere possent naturas rerum , } et mores hominum , et cursus astrorum , & nin acabado \textbf{ por al qual pudiessen conplidamente pronunçiar las natraas delas cosas e las costunbres de los omes } e los mouimientos delas estrellas \\\hline
2.2.7 & et attentos circa ipsos , \textbf{ et peruenire ad aliquam perfectionem scientiae , } ab ipsa infantia eos tradere literalibus disciplinis . & Et si quieren \textbf{ que ellos sean acuçiosos çerca dellas e que puedan venir a alguna perfeççion | e a acabamiento de sçiençia } deuenlos luego poner en su moçedat alas letros \\\hline
2.2.8 & sed indigemus ad hoc auxilio Philosophorum et Doctorum , \textbf{ expedit nos scire idioma illud , } in quo doctores et Philosophi sunt locuti : & Mas para esto auemos men ester ayuda de los philosofos e de los doctores . \textbf{ Conuiene nos de saber e de aprender aquel lenguage } en que fablaron los doctors e los philosofos . \\\hline
2.2.8 & et per debitas rationes manifestemus propositum . \textbf{ Oportuit ergo inuenire aliquam scientiam docentem modum , } quo formanda sunt argumenta , et rationes . & i anifestamos nr̃a uoluntad e nr̃a entençion . \textbf{ Et por ende conuiene de fallar algua sçiençia | que nos mostrasse } en qual manera son de enformar los argumentos e las razones . \\\hline
2.2.8 & quod filios nobilium decet \textbf{ addiscere musicam . } Sed de his forte infra tangetur . & que conuienea los fijos de los nobles \textbf{ de aprender la musica } mas destas razones \\\hline
2.2.8 & Sexta scientia liberalis est geometria , \textbf{ quae docet cognoscere mensuras et quantitates rerum . } Ad hanc autem filii nobilium , & ¶ La sexta sçiençia çia libales geometera \textbf{ que muestra conosçer las mesuras e las quantidades delas cosas . } Et aesta eran puestos \\\hline
2.2.8 & Nam Naturalis Philosophia docens \textbf{ cognoscere naturas rerum , } longe melior est , & Ca la natural ph̃ia \textbf{ que muestra conosçer las naturas delas cosas } muy meior es \\\hline
2.2.8 & non vacat eis \textbf{ subtiliter perscrutari scientias : } maxime igitur decet & Et por que non le suaga a ellos de \textbf{ escodrinnar sotilmente las sçiençias } mucho les conuiene aellos de se auer bien cerca las cosas diuinales \\\hline
2.2.8 & inquantum deseruiunt morali negocio . \textbf{ Decet igitur eos scire grammaticam , } ut intelligant idioma literale : & en quanto siruen ala ph̃ia moral . \textbf{ Et pues que assi es conuiene les a ellos de saber la guamatica } por que entiendan el lenguage delas letras \\\hline
2.2.9 & Ad huiusmodi autem prudentiam describendam , \textbf{ licet enumerare possemus omnia illa octo } quae in primo libro de prudentia tetigimus , & e de escuir \textbf{ commo quier que la podamos contar | entre aquellas ocho cosas } que dixiemos enel primero libro dela sabiduria . \\\hline
2.2.9 & recolendo praeterita . \textbf{ Nam sicut volens rectificare virgam , } nunquam eam rectificare posset & Ca primero deue ser menbrado e acordado delas colas passadas . \textbf{ Ca assi commo aquel que quiere enderesçar la pierte } ga nunca la puede enderesçar \\\hline
2.2.9 & Secundo decet \textbf{ ipsum esse prouidum futurorum . } Nam sicut aliorum director debet & Lo segundo le conuiene \textbf{ que sea prouiso en las cosas | que han de uenir . } Ca assi commo el que ha degniar los otros \\\hline
2.2.9 & Nam sicut aliorum director debet \textbf{ cogitare praeterita , } ut sciat quomodo per tempora praeterita & Ca assi commo el que ha degniar los otros \textbf{ deue penssar | lo que es passado } por que sepa en qual manera \\\hline
2.2.9 & sic et huiusmodi doctor debet \textbf{ cognoscere particulares conditiones illorum iuuenum , } quos debet dirigere . & es mas çierto en conosçer las cosas particulares e speçiales . \textbf{ Et este dector tal deue conosçer las condiçiones speçiales delos moços } a que ha de castigar e de enssennar . \\\hline
2.2.9 & de facili ad illicita declinarent . \textbf{ Patet igitur talem quaerendum esse doctorem , } qui quantum ad scientiam speculabilium & alo que les non cunple . \textbf{ Et pues que assi es paresçe | que los moços deuen tomar e buscar tal doctor e tal maestro } quanto alas sçiençias speculatiuas \\\hline
2.2.10 & ne audiant quodcunque turpium : \textbf{ quia audire , est prope ad ipsum facere . } Ideo ergo secundum Philosophum cohibendi sunt iuuenes & que non oyan cosas torpes \textbf{ por que el oyr es muy çerca del obrar ¶ } Et pues que assi es segunt el philosofo \\\hline
2.2.10 & et pulchra , \textbf{ et indecens audire turpia : } sic decet eos audire viros bonos et honestos , & cosas honestas e fermosas \textbf{ e desconuenible de oyr cosas torpes } assi les \\\hline
2.2.10 & et indecens audire turpia : \textbf{ sic decet eos audire viros bonos et honestos , } et cohibendi sunt & e desconuenible de oyr cosas torpes \textbf{ assi les | conuienea ellos de oyr a bueons omes e honestos } e son de refrenar \\\hline
2.2.11 & et qualiter se debeant \textbf{ habere iuuenes circa ipsum . Circa cibum autem contingit } sex modis peccare , vel delinquere . & e los mançebos çerta el comer \textbf{ Mas conuiene saber | que cerça el comer } pueden los omes errar en seys maneras . \\\hline
2.2.11 & si sumatur turpiter . \textbf{ Sunt enim plurimi seipsos pascere nescientes , } quod vix aut nunquam comedere possunt , & pecan si toma la uianda torpemente e suzia mente . \textbf{ Ca son muchos que non saben gouernar assi mismos . } Los quales abeso nunca pueden comer \\\hline
2.2.13 & et vitemus delectationes illicitas , \textbf{ expedit aliquando habere aliquos ludos , } et habere aliquas deductiones & que nos non conuienen . \textbf{ Conuiene a nos algunas uegadas de auer algunos trebeios } e algunos solazes conuenibles e honestos . \\\hline
2.2.13 & Nam non semper statim \textbf{ quis habere potest finem intentum : } ne ergo propter continuos labores & Ca non puede ninguno \textbf{ sienpreauer luego la fin que entiende . } Et pues que assi es por que non fallezca el omne \\\hline
2.2.13 & antequam consequatur illum , \textbf{ ideo oportet interponere aliquos ludos , } et aliquas delectationes , & Et por que algunas vegadas establesce assi fin en que trabaia luengamente ante que alcançe aquella fin \textbf{ por ende conuienele de entroponer alguons trebeios } e algunas delectaconnes \\\hline
2.2.13 & ut vult Philosophus 7 Politicorum . \textbf{ Viso qualiter iuuenes se habere debeant circa ludos . } Restat videre , & delas politicas ¶ \textbf{ Visto en qual manera los moços se deuen auer çerca los trebeios finca de ver } en qual manera se deuen auer çerca los gestos ¶ \\\hline
2.2.13 & ne aliquem motum habeant , \textbf{ ex quo quis coniecturari possit elationem animi , } vel insipientiam mentis , vel intemperantiam appetitus . & por que non ayan algun mouimiento \textbf{ del qual alguno pueda presumir | en ellos soƀua del coraçon } o non sabidia del entendimiento \\\hline
2.2.13 & Frustra ergo , \textbf{ cum quis vult audire alium , } retinet os apertum . & Ca el omne non oye con la boca mas por el oreia . \textbf{ Et pues que assi es quando alguno quiere oyr al otro } en vano tiene la boca abierta . \\\hline
2.2.13 & Sicut ergo habent indisciplinatos gestus , \textbf{ qui cum volunt audire alios , } tenent ora aperta : & Et pues que assi es assi commo aquellos \textbf{ que quieren oyr alos otros } e tienen las bocas abiertas \\\hline
2.2.13 & ut deseruiant ad opera quae intendunt . \textbf{ Nam agere aliquos motus membrorum } non deseruientes operi intento , & que entienden fazer . \textbf{ Ca fazer alguons mouimientos de los mienbros } que non siruen ala obra \\\hline
2.2.15 & quod in omni aetate videtur esse proficuum . \textbf{ Quintum , sunt recreandi per debitos ludos , } et sunt eis recitandae aliquae historiae , & Et esto es prouechoso en todas las hedades ¶ \textbf{ La quinta es que lon de recrear | por trebeios conuenibles . } Et deuen rezar ante ellos algunas bueans estorias . \\\hline
2.2.15 & et hoc maxime , \textbf{ cum incipiunt percipere significationes verborum . } Sextum , a ploratu sunt cohibendi . & Et esto les es prouechoso mayormente \textbf{ quando comiençan a entender las significa connes delas palabras . } ¶ La sexta es que deuen ser guardados de llorar . \\\hline
2.2.15 & ut plurimum pascuntur lacte \textbf{ assuescant bibere vinum . } Immo dicunt aliqui , & e se fazen de mala disposicion enel cuerpo \textbf{ si en el tienpo en que manian se acostunbraren a beuer vino } Et dize algs \\\hline
2.2.15 & et tantillos ad solidandum membra , \textbf{ et ad non defluere propter teneritudinem : } moderatum enim motum in pueris adeo laudat Philosophus , & para soldar los mienbros \textbf{ por que non los dexen caer | por que son tiernos . } Mas el mouimiento tenprado en los mocos \\\hline
2.2.15 & ut ab ipso primordio natiuitatis dicat , \textbf{ fienda esse aliqua instrumenta , } in quibus pueri vertantur , & en tanto lo alaba el philosofo que diz que luego enł comienço de su nasçençia \textbf{ deuen fazer alguons instrumentos } en que se mueun a los moços \\\hline
2.2.15 & postquam incipiunt \textbf{ percipere significationes verborum . } Vel etiam aliqui cantus honesti & Otrossi avn deuen rezar alos moços alguas estorias despues que comiençan at entender las significaçiones delas palabras . \textbf{ Et avn deuen les dezir algunos cantos } ca los cantos honestos son de cantar alos moços \\\hline
2.2.16 & quod pessimum est \textbf{ non instruere pueros ad virtutem , } et ad obseruantiam legum utilium . & dizeque muy mala cosa es de non enssennar \textbf{ e de non enduzir los mocos a uirtudes } e aguardar las leyes bueans e aprouechosas . \\\hline
2.2.17 & Sed a septimo usque ad quartumdecimum \textbf{ quia iam incipiunt habere concupiscentias aliquas illicitas , } et aliquo modo & Mas desde los siete años fasta los xiiij̊ . años \textbf{ por que ya comiencan a auer algunas cobdiçias desordenadas . | Et en alguna manera comiençan a partiçipar vso de razon } e de entendimiento \\\hline
2.2.17 & Sed a quartodecimo anno , \textbf{ quia tunc perfectius participare incipiunt rationis usum , } non solum curandum est & Mas despues del . xiiij̊ año \textbf{ por que estonçe comiençan a partiçipar | de vso de razon e de entendimiento } mas acabadamente non tan solamente deuen auer cuydado los padres de los fijos \\\hline
2.2.17 & assuescendi sunt ad labores leues : \textbf{ sed deinde debent assumere labores fortes . } Adeo enim secundum ipsum a quartodecimo anno & que fasta los . xiiij años los moços deuen ser acostunbrados a trabaios ligeros \textbf{ mas dende adelante se deuen acostunbrar a trabaios mas fuertes . } Et en tanto que segunt el philosofo desde los . \\\hline
2.2.17 & et in aliis quae ad militiam requiruntur , \textbf{ subire possint labores militares : } tunc enim est quis bene dispositus quantum ad corpus , & que pertenesçen ala caualłia \textbf{ estonçe se pue den poner alos trabaios dela caualłia } por que estonçe es alguno bien ordenado \\\hline
2.2.17 & habeant corpus sic dispositum , \textbf{ ut possint tales subire labores , } ut per eos respublica possit defendi . & en que la tierra aya meester defendimiento ayan el cuerpo o bien ordenado \textbf{ por que puedan tomar trabaios } e pueda defender la tierra . \\\hline
2.2.17 & ut habeant sic bene dispositum corpus , \textbf{ ut possint debitos subire labores , } quod maxime fieri contingit , & por que ayan el cuerpo bien ordenado \textbf{ por que puedan tomar trabaios conuenibles la qual cosa } mayormente se pue de fazer \\\hline
2.2.17 & si ad debita exercitia assuescant . \textbf{ Viso , quomodo a quartodecimo anno ultra solicitari debent patres erga filios , } ut habeant dispositum corpus . & e amouimientos conuenibles ¶ \textbf{ Visto en qual manera del . xiiij . año adelante deuen los padres auer cuydado de los fijos } por que ayan el cuerpo bien ordenado \\\hline
2.2.17 & Restat videre , \textbf{ quomodo solicitari debeant circa eos , } ut habeant ordinatum appetitum . & por que ayan el cuerpo bien ordenado \textbf{ finca de ver en qual manera de una auer cuydado dellos } por que ayan el appetito bien ordenado . \\\hline
2.2.17 & quia cum ex tunc incipiant \textbf{ habere perfectum rationis usum , } videtur eis quod digni sint dominari , & Conuiene a saber quanto al orgullo e ala locama . \textbf{ paresce que estonçe comiençan a auer vso de razon acabada } paresçe les que son dignos de enssennorear e de ser senneres \\\hline
2.2.18 & per quam impeditur mentis sublimitas . \textbf{ Eos autem qui debent regere regnum , } magis expedit esse prudentes , & por la qual cosa se enbarga la sotileza del entendimiento . \textbf{ Et aquellos que deuen gouernar el regno } mas les conuiene de ser sabios \\\hline
2.2.18 & nec sic debeant \textbf{ fugere corporales labores ; } ut effecti quasi muliebres , & que los Reyes e los prinçipeᷤ non de una de todo dexar el vso delas armas \textbf{ nin de una assi escusar los trabaios del cuerpo } por que le fagan mugeriles \\\hline
2.2.18 & qui debent alios regere , \textbf{ vitare inertiam et solicitudinem illicitam , } vacando moralibus scientiis , & ¶ Et pues que assi es conuiene aquellos \textbf{ que deuen gouernar los otros de escusar la ꝑeza } e el cuydado desconueinble estudiando enlas sçiençias morales \\\hline
2.2.19 & ad conseruandam puritatem et innocentiam , \textbf{ est vitare commoditates malefaciendi , } propter quod et prouerbialiter dicitur , & e la inoçençia de non pecar \textbf{ e para guardar las maneras de mal Razer } por la qual cosa se dize vn prouerbio \\\hline
2.2.19 & in quibus est ratio praestantior , \textbf{ est magnum periculum non vitare commoditates delictorum : } multo magis hoc est in foeminis , & e el entendimiento mayor es grant peligro \textbf{ de non escusar las azinas de los pecados much mas es esto de escusar en las mugers . } Et avn mas es en las fiias e en las moças \\\hline
2.2.19 & ex virorum consortio . \textbf{ Tollere autem a puellis verecundiam , } est tollere ab eis fraenum , & por la conpannia de los uarones \textbf{ mas toller alas moças la uerguença es toller el freno dellas } por el qual freno se retrahen \\\hline
2.2.20 & Texere enim et filare , \textbf{ et operari sericum , } satis videntur opera competentia foeminis . & segunt el departimiento delas perssonas \textbf{ cateyer e filar e obrar e coser e taiar algunas cosas sotiles } asaz paresçen obras \\\hline
2.2.21 & Ostenso , \textbf{ quod non decet puellas esse vagabundas , } nec decet eas viuere otiose : & que las mugers fuesen acuçiosas . \textbf{ ostrado que non conuiene alas moças de andar uagarosas a quande e allende } nin les conuiene de beuir ociosas \\\hline
2.2.21 & Decet ergo ipsas \textbf{ per debitam taciturnitatem adeo examinare dicenda , } ut nec dicant aliqua , & e por ende les conuiene aellas de ser callanţias en manera conuenible \textbf{ e en tanto examinar las cosas | que han de dezer } por que non digan alguas cosas \\\hline
2.3.1 & eo quod hae materiae sunt connexae , \textbf{ intendimus instruere uolentem suas domus debite gubernare , } non solum quantum ad regimen ministrorum et familiae , & por que estas materias son ayuntadas en vno entendemos de enssennar \textbf{ a aquellos que quisieren | conueinblemente gouernar sus calas } non lo lamente quanto al \\\hline
2.3.1 & quomodo deceat \textbf{ ipsos se habere circa possessiones , } et numismata , & e generalmente todos los çibdadanos . \textbf{ Et en qual manera se deuan auer çerca las possesiones } e çerca las riquezas e los dineros \\\hline
2.3.1 & per quae opera sua complere possit . \textbf{ Volens ergo tradere notitiam de arte fabrili , } oportet ipsum determinare de martello , et incude , & por los quales pueda conplir sus obras . \textbf{ Et por ende los que quieren dar conosçimiento dela arte del ferrero } conuiene les de determinar del martiello e dela yunque \\\hline
2.3.1 & et spectat ad fabrum talia instrumenta cognoscere . \textbf{ Sic volens tradere notitiam de arte textoria , } debet determinare de pectinibus , & cognosçertales estrumentos . \textbf{ Et dessa misma manera | el que quiere dar conosçimiento del arte del texer } deue determinar de los peinnes \\\hline
2.3.1 & quod spectat ad gubernatorem domus \textbf{ scire debite se habere } circa ministros et seruos : & Mas por que aquellas mismas razones sen podia prouar \textbf{ que parte nesçe al gouernamiento dela casa saber se auer } conueinblemente çerca los ofiçiales \\\hline
2.3.3 & hoc viso opinatur \textbf{ principem esse tantum , quod quasi impossibile sit ipsum inuadere : } et quia circa impossibilia non cadit electio neque consilium , & quando esto vee pienssa en su coraço \textbf{ que el prinçipe es tan grande | que en ninguna manera non podria yr contra el } e por que en las cosas \\\hline
2.3.3 & ne in contemptum habeantur a populo , \textbf{ facere aedificia magnifica , } prout requirit decentia status , & enpero conuiene alos Reyes \textbf{ e alos prinçipes de fazer moradas costosas e nobles } assi commo el su estado demanda \\\hline
2.3.5 & quod dominetur istis sensibilibus , \textbf{ et quod possit eis uti in suum obsequium , } et quia hoc est quodammodo possidere ea , & que enssennore e a estas cosas senssibles \textbf{ e que pueda vsar dellas | e resçebir seruiçio dellas } segunt quel fuere uisto \\\hline
2.3.5 & quia statim solicita est \textbf{ inducere lac in mamillis matris , } ut ex eo animalia genita nutriri possint . & quanto al nutermiento dellas \textbf{ por que luego que nasçe es acuçiosa de aduzer leche enlas teras delas madres } assi que de aquella lech̃e las aian las enrendradas se pueden cerar \\\hline
2.3.5 & ut homo est , \textbf{ ut vult Philosophus primo Polit’ habere possessionem , } et dominium aliquarum rerum exteriorum & en quanto es omne \textbf{ segunt dize el philosofo | enl primero libro delas politicas de auer possession } e sennorio de algers cosas de fuera \\\hline
2.3.7 & ordinauit enim ea ad usum et dominium nostrum ; \textbf{ licitum est ergo sumere nutrimentum ex agris , } et animalibus domesticis & e las ordeno a vso e añro sennorio . \textbf{ ¶ Et pues que assi es cosa conuenible es | de tomar nudermiento delos canpos } e delas aianlias de casa \\\hline
2.3.7 & talia facere , \textbf{ et ordinare ea in usum proprium . } Furtiua autem vita & por \textbf{ sitałs̃aianlias e ordenar las asu uso propreo } Mas la uida de furtar fablado sinplemente de ssi es desconueible \\\hline
2.3.8 & medicus ergo sanitatem quasi appetit \textbf{ inducere infinitam , } sed potionem appetit dare & Et pues que assi es el fisico \textbf{ dessea aduzir salud sin mesura e sin fin } mas la melezina dessea de dar segunt manera \\\hline
2.3.8 & Sed quod ad gubernationem domus pertineat \textbf{ non appetere infinitas possessiones , } duplici via venari possumus . & Mas que pertenezca al gouernamiento dela casa \textbf{ non dessear las riquesas e las possessiones sin mesura } e sin fin esto podemos mostrar \\\hline
2.3.8 & ergo nec gubernatiua debet \textbf{ quaerere infinitas possessiones . } Decet igitur omnes ciues & enł primero libro delas politicas . \textbf{ Et pues que assi es nin el arte del gouernamiento dela casa non deue demandar possessiones et riquezas sin mesura e sin fin . } ¶ Et por ende conuiene a todos los çibdadanos \\\hline
2.3.8 & et Principibus quam in aliis , \textbf{ quanto decet habere ordinatiorem uoluptatem , } et meliorem aestimationem finis : & que en los otros \textbf{ quanto mas conuiene aellos de auer mayor ordenamiento dela uoluntad } e meior estimacion dela finca \\\hline
2.3.8 & detestabilius est in Rege \textbf{ non habere ueram aestimationem } de fine quam in populo , & mas de denostares enl Rey \textbf{ de non auer uerdadera } estimaçonn dela fin \\\hline
2.3.8 & sicut detestabilius est in sagittante \textbf{ non cognoscere signum , } quam in sagitta : & mas de denostar es enł liallero \textbf{ de non conosçer la señal } que en la saeta \\\hline
2.3.9 & uel per se uel per procuratores intermedios , \textbf{ nam ipsius patrisfamilias est totam indigentiam subleuare domesticam . } Sed cum eiusdem ad seipsum non sit & o por si o por sus procuradores entre medianos . \textbf{ Ca al padre familias parte nesçe dereleuar toda la menguadela casa . } Mas por que non puede ser conpra \\\hline
2.3.9 & quod habetur in toto regno , \textbf{ oportuit introduci commutationem rerum ad denarios , } et econuerso . & que hades es en todo el regno \textbf{ conuiene de poner m̃udaçion delas cosas alos dineros } e de los diueros alas cosas \\\hline
2.3.9 & et utile , \textbf{ pro quo inueniri possent victualia . } Huiusmodi autem maxime est argentum , et aurum , & e que fues fermosa e aprouechable \textbf{ por que se podiessen fallar las uiandas . } Mas entre todas las otras cosas \\\hline
2.3.9 & ut volentes habere tantum vini , \textbf{ oportebat dare tantum ponderis argenti , vel auri , } vel etiam alterius metalli , & assi que los que quirien auer tunerto de vino \textbf{ conuimeles a dar tanto de peso de plata o de oro o avn de otro metal } assi commo plazia de establesçer en aquel tienpo alos pueblos e alos Reyes . \\\hline
2.3.9 & pro quo statim \textbf{ secundum ipsius valorem recipere possumus supplentia indigentiam vitae . } In toto ergo uno regno & segunt el ualor de aquellas cosas \textbf{ que cunplen | para cunplir la mengua dela uida . } Et pues que assi es por que non fuessen muy agua uiados los omes \\\hline
2.3.9 & et quomodo sunt inuentae , \textbf{ et quae fuit necessitas inuenire denarios . } Decet ergo prudentem patremfamilias , & e por que son falladas \textbf{ et qual fue la neçessidat | para fallar los des } e por ende conuiene al sabio padre familias \\\hline
2.3.10 & quae fuit necessitas \textbf{ inuenire numismata et pecuniam , restat dicere , } quot sunt species pecuniatiuae . & Et pues que assi es despues que dixiemos qual fue la neçessidat de fallar las monedas \textbf{ e los dineros finca de dezer quantas son las maneras de los dineros . } Et el philosofo en las politicas \\\hline
2.3.11 & sicut duplici nomine nominatur , \textbf{ sic duplici via inuestigare possumus eam detestabilem esse . } Vocatur enim primo denariorum partus , & talzes assi commo ella ha dos nonbres \textbf{ assi podemos prouar | por dos razones que ella es de denostar . } Ca primeramente la llamamos parto de dineros \\\hline
2.3.11 & ut aliud est domus , \textbf{ et aliud inhabitare ipsam : } in aliquibus & que otra cosa es la casa \textbf{ e otra cosa es morar enella } enpero en alguas cosas nunca se puede otorgar el uso dellas sinon \\\hline
2.3.11 & possessor domorum potest \textbf{ concedere usum domus } ut inhabitationem & et non enagenar la casa \textbf{ el señor dela casa puede otorgar el uso dela casa } para morar sin que otorgue la sustançia della . \\\hline
2.3.11 & est expendere \textbf{ et alienare denarios nunquam ergo potest } concedi usus proprius denarii , & assi ca el uso propo de los dineros es despender los e enagenar los . \textbf{ Et por ende nunca se puede otorgar el uso propreo de los dineros } si non se otorgare la sustançia \\\hline
2.3.11 & eius est usus . \textbf{ Volens ergo accipere pensionem de usu denariorum , } dicitur committere usuram , & que cuya es la sustaçia del esalulo della . \textbf{ Et pues que assi es el que quiere tomar ganançia de lisso | de los dineros dezimos } que comete usura \\\hline
2.3.11 & uel dicitur usurpare , \textbf{ et rapere ipsum usum : } quia concedendo usum denarii , & que comete usura \textbf{ e tal es dichusurar e robar uso } por que los que otorgan el uso del dinero \\\hline
2.3.11 & si volunt naturaliter Dominari , \textbf{ prohibere usuras , } ne fiant eo & si quisieren ser señors natalmente \textbf{ de defender las usuras } que non se fagan \\\hline
2.3.12 & vidit per astronomiam , \textbf{ futuram esse magnam copiam oliuarum : } et ab omnibus incolis regionis illius emit tantum oleum , & que aquel anero \textbf{ que auie de uenir | que auie de ser grant cunplimiento de oliuas e de olio . } Et el por ende conpro todo el olio \\\hline
2.3.12 & tum quia nullus poterat \textbf{ vendere oleum , } nisi ipse : & por todo el olio que auie de venir . \textbf{ Et lo vno por que ninguon non podie vender olio } si non el solo . \\\hline
2.3.12 & ( secundum Philos’ ) \textbf{ est facere monopoliam , } idest facere vendationem unius : & por que segunt el philosofo \textbf{ entre todas las cosas | que acresçientan las riquezas es fazer monopolia } que quiere dezer vendiconn de vno solo . \\\hline
2.3.12 & est facere monopoliam , \textbf{ idest facere vendationem unius : } nam quia unus solus vendit , & que acresçientan las riquezas es fazer monopolia \textbf{ que quiere dezer vendiconn de vno solo . } Ca quando vno solo uende taxa el preçio \\\hline
2.3.12 & secundum vitam politicam volentem prouidere indigentiae domesticae , \textbf{ habere curam de acquisitione pecuniae , } secundum quod exigit suus status : & que quiere proueer ala mengua dela casa \textbf{ segunt uida politica de auer cuydado de ganar dineros segunt que requiere } e demanda el su estado de cada vno . \\\hline
2.3.12 & Quare decet Reges , \textbf{ et Principes habere homines industres } tam super cultura agrorum et vinearum , & Por la qual cosa conuiene alos Reyes e alos principes \textbf{ de auer omes acuçiosos } e sabidores tan bien sobre las lauotes de los canpos \\\hline
2.3.12 & sicut alicubi consuetudo est \textbf{ habere multitudinem columbarum vel aliarum auium , } ex quibus domestica alimenta sumuntur . & assi commo veemos \textbf{ que en algunos logars han costunbres de auer palomares | e muchedunbre de palomas } e de otras aues delas quales son tomados gouernamientos para la casa \\\hline
2.3.12 & viuere melior est vita peregrina : \textbf{ et habere alimenta ex propriis , } laudabilius est , & çibdadanamente es meior que la uida pelegnina \textbf{ e auer uiandas de propreo es mas de loar } que çonprar cada vna cosa \\\hline
2.3.13 & ut si plures voces efficiunt aliquam harmoniam , \textbf{ oportet ibi dare aliquam vocem praedominantem , } secundum quam tota harmonia diiudicatur . & Assi commo si muchas uozes fiziess en alguna armonia o concordança de canto . \textbf{ Conuerna de dar y alguna bos | que enssennoreasse sobre las otras } segunt la qual serie iudgada toda aquella concordança delas uozes delas otras avn en essa misma manera \\\hline
2.3.14 & propter commune bonum oportuit \textbf{ dare leges aliquas positiuas , } secundum quas regentur regna et ciuitates : & connino de dar \textbf{ e de fazer alg̃s leyes pointiuas } legunt las quales se gouernassen los regnos e las çibdades \\\hline
2.3.14 & secundum quam ignorantes debent seruire sapientibus , \textbf{ esset dare seruitutem legalem , } et quasi positiuam , & e sin sabiduria deuen puir a los sabios . \textbf{ es de dar serudunbre legal de ley puesta por los omes } segunt la qual los flacos e los vençidos \\\hline
2.3.15 & Principaliter tamen in ministerio debet \textbf{ quis intendere bonum : } si autem intendat & e en tal scruiçio \textbf{ deue cada vno entender algun bien } mas si entiende y auer alguna merçed \\\hline
2.3.15 & hoc debet esse ex consequenti . \textbf{ Oportuit autem dare ministrationem conductam et dilectiuam } praeter ministrationem naturalem & tenporal esto deue ser despues de aquel bien que entiende . \textbf{ Mas conuiene de dar a ministraçion de alquiler e de amor sin la ministt̃ion natural et segunt ley . } Ca por que en nos es el appetito corrupto \\\hline
2.3.15 & ut ex eo possent \textbf{ acquirere aliquos ancillantes et seruos : } ne ergo tales omnino priuentur ministris , & non fazen ningua batalla iusta \textbf{ por que por ella puedan ganar algunos seruientes e sieruos . } Et pues que assi es por que tales del todo non sean priuados de seruientes \\\hline
2.3.16 & praeficiendus est unus architector ministris illis , \textbf{ cuius sit solicitare et ordinare illos . } Est autem hoc documentum maxime necessarium & mayoral que sea ordenador e mandador de todos los seruientes \textbf{ a quien parte nezca de acuçiar | e de ordenar todos los otros } Et esta regla es muy neçessaria \\\hline
2.3.16 & eo quod unus non sufficeret \textbf{ exequi opus illud . } Est igitur in commissione officiorum & por que vno non cunpliria \textbf{ para fazer aquel oficio e aquella obra . } ¶ Pues que assi es en a comne dar estos ofiçios \\\hline
2.3.16 & non magnam curam habent annexam , \textbf{ congregari possunt officia et magistratus , } ita quod eidem diuersa officia committantur . & non han grand cura anexa \textbf{ pueden se muchos ofiçios | e muchos maestradgos ayuntar en vno . } Assi que avna perssona sean acomnedados departidos ofiçios \\\hline
2.3.16 & ne per insipientiam defraudentur . \textbf{ Fidelitas autem cognosci habet per diuturnitatem : } ipsum enim cor hominis videre non possumus ; & por non saber . \textbf{ Mas la fiesdat se puede conosçer | por luengo tienpo } por que nos non podemos ver el coraçon del omne \\\hline
2.3.18 & quasi omnis virtus concomitari debet . \textbf{ Possumus enim distinguere duplicem nobilitatem : } unam secundum opinionem , & por que toda uirtud deue en algua manera ser aconpannada ala nobleza delas costunbres . \textbf{ Et nos podemos departir en dos maneras la nobleza ¶ } Vna segunt opinion de los omes \\\hline
2.3.18 & in populo progenitores suos fuisse pauperes , \textbf{ dicitur habere nobilitates generis , } et per consequens est nobilis & nin los sus auuelos fueron pobres \textbf{ e estos tales son dichos | auer nobleza de linage } e por ende se sigue \\\hline
2.3.18 & nec quod ex hoc velint \textbf{ implere legem hoc precipientem , } quod facit iustus legalis : & nin otrossi non lo faze \textbf{ por que quiera cunplir la ley } que lo manda la qual cosa faze el iusto legal . \\\hline
2.3.18 & quod facit iustus legalis : \textbf{ sed quia volunt retinere mores curiae et nobilium , } quos decet datiuos esse ; & que lo manda la qual cosa faze el iusto legal . \textbf{ Mas por que el quiere retener las costunbres dela corte | e de los no nobles omes alos } que les conuienne de ser dadores \\\hline
2.3.18 & eo quod sunt in maximo nobilitatis gradu , \textbf{ habere mores nobiles et curiales , ministros , } quos in bonis decet suos dominos imitari , & por que son en muy grand grado de nobleza \textbf{ auer buenas costunbres | e de ser curiales e nobles } assi conuiene alos seruientes dellos \\\hline
2.3.19 & Nam sicut decet ciues \textbf{ ut debitam politiam seruent } esse iustos legales , & ca assi conmo conuiene alos çibdadanos de ser iustos e legales \textbf{ para guardar su poliçia conueniblemente } assi conuiene alos sermient \\\hline
2.3.19 & esse \textbf{ magnanimos decet operari pauca et magna , } ut decet ipsos solicitari & ¶ Et pues que assi es alos Reyes \textbf{ e alos prinçipes alos quales couiene de auer altos coraçones | conuiene les de obrar pocas cosas } e grandes ca les conuiene \\\hline
2.3.19 & nullatenus decet ipsos . Hoc viso restat \textbf{ ostendere tertium , } videlicet qualiter & ca esto ꝑtenesçe alos menores . \textbf{ ¶ Esto iusto finça de demostrar lo terçero } que es en qual manera han de beuir los sennores con sus ofiçiales \\\hline
2.3.19 & sed ad eos qui sunt in dignitatibus decet \textbf{ magnanimos ostendere se magnos . } Reges ergo et Principes , & mas a aquellos que son en grandes dignidades \textbf{ los magnanimos se deuen mostrar grandes . } ¶ Et pues que assi es los Reyes \\\hline
2.3.19 & minus se exhibere quam caeteros , \textbf{ et ostendere se esse personas magis graues } et reuerendas quam alios , & que los otros \textbf{ e de se mostrar | que son personas mas pesadas } e de mayo rreuerençia \\\hline
2.3.20 & quos decet maxime temperatos esse , \textbf{ et obseruare ordinem naturalem } omnino in suis mensis , & e los prinçipeᷤ alos quales conuiene ser muy tenprados \textbf{ e guardar la orden natural en toda } meranera deuen ordenar en sus mesas \\\hline
2.3.20 & etiam \textbf{ et ipsos participare virtutes et bonos mores . } Sed si recumbentes , & Mas alos que son assentados en las mesas \textbf{ conuiene de escusar muchedunbre de palabras } por que non sea tirada la ordenn natural \\\hline
3.1.2 & quam communitas illa . \textbf{ Non sufficit dicere ciuitatem constitutam } esse gratia alicuius boni , & que la comunidat dela çibdat \textbf{ on a basta de dezer | que la çibdat es establesçida } por gera de algun bien \\\hline
3.1.2 & aliqua perfectio competens suae speciei , \textbf{ licet possit habere illa res esse aliquod , } ut imperfectum esse : & que pertenezca ala suspeno ala su semeiança \textbf{ commo quier que puede auer aquella cosa } aquel ser menguado en alguna manera . \\\hline
3.1.2 & si habeant perfectiones competentes propriae speciei : \textbf{ esse tamen virtuosum habere non possunt , } quia nequeunt participare virtute . & que parte nesçen ala su semeiançaprop̃a . \textbf{ Enpero non puede auer el ser uirtuoso } por que non puede partiçipar la uirtud ¶ Et pues que assi es en aquella manera \\\hline
3.1.2 & non tamen dicitur sufficienter viuere , \textbf{ et habere vitam sufficientem , } nisi habeat ea quae congrue sufficiunt & que el omne aya el ser biue enpero non es dicho beuir conplidamente \textbf{ e auer uida conplida } si non ouiere aquellas cosas \\\hline
3.1.4 & per quas illa ueritas confirmetur : \textbf{ intendimus in hoc capitulo adducere rationes ostendentes ciuitatem esse quid naturale , } et hominem esse naturaliter animal ciuile . & por las quales la uerdat sea confirmada . \textbf{ Por ende entendemos en este capitulo de adozir razones | que muestren } que la çibdat es cosa natural \\\hline
3.1.5 & Immo tanto principalius debet \textbf{ intendere hoc quam illud , } quanto anima est potior corpore , & e uirtuosamente ante el que faze la ley \textbf{ cantomas prinçipalmente deue tener mientes a esto } que aquello quante el alma es meior que el cuerpo \\\hline
3.1.5 & ut melius possit \textbf{ resistere impugnationem hostium : } cum ergo regnum sit & por que pueda meior cotra dezir \textbf{ e con tristar alos enemigos | que la conbaten . } ¶ Et pues que assy es commo el regno sea \\\hline
3.1.5 & cuius est quemlibet partem regni defendere , \textbf{ et ordinare ciuilem potentiam aliarum ciuitatum } ad defensionem cuiuslibet ciuitatis regni ; & so vn Rey aqui pertenesçe de defender a cada vna parte del regno \textbf{ e ordener el poderio çiuil delas otras çibdades } a defendimiento de cada vna delas çibdades del regno \\\hline
3.1.6 & magis pacifice viuere , \textbf{ et magis resistere hostibus volentibus impugnare ipsos . } Est enim huius impetus naturalis : & por que por tal establesçimiento pueden beuir mas en paz \textbf{ e pueden mas defender se de los enemigos | que les quieren mal fazer } et esta tal inclinaçion es natural \\\hline
3.1.6 & quorum quilibet dici potest naturalis : \textbf{ possumus addere modum tertium , } qui quasi est simpliciter violentus . & e del tegno delas \textbf{ quales cada vna puede ser dichͣ natural Podemos eñader la terçera manera que es sinplemente } assi commo manera forcada \\\hline
3.1.6 & et ut facilius eis dominaretur , \textbf{ poterat eos uiolenter congregare in unum , } et constituere inde ciuitatem . & Et por que mas ligeramente \textbf{ sennoreasse sobre ellos podrie ayuntarlos en vno | por fuerça } e establesçer ende çibdat . \\\hline
3.1.6 & et ad pacifice uiuere , \textbf{ et ad resistendum uolentibus turbare pacem , } et impugnare ciues . & otdenadonatanlmente a bien beuir e a beuir en paz \textbf{ e para yr | contra los que quisieren turbar la paz } e qualieren lidiar contra los çibdadanos \\\hline
3.1.6 & Ostendendum est ergo qualiter possit \textbf{ bene regi ciuitas siue regnum tempore pacis , } et qualiter impugnandi sint hostes tempore belli . & qual es la meior manera de gouernamiento de çibdat e de regno \textbf{ e en qual manera se puede bien gouernar la çibdat | e el regno en tp̃o de paz } e en qual manera deuemos lidiar contra los enemigos entp̃o de guerra . \\\hline
3.1.6 & quomodo debeant \textbf{ regere ciuitates et regna . } Secundo ostendetur , & por que por el conosçimiento dellos sean endozidos los Reyes e los prinçipes \textbf{ en qual manera de una gouernar las çibdades e los regnos ¶ } Lo segundo mostraremos qual es la muy buean politica o çibdat o muy vuen regno \\\hline
3.1.8 & secundum suum statum sit maxime perfectum , \textbf{ oportet ibi dare diuersa secundum speciem . } Nam quia tota bonitas uniuersi non potest & e por que el mundo segunt su estado sea muy acabado \textbf{ conuietie de dar en el | departidas speçias } e departidas semeianças \\\hline
3.1.8 & reseruari in una specie , \textbf{ oportet ibi dare species diuersas ; } ut in pluribus speciebus entium reseruetur maior perfectio , & nin en vna semeiança \textbf{ conuiene de dar | y deꝑ tidas espeçies } e departidas semeianças \\\hline
3.1.8 & esse perfectum , \textbf{ oportet dare diuersitatem aliquam , nec oportet ibi esse } omnimodam conformitatem et aequalitatem , & para que aya ser acabada \textbf{ conuiene de dar ay algun departimiento | nin conuiene de ser } y en toda manera confirmada egualdat \\\hline
3.1.8 & Dicere ergo in ciuitate \textbf{ vel in regno esse debere omnem unitatem , } est dicere ciuitatem & e si se estendiere a mayor vnidat paresçra el ser dela çibdat . \textbf{ Et pues que assi es dezer que enla çibdat o en el regno deua ser tan grant vnidat } commo dizian socrates e platones dezer que la çibdat non sea çibdat \\\hline
3.1.8 & est dicere ciuitatem \textbf{ non esse ciuitatem , } et regnum non esse regnum . & Et pues que assi es dezer que enla çibdat o en el regno deua ser tan grant vnidat \textbf{ commo dizian socrates e platones dezer que la çibdat non sea çibdat } e el regno non sea regno . \\\hline
3.1.8 & oportet in ciuitate \textbf{ dare diuersitatem aliqua , } ut in ea reperiatur sufficientia ad vitam . & auemos mester casas e uestid̃as e viandas e otras cosas tales \textbf{ por ende conuiene de dar algun departimiento en la çibdat por que en ella sean falladas todas las cosas } que cunplen ala uida . \\\hline
3.1.8 & ut cum in ciuitate oporteat \textbf{ dare aliquos magistratus , } et aliquas praeposituras , & delos çibdadanos a algun prinçipe o algun sennor \textbf{ e commo en la çibdat conuenga de dar alguons ofiçioso } alguons maestradgos o algunas alcaldias \\\hline
3.1.8 & oportet in ciuitate \textbf{ dare diuersitatem aliquam . } Quinta uia sumitur & por ende commo estas cosas demanden departimiento \textbf{ conuiene de dar en la çibdat algun departimiento . } La quanta razon se toma \\\hline
3.1.8 & Nam finis ciuitatis est bene viuere , \textbf{ et habere sufficientiam in vita ; } nam ciuitas est communitas & por conpaçion dela finca la fin dela çibdat es bien beuir \textbf{ e auer abastamiento en la uida } ca la çibdat escomunidat \\\hline
3.1.8 & ideo oportet ciuitatem \textbf{ habere aliquam diuersitatem in se , } et diuersos habere vicos , & para abastamiento deuida son meester muchͣs cosas departidas \textbf{ por ende conuiene enla çibdat de auer en ssi algun departimiento } e de auer departidos uarrios \\\hline
3.1.8 & habere aliquam diuersitatem in se , \textbf{ et diuersos habere vicos , } ut expediens ad vitam & por ende conuiene enla çibdat de auer en ssi algun departimiento \textbf{ e de auer departidos uarrios } assi que abonden a la uida \\\hline
3.1.8 & sed ad rectam consonantiam oportet \textbf{ ibi dare diuersitatem tonorum . } Sic pictura nunquam est bene ordinata , & mas ala derecha consonançia delas bozes \textbf{ conuiene de dar y departimiento de los tonos } assi commo la pintura non es bien ordenada \\\hline
3.1.8 & Decet ergo hoc Reges , et Principes cognoscere , \textbf{ quod nunquam quis bene nouit regere ciuitatem , } nisi sciuerit qualiter constituitur ; & e alos prinçipes de sabesto \textbf{ por que munca ninguno sopo bien gouernar çibdat } si non sopiere en qual manera es establesçida la çibdat \\\hline
3.1.9 & diu inuestigandum est , \textbf{ qualiter ciuitatem oportet esse unam , } et quam diuersitatem habere debet , & muy luengamente es de buscar \textbf{ e de escodrinnar | en qual manera la çibdat conuiene de ser vna } e qual departimiento deue auer enlła \\\hline
3.1.9 & et quasi quaedam praeambula ad sequentia . Volumus autem in hoc capitulo ostendere , \textbf{ quod non expedit ciuitati habere omnia communia } ut Socrates ordinauit : & mas nos queremos en este capitulo mostrar primeramente \textbf{ que conuiene ala çibdat | quer todas las cosas comunes } assi commo socrates ordeno . \\\hline
3.1.9 & quia quilibet crederet \textbf{ plus esse accepturum : } ut dum unus ciuis iudicaret & por que cada vno cuydaria \textbf{ que deuia mas resçebir de } quanto resçibe camientra \\\hline
3.1.10 & vel propter paucos pueros velle magnam multitudinem diligere puerorum tanquam proprios filios , \textbf{ hoc est ponere parum de melle in multa aqua . } Sicut ergo parum mellis totum unum fluuium & assi conmo a fijos propreos \textbf{ esto es poner poco de miel en muchͣ agua . } Et pues que assi es \\\hline
3.1.10 & Sicut ergo parum mellis totum unum fluuium \textbf{ non posset facere dulcem , } sic amor duorum & assi commo poca miel puesta en vn grant rio \textbf{ non puede fazer todo el rio dulçe } assi amor de dos o de tro fiios \\\hline
3.1.10 & innumerabilem multitudinem puerorum existentium in ciuitate una , \textbf{ non posset reddere placibilem et dilectam . } Sed non existente dilectione ciuium ad pueros , & de que son en vna çibdat \textbf{ nin puede fazer | que aquella muchedunbre sea plazible } e amada non estando el amost de los çibdadanos alos moços \\\hline
3.1.10 & adhuc est valde difficile debite \textbf{ et temperate se habere erga illam . } Sicut ergo prouocata gula & que avn que el o en non ouiesse si non vna mugnia vn seria cosa guaue que el se ouiesse conueinblemente \textbf{ e tenpradamente en vsar della } e por ende assi commo la garganteria \\\hline
3.1.10 & quod spectabat ad Principem ciuitatis \textbf{ habere curam et diligentiam , } ne filii coirent cum matribus , & Empero socrates quariendo escusar este mal dix̉o \textbf{ que al prinçipe dela çibdat pertenesçia de auer cuydado e acuçia } por que los fijos non yoguiessen con sus madres \\\hline
3.1.10 & Decet ergo Reges et Principes \textbf{ sic ordinare ciuitatem , } ut prohibita communitate foeminarum et uxorum certificentur parentes de propriis filiis . & Et pues que assi es conuiene alos Reyes \textbf{ e alos prinçipes de ordenar assi la çibdat } por que defendia la comunidat delas fenbras e delas mugeres casadas \\\hline
3.1.11 & Nam in rebus deseruientibus \textbf{ ad victum est considerare duo : } videlicet res fructiferas , & que siruen ala uida \textbf{ e ala uianda del omne | auemos de penssar dos cosas } Conuiene a saber las cosas \\\hline
3.1.11 & multo magis infra huiusmodi dissensionem \textbf{ non posset tollere inter multos , } ut inter omnes ciues . & que non es çierto \textbf{ non podria tirar tales varaias entre muchos } assi commo entre los çibdadanos . \\\hline
3.1.12 & quae requiritur in bellantibus , \textbf{ arguere possumus mulieres instruendas non esse ad opera bellica . } Secunda via ad inuestigandum hoc idem , & que es meester en las batallas \textbf{ podemos tomar argumento | que las mugers non son de enssennar } nin de pouer alas batallas \\\hline
3.1.12 & ne igitur reddantur bellantes pusillanimes , \textbf{ quos constat esse timidos oportet } ab exercitu expelli . & Et por ende por que los lidiadores non se enflaquezcan en las batallas \textbf{ conuiene de echar dela batalla } e dela fazienda alos de flaco \\\hline
3.1.12 & sustinere armorum pondera , \textbf{ et dare magnos ictus , } expedit eos habere magnos humeros et renes & ca commo los lidiadores ayan de sofrir el peso delas armas \textbf{ e ayan de dar grandes colpes } conuieneles de auer fuertes honbros e fuertes rennes \\\hline
3.1.12 & et dare magnos ictus , \textbf{ expedit eos habere magnos humeros et renes } ad sustinendum armorum grauedinem , & e ayan de dar grandes colpes \textbf{ conuieneles de auer fuertes honbros e fuertes rennes } para sofrir la pesadura delas armas \\\hline
3.1.12 & ad sustinendum armorum grauedinem , \textbf{ et habere fortia brachia } ad faciendum percussiones fortes : & para sofrir la pesadura delas armas \textbf{ e conuiene les de auer fuertes braços } para fazer fuertes colpes . \\\hline
3.1.12 & secundum debitam oeconomiam \textbf{ et secundum debitam dispensationem ordinare domum et ciuitatem . } Quare in iis , & a quien parte nesçe \textbf{ de ordenar la casa | e la çibdat } segunt ordenamiento conueinble en aquellas cosas \\\hline
3.1.13 & et praeposituras distribuere , \textbf{ cognoscere quales praeficiant praepositos et magistros ; } si principatus et magistratus & e partir los maestradgos e las diguidades \textbf{ de conosçer | quales pone en los prinçipados e enlos maestradgos } Si los prinçipados e los maestradgos \\\hline
3.1.13 & et uniuersaliter quaelibet praepositura virum ostendit et manifestat , \textbf{ expedit tribuentem praeposituras et magistratus super ciues prius experiri quales sint , } quos praefecit in praepositos vel magistros , & e manifiesta qual es el uaron siguese \textbf{ que conuiene a qual quier partidor o dador de las dignidades | e de los maestradgos de auer primero prueua de los çibdadanos } quales son aquellos \\\hline
3.1.13 & et concordiam ciuium debet \textbf{ intendere rector ciuitatis tanquam finem . } Sic ergo disponenda est ciuitas , & e la concordia dela çibdat es fin \textbf{ que deue entender todo gouernador dela çibdat } e por ende non ha de tomar conseio sobre ella \\\hline
3.1.14 & tot esse bellatores et defensores patriae , \textbf{ quot sunt ibi ciues valentes portare arma , } quam seperare bellatores ab aliis ciuibus . & e los defenssores dela tierra \textbf{ quantos son y çibdadanos | que pue dan tomar armas } e esto es meior \\\hline
3.1.14 & quot sunt ibi ciues valentes portare arma , \textbf{ quam seperare bellatores ab aliis ciuibus . } Secunda via sic patet , & e esto es meior \textbf{ que dezir que sean apartados los bdiadores | de los otros çibdadanos ¶ } La segunda razon se prueua \\\hline
3.1.14 & esse valde difficile et onerosum ipsis ciuibus . \textbf{ Onerosum enim et difficile esset ciuibus unius ciuitatis sustentare mille viros in stipendiis communibus , } quorum nullum esset aliud officium , & por que grant carga serie e graue cosa serie alos çibdadanos de vna \textbf{ çibdat mantener mill caualleros | de las rentas comunes de vna } çibdat los quales caualleros non ouiessen otro ofiçio ninguno \\\hline
3.1.14 & cum adesset oportunitas : \textbf{ et onerosius et quasi omnino importabile esset sustentare sic quinque milia : } oporteret enim ciuitatem illam habere possessiones quasi ad votum , & quando fuesse me este \textbf{ Et muy mayor carga e peor de sofrir serie | si ouiessen a mantener cinco mill caualleros } ca conuerne \\\hline
3.1.14 & volens ponere leges \textbf{ vel facere ordinationem aliquam in ciuitate , } ad tria debet respicere , & en el segundo libro delas politicas \textbf{ el que quiere poner leyes o fazer ordenaçion alguna en la çibdat a tres } co sas deue deuer mietes . \\\hline
3.1.14 & et maiori terrarum spatio potiretur , \textbf{ tanto sustentare posset maiorem numerum bellantium . } Tertio aspiciendum esset ad loca vicina , & e vsa de mayor espaçio de tierras \textbf{ tanto mayor cuento de lidiadores pueden mantener ¶ } Lo terçero deuen tener mientes alos logares \\\hline
3.1.14 & esse circa particularia signata , \textbf{ volens tradere artem de regimine ciuitatum , } non potest statuere determinatum numerum bellatorum : & cerca las cosas particulares e senñaladas . \textbf{ El que quiere dar arte e sçiençia de gouernamiento dela çibdat } non puede \\\hline
3.1.15 & intelligere dicta Socratica , \textbf{ saluare poterimus positionem eius . } Omnia enim esse ciuibus communia & Si quisieremos entender los dichos de socrates \textbf{ non assi conmo suena las palabras podremos entender la su opinion diziendo | que non es cosa que pueda ser } nin es cosa aprouechable \\\hline
3.1.15 & Sicut enim quilibet ciuis debet \textbf{ diligere ciues alios , } sicut seipsum : & y la comu indat \textbf{ por que si cada vn çibdada no deue amar tos otros | çibdảdanos } assy commo assi mesmo . \\\hline
3.1.15 & sicut seipsum : \textbf{ sic debent diligere uxores , filios , } et possessiones aliorum , & assy commo assi mesmo . \textbf{ En essa misma manera deue amar los fijos e las mugers } e las possessiones de los otros çibdadanos \\\hline
3.1.15 & quantum ad communitatem ciuium : \textbf{ sic etiam saluare possumus dictum eius quantum ad unitatem ciuitatis . } Nam cum dixit ciuitatem debere esse maxime unam , & quanto ala comunidat de los çibdadanos \textbf{ En essa misma manera podemos saluar el su dicho | del quanto ala vnidat dela çibdat } ca quando dixo \\\hline
3.1.15 & et mulieres \textbf{ propter penuriam ciuium defendere ciuitatem . } Quod autem ulterius addebat , & por la qual cosa conuenio alas mugers \textbf{ por mengua de los çibdadanos de defender la çibdat mas lo que enandio adelante diziendo } que sienpre conuenia \\\hline
3.1.15 & ut nobiles : \textbf{ hi videlicet nobiles potissime debent defendere patriam , } et eorum maxime est vacare & assi commo los nobles \textbf{ Por ende conuiene que estos nobles de una | prinçipalmente defender la tierra entre los otros } e a ellos parte nesçe mayormente de entender çerca la sabiduria delas armas . \\\hline
3.1.16 & aequari eis in possessionibus . \textbf{ Potuit autem Phaleas triplici via moueri ad hoc statuendum . } Primo quidem moueri potuit , & podria se ygualar alos ricos en las possessiones . \textbf{ Et este pho felleas pudo se mouer a establesçer esto por tres razones } La primera desta pudo ser \\\hline
3.1.18 & sed principalius debet \textbf{ intendere reprehensionem concupiscentiarum , } eo quod radix malitiarum & del que faze la ley deue ser çerca delas possessiones \textbf{ mas mayormente deue entender en la reprehension delas cobdiçias } por que larays delas maldades \\\hline
3.1.18 & Et tanto principalius debet \textbf{ hoc intendere circa honores , } quanto litigia inter personas honorabiles sunt magis detestanda , & e tanto mas prinçipalmente deue entender \textbf{ en partir las honrras } quanto las peleas entre las perssonas honrradas son de mayor periglo \\\hline
3.1.18 & sed quia existimant alios posse eorum delectationibus impedire , \textbf{ vel quia existimant eis posse tristitiam inferre . } non ergo solum propter possessiones sunt instituendae leges , & Mas por que cuydan que los otros pueden enbargar sus delecta connes \textbf{ o porque cuydan que les pueden fazer tristeza . } Et pues que assi es non solamente son de esta \\\hline
3.1.19 & Dicebat autem debere \textbf{ esse aliquod territorium commune , de quo bellatores viuerent } quasi de communi aerario . & Enpero dizie que algunte rrectorio deuie ser comun \textbf{ del qual deuian beuir los lidiadores } assi commo de cosa comun¶ \\\hline
3.1.19 & uniuersaliter omnes personas impotentes , \textbf{ non valentes per se ipsas sua iura conquirere . Spectat enim ad Regem et Principem , } qui debet esse custos iusti , & nin podian \textbf{ por si mismas guardar su | derechca parte nesçe al Rey e al } prinçipeque deue ser guardador dela iustiçia de auer cuydado espeçial delas cosas comunes \\\hline
3.1.19 & eo quod talibus alii de facili iniuriantur , \textbf{ cum non possint defendere iura sua . } Multa bona consequimur & por que tales perssonas los otros de ligero les fazen tuerto \textbf{ por que non pueden defender su derecho } uchos bienes se nos siguen delas opiniones de los phos antigos \\\hline
3.1.20 & tangentes diuersa genera personatum . \textbf{ Primo enim dictus Phil’ deferre fecit statuendo impossibilia . } Nam statutum de distinctione ciuium stare & tanniendo departidos linages de perssonas . \textbf{ Ca lo primero el dicho philosofo fallesçio | establesçien do establesçimientos que non podian ser nin estar en vno . } ca el establesçimiento del departimiento de los çibdadanos \\\hline
3.1.20 & Si enim ciuitas \textbf{ secundum ipsum distingui debeat in tres partes , } videlicet in bellatores , artifices , et agricolas ; & por que si la çibdat \textbf{ segunt el dixo | se deuia partir en tres partes . } Oon uiene a saber enlidiadores \\\hline
3.1.20 & loqui sibi inuicem publice , \textbf{ non tamen posse ad inuicem habere consilium in priuato . } Rursus deficit dictus modus , & que los iezes puedan fablar vno con otro en publico \textbf{ e non puedan auer conseio vno con otro en ascondido . } Otrossi fallesçe la dichͣ manera \\\hline
3.1.20 & conarentur sapientes ad inueniendum nouas leges , \textbf{ et ad ostendendum nouas leges inuentas esse proficuas ciuitati : } quare continue mutarentur leges , & de fallar nueuas leyes \textbf{ para mostrar que las leys nueuas | que ellos fallan son muy prouechosas ala çibdat } por la qual cosa cadal dia se aurian de mudar las leyes la qual cosa seria muy dannosa e muy peligrosa ala çibdat \\\hline
3.2.1 & Bene vero iudicare \textbf{ secundum leges inuentas per consiliarios , } et custoditas per principem , & que el pueblo ha de guardar \textbf{ mas bien iudgar segt las leyes falladas } por los conseieros e guardadas \\\hline
3.2.2 & tunc dicitur Monarchia siue Regnum : \textbf{ regis autem est intendere commune bonum . } Si vero ille unus dominans & e estonçe es dicho tal sennorio monarch̃ia o e egno \textbf{ ca al Rey parte nesçe de enteder el bien comun . } Et li aquel vno assi \\\hline
3.2.2 & Politia dicitur . \textbf{ Nos autem talem principatum appellare possumus gubernationem populi , } si rectus sit . & por que non ha nonbre comun es dich poliçia \textbf{ e nos podemos llamar atal prinçipado gouernamiento del pueblo } si derecho es . \\\hline
3.2.2 & et in graeco nomine dicitur Democratia . \textbf{ Nos autem ipsum appellare possumus peruersionem populi . } Patet ergo quot sunt principatus , & es dich prinçipado de malos \textbf{ e en nonbre gniego es dicho democraçia mas nos podemos le llamar destruymiento e desordenamiento del pueblo . } Et pues que assi es paresçe \\\hline
3.2.3 & ad diuersa officia et diuersos motus , \textbf{ est dare aliquod unum membrum } ut cor , & a departidos ofiçios \textbf{ e departidos mouimientos conuiene de dar algun mienbro vno } assi commo es el coraçon \\\hline
3.2.3 & Rursus si ad constitutionem eiusdem concurrunt diuersa elementa , \textbf{ est dare ibi unum aliquid , } ut animam regentem & Otrossi si a conposicion de vn cuerpo vienen departidos helementos \textbf{ conuiene de dar | y alguna cosa vna } assi commo es el alma \\\hline
3.2.4 & ut bene regat populum sibi commissum . \textbf{ Primo enim debet habere perspicacem rationem . } Secundo rectam intentionem . & que lees acomnedado . \textbf{ Lo primero deue auer razon abiuada e sotil ¶Lo segundo entencion derecha . } Lo terçero firmeza acabada . \\\hline
3.2.4 & inquantum tenent locum unius : \textbf{ dominari unum et facere monarchiam , } si debito modo fiat , & en quanto ellos tienen logar de vno \textbf{ el sennorio de vno es meior | e fazer tal monarchia de vno } si se faze en manera \\\hline
3.2.4 & ( secundum Philosophum 3 Politicor’ ) \textbf{ debet sibi associare multos sapientes , } ut habeat multos oculos & que dize el philosofo enel terçero libro delas politicas \textbf{ que deue aconpannar assi e tomar consigo muchos sabios } por que ayan muchs oios \\\hline
3.2.4 & totum ipse Rex cognoscere dicitur . \textbf{ Nec etiam dici poterit ipsum de leui posse corrumpi et peruerti : } nam si Rex recte dominari desiderat , & e saber el Rey . \textbf{ nin avn se puede dezer | que aquel vn prinçipe de ligero se pueda trastornar } e coe ronper se ca si el Rey dessea enssennorear \\\hline
3.2.4 & et dimissa societate sapientum et bonorum , \textbf{ vellet sequi caput proprium , } et appetitum priuatum , iam non esset Rex sed tyrannus : & e de los omes buenos \textbf{ e quisiere seguir su cabeça proprea } e su appetito corrupto \\\hline
3.2.5 & ad quem spectabit \textbf{ habere regiam dignitatem . } Absolute ergo loquendo , & al que pertenesçe de tegnar \textbf{ e de auer la dignidat real . } Et por ende paresça e a alguons que fablando sueltamente meiores \\\hline
3.2.5 & quare si Rex videat \textbf{ debere se principari super regnum non solum ad vitam , } sed etiam per haereditatem in propriis filiis , & Por la qual cosa si el Rey viere \textbf{ que deue regnar sobre el regno | non solamente en su uida } mas avn por heredat en sus fijos . \\\hline
3.2.5 & in haereditatem paternam . \textbf{ Vel simpliciter dicitur hoc esse diuinum , } quia nisi Reges et Principes & assi que los fijos en esta manera ouiessen la heredat de los padres \textbf{ o podemos dezir sinplemente | que es por la uirtud de dios . } si los Reyes e los prinçipes non regnaren \\\hline
3.2.5 & ad quem spectat \textbf{ suscipere curam regni . } Nam sicut mores nuper ditatorum & a quien parte nesçe de regnar \textbf{ e de tomar cuydado del regno . } ca assi commo las costunbres \\\hline
3.2.5 & et quia hoc est esse tyrannum , \textbf{ non intendere bonum regni , } tales facilius tyrannizant . & e vn por que esta es condiçion de thirano \textbf{ non tener mientes en el bien del regno } los tales que son tomados \\\hline
3.2.5 & ex qua praeficiendus est dominus , \textbf{ sed etiam oportet determinare personam . } Nam sicut oriuntur dissentiones et lites , & linage donde ha de ser tomado el sennor . \textbf{ Mas avn conuiene de determinar la perssona . } Ca assi commo nasçen discordias \\\hline
3.2.5 & in illa prosapia debeat principari . \textbf{ Talem autem determinare personam , } difficultatem non habet : & qual perssona en qual linage deua ser prinçipe \textbf{ e auer el senorio . | Mas determinar tal perssona } non hagniueza ninguna \\\hline
3.2.5 & Quod vero superius tangebatur , \textbf{ videlicet quod ire per haereditatem , dignitatem regiam , } est exponere fortunae , & Mas aqual lo que dessuso fue dich \textbf{ conuiene saber | que quando va el regno } por hedat \\\hline
3.2.5 & in talibus regiminibus vidimus , \textbf{ quae enumerare per singula longum esset . } Dicere ergo possumus & de los quales non podemos fablar \textbf{ nin contar de cada vno } por menudo pues que assi es podemos dezir \\\hline
3.2.5 & ad quem deberet regia cura peruenire , \textbf{ suppleri poterit per sapientes et bonos , } quos tanquam manus et oculos debet & auer cuydado del regno \textbf{ este fallesçimiento se puede conplir por sabios et por buenos omes } lo quales deue el Rey ayuntar \\\hline
3.2.6 & ab excessu virtuosarum actionum : \textbf{ nam quia bonorum virtuosorum est diligere bonum commune potius quam priuatum , } ideo reputatur dignus & ca por que de los buenos \textbf{ e de los uirtuosos es de amar | mas el bien comun } que el bien propreo . \\\hline
3.2.6 & Nam quia probabile est nobiles et potentes , \textbf{ magis verecundari operari turpia quam alios : } et quia tales & que los nobles e los poderosos toman mayor uerguença \textbf{ de obrar cosas torpes | e feas que los otros } e por que tales por la mayor parte \\\hline
3.2.6 & quia si virtutis est , \textbf{ tendere in bonum , } eius erit magis tendere in maius bonum , & en obras uirtuosas procurara el bien comun . \textbf{ ca si la uirtud parte nesçede se estender a mayor bien } e en mayor bien dela gente es el bien comun \\\hline
3.2.7 & Unde 3 Polit’ dicitur , \textbf{ quod principari talem , } est quasi partiri principatum in multos . & dize el philosofo \textbf{ que tal sennor auer prinçipado es } assi commo partir vn prinçipado en muchs . \\\hline
3.2.7 & quod principari talem , \textbf{ est quasi partiri principatum in multos . } Vel ( quod idem est ) & que tal sennor auer prinçipado es \textbf{ assi commo partir vn prinçipado en muchs . } o que es esso mismo \\\hline
3.2.7 & Vel ( quod idem est ) \textbf{ est quasi principari multitudinem , } eo quod in tali principatu & o que es esso mismo \textbf{ que auer muchs el prinçipado . } por que en tal prinçipado es entendido el bien de muchs . \\\hline
3.2.7 & sed etiam satagit \textbf{ impedire eorum maxima bona . } Tangit autem Philosophus 5 Polit’ tria maxima bona , & de aquellos que son en el regno \textbf{ mas avn esfuercasse para enbargar los bienes dellos } e tanne espho en el quinto libro delas politicas muy grandes tres bienes \\\hline
3.2.8 & quod natura primo dat rebus ea per quae possunt \textbf{ consequi finem suum . } Secundo dat eis ea & que la natura primeramente da a todas las cosas \textbf{ aquello por que pueden alcançar su fin . } ¶ Lo segundo les da aquellas cosas \\\hline
3.2.8 & ut melius aerem scindat \textbf{ ne prohibeatur tendere in ipsum signum : } tertio a sagittante sagittatur & meiorfender el ayre \textbf{ por que non sea enbargada de yr a su señal | ¶ } Lo terçero es puesta en la ballesta del saetero \\\hline
3.2.8 & per quae possit \textbf{ consequi finem intentum . } Secundo debet prohibentia remouere . & por que la gente que les acomnedada aya aquellas cosas \textbf{ por las quales puede alcançar la fin que entiende . } ¶ Lo segundo deue arredrar todas aquellas cosas \\\hline
3.2.8 & ut populus possit \textbf{ consequi finem intentum } et bene viuere , & Mas aquellas cosas que siruen aesto \textbf{ por que el pueblo pueda alcançar su fin } e pueda bien beuir son estas . \\\hline
3.2.8 & et habeat ordinatum appetitum , \textbf{ ut velit consequi finem illum : } spectat igitur ad rectorem regni ordinare & en tal manera \textbf{ que quiera | e pueda alcançar aquella fin . } Et por ende parte nesçe al gouernador del regno de otdenar sus \\\hline
3.2.8 & circa ea per quae possit populus \textbf{ consequi finem intentum : } restat ostendere , & de ser acuçioso çerca aquellas cosas \textbf{ por las quales el pueblo puede alcançar su fin } que entiende finca de demostrar \\\hline
3.2.8 & in haereditatem priorum : \textbf{ remouere igitur unum maxime prohibentium bonam vitam politicam , } est bene ordinare & e los postrimeros de los primeros \textbf{ Et pues que assi es tirar vna cosa } que enbarga much la buena uida çiuil es ordenar bien \\\hline
3.2.9 & ideo bene se habet \textbf{ illa decem narrare per singula . } Est autem primum quod spectat & los sermones generales poco proprouechan . \textbf{ por ende sera bien de contar estas diez cosas cada vna } por si¶ Et la primera \\\hline
3.2.9 & non solum habere familiares , \textbf{ et diligere nobiles , et barones , et alios } per quos bonus status regni conseruari potest , & non solamente de auer buenos familiares \textbf{ e de amar los nobles | e los ricos omes } e todos los otros omes . \\\hline
3.2.9 & Nono decet verum Regem per usurpationem et iniustitiam \textbf{ non dilatare suum dominium . } Nam ut dicitur Polit’ & ¶ Loye conuiene al Rey uerdadero de non enssanchar su regno \textbf{ por tomar lo ageno | por fuerça e sin iustiçia . } Ca assi commo dize el philosofo \\\hline
3.2.9 & Nam ut dicitur Polit’ \textbf{ durabilius est regnare super paucos , } quam super multos . & en el tercero libro delas politicas \textbf{ mas durable es regnar sobre pocos que sobre muchos . } la qual cosa es muy uerdadera mayor mente \\\hline
3.2.10 & Quinta cautela tyrannica , \textbf{ est habere multos exploratores , } et tenptare non latere ipsum & si el esso mismo non amasse mucho al Rey \textbf{ La quinta cautela del tirano es auer muchs assechadores } e escodrinnar \\\hline
3.2.10 & eo quod in multis offendant ipsum , \textbf{ volunt habere exploratores multos , } ut si viderent aliquos ex populo machinari aliquid contra eos , & que non lon amados del pueblo . \textbf{ por que en muchͣs cosas le aguauian quieren auer muchs assechadores } por que si vieren \\\hline
3.2.10 & Sexta cautela tyrannica , \textbf{ est non solum non permittere fieri sodalitates et amicitias , } sed etiam amicitias iam factas , & ¶ La sesta cautela del tirano es \textbf{ non solamente non conssentir las conpannias e las amistanças . } Mas avn las conpannias e las amistancas \\\hline
3.2.10 & quia non intenderet commune bonum . \textbf{ Septima , est pauperes facere subditos adeo } ut ipse tyrannus nulla custodia egeat . & ca non entendrie en el bien comun . \textbf{ La vij ͣ̊ cautela del tirano es fazer los subditos pobres } en tanto que el non aya menester guarda \\\hline
3.2.10 & quibus indigent , \textbf{ ut non vacet eis aliquid machinari contra ipsos , nec oporteat ipsos habere aliquam custodiam propter illos . } Verus autem Rex & en que han de de beuir de cada dia \textbf{ por que no les uague de fazer ayuntamiento contra ellos | nin los tiranos non ayan menester ninguna guarda } por temor dellos . \\\hline
3.2.10 & Octaua , est procurare bella , \textbf{ mittere bellatores ad partes extraneas , } et semper facere bellare & ¶ La . viij n . cautela del tirano \textbf{ es procurar guerras e enbiar | guerrasa partes estrannas } e sienpre faze lidiar sus çibdadanos \\\hline
3.2.10 & quatenus semper circa bellorum onera intenti , \textbf{ non vacet eis aliquid machinari contra tyrannum . } Verus autem Rex non intendit affligere subditos , & por razon que ellos en tal manera sean ocupados en las guercas \textbf{ que non les vague de seleunatar contra el tirano . } Mas el uerdadero rey non entiende de atormentar los subditos mouiendo les \\\hline
3.2.10 & non vacet eis aliquid machinari contra tyrannum . \textbf{ Verus autem Rex non intendit affligere subditos , } suscitando et procurando bella , & que non les vague de seleunatar contra el tirano . \textbf{ Mas el uerdadero rey non entiende de atormentar los subditos mouiendo les } e procurado les guerras \\\hline
3.2.11 & excogitant seditiones , \textbf{ quomodo possint turbare ciuitatem , et insurgere contra rectorem ciuium . } Contra haec ergo quatuor procurant tyranni perimere excellentes , & por el conplimiento que han pienssan maneras e carreras \textbf{ por las quales podran defender su çibdat | e leunatarse contra el mal gouernador de los çibdadanos } Et pues que assi es contra estas quatto cosas procuran los tyranos de matar los nobles e los grandes \\\hline
3.2.11 & destruere sapientes , \textbf{ impedire scholas , et studium , } ut existentes in regno sint ignorantes et inscii : & Otrossi procuran de destroyr los sabios \textbf{ e enbargar las escuelas e el estudio . } por que los que fueren en el regno sean nesçios e sin sabiduria \\\hline
3.2.11 & ut existentes in regno sint ignorantes et inscii : \textbf{ non permittere sodalitates ; } turbare ciues inter se & Otrossi procuran de enbargar \textbf{ e non consentir las conpannias } e de turbar los çibdadanos entre ssi \\\hline
3.2.11 & ut de se inuicem non confidant : \textbf{ depauperare eos : } occupare eos in bello , & por que non fiende ssi mismos los vnos de los otros . \textbf{ Otrossi procuran de los en pobresçer | por que sean sienpre menesterosos } e delos poner en guerras \\\hline
3.2.11 & depauperare eos : \textbf{ occupare eos in bello , } et in aliis exercitiis , & por que sean sienpre menesterosos \textbf{ e delos poner en guerras } e en otros trabaios \\\hline
3.2.11 & Ex hoc autem manifeste patet , \textbf{ tyrannidem maxime esse fugiendam a regibus : } quia pessimum de se , & e desto paresçe manifiesta miente \textbf{ que la tirama es much de escusar alos Reyes | e mucho han de foyr della } por que es . muy mala . \\\hline
3.2.12 & et quare nunquam hylarem vultum ostenderet . \textbf{ Tyrannus ille volens reddere causam quaesiti , } eum expoliari fecit , & que nunca mostraua la cara alegte \textbf{ e aquel tirano quariendo dar razon desto fizo despoiar a su hͣmano } e fizola tar \\\hline
3.2.12 & quot veri reges . \textbf{ Nam habere amicos , } et diligi a populo , & quantas han los uerdaderos los Reyes \textbf{ por que auer amigos } e ser muy amado del pueblo es muy delectable \\\hline
3.2.12 & non confidere de aliquo \textbf{ et credere se odiosum esse multitudini , } est maxime tristabile . & e non fiar de alguno \textbf{ e creer que es odioso | e aborresçido dela muchedunbre del pueblo } esto es muy derstable \\\hline
3.2.12 & congregantur in ea : \textbf{ restat videre esse eam cauendam , } eo quod etiam in ipsa congregantur & e los malos señorios de los rricos son ayuntados en ella , \textbf{ finca de ver que es de foyr e de aborresçer avn } por que en ella son ayuntados los males del mal priͥnçipado del pueblo \\\hline
3.2.13 & et iniuriari subditis , \textbf{ et non intendere commune bonum ; } licet pluribus viis ostenderimus & e fazer tuerto alos subditos \textbf{ e non entender al bien comun | commo quier } que por munchons rrazones mostramos \\\hline
3.2.13 & et periculosum esse regiae maiestati tyrannizare , \textbf{ et non recte gubernare populum : } non piget adhuc nouas vias adducere & e aborresçible deue ser ala rreal maiestad tiranzar \textbf{ e non gouernar derechamente el pueblo } aun non tomamos pereza de \\\hline
3.2.13 & est iniuria quam passi sunt ab ipso : \textbf{ naturale est enim desiderare vindictam , } propter quod Homerus dicebat , & que rresçibien \textbf{ dehcanatanl cosa | e ᷤalos omes desear uengança del mal que rresçibe } Por la qual cosa omero aquel poeta dezia \\\hline
3.2.13 & nisi honorem et gloriam propriam , \textbf{ et non honorare subditos , } et non quaerere commune bonum , & e de sugłia propria \textbf{ e non quiere honrrar los subditos } njn quiere el bien comun \\\hline
3.2.14 & volumus alias rationes adducere , \textbf{ ostendentes quod si reges cupiant suum durare dominium , } summo opere studere debent & avn en este cpleo queremos adozjr otras rrazones \textbf{ para mostrar que si los rrey e cobdiçian de duar muncho } el su señorio es toda manera deuen estudiar \\\hline
3.2.14 & dicens , \textbf{ Tyrannidem corrumpi a se , } a tyrannide alia , et a regno . & Ca cuenta el phon enel quinto libro delas politicas tres maneras dela corrupçion dela tiranja \textbf{ e dize que la tiranja corrope de si mismar coronpese desta çirana } e corronpese por El regno ¶ \\\hline
3.2.14 & Reges ergo et principes \textbf{ si volunt suum durare dominium , } summe cauere debent & e los prinçipes \textbf{ si quieren durar en su sennorio } mucho deuen escusar \\\hline
3.2.14 & gens illa oppressa non valens \textbf{ sustinere tyrannidem Principis , } insurgit et tyrannizat in ipsum , & enparadoro algun \textbf{ prinçipe vno titaniza en el pueblo aquella gente apremiada non podie do sofrir su tira } maleunatasse e tiraniza contrael prinçipe matandol o echandol del prinçipado . \\\hline
3.2.15 & Quartum autem quod politiam saluare videtur , \textbf{ est cauere seditiones et contentiones nobilium ; } et hoc ponendo eis leges , & que salua la poliçia \textbf{ es escusar las discordias | e las contiendas delos nobles } e esto poniendo les leyes \\\hline
3.2.15 & Sunt enim in regno tales leges instituendae , \textbf{ ut per eas sedari possint contentiones nobilium . } Nam baronibus dissentientibus & e son de poner tales leyes en el regno \textbf{ que por ellas se puedan tirar las discordias | e las tales contiendas de los nobles } ca desacordados \\\hline
3.2.15 & et politiam saluat , \textbf{ sicut praeficere homines bonos et virtuosos , } et conferre eis dominia et principatus . & e la poliçia \textbf{ commo poner los bueons e los uirtuosos en las dignidades } e dar les los señorios e los prinçipados . \\\hline
3.2.15 & Octauum saluans regnum et politiam , \textbf{ est habere ciuilem potentiam . } Nam ( ut dicitur in Magnis moralibus ) & cosa que salua el regno \textbf{ e la poliçia es auer poderio | çiuilca } assi commo dize el philosofo \\\hline
3.2.15 & si vult seruare iustitiam \textbf{ et vult punire transgressores iusti , } habere multos exploratores , & si quisiere bien guardar la iustiçia \textbf{ e si asi ere dar pena alos malos | que trasgre en passando la iustiçia } auer much sassechadores e muchs pesquiridores \\\hline
3.2.15 & et vult punire transgressores iusti , \textbf{ habere multos exploratores , } et multos inquisitores inuestigantes facta ciuium , & que trasgre en passando la iustiçia \textbf{ auer much sassechadores e muchs pesquiridores } que escodrinen \\\hline
3.2.15 & Sic enim faciendo ista , \textbf{ poterit seruare iustitiam , } et praeseruare regnum a maleficis , & que biue de furto o de rapina . \textbf{ ca assi fazie do podra guardar la iustiçia } e guardar el regno de malefiçios \\\hline
3.2.15 & poterit seruare iustitiam , \textbf{ et praeseruare regnum a maleficis , } et transgressoribus iusti . & ca assi fazie do podra guardar la iustiçia \textbf{ e guardar el regno de malefiçios } e de los malos \\\hline
3.2.15 & saluare et corrumpere . \textbf{ Talia autem maxime sciri poterunt per experientiam : } nam cum quis diu expertus est regni negocia , & que la pueden saluar e corronper \textbf{ Mas si tales cosas se han de saber mucho | por esperiençia } e por prueua \\\hline
3.2.15 & et saluant . \textbf{ Decet ergo Regem frequenter meditari et habere memoriam praeteritorum } quae contigerunt in regno , & e qual cosa lo salua . \textbf{ Et pues que assi es conuiene al Rey de penssar mucha menudo | e muchͣs uezes delas cosas que passaron . } Et conuiene le de auer memoria de los fecho passados \\\hline
3.2.16 & quae sepe contingunt tempore aestiuali , \textbf{ non habet esse consilium : } quia talia naturalia sunt , & que muchas uezes se fazen en el tro del \textbf{ estiuo non auemos a tomar consseio } por que tales cosas \\\hline
3.2.16 & consiliantur de iis operabilibus , \textbf{ quae fieri possunt per ipsos . } Sexto non sunt consiliabilia & cada vne de los omes toma conseio de aquellas obras \textbf{ que se puden fazer por el } ¶Lo vi̊ non caen sosico consseio todas aqllas cosas \\\hline
3.2.16 & sed de iis per quae \textbf{ consequi possumus illud . } Medicus enim quia finaliter intendit sanitatem , & mas de aquellas cosas \textbf{ por que podemos alcançar aquella fin . } Ca el fisico que entiende la salut del ome \\\hline
3.2.17 & quia est quaestio agibilium humanorum : \textbf{ restat videre qualiter est consiliandum , } et quem modum in consiliis habere debemus . & ca es question delas obras \textbf{ que pueden fazer los omes finca de ver | en qual manera es de tomar el conseio } e qual manera deuemos tener en los conseios \\\hline
3.2.17 & non enim consiliatur scriptor \textbf{ ( nisi sit omnino ignorans ) qualiter debeat scribere litteras , } quia hoc sufficienter determinatum est & Ca el esceruano non toma consseio \textbf{ commo esceruir a las letris | si non fuere del todo nesçio } que non sepa en \\\hline
3.2.17 & Nam licet homo inter seipsum possit \textbf{ inuenire vias et modos ad aliquid peragendum , } attamen imprudens est & que auemos de fazer \textbf{ ca commo quier que el omne entre ssi mismo pueda fallar carreras e maneras para fazer alguna cosa } enpero non es sabio aquel \\\hline
3.2.17 & qui solo suo capiti innittitur , \textbf{ et renuit aliorum audire sententias . } Magnae enim prudentiae est & que se esfuerça en su cabeça sola \textbf{ e menospreçia de oyr las suinas de los otros } ca de grant sabiduria es en los consseios tener esta manera \\\hline
3.2.17 & Magnae enim prudentiae est \textbf{ in consiliis hunc habere modum : } ut cum aliis conferamus quid agendum , & e menospreçia de oyr las suinas de los otros \textbf{ ca de grant sabiduria es en los consseios tener esta manera } que con los otros ayamos acuerdo \\\hline
3.2.17 & a Con et Sileo \textbf{ ut illud dicatur esse Consilium , } quod simul aliqui plures silent et tacent . & mas por auentura meior po demos \textbf{ dezir que cosseio sea dicha conssilendo } que quiere tanto dezir commo cosa que se deue callar entre muchs camuches eston de guardar en los consseios \\\hline
3.2.17 & ut quod essent adulatores , \textbf{ plus curantes loqui placentia , } quam vera . & ¶ La otra que non sean plazenteros \textbf{ assi que parezcan lisongeros auiendo mayor cuydado de fablar cosas plazenteras que uerdaderas . } En essa misma manera abn segunt dize el pho \\\hline
3.2.18 & et creditur dictis eius , \textbf{ quia existimatur bonus consiliarius esse ad persuadendum . } Sed ad hoc quod aliquis sit bene creditiuus , & por que cuydan los omes \textbf{ que es buen consseiero | para dar razon de su consseio . } mas para que alguno sea bien de creer \\\hline
3.2.18 & Nam prudentis est , \textbf{ scire et cognoscere ipsas res , } et ipsa negocia agibilia : & ca del sabio es de sabra \textbf{ e de conosçer | daquellas cosas de que fabla } et de aquellos negoçios de que obra . \\\hline
3.2.18 & et haec persuasio est per se : \textbf{ nam reddere se credibilem } et bene persuadere per se , & mas esta creençia et este amonestamiento es por si . \textbf{ Ca fazerse el omne digno de creer } e buen amonestador e razonador por si . \\\hline
3.2.18 & et ex ipsis negotiis \textbf{ de quibus loquitur scire assumere rationes et argumenta , } per quae fides fiat audientibus . & de que fabla el amonestador \textbf{ e el | razonadorca sabe tomar razones e argumentos } por los quales faga fe alos oydores \\\hline
3.2.18 & satis apparet quales consiliatores deceat \textbf{ quaerere regiam maiestatem ; } quia debet quaerere tales & que ha todo buen conseiero en ssi de fecho . \textbf{ Et por ende assaz parelçe quales conseieros deue auer el rey } ca deue tomar tales \\\hline
3.2.19 & expedit enim regium consilium \textbf{ pro viribus saluare iura Regis , } eo quod huiusmodi bona ordinanda sunt & ca conuiene que el conseio del Rey sea bue no para saluar \textbf{ por todo su ponder los derechs del Rey . } por que tales biens deuen ser ordenados a bien comun \\\hline
3.2.19 & et prouentus regni , \textbf{ quos oportet peruenire ad regem , } qui et quanti sunt : & Et conuiene que sepan las rentas del regno \textbf{ las que han de venir al Rey quales e quantas son } por que si alguͣ cosa es superflua \\\hline
3.2.19 & quia aliter iam non esset ciuitas : \textbf{ ut in huiusmodi sufficientibus ad vitam fieri debent debitae commutationes , } ut debitae emptiones , & al que en otra manera non sene çibdat . \textbf{ Et en estas cosas | que parte nesçen para la uida deuense fazer mudaçiones } e canbios conuenibles \\\hline
3.2.19 & vel etiam totaliter extirpentur , \textbf{ quia Reges et Principes non debent pati maleficos viuere . } Sunt etiam consideranda loca & por pena o echadas dela çibdat o muertos . \textbf{ ca los Reyes e los prinçipes non deuen sofrir | que los malfechores bi una . } Avn son de penssar los logares \\\hline
3.2.19 & ab extraneis possit \textbf{ suscipere detrimentum : } ideo passagia , portus , introitus et caetera & Et si alguna çibdat del regno \textbf{ puede resçebir danno de los estran nos } por ende los passaies e los puertos e las entradas \\\hline
3.2.19 & et de hoc quod principaliter intenditur , \textbf{ nullus dubitat ipsum esse prosequendum . } De eius autem opposito & e de aquello que prinçipalmente omne entiende ninguno \textbf{ non dubda delo segnir . } Et del contrario della cada vno sabe \\\hline
3.2.19 & Primo , ut nunquam capiatur iniustum bellum , \textbf{ quia iniustificari in alios , } et eos indebite opprimere , & que non sea con razon e con derecho . \textbf{ por que fazer tuerto alos otros } e apremiar sos sin derecho es mala cosa por si \\\hline
3.2.19 & ad deteriores autem nobis est expugnare \textbf{ vel non pugnare contra eos . } Posito enim quod potentiores , & con los meiores deuemos auer paz \textbf{ mas con los peores en nos es de lidiar o de non lidiar . } Ca puesto que los mas \\\hline
3.2.19 & in nos forefaciant , prudentiae est , \textbf{ non insurgere in ipsos , } nisi occurrat opportunitas temporis , & poderolos alos quales non podemos contradezer nos fagan alguna fuerca o algun tuerto grant sabiduria \textbf{ es non nos leunatar contra ellos } si non fuere en tien \\\hline
3.2.20 & Quare si legum conditores respectu iudicum sunt pauci , \textbf{ quia facilius est inuenire paucos sapientes , } quam multos , ut omnia sapienter disponantur , & por la qual cosa si los fazedores son pocos en conparaçion de los iuezes \textbf{ porque mas ligera cosa es de fallar pocos sabios que muchs . } por que todas las cosas sean ordenadas sabiamente \\\hline
3.2.20 & vel inimicus sit illa facturus , \textbf{ et debeat illam subire sententiam . } Nam si scirent quod amicus , & aquel que auie de fazer aquella cosa \textbf{ e deuie passar por tal suina . } ca si por auentra asopiessen ellos \\\hline
3.2.21 & ex eo quod huiusmodi sermones \textbf{ obligare habent iudicem , } quem esse oportet & ante los alcalłs rodemos lo prouar por tres razones ¶ \textbf{ La primera seqma par aquello que tales palabras han de to terçeres desegualar eliez } el qual conuiene de ser \\\hline
3.2.21 & ut recte iudicet , \textbf{ sic debet se habere inter partes litigantes , } sicut lingua volens discernere de saporibus , & para que derechamente iudgue \textbf{ assi se deue auer entre las partes | que contienden } commo la lengua \\\hline
3.2.21 & de proprii sensibilibus , \textbf{ habere se debet } inter ipsos sapores , & que sienten propriamente \textbf{ ca la lengua se deue auer entre los sabores } o cada vno de los otros sesos en las cosas \\\hline
3.2.21 & et quia hoc faciunt sermones passionales , \textbf{ permittere talia in iudicio nihil est aliud quam regulam obliquare : } quasi si inconueniens est & Et por que esto fazen las palabras desiguales \textbf{ et malas consentir tales palabras en el iuyzio non es otra cosa | si non torçer la regla } que non iudgue derecho \\\hline
3.2.21 & Dato tamen quod contingat \textbf{ sustinere aliquos passionales sermones , } quia ( ut in sequentibus patebit ) & que enco el uuzio se digan palabras malas e desiguales \textbf{ ca assi commo parezçca en lo que es de dezer los mezes } mas enclinados deuen sera auer piedat \\\hline
3.2.21 & inclinent voluntatem , \textbf{ et faciant apparere aliquid iustum , } vel non iustum , & enclinan la uoluntad de los omes \textbf{ e fagan paresçer alguna cosa derecha . } por que los que assi son munnidos \\\hline
3.2.21 & non pariter iudicamus , \textbf{ permittere passionales sermones in iudicio , } est peruertere ordinem iudicandi : & Et por ende iudgan ygual mente . \textbf{ Et pues que assi es conssentir tales palabras } en iuyzio es trastornar la orden de iudgar . \\\hline
3.2.21 & permittere passionales sermones in iudicio , \textbf{ est peruertere ordinem iudicandi : } quia est facere & Et pues que assi es conssentir tales palabras \textbf{ en iuyzio es trastornar la orden de iudgar . } Ca es iudgar las partes \\\hline
3.2.21 & et quod teneant supremum gradum in iudicando , \textbf{ quae debent tenere infimum . } Peruertitur ibi talis ordo , & e tener elguado primero en iudgar \textbf{ aquellos que deuen tener el postrimero } e assi se trastorna \\\hline
3.2.21 & quia partes passionando iudicem , \textbf{ ei faciunt apparere aliquid iustum vel iniustum , } quod non est officium partium , & Ca las partes mouiendo el iuez \textbf{ assi fazen paresçer | alguͣ cosa derechͣo non de rethica . } la qual cosa es ofiçio del ponedor dela ley . \\\hline
3.2.21 & de quo est litigium : \textbf{ passionare autem iudicem , } aut narrare iniurias & de que contienden \textbf{ Mas enduzir al iiez } por palabras contando le las miurias . \\\hline
3.2.23 & et supra iustitiam . \textbf{ Secundum quod inclinare debet iudicem ad clementiam , } est ipse legislator . & nin que la iustiçia afincada ¶ \textbf{ Lo segundo que deue inclinar al iues a piedat } es el establesçedor dela ley . \\\hline
3.2.23 & quod iudicans potius debet \textbf{ respicere ad legislatorem , } quam ad leges . & que el iuez \textbf{ mas deue tener mientes al ponedor dela ley } que alas leyes . \\\hline
3.2.23 & ideo dicitur 1 Rhetor’ \textbf{ quod iudicans non debet respicere ad partem , } sed ad totum . & en el primero libro de la rectonca \textbf{ que el uiez non deue catar ala parte mas al todo | nin deue tener mientes a vna obra } que fizo mas a todas las buenas obras \\\hline
3.2.23 & si viderit delinquentem \textbf{ magis velle ire ad arbitrium , } quam ad disceptationem : & si uiere \textbf{ que el que pecaua mas al aluedrio del iuez | que non a escusar se } e a disputar con el . \\\hline
3.2.24 & quod eo modo quo distinguimus ius siue iustum , \textbf{ distinguere possumus leges ipsas , } et econuerso . & e usta podemos departir las leyes \textbf{ e por el contrario por las leyes podemos deꝑtir | que cosa es derecho } e que cosa non es derecho . \\\hline
3.2.24 & ius naturale a iure gentium , \textbf{ possemus separare nos ius naturale } a iure animalium : & era que los iuristas apartan el derecho natural del derecho delas \textbf{ gentes podemos nos apartar el derecho natural del derecho delas animalias } e darla quanta distinçion \\\hline
3.2.24 & a iure animalium : \textbf{ et dare quintam distinctionem iuris , } dicendo quod quadruplex est ius , & gentes podemos nos apartar el derecho natural del derecho delas animalias \textbf{ e darla quanta distinçion | e el quinto departimiento del derech̃ . } diziendo que en quatro maneras se departe el derech . \\\hline
3.2.24 & postquam autem est editum incipit \textbf{ habere ligandi efficaciam . } Ratio autem , & mas despues que es puesto a fuerça \textbf{ de obligar alos omes . } Mas la razon por que al derech natural conuinio anneder derecho positiuo es esta \\\hline
3.2.24 & sermonem nobis esse datum a natura . \textbf{ Sicut ergo loqui est naturale , } sic autem loqui vel sic , est positiuum et ad placitum . & que la palabra non es dada por natura . \textbf{ Et pues que assi es | assi commo fablar es cosa natural alos omes } assi fablar tal \\\hline
3.2.24 & Sic , fures punire , \textbf{ non pati maleficos viuere , } et cetera huiusmodi sunt , & bien assi dar pena alos ladrones \textbf{ e non sofrir beuir los malos } e o tristales cosas son de \\\hline
3.2.24 & Ubi ergo terminatur ius naturale , \textbf{ ibi incipit oriri ius positiuum : } quia semper quae sunt & ose termina el derecho natural \textbf{ alli comiença a naçer | el derech posituio } por que sienpre aquellas cosas que son falladas \\\hline
3.2.25 & secundum quem modum loquendi potest \textbf{ ibi addi membrum quartum , } ut ius animalium . & assi commo es el derecho delas gentes . \textbf{ Et segunt esta manera de fablar podemos ennader el quarto mienbro } que es derecho delas aian lias . \\\hline
3.2.25 & Poterit ergo inclinatio naturalis \textbf{ sequi naturam hominis } vel ut homo est , & Et por ende la inclinacion natural \textbf{ puede seguir la natura del ome } o en quanto es omne o en quanto conuiene con todas las \\\hline
3.2.25 & quod et omnia entia alia appetunt : \textbf{ naturaliter appetit producere filios , educare prolem , } quod et alia animalia concupiscunt : & naturalnse te dessea ser guardado en su ser Ra qual cola avn del sean todas las otras cosas que son . \textbf{ avn el omne natutalmente dessea de auer fijos | e de criar los . } Ca esto dessean todas las otras aina las natraalmente \\\hline
3.2.25 & et communius illo : \textbf{ nam appetere bonum et esse , } et fugere malum & e es mas comun qual otro . \textbf{ Ca dessear bien e dessear ser } e foyr el mal \\\hline
3.2.25 & est plus de iure naturali , \textbf{ quam appetere procreare filios , } et nutrire prolem . & mas de derech natural \textbf{ que dessear de engendrar fijos e criar los . } Et pues que assi es esta sera la orden entre estos de ti xu rechos \\\hline
3.2.25 & quam appetere procreare filios , \textbf{ et nutrire prolem . } Erit igitur hic ordo , & mas de derech natural \textbf{ que dessear de engendrar fijos e criar los . } Et pues que assi es esta sera la orden entre estos de ti xu rechos \\\hline
3.2.25 & quod per antonomasiam dicitur esse naturale . \textbf{ Appetere enim esse et bonum , } et fugere non esse et malum , & que es dicho natural pora un ataia de los otros derechos . \textbf{ Por que dessear el bien e el ser } e foyr el non ser \\\hline
3.2.26 & ut saltem metu poenae volentes \textbf{ impedire pacem ciuium , } desisterent agere peruerse . & por que si quier \textbf{ por miedo dela pena los que quesiessen enbargar la paz de los çibdadanos dexassen de obrar mal } as leyes assi commo paresçe \\\hline
3.2.27 & nam cuius est ordinare \textbf{ et dirigere aliquos in aliquod bonum , } eiusdem est condere leges , & La primera razon assi . \textbf{ Ca aquel cuyo es de ordenar e enderesçar a alguos en algun bien atlgun aquel parte nesçe } establesçer leyes e reglas delas nuestras obras . \\\hline
3.2.27 & et dirigere aliquos in aliquod bonum , \textbf{ eiusdem est condere leges , } et regulas agibilium & Ca aquel cuyo es de ordenar e enderesçar a alguos en algun bien atlgun aquel parte nesçe \textbf{ establesçer leyes e reglas delas nuestras obras . } por las quales leyes ymosa aquel bien . \\\hline
3.2.27 & cuius est ordinare \textbf{ et dirigere alios in tale bonum , } vel condendae sunt a toto populo , & deuen ser establesçidas del prinçipe \textbf{ a quien parte nesçe ordenar e enderesçar los otros atal bien } o deuen ser establesçidas de todo el pueblo \\\hline
3.2.27 & si totus populus principetur , \textbf{ et sit in potestate eius eligere principantem : } Nulla est ergo lex , & si todo el pueblo en ssennorea \textbf{ e si en su poder es de escoger el prinçipe . Et pues que assi es la ley non es ninguna } si non es establesçida \\\hline
3.2.27 & quae non sit edita \textbf{ ab eo cuius est dirigere in bonum commune : } nam si est lex diuina et naturalis , & por aquel \textbf{ a quien parte nesçe | de enderesçar los omes al bien comun . } Ca si es ley diuinal e natural establesçida es de dios \\\hline
3.2.27 & Princeps enim aut totus populus cum principatur , \textbf{ habet dirigere et ordinare alios in commune bonum . } Quaelibet ergo persona particularis , & Ca el prinçipe o avn todo el pueblo \textbf{ quando enssennorea ha de ordenar | e de enderesçar todos los otros al bien comun . } Et pues que assi es cada vna \\\hline
3.2.27 & ideo in quolibet homine haec promulgatur et propalatur , \textbf{ quando incipit habere rationis usum , } per quam cognoscit & que en cada vn omne es publicada e manifestada \textbf{ quando comiença de auer uso de razon e de entendimiento } por el qual conosçe qual cosa ha de fazer e de escoger \\\hline
3.2.28 & et quae et quot opera debent \textbf{ continere huiusmodi leges . } Dicuntur autem quinque esse effectus legum , & quales son los fechs delas leyes \textbf{ e quales e quantas obras deuen contener estas leyes } e conuiene de sabra \\\hline
3.2.28 & licet forte ex intentione operantium possint \textbf{ esse bona et laudabilia , } vel mala et vituperabilia . & por la entençion \textbf{ del que obra pueden ser buenas e de loar o malas e de denostar assi } conmoleunatar la paia de tierra \\\hline
3.2.28 & vel mala et vituperabilia . \textbf{ Ut sic eleuare festucam de terra , } de se est opus indifferens : & del que obra pueden ser buenas e de loar o malas e de denostar assi \textbf{ conmoleunatar la paia de tierra } de si es obra \\\hline
3.2.28 & si quis tamen mala intentione eleuaret illam , \textbf{ ut quia vellet ponere in oculum socii , } esset opus prauum et vituperabile : & leunatare con mala entençion \textbf{ para poner la en el oio a su } conpannon es mala obra e de deno star . \\\hline
3.2.28 & si vero eleuando eam vellet purgare domum \textbf{ vel facere aliquod aliud opus pium , } propter bonam intentionem operantis , & para alinpiat la casa \textbf{ o para fazer alguna otra obra buean } por la buena entençion \\\hline
3.2.28 & quod de se est quasi indifferens , \textbf{ esse potest virtuosum et laudabile . } Secundum igitur haec tria genera fiendorum , & que dessi non es buena nin mala \textbf{ enpero puede ser uirtuosa e de loar . } pues que assi es segunt estas tres maneras delas obras que son de fazer podemos a podar \\\hline
3.2.28 & et ciuitatis cura peruigili \textbf{ insudare quas leges , } et quae instituta imponant ciuibus , & e delas çibdadeᷤ \textbf{ assi que con grant cuydado e con grant estudio deuen trabaiar quales leyes } e quales establesçimientos pongan a sus çibdadanos . \\\hline
3.2.29 & Secunda ex eo quod facilius est \textbf{ corrumpi Regem quam legem . } Prima via sic patet . & La segunda se toma de aquello que mas ligera cosa es es de se \textbf{ corronper el rey que la ley ¶ la primera razon paresçe assi . Ca assi commo dize el philosofo } en el quinto libro delas ethicas \\\hline
3.2.29 & quia cum optimus homo incipit furire \textbf{ et concupiscere peruersa , } et si non interimitur quantum ad esse simpliciter , & enpero matase quanto al ser muy bueno . \textbf{ por que quando el muy bueno en se en comiença de enssennar e de cobdiçiar las cosas malas } si se non mata \\\hline
3.2.29 & quod qui iubet principari intellectum , \textbf{ iubet principari deum et legem ; } sed qui iubet principari hominem , & enssenerorear al entendimiento \textbf{ manda | enssennorear a dios e ala ley . } as quien manda \\\hline
3.2.29 & iubet principari deum et legem ; \textbf{ sed qui iubet principari hominem , } propter concupiscentiam annexam apponit & enssennorear a dios e ala ley . \textbf{ as quien manda | enssennorear al omne } por la cobdiçia se allega ael manda que \\\hline
3.2.29 & Quare si nomen regis a regendo sumptum est , \textbf{ et decet Regem regere alios , } et esse regulam aliorum , & sy el nonbre del Rey es tomado de gouernamiento . \textbf{ Conuiene al rey de gouernar los otros } e de ser regla de los otros . \\\hline
3.2.29 & cum quis non innititur \textbf{ regere alios ratione } sed passione et concupiscentia , & que la bestia en ssennorea \textbf{ quando alguno non se esfuerça de gouernar los otros } por razon e por entendimiento mas por passion \\\hline
3.2.30 & Prima est , quia communiter populus non potest \textbf{ attingere punctalem formam viuendi , } ideo oportet aliqua peccata dissimulare & la qual cosa contesçe por dos razones La primera es por que el pueblo \textbf{ comunalmente non puede alcançar forma de beuir en punto . } Por ende conuiene que \\\hline
3.2.30 & ut in prosequendo patebit : \textbf{ oportuit igitur dare legem euangelicam et diuinam , } secundum quam prohiberentur & assi commo paresçra adelante . \textbf{ Et por ende conuiene de dar ley diuinal } e e un agłical segunt la qual fuessen vedados los pecados todos . \\\hline
3.2.30 & et regulam agibilium , \textbf{ sic se habere ad legem diuinam , } naturalem , et humanam : & e de ser forma de beuir e regla de todas las obras . \textbf{ assi se auer ala leyna traal e diuinal e humanal } por que assi commo sobrepuian los otros en poderio e en diuinidat \\\hline
3.2.31 & utrum sit expediens ciuitatibus \textbf{ innouare patrias leges , } et inducere nouas consuetudines . & si es cosa conuenible alas çibdades \textbf{ de renouar las leyes dela tierra } e de enduzir nueuas costunbres \\\hline
3.2.31 & innouare patrias leges , \textbf{ et inducere nouas consuetudines . } Ordinauerat enim Hippodamus & de renouar las leyes dela tierra \textbf{ e de enduzir nueuas costunbres } por que ypodomio ordenara \\\hline
3.2.31 & an positio Hippodami esset bona , \textbf{ et an expediat saepe saepius immutare leges : } dato etiam quod occurrant leges aliquae & Et por ende non sin razon dubdauna si la opinion de ypodomio era buena \textbf{ e si conuinie de renouar | e mudar las leyes muchͣ suegadas } puesto avn que algunas leyes fuessen falladas \\\hline
3.2.31 & per quas videtur ostendi , \textbf{ quod expediat innouare leges . } Prima sumitur ex parte scientiarum et artium . & que muestra \textbf{ que conuiene de renouar las leyes ¶ } La primera se toma de parte delas sçiençias e delas artes . \\\hline
3.2.31 & si posteriores sapientiores non possent \textbf{ immutare leges paternas } per simpliciores conditas : & Et por ende cosa sin razon seria \textbf{ si los sabios postrimos non pudiessen mudar las leyes } delatrraque fueron establesçidas \\\hline
3.2.31 & occurrit aliquid melius , \textbf{ inconueniens est non remouere leges paternas } et antiquas propter meliores leges nouiter inuentas . & por la esperiençia delas obras particulares alguna cosa fallar en meior \textbf{ non es cosa sin razon de tirar las leyes dela tierra antiguas } por las meiores leyes falladas nueuamente por ellos . \\\hline
3.2.31 & valde periculosum ciuitati et regno . \textbf{ Nam assuescere inducere nouas leges } ( ut innuit Philosophus 2 Pol’ ) & sinplemente es muy perigloso ala çibdat e altegno . \textbf{ Ca acostunbrar se los omes | afaznueuas leyes } assi commo dize elpho \\\hline
3.2.31 & volunt enim corpora \textbf{ inducere ad sanitatem . } Sed veri legislatores & Ca los fisicos entienden enla sanidat del cuerpo . \textbf{ por que quieren adozir el cuerpo a sanidat . Mas los uerdaderos ordenadores delas leyes } e los uerdaderos Reyes sinplemente entienden en el bien del alma \\\hline
3.2.31 & quia intendunt ciues \textbf{ inducere ad virtutem . } Ut ergo appareat & e los uerdaderos Reyes sinplemente entienden en el bien del alma \textbf{ por que entienden de adozir los çibdadanos a uirtud Et pues que assi es por que paresça lo que deuemostener desta question } e qual es la soluçion della . \\\hline
3.2.31 & dato quod occurrant leges meliores et magis sufficientes , \textbf{ non est assuescendum innouare leges . } Primo , quia aliquando contingit & que sean falladas leyes meiores e mas conplidas . \textbf{ Enpero non nos auemos a acostunbrar a renouar las leyes . } Lo primero por que algunans vegadas contesçe que se engannan los omes \\\hline
3.2.31 & Immo magnam efficaciam habent ex diuturnitate et assuefactione . \textbf{ Decet ergo reges et principes obseruare bonas consuetudines principatus et regni , } et non innouare patrias leges , & Et por ende conuiene alos Reyes \textbf{ e alos prinçipes | de guardar las bueans costunbres del prinçipado e del regno } e non renouar las leyes dela tierra saluo \\\hline
3.2.31 & Decet ergo reges et principes obseruare bonas consuetudines principatus et regni , \textbf{ et non innouare patrias leges , } nisi fuerit rectae rationi contrariae . & de guardar las bueans costunbres del prinçipado e del regno \textbf{ e non renouar las leyes dela tierra saluo } si fuessen contrarias ala razon natural \\\hline
3.2.32 & et quomodo debeat \textbf{ se habere ad principantem , } non modicum amminiculetur & Mas commo para saber qual deua ser el puebło \textbf{ e commo se deua auer al prinçipe } e conuenga de saber \\\hline
3.2.32 & cum de legibus tractabamus , \textbf{ quod facere commutationes , } et contractus erant & quando fablauamos delas leyes \textbf{ que fazer mudaçiones } e contracto sera \\\hline
3.2.32 & quam in ciuitate una . \textbf{ Potest ergo sic diffiniri regnum , } quod est multitudo magna , & que en vna çibdat . \textbf{ Et pues que assi es el regno puede se | assi declarar e demostrar } diziendo \\\hline
3.2.33 & ad nimium diuites nesciunt se rationabiliter gereres insidiantur enim eis quomodo possint astute \textbf{ et latenter eorum depraedari bona . } Sed si in populo sint multae personae mediae , & Ca sienpre les asecha commo puedan faldridamente \textbf{ e encobiertamente tomar e robar de sus biens . } Mas si en el pueblo fueren muchas perssonas medianeras quedaran todo estos enpeesçimientos \\\hline
3.2.33 & sed diuites penitus volent principari , \textbf{ et suppeditare alios . } Alii vero contra nitentes dissensionem faciunt , & Ca los ricos en toda manera quarran en ssennorear \textbf{ e poner so pie alos otros } e los pobres contradiziendo alos ricos faran discordia en la çibdat \\\hline
3.2.34 & obedire Regibus et Principibus , \textbf{ et obseruare leges . } Primo enim ex hoc consequitur populus virtutes , & quanto es prouechoso e conuenible al pueblo de obedesçer alos Reyes \textbf{ e guardar las leyes . } Ca lo primero desto alçança el pueblo uirtudes e grandes bienes \\\hline
3.2.34 & Nam ( ut dicebatur in praecedentibus ) \textbf{ intentio legislatoris est inducere ciues ad virtutem . } In recta enim Politia & dich̃en los capitulos \textbf{ sobredichos la entençion del ponedor dela ley es enduzer los çibdadanos o uirtud . } Ca en la derecha poliçia \\\hline
3.2.34 & quia intentio eius est \textbf{ inducere alios ad virtutem , } cum virtus faciat habentem bonum ; & Por la qual cosa si el prinçipe gouernar e derechamente el pueblo qual es acomne dado \textbf{ por que la su entençion es enduzir los otros a uirtud . } Et la uirtud faze al que la ha buenon \\\hline
3.2.34 & et obedire Regibus , \textbf{ esse seruitutem . } Cum enim bestiae sint naturae seruilis : & aquellos que dizen que guardar las leyes \textbf{ e obedesçer alos Reyes es seruidunbre . } Ca commo las bestias sean de natura seruil \\\hline
3.2.34 & sic Rex si recte principetur est salus et vita regni . \textbf{ Quare sicut pessimum est corpori delinquere animam , } et non regi per eam , & enssennoreare es salud et uida del regno . \textbf{ Por la qual cosa | assi commo es muy mala cosa al cuerpo desmanparar el alma } e non se gouernar por ella . \\\hline
3.2.34 & Quare sicut pessimum est corpori delinquere animam , \textbf{ et non regi per eam , } sic pessimum est regno & assi commo es muy mala cosa al cuerpo desmanparar el alma \textbf{ e non se gouernar por ella . } assi es muy mala cosa \\\hline
3.2.34 & Expediens enim fuit regno et ciuitati \textbf{ habere aliquem Regem } vel aliquem principantem , & Et por ende cosa conuenible fue al regno \textbf{ e ala çibdat de auer algun Rey o algun prinçipe } por que los malos non pudiessen turbar la paz de los çibdadanos . \\\hline
3.2.34 & Nam sicut medicus intendit \textbf{ sedare humores , } ne insurgat morbus & Ca assi commo el fisico entiende de amanssar \textbf{ e de egualar los humores } por que se non le una te enfermedat̃ \\\hline
3.2.34 & et bellum in corpore : \textbf{ sic legislator intendit placare corda , } sedare animas , & nin batalla dellos en el cuerpo . \textbf{ assi el fazedor delas leyes entiende de amanssar los coraçones } e abenir las almas \\\hline
3.2.34 & sic legislator intendit placare corda , \textbf{ sedare animas , } ne insurgat rixa & assi el fazedor delas leyes entiende de amanssar los coraçones \textbf{ e abenir las almas } porque se non le uate pelea nin contienda en el regno o en la çibdat . \\\hline
3.2.35 & ut non incurrant regiam iram , \textbf{ non forefacere in ipsum Regem . } Regi autem duo debentur , & por que non cayan en sanna del reyes \textbf{ non fazer ninguna cosa | mala contra el Rey } Ca al Rey deuemos dos cosas . \\\hline
3.2.35 & ad ipsum spectat per se \textbf{ et per alios dirigere eos , } qui sunt in regno . & Et por ende a el parte nesçe \textbf{ prinçipalmente gouernar e gniar todos } los que son en el regno \\\hline
3.2.35 & Ratione vero , \textbf{ quia ipsius est dirigere alios , } debetur ei subiectio et obedientia . & que los otros ael deue ser dada honrra e reuerençia . \textbf{ Mas por razon que a el parte nesçe degniar los otros } ael deue ser fecha subiectiuo e obediençia \\\hline
3.2.35 & Viso quomodo habitatores regni non debent \textbf{ prouocare Regem ad iram , } forefaciendo in ipsum , & Visto en qual manera los moradores del regno \textbf{ non deuen mouer el Rey a saña errando contra el } e non le faziendo \\\hline
3.2.35 & et qui pertinent ad ipsum . \textbf{ Ad Regem autem pertinere videntur quatuor genera personarum } videlicet parentes et uniuersaliter omnes cognati , & Et deuedes saber \textbf{ que al Rey parte nesçen quatro maneras de perssonas . | Conuiene a saber . } El padir . Et la madre . \\\hline
3.2.35 & ad iram prouocare , \textbf{ non solum non forefacere in ipsum Regem , } sed etiam non forefacere in cognatos , & quasieren mouer al Rey a saña \textbf{ non solamente de non fazer ningun tuerto contra el rey en su perssona . } Mas avn de non fazer contra sus parientes \\\hline
3.2.35 & non solum non forefacere in ipsum Regem , \textbf{ sed etiam non forefacere in cognatos , } uxorem , filios , & non solamente de non fazer ningun tuerto contra el rey en su perssona . \textbf{ Mas avn de non fazer contra sus parientes } nin contra su muger \\\hline
3.2.35 & instruere eos \textbf{ quomodo debeant honorare Regem , } obedire ei : & que amen al Rey \textbf{ e queles enssennen en qual manera de una honnar al Rey } e obedescerle \\\hline
3.2.35 & obedire ei : \textbf{ non forefacere in cognatos eius , } nec in filios , & e obedescerle \textbf{ e non fazer tuerto contra los sus parientes } nin contra sus fijos \\\hline
3.2.35 & ( ut dicitur 7 Pol’ ) \textbf{ non instruere pueros ad virtutem , } et obseruantiam legum utilium : & en el quinto libro delas positicas \textbf{ non enssennar los mocos auertudes } e aguarda delas leyes prouechosas \\\hline
3.2.36 & Timet igitur tunc quilibet ex populo forefacere , \textbf{ cogitans se non posse punitionem effugere . } Imo , ut vult Philos’ 7 Polit’ & Et pues que assi es cada vno del pueblo teme de mal fazer cuydando \textbf{ que non podra escapar dela pena . } Ante assi conmo dize el philosofo \\\hline
3.2.36 & et cuiuscumque principantis esse debet , \textbf{ inducere alios ad virtutem . } Omne ergo bonum & e de cada vn prinçipe deue ser \textbf{ e non duzir alos otros a uirtud . } Et pues que assi es todo bien fazer \\\hline
3.2.36 & et ex dilectione legislatoris , \textbf{ cuius est intendere commune bonum , } quiescant male agere : & e por amor del prinçipe ponedor dela ley \textbf{ cuya entençiones de tener mientes al bien comun } que por ende queden los omes de mal fazer . \\\hline
3.3.1 & scire regere seipsum , \textbf{ quam scire regere familiam , } et ciuitatem , aut regnum . & Ca menos es saber gouernar a ssi mismo \textbf{ que saber gouernar la conpaña de casa o la cibdat o el regno . } La segunda manera de la prudençia es dicha yconomica \\\hline
3.3.1 & quia scit bene consiliari , \textbf{ et bene dirigere ad bonum finem : } ubi ergo reperitur alia & por que sabe bien consseiar \textbf{ e bien guiar a buena fin . } Et pues que assi es do son falladas \\\hline
3.3.1 & oeconomicam prudentiam , \textbf{ per quam quis scit regere domum et familiam , } oportet esse aliam a prudentia , & Et por ende la sabiduria \textbf{ por la qual cada vno sabe gouernar la casa e la conpaña . } Conuiene que sea otra e departida de la sabiduria \\\hline
3.3.1 & cuius est leges ferre , \textbf{ et regere regnum et ciuitatem , } est alia a prudentia oeconomica & a quien pertenesçe de fazer leyes \textbf{ e de gouernar el regno e la çibdat . } esta es departida de la sabiduria yconomica \\\hline
3.3.1 & quae requiritur in patrefamilias , \textbf{ cuius est gubernare domum : } immo quanto bonum ciuitatis & que es menester en el padre familias \textbf{ a quien pertenesçe de gouernar la casa . } Mas en quanto el bien de la çibdat e del regno \\\hline
3.3.1 & in Rege oportet \textbf{ excedere prudentiam patrisfamilias , } vel prudentiam alicuius particularis hominis . & que pertenesçe al Rey deue \textbf{ sobrepuiar la sabiduria | del gouernamiento de vna casa } o la sabiduria de algun omne particular . \\\hline
3.3.1 & ut est paterfamilias , \textbf{ et ut habet dispensare bona domestica . } In tertio vero eruditur Rex aut Princeps & en quanto es padre familias \textbf{ que ha de gouernar la casa | e ha de despenssar los bienes de la casa . } Et enel teçero libro ensseñamos al Rey o al prinçipe \\\hline
3.3.1 & adhuc oporteret \textbf{ ipsum habere aliqualem prudentiam } qua sciret se regere et gubernare : & e morasse solo avn conuenir le \textbf{ ya de auer alguna sabiduria } por la qual se sopiesse gouernar . \\\hline
3.3.1 & bene se habere in opere bellico , \textbf{ et per actiones bellicas opprimere impedimenta hostium : } ex consequenti vero spectat ad ipsos & assi a los caualleros pertenesçe principalmente de se auer bien en obras de batallas . \textbf{ Et por las obras de batallas desenbargar todos aquellos enbargos } que pueden venir de los enemigos . \\\hline
3.3.1 & et secundum mandata principantis \textbf{ impedire omnes seditiones ciuium } et omnes oppressiones eorum & Et de si a ellos pertenesce desenbargar \textbf{ e tirar todas las discordias de los çibdadanos } e todos agrauiamientos \\\hline
3.3.1 & qui sunt in regno , \textbf{ per quas turbari potest tranquillitas ciuium et commune bonum . } Hanc autem prudentiam videlicet militarem , & aquellos que son el regno segunt \textbf{ por las quales cosas se puede turbar la paz . | el assessiego de los çibdadanos e el bien comun . } Et esto deue fazer los caualleros \\\hline
3.3.1 & maxime decet habere Regem . \textbf{ Nam licet executio bellorum , et remouere impedimenta ipsius communis boni , } spectet ad ipsos milites , & Ca commo quier que pertenezca a los caualleros \textbf{ la essecuçion de las batallas | e tirar e arredrar los enbargos del bien comun . } Et tales cosas commo estas pertenezcan a aquellos \\\hline
3.3.1 & spectet ad ipsos milites , \textbf{ et etiam ad eos quibus ipse Rex aut Princeps voluerit committere talia : } scire tamen quomodo committenda sint bella , & Et tales cosas commo estas pertenezcan a aquellos \textbf{ a quien lo quisiere acomendar el rey | o el prinçipe } Enpero saber en qual manera son de acometer las batallas \\\hline
3.3.1 & et etiam ad eos quibus ipse Rex aut Princeps voluerit committere talia : \textbf{ scire tamen quomodo committenda sint bella , } et qualiter caute remoueri possint & o el prinçipe \textbf{ Enpero saber en qual manera son de acometer las batallas } e en qual manera se pueden sabiamente tirar e arredrar los enbargos \\\hline
3.3.1 & secundum iussionem principantis \textbf{ impedire seditiones ciuium , } pugnare pro iustitia et pro iuribus , & que el sea bueno en la obra de la batalla \textbf{ e que quiera segunt el mandamiento del prinçipe desenbargar las discordias de los cibdidanos } e lidiar por la iustiçia \\\hline
3.3.1 & omnem bellicam operationem contineri sub militari . \textbf{ Nam licet bellare contingat homines pedites , } vel etiam equestres non existentes milites : & so la arte de la caualleria \textbf{ ca commo quier que contezca de lidiar los peones } e los omnes de cauallo \\\hline
3.3.2 & in quibus regionibus meliores sunt bellatores , \textbf{ oportet attendere circa praedicta duo . } In partibus igitur nimis propinquis soli , & o en quales tierras son meiores lidiadores . \textbf{ Conuiene de tener mientes en estas dos cosas sobredichas . } Et pues que asy es en las partes \\\hline
3.3.2 & Sciendum ergo quod cum bellantes debeant \textbf{ habere membra apta } et assueta ad percutiendum , & Et pues que assi es conuiene de saber \textbf{ que commo los lidiadores deuan auer los mienbros apareiados } e acostunbrados a ferir \\\hline
3.3.2 & et assueta ad percutiendum , \textbf{ non debeant horrere sanguinis effusionem , } debeant esse animosi ad inuadendum , & e acostunbrados a ferir \textbf{ et non deuan | aborresçer el derramamiento de la sangre } e deuan ser animosos \\\hline
3.3.2 & Nam nunquam bene vibrant clauam , \textbf{ aut ensem qui debet habere manum leuem , } et non est assuetus retinere & por que nunca bien leuantar a la maça \textbf{ nin esgrimira la espada | aquel que deue auer la mano liuiana . } Et non es acostubrado de tener en la mano \\\hline
3.3.3 & si vult legislator ciues bellatores facere , \textbf{ et reddere ipsos aptos ad pugnandum , } potius debet & Et si quisiere el ponedor de la ley fazer los çibdadanos buenos lidiadores \textbf{ e fazer los apareiados para la batalla } deue ante tomar el tienpo de la mançebia \\\hline
3.3.3 & esse videtur armorum industria . \textbf{ Nam siue equitem siue peditem oportet esse bellantem , } quasi fortuito videtur & nin ligera arte auer sabiduria de las armas . \textbf{ Ca si quier sea cauallero si quier peon el que ha de lidiar paresçe } que por uentura alcaça uictoria \\\hline
3.3.3 & Quare si legislator \textbf{ ut Rex aut Princeps debeat committere bellum , } viros exercitatos et bellatores strenuos debet assumere . & Por la qual cosa si el fazedor de la ley \textbf{ assi commo el Rey o el prinçipe ouiere de acometer batalla deue tomar e escoger varones vsados } e lidiadores escogidos e fuertes . \\\hline
3.3.3 & videre restat , \textbf{ ex quibus signis cognosci habeant homines bellicosi . } Sciendum igitur viros audaces et cordatos & aquellos que se deuen fazir caualleros e ser lidiadores . \textbf{ finca de ver por quales señales se han de conosçer los buenos lidiadores . } Et para esto conuiene de saber \\\hline
3.3.3 & Tribus igitur generibus signorum \textbf{ cognoscere possumus bellicosos viros . } Primo quidem per signa , & por tres maneras de señales \textbf{ podemos conosçer los omnes lidiadores } Lo primero por aquellas señales \\\hline
3.3.4 & enumerare possumus octo , \textbf{ quae habere debent homines bellatores : } secundum quae ( quantum ad praesens spectat ) & q quanto pertenesçe a lo presente ocho cosas podemos contar \textbf{ que deuen auer los omnes lidiadores } segunt las quales cosas \\\hline
3.3.4 & Primo enim oportet pugnatiuos homines posse \textbf{ sustinere magnitudinem ponderis . } Secundo posse sufferre & Lo primero conuiene \textbf{ que los omnes lidiadores puedan sofrir grandes pesos . } lo segundo que puedan sofrir grandes trabaios \\\hline
3.3.4 & et labores magnos . \textbf{ Tertio posse tolerare parcitatem victus . } Quarto non curare de incommoditate iacendi et standi . & e continuados mouimientos de los mienbros \textbf{ Lo terçero que puedan sofrir | escasseza de vianda e fanbre e sed . } Lo quarto que non aya cuydado de mal yazer \\\hline
3.3.4 & Quinto quasi non appretiare corporalem vitam . \textbf{ Sexto non horrere sanguinis effusionem . } Septimo habere aptitudinem , & que por razon de la iustiçia e del bien comun despreçien la uida corporal . \textbf{ Lo sexto que non teman | nin aborrezcan de derramar su } sangreLo septimo conuiene \\\hline
3.3.4 & et industriam ad protegendum se et feriendum alios . \textbf{ Octauo verecundari et erubescere eligere turpem fugam . } Est enim primo necessarium bellantibus posse & e para ferir a los enemigos . \textbf{ lo octauo que ayan uerguença de foyr torpemente . } Pues que assi es \\\hline
3.3.4 & Est enim primo necessarium bellantibus posse \textbf{ sustinere ponderis magnitudinem . } Nam inermes a quacunque parte foriantur , & Lo primero que es meester a los lidiadores \textbf{ es que puedan sofrir grandes pesos . } Ca los desarmados \\\hline
3.3.4 & et commune bonum \textbf{ quasi non appretiari corporalem vitam . } Nam cum tota operatio bellica exposita sit periculis mortis , & Lo quinto conuiene a los lidadores \textbf{ de non preçiar la vianda corporal | por la iustiçia e por el bien comun . } Ca commo toda la hueste sea puesta \\\hline
3.3.4 & Sexto pugnantes non debent \textbf{ horrere sanguinis effusionem . } Nam si quis cor molle habens , & Lo sexto los lidiadores non deuen \textbf{ aborresçer el derramamiento de la sangre } por que si alguno ouiere el coraçon muele \\\hline
3.3.4 & muliebris existens , \textbf{ horreat effundere sanguinem ; } non audebit hostibus plagas infligere , & e fuere assi commo mugeril \textbf{ aborresçra esparzer la sangre } e non osara fazer llagas a los enemigos . \\\hline
3.3.4 & et per consequens bene bellare non potest . \textbf{ Septimo decet eos habere aptitudinem , } et industriam ad protegendum se , & que non podra bien lidiar . \textbf{ Lo vij° . conuiene a los lidiadores de auer disposiçion e sabiduria para cobrirse e defender se } e para ferir a los otros . \\\hline
3.3.4 & Octauo decet bellatores verecundari , \textbf{ et erubescere turpem fugam . } Nam , ut dicitur 3 Ethic’ & es de auer uerguença \textbf{ e de guardar se | de non foyr torpemente de la batalla . } Ca assi commo dize el philosofo \\\hline
3.3.4 & est diligere honorari expugna , \textbf{ et erubescere turpem fugam . } Aduertendum autem quod cum dicimus , & de ser honrado por batalla \textbf{ e de auer uerguenna de foyr | torpemente de la batalla . } Mas deuemos parar mientes \\\hline
3.3.4 & Aduertendum autem quod cum dicimus , \textbf{ bellatores non habere effusionem sanguinis , } non debere multum appretiari corporalem vitam , & que quando dezimos \textbf{ que los lidiadores non deuen aborresçer el esparzimiento de la sangre } nin deuen mucho preçiar la vida corporal \\\hline
3.3.4 & bellatores non habere effusionem sanguinis , \textbf{ non debere multum appretiari corporalem vitam , } et caetera quae diffusius connumerauimus : & que los lidiadores non deuen aborresçer el esparzimiento de la sangre \textbf{ nin deuen mucho preçiar la vida corporal } et las otras cosas tales \\\hline
3.3.5 & de iis quae requiruntur ad pugnam . \textbf{ Numeratis iis quae habere debent bellatores viri : } restat inquirere , & que partenesçen a la batalla . \textbf{ ontadasontadas cosas que los lidiadores deuen auer finca de demandar } quales son los meiores lidiadores . \\\hline
3.3.5 & viros pugnatiuos tales esse debere , \textbf{ qui possent sustinere magnitudinem ponderis , } continuum laborem membrorum , parcitatem victus , & que los omnes lidiadores sean tales \textbf{ que puedan sofrir grandes pesos } e grandes trabaios continuados en los sus cuerpos \\\hline
3.3.5 & et incommoditatem iacendi et standi , \textbf{ non timere mortem , } non horrere sanguinis effusionem , & e sofrir mal yazer e mal estar \textbf{ e non temer la muerte | e non temer } nin aborresçer el esparzimiento de la sangre \\\hline
3.3.5 & non timere mortem , \textbf{ non horrere sanguinis effusionem , } et cetera alia & e non temer \textbf{ nin aborresçer el esparzimiento de la sangre } e las otras cosas \\\hline
3.3.5 & sequitur hos meliores esse pugnantes . \textbf{ Videntur enim haec duo maxima esse ad obtinendam victoriam , } videlicet erubescentia fugiendi , & siguese que son meiores lidiadores . \textbf{ Ca paresçe que estas dos cosas son prinçipales | para auer victoria . } Conuiene a saber uerguença de foyr . \\\hline
3.3.5 & quem patiuntur nobiles \textbf{ in non posse tantos sustinere labores , } quantos consueuerunt sustinere rurales . & que han los nobles \textbf{ en non poder sofrir tantos trabaios } quantos se acostunbraron de sofrir los villanos . \\\hline
3.3.5 & in non posse tantos sustinere labores , \textbf{ quantos consueuerunt sustinere rurales . } In huiusmodi enim pugna & en non poder sofrir tantos trabaios \textbf{ quantos se acostunbraron de sofrir los villanos . } por que en tal batalla mucho vale la sabidura de lidiar \\\hline
3.3.7 & enumerare octo alia , \textbf{ ad quae exercitari debent homines bellicosi . } Primo enim exercitandi sunt & aque dixiemos que son de vsar los lidiadores contar otras ocho cosas \textbf{ aque se deuen vsar los lidiadores . } Lo primero se deuen vsar aleuar grandes pesos . \\\hline
3.3.7 & non est inutile \textbf{ assuescere bellatores . } Secundo exercitandi sunt bellantes & Et por ende prouechosa cosa es de se acostunbrar los lidiadores a leuar grandes pesos . \textbf{ Lo segundo se deuen usar los lidiadores } a acometer \\\hline
3.3.7 & et iuuenes quos volebant \textbf{ facere optimos bellatores exercitabant ad palos illos ita , } ut quilibet haberet scutum dupli ponderis & Et los moços que querian \textbf{ acostunbrara fazer los buenos lidiadores . | vsauan los a ferir en aquellos palos } assi que cada vno de aquellos moços \\\hline
3.3.7 & et solertem mentem latere non potest . \textbf{ Nam ascendere equos , } est proprium equitibus : & Ca non se puede asconder a ome sabio . \textbf{ Ca sobir en los cauallos pertenesçe a los caualleros . } Et lançar piedas con fondas pertenesçe a los peones . \\\hline
3.3.7 & est proprium equitibus : \textbf{ proiicere lapides cum funda , } videtur esse proprium peditibus . & Ca sobir en los cauallos pertenesçe a los caualleros . \textbf{ Et lançar piedas con fondas pertenesçe a los peones . } Mas las otras cosas en alguna manera puenden pertenesçer a todos . \\\hline
3.3.8 & quandocunque et undecunque superuenientes hostes obsideant . \textbf{ Debet enim exercitus secum ferre munitiones congruas , } ut cum castrametari voluerit , & e donde quier que vengan los enemigos acercar los o acometer los . \textbf{ Ca deue sienpre la hueste leuar consigo guarniciones conuenibles . } por que quando quisiere la hueste folgar en algun logar parezca \\\hline
3.3.8 & circa exercitum facere fossas \textbf{ et construere castra : } restat ostendere , & que lieuan consigo assi commo vna çibdat guarnida . \textbf{ Visto commo es cosa prouechable a la hueste fazer carcauas e costruir guarniçiones e castiellos . } finca de demostrar en qual manera las tales guarniciones \\\hline
3.3.8 & facile est fossas circa exercitum fodere , \textbf{ munitiones erigere et castra construere . } Sed si aduersarii praesentes adsint , & Ca si los enemigos non estudieren cerca de ligero pueden fazer carcauas çerca de la hueste \textbf{ e leuantar | guarnicoñes e fazer castiellos . } Mas si los enemigos fueren cerca \\\hline
3.3.8 & quod ipsum oporteat facere . \textbf{ Ostenso utile esse castra construere , } et qualiter etiam praesentibus hostibus construenda sint castra : & e manden a cada vno qual cosa deua fazer . \textbf{ Mostrado que prouechosa cosa es de fazer los castiellos . } avn en qual manera los enemigos presentes son de fazer los castiellos \\\hline
3.3.9 & etiam postquam gustauerunt bella , \textbf{ appetere pugnam ; } hoc est ut raro . & Mas si contesçiere que los lidiadores sean de carnes blandas \textbf{ avn que ayan prouado las batallas pocas vezes quieren lidiar . } Ca assi commo dixiemos de suso \\\hline
3.3.9 & aut deficere : \textbf{ poterit accelerare pugnam , } vel differre : & ha conplimiento en estas seys condiçiones \textbf{ et fallesçe en ellas podra acometer la vatalla mas ayna o prolongar la } e lidiar publicamente o manifiestamente \\\hline
3.3.9 & vel deficere , \textbf{ sic se habere poterit erga bellum : } forte enim nunquam contingeret & que abonda o fallesçe en las mas destas condiçiones \textbf{ assi se podra auer en la batalla } e por auentura nunca contezçra que todas estas condiçones puedan ser de la vna parte . \\\hline
3.3.10 & non sufficiunt ad dirigendum bellantes , \textbf{ sed oportet dare euidentia signa ; } ut quilibet solo intuitu sciat & para guiar los lidiadores . \textbf{ Mas conuiene de dar otras seña les manifiestas . por que cada vno viendo aquellas señales } se sepa tener ordenadamente en su az \\\hline
3.3.10 & proceri statura , \textbf{ scientes proiicere hastas et tela : } scire etiam debeant & e altos en el estado del cuerpo \textbf{ e que sepan lançar lanças e dardos . } e avn que sepan esgrimir las espadas \\\hline
3.3.10 & gladium vibrare ad percutiendum , \textbf{ portare scutum ad se protegendum : } et cum debeant esse vigilantes , agiles , sobrii , & para ferir meior \textbf{ e rodear el escudo | para encobrirse meior } e avn que ayan los oios bien espiertos \\\hline
3.3.10 & portare scutum ad se protegendum : \textbf{ et cum debeant esse vigilantes , agiles , sobrii , } habentes armorum experientiam : & para encobrirse meior \textbf{ e avn que ayan los oios bien espiertos | e que sean ligeros e mesurados en beuer e gerrdados de vino } e avn que ayan vso de las armas . \\\hline
3.3.10 & procer statura , \textbf{ sciens eiicere hastas et iacula , } sciens dimicare gladio ad percutiendum , & deue ser fuerte en el cuerpo . \textbf{ grande en su estado e sabidor en lançar lanças e dardos } e sabio en lidiar \\\hline
3.3.10 & sciens dimicare gladio ad percutiendum , \textbf{ portare scutum ad se protegendum , } vigilans , agilis , sobrius , & e sepa esgrimir el espada \textbf{ para ferir meior . | Rodearse e cobrirse del escudo } para se guardar e despierto e vigilante e ligero e mesurado \\\hline
3.3.11 & apponere custodias \textbf{ ne possint fugere . Debet etiam eis minari mortem , } si in aliquo fraudulenter se habeant , & deue el señor de la hueste poner en ellos buenas guardas \textbf{ porque non puedan foyr . | Avn deuen los amenazar de muerte } si en alguna cosa se ouieren engañosamente \\\hline
3.3.11 & et in qualibet acie \textbf{ habere aliquos equites fidelissimos et strenuissimos , } habentes equos veloces et fortes ; & que deue el señor de la hueste en cada conpaña \textbf{ e en cada vna az auer vnos caualleros muy fieles | e muy estremados } que ayan cauallos muy ligeros \\\hline
3.3.11 & vel in peditibus , \textbf{ eligere poterit vias campestres et amplas , } vel montanas , syluestres , et nemorosas , & que ha conplimiento de caualleros o de peones \textbf{ podra escoger los caminos de los canpos | e carreras anchas o las de los montes } o de las siluas \\\hline
3.3.12 & nisi ut doceamus ordinare acies , \textbf{ percutere aduersarios , } et inuadere hostes . & si non que mostremos en commo se deuen ordenar las azes \textbf{ e ferir los contrarios } e acometer los enemigos . \\\hline
3.3.12 & His visis sciendum quadrangularem formam aciei \textbf{ inter caeteras formas esse magis inutilem : } ideo secundum hanc formam nunquam formanda est acies simpliciter , & Et vistas estas cosas \textbf{ conuiene de saber | que entre todas las otras formas de la az la quadrada es mas sin prouecho . } Et por ende nunca es de formar el az \\\hline
3.3.12 & reseruentur aliqui milites strenui et audaces , \textbf{ qui possint succurrere ad partem illam , } erga quam viderint & sean guardados algunos estremados caualleros \textbf{ e osados que puedan acorrer a aquella parte } que vieren \\\hline
3.3.13 & Inde est quod bellorum experti dicunt pugnantes \textbf{ semper debere habere loricas amplas ita , } ut annuli loricarum se constringant : & que los que son prouados en las batallas . \textbf{ dizen que los lidiadores sienpre deuen auer las lorigas anchas . } assi que les aniellos de las lorigas se ayunten \\\hline
3.3.13 & In percutiendo autem caesim , \textbf{ quia oportet fieri magnum brachiorum motum prius quam infligatur plaga , } aduersarius ex longinquo potest prouidere vulnus , & Mas en feriendo cortando . \textbf{ por que conuiene de fazer grand mouimiento de los braços | ante que se de el colpe el enemigo } o el contrario de \\\hline
3.3.13 & ideo magis sibi cauere potest \textbf{ et cooperire se ictibus . } Ideo ait Vegetius , & Et por ende puede se mas guardar \textbf{ e encobrirse de aquellos colpes . } Et por esso dize vegeçio \\\hline
3.3.13 & Percutiendo enim caesim oportet \textbf{ eleuare brachium dextrum : } quo eleuato dextrum latus nudatur et discooperitur , & por que firiendo taiando \textbf{ conuiene de leuantar el braço derecho e diestro . } Et leuantando el braço derecho paresçe descubierto el costado derecho \\\hline
3.3.14 & quia sic facilius deuincentur . \textbf{ Tertio debet aspicere ad ipsum tempus : } quando sol reuerberat & mas ligeramente seran vençidos . \textbf{ Lo terçero deue catar el cabdiello al tienpo . } assi que quando el sol fiere en los oios de los enemigos \\\hline
3.3.14 & vel per se , \textbf{ vel per alios mittere dissensiones , iurgia , } commouere eos ad lites , & por otros discordias entre los enemigos \textbf{ e boluer contiendas } e lides o enemistades entre ellos \\\hline
3.3.14 & Septimo debet diligenter \textbf{ explorare conditiones hostium : } qualiter se gerant , & deue el cabdiello \textbf{ con grant acuçia escudriñar las condiçiones de los sus enemigos } en qual manera andan \\\hline
3.3.15 & quod manu ad manum se percutiunt . \textbf{ Aliter autem debent stare bellatores viri , } cum a remotis iacula iaciunt , & quando vienen a las manos \textbf{ e en otra manera deuen estar los lidiadores } quando lançan los dardos de lueñe . \\\hline
3.3.15 & Nam iaciendo iacula a remotis , \textbf{ debent habere ipsos pedes sinistros ante , } et dextros retro . & Ca lançando dardos de lueñe \textbf{ deuen tener los pies esquierdos } delante e los derechos detras \\\hline
3.3.15 & cum ad manum pugnant , \textbf{ tenere pedem sinistrum immobiliter : } et cum volunt percutere , & Ca deuen los lidiadores \textbf{ quando vienen a las manos tener el pie esquierdo firme } e quando quieren ferir \\\hline
3.3.15 & Inde est quod laudatur Scipionis sententia , dicentis : \textbf{ Nunquam sic esse claudendos hostes , } quod non pateat eis aditus fugiendi . & Et por ende es alabada la sentençia de çipion \textbf{ por la qual dizia que nunca eran de encerrar los enemigos } assi que les non fincasse logar para foyr . \\\hline
3.3.15 & Cum ergo supra diximus , \textbf{ formandam aliquando esse aciem sub forma forficulari , } ut quando hostes pauci , & Et pues que assi es quando dixiemos dessuso que \textbf{ alguans vezes el az es de formar so forma de tiieras e esto quando los enemigos son pocos } para que meior sean ençerrados e çerrados . \\\hline
3.3.15 & ex parte exercitus hostium . \textbf{ Nam sic debet deducere bellum , } ut hoc hostes lateat . & Mas la segunda cautela es de tomar de parte de la hueste de los enemigos . \textbf{ Ca assi deue escusar la batalla | pues non ha conseio de lidiar } que esto non lo sepan los enemigos . \\\hline
3.3.15 & qua recedente , equites postea melius possunt \textbf{ vitare hostium percussiones . } Est etiam aduertendum & encubiertamente se escusa yendo se los peones . \textbf{ Et ellos ydos los caualleros pueden meior despues escusar los colpes de los enemigos . } Avn conuiene de saber \\\hline
3.3.16 & quod hostes de munitionibus exeuntes vadant \textbf{ bellare ad campum , } sed ipsi munitiones inuadunt & que los enemigos salgan a ellos de las villas \textbf{ e de las çibdades | e de los castiellos a lidiar al canpo . } Mas ellos acometen aquellas villas \\\hline
3.3.16 & et ne terrae marinae impugnentur , \textbf{ expedit regibus et principibus aliquando ordinare bella naualia . } Dicto itaque de bello campestri , & e a los principes \textbf{ algunas vezes de ordenar | e de fazer batallas nauales e de naues . } Et pues que assi es dicho de la lid canpal \\\hline
3.3.16 & cum per huiusmodi pugnam \textbf{ contingat obtineri et deuinci munitiones et urbanitates : } restat dicere quot modis talia deuinci possunt . & por que por tal lid \textbf{ contesçe tomar e vençer las villas | e los castiellos e fortalezas . } fincanos de dezir \\\hline
3.3.16 & contingat obtineri et deuinci munitiones et urbanitates : \textbf{ restat dicere quot modis talia deuinci possunt . } Est autem triplex modus obtinendi & e los castiellos e fortalezas . \textbf{ fincanos de dezir | en quantas maneras tales fortalezas pueden ser vençidas . } Et conuiene de saber \\\hline
3.3.16 & vel munitiones reddere . \textbf{ Quare diligenter excogitare debent obsidentes munitiones aliquas , } utrum per aliqua ingenia , & de sedo de dar las fortalezas . \textbf{ Por la qual cosa con grant acuçia deuen cuydar los | que cercan algunas fortalezas } si por algunos engeñios \\\hline
3.3.16 & quare si sit a munitionibus remota , \textbf{ debent obsidentes adhibere omnem diligentiam , } quomodo possint obsessis prohibere aquam . & por la qual cosa si el agua fuere lueñe de la fortaleza \textbf{ los que cercan deuen auer grant acuçia } en commo defiendan el agua a los cercado o gela tiren . \\\hline
3.3.16 & Inde est quod multotiens obsidentes \textbf{ volentes citius opprimere munitiones , } si contingat eos capere aliquos de obsessis , & Et por ende contesçe que muchas uegadas \textbf{ los que çercan queriendo | mas ayna ganar las fortalezas } si contezca \\\hline
3.3.16 & quo tempore melius est \textbf{ obsidere ciuitates et castra . } Sciendum itaque quod tempore aestiuo & las fortalezas cerradas \textbf{ finca de demostrar en que tienpo es meior de çercar las çibdades e las castiellos . } Et por ende conuiene de saber \\\hline
3.3.16 & Nam si per sitim sunt munitiones obtinendae , \textbf{ melius est facere obsessionem tempore aestiuo , } eo quod tunc magis desiccantur aquae , & Ca si por sed son de ganar las fortalezas \textbf{ meior es de fazer la çerca | en el tienpo del estiuo } por que entonçe mas se dessecan las aguas \\\hline
3.3.16 & Quare si obsessi non possunt \textbf{ gaudere fructibus anni aduenientis , } citius peribunt inopia . & Por la qual cosa si los que estan çercados \textbf{ non se pueden acorrer de los fuctos | de esse aneo en que esta . } Mas ayna pereres çran \\\hline
3.3.17 & Nam cum contingat obsessiones \textbf{ per multa aliquando durare tempora , } non est possibile obsidentes & Ca commo contezca \textbf{ que las çercas puedan durar algunas vezes } e por muchos tienpos non puede ser \\\hline
3.3.17 & vel iaculi debent castrametari , \textbf{ et circa se facere fossas , } et figere ibi ligna , & quanto podrie lançar la vallesta o el dardo \textbf{ e fazer carcauas enderredor de ssi } e finçar y grandes palos \\\hline
3.3.17 & et circa se facere fossas , \textbf{ et figere ibi ligna , } et construere propugnacula : & e fazer carcauas enderredor de ssi \textbf{ e finçar y grandes palos } e fazer algunas fortalezas \\\hline
3.3.17 & et construere propugnacula : \textbf{ ut si oppidani eos repente vellent inuadere , resistentiam inuenirent . } Viso quomodo se munire debent obsidentes , & assi que si los que estan çercados \textbf{ a desora los quisieren acometer fallen enbargo | por que los non puedan enpesçer . } visto en qual manera se deuen guarnesçer los çercadores \\\hline
3.3.17 & restat ostendere \textbf{ quot modis impugnare debent obsessos . } Est autem unus modus impugnandi communis et publicus , & finca de demostrar \textbf{ en quantas maneras se deuen acometer | los que estan cercados } et ay vna manera comun e publica de acometer e de lidiar \\\hline
3.3.17 & apponunt scalas ad muros , \textbf{ ut si possint ascendere ad partes illas . } Praeter tamen hos modos impugnationis apertos , & e ponen escaleras a los muros \textbf{ assi que si podieren sobir sean eguales dellos | para se dar con ellos } e para entrar los . \\\hline
3.3.17 & quam sint fossae munitionis deuincendae , \textbf{ pergere usque ad muros munitionis praedictae : } quod si hoc fieri potest , & que esta cercada . \textbf{ Et assi deuen yr | fasta los muros de aquel logar . } Et si esto se puede fazer \\\hline
3.3.17 & quod per solum casum murorum possint munitionem obtinere , \textbf{ statim debent apponere ignem in lignis sustinentibus muros } et facere omnes muros & luego sin detenemiento \textbf{ ninguno deuen poner fuego en la madera | que sotienen los muros } e fazer que todos los muros o grand parte dellos cayan en vno a desora . \\\hline
3.3.17 & et facere omnes muros \textbf{ vel facere magnam eorum partem cadere , } et replere fossas : & que sotienen los muros \textbf{ e fazer que todos los muros o grand parte dellos cayan en vno a desora . } Otrossi deuen fenchir las carcauas \\\hline
3.3.17 & vel facere magnam eorum partem cadere , \textbf{ et replere fossas : } quo simul ( quasi ex inopinato facto ) & e fazer que todos los muros o grand parte dellos cayan en vno a desora . \textbf{ Otrossi deuen fenchir las carcauas } assi que los que estan cercados a desora sean espantados \\\hline
3.3.17 & ut per eas possit \textbf{ haberi ingressus ad ciuitatem et castrum : } quae omnia latenter fieri possunt & por las cueuas soterrañas \textbf{ assi que por ellas puedan entrar a la çibdat o al castiello . } Et estas cosas todas deuense fazer muy encubiertamente \\\hline
3.3.17 & vel ciuitatem obsessam : \textbf{ et sic poterunt obtinere illam . } Contingit autem pluries , & en el castiello çercado \textbf{ e assi podran ganar aquellas fortalezas . } m muchas uegadas contesçe que algunas fortalezas çercadas son fundadas sobre pennas muy fuertes \\\hline
3.3.18 & Nam in omni tali machina est \textbf{ dare aliquid trahens } et eleuans virgam machinae , & puedense adozer a quatro maneras . \textbf{ Ca en todo tal engeñio es de dar alguna cosa que traya } e leuante el pertegal del engennio \\\hline
3.3.18 & Est etiam aduertendum \textbf{ quod die et nocte per lapidarias machinas impugnari possunt munitiones obsessae . } Tamen , ut videatur qualiter in nocte percutiunt lapides emissi a machinis , & Et avn conuiene de saber \textbf{ que tan bien de noche commo de dia se pueden acometer las fortalezas cercadas | por los engeñios } que lançan piedras . \\\hline
3.3.19 & ad impugnandum munitionem aliquam , \textbf{ dato quod quis non possit pertingere usque ad muros eius . } Nam quia huiusmodi trabs habens caput sic ferratum retrahitur et impingitur , & Et uale este artifiçio para acometer alguna fortaleza . \textbf{ puesto que non puedan llegar a los muros della . } Ca por que esta viga ha la cabeça \\\hline
3.3.19 & Nam si nec per arietes , \textbf{ nec per vineas capi possunt munitiones obsessae , } accipienda est mensura murorum munitionis illius , & Ca si las fortalezas cercadas non se pueden tomar par los carneros \textbf{ nin por las viñas sobredichas deuen tomarla mesura } e el alteza de los muros de aquella fortaleza \\\hline
3.3.19 & ad muros munitionis obsessae , \textbf{ illi qui sunt in parte superiori debent proiicere lapides , } et fugare eos , & quanto deuen a los muros de la fortaleza los cercados \textbf{ aquellos que estan en la parte mas alta deuen lançar piedras } e fazer foyr \\\hline
3.3.19 & illi qui sunt in parte superiori debent proiicere lapides , \textbf{ et fugare eos , } qui sunt in muris . & aquellos que estan en la parte mas alta deuen lançar piedras \textbf{ e fazer foyr } los que estan en los muros \\\hline
3.3.19 & debent pontes dimittere , \textbf{ et inuadere muros . } Sed qui sunt in parte infima et sub musculis , & mas los que estan en el soberado de medio deuen echar puentes \textbf{ e acometer por los adarues . } mas los que estan en la parte mas baxa \\\hline
3.3.19 & ut etiam et sic obsidentes \textbf{ intrare possint obsessam munitionem . } Sunt etiam , ballistae arcus , & e foradar los \textbf{ por que puedan entrar en la fortaleza cercada } Avn son menester ballestas \\\hline
3.3.20 & ut postquam docuimus obsidentes qualiter debeant inuadere obsessos , \textbf{ volumus docere ipsos obsessos qualiter } se debeant & en qual manera \textbf{ los que çercan deuen acometer los | cercados queremos ensseñar en qual } lomanera los cercados se deuen defender de los que çercan . \\\hline
3.3.20 & ut obsessi faciliter possint \textbf{ defendere munitionem aliquam , } est scire , & e lo que mas faze \textbf{ para que los çercados ligeramente puedan defender las fortalezas } e saber en qual manera son de construyr \\\hline
3.3.20 & Rursus supra cataractam debet \textbf{ esse murus perforatus recipiens ipsam , } per quem locum poterunt proiici lapides , & Otrossi sobre las puertas de la trayçion \textbf{ deue ser el muro foradado de guisa | que la puedan leuantar arriba et baxarla cada que quisieren . } Et por aquel logar pueden lançar piedras . \\\hline
3.3.20 & per quem locum poterunt proiici lapides , \textbf{ emitti poterit aqua ad extinguendum ignem , } si contingeret ipsum ab obsidentibus esse appositum . & Et por aquel logar pueden lançar piedras . \textbf{ e echar agua | por matar el fuego } si contesçiesse que los enemigos pusiessen fuego a las puertas . \\\hline
3.3.21 & Dicebatur enim supra , \textbf{ triplicem esse modum deuincendi munitiones : } videlicet per famem , sitim , et pugnam . & por que non puedan de ligero ser tomadas . \textbf{ Et dicho fue dessuso que tres maneras ay para tomar las fortalezas . } Conuiene de saber . \\\hline
3.3.21 & Per ferra vero etiam reparari possint arma , \textbf{ et fieri tela ; } et sagittae , et alia per quae impugnari valeant obsidentes . & Et del fierro puedan las armas \textbf{ e fazer fierros de dardos et de saetas | e las otras cosas } que son menester \\\hline
3.3.21 & quod si nerui deficiant , \textbf{ loco eorum adhiberi poterunt crines equi , } vel capilli mulierum . & E si por auentura fallesçieren los neruios en logar \textbf{ dellos pueden tomar las çerneias | e las colas dellos cauallos } e los cabellos de las mugeres , \\\hline
3.3.22 & Quare si docuimus per praefatos modos \textbf{ inuadere obsidentes obsessos : } reliquum est & por las maneras sobredichas \textbf{ aquellos que çercan commo han de acometer a los que estan cercados . } fincanos de demostrar \\\hline
3.3.22 & profundae foueae aquis repletae , impediuntur obsidentes ; \textbf{ ne obsessos impugnare possint per cuniculos , } et vias subterraneas . & e muy fondas enbargansse los que çercan \textbf{ por que non puedan acometer los | quiestan cercados } por carteras soterrañas . \\\hline
3.3.22 & et vias subterraneas est , \textbf{ facere in munitione obsessa viam aliam } correspondentem viae subterraneae factae ab obsidentibus . & El segundo remedio contra las cueuas conegeras \textbf{ e contra las carreras soterrañas es de fazer vna fortaleza cercada otra carrera } que responda a la carrera soterraña \\\hline
3.3.22 & et utrum per aliqua signa cognoscere possint \textbf{ obsidentes inchoare cuniculos : } quod cum perceperint , & o si por algunas señales pudieren conosçer \textbf{ que los que cercan comiençan a fazer cueuas coneieras . } Et quando esto entendieren \\\hline
3.3.22 & et partem obsessi ) \textbf{ debet esse bellum continuum , } ne obsidentes per viam illam munitionem ingrediantur . & e otra fezieron los cercados se \textbf{ deue acometer la batalla continuadamente | por que los -\-> } que cercan non puedan entrar \\\hline
3.3.22 & Debent etiam obsessi \textbf{ iuxta inchoationem viae subterraneae habere magnas tinnas plenas aquis vel etiam urinis : } et cum bellant contra obsidentes , & Avn deuen los cercados \textbf{ çerca el comienço de las carreras soterrañas | auer tiñas lleñas de agua o de oriñas . } En quando lidian contra los que los çercan deuen fingir que fuyen \\\hline
3.3.22 & et prius quam exercitus possit \textbf{ succurrere ad defendendum eam , } succendunt ipsam . & Et Ante que pueda la hueste acorrer a \textbf{ defenderle quemenle con fuego . } Mas si non osan salir \\\hline
3.3.23 & in aliis generibus bellorum , \textbf{ applicari poterunt ad naualem pugnam . } Circa hoc autem pugnandi genus , & en las otras maneras de las batallas se podran traer \textbf{ e ayuntar a esta lid de las naues . } Mas çerca esta manera de lidiar primeramente es de veer \\\hline
3.3.23 & in arboribus abundare , \textbf{ non est bonum incidere arbores , } ex quibus fabricanda est nauis . & en que el humor comiença de abondar e de cresçer en los arboles \textbf{ non es bueno de taiar los arboles } de los quales deue ser fecha la naue . \\\hline
3.3.23 & quem Incendiarium vocant . \textbf{ Expedit enim eis habere multa vasa plena pice , sulphure , rasina , oleo ; } quae omnia sunt cum stupa conuoluenda . & ca conuine de auer en las naues \textbf{ mucha usija de tierra | assi commo cantaros e o las } e otros belhezos tales \\\hline
3.3.23 & committendum durum bellum , \textbf{ ne possint currere ad extinguendum ignem . } Secundo ad committendum marinum bellum multum valent insidiae . & Et entonçe deuen acometer muy fuerte batalla contra los enemigos \textbf{ por que se non puedan acorrer | para matar el fuego . } Lo segundo para acometer batalla en la mar valen mucho las çeladas . \\\hline
3.3.23 & Nam velis eorum perforatis , \textbf{ et non valentibus retinere ventum ; } non tantum possunt ipsi hostes impetum habere pugnandi , & ca foradadas las velas \textbf{ e los treos non pueden retener el viento . } Et assi non pueden los enemigos auer tanta fuerça \\\hline
3.3.23 & Sexto consueuerunt nautae \textbf{ habere ferrum quoddam curuatum } ad modum falcis bene incidens , & si quisiere foyr de la batalla . \textbf{ Et lo sexto suelen los marineros auer vn fierro coruo bien agudo } e bien taiante a manera de foz . \\\hline
3.3.23 & Septimo consueuerunt \textbf{ e iam nautae habere uncos ferreos fortes , } ut cum vident se esse plures hostibus , & lo . vij° . \textbf{ suelen avn los marineros | auer coruos de fierro muy fuertes } e quando veen que son mas \\\hline
3.3.23 & Nona cautela est \textbf{ habere multa vasa plena } ex molli sapone , & o seran anegados en la mar . \textbf{ La . ix . cautela es auer muchos cantaros llenos } dexabon muelle \\\hline

\end{tabular}
