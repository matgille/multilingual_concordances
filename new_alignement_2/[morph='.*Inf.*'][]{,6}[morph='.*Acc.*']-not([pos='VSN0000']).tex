\begin{tabular}{|p{1cm}|p{6.5cm}|p{6.5cm}|}

\hline
1.1.1 & quod modum procedendi in hac scientia oportet \textbf{ esse figuralem et grossum . } Prima via sumitur ex parte materiae , & e en esta sçiençia conujene \textbf{ que sea figural e gruesa , } ¶ la primera rrazon se toma de parte dela materia \\\hline
1.1.1 & suis subditis imperare , \textbf{ oportet doctrinam hanc extendere usque ad populum , } ut sciat qualiter debeat & E si por qual manera Deuen mandar a los sus Subditos \textbf{ conujene esta doctrina | e esta sçiençia estender la fasta el pueblo } por que Sepa commo ha de obedesçer a sus prinçipes \\\hline
1.1.1 & oportet modum procedendi in hoc opere , \textbf{ esse grossum et figuralem . } Cum omnis doctrina & Conuie ne \textbf{ que la manera que deuemos tener enesta obra sea gruesa e figural e exenplar } a asi commo dize el philosopho \\\hline
1.1.2 & bene se habet \textbf{ narrare ordinem dicendorum , } ut de ipsis quaedam praecognitio habeatur . & asi commo el conosçimiento del entendimjento nasçe del conosçimjento Delos sesos \textbf{ Por ende buena cosa es de Recontar la orden de las cosas } que se han de dezir \\\hline
1.1.2 & ordine naturali decet regiam maiestatem \textbf{ primo scire se ipsum regere , } secundo scire suam familiam gubernare , & primeramente que el Ruy sepa gouernar asy mesmo ¶ \textbf{ Lo segundo que sepa gouernar su conpanna¶ | Lo terçero que sepa } gouernả su rregno \\\hline
1.1.2 & primo scire se ipsum regere , \textbf{ secundo scire suam familiam gubernare , } tertio scire regere regnum , et ciuitatem . In primo autem libro in quo agetur de regimine sui , & Lo terçero que sepa \textbf{ gouernả su rregno | e sus çibdades ¶ } pues que asy es en el primo libro \\\hline
1.1.2 & secundo scire suam familiam gubernare , \textbf{ tertio scire regere regnum , et ciuitatem . In primo autem libro in quo agetur de regimine sui , } sunt quatuor declaranda . & e sus çibdades ¶ \textbf{ pues que asy es en el primo libro | en el qual tractaremos del gouerna mjeto del omne . } En sy mesmo son quatro cosas de declarar e de demostrͣ \\\hline
1.1.2 & proficuum esse morali negocio , \textbf{ scrutari ea quae sunt circa operationes , } quomodo faciendum sit eas . & que prouechosa cosa es en la scian moral \textbf{ e en la sçiençia de costunbres escod̀nar aquellas cosas | que son cerca delas obras } commo las deue omne fazer \\\hline
1.1.2 & Videntur autem haec quatuor \textbf{ habere aliquam analogiam adinuecem . } Nam ex aliis , et aliis motibus , & pues que asy es paresçe \textbf{ que estas quatro cosas dichas han alguna conparaçion entre sy } por que departidas costunbres han de nasçer departidos pasiones \\\hline
1.1.3 & ostendendo , quae dicenda sunt , \textbf{ nos esse faciliter tractaturos : } et in secundo reddidimus eam docilem , & que son de dezer en este libro \textbf{ que nos prometiemos de tractar ligniamente ¶ } et en el segundo capitulo fiziemos \\\hline
1.1.3 & qui est rationalis per essentiam : \textbf{ sicut ergo rex non dicitur habere regnum , } nec dux dicitur habere ciuitatem , & que es rrazonable \textbf{ por sy mesmo peren asy es asy commo el rrey non puede auer el Regno } njn el caudiello non puede auer la çibdat \\\hline
1.1.3 & sicut ergo rex non dicitur habere regnum , \textbf{ nec dux dicitur habere ciuitatem , } si in regno vel ciuitate sunt aliqui , & por sy mesmo peren asy es asy commo el rrey non puede auer el Regno \textbf{ njn el caudiello non puede auer la çibdat | sy en el rregno } o en la çibdat ouiere discordia \\\hline
1.1.3 & et maxime regiam maiestatem \textbf{ implorare diuinam gratiam . } Nam quanto maiestas regia in loco altiori consistit , & e mayormente prinçipe o Rey \textbf{ que demande mucho afincadomente la gera de dios } Ca quanto la magestad rreal esta en logar \\\hline
1.1.3 & ut possit virtutum opera exercere , \textbf{ et sibi subditos valere inducere ad virtutem . } Quot sunt modi viuendi , & por que pueda usar de obras de uirtudes \textbf{ e por que pueda enduzir } e traher los sus subienctos a uirtudes . \\\hline
1.1.4 & videlicet , voluptuosam , politicam , et contemplatiuam . \textbf{ Videbant enim hominem esse medium inter superiora , et inferiora : } est autem homo naturaliter medius & Ca veyen los philosofos \textbf{ que el omne es medianero entre las cosas | que son desuso } que son çelestiałs e las cosas \\\hline
1.1.4 & nam in vita voluptuosa \textbf{ negauerunt esse felicitatem , } quod et Theologi negant : & Ca en la vida delectos a \textbf{ negaron algunos philosofos la feliçidat | e la bien andança } diziendo que non es en elła . \\\hline
1.1.4 & ut homo est : \textbf{ sed speculari et cognoscere veritatem , } competit ei & en quanto es omne . \textbf{ Mas estudiar e conosçer la uerdat conviene al omne en quanto es en el entedimjento especłatino } e escodrinador que es alg̃cos e diujnal . \\\hline
1.1.4 & et per omnem modum potuerunt \textbf{ attingere veritatem . } Nam licet vere dixerunt & e segunt manera acabada . \textbf{ Ca mager que dissiese nudat } que en la vida seliçonsa non es de poner bien andança \\\hline
1.1.4 & Quod autem vitam contemplatiuam dixerint \textbf{ esse potiorem , } quam vitam politicam et actiuam , & Mas en lo que ellos dixieron \textbf{ que la vida contenplatian es mejor que la vida politica e actiua } que esta en las obras en esto non descordaron de los cheologos njn dela uerdat . \\\hline
1.1.4 & tanto magis decet \textbf{ habere reges et principes , } quanto apud tribunal summi Iudicis & tantomas conviene dela auer los rreys \textbf{ e los prinçipes } por quanto han de dar mayor cuenta \\\hline
1.1.5 & quod sicut materia \textbf{ per debitas transmutationes } consequitur suam perfectionem et formam , & que asy commo la materia \textbf{ por sus conueientes e ordenadas } t̃ns muta connes viene a rresçebir su forma \\\hline
1.1.5 & sic homo per rectas \textbf{ et debitas operationes } consequitur suam perfectionem et felicitatem . & asi el omne por derechas \textbf{ e conueni entes obras } viene a auer su perfecçion \\\hline
1.1.5 & expedit volenti \textbf{ consequi suum finem , } vel suam felicitatem , & Conviene a todo omne \textbf{ que quiera alcançar e auer su fin } e la su bien andança de auer \\\hline
1.1.5 & vel suam felicitatem , \textbf{ habere praecognitionem ipsius finis . } Possumus autem dicere & e la su bien andança de auer \textbf{ ante algun conosçimjento dela su fin e dela su bien andança . } Mas podemosdezer quanto pertenesçe alo presente \\\hline
1.1.5 & duplici via venari possumus , \textbf{ quod expedit regi suum finem cognoscere . } Prima est , & e cobrar prouar \textbf{ que conujene al rrey | en toda manera de conosçer la su fin ¶ } La primera rrazon es en quanto el rrey \\\hline
1.1.5 & ut per opera nostra mereamur \textbf{ consequi finem , vel felicitatem . } Secundo requiritur & pues que asi es conviene bien fazer de fecho \textbf{ por que por las nr̃as obras merescamos de auer buena fino buena ventura } segunt deuemos los omes fas̉ bien \\\hline
1.1.5 & Unde Philosophus 2 Ethic’ vult , \textbf{ quod non sufficit agere bona , } sed bene : nec sufficit operari iusta , & en el segundo libro delas ethicas \textbf{ que non cunple solamente fazer buenas obras } mas fazerlas bien njn cunple de obrar obras iustas \\\hline
1.1.5 & eos \textbf{ consequi finem vel felicitatem . } Cum ergo ista tria contingunt , & delectosamente non les conuiene \textbf{ que por aquellas obras alcançen buena fin | nin bue an uentura ¶ } Et quando estas tres cosas todas uienen en vno \\\hline
1.1.5 & volens ostendere \textbf{ necessariam esse praecognitionem finis , ait , } quod cognitio finis & a¶ Onde el philosofo quariendo mostrar en el primero libro delas ethicas \textbf{ que es neçesario de connosçer ante la fin | dize } que para lanr̃auida grant acresçentamiento \\\hline
1.1.5 & quia in operibus suis debet \textbf{ intendere bonum gentis et commune , } quod est magis expediens et diuinius , & por que entondas sus obras \textbf{ deue entender al bien dela gente | e al bien comun } que es mas conuenible \\\hline
1.1.5 & quod maxime decet regiam maiestatem \textbf{ cognoscere suam felicitatem , } ut opera communia , & que muy mas conuiene al Reio al prinçipe conosçer la su fin \textbf{ e la su bien andança | que a otro ninguno . } por que pueda fazer buenas obras e comunes \\\hline
1.1.5 & eo quod sit sagittae director : sic magis expedit regiam maiestatem felicitatem , \textbf{ et finem cognoscere quam populum , } eo quod fit populi directiua . & que mas conuiene al Rei o al prinçipe de conosçer la su bien andança e la su \textbf{ finque non al pueblo . } Por que el es ginador del pueblo \\\hline
1.1.6 & In huiusmodi autem voluptatibus sensibilibus \textbf{ non esse felicitatem ponendam , } triplici via venari possumus . & que en estas delecta çonnes sensibles de los sesos . \textbf{ non es de poner la feliçidat e la bien andança . } Ca quanto parte nesçe alo prèsente la bien andança ençierra en si tres cosas \\\hline
1.1.6 & Felicitas enim dicit perfectum , \textbf{ et per se sufficiens bonum . Nam tunc dicimus aliquem esse felicem , } quando assecutus est id , & La primera es que la bien andança es bien acabado \textbf{ e tal bien que por si mesmo faze el omne acabado . } Ca estonçe dezimos \\\hline
1.1.6 & si felicitas ponitur \textbf{ esse perfectum bonum , oportet quod sit bonum } secundum intellectum , et rationem : & Pues si la bien auentraança es bien acabado e conplido . \textbf{ Conuiene que sea bien segunt el en tedimiento } e segunt Razon \\\hline
1.1.6 & constat in talibus \textbf{ non esse felicitatem ponendam . } Quod autem huiusmodi voluptates , & que en las tales delectaçiones \textbf{ non auemos nos de poner lanr̃a feliçidat | nin lanr̃a bien andança } Mas que estas plazenterias \\\hline
1.1.6 & nulla tamen delectatio est essentialiter ipsa felicitas , \textbf{ licet possit esse aliquid felicitatem consequens , } sed hoc declarare non est praesentis negocii . & por si mesma \textbf{ maguera que se pueda conseguir | ala feliçidat e ala bien andança . } Mas declarar esto non parte nesçe a esta arte presente . \\\hline
1.1.6 & licet possit esse aliquid felicitatem consequens , \textbf{ sed hoc declarare non est praesentis negocii . } Forte tamen de hoc aliquid infra dicetur . & ala feliçidat e ala bien andança . \textbf{ Mas declarar esto non parte nesçe a esta arte presente . } Enpero que por auentra a adelante diremos alguna cosa desto . \\\hline
1.1.6 & Est ergo detestabile cuilibet Homini \textbf{ ponere suam felicitatem in voluptatibus . } Sed maxime hoc est detestabile Regiae maiestati : & ̉ qual quier omne \textbf{ que toda su bien andança pone en delecta connes dela carne } mas mucho mas es de denostar el Rey \\\hline
1.1.6 & quod sicut non differt \textbf{ esse Puerum aetate , et moribus : } sic non refert & pues que asi es muy acuçiosamente deuemos notar \textbf{ que asy commo non ay departimento entre moço en hedat e moço en constunbres . } asy non ay deꝑtimiento \\\hline
1.1.6 & sic non refert \textbf{ esse Senem moribus et aetate , } propter quod sicut si sit Senex tempore , & asy non ay deꝑtimiento \textbf{ entre vieio en costunbres e uieio en hedat . } por la qual cosa asy commo sy \\\hline
1.1.6 & quare si constat \textbf{ eos habere mores seniles , } et vigere Prudentia , & por la qual cosa si fuer çierto \textbf{ qualos mançebos han costunbres de me nos } e han sabiduria e entendimiento \\\hline
1.1.7 & in artificialibus diuitiis \textbf{ felicitatem non esse ponendam . } Primo , quia artificiales diuitiae & por las quales nos pondemos prouar \textbf{ que la feliçidat e la bien andança non es de poner en les riquezas artifiçiales ¶ } La primera razon es por que las riquezas artifiçiales son orderandas alas riquezas natraales ¶la segunda \\\hline
1.1.7 & Dicebatur enim supra , \textbf{ felicitatem esse illud bonum , } ad quod alia bona ordinantur , & Ca dichones de suso \textbf{ que la feliçidat | e la bien andança es de poner en aquel bien } aque todos los otros bienes son ordenados \\\hline
1.1.7 & Cuilibet ergo Homini detestabile est \textbf{ ponere suam felicitatem in diuitiis , } sed maxime detestabile est regiae maiestati . & ¶pres que assi es mucho es de denostar todo en que pone su feliçidat \textbf{ e su bien andança en las riquezas corporales . } Mas mayor mente es de denostar la Real magestad \\\hline
1.1.7 & principaliter intendit reseruare sibi , \textbf{ et congregare pecuniam . } Non ergo est Rex , & e en los aueres \textbf{ prinçipalmente entiende de thesaurizar e fazer thesoro e llegar muchos dineros } Et por ende se sigue \\\hline
1.1.7 & quo potest , \textbf{ consequi finem suum . } Ponens igitur suam felicitatem in diuitiis , & que pudiere \textbf{ por que pueda alcançar aquella fin | e aquel bien ¶Donde se sigue } que el prinçipe \\\hline
1.1.7 & quid importatur nomine finis , \textbf{ non potest eum latere quemlibet , } omni via qua potest , & que quiere dezir e quanto lieua este nonbre fin \textbf{ e bien andança non se le puede esconder } por ninguna manera \\\hline
1.1.7 & omni via qua potest , \textbf{ velle consequi suum finem . } Est igitur Rex Tyrannus , & por ninguna manera \textbf{ que non pueda querer seguir la su fin | ante se trabaia dela alcançar quanto puede ¶ } Pues que assi es el Rei es tirano \\\hline
1.1.7 & Quare si detestabile est , \textbf{ Regem admittere maxima bona , } esse Tyrannum , & mas faze les mal . \textbf{ Por la qual cosa si es muy contra razon que el Rey dexe muy grandes bienes . Et si es contra razon otrosi } que sea robador del pueblo \\\hline
1.1.7 & Regem admittere maxima bona , \textbf{ esse Tyrannum , } et depraedatorem detestabile & Por la qual cosa si es muy contra razon que el Rey dexe muy grandes bienes . Et si es contra razon otrosi \textbf{ que sea robador del pueblo } e sea tyrano ¶ Bien asi es cosa contra razon \\\hline
1.1.8 & qui vult honorem \textbf{ esse exhibitionem reuerentiae } in testimonium virtutis . & por el philosofo en el primero libro delas ethicas \textbf{ do dize que honrra es reuerençia } que fazen unos omes a otros en testimoino de uirtud \\\hline
1.1.8 & quam in eo qui per huiusmodi reuerentiam honoratur . \textbf{ Apparet autem hoc esse sensibiliter verum : } nam si aliquis inclinat se reuerenter ad alium , & que non en aquel que la resçibe¶ \textbf{ Et esso mesmo paresçe manifiestamente al seso . } Por que sy alguno inclinando se faze reuerençia a otro o le honrra . \\\hline
1.1.8 & Indecens est ergo cuilibet homini \textbf{ ponere suam felicitatem in honoribus , } ut credat se esse felicem , & es que ningun omne . \textbf{ ponga su bien andança en las honrras } assi que crea que es bien andante \\\hline
1.1.8 & ponere suam felicitatem in honoribus , \textbf{ ut credat se esse felicem , } si ab hominibus honoratur . & ponga su bien andança en las honrras \textbf{ assi que crea que es bien andante } si los omes le honrran \\\hline
1.1.8 & curantes de honore tantum , \textbf{ dicit esse fictos , et superficiales . } Si ergo maxime decet & que los que fazen fuerça tan solamente de ser honrrados \textbf{ que estos son infintos e superfiçiales . } Et pues que assi es si mucho conuiene al Rey de ser bueno uerdaderamente \\\hline
1.1.8 & maxime indecens est \textbf{ ipsum ponere felicitatem in honoribus , } ne sit fictus , et superficialis . & muy mas desconueinble cosaes ael \textbf{ que otro ninguon de poner su bienandança en las honrras } por que non paresca infinto e superfiçial ¶ \\\hline
1.1.8 & Secundo indecens est Regi , \textbf{ ponere suam felicitatem in honoribus , } quia ex hoc efficietur periclitator Populi , et praesumptuosus : & assi que muy desconueible cosa es al Rey \textbf{ poner su bien andança en las honrras | Ca por esso seria prisuptuoso } e sob̃uio \\\hline
1.1.8 & quia finem summo ardore diligit , \textbf{ non curabit remunerare personas } secundum propriam dignitatem , & por que la fin e la bien andança es muy amada \textbf{ e muy desseada non fara fuerça de dar } gualardon alas personas segunt sus dignidades \\\hline
1.1.8 & Quod non decet regiam maiestatem , \textbf{ suam ponere felicitatem in gloria , } vel in & ø \\\hline
1.1.9 & et pro eodem , \textbf{ posset forte alicui videri felicitatem ponendam } esse in fama et gloria , & alguon que la feliçidat \textbf{ e la bien andança es de poner } e en fama e en eglesia \\\hline
1.1.9 & Quare cum Regem deceat \textbf{ esse totum diuinum , } et quasi semideum , & por la qual cosa commo al Rey conuenga ser todo diuinal e semeiante a dios \textbf{ si non es cosa conuenible | de poner la feliçidat } e la bien andança \\\hline
1.1.9 & inconueniens enim est , \textbf{ quod Rex se credat esse felicem , } si sit famosus apud Homines , & Et non es cosa conuenible \textbf{ nin cosa con razon | que el Rey crea } que es bien auentra ado si es famoso entre los omes \\\hline
1.1.9 & vel si sit in populis gloriosus . \textbf{ Non igitur debet Rex se credere esse beatum , } si sit in gloria apud homines : & o si es głioso en los pueblos ¶ \textbf{ Et pues que assi es el rey | non deue creer } que es bien auenturado \\\hline
1.1.9 & mercedem tribuendam esse Regibus , \textbf{ et hunc esse honorem et gloriam . } Non est intelligendus textus Philosophi , & gualardon que los Reis deuian \textbf{ auer era en eglesia e en honrra segunt el philosofo dezie . } El testo del philosofo non se deue \\\hline
1.1.9 & Non est intelligendus textus Philosophi , \textbf{ quod Reges principaliter pro suo merito quaerere debeant gloriam , } et famam Hominum , & assi entender \textbf{ que los Reys prinçipalmente por su meresçimiento deuen demandar } e quere reglesia e fama de los omes . \\\hline
1.1.9 & honor tamen eos consequitur , \textbf{ et decet eos acceptare honorem sibi exhibitum , } non habentibus Hominibus aliquid maius , & enpero la honrra les parte nesçe a ellos . \textbf{ Et conuiene les alos Reys de resçebir la honrra | que les fazenn los omes } por que los omes non les pueden dar mayor cosa que honrra \\\hline
1.1.10 & Vegetius in libro De re militari , \textbf{ super omnia commendare videtur bellorum industriam . } Hoc enim ( secundum ipsum ) est & que fizo dela caualleria \textbf{ que sobre todas las cosas es de alabar la maestria | e la sabiduria delas batallas . } Et esta es vna cosa segunt que el dize \\\hline
1.1.10 & et summo opere studuerunt , \textbf{ quomodo possent sibi subiicere nationes . } Propter quod & Et sobre todas las cosas estudiaron \textbf{ commo pudiessen subiugar todas las naçiones } e todas las gentes . \\\hline
1.1.10 & in 7 Pol’ quinque rationibus felicitatem \textbf{ non esse ponendam } in ciuili potentia . & Por que el philosofo praeua en el septimo libro delas politicas \textbf{ por çinco razones | que la feliçidat } e la bien andança de los Reyes \\\hline
1.1.10 & Nam per ciuilem potentiam \textbf{ velle sibi subiicere nationes , } hoc est , velle dominari per violentiam . & La primera razon se puede assi declarar \textbf{ Ca querer subiugar las naconnes | e las gentes } por poderio çiuil esto esquerer \\\hline
1.1.10 & velle sibi subiicere nationes , \textbf{ hoc est , velle dominari per violentiam . } Violentia autem perpetuitatem nescit . & e las gentes \textbf{ por poderio çiuil esto esquerer | enssen onrear por fuerça } e non pornatraa \\\hline
1.1.10 & Non ergo Rex debet \textbf{ se credere esse felicem , } si per violentiam , & nin el prinçipe \textbf{ que es bien auentraado | si enseñorea sobre el pueblo } por fuerca e por poderio çiuil . \\\hline
1.1.10 & Impossibile est autem \textbf{ in aliquo esse maximum bonum , } nisi ille bene viuat , & Mas esto non puede ser \textbf{ que en alguno sea muy grand bien } si el non biuiere bien \\\hline
1.1.10 & sicut impossibile est in aliquo \textbf{ esse intensam albedinem , } nisi ille fit intense albus . & assi commo non puede seer \textbf{ que en alguna cosa sea grand blancura } si aquella cosa non fuere muy blanca . \\\hline
1.1.10 & ut non videretur in eis \textbf{ esse aliquid molle , } nec clementia aliqua . & Et fueron de tan grant crueldat \textbf{ que non paresçia en ellos ningunan cosablanda } nin de piedat \\\hline
1.1.10 & quia si Princeps se crederet \textbf{ esse felicem , } si abundet in ciuili potentia , & Por que si el prinçipe o el Rey crea \textbf{ que es bien auen traado } por que abonda en poderio çiuil non ordenara los çibdadanos \\\hline
1.1.10 & et ad ea , \textbf{ per quae sibi possit subiicere nationes . } Inducet ergo ciues & Et aquellas cosas \textbf{ por que pueda subiugar | assi las naçiones e los pueblos . } Et por ende non induzir a los çibdadanos \\\hline
1.1.10 & Quare si inconueniens est \textbf{ ponere felicitatem } in aliquo non diuturno , & por las quales cosas ya dichas \textbf{ si non es cosa conuenible de poner la bien andança en alguna cosa } que non sea duradera \\\hline
1.1.10 & inconueniens est etiam Principem \textbf{ ponere suam felicitatem } in ciuili potentia , & que non es cosa conuenible \textbf{ que el prinçipe ponga su bien andança en poderio çiuil . } Et por que cuyde que es bien \\\hline
1.1.10 & in ciuili potentia , \textbf{ et quod credat se esse felicem , } si possit sibi subiicere nationes multas . & que el prinçipe ponga su bien andança en poderio çiuil . \textbf{ Et por que cuyde que es bien } auentraado \\\hline
1.1.10 & et quod credat se esse felicem , \textbf{ si possit sibi subiicere nationes multas . } Quod non deceat Regiam maiestatem & Et por que cuyde que es bien \textbf{ auentraado | quando pudiere subiugar } assi las naçiones . \\\hline
1.1.11 & quam talia bona praegustentur , \textbf{ creduntur esse maiora , } quam sint : & Ca primeramente que omne goste e siente \textbf{ que le son estos bienes corporales paresçen le mayores de quanto son } Mas despues que los ha auido paresçen meno res de quanto el cuydaua . \\\hline
1.1.11 & nec in pulchritudine corporis , sed animae . \textbf{ Non igitur quis credat se esse felicem , } si habeat aequatos humores , & nin en fortaleza del cuerpo mas del alma . \textbf{ Et pues que assi es non crea ninguno | que es bien auentra ado } si ouiere los humores egualados \\\hline
1.1.11 & Non igitur quis credat se esse felicem , \textbf{ si habeat aequatos humores , } et sit sanus corpore : & que es bien auentra ado \textbf{ si ouiere los humores egualados } e fuere sano del cuerpo \\\hline
1.1.11 & tunc ( ut exigit suus status ) \textbf{ credat se esse felicem . } Dicimus autem & Estonçe crea el \textbf{ que es bien auentraado segunt su estado } Et dezimos segunt \\\hline
1.1.11 & Debet enim Princeps \textbf{ possidere sufficientes diuitias , } ut possit regnum defendere , & e fazen grand discordia en el pueblo ¶ \textbf{ Otrossi deuen los prinçipes auer riquezas sufiçientes } por que puedan defender los regnos \\\hline
1.1.11 & ut possit regnum defendere , \textbf{ et exercere operationes virtutum : } decet enim Regem esse magnificum , & por que puedan defender los regnos \textbf{ e fazer obras de uertudes . } E conuiene al Rey de seer magnifico e largo \\\hline
1.1.11 & est Rex dignus honore , \textbf{ et expedit ei habere ciuilem potentiam : } nam propter paruipensionem Principis , & por que non sea menospreçiada la Real magestad . \textbf{ Et por ende le conuiene de auer poderio çeuil . | Ca por el } menospreçiamientodel prinçipe muchͣs vezes contesçe que alguons fazen e obran malas cosas \\\hline
1.1.11 & sed quia possunt \textbf{ esse organa ad felicitatem . } Talia ergo diligenda sunt , & mas por que son instru mentos \textbf{ para ganar la feliçidat e la bien andança . } Et pues que assi es estas cosas tales son de amar \\\hline
1.1.12 & Voluit autem felicitatem \textbf{ non esse ponendam in viribus , } siue in potentiis animae , & que la feliçidat \textbf{ e la bien andança non se deue poner en las fuerças corporales } nin en las potençias del alma senssetuias \\\hline
1.1.12 & sed etiam mali participant . \textbf{ Nec etiam voluit esse ponendam eam in habitibus , } quia habens habitum , & en las disposiconnes \textbf{ nin en las scinas | que son en el alma } que el que ha las scians \\\hline
1.1.12 & quomodo deceat regiam maiestatem \textbf{ ponere suam felicitatem } in actu prudentiae , & en qual manera conuenga ala Real magestad \textbf{ de poner la primera feliçidat } en las obras de pradençia . \\\hline
1.1.12 & et rationem participat , \textbf{ ponere suam felicitatem } in bono maxime uniuersali , & e ha razon e entendimiento \textbf{ de poner la su bien andança en bien muy comun } e muy entelligible \\\hline
1.1.12 & nam regens multitudinem debet \textbf{ intendere commune bonum . } In eo ergo debet & Ca el que gouienna a muchos deue tener \textbf{ mientesal bien comun de todos . } Et por ende deue poner la su feliçidat \\\hline
1.1.12 & Si ergo Rex debet in Deo \textbf{ ponere suam felicitatem , } oportet ipsum huiusmodi felicitatem ponere & ¶ Et pues que el Rey deue poner la su feliçidat \textbf{ e la su bien andança en dios . } Conuiene le dela poner en la obra de aquella uirtud \\\hline
1.1.12 & unitiuam \textbf{ quandam dicimus esse virtutem . } In amore ergo diuino est ponenda felicitas . & commo aquel commo humanal o angelical o diuinal ha muy grant uirtud de ayuntar al que ama con lo que ama . \textbf{ Pues que assi es en el amor de dios } es de poner la feliçidat en la bien andança \\\hline
1.1.12 & si Princeps est felix diligendo Deum , \textbf{ debet credere se esse felicem operando } quae Deus vult . & Si el prinçipe es bien auenturado \textbf{ amando a dios deue creer | que es bien auenturado si obra } segunt que dios quiere e manda . \\\hline
1.1.13 & aliqualiter felicitas sit ponenda . \textbf{ Magnum autem esse praemium Regis , } et magnam eius esse felicitatem , & segunt dicho es \textbf{ or çinco razones podemos prouar | quant grant es el gualardon de los reyes } e quant grande es la su feliçidat \\\hline
1.1.13 & Magnum autem esse praemium Regis , \textbf{ et magnam eius esse felicitatem , } si per prudentiam , & quant grant es el gualardon de los reyes \textbf{ e quant grande es la su feliçidat | e las un bien andança } que han de auer \\\hline
1.1.13 & magna ergo debet esse virtus Regis , \textbf{ ad quem spectat regere non solum se , } et suam familiam , & pues que assi es grande deue ser la uirtud del Rey \textbf{ a quien parte nesçe de gouernar | non solamente assi mesmo } e asu conpanna \\\hline
1.2.1 & ( ut supra plenius probabatur ) \textbf{ debent uti tanquam organis ad felicitatem . } Suam autem felicitatem ponere debent & Mas assi commo prouamos conplidamente de suso deuen husar de todas estas cosas \textbf{ assi commo de instrumentos | para ganar la feliçidat e la bien andança . } Mas la su bien andança deuen poner en obras de pradençia e de sabiduria \\\hline
1.2.2 & Dictum est enim , \textbf{ virtutem esse aliquid secundum rationem : } oportet ergo esse rationalem potentiam , & Ca dicho es \textbf{ que la uirtud es alguna cosa | segunt razon } e por ende conuiene \\\hline
1.2.2 & virtutem esse aliquid secundum rationem : \textbf{ oportet ergo esse rationalem potentiam , } in qua potest esse virtus . & segunt razon \textbf{ e por ende conuiene | que sea poderio razonable } aquel en que esta la uirtud ¶ \\\hline
1.2.2 & vel ipsa ratio essentialiter , \textbf{ dicitur tamen participare rationem , } quia est aptus natus rationi obedire . & nin sea la razon por eennçia \textbf{ Enpero partiçipa con la razon } por que es apareiado e inclinado de obedesçer al entendimiento \\\hline
1.2.2 & videlicet , sensitiuus , \textbf{ qui sequitur formam apprehensam per sensum : } et intellectiuus , & assi en nos ay dos apetitos vno \textbf{ senssitiuo que sigue la forma tomada e conosçida por el seso . } Et el otro intellectiuo \\\hline
1.2.2 & nam arduitas , et difficultas potissime sunt repugnantia , et prohibentia , \textbf{ ne possimus consequi bonum , } et vitare malum . & e nos retienen \textbf{ por que non podamos seguir el bien } e esquiuar el mal . \\\hline
1.2.2 & secundum modum sibi conuenientem , \textbf{ prout bene possint delectari per concupiscibilem , } data est eis irascibilis , & Pues que assi es en quanto las . \textbf{ aian las | segund su manera conuenible se pueden delectar conplidamente } por el appetito \\\hline
1.2.3 & nisi circa ea quae sunt in potestate nostra , \textbf{ in quibus decet nos ponere medietatem , } vel aequalitatem , siue rectitudinem : & que son en nuestro poder . \textbf{ En las quales cosas nos conuiene } de poner meatado ygualdat o derechura . \\\hline
1.2.3 & per virtutes enim debemus \textbf{ habere rationes rectas , } passiones moderatas , & e son obras que fazemos de fuera Et en estas nos conuiene \textbf{ e poner meatad e egualdat . | Ca por las uirtudes deuemos auer las razones derechas } e las pasiones ordenadas e tenprados . \\\hline
1.2.3 & Dictum est enim virtutes \textbf{ illas esse circa passiones : } quas per se habent moderare , & e la sufiçiençia se puede tomar en esta manera . Ca dicho es ya que aquellas uirtudes han de seer çerca delas passiones \textbf{ las quales passiones han de mesurar e de ygualar por si ¶ } Pues que assi es en \\\hline
1.2.3 & non sic possunt \textbf{ habere rationem ardui sicut utilia , et honesta , } licet tam ex bonis utilibus & Et daqui paresçe que por que los bienes delectables non pueden auer \textbf{ assi manera de guaueza | commo han los bienes prouechosos } e los honestos maguera \\\hline
1.2.3 & videlicet , Veritas , Affabilitas , et Eutrapelia , \textbf{ quae potest dici bona versio . } Est autem Veritas & La otra es eutropolia \textbf{ que quiere dezir buena conuerssaçion o buena manera de beuir . | Mas la uerdat assi conma aqui fablamos de uerdat } non en quanto es uirtud \\\hline
1.2.4 & Ne ergo aliquis crederet \textbf{ non esse aliquas } alias bonas dispositiones & Et pues que assi es \textbf{ por que non cuydasse alguno que non auia otras buenas disposiconnes } si non estas uirtudes que ya dixiemos . \\\hline
1.2.5 & ratiocinari recte et non recte , \textbf{ oportet dare virtutem aliquam , } quae sit recta ratio , & Ca commo contesca de razonar derechamente \textbf{ e non derechamente conuiene de dar alguna uirtud } que sea razon derecha . \\\hline
1.2.5 & Rursus cum contingat operari recte et non recte , \textbf{ sic ut est dare virtutem , } per quam dirigimur & e non derechamente \textbf{ assi commo auemos a dar uirtud . } Por la qual cosa somos endereçados en razonando delans obras en essa mis ma guas a auemos de dar uirtud \\\hline
1.2.5 & Amplius quia contingit nos passionari recte et non recte , \textbf{ oportet dare virtutes aliquas , } per quas modificentur in ipsis passionibus . & e non derecha mente . \textbf{ Conuiene nos de dar uirtudes algunas } por las quales seamos tenprados e reglados en aquellas passiones ¶ \\\hline
1.2.5 & circa passiones oportet \textbf{ dare virtutem aliquam , } ne passiones nos impellant & assi commo son las passiones dela saña . \textbf{ Conuiene dar alguna uirtud en las passiones } por la qual las passiones non nos pueden mouer \\\hline
1.2.5 & ad id quod ratio vetat : \textbf{ et oportet dare virtutem aliam , } ne passiones retrahant nos ab eo , & nin inclinar a aquelo que uieda la razon ¶ \textbf{ Et otrosi nos conuiene de dar otra uirtud } por la qual las passiones non nos pueden arredrar \\\hline
1.2.6 & quomodo possumus \textbf{ consequi talem finem , } quod fit per prudentiam . & si non sopiere \textbf{ en qual manera el puede alcançar tal fin . } e esto ha de saber \\\hline
1.2.6 & Per virtutes ergo morales \textbf{ praestituimus nobis debitos fines : } sed per prudentiam & por la pradençia ¶ Et pues que assi es por las uirtudes morales somos ordenados \textbf{ a nuestros fines buenos e conuenibles . } Mas por la pradençia somos reglados \\\hline
1.2.6 & secundum inuenta et iudicata , \textbf{ et hanc dicimus esse prudentiam . } Prudentia ergo respectu virtutis inuentiuae et iudicatiuae & por la qual mandamos que se fagan las obras todas segunt las cosas falladas e iudgadas \textbf{ e esta dize el philosofo | que es pradençia . } Et pues que assi es la pradençia \\\hline
1.2.6 & Cum ergo in moralibus actus \textbf{ et opera dicantur esse potiora , } Prudentia , quae immediatius se habet ad ea , & Et pues que assi es commo en las uirtudes morales las obras \textbf{ e los fechos sean dichos mayores e meiors . } la pradençia que mas derechanmente cata alas obras es mayor e meior \\\hline
1.2.6 & sic prudentiae est praecipere . \textbf{ Tertio , Prudentia comparari potest ad materiam , } circa quam versatur . & assi la pradençia es uirtud para mandar . \textbf{ Lo terçero la pradençia se pue de conparar } ala materia en que obra . \\\hline
1.2.6 & quae sunt in potestate nostra . \textbf{ Quinto comparari potest prudentia ad artem , } a qua etiam distingui habet . & o non las fazer . \textbf{ Et aquella sabiduria es dicha pradençia | por la qual cosa seg̃t que la pradençia ha departimiento dela sçina } pue dese \\\hline
1.2.7 & nomen enim regum a regendo sumptum est : \textbf{ regere autem alios , } et dirigere ipsos in finem debitum , & Ca el nonbre del Rey es tomado de gouernamiento . \textbf{ Mas gouernar alos otros } e guiar los en su fin conuenible . \\\hline
1.2.7 & regere autem alios , \textbf{ et dirigere ipsos in finem debitum , } sit per prudentiam . & Mas gouernar alos otros \textbf{ e guiar los en su fin conuenible . } Esto ha de ser por la pradençia \\\hline
1.2.7 & Qui ergo hoc oculo caret , \textbf{ non sufficienter videre potest ipsum bonum , } nec ipsum debitum finem , & que catamos el bien e la fin conuenible . \textbf{ Et el que non ha este oio non puede conplidamente ueer el bien } nin la su fin conuenible \\\hline
1.2.7 & non solum nomine sed re , \textbf{ decet ipsum habere prudentiam . } Secundo hoc decet eum , & non solamente segunt el nonbre \textbf{ mas segunt el fech̃o | conuiene le de auer sabiduria . } La segunda manera por que conuiene al Rey de ser sabio \\\hline
1.2.7 & Est enim prudentis , \textbf{ prouidere bona sibi et aliis , } et dirigere se et alios in optimum finem . & ala qual nos inclinan las uirtudes morales . \textbf{ Ca de omne sabio es proueer buenas cosas . } assi e alos otros e de guiar \\\hline
1.2.7 & prouidere bona sibi et aliis , \textbf{ et dirigere se et alios in optimum finem . } Si ergo aliquis prudentia careat , & Ca de omne sabio es proueer buenas cosas . \textbf{ assi e alos otros e de guiar | assi e alos otros a buena fin ¶ } pues si alguno non ouiere sabiduria \\\hline
1.2.7 & Tertio decet Reges , \textbf{ et Principes habere prudentiam , } quia sine ea non possunt naturaliter dominari . & si non commo podra sacardes e algo del su pueblo . \textbf{ La terçera manera por que conuiene al Rey de auer sabiduria es } por que sin ella non puede ser señor \\\hline
1.2.7 & Hoc etiam modo iuuenes naturaliter decet \textbf{ antiquioribus esse subiectos , } quia inexperti agibilium & en esta misma gusa las moços \textbf{ e los mançebos | conuiene que naturalmente sean subiectos de los mas antigos } por que non son espiertos \\\hline
1.2.7 & quod polleat prudentia , et intellectu . \textbf{ Quot , et quae oporteat habere Regem , } si & ø \\\hline
1.2.8 & si debeat aliquis esse perfecte prudens , \textbf{ oportet ipsum habere omnia } quae concurrunt ad prudentiam , & si alguno ouiere aser sabio conplida mente . \textbf{ Conuienel e de auer todas aquellas cosas } que son necessarias ala sabiduria . \\\hline
1.2.8 & et omnes partes eius . \textbf{ Consueuerunt autem assignari octo partes prudentiae , } videlicet , memoria , prouidentia , intellectus , ratio , solertia , docilitas , experientia , et cautio . & Et connuiene le de auer todas las partidas de la sabiduria . \textbf{ Mas suele le sennalar e departir ocho partes dela praderçia e dela sabiduria¶ | La primera es memoria ¶ } La segunda prouidençia ¶ \\\hline
1.2.8 & Haec autem octo , \textbf{ quae dicuntur esse partes prudentiae , } sic accipi possunt . & Mas estas ocho cosas \textbf{ que son dichas part | s̃ dela sabiduria } assi se pueden tomar . \\\hline
1.2.8 & ex hoc aliquis dicitur esse prudens , \textbf{ quia est sufficiens dirigere se , } et alios in aliqua bona , & por esso es alguon dicho sabio \textbf{ porque es suficiente para enderesçar assi e alos otros e de guiar assi e alos otros a alguons bienes o a algunas buenas fines ¶ } Pues que assi es quatro cosas nos \\\hline
1.2.8 & et prouidentiam futurorum . \textbf{ Debet enim habere praeteritorum memoriam , } non quod possit praeterita immutare , & e que han de venir \textbf{ Ca deue el Rey auer memoria e remenbrança delas cosas passadas } non por que las pueda mudar . \\\hline
1.2.8 & quia nulli agenti hoc est possibile , \textbf{ sed decet Regem habere praeteritorum memoriam , } ut possit ex praeteritis cognoscere , & Ca esto ninguno non lo pie de fazer . \textbf{ Mas conuiene al Rey de auer memoria delans cosas passadas | por que pue da } por las cosas passadas conosçer e tomar \\\hline
1.2.8 & ut possit ex praeteritis cognoscere , \textbf{ quid euenire debeat in futurum . } Nam ( ut scribitur secundo Rhetoricorum ) & e ꝑcebimiento delas cosas \textbf{ que han de venir | ¶ } Ca assi commo dize el philosofo \\\hline
1.2.8 & ut plurimum futura sunt praeteritis similia . \textbf{ Secundo decet ipsum habere prouidentiam futurorum : } quia homines prouidentes futura bona , & que son passadas \textbf{ ¶lo segundo conuiene al Rey de auer | prouisionde las cosas } que han de venir . \\\hline
1.2.8 & ut ex actis praeteritis sciat \textbf{ quid agere debeat in futurum . } Ratione vero modi & por que delas cosas passadas \textbf{ sepa lo que ha de fazer en lo que ha de venir . } Mas por razon dela manera \\\hline
1.2.8 & oportet quod sit industris , et solers , \textbf{ ut sciat ex se inuenire bona gentis sibi commissae . } Verum quia nullus homo sufficit & por que sepa \textbf{ por si buscar e fallar aquellos bienes | que conuiene a su pueblo e asu gente ¶ } Mas porque ningun omne non puede conplidamente penssar aquellas cosas \\\hline
1.2.8 & quae possunt \textbf{ esse utilia toti regno , } cum hoc quod Regem expedit & Mas porque ningun omne non puede conplidamente penssar aquellas cosas \textbf{ que son aprouechables a todo el regno . } Enpero con esto que conuiene al Rey de ser sotil e agudo de si penssando los bienes \\\hline
1.2.8 & quod non decet \textbf{ ipsum fugere commouentem . } Non enim decet Regem & do dize que non conuiene al magnanimo \textbf{ menospreçiara | aquel que bien le conseia } por la qual cosa non le conuiene al Rey de seguir en todas cosas su cabeça \\\hline
1.2.8 & Non enim decet Regem \textbf{ in omnibus sequi caput suum , } nec inniti semper solertiae propriae : & aquel que bien le conseia \textbf{ por la qual cosa non le conuiene al Rey de seguir en todas cosas su cabeça } nin atener se sienpre al su engennio propio . \\\hline
1.2.8 & et eligendo bona simpliciter , \textbf{ ad quae debet dirigere gentem sibi commissam . } Quomodo Reges , & ¶Et otrosi para escoger las buenas \textbf{ que son dessi buenas } alas quales deue el Rey guiar \\\hline
1.2.9 & fiunt magis prudentes in agibilibus . \textbf{ Secundo debent diligenter intueri futura bona , } quae possunt esse proficua regno : & que han de fazer ¶ \textbf{ La segunda manera es esta | que deuen los reyes muy acuçiosamente catar las bueans cosas } e los bueons fechos \\\hline
1.2.9 & Tertio debent saepe \textbf{ recogitare bonas consuetudines , } et bonas leges : & La terçera manera es que los Reyes et los prinçipes deuen penssar muchas uezes \textbf{ e traer a su memoria | las buenas costun bres } e las buenas leyes . \\\hline
1.2.9 & et consuetudines debite regnum regat , \textbf{ eliciendo ex eis debitas conclusiones agibilium . } Non enim sufficit esse intelligentem , & puede bien gouernar su regno \textbf{ tomando delas razones conuenibles conclusiones | para todas las cosas } que ha de fazer . \\\hline
1.2.9 & Verum quia malitia est corruptiua principii . \textbf{ Sicut enim quis habens corruptum gustum , } male iudicat de saporibus , & corronpadera dela razon e del comienco para obrar . \textbf{ Ca assi commo aquel que ha el gosto } corronpido mal iudga delos sabores . \\\hline
1.2.9 & sic habens infectam , \textbf{ et deprauatam voluntatem , excoecatur in intellectu , } ut male iudicet de agibilibus : & assi aquel que ha corrupta e desordenada la uoluntad \textbf{ por maliçia es ciego en el entendimiento } e en la razon por que iudge mal en lo que ha de fazer \\\hline
1.2.9 & quae superius diximus , \textbf{ oportet ipsos esse bonos , } et non habere voluntatem deprauatam : & que dixiemos de suso \textbf{ conuieneles | que sean buenos } e que non ayan uoluntad mala nin desordenada \\\hline
1.2.9 & oportet ipsos esse bonos , \textbf{ et non habere voluntatem deprauatam : } ne propter malitiam appetitus , imprudenter agant , & que sean buenos \textbf{ e que non ayan uoluntad mala nin desordenada } por que por la maliçia dela uoluntad fagan las cosas sin razon \\\hline
1.2.10 & lex praecipit actus omnium virtutum . \textbf{ Praecipit enim lex operari fortia et temperata , } et uniuersaliter omnia & La ley manda fazer las obras de todas las uirtudes . \textbf{ Ca manda la ley obrar obras fuertes e obras tenpradas . } Et generalmente todas las obras \\\hline
1.2.10 & Sic etiam Ethicorum 5 scribitur , \textbf{ quod lex praecipit non derelinquere aciem , } neque fugere , & en el quinto libro delas ethicas \textbf{ que la ley manda non del enparar elaz en la fazienda } nin foyr dela fazienda \\\hline
1.2.10 & neque fugere , \textbf{ neque obiicere arma , } quod spectat ad fortitudinem . & nin foyr dela fazienda \textbf{ nin echar las armas dessi altp̃o del mester } las quales cosas pertenesçen ala fortaleza ¶ \\\hline
1.2.10 & Esse igitur Iustum secundum legem , \textbf{ et implere legalem Iustitiam , } est sequi omne bonum , & e ordenan toda manera de bondat Et por ende seer el omne iusto segunt la ley \textbf{ e conplir la iustiçia legales } segnir todo bien \\\hline
1.2.10 & et implere legalem Iustitiam , \textbf{ est sequi omne bonum , } et fugere omne vitium , & e conplir la iustiçia legales \textbf{ segnir todo bien } e esquiuar todo mal . \\\hline
1.2.10 & et fugere omne vitium , \textbf{ et habere quodammodo omnem virtutem , } propter quod legalis Iustitia dicta est & e esquiuar todo mal . \textbf{ Et es auer en alguna manera toda uirtud } ¶Et por ende la iustiçia legales \\\hline
1.2.10 & sed quia ea lex praecipit , \textbf{ et vult implere legem , } iustus legalis est . & mas en quanto las manda fazer la ley \textbf{ e el quiere conplir la ley es dicho iusto legal . } Et pues que assi es el iusto legal \\\hline
1.2.10 & Perfici ergo in ordine ad leges , \textbf{ est perfici in ordine ad Principem , } cuius est legem ferre , & alas leyes es seer acabado en orden al prinçipe \textbf{ al qual parte nesçe } commo dicho es confirmar la ley \\\hline
1.2.10 & Sic etiam dicitur \textbf{ unicuique tribuere quod suum est : } quia aequum est , & e lo que es igual \textbf{ Et assi es dicha dar a cada vno | lo que es suyo . } Ca cosa igual es \\\hline
1.2.11 & Dicebatur in praecedenti capitulo \textbf{ duas esse Iustitias , } unam generalem , et aliam specialem . & a assi commo dicho es en este capitulo \textbf{ sobredicho dos son las iustiçias vna general e otra espeçial . } Mas para que los regnos esten en su estado \\\hline
1.2.11 & Habere enim huiusmodi Iustitiam , \textbf{ est implere legem . } Si ergo lex iubet omne bonum , & sobredicho la iustiçia legales en alguna meranera toda uirtud \textbf{ Ca auer esta iustiçia es conplir la ley ¶ } Pues que assi es si la ley manda \\\hline
1.2.11 & et perfecta malitia . \textbf{ In nullo ergo obseruare leges , } et ciues non participare & es maliçia entera e acabada¶ pues que assi es quando los çibdadanos \textbf{ ennigua cosa non guardan las leyes } nin toma ninguna parte dela iustiçia legal . \\\hline
1.2.11 & cuius ciues integre essent mali , \textbf{ et in nullo vellent implere legem , } nec vellent in aliquo participare legalem Iustitiam . & si los çibdadanos fuessen enteramente malos . \textbf{ en ninguna cosa non quisi es en cunplir la ley } ni quisiesen tomar ninguna parte dela ley nin dela iustiçia . \\\hline
1.2.11 & et in nullo vellent implere legem , \textbf{ nec vellent in aliquo participare legalem Iustitiam . } Ex parte igitur ipsius legalis Iustitiae , & en ninguna cosa non quisi es en cunplir la ley \textbf{ ni quisiesen tomar ninguna parte dela ley nin dela iustiçia . | el regno no los podrie sos rir } ni la su çibdat non podrie mucho durar . \\\hline
1.2.11 & et in membris eiusdem corporis possumus \textbf{ aliquo modo contemplari Iustitiam . } Unius enim , & que la iustiçia es de vno assi mismo \textbf{ e en mienbros de vn cuerpo podemos entender la iustiçia en alguno manera . } Ca los mienbros de vn cuerpo mismo han ordenamiento entre si mismos \\\hline
1.2.12 & Quantum ergo animatum inanimatum superat , \textbf{ tantum Rex siue Princeps debet superare legem . } Debet etiam Rex esse tantae Iustitiae , & que ha alma sobrepiua ala que non ha alma . \textbf{ tanto el Rey o el prinçipe deue sobrepuiar la ley . } Ca deue el prinçipe o el Rey ser de tan grant iustiçia \\\hline
1.2.12 & maxime decet \textbf{ ipsum seruare Iustitiam . } Secundo possumus inuestigare hoc idem & que conuiene mucho al Rey \textbf{ de guardar la iustiçia¶ } La segunda manera por que podemos prouar \\\hline
1.2.12 & Si ergo decet Reges et Principes \textbf{ habere clarissimas virtutes } ex parte ipsius Iustitiae , & Et pues que assi es si conuiene alos Reyes \textbf{ e alos prinçipes de auer | muy claras } uirtudes paresçe de parte dela iustiçia \\\hline
1.2.12 & quae est quaedam clarissima virtus , probari potest , \textbf{ quod decet eos obseruare Iustitiam . } Tertio hoc probari potest & que se puede prouar \textbf{ que conuiene alos Reyes | de guardar la iustiçia . } lo terçero esso mismo se puede prouar \\\hline
1.2.12 & ostendit \textbf{ eos esse perfecte bonos . } Sic enim videmus in aliis rebus & muestra \textbf{ que ellos son acabados e buenos . } Ca assi lo veemos en todas las otras cosas \\\hline
1.2.12 & quae perficiunt hominem in se , \textbf{ se videntur habere ad Iustitiam , } quae perficit hominem in ordine ad alterum , & Et por ende todas las otras uirtudes morałs̃ que acaban el omne en ssi deuen auer la iustiçia \textbf{ assi commo reina e sennora } porque acaba el omne en orden alos otros \\\hline
1.2.12 & sicut subditi , \textbf{ qui quodammodo solum habent regere seipsos , } se habent ad Principem , & assi conmo los subditos \textbf{ que en alguna manera solamente han de gouernar assi mismos . } han se a su prinçipe \\\hline
1.2.12 & quanto ex eorum Iustitia potest \textbf{ consequi maius malum , } et potest inferri pluribus nocumentum . & e para escusar la mi ustiçia e el mal quanto por la mengua dela su iustiçia se puede seguir mayor mal \textbf{ Et puede venir mayor deño a muchos . } Mas avn conuiene mas de declarar commo los Reyes \\\hline
1.2.13 & et bene agere , \textbf{ oportet dare virtutem aliquam , } per quam regulentur in agendo . & e pecar en obrando . \textbf{ Conuiene de dar e de ponetur algua uirtud } por la qual seamos reglados en las obras \\\hline
1.2.13 & et non recte , \textbf{ oportet dare virtutem aliquam } circa timores , et audacias . & e en las osadias \textbf{ por la qual sea el omne reglado en ellos . | por que contesce que algunos remen algunas cosas } que han de temer e alas uegadas temen alguas cosas \\\hline
1.2.13 & et in aegritudinibus , \textbf{ et in aliis circa quae conuenit esse pericula . } Rursus in periculis bellorum homines diuersimode se habent . & e en las enfermedades \textbf{ e en los otros negoçios | en los quales pueden conteçer periglos . } Otrosi en los periglos delas faziendas \\\hline
1.2.13 & et etiam quia in periculis bellicis \textbf{ difficilius est reprimere timores , } quam moderare audacias : & Et ahun por que en los periglos delas batallas \textbf{ mas fuerte cosa es de repremer los temores } que de restenar las osadias . \\\hline
1.2.13 & rursus quia in audendo \textbf{ non tam difficile est aggredi pugnam , } sicut tolerare , & Otrosi por que en auiendo osadia \textbf{ non es tan fuerte nin tan graue cosa acometer la batalla } e la pellea commo sofrir \\\hline
1.2.13 & quod per fugam ea de facili vitare non possumus . \textbf{ Non enim sic per fugam vitare possumus aegritudines : } quia cum aegritudo sit aliquid in nobis existens , & que por foyr podemos ligeramente escapar dellos . \textbf{ Ca nos non podemos | assi por foyr escapar las enfermedades } por que la enfermedat es alguna cosa \\\hline
1.2.13 & sicut pericula belli . \textbf{ Cum ergo difficilius sit durare , et sustinere pericula illa } quae per fugam vitare possumus , & assi escusar commo los periglos delas batallas . \textbf{ Et pues que assi es commo sea mas | guaue cosa de endurar } e de sufrir aquellos periglos que podemos escusar \\\hline
1.2.13 & Cum ergo naturaliter tristia fugiamus , \textbf{ difficile est reprimere timores , } per quos tristia fugimus . & Et pues que assi es commo nos natural mente fuyamos dela tristeza \textbf{ graue cosa es de repmir los temores } por los quales fuyamos dela tristeza . \\\hline
1.2.13 & Propter quod difficilius est \textbf{ sustinere pugnam , } quam aggredi pugnantes . & Et por çierto mas \textbf{ guauecosa es de sefrir lo batalla } que de acometer los lidiadores ¶ \\\hline
1.2.13 & Difficilius autem est inniti , \textbf{ et habere se fortiter } contra mala praesentia , & guaue cosa es de esforçar se el omne \textbf{ e auer se fuertemente contra los males presentes } que contra los males \\\hline
1.2.13 & quia aggredi potest fieri subito : \textbf{ sed sustinere requirit diuturnitatem , et tempus . } Difficilius est autem habere se fortiter , & ¶ lo terçero esto es mas guaue cosa por que acometer puede se fazer \textbf{ adesora mas sofrir requiere mas luengotron . } Et por ende mas guaue cosa es auerse ome fuertemente \\\hline
1.2.13 & sed sustinere requirit diuturnitatem , et tempus . \textbf{ Difficilius est autem habere se fortiter , } et constanter in sustinendo bella , & adesora mas sofrir requiere mas luengotron . \textbf{ Et por ende mas guaue cosa es auerse ome fuertemente } e firmemente en sufriendo las batallas \\\hline
1.2.13 & ( et subdit ) \textbf{ Fortitudinem esse in sustinendo tristia . } Declaratum est igitur , & Et adelante dize \textbf{ que la fortaleza es en sofrir las cosas tristes . } Et por ende ya declarado es cerca quales cosas ha de seer la fortaleza . \\\hline
1.2.13 & restat ergo declarandum , \textbf{ quomodo possumus facere nos ipsos fortes . } Notandum ergo , & Pues que assi es fincanos de declarar \textbf{ en qual manera podemos fazer anos mismos fuertes } Pues que assi es deuen dos notar e entender que commo quier que la uirtud sea contraria . \\\hline
1.2.13 & Sed quia difficilius est \textbf{ reprimere timores , } quam moderare audacias : & assi commo mas \textbf{ guaue cosa es de repremir los temores } que refrenar las osadias . \\\hline
1.2.13 & Quia igitur non possumus punctualiter \textbf{ attingere medium inter audaciam , } et timorem : & Et mas contradize el temor ala fortaleza que la osadia \textbf{ ¶Pues que assi es porque non podemos en punto alcançar el medio entre la osadia e el temor . } por ende auemos de inclinar nos mas ala osadia \\\hline
1.2.13 & Tertio declaratum fuit , \textbf{ quomodo possumus facere nos ipsos fortes : } quia maxime hoc faciemus , declinando magis ad audaciam , & ¶ Lo terçero ya declaramos \textbf{ en qual manera podemos fazer a nos mismos fuertes . } Ca mayormente nos podemos fazer fuertes \\\hline
1.2.14 & quum sunt in ignotis partibus , \textbf{ committere aliqua turpia , } quae inter ciues et notos nullatenus attentarent . & do non son conosçidos \textbf{ acometen alguas torpedades } las quales non quarrian acometer nin tentar entre los sus çibdadanos en ningunan manera \\\hline
1.2.14 & Hoc autem modo quidam Dux dicitur exercitum suum \textbf{ coegisse ad Fortitudinem . } Nam , cum nauigiis , & Et en esta manera vn caudiello dizen \textbf{ que costrino sus conpannas a fortaleza . | por que fuesen fuertes . } Ca commo el estudiese en sus naues \\\hline
1.2.15 & quam ratio dictet , \textbf{ fugere delectationes corporales sensibiles . } Qui igitur omnes delectationes insequitur , & Et pues que assi es el non sentirse es foyr delas delecta con \textbf{ nessenssibles e corporales | mas que la razon manda } Et por ende aquel que se da a todas las delecta \\\hline
1.2.15 & quam delectationibus aliorum sensuum . \textbf{ Possumus enim videre , audire , et odorare distantia : } sed non possumus gustare , & que en los otros . \textbf{ Ca podemos veer e oyr | e oler cosas que estan arredradas de nos . } mas non podemos gostar nin tanner \\\hline
1.2.15 & et tactus magis directe \textbf{ et immediate videntur ordinari ad conseruationem nostram : } ut delectabilia gustus & son mas derechamente \textbf{ e mas ayuntadamente ordenades alanr̃auida | e alanr̃a conseruaçion } assi commo las cosas \\\hline
1.2.15 & Oportet enim vere temperatum \textbf{ non exercere opera venerea , neque gestus . } Prout ergo abstinet & e la linpieza refreña \textbf{ e abaxan las delecta connes | e los uicios dela carne } Et por ende conuiene aquel que uerdaderamente es \\\hline
1.2.15 & His visis de leui patet , \textbf{ quomodo nosipsos facere possumus temperatos . } Nam Temperantia , & Et estas cosas vistas \textbf{ que dichas son de ligero paresçe commo nos mismos nos podemos fazer tenprados . } Ca la tenpranca e la fortaleza se ha \\\hline
1.2.15 & sic Temperantia plus conuenit cum insensibilitate . \textbf{ Si ergo volumus nosipsos facere temperatos , } ad illam partem declinandum est , & que con la senssiblidat de los sesos . \textbf{ Et por ende si nos quisieremos fazer a nos mismos tenprados deuemos } declinara aquella parte \\\hline
1.2.15 & Quarto vero declaratum fuit , \textbf{ quomodo possumus nosipsos facere temperatos : } quia hoc maxime faciemus & ¶Lo quarto declaramos \textbf{ en qual manera podemos fazer a nos mismos tenprados . } Ca esto podemos fazer mayormente \\\hline
1.2.16 & tum etiam quia facilius est \textbf{ ei facere bonum , } et acquirere temperantiam , & Lo vno por que peta mas de uoluntad \textbf{ ¶Lo otro por que mas ligeramente puede bien fazer e ganar tenprança que fortaleza . } Mas que el \\\hline
1.2.16 & ei facere bonum , \textbf{ et acquirere temperantiam , } quam sit acquirere fortitudinem . & ¶Lo otro por que mas ligeramente puede bien fazer e ganar tenprança que fortaleza . \textbf{ Mas que el } destenprado pequemas de voluntad que el temeroso puede se demostrar \\\hline
1.2.16 & et acquirere temperantiam , \textbf{ quam sit acquirere fortitudinem . } Quod autem magis voluntarie peccet intemperatus & Mas que el \textbf{ destenprado pequemas de voluntad que el temeroso puede se demostrar } por dos razons¶ \\\hline
1.2.16 & quam timidius dupliciter ostendi potest . \textbf{ Primo , quia insequi voluntates intemperatas , } est delectabile : & por dos razons¶ \textbf{ La primera es por que segnir } plazenterias desfenpradas es cosa delectable \\\hline
1.2.16 & est delectabile : \textbf{ fugere autem et timere , est tristabile . } Magis quis voluntarie agit & plazenterias desfenpradas es cosa delectable \textbf{ Mas fuyr e temer es cosatste . } Et mas de uoluntad faze cada vno \\\hline
1.2.16 & nec voluntarie et deliberate agit quod agit . \textbf{ Tolerabilius est igitur peccare per timorem , } quam per intemperantiam : & e esta fuera de ssi non faz aquello que faze por uoluntad ñcon delibramiento . \textbf{ Et por ende mas de foyr | e de escusares de pecar } por temor o por miedo \\\hline
1.2.16 & sed aggredi terribilia , \textbf{ et experiri bellum , sine periculo non potest . } Valde est ergo increpandus carens tempesantia , & mas acometer las cosas espantables \textbf{ e puar las batallas non se puede fazer sin periglo . } Et pues que assi es mucho es de denostar el \\\hline
1.2.16 & quam non esse fortes . \textbf{ Si ergo Regem non esse virilem , } et non esse constantem & que por non ser fuertes . \textbf{ Et por ende si el Rey non fuere fuerte } e non fuer firme en el coraçon es de deno star por ello . \\\hline
1.2.16 & Si ergo Regem non esse virilem , \textbf{ et non esse constantem } animo est exprobrabile , & Et por ende si el Rey non fuere fuerte \textbf{ e non fuer firme en el coraçon es de deno star por ello . } Et es mas de denostar si fuer deste prado \\\hline
1.2.16 & patet quod est exprobrabilius \textbf{ ipsum esse intemperatum , } et insecutorem passionum . & e non fuer firme en el coraçon es de deno star por ello . \textbf{ Et es mas de denostar si fuer deste prado } e segnidor de passiones . \\\hline
1.2.16 & esse bestialem et seruilem : \textbf{ indecens est ipsum esse intemperatum . } Secundo intemperantia est vitium maxime puerile . & enssennorear alos otros de ser bestial e sieruo \textbf{ e non es cosa conuenible | que el sea destenp̃do ¶ } Lo segundo la \\\hline
1.2.16 & Ideo maxime videmus eos sequi delectabilia , \textbf{ et esse insecutores passionum . } Unde Philosophus 3 Ethicorum vim concupiscibilem , & mas sigue las cosas delecta bles \textbf{ e son seguidores delas pasiones | e de los apetitos que los otros } ¶ Onde el philosofo en el terçero libro delas ethicas \\\hline
1.2.16 & esse vitium intemperantiae assimilat puero : \textbf{ quia sicut puer debet regi per paedagogum , } sic vis concupiscibilis est regenda , & destenpranca al moço . \textbf{ Ca assi commo el moço se deue gouernar | por su ayo o por su maestro } assi el apetito cobdiciador \\\hline
1.2.16 & esse puerum moribus , \textbf{ et non sequi rationem , sed passionem : } indecens est ipsum esse intemperatum . & de ser el Rey moço en costunbres \textbf{ e de non segnir razon | e en entendimiento mas passiones } e delectaçiones \\\hline
1.2.16 & et non sequi rationem , sed passionem : \textbf{ indecens est ipsum esse intemperatum . } Tertio est hoc indecens Regi : & e en entendimiento mas passiones \textbf{ e delectaçiones | non es cosa conuenible } que el Rey sea es tenprado ¶ \\\hline
1.2.16 & se reuerendam et honore dignam , \textbf{ maxime indecens est eam esse intemperatam . } Exemplum autem huius habemus & e muy digna de honrra . \textbf{ Much̃o desconuenible cosa es | que el Rey sea destenprado } e desto auemos \\\hline
1.2.16 & Dux autem ille assuetus rebus bellicis , \textbf{ videns Regem suum esse totum muliebrem et bestialem , } statim ipsum habuit in contemptum : & veyendo \textbf{ que el su Rey era todo mugeril | e toda su } conuerssaçion era entre mugers e era bestial . \\\hline
1.2.17 & et ostendimus quomodo Reges et Principes illis virtutibus decet \textbf{ esse ornatos . } Reliquum est pertransire & e los prinçipes deuen ser conpuestos e honrrados \textbf{ dellas fincanos } de dezir delas otras och̃o uirtudes . \\\hline
1.2.17 & magnificentia vero dicitur \textbf{ respicere magnos sumptus ; } quod quomodo sit intelligendum , & e non son grandes nin pequenas . \textbf{ Mas la magnificençia es tal uirtud que cata alas grandes despenssas . } Et esto en qual manera se deue entender adelante lo mostrͣemos¶ \\\hline
1.2.17 & contra rectam regulam rationis , \textbf{ oportet dare virtutem aliquam mediam } inter auaritiam , et prodigalitatem : & escontra regla derecha de razon e de entendimiento . \textbf{ Conuiene de dar alguna uirtud medianera } entre la auariçia e el gastamiento . \\\hline
1.2.17 & Spectat autem ad liberalem \textbf{ non usurpare alios redditus , } et custodire proprios . & Ca pertenesçe al franco \textbf{ que non tome | nin fuerce las rentas de los otros } e que guarde \\\hline
1.2.17 & non usurpare alios redditus , \textbf{ et custodire proprios . } Nam licet liberales & nin fuerce las rentas de los otros \textbf{ e que guarde | e tome las suyas . } Ca maguera que el franco non ha menos \\\hline
1.2.17 & non debet proprios redditus inaniter dispergere . \textbf{ Ergo non usurpare redditus alienos , } habere debitam curam de propriis , & nin espender vanamente \textbf{ nin deue tomar las rentas agenas por fuerca } mas deue auer cuydado de su fazienda \\\hline
1.2.17 & Ergo non usurpare redditus alienos , \textbf{ habere debitam curam de propriis , } et ex eis debitos sumptus facere : & nin deue tomar las rentas agenas por fuerca \textbf{ mas deue auer cuydado de su fazienda } e delas sus rentas propias e fazer dellas sus espenssas quales conuiene ¶ \\\hline
1.2.17 & Circa autem proprios redditus custodire , \textbf{ et circa non accipere alienos , } est ex consequenti . & Mas despues desto es en guardar las tentas propias . \textbf{ Et despues es en non tomar nin forcar los bienes agenos . } Ca aquel que vsurpa e toma los bienes prouechosos agenos malamente commo non deue . \\\hline
1.2.17 & expoliatores mortuorum , et aleatores , \textbf{ dicit esse turpia lucra : } et omnes tales appellat illiberales . & e los iugadores delas tablas \textbf{ e de los otros iuegos | Dize } que talon iuegos e ganançias commo estas son torpes e de sone stos . \\\hline
1.2.17 & ex consequenti autem est \textbf{ circa custodire proprios redditus , } et circa non usurpare alienos . & que mas prinçipalmente es la franqueza en espender e en fazer bien alos otros . \textbf{ Et despues desto es en guardar las sus rentas propreas } e non vsurpar nin tomar las agenas . \\\hline
1.2.17 & circa custodire proprios redditus , \textbf{ et circa non usurpare alienos . } Probat enim Philosophus & Et despues desto es en guardar las sus rentas propreas \textbf{ e non vsurpar nin tomar las agenas . } Ca el philosofo praeua en el quarto libro delas ethicas \\\hline
1.2.17 & Uti autem pecunia , \textbf{ est expendere eam } et tribuere eam aliis . & en uso conueinble de espender del auer . \textbf{ Ca husar del auer es en espender lo } e partir lo alos otros \\\hline
1.2.17 & est expendere eam \textbf{ et tribuere eam aliis . } Custodire autem proprios redditus , & Ca husar del auer es en espender lo \textbf{ e partir lo alos otros } mas guardar el omne \\\hline
1.2.17 & et tribuere eam aliis . \textbf{ Custodire autem proprios redditus , } non est uti pecunia , & e partir lo alos otros \textbf{ mas guardar el omne } lo suyo non es husar del auerante es mas ganar lo e allegar lo . \\\hline
1.2.17 & sed magis est acquirere \textbf{ et generare ipsam . } Propter quod patet liberalitatem & lo suyo non es husar del auerante es mas ganar lo e allegar lo . \textbf{ por la qual cosa paresçe } que la franqueza es mas en espender \\\hline
1.2.17 & esse magis circa expendere \textbf{ et circa tribuere pecuniam aliis , } quam circa proprios redditus custodire . & que la franqueza es mas en espender \textbf{ e partir el auer alos otros que en guardar las rentas propias } ¶ \\\hline
1.2.17 & quia ad virtutem principalius spectat \textbf{ facere maius bonum . } Maius autem bonum est benefacere , & Lo segundo esto mismo se praeua assi por que ala uirtud \textbf{ mas prinçipal parte nesçe de fazer mayor bien . } Et mayor bien es en bien fazer \\\hline
1.2.17 & circa debitos sumptus , \textbf{ et circa debitas rationes ; } ex consequenti autem est & Ca prinçipalmente es en las espenssas conuenibles \textbf{ e en las dona connes conuenibles } e despues desto ha de ser \\\hline
1.2.18 & ad possessiones dantis . \textbf{ Ideo Philosophus ait Tyrannos non esse prodigos : } quia non videntur posse & en conparaçion delas possesiones e de las rentas del queda \textbf{ Et por ende dize el philosofo | que los Reyes non son gastadores } por razon que non pueden \\\hline
1.2.18 & propter quod omnino detestabile est \textbf{ Reges et Principes esse auaros : } tam enim fugienda est auaritia a principibus & por la qual razon muy de denostar son los Reyes \textbf{ e los prinçipes | si fueren auarientos . } Ca en tanto deue ser arredrada la auariçia de los prinçipes \\\hline
1.2.18 & omnino detestabile est \textbf{ Regem esse auarum , } et quod melius esset & de que non pueda guaresçer \textbf{ por ende mucho de denostar es el Rey } sy fuere auariento ¶ Et pues que assi es paresçe \\\hline
1.2.18 & Si ergo omnino decens est \textbf{ Regem esse virtuosum , } tanto detestabilius est & qual quier gastador liberal e franco ¶ \textbf{ pues que assi es si es conueinble al Rey } en toda manera de ser uirtuoso tanto \\\hline
1.2.18 & tanto detestabilius est \textbf{ ipsum esse auarum , } quam prodigum : & en toda manera de ser uirtuoso tanto \textbf{ mas de denostar es el Rey si fuer auariento } que si fuere gastador \\\hline
1.2.18 & Omnino ergo detestabile est , \textbf{ Regem esse auarum . } Viso quod quasi impossibile est & Et el gastadora muchos aprouecha dando . \textbf{ Et por ende muy de depostar es el Rey si fuer auariento ¶ visto } que los Reyes non pueden ser gastadores \\\hline
1.2.18 & et quod omnino detestabile est \textbf{ eos esse auaros : } restat ostendere , & e que muchon son de denostar \textbf{ si fueren auarientos fincanos de demostrar } que conuiene alos Reyes \\\hline
1.2.18 & quae continet . \textbf{ Cum ergo tanto deceat fontem habere os largius , } quanto ex eo plures participare debent : & Ca ha . manera daua so ancho e largo e da conplidamente lo que tiene \textbf{ ¶pues que assi es conmo tanto conuenga ala fuente auer la boca | mas ancha } quanto della deuen \\\hline
1.2.18 & Spectat autem ad liberalem primo \textbf{ respicere quantitatem dati , } ut non det minus , & Mas par tenesce al libal e alstan ço de catar tres cosas ¶ \textbf{ La primera deue catar la quantidat | delo que da } por que non de menos o mas \\\hline
1.2.18 & quia magnitudo expensarum vix potest \textbf{ excedere multitudinem reddituum . } Imo si contingat liberalem & por que la grandeza delas espenssas \textbf{ apenas puede sobrepuiar ala muchedunbre de las sus rentas . Por ende si contesçe algunas uegadas al liberal de dar } mas de quanto deue legunt \\\hline
1.2.19 & et minus non videantur \textbf{ diuersificare speciem , } et naturam rerum , & Mas commo en cada cosa \textbf{ mas e menos non fagan departimiento en la naturaleza } e en la semeiança delas cosas \\\hline
1.2.19 & in mediocribus sumptibus , \textbf{ dici potest magnanimitatem } quae est circa magnos sumptus , & que en las espenssas medianas e mesuradas . \textbf{ Conuiene de dezir que la magnifiçençia } que es en las grandes \\\hline
1.2.19 & quae est circa magnos sumptus , \textbf{ esse virtutem aliam a liberalitate , } quae est circa mediocres . & que es en las grandes \textbf{ espenssassea otra uirtud e apartada dela liberalidat } que es çerca delas medianas mesuradas espenssas . \\\hline
1.2.19 & Sumptus enim , \textbf{ vel possunt considerari secundum se , } vel ut proportionantur facultatibus . & Por ende la liłalidat se estiende alas espenssas mesuradas \textbf{ ca las espenssas o se pueden penssar | segunt } si o se pueden penssar \\\hline
1.2.19 & ista decenter se habere debet : \textbf{ non tamen aeque principaliter intendere debet circa omnia ista . } Nam principaliter et primo , & conueniblemente çerca estas quatro cosas . \textbf{ Mas enpero non deue entender egualmente nin prinçipalmente cerca estas quatro cosas . } Ca primero e prinçipalmente deue seer el omne magnifico \\\hline
1.2.20 & Videtur enim ei , \textbf{ quod remouere a se pecuniam , } sit abscindere membra a proprio corpore . & sienpre las faze tardando . \textbf{ Ca paresçe leal paruifico } que tirar el auer de ssi \\\hline
1.2.20 & quod remouere a se pecuniam , \textbf{ sit abscindere membra a proprio corpore . } Ideo sicut dato & Ca paresçe leal paruifico \textbf{ que tirar el auer de ssi | estaiarle los mienbros de su cuerpo . } Por ende assi commo si fuesse menester \\\hline
1.2.20 & qualiter faciat magnum opus , \textbf{ ut qualiter faciat debitas largitiones , } vel quomodo faciat decentes nuptias : & en qual manera faga granada obra \textbf{ e en qual manera faga sus dones granados e conuenibles } o en qual manera faga sus bodas conuenibles \\\hline
1.2.20 & non potest paruificus \textbf{ ita modicum sumptum facere erga quodcunque opus , } quin semper videatur ei & assi en essa misma manera non puede el \textbf{ parufico fazer despenssas tan pequanas } en qual si quier obra que faga que non le paresca sienpre a el \\\hline
1.2.20 & cum tristitia et dolore ; et cum nihil facit , \textbf{ credere se magna operari , } quia omnia haec valde derogant regiae maiestati , & Et quando non faze ningunan cosa cree el \textbf{ que faze grandescosas e grandes obras . } Et por que todas estas cosas ponen grand mengua en la Real magestad \\\hline
1.2.20 & Quod autem deceat \textbf{ ipsum esse magnificum , } sufficienter probant superiora dicta : & que el Rey sea periufico mas que conuengaal Rey de ser magnifico \textbf{ e de fazer grandes espenssas } conplidamente es prouado \\\hline
1.2.20 & distribuere bona regni , \textbf{ maxime decet ipsum esse magnificum . Nam quia est caput regni , } et gerit in hoc Dei vestigium , & e a el pertenesca de partir los bienes del regno mucho le conuiene a el de ser magnifico . \textbf{ Ca porque es cabeça del regno } e ha en esto semeiança de dios \\\hline
1.2.20 & maxime spectat ad eum magnifice \textbf{ se habere circa bona communia , } et circa ea quae respiciunt regnum totum . & mucho parte nesçe a el de se auer granadamente \textbf{ e honrradamente çerca los bien es comuns } e cerca todas aquellas cosas \\\hline
1.2.20 & distribuere bona regni , \textbf{ omnino decet eum magnifice se habere erga personas dignas , } quibus digne competunt illa bona . & prinçipalmente partir los bienes del regno \textbf{ en todas maneras le conuiene ael de se auer grande | e honrradamente a aquellas personas } que son dignas \\\hline
1.2.20 & regia persona debet esse reuerenda et honore digna , \textbf{ spectat ad Regem magnifice se habere erga personam propriam , } et erga personas sibi coniunctas , & La persona del Rey deue ser de grand reuerençia \textbf{ e digna de grand honrra parte nesçe mucho al Rey de se auer granadamente } e honrradamente çerca dela su persona propia \\\hline
1.2.21 & quia magnificus assimilatur scienti . \textbf{ Dicebatur enim spectare ad magnificum } in magnis operibus facere decentes sumptus . & La primera es que el magnifico es semeiante al sabio \textbf{ Ca dixiemos de suso } que conuenia al magnifico de fazer conuenientes espenssas en las grandes obras . \\\hline
1.2.21 & Secunda proprietas magnifici , \textbf{ est facere magnos sumptus , } non ut ostendat seipsum , & entendimiento¶ \textbf{ La segunda propiedat del magnifico es fazer grandes espenssas } non por que se muestre \\\hline
1.2.21 & Est enim hoc commune cuilibet virtuti , \textbf{ agere non propter fauorem , } vel propter gloriam hominum , & nin por vanagłia mas por razon de algun bien \textbf{ ca esto es comun a cada vna delas uirtudes obrar non } por honrra o por vanagłoia de los omes \\\hline
1.2.21 & difficile est in talibus \textbf{ non quaerere humanam laudem . } Et quia virtus est & Enpero guaue cosa es en tales cosas \textbf{ non demandar loor delas gentes } Et por quela uirtud es cerca bien e cerca la cosaguaue . \\\hline
1.2.21 & in suis magnificis operibus , \textbf{ et distributionibus intendere finaliter bonum , } et non fauorem , et gloriam hominum . & en las sus muy grandes obras \textbf{ e en las sus parti | connsenteder finalmente el bien } e non honrra \\\hline
1.2.21 & quam quot et quanta numismata oporteat \textbf{ ipsum consumere propter huiusmodi opera . } Quinta proprietas est , & e en qual manera aquellos dones sean grandes e conuenibles que entender e cuydar quantos des \textbf{ e quanto auer le conuiene ael de despender en estas obras ¶ } La quinta propiedates \\\hline
1.2.21 & esse liberalem , \textbf{ facere maximos decentes sumptus , } quos facit magnificus , & si faz conuenibles espenssas faz omne ser liberal fazer muy grandes \textbf{ e muy conuenibles espenssas } lo que faze el magnifico es ser mucho mas liberal ¶ \\\hline
1.2.21 & Ad eos autem maxime spectat \textbf{ facere magnas largitiones , } et excellentes sumptus boni gratia & e conosçedores quales despenssas a quales obras conuienen . \textbf{ Et aellos otrosi mucho mas pertenesçe de fazer grandes donaconnes } e lobre puiantes de espenssas \\\hline
1.2.21 & Oportet etiam eos esse excellenter liberales , \textbf{ et semper facere magnifica opera . } Omnes igitur proprietates magnifici per amplius , & e alos prinçipes de ser liberales muy altamente \textbf{ e de fazer sienpre obras muy grandes e magnificas . } Et pues que assi es todas las propiedades del magnifico \\\hline
1.2.21 & quia non quilibet potest \textbf{ facere magnos sumptus . } Sed , ut ibidem dicitur , & que non pue de cada vno ser magnifico \textbf{ por que non puede cada vno fazer grandes espenssas } Mas assi commo alli dize el philosofo tales son los nobles e los głiosos . \\\hline
1.2.21 & Sed , ut ibidem dicitur , \textbf{ tales oportet esse nobiles et gloriosos . } Quare quanto est nobilior aliis , & por que non puede cada vno fazer grandes espenssas \textbf{ Mas assi commo alli dize el philosofo tales son los nobles e los głiosos . } por la quel cosa en quanto el Rey es mas noble \\\hline
1.2.21 & tanto decet ipsum pollere magnificentia , \textbf{ et habere proprietates magnifici . } Bonorum exteriorum & mas le conuiene ael de resplandesçer \textbf{ por magnificençia e auer propiedades de magnifico | e de ome muy guanade } ssi commo dicho es de suso algunos de los bienes de fuera son aprouechosos \\\hline
1.2.22 & 4 Ethicor’ velle , \textbf{ magnanimitatem esse circa honores , } circa diuitias , et principatus , & en el quarto libro delas ethicas dize \textbf{ que la magranimidat | es cerca las honrras } et cerca las riquezas \\\hline
1.2.22 & Ad pusillanimem enim pertinet \textbf{ nescire fortunas ferre . } Ideo Andron’ Perip’ ait : & Mas al pusill animo \textbf{ e de flaco coraçon pertenesçe non saber sofrir buenas uenturas . } Por ende dize andronico el sabio philosofo \\\hline
1.2.22 & quae nos ad magnanimitatem trahunt , \textbf{ est parua pretiari exteriora bona , } quaecunque sint illa , & que inclinan anos a magnanimidat \textbf{ es poco preçiar todos los bienes } de fuera quales se quier que sean . \\\hline
1.2.22 & siue quaecunque alia huiusmodi bona . \textbf{ Dictum est enim pusillanimem nescire fortunas ferre , } sed ex modico fortunio extolli , & si quier quales si quier otros tales bienes \textbf{ Ca dicho es de suso | que el pusillanimo non sabe sofrir buenas uenfas } mas de muy \\\hline
1.2.22 & cum non reputamus \textbf{ talia esse simpliciter maxima bona . } Et si contingat nos infortunari circa ea , & por que non cuydaremos \textbf{ que tales bienes son los mayores bienes . } Et si contesçiere quenos \\\hline
1.2.23 & Prima proprietas magnanimi , \textbf{ est bene se habere circa pericula . } Bene autem se habere circa ea , & La primera propriedat del magnanimo es \textbf{ que se deue bien auer çerca los periglos } Mas auer se bien çerca los periglos \\\hline
1.2.23 & non parcat vitae , \textbf{ ut Philosophus ait 4 Ethic’ . Secundo competit magnanimo se habere bene circa retributiones . } Magnanimus enim parum appreciatur exteriora bona , & assi commo dize el philosofo \textbf{ en el quarto libro delas ethicas¶ | La segunda propiedat que parte nesçe al magnanimo es } auer se bien çerca las particones delos dones dando a cada vno \\\hline
1.2.23 & Propter quod , quia esse plurimum retributiuum , \textbf{ est agere opera virtutum , } conuenit magnanimo esse plurimum retributiuum , & e dador de los galardones \textbf{ es fazer obras de uirtudes . } Ca assi commo dize el philosofo en el quarto libro delas ethicas pertenesce mucho \\\hline
1.2.23 & est agere opera virtutum , \textbf{ conuenit magnanimo esse plurimum retributiuum , } ut dicitur 4 Ethicor’ . & es fazer obras de uirtudes . \textbf{ Ca assi commo dize el philosofo en el quarto libro delas ethicas pertenesce mucho } almagnanimo ser mucho partidor e dador de gualardones ¶ \\\hline
1.2.23 & ut circa ea , \textbf{ ex quibus consurgere possunt magni honores ; } talia autem non multotiens occurrunt , & assi commo cerca aquellas \textbf{ de que se pueden le unatar grandeshonrras . } Et tales cosas commo estas non contesçen muchas vezes . \\\hline
1.2.23 & omnia negocia \textbf{ quantumcumque modica expedire per seipsos , } nec decet eos omnium esse operatiuos ; & por si mismos todos los negoçios \textbf{ mayormente los que son pequa nons } nin conuiene aellos de seer obradores de todas las cosas \\\hline
1.2.23 & quae sunt multa . \textbf{ Quarto decet esse apertos , } ut esse veridicos ; & que son muchos alos otros ¶ \textbf{ Lo quarto conuiene alos Reyes } de seer manifiestos e claros e seer uerdaderos \\\hline
1.2.24 & quod non congruit \textbf{ magnanimo fugere commouentem , } quod est actus prudentiae , & que non pertenesce almagranimo foyr \textbf{ de aquel qual bien conseia . } Ca esto es obra de pradençia \\\hline
1.2.24 & et honoris amatiuos . Reges enim et Principes decet honores diligere modo quo dictum est ; \textbf{ videlicet , ut diligant et cupiant facere opera , } quae sint honore digna . & e alos prinçipes amar las honrras \textbf{ en la manera que dich̃ones de suso . | Conuiene saber que amen e cobdicien fazer lobras } que sean dignas de honrra . \\\hline
1.2.24 & quae sint honore digna . \textbf{ Videtur enim honoris amatiua se habere ad magnanimitatem , } sicut formositas corporis & que sean dignas de honrra . \textbf{ por que paresçe que la uirtud es dicha amadora de honrra se ha ala magnanimidat | assi commo la fermosura cortoral se ha } ala apostura grande de todo el cuerpo . \\\hline
1.2.25 & non appellat magnanimum , sed temperatum . \textbf{ Cum igitur habere temperantiam in honoribus , } sit idem , & mas llamale tenprado . \textbf{ Et por ende auer algun tenpramiente en las honrras } es esso mismo \\\hline
1.2.25 & sit idem , \textbf{ quod habere humilitatem : } virtus illa , & es esso mismo \textbf{ que auer humildat . } Et aquella uirtud o razon de tenprança \\\hline
1.2.25 & si unum et idem aliter et aliter acceptum nos retrahit et impellit , \textbf{ oportebit circa illud dare duas virtutes , } unam impellentem , & e nos allega a aquello que la razon manda o uieda . \textbf{ Conuiene de dar en aquella cosa dos uirtudes ¶ La vna que nos allegue . } Et la otra qua nos arriedre dello . \\\hline
1.2.25 & Nam cum impossibile sit \textbf{ esse magnanimum non existentem bonum , } ut probat Philosophus 4 Ethic’ & Ca commo non pueda seer \textbf{ que alguno sea magnanimo | e que non sea bueno } assi \\\hline
1.2.25 & Utrum autem humilitas sit idem simpliciter \textbf{ quod diligere mediocres honores , } vel utrum sit idem simpliciter & Mas si la humildat es essa misma cosa \textbf{ sinplemente que amar las honrras medianeras . } O si es essa misma cosa \\\hline
1.2.25 & de qua loquitur Philosophus , \textbf{ non esse per omnem modum idem cum humilitate : } quia illa de qua Philosophus loquitur , & mostrariamos que la uirtud de que fabla el philosofo \textbf{ non es en toda manera vna cosa misma con la humildat } por que aquella de que fabla el philosofo \\\hline
1.2.26 & haec duo eidem virtuti competere possunt . \textbf{ Spectat igitur ad magnanimitatem reprimere desperationem , } ne desperemus de bonis arduis , & e prinçipalmente pueden parte nesçer a vna uirtud \textbf{ e por ende pertenesçe ala magranimidat repremir la desparaçion } por que non desesꝑemos de los bienes muy altos . \\\hline
1.2.26 & circa quae habet esse . \textbf{ Intendit enim humilis reprimere superbias , } et moderare deiectiones . & çerca quales cosas ha de seer . \textbf{ Ca el humildoso entiende repremir las soƀͣiuas } e tenprar los despreçiamientos e los decaemientos . \\\hline
1.2.26 & circa haec aeque principaliter . \textbf{ Nam humilitas principaliter intendit reprimere superbias , } ex consequenti vero moderare deiectiones . & Enpo non es cerca desto egualmente nin prinçipalmente \textbf{ por que la humildat prinçipalmente entiende repremir las soƀͣmas . } mas despues desto entiende tenprar los despreçiamientos \\\hline
1.2.26 & Inquirendo enim opera honore digna , \textbf{ non solum contingit peccare per superbiam , } sed etiam per deiectionem . & que son dignas de grant honrra \textbf{ non solamente pueden pecar | por sobrepuiamiento } mas ahun puede pecar \\\hline
1.2.26 & Debent enim Reges \textbf{ sic quaerere opera honore digna , } non ultra quam ratio dictet , & Por que conuiene alos Reyes \textbf{ et alos prinçipes | assi de madar las obras dignas de honrra } que non sean mas que la razon \\\hline
1.2.26 & quod faciunt superbi . \textbf{ Debent enim agere bona opera } et honore digna boni gratia , & en sobrepuiança de honrra lo que fazen los sobuios \textbf{ por que deuen fazer los Reyes bueans obras e dignas de honrra | non por alabança } e por sobrauia \\\hline
1.2.27 & et deficere , \textbf{ oportet ibi dare virtutem aliquam , } per quam dirigamur ad bene agendum , & e tal sesçer conuiene de dar y . \textbf{ alguna uirtud por la qual seamos enderesçados } abien obrar \\\hline
1.2.27 & praeter ordinem rationis . \textbf{ Ratio enim dictat punitiones aliquas esse faciendas , } et quod est irascendum , & fuera de orden de razon e de entendimiento \textbf{ por que la razon demanda | que algunas penas sean dadas } e algunas venganças sean fechas \\\hline
1.2.27 & contingit superabundare et deficere : \textbf{ oportet ibi dare virtutem } aliquam reprimentem superabundantias , & e uenganças del contesçe de sobrepiuar e de fallesçer . \textbf{ Conuiene de dar y alguna uirtud } que reprima las sobrepuianças \\\hline
1.2.27 & Mansuetudo enim principaliter \textbf{ et primo intendit reprimere iras , } ex consequenti autem intendit moderare passiones oppositas irae . & e prinçipalmente entiende repremir las sañas \textbf{ mas despues desto entiende tenprar las passiones contrarias dela sana | que es nunca se enssanar } por lo que ha razon de se ensannar . Ca natural cosa es anos \\\hline
1.2.27 & quam puniendi sint . \textbf{ Difficile est ergo valde reprimere iras , } et non appetere punitiones iniuriarum & por el mal que nos fazen . \textbf{ Et por que muy | guaue cosa es de repremir las sannas } e de non dessear uengança delas iniurias \\\hline
1.2.27 & Difficile est ergo valde reprimere iras , \textbf{ et non appetere punitiones iniuriarum } ultra quam dictet ratio . Plures ergo peccant in appetendo plus : & guaue cosa es de repremir las sannas \textbf{ e de non dessear uengança delas iniurias } mas que la razon e el entendimiento muestra \\\hline
1.2.28 & nisi recte conuersari cum hominibus , \textbf{ et ordinare opera , } et verba nostra & si non derechamente beuir con todos \textbf{ e ordenar las nr̃as palauras } e las nuestras obras a buena conuerssaçion e conuenible . \\\hline
1.2.28 & et verba nostra \textbf{ ad debitam conuersationem . } Secundo , verba , et opera nostra & e las nuestras obras a buena conuerssaçion e conuenible . \textbf{ lo segundo las nr̃as palauras } e las nr̃as obras siruennos ala uerdat \\\hline
1.2.28 & quia nec quis se debet \textbf{ tantum aliis ostendere socialem , } ut videatur placidus , & Mas cada vno destos fallesçen en cada vna destas razo nes \textbf{ por que ninguno non le deue en tanto mostrar conpanero alos otros } por que sea visto plazentero e falaguero . \\\hline
1.2.28 & circa quam contingit abundare et deficere , \textbf{ oportet dare uirtutem } aliquam reprimentem superabundantias , & cerca la qual contesçe de sobrepuiar e de fallesçer . \textbf{ Conuiene de dar uirtud alguna } que reprima las sobrepuianças \\\hline
1.2.28 & et uerba , \textbf{ ut ordinantur ad debitam conuersationem in uita . } Si enim homo est naturaliter animal sociale , & Ca ha de seer \textbf{ çerca las palauras en quanto son ordenadas a buena conuerssaçion en la uida del omne . } Ca si el omne es naturalmente animalia aconpanable \\\hline
1.2.28 & in quibus communicat cum aliis , \textbf{ dare uirtutem aliquam , } per quam debite conseruetur . & en las quales el omne partiçipa con los otros \textbf{ de dar alguna uirtud } por la qual conueniblemente sepa conuerssar e beuir con los otros . \\\hline
1.2.28 & ait , quod decet Reges et Principes \textbf{ apparere personas reuerendas , } ne contemptibiles habeantur . & e alos prinçipes de paresçer \textbf{ perssonas reuerendas a quien deuen fazer reuerençia } por qua non sean auidos en despreçiamiento . \\\hline
1.2.28 & sic in conuersatione hominum . \textbf{ Aliqua enim familiaritas reputatur regi ad virtutem , } et dicitur ex hoc amicabilis esse : & el qual seria pequano para el sanno assi en essa misma manera en la conuerssacion de los omes \textbf{ alguna familiaridat es contada al Rey a uirtud } e es dicho por ende amigable \\\hline
1.2.29 & nisi non esse apertum , \textbf{ et non ostendere se talem , } qualis est . & por que vn mentires non ser el omne manifiesto \textbf{ nin se mostrar tal qual es . } Et pues que assi es desta uirtud \\\hline
1.2.29 & idest irrisores , et despectores . \textbf{ Oportet ergo dare aliquam virtutem mediam , } per quam moderentur diminuta , & que quiere dezir escarnidores e despreçiadores dessi mismos . \textbf{ Et pues que assi es conuiene de dar alguna uirtud medianera } por la qual sean tenpradas las cosas menguadas \\\hline
1.2.29 & Volens igitur esse verax , \textbf{ non debet de se fingere habere bonitatem } quam non habet , & Pues que assi es el que quiere ser uerdadero \textbf{ non deue dezir nin segnit de ssi bondat } la qual non ha \\\hline
1.2.29 & quod declinare ad minus , \textbf{ et dicere de se minora quam sint , } est opus prudentis . & dize que declinar alo menos \textbf{ e dezir dessi menores cosas } que sean es obra de sabio . \\\hline
1.2.29 & est opus prudentis . \textbf{ Spectat igitur ad veracem nullo modo dicere de se maiora , } quam sint , & que sean es obra de sabio . \textbf{ Pues que assi es parte nesce al uerdadero | non dezir dessi mayores cosas } que sean en el \\\hline
1.2.29 & moderare huiusmodi derisiones , \textbf{ et reprimere iactantias . } Principalius tamen spectat & de tenprar estos tales escarnesçimientos \textbf{ e de repremir los alabamientos . } Enpero mas prinçipalmente parte nesçe ala uerdat \\\hline
1.2.29 & quod affecti ad propria bona , \textbf{ videntur nobis illa esse maiora , } quam sint . & Ca deuemos cuydar que nos \textbf{ por que somos inclinados alos nuestros bienes propreos paresçen nos mayores de quanto son . } Et esta razon tanne el philosofo en la auctoridat \\\hline
1.2.29 & cognoscere seipsum , \textbf{ et sciri quod propria bona } semper aestimantur maiora quam sint . & Ca muy grand pradençia \textbf{ e grant sabiduria es conosçer assi mismo . omne e saber que los sus bienes propreos } sienpreles son vistos mayores que son ¶ \\\hline
1.2.29 & declinandum esse in minus \textbf{ propter onerosas esse superabundantias . } Ostenso quid est veritas & declinaralo menos \textbf{ por la sobrepuiança de carga . } ¶ Visto que cosa es la uerdat \\\hline
1.2.29 & quam sint , \textbf{ videntur esse derisores , et contemptibiles . } Excedentes vero in plus , & e mas viles de quanto son . \textbf{ paresçe que son escarnidores e despreçiadores de ssi mismos . } mas aquellos que sobrepuian en lo mas \\\hline
1.2.29 & vel promittendo aliis maiora quam faciant . \textbf{ Immo tanto magis decet Reges et Principes cauere iactantiam , } quanto plures habent incitantes ipsos ad iactantiam , & nin prometiendo alos otros mayores cosas que faran . \textbf{ Mas por tanto conuiene alos Reyes | e alos prinçipes de escusar } e de foyr el alabança \\\hline
1.2.30 & honestus , et modestus , \textbf{ ordinari habet in bonum finem : } quia est quodammodo necessarius in vita . & si es liberal e honesto \textbf{ e tenprado hase de ordenara buena fin . } por que es en alguna manera necessario ala uida del omne . \\\hline
1.2.30 & quod videtur requies \textbf{ et ludus esse aliquid necessarium in vita . } Sicut ergo sensus corporales , & que paresce que la folgura \textbf{ e el trebeio es vna cosa necessaria en la uida de los omes . } Et pues que assi es \\\hline
1.2.30 & circa ipsos iocos \textbf{ dare virtutem aliquam , } per quam debite nos habeamus ad ludos . & conuiene erca tales iuegos \textbf{ e cerca tales delecta connes deuiegos dar alguna uirtud . } por la qual conueniblemente nos ayamos alos iuegos e alos trabaios . \\\hline
1.2.30 & qualitercunque possent aliquid de illa praeda capere : \textbf{ sic volentes omnino facere risum , } et prouocare alios ad cachinnum , & de aquella prea alguna cosa en essa misma manera \textbf{ los que quieren fazer de todo en todo riso } e enduzir alos otros a escarnio \\\hline
1.2.30 & sic volentes omnino facere risum , \textbf{ et prouocare alios ad cachinnum , } non curant & los que quieren fazer de todo en todo riso \textbf{ e enduzir alos otros a escarnio } non curan en qual se quier manera puedan tomar los dichos o los fechos de los otros \\\hline
1.2.30 & aliquam puerilitatem videtur \textbf{ habere annexam . } Tanto igitur decet Reges et Principes moderate & e honesto pareste \textbf{ que aya en el alguna moçedat ayuntada Et pues que assi es en tanto conuiene alos Reyes } e alos prinçipes de vsar \\\hline
1.2.30 & uti delectationibus ludorum , \textbf{ quanto detestabilius est eos esse pueriles . } Amplius ( ut patet ex habitis ) & tenpradamente delas delecta connes de los iuegos \textbf{ en quanto mas de denostar es aellos de paresçer moços . } Otrossi assi commo paresçe \\\hline
1.2.31 & quae sine aliis virtutibus \textbf{ omnibus perfecte possit haberi . Sic etiam tractatores veritatis senserunt } dicentes virtutes connexas esse . & sin las otras uirtudes todas \textbf{ Et en esta misma manera avn todos los que tractaron delas uirtudes sentieron esto } e dixieron \\\hline
1.2.31 & Philosophum circa finem 6 Ethicor’ \textbf{ manifeste probare virtutes connexas esse . } Sed ut soluat huiusmodi obiectiones , & Et pues que assi es deuedes saber \textbf{ quel philosofo çerca la fin del sexto libro delas ethicas prueua manifiestamente que todas las uirtudes son conexas } e ayuntadas vna con otra . \\\hline
1.2.31 & Sic etiam ex ipsa pueritia \textbf{ videmus aliquos mox inclinari ad opera largitatis , } qui non sunt casti : & que en el tp̃o dela su moçedat \textbf{ luego son inclinados a obras de largueza e de franqueza } los quales non son castos . \\\hline
1.2.31 & non tamen perfecte liberales \textbf{ dici debent : quia ad perfectam virtutem spectat } non solum proponere bonum finem , & acabadamente liberales nin franços . \textbf{ por que pertenesçe ala uirtud acabada } non solamente establesçer fin conuenible \\\hline
1.2.31 & dici debent : quia ad perfectam virtutem spectat \textbf{ non solum proponere bonum finem , } sed etiam debite tendere in illum finem . & por que pertenesçe ala uirtud acabada \textbf{ non solamente establesçer fin conuenible } mas ahun deuen yr conueniblemente a aquella fin . \\\hline
1.2.31 & non solum proponere bonum finem , \textbf{ sed etiam debite tendere in illum finem . } Indigemus ergo virtutibus moralibus , & non solamente establesçer fin conuenible \textbf{ mas ahun deuen yr conueniblemente a aquella fin . } Et pues que assi es auemos meester las uirtudes morales \\\hline
1.2.31 & si vero sit multum infecta phlegmate dulci , \textbf{ videtur participare quandam dulcedinem . } Sic quales sumus & si mucho es llena de flema \textbf{ dulçe paresçen le todas las cosas dulçes . } En essa misma manera quales somos segunt nr̃a uoluntad \\\hline
1.2.31 & et ad bonum opus , \textbf{ sufficiat proponere bonum finem , } nisi per bonam viam eatur in finem illum , & e a buena obra fazer \textbf{ non abasta de entender buena fin } si non fuere a aquella fin \\\hline
1.2.31 & et denidici , \textbf{ si sciant excogitare vias , } per quas consequantur venerea et turpia , & e de moticos \textbf{ si sopieren cuydar las carreras e los caminos | por los quales pueden alcançar las cosas delectables } segunt la carne \\\hline
1.2.31 & habeat omnes virtutes morales . \textbf{ Potest enim quis habere perfecte temperantiam , } et habere prudentiam , & aya todas las uirtudes morales \textbf{ por que puede alguno auer acabadamente la tenpnca } e auer la pradençia \\\hline
1.2.31 & Potest enim quis habere perfecte temperantiam , \textbf{ et habere prudentiam , } ut deseruit temperantiae : & por que puede alguno auer acabadamente la tenpnca \textbf{ e auer la pradençia } e la sabiduria en quanto sirue ala tenprança . \\\hline
1.2.31 & non potest autem aliquis \textbf{ habere aliquam virtutem , } nisi habeat omnes virtutes . & e si non ouiere las otras uirtudes \textbf{ nin puede ninguno acabadamente auer alguna uirtud } si non ouiere todas las uirtudes . \\\hline
1.2.31 & ei displicerent venerea : \textbf{ tamen si posset lucrari pecuniam , } quam intenderet ut finem , & que por auentra asi non plogeres en ael las cosas de luxuria . \textbf{ Enpero si pudiesse ganar el auer | e los dineros la qual cosa entendie } assi commo su fin \\\hline
1.2.31 & et Principes esse quasi semideos , \textbf{ et habere virtutes perfectas : } decet eos habere omnes virtutes , & Por la qual cosa si conuiene alos Reyes e alos prinçipes de ser \textbf{ assi commo medios dioses | e auer las uirtudesacabadas . } Conuiene a ellos de auer todas las uirtudes \\\hline
1.2.32 & contra aliud se tenere . \textbf{ Incontinere ergo est aggredi pugnam , } et in pugna non posse se tenere , & Ca contener se este nerse contra alguna cosa . \textbf{ Et por enerde el non contener se es acometer algua batalla } e en aquella batalla non se poder tener mas fallesçer en el ła¶ \\\hline
1.2.32 & Incontinere ergo est aggredi pugnam , \textbf{ et in pugna non posse se tenere , } sed deficere . & Et por enerde el non contener se es acometer algua batalla \textbf{ e en aquella batalla non se poder tener mas fallesçer en el ła¶ } En el terçero guado de malos son los destenprados . \\\hline
1.2.32 & Tales autem Philosophus assimilat paralyticis , \textbf{ qui eligentes ire in dextram , } propter dissolutionem corporis , et non valentes corpus regere , & assemeia los alos paraliticos \textbf{ los quales quieran yr ala diestra parte . | Enpero por la dissoluçion del } que non puede bien gouernar el cuerpo van ala simestro En essa misma manera los muelles \\\hline
1.2.32 & Sic molles et incontinentes proponunt benefacere , \textbf{ et eligunt ire in dextram : } tamen quia habent potentias animae dissolutas , & e los non continentes proponen de bien fazer \textbf{ e escogen de yr ala diestra . } Empero por que han los poderios del alma dessoluidos \\\hline
1.2.32 & tamen quia habent potentias animae dissolutas , \textbf{ nec habent eas bene regulatas et ordinatas } secundum ordinem rationis , & Empero por que han los poderios del alma dessoluidos \textbf{ e non los han bien reglados | nin bien ordenados } segunt orden de razon \\\hline
1.2.32 & per quam quis debet \textbf{ esse bonus ultra modum humanum , } appellatur a Philosopho heroica & Mas aquella uirtud por la qual alguno es dich̃o bueno \textbf{ sobre la manera comunal de los omes } es llamada del philosofo eroyca \\\hline
1.2.33 & Virtutes autem politicas , \textbf{ esse virtutus acquisitas , } per quas homines bene se habent & Et las uirtudes politicas son uirtudes g̃nadas \textbf{ que ganan los . | omes por buean sobras } por las quales los omes bien se han \\\hline
1.2.33 & et purgati animi dicunt \textbf{ esse virtutes infusas , } per quis quis bene se habet ad diuina . & e las de pragado coraçon \textbf{ son dichas uirtudes enuiadas } que enuia dios en el alma del omne \\\hline
1.2.33 & et tales dicuntur \textbf{ habere virtutes purgatorias . Aliqui vero sunt } quodammodo iam assecuti similitudinem illam : & e tales son dichos auer uirtudes pgatorias . \textbf{ Mas otros algunos son que en algua manera han ya conssigo esta semeiança diuinal } e tales son dichos auer uirtudes de pgado coraçon . \\\hline
1.2.33 & secundum se vera dicant , \textbf{ non tamen videntur accedere ad intentionem eorum , } qui hoc modo de virtutibus sunt locuti . & Mas commo quier que estos digan cosas uerdaderas dessi \textbf{ enpero non paresçe que se allegan ala entençion } de aquellos \\\hline
1.2.33 & quam ponebant , \textbf{ dicebant esse acquisitam . } Sectando ergo Philosophorum viam , & que los philosofos ponian dizian \textbf{ que era ganada de los omes | por vso de buenas obras } ¶Et pues que assi es siguiendo el camino \\\hline
1.2.33 & quod sicut est \textbf{ dare diuersos gradus bonorum , } sic est dare diuersa virtutum genera , & podemos dezir \textbf{ que assi commo contesçe de dar guados de ptidos de bueons } assi conuiene de dar den parti dos linages de uirtudes . \\\hline
1.2.33 & dare diuersos gradus bonorum , \textbf{ sic est dare diuersa virtutum genera , } ita quod secundum quod aliquis est excellentior bonus , & que assi commo contesçe de dar guados de ptidos de bueons \textbf{ assi conuiene de dar den parti dos linages de uirtudes . | Conuiene a saber } que segunt que cada vno es mas altamente bueno ha mas alto grado de uirtudes . \\\hline
1.2.33 & aliquos vero diuinos , \textbf{ sic possumus distinguere quatuor ordines virtutum , } ita quod cuilibet generi bonorum & e algunos tenprados e algunos diuinales . \textbf{ Et en essa misma manera podemos departir quatro ordenes de uirtudes } assi que a cada vn linage de los bueons demos su orden de uirtudes . \\\hline
1.2.33 & Modus autem vincendi eas est auferre , \textbf{ et remouere se de eis . } Ideo continentibus dicuntur & Mas la manera para vençer estas passiones \textbf{ es tirar se e partir se dellas . } Et por ende es dich̃o \\\hline
1.2.33 & habere virtutes purgati animi facientes \textbf{ ipsum obliuisci passiones illas crebras , } quia iam habet animum purgatum et castigatum , & las quales le fazen escaeçer e oluidar las passiones \textbf{ e las delectaçiones desordenadas } por que ya ha el \\\hline
1.2.33 & sed etiam nominare , \textbf{ et audire turpia nefas esse debet . } Bene ergo eis competunt exemplares virtutes , & mas avn delas oyr no obra ¶ \textbf{ Et pues que assi es mucho ꝑ } tenesçen aellos las uirtudes exenplares \\\hline
1.2.34 & Sed visis praehabitis , \textbf{ ostendere quomodo haec sic se habent , } non est difficile . & Mas iustas las cosas dichas de suso non es cosa \textbf{ guaue demostrar en qual manera estas cosas se han assi . } Ca enbolia que es uirtud para conseiar \\\hline
1.2.34 & magis tamen videtur \textbf{ esse dispositio ad virtutem , } quam virtus . & largamente tomado la uirtud . \textbf{ Empero mas paresçe que sea disposiçion } ala uirtud que uirtud . \\\hline
1.3.1 & sicut dicebamus esse duodecim virtutes , \textbf{ sic dicere possumus quod sunt duodecim passiones : } videlicet , amor , odium , desiderium , abominatio , delectatio , tristitia , spes , desperatio , timor , audacia , ira , et mansuetudo . & que eran doze uirtudes \textbf{ assi podemos dezinr que las passiones son doze | conuiene saber amor e mal querençia e desseo . } e aborrençia er delectacion . \\\hline
1.3.1 & sed prout tendimus in ipsum , \textbf{ habet esse in nobis desiderium : } prout vero quietamur in eo , & Mas en quanto aquel bien ha de ser \textbf{ en nos es en nos desseo . } Et en quanto folgamos en aquel bien esta en nos gozo e delectaçion . \\\hline
1.3.1 & passiones concupiscibiles : \textbf{ restat videre , quomodo sumendae sunt passiones irascibiles . } Differunt autem hae passiones ab illis , & en qual manera se toman las passiones del appetito desseador \textbf{ fincanos deuer en qual manera se han de tomar las passiones del appetito enssannador } Mas estas passiones han diferençia e departimiento de aquellas otras . \\\hline
1.3.1 & Cum ergo non possint \textbf{ pluribus modis variari nostri motus et nostrae affectiones , } in uniuerso duodecim erunt passiones : & ¶ Et pues que assi es conmo los nuestros mouimientos del alma \textbf{ et las nr̃as afectiones e passiones | non se puedan departir en mas maneras } que dichas son seran \\\hline
1.3.2 & Accipiendo ergo huiusmodi ordinem secundum combinationem , \textbf{ dicere possumus primas passiones esse , amor , et odium . } In secundo vero gradu sunt desiderium , et abominatio . & tal segunt conbinaçion \textbf{ e ayuntamiento podemos dezir | que las primeras passiones son amor e mal querençia . } Et en el segundo guado son el desseo et el aborrençia . \\\hline
1.3.2 & uel si ipsum habemus , \textbf{ desideramus conseruari in habendo ipsum . } Abominatio uero immediate innititur odio : & O si la ouieremos \textbf{ desseamos la de guardar en auiendo la . } Mas la aborrençia sin ningun medio se ayunta ala mal querençia \\\hline
1.3.2 & praecedit alias passiones : \textbf{ et quia tendere in bonum est } magis coniungi bono & que todas las otraspassiones . \textbf{ Et por que yr al bien nos ayunta mas al bien } que fallesçer del bien \\\hline
1.3.2 & quae deficit ab ipso : \textbf{ sic quia refugere malum habet rationem boni , } ideo timor per quem refugimus malum , & que fallesce del bien \textbf{ En essa misma guisa | por que fuyr del mal ha razon de bien } por ende el temor \\\hline
1.3.2 & Utrum autem secundum aliquem alium modum mansuetudo praecedat iram , \textbf{ inuestigare non est praesentis negocii . } Delectatio autem , & Mas si en alguna manera la mansedunbrees primero que la saña \textbf{ esto non lo auemos de escrudinar aqui ¶ } Otrosi la delectaçion \\\hline
1.3.3 & ideo necessarium est ostendere \textbf{ quomodo nos habere debeamus ad illas . } Oportebat ergo enumerare omnes passiones , & por ende escoła neçesaria de mostrar \textbf{ en qual manera nos deuemos auer a aquellas passiones } Et por ende conuena de contar tondas las passiones \\\hline
1.3.3 & quomodo nos habere debeamus ad illas . \textbf{ Oportebat ergo enumerare omnes passiones , } ut sciremus numerum passionum , & en qual manera nos deuemos auer a aquellas passiones \textbf{ Et por ende conuena de contar tondas las passiones } por que sopiessemos el cuento dellas delas \\\hline
1.3.3 & de quibus determinare debemus . \textbf{ Oportebat etiam ostendere ordinem earum , } ut sciremus quo ordine determinaremus de illis . & quales auemos de determinar e de dezir . \textbf{ ¶ Otrosi conuenia avn demostrar la orden dellas } por que sopiessemos \\\hline
1.3.3 & Sic etiam antiquitus \textbf{ si perspeximus ciuitatem aliquam dominari et tenere monarchiam : } hoc erat , quia ciues pro Republica non dubitabant & Et en essa misma manera avn si cataremos al tp̃o \textbf{ quando alguna çibdat auie señorio e tenie sennorio sobre las otras esto era } por que los çibdadanos non duda una de se poner ala muerte \\\hline
1.3.3 & Inter caetera autem , \textbf{ quae inducere possent alios ad virtutes , } est , ut bonum diuinum & Mas entre todas las uirtudes \textbf{ que pueden los Reyes } e los prinçipes aduzir a uirtudes es que amen prinçipalmente el bien diuianl e el bien comun \\\hline
1.3.3 & tales enim sunt tyranni , \textbf{ volentes explere voluptatem propriam , } et quaerentes excellentiam singularem : & Et tales commo estos son los tiranos \textbf{ que quieren conplir su uoluntad proprea } e demandan grandia singular de su persona \\\hline
1.3.3 & et depraedabat sacra . \textbf{ Viso quomodo Reges et Principes se habere debeant ad amorem , } quia principaliter debent amare bonum diuinum et commune : & e dannaua las eglesias e las casas santas \textbf{ ¶ visto en qual manera los Reyes | e los prinçipes se de una auer al amor } Ca nal e comun de ligero puede paresçer \\\hline
1.3.3 & Viso quomodo Reges et Principes se habere debeant ad amorem , \textbf{ quia principaliter debent amare bonum diuinum et commune : } de facili patere potest , & e los prinçipes se de una auer al amor \textbf{ Ca nal e comun de ligero puede paresçer } en qual manera \\\hline
1.3.3 & potissimum ergo in intentione cuiuslibet \textbf{ esse debet quid amandum . } Ostenso ergo quomodo Reges et Principes & Et pues que assi es la prinçipal entençion de cada vno deue ser \textbf{ que cosa ha de amar } Et por ende mostrado \\\hline
1.3.3 & quodam speciali modo prae aliis debent \textbf{ diligere bonum diuinum et commune , } et quodam speciali modo & que los Reyes et los prinçipes \textbf{ por alguna manera especial sobre todos los otros deuen amar el bien diuinal } e el bien comunal en alguna manera espeçial sobre todos los otros deuen aborresçer todas aquellas cosas \\\hline
1.3.3 & decet Reges \textbf{ et Principes amare Iustitiam , } et odit vitia , & Conuiene alos Reyes \textbf{ e alos prinçipes | assi saber amar iustiçia } e aborresçer todos los pecados \\\hline
1.3.3 & aliter vitia extirpari , \textbf{ nec potest aliter durare commune bonum , } nisi exterminando maleficos homines , & si por auentra a non pueden en otra manera destroyr los males \textbf{ nin puede en otra manera durar el bien comun } si non destruiendo \\\hline
1.3.4 & dicere restat , \textbf{ quomodo Reges et Principes se habere debeant ad desiderium , } et abominationem , & que son las primeras passiones finca de dezir \textbf{ en qual manera los Reyes et los pnçipes se deue auer al desseo } e ala aborrençia \\\hline
1.3.4 & per quam conformatur loco sursum vel deorsum . \textbf{ Secundo est ibi considerare motum , } per quem tendunt in talem locum . & al su logar de yuso o de suso ¶ \textbf{ Lo segundo es hy de penssar el mouimiento } por el qual van a aquel lugar ¶ \\\hline
1.3.4 & et mensuram ipsius finis : \textbf{ ut medicus intendit inducere sanitatem , } quantum potest , & e la mesura de aquella fin \textbf{ assi commo el fisico entiende enduzir | quanto puede sanidat en el enfermo } la mayor e meior que pudiere \\\hline
1.3.4 & intendere \textbf{ et amare bonum regni et commune . } Quare si desiderium debet & Conuiene alos Reyes e alos prinçipes entender e amar \textbf{ prinçipalmente el bien del regno e el bien comun . } Por la qual cosa si el desseo deue tomar mesura del amor \\\hline
1.3.4 & ex amore , \textbf{ principaliter Reges et Principes debent desiderare bonum statum regni : } ut quod qui in regno sunt , & Por la qual cosa si el desseo deue tomar mesura del amor \textbf{ Conuiene que los Reyes e los prinçipes desse en prinçipalmente el buen estado del regno } assi que todos quantos son en el regno \\\hline
1.3.4 & inquantum per ea possunt \textbf{ cohercere malos , } punire iniusta , & e todos los otros dales bienes \textbf{ commo estos deuen los Reyes dessear en tanto en quanto por ellos pueden apremiar los malos } e dar las penas \\\hline
1.3.4 & punire iniusta , \textbf{ et facere talia , } a quibus regni salus dependere videtur . & por las cosas desiguales e malas . \textbf{ Et fazer o tris cosas tales delas quales nasçe e cuelga la salud del regno } ¶ \\\hline
1.3.4 & tanto magis decet Reges et Principes , quanto magis eos decet \textbf{ habere curam de bono regni et communi . } Quae sunt autem illa & quanto mas conuiene a ellos \textbf{ de auer cuydado del regno e del bien comun . } Mas quales cosas son aquellas que guardan el regno en buen estado \\\hline
1.3.4 & quae regnum conseruant in bono statu , \textbf{ et quomodo Rex se debeat habere ad ipsum regnum , } in tertio libro diffusius ostendetur . & Mas quales cosas son aquellas que guardan el regno en buen estado \textbf{ e en qual manera el Rey se deue auer a su regno } en el terçero libro lo mostraremos mas conplidamente \\\hline
1.3.5 & Possumus autem quadrupliciter ostendere , \textbf{ quod deces Reges et Principes decenter se habere circa spem , } et sperare speranda , & Mas nos podemos mostrar en quatro maneras \textbf{ que conuiene alos Reyes | e alos prinçipes de se auer } conueniblemente cerca la esperança \\\hline
1.3.5 & quodcunque bonum possit \textbf{ esse amor vel desiderium : } spes tamen esse non habet , & Ca commo quier que el amor e el desseo puedan ser cerca \textbf{ qual si quier bien . } Enpero la esperança non ha de ser \\\hline
1.3.5 & nisi sibi videatur \textbf{ esse bonum arduum , } et difficile . & si non cerca de bien alto e guaue de alcançar . \textbf{ Ca ninguno non es para si non bien alto } e guaue de alcançar \\\hline
1.3.5 & ad Reges et Principes leges ponere , \textbf{ spectat ad eos sperare bonum . } Rursus quia principale intentum & e alos prinçipes de poner las leyes . \textbf{ Et parte nesçe a ellos de es par algun bien . } Otrosi por que la prinçipal entençion del fazedor delas leyes \\\hline
1.3.5 & non solum spectat ad Reges \textbf{ et Principes tendere in bonum , } sed etiam decet eos tendere in bonum arduum . & Por ende non solamente parte nesçe alos Reyes \textbf{ e alos prinçipes de entender en el bien } Mas avn les conuiene de entender \\\hline
1.3.5 & et Principes tendere in bonum , \textbf{ sed etiam decet eos tendere in bonum arduum . } Amplius quanto maior est communitas , & e alos prinçipes de entender en el bien \textbf{ Mas avn les conuiene de entender | en bien alto e grande e guaue de fazer De mas desto } quanto mayor es la comunidat \\\hline
1.3.5 & Decet ergo Reges et Principes \textbf{ considerare bona non solum } ut sunt ardua , & Et pues que assi es conuiene alos Reyes \textbf{ e alos prinçipes de penssar los bienes | non solamente en quanto son altos e grandes mas avn les conuiene de los penssar } en quanto son bienes \\\hline
1.3.5 & sed ut sunt futura . \textbf{ Congruit etiam eos considerare talia , } ut possibilia . & que han de venir \textbf{ Et avn les conuiene de penssar tales bienes } en quanto pueden ser . \\\hline
1.3.5 & ab aliquibus bonis arduis , \textbf{ videntur mereri indulgentiam , } quia ciuilis potentia , diuitiae , & e si se tiran de algunos bienes altos \textbf{ e grandes meresçen perdon } por que el poderio ciuil e las riquezas \\\hline
1.3.5 & et nobilitas non adminiculantur eis , \textbf{ ut possint prosequi talia bona : } Reges autem et Principes , & non les siruen aellos \textbf{ por que puedan alcançar tales bienes . } Mas los Reyes e los prinçipes \\\hline
1.3.5 & et non credant eis esse possibile \textbf{ prosequi bona ardua } et magno honore digna . & si non creyeren \textbf{ que ellos pueden alcançar tan grandes bienes } e tan dignos de grand honrra . \\\hline
1.3.5 & Quare cum Reges et Principes \textbf{ tendere debeant in bona ardua , } et debeant prouidere bona futura possibilia ipsi regno : & e los prinçipes \textbf{ de una entender alos bienes altos e grandes } e de una proueer los biens \\\hline
1.3.5 & tendere debeant in bona ardua , \textbf{ et debeant prouidere bona futura possibilia ipsi regno : } decet eos esse bene sperantes per magnanimitatem , & de una entender alos bienes altos e grandes \textbf{ e de una proueer los biens | que han de venir e los bienes que pueden acahesçer a su regno . } Por ende conuiene a ellos de serbine esparautes \\\hline
1.3.5 & et debeant prouidere bona futura possibilia ipsi regno : \textbf{ decet eos esse bene sperantes per magnanimitatem , } quia habent omnia & que han de venir e los bienes que pueden acahesçer a su regno . \textbf{ Por ende conuiene a ellos de serbine esparautes | por la magnanimidat } que han en ssi \\\hline
1.3.5 & Sperare enim ultra quam sit sperandum , \textbf{ et aggredi opus ultra vires suas , } videtur ex imprudentia procedere , & que deue omne esparar \textbf{ e acometer alguna obra | mas que la su fuerca demanda paresçe } que esto uiene mas de mengua de sabiduria \\\hline
1.3.5 & inexperti enim non possunt \textbf{ cognoscere arduitatem operis . } Contingit etiam hoc ex immoderatione passionis : & enlos fechͣs non pueden saber las cosas altas e grandes abiertamente \textbf{ por que non sopieron la guaueza | nin la alteza delas obras } Et esto mismo contesçe avn \\\hline
1.3.5 & decet Reges et Principes \textbf{ non aggredi aliquid ultra vires , } et non sperare aliqua non speranda . & Conuiene alos Reyes e alos prin çipes \textbf{ de non acometer ninguna cosa | mas que la su fuerça demanda . } Otrossi les conuiene de non esparar alguas cosas \\\hline
1.3.6 & circa spem et desperationem \textbf{ quae respiciunt futurum bonum ; } sic secundum eandem methodum & e cerca la desesperanca \textbf{ que catan al bien | que ha de venir . } En essa misma manera seg̃t essa misma sciençia los podemos ensseñar \\\hline
1.3.6 & ut consiliatiui reddantur , \textbf{ habere aliquem moderatum timorem . } Secundo hoc idem inuestigare possumus & conuiene alos Reyes e alos prinçipes de auer algun temor tenprado \textbf{ sienpre tomado conseio sobrello ¶ } Lo segundo podemos esso mismo mostrar \\\hline
1.3.6 & diligentius agimus opera , \textbf{ per quae fugere credimus timorem illum . } Ostensum est ergo , & acuciosamente fazemos las obras \textbf{ por las quales queremos foyr de aquel temor } Et por ende mostrado es que los Reyes e los prinçipes deuen auer temor tenprado . \\\hline
1.3.6 & quod decet Reges , \textbf{ et Principes moderatum habere timorem . } Attamen immoderate timere & por las quales queremos foyr de aquel temor \textbf{ Et por ende mostrado es que los Reyes e los prinçipes deuen auer temor tenprado . } Enpero temer destenpradamente en ninguna manera \\\hline
1.3.6 & in seipso contrahitur , \textbf{ et redditur immobilis . Quare si indecens est caput regni siue Regem esse immobilem et contractum , } indecens est ipsum timere timore immoderato . & et pierde el mouimiento . \textbf{ Et por ende si es cosa desconuenible | que la cabeca del regno o el Rey } sea tal que se non mueua \\\hline
1.3.6 & Nerui ergo fiunt frigefacti , \textbf{ et non valentes sustinere membra , } quare accidit ei tremor . & Et por ende quando los neruios son enfriados \textbf{ e non pueden sofrir los mienbros del cuerpo acahesçeles } e viene les luego el tremer ¶ \\\hline
1.3.6 & si Rex sit inoperatiuus , \textbf{ et imperare non valeat propter immoderatum timorem , } toti regno praeiudicium gignitur : & si el rey fuere tal que no nobre \textbf{ e non pueda mandar | por el temor destenprado } e sin razon \\\hline
1.3.6 & Regem immoderato timore timere . \textbf{ Viso quomodo Reges se habere debeant ad timorem , } quia difficilius est reprimere timorem , & e sin razon . \textbf{ ¶ visto en qual manera los Reyes se deuen auer al temor } por que cosa mas guaue es de repremir el temor que tenprar la osadia \\\hline
1.3.6 & Viso quomodo Reges se habere debeant ad timorem , \textbf{ quia difficilius est reprimere timorem , } quam moderare audaciam , & ¶ visto en qual manera los Reyes se deuen auer al temor \textbf{ por que cosa mas guaue es de repremir el temor que tenprar la osadia } assi commo fue dicho de suso \\\hline
1.3.6 & ad audacias decet \textbf{ enim eos non habere audaciam immoderatam , } sed moderatam . & en qual manera se deuen auer los Reyes ala osadia . \textbf{ Ca conuiene aellos de non auer osadia destenprada } e sin razon mas tenprada \\\hline
1.3.6 & quia tunc nihil aggreditur . \textbf{ Moderate ergo se habere ad timorem , } et ad audaciam Regibus et Principibus omnino congruit . & por que estonçe non acometria ninguna cosa¶ Et pues \textbf{ que assi es auersse tenpradamente al temor } e ala osadia es cosa en todo en todo conuenible alos Reyes e alos prinçipes . \\\hline
1.3.7 & est idem quod velle alicui bonum secundum se . \textbf{ Sic odire aliquem est velle malum ei simpliciter , } et absolute . Ira autem non sit : & que querera alguno algun bien segunt si \textbf{ En essa misma manera querer mala alguna cosa esquerer | que luenga algun mal siplemente e suelta mente . } mas la saña non es assi . \\\hline
1.3.7 & nisi credat \textbf{ ipsum fore fecisse vel in se , vel in filios , } vel in amicos , & otrosi non creyere \textbf{ quel fizo algun mal } o en si o en sus fios o en amigos o en algunos otros \\\hline
1.3.7 & cum scimus aliquem esse malum , \textbf{ ut cum scimus aliquem esse furem , } possumus ipsum odire , & Por que luego quando sabemos \textbf{ que alguno es ladron podemos le mal querer } si quiera aya fecho mal a nos o a \\\hline
1.3.7 & per aliquem hominem specialem : \textbf{ licet odire possumus fures uniuersaliter ; } non tamen irascimur , & por algun omne espeçial . \textbf{ Et por ende podemos querer mal generalmente alos ladrones } enpero non nos enssannamos si non a alguna perssona singular . \\\hline
1.3.7 & Vult enim iratus \textbf{ inferre dolorem , et tristitiam : } sed odiens vult & Mas el que quiere mal a alguno tenssea dele enpeçer . \textbf{ Ca el sannudo quiere dar dolor e tsteza } mas el mal quariente quiere fazer danno e enpeçemiento¶ \\\hline
1.3.7 & sed odiens vult \textbf{ inferre damnum , et nocumentum . } Quinta differentia est , & Ca el sannudo quiere dar dolor e tsteza \textbf{ mas el mal quariente quiere fazer danno e enpeçemiento¶ } La quinta diferençia es \\\hline
1.3.7 & magis cauendum est odium quam ira . \textbf{ Immo iram transire in odium secundum Augustinum , } hoc est , trabem facere de festuca . & que dela sanna ante segunt que dize \textbf{ sat̃ agostin la saña passar se en mal querençia } esto es de vna paia fazer ugalagar ¶ \\\hline
1.3.7 & statim enim cum ratio dicit \textbf{ vindictam esse fiendam , } statim vult currere , & ento dize \textbf{ que sea techa vengança } luego quiere correr \\\hline
1.3.7 & cauendum est \textbf{ habere rationem obnubilatam , } et non plene rationi obedire : & Et ponen de si en cada vn omne es de esquiuar \textbf{ que aya la razon | e el entendimiento oscuresçido } e non obedezca conplidamente ala razon a cada vno es de foyr \\\hline
1.3.7 & a Regibus , et Princibus , \textbf{ quia eos maxime decet sequi imperium rationis . } Cauenda est ergo ira inordinata , & e alos prinçipes \textbf{ por que mucho mas conuiene aellos | de segnir el iuyzio dela razon e del entendimiento . } Et pues que assi es paresçe \\\hline
1.3.8 & Dicebatur enim supra , delectationes , \textbf{ et tristitias tenere ultimum gradum } in ordine passionum : & que las delecta connes \textbf{ e las tristezas tienen el } postrimero grado en la orden delas \\\hline
1.3.8 & ut patet per Philos’ 10 Ethicor’ . \textbf{ Eudoxus autem posuit omnem delectationem esse bonam : } quia quod ab omnibus appetitur & assi commo paresçe por el philosofo en el decimo libro delas ethicas \textbf{ Es heudoxio puso | que todas las delectaçiones eran buenas } e fazia esta razon que aquella cola que es desseada de todos \\\hline
1.3.8 & Alii autem econtrario , \textbf{ dicebant omnem delectationem esse fugiendam . } Sed hi omnem delectationem condemnantes , & Mas otros dizian todo el contrario desto \textbf{ diziendo | que toda delectaçion era mala de foyr e de esquiuar } Mas todos estos que despreçia un a todas las delectaçiones \\\hline
1.3.8 & ( ut patet per Philos 4 Metaphy’ ) \textbf{ sic ponens omnem delectationem esse fugiendam , } ponit aliquam delectationem esse prosequendam . & en el quarto libro delas ethicas \textbf{ En essa misma manera el que pone que toda delectaçiones de esquiuar e de foyr pone que alguna delectaciones de segnir . } Ca assy commo la fabla non puede ser negada sinon por la fabla . \\\hline
1.3.8 & sic ponens omnem delectationem esse fugiendam , \textbf{ ponit aliquam delectationem esse prosequendam . } Nam cum loquela non possit negari , & en el quarto libro delas ethicas \textbf{ En essa misma manera el que pone que toda delectaçiones de esquiuar e de foyr pone que alguna delectaciones de segnir . } Ca assy commo la fabla non puede ser negada sinon por la fabla . \\\hline
1.3.8 & Quanto ergo detestabilius est Reges , \textbf{ et Principes eligere vitam pecudum , } tanto detestabilius est eos & Et por ende quanto mas de deno stares alos Reyes \textbf{ e alos prinçipes de escoger uida de bestias . } Tanto mas de deno stares a ellos segnir las delecta connes bestiales \\\hline
1.3.8 & tanto detestabilius est eos \textbf{ sequi bestiales delectationes . } Patet igitur quomodo Reges , & e alos prinçipes de escoger uida de bestias . \textbf{ Tanto mas de deno stares a ellos segnir las delecta connes bestiales } ¶ Et pues que assi es paresçe \\\hline
1.3.8 & et in operibus virtuosis , \textbf{ expeditiori modo et magis perfecte efficere poterunt huiusmodi opera . Nam quanto quis vehementiori modo delectatur } in actibus virtuosis , & e en las obras uirtuosas mas desenbargadamente \textbf{ e mas acabadamente podrian fazer estas tales obras . | Ca quando alguno mas fuertemente se delecta en las obras uirtuosas } tanto \\\hline
1.3.8 & Viso , quomodo Reges , \textbf{ et Principes se habere debeant ad delectationes : } videre restat , & ¶ Visto en qual manera los Reyes \textbf{ e los prinçipes se deuen auer alas delectaçiones } finça deuer en qual maneras \\\hline
1.3.8 & videre restat , \textbf{ quomodo se habere debent ad tristitias . } Tristitia autem nunquam est assumenda , & e los prinçipes se deuen auer alas delectaçiones \textbf{ finça deuer en qual maneras } e de una auer alas tristezas . \\\hline
1.3.8 & quod sint amici : \textbf{ et quia delectabile est habere amicos , } delectamur : & que son nros amigos . \textbf{ Et por que es cosa delectable } auer amigos delectamos nos \\\hline
1.3.8 & Nam per huiusmodi considerationem cognoscimus \textbf{ talia esse modica bona : } ideo eis amissis non dolebimus , & que tales bienes \textbf{ commo estos son muy pequa nons bienes . | Et por ende aquellos perdudos } non nos dolemos dellos sinon por auentura por accidente alguno en \\\hline
1.3.8 & eorum impedimur \textbf{ ab operibus virtuosis . Patet ergo non esse dolendum , } nisi de turpibus , & quanto por perdimiento de aquellos bienes somos enbargados delas obras uirtuosas . \textbf{ Et pues que assi es paresçe | que non nos deuemos doler } sinon delas cosas torpes \\\hline
1.3.8 & vel per cognitionem veritatis . \textbf{ Consueuit etiam ad hoc dari quartum subsidium , } videlicet , remedia corporalia , & o por conosçimiento de uerdat . \textbf{ Mas avn suel en dar otro remedio quarto a esto . } Conuien e saber remedios corporales . \\\hline
1.3.9 & Nam omnes aliae passiones videntur \textbf{ ordinari ad istas ; } ut passiones sumptae respectu boni , & assi commo las passiones \textbf{ que son tomadas } en conparacion de algun bien \\\hline
1.3.9 & ut passiones sumptae respectu boni , \textbf{ ordinari videntur ad spem , } et gaudium , & en conparacion de algun bien \textbf{ son ordenadas ala esperança e al gozo . } Mas las que son tomadas \\\hline
1.3.9 & sumptae autem respectu mali , \textbf{ ordinari videntur ad timorem , et tristitiam . } Nam passio sumpta respectu boni , & en conparaçion de algun mal \textbf{ son ordenadas al temor e ala tristeza . } Ca la passion tomada en conparaçion de algun bien . \\\hline
1.3.9 & vel ut est iam praesens . \textbf{ Secundum hoc ergo sumi habent hae quatuor passiones ; } quia de bono futuro est spes , & que es de venir o en quanto es presente . \textbf{ Et por ende segunt esto se han de tomar estas quetro passions } por que del bien futuro es la esperança \\\hline
1.3.9 & oportet delectationem et tristitiam \textbf{ esse principales passiones respectu concupiscibilis . } Spes autem et timor sunt principales passiones respectu irascibilis . & Conuiene que la delectaçion e la tristeza \textbf{ sean prinçipales passiones | en conparaçion del appetito cobdiçiador . } Mas la esperança e el temor son passiones prinçipales \\\hline
1.3.10 & quod aliqui dicuntur Zelotypi de persona aliqua , \textbf{ si noluerint in ea habere aliquod consortium . } Intensus ergo amor corporalium videtur esse amor priuatus , et reprehensibilis , & Et por ende dende sallio el vso que algunos son dichos çelosos de alguna \textbf{ personasi non quieren auer alguna conparia en ella ¶ } Et pues que assi es el amor grande de las colas corporales parelçe \\\hline
1.3.10 & nec proprie virtutes diligeret , \textbf{ si nollet in eis habere consortium . } Huiusmodi ergo zelus respectu bonorum honorabilium & nin amaria propiamente las \textbf{ uirtudessi non quisiesse auer conpania en aquellas uirtudes . } Et por ende este tal zelo \\\hline
1.3.10 & sicut timentes pallescunt . \textbf{ Nam ex eo , quod aliquis credit se amittere vitam , } quod est bonum interius , & assi commo los temerosos se tornan amariellos . \textbf{ Ca por tanto que alguno cree | que perdera la uida } que es bien de dentro teme mas \\\hline
1.3.10 & timet : \textbf{ sed ex eo , quod credit se amittere gloriam et honorem , } quae sunt bona exteriora , & que es bien de dentro teme mas \textbf{ por que alguno teme perder la honrra | e la eglesia } que son bienes de fuera ha uerguença . \\\hline
1.3.10 & quod quis se credit \textbf{ amittere interiora bona . } Sed cum quis verecundatur , & que perdera los bienes de dentro . \textbf{ Et por ende la sangre core al coraçon | para esforçar los mienbros de dentro . } Mas quando alguno ha uerguença \\\hline
1.3.10 & quia verecundia consurgit \textbf{ ex eo quod quis se credit amittere exteriora bona . } Duplex ergo est timor , & por que la uerguença se leunata de aquello que alguno cree \textbf{ que pierde los bienes de fuera . } Et pues que assi es dos son los temores . \\\hline
1.3.10 & et maxime si credit \textbf{ ipsum indigne pati illud malum , } sic est misericordia . & e mayormente si cree \textbf{ que alguno sufre aquel mala tuerto } assi es miscderia . \\\hline
1.3.10 & Si ergo omnes hae passiones \textbf{ diuersificare habent omnes operationes nostras , } decet nos omnes eas cognoscere ; & Et pues que assi es si todas estas passiones han de partir \textbf{ todasnr̃as obras conuiene a nos delas cognosçer todas . } Et tanto mas esta conuiene alos Reyes \\\hline
1.3.10 & et tanto magis hoc decet Reges et Principes , \textbf{ quanto habere debent operationes maxime excellentes . } Praedictarum passionum & e alos prinçipes \textbf{ quantomas deuen auer las obras mas altas e mas nobles . } lgunas delas passiones sobredichas paresçen ser de loar \\\hline
1.3.11 & licet videantur \textbf{ esse laudabiles passiones , } non tamen simpliciter & Mas la uerguença e la nemessis \textbf{ commo quier que parescan passiones de loar } Enpero non son sinplemente de loar \\\hline
1.3.11 & nisi ex suppositione : \textbf{ nam si contingeret eos operari turpia , } verecundari deberent . Nemesis etiam non multum videtur esse laudabilis , & si non por alguna condiçion . \textbf{ Ca si les contesçiesse a ellos de obrar algunas cosas torpes | e malas deuen se enuergonçar . } Ahun la nemessis non paresçe mucho \\\hline
1.3.11 & Nam mali non possunt \textbf{ possidere maxima bona , } cuiusmodi sunt virtutes : & o delas bien andanças de lons malos \textbf{ porque los malos non pueden auer grandes bienes } assi commo son las uirtudes . \\\hline
1.3.11 & cuiusmodi sunt virtutes : \textbf{ sed forte possidere possunt bona media , } vel bona minima , & assi commo son las uirtudes . \textbf{ Mas por auentra a pueden auer bienes medianeros o bienes muy pequanos } los quales son bienes de fuera . \\\hline
1.4.1 & quia non credunt \textbf{ alios esse malos , } sed ut plurimum credunt & por que non creen \textbf{ que los otros sean malos . } mas por la mayor parte creen \\\hline
1.4.1 & sed ut plurimum credunt \textbf{ omnes homines esse bonos . } Cuius ratio est , & mas por la mayor parte creen \textbf{ que todos los omes son buenos . } Et la razon desto es \\\hline
1.4.1 & quomodo Reges et Principes \textbf{ se debeant habere ad illos . } Nam non quicquid est laudabile in hoc , & puede paresçer en qual manera los Reyes \textbf{ e los prinçipesse de una auera aquellas costunbres . | Ca algunas costunbres son de loar en los mançebos } que non son de loar en los uieios nin en los Rleyes . \\\hline
1.4.1 & est laudabile simpliciter : \textbf{ uidemus enim quod esse furibundum , } est laudabile in cane , & Mas qual si quier cosa \textbf{ que sea de loar en vno | e non en otro o es de loar } por alguna condicion non es de loar sinple mente . \\\hline
1.4.1 & Reges tamen et Principes , \textbf{ quos decet esse quasi semideos , } non solum quod turpia committant , & por que los Reyes e los prinçipes alos quales conuiene de ser \textbf{ assi commo medios dioses } non solamente non les conuiene de fazer cosas torpes \\\hline
1.4.1 & nisi ex suppositione : \textbf{ quia si contingeret eos operari turpia , } uerecundari deberent etiam plus quam alii , & por alguna razon \textbf{ assi commo si contesçiesse | que ellos obrassen algunas cosas } torꝑes deuen auer uerguença ahun \\\hline
1.4.1 & si multitudinem diuitiarum qua pollent , \textbf{ non multiplicarent in debitos et pios usus , } ut supra in tractatu de liberalitate sufficienter tetigimus . & que ellos han \textbf{ e por las quales los preçian en las non espender en vsos o en obras conuenibles e piadosas } assi commo dessuso dixiemos \\\hline
1.4.1 & sunt digniora quam alia . \textbf{ Rursus decet eos esse magnanimos : } quia ( ut dicebatur & e alos prinçipes de ser magranimos \textbf{ e de grand coraçon } Ca assi commo es dicho dessuso \\\hline
1.4.1 & in malam partem , \textbf{ contingeret eos esse tyrannos , } et esse vastatores gentium . & Ca si los fechos de los subditos lienpre le interpetrassen en mala parte contesçrie \textbf{ que los Reyes serien tiranos e destruydores delas gentes } Et otrossi conuiene alos Reyes \\\hline
1.4.2 & Primo , quia non sunt maligni moris . \textbf{ Non enim putant alios esse malos , } sed sua innocentia alios mensurant . & por que non son maliçiosos de uoluntad \textbf{ nin cuydan de los otros | que son malos . } Mas por la su moçençia \\\hline
1.4.2 & quod quis de facili credat ei , \textbf{ quem credit esse bonum , } et quem non cogitat & Et pues que assi es commo natural cosa sea \textbf{ que qual quier omne de ligero cree a aquel que cuyda que es bueno . } et aquel que cuyda que non fabla en enganno \\\hline
1.4.2 & sunt pertinaces in mendatio cogitant enim \textbf{ se esse ingloriosos , } si appareat non sic esse & por que cuydan \textbf{ que ellos ya son en eglesia } si paresçiere alos otros \\\hline
1.4.2 & in Regibus et Principibus , \textbf{ qui debent esse caput et regula aliorum . } Indecens enim est Reges et Principes & e en los prinçipes \textbf{ que deuen lercabesca e regla de todos los otros . } Et por ende cosa desconuenible es alos Reyes de ser segnidores delas passiones \\\hline
1.4.2 & esse passionum insecutores , \textbf{ et venereorum habere concupiscentias vehementes : } quia in eis maxime dominari habet ratio , et intellectus . & Et por ende cosa desconuenible es alos Reyes de ser segnidores delas passiones \textbf{ e de auer cobdiçias afincadas de lux̉ia } por que en ellos mucho mas se deue apoderar la razon e el \\\hline
1.4.2 & Nam cum inconueniens sit \textbf{ regulam esse obliquam , } Reges et Principes , & Porque cosa desconuenible es que la regla sea tuerta \textbf{ e ellos son commo regla . } Et por ende los Reyes e los prinçipes \\\hline
1.4.2 & Tertio indecens est \textbf{ eos esse nimis creditiuos . } Nam cum multos habeant adulatores , & Lo terçero cosa desconuenible es alos Reyes \textbf{ e alos prinçipes de çreer de ligero . } Ca commo ellos ayan muchos lisongeros \\\hline
1.4.2 & Quarto indecens est \textbf{ eos esse iniuriatores et contumeliosos . Nam poenas inferre debent , } non iniuriam , & Lo quarto non es cosa conueniente aellos de ser tortizeros \textbf{ e deno stadores | por que ellos deuen dar penas } e non deuen dar iniurias \\\hline
1.4.2 & Sexto indecens est \textbf{ eos non habere modum } in actionibus suis : & Le seyto cosa desconuenible es alos Reyes \textbf{ non auer manera en las sus obras . } Ca commo todas las sus obras \\\hline
1.4.3 & qui sunt mores senum , \textbf{ et quomodo Reges et Principes se debeant habere ad mores illos . } Senum autem quidam mores sunt laudabiles , & que son de denostar en los vieios \textbf{ e en qual manera los Reyes | e los prinçipes se deuan auer a aquellas costunbres . } Et deuedessaber que delas costunbres de los uieios \\\hline
1.4.3 & non de facili fit eis fides , \textbf{ sed credunt omnes alios esse deceptores . } Ideo dicitur 2 Rhetoricorum , & nin dan fe alos sy dichos . \textbf{ Mas cuydan que todos los otros | sanmint tosos e engannadores . } Por ende dize el philosofo \\\hline
1.4.3 & sequitur quod sit naturaliter formidolosus . \textbf{ Sequitur ergo senes esse naturaliter timidos , } quia deficit in eis naturalis calor , & Et por ende siguese \textbf{ que los uieios son naturalmente temerosos . Ca fallesçe enellos la calentura natural } e han los mienbs naturalmente frios ¶ \\\hline
1.4.3 & quia enim multis annis vixerunt , \textbf{ credibile est eos fuisse passos indigentias multas . } Timentes ergo indigentiam pati , & Ca por que biuieron muchos a nons de creer es \textbf{ que ellos sufrieron muchͣs menguas . } Et por ende temiendo que auran adelante menguas son escassos . \\\hline
1.4.3 & quae quasi communis est ad omnia tacta . \textbf{ Dictum est enim senes esse frigidos . } Frigidus enim omnia constipat , & Mas puede aqui ser fallada vna razon que es a comun a todas estas cosas de suso dichͣs . \textbf{ Ca dicho es de suso | que los uieios son frios } e el frio tondas las cosas estrinne e aprieta e costͥmedolas \\\hline
1.4.3 & reddit ipsa grauiora , \textbf{ et facit ea appetere inferiorem locum . } Videmus enim quod elementa frigida & e apretandolas torna las colas mas pesadas \textbf{ e faz las dessear el logar mas bayo . } Ca nos veemos \\\hline
1.4.3 & quia frigidi non est \textbf{ appetere locum superiorem , sed inferiorem . } Viso qui sunt mores senum vituperabiles ; & nin de ser tenidos en muchͣ \textbf{ por que la cosa fria non ha de querer logar alto mas baxo . } visto quales son las costunbres de los uieios \\\hline
1.4.3 & sed consideratis conditionibus personarum , \textbf{ debent adhibere fidem } iis quae eis dicuntur & Mas penssadas las condiconnes delas personas \textbf{ deuen dar fe a aquellas cosas } que les dizen \\\hline
1.4.3 & faciendo mediocres sumptus : \textbf{ sed etiam congruit eos esse magnificos , } magnifica faciendo . & e alos prinçipes de ser francos faziendo espenssas medianas \textbf{ mas ahun les conuiene de sern magnificos } e granados fazie do grandes cosas ¶ \\\hline
1.4.3 & Non decet tamen eos verecundari : \textbf{ quia indecens est ipsos operari turpia , } ex quibus verecundia consurgit . & e assi non les conuiene aellos de auer uerguenna \textbf{ por que non les conuiene de obrar cosas torꝑes } delas quales se le unata la uerguença . \\\hline
1.4.4 & Positis moribus senum vituperabilibus , \textbf{ restat enumerare mores ipsorum laudabiles . } Videtur autem Philosophus 2 Rhetoricorum , & que non son de loar \textbf{ fincanos de poner las costunbres dellos qson de loar } Mas paresçe que el philosofo en el segundo libro dela rectorica pone quatro costunbres de los uieios \\\hline
1.4.4 & Videtur autem Philosophus 2 Rhetoricorum , \textbf{ circa senes tangere quatuor mores , } qui possunt esse laudabiles . & fincanos de poner las costunbres dellos qson de loar \textbf{ Mas paresçe que el philosofo en el segundo libro dela rectorica pone quatro costunbres de los uieios } que pueden ser de loar ¶ \\\hline
1.4.4 & viuere ratione quam passione , \textbf{ decet eos habere concupiscentias temperatas : } quia ( ut supra dicebatur ) & por razon que por passion dela carne \textbf{ conuiene les aellos | de auer las cobdiçias tenpdas . } Ca assy commo es dicho desuso las cobdiçias \\\hline
1.4.4 & ne per hoc iudicentur leues et indiscreti . \textbf{ Quarto in suis actionibus debent habere moderationem et temperamentum : } quia ( ut dictum est ) & e de poco saber ¶ lo quarto los Reyes \textbf{ e los prinçipes deuen auer | en las sus obras mesura e tenpramiento } por que assi commo dicho es ellos \\\hline
1.4.4 & possunt tamen contra illam pronitatem facere \textbf{ consequi laudabiles mores . } Sic et illi & Empero pueden fazer contra aquella disposiconn \textbf{ e inclina conn natural } e segnir bueans costunbrs e de loar . \\\hline
1.4.4 & qui aliis dominantur , \textbf{ sequi mores laudabiles } secundum dictamen , & e por entendemiento Conuiene alos Reyes e alos prinçipes \textbf{ que son senores de los otros segnir costunbres bueans e de loar } segunt \\\hline
1.4.5 & dicere possumus , \textbf{ ipsorum nobilium esse quatuor mores laudabiles . } Primo enim sunt magnanimi . & quanto parte nesçe alo prasente podemos dezir \textbf{ que quatro son las costunbres bueans | e de lapña esboar delos nobles omes ¶ } que son de grand coraçon ¶ \\\hline
1.4.5 & si ab antiquo affluebat diuitiis . \textbf{ Cum ergo semper sit dare initium , } in quo genitores alicuius ditari inceperunt : & si de antigo tienpo abondo en riquezas . \textbf{ Et pues que assi es comm sienpre ayamos de dar comienço } en que los padres de alguons comne caron de se enrriqueçer \\\hline
1.4.5 & quod magnanimos et magnificos decet \textbf{ esse nobiles et gloriosos . } Vult enim ibidem , & e de grandes coraçones e magnificos \textbf{ e de grandes fechos e głiosos e much̃ honrrados } por que dize alli el philosofo \\\hline
1.4.5 & rationabile est , \textbf{ eos habere corpus bene dispositum , } et bene complexionatum . & con razon es \textbf{ que ellos ayan los cuerpos bien ordenados e bien conplissionados . } Et pues que assi es conmolos bien conplissionados \\\hline
1.4.5 & ex diligenti consideratione suorum agibilium \textbf{ esse dociles , et industres . } Ex hoc autem apparere potest , & en todas sus obras \textbf{ que deuen fazer . } Et desto puede paresçer \\\hline
1.4.5 & diligenter considerent quid agendum . \textbf{ Quarto nobiles contingit esse politicos , et affabiles . } Nam quia ut plurimum in curiis nobilium consueuit & e si temieren de fazer cosas reprehenssibles e si cuydaren sotilmente todo lo que han de fazer \textbf{ ¶La quatta condicion de los nobles es | que son corteses e amigables . } Ca porque en la mayor parte en las cortes delos nobles \\\hline
1.4.5 & Non enim debemus \textbf{ appetere ipsos honores in se , } quia hoc faciunt elati et superbi : & paresçe de ser malas costunbres \textbf{ por que non deuemos dessear las honrras en lli . } Ca esto fazen los orgullolos et los sob̃uios . \\\hline
1.4.5 & quia hoc faciunt elati et superbi : \textbf{ sed debemus appetere opera honore digna , } quod faciunt virtuosi et magnanimi . & Ca esto fazen los orgullolos et los sob̃uios . \textbf{ Mas deuemos dessear las obras | que son dignas de honrra } la qual cosa fazen los uirtuosos \\\hline
1.4.5 & nisi sint boni et virtuosi , \textbf{ decet eos sequi bonos mores nobilium , } ut sint magnanimi et magnifici , prudentes et affabiles : & e uirtuosos conuiene les aellos \textbf{ de segnir las bueans costunbres de los nobles } por que sean de grand coraçon e de grand fazienda \\\hline
1.4.5 & ut sint magnanimi et magnifici , prudentes et affabiles : \textbf{ et fugere malos mores , } ut non sint elati , & e muy sabios e bien razonados . \textbf{ Otrossi les conuiene de foyr malas costunbres } por que non senas obrauios e deipreçiadores de los otros . \\\hline
1.4.6 & quia habendo diuitias aliquas , \textbf{ credunt se acquisiuisse omnia bona . } Unde et 2 Rhetoricorum dicitur , & por que auyendo las riquezas \textbf{ cuydan | que han todos los bienes . } Onde en el segundo libro de la rectorica dize \\\hline
1.4.6 & Habentes ergo numismata , \textbf{ aestimant se habere omnia bona , } eo quod reputent & Et por ende los que han las monedas cuydan \textbf{ que han todos los bienes } por que cuydan que los dineros son dignidat e preçio de todas las otras cosas . \\\hline
1.4.6 & eo quod reputent \textbf{ pecuniam esse dignitatem , } et pretium omnium aliorum ; & que han todos los bienes \textbf{ por que cuydan que los dineros son dignidat e preçio de todas las otras cosas . } Por la qual cosa en los susco raçons se ensoƀueçen \\\hline
1.4.6 & dispicientes alios , \textbf{ et credentes se esse super eos , } eo quod videant illos indigere bonis eorum . & despreçiando alos otros \textbf{ e cuydando que son mayores que ellos } por que veen \\\hline
1.4.6 & Utrum esset melius , \textbf{ fieri diuitem , } quam sapientem . & Et ella respondio \textbf{ que ma veye yr los sabios alas puertas de los ricos } que los ricos alas puertas de los sabios . \\\hline
1.4.6 & et quod decet Reges , \textbf{ et Principes fugere tales mores : } videre restat , & e que conuiene alos Reyes e alos \textbf{ prinçipesarredrar se de tales costunbrs finca de ueer } quales son las bueans costunbres de los ricos . \\\hline
1.4.6 & Diuitiae enim , \textbf{ quia videntur esse bona fortunae , } non videtur sufficere industria humana & e que parte nesçen a dios . \textbf{ Ca las riquezas | por que paresçen biens de auentura } non paresçe \\\hline
1.4.6 & per ordinationem diuinam \textbf{ habere huiusmodi bona . } Hoc autem dictum Philosophicum & e por los ordenamientos de dios \textbf{ han estos bienes tenporales e estas riquezas . } Et por ende este dicho tan sotil del philosofo \\\hline
1.4.6 & tanto magis decet Reges , et Principes , \textbf{ quanto summo Deo iudici de pluribus debent reddere rationem . } Nobilitas , diuitiae , et ciuilis potentia , & e alos prinçipes \textbf{ quanto ellos han de dar cuenta de mas cosas a dios | que asuiez de todas las cosas . } ra nobleza e la riqueza e el poderio çiuil \\\hline
1.4.7 & sed sunt nuper ditati . \textbf{ Differunt ergo esse nobilem , } et esse diuitem . & mas fezieron sericos del otro dia aca . \textbf{ Pues que assi es diferençia ay } entre ser noble e ser rico \\\hline
1.4.7 & et esse diuitem . \textbf{ Differt etiam esse nobilem , } et esse diuitem , & entre ser noble e ser rico \textbf{ e ahun disferençia ay entre ler noble e rico } e entre ser poderoso . \\\hline
1.4.7 & Differt etiam esse nobilem , \textbf{ et esse diuitem , } ab esse potentem . & entre ser noble e ser rico \textbf{ e ahun disferençia ay entre ler noble e rico } e entre ser poderoso . \\\hline
1.4.7 & verecundatur omnino declinare a medio , \textbf{ et non agere opera virtuosa . } Ipse ergo principatus & en que esta la uirtud \textbf{ e de non fazer obras uirtuosas } e pues que assi es el prinçipado \\\hline
1.4.7 & Quare contingit potentes \textbf{ magis esse temperatos , } quam diuites . & por la qual cosa contesçe \textbf{ que los poderosos son mas tenprados que los ricos ¶ } Lo terçero los poderosos son menos peleadores que los ricos . \\\hline
1.4.7 & Non enim curabunt \textbf{ facere paruam offensam , } sed vel in nullo damnificabunt alios , & por que non curan de fazer \textbf{ pequano tuerto nin pequeno danno . } Mas o en ninguna cosa non faran danno alos otros o les faran grand danno . \\\hline
1.4.7 & et dulcedine scientiarum , \textbf{ statim percipit ea esse maiora bona , } quam crederent . & por que aquel que comiença agostar dela bondat delas uirtudes e dela dulçedunbre delas sçiençias \textbf{ luego entiende | que aquellas cosas son mayores } e meiores bienes que el cuydaua . \\\hline
1.4.7 & et magnam pronitatem habent , \textbf{ ut sequantur praedictos mores . } Iuuenes ergo et senes non indignentur , & e han grand disposiçion \textbf{ para segnir las costunbres sobredichͣs . } Et por ende non se deuen enssonnar los mançebos \\\hline
1.4.7 & quin possint omnes malos mores vitare , \textbf{ et sequi ordinem rationis . } Sic etiam nec nobiles , & que non puedan ellos esquiuar todas estas malas costunbres \textbf{ e segnir orden de razon | e de entendemiento } e auer las bueans . \\\hline
1.4.7 & quia non oportet omnes esse tales , \textbf{ sed sufficit reperiri illud in pluribus : } pronitatem enim quandam , et non necessitatem , & ca non conuiene que todos seantales . \textbf{ Mas abasta que aquellas costunbres sean falladas en muchos por que non | entendiemosponer en ellos } por estas costunbres neçessidat ninguna mas alguna disposicion e inclinaçion para auer las \\\hline
1.4.7 & decet omnes homines \textbf{ sequi mores laudabiles , } et fugere vituperabiles . & Conuiene a todos los omes \textbf{ de segnir las costunbres | que son de loar } e desse arredrar delas \\\hline
2.1.1 & quod Reges debite seipsos regant , \textbf{ nisi regere sciant domum , ciuitatem , et regnum . } In hoc ergo secundo libro determinabitur de regimine domus . & e los prinçipes gouiernen assi mismos conueniblemente \textbf{ sinon sopieren gouernar su casa | e la çibdat e el regno . } Et por ende en este segundo libro \\\hline
2.1.1 & ex quibus quadruplici via venari possumus , \textbf{ ipsum esse communicatiuum et sociale . } Prima via sumitur ex victu , & que el omne es natra \textbf{ alnen te comunal a todos | e conpannero ¶ } La primera manera se toma \\\hline
2.1.1 & Nam quia non habent complexionem ita puram , \textbf{ et ita redactam ad medium , } ut homo , non indigent cibo ita deputato , & Ca por que non han la conplission tan pura las oinanlias \textbf{ nin assi trayda atenpmiento } medianere commo el omne non han \\\hline
2.1.1 & quia magis habet complexionem puram \textbf{ et redactam ad medium , } indiget alimento praeparato et depurato . & Mas el omne por que ha conplission mas apurada \textbf{ e mas aducha atenpramiento medianero } por esso ha mester vianda mas apareiada e mas apurada \\\hline
2.1.1 & quasi naturale indumentum , \textbf{ habere videntur lanam , vel pennas . } Homini autem non sufficienter prouidet natura in vestitu : & que han natural uestido \textbf{ assi commo las bestias han la lana e las aues las pennolas . } Mas la natura non prouee al omne tan conplidamente en uestidura \\\hline
2.1.1 & quia non per aliam viam possunt \textbf{ euadere mortis pericula } nisi per corporis agilitatem et per fugam , & Ca las corcas e las liebres saben \textbf{ que no pueden escapar por otro camino los periglos dela muerte } si non por lignieza del su cuerpo e por foyr . \\\hline
2.1.1 & fabricare valemus . \textbf{ Quare si naturale est homini desiderare conseruationem vitae , } cum homo solitarius non sufficiat sibi & para nuestro defendemiento . \textbf{ Par la qual cosa si natural cosa es al omne de dessear conseruaçion e guarda de su uida } commo el omne \\\hline
2.1.1 & ut aranea ex instinctu naturae \textbf{ debitam telam faceret , } si nunquam vidisset & por inclinaçion natra al \textbf{ faze su tela conuenible } avn que nunca aya visto otras arannas texer en essa misma manera \\\hline
2.1.2 & si sit recte ordinata , \textbf{ continere debet expedientia in tota vita , } ut in tertio libro plenius ostendetur . & Et pues que assi es si la çibdat es derechamente ordenada \textbf{ deue auer en ssi todas las cosas | que son neçessarias . } a toda la uida humanal \\\hline
2.1.2 & In praecedenti ergo capitulo determinauimus de societate humana , \textbf{ ostendentes eam esse necessariam ad vitam nostram : } quia per hoc manifeste ostenditur & Et pues que assi es en el capitulo sobredicho auemos determinado dela conpannia humanal \textbf{ mostrando que es neçessario a lanr̃auida } por que por esta razon se muestra manifiestamente \\\hline
2.1.2 & quia per hoc manifeste ostenditur \textbf{ necessariam esse communitatem domesticam : } cum omnis alia communitas communitatem illam praesupponat . & por que por esta razon se muestra manifiestamente \textbf{ que la comunidat dela casa es neçessaria } por que todas las comuni dades ençierran en ssi \\\hline
2.1.2 & si dicta Politica diligenter consideremus , \textbf{ apparebit quadruplicem esse communitatem ; } videlicet , domus , vici , ciuitatis , et regni . & en las politicas paresçra \textbf{ que son quatro las comuindades Conuiene a saber . | Comuidat dela casa } Et comunidat de uarrio . \\\hline
2.1.2 & et non valentibus habitare in una domo , \textbf{ compulsi sunt facere domos plures , } et constituere vicum . & por que non podien todos morar en vna casa \textbf{ por fuerça ouieron de fazer muchas casas } e fizieron vn uarrio . \\\hline
2.1.2 & Reges ergo et Principes , \textbf{ quorum officium est dirigere alios } ad bene viuere , ignorare non debent , & Et pues que assi es los Reyes e los prinçipes \textbf{ cuyo ofiçio es de gerar | e enderescar los otros } a bien beuir \\\hline
2.1.3 & et typo ostendere , \textbf{ quod decet homines habere habitationes decentes } secundum suam possibilem facultatem ; & por que ael parte nesçe generalmente demostrar \textbf{ por figera e por exienplo que conuiene alos omes de auer conueibles moradas } segunt el su poder e la su riqueza . \\\hline
2.1.3 & Agens enim primo et principaliter intendit finem . \textbf{ Verum quia non potest habere finem , } nisi per ea , & assi commo el carpento el arca \textbf{ que faz de los maderos ¶as | por que non puede alcançar la fin } sin aquellas cosas \\\hline
2.1.3 & communitatem domus \textbf{ esse priorem aliis tempore et generatione ; } esse tamen posteriorem & Bien dicho es que la comuidat dela casa es primero \textbf{ por generaçion | e por tienpo que las otras . } Enpero es postrima \\\hline
2.1.3 & esse priorem aliis tempore et generatione ; \textbf{ esse tamen posteriorem } illis perfectione et complemento . & e por tienpo que las otras . \textbf{ Enpero es postrima } por perfeccion e por conplimiento . \\\hline
2.1.3 & spectat enim non solum ad Principem siue ad legislatorem , \textbf{ sed etiam ad quemlibet ciuem prius intendere bonum ciuitatis et regni , } quam etiam bonum propriae domus , & Por ende non solamente pertenesçe al prinçipe o al fazedor delas leyes \textbf{ mas ahun a cada vno de los çibdadanos enparar mientes | primero abalbien dela çibdat e del regno } que al bien dela su casa proprea \\\hline
2.1.3 & scire gubernare domestica , \textbf{ et regere familiam siue domum , } non solum inquantum esse debent viri sociales et politici , & de saber gouernar las cosas dela casa \textbf{ e gouernar la conpanna dela casa } non solamente en quanto deuen ser uarones aconpannables e bien acostunbrados . \\\hline
2.1.3 & non solum inquantum esse debent viri sociales et politici , \textbf{ quia sic scire gubernationem domus pertinet ad omnes ciues : } sed spectat specialiter & non solamente en quanto deuen ser uarones aconpannables e bien acostunbrados . \textbf{ Ca en esta manera saber gouernar la casa | par tenesçe a todos los çibradadanos . } Mas espeçialmente esto parte nesçe a los Reyes e alos prinçipes . \\\hline
2.1.3 & ad Reges et Principes , \textbf{ quia sicut regnum vel ciuitas praesupponunt esse domum , } sic regimen regni et ciuitatis & Mas espeçialmente esto parte nesçe a los Reyes e alos prinçipes . \textbf{ Ca assy commo el regno | e la çibdat } ante ponen la comunidat dela casa \\\hline
2.1.3 & Quare si specialiter spectat ad Reges et Principes \textbf{ regere regnum et ciuitates , } specialiter spectat ad eos , & Por la qual cosasi espeçialmente pertenesçe alos Reyes \textbf{ e alos prinçipes de gouernar el regno } e la çibdat espeçialmente pertenesçe a ellos \\\hline
2.1.3 & ad Reges et Principes spectat \textbf{ intendere bonum regni et principatus : } attamen huiusmodi bonum intendere & e alos prinçipes de entender \textbf{ e cuydar çerca el bien del regno | e del prinçipado } empero pertenesçe entender este bien \\\hline
2.1.3 & tum etiam quia spectat ad omnes ciues \textbf{ intendere bonum regni : } spectat ad unumquemque ciuem scire regere domum suam , & çibdadano de \textbf{ entenderal bien del regno . } Por ende pertenesçe a cada vn \\\hline
2.1.3 & intendere bonum regni : \textbf{ spectat ad unumquemque ciuem scire regere domum suam , } non solum inquantum huiusmodi regimen est bonum proprium , & entenderal bien del regno . \textbf{ Por ende pertenesçe a cada vn | çibdadano saber gouernar su casa } non solamente en quanto este gouernamiento es bien propreo suyo \\\hline
2.1.4 & Sciendum ergo , \textbf{ Philosophum 1 Politicorum sic describere communitatem domus : } videlicet , quod domus est communitas secundum naturam , & Pues que assi es deuedes saber \textbf{ que el philosofo en el primero libro delas | politicasasse declara } e difine la comunidat dela casa \\\hline
2.1.4 & non sufficiebat communitas domestica , \textbf{ sed oportuit dare communitatem vici , } ita quod cum vicus constet & non cunplie la comunidat de vna casa \textbf{ mas conuiene de dar comunidat de varrio . } Por que commo el uarrio sea fech̃ de muchas casas \\\hline
2.1.4 & oportuit \textbf{ dare communitatem ciuitatis . } Communitas ergo ciuitatis esse videtur & conuiene de dar comunidat ala çibdat \textbf{ sobre la comunidat deluarrio . } Et por ende \\\hline
2.1.4 & ad expugnandam ciuitatem aliam \textbf{ confoederare se alteri ciuitati ; } quare cum confoederatio ciuitatum utilis sit & para que pueda lidiar con otra çibdat \textbf{ que aya conpanna e amistança con otra çibdat } que la pueda ayudar . \\\hline
2.1.4 & sed in domo oportet \textbf{ dare plures communitates : } quod sine pluralitate personarum & non solamente la casa es vna comiundat \textbf{ mas en la casa conuiene de dar muchͣs comunidades } la qual cosa non puede ser sin muchͣs perssonas . \\\hline
2.1.4 & tam necessaria in vita ciuili , \textbf{ spectat ad quemlibet ciuem scire debite regere suam domum : } tanto tamen magis hoc spectat ad Reges et Principes , & sea tan neçessaria \textbf{ enla uida çiuil pertenesçe a cada vn çibdada | no de laber gouernar conueniblemente lucasa . } Et tanto mas esto parte nesçe alos Reyes \\\hline
2.1.4 & quanto ex incuria propriae domus magis potest \textbf{ insurgere praeiudicium ciuitati et regno , } quam ex incuria aliorum . & quanto por mal gouernamiento de su casa proprea \textbf{ mas se puede leuna tar piuyzio ala çibdat | e al regno } por mal gouernamiento de los otros . \\\hline
2.1.5 & quod Damascenus ait , \textbf{ generationem esse quid naturale , } et esse opus naturae . & a qual lo que dize damasçeno \textbf{ que la generaçion es cosa natural } e es obra de natura . \\\hline
2.1.5 & generationem esse quid naturale , \textbf{ et esse opus naturae . } Rursus rerum conseruatio , & que la generaçion es cosa natural \textbf{ e es obra de natura . } Otrossy la conseruaçion \\\hline
2.1.5 & Hoc ergo modo hae duae communitates faciunt domum esse quid naturale : \textbf{ quia communitas viri et uxoris ordinatur ad generationem , } communitas vero domini & fazen ser la casa cosa natural . \textbf{ Ca la comunidat del uaron e dela mugnies ordenada ala generacion . } Mas la comunidat del sennor e del sieruo es ordenada ala \\\hline
2.1.5 & quia sine eis domus congrue esse non valet . \textbf{ Quod autem communicatio viri et uxoris sit propter generationem , } videre non habet dubium : & por que sin ellas non puede ser la cosa conueniblemente . \textbf{ Mas que la comunidat del uaron | e dela muger sea } para la generaçion non adubda ninguna \\\hline
2.1.5 & requiritur \textbf{ communitas viri et uxoris propter generationem , } sic requiritur ibi communitas domini & es men ester la comunidat del uaron \textbf{ e dela mugni | para la generaçion } en essa misma manera es \\\hline
2.1.5 & et alia conseruationi , \textbf{ dicuntur facere primam domum : } quia sine eis domus congrue & e la otra ala con leruaçion \textbf{ fazen la primera cala } por que sin ellas la primera casa non puede estar conueniblemente . \\\hline
2.1.5 & Nam ut ait , pauperes homines , \textbf{ qui non possunt habere seruum } et ministrum rationalem , & Ca assi commo alli dize los omes pobres \textbf{ non pue den auer sienpre sieruos } e aministdores que los siruna . \\\hline
2.1.5 & qui non solum non possunt \textbf{ habere ministrum rationalem , } sed etiam habere non possunt ministrum animatum : & Et avn algunos son tan pobres \textbf{ que non solamente pueden auer seruidor razonable } mas avn non pueden auer huidor con alma . \\\hline
2.1.5 & habere ministrum rationalem , \textbf{ sed etiam habere non possunt ministrum animatum : } sed loco eius habent aliquid inanimatum ; & que non solamente pueden auer seruidor razonable \textbf{ mas avn non pueden auer huidor con alma . | assi commo bestia anas en logar de tal seruidor } ponen alguna cosa \\\hline
2.1.5 & vel habent aliquid aliud loco bonis . \textbf{ Decet autem omnes ciues cognoscere partes , } ex quibus componitur domus : & o alguna otra cosa en logar de bueye . \textbf{ Et conuiene a todos los çibdadanos de conosçer } e saber las partes de que se conpone la casa \\\hline
2.1.5 & Quia ergo cognitio partium domus , \textbf{ et scire quot genera personarum , } et quot communitates requiruntur ad domum : & Ca saber las partes dela casa \textbf{ e saƀ quantos son los linages delas perssonas | e quintas son las comuindades } que son menester ala casa \\\hline
2.1.6 & videlicet , viri et uxoris , domini et serui , \textbf{ facere domum primam . } Sed tamen , & sobredicho dos comuni dades \textbf{ fazen la primera casa } o nuene de saber et uaron \\\hline
2.1.6 & si domus debet esse perfecta , \textbf{ oportet ibi dare communitatem tertiam , } scilicet patris et filii . & Emposi la casa fuere acabada conuiene de dar \textbf{ y la terçera comunidat } que es de padre e de fijo . \\\hline
2.1.6 & potest sibi simile producere , \textbf{ sed oportet prius ipsam esse perfectam . } statim enim , & luego que es fecha fazer otra semeiante \textbf{ assi mas conuiene que ella primeramente sea acabada } enssi \\\hline
2.1.6 & sed oportet prius ipsum esse perfectum : \textbf{ producere ergo sibi similem , } non est de ratione rei naturalis & luego otro su semeinante \textbf{ mas conuiene que primeramente el sea acabado . } Et pues que assi es engendrar su semeiante non pertenesçe a cosa natural tomada en qual quier manera mas pertenesçe a cosa natural en quanto ella es acabada . \\\hline
2.1.6 & et ea quae videmus in domo , \textbf{ reducere volumus in naturales causas , } dicemus duas communitates , & e las cosas que veemos en la casa queremos traer \textbf{ a razones naturales diremos que las dos comuidades } que son de varon e de muger e de señor e de sieruo \\\hline
2.1.6 & cum prius dixisset \textbf{ communitatem viri et vxoris , domini et serui facere communitatem primam : } postea in sequenti capitulo praedicti libri ait , & commo ouiesse dicho primeramente \textbf{ que la comiundat del omne e dela muger e del sennor e del sieruo fazen la primera casa . } Despues en el segundo capitulo desse dicho libro \\\hline
2.1.6 & Patet ergo quod ad hoc quod domus habeat esse perfectum , \textbf{ oportet ibi esse tres communitates : } unam viri et uxoris , aliam domini et serui , & Et por ende paresçe que para que la casa sea acabada \textbf{ que conuiene que sean enlla tres comuundades . } ¶ La vna del uaron e dela muger ¶ \\\hline
2.1.6 & oportet ibi esse tres communitates : \textbf{ unam viri et uxoris , aliam domini et serui , } tertiam patris et filii . & que conuiene que sean enlla tres comuundades . \textbf{ ¶ La vna del uaron e dela muger ¶ | La otra del senor e del sieruo ¶ } La terçera del padre e del fij̉o . \\\hline
2.1.6 & in domo perfecta esse tria regimina . \textbf{ Nam nunquam est dare communitatem } aliquam bene ordinatam , & deuen ser tres gouernamientos . \textbf{ Ca nunca podemos dar comunindat ninguna bien ordenada } si non fuere . \\\hline
2.1.6 & quod ibi oportet \textbf{ esse quatuor genera personarum . } Videretur tamen forte alicui & que conuiene que sean y . \textbf{ quatro linages de perssonas } Empero podrie paresçer a alguno por auentura que deuen y ser seys linages de perssonas \\\hline
2.1.6 & Nam cum in domo perfecta sint tria regimina , \textbf{ oportet hunc librum tres habere partes ; } in quarum prima tractetur primo de regimine coniugali : & Ca commo en la casa acabada sean tres gouernamientos . \textbf{ Ca conuiene que este libro sea partido en tres partes . } ¶ En la primera delas quales tractaremos del gouernamiento mater moianl . \\\hline
2.1.6 & magnum adminiculum habebunt , \textbf{ ut bene sciant regere regnum , et ciuitatem . } Dicebatur in praecedenti capitulo , & por que estos gouernamientos catados con grant acuçia auran grant ayuda \textbf{ por que sepan bien gouernar sus regnos et sus çibdades } icho es en el capitulo sobredicho \\\hline
2.1.7 & primum oportet \textbf{ congregare marem , et foeminam . } Est autem hic ordo rationabilis . & en la comunidat dela casa \textbf{ primeramente conuiene de ayuntar el uaron con la mugni } e esta orden es muy con razon . \\\hline
2.1.7 & Deinde ostendemus , \textbf{ qualiter viri suas uxores regere debeant } et ad quas virtutes , & e mayormente los Reyes e los prinçipes . \textbf{ Despues mostraremosen qual manera los uarones deuen gouernar sus mugers } e a quales uirtudes \\\hline
2.1.7 & primo declarandum occurrit , \textbf{ coniugium esse aliquid secundum naturam , } et quod homo naturaliter est animal coniugale . & Mas en demostrando quales el ayuntamiento del uaron e dela muger \textbf{ pmeramente nos conuiene de declarar en qual manera el mater moino es alguna cosa segunt natura . } Et que el omne naturalmente \\\hline
2.1.7 & ostendere \textbf{ qualis amicitia sit viri ad uxorem , } probat amicitiam illam esse secundum naturam : & quariendo mostrar \textbf{ qual es el amistança del uaron } a la muger prueua \\\hline
2.1.7 & qualis amicitia sit viri ad uxorem , \textbf{ probat amicitiam illam esse secundum naturam : } adducens triplicem rationem & qual es el amistança del uaron \textbf{ a la muger prueua | que aquella amistança es segunt natura . } Et aduze para esto tres razones \\\hline
2.1.7 & Probabatur enim in primo capitulo huius secundi libri , \textbf{ hominem esse naturaliter animal sociale et communicatiuum . } Communitas autem in vita humana & que el omne es naturalmente \textbf{ aina l aconpannable e comun incatiuo | que quiere dezir ꝑtiçipante con otro } Mas la comunidat en la uida humanal \\\hline
2.1.7 & Hanc autem rationem tangit Philosophus 1 Politicorum , et 8 Ethicorum , \textbf{ ubi probat coniugium competere homini secundum naturam , } quia naturale est homini , & e en el octauo delas ethicas do prueua \textbf{ que el casamiento conuiene alos omes segunt natura } por que natural cola es al omne \\\hline
2.1.7 & et omnibus animalibus , \textbf{ habere naturalem impetum } ad producendum sibi simile . & e atondas las aianlias \textbf{ auer natural inclinaçion e appetito } para engendrar cosa semeiable \\\hline
2.1.7 & Quare si naturale est homini , \textbf{ habere impetum ad sufficientiam vitae : } naturale est ei , & ø \\\hline
2.1.8 & quod decet coniugia indiuisibilia esse . \textbf{ Ad quod ostendendum adducere possumus duas vias , } quas philosophi tetigerunt . & que los casamientos sean sin departimiento ninguno \textbf{ e que non le puedan partir . | Et para esto mostrar podemos dezir dos razones } las quales posieron los philosofos \\\hline
2.1.8 & si ab amicitia eius discedat : \textbf{ si inter virum et uxorem debitam fidem , } vel fidelem amicitiam saluare volumus , & por amistança \textbf{ si se departe della . Si entre el marido e la mugier queremos saluar fe conuenible } e amistan ça leal \\\hline
2.1.8 & cum non fit naturalis amicitia \textbf{ inter aliquos nisi obseruent sibi debitam fidem ; } ad hoc quod coniugium sit secundum naturam , & ca entre algunos \textbf{ si non guardaren | assi mesmos fe conuenible } para el casamiento \\\hline
2.1.8 & tanto magis hoc decet reges et principes , \textbf{ quanto magis in eis relucere debet fidelitas , et ceterae bonitates . } Secunda via ad inuestigandum & tantomas parte nesçe alos Reyes e alos prinçipes \textbf{ quanto mas deue en ellos reluzir la fialdat | e todas las otras bondades ¶ } La segunda razon para prouar esto \\\hline
2.1.8 & prae omnibus aliis \textbf{ debent diligentiorem habere curam . } Incuria enim regiae prolis & que alos otros \textbf{ ca non auer cuydado de los fijos del Rey } mas puede fazer danno a todo el regno \\\hline
2.1.9 & nimis vacare venereis , \textbf{ et retrahere se ab actibus prudentiae , } et ab operibus ciuilibus , & mucho alos deleytes de lux̉ia \textbf{ e arredrar se delas obras dela sabiduria } e delas obras ciuiles \\\hline
2.1.9 & ab huiusmodi cura , \textbf{ indecens est eos plures habere uxores . } Secunda via sumitur & que deuen tomar en el gouernar aiento del regno \textbf{ non les conuiene de auer muchͣs | mugiers¶ } La segunda razon se toma de parte dela muger . \\\hline
2.1.9 & ut vult Philosophus 9 Ethicor’ , \textbf{ indecens est quoscunque ciues plures habere uxores : } quia eas non tanta amicitia diligerent , & assi conmo dize el philosofo en elix̊ . \textbf{ delas ethicas cosa desconuenible es | a quales si quier çibdadanos } e a quales se quier uatones de auer muchͣs mugieres \\\hline
2.1.9 & non potest \textbf{ portare onera matrimonii , } nec sufficit ad praestandum filiis omnia necessaria & por que en la mayor parte vna sola fenbra \textbf{ non puede sofrir las cargas del matermonio } nin abonda para dar todas las cosas neçessarias alos fijos \\\hline
2.1.9 & ut unus masculus uni adhaereat foeminae , \textbf{ sequitur in hominibus esse quid naturale , } ut quam diu filii indigent parentibus , & Por en de liguele \textbf{ que en los omes le acola natural } que mientra que los fijos han menester ayuda del padre \\\hline
2.1.9 & huius secundi libri plenius dicebatur . \textbf{ Postquam ergo pulli auium apposuerunt debitas pennas , } et peruenerunt ad debitum incrementum : & assi commo es dichon mas conplidamente ençima en el primero capitulo deste segundo libro ¶ \textbf{ Et pues que assi es despues que alos pollos delas aues cresçieren pennolas conuenibles } e vinieren acres çentamiento conueni \\\hline
2.1.9 & Quare si nolentes adhaerere coniugio , \textbf{ decens est eos adhaerere secundum modum , } et ordinem naturalem , & por casamiento es cosa conuenible \textbf{ que se ellos ayunten | segunt manera conuenible } e segunt ordenna traal . \\\hline
2.1.9 & decet omnes ciues \textbf{ una sola uxore esse contentos . } Et tanto magis hoc decet Reges et Principes , & Conuiene que todos los çibdadanos sean pagados \textbf{ cada vno de vna sola mugier . } Et tanto esto mas pertenesçe a los Reyes \\\hline
2.1.9 & quanto decet eos meliores esse aliis , \textbf{ et magis sequi ordinem naturalem . } Patet ergo quod non solum ex parte viri et uxoris , & que todos los otros . \textbf{ Et pues que assi es paresçe } que non lolamente es conuenible de parte del uaron \\\hline
2.1.10 & in congruum unum virum \textbf{ etiam simul habere plures uxores ; } apud nullas tamen gentes & sts̃ naçiones baruaras non se tenido \textbf{ por desconuenible que vn uaron aya muchͣs mugieres en vno . } Empero entre ningunas gentes que biuen \\\hline
2.1.10 & vel propter aliquam figuram \textbf{ et signationem legimus , unius viri plures fuisse uxores : } tamen quod reperitur in paucis , & o por alguna significaçion o figera . \textbf{ Leemos que vn omne ouo muchͣs mugers . } Enpero aquella cosa que es fallada en pocos \\\hline
2.1.10 & Secundum enim commune dictamen rationis detestabile est \textbf{ unum virum simul plures habere uxores : } detestabilius tamen esset , & e de entendemiento cosa de denostares \textbf{ que vn uaron aya en vno muchͣ̃s mugers . } Empero mas de denostares \\\hline
2.1.10 & Coniugium enim ad quatuor comparari potest , \textbf{ ex quibus sumi possunt quatuor rationes , } per quas inuestigare possumus , & que vna muger aya muchos maridos en vno . Ca el casamiento puede ser conparado a quatro cosas \textbf{ delas | quales podemos tomar quatro razones } Por las quales podemos prouar \\\hline
2.1.10 & per quas inuestigare possumus , \textbf{ omnino detestabile esse unam foeminam nuptam } esse pluribus viris . & Por las quales podemos prouar \textbf{ que es de denostar en todo en todo que vna muger sea casada con muchos uarones } Ca en el mater moino \\\hline
2.1.10 & nam naturale est foeminam \textbf{ esse subiectam viro , } eo quod vir prudentia et intellectu sit praestantior ipsa . & por que el uaron es meior \textbf{ que la muger } por sabiduria e por entendemiento¶ \\\hline
2.1.10 & ad conseruationem ordinis naturalis , \textbf{ et ad debitam pacem , } sed etiam ordinatur ad procreationem filiorum . & e aguarda dela orden natural \textbf{ e apaz conueinble mas avn es ordenado a generacion de los fijos . } ¶ Lo quarto \\\hline
2.1.10 & ad filiorum procreationem : \textbf{ sic ordinatur ad eorum debitam nutritionem . } Inconueniens est ergo unam foeminam , & assi commo el casamiento es ordenado a generaçion de los fijos \textbf{ assi es ordenado anudermiento conuenible dellos . } Et por ende non es cosa conueniente \\\hline
2.1.10 & simul viris pluribus detestabilius esse debet . \textbf{ Decet ergo coniuges omnium ciuium uno viro esse contentas : } multo magis tamen hoc decet & que vna muger case en vno con mugons varones . \textbf{ ¶ Et pues que assi es conuiene | quelas mugers de todos los çibdadanos sean pagadas de vn uaron . } Enpero mucho mas conuiene esto alas mugers de los Reyes \\\hline
2.1.10 & Igitur ex parte procreationis filiorum omnino indecens est \textbf{ unam foeminam plures habere uiros . } Nam etsi unus masculus potest & Et por ende departe dela generaçion de lons fijos es \textbf{ cosadesconuenible en qua vna fenbra aya muchos maridos } ca commo quier que vn mas lo puede enprenniar muchͣs fenbras . \\\hline
2.1.10 & Nam etsi unus masculus potest \textbf{ plures foecundare foeminas : } una tamen foemina non sic foecundari potest & cosadesconuenible en qua vna fenbra aya muchos maridos \textbf{ ca commo quier que vn mas lo puede enprenniar muchͣs fenbras . } En pero vna muger non puede \\\hline
2.1.10 & ne impediatur earum foecunditas , \textbf{ uno viro esse contentas . Tanto tamen hoc magis decet Regum , } et Principum coniuges , & por que non sea enbargado el \textbf{ sunconçebemiento sean paragadas de vn marido solo . | Enpero tanto mas conuiene esto } alas mugers de los Reyes e de los prinçipes \\\hline
2.1.10 & Nam ex hoc parentes solicitantur circa pueros , \textbf{ quia firmiter credunt eos esse eorum filios : } quicquid ergo impedit certitudinem filiorum , & çerca los moços \textbf{ por que creen firmemente | que ellos son sus fiios . } Et pues que assi es \\\hline
2.1.10 & et in haereditate prouideant . \textbf{ Detestabile est ergo unum virum plures habere uxores : } sed detestabilius est unam uxorem plures habere viros , & nin en proueer los dela hedat ¶ \textbf{ Et pues que assi es cosa de denostar | que vn ome aya muchas mugers . } Et mucho mas de denostares \\\hline
2.1.10 & Detestabile est ergo unum virum plures habere uxores : \textbf{ sed detestabilius est unam uxorem plures habere viros , } quia per hoc magis impeditur certitudo filiorum . & que vn ome aya muchas mugers . \textbf{ Et mucho mas de denostares | que vna muger aya muchos maridos } ca por esto se enbargaria mas la çertidunbre de los fijos . \\\hline
2.1.11 & ( dum tamen una foemina per coniugium uni copuletur viro ) \textbf{ licitum esse illud coniugium , } cuiuscunque generis , & que vna fenbra sea ayuntada avn uaron \textbf{ por matermonio a aqlmat monio es connenible } de qual se quier linage \\\hline
2.1.11 & Prima via sic patet . \textbf{ Nam cum ex naturali ordine debeamus parentibus debitam subiectionem , } et consanguineis debitam reuerentiam , & La primera razon se declara assi . \textbf{ Ca commo por la orden natural deuamos auer | subiectiuo al padre e ala madre } e reuerençia conueible alos parientes \\\hline
2.1.11 & Nam cum ex naturali ordine debeamus parentibus debitam subiectionem , \textbf{ et consanguineis debitam reuerentiam , } cum huiusmodi reuerentia debita non reseruetur & subiectiuo al padre e ala madre \textbf{ e reuerençia conueible alos parientes } e commo esta reuerençia conueinble non sea guardada \\\hline
2.1.11 & inconueniens esset \textbf{ sic matrem filio esse subiectam . } Non licet ergo filiis contrahere cum parentibus & Et cosa desconueinente \textbf{ si e que la madre fuesse subiecta al fijo . } ¶ Et pues que assi es non conuiene alas fijas de casar con su padre nin alos fijos con su madre \\\hline
2.1.11 & Decet ergo omnes ciues \textbf{ non contrahere coniugia } cum quibuscunque personis ; & que son muy cercanas por parentesto . \textbf{ Et pues que assi es conuiene a todos los çibdadanos de non fazer matermonios } con quales quier perssonas . \\\hline
2.1.11 & dictat naturalis \textbf{ ratio coniugia contrahenda esse inter illos } qui non sunt nimia consanguineitate coniuncti : & Por ende la razon natural dize \textbf{ que los matermonios non son de fazer | entre estos tales } que son muy allegados por parentesco \\\hline
2.1.12 & ( ut patet ex dictis ) \textbf{ ad debitam societatem , } ad pacificum esse , & assi commo paresçe en las cosas \textbf{ sobredichͣs | aconpannia conuenible } e abien de paz e abastamiento dela uida . \\\hline
2.1.12 & Prout ergo coniugium ordinatur \textbf{ ad debitam societatem , } apud Reges , & que el mater moion es ordenado \textbf{ ala conpannia conuenible } conuiene alos Reyes \\\hline
2.1.12 & coniugibus quaerenda est nobilitas generis : \textbf{ sed prout ordinatur ad esse pacificum , } quaerenda est multitudo amicorum : & en sus mugi eres nobleza de liuage \textbf{ mas en quanto el matermoino es ordenado abien de paz } deuen querer \\\hline
2.1.12 & nisi coniugium ordinaretur \textbf{ in quandam societatem debitam et naturalem . } Cum ergo debite et congrue nobili societur : & Mas esto non si asi el casamiento non fuesen ordenado a algua conpanna conuenible e natural . \textbf{ ¶ Et pues que assi es commo deuidamente | e conueniblemente } el noble deua ser aconpannado \\\hline
2.1.12 & quae sint ex nobili genere . \textbf{ Secundo propter esse pacificum quaerenda est amicorum multitudo . } Nam pax inter homines se habet & que sean de noble linage¶ \textbf{ Lo segundo por el bien dela paz es de querer en el mater moino la muchedunbe de los amigos . } Ca la paz se ha entre los omes \\\hline
2.1.12 & ut possit nociua expellere , \textbf{ sic ad esse pacificum requiritur } abundantia ciuilis potentiae & quel enpesçen \textbf{ assi para el bien de paz | conuiene } que aya abondança de poderio çiuil \\\hline
2.1.12 & in esse pacifico . \textbf{ Coniugium igitur prout ordinatur ad esse pacificum , } quaerenda est ex eo amicorum pluralitas . & e non le dexan beuir en paz . \textbf{ Et pues que assy es el casamiento en quanto es ordenado a bien de paz } por ende es de querer en ellos bienes \\\hline
2.1.12 & et Principes in suis coniugibus \textbf{ debent quaerere exteriora bona : } de leui patet & e los prinçipes \textbf{ deuen demandar con sus mugers los bienes de fuera } que pertenesçen a honrramiento del cuerpo . \\\hline
2.1.13 & et quod non amet esse ociosa , \textbf{ sed diligat facere opera non seruilia . } Quae autem sunt opera non seruilia & e que non ame ser uagarosa . \textbf{ mas que ame fazer obras non seruiles | nin de sieruo . } Mas quales son estas obras \\\hline
2.1.13 & ut in praecedenti capitulo dicebatur , \textbf{ ordinetur ad societatem debitam , et ad esse pacificum , } et ad sufficientiam vitae : & sobredicho sea ordenado \textbf{ aconpania conuenible e abien de paz e aconplimiento deuida . } Enpero avn es ordenado a generaçion conuenible de los fijos \\\hline
2.1.13 & ordinatur etiam nihilominus \textbf{ ad debitam prolis productionem , } et ad fornicationem vitandam . & aconpania conuenible e abien de paz e aconplimiento deuida . \textbf{ Enpero avn es ordenado a generaçion conuenible de los fijos } e a esquiua la fornicaçion . \\\hline
2.1.13 & Immo principalius videtur \textbf{ ordinari coniugium } ad haec duo bona , & ø \\\hline
2.1.13 & quam seruando fornicationem vitant ) \textbf{ et bonum prolis magis directe pertinere videntur ad coniugium , } quam ea quae in praecedenti capitulo diximus . & La qual fialdat guardando escusan la fornicaçion \textbf{ e avn el bien dela generaçion de los fijos . | Mas paresçe parte nesçer derechamente al casamiento } que aquellas cosas \\\hline
2.1.13 & ut filii polleant magnitudine corporali , \textbf{ quaerere in suis uxoribus magnitudinem corporis : } tanto tamen magis hoc decet Reges et Principes , & e por que los fijos dellos \textbf{ resplandezcan por grandeza de cuepo de demandar en las sus mugers grandeza de cuerpo . } Enpero tanto mas esto conuiene alos Reyes \\\hline
2.1.13 & decet eos \textbf{ in suis uxoribus quaerere magnitudinem , } et pulchritudinem corporalem : & por fiios grandes e fermosos . \textbf{ Conuiene a ellos de demandar en las sus mugieres grandeza e fermosura corporal . } Ca paresçe que la fermosura dela muger \\\hline
2.1.13 & Dicebatur autem supra , \textbf{ quod agere secundum rationem , } et insequi passiones , & a cuyo contrario son las mugers mas inclinadas . \textbf{ Et dicho es de ssuso que obrar segunt razon e seguir las passiones } lonco las contrarias \\\hline
2.1.13 & quod agere secundum rationem , \textbf{ et insequi passiones , } modo opposito se habent , & Et dicho es de ssuso que obrar segunt razon e seguir las passiones \textbf{ lonco las contrarias } alli que quanto alguno mas sigue las passiones \\\hline
2.1.13 & ( quantum est de se ) \textbf{ magis videntur esse insecutores passionum , } quam viri , & quanto es dessi \textbf{ mas son seguidoras de passiones que los varones . } Ca el uaron es mas acabado en razon \\\hline
2.1.13 & Decet ergo coniuges temperatas esse . \textbf{ Decet eas etiam amare operositatem : } quia cum aliqua persona ociosa existat , & que las muger ssean tenpradas . \textbf{ Et avn les conuiene aellas de amar fazer buenas obras . } Ca quando alguna persona esta de uagar mas ligeramente es inclinada a aquellas cosas \\\hline
2.1.13 & Patet ergo ex iam dictis , \textbf{ quale debet esse coniugium , } et qualiter omnes ciues , & ya dichas qual deueler el casamiento . \textbf{ Et en qual manera todos los çibdadanos } e mayormente los Reyes \\\hline
2.1.13 & et quomodo per tale coniugium \textbf{ consequi possint ciuilem potentiam , } et multitudinem amicorum . & Et en qual manera por tal casamiento \textbf{ pueden auer poderio çiuil } e muchedunbre de amigos . \\\hline
2.1.14 & Quia non sufficit \textbf{ scire quale coniugium , } et qualiter quis se habere debeat & que sean conuenibles e honestas . \textbf{ or que non abasta saber qual es el casamiento } e en qual manera se deue cada vno auer en tomar sumus \\\hline
2.1.14 & et sermones quidam , \textbf{ quomodo vir habere se debeat circa ipsam . } Dicitur ergo tale regimen politicum : & e algunas palauras \textbf{ en qual manera el marido se deua auer çerca la muger . } Et por ende es dicho tal gouernamiento politico e çiuil por que es semeiado a aquel gouernamiento \\\hline
2.1.15 & Dicebatur superius \textbf{ in domo esse tria distincta regimina : } nuptiale , & e la razon natural muestra . \textbf{ a dixiemos de suso que en la casa ay tres gouernamientos departidos } de los quales el vno es mater moian \\\hline
2.1.15 & ita quod unus gladius deseruiebat pluribus officiis : \textbf{ utputa pauperes non valentes plura habere instrumenta , } faciebant aliquod instrumentum fabricari , & assi que vn cuchiello sirue a muchos ofiçios . \textbf{ Conuiene saber que por que los pobres non podian auer } muchos instrumentos fazian fazer vn instrͤde \\\hline
2.1.15 & faciebant aliquod instrumentum fabricari , \textbf{ quo possent ad plura uti officia . } Natura autem non sic agit , & muchos instrumentos fazian fazer vn instrͤde \textbf{ que podiesen vsar en muchos ofiçies } mas la natura non faze \\\hline
2.1.15 & sed idem est \textbf{ esse naturaliter barbarum et seruum ; } esse enim barbarum ab aliquo , & por que entre los barbaros non era ninguno natrealmente sennor \textbf{ mas vn omne era naturalmente barbaro e sieruo . } Ca ser barbaro de alguno este es ser estranno del \\\hline
2.1.15 & et expedit ei quod ab aliquo alio dirigatur , \textbf{ idem est esse natura barbarum et seruum . } Quare si apud Barbaros eundem habent ordinem uxor et seruus , & e conuiene que sea gado de otro este \textbf{ tal es naturalmente barbaro e sieruo } Por la qual cosa sient los barbaros han vna orden la muger \\\hline
2.1.15 & Quare si decet ciues esse industres , \textbf{ et cognoscere modum } et ordinem naturalem ; & Et por ende si conuiene alos çibdadanos de ser sabidores \textbf{ e conosçer la manerar la orden natural } cosa muy desconuenble es a ellos de vsar delas muger \\\hline
2.1.15 & Ex parte igitur ordinis naturalis \textbf{ patet aliud esse regimen coniugale quam seruile : } et non esse utendum uxoribus tanquam seruis . & e de ser menguados de razon e de encendemiento . \textbf{ Et pues que assi es de parte de la orden natural paresçe que otra cosa es el gouernamiento del marido ala mug̃r } que del señor al sieruo . \\\hline
2.1.15 & patet aliud esse regimen coniugale quam seruile : \textbf{ et non esse utendum uxoribus tanquam seruis . } Secunda via ad inuestigandum hoc idem , & Et pues que assi es de parte de la orden natural paresçe que otra cosa es el gouernamiento del marido ala mug̃r \textbf{ que del señor al sieruo . | Et paresçe que non deuen vsar los omes delas mugers } assi commo de sieruas ¶ \\\hline
2.1.15 & sumitur ex parte perfectionis domus . \textbf{ Videtur enim domus esse imperfecta , et habere penuriam rerum , } et non sufficere sibi in vita , & e del abastamiento dela casa . \textbf{ Ca paresçe que la casa non es acabada | e que a ninguna delas cosas } e non abasta assi en la uida \\\hline
2.1.16 & quia regimen coniugale est aliud a paternali et seruili : \textbf{ et ostendere quod aliter debet se habere vir } tam erga uxorem , & matermoian les otro que el paternal \textbf{ e que el suil e mostrar } que en otra manera se deua auer el uaron cerca la mugni \\\hline
2.1.16 & ne possit bene speculari , \textbf{ et ne possit libere exequi actiones suas . } Nascentes ergo ex tali coniugio & por que non pueda bien entender \textbf{ e que non pueda faze sus obras libremente . } ¶ Et pues que assi es los que nasçen de tal casamiento \\\hline
2.1.16 & esse debitum tempus \textbf{ dare operam copulae coniugali : } verum quia vis generatiua est & terçerosetenario de los uarones es tienpo conueinble \textbf{ para dar obra al | ayuntamientod el casamiento . } Empero por que la fu erça de engendtar es muy corrupta \\\hline
2.1.17 & postquam probauit per rationes plurimas , \textbf{ non esse dandam operam coniugio in aetate nimis iuuenili : } inquirit quo tempore & por muchos razones \textbf{ que los omes non deue dar obra al casamiento } en la he perdat de grand mançebia demanda en quet pon deuen dar mas obra ala generaçion delos fijos . \\\hline
2.1.17 & quo flant venti boreales , \textbf{ melius est dare operam coniugio , } quam calido tempore quo flant australes . & en que vientan los vientos del çierco \textbf{ es meior de dar obra al casamiento . } que en el tp̃o caliente \\\hline
2.1.17 & propter roborationem caloris materni uteri , \textbf{ magis possunt conseruare suos foetus , } et eos perfectiores faciunt . & calentraa del uientre de la \textbf{ madremas pueden guardar las ceraturas } e fazer las mas fuertes \\\hline
2.1.17 & tanto tamen hoc magis decet Reges et Principes , \textbf{ quanto decet eos elegantiores habere filios . } Mulierum autem mores & e alos prinçipes \textbf{ quanto mas les conuiene aellos de auer los fijos grandes e esforcados de cuerpo } euedes saber \\\hline
2.1.18 & et quantum ad rationis usum , \textbf{ foemina et puer quodammodo rationem eandem habent , sequitur mores foemineos esse quodammodo pueriles . } In primo ergo libro & e el moço en alguna manera han vna conparaçion . \textbf{ Siguese que las costunbres delas | mugersen alguna manera son costunbres de meços . } ¶ Et pues que assi es en el primero libto \\\hline
2.1.18 & quia timent inglorificari \textbf{ et amittere laudem } quam nimia affectione desiderant . & por que temen de non ser gliadas \textbf{ o temen perder alabança } la qual dessean con muy grand apetito ¶ pueᷤ \\\hline
2.1.18 & ut superius dicebatur . \textbf{ Ex diuersis ergo causis probare possumus mulieres uerecundas esse : } quicquid tamen sit de eius causis , & assi commo dicho es de suso . \textbf{ Et pues que assi es por muchos razones podemos prouar | que las muger sson uergonçosas . } Enpero que quier que sea destas razones \\\hline
2.1.18 & restat narrare quae sunt vituperabilia in eis . \textbf{ Possumus autem narrare tria } in mulieribus vituperabilia . & s finca de dezir que cosas son de denostar en ellas . \textbf{ Et podemos dez que tres cosas son de denostar en ellas ¶ } Lo primero que por la mayor parte son destep̃das e seguidoras delas passiones ¶ \\\hline
2.1.19 & dicere aliqua de regimine coniugum . \textbf{ Sciendum ergo unam esse communem regulam } ad omne regimen . & por qual gouernamiento se han de gouernar las mugers . Por ende deuemos dezir en espeçial algunas cosas del gouernamiento del casamiento . \textbf{ Et pues que assi es deuedes saber | que es vna regla comunal } para todo gouernamiento \\\hline
2.1.19 & Nam quicunque vult aliquid bene regere , \textbf{ oportet ipsum speciales habere cautelas ad ea , } circa quae videt ipsum magis deficere . & conuiene \textbf{ que el aya algunas cautelas espeçiales | para aquellas cosas } en las quales vee \\\hline
2.1.19 & vel dissensio oriri . \textbf{ Secundo decet eas esse pudicas } et honestas . & e mayor discordia \textbf{ que lo segundo couiene a el de los otros . } las de ser linpias e honestas \\\hline
2.1.19 & et honestas . \textbf{ Nam non sufficit coniuges esse castas , } et cauere sibi ab operibus illicitis : & las de ser linpias e honestas \textbf{ por que non abasta | que las muger ssean castas } e se guarden de malas obras . \\\hline
2.1.19 & ( ut recitat Valerius Maximus libro II capitulo de Institutis antiquis ) \textbf{ quodammodo nefas erat bibere vinum . } Unde ait , & constitucon nes antigas entre las mugers romana \textbf{ sera grand denuesto beuer el vino . } Ende dize que el vso del uino en el tp̃o trispassado \\\hline
2.1.19 & debent per seipsos suas instruere coniuges , \textbf{ et debitas cautelas adhibere , } ut polleant bonitatibus supradictis . & por si mesmos \textbf{ e dar les castigos conuenibles } por que puedan resplandesçer en las bondades sobredichͣ̃s . \\\hline
2.1.19 & et ciuili potentia , \textbf{ decet inquirere matronas } aliquas boni testimonii & e en poderio çiuil \textbf{ conuiene les de bulcar buenas mugers } e antiguas de buen testimoino prouadas \\\hline
2.1.19 & tanto maior credulitas adgeneratur viro , \textbf{ ut ei debitam fidem seruet . } Tali ergo regimine regendae sunt coniuges , & tanto mayor firmeza faze en su marido \textbf{ para quel guarde fialdat . } ¶ Et pues que assi es por tal gouernamiento \\\hline
2.1.20 & et tanto magis hoc decet Reges , et Principes , \textbf{ quanto indecentius est eos propter huiusmodi actus habere corpus debilitatum , } mentem depressam , & e alos prinçipes \textbf{ quanto mas desconuenible es aellos | por tales obras carnales } auer el cuerpo enflaqueçido \\\hline
2.1.20 & qualiter cum eis debeant conuersari . \textbf{ Tunc autem viri ad uxorem est conuersatio congrua , } si ei ostendat debita signa amicitiae , & en qual manera deuen beuir conellas \textbf{ mas estonçe es dicha la conuersaçion | e la uida conuenible e buena entre el marido e la muger } si se mostraren sennales conuenibles de amistança e de amor . \\\hline
2.1.20 & si ei ostendat debita signa amicitiae , \textbf{ et si eas per debitas monitiones instruat . } Declarare autem quae sunt signa amicitiae debita , & si se mostraren sennales conuenibles de amistança e de amor . \textbf{ Et si el marido enssencare ala muger | por conuenibles castigos . } Mas declarar quales son las señales conueinbles dela mistança \\\hline
2.1.20 & et si eas per debitas monitiones instruat . \textbf{ Declarare autem quae sunt signa amicitiae debita , } et quae sunt monitiones congruae , & por conuenibles castigos . \textbf{ Mas declarar quales son las señales conueinbles dela mistança } e quales son las moniçonnes e castigos conuenibles \\\hline
2.1.20 & et inspectis conditionibus personarum , \textbf{ suis uxoribus ostendere debita amicitiae signa , } et eas ( ut expedit ) & e catadas las condiconnes delas perssonas mostrar a sus \textbf{ mugersseñales conueibles de amor } e enssennarlas \\\hline
2.1.20 & et eas ( ut expedit ) \textbf{ per debitas monitiones instruere . } Decet Reges , et Principes , & e enssennarlas \textbf{ assi commo conuiene | por conuenibłs castigos . } onuiene alos Reyes e alos prinçipes \\\hline
2.1.21 & non ornans se propter vanam gloriam , \textbf{ posset delinquere in ornatum , } si non esset moderata , & e non se posie con sse \textbf{ por uana eglesia podria pecar | en el conponimiento del cuerpo } si non fuese tenprada . \\\hline
2.1.21 & Tertio decet foeminas \textbf{ circa ornatum corporis esse simplices , } ut non nimia solicitudine ornamenta requirant . & ¶ Lo terçero conuiene alas mugers de ser \textbf{ sinples enel conponimiento de su cuerpo } por que non demanden \\\hline
2.1.22 & increpantur a viris , \textbf{ si contingat eos nimis esse zelotypos : } eo quod ipsi zelotypi & e fazen lo que deuen son denostadas a tuerto de sus maridos \textbf{ si ellos fueren muy çelosos } por que los çelosos son acostunbrados \\\hline
2.1.22 & ostendentes nimis \textbf{ zelotypos non esse laudandos . } Primum est , quia viri in seipsis nimia turbatione vexantur . & para prouar \textbf{ que los muy çelosos non son de loar ¶ } La primera se toma de esto \\\hline
2.1.22 & si contingat suos viros \textbf{ esse nimis zelotypos . } Commune est enim & que por ende las muger sson mas abiuadas a mal \textbf{ quando sus maridos son muy çelosos dellas . } Ca comunal cosa es sienpre \\\hline
2.1.22 & de suis coniugibus \textbf{ esse nimis zelotypos . } Nec etiam decet eos & ø \\\hline
2.1.22 & circa suas coniuges nullam habere custodiam \textbf{ et nullum habere zelum , } sed consideratis conditionibus personarum , & nin les conuiene avn \textbf{ de non auer algun çelo dellas } Mas penssadas las condiconnes delas perssonas \\\hline
2.1.22 & circa propriam coniugem \textbf{ debet debitam curam , } et debitam diligentiam adhibere . & e catadas las costunbres dela tr̃ra \textbf{ ca da vno deue auer cura } e cuydado conuenible de su muger \\\hline
2.1.22 & debet debitam curam , \textbf{ et debitam diligentiam adhibere . } Sic enim decet uirum quemlibet & ca da vno deue auer cura \textbf{ e cuydado conuenible de su muger | e deue auer acuçia conuenible de su casa . } Ca assi conuiene a cada vn marido de auer çelo ordenado de su mugni \\\hline
2.1.22 & Sic enim decet uirum quemlibet \textbf{ erga suam coniugem ornatum habere zelum , } ut sit inter eos amicitia naturalis delectabilis , et honesta . & e deue auer acuçia conuenible de su casa . \textbf{ Ca assi conuiene a cada vn marido de auer çelo ordenado de su mugni } por que sea entre ellos amistança natural delectable e honesta \\\hline
2.1.23 & oportet foeminas deficere a ratione , \textbf{ et habere consilium inualidum . } Nam quantum corpus est melius complexionatum , & que las mugers \textbf{ que fallezcan de vso de razon e que ayan el conseio flaço . } Ca quando el cuerpo es meior conplissionado tanto \\\hline
2.1.23 & In casu tamen potest \textbf{ esse muliebre consilium melius quam virile : } ut quia illud est citius in suo complemento , & que del omne \textbf{ por que el consseio de la muger es mas ayna el su conplimiento que deluats . } por que si acaesçiesse de obrar alguna cosa adesora \\\hline
2.1.24 & ut operemur illud . \textbf{ Quare cum ponere aliquid in praecepto , } sit prohibere , & para obrar aquella cosa . \textbf{ Por la qual razon commo poner alguna cosa } en poridat se a uedar \\\hline
2.1.24 & ab usu rationis deficiunt , \textbf{ nec possunt sic refraenare concupiscentias , } sunt magis propalatiuae secretorum , & por que fallesçen de vso de razon \textbf{ non pueden | assi refrenar sus cobdiçias e sus appetitos } e por ende son mas reueladoras delas poridades que los uatones . \\\hline
2.1.24 & quanto ab usu rationis deficientes , \textbf{ minus possunt refraenare incitamenta concupiscentiarum quam viri . } Secunda via ad inuestigandum hoc idem , & e tan comenos pueden refrenar los abiuamientos de las cobdiçias \textbf{ que los uarones | quanto mas fallesçe enlłas razon } que en los omes \\\hline
2.1.24 & et ridet in facie earum , \textbf{ credunt ipsam esse amicam , } et reuelant ei omnia secreta cordis . & e a Reyr en su faz dellas \textbf{ luego ellas a aquella perssona tienen por amiga } e descubienle todas las poridades de su coraçon \\\hline
2.1.24 & reuelare secreta . \textbf{ Nam cum dicimus hos esse mores iuuenum , } hos mulierum , hos senum . & en qual manera los maridos de una descobrir a sus mugieres los sus secretos . \textbf{ Ca quando nos dezimos | que estas son las costunbres de los mançebos } e estas las de los uieios \\\hline
2.1.24 & esse constantes , \textbf{ et vincere huiusmodi impetus et inclinationes . } Nam licet sit difficile & si quisieren ser constantes e firmes \textbf{ e vençer estos appetitos natraales e estas iclinaçiones . } Ca conmoquier que sea cosa \\\hline
2.1.24 & Nam licet sit difficile \textbf{ superare incitamenta concupiscentiarum , } et sit hoc magis difficile in foeminis & Ca conmoquier que sea cosa \textbf{ guaue sobrepuiar | e vençer los entendimientos delas cobdiçias } et maguer esto sea mas guaue en las mugers \\\hline
2.1.24 & nisi per diuturna tempora sint experti , \textbf{ eas esse discretas , prudentes , et stabiles , } et non esse secretorum propalatiuas . & saluo a aquellas de que han prouado de luengot \textbf{ pon que son sabias e entendidas e estables en vn proponimiento } e que non son descobrideras delos secretos \\\hline
2.1.24 & eas esse discretas , prudentes , et stabiles , \textbf{ et non esse secretorum propalatiuas . } His visis , & pon que son sabias e entendidas e estables en vn proponimiento \textbf{ e que non son descobrideras delos secretos } ¶ vistas estas cosas \\\hline
2.1.24 & quomodo Reges et Principes , \textbf{ et uniuersaliter omnes ciues se habere debeant ad suas coniuges , } et quomodo cum eis debeant conuersari , & en qual manera los Reyes e los prinçipes \textbf{ e generalmente todos los çibdad a uos se de una auer alus mugers } e como de una beuir con ellas . \\\hline
2.1.24 & non sufficienter \textbf{ esse traditam notitiam regiminis nuptialis , } eo quod non ostensum fit , & que non auiemos dado \textbf{ conplidamente sabiduria del gouernamiento de los casados e del casamiento } por que non es mostrado a vna \\\hline
2.1.24 & Sed quia de eis infra dicetur , \textbf{ volumus ea hic silentio praeterire . Primae partis secundi libri de regimine Principum finis , } in qua traditum fuit , & suso deuemos mostrar \textbf{ quales obras conuiene que vsen las mugers } Mas por que dellas diremos adelante \\\hline
2.2.1 & non enim sufficit patrifamilias , \textbf{ scire coniugem regere , } nisi nouerit filios debite gubernare . & en la qual ¶ diremos del gouernamiento del padre alos fijos . \textbf{ Ende non abasta al padre dela casa saber gouernar a su muger } si non sopiere gouernar conueniblemente asus fijos . \\\hline
2.2.1 & et filii pertineat \textbf{ ad domum iam inesse perfectam : } quia domus prima praecedit & Mas la comunindat del padre e del fijo parte nesçan ala casa ya acabada en su ser \textbf{ por que la casa primera es ante que la } casaque es ya acabada . \\\hline
2.2.1 & quia domus prima praecedit \textbf{ domum iam in esse perfectam , } ideo forte videretur alicui statim & por que la casa primera es ante que la \textbf{ casaque es ya acabada . } por ende por auentura parescria a alguno que luego despues que dixiemos del gouernamiento del casamiento \\\hline
2.2.1 & post determinationem de regimine nuptiali , \textbf{ determinandum esse de regimine seruorum . Verum quia , } ut dicitur primo Politicorum , & por ende por auentura parescria a alguno que luego despues que dixiemos del gouernamiento del casamiento \textbf{ deuiemos determinar del gouernamiento de los sieruos . } Enpero assi commo dize el philosofo en el primero delas politicas en el gouernamiento dela \\\hline
2.2.1 & statim est solicita \textbf{ dare ei leuitatem } quandam , & naturada ser al fuego \textbf{ luego es cuydadosa de dar le liuiandat } por que pueda sobir suso \\\hline
2.2.1 & natura est solicita \textbf{ dare animalibus ora et alia organa , } per quae possunt sumere cibum et nutrimentum . & luego es cuydados a de dar a todas las aian lias bocas \textbf{ e todos los organos e instrumentos } por los quales puedan tomar la uianda qual les conuiene . \\\hline
2.2.1 & dare animalibus ora et alia organa , \textbf{ per quae possunt sumere cibum et nutrimentum . } Quare si patres sunt causa filiorum , & e todos los organos e instrumentos \textbf{ por los quales puedan tomar la uianda qual les conuiene . } Por la qual cosa sy los padres son comienço \\\hline
2.2.1 & a patribus esse habent , \textbf{ decet patres habere curam filiorum , } et solicitari erga eos , & e razon de los fijos \textbf{ e los fijos naturalmente han el ser de los padres . Conuiene alos padres de auer cuydado de los fijos } e ser cuydadosos dellos \\\hline
2.2.1 & et ea regulant et conseruant : \textbf{ videmus enim super caelestia corpora influere in haec inferiora , } et ea regere , et conseruare . & e guardan las en su ser . \textbf{ Ca ueemos que los cuerpos | çelestiales enbian de suso su uirtud enlos cuerpos de yuso } e gouiernan los \\\hline
2.2.1 & quem habent ad filios , \textbf{ solicitari circa eos . } Licet omnes patres deceat solicitari & Conuiene que los padres por amor natural \textbf{ que han alos fijos sean cuy dadosos della } aguer que todos los padres de una \\\hline
2.2.2 & magis habet solicitudinem circa filios : \textbf{ naturale est enim quemlibet diligere sua opera , } ut Philosophus in Ethicorum & mas ha cuydado de sus fijos \textbf{ Ca natural cosa es que cada vno ame sus obras } assi commo dize el philosofo en las ethicas \\\hline
2.2.2 & tanto maiori solicitudine et dilectione mouetur circa illud . \textbf{ Patres ergo tanto magis debent solicitari circa filios , } quanto predentiores sunt , & e con mayor amor se deue mouer a ella . \textbf{ Et por ende los padres | tanto mayor cuydado deuen auer de los fijos } quanto mas sabios son \\\hline
2.2.2 & Decet enim filios Regum et Principum \textbf{ maiori bonitate pollere quam alios : } quia secundum Philosophum in Politic’ & e de los prinçipes \textbf{ de auer mayor bondat | e mayor nobleza que los otros . } Ca segunt el philosofo enlas politicas . \\\hline
2.2.2 & et dominantur in regno . \textbf{ Utile est ergo toti regno habere bonos ciues , } sed utilius est habere bonos principantes , & e son sennors en el regno ¶ \textbf{ pues que assi es prouechosa cosa es a todo el regno | de auer bueon sçibdadanos . } Mas mas prouechosa cosa es de auer bueons prinçipes \\\hline
2.2.2 & Utile est ergo toti regno habere bonos ciues , \textbf{ sed utilius est habere bonos principantes , } eo quod principantis sit alios regere et gubernare : & de auer bueon sçibdadanos . \textbf{ Mas mas prouechosa cosa es de auer bueons prinçipes } por que alos prinçipes parte nesçe de gouernar e de garalo sots . \\\hline
2.2.2 & ex bonitate filiorum Regum , \textbf{ qui debent habere principatum et dominium in regno ; } quam ex bonitate et prudentia aliorum . & quanto mayor prouecho seleunata al regno dela bodat de los fijos delos Reyes \textbf{ que deuen auer el prinçipado | e el senorio en el regno } que dela bondat e dela sabiduria de los otros \\\hline
2.2.3 & Nam pacta et conuentiones non interueniunt inter subditum et praeeminentem , \textbf{ nisi sit in potestate subiecti eligere sibi rectorem : } non est autem in potestate filiorum eligere sibi patrem , & que caen entre el subdito e el sennar \textbf{ si non fuere en poderio del subdito | de esceger su gouernador . } Mas non es en poderio de los fijos de escoger \\\hline
2.2.3 & nisi sit in potestate subiecti eligere sibi rectorem : \textbf{ non est autem in potestate filiorum eligere sibi patrem , } si ex naturali origine filii procederent a parentibus . & de esceger su gouernador . \textbf{ Mas non es en poderio de los fijos de escoger | assi mismos padres } mas por natra al nasçençia \\\hline
2.2.3 & et propter bonum ipsorum : \textbf{ cum amare aliquod , } idem sit quod velle ei bonum , & enssennorear alos fiios realmente \textbf{ e por el bien dollos commo amar a alguno sea esso mismo } que querer qual bien . \\\hline
2.2.3 & praeesse aliquibus dominatiue , \textbf{ non intendere bonum ipsorum , } sed proprium : & Et esto es segunt el philosofo enssennarear a algunos seruilmente \textbf{ e non çibdadanamente non entender | enssennorear el bien de los sieruos } mas por el luyo propreo . \\\hline
2.2.3 & patet , paternale regimen \textbf{ non esse idem quod dominatiuum , } sed sumit originem ex amore tamen , & paresçe que el gouernamiento del padre \textbf{ non es tal commo el gouernamiento a uereruo } ca toma comienço de amor . \\\hline
2.2.3 & possumus duplici via venare , \textbf{ paternale regimen trahere originem ex amore . } Prima via sumitur ex ordine naturali . & todemos prouar pardas rasones \textbf{ quel gouernamiento | qł padre toma comie y de amor¶ } La primera razon se torna de la orden natural \\\hline
2.2.3 & patet quod filiis debet \textbf{ praeesse pater propter bonum filiorum . } Non ergo regendi sunt filii eodem regimine , & que el padre deue \textbf{ enssennorear alos fiios | por el bien de los fijos . } Et por ende non son de gouernar los fuos \\\hline
2.2.4 & Dicebatur in praecedenti capitulo , \textbf{ paternale regimen sumere originem ex amore . } Videndum est igitur quantus sit amor patrum ad filios , & ssi commo es dicho en el capitulo sobredich̃ . \textbf{ El gouernamiento patrinal toma comienço del amor Et } pues que assi es deuemos uer \\\hline
2.2.4 & ut nobis innotescat , \textbf{ quomodo patres debeant regere filios , } et filii patribus obedire . & por que nos conosca mos \textbf{ en qual manera de una los padres gouernar alos fijos } e los fujos obedesçer alos padres¶ Et \\\hline
2.2.4 & Sciendum ergo per Philosophum 8 Ethic’ triplici ratione probare , \textbf{ parentes plus diligere filios quam econtra . } Prima via sumitur & que por tres razones podemos prouar \textbf{ que los padres aman mas alos fijos | que los fijos alos padres } ¶La primera razon se toma del alongamiento del tp̃o¶ \\\hline
2.2.4 & nisi ex quibusdam signis , \textbf{ ut quia puer videt personas aliquas magis affici ad eum quam alias , } arguit illas parentes eius , & si non por algunas sennales o por oydo \textbf{ o por que el moço vio algunas perssonas } que se inclina una \\\hline
2.2.4 & quod ab ea comprehendi non potest . \textbf{ In toto enim est assignare aliquid , } quod multum distat a parte : & por que non puie des e conphendido della . \textbf{ Por que en el todo non se puede | sennalar alguna cosa } que es alongada mucho dela parte . \\\hline
2.2.4 & quam econuerso ; \textbf{ cum diligere aliquem , } idem sit quod velle ei bonum , & que los fijos alos padres \textbf{ commo amara alguno sea essa misma cosa } que querer bien \\\hline
2.2.4 & ut congregant eis bona : \textbf{ et congregare aliis bona , } et solicitari circa eorum vitam , & para allegar les los bienes \textbf{ que les faz menestra . | Commo allegar les los bienes } e ser cuydadosos \\\hline
2.2.4 & quam econuerso . \textbf{ Simpliciter tamen parentes plus dicuntur diligere filios , } quam filii ipsos : & Enpero sienpre dezimos \textbf{ que los padres mas aman alos fijnos } que los fijos alos padres \\\hline
2.2.5 & quam parentes tenent . \textbf{ Si enim in aliis legibus parentes statim sunt soliciti erudire proprios filios } in iis quae sunt fidei suae , & que tienen el padre e la madre . \textbf{ Ca si e las otras leyes los padres son acuçiosos de enssennar sus fijos en aquellas cosas } que son de su fe \\\hline
2.2.6 & retrahantur a lasciuiis . \textbf{ Quare cum rationis sit concupiscentias refraenare et lasciuias , } quanto aliquis magis a ratione deficit , & por que de la razon \textbf{ e del entendimiento | es de refrenar los desseos e las locanias . } Et por ende quanto alguon mas fallesçe en razon \\\hline
2.2.6 & a lasciuiis retrahantur . \textbf{ Decet ergo omnes ciues solicitari erga filios , } ut ab ipsa infantia instruentur ad bonos mores . & e por bueons castigos sean tirados delas loçanias . \textbf{ Et pues que assi es | conuieneque todos los çibdadanos ayan grand cuydado de sus fijos } assi que luego en su moçedat \\\hline
2.2.7 & quis debite et distincte \textbf{ proferre aliquod idioma , } nisi sit in eo in ipsa infantia assuefactus ; & e departidamente algun \textbf{ lenageiaie si non fuere acostunbrado ael de su moçedat . } Ca aquellos que se mudan en hedat acabada a tierras luengas do los legunaies son departidos del \\\hline
2.2.7 & vix aut nunquam potest \textbf{ recte loqui linguam illam ; } et ab incolis illius terrae semper cognoscitur & que esten luengo tienpo en aquellas \textbf{ tierrasapenas o nunca pueden fablar derechamente aquella lengua . } Mas luego son conosçidos de los moradores de aqual la tierra \\\hline
2.2.7 & et ab incolis illius terrae semper cognoscitur \textbf{ ipsum fuisse aduenam , } et non fuisse in illis partibus oriundus . & Mas luego son conosçidos de los moradores de aqual la tierra \textbf{ que son auenedizos } e que non nasçieron en aquella tierra . \\\hline
2.2.7 & esse completum et perfectum , \textbf{ per quod perfecte exprimere possent naturas rerum , } et mores hominum , et cursus astrorum , & nin acabado \textbf{ por al qual pudiessen conplidamente pronunçiar las natraas delas cosas e las costunbres de los omes } e los mouimientos delas estrellas \\\hline
2.2.7 & ad perfectionem scientiae , \textbf{ nisi quasi ab ipsis cunabulis vacare incipiat ad ipsam . } Nam licet intelligentiae & a perfectiuo de sçiençia \textbf{ si non lo comne care de pequanon . } Ca commo quier que los angeles \\\hline
2.2.7 & si volunt suos filios distincte \textbf{ et recte loqui literales sermones , } et si volunt eos esse feruentes , & que los sus fijos departidamente \textbf{ e derechamente fablen las palabras delas letras } Et si quieren \\\hline
2.2.7 & et attentos circa ipsos , \textbf{ et peruenire ad aliquam perfectionem scientiae , } ab ipsa infantia eos tradere literalibus disciplinis . & Et si quieren \textbf{ que ellos sean acuçiosos çerca dellas e que puedan venir a alguna perfeççion | e a acabamiento de sçiençia } deuenlos luego poner en su moçedat alas letros \\\hline
2.2.8 & ut vigere possint prudentia et intellectu . \textbf{ Septem scientias esse famosas apud antiquos , } antiqua auctoritas protestatur . & ø \\\hline
2.2.8 & sed indigemus ad hoc auxilio Philosophorum et Doctorum , \textbf{ expedit nos scire idioma illud , } in quo doctores et Philosophi sunt locuti : & Mas para esto auemos men ester ayuda de los philosofos e de los doctores . \textbf{ Conuiene nos de saber e de aprender aquel lenguage } en que fablaron los doctors e los philosofos . \\\hline
2.2.8 & est ut per debita argumenta , \textbf{ et per debitas rationes manifestemus propositum . } Oportuit ergo inuenire aliquam scientiam docentem modum , & que por argumentos conuenibles \textbf{ e por razones derechas | i anifestamos nr̃a uoluntad e nr̃a entençion . } Et por ende conuiene de fallar algua sçiençia \\\hline
2.2.8 & et per debitas rationes manifestemus propositum . \textbf{ Oportuit ergo inuenire aliquam scientiam docentem modum , } quo formanda sunt argumenta , et rationes . & i anifestamos nr̃a uoluntad e nr̃a entençion . \textbf{ Et por ende conuiene de fallar algua sçiençia | que nos mostrasse } en qual manera son de enformar los argumentos e las razones . \\\hline
2.2.8 & quod filios nobilium decet \textbf{ addiscere musicam . } Sed de his forte infra tangetur . & que conuienea los fijos de los nobles \textbf{ de aprender la musica } mas destas razones \\\hline
2.2.8 & Sexta scientia liberalis est geometria , \textbf{ quae docet cognoscere mensuras et quantitates rerum . } Ad hanc autem filii nobilium , & ¶ La sexta sçiençia çia libales geometera \textbf{ que muestra conosçer las mesuras e las quantidades delas cosas . } Et aesta eran puestos \\\hline
2.2.8 & Nam Naturalis Philosophia docens \textbf{ cognoscere naturas rerum , } longe melior est , & Ca la natural ph̃ia \textbf{ que muestra conosçer las naturas delas cosas } muy meior es \\\hline
2.2.8 & non vacat eis \textbf{ subtiliter perscrutari scientias : } maxime igitur decet & Et por que non le suaga a ellos de \textbf{ escodrinnar sotilmente las sçiençias } mucho les conuiene aellos de se auer bien cerca las cosas diuinales \\\hline
2.2.8 & inquantum deseruiunt morali negocio . \textbf{ Decet igitur eos scire grammaticam , } ut intelligant idioma literale : & en quanto siruen ala ph̃ia moral . \textbf{ Et pues que assi es conuiene les a ellos de saber la guamatica } por que entiendan el lenguage delas letras \\\hline
2.2.9 & quam doctor : \textbf{ decet igitur ipsum esse inuentiuum . } Secundo decet ipsum esse intelligentem et perspicacem . & este tal mas es rezador que doctor ¶ \textbf{ Et pues que assi es conuiene al maestro | que non tan solamente sea fallador delas cosas } mas que sea entendido e sotil . Ca assi commo ninguon non puede abastar \\\hline
2.2.9 & decet igitur ipsum esse inuentiuum . \textbf{ Secundo decet ipsum esse intelligentem et perspicacem . } Nam sicut nullus bene et perfecte & que non tan solamente sea fallador delas cosas \textbf{ mas que sea entendido e sotil . Ca assi commo ninguon non puede abastar } asi en la uida bien \\\hline
2.2.9 & Ad huiusmodi autem prudentiam describendam , \textbf{ licet enumerare possemus omnia illa octo } quae in primo libro de prudentia tetigimus , & e de escuir \textbf{ commo quier que la podamos contar | entre aquellas ocho cosas } que dixiemos enel primero libro dela sabiduria . \\\hline
2.2.9 & recolendo praeterita . \textbf{ Nam sicut volens rectificare virgam , } nunquam eam rectificare posset & Ca primero deue ser menbrado e acordado delas colas passadas . \textbf{ Ca assi commo aquel que quiere enderesçar la pierte } ga nunca la puede enderesçar \\\hline
2.2.9 & Secundo decet \textbf{ ipsum esse prouidum futurorum . } Nam sicut aliorum director debet & Lo segundo le conuiene \textbf{ que sea prouiso en las cosas | que han de uenir . } Ca assi commo el que ha degniar los otros \\\hline
2.2.9 & Nam sicut aliorum director debet \textbf{ cogitare praeterita , } ut sciat quomodo per tempora praeterita & Ca assi commo el que ha degniar los otros \textbf{ deue penssar | lo que es passado } por que sepa en qual manera \\\hline
2.2.9 & sic et huiusmodi doctor debet \textbf{ cognoscere particulares conditiones illorum iuuenum , } quos debet dirigere . & es mas çierto en conosçer las cosas particulares e speçiales . \textbf{ Et este dector tal deue conosçer las condiçiones speçiales delos moços } a que ha de castigar e de enssennar . \\\hline
2.2.9 & esse doctor iuuenum , \textbf{ ut eos per debitos sermones , } et per debitas monitiones & Et pues que assi es tal deue ser el doctor \textbf{ e el maestro de los mocos } que los pueda endozir \\\hline
2.2.9 & ut eos per debitos sermones , \textbf{ et per debitas monitiones } inducat ad bonum . & Et pues que assi es tal deue ser el doctor \textbf{ e el maestro de los mocos } que los pueda endozir \\\hline
2.2.9 & de facili ad illicita declinarent . \textbf{ Patet igitur talem quaerendum esse doctorem , } qui quantum ad scientiam speculabilium & alo que les non cunple . \textbf{ Et pues que assi es paresçe | que los moços deuen tomar e buscar tal doctor e tal maestro } quanto alas sçiençias speculatiuas \\\hline
2.2.10 & per se prauum et fugiendum , \textbf{ per debitas monitiones et correptiones inducendi sunt } ut relinquentes mendacium adhaereant veritati , & Por ende son de endozir \textbf{ por castigos | e por conseios conuenibles } que dexen la mentira \\\hline
2.2.10 & in illa quae vident . \textbf{ Quare si contingat eos videre turpia , } magis recordantur de illis , & e mayor acuçiavan a aquellas cosas que veen . \textbf{ Por la qual cosa si contesca que ellos vean cosas torpes } mas se acuerdan dellas \\\hline
2.2.10 & ne audiant quodcunque turpium : \textbf{ quia audire , est prope ad ipsum facere . } Ideo ergo secundum Philosophum cohibendi sunt iuuenes & que non oyan cosas torpes \textbf{ por que el oyr es muy çerca del obrar ¶ } Et pues que assi es segunt el philosofo \\\hline
2.2.10 & quia sicut decens est \textbf{ audire eos honesta , } et pulchra , & quanto a aquellos que oyen . \textbf{ Ca assi commo es cosa conuenible a ellos de oyr } cosas honestas e fermosas \\\hline
2.2.10 & et pulchra , \textbf{ et indecens audire turpia : } sic decet eos audire viros bonos et honestos , & cosas honestas e fermosas \textbf{ e desconuenible de oyr cosas torpes } assi les \\\hline
2.2.10 & et indecens audire turpia : \textbf{ sic decet eos audire viros bonos et honestos , } et cohibendi sunt & e desconuenible de oyr cosas torpes \textbf{ assi les | conuienea ellos de oyr a bueons omes e honestos } e son de refrenar \\\hline
2.2.11 & et qualiter se debeant \textbf{ habere iuuenes circa ipsum . Circa cibum autem contingit } sex modis peccare , vel delinquere . & e los mançebos çerta el comer \textbf{ Mas conuiene saber | que cerça el comer } pueden los omes errar en seys maneras . \\\hline
2.2.11 & oportet \textbf{ ipsum esse proportionatum calori naturali . } Quare si in tanta quantitate sumatur , & Ca si la vianda se ouiere bien a cozer \textbf{ conuiene que sea bien proporçionada ala calentura natural } Por la qual cosa si en tan grand quantia se \\\hline
2.2.11 & si sumatur turpiter . \textbf{ Sunt enim plurimi seipsos pascere nescientes , } quod vix aut nunquam comedere possunt , & pecan si toma la uianda torpemente e suzia mente . \textbf{ Ca son muchos que non saben gouernar assi mismos . } Los quales abeso nunca pueden comer \\\hline
2.2.11 & ideo cum quis assuescit , \textbf{ sumere cibum in aliqua hora , } ut plurimum appetit sumptionem eius in eadem hora . & por ende quando alguno se acostunbra a tomar la uianda \textbf{ en algua ora desordenada } por la mayor parte dessea dela tomar en aquella misma ora . \\\hline
2.2.11 & Nam etiam in vilibus cibariis potest \textbf{ quis ostendere se nimis gulosum , } si nimio studio velit ea esse parata . & por que avn en las viles viandas cada vno se puede mostrar \textbf{ por muy goloso } si las quisiere auer apareiadas con quant estudio . \\\hline
2.2.11 & Sufficit autem eos paulatim et pedetentim instruere , \textbf{ ut cum ad debitam aetatem peruenerint , } sint sufficienter instructi , & que poco a poco sean enformados e enssennados \textbf{ por que quando venieren a hedat } conueinble e acabada puedan ser enssennados \\\hline
2.2.13 & et vitemus delectationes illicitas , \textbf{ expedit aliquando habere aliquos ludos , } et habere aliquas deductiones & que nos non conuienen . \textbf{ Conuiene a nos algunas uegadas de auer algunos trebeios } e algunos solazes conuenibles e honestos . \\\hline
2.2.13 & expedit aliquando habere aliquos ludos , \textbf{ et habere aliquas deductiones } licitas et honestas . & Conuiene a nos algunas uegadas de auer algunos trebeios \textbf{ e algunos solazes conuenibles e honestos . } Mas quales son estos trebeios \\\hline
2.2.13 & Nam non semper statim \textbf{ quis habere potest finem intentum : } ne ergo propter continuos labores & Ca non puede ninguno \textbf{ sienpreauer luego la fin que entiende . } Et pues que assi es por que non fallezca el omne \\\hline
2.2.13 & antequam consequatur illum , \textbf{ ideo oportet interponere aliquos ludos , } et aliquas delectationes , & Et por que algunas vegadas establesce assi fin en que trabaia luengamente ante que alcançe aquella fin \textbf{ por ende conuienele de entroponer alguons trebeios } e algunas delectaconnes \\\hline
2.2.13 & ut vult Philosophus 7 Politicorum . \textbf{ Viso qualiter iuuenes se habere debeant circa ludos . } Restat videre , & delas politicas ¶ \textbf{ Visto en qual manera los moços se deuen auer çerca los trebeios finca de ver } en qual manera se deuen auer çerca los gestos ¶ \\\hline
2.2.13 & Videmus enim prudentes et bonos habere \textbf{ gestus ordinatos et honestos : } cohibent enim sua membra , & Ca veemos que los sabios \textbf{ e los buenos han gestos ordenados e honestos } por que estos tales costramnen e apetan sus mienbros \\\hline
2.2.13 & ne aliquem motum habeant , \textbf{ ex quo quis coniecturari possit elationem animi , } vel insipientiam mentis , vel intemperantiam appetitus . & por que non ayan algun mouimiento \textbf{ del qual alguno pueda presumir | en ellos soƀua del coraçon } o non sabidia del entendimiento \\\hline
2.2.13 & Frustra ergo , \textbf{ cum quis vult audire alium , } retinet os apertum . & Ca el omne non oye con la boca mas por el oreia . \textbf{ Et pues que assi es quando alguno quiere oyr al otro } en vano tiene la boca abierta . \\\hline
2.2.13 & Sicut ergo habent indisciplinatos gestus , \textbf{ qui cum volunt audire alios , } tenent ora aperta : & Et pues que assi es assi commo aquellos \textbf{ que quieren oyr alos otros } e tienen las bocas abiertas \\\hline
2.2.13 & ut deseruiant ad opera quae intendunt . \textbf{ Nam agere aliquos motus membrorum } non deseruientes operi intento , & que entienden fazer . \textbf{ Ca fazer alguons mouimientos de los mienbros } que non siruen ala obra \\\hline
2.2.13 & tempora , et aetates . \textbf{ Nam habentes complexiones magis depressas et minus porosas , } non sic laeduntur a calore et frigore , & departiendo entre las conplissiones e los tienpos e las hedades . \textbf{ Ca los que han las conplissiones espessas | e menos ralas non rsçiben } assi danno dela calentura \\\hline
2.2.14 & maxime competere iuuenibus , \textbf{ fugere societatem prauam , } sumitur ex eo quod iuuenes sunt & que conuiene alos mançebos de foyr \textbf{ la mala conpannia se toma desto que los mançebos son muy muelles } e muy tristor nabłs \\\hline
2.2.14 & est virtus organica siue corporalis . \textbf{ Quare oportet talem appetitum sumere modum , } et mensuram ex ipso corpore . & mas el appetito de los sesos es uirtud organica o corporal . \textbf{ Por la qual cosa conuiene } que tal desseo tome manera e mesura del cuerpo . \\\hline
2.2.15 & quod in omni aetate videtur esse proficuum . \textbf{ Quintum , sunt recreandi per debitos ludos , } et sunt eis recitandae aliquae historiae , & Et esto es prouechoso en todas las hedades ¶ \textbf{ La quinta es que lon de recrear | por trebeios conuenibles . } Et deuen rezar ante ellos algunas bueans estorias . \\\hline
2.2.15 & et hoc maxime , \textbf{ cum incipiunt percipere significationes verborum . } Sextum , a ploratu sunt cohibendi . & Et esto les es prouechoso mayormente \textbf{ quando comiençan a entender las significa connes delas palabras . } ¶ La sexta es que deuen ser guardados de llorar . \\\hline
2.2.15 & maxime videtur \textbf{ esse proportionatum proprio filio . } Secundo pueri sunt prohibendi a vino , & por que la leche dela madre paresçe mas mucho proporçio nada \textbf{ e mas conueinble al fijo | que otra } ning¶lo segundo alos moços es de defendeᷤ el vino mayormente en aquel tienpo \\\hline
2.2.15 & ut plurimum pascuntur lacte \textbf{ assuescant bibere vinum . } Immo dicunt aliqui , & e se fazen de mala disposicion enel cuerpo \textbf{ si en el tienpo en que manian se acostunbraren a beuer vino } Et dize algs \\\hline
2.2.15 & unde Philosophus septimo Politi’ ait , \textbf{ quod mox expedit pueris paruis consuescere ad frigora . } Assuescere enim pueros ad frigora utile est ad duo . & Onde el philosofo en el septimo libro delas politicas \textbf{ dizeque luego conuiene alos mocos pequanos } de acostunbrar los alos frios \\\hline
2.2.15 & quod expedit in pueris \textbf{ facere motus quoscunque } et tantillos ad solidandum membra , & dize \textbf{ que conuiene alos moços de faz quales quier mouimientos pequanos } para soldar los mienbros \\\hline
2.2.15 & et tantillos ad solidandum membra , \textbf{ et ad non defluere propter teneritudinem : } moderatum enim motum in pueris adeo laudat Philosophus , & para soldar los mienbros \textbf{ por que non los dexen caer | por que son tiernos . } Mas el mouimiento tenprado en los mocos \\\hline
2.2.15 & ut ab ipso primordio natiuitatis dicat , \textbf{ fienda esse aliqua instrumenta , } in quibus pueri vertantur , & en tanto lo alaba el philosofo que diz que luego enł comienço de su nasçençia \textbf{ deuen fazer alguons instrumentos } en que se mueun a los moços \\\hline
2.2.15 & postquam incipiunt \textbf{ percipere significationes verborum . } Vel etiam aliqui cantus honesti & Otrossi avn deuen rezar alos moços alguas estorias despues que comiençan at entender las significaçiones delas palabras . \textbf{ Et avn deuen les dezir algunos cantos } ca los cantos honestos son de cantar alos moços \\\hline
2.2.15 & Nam ipsi nihil tristes sustinere possunt : \textbf{ ideo bonum est , eos assuescere ad aliquos moderatos ludos , } et ad honestas aliquas & por que los moços non pueden sostir ninguna cosa triste . \textbf{ Por ende es bien de los acostunbrara algs trebeios tenprados } e a alguas delectaçiones honestas \\\hline
2.2.16 & ut cum dicimus , \textbf{ usque ad septem annos sic esse regendos : } a septimo usque ad decimumquartum sic esse instruendos , & assi deuian ser gouernados los moços . \textbf{ Et de los siete años | fasta los que torze } assi deuian ser enssennados . \\\hline
2.2.16 & exercitandi sunt per debita exercitia , \textbf{ et per debitos motus . } Ut habeant voluntatem bene ordinatam , & e bien ordenado son de vsar \textbf{ por bsos e por mouimientos conuenibles . } Mas por que ayan la uoluntad bien dispuesta \\\hline
2.2.16 & eos ordinari ad virtutes , \textbf{ ut habeant dispositam voluntatem . } Sciendum ergo , & finca de demostrar en qual manera conuiene alos moços de ser dispuestos e ordenados alas uirtudes \textbf{ porque ayan bien dispuesta e bien ordenada la uoluntad e el entendimiento . } Et pues que assi łes deuedes saber \\\hline
2.2.16 & quod pessimum est \textbf{ non instruere pueros ad virtutem , } et ad obseruantiam legum utilium . & dizeque muy mala cosa es de non enssennar \textbf{ e de non enduzir los mocos a uirtudes } e aguardar las leyes bueans e aprouechosas . \\\hline
2.2.16 & perfecte scire non possunt . \textbf{ Ne tamen cum incipiunt habere rationis usum , } omnino sint indispositi ad scientiam , & fallesçe de vso de razon non pueden saber las sçiençias acabada mente . \textbf{ Enpero por que quando comiençan a auer } vso de razon non seanda todo mal apareiados ala sçiençia deuen ser acostunbrados alas otras artes delas \\\hline
2.2.17 & Sed a septimo usque ad quartumdecimum \textbf{ quia iam incipiunt habere concupiscentias aliquas illicitas , } et aliquo modo & Mas desde los siete años fasta los xiiij̊ . años \textbf{ por que ya comiencan a auer algunas cobdiçias desordenadas . | Et en alguna manera comiençan a partiçipar vso de razon } e de entendimiento \\\hline
2.2.17 & ( licet imperfecte ) \textbf{ incipiunt participare rationis usum , } ideo in illo tempore non solum curandum est & commo quier \textbf{ que non acabada mente } por ende en aqł tienpo \\\hline
2.2.17 & Sed a quartodecimo anno , \textbf{ quia tunc perfectius participare incipiunt rationis usum , } non solum curandum est & Mas despues del . xiiij̊ año \textbf{ por que estonçe comiençan a partiçipar | de vso de razon e de entendimiento } mas acabadamente non tan solamente deuen auer cuydado los padres de los fijos \\\hline
2.2.17 & assuescendi sunt ad labores leues : \textbf{ sed deinde debent assumere labores fortes . } Adeo enim secundum ipsum a quartodecimo anno & que fasta los . xiiij años los moços deuen ser acostunbrados a trabaios ligeros \textbf{ mas dende adelante se deuen acostunbrar a trabaios mas fuertes . } Et en tanto que segunt el philosofo desde los . \\\hline
2.2.17 & et in aliis quae ad militiam requiruntur , \textbf{ subire possint labores militares : } tunc enim est quis bene dispositus quantum ad corpus , & que pertenesçen ala caualłia \textbf{ estonçe se pue den poner alos trabaios dela caualłia } por que estonçe es alguno bien ordenado \\\hline
2.2.17 & Cum ergo omnes volentes viuere vita politica , \textbf{ oporteat aliquando sustinere fortes labores } pro defensione reipublicae : & que quieren beuir uida çiuil \textbf{ conuiene les de sofrir alguas uegadas fuertes trabaios } por defendemiento dela tierra . \\\hline
2.2.17 & habeant corpus sic dispositum , \textbf{ ut possint tales subire labores , } ut per eos respublica possit defendi . & en que la tierra aya meester defendimiento ayan el cuerpo o bien ordenado \textbf{ por que puedan tomar trabaios } e pueda defender la tierra . \\\hline
2.2.17 & ut habeant sic bene dispositum corpus , \textbf{ ut possint debitos subire labores , } quod maxime fieri contingit , & por que ayan el cuerpo bien ordenado \textbf{ por que puedan tomar trabaios conuenibles la qual cosa } mayormente se pue de fazer \\\hline
2.2.17 & si ad debita exercitia assuescant . \textbf{ Viso , quomodo a quartodecimo anno ultra solicitari debent patres erga filios , } ut habeant dispositum corpus . & e amouimientos conuenibles ¶ \textbf{ Visto en qual manera del . xiiij . año adelante deuen los padres auer cuydado de los fijos } por que ayan el cuerpo bien ordenado \\\hline
2.2.17 & Restat videre , \textbf{ quomodo solicitari debeant circa eos , } ut habeant ordinatum appetitum . & por que ayan el cuerpo bien ordenado \textbf{ finca de ver en qual manera de una auer cuydado dellos } por que ayan el appetito bien ordenado . \\\hline
2.2.17 & quia cum ex tunc incipiant \textbf{ habere perfectum rationis usum , } videtur eis quod digni sint dominari , & Conuiene a saber quanto al orgullo e ala locama . \textbf{ paresce que estonçe comiençan a auer vso de razon acabada } paresçe les que son dignos de enssennorear e de ser senneres \\\hline
2.2.17 & Nam quia a decimoquarto anno ultra incipiunt \textbf{ habere perfectum rationis usum , } ut dicebatur , & por que del xiiij ̊ año adelante comiença los mançebos de auer \textbf{ mas acabadamente vso de razon . } Ca assi commo es dicho desde \\\hline
2.2.17 & ex tunc potest \textbf{ instrui non solum in grammatica } quae est scientia verborum , & estonçe pueden ser enssennados \textbf{ non tan solamente en la guamatica } que es assi commo sçiençia de palabras o en logica \\\hline
2.2.18 & per quam impeditur mentis sublimitas . \textbf{ Eos autem qui debent regere regnum , } magis expedit esse prudentes , & por la qual cosa se enbarga la sotileza del entendimiento . \textbf{ Et aquellos que deuen gouernar el regno } mas les conuiene de ser sabios \\\hline
2.2.18 & nec sic debeant \textbf{ fugere corporales labores ; } ut effecti quasi muliebres , & que los Reyes e los prinçipeᷤ non de una de todo dexar el vso delas armas \textbf{ nin de una assi escusar los trabaios del cuerpo } por que le fagan mugeriles \\\hline
2.2.18 & qui debent alios regere , \textbf{ vitare inertiam et solicitudinem illicitam , } vacando moralibus scientiis , & ¶ Et pues que assi es conuiene aquellos \textbf{ que deuen gouernar los otros de escusar la ꝑeza } e el cuydado desconueinble estudiando enlas sçiençias morales \\\hline
2.2.19 & qualis cura gerenda sit circa filias . \textbf{ Nam sicut decet coniuges esse continentes , } pudicas , abstinentes , et sobrias : & çerca delas fijas \textbf{ ca assi commo conuiene alas madres } de ser continentes e castas e guardadas e mesuradas en essa misma manera conuiene alas fijas de ser tales \\\hline
2.2.19 & ad conseruandam puritatem et innocentiam , \textbf{ est vitare commoditates malefaciendi , } propter quod et prouerbialiter dicitur , & e la inoçençia de non pecar \textbf{ e para guardar las maneras de mal Razer } por la qual cosa se dize vn prouerbio \\\hline
2.2.19 & in quibus est ratio praestantior , \textbf{ est magnum periculum non vitare commoditates delictorum : } multo magis hoc est in foeminis , & e el entendimiento mayor es grant peligro \textbf{ de non escusar las azinas de los pecados much mas es esto de escusar en las mugers . } Et avn mas es en las fiias e en las moças \\\hline
2.2.19 & ex virorum consortio . \textbf{ Tollere autem a puellis verecundiam , } est tollere ab eis fraenum , & por la conpannia de los uarones \textbf{ mas toller alas moças la uerguença es toller el freno dellas } por el qual freno se retrahen \\\hline
2.2.19 & videtur esse verecundia . \textbf{ Decens ergo est cohibere puellas } a discursu et euagatione , & por que non puedan sallir a fazer cosas torpes . \textbf{ Et pues que assi es cosa conuenible es de defender es alas mocas } que non corran \\\hline
2.2.20 & Texere enim et filare , \textbf{ et operari sericum , } satis videntur opera competentia foeminis . & segunt el departimiento delas perssonas \textbf{ cateyer e filar e obrar e coser e taiar algunas cosas sotiles } asaz paresçen obras \\\hline
2.2.20 & infra declarandum esse , \textbf{ circa quae opera deceat foeminas esse intentas . } Ostenso , & casamien toca y dixiemos que adelante serie de declarar cerca quales obras conuenia \textbf{ que las mugers fuesen acuçiosas . } ostrado que non conuiene alas moças de andar uagarosas a quande e allende \\\hline
2.2.21 & Ostenso , \textbf{ quod non decet puellas esse vagabundas , } nec decet eas viuere otiose : & que las mugers fuesen acuçiosas . \textbf{ ostrado que non conuiene alas moças de andar uagarosas a quande e allende } nin les conuiene de beuir ociosas \\\hline
2.2.21 & ut foeminae etiam a puellari aetate discant \textbf{ cautos proferre sermones , } decet eas non esse loquaces : & tomadesto \textbf{ que las mugrͣ̃s non sean prestas avaraias e apeleas } ca commo las muger se mayormente las mocas \\\hline
2.2.21 & Decet ergo ipsas \textbf{ per debitam taciturnitatem adeo examinare dicenda , } ut nec dicant aliqua , & e por ende les conuiene aellas de ser callanţias en manera conuenible \textbf{ e en tanto examinar las cosas | que han de dezer } por que non digan alguas cosas \\\hline
2.3.1 & eo quod hae materiae sunt connexae , \textbf{ intendimus instruere uolentem suas domus debite gubernare , } non solum quantum ad regimen ministrorum et familiae , & por que estas materias son ayuntadas en vno entendemos de enssennar \textbf{ a aquellos que quisieren | conueinblemente gouernar sus calas } non lo lamente quanto al \\\hline
2.3.1 & vel ad sufficientiam vitae , \textbf{ quae supplere videntur indigentiam corporalem . } Determinabimus igitur & e a conplimiento dela uida \textbf{ e que paresçen que cunplen la mengua corporal . } Et por ende determinaremos en esta terçera parte deste segundo libro \\\hline
2.3.1 & quomodo deceat \textbf{ ipsos se habere circa possessiones , } et numismata , & e generalmente todos los çibdadanos . \textbf{ Et en qual manera se deuan auer çerca las possesiones } e çerca las riquezas e los dineros \\\hline
2.3.1 & per quae opera sua complere possit . \textbf{ Volens ergo tradere notitiam de arte fabrili , } oportet ipsum determinare de martello , et incude , & por los quales pueda conplir sus obras . \textbf{ Et por ende los que quieren dar conosçimiento dela arte del ferrero } conuiene les de determinar del martiello e dela yunque \\\hline
2.3.1 & et spectat ad fabrum talia instrumenta cognoscere . \textbf{ Sic volens tradere notitiam de arte textoria , } debet determinare de pectinibus , & cognosçertales estrumentos . \textbf{ Et dessa misma manera | el que quiere dar conosçimiento del arte del texer } deue determinar de los peinnes \\\hline
2.3.1 & talia instrumenta cognoscere . \textbf{ Quare volens tradere notitiam } de arte gubernationis domus , & e pertenesçe al texedor de conosçer tales estrumentos . \textbf{ Por la qual cosa el que quisiere dar conosçimiento del arte del gouernamiento dela casa } deue determinar de los hedifiçios \\\hline
2.3.1 & quod spectat ad gubernatorem domus \textbf{ scire debite se habere } circa ministros et seruos : & Mas por que aquellas mismas razones sen podia prouar \textbf{ que parte nesçe al gouernamiento dela casa saber se auer } conueinblemente çerca los ofiçiales \\\hline
2.3.2 & Suprema autem in quolibet negotio \textbf{ esse videntur architectores et domini : } infima vero sunt organa inanimata : & Mas las cosas mas altas en cada vn negoçio \textbf{ son los maestros | e los gouernadores e los sennores . } Et las cosas mas baxas son instrumentos sin alma . \\\hline
2.3.2 & per se ipsos esse praeparatores mensarum , \textbf{ vel esse ostiarios , } aut aliqua talia exercere : & por si mismos sean apareiadores delas mesas \textbf{ o que sean porteroso } que vsen de o tristales cosas . \\\hline
2.3.3 & Quod autem Reges et Principes debeant \textbf{ habere habitationes mirabiles , } et subtili industria constructas , & Mas que los Reyes e los prinçipes de una auer \textbf{ moradas marauillosas e labradas } por engennio muy sotil \\\hline
2.3.3 & nam secundum Philosophum 4 Ethicorum capitulo de Magnificentia , \textbf{ maxime gloriosos et nobiles decet esse magnificos : } Reges ergo et Principes , & en el quarto libro delas ethicas \textbf{ enł capitulo dela magnifiçençia | que much mas conuiene alos Reyes } e alos prinçipes \\\hline
2.3.3 & quantum ad industriam operis , \textbf{ decet habere habitationes mirabiles . } Alii vero ciues tales habitationes & e alos prinçipes \textbf{ quanto ala maestera dela obra pertenesçe auer moradas matauillosas } e por ende los otros çibdadanos deuen auer tales moradas \\\hline
2.3.3 & hoc viso opinatur \textbf{ principem esse tantum , quod quasi impossibile sit ipsum inuadere : } et quia circa impossibilia non cadit electio neque consilium , & quando esto vee pienssa en su coraço \textbf{ que el prinçipe es tan grande | que en ninguna manera non podria yr contra el } e por que en las cosas \\\hline
2.3.3 & ne in contemptum habeantur a populo , \textbf{ facere aedificia magnifica , } prout requirit decentia status , & enpero conuiene alos Reyes \textbf{ e alos prinçipes de fazer moradas costosas e nobles } assi commo el su estado demanda \\\hline
2.3.3 & In domibus ergo Regum et Principum \textbf{ oportet multos abundare ministros , } ut ergo non solum personas Regis et Principis , & e de los prinçipes conuiene \textbf{ que ayan muchos ofiçiales | e much ssiruient s̃ Et pues que assi es } por que non solamente la persona del Rey o del prinçipe mas avn \\\hline
2.3.3 & propter circumstantiam montium contingit \textbf{ ipsum non esse salubrem . } Sic enim imaginari debemus , & et por ende contesçe \textbf{ que el ayre | y non sea sano } porque deuemos assi ymaginar \\\hline
2.3.4 & illas aquas generari , \textbf{ vel transire per aliqua loca infecta , } a quibus talem odorem , & que aquellas aguaas son engendradas en logares corruptoso \textbf{ que passan por algunos logares non sanos } delos quales trahen tal color o tal sabor \\\hline
2.3.4 & In ordine autem Uniuersi , \textbf{ prout requiri aedificium construendum , } sunt tria consideranda , & mas en la arden del mundo \textbf{ segunt que demanda la morada } que es de fazer son de penssar tres cosas \\\hline
2.3.5 & quod dominetur istis sensibilibus , \textbf{ et quod possit eis uti in suum obsequium , } et quia hoc est quodammodo possidere ea , & que enssennore e a estas cosas senssibles \textbf{ e que pueda vsar dellas | e resçebir seruiçio dellas } segunt quel fuere uisto \\\hline
2.3.5 & quia statim solicita est \textbf{ inducere lac in mamillis matris , } ut ex eo animalia genita nutriri possint . & quanto al nutermiento dellas \textbf{ por que luego que nasçe es acuçiosa de aduzer leche enlas teras delas madres } assi que de aquella lech̃e las aian las enrendradas se pueden cerar \\\hline
2.3.5 & ut homo est , \textbf{ ut vult Philosophus primo Polit’ habere possessionem , } et dominium aliquarum rerum exteriorum & en quanto es omne \textbf{ segunt dize el philosofo | enl primero libro delas politicas de auer possession } e sennorio de algers cosas de fuera \\\hline
2.3.6 & et suprema dilectio in ciuitate . \textbf{ Tunc enim omnes viri diligerent omnes foeminas tanquam proprias , } sic etiam omnes homines diligerent omnes pueros tanquam filios proprios , & e grand amorio en la çibdat \textbf{ por que estonçe todos los omes | a marien a todas las mugers } assi commo sus prop̃as mugers . \\\hline
2.3.6 & Possumus autem ex diuersis locis \textbf{ in libro Polit’ accipere tria , } per quae triplici via venari possumus , & de departidos logares \textbf{ enł libro delas politicas tres cosas } por las quales podemos prouar \\\hline
2.3.6 & per quae triplici via venari possumus , \textbf{ quod expedit ciuitati ciues habere proprias possessiones . } Prima via sumitur , & por tres razons \textbf{ que conuiene ala çibdat | que los çibdadanos ayan possessiones propias } ¶ \\\hline
2.3.6 & ut plurimum contingeret ciuitatem \textbf{ illam sic ordinatam venire ad inopiam , } ut ciues non possent sibi in vita sufficere ; & por ende contesçeria en la mayor parte \textbf{ que aquella çibdat | assi orde nada uerme a grant pobreza } por que los çibdadanos non podrien abondar assi enla uida \\\hline
2.3.6 & utile est ciuitati ciues \textbf{ habere possessiones proprias , } ne propter ignauiam circa communia , & assi commo dicho es pro prouechosa cosa es ala çibdat \textbf{ que los çibdadanos ayan | possessionspropreas } por que non auiendo cuy dado çerca las cosas comunes dela casa \\\hline
2.3.7 & ordinauit enim ea ad usum et dominium nostrum ; \textbf{ licitum est ergo sumere nutrimentum ex agris , } et animalibus domesticis & e las ordeno a vso e añro sennorio . \textbf{ ¶ Et pues que assi es cosa conuenible es | de tomar nudermiento delos canpos } e delas aianlias de casa \\\hline
2.3.7 & talia facere , \textbf{ et ordinare ea in usum proprium . } Furtiua autem vita & por \textbf{ sitałs̃aianlias e ordenar las asu uso propreo } Mas la uida de furtar fablado sinplemente de ssi es desconueible \\\hline
2.3.7 & quia sapientes naturaliter debent dominari insipientibus , \textbf{ iustum habere bellum contra ipsos , } si eis nolint esse subiecti . & sennorsnaturalmente de los non sabios \textbf{ e por ende han batalla decha contra ellos } si non quisieren sorsus subiectos . \\\hline
2.3.8 & Communiter videntur \textbf{ delinquere homines } circa appetitum possessionum , & o munalmente paresçe \textbf{ que los omes pecan çerca el appetito delas possessiones } e çerca las cobdiçias de las riquizas \\\hline
2.3.8 & infinita est diuitiarum concupiscentia , \textbf{ cuius causa est studere homines circa viuere , } non circa bene viuere ; & que la cobdiçia delas riquezas es sin fin e sin mesura \textbf{ e la razon desto es | por que los omes estudian cerca beuir } e non çerca bien beuir . \\\hline
2.3.8 & medicus ergo sanitatem quasi appetit \textbf{ inducere infinitam , } sed potionem appetit dare & Et pues que assi es el fisico \textbf{ dessea aduzir salud sin mesura e sin fin } mas la melezina dessea de dar segunt manera \\\hline
2.3.8 & et putant ipsum finem \textbf{ in diuitiis esse ponendum , } appetunt eas in infinitum . & que han de poner su fin \textbf{ e su bien andança en las riquezas } dessean las sin mesura e sin fin . \\\hline
2.3.8 & Sed quod ad gubernationem domus pertineat \textbf{ non appetere infinitas possessiones , } duplici via venari possumus . & Mas que pertenezca al gouernamiento dela casa \textbf{ non dessear las riquesas e las possessiones sin mesura } e sin fin esto podemos mostrar \\\hline
2.3.8 & ergo nec gubernatiua debet \textbf{ quaerere infinitas possessiones . } Decet igitur omnes ciues & enł primero libro delas politicas . \textbf{ Et pues que assi es nin el arte del gouernamiento dela casa non deue demandar possessiones et riquezas sin mesura e sin fin . } ¶ Et por ende conuiene a todos los çibdadanos \\\hline
2.3.8 & et Principibus quam in aliis , \textbf{ quanto decet habere ordinatiorem uoluptatem , } et meliorem aestimationem finis : & que en los otros \textbf{ quanto mas conuiene aellos de auer mayor ordenamiento dela uoluntad } e meior estimacion dela finca \\\hline
2.3.8 & detestabilius est in Rege \textbf{ non habere ueram aestimationem } de fine quam in populo , & mas de denostares enl Rey \textbf{ de non auer uerdadera } estimaçonn dela fin \\\hline
2.3.8 & sicut detestabilius est in sagittante \textbf{ non cognoscere signum , } quam in sagitta : & mas de denostar es enł liallero \textbf{ de non conosçer la señal } que en la saeta \\\hline
2.3.9 & uel per se uel per procuratores intermedios , \textbf{ nam ipsius patrisfamilias est totam indigentiam subleuare domesticam . } Sed cum eiusdem ad seipsum non sit & o por si o por sus procuradores entre medianos . \textbf{ Ca al padre familias parte nesçe dereleuar toda la menguadela casa . } Mas por que non puede ser conpra \\\hline
2.3.9 & quod in prima communitate quae est domus , \textbf{ manifestum est nullum esse opus ipsius commutationis igitur } propter communitates & que es comuidat dela casa es cosa prouada \textbf{ que non es menest obra de muda conn ninguna } Et por ende por las otras comuidades \\\hline
2.3.9 & quod habetur in toto regno , \textbf{ oportuit introduci commutationem rerum ad denarios , } et econuerso . & que hades es en todo el regno \textbf{ conuiene de poner m̃udaçion delas cosas alos dineros } e de los diueros alas cosas \\\hline
2.3.9 & commode ad partes longinquas portari non possunt . \textbf{ Oportuit ergo inuenire aliquid } quod esset portabile , & non las poderemos leuar conueniblemente a luengas tierras . \textbf{ Et pues que assi es conuiene de fablar alguna cosa } que se podiesse leuar \\\hline
2.3.9 & et utile , \textbf{ pro quo inueniri possent victualia . } Huiusmodi autem maxime est argentum , et aurum , & e que fues fermosa e aprouechable \textbf{ por que se podiessen fallar las uiandas . } Mas entre todas las otras cosas \\\hline
2.3.9 & Primitus ergo inuentae fuerunt commutationes ad metalla solum secundum pondera : \textbf{ ut volentes habere tantum vini , } oportebat dare tantum ponderis argenti , vel auri , & tan solamente segt̃ sus pesos \textbf{ assi que los que quirien auer tunerto de vino } conuimeles a dar tanto de peso de plata o de oro o avn de otro metal \\\hline
2.3.9 & ut volentes habere tantum vini , \textbf{ oportebat dare tantum ponderis argenti , vel auri , } vel etiam alterius metalli , & assi que los que quirien auer tunerto de vino \textbf{ conuimeles a dar tanto de peso de plata o de oro o avn de otro metal } assi commo plazia de establesçer en aquel tienpo alos pueblos e alos Reyes . \\\hline
2.3.9 & pro quo statim \textbf{ secundum ipsius valorem recipere possumus supplentia indigentiam vitae . } In toto ergo uno regno & segunt el ualor de aquellas cosas \textbf{ que cunplen | para cunplir la mengua dela uida . } Et pues que assi es por que non fuessen muy agua uiados los omes \\\hline
2.3.9 & cum ex una parte regni oportebat \textbf{ eos accedere ad aliam , } portando secum victualia onerosa , & e que estan en vn logar del regno \textbf{ commo contesçe alas vezes | e los que estan en vna parte del regno an de yr } ala otra parte del regno leunado \\\hline
2.3.9 & et rerum ad numismata , \textbf{ oportuit inuenire commutationem numismatum ad numismata . } Patet ergo quot sunt commutationes , & e delas cosas alos \textbf{ diueros otra mudaçiones | que es de monedas alas monedas . } Et pues que assi es paresçe \\\hline
2.3.9 & et quomodo sunt inuentae , \textbf{ et quae fuit necessitas inuenire denarios . } Decet ergo prudentem patremfamilias , & e por que son falladas \textbf{ et qual fue la neçessidat | para fallar los des } e por ende conuiene al sabio padre familias \\\hline
2.3.10 & tractatur de numismatibus , \textbf{ nam habere possessiones } et abundare vino et frumento , & se sigue el tractado delas monedas \textbf{ ca aun possessiones } e abondar en vino \\\hline
2.3.10 & quae fuit necessitas \textbf{ inuenire numismata et pecuniam , restat dicere , } quot sunt species pecuniatiuae . & Et pues que assi es despues que dixiemos qual fue la neçessidat de fallar las monedas \textbf{ e los dineros finca de dezer quantas son las maneras de los dineros . } Et el philosofo en las politicas \\\hline
2.3.10 & ex totidem denariis numero , \textbf{ confici massam maioris ponderis : } ex quo casu ars sumpsit originem , & e por esta razon acaesçe por auentura \textbf{ que de tantos dineros en cuento se faze massa de mayor peso . | Et desta } auentraa tomo comienço esta arte \\\hline
2.3.10 & Quarta species pecuniatiuae \textbf{ dicitur esse tachos , } quod in latino idem sonat & La terçera manera del arte pecuniatiua \textbf{ es dicha en gniego | tal zez que es usura . } Et en latin tato suena commo parto \\\hline
2.3.10 & Videtur enim haec ars parere \textbf{ et generare denarios , } quam nos communi nomine appellamus usuram : & Et en latin tato suena commo parto \textbf{ por que pare por que paresçe que esta pare e engendradinos la qual arte nos } por nonbre comunal llamamos usura \\\hline
2.3.11 & sicut duplici nomine nominatur , \textbf{ sic duplici via inuestigare possumus eam detestabilem esse . } Vocatur enim primo denariorum partus , & talzes assi commo ella ha dos nonbres \textbf{ assi podemos prouar | por dos razones que ella es de denostar . } Ca primeramente la llamamos parto de dineros \\\hline
2.3.11 & ut aliud est domus , \textbf{ et aliud inhabitare ipsam : } in aliquibus & que otra cosa es la casa \textbf{ e otra cosa es morar enella } enpero en alguas cosas nunca se puede otorgar el uso dellas sinon \\\hline
2.3.11 & possessor domorum potest \textbf{ concedere usum domus } ut inhabitationem & et non enagenar la casa \textbf{ el señor dela casa puede otorgar el uso dela casa } para morar sin que otorgue la sustançia della . \\\hline
2.3.11 & est expendere \textbf{ et alienare denarios nunquam ergo potest } concedi usus proprius denarii , & assi ca el uso propo de los dineros es despender los e enagenar los . \textbf{ Et por ende nunca se puede otorgar el uso propreo de los dineros } si non se otorgare la sustançia \\\hline
2.3.11 & eius est usus . \textbf{ Volens ergo accipere pensionem de usu denariorum , } dicitur committere usuram , & que cuya es la sustaçia del esalulo della . \textbf{ Et pues que assi es el que quiere tomar ganançia de lisso | de los dineros dezimos } que comete usura \\\hline
2.3.11 & Volens ergo accipere pensionem de usu denariorum , \textbf{ dicitur committere usuram , } uel dicitur usurpare , & de los dineros dezimos \textbf{ que comete usura } e tal es dichusurar e robar uso \\\hline
2.3.11 & uel dicitur usurpare , \textbf{ et rapere ipsum usum : } quia concedendo usum denarii , & que comete usura \textbf{ e tal es dichusurar e robar uso } por que los que otorgan el uso del dinero \\\hline
2.3.11 & Cum ergo quis possit \textbf{ concedere denarios } ad talem usum & dellos sus dineros ponen ante ssi muchedunbre de dineros . \textbf{ Et pues que assi es commo alguon pueda otorgar los dineros para tal uso . } Conuiene a sabra para paresçer con ellos \\\hline
2.3.11 & si volunt naturaliter Dominari , \textbf{ prohibere usuras , } ne fiant eo & si quisieren ser señors natalmente \textbf{ de defender las usuras } que non se fagan \\\hline
2.3.12 & oeconomicum et dispensatorem domus \textbf{ esse expertum circa possessiones , } sciendo quae sunt magis fructiferae , & Ca conuiene segunt el philosofo al mayordomo e al despenssero dela casa de ser prouado \textbf{ e sabio | derca las possessiones } sabien \\\hline
2.3.12 & quibus pecuniam sunt lucrati , \textbf{ dicitur scire lucratiuam experimentalem . } Recitat enim Philosophus & por los quales fechos ganaron alguas riquezas \textbf{ Esta prueua tal es dicho ganançiosa por prueua . } Ca el philosofo cuenta dos fechos particulares \\\hline
2.3.12 & tum quia nullus poterat \textbf{ vendere oleum , } nisi ipse : & por todo el olio que auie de venir . \textbf{ Et lo vno por que ninguon non podie vender olio } si non el solo . \\\hline
2.3.12 & ( secundum Philos’ ) \textbf{ est facere monopoliam , } idest facere vendationem unius : & por que segunt el philosofo \textbf{ entre todas las cosas | que acresçientan las riquezas es fazer monopolia } que quiere dezer vendiconn de vno solo . \\\hline
2.3.12 & est facere monopoliam , \textbf{ idest facere vendationem unius : } nam quia unus solus vendit , & que acresçientan las riquezas es fazer monopolia \textbf{ que quiere dezer vendiconn de vno solo . } Ca quando vno solo uende taxa el preçio \\\hline
2.3.12 & secundum vitam politicam volentem prouidere indigentiae domesticae , \textbf{ habere curam de acquisitione pecuniae , } secundum quod exigit suus status : & que quiere proueer ala mengua dela casa \textbf{ segunt uida politica de auer cuydado de ganar dineros segunt que requiere } e demanda el su estado de cada vno . \\\hline
2.3.12 & vel per se , \textbf{ vel per alios esse expertos , } sciendo particulares conditiones regni , & por que conuiene a ellos \textbf{ que por si o por otros ayan prouada } de saber las condiconnes particulares del regno \\\hline
2.3.12 & qui tantae sapientiae secularis praedicabatur , \textbf{ habuisse massaritias multas . } Non obstante enim quod terrae fertilissimae dominabatur , & que era de tan grand sabiduria del sieglo que auie greyes \textbf{ maguer que fuesse señor de tierra muy abondosa } en la qual auya muchͣs uiandas e de grand mercado . \\\hline
2.3.12 & Quare decet Reges , \textbf{ et Principes habere homines industres } tam super cultura agrorum et vinearum , & Por la qual cosa conuiene alos Reyes e alos principes \textbf{ de auer omes acuçiosos } e sabidores tan bien sobre las lauotes de los canpos \\\hline
2.3.12 & sicut alicubi consuetudo est \textbf{ habere multitudinem columbarum vel aliarum auium , } ex quibus domestica alimenta sumuntur . & assi commo veemos \textbf{ que en algunos logars han costunbres de auer palomares | e muchedunbre de palomas } e de otras aues delas quales son tomados gouernamientos para la casa \\\hline
2.3.12 & viuere melior est vita peregrina : \textbf{ et habere alimenta ex propriis , } laudabilius est , & çibdadanamente es meior que la uida pelegnina \textbf{ e auer uiandas de propreo es mas de loar } que çonprar cada vna cosa \\\hline
2.3.13 & Ostendemus enim primo seruitutem aliquam naturalem esse , \textbf{ et quod naturaliter expedit aliquibus aliis esse subiectos : } quod probat Philosophus primo Polit’ quadruplici via , & que alguna suidunbre es dichͣ natural \textbf{ e que conuiene que alg ssean subietos naturalmente a algunos otros } la qual cosa praeua el philosofo \\\hline
2.3.13 & ut si plures voces efficiunt aliquam harmoniam , \textbf{ oportet ibi dare aliquam vocem praedominantem , } secundum quam tota harmonia diiudicatur . & Assi commo si muchas uozes fiziess en alguna armonia o concordança de canto . \textbf{ Conuerna de dar y alguna bos | que enssennoreasse sobre las otras } segunt la qual serie iudgada toda aquella concordança delas uozes delas otras avn en essa misma manera \\\hline
2.3.13 & Corpus enim non posset \textbf{ seipsum dirigere ad operationes debitas , } sed dirigitur ad huiusmodi opera & Ca el cuerpo non puede enderesçat \textbf{ assi mismo a obras conueinbles } si non fuere enderescado atales obras \\\hline
2.3.13 & quasi corpus ad animam , \textbf{ sequitur eos esse naturaliter seruos . } Sunt enim aliqui carentes prudentia et intellectu , & assi commo el cuerpo al alma siguesse \textbf{ que aquellos sean naturalmente sieruos } Et por que algunos son menguados de entendimiento e de sabideria \\\hline
2.3.13 & ut canes , et equos in multis \textbf{ consequi salutem propter prudentiam hominum , } quam ex propria industria & assi commo los canes e los cauallos \textbf{ que en muchͣs cosas han salud | por la sabiduria de los omes } la qual non podrian auer \\\hline
2.3.13 & a rationis usu quam foeminae a viris , \textbf{ sequitur eos naturaliter esse subiectos . } Quare seruitus est & que las fenbras de los uarones \textbf{ por ende se sigue | que algunos sean naturalmente subietos e sieruos } Pot la qual cosa la piudunbre es en alguna manera cosa natural \\\hline
2.3.14 & propter commune bonum oportuit \textbf{ dare leges aliquas positiuas , } secundum quas regentur regna et ciuitates : & connino de dar \textbf{ e de fazer alg̃s leyes pointiuas } legunt las quales se gouernassen los regnos e las çibdades \\\hline
2.3.14 & secundum quam ignorantes debent seruire sapientibus , \textbf{ esset dare seruitutem legalem , } et quasi positiuam , & e sin sabiduria deuen puir a los sabios . \textbf{ es de dar serudunbre legal de ley puesta por los omes } segunt la qual los flacos e los vençidos \\\hline
2.3.14 & ( ut dicitur in Politic’ ) \textbf{ habere aliquem excessum respectu serui . } Huiusmodi autem excessus dupliciter esse potest , & en las politicas \textbf{ aya algua auentaia sobre el su sieruo . } Et esta auentaia puede ser en dos maneras \\\hline
2.3.14 & Videtur tamen huiusmodi iustum \textbf{ aliquo modo esse congruum , } si considerentur legum conditores . & mas es derech segunt prigon de ley . \textbf{ Enpero paresçe que este derecho en algua manera sea conuenible } si pararemos mientes alos establesçedores delas leyes . \\\hline
2.3.14 & sed ( ut ait ) non similiter esse facile , \textbf{ videre pulchritudinem animae , et corporis . } Secunda congruitas sumitur & Mas assi commo el dize non es semeiante cosa fazer paresçer la fermosura del alma \textbf{ e la fermosura del cuerpo ¶ } La segunda razon se toma dela defenssion delatrra . \\\hline
2.3.14 & si scirent se ex eis nullam utilitatem consecuturos ; \textbf{ sed cum cogitant eos acquirere in seruos , } reseruant ipsos propter utilitatem & soperiessen que nigunt pro non aurian de tal uençimiento . \textbf{ Mas quando pienssan que aquellos a quien vençe } que los gana \\\hline
2.3.15 & propter quod tales contingit \textbf{ esse naturaliter seruos , } ut est per habita manifestum . & Por la qual cosa conuiene \textbf{ que tales natraalmente sean sieruos } assi commo es manifiesto e prouado por las cosas ya dichͣs \\\hline
2.3.15 & Impotentes vero contingit \textbf{ esse ministros ex lege : } ut si qui in potentia deficientes ; & Mas aquellos que non son poderosos \textbf{ conuiene que sean ministros e siruientes por ley } assi commo si algunos fallesciessen enl poderio \\\hline
2.3.15 & Mercenarios vero contingit \textbf{ esse ministros ex conducto : } ille enim mercenarius dicitur , & Otrossi los merçenarios conuiene \textbf{ que sean ministros | por alquiler } por que aquel es dich merçenario \\\hline
2.3.15 & Principaliter tamen in ministerio debet \textbf{ quis intendere bonum : } si autem intendat & e en tal scruiçio \textbf{ deue cada vno entender algun bien } mas si entiende y auer alguna merçed \\\hline
2.3.15 & hoc debet esse ex consequenti . \textbf{ Oportuit autem dare ministrationem conductam et dilectiuam } praeter ministrationem naturalem & tenporal esto deue ser despues de aquel bien que entiende . \textbf{ Mas conuiene de dar a ministraçion de alquiler e de amor sin la ministt̃ion natural et segunt ley . } Ca por que en nos es el appetito corrupto \\\hline
2.3.15 & Rursus , quia contingit aliquando plures etiam ex nobili genere ortos toto tempore vitae suae \textbf{ non agere aliquod iustum bellum , } ut ex eo possent & que en todo tienpo de su uida \textbf{ non fazen ningua batalla iusta } por que por ella puedan ganar algunos seruientes e sieruos . \\\hline
2.3.15 & ut ex eo possent \textbf{ acquirere aliquos ancillantes et seruos : } ne ergo tales omnino priuentur ministris , & non fazen ningua batalla iusta \textbf{ por que por ella puedan ganar algunos seruientes e sieruos . } Et pues que assi es por que tales del todo non sean priuados de seruientes \\\hline
2.3.15 & ad supplendum indigentiam domesticam oportuit \textbf{ esse aliquos ministros conductos seruientes } intuitu mercedis , & conuiene para cunplimiento dela mengua dela casa \textbf{ que ouiessen algunos seruientes alquilados | que los seruiessen por el gualardon } e por la merçed \\\hline
2.3.15 & quos virtus et amor boni inclinat ad seruiendum , \textbf{ decet principantes se habere quasi ad filios , } et decet eos regere non regimine seruili , & e el amor de bien los inclina asuir . \textbf{ Conuiene que los prinçipes se ayan çerca ellos | assi commo cerca de fijos . } Et conuiene les alos prinçipes delos gouernar non \\\hline
2.3.16 & si debet esse ordinata , \textbf{ oportet reduci in unum aliquem , } a quo ordinetur . & En essa misma manera cada vna muchedunbre si bien ordenada es \textbf{ conuiene que sea aduchͣa vn ordenador } de quien ella sea ordenada . \\\hline
2.3.16 & praeficiendus est unus architector ministris illis , \textbf{ cuius sit solicitare et ordinare illos . } Est autem hoc documentum maxime necessarium & mayoral que sea ordenador e mandador de todos los seruientes \textbf{ a quien parte nezca de acuçiar | e de ordenar todos los otros } Et esta regla es muy neçessaria \\\hline
2.3.16 & eo quod unus non sufficeret \textbf{ exequi opus illud . } Est igitur in commissione officiorum & por que vno non cunpliria \textbf{ para fazer aquel oficio e aquella obra . } ¶ Pues que assi es en a comne dar estos ofiçios \\\hline
2.3.16 & non multi possunt \textbf{ praesidere in officiis et ubi officia commissa } non magnam curam habent annexam , & do non pueden muchs auer los ofiçios \textbf{ por la poquedat de los moradores . | Et do los ofiçios acomnedados } non han grand cura anexa \\\hline
2.3.16 & non magnam curam habent annexam , \textbf{ congregari possunt officia et magistratus , } ita quod eidem diuersa officia committantur . & non han grand cura anexa \textbf{ pueden se muchos ofiçios | e muchos maestradgos ayuntar en vno . } Assi que avna perssona sean acomnedados departidos ofiçios \\\hline
2.3.16 & ne per insipientiam defraudentur . \textbf{ Fidelitas autem cognosci habet per diuturnitatem : } ipsum enim cor hominis videre non possumus ; & por non saber . \textbf{ Mas la fiesdat se puede conosçer | por luengo tienpo } por que nos non podemos ver el coraçon del omne \\\hline
2.3.17 & et quia debita prouisio maxime videtur \textbf{ facere ad honoris statum , } ut instruantur Reges , et Principes , & e por que prouision conueible delas uestiduras \textbf{ mayormente parte nesçe a estado de onrra } por ende por que los Reyes e los prinçipes sean ensennados \\\hline
2.3.17 & cognoscatur \textbf{ eos esse unius Principis ministros . } Tertio circa prouisionem indumentorum & por que por la semeiança delas uestidas sea conosçidos \textbf{ que son seruientes de vn prinçipe¶ } Lo terçero çerca la prouision delas uestidas es de penssar la condiçion delas personas \\\hline
2.3.17 & Nam non omnes decet \textbf{ habere aequalia indumenta . } In tantis enim domibus & por que non conuiene que todos sean uestidos \textbf{ de eguales uestiduras caenta } grandescasas non solamente son legos mas avn aycłigos \\\hline
2.3.17 & eos aliter \textbf{ et aliter esse ordinatos . } Sic enim videmus & ø \\\hline
2.3.17 & Videmus enim communiter homines \textbf{ adeo affici ad patrias consuetudines , } et ad conuersationes regionis propriae , & mas de buenamente la veemos \textbf{ e por ende los ons en tanto lon mas inclinados | alas costunbres propreas dela su tierra } e alas conuersaconnes de su regno \\\hline
2.3.18 & quasi omnis virtus concomitari debet . \textbf{ Possumus enim distinguere duplicem nobilitatem : } unam secundum opinionem , & por que toda uirtud deue en algua manera ser aconpannada ala nobleza delas costunbres . \textbf{ Et nos podemos departir en dos maneras la nobleza ¶ } Vna segunt opinion de los omes \\\hline
2.3.18 & in populo progenitores suos fuisse pauperes , \textbf{ dicitur habere nobilitates generis , } et per consequens est nobilis & nin los sus auuelos fueron pobres \textbf{ e estos tales son dichos | auer nobleza de linage } e por ende se sigue \\\hline
2.3.18 & et secundum quem decet \textbf{ eos esse meliores aliis ; } si probabile est ex bonis bonos , & Et segunt el su estado conuienel es de ser meiores \textbf{ que los | otrossi cosa prouable es } que de los bueons nasçen buenos \\\hline
2.3.18 & qui sunt ex nobilibus natalibus orti dicuntur \textbf{ esse nobiles secundum opinionem , } quia opinio probabilitati innititur , & aquellos tales son dichos ser nobles \textbf{ segunt opinion de los omes } por que aquella opinion se funda en alguna nobleza \\\hline
2.3.18 & et tales ( ut patet per praehabita ) \textbf{ probabile est esse prudentes , et bonos . } Huic autem probabilitati aliquando subest falsitas , & assi commo paresçe por lo que dicho es \textbf{ Et por ende cosa prouable es | que ellos son sabios e buenons . } Enpero en esta opinion prouable algunas uezes \\\hline
2.3.18 & esse tales secundum veritatem , \textbf{ decens est nobiles genere esse nobiles secundum mores . } Ex hoc ergo curialitas venisse videtur . & Et por ende cosa conueinble es \textbf{ que los nobles | por linage sean nobles } por costunbres e desto paresçe \\\hline
2.3.18 & et quia decet nobiles \textbf{ et magnos esse nobiles secundum mores , } inde sumptum est , & e por que conuiene \textbf{ que los nobles e los guaades sean nobles en costunbres } dende fue tomado \\\hline
2.3.18 & inde sumptum est , \textbf{ ut dicantur esse curiales habentes mores nobiles : } propter quod curialitas morum & dende fue tomado \textbf{ que sean dichs curiales e corteses | los que han nobles costunbres } por la qual cosa la curialidat e la cortesia \\\hline
2.3.18 & quod est opus temperantiae : \textbf{ non dimittere aciem , } quod est opus fortitudinis : & que es obra de tenprança . \textbf{ Otrossi manda que los caualleros non descçian par en el az } que es obra de fortaleza \\\hline
2.3.18 & quod est opus temperantiae . \textbf{ Curiales etiam dicuntur homines se habere erga suos ciues , } si non eis iniuriam inferant in uxoribus , & nin torpemente la qual cosa es obra de tenprança . \textbf{ avn los omes son dicho | que } seancurialmente contra sus çibdadanos sinon les fezieren tuerto en las mugers \\\hline
2.3.18 & nec quod ex hoc velint \textbf{ implere legem hoc precipientem , } quod facit iustus legalis : & nin otrossi non lo faze \textbf{ por que quiera cunplir la ley } que lo manda la qual cosa faze el iusto legal . \\\hline
2.3.18 & quod facit iustus legalis : \textbf{ sed quia volunt retinere mores curiae et nobilium , } quos decet datiuos esse ; & que lo manda la qual cosa faze el iusto legal . \textbf{ Mas por que el quiere retener las costunbres dela corte | e de los no nobles omes alos } que les conuienne de ser dadores \\\hline
2.3.19 & Nam sicut decet ciues \textbf{ ut debitam politiam seruent } esse iustos legales , & ca assi conmo conuiene alos çibdadanos de ser iustos e legales \textbf{ para guardar su poliçia conueniblemente } assi conuiene alos sermient \\\hline
2.3.19 & in commissis officiis solicitandi . \textbf{ Per se enim ipsos habere curam } et solicitudinem de ministris , & en qual manera son de acuçiar \textbf{ por que cunplan bien sus ofiçios } que les son acomnedados \\\hline
2.3.19 & esse \textbf{ magnanimos decet operari pauca et magna , } ut decet ipsos solicitari & ¶ Et pues que assi es alos Reyes \textbf{ e alos prinçipes alos quales couiene de auer altos coraçones | conuiene les de obrar pocas cosas } e grandes ca les conuiene \\\hline
2.3.19 & in tertio Libro patebit . \textbf{ Solicitari vero circa quosdam ministros } et velle se de quibuscumque inimicis intrommittere , & mas non conuiene alos reyes e alos prinçipes de ser acuçiosos \textbf{ çerca quales si quier ofiçiales | nin cerca de sus ofiçios } nin se deuen entremeter \\\hline
2.3.19 & nullatenus decet ipsos . Hoc viso restat \textbf{ ostendere tertium , } videlicet qualiter & ca esto ꝑtenesçe alos menores . \textbf{ ¶ Esto iusto finça de demostrar lo terçero } que es en qual manera han de beuir los sennores con sus ofiçiales \\\hline
2.3.19 & sed ad eos qui sunt in dignitatibus decet \textbf{ magnanimos ostendere se magnos . } Reges ergo et Principes , & mas a aquellos que son en grandes dignidades \textbf{ los magnanimos se deuen mostrar grandes . } ¶ Et pues que assi es los Reyes \\\hline
2.3.19 & qui respectu eorum sunt inferiores et humiles , \textbf{ debent se ostendere moderatos : } quia erga eos velle se habere & tonprados a los sus seruientes propreos \textbf{ los quales en conparacion dellos son humillosos e baxos } por que contra ellos non se deuen mostrar en grand alteza \\\hline
2.3.19 & omnino enim decet Reges et Principes \textbf{ minus se exhibere quam caeteros , } et ostendere se esse personas magis graues & e alos prinçipes \textbf{ de se faz menos familiares | que los otros } e de se mostrar \\\hline
2.3.19 & minus se exhibere quam caeteros , \textbf{ et ostendere se esse personas magis graues } et reuerendas quam alios , & que los otros \textbf{ e de se mostrar | que son personas mas pesadas } e de mayo rreuerençia \\\hline
2.3.19 & si per diuturnitatem temporis constet \textbf{ ipsos esse beniuolos , } fideles , et prudentes , & e por dilectonn del prinçipe \textbf{ si fueren çiertos | por tpoluengo } que ellos son beniuolos e fieles e sabios atales podran descobrir sus poridades \\\hline
2.3.20 & Dicto quales debent \textbf{ esse ministri Regum et Principum , } et qualiter Reges et principes debeant se habere de ipsis : & ich quales deuen ser los seruientes de los Reyes \textbf{ e de los prinçipes | e en qual manera los Reyes } e los prinçipes se de una auer \\\hline
2.3.20 & restat ut dicamus qualiter in mensis . \textbf{ Principum circa eloquia habere se debeant } tam ipsi Reges et Principes & a ellos finca \textbf{ que digamos en qual manera en las mesas de los prinçipes se de una auer } enl fablar tan bien los Reyes \\\hline
2.3.20 & quos decet maxime temperatos esse , \textbf{ et obseruare ordinem naturalem } omnino in suis mensis , & e los prinçipeᷤ alos quales conuiene ser muy tenprados \textbf{ e guardar la orden natural en toda } meranera deuen ordenar en sus mesas \\\hline
2.3.20 & etiam \textbf{ et ipsos participare virtutes et bonos mores . } Sed si recumbentes , & Mas alos que son assentados en las mesas \textbf{ conuiene de escusar muchedunbre de palabras } por que non sea tirada la ordenn natural \\\hline
2.3.20 & et ne intemperati appareant , \textbf{ decet in mensis vitare sermonum multitudinem , } decet etiam hoc ipsos ministrantes , & e por que non parezcan destenprados \textbf{ assi commo dicho es . } Avn esto mismo conuiene alos seruientes por que la orden e la manera del seruir \\\hline
3.1.1 & Quoniam omnem ciuitatem contingit \textbf{ esse communitatem quandam , } cum omnis communitas fit & e todo cunplimiento ha de ser \textbf{ or que toda çibdat conuiene que sea alguna comunindat } commo toda comunidat sea por graçia de algun bien . \\\hline
3.1.1 & gratia alicuius boni , \textbf{ oportet ciuitatem ipsam constitutam esse propter aliquod bonum . } Probat autem Philosophus primo Polit’ duplici via , & commo toda comunidat sea por graçia de algun bien . \textbf{ Conuiene que la çibdat sea establesçida por algun bien | Ca pruena el pho } enl primero libro delas politicas \\\hline
3.1.1 & Aduertendum tamen , \textbf{ communitatem ciuitatis esse principalissimam } non simpliciter et per omnem modum , & Enpero conuiene de saber \textbf{ que la comuidat dela çibdat } non es la mas prinçipal sinplemente \\\hline
3.1.1 & utilem esse in vita humana , \textbf{ et esse principaliorem communitate ciuitatis . } Videtur enim suo modo communitas regni & que la comunidat del regno es prouechosa en la uida humanal \textbf{ e es mas prinçipal | que la comunidat dela çibdat ca paresçe } que assi se ha la comunidat del regno \\\hline
3.1.2 & quam communitas illa . \textbf{ Non sufficit dicere ciuitatem constitutam } esse gratia alicuius boni , & que la comunidat dela çibdat \textbf{ on a basta de dezer | que la çibdat es establesçida } por gera de algun bien \\\hline
3.1.2 & et aliud virtuose viuere . \textbf{ Nam esse latissimum est , } ut dicitur in libro de Causis : & e otra cosa es beuir uirtuosamente \textbf{ por que el seres cosa muy general | e muy ancha } assi commo es dicho en el libro de causis \\\hline
3.1.2 & aliqua perfectio competens suae speciei , \textbf{ licet possit habere illa res esse aliquod , } ut imperfectum esse : & que pertenezca ala suspeno ala su semeiança \textbf{ commo quier que puede auer aquella cosa } aquel ser menguado en alguna manera . \\\hline
3.1.2 & non tamen dicitur sufficienter viuere , \textbf{ et habere vitam sufficientem , } nisi habeat ea quae congrue sufficiunt & que el omne aya el ser biue enpero non es dicho beuir conplidamente \textbf{ e auer uida conplida } si non ouiere aquellas cosas \\\hline
3.1.3 & Videmus autem multos societatem politicam retinentes , \textbf{ eligere solitariam vitam , et campestrem . } Sed hae et aliae dubitationes & que muchos fuyen la conpanna politicas \textbf{ e çiuil | e escogen uida solitaria e montanensa } assi commo los hermitaons . . \\\hline
3.1.3 & aliqui enim sic habent \textbf{ appetitum corruptum et voluntatem peruersam , } quod nequeunt viuere in societate & ca algunos omes assi han el appetito corrupto \textbf{ e la uoluntad desordenada } que non pue den beuir en conpannia e segunt ley . \\\hline
3.1.4 & per quas probari uidebatur , \textbf{ ciuitatem non esse aliquid secundum naturam , } et hominem non esse naturaliter animal ciuile . & por las quales se prouaua \textbf{ que la çibdat non era cosa natural } e que el oen non era naturalmente aianlçiuil . \\\hline
3.1.4 & ciuitatem non esse aliquid secundum naturam , \textbf{ et hominem non esse naturaliter animal ciuile . } Cum ergo non satis sit & que la çibdat non era cosa natural \textbf{ e que el oen non era naturalmente aianlçiuil . } Et pues que assi es commo non cunpla soluer la \\\hline
3.1.4 & Cum ergo non satis sit \textbf{ remouere omnes obiectiones } contrarias veritatem aliquam impugnantes , & e que el oen non era naturalmente aianlçiuil . \textbf{ Et pues que assi es commo non cunpla soluer la } sobiecconnes contrarias \\\hline
3.1.4 & per quas illa ueritas confirmetur : \textbf{ intendimus in hoc capitulo adducere rationes ostendentes ciuitatem esse quid naturale , } et hominem esse naturaliter animal ciuile . & por las quales la uerdat sea confirmada . \textbf{ Por ende entendemos en este capitulo de adozir razones | que muestren } que la çibdat es cosa natural \\\hline
3.1.4 & intendimus in hoc capitulo adducere rationes ostendentes ciuitatem esse quid naturale , \textbf{ et hominem esse naturaliter animal ciuile . } Possumus autem duplici uia ostendere & que muestren \textbf{ que la çibdat es cosa natural | e que el omne es naturalmente aianlçiuil } e podemos por dos razones mostrat \\\hline
3.1.4 & communitatem politicam \textbf{ siue ciuitatem esse aliquid secundum naturam . } Prima uia sumitur & que la comiundat politicas \textbf{ o la çibdat es algunan cosa segunt natura . } ¶ La primera razon se tomadesto \\\hline
3.1.4 & Probabatur enim supra , \textbf{ quod uiuere erat homini secundum naturam ; } ut natura non deficiat in necessariis , & ¶Lo primero se prueua \textbf{ assi ca fue prouado dessuso | que beuir es cosa natural al omne } e por que la nafa non pueda fallesçer en las cosas neçessarias \\\hline
3.1.4 & quam communitates illae , \textbf{ oportet eam esse secundum naturam . } Secunda uia ad inuestigandum hoc idem , & que estas dos comuidades \textbf{ por ende conuiene | que la çibdat lea comuidat natraal ¶ } La segunda razon para prouar \\\hline
3.1.4 & tanquam ad finem et complementum , \textbf{ ordinatur ad ciuitatem . Viso , ciuitatem esse aliquid secundum naturam : } reliquum est ostendere , & assi commo assu fin e asu conplimiento . \textbf{ Et pues que assi es iusto | que la çibdat es cosa natural . } finca de demostrar \\\hline
3.1.4 & reliquum est ostendere , \textbf{ hominem esse naturaliter animal politicum et ciuile , } quod etiam duplici via inuestigare possumus . & finca de demostrar \textbf{ que el omne es naturalmente | aianl politicas } e ciuilla qual cosa podemos demostrar \\\hline
3.1.4 & Hoc autem ex parte sermonis ostendere possumus \textbf{ hominem esse naturaliter animal politicum et ciuile , } ex eo quod vox humana , & aqui deꝑte dela palabra podemos demostrar \textbf{ que el omne es natraalmente | aianl politicas e ciuil } por que la boz del omne \\\hline
3.1.4 & alteri cani per suum latratum \textbf{ significare tristitiam , } vel delectationem quam habet . & quando se trista puede a otro can de mostrar \textbf{ por su ladrado sutsteza o su delecta conn que ha mas al omne } sobre todo esto le es dada la palabra \\\hline
3.1.4 & oportet communitatem domesticam \textbf{ et ciuilem esse quid naturale . } Nam si natura dedit homini sermonem , & que la comunidat dela casa \textbf{ e la comunidat dela çibdat sean cosas naturales } ca si la natura dio al omne palabra natural aquella comunidat \\\hline
3.1.4 & quae ordinatur ad illa , \textbf{ quae sunt apta nata exprimi per sermonem : } iustum enim et iniustum non proprie habet & que se han de demostrar \textbf{ conueinblemente | por la palabra conuiene } que sea natural \\\hline
3.1.4 & oportet ciuitatem \textbf{ esse quid naturale , } vel esse aliquid secundum naturam . & conuiene quela çibdat sea cosa natural \textbf{ e sea alguna cosa segunt natura } e podemos mostrar por tres razones \\\hline
3.1.4 & esse quid naturale , \textbf{ vel esse aliquid secundum naturam . } Possumus autem triplici via ostendere , & conuiene quela çibdat sea cosa natural \textbf{ e sea alguna cosa segunt natura } e podemos mostrar por tres razones \\\hline
3.1.5 & quod semper oporteat ciuitatem \textbf{ ex propriis possessionibus habere omnia quae requiruntur ad vitam : } sed sufficit ciuitatem sic esse sitam , quod per mercationes , & para aquellas cosas \textbf{ que son menester ala uida | mas cunple } que assi sea la çibdat establesçida \\\hline
3.1.5 & ex propriis possessionibus habere omnia quae requiruntur ad vitam : \textbf{ sed sufficit ciuitatem sic esse sitam , quod per mercationes , } et ponderis portatiuam , & mas cunple \textbf{ que assi sea la çibdat establesçida | que por mercadores } e por acarreo trayendo cargas \\\hline
3.1.5 & in eadem ciuitate \textbf{ congregari diuersos vicos , } ut facilius habeantur & assi commo cosa apuechosa es ala uida humanal \textbf{ que en vna çibdat sean ayuntados muchos uarrios } por que mas ligeramente e meior se puedan acoirer los vnos alos otros \\\hline
3.1.5 & Immo tanto principalius debet \textbf{ intendere hoc quam illud , } quanto anima est potior corpore , & e uirtuosamente ante el que faze la ley \textbf{ cantomas prinçipalmente deue tener mientes a esto } que aquello quante el alma es meior que el cuerpo \\\hline
3.1.5 & Oportet ergo rectores ciuitatis \textbf{ habere ciuilem potentiam , } ut possint cogere et punire & Pues que assi es conuiene \textbf{ que los gouernadores de la çibdat ayan poderio çiuil } por que puedan costrennir e fazer iustiçia \\\hline
3.1.5 & Quare cum peruersi in ciuitate aliqua \textbf{ non audeant insurgere contra principem , } si sciant ipsum magnam habere ciuilem potentiam , & por la qual cosa commo los malos en alguna çibdat \textbf{ non se osenle unatat | contra el prinçipesi sopieren } que el ha grant poderio en la çibdat \\\hline
3.1.5 & non audeant insurgere contra principem , \textbf{ si sciant ipsum magnam habere ciuilem potentiam , } et dominare in ciuitatibus multis , & contra el prinçipesi sopieren \textbf{ que el ha grant poderio en la çibdat } e que han grant señorio en muchͣs cibdades \\\hline
3.1.5 & ut melius possit \textbf{ resistere impugnationem hostium : } cum ergo regnum sit & por que pueda meior cotra dezir \textbf{ e con tristar alos enemigos | que la conbaten . } ¶ Et pues que assy es commo el regno sea \\\hline
3.1.5 & cuius est quemlibet partem regni defendere , \textbf{ et ordinare ciuilem potentiam aliarum ciuitatum } ad defensionem cuiuslibet ciuitatis regni ; & so vn Rey aqui pertenesçe de defender a cada vna parte del regno \textbf{ e ordener el poderio çiuil delas otras çibdades } a defendimiento de cada vna delas çibdades del regno \\\hline
3.1.6 & magis pacifice viuere , \textbf{ et magis resistere hostibus volentibus impugnare ipsos . } Est enim huius impetus naturalis : & por que por tal establesçimiento pueden beuir mas en paz \textbf{ e pueden mas defender se de los enemigos | que les quieren mal fazer } et esta tal inclinaçion es natural \\\hline
3.1.6 & poterat \textbf{ quis insurgere per tyrannidem } et per violentiam , & e cada vno podie le una tarse \textbf{ por tirania } e por fuerça e ser sennar dellos . \\\hline
3.1.6 & et ut facilius eis dominaretur , \textbf{ poterat eos uiolenter congregare in unum , } et constituere inde ciuitatem . & Et por que mas ligeramente \textbf{ sennoreasse sobre ellos podrie ayuntarlos en vno | por fuerça } e establesçer ende çibdat . \\\hline
3.1.6 & et faciat se Regem constitui super illas . \textbf{ Viso diuersos esse modos generationis ciuitatis et regni , } restat uidere & ¶ Et pues que assi es visto \textbf{ que son maneras departidas | de establesçimiento de çibdat e de Regno . } finca de ver en quantas partes conuiene \\\hline
3.1.6 & et ad pacifice uiuere , \textbf{ et ad resistendum uolentibus turbare pacem , } et impugnare ciues . & otdenadonatanlmente a bien beuir e a beuir en paz \textbf{ e para yr | contra los que quisieren turbar la paz } e qualieren lidiar contra los çibdadanos \\\hline
3.1.6 & Ostendendum est ergo qualiter possit \textbf{ bene regi ciuitas siue regnum tempore pacis , } et qualiter impugnandi sint hostes tempore belli . & qual es la meior manera de gouernamiento de çibdat e de regno \textbf{ e en qual manera se puede bien gouernar la çibdat | e el regno en tp̃o de paz } e en qual manera deuemos lidiar contra los enemigos entp̃o de guerra . \\\hline
3.1.6 & quomodo debeant \textbf{ regere ciuitates et regna . } Secundo ostendetur , & por que por el conosçimiento dellos sean endozidos los Reyes e los prinçipes \textbf{ en qual manera de una gouernar las çibdades e los regnos ¶ } Lo segundo mostraremos qual es la muy buean politica o çibdat o muy vuen regno \\\hline
3.1.7 & arguitur esse summe bonus . \textbf{ Videtur ergo ciuitas esse potissime bona , } si sit potissime una ; & que dios es muy bueno \textbf{ e por ende paresçe que la çibdat es muy | buenasi fuere muy vna . } Et pues que assi es quanto mas se allega a vnidat \\\hline
3.1.7 & reputarent \textbf{ quemlibet puerorum esse filium proprium . } Videbatur enim Socrati et Platoni totam dissensionem ciuium consurgere & mas cuydarian de cada vno moço \textbf{ que era su fijo propreo } por que paresçia a socrates e a platon \\\hline
3.1.7 & antiquiores reputarent se \textbf{ habere maximam unitatem } cum iunioribus , & e grant ayuntamiento de los padres alos fijos los mas antiguos \textbf{ cuydarian | que auian muy grant vnidat con los moços } e esso mismo los mocos cuydarian \\\hline
3.1.7 & et quod crederent \textbf{ eos esse suos filios , } illi vero opinarentur & por que los antiguos creerian \textbf{ que los moços eran sus fiios } e los mocos cuydarian \\\hline
3.1.7 & illi vero opinarentur \textbf{ eos esse suos patres . } Tertium vero quod senserunt & e los mocos cuydarian \textbf{ que ellos eran sus padres . } ¶ Lo terçero que sintieron los dichs philosofos cerca el gouernamiento dela çibdat . \\\hline
3.1.7 & est , quia dixerunt ciuitatem \textbf{ quamlibet diuidendam esse in quinque partes , } videlicet in agricolas , artifices , bellatores , consiliarios , et principem . & que dixieron \textbf{ que cada vna çibdat | demaser partida en çinco partes . } Conuiene a saber en labradores \\\hline
3.1.7 & ( si bene ordinata erat ) \textbf{ ad minus deberet habere mille bellatores , } et ad plus quinque millia . & que si la çibdat fuesse bien ordenada \textbf{ a todo lo menos deuia auer milł batalladores o caualleros } e alo mas çinco milł . \\\hline
3.1.8 & secundum suum statum sit maxime perfectum , \textbf{ oportet ibi dare diuersa secundum speciem . } Nam quia tota bonitas uniuersi non potest & e por que el mundo segunt su estado sea muy acabado \textbf{ conuietie de dar en el | departidas speçias } e departidas semeianças \\\hline
3.1.8 & reseruari in una specie , \textbf{ oportet ibi dare species diuersas ; } ut in pluribus speciebus entium reseruetur maior perfectio , & nin en vna semeiança \textbf{ conuiene de dar | y deꝑ tidas espeçies } e departidas semeianças \\\hline
3.1.8 & vel in regno esse debere omnem unitatem , \textbf{ est dicere ciuitatem } non esse ciuitatem , & Et pues que assi es dezer que enla çibdat o en el regno deua ser tan grant vnidat \textbf{ commo dizian socrates e platones dezer que la çibdat non sea çibdat } e el regno non sea regno . \\\hline
3.1.8 & est dicere ciuitatem \textbf{ non esse ciuitatem , } et regnum non esse regnum . & Et pues que assi es dezer que enla çibdat o en el regno deua ser tan grant vnidat \textbf{ commo dizian socrates e platones dezer que la çibdat non sea çibdat } e el regno non sea regno . \\\hline
3.1.8 & non esse ciuitatem , \textbf{ et regnum non esse regnum . } Secunda via ad inuestigandum hoc idem , & commo dizian socrates e platones dezer que la çibdat non sea çibdat \textbf{ e el regno non sea regno . } La segunda razon para prouar esto mismo se toma \\\hline
3.1.8 & et auditus ideo oportet \textbf{ ibi dare diuersa membra exercentia diuersos actus : } sic quia ad indigentiam vitae & y departidos mienbros \textbf{ que fagan estas obras departidas . } En essa misma manera por que para conplir la mengua dela uida \\\hline
3.1.8 & oportet in ciuitate \textbf{ dare diuersitatem aliqua , } ut in ea reperiatur sufficientia ad vitam . & auemos mester casas e uestid̃as e viandas e otras cosas tales \textbf{ por ende conuiene de dar algun departimiento en la çibdat por que en ella sean falladas todas las cosas } que cunplen ala uida . \\\hline
3.1.8 & ut cum in ciuitate oporteat \textbf{ dare aliquos magistratus , } et aliquas praeposituras , & delos çibdadanos a algun prinçipe o algun sennor \textbf{ e commo en la çibdat conuenga de dar alguons ofiçioso } alguons maestradgos o algunas alcaldias \\\hline
3.1.8 & oportet in ciuitate \textbf{ dare diuersitatem aliquam . } Quinta uia sumitur & por ende commo estas cosas demanden departimiento \textbf{ conuiene de dar en la çibdat algun departimiento . } La quanta razon se toma \\\hline
3.1.8 & Nam finis ciuitatis est bene viuere , \textbf{ et habere sufficientiam in vita ; } nam ciuitas est communitas & por conpaçion dela finca la fin dela çibdat es bien beuir \textbf{ e auer abastamiento en la uida } ca la çibdat escomunidat \\\hline
3.1.8 & ideo oportet ciuitatem \textbf{ habere aliquam diuersitatem in se , } et diuersos habere vicos , & para abastamiento deuida son meester muchͣs cosas departidas \textbf{ por ende conuiene enla çibdat de auer en ssi algun departimiento } e de auer departidos uarrios \\\hline
3.1.8 & habere aliquam diuersitatem in se , \textbf{ et diuersos habere vicos , } ut expediens ad vitam & por ende conuiene enla çibdat de auer en ssi algun departimiento \textbf{ e de auer departidos uarrios } assi que abonden a la uida \\\hline
3.1.8 & sed ad rectam consonantiam oportet \textbf{ ibi dare diuersitatem tonorum . } Sic pictura nunquam est bene ordinata , & mas ala derecha consonançia delas bozes \textbf{ conuiene de dar y departimiento de los tonos } assi commo la pintura non es bien ordenada \\\hline
3.1.8 & Decet ergo hoc Reges , et Principes cognoscere , \textbf{ quod nunquam quis bene nouit regere ciuitatem , } nisi sciuerit qualiter constituitur ; & e alos prinçipes de sabesto \textbf{ por que munca ninguno sopo bien gouernar çibdat } si non sopiere en qual manera es establesçida la çibdat \\\hline
3.1.9 & et quasi quaedam praeambula ad sequentia . Volumus autem in hoc capitulo ostendere , \textbf{ quod non expedit ciuitati habere omnia communia } ut Socrates ordinauit : & mas nos queremos en este capitulo mostrar primeramente \textbf{ que conuiene ala çibdat | quer todas las cosas comunes } assi commo socrates ordeno . \\\hline
3.1.9 & cessarent litigia , \textbf{ quia crederent ciues omnes pueros esse filios suos , } et sic esset in ciuitate maximus amor . & e las contiendas enla çibdat \textbf{ por que cuydarian los çibdadanos | que todos los moços eran sus fijos propreos } e por ende en la çibdat seria muy grant amor Et pues que assi es nos podemos mostrar \\\hline
3.1.9 & Immo quia impossibile est omnes ciues \textbf{ aequaliter esse prudentes et bonos , } et esse aequaliter utiles ciuitati , & qua non puede ser \textbf{ que todos los çibdadanos sean egualmente sabios } e todos sean bien egualmente prouechosos ala çibdat \\\hline
3.1.9 & quia quilibet crederet \textbf{ plus esse accepturum : } ut dum unus ciuis iudicaret & por que cada vno cuydaria \textbf{ que deuia mas resçebir de } quanto resçibe camientra \\\hline
3.1.9 & uellet secundum dignitatem suam \textbf{ ei fieri retributionem . } Hanc autem aequalitatem non de facili esset possibile reseruari inter ciues & por meior que el otro quarria \textbf{ que segunt la su dignỉdat le diessen mayor gualardo delas cosas comunes } mas esta egualdat non se podria guardar de ligero \\\hline
3.1.9 & non oporteret ciues \textbf{ omnes pueros reputare filios proprios . } Immo quia puerorum aliqui essent & assi comunes non conuernia \textbf{ que los çibdadanos cuydassen | que todos los moços fuessen sus fijos propreos } por que alguons de los moços son semeiantes \\\hline
3.1.9 & quilibet ciuis appropriaret sibi in filium , \textbf{ quem videret sibi esse similem . } Unde et Philosophus narrat 2 Politicor’ & e cada vno de los çibdadanos apropriaria \textbf{ assi por fijo a aquel que viesse | que lo semeiaua . } Onde el pho cuenta en el segundo libro delas politicas \\\hline
3.1.9 & Videlicet , ut ciues omnes pueros crederent \textbf{ esse proprios filios . } Tertia via sic patet . & que todos los çibdadanos creyessen \textbf{ que todos los moços serian sus fijos propreos } La terçera razon paresçe \\\hline
3.1.9 & si firmiter crederet \textbf{ ipsum esse talem ; } quam pater filium , & si creyesse uerdaderamente \textbf{ que era su nieto } que el padre amaria al fijo \\\hline
3.1.10 & Nam oportet in ciuitate \textbf{ consurgere lites , vulnerationes , et contumelias , } quae tanto detestabiliores sunt , & El quinto es abusion de los parientes ¶ Lo primero paresçe assi ca conuiene \textbf{ que en la çibdat se leunatenlides e feridas e deniestos } las quales cosas tanto son \\\hline
3.1.10 & propter honestatem \textbf{ et bonitatem morum parentes esse certos de suis filiis , } et quoslibet certificari de eorum consanguineis , & por bondat e honestad de costunbres \textbf{ que los padres sean çiertos de sus fiios } e cada vnos sean çiertos de sus parientes \\\hline
3.1.10 & et exaltare ignobiles , \textbf{ et non saluare amicitiam inter eos . } Tertium malum sic declaratur . & e enxalcar los viles \textbf{ e assi se salua la amistança entre ellos | ¶ } El terçero mal se declara \\\hline
3.1.10 & Quare si quilibet ciuis crederet \textbf{ quemlibet puerorum esse proprium filium , quia partiretur eius amor in tantam multitudinem , } modicum diligeret unumquemque , & por la qual cosa si cada vno de los çibdadanos cuydasse que cada vno de los moços era su fijo propreo \textbf{ por que el amor del se partia en tanta muchedunbre de fijos } muy \\\hline
3.1.10 & nullo modo suspicari posset \textbf{ omnes pueros esse suos filios . } Si ergo omnes diligerent tanquam filios , & si non fuesse loco en ninguna manera non podria sospechͣr \textbf{ que todos los moços fuessen sus fijos . } Et pues que assi es si a todos amassen \\\hline
3.1.10 & hoc esset ratione duorum vel trium puerorum , \textbf{ quos crederent esse proprios filios : } et quia illi non essent eis certitudinaliter noti , & por razon de dos o tres moços \textbf{ los quales cuydaria que eran suᷤ fijos propreos } e por que aquellos nonl serian conosçidos çiertamente \\\hline
3.1.10 & propter duos vel tres \textbf{ vel propter paucos pueros velle magnam multitudinem diligere puerorum tanquam proprios filios , } hoc est ponere parum de melle in multa aqua . & Mas esto reprahende el philosofo enel segundo libro delas politicas \textbf{ ca por dos o por tres o por pocos mocos querera mar grant muchedunbre de moços | assi conmo a fijos propreos } esto es poner poco de miel en muchͣ agua . \\\hline
3.1.10 & vel propter paucos pueros velle magnam multitudinem diligere puerorum tanquam proprios filios , \textbf{ hoc est ponere parum de melle in multa aqua . } Sicut ergo parum mellis totum unum fluuium & assi conmo a fijos propreos \textbf{ esto es poner poco de miel en muchͣ agua . } Et pues que assi es \\\hline
3.1.10 & Sicut ergo parum mellis totum unum fluuium \textbf{ non posset facere dulcem , } sic amor duorum & assi commo poca miel puesta en vn grant rio \textbf{ non puede fazer todo el rio dulçe } assi amor de dos o de tro fiios \\\hline
3.1.10 & innumerabilem multitudinem puerorum existentium in ciuitate una , \textbf{ non posset reddere placibilem et dilectam . } Sed non existente dilectione ciuium ad pueros , & de que son en vna çibdat \textbf{ nin puede fazer | que aquella muchedunbre sea plazible } e amada non estando el amost de los çibdadanos alos moços \\\hline
3.1.10 & adhuc est valde difficile debite \textbf{ et temperate se habere erga illam . } Sicut ergo prouocata gula & que avn que el o en non ouiesse si non vna mugnia vn seria cosa guaue que el se ouiesse conueinblemente \textbf{ e tenpradamente en vsar della } e por ende assi commo la garganteria \\\hline
3.1.10 & quod spectabat ad Principem ciuitatis \textbf{ habere curam et diligentiam , } ne filii coirent cum matribus , & Empero socrates quariendo escusar este mal dix̉o \textbf{ que al prinçipe dela çibdat pertenesçia de auer cuydado e acuçia } por que los fijos non yoguiessen con sus madres \\\hline
3.1.10 & Decet ergo Reges et Principes \textbf{ sic ordinare ciuitatem , } ut prohibita communitate foeminarum et uxorum certificentur parentes de propriis filiis . & Et pues que assi es conuiene alos Reyes \textbf{ e alos prinçipes de ordenar assi la çibdat } por que defendia la comunidat delas fenbras e delas mugeres casadas \\\hline
3.1.11 & Nam in rebus deseruientibus \textbf{ ad victum est considerare duo : } videlicet res fructiferas , & que siruen ala uida \textbf{ e ala uianda del omne | auemos de penssar dos cosas } Conuiene a saber las cosas \\\hline
3.1.11 & et firmiter credunt \textbf{ se esse tanta consanguinitate coniunctos , } multa habent litigia , & pocoscreen firmemente \textbf{ que son ayuntados en tan grant parentesco } e han muchas contiendas \\\hline
3.1.11 & propter communitatem mulierum et uxorum crederent \textbf{ se esse consanguinitate coniunctos ; } attamen inter eos & e delas mugers creyessen \textbf{ que eran ayuntados | por mayor parentesco . } Enpero segunt uerdat entre ellos \\\hline
3.1.11 & multo magis infra huiusmodi dissensionem \textbf{ non posset tollere inter multos , } ut inter omnes ciues . & que non es çierto \textbf{ non podria tirar tales varaias entre muchos } assi commo entre los çibdadanos . \\\hline
3.1.11 & eo quod oporteat eos valde ad inuicem conuersari , \textbf{ ostenditur ut plurimum homines habere lites et iurgia } propter quod Philosophus ait 2 Polit’ & por que han de beuir en vno \textbf{ que por la mayor parte han contiendas e uaraias por la qual cosa dize el philosofo } en el segundo libro de las politicas \\\hline
3.1.11 & Cum ergo custodes ciuitatis nobiliores sint agricolis , \textbf{ tanquam meliores et nobiliores estimarent se plus esse accepturos } de fructibus possessionum , & e los defendedores dela çibdat sean mas nobles que los labradores \textbf{ assi commo meiores | e mas nobles cuydaria } que aurian de reçebir mas de los fructos delas possessiones que los labradores \\\hline
3.1.11 & ne inter ciues oriantur dissensiones et iurgia , \textbf{ non sic esse possessiones communes , } ut Socrates statuebat . & por que non nazcan entre los çibdadanos uarias e contiendas \textbf{ qua non sean las possessio nes } assi comunes commo establesçio socrates mas lamzon delpho \\\hline
3.1.11 & adhibebit \textbf{ debitam diligentiam circa illa . } Expedit autem talia esse communia secundum liberalitatem : & ca cada vn sennor de sus bienes propreos aura mayor acuçia de aquellos bienes \textbf{ que si fuessen comunes } mas conuiene que las cosas sean comunes \\\hline
3.1.11 & debitam diligentiam circa illa . \textbf{ Expedit autem talia esse communia secundum liberalitatem : } quia cives inter se debent liberales esse , & que si fuessen comunes \textbf{ mas conuiene que las cosas sean comunes | segunt uirtud de franqueza } por que los çibdadanos entre ssi deuen ser francos ꝑtiendo sus bienes entre ssi . \\\hline
3.1.12 & Homines enim bellatores decet \textbf{ esse mente cautos et prouidos : } corde viriles et animosos : & ca los omes lidiadores conuiene que sean cuerdos \textbf{ por entendimiento e sabios } e sean rezios e esforcados de coraçon e fuertes e ualientes de cuerpo \\\hline
3.1.12 & quae requiritur in bellantibus , \textbf{ arguere possumus mulieres instruendas non esse ad opera bellica . } Secunda via ad inuestigandum hoc idem , & que es meester en las batallas \textbf{ podemos tomar argumento | que las mugers non son de enssennar } nin de pouer alas batallas \\\hline
3.1.12 & quam eos in societate habere \textbf{ nam cum humanum sit timere mortem , } viriles etiam et animosi trepidant & que auerlos en su conpannia \textbf{ ca commo todos los omes | teman la muerte los esforçados } e de grandes coraçones temen \\\hline
3.1.12 & ne igitur reddantur bellantes pusillanimes , \textbf{ quos constat esse timidos oportet } ab exercitu expelli . & Et por ende por que los lidiadores non se enflaquezcan en las batallas \textbf{ conuiene de echar dela batalla } e dela fazienda alos de flaco \\\hline
3.1.12 & sustinere armorum pondera , \textbf{ et dare magnos ictus , } expedit eos habere magnos humeros et renes & ca commo los lidiadores ayan de sofrir el peso delas armas \textbf{ e ayan de dar grandes colpes } conuieneles de auer fuertes honbros e fuertes rennes \\\hline
3.1.12 & et dare magnos ictus , \textbf{ expedit eos habere magnos humeros et renes } ad sustinendum armorum grauedinem , & e ayan de dar grandes colpes \textbf{ conuieneles de auer fuertes honbros e fuertes rennes } para sofrir la pesadura delas armas \\\hline
3.1.12 & ad sustinendum armorum grauedinem , \textbf{ et habere fortia brachia } ad faciendum percussiones fortes : & para sofrir la pesadura delas armas \textbf{ e conuiene les de auer fuertes braços } para fazer fuertes colpes . \\\hline
3.1.12 & Hominis ergo est \textbf{ secundum debitam oeconomiam } et secundum debitam dispensationem ordinare domum et ciuitatem . & Et pues que assi es los omes \textbf{ a quien parte nesçe } de ordenar la casa \\\hline
3.1.12 & secundum debitam oeconomiam \textbf{ et secundum debitam dispensationem ordinare domum et ciuitatem . } Quare in iis , & a quien parte nesçe \textbf{ de ordenar la casa | e la çibdat } segunt ordenamiento conueinble en aquellas cosas \\\hline
3.1.13 & et praeposituras distribuere , \textbf{ cognoscere quales praeficiant praepositos et magistros ; } si principatus et magistratus & e partir los maestradgos e las diguidades \textbf{ de conosçer | quales pone en los prinçipados e enlos maestradgos } Si los prinçipados e los maestradgos \\\hline
3.1.13 & et uniuersaliter quaelibet praepositura virum ostendit et manifestat , \textbf{ expedit tribuentem praeposituras et magistratus super ciues prius experiri quales sint , } quos praefecit in praepositos vel magistros , & e manifiesta qual es el uaron siguese \textbf{ que conuiene a qual quier partidor o dador de las dignidades | e de los maestradgos de auer primero prueua de los çibdadanos } quales son aquellos \\\hline
3.1.13 & sed eius est \textbf{ intendere sanitatem tanquam finem : } sic cuius est ciuitatem ordinare , & mas ha de tener mienteᷤ en la sanidat \textbf{ assi comm̃en su fin bien } assi aquel que ha de ordenar la çibdat \\\hline
3.1.13 & et concordiam ciuium debet \textbf{ intendere rector ciuitatis tanquam finem . } Sic ergo disponenda est ciuitas , & e la concordia dela çibdat es fin \textbf{ que deue entender todo gouernador dela çibdat } e por ende non ha de tomar conseio sobre ella \\\hline
3.1.13 & ut innuit Philosophus \textbf{ 2 Polit’ videtur facere ad quietum } et pacificum statum ciuium . & e partiendo los a departidas \textbf{ ꝑsonas faze a buen estado e paçifico dela çibdat } e de los çibdadanos \\\hline
3.1.13 & Nam si spretis aliis semper iidem in magistratibus et praeposituris praeficiantur , \textbf{ alii videntes se esse despectos } ad seditionem consurgunt , & ca si menospreçiando alos vnos sienpre los otros fueren puestos en los ofiçios \textbf{ e en las dignidades los otros viendo se menospreçiados le una tan se en vandos } e en peleas \\\hline
3.1.13 & videntes enim nullam dignitatem possidere , \textbf{ si contingat eos esse viriles et animosos , } seditiones mouent . & nin gͤdignidat en la çibdat \textbf{ si contezca | que ellos sean tales } que sean poderosos \\\hline
3.1.14 & Quare cum patefactum sit in praecedentibus , \textbf{ non expedire ciuitati possessiones , } uxores , et filios esse communes , & por las cosas dichͣs de suso \textbf{ que non conuiene ala çibdat } que las possessiones nin los fijos nin las mugieres sean comunes \\\hline
3.1.14 & nec esse decens , \textbf{ mulieres ordinari ad opera bellica ; nec esse utile , } eosdem semper in eisdem magistratibus praefici , & que las mugieres sean ordenadas \textbf{ alas obras de batalla | nin es prouechoso } que sienpre vnos ofiçialon sean puestos en essos mismos ofiçios \\\hline
3.1.14 & tot esse bellatores et defensores patriae , \textbf{ quot sunt ibi ciues valentes portare arma , } quam seperare bellatores ab aliis ciuibus . & e los defenssores dela tierra \textbf{ quantos son y çibdadanos | que pue dan tomar armas } e esto es meior \\\hline
3.1.14 & quot sunt ibi ciues valentes portare arma , \textbf{ quam seperare bellatores ab aliis ciuibus . } Secunda via sic patet , & e esto es meior \textbf{ que dezir que sean apartados los bdiadores | de los otros çibdadanos ¶ } La segunda razon se prueua \\\hline
3.1.14 & esse valde difficile et onerosum ipsis ciuibus . \textbf{ Onerosum enim et difficile esset ciuibus unius ciuitatis sustentare mille viros in stipendiis communibus , } quorum nullum esset aliud officium , & por que grant carga serie e graue cosa serie alos çibdadanos de vna \textbf{ çibdat mantener mill caualleros | de las rentas comunes de vna } çibdat los quales caualleros non ouiessen otro ofiçio ninguno \\\hline
3.1.14 & cum adesset oportunitas : \textbf{ et onerosius et quasi omnino importabile esset sustentare sic quinque milia : } oporteret enim ciuitatem illam habere possessiones quasi ad votum , & quando fuesse me este \textbf{ Et muy mayor carga e peor de sofrir serie | si ouiessen a mantener cinco mill caualleros } ca conuerne \\\hline
3.1.14 & et onerosius et quasi omnino importabile esset sustentare sic quinque milia : \textbf{ oporteret enim ciuitatem illam habere possessiones quasi ad votum , } ut posset ex communibus sumptibus & ca conuerne \textbf{ que aquella çibdat ouiesse tantas possessiones | quantas quisiesse a ssu uoluntad } por que pudiesse de las rentas comunes abondar atanta muchedunbre \\\hline
3.1.14 & famulos et liberos , \textbf{ pascere tantam multitudinem bellatorum . } Tertio delinquebat Socrates & assi commo dize el philosofo gouernar atanta muchedunbre de lidiadores \textbf{ sin çibdadanos e sin mugers e sin siruientes e sin fijos } Lo terçero erraua socrates \\\hline
3.1.14 & Nam secundum Philosophum secundo Politicorum , \textbf{ volens ponere leges } vel facere ordinationem aliquam in ciuitate , & en el segundo libro delas politicas \textbf{ el que quiere poner leyes o fazer ordenaçion alguna en la çibdat a tres } co sas deue deuer mietes . \\\hline
3.1.14 & volens ponere leges \textbf{ vel facere ordinationem aliquam in ciuitate , } ad tria debet respicere , & en el segundo libro delas politicas \textbf{ el que quiere poner leyes o fazer ordenaçion alguna en la çibdat a tres } co sas deue deuer mietes . \\\hline
3.1.14 & et maiori terrarum spatio potiretur , \textbf{ tanto sustentare posset maiorem numerum bellantium . } Tertio aspiciendum esset ad loca vicina , & e vsa de mayor espaçio de tierras \textbf{ tanto mayor cuento de lidiadores pueden mantener ¶ } Lo terçero deuen tener mientes alos logares \\\hline
3.1.14 & esse circa particularia signata , \textbf{ volens tradere artem de regimine ciuitatum , } non potest statuere determinatum numerum bellatorum : & cerca las cosas particulares e senñaladas . \textbf{ El que quiere dar arte e sçiençia de gouernamiento dela çibdat } non puede \\\hline
3.1.15 & saluare poterimus positionem eius . \textbf{ Omnia enim esse ciuibus communia } secundum rei veritatem & que non es cosa que pueda ser \textbf{ nin es cosa aprouechable | que todas las cosas sean comunes } alos çibdadanos segunt uerdat . \\\hline
3.1.15 & Sicut enim quilibet ciuis debet \textbf{ diligere ciues alios , } sicut seipsum : & y la comu indat \textbf{ por que si cada vn çibdada no deue amar tos otros | çibdảdanos } assy commo assi mesmo . \\\hline
3.1.15 & sicut seipsum : \textbf{ sic debent diligere uxores , filios , } et possessiones aliorum , & assy commo assi mesmo . \textbf{ En essa misma manera deue amar los fijos e las mugers } e las possessiones de los otros çibdadanos \\\hline
3.1.15 & sed in possessionibus non solum debet \textbf{ reseruari communitas quantum ad amorem } ut quod omnes ciues communiter bonum diligant & Mas en las mugers e en los fijos deue ser guardada comunidat \textbf{ non solamente quanto al amor } assi que todos los çibdadanos \\\hline
3.1.15 & quantum ad communitatem ciuium : \textbf{ sic etiam saluare possumus dictum eius quantum ad unitatem ciuitatis . } Nam cum dixit ciuitatem debere esse maxime unam , & quanto ala comunidat de los çibdadanos \textbf{ En essa misma manera podemos saluar el su dicho | del quanto ala vnidat dela çibdat } ca quando dixo \\\hline
3.1.15 & et mulieres \textbf{ propter penuriam ciuium defendere ciuitatem . } Quod autem ulterius addebat , & por la qual cosa conuenio alas mugers \textbf{ por mengua de los çibdadanos de defender la çibdat mas lo que enandio adelante diziendo } que sienpre conuenia \\\hline
3.1.15 & volens ciuitatem \textbf{ ad minus mille continere bellatores . } Forte per bellatores intendebat nobiles , & e en batalladores quariendo \textbf{ que alo menos la çibdat ouiesse mil ł batalladores } por auentura \\\hline
3.1.15 & ut nobiles : \textbf{ hi videlicet nobiles potissime debent defendere patriam , } et eorum maxime est vacare & assi commo los nobles \textbf{ Por ende conuiene que estos nobles de una | prinçipalmente defender la tierra entre los otros } e a ellos parte nesçe mayormente de entender çerca la sabiduria delas armas . \\\hline
3.1.16 & quomodo ciues habeant possessiones aequatas . \textbf{ Volebat enim tunc esse ciuitatem optime ordinatam , } si nullus ciuium haberet plures redditus , & en qual manera los çibdadanos de una auer las possesipnes igualadas \textbf{ ca creya | que estonçe seria la çibdat muy bien ordenada } si ninguno de los çibdadanos non ouiesse mayores rentas o mayores possessiones \\\hline
3.1.16 & de facili rector ciuitatis posset \textbf{ diuidere aequaliter possessiones illas inter ciues . } Sed ciuitate iam constituta , & podria partir el rectoor dela çibdat \textbf{ egualmente aquellas possessions entre los çibdadanos } mas la çibdat ya establesçida los çibdadanos \\\hline
3.1.16 & Statuit enim Phaleas ciuitatis rectorem \textbf{ hoc modo reducere hanc inaequalitatem } ad aequalitatem mediantibus dotibus statuendo & que el rector dela çibdat \textbf{ en esta manera aduxiesse esta desegualdat a egualdat | Conuiene a saber } por las arras \\\hline
3.1.16 & aequari eis in possessionibus . \textbf{ Potuit autem Phaleas triplici via moueri ad hoc statuendum . } Primo quidem moueri potuit , & podria se ygualar alos ricos en las possessiones . \textbf{ Et este pho felleas pudo se mouer a establesçer esto por tres razones } La primera desta pudo ser \\\hline
3.1.16 & et scirent se non posse \textbf{ excedere suos conciues in possessionibus , } frustra propter hoc insurgerent lites et placita . & que vno non podia sobrepuiar \textbf{ los otros çibdadanos sus uesnos en possessiones } por esto debalde se leunatarian entre ellos las contiendas e las uaraias \\\hline
3.1.17 & non est possibile statuere in ciuitate \textbf{ omnes ciues habere aequalem numerum filiorum . } Propter quod ex parte procreationis prolis manifeste ostenditur & que otros por ende non se puede poner ley en la çibdat \textbf{ que todos los çibdadanos ayan ygual cuento de fijos } por la qual cosa se muestra manifiesta miente de parte dela generaçion de los fijos \\\hline
3.1.17 & Propter quod ex parte procreationis prolis manifeste ostenditur \textbf{ praedictam legem } non esse congruentem ; & por la qual cosa se muestra manifiesta miente de parte dela generaçion de los fijos \textbf{ quela ley puesta por felleas non es conuenible } por que se non puede guardar conueinblemente ¶ \\\hline
3.1.17 & praedictam legem \textbf{ non esse congruentem ; } eo quod congrue obseruari non possit . & por la qual cosa se muestra manifiesta miente de parte dela generaçion de los fijos \textbf{ quela ley puesta por felleas non es conuenible } por que se non puede guardar conueinblemente ¶ \\\hline
3.1.17 & ut plurimum sint magnanimi et magni cordis \textbf{ si videant se esse despectos } et pauperes exaltatos & e de grant coraçon \textbf{ si uieren que son despreçiados } e los pobres enxalçados \\\hline
3.1.17 & Tertia via ad ostendendum legem Phaleae \textbf{ non esse decentem } de aequatione possessionum , & e por ende fazer se yan sobuios e turbarian los otros . \textbf{ ¶ La terçera razon para mostrar } que la ley de felleas del ordenamiento delas possessiones \\\hline
3.1.17 & decet enim ipsos \textbf{ esse liberales et temperatos : } non ergo bene dictum est quod ad bonum regimen ciuitatis sufficit ciues habere possessiones aequatas , & por que conuiene \textbf{ que los çibdadanos sean liberales e francos } e por ende non es bien dicho \\\hline
3.1.17 & esse liberales et temperatos : \textbf{ non ergo bene dictum est quod ad bonum regimen ciuitatis sufficit ciues habere possessiones aequatas , } nisi aliquid determinetur & que los çibdadanos sean liberales e francos \textbf{ e por ende non es bien dicho | que a buen gouernamiento dela çibdat } cunple de ser las possessiones egualadas \\\hline
3.1.18 & sed principalius debet \textbf{ intendere reprehensionem concupiscentiarum , } eo quod radix malitiarum & del que faze la ley deue ser çerca delas possessiones \textbf{ mas mayormente deue entender en la reprehension delas cobdiçias } por que larays delas maldades \\\hline
3.1.18 & et in causa quam alibi , \textbf{ magis debent intendere rectores ciuium } circa reprimendas concupiscentias quam circa alia , & e en el comienço donde nasçe \textbf{ que en otra cosa los rectores delas çibdades } mas deuen entender en repreheder las cobdiçias \\\hline
3.1.18 & Meminimus tamen , \textbf{ nos edidisse quendam tractatum . } De differentia Ethicae Rhetoricae et Politicae , & enpero mienbranos \textbf{ que fiziemos vn tractado del partimiento } que es entre la ethica e la rectorica e la politica \\\hline
3.1.18 & si non possint \textbf{ consequi honorem debitum et condignum . } Quare si ciuium quidam personae sunt pauperes , & e con ueinble a ellos . \textbf{ Et por ende mas contienden los honrrados sobre la honra | que sobre la sustançia } por la qual cosa \\\hline
3.1.18 & Et tanto principalius debet \textbf{ hoc intendere circa honores , } quanto litigia inter personas honorabiles sunt magis detestanda , & e tanto mas prinçipalmente deue entender \textbf{ en partir las honrras } quanto las peleas entre las perssonas honrradas son de mayor periglo \\\hline
3.1.18 & sed quia existimant alios posse eorum delectationibus impedire , \textbf{ vel quia existimant eis posse tristitiam inferre . } non ergo solum propter possessiones sunt instituendae leges , & Mas por que cuydan que los otros pueden enbargar sus delecta connes \textbf{ o porque cuydan que les pueden fazer tristeza . } Et pues que assi es non solamente son de esta \\\hline
3.1.19 & Volebat autem bellatores \textbf{ debere habere arma , } et non terram . & e en labradores caquaria \textbf{ que los lidiadores troxiessen armas } e que non labrassen la tr̃ra \\\hline
3.1.19 & Dicebat autem debere \textbf{ esse aliquod territorium commune , de quo bellatores viuerent } quasi de communi aerario . & Enpero dizie que algunte rrectorio deuie ser comun \textbf{ del qual deuian beuir los lidiadores } assi commo de cosa comun¶ \\\hline
3.1.19 & uniuersaliter omnes personas impotentes , \textbf{ non valentes per se ipsas sua iura conquirere . Spectat enim ad Regem et Principem , } qui debet esse custos iusti , & nin podian \textbf{ por si mismas guardar su | derechca parte nesçe al Rey e al } prinçipeque deue ser guardador dela iustiçia de auer cuydado espeçial delas cosas comunes \\\hline
3.1.19 & eo quod talibus alii de facili iniuriantur , \textbf{ cum non possint defendere iura sua . } Multa bona consequimur & por que tales perssonas los otros de ligero les fazen tuerto \textbf{ por que non pueden defender su derecho } uchos bienes se nos siguen delas opiniones de los phos antigos \\\hline
3.1.20 & Si enim ciuitas \textbf{ secundum ipsum distingui debeat in tres partes , } videlicet in bellatores , artifices , et agricolas ; & por que si la çibdat \textbf{ segunt el dixo | se deuia partir en tres partes . } Oon uiene a saber enlidiadores \\\hline
3.1.20 & oportebat bellatores \textbf{ habere maiorem potentiam , } quam agricolae , & Et segunt esto conuiene \textbf{ que los lidiadores ouiessen mayor poderio } que los menestrales nin los labradores todos en vno . \\\hline
3.1.20 & loqui sibi inuicem publice , \textbf{ non tamen posse ad inuicem habere consilium in priuato . } Rursus deficit dictus modus , & que los iezes puedan fablar vno con otro en publico \textbf{ e non puedan auer conseio vno con otro en ascondido . } Otrossi fallesçe la dichͣ manera \\\hline
3.1.20 & conarentur sapientes ad inueniendum nouas leges , \textbf{ et ad ostendendum nouas leges inuentas esse proficuas ciuitati : } quare continue mutarentur leges , & de fallar nueuas leyes \textbf{ para mostrar que las leys nueuas | que ellos fallan son muy prouechosas ala çibdat } por la qual cosa cadal dia se aurian de mudar las leyes la qual cosa seria muy dannosa e muy peligrosa ala çibdat \\\hline
3.1.20 & et utrum principatus debeat \textbf{ ire per electionem } vel per haereditatem , & e si los prinçipes deuen yr \textbf{ por elecçion o por suçession de heredamiento . } Et si las leyes son assi de renouar \\\hline
3.2.1 & quae vim legum obtinent . \textbf{ Videntur ergo sic se habere arma ad tempus belli , } sicut leges ad tempus pacis . & que han fuerça de leyes \textbf{ et por ende assi sean las armas | alt pon de guerra } commo las leyes al tro de paz . \\\hline
3.2.1 & Viso igitur , \textbf{ tempore pacis gubernandam esse ciuitatem } et regnum per leges iustas , & Et pues que assi es visto \textbf{ que en elt pon dela paz es de gouernar la çibdat } e el regno \\\hline
3.2.1 & quae traduntur in legibus sint debite obseruata . \textbf{ Bene autem custodire leges } per ciuilem potentiam & commo deuen . \textbf{ Mas guardar bien las leyes } por el poderio çiuil esto parte nesçe alos prinçipes \\\hline
3.2.1 & quaedam dici possunt . \textbf{ Bene quidem inuenire leges } per sapientiam & ca los establesçimientos delas leyes pueden ser dichͣs \textbf{ leyes | mas fallar bien las leyes pertenesçe al conseio } por sabiduria \\\hline
3.2.1 & Bene vero iudicare \textbf{ secundum leges inuentas per consiliarios , } et custoditas per principem , & que el pueblo ha de guardar \textbf{ mas bien iudgar segt las leyes falladas } por los conseieros e guardadas \\\hline
3.2.1 & acta ciuium iudicare debent . \textbf{ Sed bene obseruare leges } spectat ad omnes ciues , & los que segunt tales leyes deuen iudgar los fechs de los çibdadanos . \textbf{ Mas bien guardar las leyes puestas } eston pertenesce a todos los çibdadanos o a todo el pueblo \\\hline
3.2.1 & sed de laudabili et vituperabili est exclamatio siue concionatio , \textbf{ quae potest respicere totum populum : } populus enim ad bene agendum , & e de denostar es llamamiento e conuiramiento \textbf{ que parte nesçe a todo el pueblo } ca el pueblo es de abiuar \\\hline
3.2.2 & tunc dicitur Monarchia siue Regnum : \textbf{ regis autem est intendere commune bonum . } Si vero ille unus dominans & e estonçe es dicho tal sennorio monarch̃ia o e egno \textbf{ ca al Rey parte nesçe de enteder el bien comun . } Et li aquel vno assi \\\hline
3.2.2 & vocat eum Philosophus nomine communi , \textbf{ et dicit ipsum esse Politiam . } Politia enim quasi idem est , & llamalle el philosofo nonbre comun \textbf{ e diz el poliçia } por que poliçia es \\\hline
3.2.2 & Politia dicitur . \textbf{ Nos autem talem principatum appellare possumus gubernationem populi , } si rectus sit . & por que non ha nonbre comun es dich poliçia \textbf{ e nos podemos llamar atal prinçipado gouernamiento del pueblo } si derecho es . \\\hline
3.2.2 & et in graeco nomine dicitur Democratia . \textbf{ Nos autem ipsum appellare possumus peruersionem populi . } Patet ergo quot sunt principatus , & es dich prinçipado de malos \textbf{ e en nonbre gniego es dicho democraçia mas nos podemos le llamar destruymiento e desordenamiento del pueblo . } Et pues que assi es paresçe \\\hline
3.2.3 & ad diuersa officia et diuersos motus , \textbf{ est dare aliquod unum membrum } ut cor , & a departidos ofiçios \textbf{ e departidos mouimientos conuiene de dar algun mienbro vno } assi commo es el coraçon \\\hline
3.2.3 & Rursus si ad constitutionem eiusdem concurrunt diuersa elementa , \textbf{ est dare ibi unum aliquid , } ut animam regentem & Otrossi si a conposicion de vn cuerpo vienen departidos helementos \textbf{ conuiene de dar | y alguna cosa vna } assi commo es el alma \\\hline
3.2.4 & Philosophus 3 Politicorum videtur \textbf{ tangere tres rationes , } per quas probari videtur , & e han cunplimiento delas cosas \textbf{ lpho en el terçero libro delas politicas tanne tres razons } por las quales paresçe que se puede prouar \\\hline
3.2.4 & ut bene regat populum sibi commissum . \textbf{ Primo enim debet habere perspicacem rationem . } Secundo rectam intentionem . & que lees acomnedado . \textbf{ Lo primero deue auer razon abiuada e sotil ¶Lo segundo entencion derecha . } Lo terçero firmeza acabada . \\\hline
3.2.4 & quasi omnino a communi bono . \textbf{ Peius est igitur principari unum , } quam plures . & o apartasse much del bien comun . \textbf{ Et pues que assi es meior cosa es de prinçipar muchs que vno¶ } La terçera razon se toma dela firmeza \\\hline
3.2.4 & decet enim Principem \textbf{ esse regulam rectam et stabilem , ut per iram et concupiscentias } et per alias passiones non corrumpatur nec peruertatur . & por que conuiene \textbf{ que el prinçipe sea regla derecha e firme et estable | assi que por ira } nin por cobdiçia \\\hline
3.2.4 & assignans rationes multas , \textbf{ quod melius sit dominari multitudinem : } postea in eodem 3 tangit quaedam , & e poniendo muchͣs razones para esto \textbf{ que meior es que much | senssennore en que vno . } Despues en esse mismo terçero tanne algunas cosas \\\hline
3.2.4 & cum ipse pluries dicat in eisdem politicis , \textbf{ regnum esse dignissimum principatum : } inter principatus enim rectos , & en esse mismo libro delas politicas \textbf{ que el regno es prinçipado muy digno } por que entre los prinçipados derechs el prinçipado de vno \\\hline
3.2.4 & non est dignius , \textbf{ quam dominari unum ; } cum nunquam plures recte dominari possint , & derech non es mas digna cosa nin meior \textbf{ que si enssennoreas se vno . } ca nunca pueden much \\\hline
3.2.4 & inquantum tenent locum unius : \textbf{ dominari unum et facere monarchiam , } si debito modo fiat , & en quanto ellos tienen logar de vno \textbf{ el sennorio de vno es meior | e fazer tal monarchia de vno } si se faze en manera \\\hline
3.2.4 & Censendum est igitur , \textbf{ regnum esse dignissimum principatum , } et secundum rectum dominium melius est dominari unum , & Et pues̃ que assi es deuemos otorgar \textbf{ que el regno es prinçipado muy digno } e segut derech \\\hline
3.2.4 & regnum esse dignissimum principatum , \textbf{ et secundum rectum dominium melius est dominari unum , } quam plures . & que el regno es prinçipado muy digno \textbf{ e segut derech | sennorio meior es } que sea vn sennor que muchos . \\\hline
3.2.4 & ( secundum Philosophum 3 Politicor’ ) \textbf{ debet sibi associare multos sapientes , } ut habeat multos oculos & que dize el philosofo enel terçero libro delas politicas \textbf{ que deue aconpannar assi e tomar consigo muchos sabios } por que ayan muchs oios \\\hline
3.2.4 & Non ergo dici poterit \textbf{ talem unum monarchiam non cognoscere multa ; } quia quantum spectat & Et pues que assi es non se puede dezer \textbf{ que vn tal monarchia | o tal prinçipe assi fech̃ de muchos que non conogca } e non sepa muchͣs cosas . \\\hline
3.2.4 & totum ipse Rex cognoscere dicitur . \textbf{ Nec etiam dici poterit ipsum de leui posse corrumpi et peruerti : } nam si Rex recte dominari desiderat , & e saber el Rey . \textbf{ nin avn se puede dezer | que aquel vn prinçipe de ligero se pueda trastornar } e coe ronper se ca si el Rey dessea enssennorear \\\hline
3.2.4 & et bonos quos sibi associauit , \textbf{ contingeret esse peruersos : } talis enim maxime intendit commune bonum ; & e todos los buenos omes \textbf{ que assi ouo aconpannado fuessen tristornados e corronpidos } por que tales prinçipalmente entienden el bien comun . \\\hline
3.2.4 & et dimissa societate sapientum et bonorum , \textbf{ vellet sequi caput proprium , } et appetitum priuatum , iam non esset Rex sed tyrannus : & e de los omes buenos \textbf{ e quisiere seguir su cabeça proprea } e su appetito corrupto \\\hline
3.2.5 & omnino esse melius \textbf{ et dignius dominationem regiam et principatum ire per electionem } quam per haereditatem . & e mas digna cosa \textbf{ qua el senñorio real | e el prinçipado venga por elecçio } que non por heredamiento \\\hline
3.2.5 & An melius sit regiam dignitatem \textbf{ ire per electionem , } an per haereditatem : & si es meior de ser la dignidat real \textbf{ por elecçion } que ꝑ non por heredamiento paresçe \\\hline
3.2.5 & videtur tale regnum non esse expositum casui et fortunae , \textbf{ sed factum esse per artem , } eo quod praeficietur melior et industrior . & de si nin auentura \textbf{ mas es fecho por arte e por sabiduria por el que meior e el mas sabio sera puesto en el sennorio } mas si esto fuere por heredat es pone se el regno \\\hline
3.2.5 & ad quem spectabit \textbf{ habere regiam dignitatem . } Absolute ergo loquendo , & al que pertenesçe de tegnar \textbf{ e de auer la dignidat real . } Et por ende paresça e a alguons que fablando sueltamente meiores \\\hline
3.2.5 & Absolute ergo loquendo , \textbf{ melius est Principem praestituendum esse per electionem , } quam per haereditatem . & Et por ende paresça e a alguons que fablando sueltamente meiores \textbf{ que el prinçipe sea establesçido | por elecçion } que por heredat . \\\hline
3.2.5 & quanto credit ipsum regnum \textbf{ magis esse bonum suum et bonum proprium : } quare si Rex videat & que el regno es mas su bien \textbf{ e mas su bien propo | que de otro ninguno } Por la qual cosa si el Rey viere \\\hline
3.2.5 & quare si Rex videat \textbf{ debere se principari super regnum non solum ad vitam , } sed etiam per haereditatem in propriis filiis , & Por la qual cosa si el Rey viere \textbf{ que deue regnar sobre el regno | non solamente en su uida } mas avn por heredat en sus fijos . \\\hline
3.2.5 & magis reputabit bonum regni \textbf{ esse bonum suum , } et ardentius solicitabitur & mas avn por heredat en sus fijos . \textbf{ mas terna que el bien del regno es su bien propreo } e con mayor \\\hline
3.2.5 & Innuit enim hoc esse \textbf{ quasi virtutis diuinae , et excedere humanum modum . } Sed forte ideo hoc Philosophus dixit , & ca dize que esto es assi commo por uirtud de dios \textbf{ e que sobrepiua la manera | humanabeas por auentura esto } por ende lo dize el philosofo \\\hline
3.2.5 & in haereditatem paternam . \textbf{ Vel simpliciter dicitur hoc esse diuinum , } quia nisi Reges et Principes & assi que los fijos en esta manera ouiessen la heredat de los padres \textbf{ o podemos dezir sinplemente | que es por la uirtud de dios . } si los Reyes e los prinçipes non regnaren \\\hline
3.2.5 & ad quem spectat \textbf{ suscipere curam regni . } Nam sicut mores nuper ditatorum & a quien parte nesçe de regnar \textbf{ e de tomar cuydado del regno . } ca assi commo las costunbres \\\hline
3.2.5 & sicut haereditarie principantes : \textbf{ et quia hoc est esse tyrannum , } non intendere bonum regni , & conmo aquellos que enssennorean por h̃edamiento \textbf{ e vn por que esta es condiçion de thirano } non tener mientes en el bien del regno \\\hline
3.2.5 & et quia hoc est esse tyrannum , \textbf{ non intendere bonum regni , } tales facilius tyrannizant . & e vn por que esta es condiçion de thirano \textbf{ non tener mientes en el bien del regno } los tales que son tomados \\\hline
3.2.5 & ex qua praeficiendus est dominus , \textbf{ sed etiam oportet determinare personam . } Nam sicut oriuntur dissentiones et lites , & linage donde ha de ser tomado el sennor . \textbf{ Mas avn conuiene de determinar la perssona . } Ca assi commo nasçen discordias \\\hline
3.2.5 & in illa prosapia debeat principari . \textbf{ Talem autem determinare personam , } difficultatem non habet : & qual perssona en qual linage deua ser prinçipe \textbf{ e auer el senorio . | Mas determinar tal perssona } non hagniueza ninguna \\\hline
3.2.5 & per haereditatem transferatur ad posteros , \textbf{ oportet eam transferre in filios , } quia secundum lineam consanguinitatis filii parentibus maxime sunt coniuncti : & por hedamiento conuiene alos pueblos \textbf{ que tomne alos fijos } ca segunt el linage del patente \\\hline
3.2.5 & ut pater ampliori solicitudine curet de bono regni , \textbf{ sciens ipsum peruenire ad filium plus dilectum . } Et si dicatur quod contingit aliquando magis diligere minores . & por que el padre con mayor acuçia aya cuydado del bien del regno \textbf{ sabiendo | que el regno parte nesçe al su fiio mas amado } Et si dixiere alguno \\\hline
3.2.5 & sciens ipsum peruenire ad filium plus dilectum . \textbf{ Et si dicatur quod contingit aliquando magis diligere minores . } Talibus obiectionibus de facili respondetur : & que el regno parte nesçe al su fiio mas amado \textbf{ Et si dixiere alguno | que contesçe algunas uezes } que los padres mas aman alos menores \\\hline
3.2.5 & Quod vero superius tangebatur , \textbf{ videlicet quod ire per haereditatem , dignitatem regiam , } est exponere fortunae , & Mas aqual lo que dessuso fue dich \textbf{ conuiene saber | que quando va el regno } por hedat \\\hline
3.2.5 & in talibus regiminibus vidimus , \textbf{ quae enumerare per singula longum esset . } Dicere ergo possumus & de los quales non podemos fablar \textbf{ nin contar de cada vno } por menudo pues que assi es podemos dezir \\\hline
3.2.5 & ad quem deberet regia cura peruenire , \textbf{ suppleri poterit per sapientes et bonos , } quos tanquam manus et oculos debet & auer cuydado del regno \textbf{ este fallesçimiento se puede conplir por sabios et por buenos omes } lo quales deue el Rey ayuntar \\\hline
3.2.6 & nimis ardenter mouetur in eorum amorem , \textbf{ et optat eos habere in dominos . } Inde est quod antiquitus plures sic praeficiebantur in Reges . & e bien fechores mueuense con grant ardor alos amar \textbf{ e dessean de los auer | por sennores } e por ende antiguamente los mas de los sennores fueron tomados en Reyes . \\\hline
3.2.6 & praeficiebat ipsum in Regem . \textbf{ Secundo potest aliquis praefici in Regem } ab excessu virtuosarum actionum : & por grant amor tomauas lo por su Rey \textbf{ ¶Lo segundo puede alguno sobrepuiar a otro } por a unataia de obras uirtuosas . \\\hline
3.2.6 & ab excessu virtuosarum actionum : \textbf{ nam quia bonorum virtuosorum est diligere bonum commune potius quam priuatum , } ideo reputatur dignus & ca por que de los buenos \textbf{ e de los uirtuosos es de amar | mas el bien comun } que el bien propreo . \\\hline
3.2.6 & qui a populo creditur virtuosus . \textbf{ Tertio consueuit praefici aliquis in Regem } ab excessu potentiae et dignitatis . & por digno que sea tomado por Rey \textbf{ Lo terçero fue acostunbrado de tomar alguno | por Rey } pora una taia de poderio e de dignidat \\\hline
3.2.6 & Nam quia probabile est nobiles et potentes , \textbf{ magis verecundari operari turpia quam alios : } et quia tales & que los nobles e los poderosos toman mayor uerguença \textbf{ de obrar cosas torpes | e feas que los otros } e por que tales por la mayor parte \\\hline
3.2.6 & Quare expedit regem \textbf{ habere praedictos tres excessus . } Nam si abundet in beneficiis tribuendis , & que el rey aya aquellas tres aun ataias \textbf{ e aquellas tres condiçiones buenas sobredichͣs . } ca si abondare en bien fazer seria muy amado del pueblo \\\hline
3.2.6 & quia si virtutis est , \textbf{ tendere in bonum , } eius erit magis tendere in maius bonum , & en obras uirtuosas procurara el bien comun . \textbf{ ca si la uirtud parte nesçede se estender a mayor bien } e en mayor bien dela gente es el bien comun \\\hline
3.2.6 & tendere in bonum , \textbf{ eius erit magis tendere in maius bonum , } bonum ergo gentis et commune & ca si la uirtud parte nesçede se estender a mayor bien \textbf{ e en mayor bien dela gente es el bien comun } que es mas diuinal \\\hline
3.2.6 & in ciuili potentia , ut possit corrigere volentes insurgere , \textbf{ et turbare pacem regni . } Viso in quibus Rex alios debet excedere : & por que pueda castigar los que se quisieren le una tar \textbf{ contra la paz del regno } Visto en quales cosas el Rey deue sobrepuiar \\\hline
3.2.6 & nisi de delectationibus propriis , \textbf{ videns se esse onerosum et tediosum } ab iis & si non delas sus delecta connes proprias . \textbf{ veyendo se en carga e en aborresçimiento } de aquellos que son en el regno . \\\hline
3.2.7 & Quadruplici via venari possumus , \textbf{ tyrannidem esse pessimum principatum . } Prima sumitur ex eo quod tale dominium maxime recedit & or quatro razones podemos prouar \textbf{ que la thirama es muy mal prinçipado | ¶La primera se toma } por razon que tal sennorio mucho se arriedra dela entençion del bien comun . \\\hline
3.2.7 & Unde 3 Polit’ dicitur , \textbf{ quod principari talem , } est quasi partiri principatum in multos . & dize el philosofo \textbf{ que tal sennor auer prinçipado es } assi commo partir vn prinçipado en muchs . \\\hline
3.2.7 & quod principari talem , \textbf{ est quasi partiri principatum in multos . } Vel ( quod idem est ) & que tal sennor auer prinçipado es \textbf{ assi commo partir vn prinçipado en muchs . } o que es esso mismo \\\hline
3.2.7 & Vel ( quod idem est ) \textbf{ est quasi principari multitudinem , } eo quod in tali principatu & o que es esso mismo \textbf{ que auer muchs el prinçipado . } por que en tal prinçipado es entendido el bien de muchs . \\\hline
3.2.7 & in eodem 4 Politicorum ubi ait , \textbf{ tyrannidem esse pessimum principatum , } quia nullus liberorum voluntarie sustinet principatum talem . & en el quarto libro delas politicas \textbf{ do dize que la tirania es muy mal prinçipado } por que ninguno de los omes francos e libres non sufre \\\hline
3.2.7 & Hanc autem rationem tangit Philosophus quinto Politicorum ubi ait , \textbf{ tyrannidem esse oligarchiam } extremam idest pessimam : & en el quinto libro delas politicas \textbf{ do dize que la tirnia es la postrimera obligarçia } que quiere dezer muy mala obligacion \\\hline
3.2.7 & sed etiam satagit \textbf{ impedire eorum maxima bona . } Tangit autem Philosophus 5 Polit’ tria maxima bona , & de aquellos que son en el regno \textbf{ mas avn esfuercasse para enbargar los bienes dellos } e tanne espho en el quinto libro delas politicas muy grandes tres bienes \\\hline
3.2.7 & videlicet pacem , virtutes , et scientias . \textbf{ Tyranni enim nolunt ciues habere pacem } et concordiam adinuicem : & Paz ¶ Virtudes . \textbf{ Et sçiençias | Ca los tirannos non quieren } que los çibdadanos ayan paz nin concordia entre ssi . \\\hline
3.2.7 & et concordiam adinuicem : \textbf{ rursus nolunt eos esse magnanimos et virtuosos : } nec etiam volunt ipsos esse sapientes et disciplinatos . & que los çibdadanos ayan paz nin concordia entre ssi . \textbf{ Otrossi non quiere | que ellos se que de grandes coraçones e uirtuosos } e avn non quiere \\\hline
3.2.7 & rursus nolunt eos esse magnanimos et virtuosos : \textbf{ nec etiam volunt ipsos esse sapientes et disciplinatos . } Quare autem tyranni praedicta bona & que ellos se que de grandes coraçones e uirtuosos \textbf{ e avn non quiere | que los çibdadanos sean sabios e entendidos . } Et la razon por que los tiranos enbargan estos bienes sobredichos en las çibdades ayuso se dira . \\\hline
3.2.7 & Sufficiat autem ad praesens scire , \textbf{ tyrannidem esse pessimum principatum } propter rationes tactas . & Et cunpla agora de saber \textbf{ que la tirania es muy mal prinçipado } por las razones sobredichͣs . \\\hline
3.2.8 & quod natura primo dat rebus ea per quae possunt \textbf{ consequi finem suum . } Secundo dat eis ea & que la natura primeramente da a todas las cosas \textbf{ aquello por que pueden alcançar su fin . } ¶ Lo segundo les da aquellas cosas \\\hline
3.2.8 & ut possit \textbf{ consequi finem intentum . } Secundo , ut remoueantur prohibentia et deuiantia & Lo primero que en tal manera sea el pueblo apareiado e ordenado por que pue da alcançar su fin que entiende . \textbf{ Lo segundo conuiene que sean arredradas todas aquellas cosas } que enbargan de alcançar aquella fin \\\hline
3.2.8 & ut melius aerem scindat \textbf{ ne prohibeatur tendere in ipsum signum : } tertio a sagittante sagittatur & meiorfender el ayre \textbf{ por que non sea enbargada de yr a su señal | ¶ } Lo terçero es puesta en la ballesta del saetero \\\hline
3.2.8 & per quae possit \textbf{ consequi finem intentum . } Secundo debet prohibentia remouere . & por que la gente que les acomnedada aya aquellas cosas \textbf{ por las quales puede alcançar la fin que entiende . } ¶ Lo segundo deue arredrar todas aquellas cosas \\\hline
3.2.8 & ut populus possit \textbf{ consequi finem intentum } et bene viuere , & Mas aquellas cosas que siruen aesto \textbf{ por que el pueblo pueda alcançar su fin } e pueda bien beuir son estas . \\\hline
3.2.8 & et habeat ordinatum appetitum , \textbf{ ut velit consequi finem illum : } spectat igitur ad rectorem regni ordinare & en tal manera \textbf{ que quiera | e pueda alcançar aquella fin . } Et por ende parte nesçe al gouernador del regno de otdenar sus \\\hline
3.2.8 & organice deseruiunt res exteriores . \textbf{ Decet ergo Reges et Principes sic regere ciuitates et regna , } ut sibi subiecti abundent rebus exterioribus & assi commo son las riquezas e los algos . \textbf{ Et por ende conuiene alos Reyes | e alos prinçipes de gouernar } assi las çibdades e los regnos \\\hline
3.2.8 & circa ea per quae possit populus \textbf{ consequi finem intentum : } restat ostendere , & de ser acuçioso çerca aquellas cosas \textbf{ por las quales el pueblo puede alcançar su fin } que entiende finca de demostrar \\\hline
3.2.8 & in haereditatem priorum : \textbf{ remouere igitur unum maxime prohibentium bonam vitam politicam , } est bene ordinare & e los postrimeros de los primeros \textbf{ Et pues que assi es tirar vna cosa } que enbarga much la buena uida çiuil es ordenar bien \\\hline
3.2.8 & quasi enim nihil esset \textbf{ vitare interiora discrimina , } nisi prohibentur exteriora pericula . & por que non seria nada escusar los males \textbf{ de dentro del alma | e los pecados } si non fuessen tirados \\\hline
3.2.9 & ideo bene se habet \textbf{ illa decem narrare per singula . } Est autem primum quod spectat & los sermones generales poco proprouechan . \textbf{ por ende sera bien de contar estas diez cosas cada vna } por si¶ Et la primera \\\hline
3.2.9 & et regni redditus studeat \textbf{ expendere in bonum commune , } vel in bonum regni : & que las rentas del regno se pongan \textbf{ enł bien comun } e en el bien del regno \\\hline
3.2.9 & Tertio decet Regem , \textbf{ et Principem non ostendere se nimis terribilem et seuerum , } nec decet se nimis familiarem exhibere , & nin los derechos del regno . \textbf{ ¶ Lo terçeto conuiene al Rey et al prinçipe de non mostrarsse muy espantable nin muy cruel . } nin le conuiene otrosi de se fazer muy familiar alos omnes \\\hline
3.2.9 & tyrannus autem non est , \textbf{ sed esse se simulat . } Quarto spectat ad Regem , & Mas el tirano non es tal \textbf{ mas quiere pare sçertal . } lo quarto parte nesçe a \\\hline
3.2.9 & non solum habere familiares , \textbf{ et diligere nobiles , et barones , et alios } per quos bonus status regni conseruari potest , & non solamente de auer buenos familiares \textbf{ e de amar los nobles | e los ricos omes } e todos los otros omes . \\\hline
3.2.9 & per quos bonus status regni conseruari potest , \textbf{ sed etiam ut ait Philosophus in Polit’ inducere debent uxores proprias } ut sint familiares et beniuolae uxoribus praedictorum : & por los quales se puede guardar el buen estado del regno . \textbf{ Mas avn assi commo dize el philosofo | en el terçero libro delas politicas deue enduziras Ꝯmugres propraas } por que sean familiares e bien querençiosas alas mugers \\\hline
3.2.9 & ut seditiones mouerent in regno aut principatu ; \textbf{ sic ergo gerere se debet } bonus rector regni aut ciuitatis . & e enl prinçipado \textbf{ Et pues que assi es . } assi se deue auer buen Rey e buen gouernador deregas e de çibdat \\\hline
3.2.9 & Nono decet verum Regem per usurpationem et iniustitiam \textbf{ non dilatare suum dominium . } Nam ut dicitur Polit’ & ¶ Loye conuiene al Rey uerdadero de non enssanchar su regno \textbf{ por tomar lo ageno | por fuerça e sin iustiçia . } Ca assi commo dize el philosofo \\\hline
3.2.9 & Nam ut dicitur Polit’ \textbf{ durabilius est regnare super paucos , } quam super multos . & en el tercero libro delas politicas \textbf{ mas durable es regnar sobre pocos que sobre muchos . } la qual cosa es muy uerdadera mayor mente \\\hline
3.2.9 & omnino est subiectus Regi \textbf{ quem credit esse deicolam , } et habere amicum Deum : & Ca el pueblo segunt que dize el philosofo es del todo subiecto al Rey \textbf{ quando vor que es honrrador de dios } e que ha a dios por amigo \\\hline
3.2.9 & quem credit esse deicolam , \textbf{ et habere amicum Deum : } existimat enim talem semper iuste agere , & quando vor que es honrrador de dios \textbf{ e que ha a dios por amigo } ca sienpre cuyda \\\hline
3.2.10 & Vident enim se contra dictamen rectae rationis agere , \textbf{ et non intendere bonum commune sed proprium : } ideo vellent omnes suos subditos & contra razon derech̃tu eyendo \textbf{ que ellos non entienden enl bien comun | mas en el su bien propreo . } Por ende querrien que todos los sus subditos fuessen sin sabiduria e nesçios \\\hline
3.2.10 & ideo vellent omnes suos subditos \textbf{ esse ignorantes et inscios , } ne cognoscentes eorum nequitiam , & mas en el su bien propreo . \textbf{ Por ende querrien que todos los sus subditos fuessen sin sabiduria e nesçios } por qua non conosçiessen \\\hline
3.2.10 & et eos qui sunt in regno \textbf{ non esse sodales , } nec esse ad inuicem notos : & que los çibdadanos que son en el regno \textbf{ non sean conpannones nin amigos } nin sean conosçidos vnos con otros . \\\hline
3.2.10 & non esse sodales , \textbf{ nec esse ad inuicem notos : } nam ( ut ait Philosophus ) & non sean conpannones nin amigos \textbf{ nin sean conosçidos vnos con otros . } Ca assi commo dize el pho la conosçençia faze fe . \\\hline
3.2.10 & Verus autem Rex econtrario permittit sodalitates ciuium , \textbf{ et vult ciues sibi inuicem esse notos , } et de se confidere ; & ca consiente todas las conpannias \textbf{ e quiere que los çibdada nos sean conosçidos vnos con otros } e que fien vnos de otros . \\\hline
3.2.10 & ut diligatur ab eis : \textbf{ quare vult eos esse confoederatos et coniunctos , } quia tunc magis unanimiter diligunt bonum Regis . & cosaes que sea amado dellos . \textbf{ Por la qual cosa quiere | que los çibdadanos sean amigos et ayuntados } por conpannias por que estonçe los çibdadanos aman \\\hline
3.2.10 & Omnino enim esset peruersus populus , \textbf{ si cognosceret se habere verum Regem , } et diligere commune bonum , & Ca en todo en todo sia malo el pueblo \textbf{ si conosçiesse | que auia buen Rey e uerdadero } e que amaua el bien comun \\\hline
3.2.10 & si cognosceret se habere verum Regem , \textbf{ et diligere commune bonum , } si viceuersa non diligeret ipsum Regem . & que auia buen Rey e uerdadero \textbf{ e que amaua el bien comun } si el esso mismo non amasse mucho al Rey \\\hline
3.2.10 & Quinta cautela tyrannica , \textbf{ est habere multos exploratores , } et tenptare non latere ipsum & si el esso mismo non amasse mucho al Rey \textbf{ La quinta cautela del tirano es auer muchs assechadores } e escodrinnar \\\hline
3.2.10 & est habere multos exploratores , \textbf{ et tenptare non latere ipsum } quicquid a ciuibus agitur . & La quinta cautela del tirano es auer muchs assechadores \textbf{ e escodrinnar | que non se le encubra ninguna cosa } delo que fazen los çibdadanos . \\\hline
3.2.10 & eo quod in multis offendant ipsum , \textbf{ volunt habere exploratores multos , } ut si viderent aliquos ex populo machinari aliquid contra eos , & que non lon amados del pueblo . \textbf{ por que en muchͣs cosas le aguauian quieren auer muchs assechadores } por que si vieren \\\hline
3.2.10 & volunt habere exploratores multos , \textbf{ ut si viderent aliquos ex populo machinari aliquid contra eos , } possint obuiare illis . & por que en muchͣs cosas le aguauian quieren auer muchs assechadores \textbf{ por que si vieren | que alguon ssele una tan contra ellos } que los puedan contradezer ante \\\hline
3.2.10 & Immo eo ipso quod ciues credunt \textbf{ tyrannum habere exploratores multos , } non audent congregari & que los çibdadanos creen \textbf{ que ay muchos assechadores } non se osan ayuntar \\\hline
3.2.10 & Utrum autem deceat Reges \textbf{ habere exploratores in regno propter aliam causam , } quam ne populus insurgat in ipsum , & Mas si conuiene al Rey auer assechadores en el regno \textbf{ por otra razon que por la que dichͣes . } que el pueblo non se leunate contra el adelante se dira . \\\hline
3.2.10 & Sexta cautela tyrannica , \textbf{ est non solum non permittere fieri sodalitates et amicitias , } sed etiam amicitias iam factas , & ¶ La sesta cautela del tirano es \textbf{ non solamente non conssentir las conpannias e las amistanças . } Mas avn las conpannias e las amistancas \\\hline
3.2.10 & quia non intenderet commune bonum . \textbf{ Septima , est pauperes facere subditos adeo } ut ipse tyrannus nulla custodia egeat . & ca non entendrie en el bien comun . \textbf{ La vij ͣ̊ cautela del tirano es fazer los subditos pobres } en tanto que el non aya menester guarda \\\hline
3.2.10 & quibus indigent , \textbf{ ut non vacet eis aliquid machinari contra ipsos , nec oporteat ipsos habere aliquam custodiam propter illos . } Verus autem Rex & en que han de de beuir de cada dia \textbf{ por que no les uague de fazer ayuntamiento contra ellos | nin los tiranos non ayan menester ninguna guarda } por temor dellos . \\\hline
3.2.10 & sed magis procurat eorum bona . \textbf{ Octaua , est procurare bella , } mittere bellatores ad partes extraneas , & Mas ha cuydado de acrescentar sus bienes \textbf{ ¶ La . viij n . cautela del tirano } es procurar guerras e enbiar \\\hline
3.2.10 & Octaua , est procurare bella , \textbf{ mittere bellatores ad partes extraneas , } et semper facere bellare & ¶ La . viij n . cautela del tirano \textbf{ es procurar guerras e enbiar | guerrasa partes estrannas } e sienpre faze lidiar sus çibdadanos \\\hline
3.2.10 & quatenus semper circa bellorum onera intenti , \textbf{ non vacet eis aliquid machinari contra tyrannum . } Verus autem Rex non intendit affligere subditos , & por razon que ellos en tal manera sean ocupados en las guercas \textbf{ que non les vague de seleunatar contra el tirano . } Mas el uerdadero rey non entiende de atormentar los subditos mouiendo les \\\hline
3.2.10 & non vacet eis aliquid machinari contra tyrannum . \textbf{ Verus autem Rex non intendit affligere subditos , } suscitando et procurando bella , & que non les vague de seleunatar contra el tirano . \textbf{ Mas el uerdadero rey non entiende de atormentar los subditos mouiendo les } e procurado les guerras \\\hline
3.2.10 & non vult ergo eos \textbf{ qui sunt in regno assumere arma } nisi pro defensione regni , & e por ende non quiere \textbf{ que los que son en el regno | tomne armas sinon } por defendimiento del regno \\\hline
3.2.11 & Secundo ex industria et sagacitate , \textbf{ ut quia credit se tot adinuenire vias et versutias posse , } ut valeat tyrannum perimere . & La segunda paresçe que salle de omne de grant coraçon . \textbf{ por que el que se leunata contra el tyra | no es de tan grant coraçon } que non tiene por grant cosa de \\\hline
3.2.11 & excogitant seditiones , \textbf{ quomodo possint turbare ciuitatem , et insurgere contra rectorem ciuium . } Contra haec ergo quatuor procurant tyranni perimere excellentes , & por el conplimiento que han pienssan maneras e carreras \textbf{ por las quales podran defender su çibdat | e leunatarse contra el mal gouernador de los çibdadanos } Et pues que assi es contra estas quatto cosas procuran los tyranos de matar los nobles e los grandes \\\hline
3.2.11 & destruere sapientes , \textbf{ impedire scholas , et studium , } ut existentes in regno sint ignorantes et inscii : & Otrossi procuran de destroyr los sabios \textbf{ e enbargar las escuelas e el estudio . } por que los que fueren en el regno sean nesçios e sin sabiduria \\\hline
3.2.11 & ut existentes in regno sint ignorantes et inscii : \textbf{ non permittere sodalitates ; } turbare ciues inter se & Otrossi procuran de enbargar \textbf{ e non consentir las conpannias } e de turbar los çibdadanos entre ssi \\\hline
3.2.11 & non permittere sodalitates ; \textbf{ turbare ciues inter se } ut de se inuicem non confidant : & e non consentir las conpannias \textbf{ e de turbar los çibdadanos entre ssi } por que non fiende ssi mismos los vnos de los otros . \\\hline
3.2.11 & ut de se inuicem non confidant : \textbf{ depauperare eos : } occupare eos in bello , & por que non fiende ssi mismos los vnos de los otros . \textbf{ Otrossi procuran de los en pobresçer | por que sean sienpre menesterosos } e delos poner en guerras \\\hline
3.2.11 & depauperare eos : \textbf{ occupare eos in bello , } et in aliis exercitiis , & por que sean sienpre menesterosos \textbf{ e delos poner en guerras } e en otros trabaios \\\hline
3.2.11 & Ex hoc autem manifeste patet , \textbf{ tyrannidem maxime esse fugiendam a regibus : } quia pessimum de se , & e desto paresçe manifiesta miente \textbf{ que la tirama es much de escusar alos Reyes | e mucho han de foyr della } por que es . muy mala . \\\hline
3.2.12 & sed pecuniam , \textbf{ intendere corporales delectationes . } Tertio diuites sic principantes & e non entienden en el bien comun \textbf{ mas en las riquezas entienden en las delectaçonnes corporales } ¶ \\\hline
3.2.12 & quantum ad uxores et filias . \textbf{ Videt ergo se esse odiosum populo , } ideo non credit se multitudini , & e faze much stuertos alos çibdadanos e enlas mugres e en las fijnas . \textbf{ Et por ende veyendo se aborresçido del pueblo } non fia dela muchedunbre de los çibdadanos \\\hline
3.2.12 & et quare nunquam hylarem vultum ostenderet . \textbf{ Tyrannus ille volens reddere causam quaesiti , } eum expoliari fecit , & que nunca mostraua la cara alegte \textbf{ e aquel tirano quariendo dar razon desto fizo despoiar a su hͣmano } e fizola tar \\\hline
3.2.12 & quot veri reges . \textbf{ Nam habere amicos , } et diligi a populo , & quantas han los uerdaderos los Reyes \textbf{ por que auer amigos } e ser muy amado del pueblo es muy delectable \\\hline
3.2.12 & non confidere de aliquo \textbf{ et credere se odiosum esse multitudini , } est maxime tristabile . & e non fiar de alguno \textbf{ e creer que es odioso | e aborresçido dela muchedunbre del pueblo } esto es muy derstable \\\hline
3.2.12 & Priuatur ergo tyrannus a maxima delectatione , \textbf{ cum videat se esse populis odiosum . } Viso tyrannidem cauendam esse , & e por ende el tirano es pri uado de grant delectaçion \textbf{ quando bee | que es aborresçido delos pueblos } Disto que la tirauja es de esquiuar e de aborresçer \\\hline
3.2.12 & congregantur in ea : \textbf{ restat videre esse eam cauendam , } eo quod etiam in ipsa congregantur & e los malos señorios de los rricos son ayuntados en ella , \textbf{ finca de ver que es de foyr e de aborresçer avn } por que en ella son ayuntados los males del mal priͥnçipado del pueblo \\\hline
3.2.13 & et iniuriari subditis , \textbf{ et non intendere commune bonum ; } licet pluribus viis ostenderimus & e fazer tuerto alos subditos \textbf{ e non entender al bien comun | commo quier } que por munchons rrazones mostramos \\\hline
3.2.13 & et periculosum esse regiae maiestati tyrannizare , \textbf{ et non recte gubernare populum : } non piget adhuc nouas vias adducere & e aborresçible deue ser ala rreal maiestad tiranzar \textbf{ e non gouernar derechamente el pueblo } aun non tomamos pereza de \\\hline
3.2.13 & quod nimis fugans timidum , \textbf{ vi compellit esse audacem . } Sic etiam et alia animalia & que quien muncho faze foyr al temeroso \textbf{ por fuerça lo costrange desee oscido en essa misma manera } avn en las otras ainalias \\\hline
3.2.13 & est iniuria quam passi sunt ab ipso : \textbf{ naturale est enim desiderare vindictam , } propter quod Homerus dicebat , & que rresçibien \textbf{ dehcanatanl cosa | e ᷤalos omes desear uengança del mal que rresçibe } Por la qual cosa omero aquel poeta dezia \\\hline
3.2.13 & quidam enim nomine Dion videns ipsum \textbf{ quasi semper esse ebrium , } propter despectionem insurrexit in ipsum . & que auja nonbre dion viendol \textbf{ que sienpre estaua enbriago | por despechon quel auje } e despreçiandol \\\hline
3.2.13 & nisi honorem et gloriam propriam , \textbf{ et non honorare subditos , } et non quaerere commune bonum , & e de sugłia propria \textbf{ e non quiere honrrar los subditos } njn quiere el bien comun \\\hline
3.2.13 & et non honorare subditos , \textbf{ et non quaerere commune bonum , } volentes adipisci honorem & e non quiere honrrar los subditos \textbf{ njn quiere el bien comun } quariendo algunos alcançar la gloriar la honrra \\\hline
3.2.13 & et non quaerere commune bonum , \textbf{ volentes adipisci honorem } et gloriam quam conspiciunt in tyranno , & njn quiere el bien comun \textbf{ quariendo algunos alcançar la gloriar la honrra } que veen enel tirano acometen ler matanle , \\\hline
3.2.13 & et perimunt ipsum . \textbf{ Sic etiam quia multi reputant pecuniam esse maximum bonum , } videntes tyrarannum non intendere & Ca essa misma manera avri \textbf{ por que munchos cuda | que el auer es muy grand bien veyendo } que el tirano non entiende \\\hline
3.2.13 & sed ut videantur \textbf{ facere actiones aliquas singulares . } Volunt enim aliqui esse in aliquo nomine & por que ayan el su señorio mas por que paresca alos omes \textbf{ que fazen algunos omes apartadas } ca algunos quieren ser en alguna nonbrada o en alguna fama \\\hline
3.2.13 & et aliquod singulare factum : \textbf{ et quia insurgere contra tyrannum } reputant populi valde stupendum , & e algund fhon estraño \textbf{ E por que leunatase contra el tirano } tieuenlo los omes \\\hline
3.2.13 & Sexto contingit aliquos insidiari tyrannis \textbf{ et perimere ipsos , } ut liberent patriam & que al gunos a echa alos tiranos \textbf{ e los matan } por que libren la trrͣa dela grad \\\hline
3.2.14 & volumus alias rationes adducere , \textbf{ ostendentes quod si reges cupiant suum durare dominium , } summo opere studere debent & avn en este cpleo queremos adozjr otras rrazones \textbf{ para mostrar que si los rrey e cobdiçian de duar muncho } el su señorio es toda manera deuen estudiar \\\hline
3.2.14 & dicens , \textbf{ Tyrannidem corrumpi a se , } a tyrannide alia , et a regno . & Ca cuenta el phon enel quinto libro delas politicas tres maneras dela corrupçion dela tiranja \textbf{ e dize que la tiranja corrope de si mismar coronpese desta çirana } e corronpese por El regno ¶ \\\hline
3.2.14 & Reges ergo et principes \textbf{ si volunt suum durare dominium , } summe cauere debent & e los prinçipes \textbf{ si quieren durar en su sennorio } mucho deuen escusar \\\hline
3.2.14 & eo quod esset habet multipliciter : \textbf{ contingit enim uno modo percutere signum , } propter quod in hoc non est diuersitas nec contrarietas : & por que ha de ser en muchͣs maneras . \textbf{ ca contesçe que en vna manera tiran ala sennal derechomente } por esso en tal cosa commo esta non ha contrariedat nin departimiento . \\\hline
3.2.14 & gens illa oppressa non valens \textbf{ sustinere tyrannidem Principis , } insurgit et tyrannizat in ipsum , & enparadoro algun \textbf{ prinçipe vno titaniza en el pueblo aquella gente apremiada non podie do sofrir su tira } maleunatasse e tiraniza contrael prinçipe matandol o echandol del prinçipado . \\\hline
3.2.14 & Decet ergo regiam maiestatem \textbf{ summo studio cauere tyrannidem , } ne praedictis periculis exponatur . & Et pues que assi es conuiene ala Real magestad de escusar con grant estudio \textbf{ e con grant acuçia la tirama } por que non se pongan alos peligros sobredichos \\\hline
3.2.15 & quae politiam saluant , \textbf{ et quae oportet facere Regem ad hoc } ut se in suo principatu praeseruet . & e el gouernamiento del regno \textbf{ e dela çibdat | las quals conuiene al Rey de fazer } para que se pueda man tener en lu prinçipado e en lu lennorio ¶ \\\hline
3.2.15 & ut se in suo principatu praeseruet . \textbf{ Primo est , non permittere in suo regno transgressiones modicas . } Nam multae modicae transgressiones & para que se pueda man tener en lu prinçipado e en lu lennorio ¶ \textbf{ La primera es que non consienta en su regno muchos pequanos males } ca muchs pequannos males \\\hline
3.2.15 & Tertium est , \textbf{ incutere timorem } iis & puesto que en el sea alguna cosa meztlada de maldat ¶ \textbf{ La terçera cosa que guarda al gouernamiento del regno es meter mie } do aquellos que son enla çibdat \\\hline
3.2.15 & Quartum autem quod politiam saluare videtur , \textbf{ est cauere seditiones et contentiones nobilium ; } et hoc ponendo eis leges , & que salua la poliçia \textbf{ es escusar las discordias | e las contiendas delos nobles } e esto poniendo les leyes \\\hline
3.2.15 & Sunt enim in regno tales leges instituendae , \textbf{ ut per eas sedari possint contentiones nobilium . } Nam baronibus dissentientibus & e son de poner tales leyes en el regno \textbf{ que por ellas se puedan tirar las discordias | e las tales contiendas de los nobles } ca desacordados \\\hline
3.2.15 & et politiam saluat , \textbf{ sicut praeficere homines bonos et virtuosos , } et conferre eis dominia et principatus . & e la poliçia \textbf{ commo poner los bueons e los uirtuosos en las dignidades } e dar les los señorios e los prinçipados . \\\hline
3.2.15 & Septimum saluans regnum et politiam , \textbf{ est Regem siue principantem habere dilectionem } et amorem ad bonum regni , & e la poliçia es \textbf{ que el Rey | e el prinçipe } aya grant amor al bien del regno \\\hline
3.2.15 & Octauum saluans regnum et politiam , \textbf{ est habere ciuilem potentiam . } Nam ( ut dicitur in Magnis moralibus ) & cosa que salua el regno \textbf{ e la poliçia es auer poderio | çiuilca } assi commo dize el philosofo \\\hline
3.2.15 & Debet enim Rex aut Princeps \textbf{ si vult seruare iustitiam } et vult punire transgressores iusti , & ca deue el Rey o el prinçipe \textbf{ si quisiere bien guardar la iustiçia } e si asi ere dar pena alos malos \\\hline
3.2.15 & si vult seruare iustitiam \textbf{ et vult punire transgressores iusti , } habere multos exploratores , & si quisiere bien guardar la iustiçia \textbf{ e si asi ere dar pena alos malos | que trasgre en passando la iustiçia } auer much sassechadores e muchs pesquiridores \\\hline
3.2.15 & et vult punire transgressores iusti , \textbf{ habere multos exploratores , } et multos inquisitores inuestigantes facta ciuium , & que trasgre en passando la iustiçia \textbf{ auer much sassechadores e muchs pesquiridores } que escodrinen \\\hline
3.2.15 & quod expendunt , et quomodo possunt \textbf{ reddere rationem sui victus : } nam qui huiusmodi rationem non potest reddere , & et commo pueden dar razon de su uida \textbf{ e de comm̃ se mantienen } ca aquel que non puede dar razon desto señal \\\hline
3.2.15 & Sic enim faciendo ista , \textbf{ poterit seruare iustitiam , } et praeseruare regnum a maleficis , & que biue de furto o de rapina . \textbf{ ca assi fazie do podra guardar la iustiçia } e guardar el regno de malefiçios \\\hline
3.2.15 & poterit seruare iustitiam , \textbf{ et praeseruare regnum a maleficis , } et transgressoribus iusti . & ca assi fazie do podra guardar la iustiçia \textbf{ e guardar el regno de malefiçios } e de los malos \\\hline
3.2.15 & Nonum maxime saluans regnum , \textbf{ est esse regem bonum et virtuosum . } Nam ut dicitur 5 Politicorum , & La ixͣ cosa que much salua el regno es \textbf{ que el rey sea bueno e uirtuoso . } ca assi commo dize el philosofo \\\hline
3.2.15 & et epiikis idest super iustus : \textbf{ decet enim talem esse quasi semideum , } ut sicut alios dignitate et potentia excellit , & ca conuiene \textbf{ que el tal que sea | assi commo dios } assi que commo lieua auna taia de los otros en dignidat e en poderio \\\hline
3.2.15 & saluare et corrumpere . \textbf{ Talia autem maxime sciri poterunt per experientiam : } nam cum quis diu expertus est regni negocia , & que la pueden saluar e corronper \textbf{ Mas si tales cosas se han de saber mucho | por esperiençia } e por prueua \\\hline
3.2.15 & et saluant . \textbf{ Decet ergo Regem frequenter meditari et habere memoriam praeteritorum } quae contigerunt in regno , & e qual cosa lo salua . \textbf{ Et pues que assi es conuiene al Rey de penssar mucha menudo | e muchͣs uezes delas cosas que passaron . } Et conuiene le de auer memoria de los fecho passados \\\hline
3.2.16 & et qui illorum peruersi , \textbf{ et declarauimus regnum esse optimum principatum , } et tyrannidem pessimum ; & e quales tuertos . \textbf{ Et declaramos en commo el regno era muy buen prinçipado } e la tirania muy malo . \\\hline
3.2.16 & nec propter nostra opera \textbf{ immutari possunt eorum cursus , } ideo circa talia non est consilium adhibendum . & e non se pueden mudar los sus mouimientos \textbf{ por las nuestras obras } por ende nos deuemos tomar conseio \\\hline
3.2.16 & quae sepe contingunt tempore aestiuali , \textbf{ non habet esse consilium : } quia talia naturalia sunt , & que muchas uezes se fazen en el tro del \textbf{ estiuo non auemos a tomar consseio } por que tales cosas \\\hline
3.2.16 & Ideo dicitur in Ethic’ \textbf{ non esse consilium de his , } quae sunt a fortuna , & e por ende dize el philosofo enl terçero libro delas ethins \textbf{ que non ay consero de aquellas cosas } que son auentura \\\hline
3.2.16 & consiliantur de iis operabilibus , \textbf{ quae fieri possunt per ipsos . } Sexto non sunt consiliabilia & cada vne de los omes toma conseio de aquellas obras \textbf{ que se puden fazer por el } ¶Lo vi̊ non caen sosico consseio todas aqllas cosas \\\hline
3.2.16 & oportet enim in consilio \textbf{ praesupponere finaliter intentum , } et non consiliari de ipso , & ca el nuestro consseio non es dela fu . \textbf{ por que conuiene que en el conseio sorongamos la fin } e que non tomemos consseio della \\\hline
3.2.16 & sed de iis per quae \textbf{ consequi possumus illud . } Medicus enim quia finaliter intendit sanitatem , & mas de aquellas cosas \textbf{ por que podemos alcançar aquella fin . } Ca el fisico que entiende la salut del ome \\\hline
3.2.16 & utrum debeat \textbf{ sanare egrum } sed hoc accipit & por su fin non toma conseio \textbf{ si deua sanar el doliente } mas esto toma \\\hline
3.2.17 & et circa naturas rerum , \textbf{ et circa aeterna fieri quaestiones multae , } sed huiusmodi quaestiones consilia & e en las sçiençias delas naturas delas cosas \textbf{ e enlas sçiençias delas cosas perdurables . } Mas tales quastions \\\hline
3.2.17 & quia est quaestio agibilium humanorum : \textbf{ restat videre qualiter est consiliandum , } et quem modum in consiliis habere debemus . & ca es question delas obras \textbf{ que pueden fazer los omes finca de ver | en qual manera es de tomar el conseio } e qual manera deuemos tener en los conseios \\\hline
3.2.17 & non enim consiliatur scriptor \textbf{ ( nisi sit omnino ignorans ) qualiter debeat scribere litteras , } quia hoc sufficienter determinatum est & Ca el esceruano non toma consseio \textbf{ commo esceruir a las letris | si non fuere del todo nesçio } que non sepa en \\\hline
3.2.17 & quanto pluribus modis fieri potest \textbf{ et quanto minus habet certas et determinatas vias , } tanto per plus tempus est consiliandum , & por mas maneras se puede fazer . \textbf{ Et quanto menos ha çiertas | e determinadas carreras } para se fazer tanto mayor tienpo ha menester omne \\\hline
3.2.17 & ut quae sunt apta nata \textbf{ efficere paruum bonum , } vel prohibere modicum malum , & Et por ende las cosas que son muy pequan ans assi que pueden acarrear muy \textbf{ pequano mal o enbargar } pequano bien non son de poner en consseio . \\\hline
3.2.17 & Nam licet homo inter seipsum possit \textbf{ inuenire vias et modos ad aliquid peragendum , } attamen imprudens est & que auemos de fazer \textbf{ ca commo quier que el omne entre ssi mismo pueda fallar carreras e maneras para fazer alguna cosa } enpero non es sabio aquel \\\hline
3.2.17 & qui solo suo capiti innittitur , \textbf{ et renuit aliorum audire sententias . } Magnae enim prudentiae est & que se esfuerça en su cabeça sola \textbf{ e menospreçia de oyr las suinas de los otros } ca de grant sabiduria es en los consseios tener esta manera \\\hline
3.2.17 & Magnae enim prudentiae est \textbf{ in consiliis hunc habere modum : } ut cum aliis conferamus quid agendum , & e menospreçia de oyr las suinas de los otros \textbf{ ca de grant sabiduria es en los consseios tener esta manera } que con los otros ayamos acuerdo \\\hline
3.2.17 & a Con et Sileo \textbf{ ut illud dicatur esse Consilium , } quod simul aliqui plures silent et tacent . & mas por auentura meior po demos \textbf{ dezir que cosseio sea dicha conssilendo } que quiere tanto dezir commo cosa que se deue callar entre muchs camuches eston de guardar en los consseios \\\hline
3.2.17 & ut quod essent adulatores , \textbf{ plus curantes loqui placentia , } quam vera . & ¶ La otra que non sean plazenteros \textbf{ assi que parezcan lisongeros auiendo mayor cuydado de fablar cosas plazenteras que uerdaderas . } En essa misma manera abn segunt dize el pho \\\hline
3.2.17 & quidam poeta nomine Alexander videns \textbf{ Priamum in consiliis esse secretarium et veracem , commendans eum dicebat , } Iste est qui consuluit . & que auie nonbre alixandre veyendo \textbf{ que primero era muy guardado enlos conseios | e muy uerdadero } alabandolo dize del este es aquel que conseia \\\hline
3.2.18 & et creditur dictis eius , \textbf{ quia existimatur bonus consiliarius esse ad persuadendum . } Sed ad hoc quod aliquis sit bene creditiuus , & por que cuydan los omes \textbf{ que es buen consseiero | para dar razon de su consseio . } mas para que alguno sea bien de creer \\\hline
3.2.18 & Sed ad hoc quod aliquis sit bene creditiuus , \textbf{ non oportet ipsum esse existenter talem , } sed sufficit quod videatur & mas para que alguno sea bien de creer \textbf{ non conuiene | que el sea tal fechmas cunple } que parezca tal cael o en iudga las cosas que paresçen de fuera por las cosas que vee \\\hline
3.2.18 & ut credamus eos plus valere quam valeant , \textbf{ et esse meliores quam sint . } Quare si auditores credunt & que valen mas de quanto ualen \textbf{ e que son meiores de quanto son . . } por la qual cosa si los oydores creen alos bien querençiosos \\\hline
3.2.18 & Nam prudentis est , \textbf{ scire et cognoscere ipsas res , } et ipsa negocia agibilia : & ca del sabio es de sabra \textbf{ e de conosçer | daquellas cosas de que fabla } et de aquellos negoçios de que obra . \\\hline
3.2.18 & et haec persuasio est per se : \textbf{ nam reddere se credibilem } et bene persuadere per se , & mas esta creençia et este amonestamiento es por si . \textbf{ Ca fazerse el omne digno de creer } e buen amonestador e razonador por si . \\\hline
3.2.18 & nam reddere se credibilem \textbf{ et bene persuadere per se , } est ex ipsis rebus , & Ca fazerse el omne digno de creer \textbf{ e buen amonestador e razonador por si . } nasçe de aqual las cosas \\\hline
3.2.18 & et ex ipsis negotiis \textbf{ de quibus loquitur scire assumere rationes et argumenta , } per quae fides fiat audientibus . & de que fabla el amonestador \textbf{ e el | razonadorca sabe tomar razones e argumentos } por los quales faga fe alos oydores \\\hline
3.2.18 & satis apparet quales consiliatores deceat \textbf{ quaerere regiam maiestatem ; } quia debet quaerere tales & que ha todo buen conseiero en ssi de fecho . \textbf{ Et por ende assaz parelçe quales conseieros deue auer el rey } ca deue tomar tales \\\hline
3.2.19 & circa haec ergo quinque oportet \textbf{ consiliatores esse instructos . } Primo enim contingit esse Regis consilium circa prouentus , & Et en estas çinco cosas pueden ser enformados \textbf{ e enssenados los conseieros e los sabidores dellas . } Lo primero conuiene que el conseio del Rey \\\hline
3.2.19 & consiliatores esse instructos . \textbf{ Primo enim contingit esse Regis consilium circa prouentus , } in quo duo sunt attendenda . & e enssenados los conseieros e los sabidores dellas . \textbf{ Lo primero conuiene que el conseio del Rey | sea cerca las sus rentas } en la qual cosa dos cosas conuiene de penssar \\\hline
3.2.19 & expedit enim regium consilium \textbf{ pro viribus saluare iura Regis , } eo quod huiusmodi bona ordinanda sunt & ca conuiene que el conseio del Rey sea bue no para saluar \textbf{ por todo su ponder los derechs del Rey . } por que tales biens deuen ser ordenados a bien comun \\\hline
3.2.19 & et prouentus regni , \textbf{ quos oportet peruenire ad regem , } qui et quanti sunt : & Et conuiene que sepan las rentas del regno \textbf{ las que han de venir al Rey quales e quantas son } por que si alguͣ cosa es superflua \\\hline
3.2.19 & apponatur et augeatur . \textbf{ Secundo debet esse consilium de alimento , } ut sciatur utrum ciuitas vel regnum & e sean acresçentadas las sus rentas ¶ \textbf{ Lo segundo deueser tomado consseio en fech delas uiandas | e enla mantenençia de los omes . } por que sea sabido \\\hline
3.2.19 & quia aliter iam non esset ciuitas : \textbf{ ut in huiusmodi sufficientibus ad vitam fieri debent debitae commutationes , } ut debitae emptiones , & al que en otra manera non sene çibdat . \textbf{ Et en estas cosas | que parte nesçen para la uida deuense fazer mudaçiones } e canbios conuenibles \\\hline
3.2.19 & vel etiam totaliter extirpentur , \textbf{ quia Reges et Principes non debent pati maleficos viuere . } Sunt etiam consideranda loca & por pena o echadas dela çibdat o muertos . \textbf{ ca los Reyes e los prinçipes non deuen sofrir | que los malfechores bi una . } Avn son de penssar los logares \\\hline
3.2.19 & ut in ciuitate contingit \textbf{ esse vicos aliquos } magis esse suspectos quam alios : & assi commo contesçe \textbf{ que en la çibdat son alguon suarrios mas sospethosos que los otros . } por que los mal fechores se acostunbraron de esconder se \\\hline
3.2.19 & esse vicos aliquos \textbf{ magis esse suspectos quam alios : } quia iniustificantes ibidem possunt magis latere , & assi commo contesçe \textbf{ que en la çibdat son alguon suarrios mas sospethosos que los otros . } por que los mal fechores se acostunbraron de esconder se \\\hline
3.2.19 & ab extraneis possit \textbf{ suscipere detrimentum : } ideo passagia , portus , introitus et caetera & Et si alguna çibdat del regno \textbf{ puede resçebir danno de los estran nos } por ende los passaies e los puertos e las entradas \\\hline
3.2.19 & et de hoc quod principaliter intenditur , \textbf{ nullus dubitat ipsum esse prosequendum . } De eius autem opposito & e de aquello que prinçipalmente omne entiende ninguno \textbf{ non dubda delo segnir . } Et del contrario della cada vno sabe \\\hline
3.2.19 & Primo , ut nunquam capiatur iniustum bellum , \textbf{ quia iniustificari in alios , } et eos indebite opprimere , & que non sea con razon e con derecho . \textbf{ por que fazer tuerto alos otros } e apremiar sos sin derecho es mala cosa por si \\\hline
3.2.19 & per se est malum , et fugiendum . \textbf{ Deinde , si visum sit bellum esse iustum , } consideranda est potentia regni , vel ciuitatis , & e es muchͣ de escusar \textbf{ despues si fuere iusto | que la guerra es derechͣ } deue ser penssado el poderio del regno o dela çibdat \\\hline
3.2.19 & ad deteriores autem nobis est expugnare \textbf{ vel non pugnare contra eos . } Posito enim quod potentiores , & con los meiores deuemos auer paz \textbf{ mas con los peores en nos es de lidiar o de non lidiar . } Ca puesto que los mas \\\hline
3.2.19 & in nos forefaciant , prudentiae est , \textbf{ non insurgere in ipsos , } nisi occurrat opportunitas temporis , & poderolos alos quales non podemos contradezer nos fagan alguna fuerca o algun tuerto grant sabiduria \textbf{ es non nos leunatar contra ellos } si non fuere en tien \\\hline
3.2.19 & quia absque iustitia nequeunt regna subsistere . \textbf{ Decet autem scire Regem } quot sunt genera dominorum , & nin mucho durar sin iustiçia . \textbf{ Et por ende conuiene al Rey } de saber quantas son las . \\\hline
3.2.20 & Nam in qualibet ciuitate oporteret \textbf{ esse aliquod praetorium ordinarium } ad quod causae reducantur & ca en cada vna çibdat conuiene \textbf{ que aya vna alcalłia otdinaria } ala qual deuen venir todos los pleitos \\\hline
3.2.20 & Quare si legum conditores respectu iudicum sunt pauci , \textbf{ quia facilius est inuenire paucos sapientes , } quam multos , ut omnia sapienter disponantur , & por la qual cosa si los fazedores son pocos en conparaçion de los iuezes \textbf{ porque mas ligera cosa es de fallar pocos sabios que muchs . } por que todas las cosas sean ordenadas sabiamente \\\hline
3.2.20 & vel inimicus sit illa facturus , \textbf{ et debeat illam subire sententiam . } Nam si scirent quod amicus , & aquel que auie de fazer aquella cosa \textbf{ e deuie passar por tal suina . } ca si por auentra asopiessen ellos \\\hline
3.2.20 & quod maxime quidem contingit \textbf{ recte positas leges , } quaecunque possibile est determinare : & que mucho conuiene \textbf{ que las leyes | que son derechamente puestas determinen } quanto pueden ser todas las cosas \\\hline
3.2.20 & sufficienter enim iudex excusatur , \textbf{ cum secundum positas leges aliquid iudicat ; } quia non videtur ipse & e muy pocas cosas son de dexar en aluedrio de los mueze ᷤ \textbf{ quando iudga alguna cosa | segunt las leyes puestas } ca non paresçe \\\hline
3.2.21 & in iudicio prohibeantur : \textbf{ multi enim litigantium cognoscentes se habere malam causam , } non narrant quid factum et quid non factum , & assi comm̃ayra a abortençia sean defendidas en łmyzio \textbf{ ca muchos de los que contienden en iuyzio | sabiendo que tienen mal pleito } non cuentan lo que es fecho \\\hline
3.2.21 & ex eo quod huiusmodi sermones \textbf{ obligare habent iudicem , } quem esse oportet & ante los alcalłs rodemos lo prouar por tres razones ¶ \textbf{ La primera seqma par aquello que tales palabras han de to terçeres desegualar eliez } el qual conuiene de ser \\\hline
3.2.21 & ut recte iudicet , \textbf{ sic debet se habere inter partes litigantes , } sicut lingua volens discernere de saporibus , & para que derechamente iudgue \textbf{ assi se deue auer entre las partes | que contienden } commo la lengua \\\hline
3.2.21 & de proprii sensibilibus , \textbf{ habere se debet } inter ipsos sapores , & que sienten propriamente \textbf{ ca la lengua se deue auer entre los sabores } o cada vno de los otros sesos en las cosas \\\hline
3.2.21 & infecto aliquo humore , recte iudicat , \textbf{ dicens amarum esse amarum , } et dulce dulce . & por algun humor iudga derechͣmente diziendo \textbf{ que lo amargo es amargo } e lo dulçe es dulçe . \\\hline
3.2.21 & peruersae iudicat , \textbf{ dicens dulce esse amarum , } et econuerso , & algundelas partes contrarias iudga mal diziendo \textbf{ que lo dulçe es amargo } e lo amargo es dulçe . \\\hline
3.2.21 & quasi regula recta decet \textbf{ iustum esse iustum } et iniustum iniustum . & assi commo regla derecha diziendo \textbf{ e mostrando lo que es derecho | que es derech } e lo que es tuerto que es tuerto . \\\hline
3.2.21 & et quia hoc faciunt sermones passionales , \textbf{ permittere talia in iudicio nihil est aliud quam regulam obliquare : } quasi si inconueniens est & Et por que esto fazen las palabras desiguales \textbf{ et malas consentir tales palabras en el iuyzio non es otra cosa | si non torçer la regla } que non iudgue derecho \\\hline
3.2.21 & quasi si inconueniens est \textbf{ permittere obliquari regulam , } inconueniens est sustinere & cosasi non es cosa conueible \textbf{ que la regla se tuerca } non es cosa conuenible de sofrir \\\hline
3.2.21 & Dato tamen quod contingat \textbf{ sustinere aliquos passionales sermones , } quia ( ut in sequentibus patebit ) & que enco el uuzio se digan palabras malas e desiguales \textbf{ ca assi commo parezçca en lo que es de dezer los mezes } mas enclinados deuen sera auer piedat \\\hline
3.2.21 & inclinent voluntatem , \textbf{ et faciant apparere aliquid iustum , } vel non iustum , & enclinan la uoluntad de los omes \textbf{ e fagan paresçer alguna cosa derecha . } por que los que assi son munnidos \\\hline
3.2.21 & non pariter iudicamus , \textbf{ permittere passionales sermones in iudicio , } est peruertere ordinem iudicandi : & Et por ende iudgan ygual mente . \textbf{ Et pues que assi es conssentir tales palabras } en iuyzio es trastornar la orden de iudgar . \\\hline
3.2.21 & permittere passionales sermones in iudicio , \textbf{ est peruertere ordinem iudicandi : } quia est facere & Et pues que assi es conssentir tales palabras \textbf{ en iuyzio es trastornar la orden de iudgar . } Ca es iudgar las partes \\\hline
3.2.21 & et quod teneant supremum gradum in iudicando , \textbf{ quae debent tenere infimum . } Peruertitur ibi talis ordo , & e tener elguado primero en iudgar \textbf{ aquellos que deuen tener el postrimero } e assi se trastorna \\\hline
3.2.21 & quia partes passionando iudicem , \textbf{ ei faciunt apparere aliquid iustum vel iniustum , } quod non est officium partium , & Ca las partes mouiendo el iuez \textbf{ assi fazen paresçer | alguͣ cosa derechͣo non de rethica . } la qual cosa es ofiçio del ponedor dela ley . \\\hline
3.2.21 & de quo est litigium : \textbf{ passionare autem iudicem , } aut narrare iniurias & de que contienden \textbf{ Mas enduzir al iiez } por palabras contando le las miurias . \\\hline
3.2.21 & passionare autem iudicem , \textbf{ aut narrare iniurias } quas pars aduersa iudici intulit , & Mas enduzir al iiez \textbf{ por palabras contando le las miurias . } las quales la parte contraria fizo al iuez \\\hline
3.2.21 & quas pars aduersa iudici intulit , \textbf{ vel narrare bona } quae ipsi iudici contulerunt , & las quales la parte contraria fizo al iuez \textbf{ o contando le los bienes } que ellos fizieron a liiez . \\\hline
3.2.21 & quae ipsi iudici contulerunt , \textbf{ et hoc modo prouocare iudicem } ad maliuolentiam partis aduersae , & que ellos fizieron a liiez . \textbf{ Et en esta manera inclinar al iuez a malenconia } e a mal querençia dela parte contraria \\\hline
3.2.22 & faciunt iudicium usurpatum . \textbf{ Tunc quidem dicuntur iudices non recte se habere ad legislatorem , } quando excedunt auctoritatem sibi commissam . & fazen que el iuyzio sea fortado . \textbf{ Et entonçe los iuezes non se han derechamente al fazedor dela ley } quando sobressallen dela auctoridat \\\hline
3.2.22 & Secundo dicuntur iudices \textbf{ facere iudicium temerarium , } si non recte se habent ad leges , & que les es a comne dada . \textbf{ Lo segundo los iuezes fazen iuyzio loco } quando non se han derechamente \\\hline
3.2.23 & ad quae decet \textbf{ respicere iudicem , } ut humanis indulgeat , & en el primero libro \textbf{ uiene que tenga el iuez sienpre mientes } para que perdone alas obras de los omes \\\hline
3.2.23 & et supra iustitiam . \textbf{ Secundum quod inclinare debet iudicem ad clementiam , } est ipse legislator . & nin que la iustiçia afincada ¶ \textbf{ Lo segundo que deue inclinar al iues a piedat } es el establesçedor dela ley . \\\hline
3.2.23 & quod iudicans potius debet \textbf{ respicere ad legislatorem , } quam ad leges . & que el iuez \textbf{ mas deue tener mientes al ponedor dela ley } que alas leyes . \\\hline
3.2.23 & quod iudicans debet \textbf{ aspicere non ad actionem , } sed ad electionem . & que el que iudga non deue tener \textbf{ mientesa la obra mas ala entençion . } Lo quinto que enduze el iuez ami bicordia \\\hline
3.2.23 & multa bona opera prius fecit : \textbf{ debet ergo iudex non ita respicere ad partem } ut ad hoc particulare negocium in quo delinquunt , & ca por auentra a aquel que agora peca fizo ante muchas bueans obras \textbf{ Et por ende eliez non deue } assi catara vna obra particular \\\hline
3.2.23 & debet ergo iudex non ita respicere ad partem \textbf{ ut ad hoc particulare negocium in quo delinquunt , } sicut ad totum & Et por ende eliez non deue \textbf{ assi catara vna obra particular } en que peco commo a todos los bienes \\\hline
3.2.23 & ideo dicitur 1 Rhetor’ \textbf{ quod iudicans non debet respicere ad partem , } sed ad totum . & en el primero libro de la rectonca \textbf{ que el uiez non deue catar ala parte mas al todo | nin deue tener mientes a vna obra } que fizo mas a todas las buenas obras \\\hline
3.2.23 & Sextum est diuturnitas temporis retroacti . \textbf{ Nam contingit etiam in pauco tempore facere multa bona opera : } duo ergo debent inducere Regem & Lo sexto que inclina ali es a piedat es alongamiento \textbf{ detpo passado por que contesçe que alas uezes alguno en poco tp̃o faze muchas buenas obras . } Et por ende dos cosas deuen endozir al Rey o al prinçipe \\\hline
3.2.23 & Nam contingit etiam in pauco tempore facere multa bona opera : \textbf{ duo ergo debent inducere Regem } aut quemcunque alium dominum & detpo passado por que contesçe que alas uezes alguno en poco tp̃o faze muchas buenas obras . \textbf{ Et por ende dos cosas deuen endozir al Rey o al prinçipe } o a qual quier otro sennor \\\hline
3.2.23 & si viderit delinquentem \textbf{ magis velle ire ad arbitrium , } quam ad disceptationem : & si uiere \textbf{ que el que pecaua mas al aluedrio del iuez | que non a escusar se } e a disputar con el . \\\hline
3.2.23 & Patet ergo quomodo decet \textbf{ iudices esse magis clementes quam seueros : } et si hoc decet iudices , & en qual manera conuiene \textbf{ que los miezes sean mas piadosos que crueles } Et si esto conuiene alos iuezes mucho \\\hline
3.2.24 & quod eo modo quo distinguimus ius siue iustum , \textbf{ distinguere possumus leges ipsas , } et econuerso . & e usta podemos departir las leyes \textbf{ e por el contrario por las leyes podemos deꝑtir | que cosa es derecho } e que cosa non es derecho . \\\hline
3.2.24 & ius naturale a iure gentium , \textbf{ possemus separare nos ius naturale } a iure animalium : & era que los iuristas apartan el derecho natural del derecho delas \textbf{ gentes podemos nos apartar el derecho natural del derecho delas animalias } e darla quanta distinçion \\\hline
3.2.24 & a iure animalium : \textbf{ et dare quintam distinctionem iuris , } dicendo quod quadruplex est ius , & gentes podemos nos apartar el derecho natural del derecho delas animalias \textbf{ e darla quanta distinçion | e el quinto departimiento del derech̃ . } diziendo que en quatro maneras se departe el derech . \\\hline
3.2.24 & postquam autem est editum incipit \textbf{ habere ligandi efficaciam . } Ratio autem , & mas despues que es puesto a fuerça \textbf{ de obligar alos omes . } Mas la razon por que al derech natural conuinio anneder derecho positiuo es esta \\\hline
3.2.24 & sermonem nobis esse datum a natura . \textbf{ Sicut ergo loqui est naturale , } sic autem loqui vel sic , est positiuum et ad placitum . & que la palabra non es dada por natura . \textbf{ Et pues que assi es | assi commo fablar es cosa natural alos omes } assi fablar tal \\\hline
3.2.24 & Sic , fures punire , \textbf{ non pati maleficos viuere , } et cetera huiusmodi sunt , & bien assi dar pena alos ladrones \textbf{ e non sofrir beuir los malos } e o tristales cosas son de \\\hline
3.2.24 & Ubi ergo terminatur ius naturale , \textbf{ ibi incipit oriri ius positiuum : } quia semper quae sunt & ose termina el derecho natural \textbf{ alli comiença a naçer | el derech posituio } por que sienpre aquellas cosas que son falladas \\\hline
3.2.24 & quae sunt naturae . \textbf{ Quare si ius naturale dictat fures et maleficos esse puniendos , } hoc praesupponens ius positiuum procedit ulterius , & que son dela natura \textbf{ Por la qual cosa si el derecho natural manda | que los ladrones } e los mas fechores sean castigados \\\hline
3.2.25 & secundum quem modum loquendi potest \textbf{ ibi addi membrum quartum , } ut ius animalium . & assi commo es el derecho delas gentes . \textbf{ Et segunt esta manera de fablar podemos ennader el quarto mienbro } que es derecho delas aian lias . \\\hline
3.2.25 & ut conuenimus cum animalibus aliis : \textbf{ sic dicitur esse ius naturale . } Ideo in Instituta , & siguiere la nuestra natura en quanto auemos conueniençia con las otras aian lias \textbf{ assi es dich derech natural . } Et por ende en la instituta del derecho natural \\\hline
3.2.25 & ut communicamus cum animalibus aliis , \textbf{ respectu iuris gentium dicitur esse naturale . } Nam si considerentur dicta in praecedenti capitulo , & en quanto participamos con las otras aianlas \textbf{ en conparacion del derecho delas gentes es dicho derecho natural . } Ca si penssaremos los dichos del capitulo \\\hline
3.2.25 & Poterit ergo inclinatio naturalis \textbf{ sequi naturam hominis } vel ut homo est , & Et por ende la inclinacion natural \textbf{ puede seguir la natura del ome } o en quanto es omne o en quanto conuiene con todas las \\\hline
3.2.25 & quod et omnia entia alia appetunt : \textbf{ naturaliter appetit producere filios , educare prolem , } quod et alia animalia concupiscunt : & naturalnse te dessea ser guardado en su ser Ra qual cola avn del sean todas las otras cosas que son . \textbf{ avn el omne natutalmente dessea de auer fijos | e de criar los . } Ca esto dessean todas las otras aina las natraalmente \\\hline
3.2.25 & in societate viuere \textbf{ secundum debitas conuentiones et pacta ; } sic erit de iure naturali , & dessea beuir en conpannia \textbf{ segunt | establesçimientos e posturas cs̃uenbles } assi seran de derecho natural \\\hline
3.2.25 & et communius illo : \textbf{ nam appetere bonum et esse , } et fugere malum & e es mas comun qual otro . \textbf{ Ca dessear bien e dessear ser } e foyr el mal \\\hline
3.2.25 & est plus de iure naturali , \textbf{ quam appetere procreare filios , } et nutrire prolem . & mas de derech natural \textbf{ que dessear de engendrar fijos e criar los . } Et pues que assi es esta sera la orden entre estos de ti xu rechos \\\hline
3.2.25 & quam appetere procreare filios , \textbf{ et nutrire prolem . } Erit igitur hic ordo , & mas de derech natural \textbf{ que dessear de engendrar fijos e criar los . } Et pues que assi es esta sera la orden entre estos de ti xu rechos \\\hline
3.2.25 & quod ius consequens naturam nostram \textbf{ prout appetimus esse et bonum , } est naturale respectu iuris animalium , & que el derecho que ligue lanr̃a natura en quanto desseamos ser \textbf{ e desseamos bien es natural en conparaçion del derech delas aianlias } o en conparaçion del derech \\\hline
3.2.25 & Si ut conuenit cum animalibus aliis , \textbf{ sic habet esse ius illud , } quod natura omnia animalia docuit . & Mas en quanto conuiene el omne con todas las otras \textbf{ ai alias | assi se toma aquel derecho } que la natura demostro \\\hline
3.2.25 & cum omnibus entibus , \textbf{ sic habet esse ius illud , } quod per antonomasiam dicitur esse naturale . & con todas las sustançias \textbf{ assi se toma aquel derech } que es dicho natural pora un ataia de los otros derechos . \\\hline
3.2.25 & sic habet esse ius illud , \textbf{ quod per antonomasiam dicitur esse naturale . } Appetere enim esse et bonum , & assi se toma aquel derech \textbf{ que es dicho natural pora un ataia de los otros derechos . } Por que dessear el bien e el ser \\\hline
3.2.25 & quod per antonomasiam dicitur esse naturale . \textbf{ Appetere enim esse et bonum , } et fugere non esse et malum , & que es dicho natural pora un ataia de los otros derechos . \textbf{ Por que dessear el bien e el ser } e foyr el non ser \\\hline
3.2.26 & ad quem debet applicari \textbf{ et debet regulari per huiusmodi legem , } oportet quod sit competens & a que es dada \textbf{ el qual pueblo deueser reglado por aquella ley . } conuiene que sea conuenible \\\hline
3.2.26 & Primo igitur oportet legem humanam \textbf{ siue positiuam esse iustam } ut comparatur ad rationem naturalem & ues que assi es . \textbf{ Lo primero conuiene que la ley humanal o positiua sea derecha } en quanto es conparada ala razon natural o ala ley de natura . \\\hline
3.2.26 & quod non oportet \textbf{ adaptare politias legibus , } sed leges politiae , & que non conuiene de apropar las comunidades \textbf{ delas çibdades alas leyes . } Mas las leyes alas comunidades \\\hline
3.2.26 & ut saltem metu poenae volentes \textbf{ impedire pacem ciuium , } desisterent agere peruerse . & por que si quier \textbf{ por miedo dela pena los que quesiessen enbargar la paz de los çibdadanos dexassen de obrar mal } as leyes assi commo paresçe \\\hline
3.2.27 & nam cuius est ordinare \textbf{ et dirigere aliquos in aliquod bonum , } eiusdem est condere leges , & La primera razon assi . \textbf{ Ca aquel cuyo es de ordenar e enderesçar a alguos en algun bien atlgun aquel parte nesçe } establesçer leyes e reglas delas nuestras obras . \\\hline
3.2.27 & et dirigere aliquos in aliquod bonum , \textbf{ eiusdem est condere leges , } et regulas agibilium & Ca aquel cuyo es de ordenar e enderesçar a alguos en algun bien atlgun aquel parte nesçe \textbf{ establesçer leyes e reglas delas nuestras obras . } por las quales leyes ymosa aquel bien . \\\hline
3.2.27 & cuius est ordinare \textbf{ et dirigere alios in tale bonum , } vel condendae sunt a toto populo , & deuen ser establesçidas del prinçipe \textbf{ a quien parte nesçe ordenar e enderesçar los otros atal bien } o deuen ser establesçidas de todo el pueblo \\\hline
3.2.27 & si totus populus principetur , \textbf{ et sit in potestate eius eligere principantem : } Nulla est ergo lex , & si todo el pueblo en ssennorea \textbf{ e si en su poder es de escoger el prinçipe . Et pues que assi es la ley non es ninguna } si non es establesçida \\\hline
3.2.27 & quae non sit edita \textbf{ ab eo cuius est dirigere in bonum commune : } nam si est lex diuina et naturalis , & por aquel \textbf{ a quien parte nesçe | de enderesçar los omes al bien comun . } Ca si es ley diuinal e natural establesçida es de dios \\\hline
3.2.27 & Princeps enim aut totus populus cum principatur , \textbf{ habet dirigere et ordinare alios in commune bonum . } Quaelibet ergo persona particularis , & Ca el prinçipe o avn todo el pueblo \textbf{ quando enssennorea ha de ordenar | e de enderesçar todos los otros al bien comun . } Et pues que assi es cada vna \\\hline
3.2.27 & ideo in quolibet homine haec promulgatur et propalatur , \textbf{ quando incipit habere rationis usum , } per quam cognoscit & que en cada vn omne es publicada e manifestada \textbf{ quando comiença de auer uso de razon e de entendimiento } por el qual conosçe qual cosa ha de fazer e de escoger \\\hline
3.2.28 & Ostendimus in praecedentibus capitulis , \textbf{ quales debent esse leges condendae } a Regibus et Principibus & a demostramos enlos capitulos sobredichos quales deuen ser las leyes \textbf{ que son de poner } por los Reyes \\\hline
3.2.28 & et quae et quot opera debent \textbf{ continere huiusmodi leges . } Dicuntur autem quinque esse effectus legum , & quales son los fechs delas leyes \textbf{ e quales e quantas obras deuen contener estas leyes } e conuiene de sabra \\\hline
3.2.28 & continere huiusmodi leges . \textbf{ Dicuntur autem quinque esse effectus legum , } vel quinque esse opera legalia , & e quales e quantas obras deuen contener estas leyes \textbf{ e conuiene de sabra | que çinco son los fechos } o las obras delas leyes \\\hline
3.2.28 & Dicuntur autem quinque esse effectus legum , \textbf{ vel quinque esse opera legalia , } videlicet praecipere , permittere , prohibere , praemiare , et punire . & que çinco son los fechos \textbf{ o las obras delas leyes | que son estas . } Mandar ¶ Conssentir . \\\hline
3.2.28 & quae in ea traduntur , \textbf{ vult regulare et aequare humanos humores : } sic scientia politica & ø \\\hline
3.2.28 & vult aequare \textbf{ et regulare actiones humanas , } ut ciues iuste viuant , & que le contienen en aquella sçiençia \textbf{ por que los çibdadanos biuna derechamente } e saayan conmose deuen auer . \\\hline
3.2.28 & vel mala et vituperabilia . \textbf{ Ut sic eleuare festucam de terra , } de se est opus indifferens : & del que obra pueden ser buenas e de loar o malas e de denostar assi \textbf{ conmoleunatar la paia de tierra } de si es obra \\\hline
3.2.28 & si quis tamen mala intentione eleuaret illam , \textbf{ ut quia vellet ponere in oculum socii , } esset opus prauum et vituperabile : & leunatare con mala entençion \textbf{ para poner la en el oio a su } conpannon es mala obra e de deno star . \\\hline
3.2.28 & esset opus prauum et vituperabile : \textbf{ si vero eleuando eam vellet purgare domum } vel facere aliquod aliud opus pium , & conpannon es mala obra e de deno star . \textbf{ Mas si la leuna tare | para alinpiat la casa } o para fazer alguna otra obra buean \\\hline
3.2.28 & si vero eleuando eam vellet purgare domum \textbf{ vel facere aliquod aliud opus pium , } propter bonam intentionem operantis , & para alinpiat la casa \textbf{ o para fazer alguna otra obra buean } por la buena entençion \\\hline
3.2.28 & et ciuitatis cura peruigili \textbf{ insudare quas leges , } et quae instituta imponant ciuibus , & e delas çibdadeᷤ \textbf{ assi que con grant cuydado e con grant estudio deuen trabaiar quales leyes } e quales establesçimientos pongan a sus çibdadanos . \\\hline
3.2.29 & Secunda ex eo quod facilius est \textbf{ corrumpi Regem quam legem . } Prima via sic patet . & La segunda se toma de aquello que mas ligera cosa es es de se \textbf{ corronper el rey que la ley ¶ la primera razon paresçe assi . Ca assi commo dize el philosofo } en el quinto libro delas ethicas \\\hline
3.2.29 & quia cum optimus homo incipit furire \textbf{ et concupiscere peruersa , } et si non interimitur quantum ad esse simpliciter , & enpero matase quanto al ser muy bueno . \textbf{ por que quando el muy bueno en se en comiença de enssennar e de cobdiçiar las cosas malas } si se non mata \\\hline
3.2.29 & quia est aliquid pertinens ad rationem , \textbf{ videtur dicere intellectum solum : } ideo dicitur 3 Polit’ & que parte nesçe \textbf{ a razon paresçe | que diga entendimiento solo . } Et por ende dize el pho en el terçero delas politicas \\\hline
3.2.29 & quod qui iubet principari intellectum , \textbf{ iubet principari deum et legem ; } sed qui iubet principari hominem , & enssenerorear al entendimiento \textbf{ manda | enssennorear a dios e ala ley . } as quien manda \\\hline
3.2.29 & iubet principari deum et legem ; \textbf{ sed qui iubet principari hominem , } propter concupiscentiam annexam apponit & enssennorear a dios e ala ley . \textbf{ as quien manda | enssennorear al omne } por la cobdiçia se allega ael manda que \\\hline
3.2.29 & His ergo rationibus videtur ostendi , \textbf{ melius esse regnum et ciuitatem Regi lege , } quam Rege . & Et por esta razon e paresçe ser mostrado \textbf{ que meior es que el regno o la çibdat se gouernada } por ley que por Rey . \\\hline
3.2.29 & Quare si nomen regis a regendo sumptum est , \textbf{ et decet Regem regere alios , } et esse regulam aliorum , & sy el nonbre del Rey es tomado de gouernamiento . \textbf{ Conuiene al rey de gouernar los otros } e de ser regla de los otros . \\\hline
3.2.29 & oportet Regem in regendo alios \textbf{ sequi rectam rationem , } et per consequens sequi naturalem legem , & Et assi se sigue \textbf{ que sigua la ley natural | la qual se leunata de razon derecha e de entendumento derecho . } Et por ende el rey en gouernando es a \\\hline
3.2.29 & sequi rectam rationem , \textbf{ et per consequens sequi naturalem legem , } quia in tantum recte regit , & la qual se leunata de razon derecha e de entendumento derecho . \textbf{ Et por ende el rey en gouernando es a | quande dela ley natural } por que en tanto gouierna derechamente \\\hline
3.2.29 & cum quis non innititur \textbf{ regere alios ratione } sed passione et concupiscentia , & que la bestia en ssennorea \textbf{ quando alguno non se esfuerça de gouernar los otros } por razon e por entendimiento mas por passion \\\hline
3.2.29 & in mente cuiuslibet hominis , \textbf{ dirigere legem positiuam , } et esse supra iustitiam legalem , & la qual dios puso en voluntad de cada vn omne \textbf{ que enderesçe la ley positiua } e que sea sobre la iustiçia legal \\\hline
3.2.29 & dirigere legem positiuam , \textbf{ et esse supra iustitiam legalem , } et non obseruare legem , & que enderesçe la ley positiua \textbf{ e que sea sobre la iustiçia legal } e qua non guarde la ley positiua \\\hline
3.2.29 & et esse supra iustitiam legalem , \textbf{ et non obseruare legem , } ubi non est obseruanda . & e que sea sobre la iustiçia legal \textbf{ e qua non guarde la ley positiua } do non la deue guardar . \\\hline
3.2.29 & quae sit applicabilis humanis actibus . \textbf{ Oportet igitur aliquando legem plicare ad partem unam , } et agere mitius cum delinquente , & e allegar alas obras delos omes . \textbf{ Et por ende conuiene quela ley que se ençorue } e se allegue algunas vezes ala vna parte e que obre mas manssamente con el que peca \\\hline
3.2.29 & quam lex dictat : \textbf{ aliquando etiam oportet eam plicare ad partem oppositam , } et rigidius punire peccantem , & quela ley demanda o que la ley nidga . \textbf{ Et algunas vezes conuiene que la regla se encorue | ala parte contraria } e que mas reziamente de pena \\\hline
3.2.29 & aliquando etiam oportet eam plicare ad partem oppositam , \textbf{ et rigidius punire peccantem , } quam lex determinet . & ala parte contraria \textbf{ e que mas reziamente de pena } al que peca que la ley demandan que determina . \\\hline
3.2.30 & expediens \textbf{ dare legem euangelicam et diuinam , } triplici via possumus venari & ø \\\hline
3.2.30 & Prima est , quia communiter populus non potest \textbf{ attingere punctalem formam viuendi , } ideo oportet aliqua peccata dissimulare & la qual cosa contesçe por dos razones La primera es por que el pueblo \textbf{ comunalmente non puede alcançar forma de beuir en punto . } Por ende conuiene que \\\hline
3.2.30 & Oportuit igitur praeter legem humanam \textbf{ dari aliquam legem , } ut nullum malum remaneret impunitum , & Et por ende conuiene que sin la ley humanal fuesse \textbf{ dada otra ley diuinal | e e un agłica l . } por que ningun mal non fincasse sin pena \\\hline
3.2.30 & ut in prosequendo patebit : \textbf{ oportuit igitur dare legem euangelicam et diuinam , } secundum quam prohiberentur & assi commo paresçra adelante . \textbf{ Et por ende conuiene de dar ley diuinal } e e un agłical segunt la qual fuessen vedados los pecados todos . \\\hline
3.2.30 & sed quantum ad punitionem , \textbf{ sic dicitur non prohibere mentem et animum , } eo quod talia delicta non puniat . & humanal non quanto ala entençion del ponedor della \textbf{ mas quanto ala pena que pone assi digo que non defiende la uoluntad e el coraçon . } por que non condep̃na tales pecados \\\hline
3.2.30 & ut vitentur adulteria . \textbf{ Secunda via ostendens necessariam esse legem euangelicam et diuinam , } sumitur ex parte cognitionis nostrae , & por que sean escusados los adulterios . \textbf{ ¶ La segunda razon que muestra la ley en angelical } e diuinal ser neçessaria es tomada deꝑte del nuestro conosçimiento \\\hline
3.2.30 & et regulam agibilium , \textbf{ sic se habere ad legem diuinam , } naturalem , et humanam : & e de ser forma de beuir e regla de todas las obras . \textbf{ assi se auer ala leyna traal e diuinal e humanal } por que assi commo sobrepuian los otros en poderio e en diuinidat \\\hline
3.2.31 & utrum sit expediens ciuitatibus \textbf{ innouare patrias leges , } et inducere nouas consuetudines . & si es cosa conuenible alas çibdades \textbf{ de renouar las leyes dela tierra } e de enduzir nueuas costunbres \\\hline
3.2.31 & innouare patrias leges , \textbf{ et inducere nouas consuetudines . } Ordinauerat enim Hippodamus & de renouar las leyes dela tierra \textbf{ e de enduzir nueuas costunbres } por que ypodomio ordenara \\\hline
3.2.31 & ut inuenientes consuetudines nouas , \textbf{ dicentes eas esse utiles et proficuas ciuitati , } soluerent leges patrias et antiquas . & para fallar costunbres nueuas \textbf{ diziendo que aquellas eran prouechosas ala çibdat . } Et en esto desfazien e destruyen las leyes antiguas dela tr̃ra . \\\hline
3.2.31 & an positio Hippodami esset bona , \textbf{ et an expediat saepe saepius immutare leges : } dato etiam quod occurrant leges aliquae & Et por ende non sin razon dubdauna si la opinion de ypodomio era buena \textbf{ e si conuinie de renouar | e mudar las leyes muchͣ suegadas } puesto avn que algunas leyes fuessen falladas \\\hline
3.2.31 & per quas videtur ostendi , \textbf{ quod expediat innouare leges . } Prima sumitur ex parte scientiarum et artium . & que muestra \textbf{ que conuiene de renouar las leyes ¶ } La primera se toma de parte delas sçiençias e delas artes . \\\hline
3.2.31 & contingit \textbf{ esse malas et barbaricas : } sicut erat lex olim apud Graecos , & Ca contesçe que algunas leyes dela tr̃ra \textbf{ assi commo dize el philosofo son deseguales e malas e barbaricas } assi commo fue aquella ley que era establesçida entre los gniegos . \\\hline
3.2.31 & ut ciues possent uxores suas vendere . \textbf{ Sic etiam contingit leges aliquas esse stultas , } utputa legem illam quam & por las quales los çibdadanos pudiessen vender so mugers . \textbf{ assi avn contesçe que algunas leyes son locas } assi commo aquella ley \\\hline
3.2.31 & Nam si aliquando condentes leges contingit \textbf{ esse simplices , } irrationale esset , & que los fazedores delas leyes son sinples \textbf{ e de poco saber . } Et por ende cosa sin razon seria \\\hline
3.2.31 & si posteriores sapientiores non possent \textbf{ immutare leges paternas } per simpliciores conditas : & Et por ende cosa sin razon seria \textbf{ si los sabios postrimos non pudiessen mudar las leyes } delatrraque fueron establesçidas \\\hline
3.2.31 & occurrit aliquid melius , \textbf{ inconueniens est non remouere leges paternas } et antiquas propter meliores leges nouiter inuentas . & por la esperiençia delas obras particulares alguna cosa fallar en meior \textbf{ non es cosa sin razon de tirar las leyes dela tierra antiguas } por las meiores leyes falladas nueuamente por ellos . \\\hline
3.2.31 & valde periculosum ciuitati et regno . \textbf{ Nam assuescere inducere nouas leges } ( ut innuit Philosophus 2 Pol’ ) & sinplemente es muy perigloso ala çibdat e altegno . \textbf{ Ca acostunbrar se los omes | afaznueuas leyes } assi commo dize elpho \\\hline
3.2.31 & et per consequens est \textbf{ tollere principatum et regnum . } Quantum autem malum sequitur & e alos prinçipes dela qual cosa se signirie \textbf{ que se tiraria el prinçipado e el regno . } Mas quanto mal se se sigue \\\hline
3.2.31 & volunt enim corpora \textbf{ inducere ad sanitatem . } Sed veri legislatores & Ca los fisicos entienden enla sanidat del cuerpo . \textbf{ por que quieren adozir el cuerpo a sanidat . Mas los uerdaderos ordenadores delas leyes } e los uerdaderos Reyes sinplemente entienden en el bien del alma \\\hline
3.2.31 & quia intendunt ciues \textbf{ inducere ad virtutem . } Ut ergo appareat & e los uerdaderos Reyes sinplemente entienden en el bien del alma \textbf{ por que entienden de adozir los çibdadanos a uirtud Et pues que assi es por que paresça lo que deuemostener desta question } e qual es la soluçion della . \\\hline
3.2.31 & per leges posita . \textbf{ Sed sic appellare aliquid } contra naturam esse , & por las leyes aprouecho de los ons \textbf{ Mhas dezir que alguna cosa es } assi contra natura el fablar rudamente e nesçiamente \\\hline
3.2.31 & dato quod occurrant leges meliores et magis sufficientes , \textbf{ non est assuescendum innouare leges . } Primo , quia aliquando contingit & que sean falladas leyes meiores e mas conplidas . \textbf{ Enpero non nos auemos a acostunbrar a renouar las leyes . } Lo primero por que algunans vegadas contesçe que se engannan los omes \\\hline
3.2.31 & Immo magnam efficaciam habent ex diuturnitate et assuefactione . \textbf{ Decet ergo reges et principes obseruare bonas consuetudines principatus et regni , } et non innouare patrias leges , & Et por ende conuiene alos Reyes \textbf{ e alos prinçipes | de guardar las bueans costunbres del prinçipado e del regno } e non renouar las leyes dela tierra saluo \\\hline
3.2.31 & Decet ergo reges et principes obseruare bonas consuetudines principatus et regni , \textbf{ et non innouare patrias leges , } nisi fuerit rectae rationi contrariae . & de guardar las bueans costunbres del prinçipado e del regno \textbf{ e non renouar las leyes dela tierra saluo } si fuessen contrarias ala razon natural \\\hline
3.2.32 & et quomodo debeat \textbf{ se habere ad principantem , } non modicum amminiculetur & Mas commo para saber qual deua ser el puebło \textbf{ e commo se deua auer al prinçipe } e conuenga de saber \\\hline
3.2.32 & non est sufficiens resistere impugnantibus , \textbf{ et vitare iniurias } et iniustitias sibi factas ; & defenderde los que mal le quisiessen \textbf{ nin poder a escusar las imiurias } e los tuertos \\\hline
3.2.32 & cum de legibus tractabamus , \textbf{ quod facere commutationes , } et contractus erant & quando fablauamos delas leyes \textbf{ que fazer mudaçiones } e contracto sera \\\hline
3.2.32 & Nam regnum supra ciuitatem videtur \textbf{ addere multitudinem nobilium et ingenuorum . } Est enim ciuitas pars regni ; & Ca el regno eñade sobre la çibdat muchedunbre \textbf{ de nobles omes e de alto linage . } por que la çibdat es parte del regno . \\\hline
3.2.32 & quam in ciuitate una . \textbf{ Potest ergo sic diffiniri regnum , } quod est multitudo magna , & que en vna çibdat . \textbf{ Et pues que assi es el regno puede se | assi declarar e demostrar } diziendo \\\hline
3.2.32 & decet nobiles et ingenuos \textbf{ esse magis bonos et virtuosos } quam ciues alios : & que los nobles e los altos \textbf{ e los mas fijos dalgo sean mas buenos | e mas uirtuosos } que los otros çibdadanos . \\\hline
3.2.32 & propter quod regem ipsum tanquam omnibus excellentiorem \textbf{ decet esse optimum , } et quasi semideum . & assi commo aquel que sobrepula todos los otros en dignidat \textbf{ e en pero de rio sea muy bueno } e sea assi commo medio dios . \\\hline
3.2.32 & in ciuitate et regno , \textbf{ oportet esse talem , } quod viuat bene et virtuose . & que es en el regno e enla çibdat . \textbf{ conuiene que sea atal que biuna bien e uirtuosamente . } Et por ende assi conmo dize el philosofo en el terçero libro delas politicas \\\hline
3.2.33 & quod tres oportet \textbf{ esse partes ciuitatis . } Nam alii quidem sunt opulenti valde , & uenta el philosofo en el quarto libro delas politicas \textbf{ que conuiene que sean tres partes dela çibdat . } Ca alguons son muy ricos . \\\hline
3.2.33 & ostendere \textbf{ optimam esse ciuitatem et regnum , } si ibi sit populus & Mas la entençion deste capitulo es mostrar \textbf{ que es muy buena la çibdat e el regno } si y fuere pueblo establesçido de muchͣs perssonas medianeras \\\hline
3.2.33 & ex quibus sumi possunt quatuor viae , \textbf{ ostendentes meliorem esse politiam , } vel melius esse regnum et ciuitatem , & delas quales se pueden tomar quatro razonnes \textbf{ que muestran que meior es la poliçia } o meior es el regno o la çibdat \\\hline
3.2.33 & ostendentes meliorem esse politiam , \textbf{ vel melius esse regnum et ciuitatem , } si ibi sit populus abundans & que muestran que meior es la poliçia \textbf{ o meior es el regno o la çibdat } si y fuere pueblo \\\hline
3.2.33 & ad nimium diuites nesciunt se rationabiliter gereres insidiantur enim eis quomodo possint astute \textbf{ et latenter eorum depraedari bona . } Sed si in populo sint multae personae mediae , & Ca sienpre les asecha commo puedan faldridamente \textbf{ e encobiertamente tomar e robar de sus biens . } Mas si en el pueblo fueren muchas perssonas medianeras quedaran todo estos enpeesçimientos \\\hline
3.2.33 & sed diuites penitus volent principari , \textbf{ et suppeditare alios . } Alii vero contra nitentes dissensionem faciunt , & Ca los ricos en toda manera quarran en ssennorear \textbf{ e poner so pie alos otros } e los pobres contradiziendo alos ricos faran discordia en la çibdat \\\hline
3.2.33 & videns se ei quasi aequalem existere , \textbf{ et non esse magnum excessum inter ipsos . } Huic auctoritati attestatur , & ueyendo que es su egual e veyendo \textbf{ qua non ay grant auna taia entre el vno e el otro . } Et a esta uerdat da testimo \\\hline
3.2.33 & ex personis mediis . \textbf{ Decet ergo Reges et Principes adhibere cautelas , } ut in regno suo abundent multae personae mediae ; & establesçidas de perssonas medianeras . \textbf{ Et pues que assi es conuiene | que los reyes e los prinçipes ayan cautelas e sabidurias . } por que en el su regno sean muchͣs perssonas medianeras \\\hline
3.2.34 & obedire Regibus et Principibus , \textbf{ et obseruare leges . } Primo enim ex hoc consequitur populus virtutes , & quanto es prouechoso e conuenible al pueblo de obedesçer alos Reyes \textbf{ e guardar las leyes . } Ca lo primero desto alçança el pueblo uirtudes e grandes bienes \\\hline
3.2.34 & Nam ( ut dicebatur in praecedentibus ) \textbf{ intentio legislatoris est inducere ciues ad virtutem . } In recta enim Politia & dich̃en los capitulos \textbf{ sobredichos la entençion del ponedor dela ley es enduzer los çibdadanos o uirtud . } Ca en la derecha poliçia \\\hline
3.2.34 & quia intentio eius est \textbf{ inducere alios ad virtutem , } cum virtus faciat habentem bonum ; & Por la qual cosa si el prinçipe gouernar e derechamente el pueblo qual es acomne dado \textbf{ por que la su entençion es enduzir los otros a uirtud . } Et la uirtud faze al que la ha buenon \\\hline
3.2.34 & quanto decentius est \textbf{ eos esse bonos , et virtuosos . } Secunda via ad inuestigandum hoc idem , & quanto mas conuenible es \textbf{ que ellos sean bueons e uirtuosos . } la segunda razon para prouar esto mesmo se tomadesto \\\hline
3.2.34 & Credunt enim aliqui , \textbf{ quod obseruare leges , } et obedire regi , & Ca assi commo dize el philosofo \textbf{ en el primero libro de la rectorica enlas leyes es salud dela çibdat . | Et maguer algunos cuyden que guardan las leyes } e obedesçer el Rey lea algunasiudunbre . \\\hline
3.2.34 & Ignorant enim quid est libertas , \textbf{ dicentes obseruare leges } et obedire Regibus , & Ca non saben que cosa es libertad \textbf{ aquellos que dizen que guardar las leyes } e obedesçer alos Reyes es seruidunbre . \\\hline
3.2.34 & et obedire Regibus , \textbf{ esse seruitutem . } Cum enim bestiae sint naturae seruilis : & aquellos que dizen que guardar las leyes \textbf{ e obedesçer alos Reyes es seruidunbre . } Ca commo las bestias sean de natura seruil \\\hline
3.2.34 & sic Rex si recte principetur est salus et vita regni . \textbf{ Quare sicut pessimum est corpori delinquere animam , } et non regi per eam , & enssennoreare es salud et uida del regno . \textbf{ Por la qual cosa | assi commo es muy mala cosa al cuerpo desmanparar el alma } e non se gouernar por ella . \\\hline
3.2.34 & Quare sicut pessimum est corpori delinquere animam , \textbf{ et non regi per eam , } sic pessimum est regno & assi commo es muy mala cosa al cuerpo desmanparar el alma \textbf{ e non se gouernar por ella . } assi es muy mala cosa \\\hline
3.2.34 & sic pessimum est regno \textbf{ deserere leges regias } et praecepta legalia , & assi es muy mala cosa \textbf{ que el regno desanpare las leyes } e los mandamientos reales \\\hline
3.2.34 & et praecepta legalia , \textbf{ et non regi per Regem . } Tertia via ad ostendendum hoc idem , & e los mandamientos reales \textbf{ e que non se gouierne por el Rey ¶ } La terçera razon para mostrar esto mismo se \\\hline
3.2.34 & Expediens enim fuit regno et ciuitati \textbf{ habere aliquem Regem } vel aliquem principantem , & Et por ende cosa conuenible fue al regno \textbf{ e ala çibdat de auer algun Rey o algun prinçipe } por que los malos non pudiessen turbar la paz de los çibdadanos . \\\hline
3.2.34 & Nam sicut medicus intendit \textbf{ sedare humores , } ne insurgat morbus & Ca assi commo el fisico entiende de amanssar \textbf{ e de egualar los humores } por que se non le una te enfermedat̃ \\\hline
3.2.34 & et bellum in corpore : \textbf{ sic legislator intendit placare corda , } sedare animas , & nin batalla dellos en el cuerpo . \textbf{ assi el fazedor delas leyes entiende de amanssar los coraçones } e abenir las almas \\\hline
3.2.34 & sic legislator intendit placare corda , \textbf{ sedare animas , } ne insurgat rixa & assi el fazedor delas leyes entiende de amanssar los coraçones \textbf{ e abenir las almas } porque se non le uate pelea nin contienda en el regno o en la çibdat . \\\hline
3.2.35 & Appetit autem iratus apparenter , \textbf{ idest manifeste punire eos } qui paruipendunt ipsum , & por que . nunca es sanna sin alguna tristeza . \textbf{ ca dessea el sannudo dar pena manifiestamente a aquellos que menospreçian a el o algunas cosas } que son de lo algunas cosas \\\hline
3.2.35 & ut non incurrant regiam iram , \textbf{ non forefacere in ipsum Regem . } Regi autem duo debentur , & por que non cayan en sanna del reyes \textbf{ non fazer ninguna cosa | mala contra el Rey } Ca al Rey deuemos dos cosas . \\\hline
3.2.35 & ad ipsum spectat per se \textbf{ et per alios dirigere eos , } qui sunt in regno . & Et por ende a el parte nesçe \textbf{ prinçipalmente gouernar e gniar todos } los que son en el regno \\\hline
3.2.35 & Ratione vero , \textbf{ quia ipsius est dirigere alios , } debetur ei subiectio et obedientia . & que los otros ael deue ser dada honrra e reuerençia . \textbf{ Mas por razon que a el parte nesçe degniar los otros } ael deue ser fecha subiectiuo e obediençia \\\hline
3.2.35 & debetur ei subiectio et obedientia . \textbf{ Quare dupliciter potest forefieri ad Regem } ab iis qui sunt in regno . & ael deue ser fecha subiectiuo e obediençia \textbf{ Por la qual cosa en dos maneras pueden los que son en el regno errar contra el Rey¶ } La primera si non le fizieren honrra \\\hline
3.2.35 & Viso quomodo habitatores regni non debent \textbf{ prouocare Regem ad iram , } forefaciendo in ipsum , & Visto en qual manera los moradores del regno \textbf{ non deuen mouer el Rey a saña errando contra el } e non le faziendo \\\hline
3.2.35 & forefaciendo in ipsum , \textbf{ non exhibere ei debitum honorem } et obedientiam condignam . & non deuen mouer el Rey a saña errando contra el \textbf{ e non le faziendo } obediençia qual deuen e honrra conuenble . \\\hline
3.2.35 & et qui pertinent ad ipsum . \textbf{ Ad Regem autem pertinere videntur quatuor genera personarum } videlicet parentes et uniuersaliter omnes cognati , & Et deuedes saber \textbf{ que al Rey parte nesçen quatro maneras de perssonas . | Conuiene a saber . } El padir . Et la madre . \\\hline
3.2.35 & ad iram prouocare , \textbf{ non solum non forefacere in ipsum Regem , } sed etiam non forefacere in cognatos , & quasieren mouer al Rey a saña \textbf{ non solamente de non fazer ningun tuerto contra el rey en su perssona . } Mas avn de non fazer contra sus parientes \\\hline
3.2.35 & non solum non forefacere in ipsum Regem , \textbf{ sed etiam non forefacere in cognatos , } uxorem , filios , & non solamente de non fazer ningun tuerto contra el rey en su perssona . \textbf{ Mas avn de non fazer contra sus parientes } nin contra su muger \\\hline
3.2.35 & ab ipsa infantia \textbf{ prouocare filios } ad dilectionem Regis : & e generalmente a todos los moradores del regno \textbf{ que enssennen a sus fijos en sudlxxi ninnes } que amen al Rey \\\hline
3.2.35 & ad dilectionem Regis : \textbf{ instruere eos } quomodo debeant honorare Regem , & que amen al Rey \textbf{ e queles enssennen en qual manera de una honnar al Rey } e obedescerle \\\hline
3.2.35 & instruere eos \textbf{ quomodo debeant honorare Regem , } obedire ei : & que amen al Rey \textbf{ e queles enssennen en qual manera de una honnar al Rey } e obedescerle \\\hline
3.2.35 & obedire ei : \textbf{ non forefacere in cognatos eius , } nec in filios , & e obedescerle \textbf{ e non fazer tuerto contra los sus parientes } nin contra sus fijos \\\hline
3.2.35 & ( ut dicitur 7 Pol’ ) \textbf{ non instruere pueros ad virtutem , } et obseruantiam legum utilium : & en el quinto libro delas positicas \textbf{ non enssennar los mocos auertudes } e aguarda delas leyes prouechosas \\\hline
3.2.36 & Nam maxime prouocatur populus ad odium Regis , \textbf{ si viderit ipsum non obseruare iustitiam : } Ideo dicitur 2 Rhet’ & Ca el pueblo mayormente se le una taria a mal querençia del Rey \textbf{ si viesse | que el non guardaua nistiçia . } Et por ende el philosofo \\\hline
3.2.36 & quod homines timent eos , \textbf{ de quibus sunt conscii fecisse aliquid dirum . } Secundo timentur Reges & que los omes temen a aquellos de que son sabidores \textbf{ que fizien es alguna cosa muy cruel ¶ } Lo segundo son temidos los Reyes \\\hline
3.2.36 & Timet igitur tunc quilibet ex populo forefacere , \textbf{ cogitans se non posse punitionem effugere . } Imo , ut vult Philos’ 7 Polit’ & Et pues que assi es cada vno del pueblo teme de mal fazer cuydando \textbf{ que non podra escapar dela pena . } Ante assi conmo dize el philosofo \\\hline
3.2.36 & magis punire , \textbf{ et seuerius se gerere contra amicos , } si contingat eos valde forefacere , & que mayor penaden \textbf{ e mas cruelmente se ayan contra los amigos } quando mal fizieren \\\hline
3.2.36 & et cuiuscumque principantis esse debet , \textbf{ inducere alios ad virtutem . } Omne ergo bonum & e de cada vn prinçipe deue ser \textbf{ e non duzir alos otros a uirtud . } Et pues que assi es todo bien fazer \\\hline
3.2.36 & per quod ciues sunt magis boni et virtuosi , \textbf{ debet esse magis intentum a legislatore . } Cum ergo ciues et existentes in regno & e mas uirtuo los deue ser \textbf{ prinçipalmente quarido e entendido del ponedor dela ley . } Et pues que assi es quando los çibdadanos \\\hline
3.2.36 & et ex dilectione legislatoris , \textbf{ cuius est intendere commune bonum , } quiescant male agere : & e por amor del prinçipe ponedor dela ley \textbf{ cuya entençiones de tener mientes al bien comun } que por ende queden los omes de mal fazer . \\\hline
3.2.36 & quiescant male agere : \textbf{ oportuit ergo aliquos inducere ad bonum , } et retrahere a malo timore poenae . & Por la qual cosa conuiene \textbf{ que alguon s | enduxiessemosa bien } e arredrassemos del mal \\\hline
3.3.1 & et ad quid sit instituta . \textbf{ Sciendum igitur militiam esse quandam prudentiam , } siue quandam speciem prudentiae . & Et pues que assi es deuedes saber \textbf{ que la caualleria es vna prudençia o vna manera de sabiduria . . | Mas podemos quanto pertenesçe a lo presente } departir çinco maneras de prudençia e de sabiduria . \\\hline
3.3.1 & videlicet prudentiam singularem , oeconomicam , regnatiuam , politicam siue ciuilem , et militarem . \textbf{ Dicitur enim aliquis habere singularem } vel particularem prudentiam , & Et prudencia politica o çiuil \textbf{ para gouernar la çibdat . } Et prudençia caualleril \\\hline
3.3.1 & scire regere seipsum , \textbf{ quam scire regere familiam , } et ciuitatem , aut regnum . & Ca menos es saber gouernar a ssi mismo \textbf{ que saber gouernar la conpaña de casa o la cibdat o el regno . } La segunda manera de la prudençia es dicha yconomica \\\hline
3.3.1 & quia scit bene consiliari , \textbf{ et bene dirigere ad bonum finem : } ubi ergo reperitur alia & por que sabe bien consseiar \textbf{ e bien guiar a buena fin . } Et pues que assi es do son falladas \\\hline
3.3.1 & oeconomicam prudentiam , \textbf{ per quam quis scit regere domum et familiam , } oportet esse aliam a prudentia , & Et por ende la sabiduria \textbf{ por la qual cada vno sabe gouernar la casa e la conpaña . } Conuiene que sea otra e departida de la sabiduria \\\hline
3.3.1 & per quam quis scit regere domum et familiam , \textbf{ oportet esse aliam a prudentia , } qua quis nouit seipsum regere . & por la qual cada vno sabe gouernar la casa e la conpaña . \textbf{ Conuiene que sea otra e departida de la sabiduria } por la qual cada vno sabe gouernar a ssi mismo . \\\hline
3.3.1 & cuius est leges ferre , \textbf{ et regere regnum et ciuitatem , } est alia a prudentia oeconomica & a quien pertenesçe de fazer leyes \textbf{ e de gouernar el regno e la çibdat . } esta es departida de la sabiduria yconomica \\\hline
3.3.1 & quae requiritur in patrefamilias , \textbf{ cuius est gubernare domum : } immo quanto bonum ciuitatis & que es menester en el padre familias \textbf{ a quien pertenesçe de gouernar la casa . } Mas en quanto el bien de la çibdat e del regno \\\hline
3.3.1 & in Rege oportet \textbf{ excedere prudentiam patrisfamilias , } vel prudentiam alicuius particularis hominis . & que pertenesçe al Rey deue \textbf{ sobrepuiar la sabiduria | del gouernamiento de vna casa } o la sabiduria de algun omne particular . \\\hline
3.3.1 & ut est paterfamilias , \textbf{ et ut habet dispensare bona domestica . } In tertio vero eruditur Rex aut Princeps & en quanto es padre familias \textbf{ que ha de gouernar la casa | e ha de despenssar los bienes de la casa . } Et enel teçero libro ensseñamos al Rey o al prinçipe \\\hline
3.3.1 & ut est caput regni aut principatus , \textbf{ et ut habet ferre leges } et gubernare ciues . & en quanto es cabeça del regno o del prinçipado \textbf{ e en quanto ha de poner leyes e gouernar los çibdadanos . } Et todas estas tres sabidurias \\\hline
3.3.1 & et gubernare ciues . \textbf{ Omnes autem tres prudentias decet habere Regem , } videlicet particularem , oeconomicam et regnatiuam . & e en quanto ha de poner leyes e gouernar los çibdadanos . \textbf{ Et todas estas tres sabidurias | conuiene que aya el Rey . } Conuiene a saber . \\\hline
3.3.1 & sic in quolibet ciue requiritur prudentia aliqualis \textbf{ qua noscat adimplere leges } et mandata principantis . & assi en cada cibdadno es meester alguna sabiduria \textbf{ por la qual sepa conplir las leyes } e los mandamientos del principe \\\hline
3.3.1 & adhuc oporteret \textbf{ ipsum habere aliqualem prudentiam } qua sciret se regere et gubernare : & e morasse solo avn conuenir le \textbf{ ya de auer alguna sabiduria } por la qual se sopiesse gouernar . \\\hline
3.3.1 & Quare cum commune bonum directe videatur \textbf{ impediri per impugnationem hostium , } ex consequenti vero & derechamente parezca de ser enbargado \textbf{ por la guerra de los enemigos } e de si por la turbacion \\\hline
3.3.1 & bene se habere in opere bellico , \textbf{ et per actiones bellicas opprimere impedimenta hostium : } ex consequenti vero spectat ad ipsos & assi a los caualleros pertenesçe principalmente de se auer bien en obras de batallas . \textbf{ Et por las obras de batallas desenbargar todos aquellos enbargos } que pueden venir de los enemigos . \\\hline
3.3.1 & et secundum mandata principantis \textbf{ impedire omnes seditiones ciuium } et omnes oppressiones eorum & Et de si a ellos pertenesce desenbargar \textbf{ e tirar todas las discordias de los çibdadanos } e todos agrauiamientos \\\hline
3.3.1 & qui sunt in regno , \textbf{ per quas turbari potest tranquillitas ciuium et commune bonum . } Hanc autem prudentiam videlicet militarem , & aquellos que son el regno segunt \textbf{ por las quales cosas se puede turbar la paz . | el assessiego de los çibdadanos e el bien comun . } Et esto deue fazer los caualleros \\\hline
3.3.1 & Hanc autem prudentiam videlicet militarem , \textbf{ maxime decet habere Regem . } Nam licet executio bellorum , et remouere impedimenta ipsius communis boni , & Et esta sabiduria de caualleria \textbf{ mas pertenesçe al rey que a otro ninguno . | Ca commo quier que pertenezca a los caualleros } la essecuçion de las batallas \\\hline
3.3.1 & maxime decet habere Regem . \textbf{ Nam licet executio bellorum , et remouere impedimenta ipsius communis boni , } spectet ad ipsos milites , & Ca commo quier que pertenezca a los caualleros \textbf{ la essecuçion de las batallas | e tirar e arredrar los enbargos del bien comun . } Et tales cosas commo estas pertenezcan a aquellos \\\hline
3.3.1 & spectet ad ipsos milites , \textbf{ et etiam ad eos quibus ipse Rex aut Princeps voluerit committere talia : } scire tamen quomodo committenda sint bella , & Et tales cosas commo estas pertenezcan a aquellos \textbf{ a quien lo quisiere acomendar el rey | o el prinçipe } Enpero saber en qual manera son de acometer las batallas \\\hline
3.3.1 & et etiam ad eos quibus ipse Rex aut Princeps voluerit committere talia : \textbf{ scire tamen quomodo committenda sint bella , } et qualiter caute remoueri possint & o el prinçipe \textbf{ Enpero saber en qual manera son de acometer las batallas } e en qual manera se pueden sabiamente tirar e arredrar los enbargos \\\hline
3.3.1 & ad dignitatem militarem , \textbf{ nisi constet ipsum diligere bonum regni et commune , } et nisi spes habeatur & para dignidat de caualleria \textbf{ si non fueren çiertos | que el ama el bien del regno e el bien comun } e si non ouieren esperança \\\hline
3.3.1 & secundum iussionem principantis \textbf{ impedire seditiones ciuium , } pugnare pro iustitia et pro iuribus , & que el sea bueno en la obra de la batalla \textbf{ e que quiera segunt el mandamiento del prinçipe desenbargar las discordias de los cibdidanos } e lidiar por la iustiçia \\\hline
3.3.1 & remouere quaecunque \textbf{ impedire possunt commune bonum . } Ex hoc etiam patere potest & quales se quier cosas \textbf{ que enbarguen el bien comun . } Et desto puede parescer \\\hline
3.3.1 & omnem bellicam operationem contineri sub militari . \textbf{ Nam licet bellare contingat homines pedites , } vel etiam equestres non existentes milites : & so la arte de la caualleria \textbf{ ca commo quier que contezca de lidiar los peones } e los omnes de cauallo \\\hline
3.3.2 & in quibus regionibus meliores sunt bellatores , \textbf{ oportet attendere circa praedicta duo . } In partibus igitur nimis propinquis soli , & o en quales tierras son meiores lidiadores . \textbf{ Conuiene de tener mientes en estas dos cosas sobredichas . } Et pues que asy es en las partes \\\hline
3.3.2 & Sciendum ergo quod cum bellantes debeant \textbf{ habere membra apta } et assueta ad percutiendum , & Et pues que assi es conuiene de saber \textbf{ que commo los lidiadores deuan auer los mienbros apareiados } e acostunbrados a ferir \\\hline
3.3.2 & et assueta ad percutiendum , \textbf{ non debeant horrere sanguinis effusionem , } debeant esse animosi ad inuadendum , & e acostunbrados a ferir \textbf{ et non deuan | aborresçer el derramamiento de la sangre } e deuan ser animosos \\\hline
3.3.2 & quia non sine magna audacia contingit \textbf{ aliquos inuadere apros . } Sunt ergo tales animosi et strenui ad bellandum . & sin grant osadia acometer los puercos monteses \textbf{ e les otros fuertes venados . } Et por ende tales son de grant coraçon \\\hline
3.3.2 & Nam non timentes aprorum pericula , \textbf{ signum est eos non timere hostium bella . } Rursus venatores ceruorum non sunt repudiandi & e de las otras bestias fuertes . \textbf{ señal es que non | temerien lasbatallas de los enemigos . } Otrossi los caçadores de los çieruos non son de refusar \\\hline
3.3.2 & Nam nunquam bene vibrant clauam , \textbf{ aut ensem qui debet habere manum leuem , } et non est assuetus retinere & por que nunca bien leuantar a la maça \textbf{ nin esgrimira la espada | aquel que deue auer la mano liuiana . } Et non es acostubrado de tener en la mano \\\hline
3.3.3 & si vult legislator ciues bellatores facere , \textbf{ et reddere ipsos aptos ad pugnandum , } potius debet & Et si quisiere el ponedor de la ley fazer los çibdadanos buenos lidiadores \textbf{ e fazer los apareiados para la batalla } deue ante tomar el tienpo de la mançebia \\\hline
3.3.3 & esse videtur armorum industria . \textbf{ Nam siue equitem siue peditem oportet esse bellantem , } quasi fortuito videtur & nin ligera arte auer sabiduria de las armas . \textbf{ Ca si quier sea cauallero si quier peon el que ha de lidiar paresçe } que por uentura alcaça uictoria \\\hline
3.3.3 & quasi fortuito videtur \textbf{ peruenire ad palmam , } si caret industria bellandi . & Ca si quier sea cauallero si quier peon el que ha de lidiar paresçe \textbf{ que por uentura alcaça uictoria } si non ouiere sabiduria de lidiar \\\hline
3.3.3 & Quare si legislator \textbf{ ut Rex aut Princeps debeat committere bellum , } viros exercitatos et bellatores strenuos debet assumere . & Por la qual cosa si el fazedor de la ley \textbf{ assi commo el Rey o el prinçipe ouiere de acometer batalla deue tomar e escoger varones vsados } e lidiadores escogidos e fuertes . \\\hline
3.3.3 & videre restat , \textbf{ ex quibus signis cognosci habeant homines bellicosi . } Sciendum igitur viros audaces et cordatos & aquellos que se deuen fazir caualleros e ser lidiadores . \textbf{ finca de ver por quales señales se han de conosçer los buenos lidiadores . } Et para esto conuiene de saber \\\hline
3.3.3 & Sciendum igitur viros audaces et cordatos \textbf{ utiliores esse ad bellum , } quam timidos . & que los omnes osados eatreuidos \textbf{ e de grandes coraçones son mas prouechosos para la batalla } que los temerosos e de flacos coraçones . \\\hline
3.3.3 & Tribus igitur generibus signorum \textbf{ cognoscere possumus bellicosos viros . } Primo quidem per signa , & por tres maneras de señales \textbf{ podemos conosçer los omnes lidiadores } Lo primero por aquellas señales \\\hline
3.3.3 & Sed e contrario dicimus , \textbf{ duri carne habentes compactos neruos , } et lacertos , & son sotiles de coraçon . \textbf{ Et por el contrario los que han las carnes duras } e los neruios espessos e firmes \\\hline
3.3.3 & et latitudo pectoris . \textbf{ Videmus enim leones animalium fortissimos habere magna brachia , } et latum pectus . & son grandeza de los mienbros e anchura de los pechos . \textbf{ Ca veemos que los leones | que son mas fuertes que todas las otras animalas } por que han grandes braços e anchos pechos . \\\hline
3.3.3 & debemus arguere \textbf{ ipsum esse bellicosum , } et aptum ad pugnam . & estos tales deuemos iudgar \textbf{ por lidiadores e apareiados para la batalla . } Et pues que assi es tales son de escoger los lidiadores \\\hline
3.3.3 & quia ut plurimum contingit \textbf{ eos esse aptos ad actiones bellicas . } Quantum ad praesens spectat , & que por la mayor parte sean apareiados \textbf{ para las obras de la batalla . |  } q quanto pertenesçe a lo presente ocho cosas podemos contar \\\hline
3.3.4 & enumerare possumus octo , \textbf{ quae habere debent homines bellatores : } secundum quae ( quantum ad praesens spectat ) & q quanto pertenesçe a lo presente ocho cosas podemos contar \textbf{ que deuen auer los omnes lidiadores } segunt las quales cosas \\\hline
3.3.4 & Primo enim oportet pugnatiuos homines posse \textbf{ sustinere magnitudinem ponderis . } Secundo posse sufferre & Lo primero conuiene \textbf{ que los omnes lidiadores puedan sofrir grandes pesos . } lo segundo que puedan sofrir grandes trabaios \\\hline
3.3.4 & et labores magnos . \textbf{ Tertio posse tolerare parcitatem victus . } Quarto non curare de incommoditate iacendi et standi . & e continuados mouimientos de los mienbros \textbf{ Lo terçero que puedan sofrir | escasseza de vianda e fanbre e sed . } Lo quarto que non aya cuydado de mal yazer \\\hline
3.3.4 & Quarto non curare de incommoditate iacendi et standi . \textbf{ Quinto quasi non appretiare corporalem vitam . } Sexto non horrere sanguinis effusionem . & nin de mal estar \textbf{ Lo quinto | que por razon de la iustiçia e del bien comun despreçien la uida corporal . } Lo sexto que non teman \\\hline
3.3.4 & Quinto quasi non appretiare corporalem vitam . \textbf{ Sexto non horrere sanguinis effusionem . } Septimo habere aptitudinem , & que por razon de la iustiçia e del bien comun despreçien la uida corporal . \textbf{ Lo sexto que non teman | nin aborrezcan de derramar su } sangreLo septimo conuiene \\\hline
3.3.4 & Sexto non horrere sanguinis effusionem . \textbf{ Septimo habere aptitudinem , } et industriam ad protegendum se et feriendum alios . & nin aborrezcan de derramar su \textbf{ sangreLo septimo conuiene | que ayan buena disposiçion e buena sabidura } para defender assi \\\hline
3.3.4 & et industriam ad protegendum se et feriendum alios . \textbf{ Octauo verecundari et erubescere eligere turpem fugam . } Est enim primo necessarium bellantibus posse & e para ferir a los enemigos . \textbf{ lo octauo que ayan uerguença de foyr torpemente . } Pues que assi es \\\hline
3.3.4 & Est enim primo necessarium bellantibus posse \textbf{ sustinere ponderis magnitudinem . } Nam inermes a quacunque parte foriantur , & Lo primero que es meester a los lidiadores \textbf{ es que puedan sofrir grandes pesos . } Ca los desarmados \\\hline
3.3.4 & ad defensionem exercitus , \textbf{ deferre in abundantia victualium copiam : } immo et si adesset pugnantibus ciborum ubertas , & para defendimieto de la hueste \textbf{ que lo pongan | e que lo lieuen en muchedunbre de uiandas } Ca ante deuen dexar el peso de las talegas \\\hline
3.3.4 & et commune bonum \textbf{ quasi non appretiari corporalem vitam . } Nam cum tota operatio bellica exposita sit periculis mortis , & Lo quinto conuiene a los lidadores \textbf{ de non preçiar la vianda corporal | por la iustiçia e por el bien comun . } Ca commo toda la hueste sea puesta \\\hline
3.3.4 & Sexto pugnantes non debent \textbf{ horrere sanguinis effusionem . } Nam si quis cor molle habens , & Lo sexto los lidiadores non deuen \textbf{ aborresçer el derramamiento de la sangre } por que si alguno ouiere el coraçon muele \\\hline
3.3.4 & muliebris existens , \textbf{ horreat effundere sanguinem ; } non audebit hostibus plagas infligere , & e fuere assi commo mugeril \textbf{ aborresçra esparzer la sangre } e non osara fazer llagas a los enemigos . \\\hline
3.3.4 & et per consequens bene bellare non potest . \textbf{ Septimo decet eos habere aptitudinem , } et industriam ad protegendum se , & que non podra bien lidiar . \textbf{ Lo vij° . conuiene a los lidiadores de auer disposiçion e sabiduria para cobrirse e defender se } e para ferir a los otros . \\\hline
3.3.4 & Octauo decet bellatores verecundari , \textbf{ et erubescere turpem fugam . } Nam , ut dicitur 3 Ethic’ & es de auer uerguença \textbf{ e de guardar se | de non foyr torpemente de la batalla . } Ca assi commo dize el philosofo \\\hline
3.3.4 & est diligere honorari expugna , \textbf{ et erubescere turpem fugam . } Aduertendum autem quod cum dicimus , & de ser honrado por batalla \textbf{ e de auer uerguenna de foyr | torpemente de la batalla . } Mas deuemos parar mientes \\\hline
3.3.4 & Aduertendum autem quod cum dicimus , \textbf{ bellatores non habere effusionem sanguinis , } non debere multum appretiari corporalem vitam , & que quando dezimos \textbf{ que los lidiadores non deuen aborresçer el esparzimiento de la sangre } nin deuen mucho preçiar la vida corporal \\\hline
3.3.4 & bellatores non habere effusionem sanguinis , \textbf{ non debere multum appretiari corporalem vitam , } et caetera quae diffusius connumerauimus : & que los lidiadores non deuen aborresçer el esparzimiento de la sangre \textbf{ nin deuen mucho preçiar la vida corporal } et las otras cosas tales \\\hline
3.3.5 & de iis quae requiruntur ad pugnam . \textbf{ Numeratis iis quae habere debent bellatores viri : } restat inquirere , & que partenesçen a la batalla . \textbf{ ontadasontadas cosas que los lidiadores deuen auer finca de demandar } quales son los meiores lidiadores . \\\hline
3.3.5 & Numquam credo potuisse dubitari \textbf{ aptiorem armis esse rusticam plebem . } Ad hoc etiam videntur facere & Creo que ninguno nunca pudo dubdar \textbf{ que los omnes rusticos e aldeanos non fuessen meiores para las armas | que los que son delicadamente criados . } Et a esto fazen avn aquellas cosas \\\hline
3.3.5 & viros pugnatiuos tales esse debere , \textbf{ qui possent sustinere magnitudinem ponderis , } continuum laborem membrorum , parcitatem victus , & que los omnes lidiadores sean tales \textbf{ que puedan sofrir grandes pesos } e grandes trabaios continuados en los sus cuerpos \\\hline
3.3.5 & et incommoditatem iacendi et standi , \textbf{ non timere mortem , } non horrere sanguinis effusionem , & e sofrir mal yazer e mal estar \textbf{ e non temer la muerte | e non temer } nin aborresçer el esparzimiento de la sangre \\\hline
3.3.5 & non timere mortem , \textbf{ non horrere sanguinis effusionem , } et cetera alia & e non temer \textbf{ nin aborresçer el esparzimiento de la sangre } e las otras cosas \\\hline
3.3.5 & Hos etiam probabile est \textbf{ non multum timere mortem . } Nam tanto quis magis mortem timere videtur , & Et avn asas es cosa prouada \textbf{ que estos poco temen la muerte } ca tanto mas teme cada vno la muerte \\\hline
3.3.5 & Hi etiam non videntur \textbf{ horrere effusionem sanguinis . } Nam inter ceteras gentes & Et avn paresçe \textbf{ que estos non aboresçen derramamiento de sangre . } Ca entre todas las gentes \\\hline
3.3.5 & Ad haec igitur intendentibus videtur \textbf{ censendum esse meliores bellatores esse rurales . } Sunt autem alia , & e los villanos son meiores \textbf{ para las batallas | que los nobles nin los fijosdalgo . } Mas ay otras cosas \\\hline
3.3.5 & velle honorari ex pugna , \textbf{ et erubescere turpem fugam . } Hoc est enim & e tomar uerguença de foyr \textbf{ torpemente | assi commo dicho es dessuso . } Et esto es segunt \\\hline
3.3.5 & quam rusticis , \textbf{ ii meliores esse videntur ad pugnam , } eo quod verecundentur fugere . & Porende meiores son los nobles \textbf{ e los | fijosdalgo para las batallas } que los villanos e los aldeanos \\\hline
3.3.5 & sequitur hos meliores esse pugnantes . \textbf{ Videntur enim haec duo maxima esse ad obtinendam victoriam , } videlicet erubescentia fugiendi , & siguese que son meiores lidiadores . \textbf{ Ca paresçe que estas dos cosas son prinçipales | para auer victoria . } Conuiene a saber uerguença de foyr . \\\hline
3.3.5 & quem patiuntur nobiles \textbf{ in non posse tantos sustinere labores , } quantos consueuerunt sustinere rurales . & que han los nobles \textbf{ en non poder sofrir tantos trabaios } quantos se acostunbraron de sofrir los villanos . \\\hline
3.3.5 & in non posse tantos sustinere labores , \textbf{ quantos consueuerunt sustinere rurales . } In huiusmodi enim pugna & en non poder sofrir tantos trabaios \textbf{ quantos se acostunbraron de sofrir los villanos . } por que en tal batalla mucho vale la sabidura de lidiar \\\hline
3.3.6 & Viso armorum exercitium \textbf{ esse perutile ad opera bellica , } restat ostendere & Visto en qual manera el uso de las armas es muy prouechoso \textbf{ para las obras de la batalla } finca de demostrar \\\hline
3.3.6 & Et cum viderit magister bellorum \textbf{ aliquem non tenere ordinem debitum in acie , } ipsum increpet et corrigat : & Et quando vieren los caudiellos maestros de las batallas \textbf{ que alguno non guarda orden en la az } deuenle denostar e castigar \\\hline
3.3.6 & ut sint habiles in praecurrendo . \textbf{ Videtur enim hoc valere ad tria . } Primo ad explorandum inimicorum facta . & quando venieren a la fazienda . \textbf{ Ca paresçe que esto les vale atres cosas } Lo primero para assechar e ascuchar el estado de los enemigos . \\\hline
3.3.6 & Tertio ad infligendum maiores plagas . \textbf{ Contingit enim aliquando inuenire fossas } et alia impedimenta in via , & Lo tercero para fazer mayores llagas . \textbf{ Ca contesçe algunas vezes de fallar algunas carcauas e arroyos e açequias } e algunos otros enbargos en la carrera \\\hline
3.3.7 & enumerare octo alia , \textbf{ ad quae exercitari debent homines bellicosi . } Primo enim exercitandi sunt & aque dixiemos que son de vsar los lidiadores contar otras ocho cosas \textbf{ aque se deuen vsar los lidiadores . } Lo primero se deuen vsar aleuar grandes pesos . \\\hline
3.3.7 & non est inutile \textbf{ assuescere bellatores . } Secundo exercitandi sunt bellantes & Et por ende prouechosa cosa es de se acostunbrar los lidiadores a leuar grandes pesos . \textbf{ Lo segundo se deuen usar los lidiadores } a acometer \\\hline
3.3.7 & et iuuenes quos volebant \textbf{ facere optimos bellatores exercitabant ad palos illos ita , } ut quilibet haberet scutum dupli ponderis & Et los moços que querian \textbf{ acostunbrara fazer los buenos lidiadores . | vsauan los a ferir en aquellos palos } assi que cada vno de aquellos moços \\\hline
3.3.7 & postquam per magnam partem \textbf{ dici exercitati essent ad arma , } si tempus erat natationi congruum , & que auian de ser lidiadores \textbf{ despues que por vna grant parte del dia eran usados en las armas } si tienpo era conuenible para nadar \\\hline
3.3.7 & et solertem mentem latere non potest . \textbf{ Nam ascendere equos , } est proprium equitibus : & Ca non se puede asconder a ome sabio . \textbf{ Ca sobir en los cauallos pertenesçe a los caualleros . } Et lançar piedas con fondas pertenesçe a los peones . \\\hline
3.3.7 & est proprium equitibus : \textbf{ proiicere lapides cum funda , } videtur esse proprium peditibus . & Ca sobir en los cauallos pertenesçe a los caualleros . \textbf{ Et lançar piedas con fondas pertenesçe a los peones . } Mas las otras cosas en alguna manera puenden pertenesçer a todos . \\\hline
3.3.8 & quam circa ipsa in aliquo neglexisse . \textbf{ Nam recitat Vegetius dixisse Catonem , } quod in aliis rebus & Ca en ninguna guisa non deuen ser negligentes en ellas . \textbf{ Onde dize Uegeçio | que gaton el sabio dixo } que en las otras cosas \\\hline
3.3.8 & quod , exercitu absque fossis et castris existente , \textbf{ et non credentes hostes esse propinquos , } superuenientibus hostibus fugit exercitus debellatus . & e sin castiellos o otros defendimientos non cuydando \textbf{ que sus enemigos estan çerca vienen a desora los } e es vençida la hueste . \\\hline
3.3.8 & quandocunque et undecunque superuenientes hostes obsideant . \textbf{ Debet enim exercitus secum ferre munitiones congruas , } ut cum castrametari voluerit , & e donde quier que vengan los enemigos acercar los o acometer los . \textbf{ Ca deue sienpre la hueste leuar consigo guarniciones conuenibles . } por que quando quisiere la hueste folgar en algun logar parezca \\\hline
3.3.8 & Viso utile esse \textbf{ circa exercitum facere fossas } et construere castra : & que lieuan consigo assi commo vna çibdat guarnida . \textbf{ Visto commo es cosa prouechable a la hueste fazer carcauas e costruir guarniçiones e castiellos . } finca de demostrar en qual manera las tales guarniciones \\\hline
3.3.8 & circa exercitum facere fossas \textbf{ et construere castra : } restat ostendere , & que lieuan consigo assi commo vna çibdat guarnida . \textbf{ Visto commo es cosa prouechable a la hueste fazer carcauas e costruir guarniçiones e castiellos . } finca de demostrar en qual manera las tales guarniciones \\\hline
3.3.8 & facile est fossas circa exercitum fodere , \textbf{ munitiones erigere et castra construere . } Sed si aduersarii praesentes adsint , & Ca si los enemigos non estudieren cerca de ligero pueden fazer carcauas çerca de la hueste \textbf{ e leuantar | guarnicoñes e fazer castiellos . } Mas si los enemigos fueren cerca \\\hline
3.3.8 & quod ipsum oporteat facere . \textbf{ Ostenso utile esse castra construere , } et qualiter etiam praesentibus hostibus construenda sint castra : & e manden a cada vno qual cosa deua fazer . \textbf{ Mostrado que prouechosa cosa es de fazer los castiellos . } avn en qual manera los enemigos presentes son de fazer los castiellos \\\hline
3.3.8 & Attamen quia figura circularis est capacissima , \textbf{ est elegibilius facere munitiones } secundum circularem formam , & mas que las otras \textbf{ por ende es | mas de escoger de fazer las guarniçiones } segunt la figura redonda \\\hline
3.3.8 & Nam contingit aliquando situm \textbf{ illum non pati talem formam . } In tali ergo casu construenda sunt castra semicircularia , & Ca algunas uegadas \textbf{ contesçe que el assentamiento non sufre tal figura . } Et por ende en tal caso deuen se fazer los castiellos \\\hline
3.3.9 & etiam postquam gustauerunt bella , \textbf{ appetere pugnam ; } hoc est ut raro . & Mas si contesçiere que los lidiadores sean de carnes blandas \textbf{ avn que ayan prouado las batallas pocas vezes quieren lidiar . } Ca assi commo dixiemos de suso \\\hline
3.3.9 & aut deficere : \textbf{ poterit accelerare pugnam , } vel differre : & ha conplimiento en estas seys condiçiones \textbf{ et fallesçe en ellas podra acometer la vatalla mas ayna o prolongar la } e lidiar publicamente o manifiestamente \\\hline
3.3.9 & vel deficere , \textbf{ sic se habere poterit erga bellum : } forte enim nunquam contingeret & que abonda o fallesçe en las mas destas condiçiones \textbf{ assi se podra auer en la batalla } e por auentura nunca contezçra que todas estas condiçones puedan ser de la vna parte . \\\hline
3.3.10 & si careant centurione et duce , \textbf{ qui debet esse eorum caput et eorum directiuum . } Inde est quod antiquitus & e \textbf{ mayorales que sean cabeças dellos e guiadores en la hueste . } Et por ende antiguamente \\\hline
3.3.10 & non sufficiunt ad dirigendum bellantes , \textbf{ sed oportet dare euidentia signa ; } ut quilibet solo intuitu sciat & para guiar los lidiadores . \textbf{ Mas conuiene de dar otras seña les manifiestas . por que cada vno viendo aquellas señales } se sepa tener ordenadamente en su az \\\hline
3.3.10 & proceri statura , \textbf{ scientes proiicere hastas et tela : } scire etiam debeant & e altos en el estado del cuerpo \textbf{ e que sepan lançar lanças e dardos . } e avn que sepan esgrimir las espadas \\\hline
3.3.10 & gladium vibrare ad percutiendum , \textbf{ portare scutum ad se protegendum : } et cum debeant esse vigilantes , agiles , sobrii , & para ferir meior \textbf{ e rodear el escudo | para encobrirse meior } e avn que ayan los oios bien espiertos \\\hline
3.3.10 & portare scutum ad se protegendum : \textbf{ et cum debeant esse vigilantes , agiles , sobrii , } habentes armorum experientiam : & para encobrirse meior \textbf{ e avn que ayan los oios bien espiertos | e que sean ligeros e mesurados en beuer e gerrdados de vino } e avn que ayan vso de las armas . \\\hline
3.3.10 & procer statura , \textbf{ sciens eiicere hastas et iacula , } sciens dimicare gladio ad percutiendum , & deue ser fuerte en el cuerpo . \textbf{ grande en su estado e sabidor en lançar lanças e dardos } e sabio en lidiar \\\hline
3.3.10 & sciens dimicare gladio ad percutiendum , \textbf{ portare scutum ad se protegendum , } vigilans , agilis , sobrius , & e sepa esgrimir el espada \textbf{ para ferir meior . | Rodearse e cobrirse del escudo } para se guardar e despierto e vigilante e ligero e mesurado \\\hline
3.3.10 & et ducem militaris belli \textbf{ esse habilem corpore , } ut possit etiam armatus agiliter equum conscendere : & que el que es antepuesto es cabdiello de la caualleria en la batalla \textbf{ que sea ligero en el cuerpo } por que pueda avn que sea armado sobir ligeramente en el cauallo \\\hline
3.3.11 & per viam aliquam \textbf{ in qua pati possit insidias , } nisi qualitates viarum , montes , flumina , & nin de las çeladas \textbf{ si el cabdiello de la batalla non ouiere escriptas . } o pintadas las qualidades de los caminos . \\\hline
3.3.11 & ut simul cum hoc quod habet vias \textbf{ et qualitates viarum conscriptas et depictas , } ducat dux belli conductores aliquos & que deue auer las carreras \textbf{ e las qualidades de los caminos escriptas e pintadas . } avn ayan otros guiadores \\\hline
3.3.11 & apponere custodias \textbf{ ne possint fugere . Debet etiam eis minari mortem , } si in aliquo fraudulenter se habeant , & deue el señor de la hueste poner en ellos buenas guardas \textbf{ porque non puedan foyr . | Avn deuen los amenazar de muerte } si en alguna cosa se ouieren engañosamente \\\hline
3.3.11 & si in aliquo fraudulenter se habeant , \textbf{ et promittere dona } si se fideliter gesserint . & si en alguna cosa se ouieren engañosamente \textbf{ e deuenles prometer dones } si fueren fieles a su prinçipe . \\\hline
3.3.11 & Tertia est , \textbf{ habere secum plures sapientes fideles principi , exercitatos in bellis , } de quorum consilio agat & La terçera cautela es \textbf{ que el cabdiello aya consigo muchos sabios | e fieles al prinçipe } e vsados en las batallas \\\hline
3.3.11 & et vias illas Dux habet \textbf{ conscriptas et depictas , } et habentur conductores aliqui fideles , & por quales caminos deue yr la hueste \textbf{ e aquellas carreras touiere el cabdiello escriptas e pintadas } e ouiere algunos omes guiadores fieles \\\hline
3.3.11 & et in qualibet acie \textbf{ habere aliquos equites fidelissimos et strenuissimos , } habentes equos veloces et fortes ; & que deue el señor de la hueste en cada conpaña \textbf{ e en cada vna az auer vnos caualleros muy fieles | e muy estremados } que ayan cauallos muy ligeros \\\hline
3.3.11 & Octaua cautela est , \textbf{ considerare exercitum } in quibus sit copiosior , & por que estauan aperçebidos . \textbf{ La . viij° . cautela es penssar } de quales ha mayor conplimiento la hueste de peones o de caualleros . \\\hline
3.3.11 & vel in peditibus , \textbf{ eligere poterit vias campestres et amplas , } vel montanas , syluestres , et nemorosas , & que ha conplimiento de caualleros o de peones \textbf{ podra escoger los caminos de los canpos | e carreras anchas o las de los montes } o de las siluas \\\hline
3.3.12 & si debeat publica pugna committi , \textbf{ et quibus cautelis abundare decet bellorum ducem } ne suus exercitus laedatur & si se deue la batalla acometer publicamente . \textbf{ Et quales cautelas ha de auer el señor de la batalla } por que la su hueste non sea dañada en el camino . \\\hline
3.3.12 & nisi ut doceamus ordinare acies , \textbf{ percutere aduersarios , } et inuadere hostes . & si non que mostremos en commo se deuen ordenar las azes \textbf{ e ferir los contrarios } e acometer los enemigos . \\\hline
3.3.12 & assuefacere bellantes , \textbf{ ut sciant construere aciem } secundum quamcunque formam . & Et assi de las otras maneras deuen costunbrar los lidiadores \textbf{ por que sepan parar el az } segun qua si quier forma o figura \\\hline
3.3.12 & His visis sciendum quadrangularem formam aciei \textbf{ inter caeteras formas esse magis inutilem : } ideo secundum hanc formam nunquam formanda est acies simpliciter , & Et vistas estas cosas \textbf{ conuiene de saber | que entre todas las otras formas de la az la quadrada es mas sin prouecho . } Et por ende nunca es de formar el az \\\hline
3.3.12 & Nam pugnantes vel solum volunt se defendere \textbf{ et sustinere ictus , } vel volunt alios inuadere . & ø \\\hline
3.3.12 & reseruentur aliqui milites strenui et audaces , \textbf{ qui possint succurrere ad partem illam , } erga quam viderint & sean guardados algunos estremados caualleros \textbf{ e osados que puedan acorrer a aquella parte } que vieren \\\hline
3.3.13 & Inde est quod bellorum experti dicunt pugnantes \textbf{ semper debere habere loricas amplas ita , } ut annuli loricarum se constringant : & que los que son prouados en las batallas . \textbf{ dizen que los lidiadores sienpre deuen auer las lorigas anchas . } assi que les aniellos de las lorigas se ayunten \\\hline
3.3.13 & In percutiendo autem caesim , \textbf{ quia oportet fieri magnum brachiorum motum prius quam infligatur plaga , } aduersarius ex longinquo potest prouidere vulnus , & Mas en feriendo cortando . \textbf{ por que conuiene de fazer grand mouimiento de los braços | ante que se de el colpe el enemigo } o el contrario de \\\hline
3.3.13 & ideo magis sibi cauere potest \textbf{ et cooperire se ictibus . } Ideo ait Vegetius , & Et por ende puede se mas guardar \textbf{ e encobrirse de aquellos colpes . } Et por esso dize vegeçio \\\hline
3.3.13 & Percutiendo enim caesim oportet \textbf{ eleuare brachium dextrum : } quo eleuato dextrum latus nudatur et discooperitur , & por que firiendo taiando \textbf{ conuiene de leuantar el braço derecho e diestro . } Et leuantando el braço derecho paresçe descubierto el costado derecho \\\hline
3.3.14 & Secundo debet diligenter \textbf{ explorare eorum itinera , } ut ad transitus fluuiorum , & Lo segundo deue escudriñar \textbf{ con grand acuçia los caminos dellos } assi commo el passo de los rios \\\hline
3.3.14 & quia sic facilius deuincentur . \textbf{ Tertio debet aspicere ad ipsum tempus : } quando sol reuerberat & mas ligeramente seran vençidos . \textbf{ Lo terçero deue catar el cabdiello al tienpo . } assi que quando el sol fiere en los oios de los enemigos \\\hline
3.3.14 & quando hostes magnam fecerunt dietam , \textbf{ sunt fatigati habent laxatos equos : } tunc enim , & quando los enemigos fizieren grant iornada \textbf{ e touieren los cauallos canssados . } Ca estonçe si los quisieren acometer \\\hline
3.3.14 & vel per se , \textbf{ vel per alios mittere dissensiones , iurgia , } commouere eos ad lites , & por otros discordias entre los enemigos \textbf{ e boluer contiendas } e lides o enemistades entre ellos \\\hline
3.3.14 & Septimo debet diligenter \textbf{ explorare conditiones hostium : } qualiter se gerant , & deue el cabdiello \textbf{ con grant acuçia escudriñar las condiçiones de los sus enemigos } en qual manera andan \\\hline
3.3.15 & quod manu ad manum se percutiunt . \textbf{ Aliter autem debent stare bellatores viri , } cum a remotis iacula iaciunt , & quando vienen a las manos \textbf{ e en otra manera deuen estar los lidiadores } quando lançan los dardos de lueñe . \\\hline
3.3.15 & Nam iaciendo iacula a remotis , \textbf{ debent habere ipsos pedes sinistros ante , } et dextros retro . & Ca lançando dardos de lueñe \textbf{ deuen tener los pies esquierdos } delante e los derechos detras \\\hline
3.3.15 & cum ad manum pugnant , \textbf{ tenere pedem sinistrum immobiliter : } et cum volunt percutere , & Ca deuen los lidiadores \textbf{ quando vienen a las manos tener el pie esquierdo firme } e quando quieren ferir \\\hline
3.3.15 & Inde est quod laudatur Scipionis sententia , dicentis : \textbf{ Nunquam sic esse claudendos hostes , } quod non pateat eis aditus fugiendi . & Et por ende es alabada la sentençia de çipion \textbf{ por la qual dizia que nunca eran de encerrar los enemigos } assi que les non fincasse logar para foyr . \\\hline
3.3.15 & Cum ergo supra diximus , \textbf{ formandam aliquando esse aciem sub forma forficulari , } ut quando hostes pauci , & Et pues que assi es quando dixiemos dessuso que \textbf{ alguans vezes el az es de formar so forma de tiieras e esto quando los enemigos son pocos } para que meior sean ençerrados e çerrados . \\\hline
3.3.15 & ex parte exercitus hostium . \textbf{ Nam sic debet deducere bellum , } ut hoc hostes lateat . & Mas la segunda cautela es de tomar de parte de la hueste de los enemigos . \textbf{ Ca assi deue escusar la batalla | pues non ha conseio de lidiar } que esto non lo sepan los enemigos . \\\hline
3.3.15 & qua recedente , equites postea melius possunt \textbf{ vitare hostium percussiones . } Est etiam aduertendum & encubiertamente se escusa yendo se los peones . \textbf{ Et ellos ydos los caualleros pueden meior despues escusar los colpes de los enemigos . } Avn conuiene de saber \\\hline
3.3.15 & hostes \textbf{ insequi fugientes a bello , } plures occiderent ; & ca podrie contesçer \textbf{ que los enemigos persiguirien a los que fuyessen de la batalla } e matarian muchos dellos \\\hline
3.3.16 & quod hostes de munitionibus exeuntes vadant \textbf{ bellare ad campum , } sed ipsi munitiones inuadunt & que los enemigos salgan a ellos de las villas \textbf{ e de las çibdades | e de los castiellos a lidiar al canpo . } Mas ellos acometen aquellas villas \\\hline
3.3.16 & in tanta multitudine esse , \textbf{ et tantam habere potentiam , } ut non expectant hostes exire ad campum , & que algunos lidiadores son en tan grand muchedunbre \textbf{ e de tan grant poder } que no esperan \\\hline
3.3.16 & et tantam habere potentiam , \textbf{ ut non expectant hostes exire ad campum , } sed ipsas munitiones obsideant et inuadant : & e de tan grant poder \textbf{ que no esperan | que salgan los enemigos al canpo . } Mas ellos acometen las villas \\\hline
3.3.16 & sed ipsas munitiones obsideant et inuadant : \textbf{ sic contingit aliquos esse adeo paucos et tam debiles , } ut non putent in campo & e los castiellos o las fortalezas e los çercan . \textbf{ assi contesçe que alguno lidiadores son tan pocos et tan flacos } que non cuydan \\\hline
3.3.16 & Contingit etiam aliquando aliquos \textbf{ inuadere aliquas munitiones eorum ; } propter quod eos oportet & Et algunas vezes çerca villas o castiellos o fortalezas . \textbf{ Et avn algunas vezes contesçe que algunos otros çercan sus villas o sus castiellos . } Por la qual cosa les conuiene de vsar de batalla defenssiua para se defender . \\\hline
3.3.16 & et ne terrae marinae impugnentur , \textbf{ expedit regibus et principibus aliquando ordinare bella naualia . } Dicto itaque de bello campestri , & e a los principes \textbf{ algunas vezes de ordenar | e de fazer batallas nauales e de naues . } Et pues que assi es dicho de la lid canpal \\\hline
3.3.16 & cum per huiusmodi pugnam \textbf{ contingat obtineri et deuinci munitiones et urbanitates : } restat dicere quot modis talia deuinci possunt . & por que por tal lid \textbf{ contesçe tomar e vençer las villas | e los castiellos e fortalezas . } fincanos de dezir \\\hline
3.3.16 & vel munitiones reddere . \textbf{ Quare diligenter excogitare debent obsidentes munitiones aliquas , } utrum per aliqua ingenia , & de sedo de dar las fortalezas . \textbf{ Por la qual cosa con grant acuçia deuen cuydar los | que cercan algunas fortalezas } si por algunos engeñios \\\hline
3.3.16 & vel per aliquam industriam possint \textbf{ ab obsessis accipere aquam . Nam multotiens euenit , } aquam a remoto principio deriuari & o por alguna sotileza puedan tomar el agua de los cercados . \textbf{ Ca muchas uegadas contesçe } que el agua viene de lueñe fasta las fortalezas cercadas . \\\hline
3.3.16 & per quam pergit aqua ad obsessos , \textbf{ oportebit ipsos pati aquarum penuriam . } Rursus , aliquando munitiones sunt altae , & por do viene el agua a los cercados han de auer los cercados \textbf{ por fuerça mengua de agua . } Otrossi algunas uegadas las fortalezas son altas \\\hline
3.3.16 & quare si sit a munitionibus remota , \textbf{ debent obsidentes adhibere omnem diligentiam , } quomodo possint obsessis prohibere aquam . & por la qual cosa si el agua fuere lueñe de la fortaleza \textbf{ los que cercan deuen auer grant acuçia } en commo defiendan el agua a los cercado o gela tiren . \\\hline
3.3.16 & debent obsidentes adhibere omnem diligentiam , \textbf{ quomodo possint obsessis prohibere aquam . } Secundus modus impugnandi munitiones , & los que cercan deuen auer grant acuçia \textbf{ en commo defiendan el agua a los cercado o gela tiren . } La segunda manera para ganar las fortalezas es por fanbre . \\\hline
3.3.16 & Inde est quod multotiens obsidentes \textbf{ volentes citius opprimere munitiones , } si contingat eos capere aliquos de obsessis , & Et por ende contesçe que muchas uegadas \textbf{ los que çercan queriendo | mas ayna ganar las fortalezas } si contezca \\\hline
3.3.16 & volentes citius opprimere munitiones , \textbf{ si contingat eos capere aliquos de obsessis , } non occidunt illos & mas ayna ganar las fortalezas \textbf{ si contezca | que prendan algunos de los cercados } non los matan \\\hline
3.3.16 & quo tempore melius est \textbf{ obsidere ciuitates et castra . } Sciendum itaque quod tempore aestiuo & las fortalezas cerradas \textbf{ finca de demostrar en que tienpo es meior de çercar las çibdades e las castiellos . } Et por ende conuiene de saber \\\hline
3.3.16 & Nam si per sitim sunt munitiones obtinendae , \textbf{ melius est facere obsessionem tempore aestiuo , } eo quod tunc magis desiccantur aquae , & Ca si por sed son de ganar las fortalezas \textbf{ meior es de fazer la çerca | en el tienpo del estiuo } por que entonçe mas se dessecan las aguas \\\hline
3.3.16 & Quare si obsessi non possunt \textbf{ gaudere fructibus anni aduenientis , } citius peribunt inopia . & Por la qual cosa si los que estan çercados \textbf{ non se pueden acorrer de los fuctos | de esse aneo en que esta . } Mas ayna pereres çran \\\hline
3.3.17 & Nam cum contingat obsessiones \textbf{ per multa aliquando durare tempora , } non est possibile obsidentes & Ca commo contezca \textbf{ que las çercas puedan durar algunas vezes } e por muchos tienpos non puede ser \\\hline
3.3.17 & non est possibile obsidentes \textbf{ semper esse paratos aeque . } Ideo nisi sint muniti , & que los que cercan \textbf{ sienpre esten apareiados de vna guisa | nin de vna manera . } Et por ende si non estudieren guarnesçidos puede les contesçer \\\hline
3.3.17 & vel iaculi debent castrametari , \textbf{ et circa se facere fossas , } et figere ibi ligna , & quanto podrie lançar la vallesta o el dardo \textbf{ e fazer carcauas enderredor de ssi } e finçar y grandes palos \\\hline
3.3.17 & et circa se facere fossas , \textbf{ et figere ibi ligna , } et construere propugnacula : & e fazer carcauas enderredor de ssi \textbf{ e finçar y grandes palos } e fazer algunas fortalezas \\\hline
3.3.17 & et construere propugnacula : \textbf{ ut si oppidani eos repente vellent inuadere , resistentiam inuenirent . } Viso quomodo se munire debent obsidentes , & assi que si los que estan çercados \textbf{ a desora los quisieren acometer fallen enbargo | por que los non puedan enpesçer . } visto en qual manera se deuen guarnesçer los çercadores \\\hline
3.3.17 & restat ostendere \textbf{ quot modis impugnare debent obsessos . } Est autem unus modus impugnandi communis et publicus , & finca de demostrar \textbf{ en quantas maneras se deuen acometer | los que estan cercados } et ay vna manera comun e publica de acometer e de lidiar \\\hline
3.3.17 & apponunt scalas ad muros , \textbf{ ut si possint ascendere ad partes illas . } Praeter tamen hos modos impugnationis apertos , & e ponen escaleras a los muros \textbf{ assi que si podieren sobir sean eguales dellos | para se dar con ellos } e para entrar los . \\\hline
3.3.17 & Praeter tamen hos modos impugnationis apertos , \textbf{ est dare triplicem impugnationis modum non omnibus notum . } Quorum unus est per cuniculos . & Enpero sin estas maneras manifiestas de batalla \textbf{ ay otras tres maneras | que non son manifiestas a todos } de las quales la vna es \\\hline
3.3.17 & quam sint fossae munitionis deuincendae , \textbf{ pergere usque ad muros munitionis praedictae : } quod si hoc fieri potest , & que esta cercada . \textbf{ Et assi deuen yr | fasta los muros de aquel logar . } Et si esto se puede fazer \\\hline
3.3.17 & Nam hoc facto primo debent muros fodere , \textbf{ et supponere ibi ligna } ne statim cadant . & ca esto fecho primero deuen socauar los muros \textbf{ e so poner y maderos } por que non puedan luego caer \\\hline
3.3.17 & quod per solum casum murorum possint munitionem obtinere , \textbf{ statim debent apponere ignem in lignis sustinentibus muros } et facere omnes muros & luego sin detenemiento \textbf{ ninguno deuen poner fuego en la madera | que sotienen los muros } e fazer que todos los muros o grand parte dellos cayan en vno a desora . \\\hline
3.3.17 & statim debent apponere ignem in lignis sustinentibus muros \textbf{ et facere omnes muros } vel facere magnam eorum partem cadere , & que sotienen los muros \textbf{ e fazer que todos los muros o grand parte dellos cayan en vno a desora . } Otrossi deuen fenchir las carcauas \\\hline
3.3.17 & et facere omnes muros \textbf{ vel facere magnam eorum partem cadere , } et replere fossas : & que sotienen los muros \textbf{ e fazer que todos los muros o grand parte dellos cayan en vno a desora . } Otrossi deuen fenchir las carcauas \\\hline
3.3.17 & vel facere magnam eorum partem cadere , \textbf{ et replere fossas : } quo simul ( quasi ex inopinato facto ) & e fazer que todos los muros o grand parte dellos cayan en vno a desora . \textbf{ Otrossi deuen fenchir las carcauas } assi que los que estan cercados a desora sean espantados \\\hline
3.3.17 & ut per eas possit \textbf{ haberi ingressus ad ciuitatem et castrum : } quae omnia latenter fieri possunt & por las cueuas soterrañas \textbf{ assi que por ellas puedan entrar a la çibdat o al castiello . } Et estas cosas todas deuense fazer muy encubiertamente \\\hline
3.3.17 & vel ciuitatem obsessam : \textbf{ et sic poterunt obtinere illam . } Contingit autem pluries , & en el castiello çercado \textbf{ e assi podran ganar aquellas fortalezas . } m muchas uegadas contesçe que algunas fortalezas çercadas son fundadas sobre pennas muy fuertes \\\hline
3.3.18 & munitiones aliquas obsessas \textbf{ super lapides fortissimos esse constructas , } vel esse aquis circumdatas , & e assi podran ganar aquellas fortalezas . \textbf{ m muchas uegadas contesçe que algunas fortalezas çercadas son fundadas sobre pennas muy fuertes } o son cercadas de agua \\\hline
3.3.18 & super lapides fortissimos esse constructas , \textbf{ vel esse aquis circumdatas , } vel habere profundissimas foueas , & m muchas uegadas contesçe que algunas fortalezas çercadas son fundadas sobre pennas muy fuertes \textbf{ o son cercadas de agua } o han carcauas muy fondas \\\hline
3.3.18 & vel esse aquis circumdatas , \textbf{ vel habere profundissimas foueas , } vel aliquo alio modo esse munitas : & o son cercadas de agua \textbf{ o han carcauas muy fondas } o son \\\hline
3.3.18 & vel habere profundissimas foueas , \textbf{ vel aliquo alio modo esse munitas : } ut per viculos & o han carcauas muy fondas \textbf{ o son | enfortalezidas en alguna otra manera } assi que por cueuas conegeras \\\hline
3.3.18 & et alia machinamenta obsidentium debeant prouidere . \textbf{ Quare si modus artis debet imitari naturam } quae semper faciliori via res ad effectum producit : & Por la qual cosa \textbf{ si la manera del arte deue semeiar a la natura . | la qual natura sienpre aduze } por el mas ligero camino \\\hline
3.3.18 & Nam in omni tali machina est \textbf{ dare aliquid trahens } et eleuans virgam machinae , & puedense adozer a quatro maneras . \textbf{ Ca en todo tal engeñio es de dar alguna cosa que traya } e leuante el pertegal del engennio \\\hline
3.3.18 & Est etiam aduertendum \textbf{ quod die et nocte per lapidarias machinas impugnari possunt munitiones obsessae . } Tamen , ut videatur qualiter in nocte percutiunt lapides emissi a machinis , & Et avn conuiene de saber \textbf{ que tan bien de noche commo de dia se pueden acometer las fortalezas cercadas | por los engeñios } que lançan piedras . \\\hline
3.3.19 & ideo appellatur aries , \textbf{ quia ratione ferri ibi appositi durissimam habet } frontem ad percutiendum . & Ca por razon del fierro \textbf{ que ponen y . | ha muy fuerte et muy . } dura fruente para ferir \\\hline
3.3.19 & ad impugnandum munitionem aliquam , \textbf{ dato quod quis non possit pertingere usque ad muros eius . } Nam quia huiusmodi trabs habens caput sic ferratum retrahitur et impingitur , & Et uale este artifiçio para acometer alguna fortaleza . \textbf{ puesto que non puedan llegar a los muros della . } Ca por que esta viga ha la cabeça \\\hline
3.3.19 & Nam si nec per arietes , \textbf{ nec per vineas capi possunt munitiones obsessae , } accipienda est mensura murorum munitionis illius , & Ca si las fortalezas cercadas non se pueden tomar par los carneros \textbf{ nin por las viñas sobredichas deuen tomarla mesura } e el alteza de los muros de aquella fortaleza \\\hline
3.3.19 & Potest autem per huiusmodi musculos \textbf{ quasi continuari castra } usque ad munitionem obsessam : & e pueden por estos muslos allegar los castiellos \textbf{ fasta la fortaleza que tienen cercada . } Et esto quando assi fuere fecho \\\hline
3.3.19 & ad muros munitionis obsessae , \textbf{ illi qui sunt in parte superiori debent proiicere lapides , } et fugare eos , & quanto deuen a los muros de la fortaleza los cercados \textbf{ aquellos que estan en la parte mas alta deuen lançar piedras } e fazer foyr \\\hline
3.3.19 & illi qui sunt in parte superiori debent proiicere lapides , \textbf{ et fugare eos , } qui sunt in muris . & aquellos que estan en la parte mas alta deuen lançar piedras \textbf{ e fazer foyr } los que estan en los muros \\\hline
3.3.19 & debent pontes dimittere , \textbf{ et inuadere muros . } Sed qui sunt in parte infima et sub musculis , & mas los que estan en el soberado de medio deuen echar puentes \textbf{ e acometer por los adarues . } mas los que estan en la parte mas baxa \\\hline
3.3.19 & ut etiam et sic obsidentes \textbf{ intrare possint obsessam munitionem . } Sunt etiam , ballistae arcus , & e foradar los \textbf{ por que puedan entrar en la fortaleza cercada } Avn son menester ballestas \\\hline
3.3.20 & volumus de bello defensiuo : \textbf{ ut postquam docuimus obsidentes qualiter debeant inuadere obsessos , } volumus docere ipsos obsessos qualiter & que es para se defender los çercados . \textbf{ assi que despues que ensseñamos | en qual manera } los que çercan deuen acometer los \\\hline
3.3.20 & ut postquam docuimus obsidentes qualiter debeant inuadere obsessos , \textbf{ volumus docere ipsos obsessos qualiter } se debeant & en qual manera \textbf{ los que çercan deuen acometer los | cercados queremos ensseñar en qual } lomanera los cercados se deuen defender de los que çercan . \\\hline
3.3.20 & ut obsessi faciliter possint \textbf{ defendere munitionem aliquam , } est scire , & e lo que mas faze \textbf{ para que los çercados ligeramente puedan defender las fortalezas } e saber en qual manera son de construyr \\\hline
3.3.20 & propter quod non est inconueniens \textbf{ construere huiusmodi muros } ex terra depressata ; & e fazen la fortaleza mas fuerte . \textbf{ Por la qual cosa mucho cunple fazer tales muros } e tales torres albarranas de tierra muy tapiada . \\\hline
3.3.20 & per quem locum poterunt proiici lapides , \textbf{ emitti poterit aqua ad extinguendum ignem , } si contingeret ipsum ab obsidentibus esse appositum . & Et por aquel logar pueden lançar piedras . \textbf{ e echar agua | por matar el fuego } si contesçiesse que los enemigos pusiessen fuego a las puertas . \\\hline
3.3.21 & Dicebatur enim supra , \textbf{ triplicem esse modum deuincendi munitiones : } videlicet per famem , sitim , et pugnam . & por que non puedan de ligero ser tomadas . \textbf{ Et dicho fue dessuso que tres maneras ay para tomar las fortalezas . } Conuiene de saber . \\\hline
3.3.21 & et non possit \textbf{ habere aquam nisi santam , } eo quod dulcem aquam habeat distantem , & e non podieren auer \textbf{ si non agua salada } por que el agua dulçe ba muy lueñe \\\hline
3.3.21 & et etiam aedificia necessaria munitioni fieri possint . \textbf{ Per ferra vero etiam reparari possint arma , } et fieri tela ; & cadahalsos \textbf{ los que fezieren menester en la fortaleza . | Et del fierro puedan las armas } e fazer fierros de dardos et de saetas \\\hline
3.3.21 & Per ferra vero etiam reparari possint arma , \textbf{ et fieri tela ; } et sagittae , et alia per quae impugnari valeant obsidentes . & Et del fierro puedan las armas \textbf{ e fazer fierros de dardos et de saetas | e las otras cosas } que son menester \\\hline
3.3.21 & quod si nerui deficiant , \textbf{ loco eorum adhiberi poterunt crines equi , } vel capilli mulierum . & E si por auentura fallesçieren los neruios en logar \textbf{ dellos pueden tomar las çerneias | e las colas dellos cauallos } e los cabellos de las mugeres , \\\hline
3.3.22 & vel ciuitatis obsessae . \textbf{ Quare si docuimus per praefatos modos } inuadere obsidentes obsessos : & castielloo de la çibdat cercada . \textbf{ Por la qual cosa si ya ensseñamos | por las maneras sobredichas } aquellos que çercan commo han de acometer a los que estan cercados . \\\hline
3.3.22 & Quare si docuimus per praefatos modos \textbf{ inuadere obsidentes obsessos : } reliquum est & por las maneras sobredichas \textbf{ aquellos que çercan commo han de acometer a los que estan cercados . } fincanos de demostrar \\\hline
3.3.22 & profundae foueae aquis repletae , impediuntur obsidentes ; \textbf{ ne obsessos impugnare possint per cuniculos , } et vias subterraneas . & e muy fondas enbargansse los que çercan \textbf{ por que non puedan acometer los | quiestan cercados } por carteras soterrañas . \\\hline
3.3.22 & et tunc propter duritiem lapidum \textbf{ non est facile per cuniculos debellare eam , } vel est supra petram & por la dureza de la peña \textbf{ que se non puede } cauaro esta la fortaleza assentada sobre peña blanda \\\hline
3.3.22 & et vias subterraneas est , \textbf{ facere in munitione obsessa viam aliam } correspondentem viae subterraneae factae ab obsidentibus . & El segundo remedio contra las cueuas conegeras \textbf{ e contra las carreras soterrañas es de fazer vna fortaleza cercada otra carrera } que responda a la carrera soterraña \\\hline
3.3.22 & et utrum per aliqua signa cognoscere possint \textbf{ obsidentes inchoare cuniculos : } quod cum perceperint , & o si por algunas señales pudieren conosçer \textbf{ que los que cercan comiençan a fazer cueuas coneieras . } Et quando esto entendieren \\\hline
3.3.22 & statim debent viam aliam subterraneam \textbf{ facere correspondentem illis cuniculis , } ita tamen quod via illa pendeat & ninguno deuen fazer otras cueuas soterrañas \textbf{ que respondan a aquellas cueuas coneieras | que vengan derechamente contra ellas . } Enpero assi lo deuen fazer \\\hline
3.3.22 & et partem obsessi ) \textbf{ debet esse bellum continuum , } ne obsidentes per viam illam munitionem ingrediantur . & e otra fezieron los cercados se \textbf{ deue acometer la batalla continuadamente | por que los -\-> } que cercan non puedan entrar \\\hline
3.3.22 & Debent etiam obsessi \textbf{ iuxta inchoationem viae subterraneae habere magnas tinnas plenas aquis vel etiam urinis : } et cum bellant contra obsidentes , & Avn deuen los cercados \textbf{ çerca el comienço de las carreras soterrañas | auer tiñas lleñas de agua o de oriñas . } En quando lidian contra los que los çercan deuen fingir que fuyen \\\hline
3.3.22 & debent se fingere fugere , \textbf{ et exire foueam illam } quo facto totam aquam aut urinam congregatam & En quando lidian contra los que los çercan deuen fingir que fuyen \textbf{ e deuen salir de aquella cueua } la qual cosa fecho toda aquella agua o aquella orina \\\hline
3.3.22 & quare si hoc aliquando factum fuit , \textbf{ non debemus reputare impossibile } ne iterum fieri possit . Viso quomodo resistendum sit debellationi factae per cuniculos : & Por la qual cosa si esto alguna vegada fue fecho \textbf{ non deuemos cuydar } que se non pueda fazer otra vegada . \\\hline
3.3.22 & restat videre quomodo obsessi debeant \textbf{ obuiare impugnationi factae per lapidarias machinas . } Contra has autem quadrupliciter subuenitur . & Visto en qual manera auemos de contrallar a la batalla fecha \textbf{ por los engenios que lançan las piedras . } Et podemos dar contra los engeñios quatro maneras de acorro . \\\hline
3.3.22 & et prius quam exercitus possit \textbf{ succurrere ad defendendum eam , } succendunt ipsam . & Et Ante que pueda la hueste acorrer a \textbf{ defenderle quemenle con fuego . } Mas si non osan salir \\\hline
3.3.22 & vel testam ex ferro ; \textbf{ et iuxta machinam illam construere fabricam } in qua aliquod magnum ferrum bene ignatur , & texidade fierro . \textbf{ Et cerca de aquel engeñio deuen fazer vna fragua } en que pongan vn grand pedaço de fierro \\\hline
3.3.22 & trabem ferratam percutientem muros munitionis obsessae \textbf{ propter duritiem capitis vocari Arietem . } Contra hoc autem constituitur & para ferir en los muros de la fortaleza \textbf{ por la dureza de la cabeça es llamado carnero } et contra este \\\hline
3.3.22 & et clam suffoditur terra \textbf{ unde debet transire castrum ; } qua suffossa , et castro demerso in ipsam propter magnitudinem ponderis , & e ascondidamente se puede cauar la tierra \textbf{ por que puedan passar } allende del castiello o de la villa çercada . \\\hline
3.3.22 & Quod si tamen contingeret \textbf{ per huiusmodi aedificia perforari muros munitionis obsessae : } cum de hoc dubitatur , & Enpero si por auentura contesçiere \textbf{ que por estos artifiçios fueren foradados los muros de la fortaleza çercada . } quando desto dubdaren ante que los muros sean foradados \\\hline
3.3.22 & aedificentur muri lapidei : \textbf{ ut si continget obsidentes intrare munitionem , } retineantur clausi inter muros illos ; & o si pueden deuen fazer muros de piedra \textbf{ assi que si los que çercan entraren de dentro de la fortaleza sean retenidos e ençerrados entre aquellos muros } assi que se non puedan defender \\\hline
3.3.23 & in aliis generibus bellorum , \textbf{ applicari poterunt ad naualem pugnam . } Circa hoc autem pugnandi genus , & en las otras maneras de las batallas se podran traer \textbf{ e ayuntar a esta lid de las naues . } Mas çerca esta manera de lidiar primeramente es de veer \\\hline
3.3.23 & in arboribus abundare , \textbf{ non est bonum incidere arbores , } ex quibus fabricanda est nauis . & en que el humor comiença de abondar e de cresçer en los arboles \textbf{ non es bueno de taiar los arboles } de los quales deue ser fecha la naue . \\\hline
3.3.23 & simul bellis naualibus , \textbf{ et exponere se periculis , } ne puppis per rimas naufragium patiatur . & a la batalla de las naues \textbf{ e al periglo de las aberturas . } por las quales la naua puede peresçer . \\\hline
3.3.23 & Immo in huiusmodi pugna oportet \textbf{ homines melius esse armatos , quam in terrestri : } quia cum pugnatores marini quasi fixi stent in naui , & Ante conuiene que en esta batalla de la \textbf{ naue sean los omnes . | meior armados que en la de la tierra } por que los lidiadores de la mar esten firmes \\\hline
3.3.23 & quem Incendiarium vocant . \textbf{ Expedit enim eis habere multa vasa plena pice , sulphure , rasina , oleo ; } quae omnia sunt cum stupa conuoluenda . & ca conuine de auer en las naues \textbf{ mucha usija de tierra | assi commo cantaros e o las } e otros belhezos tales \\\hline
3.3.23 & committendum durum bellum , \textbf{ ne possint currere ad extinguendum ignem . } Secundo ad committendum marinum bellum multum valent insidiae . & Et entonçe deuen acometer muy fuerte batalla contra los enemigos \textbf{ por que se non puedan acorrer | para matar el fuego . } Lo segundo para acometer batalla en la mar valen mucho las çeladas . \\\hline
3.3.23 & Nam velis eorum perforatis , \textbf{ et non valentibus retinere ventum ; } non tantum possunt ipsi hostes impetum habere pugnandi , & ca foradadas las velas \textbf{ e los treos non pueden retener el viento . } Et assi non pueden los enemigos auer tanta fuerça \\\hline
3.3.23 & Sexto consueuerunt nautae \textbf{ habere ferrum quoddam curuatum } ad modum falcis bene incidens , & si quisiere foyr de la batalla . \textbf{ Et lo sexto suelen los marineros auer vn fierro coruo bien agudo } e bien taiante a manera de foz . \\\hline
3.3.23 & Septimo consueuerunt \textbf{ e iam nautae habere uncos ferreos fortes , } ut cum vident se esse plures hostibus , & lo . vij° . \textbf{ suelen avn los marineros | auer coruos de fierro muy fuertes } e quando veen que son mas \\\hline
3.3.23 & in tali bello vident \textbf{ sibi imminere mortem : } quare si oculi bellantium & los que estan en las naues \textbf{ non veen donde les viene la muerte . } Por la qual cosa si los oios de los que lidian en la \\\hline
3.3.23 & Nona cautela est \textbf{ habere multa vasa plena } ex molli sapone , & o seran anegados en la mar . \textbf{ La . ix . cautela es auer muchos cantaros llenos } dexabon muelle \\\hline
3.3.23 & diu valentem durare sub aquis ; \textbf{ qui accepto penetrali sub aquis debet accedere ad hostilem nauem , } et eam in profundo perforare , & que puedan y mucho estar e lieuen taladros para foradar \textbf{ e llegunen se a la naue de los enemigos so el agua } e foraden la pordiuso . \\\hline
3.3.23 & et ex cupiditate eorum \textbf{ ordinari ad lucrum , } vel ad aliquam aliam satifactionem irae , vel concupiscentiae . & o nasçen de cobdiçia \textbf{ e la ordena algunan ganançia } o la ordena a alguna otra vengança de saña o de cobdiçia mundanal . \\\hline
3.3.23 & et ad commune bonum . \textbf{ Nam sic se debent habere bella } in societate hominum , & e al bien comun . \textbf{ ca assi se deuen auer las batallas } en la muchedunbre de los omnes \\\hline
3.3.23 & sic in conuersatione , et societate hominum est \textbf{ dare plures personas , et plures homines . } Et sicut quamdiu humores sunt aequati in corpore , & e en la conpañia de los omnes \textbf{ ay muchas perssonas e muchos omens } e assi commo mientra los humores son ygualados en el cuerpo del omne \\\hline
3.3.23 & et eorum qui sunt in regno . \textbf{ Supposito ergo Reges et Principes habere iustum bellum , } et hostes eorum iniuste perturbare pacem & e de todos los que son en el regno . \textbf{ Et pues que assi es puesto } que los Reyes e los prinçipes ayan batalla derecha \\\hline
3.3.23 & Supposito ergo Reges et Principes habere iustum bellum , \textbf{ et hostes eorum iniuste perturbare pacem } et commune bonum : & e de todos los que son en el regno . \textbf{ Et pues que assi es puesto } que los Reyes e los prinçipes ayan batalla derecha \\\hline
3.3.23 & et commune bonum : \textbf{ non est inconueniens docere eos omnia genera bellandi , } et omnem modum & Et pues que assi es puesto \textbf{ que los Reyes e los prinçipes ayan batalla derecha } et los sus enemigos turben la paz \\\hline

\end{tabular}
