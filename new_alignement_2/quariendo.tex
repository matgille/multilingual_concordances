\begin{tabular}{|p{1cm}|p{6.5cm}|p{6.5cm}|}

\hline
1.1.5 & mas es por auentra \textbf{ a¶ Onde el philosofo quariendo mostrar en el primero libro delas ethicas } que es neçesario de connosçer ante la fin & sed à fortuna . \textbf{ Unde Philosophus 1 Ethicor’ | volens ostendere } necessariam esse praecognitionem finis , ait , \\\hline
1.1.6 & en el primero libro delas ethicas \textbf{ quariendo mostrar } que cosa es la fe liçidat & Unde Philosophus 1 Ethicorum \textbf{ describens felicitatem , } ait , \\\hline
1.2.14 & quando alguno temiendo uerguença \textbf{ e quariendo ganar honrra } acomete alguna cosa fuerte e espantable . & quando aliquis timens verecundiam , \textbf{ et volens honorem adipisci , } aggreditur aliquod terribile , \\\hline
1.2.26 & e al menospreçiamiento \textbf{ Ca en quariendo omne obrar obras } que son dignas de grant honrra & ex consequenti vero contrariatur deiectioni . \textbf{ Inquirendo enim opera honore digna , } non solum contingit peccare per superbiam , \\\hline
1.2.26 & Ca el sobra uio demandado \textbf{ e quariendo su excellençia e sobrepuiamiento } mas que deue & Secundo decet eos esse humiles ratione operum fiendorum . \textbf{ Nam superbus quaerens suam excellentiam ultra quam debeat , } ut plurimum tendit \\\hline
1.2.32 & delas ethicas el delicamiento es vna molleza . \textbf{ Et por ende estos tales non quariendo sofrir ninguna cosa guaue } luego que padelçen o son passionados por alguna passion & delicia quaedam mollicies est . \textbf{ Tales ergo nihil difficile sustinere volentes , } statim cum passionantur , \\\hline
2.1.3 & que son ordenadas ala fin entendiendo \textbf{ e quariendo la fin . } Assi que la fin es primero quarida e entendida . & ea quae sunt ad finem , \textbf{ intendendo et volendo finem , } ita quod finis est primo volitus et intentus : \\\hline
2.1.7 & Et pues que assi es que el philosofo en el octauo delas ethicas \textbf{ quariendo mostrar } qual es el amistança del uaron & Sciendum ergo quod Philosophus 8 Ethic’ volens \textbf{ ostendere } qualis amicitia sit viri ad uxorem , \\\hline
2.2.5 & Onde el philosofo en el segundo libro dela \textbf{ methafisica quariendo prouar } que la costunbre es de grand fuerça dize assi . & Unde et Philosophus 2 Meta’ \textbf{ volens probare } consuetudinem esse magnae efficaciae , \\\hline
3.1.10 & e los padres con sus fijas . \textbf{ Empero socrates quariendo escusar este mal dix̉o } que al prinçipe dela çibdat pertenesçia de auer cuydado e acuçia & et patres filias . \textbf{ Socrates volens hoc inconueniens vitare , | dixit , } quod spectabat ad Principem ciuitatis \\\hline
3.1.15 & quando los çibdadanos amandose \textbf{ e quariendose muy bien fuessen } much ayuntados en amor . & ut quod tunc esset ciuitas optima , \textbf{ quando ciues se amando et diligendo maxime unirentur . } Sic ergo exposita mente Socratis \\\hline
3.1.15 & en menestrales e en labradores \textbf{ e en batalladores quariendo } que alo menos la çibdat ouiesse mil ł batalladores & Quod autem ciuitatem diuidebat \textbf{ in agricolas , artifices , et bellatores , } volens ciuitatem \\\hline
3.2.10 & ̃en el su regno \textbf{ non quariendo sofrir sus males leuna tanse contra el . } Et el tirano de que conosçe & excellentes et nobiles existentes \textbf{ in regno non valentes hoc pati , | insurgunt contra ipsum : } tyrannus autem ex quo talem se esse cognoscit , \\\hline
3.2.12 & que nunca mostraua la cara alegte \textbf{ e aquel tirano quariendo dar razon desto fizo despoiar a su hͣmano } e fizola tar & et quare nunquam hylarem vultum ostenderet . \textbf{ Tyrannus ille volens reddere causam quaesiti , | eum expoliari fecit , } et ligari : \\\hline
3.2.13 & que han del \textbf{ algunas vezes la acometen quariendo vengar aquellos tuertos } e aquellas miurias que rresçibieron ¶ & et propter vehementem iram aliquando inuadunt \textbf{ ipsum volentes latas iniurias vindicare . } Tertio insidiantur aliqui tyranno , \\\hline
3.2.13 & njn quiere el bien comun \textbf{ quariendo algunos alcançar la gloriar la honrra } que veen enel tirano acometen ler matanle , & et non quaerere commune bonum , \textbf{ volentes adipisci honorem } et gloriam quam conspiciunt in tyranno , \\\hline
3.2.23 & por ende el pho en el primero libro de la \textbf{ rectorica quariendo enduzir los iuezes a misericordia } contra los que yerran contra ellos dize & Ideo Philos’ 1 Rhet’ \textbf{ volens iudicantes | ad misericordiam adducere } erga delinquentes in ipsos ; \\\hline
3.2.32 & Et cuenta el philosofo enel terçero libro delas politicas \textbf{ quariendo de el arar } que cosa es la çibdat seys bienes & quod bonorum illorum sit potius . \textbf{ Narrat quidem Philosophus 3 Politic’ volens diffinire } quid sit ciuitas , \\\hline

\end{tabular}
