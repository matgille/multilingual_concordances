\begin{tabular}{|p{1cm}|p{6.5cm}|p{6.5cm}|}

\hline
1.1.2 & el que quiere tractar del gouernaiento \textbf{ e fablarde sy mesmo conujene de tractar } e de dar conosçimiento de todas aquellas cosas & volens tractare de regimine sui , \textbf{ oportet ipsum notitiam tradere de omnibus his } quae diuersificant mores et actiones . \\\hline
1.1.3 & njn guardar \textbf{ syn la gera de dios conviene de cada vn omne } e mayormente prinçipe o Rey & absque diuina gratia obseruari non possunt , \textbf{ decet quemlibet hominem , } et maxime regiam maiestatem \\\hline
1.1.9 & e muestre algua bondat de fuera \textbf{ por la qual cosa commo al Rey conuenga ser todo diuinal e semeiante a dios } si non es cosa conuenible & quod exterius bona praetendat . \textbf{ Quare cum Regem deceat } esse totum diuinum , \\\hline
1.2.1 & e la su bien andança . \textbf{ Et que non los conuiene poner la su fin en riquezas } nin en poderio çiuil & suam felicitatem debeant ponere , \textbf{ quia non decet | eos suum finem ponere in diuitiis , } nec in ciuili potentia , \\\hline
1.2.10 & assi conmosi quisiere auer mas de aquellos bienes \textbf{ de quanto le conuiene auer } por esta razon viene danno alos otros çibdadanos & ut quod velit habere plus de iis , \textbf{ quam eum deceat : } ex hoc infertur nocumentum aliis ciuibus : \\\hline
1.2.18 & que menos dan de quanto les conuiene ᷤ dar \textbf{ Et menos fazen de quanto les conuiene de fazer . } Et desto puede bien paresçer & Semper ergo cogitare debent , \textbf{ quod minora faciunt , | quam deceat . } Ex hoc autem apparere potest \\\hline
1.2.18 & los que son en el su regno \textbf{ mucho les conuiene de ser liberales e francos } Mas par tenesce al libal e alstan ço de catar tres cosas ¶ & qui sunt in Regno , \textbf{ maxime decet eos liberales esse . } Spectat autem ad liberalem primo \\\hline
1.2.18 & que si espendiere \textbf{ do non le conuiene espender } Mas los Reyes e los prinçipes de suranse & ubi oportet , \textbf{ quam si expendat | ubi non oportet . } Deuiant autem a liberalitate Reges , \\\hline
1.2.25 & e nos allega a aquello que la razon manda o uieda . \textbf{ Conuiene de dar en aquella cosa dos uirtudes ¶ La vna que nos allegue . } Et la otra qua nos arriedre dello . & si unum et idem aliter et aliter acceptum nos retrahit et impellit , \textbf{ oportebit circa illud dare duas virtutes , | unam impellentem , } et aliam retrahentem . \\\hline
1.3.1 & e los prinçipes poner su fin e su bien andança . \textbf{ Et otrosi mostrado es en commo les conuiene de ser uirtuosos } ¶ Agora finca de dezir dela tercera parte deste libro & in quo Reges et Principes suum finem ponere debeant , \textbf{ et quomodo oportet | eos virtuosos esse . } Restat exequi de tertia parte huius primi libri , \\\hline
1.3.6 & por ende en este primero libro \textbf{ conuiene de tractar delas costunbres de lons Reyes } uniuersalmente & ut dicitur 1 Physicorum , \textbf{ deo in hoc primo de moribus Regum oportet } pertransire uniuersaliter typo : \\\hline
1.3.7 & Ca nos podemos natanlmente querer mal a todos los ladrones . \textbf{ pero non conuiene de temer quetsteza se aconpanne a esta mal querençia . } ¶ La septima diferençia es & uniuersaliter omnes fures : \textbf{ non tamen oportet , | quod tristitia committetur huiusmodi odium . } Septima differentia est : \\\hline
1.4.4 & e confonden el entendemiento . \textbf{ Otrosy les conuiene de ser misconiosos non por fallesçimiento nin por llaqueza de } coraçon quales en los vieios . & rationem percutiunt . \textbf{ Decet etiam eos esse miseratiuos , | non propter defectum , } vel propter imbecillitatem : \\\hline
1.4.5 & Onde el philosofo dize en el quarto libro de la rectorica \textbf{ que conuiene de ser los nobles magranimos } e de grandes coraçones e magnificos & Unde Philos’ 4 Eth’ ait , \textbf{ quod magnanimos et magnificos decet } esse nobiles et gloriosos . \\\hline
2.1.4 & todas las cosas neçessarias ala uida \textbf{ conuiene de dar comunidat ala çibdat } sobre la comunidat deluarrio . & omnia necessaria ad vitam , \textbf{ praeter communitatem vici | oportuit } dare communitatem ciuitatis . \\\hline
2.1.10 & por que en el casamiento \textbf{ dellos conuiene de guardar la orden natural mas que en otro ninguno . } ¶ Lo segundo esso mismo pue de ser mostrada & coniuges Regum et Principum , \textbf{ quia in eorum coniugio magis quam in alio decet | naturalem ordinem conseruare . } Secundo hoc idem inuestigari potest \\\hline
2.1.16 & e en qual manera de una vsar del . \textbf{ Et pues que assi es conuiene de desçender } mas en particular & et quomodo utendum sit eo . \textbf{ Oportet ergo magis in particulari descendere , } qualiter omnes ciues \\\hline
2.1.19 & e then a sus maridos a mayor amor . \textbf{ Et por ende les conuiene de ser calladas } e en essa misma manera avn les conuiene de ser estables e firmes & et ad maiorem amorem viros inducunt : \textbf{ decet ergo eas esse taciturnas . } Sic etiam decet esse stabiles : \\\hline
2.2.2 & e en algun sennorio \textbf{ en quel conuiene gouernar los otros } mucho les conuiene de ser sabios e buenos . & et in aliquo dominio , \textbf{ in quo oportet eos alios gubernare ; } maxime decet eos esse prudentes et bonos . \\\hline
2.2.7 & e en las sçiençias liberales \textbf{ quanto mas les conuiene de ser mas entendudos } e mas sabios que los otros & insudare literalibus disciplinis , \textbf{ quanto decet eos intelligentiores et prudentiores esse , } ut possint naturaliter dominari . \\\hline
2.2.10 & ¶ La primera quanto alas cosas uisibles \textbf{ que assi commo non les conuiene de fablar cosas torpes } Et la razon desto pone el philosofo en łvij̊ libro delas ethicas & Quantum ad visibilia quidem , \textbf{ quia sicut non decet | eos turpia sequi : } sic indecens est eos turpia videre . \\\hline
2.2.20 & Mas si alguno demandare \textbf{ de que se deuen trabaiar las mugers conuiene de fablar en tales cosas departidamente } segunt el departimiento delas perssonas & Si autem quaeratur \textbf{ circa qualia opera solicitari debent : | oportet in talibus differenter loqui } secundum diuersitatem personarum . \\\hline
2.2.21 & ostrado que non conuiene alas moças de andar uagarosas a quande e allende \textbf{ nin les conuiene de beuir ociosas } finca que agora lo terçero mostremos & quod non decet puellas esse vagabundas , \textbf{ nec decet eas viuere otiose : } restat ut nunc tertio ostendamus , \\\hline
2.3.16 & en el gouernamiento delas casas de los Reyes \textbf{ En las quales por la grandeza de los offiçios conuiene de } acomne dar vn ofiçio a muchos seruientes & in gubernatione domorum regalium , \textbf{ ubi propter magnitudinem officiorum oportet } idem ministerium committi ministris multis , \\\hline
2.3.20 & Mas podemos mostrar por dos razones \textbf{ que non conuiene de fablar mucho en las mesas de los Reyes } nin de los prinçipes & Possumus autem duplici via ostendere , \textbf{ quod non decet | in mensis Regum et Principum } et uniuersaliter omnium nobilium \\\hline
2.3.20 & Et pues que assi es los Reyes \textbf{ e los prinçipeᷤ alos quales conuiene ser muy tenprados } e guardar la orden natural en toda & Reges ergo et Principes , \textbf{ quos decet maxime temperatos esse , } et obseruare ordinem naturalem \\\hline
3.1.8 & assi commo de andar e de tanner e de oyr e deuer . \textbf{ por ende conuiene de dar . } y departidos mienbros & ut ambulatione , tactu , visione , \textbf{ et auditus ideo oportet } ibi dare diuersa membra exercentia diuersos actus : \\\hline
3.1.8 & auemos mester casas e uestid̃as e viandas e otras cosas tales \textbf{ por ende conuiene de dar algun departimiento en la çibdat por que en ella sean falladas todas las cosas } que cunplen ala uida . & et aliis huiusmodi ; \textbf{ oportet in ciuitate | dare diuersitatem aliqua , } ut in ea reperiatur sufficientia ad vitam . \\\hline
3.1.8 & por ende commo estas cosas demanden departimiento \textbf{ conuiene de dar en la çibdat algun departimiento . } La quanta razon se toma & Quare cum hoc diuersitatem requirat , \textbf{ oportet in ciuitate | dare diuersitatem aliquam . } Quinta uia sumitur \\\hline
3.1.8 & si non sopiere en qual manera es establesçida la çibdat \textbf{ e si non sopiere en qual manera conuiene de auer en ella departimiento de ofiçios e de ofiçiales } l sermon en los comienços deueser luengo & nisi sciuerit qualiter constituitur ; \textbf{ et nisi cognoscat | quod oportet in ea diuersitatem esse . } Sermo in principiis debet esse longus , \\\hline
3.1.12 & Et por ende por que los lidiadores non se enflaquezcan en las batallas \textbf{ conuiene de echar dela batalla } e dela fazienda alos de flaco & ne igitur reddantur bellantes pusillanimes , \textbf{ quos constat esse timidos oportet } ab exercitu expelli . \\\hline
3.1.14 & que sienpre los çibdadanos \textbf{ non les conuenga de lidiar } por defendimiento de su tierra & ab artificibus et ab aliis ciuibus , \textbf{ quod ciues alii pro defensione patriae bellare non oporteat } melius est ergo dicere in ciuitate \\\hline
3.1.17 & por que podrian los çibdadanos auer tan pocas possessiones \textbf{ que les conuenia de beuir } assi es casamente & possent enim ciues adeo modicas possessiones habere , \textbf{ quod oporteret eos ita parce viuere } quod opera liberalitatis de facili exercere non valerent . \\\hline
3.2.1 & entp̃o dela paz \textbf{ por las leyes conuiene de fazer tractado destas quatro cosas sobredichͣs en este gouernamiento ¶ } La segunda razon para prouar & Quare si considerentur quae requiruntur ad hoc quod tempore pacis per leges bene gubernetur ciuitas , \textbf{ oportet in huiusmodi regimine | de praedictis quatuor considerationem facere . } Secunda via ad inuestigandum hoc idem sumitur ex fine \\\hline
3.2.13 & segund que dize el philosofo \textbf{ ca conujene de dar a entender } que estos tales non han cuydado de saluar su vida ¶ & ( ut ait Philos’ ) \textbf{ sunt paucissimi numero , | supponi oportet } eos nihil curare , \\\hline
3.2.17 & por la quel cosa commo muchs mas cosas ayan prouadas \textbf{ que vno solo conuiene de llamar otros } para los negoçios . por que por el conseio dellos pueda ser escogida la meior carrera & Quare cum plures plura experti sint , \textbf{ quam unus solus : | decet ad huiusmodi negocia alios aduocare , } ut per eorum consilium possit \\\hline
3.2.21 & La primera seqma par aquello que tales palabras han de to terçeres desegualar eliez \textbf{ el qual conuiene de ser } assi commo regla derecha en & obligare habent iudicem , \textbf{ quem esse oportet } quasi regulam in iudicando . \\\hline
3.2.26 & en el quarto libro delas politicas \textbf{ que non conuiene de apropar las comunidades } delas çibdades alas leyes . & Ideo dicitur 4 Politicorum \textbf{ quod non oportet } adaptare politias legibus , \\\hline
3.3.8 & e fazer muy apriessa . \textbf{ Mas conuiene de poner algunos maestros } para costruyr los castiellos & debet celeriter castra construere . \textbf{ Oportet autem semper construendis castris , } et faciendis fossis aliquos magistros praestitui , \\\hline
3.3.8 & e son de fazer mas anchas carcauas . \textbf{ mas solamente quieren y estar vna noche o por poco tienpo non conuiene de fazer tantas guarniçiones . } Mas la manera e la quantidat de las carcauas pone la vegeçio & aut ibi debet \textbf{ per modicum tempus existere , | non oportet tantas munitiones expetere . } Modum autem , \\\hline
3.3.13 & por que quanto aquellos aniellos mas son ayuntados . \textbf{ tanto conuiene de cortar mas dellos } para que los colpes enpeescan . & quia quanto illi annuli magis sunt compacti , \textbf{ tanto oportet plures ex eis frangere } ut vulnera noceant . \\\hline
3.3.14 & en qual manera deuen lidiar . \textbf{ Por la qual cosa les conuiene de foyr . } Lo quarto el señor de la hueste se deue tenprar & qualiter debeant dimicare : \textbf{ propter quod oportebit eos fugam eligere . } Quarto dux exercitus sic se temperare debet : \\\hline
3.3.16 & Ca contesçe algunas vegadas que los cercados non han agua . \textbf{ e por ende o les conuiene de peresçer o de morir } de sedo de dar las fortalezas . & Contingit enim aliquando obsessos carere aqua : \textbf{ ideo vel oportet eos siti perire , } vel munitiones reddere . \\\hline
3.3.22 & la qual tierra cauada \textbf{ conuiene de apoyar bien el castiello o la çerca } por que se non funda & qua suffossa , et castro demerso in ipsam propter magnitudinem ponderis , \textbf{ oportet castrum iterum construi , } eo quod non possit \\\hline
3.3.23 & de la batalla de las naues . \textbf{ enpero non conuiene de nos } de tener çerca esto tanto . & volumus aliqua de nauali bello : \textbf{ non tamen oportet } circa hoc tantum insistere , \\\hline

\end{tabular}
