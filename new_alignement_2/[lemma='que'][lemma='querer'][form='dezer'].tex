\begin{tabular}{|p{1cm}|p{6.5cm}|p{6.5cm}|}

\hline
1.1.5 & que en olimpiedes \textbf{ que quiere dezer en aquellas faziendas } o es aquellas batallas & quod in Olimpidiadibus , \textbf{ idest in illis bellis } et agonibus \\\hline
1.2.6 & en el sexto libro delas ethicas . \textbf{ Eubullia que quiere dezer uirtud para bien coseiar . } la otra es por la qual & quam Philosophus Ethic’ 6 appellat eubuliam , \textbf{ idest bene consiliatiua . } Alia vero per quam bene iudicamus de inuentis , \\\hline
2.3.10 & Et canssoria de canbio . \textbf{ Obolostica que quiere dezer maunera } de tornar los dineros en pasta . & quatuor species pecuniatiuae : \textbf{ videlicet naturalem , campsoriam , obolostaticam , } et tacos siue usuram : \\\hline
2.3.10 & ¶La terçera manera del arte pecuniatiua de dineros es obolostica \textbf{ que quiere dezer arte de peso sobrepuiante que por auentura fue fallada assi . } Ca assi commo la massa del metal es partida en los dineros & Tertia species pecuniatiuae est obolostatica , \textbf{ vel ponderis excessiua : | quae forte sic inuenta fuit . } Nam sicut massa metalli \\\hline
2.3.12 & que acresçientan las riquezas es fazer monopolia \textbf{ que quiere dezer vendiconn de vno solo . } Ca quando vno solo uende taxa el preçio & est facere monopoliam , \textbf{ idest facere vendationem unius : | nam quia unus solus vendit , } taxat precium \\\hline
3.2.2 & ca el regno e la aristo carçia \textbf{ que quiere dezer sennorio de buenos } e la poliçia & et tres sunt mali . \textbf{ Nam regnum aristocratia , } et politia sunt principatus boni : \\\hline
3.2.2 & e la poliçia \textbf{ que quiere dezer pueblo bien } enssenoreante son bueons prinçipados . & Nam regnum aristocratia , \textbf{ et politia sunt principatus boni : } tyrannides , oligarchia , et democratia sunt mali . \\\hline
3.2.2 & enssenoreante son bueons prinçipados . \textbf{ La thirama que quiere dezer sennorio malo } e la obligaçia que quiere dezer sennorio duro . & et politia sunt principatus boni : \textbf{ tyrannides , oligarchia , et democratia sunt mali . } Docet enim idem ibidem \\\hline
3.2.2 & La thirama que quiere dezer sennorio malo \textbf{ e la obligaçia que quiere dezer sennorio duro . } Et la democraçia & tyrannides , oligarchia , et democratia sunt mali . \textbf{ Docet enim idem ibidem } discernere \\\hline
3.2.2 & e tal prinçipado es dicħa ristrocaçia \textbf{ que quiere dezer prinçipado de buenos omes } e uir̉tuosos e dende vienen & dicitur Aristocratia , \textbf{ quod idem est | quod principatus bonorum et virtuosorum . } Inde autem venit \\\hline
3.2.7 & do dize que la tirnia es la postrimera obligarçia \textbf{ que quiere dezer muy mala obligacion } por que es muy enpesçedera alos subditos ¶ & tyrannidem esse oligarchiam \textbf{ extremam idest pessimam : } quia est maxime nociua subditis . \\\hline
3.2.12 & e es llamado anstrocraçia \textbf{ que quiere dezer señorio de buenos . } Mas si enssennorear en pocos non & et vocatur aristocratia \textbf{ siue principatus bonorum . } Si vero dominentur non quia boni , \\\hline
3.2.12 & mas por que son ricos es llamado obligarçia \textbf{ que quiere dezer señorio tuerto . } Mas quando enssennore a todo el pueblo & sed quia diuites , \textbf{ est peruersus | et vocatur oligarchia . } Sed si dominatur totus populus \\\hline
3.2.27 & Lo segundo que sean bien guardadas \textbf{ que quiere dezer tanto commo } que atales leyes & secundo , ut bene custodiantur , \textbf{ vel ( quod idem est ) } ut legibus sic institutis bene obediatur . \\\hline

\end{tabular}
