\begin{tabular}{|p{1cm}|p{6.5cm}|p{6.5cm}|}

\hline
1.1.7 & tum quia sunt diuitiae \textbf{ ex institutione Hominum , tum quia cum sint corporalia , } ipsi indigentiae corporali & Lo segundo que por que estas Riquezas son riquezas \textbf{ por ordenamiento e estableçemiento | delons omes } e non en otra gnisa¶ \\\hline
1.1.8 & Honor ergo habet rationem boni extrinseci , \textbf{ cum sit reuerentia exhibita } per quaedam exteriora signa . & que esta de fuera \textbf{ por que es reuerençia fecha } por señales mostradas de fuera¶ \\\hline
1.1.8 & Nam si Princeps suam felicitatem in honoribus ponat , \textbf{ cum sufficiat ad hoc } quod quis honoretur , & si pusiere la su bien andança en honrras \textbf{ conmoabaste acanda vno } para que sea honrrado \\\hline
1.1.10 & quia tale dominium \textbf{ cum sit violentum , } et contra naturam , & Mas es de poner en aquello \textbf{ que sienpre ha de durar ¶ } La segunda razon se declara \\\hline
1.1.13 & si non transgrediantur , \textbf{ cum possint transgredi , } maioris meriti esse videntur . & si non trispassaren los mandamientos de dios \textbf{ conmolos podiessen } trispassar son de mayor meresçimiento . \\\hline
1.2.5 & per quam de ipsis agibilibus rectas rationes faciamus . \textbf{ Rursus cum contingat operari recte et non recte , } sic ut est dare virtutem , & que fazemos fagamos razones derechas ¶ \textbf{ Otrosi commo contesca de obrar derechamente | e non derechamente } assi commo auemos a dar uirtud . \\\hline
1.2.5 & principalior omnibus aliis , \textbf{ cum sit directiua omnium aliarum , } et iustitia sit principalior & ¶Otrosi por que la prudençia es mas prinçipal que todas las otras \textbf{ por que es endereçadora e regladora de todas las otras ¶ } Et la iustiçia en pos ella es mas prinçipal que la fortaleza e la tenperança . \\\hline
1.2.15 & nomine Phyloxenus , \textbf{ qui , cum esset pultiuorax , orauit , } ut guttur eius longius quam gruis fieret . & e tomaua muy grant delectaçion en ellas g̃rago a dios \textbf{ quel feziese la garganta } mas luenga que garganta de grulla \\\hline
1.2.16 & in Rege Sardanapallo , \textbf{ qui cum esset totus muliebris , } et deditus intemperantiae ( ut recitat Iustinus Historicus , libro 1 abbreuiationis Trogi Pompeii ) & enxienplo en el Rey sardan \textbf{ apalo que por que era todo mugeril | e dado a mugers } e era muy \\\hline
1.2.18 & de leui \textbf{ quis cum sit prodigus , } fieri poterit liberalis . & desbien commo el libal non lo es de ligero se puede fazer \textbf{ qual quier gastador liberal e franco ¶ } pues que assi es si es conueinble al Rey \\\hline
1.2.23 & ut esse veridicos ; \textbf{ cum sint regula aliorum , } quae obliquari , & de seer manifiestos e claros e seer uerdaderos \textbf{ por que son regla de los otros } La qual regla non se deue torcer nin falssar \\\hline
1.3.1 & passionem oppositam irae . \textbf{ Sed cum sit quaedam virtus } inter iram et mansuetudinem , & Ca la manssedunbre nonbra propriamente passion contraria ala saña . \textbf{ Mas por que ha de ser alguna uirtud entre la sanna e la mansedunbre } la qual uirtud non podemos nonbrar \\\hline
1.3.2 & Nam spes , et desperatio , \textbf{ cum sumantur respectu boni , } praecedunt timorem , et audaciam , iram , et mansuetudinem , & Maen el tercero logar son de poner la esperançar la desesꝑança . \textbf{ Ca por que son tomadas } por razon de bien son puestas primero que el temor e la oladia \\\hline
1.3.3 & Unde et Valerius Maximus de Dionysio Ciciliano recitat , \textbf{ qui cum esset tyrannus , } erat amator proprii commodi , & Onde ualerio maximo cuenta de dionsio seziliano \textbf{ que commo fuesse tyrano } e amador de propio prouecho despoblaua \\\hline
1.3.7 & ei sed odium pro nullo miserebitur , \textbf{ cum sit quid insatiabile . } Octaua differentia est : & Mas la mal querençia de ninguno non se apiada por que es cosa \textbf{ que se non farta . } ¶ La octaua diferençia es \\\hline
1.4.1 & Iuuenes ergo , \textbf{ cum sint liberales , et cum sint animosi et bonae spei , } non habent unde retrahantur & e entremetesse de fazer grandes cosas . \textbf{ Et pues que assi es commo los mancebos } non ayan ninguna cosa \\\hline
1.4.3 & Verecundia ergo , \textbf{ cum sit timor inhonorationis , } non competit senibus ; & en el segundo libro dela Rectorica \textbf{ Ca por que la uerguença es temor de desonera non pertenesçe alos uieios } por que may orcuidado han del prouecho \\\hline
2.1.14 & quibus vacare debeant \textbf{ cum sint adulti : } ad quae non sunt instruendae uxores , & e alas obras çiuiles alas quales deuen entender \textbf{ quando fueren criados } e mayores alas quales cosas non son de enssennar las mugers \\\hline
2.1.19 & Unde et aliquos Philosophos legimus sic fecisse , \textbf{ qui cum essent impeditae linguae , } accipientes specialem conatum & Ende leemos que algunos philosofos lo fizieron \textbf{ assi los quales commo ouiessen las lenguas enbargadas } tomaron especial esfuerço çerca aquellas letras \\\hline
2.1.23 & elegibilius esset consilium muliebre quam virile . \textbf{ Natura enim cum moueatur ab intelligentiis , et a Deo , } in quo est suprema prudentia ; & e la razon es esta \textbf{ por que la natura toda es mouida delos angeles | e de dios } en que es conplimiento de sabiduria . Et por ende conuiene que aquellas cosas \\\hline
2.3.9 & et talia quibus indigemus ad vitam , \textbf{ cum sint magni ponderis , } commode ad partes longinquas portari non possunt . & que auemos menester para la uida \textbf{ por que son de grand peso } non las poderemos leuar conueniblemente a luengas tierras . \\\hline
2.3.12 & qui primo philosophari coeperunt . \textbf{ Ipse enim cum esset pauper , } et improperaretur sibi a multis cur philosopharetur , & que primeramente comneçara a philosofo far . \textbf{ Este mille sio commo fuesse muy pobre } e le denostassen sus amigos \\\hline
3.1.7 & et maxima coniunctio in ciuitate . \textbf{ Nam cum sit maxima unitas , } et maxima coniunctio patrum ad filios , & e muy grant ayuntamiento enla çibdat . \textbf{ Ca commo sea muy grant vnidat } e grant ayuntamiento de los padres alos fijos los mas antiguos \\\hline
3.1.13 & vel ad aliquem magistratum assumitur . \textbf{ Quare cum deceat regia maiestatem } et uniuersaliter omnem ciuem , & commo se conosçe despues que esle un atada en alguna dignidat o en algun maestradgo o en algun poderio \textbf{ por la qual razon commo venga ala real magestad } e generalmente a qual quier que ha de dar \\\hline
3.1.14 & nisi bellare , \textbf{ cum adesset oportunitas : } et onerosius et quasi omnino importabile esset sustentare sic quinque milia : & si non lidiar \textbf{ quando fuesse me este } Et muy mayor carga e peor de sofrir serie \\\hline
3.2.10 & et de se confidere ; \textbf{ nam cum intendat bonum ipsorum ciuium et subditorum , } naturale est & e que fien vnos de otros . \textbf{ Ca commo el entienda enl bien de los çibdadanos natural } cosaes que sea amado dellos . \\\hline
3.2.12 & Priuatur ergo tyrannus a maxima delectatione , \textbf{ cum videat se esse populis odiosum . } Viso tyrannidem cauendam esse , & e por ende el tirano es pri uado de grant delectaçion \textbf{ quando bee | que es aborresçido delos pueblos } Disto que la tirauja es de esquiuar e de aborresçer \\\hline
3.2.13 & ut recte et debite gubernent populum sibi commissum : \textbf{ cum deuiare a recto regimine sit tyrannizare , } et iniuriari subditis , & e commo desinarse \textbf{ e arredrarse los rreyes del | gouernemjento derecho sea tiranizar } e fazer tuerto alos subditos \\\hline
3.2.16 & quae tractanda sunt circa ipsum . \textbf{ Sed , cum dicat Philosophus } 3 Ethic’ & quales cosas son de trattrar çerca el . \textbf{ Mas commo el pho diga en el segundo libro delas ethicas } que por çierto alguno tomara consseio non de aquellas cosas \\\hline
3.2.29 & quam legem . \textbf{ Nam Rex cum sit homo } non dicit intellectum bonum tantum , & corconper el rey que la ley . \textbf{ Ca el Rey por que es omne } non dize entendemiento tan solamente mas dize entendimiento con cobdiçia . \\\hline
3.2.30 & et ad vitium si sint mali , \textbf{ cum procedant ex interiori appetitu . } Sed si consideretur lex humana & e a pecado si son malas . \textbf{ quando uieñe del apetito | et ple desseo del coraçon . } Mas si fuere penssa para la ley \\\hline
3.3.6 & ac si deberent pugnam committere . \textbf{ Et cum viderit magister bellorum } aliquem non tenere ordinem debitum in acie , & assi commo si ouiessen de acometer la batalla . \textbf{ Et quando vieren los caudiellos maestros de las batallas } que alguno non guarda orden en la az \\\hline
3.3.10 & portare scutum ad se protegendum : \textbf{ et cum debeant esse vigilantes , agiles , sobrii , } habentes armorum experientiam : & para encobrirse meior \textbf{ e avn que ayan los oios bien espiertos | e que sean ligeros e mesurados en beuer e gerrdados de vino } e avn que ayan vso de las armas . \\\hline
3.3.16 & nauales dicuntur . \textbf{ Quare cum sint quatuor genera pugnarum , } postquam diximus de campestri , & son dichas batallas nauales e de naues . \textbf{ Por la qual cosa commo sean quatro maneras da batallas } despues que dixiemos de la batalla de la tierra fincanos de dezir de las otras tres . \\\hline
3.3.17 & ab obsessis molestari poterunt . \textbf{ Nam cum contingat obsessiones } per multa aliquando durare tempora , & de los que estan çercados \textbf{ Ca commo contezca } que las çercas puedan durar algunas vezes \\\hline
3.3.17 & contingit quod existentes in castris \textbf{ ( cum fuerint occupati obsidentes somno , } vel ludo , vel ocio , & o en las çibdades cercadas \textbf{ quando los que çercan durmieren o comieren o estudieren de vagar o fueren derramados } por alguna neçessidat a desora pueden dar en ellos \\\hline
3.3.22 & obsidentes inchoare cuniculos : \textbf{ quod cum perceperint , } statim debent viam aliam subterraneam & que los que cercan comiençan a fazer cueuas coneieras . \textbf{ Et quando esto entendieren } luego sin detenimiento \\\hline

\end{tabular}
