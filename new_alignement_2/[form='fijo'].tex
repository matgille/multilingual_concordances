\begin{tabular}{|p{1cm}|p{6.5cm}|p{6.5cm}|}

\hline
1.1.8 & Et desto auemos \textbf{ enxienplo en vn fijo de vn prinçipe Romano } que dizien torcato & Exemplum huiusmodi habemus \textbf{ de filio cuiusdam Romani Principis nomine Torquati , } qui nimii honoris auidus , \\\hline
1.1.8 & porque los otros tomasen exienplo dello \textbf{ e que non fuesen cobdiçon sos de honrra mato a su fijo presunptuoso } e soƀuio commo quier & ne alii ex hoc exemplum assumerent , \textbf{ et ne essent nimii honoris auidi , | filium sic praesumptuosum occidit , } non obstante quod dictus filius \\\hline
1.1.8 & e soƀuio commo quier \textbf{ que aquel su fijo ouiese auido uictoria de los sus enemigos ¶ } Pues que assi es el prinçipe & non obstante quod dictus filius \textbf{ victoriam obtinuerat ab hoste . } Ne ergo Princeps se praecipitet , et ne nimis praesumat , \\\hline
1.2.32 & Ca quando alguno quaria conbidar a \textbf{ otrossi el su fiio non era en casa tomaua prestado el fijo de otro su vezino } e aprestaual para fazer el conbit & Cum enim qui alios conuiuare volebat , \textbf{ si filius suus domi non erat , | a vicino suo mutuabat filium , } et ipsum parabat in conuiuium , \\\hline
1.2.32 & por la qual cosa non le paresçia \textbf{ que era fijo de padre } e de varon meral & quod erat valde bonus : \textbf{ propter quod non videbatur existere puer , } viri moralis , sed Dei . \\\hline
1.2.32 & e bien acostunbrado \textbf{ mas que era fijo de dios } Mas aquella uirtud por la qual alguno es dich̃o bueno & ø \\\hline
2.1.6 & y la terçera comunidat \textbf{ que es de padre e de fijo . } Ca ueemos en las cosas naturales & oportet ibi dare communitatem tertiam , \textbf{ scilicet patris et filii . } Videmus enim in naturalibus rebus \\\hline
2.1.6 & por la qual cosa \textbf{ commo la comunidat del padre al fijo tome nasçençia e comienço de aquello que el padre e la madre } engendran su semeiança esta tal comunidat non es dicha de razon dela primera casa & nisi sit iam perfectus . \textbf{ Quare cum communitas patris ad filium sumat originem | ex eo quod parentes sibi simile produxerunt : } huiusmodi communitas non dicitur \\\hline
2.1.6 & que ala casa acabada parte nesçe la terçera cemuidat \textbf{ que es del padre e del fijo . } Mas que ala perfectiuo dela & quod ad domum perfectam requiritur communitas tertia , \textbf{ videlicet patris et filii . } Quod autem ad perfectionem domus requiratur haec tertia communitas , \\\hline
2.1.6 & que es del padͤ \textbf{ e del fijo pertenescan a conplimiento dela casa podemos lo prouar . } ¶Lo primero de parte dela generaçion & quae est patris et filii , \textbf{ primo possumus probare } ex parte generationis et fructificationis naturalis . \\\hline
2.1.6 & ay ayuntamiento de omne e de muger \textbf{ e non sea y generaçion de fijo o de fija } siguese que es nengua de parte del uaron & coniunctio maris et foeminae ; \textbf{ et tamen non est procreatio prolis , } vel mas est imperfectum actiuum , \\\hline
2.1.6 & por suçession e generaçion de los fiios el padre \textbf{ engendrando el fuo el fijo otro fijo } e assi biue por sienpre & sed quodammodo perpetuatur humana vita \textbf{ per successionem filiorum : } domus ubi est carentia prolis , \\\hline
2.1.6 & Mas en la comunidat del padre \textbf{ e del fijo el padre deua sienpre mandar } e el fij̉o ser obediente . & in communitate vero patris et filii , \textbf{ pater debet esse imperans , } et filius obtemperans ; \\\hline
2.1.6 & ¶ \textbf{ La quarta el fijo . } ¶ & ibi vir , \textbf{ secunda uxor , tertia pater , quarta filius , } quinta dominus , sexta seruus . \\\hline
2.1.11 & Et cosa desconueinente \textbf{ si e que la madre fuesse subiecta al fijo . } ¶ Et pues que assi es non conuiene alas fijas de casar con su padre nin alos fijos con su madre & inconueniens esset \textbf{ sic matrem filio esse subiectam . } Non licet ergo filiis contrahere cum parentibus \\\hline
2.1.14 & lons quales el vno es paternal \textbf{ que es del padre al fijo . } Et el otro mater moianl & secundum Philosophum in Polit’ \textbf{ assimilantur } duo regimina domus , paternale , et coniugale . \\\hline
2.1.14 & pues que assi es el aruernamiento matermonial soes \textbf{ assi natil commo el del padre al fijo . } Ca en ninguna manera los fijos non escogen assi su padre ¶ & Coniugale ergo regimen non est sic naturale , \textbf{ ut paternum : } quia filii nullo modo eligunt sibi patrem . \\\hline
2.1.19 & que van e demuestran desonestad . \textbf{ Ca non abasta que el fijo ageno non he de la h̃edat } de aquel que non es su padre . & quae videntur inhonestatem protendere : \textbf{ non enim sufficit } ut alius filius non succedat in haereditatem , \\\hline
2.1.19 & Mas conuiene que el \textbf{ padresea çierto de su fijo . } Et pues que assi es por que las señales desonestas & sed requiritur \textbf{ ut pater sit certus de sua prole . } Cum ergo signa inhonesta , \\\hline
2.2.1 & e la conpannia del uaron e dela muger e del sennor e del sieruo part enescan ala casa primera \textbf{ Mas la comunindat del padre e del fijo parte nesçan ala casa ya acabada en su ser } por que la casa primera es ante que la & ad domum primam : \textbf{ communitas vero patris , | et filii pertineat } ad domum iam inesse perfectam : \\\hline
2.2.1 & por la qual cosa commo entre el padre \textbf{ e el fijo sea amor natraal } assi commo se praeua en el viij delas . ethicas . & quilibet enim solicitatur circa dilectum : \textbf{ quare cum inter patrem et filium sit amor naturalis , } ut probatur 8 Ethicorum , \\\hline
2.2.3 & Por la qual cosa si el gonernamiento del padre desto tora a comienco \textbf{ por que el fijo naturalmente es vria semerança que desçende del cadre . } Canmo segunr nacsta & ex hoc sumit originem , \textbf{ quia filius naturaliter est | quaedam similitudo procedens a patre : } cum secundum naturam ad huiusmodi similia fit dilectio , \\\hline
2.2.4 & assi como vn aꝑte de los padres . \textbf{ Ca el fijo es vna part entaiada del padre . } Mas la parte mas es ayuntada al su todo & quasi quaedam pars parentum : \textbf{ Nam filii est quaedam pars | a parentibus abscisa . } Pars autem magis unitur toti , \\\hline
2.2.5 & que es vn dios poderoso todo criador de todas las cosas \textbf{ e que es padre e fijo e spunsanto } e que adam primero padre peco & quod unus est Deus omnipotens creator omnium , \textbf{ qui est pater et filius et spiritus sanctus . } Quod Adam primo parente nostro peccante , \\\hline
2.2.5 & e que el humanal linage fue ensuziado por el pecadodt \textbf{ Et por ende el fijo de dios } por que nos redimiesse tomo carne en la bien auentraada santamͣ & et humano genere per peccatum eius infecto , \textbf{ Dei filius , } ut nos redimeret , \\\hline
2.2.5 & por que nos redimiesse tomo carne en la bien auentraada santamͣ \textbf{ e nasçio della . Et que esse mismo fijo de dios padesçio } e fue muerto e soterrado & assumpsit carnem in beata Virgine , \textbf{ et est natus ex ipsa . | Quod ipse Dei filius } propter peccata nostra fuit passus , mortuus , et sepultus . \\\hline
3.1.7 & mas cuydarian de cada vno moço \textbf{ que era su fijo propreo } por que paresçia a socrates e a platon & reputarent \textbf{ quemlibet puerorum esse filium proprium . } Videbatur enim Socrati et Platoni totam dissensionem ciuium consurgere \\\hline
3.1.9 & e cada vno de los çibdadanos apropriaria \textbf{ assi por fijo a aquel que viesse } que lo semeiaua . & quilibet ciuis appropriaret sibi in filium , \textbf{ quem videret sibi esse similem . } Unde et Philosophus narrat 2 Politicor’ \\\hline
3.1.10 & que mucho plasa muchos non es cosa ligera \textbf{ por la qual cosa si cada vno de los çibdadanos cuydasse que cada vno de los moços era su fijo propreo } por que el amor del se partia en tanta muchedunbre de fijos & non est facile . \textbf{ Quare si quilibet ciuis crederet } quemlibet puerorum esse proprium filium , quia partiretur eius amor in tantam multitudinem , \\\hline
3.2.5 & por suçession \textbf{ e si algun fallesçemiento ouiere en el fijo del Rey } a quien parte nesçe & filium succedere in regimine patris : \textbf{ et si aliquis defectus esset in filio regio , } ad quem deberet regia cura peruenire , \\\hline
3.2.36 & e el iusto non perdona a ninguno por iustiçia . \textbf{ Ca nin por fijo nin por amigo } nin por otro ninguon non deue dexar de fazer iustiçia & quia nec pro patre , \textbf{ nec pro filio , | nec pro amico , } nec pro aliquo alio \\\hline

\end{tabular}
