\begin{tabular}{|p{1cm}|p{6.5cm}|p{6.5cm}|}

\hline
1.1.1 & De saber fazer ¶ \textbf{ E si por qual manera Deuen mandar a los sus Subditos } conujene esta doctrina & quomodo debeant se habere , \textbf{ et qualiter debeant | suis subditis imperare , } oportet doctrinam hanc extendere usque ad populum , \\\hline
1.1.2 & continuadamente aprouecha en La sçiençia \textbf{ fasta que segun la su manera } e quanto el puede aya de venir asçia conplida¶ & et sic dando se speculationi , \textbf{ continue in scientiam perficitur , donec secundum modum sibi possibilem habeat perfectam notitiam . } Quod ergo secundum naturalem ordinem dictum est de speculabilibus , \\\hline
1.1.3 & por que dios en la suenençia \textbf{ e en la su sustançia es vno caes muy conplidamente vno } e muy conplidamente bueno . & Est enim Deus ipse , \textbf{ essentia unitatis et bonitatis , | quia est maxime unus , } et maxime bonus . \\\hline
1.1.3 & e por que pueda enduzir \textbf{ e traher los sus subienctos a uirtudes . } puestos ya vnos preanbulos neçesarios al proposito & Quot sunt modi viuendi , \textbf{ et quomodo in } Praemissis quibusdam praeambulis necessariis ad propositum , \\\hline
1.1.4 & faziendo cosas granadas \textbf{ e honrra das et gouernando derechamente los sus . } subditos ¶ Et por la vida & ponimus in pura speculatione , \textbf{ ut Philosophi sentiebant . | Unde si in speculatione diuinorum } vita contemplatiua consistit , \\\hline
1.1.5 & commo demanda la fin que ha de seguir . \textbf{ sy la su fin non sopiere . } Conviene a todo omne & ut requirit consecutio finis , \textbf{ ignorato ipso fine , } expedit volenti \\\hline
1.1.5 & que quiera alcançar e auer su fin \textbf{ e la su bien andança de auer } ante algun conosçimjento dela su fin e dela su bien andança . & consequi suum finem , \textbf{ vel suam felicitatem , } habere praecognitionem ipsius finis . \\\hline
1.1.5 & que conujene al rrey \textbf{ en toda manera de conosçer la su fin ¶ } La primera rrazon es en quanto el rrey & duplici via venari possumus , \textbf{ quod expedit regi suum finem cognoscere . } Prima est , \\\hline
1.1.5 & La primera rrazon es en quanto el rrey \textbf{ por las sus obras ayuda asy mesmo } por que aya e alçen la su fin & Prima est , \textbf{ inquantum per sua opera cooperatur , } ut sit finis consecutiuus . \\\hline
1.1.5 & por las sus obras ayuda asy mesmo \textbf{ por que aya e alçen la su fin } ¶La segunda Razon es en quanto el mjsmo es guiador de los otros ¶ & inquantum per sua opera cooperatur , \textbf{ ut sit finis consecutiuus . } Secunda vero , inquantum est aliorum directiuus . \\\hline
1.1.5 & Ca para que cada vno \textbf{ por las sus obras } alcançe la su fin & Nam ad hoc quod aliquis \textbf{ per suas operationes } finem consequatur , \\\hline
1.1.5 & por las sus obras \textbf{ alcançe la su fin } tres cosas le son le mester ¶ & per suas operationes \textbf{ finem consequatur , } tria requiruntur . \\\hline
1.1.5 & que nos bien obremos conuiene nos de estableçer alguna buean fin e conuenible \textbf{ que todas las nuestras obras toman nasçençia dela fin } Ca non podria ser ninguna buena obra & praestituere finem bonum et debitum : \textbf{ quia ex fine opera nostra speciem summunt : } non enim esse posset \\\hline
1.1.5 & si fieren en las señal \textbf{ esto es por ventura asy los que ante non conosçen la su fin } si bien fazen & si signum percutiant , \textbf{ hoc est casu et fortuitu : | sic non praecognoscentes finem , } si bene agant , \\\hline
1.1.5 & hun omne conuiene ante \textbf{ connosçerlo su fin e la su bien andança . } por que pueda obrar bien e de uoluntad e delectosamente . & Cuilibet ergo homini , \textbf{ ut agat bene , | ex electione , } et delectabiliter , \\\hline
1.1.5 & pues que asy co asaz paresçe \textbf{ que muy mas conuiene al Reio al prinçipe conosçer la su fin } e la su bien andança & Patet ergo , \textbf{ quod maxime decet regiam maiestatem } cognoscere suam felicitatem , \\\hline
1.1.5 & que muy mas conuiene al Reio al prinçipe conosçer la su fin \textbf{ e la su bien andança } que a otro ninguno . & quod maxime decet regiam maiestatem \textbf{ cognoscere suam felicitatem , } ut opera communia , \\\hline
1.1.5 & Et asi se sigue \textbf{ que mas conuiene al Rei o al prinçipe de conosçer la su bien andança e la su } finque non al pueblo . & sagittatorem signum percipere quam sagitta , \textbf{ eo quod sit sagittae director : sic magis expedit regiam maiestatem felicitatem , } et finem cognoscere quam populum , \\\hline
1.1.6 & de poner la feliçidat suia \textbf{ e la su bien andança } enlas delectaçiones sensibles e dela carne ¶ & quod non decet aliquem hominem \textbf{ suam felicitatem ponere } in delectationibus sensibilibus . \\\hline
1.1.7 & Mas lanr̃a bien andança \textbf{ nin el nuestro bien } segunt si mesmo & nisi ex ordinatione Hominum . \textbf{ Vera enim felicitas nostra } est bonum nostrum secundum se , \\\hline
1.1.7 & Et por ende non es de poner la bien andança enellas . \textbf{ Otrosy que nos non auemos de poner la nuestra feliçidat } e lanr̃a bien andança en las riquezas natraales & in eis non est ponenda felicitas . \textbf{ Quod autem in naturalibus diuitiis , } cuiusmodi sunt cibus , et potus , \\\hline
1.1.7 & Et mucho seria de denostar \textbf{ si la su bien andança pusiese en estas riquezas corporales } ¶ & detestabile est ei \textbf{ suam felicitatem | in talibus ponere . } Secundo detestabile est Regi , \\\hline
1.1.7 & Mas si alguas vezes para mientes \textbf{ por el su bien propio } esto es por que del bien comun se le siguͤ bien ala persona & et bonum commune : \textbf{ si autem intendit bonum proprium , } hoc est ex consequenti . \\\hline
1.1.7 & a aquel que pone las un feliçidat \textbf{ e la su bien andança en las riquezas } e en los aueres & Cum ergo finis maxime diligatur , \textbf{ ponens suam felicitatem in numismate , } principaliter intendit reseruare sibi , \\\hline
1.1.7 & ¶Lo terçero se declara \textbf{ asi que poniendo el prinçipe la su feliçidat } e la su bien andança en las riquezas corporales & sed Tyrannus , \textbf{ cum non intendat principaliter bonum publicum , sed priuatum . } Tertio hoc posito sequitur \\\hline
1.1.7 & asi que poniendo el prinçipe la su feliçidat \textbf{ e la su bien andança en las riquezas corporales } dende se sigue & sed Tyrannus , \textbf{ cum non intendat principaliter bonum publicum , sed priuatum . } Tertio hoc posito sequitur \\\hline
1.1.7 & por ninguna manera \textbf{ que non pueda querer seguir la su fin } ante se trabaia dela alcançar quanto puede ¶ & omni via qua potest , \textbf{ velle consequi suum finem . } Est igitur Rex Tyrannus , \\\hline
1.1.7 & e sea tyrano ¶ Bien asi es cosa contra razon \textbf{ de poner el Rey la su feliçidat } e la su bien andança en las riquezas corporales . & esse Tyrannum , \textbf{ et depraedatorem detestabile } quoque est suam felicitatem \\\hline
1.1.7 & de poner el Rey la su feliçidat \textbf{ e la su bien andança en las riquezas corporales . } ora uentra a muchos biuen uida politica & et depraedatorem detestabile \textbf{ quoque est suam felicitatem | in diuitiis ponere . } Forte multi viuentes vita politica credunt \\\hline
1.1.8 & e non en aquel a quien se inclina \textbf{ Ca propiamente el accidente es en el su sƀiecto } e non en otro ninguno & non in eo cui inclinatio exhibetur . \textbf{ Accidens enim proprie est in suo subiecto , } non autem in obiecto : \\\hline
1.1.8 & que la real magestad \textbf{ ponga la su bien andança en las honrras . } Et ahun esto podemos prouar por tres razones . & Maxime tamen hoc est indecens regiae maiestati : \textbf{ quod etiam triplici via venari potest . } Si enim Rex suam felicitatem in honoribus ponat , \\\hline
1.1.8 & Ca el prinçipe \textbf{ si pusiere la su bien andança en honrras } conmoabaste acanda vno & erit praesumptuosus , et erit iniustus , et inaequale . \textbf{ Nam si Princeps suam felicitatem in honoribus ponat , } cum sufficiat ad hoc \\\hline
1.1.8 & e pornia los pueblos en peligro¶ \textbf{ Ca commo cada vn omne mucho ame la su fin } en que pone la su bien andança & quia ex hoc efficietur periclitator Populi , et praesumptuosus : \textbf{ nam cum finis maxime diligatur , } si Princeps suam felicitatem in honoribus ponat , \\\hline
1.1.8 & Ca commo cada vn omne mucho ame la su fin \textbf{ en que pone la su bien andança } si el prinçipe pusiere la su bien andança enlas honrras & nam cum finis maxime diligatur , \textbf{ si Princeps suam felicitatem in honoribus ponat , } ut possit honorem consequi , \\\hline
1.1.8 & en que pone la su bien andança \textbf{ si el prinçipe pusiere la su bien andança enlas honrras } por que pueda delo que feziere honrra alcançar & nam cum finis maxime diligatur , \textbf{ si Princeps suam felicitatem in honoribus ponat , } ut possit honorem consequi , \\\hline
1.1.8 & e soƀuio commo quier \textbf{ que aquel su fijo ouiese auido uictoria de los sus enemigos ¶ } Pues que assi es el prinçipe & non obstante quod dictus filius \textbf{ victoriam obtinuerat ab hoste . } Ne ergo Princeps se praecipitet , et ne nimis praesumat , \\\hline
1.1.8 & que sea iniusto nin desegual ¶ \textbf{ Mas conuiene le partir los sus bienes } alos sus vassallos & ne sit iniustus et inaequalis : \textbf{ decet enim Principem } sua bona distribuere \\\hline
1.1.8 & que dicho es en este capitulo \textbf{ si el Rei pusiere la su feliçidat } e la su bienandança en las honrras sera malo en si mesmo & quo plus honoris consequi possit . \textbf{ Si ergo Rex suam felicitatem in honoribus ponat , } erit malus in se , \\\hline
1.1.8 & si el Rei pusiere la su feliçidat \textbf{ e la su bienandança en las honrras sera malo en si mesmo } e non fara fuerça de ser bueno mas de paresçer bueno . & Si ergo Rex suam felicitatem in honoribus ponat , \textbf{ erit malus in se , } quia non curabit esse \\\hline
1.1.9 & Et ahun paresçe \textbf{ que los prinçipes deuen poner mayormente la su feliçidat } e la su bien andança en la eglesia & ipsam indelebilem esse . \textbf{ Videtur ergo quod maxime Princeps } in hoc suam felicitatem ponere debeat , \\\hline
1.1.9 & que los prinçipes deuen poner mayormente la su feliçidat \textbf{ e la su bien andança en la eglesia } e en la & Videtur ergo quod maxime Princeps \textbf{ in hoc suam felicitatem ponere debeat , } dicente Philosopho 5 Ethic’ \\\hline
1.1.9 & por que deles auido entre los omes algun loando e claro conosçimiento¶ \textbf{ Como el nuestro connosçimiento non sea aque|p{1cm}|p{6.5cm}|p{6.5cm}|la cosa de que es } nin sea razon dellas & et clara notitia , \textbf{ cum scientia nostra | non sit ipsa res , } nec sit causa rerum , \\\hline
1.1.9 & e faze todas las cosas \textbf{ mas el nuestro conostimiento es fecho delas cosas que dios fizo . } asi commo dize el philosofo & ( ut ad praesens spectat ) \textbf{ in tribus differt a notitia nostra : } nam notitia Dei causat res , \\\hline
1.1.9 & Otrosi el conosçimiento de dios es tal en qua non puede caer enganno . \textbf{ Et el nuestro conosçimiento es tal en que muchas vezes puede caer enganno ¶ } La terçera diferençia es & nam notitia Dei causat res , \textbf{ notitia nostra causatur a rebus , ut vult Commen’ 12 Met’ . } Rursus notitia Dei est infallibilis , \\\hline
1.1.9 & entre el conosçimiento de dios \textbf{ e el nuestro es esta } que al conosçimiento de dios es tan bien dellas cosas de dentro & Rursus notitia Dei est infallibilis , \textbf{ notitia nostra pluries fallitur . } Amplius notitia Dei est \\\hline
1.1.9 & si non fuere bueno en uerdat e de fecho \textbf{ assi commo demanda el su estado ¶ } Et pues que assi es maguera & nisi sit bonus , et beatus , \textbf{ ut exigit status suus . } Licet ergo sic sit de notitia Dei , \\\hline
1.1.9 & Ca segunt que dize boeçio la fama del pueblo romano \textbf{ nin la su eglesia non passo el monte de caucaso ¶ } Et pues que assi es commo fama & nam secundum Boetium fama Romani populi \textbf{ nunquam transiuit Caucasum montem . } Quomodo ergo fama unius hominis \\\hline
1.1.9 & que les fazen honrramas quesiese \textbf{ que la su gente } et los sus vasallos le diesen otros bienes & Quod si tamen Principes non acceptarent hanc affectionem dantium , \textbf{ sed requirerent a gente sibi commissa alia exteriora bona , } ut puta aurum , vel argentum , vel diuitias alias , \\\hline
1.1.9 & que la su gente \textbf{ et los sus vasallos le diesen otros bienes } asi commo oro o plata o otras riquezas & Quod si tamen Principes non acceptarent hanc affectionem dantium , \textbf{ sed requirerent a gente sibi commissa alia exteriora bona , } ut puta aurum , vel argentum , vel diuitias alias , \\\hline
1.1.10 & La quinta razon por que non conuiene al prinçipe \textbf{ poner la su bien andança en el poderio çiuiles } por que este sennorio faze grant danno en las mas cosas . & Quinto hoc non decet ipsum , \textbf{ quia huiusmodi principatus infert } ut plurimum nocumentum . \\\hline
1.1.10 & e la bien andança \textbf{ sea fin de todas las nuestras obras } ¶ Cada vno porna toda su uida & ut plurimum nocumentum . \textbf{ Nam cum felicitas sit finis omnium operatorum , } quilibet totam vitam suam , \\\hline
1.1.10 & si non en vsos de armas e de batallas . \textbf{ Et toda la su vida aya ordenada a fortaleza } e non aiustiçia nin atenperança . & nisi in exercitiis bellicis , \textbf{ et totam vitam suam | ordinauerit } ad fortitudinem , \\\hline
1.1.10 & deno stando alos griegos \textbf{ por que ponian la su bien andança en el poderio çiuil . } Et dize assi que torpe cosa es & ponentes felicitatem \textbf{ in ciuili potentia , } ait , turpe esse , \\\hline
1.1.11 & Et pues que assi es non conuiene al rey \textbf{ nin a ningun omnen poner la su feliçidat } e la su bien andança en tales cosas & Non decet ergo Regem , \textbf{ nec aliquem hominem in talibus } suam felicitatem ponere , \\\hline
1.1.11 & nin a ningun omnen poner la su feliçidat \textbf{ e la su bien andança en tales cosas } por que son corporales & nec aliquem hominem in talibus \textbf{ suam felicitatem ponere , } quae sunt corporalia , \\\hline
1.1.11 & Et ahun en essa misma guła deue auer cuydado el prinçipe de auer buena fama . \textbf{ Ca por esso se enduzen los sus subditos a seer uirtuosos . } Ca por las mas uezes contesçe & esse curae ipsi Principi de debita fama , \textbf{ quia propter hoc inducuntur | subditi ad virtutem . } Nam ( ut probatum est ) \\\hline
1.1.12 & ¶ Et la segunda commo le conuiene \textbf{ de poner er la su bien andança solamente en dios . } Esto pondemos prouar por tres razones ¶ & sciendum quod decet Regem maxime \textbf{ suam felicitatem | ponere in ipso Deo , } quod triplici via venari possumus . \\\hline
1.1.12 & e ha razon e entendimiento \textbf{ de poner la su bien andança en bien muy comun } e muy entelligible & et rationem participat , \textbf{ ponere suam felicitatem | in bono maxime uniuersali , } et maxime intelligibili : \\\hline
1.1.12 & e muy alongado de toda materia¶ \textbf{ La segunda razon por que el rey ha de poner la su bien andaça } en dios solo es esta . & et maxime a materia separatus . \textbf{ Secundo decet Principem | suam felicitatem } ponere in ipso Deo , \\\hline
1.1.12 & e que sea su ofiçial \textbf{ para fazer las sus obras . } Por la qual cosa si los ofiçiales & sit diuinum organum , \textbf{ siue sit minister Dei . } Quare si minister , \\\hline
1.1.12 & e los seruientes del señor \textbf{ deuen poner la su merçed } e el su & Quare si minister , \textbf{ suam mercedem , } et suum praemium debet \\\hline
1.1.12 & e el su \textbf{ gualardon en el su señor } e deuen la esparar del . & et suum praemium debet \textbf{ ponere in suo Domino , } et debet eam expectare ab ipso , \\\hline
1.1.12 & que es ofiçial de dios \textbf{ poner la su bien andança en dios que es prinçipal señor } e del solo deue esperar & qui est Dei minister , \textbf{ suam felicitatem ponere in ipso Deo , } et suum praemium expectare ab ipso . \\\hline
1.1.12 & mientesal bien comun de todos . \textbf{ Et por ende deue poner la su feliçidat } e la su bien andança & intendere commune bonum . \textbf{ In eo ergo debet | suam felicitatem ponere , } quod est maxime , \\\hline
1.1.12 & Et por ende deue poner la su feliçidat \textbf{ e la su bien andança } en aquel que es bien mas comun & quod est maxime , \textbf{ et commune bonum . } Huiusmodi autem est \\\hline
1.1.12 & Et lo otro por que deue te çier mientes al bien comun \textbf{ deue pener la su feliçidat } e la su bien andança en dios & et tum quia intendit bonum commune , \textbf{ debet suam felicitatem ponere in Deo , } cui seruit , \\\hline
1.1.12 & deue pener la su feliçidat \textbf{ e la su bien andança en dios } a quien deue seruir . & et tum quia intendit bonum commune , \textbf{ debet suam felicitatem ponere in Deo , } cui seruit , \\\hline
1.1.12 & e muy unun sal et muy comun \textbf{ ¶ Et pues que el Rey deue poner la su feliçidat } e la su bien andança en dios . & et maxime uniuersale , et commune . \textbf{ Si ergo Rex debet in Deo } ponere suam felicitatem , \\\hline
1.1.12 & ¶ Et pues que el Rey deue poner la su feliçidat \textbf{ e la su bien andança en dios . } Conuiene le dela poner en la obra de aquella uirtud & Si ergo Rex debet in Deo \textbf{ ponere suam felicitatem , } oportet ipsum huiusmodi felicitatem ponere \\\hline
1.1.12 & Et pues que assi es los Reyes \textbf{ et los prinçipes deuen poner la su feliçidat } e la su bien andança & et sancte regant . \textbf{ Regibus ergo , et Principibus ponenda est felicitas } in actu prudentiae , \\\hline
1.1.12 & et los prinçipes deuen poner la su feliçidat \textbf{ e la su bien andança } en las obras dela pradençia & Regibus ergo , et Principibus ponenda est felicitas \textbf{ in actu prudentiae , } non simpliciter , \\\hline
1.1.13 & quant grant es el gualardon de los reyes \textbf{ e quant grande es la su feliçidat } e las un bien andança & Magnum autem esse praemium Regis , \textbf{ et magnam eius esse felicitatem , } si per prudentiam , \\\hline
1.1.13 & que han de auer \textbf{ si gouernare el su pueblo } qual es acomnedado bien e derechamente & si per prudentiam , \textbf{ et legem recte regat populum sibi commissum , } ex quinque venari possumus . \\\hline
1.1.13 & Mas el estado del Rey demanda que sea mas acordable et mas acordable e mas semeiable a dios \textbf{ que ninguno de los sus subditos . } ante por esso mesmo & ut sit Deo conformior , \textbf{ quam eius subditi . } Immo eo ipso quod Rex studet per legem , \\\hline
1.1.13 & si bien gouernar en las gentes que les son acomne dadas \textbf{ por las sus buenas obras } e por el su buen gouierno & Reges ergo si bene regant \textbf{ gentem sibi commissam , } ex operibus eorum consequenter mercedem magnam : \\\hline
1.1.13 & por las sus buenas obras \textbf{ e por el su buen gouierno } alcançara grant merçed e grant gualardon de dios . & gentem sibi commissam , \textbf{ ex operibus eorum consequenter mercedem magnam : } quia pro bono gentis . \\\hline
1.1.13 & la qual materia es la gente e la muchedunbre delons pueblos esta muestra \textbf{ que el su gualardon es muy grande¶ } Et aqui se acaba la primera parte & et multitudo , \textbf{ indicat eius praemium esse magnum . } Primae partis primi libri \\\hline
1.2.1 & Et mostramos en que deuen poner los Reyes \textbf{ e los prinçipes la su feliçidat } e la su bien andança . & in quo agitur de regimine sui , \textbf{ ostendentes in quo Reges et Principes } suam felicitatem debeant ponere , \\\hline
1.2.1 & e los prinçipes la su feliçidat \textbf{ e la su bien andança . } Et que non los conuiene poner la su fin en riquezas & ostendentes in quo Reges et Principes \textbf{ suam felicitatem debeant ponere , } quia non decet \\\hline
1.2.1 & e la su bien andança . \textbf{ Et que non los conuiene poner la su fin en riquezas } nin en poderio çiuil & suam felicitatem debeant ponere , \textbf{ quia non decet | eos suum finem ponere in diuitiis , } nec in ciuili potentia , \\\hline
1.2.1 & para ganar la feliçidat e la bien andança . \textbf{ Mas la su bien andança deuen poner en obras de pradençia e de sabiduria } segund que tales obras son regladas & debent uti tanquam organis ad felicitatem . \textbf{ Suam autem felicitatem ponere debent | in actu prudentiae , } prout talis actus est \\\hline
1.2.1 & Ca estonçe han los Reyes \textbf{ e los prinçipes la su feliçidait } e la su feliçidat bien andança qual deuen auer & imperatus a charitate : \textbf{ nam tunc Reges habent felicitatem suo statui debitam , } et condignam , \\\hline
1.2.1 & e los prinçipes la su feliçidait \textbf{ e la su feliçidat bien andança qual deuen auer } e qual parte nesçe a su estado & nam tunc Reges habent felicitatem suo statui debitam , \textbf{ et condignam , } quando instigante Dei dilectione \\\hline
1.2.1 & segund la sabiduria del gouernamiento \textbf{ gouernan las sus gentes } e los sus pueblos & quando instigante Dei dilectione \textbf{ secundum prudentiam regitiuam , gentem sibi commissam } secundum legem , \\\hline
1.2.1 & gouernan las sus gentes \textbf{ e los sus pueblos } que les son acomendados scanmente e derechureramente & secundum prudentiam regitiuam , gentem sibi commissam \textbf{ secundum legem , } et rationem sancte , \\\hline
1.2.1 & e de los prinçipes es de poner en solo dios . \textbf{ Ca deuen ellos ordenar la su uida } e el su estado a esto & Principaliter ergo Regum felicitas ponenda est in ipso Deo , \textbf{ et ex cognitione et dilectione eius studium suum , } et vitam suam \\\hline
1.2.1 & Ca deuen ellos ordenar la su uida \textbf{ e el su estado a esto } que por çonosçimiento & et ex cognitione et dilectione eius studium suum , \textbf{ et vitam suam } ad hoc ordinare debet , \\\hline
1.2.1 & Ca assi commo el fuego tanto escalienta quato puede escalentar \textbf{ por que es determimado en la su obra } por esso nies de loar & quantum potest calefacere , \textbf{ propter quod , | quia determinatus est in actione sua , } nec laudatur , nec vituperatur , \\\hline
1.2.2 & Pues que assi es por que el fuego non fuese enbargando \textbf{ por los sus contrarios } que non podiesse sobir & ne ergo ignis \textbf{ per quaecunque contraria agentia impediretur , } ne per leuitatem in proprio loco quiesceret , \\\hline
1.2.2 & e ayan de desanparar los logares propios \textbf{ e la su propia folgura } Vien assi las ainalias e las bestias & et deserant propria loca , \textbf{ et propriam quietem . } Sic animalia per concupiscibilem \\\hline
1.2.2 & e arredrar tondas las cosas \textbf{ que enbargan la su delectaçion . } Pues que assi es dos son los . appetitos & per quam resistant , \textbf{ et aggrediantur impedientia delectationem illam . } Duplex est ergo appetitus sensitiuus , \\\hline
1.2.3 & assi commo es la perfectiuo \textbf{ en aquello que acaba o la forma en el su subiecto o en su materia } Mas la iustiçia va a bien de razon & enim Prudentia in ipso intellectu , \textbf{ tanquam perfectio in ipso perfectibili . } Iustitia vero tendit \\\hline
1.2.3 & Et este llamamos aqui uerdadero \textbf{ e la su uirtud es uerdat . } Mas la bien fablança es & et verbis et factis ostendit se talem , \textbf{ qualis est . } Affabilitas vero est , quando quis bene se habet \\\hline
1.2.3 & assi commo conuiene en los trebeios . \textbf{ Et la su uirtud es eutropolia } que ̀ere dezir buena conpanma . & ut se habeat circa ludos \textbf{ prout expedit . } Omnes autem hae tres virtutes , \\\hline
1.2.6 & segunt aquellas maneras falladas e iudgadas . \textbf{ Et pues que assi es en el nuestro entendimiento } deuen ser tres uirtudes ¶ & secundum inuenta et iudicata . \textbf{ In intellectu ergo nostro debent esse tres virtutes . } Una per quam bene inueniamus \\\hline
1.2.7 & Lo segundo deue estudiar el Rey \textbf{ que el su prinçipadgo } e el su sennorio non se torne en tirania & Secundo studere debet , \textbf{ ne suus principatus in tyrannidem conuertatur . } Tertio studere debet , \\\hline
1.2.7 & que el su prinçipadgo \textbf{ e el su sennorio non se torne en tirania } que es señorio malo e desigual & Secundo studere debet , \textbf{ ne suus principatus in tyrannidem conuertatur . } Tertio studere debet , \\\hline
1.2.7 & e de dignidat \textbf{ por que el ofiçio del Rey es que gouierne e guie la su gente . } la qual cosa muestra el nonbre del rey . & Nam Rex est nomen officii , et dignitatis . \textbf{ Est enim Regis officium , | ut suam gentem regat , } et dirigat in debitum finem . \\\hline
1.2.7 & Et el que non ha este oio non puede conplidamente ueer el bien \textbf{ nin la su fin conuenible } ala qual es de guiar el pueblo . & non sufficienter videre potest ipsum bonum , \textbf{ nec ipsum debitum finem , } in quem est populus dirigendus . \\\hline
1.2.8 & or que nunca conplidamente se pueda auer el todo \textbf{ si non se ouieren todas las sus partes } si alguno ouiere aser sabio conplida mente . & Quoniam nunquam perfecte habetur aliquod totum , \textbf{ nisi habeantur partes eius : } si debeat aliquis esse perfecte prudens , \\\hline
1.2.8 & Conuiene al Rey de ser entendido e razonable ¶ \textbf{ por razon de la su propia persona } que ha de guiar los otros . & et rationabilem : \textbf{ ratione propriae personae } quae alios est dirigens , oportet quod sit solers , et docilis : \\\hline
1.2.8 & e cauto ete sogedor de bien ¶ \textbf{ Ca si el Rey ha a guiar la su gente } e la su conpanna a alguons bienes . & congruit quod sit expertus et cautus . \textbf{ Si enim Rex debet } gentem aliquam ad bonum dirigere , \\\hline
1.2.8 & Ca si el Rey ha a guiar la su gente \textbf{ e la su conpanna a alguons bienes . } Conuiene que aya memoria de las cosas passadas . & Si enim Rex debet \textbf{ gentem aliquam ad bonum dirigere , } oportet quod habeat memoriam praeteritorum , \\\hline
1.2.8 & e razon o conuiene le que sea entendido e razonable . \textbf{ Ca la manera por que el Rey guia el su pueblo } Conuiene que sea manera de omne . & quod sit intelligens et rationale . \textbf{ Modus enim , | quo Rex suum populum dirigit , } oportet quod sit humanus , \\\hline
1.2.9 & en lo que ha de venir . \textbf{ Ca sienpre deue el Rey conformar e ordenar el su gouernamiento } segunt el gouernamiento del tp̃o passado & quid agendum sit in futurum . \textbf{ Nam semper debet suum regimen conformare regimini retroacto , } sub quo regnum tutius , \\\hline
1.2.9 & segunt el gouernamiento del tp̃o passado \textbf{ en el qual el su regno meior } e mas en paz fue gouernado . & Nam semper debet suum regimen conformare regimini retroacto , \textbf{ sub quo regnum tutius , } et melius regebatur . \\\hline
1.2.11 & que es acabada uirtud . \textbf{ Et el su contrario es coplida maliçia se puede prouar } que sin lan iustiçia legal & quae est perfecta virtus , \textbf{ cuius oppositum est perfecta malitia , | probari potest } quod absque legali Iustitia non valent regna subsistere . \\\hline
1.2.11 & enla qual abonda el otro . \textbf{ Et por ende por que cada vno pudiesse proueer a la su mengua } sue fallada la iustiçia mudadora & in quo ille abundat . \textbf{ Ideo ut quilibet suae indigentiae prouideret , } inuenta fuit commutatiua Iustitia . \\\hline
1.2.11 & pueᷤ que assi es la iustiçia mudadora es \textbf{ en quanto el vno ordena los sus bienes aprouecho del otro . Et el otro los sus bienes } que tiene aprouecho del otro & Est igitur commutatiua Iustitia , \textbf{ prout unus ordinat bona sua | in utilitatem alterius , et econuerso . } Hunc autem modo reperimus \\\hline
1.2.11 & o de vn regno han ordenamiento entre si mismos \textbf{ e se acorren a las sus menguas los vnos alos otros mudando } e dando las vnas cosas por las otras . & et sibi inuicem \textbf{ secundum quandam commutationem suis indigentiis satisfaciunt , } est in eis commutatiua Iustitia , \\\hline
1.2.11 & afincadamente deue el rey estudiar \textbf{ por que en los sus regnos sea guardada la iustiçia } non solamente aquellos que nasçieron en el regno & summo opere Rex studere debet , \textbf{ ut in suo Regno , | seruetur Iustitia , } non solum iis , \\\hline
1.2.12 & que es muy fermosa e muy clara \textbf{ e por la su fermosura } e por la su claridat es llamada renꝮ & quae est valde pulchra , et clara : \textbf{ et propter sui pulchritudinem , } et venustatem communi nomine \\\hline
1.2.12 & e por la su fermosura \textbf{ e por la su claridat es llamada renꝮ } por nonbre comunal & et propter sui pulchritudinem , \textbf{ et venustatem communi nomine } appellatur Venus . \\\hline
1.2.12 & Et quando la su obra \textbf{ e la su calentura se estiende alos otros . Et esso mismo } estonçe es dicho el ome & quando potest alia calefacere , \textbf{ et quando actio sua ad alia se extendit . } Et tunc est aliquis perfecte sciens , \\\hline
1.2.12 & conplidamente bueon \textbf{ quando la su bondat se estiende alos otros } Et por ende la bondat acabada de los omes & tunc est aliquis perfecte bonus , \textbf{ quando bonitas sua usque ad alios se extendit . } Inde est ergo , \\\hline
1.2.12 & si non assi mismo non paresçe bien \textbf{ quales nin es conostida la su bondat acabadamente ¶ } Mas quando es puesto en algun prinçipado o en algun sennorio & non plene apparet qualis sit , \textbf{ nec perfecte cognoscitur bonitas eius . } Sed quando constituitur \\\hline
1.2.12 & Mas quando es puesto en algun prinçipado o en algun sennorio \textbf{ por que la su bondat se ha de estender a otros } estonçe meior paresçe quales si es bueno o malo & in principatu aliquorum , \textbf{ quia oportet , | quod bonitas sua ad alios se extendat , } tunc melius apparet qualis sit , \\\hline
1.2.12 & estonçe meior paresçe quales si es bueno o malo \textbf{ por que las sus obras le estienden a los otros } ¶ & tunc melius apparet qualis sit , \textbf{ eo quod opera sua ad exteriora se extendant . } Si ergo nobis exteriora magis nota sunt , \\\hline
1.2.12 & e en mayor dignidat \textbf{ por que las sus obras se estienden amas } estonçe paresçe meior cada vno quales . & quanto aliquis in maiori principatu constituitur ; \textbf{ quia opera sua ad plura se extendunt , } magis apparet qualis sit . \\\hline
1.2.12 & que non solamente es bueno en si \textbf{ Mas ahun la su bendat se estiende alos otros omes . } Assi peor es el omne & qui non solum est bonus in se , \textbf{ sed etiam bonitas sua se extendit ad alios : } sic peior est , \\\hline
1.2.12 & que non solamente es malo en ssi \textbf{ mas ahun la su maliçia se estiende alos otros omes } Et quanto amas se estiende la su maliçia & qui non solum malus est in se , \textbf{ sed etiam malitia sua se extendit ad alios : } et quantum ad plures se extendit malitia eius , \\\hline
1.2.12 & mas ahun la su maliçia se estiende alos otros omes \textbf{ Et quanto amas se estiende la su maliçia } tanto peor es el . & sed etiam malitia sua se extendit ad alios : \textbf{ et quantum ad plures se extendit malitia eius , } tanto peior existit . \\\hline
1.2.12 & para guardar la iustiçia \textbf{ e para escusar la mi ustiçia e el mal quanto por la mengua dela su iustiçia se puede seguir mayor mal } Et puede venir mayor deño a muchos . & ut seruent Iustitiam , \textbf{ et iniustitiam vitent : | quanto ex eorum Iustitia potest } consequi maius malum , \\\hline
1.2.12 & et auer la iustiçia \textbf{ e de commo deuen guardar la iustiçia en los sus regnos . } Mas esto en el tercero libro aura logar & et Principes possunt Iustitiam acquirere : \textbf{ et quomodo debeant Iustitiam obseruare . } Sed in tertio libro , \\\hline
1.2.14 & acometen alguas torpedades \textbf{ las quales non quarrian acometer nin tentar entre los sus çibdadanos en ningunan manera } e entre los sus conosçientes ¶ & committere aliqua turpia , \textbf{ quae inter ciues et notos nullatenus attentarent . } Secunda Fortitudo dicitur seruilis , \\\hline
1.2.14 & las quales non quarrian acometer nin tentar entre los sus çibdadanos en ningunan manera \textbf{ e entre los sus conosçientes ¶ } La segunda fortaleza es dicha seruil & committere aliqua turpia , \textbf{ quae inter ciues et notos nullatenus attentarent . } Secunda Fortitudo dicitur seruilis , \\\hline
1.2.14 & que es mayor la batalla \textbf{ que la su praeua } que han en las armas tornan se a & nam cum adeo inualescit bellum , \textbf{ quod excedat eorum experientiam , } in fugam conuertuntur . \\\hline
1.2.14 & Enpero ellos deuen ser fuertes de fortaleza uirtuosa \textbf{ por que non pongan la su gente } e el su pueblo a periglos de batallas & ipsi tamen debent esse fortes fortitudine virtuosa , \textbf{ ut non exponant suam gentem periculis bellicis , } nisi habeant iusta bella , \\\hline
1.2.14 & por que non pongan la su gente \textbf{ e el su pueblo a periglos de batallas } si non quando ouieren razon derecha para auer batalla . & ipsi tamen debent esse fortes fortitudine virtuosa , \textbf{ ut non exponant suam gentem periculis bellicis , } nisi habeant iusta bella , \\\hline
1.2.15 & porque mas se delectase por el gusto . \textbf{ Mas rogo que la su garganta fuese mas luenga que garganta de grulla } por que comiendo e beuiendose delectase mas prolongadamente por el tannimiento ¶ pues que assi es paresçe ya & ut magis delectaretur per gustum , \textbf{ sed ut guttur eius esset longius gutture gruis , } ut comedendo , \\\hline
1.2.16 & veyendo \textbf{ que el su Rey era todo mugeril } e toda su & Dux autem ille assuetus rebus bellicis , \textbf{ videns Regem suum esse totum muliebrem et bestialem , } statim ipsum habuit in contemptum : \\\hline
1.2.17 & que catan alos bienes honestos . \textbf{ por que el nuestro conosçimiento conmienca en los sesos } e en las cosas que sentimos . & et postea de respicientibus bona honesta . \textbf{ Incipit enim nostra cognitio a sensu . } Cum ergo bona utilia \\\hline
1.2.17 & por que lo es . es dicha mastal . \textbf{ Et por ende si alguno es liberal e franco en guardando las sus rentas propreas } e tomando onde deue esto & et illud magis . \textbf{ Si enim liberalis conseruans proprios redditus , } et accipiens unde debet , \\\hline
1.2.17 & que mas prinçipalmente es la franqueza en espender e en fazer bien alos otros . \textbf{ Et despues desto es en guardar las sus rentas propreas } e non vsurpar nin tomar las agenas . & et in benefaciendo aliis ; \textbf{ ex consequenti autem est | circa custodire proprios redditus , } et circa non usurpare alienos . \\\hline
1.2.17 & que en guardar lo suyo mismo \textbf{ o las sus rentas propias } por que la franqueza & et circa beneficiare alios , \textbf{ quam circa proprios redditus custodire . } Liberalitas enim \\\hline
1.2.17 & ca la uirtud mas prinçipalmente es cerca lo mas guaue . \textbf{ Et mas guaue es dar los sus bienes e fazer bien alos otros } que guardar las sus rentas propias & circa difficilius . \textbf{ Difficilius autem est aliis dona tribuere , } quam proprios redditus custodire , \\\hline
1.2.17 & Et mas guaue es dar los sus bienes e fazer bien alos otros \textbf{ que guardar las sus rentas propias } o que non tomar los bienes agenos . & Difficilius autem est aliis dona tribuere , \textbf{ quam proprios redditus custodire , } vel quam aliena non surripere . \\\hline
1.2.17 & Ca cada hun omne es naturalmente inclinado a amar asi mismo \textbf{ e aguardar los sus biens propos } Mas dar los sus biens propios ha alguna guaueza por si . & ut se diligat , \textbf{ et ut sua bona custodiat . } Dare autem propria bona , \\\hline
1.2.17 & e aguardar los sus biens propos \textbf{ Mas dar los sus biens propios ha alguna guaueza por si . } Ca los bienes propios son & et ut sua bona custodiat . \textbf{ Dare autem propria bona , | secundum se difficultatem habet : } quia propria bona sunt aliquid ad nos pertinens , \\\hline
1.2.17 & e despues desto ha de ser \textbf{ en guardar las sus rentas } e en non tomar las agenas . & ex consequenti autem est \textbf{ in custodiendo redditus proprios , } et non in usurpando alienos : \\\hline
1.2.18 & que las sus donacones \textbf{ e las sus espenssas non pueden sobrepuiar } ala muchedunbre delas sus possesiones . & et tanta recipit , \textbf{ quod dationes et expensae multitudinem possessionum superare non possint , } quodammodo prodigus esse non potest . \\\hline
1.2.18 & muchas vezes son menguados \textbf{ por que las sus espenssas son muy mayores que las sus rentas . } Et por ende veyendo la mengua & ut plurimum egent , \textbf{ quia expensae superabundant redditibus . } Experiendo ergo indigentiam , \\\hline
1.2.18 & e deue descender todo el gouernamiento del regno \textbf{ non le conuiene de enfermar en las sus costunbres } de tal enfermedat & a quo totum regnum dirigi debet , \textbf{ indecens est aegrotare } secundum mores morbo incurabili , \\\hline
1.2.18 & tanto conuiene al Rey de ser mas largo \textbf{ quanto la su magnifiçençia } e la su largueza amas se ha de estender & tanto decet Regem largiorem esse , \textbf{ quanto influentia eius ad plures extendenda est , } quam influentia aliorum . \\\hline
1.2.18 & quanto la su magnifiçençia \textbf{ e la su largueza amas se ha de estender } que la largueza de los otros ¶ & tanto decet Regem largiorem esse , \textbf{ quanto influentia eius ad plures extendenda est , } quam influentia aliorum . \\\hline
1.2.18 & ¶ Lo terçero esta uirtud es dicha franqueza \textbf{ por que por ellas los omes parten los sus bienes } por la qual participaçion se departen estri̊madamente de los otros . & Tertio huiusmodi virtus dicitur communicabilitas : \textbf{ quia per eam homines communicant sua bona , } per quam communicationem ab aliis potissime diliguntur : \\\hline
1.2.18 & e alos prinçipes de ser amados de todos \textbf{ los que son en el su regno } mucho les conuiene de ser liberales e francos & ab iis \textbf{ qui sunt in Regno , } maxime decet eos liberales esse . \\\hline
1.2.18 & por que la grandeza delas espenssas \textbf{ apenas puede sobrepuiar ala muchedunbre de las sus rentas . Por ende si contesçe algunas uegadas al liberal de dar } mas de quanto deue legunt & quia magnitudo expensarum vix potest \textbf{ excedere multitudinem reddituum . | Imo si contingat liberalem } dare plus quam deceat , \\\hline
1.2.19 & çerca las cosas dauinołs \textbf{ Et si cunplieren las sus riquezas } deue fazer grandes eglesias e sac̀fiçios honrrados & Nam principaliter et primo , \textbf{ homo debet esse magnificus circa diuina , } constituendo ( si facultates tribuant ) templa magnifica , sacrificia honorabilia , praeparationes dignas . \\\hline
1.2.19 & Lo quarto el magnifico se deue auer \textbf{ conueinblemente cerca la su perssona propia } por que deue cada vno granadamenᷤte se auer & Quarto debet se decenter \textbf{ habere magnificus | circa personam propriam : } debet enim quis magnifice se habere \\\hline
1.2.19 & e mas deue entender \textbf{ en qual manera deue fazer muuy marauillosas e muy durables las sus casas e las sus moradas } que en qual manera las fara sufisticas e aparesçientes e muy pintadas & et magis debet intendere quomodo facere \textbf{ debeat admirabiles , | et diuturnas domus , } quam quomodo faciat eas sophysticas et apparentes . \\\hline
1.2.19 & En essa misma manera conuiene al magnifico de fazer muy \textbf{ honrradamente las sus bodas e las sus cauallerias } e aquellas cosas & quam quomodo faciat eas sophysticas et apparentes . \textbf{ Sic etiam decet magnificum , nuptias , et militias , } et talia quae raro occurrunt , \\\hline
1.2.19 & fechas cerca las personas dignas \textbf{ e çerca la su persona miłma ¶ } Mostrado que cosa es la magnificençia & circa personas dignas , \textbf{ et circa seipsum . Ostenso quid est magnificentia , } et circa quae habet esse : \\\hline
1.2.20 & por tanto que el parufico cuyda \textbf{ que el su auer } e los sus desque son & quia est ibi quasi quaedam diuisio continui , \textbf{ eo quod paruificus reputat suam pecuniam } quasi sibi incorporatam et continuatam , \\\hline
1.2.21 & Por ende mucho parte nesçe al magnifico \textbf{ en las sus muy grandes obras } e en las sus parti & ideo maxime spectat ad magnificum \textbf{ in suis magnificis operibus , } et distributionibus intendere finaliter bonum , \\\hline
1.2.21 & en las sus muy grandes obras \textbf{ e en las sus parti } connsenteder finalmente el bien & in suis magnificis operibus , \textbf{ et distributionibus intendere finaliter bonum , } et non fauorem , et gloriam hominum . \\\hline
1.2.25 & que ha en reuerençia alos otros \textbf{ por que cuydando en los sus desfallesçimientos propios en las cosas conuenibles e honestas . } faze reuerençia alos otros ¶ & Ideo humilis dicitur alios reuereri , \textbf{ quia considerans proprios defectus , | in rebus licitis et honestis alios reueretur . } Secundo differt haec ab illa , \\\hline
1.2.26 & por que se vestian de villes pannos \textbf{ mas que el su estado demandaua creyendo } por esto & vilius induebantur : \textbf{ credentes ex hoc in quendam honorem , } et in quandam excellentiam consurgere . \\\hline
1.2.26 & e esto es lo que fazen los humildosos . \textbf{ Et otrosi que non pongan la su bien andança } en sobrepuiança de honrra lo que fazen los sobuios & quod faciunt humiles : \textbf{ quod tamen suam felicitatem | non ponant } in excellentia et honore , \\\hline
1.2.26 & por la mayor parte va a mayores cosas \textbf{ de quanto puede el su poder . } Et por ende conuiene alos omes de ser humildosos & ut plurimum tendit \textbf{ ad ea quae proprias vires excellunt . } Ideo decet homines esse humiles , \\\hline
1.2.26 & por que cuydando el en su fallescimiento propio \textbf{ o el su poder non vayan a cosas mas altas que deuen . } Ca en las mas cosas los soƀ̃uios & ut considerato proprio defectu \textbf{ vel propria facultate , | non tendant in ardua ultra quam debeant , } ut plurimum enim superbi , \\\hline
1.2.26 & en quanto mas contraria es dessi \textbf{ por non poner a periglo los sus bienes e las sus gentes . } Ca el señor soƀuio en las mas cosas pone el su pueblo a periglo . ¶ & a se superbiam remouere , \textbf{ quanto peius est communia bona periculis exponere . } Superbus enim Dominus , \\\hline
1.2.26 & por non poner a periglo los sus bienes e las sus gentes . \textbf{ Ca el señor soƀuio en las mas cosas pone el su pueblo a periglo . ¶ } es pues que determinamos delas uirtudes & quanto peius est communia bona periculis exponere . \textbf{ Superbus enim Dominus , | ut plurimum periclitator efficitur populorum . } Postquam determinauimus \\\hline
1.2.27 & o por otras muy destenpradas passiones del alma \textbf{ tristorna el nuestro iuyzio de la razon e del entendemiento . } Et pues que assi es si cosa desconbenible & vel per alias immoderatas passiones , \textbf{ peruertitur nostrum iudicium rationis . } Si igitur inconueniens est \\\hline
1.2.29 & cuydando que valen mas de quantovalen . \textbf{ Et por ende en contando cada vno los sus propreos bienes } deue se sienpre inclinar alo menos . & quam valeant . \textbf{ In narrando ergo propria bona , } semper declinandum est in minus : \\\hline
1.2.29 & Ca muy grand pradençia \textbf{ e grant sabiduria es conosçer assi mismo . omne e saber que los sus bienes propreos } sienpreles son vistos mayores que son ¶ & Nam magnae prudentiae est , \textbf{ cognoscere seipsum , | et sciri quod propria bona } semper aestimantur maiora quam sint . \\\hline
1.2.30 & e el oyr t̃baian en sintiendo las cosas senssibles . \textbf{ Et por ende la natura ordeno el su enno } para lu tolgura dellos . & quia laborant in sentiendo , \textbf{ natura ordinauit somnum propter eorum requiem , } et est necessarius somnus in vita . \\\hline
1.2.31 & e acaba a aquel que la ha \textbf{ e faga la su obra buena . } Por ende commo havien escoger & et perficiat habentem , \textbf{ et opus suum bonum reddat : } cum ad bene eligere , \\\hline
1.2.31 & En essa misma manera avn si alguno fuesse auariento \textbf{ por que pusiesse la su fin en auer riquezas e dineros } commo quier & Sic etiam si esset auarus , \textbf{ quia finem suum poneret | in habendo pecuniam , } licet forte secundum se \\\hline
1.2.32 & Et otra cosa que es peor desta \textbf{ que presta un a los sus fiios en los conbites a sus vezinos que los comiessen . } Ca quando alguno quaria conbidar a & et ( quod peius est ) \textbf{ praestabant sibi filios inconuiuiis . } Cum enim qui alios conuiuare volebat , \\\hline
1.2.32 & Ca quando alguno quaria conbidar a \textbf{ otrossi el su fiio non era en casa tomaua prestado el fijo de otro su vezino } e aprestaual para fazer el conbit & Cum enim qui alios conuiuare volebat , \textbf{ si filius suus domi non erat , | a vicino suo mutuabat filium , } et ipsum parabat in conuiuium , \\\hline
1.2.33 & Ca de tanta bondat deue ser el prinçipe \textbf{ que cada vno de los sus subditos tome del forma e manera de beuir } e conosca can vno su mengua veyendo la uida & Tantae enim bonitatis debet esse Princeps , \textbf{ ut quilibet suus subditus accipiat | inde formam viuendi , } et cognoscat defectum suum , \\\hline
1.2.33 & sin la gera de dios \textbf{ et sin la su ayuda . } Por ende quanto los Reyes & absque Dei gratia , \textbf{ et eius auxilio : } quanto Reges , \\\hline
1.3.3 & or que las passiones fazen departimiento \textbf{ en el nuestro gouernamiento } e en lanr̃a uida & Passiones autem \textbf{ quia diuersificant regnum et vitam nostram , } ideo necessarium est ostendere \\\hline
1.3.3 & diuinales guardador del nuestro bien . \textbf{ Ante si el nuestro bien fuese destroydo } e perdido dios donde el quisiese lo podria refazer & Bonum enim diuinum est conseruatiuum boni nostri : \textbf{ immo et si annihilatum esset bonum nostrum , } Deus unde vellet , \\\hline
1.3.3 & Et mas altamente \textbf{ e mas noblemente es guardado el su bien en dios que en ssi mismo . } Et por que el bien comun es mas diuinal & quia bonum uniuscuiusque principaliter est a Deo , \textbf{ et excellentius reseruatur in Deo , | quam in seipso . } Et quia commune bonum est \\\hline
1.3.3 & Et entendiendo en el bien comun \textbf{ entiende en el su bien proprio } Ca quando el regno fuere & et intendendo bonum commune , \textbf{ intendit bonum proprium : } quia saluato regno , \\\hline
1.3.3 & Mas el tyrano taze todo lo contrario . \textbf{ ca prinçipalmente entiende en el su bien propio mas despues desto } e assi conmo por açidente entiende en el bien comun & Tyrannus autem econtrario , \textbf{ principaliter intendit bonum priuatum : | ex consequenti autem } et quasi per accidens intendit bonum commune , \\\hline
1.3.3 & e el bien comun \textbf{ que el su bien propio } ¶Lo segundo esto mesmo se praeua assi si pensaremos las uirtudes & et Principes bonum diuinum \textbf{ et commune praeponere cuilibet priuato bono . } Secundo hoc idem patet , \\\hline
1.3.5 & por que non acometan ninguna cosa . \textbf{ mayor de quanto demanda la su fuerça } e que non es ꝑen las cosas & sic per humilitatem debent esse moderati , \textbf{ ut non aggrediantur aliquid ultra vires proprias , } et ut non sperent non speranda . \\\hline
1.3.5 & mas que la fuerca suya \textbf{ nin la su uirtud demanda . } Otrossi non deuen esparaquellas cosas & et sperare ultra quam sit sperandum . \textbf{ Prima via sumitur } ex parte officii regis . \\\hline
1.3.5 & e acometer alguna obra \textbf{ mas que la su fuerca demanda paresçe } que esto uiene mas de mengua de sabiduria & Sperare enim ultra quam sit sperandum , \textbf{ et aggredi opus ultra vires suas , } videtur ex imprudentia procedere , \\\hline
1.3.5 & de non acometer ninguna cosa \textbf{ mas que la su fuerça demanda . } Otrossi les conuiene de non esparar alguas cosas & decet Reges et Principes \textbf{ non aggredi aliquid ultra vires , } et non sperare aliqua non speranda . \\\hline
1.3.5 & aquel que acomete alguna cosa alta \textbf{ e mayor que la su fuerça demanda¶ } Et por ende si cosa desconuenible es poner toda la gente & ut plurimum enim exponit se periculo \textbf{ qui aggreditur aliquid ultra vires . } Si ergo inconueniens est totam gentem \\\hline
1.3.5 & por que non acometan cosa mas alta \textbf{ de quanto demanda la su fuerça } e porque non es ꝑen alguna cosa & quid aggrediantur , \textbf{ ne assumant arduum aliquod ultra vires , } et ne sperent aliquid non sperandum . \\\hline
1.3.7 & ante que entiendan conplidamente el mandamiento del corren \textbf{ para fazer e cunplir el su mandado . } Por la qual cosa les & currunt , \textbf{ ut exequantur mandatum ipsius ; } quare contingit eos deficere , \\\hline
1.3.8 & luego mostra una \textbf{ que la su posicion era de reprehender . } Ca segunt el philosofo & Sed hi omnem delectationem condemnantes , \textbf{ statim suam positionem ostendebant reprehensibilem : } quia ( secundum Philosoph’ ) \\\hline
1.3.8 & que quando nos veemos los amigos doler se del nuestro dolor \textbf{ non es menguado el nuestro dolor } por aquellos que se duellen de nos & de dolore nostro , \textbf{ non quia ipsi dolent de dolore nostro } minuitur dolor noster , \\\hline
1.3.8 & e en delectado nos \textbf{ menguase el nuestro dolor } por que toda delectaçion o tuelle del todo la tristeza & et delectando , \textbf{ minuitur dolor noster : } quia omnis delectatio , \\\hline
1.3.9 & et alos prinçipes \textbf{ en quanto las sus obras } dellos son mas dignas & Sed hoc tanto magis decet Reges et Principes , \textbf{ quanto eorum opera sunt digniora , } eo quod respiciant bonum gentis . \\\hline
1.3.11 & dela bien andança de los malos \textbf{ en quanto ellos non deuen partir los sus bienes alos malos . } nin alos que non son dignos . & de prosperitatibus malorum , \textbf{ ne indignis distribuant sua bona . } Sic ergo se habere debent \\\hline
1.4.1 & por su trabaio propreo . \textbf{ Ca cada vno con mayor acuçia guarda las sus riquezas } e el su auer & non acquisiuerunt proprio labore . \textbf{ Nam quilibet cum maiori diligentia retinet facultates suas , } quando propter indigentiam passus est aliqua mala , \\\hline
1.4.1 & Ca cada vno con mayor acuçia guarda las sus riquezas \textbf{ e el su auer } quando ha sofrido alguons males & non acquisiuerunt proprio labore . \textbf{ Nam quilibet cum maiori diligentia retinet facultates suas , } quando propter indigentiam passus est aliqua mala , \\\hline
1.4.2 & que son malos . \textbf{ Mas por la su moçençia } e por la su sinpleza mesuran alos otros . & Non enim putant alios esse malos , \textbf{ sed sua innocentia alios mensurant . } Cum ergo naturale sit , \\\hline
1.4.2 & Mas por la su moçençia \textbf{ e por la su sinpleza mesuran alos otros . } Et pues que assi es commo natural cosa sea & Non enim putant alios esse malos , \textbf{ sed sua innocentia alios mensurant . } Cum ergo naturale sit , \\\hline
1.4.2 & ¶ \textbf{ Lo sexto non han manera en las sus obras } Mas todas las cosas fazen forçadamente & ut dicunt . \textbf{ Sexto in suis actionibus | non habent modum , } sed omnia faciunt valde . \\\hline
1.4.2 & Le seyto cosa desconuenible es alos Reyes \textbf{ non auer manera en las sus obras . } Ca commo todas las sus obras & Sexto indecens est \textbf{ eos non habere modum } in actionibus suis : \\\hline
1.4.2 & non auer manera en las sus obras . \textbf{ Ca commo todas las sus obras } de una ser tenp̃das & eos non habere modum \textbf{ in actionibus suis : } quia cum alia sint moderanda per mensuram , \\\hline
1.4.3 & e non han fecho muchos males \textbf{ por la su sinpleza et inoçençia iudgan todos los otros . } Et todas las cosas retuerçen ala meior parte & quia non multa mala fecerunt \textbf{ et innocentes sunt , | sua innocentia alios mensurant , } et omnia referunt in meliorem partem : \\\hline
1.4.3 & Et por ende assi commo los uieios fallesçen \textbf{ en los sus cuerpos propreos } e en los humores & ut plurimum sequitur complexiones corporis . \textbf{ Sicut ergo senes in propriis corporibus deficiunt in humoribus , } et in vita : \\\hline
1.4.3 & e non misi cordiosos \textbf{ e caerien en malquerençia de los sus subditos ¶ } Lo terçero non conuiene a ellos de ser tem̃osos e de flacos coraçones & quia ex hoc contingeret eos esse seueros et immiseratiuos , \textbf{ et incurrerent maliuolentiam subditorum . } Tertio non decet eos esse timidos et pusillanimes , \\\hline
1.4.4 & Pues que assi es los vieios \textbf{ por la su flaqueza } por que quarrien que los otros ouiessen piadat & et compatiantur aliis . \textbf{ Senes ergo propter imbecillitatem , } quia vellent sibi alios compati , \\\hline
1.4.4 & Lo quarto non fazen ninguna cosa con sobrepunaça \textbf{ mas en todas las sus obras quieren paresçer tenprados . } Ca assi conmo los mançebos & Quarto nihil agunt valde , \textbf{ sed in omnibus operibus suis videntur esse temperati . } Nam sicut iuuenes , \\\hline
1.4.4 & e los prinçipes deuen auer \textbf{ en las sus obras mesura e tenpramiento } por que assi commo dicho es ellos & ne per hoc iudicentur leues et indiscreti . \textbf{ Quarto in suis actionibus debent habere moderationem et temperamentum : } quia ( ut dictum est ) \\\hline
1.4.5 & Et por ende los nobles teniendo mientes \textbf{ que en el su linage fueron muchos nobles } e que entendien a grandes cosas & Nobiles ergo aduertentes \textbf{ quod in eorum genere fuerunt multi insignes , } et tendentes in ardua , \\\hline
1.4.5 & tanto menos es memoria \textbf{ que los sus parientes fuessen pobres . } Et por ende sienpre cresçe la nobleza & per creationem filiorum , \textbf{ tanto minus est memoria genitores suos fuisse pauperes ; } ideo semper augmentatur nobilitas , \\\hline
1.4.5 & que les conuiene de fazer \textbf{ por que las sus obras } que veen todos non parescan de reprehender . & quid decet eos facere , \textbf{ ne opera eorum , } quae multi desiderant , \\\hline
1.4.5 & Ca commo los nobles non sean reprehendidos destos lisongos \textbf{ mas los sus malos fechos sean alabados } dellos disponen los & sed ab adulatoribus \textbf{ etiam eorum mala facta commendantur , } disponuntur , \\\hline
1.4.5 & por que son honrrados \textbf{ por el su linage } por ende quieren acresçentar aquella honrra & Nobiles ergo , \textbf{ quia ex suo genere videntur } esse honorabiles , \\\hline
1.4.5 & por que sienpre es mas antigua¶ \textbf{ Mas ser sobrauios e despreçiar los sus engendradores } e ser muy cobdiçiosos de honrra & quia semper est magis antiqua . \textbf{ Esse autem elatum , | et despicere suos progenitores , } et nimis esse honoris cupidi , \\\hline
1.4.6 & por que veen \textbf{ que los otro o han menester de los sus bienes } Ca cuenta el philosofo & et credentes se esse super eos , \textbf{ eo quod videant illos indigere bonis eorum . } Recitat enim Philosophus 2 Rhetoricorum , \\\hline
1.4.6 & si non se arte draten delas malas costunbres de los ricos \textbf{ e si non ordenar en las sus riquezas abien o a obras de uirtud } ¶ & nisi fugiant malos mores ipsorum diuitum , \textbf{ et nisi suas diuitias ordinent ad bonum , | et ad opera virtuosa . } Viso qui sunt mali mores diuitum , \\\hline
1.4.7 & que es noble \textbf{ e de antigo tienpo los sus auuelos fueron ricos meior sabe sofrir las riquezas } e por ellas non se leu nata en so ƀͣuia & quod tamen est nobilis , \textbf{ et ab antiquo sui progenitores diuites extiterunt , | melius nouit diuitias supportare , } et propter eas non tantum extollitur . \\\hline
1.4.7 & por que entienden en muchos cuydados \textbf{ por el su prinçipado non pueden assi entender nin se dar alos deleytes dela lux̉ia¶ } pues que assi es los Reyes e los prinçipes & quia diuersis curis intendunt propter principatum , \textbf{ non ita possunt vacare venereis . } Reges ergo et Principes , \\\hline
2.1.3 & Bien assi alguons suelen dezir \textbf{ que las sus casas fizieron esto } o aquello non por que las piedras fizieron esto & Sic aliqui dicere \textbf{ consueuerunt domos suas hoc operatas esse , } non quia lapides illud egerint , \\\hline
2.1.3 & o aquello non por que las piedras fizieron esto \textbf{ mas por que los sus padres o los sus fujos lo fizieron . } Por la qual razon & non quia lapides illud egerint , \textbf{ sed quia sui progenitores fecerunt illud , } quare sicut communicatio ciuium ciuitas nominatur , \\\hline
2.1.3 & por figera e por exienplo que conuiene alos omes de auer conueibles moradas \textbf{ segunt el su poder e la su riqueza . } Enpero non pertenesçea el de tractar delan casa prinçipalmente & quod decet homines habere habitationes decentes \textbf{ secundum suam possibilem facultatem ; } non tamen spectat \\\hline
2.1.5 & non pueden ser conseruadas \textbf{ nin guardadas en su ser si primeramente non resçibieren el su ser por generaçion . } La conseruaçion delas cosas engendradas non puede ser & Nam cum generata non possint conseruari in esse \textbf{ nisi prius per generationem acceperint esse , } conseruatio rerum generatarum esse non potest \\\hline
2.1.7 & Et por ende poniendo ellos las cosas propreas al comun \textbf{ assi commo quando la muger orden a las sus obras propreas al bien de su marido } o al bien de toda la casa . & Ponentes ergo propria ad commune , \textbf{ ut cum uxor propria sua ordinat | in bonum uiri uel } in bonum totius domus : \\\hline
2.1.10 & que las que son ayuntadas a sus maridos \textbf{ por matmonio con quanta diligençia deuen guardar la su honestad } e con quanto esfuerço deuen guardar la fialdat & quae suis viris per coniugium copulantur , \textbf{ quanta diligentia debeant | ad suam pudicitiam obseruare , } et quanto conatu fidem suis viris obseruent : \\\hline
2.1.11 & que son muy ayuntadas en parentesço \textbf{ por que non sea la su razon menguada } nin el su entendimientollagado & cum personis nimia consanguinitate coniunctis ; \textbf{ ne dando nimis operam venereis , } percutiatur eorum ratio , \\\hline
2.1.11 & por que non sea la su razon menguada \textbf{ nin el su entendimientollagado } e se ayan de tirar de los cuydados conuenibles & ne dando nimis operam venereis , \textbf{ percutiatur eorum ratio , } et retrahantur a curis debitis \\\hline
2.1.12 & que siruen a abastamiento dela uida . \textbf{ Conuiene aellos de demandar en las sus mugers } mas prinçipalmente que ellas sean nobles de linage & quae deseruiunt ad sufficientiam vitae : \textbf{ decet eos in suis coniugibus } principalius quaerere , \\\hline
2.1.13 & e por que los fijos dellos \textbf{ resplandezcan por grandeza de cuepo de demandar en las sus mugers grandeza de cuerpo . } Enpero tanto mas esto conuiene alos Reyes & ut filii polleant magnitudine corporali , \textbf{ quaerere in suis uxoribus magnitudinem corporis : } tanto tamen magis hoc decet Reges et Principes , \\\hline
2.1.13 & por fiios grandes e fermosos . \textbf{ Conuiene a ellos de demandar en las sus mugieres grandeza e fermosura corporal . } Ca paresçe que la fermosura dela muger & decet eos \textbf{ in suis uxoribus quaerere magnitudinem , | et pulchritudinem corporalem : } videtur enim pulchritudo coniugis \\\hline
2.1.13 & Et pues que assi es conuiene a todos los çibdadanos \textbf{ de demandar esto en las sus mugers } Empero tanto conuiene esto mas alos Reyes & Decet ergo omnes ciues \textbf{ hoc in suis coniugibus quaerere : } tanto tamen hoc decet Reges et Principes , \\\hline
2.1.14 & e las con diconnes \textbf{ que deue el guardar en el su gouernamiento . } Et pues que assi es dela manera del gouernar & et conuentiones \textbf{ quasdam in suo regimine obseruare . } Ex ipso ergo modo regendi , \\\hline
2.1.18 & que fallesçen de razon \textbf{ mas el su freno dellas es passion } assi commo uerguença ça & quia communiter a ratione deficiunt : \textbf{ sed magis est passio } ut verecundia : \\\hline
2.1.20 & Et pues que assi es conuiene a cada vno de los uarones \textbf{ penssando el su estado propreo } e catadas las condiconnes delas perssonas mostrar a sus & Decet ergo quoslibet viros , \textbf{ considerato proprio statu , } et inspectis conditionibus personarum , \\\hline
2.1.21 & en todas aquellas cosas \textbf{ que pueden fallesçer e errar las sus mugers en la mayor parte } e que cosas les son conuenibles & in quibus ut plurimum delinquunt foeminae , \textbf{ diligenter aduertere } quae sunt ibi licita , \\\hline
2.1.21 & Mas estonçe son tenpradas \textbf{ quando catado el su estado non quieren } nin demandan uestiduras sobeias . & Tunc vero sunt moderatae , \textbf{ quando considerato suo statu } non superflua vestimenta quaerunt . \\\hline
2.1.21 & dessease los honrramientos e conponimientos del cuerpo \textbf{ mas que demanda el su estado . } ¶ Lo terçero conuiene alas mugers de ser & si non esset moderata , \textbf{ et ultra quam suus status requireret , } appeteret ornamenta . \\\hline
2.1.21 & por que algunas mugersen tanto son ꝑezosas \textbf{ que por la su negligençia dexan de ser acuiçiosas } cerca los conponimientos & Nam aliquae adeo sunt pigrae , \textbf{ quod ex sola negligentia omittunt } solicitari \\\hline
2.1.21 & que alguna muger se ensobuesce \textbf{ por la su mes } quandat que sufre & Contingit enim aliquando aliquem efferri et superbiri \textbf{ ex ipsa miseria , } quam sustinet . \\\hline
2.1.21 & por que quarian vestiduras mas viles \textbf{ que el su estado demandaua . } Et por esta razon se mouiana so ƀiuia e a alabança . & vituperat Laconios , \textbf{ qui infra suum statum vestimenta quaerentes } ex hoc in elationem et iactantiam mouebantur . \\\hline
2.1.21 & e non demanden uestiduras mas honrradas \textbf{ que el su estado demanda¶ } Lo quarto que sean sinples & Tertio , ut sint moderatae , \textbf{ ne ultra eorum statum vestimenta requirant . } Quarto , ut sint simplices , \\\hline
2.1.23 & que del omne \textbf{ por que el consseio de la muger es mas ayna el su conplimiento que deluats . } por que si acaesçiesse de obrar alguna cosa adesora & In casu tamen potest \textbf{ esse muliebre consilium melius quam virile : | ut quia illud est citius in suo complemento , } sic oporteret repentino operari , \\\hline
2.1.23 & por auentra a seria \textbf{ mas de escoger el su conseio que del uaron . } or tres razones podemos & sic oporteret repentino operari , \textbf{ forte elegibilius esset huiusmodi consilium . } Triplici via inuestigare possumus , \\\hline
2.1.24 & Et por ende de ligero paresçe \textbf{ en qual manera los maridos de una descobrir a sus mugieres los sus secretos . } Ca quando nos dezimos & Ex hoc autem de facili apparet , \textbf{ qualiter viri suis coniugibus debeant | reuelare secreta . } Nam cum dicimus hos esse mores iuuenum , \\\hline
2.1.24 & Et pues que assi es los maridos \textbf{ non deuen descobrir a sus mugers las sus poridades } saluo a aquellas de que han prouado de luengot & Viri igitur non debent \textbf{ suis coniugibus secreta aperire , } nisi per diuturna tempora sint experti , \\\hline
2.2.1 & por que pueda sobir suso \textbf{ Por que en logar de suso mas ha de ser guardado el su ser } que en el logar de yuso . & per quam feratur sursum , \textbf{ eo quod in loco superiori magis habet } in esse conseruari quam in inferiori : \\\hline
2.2.2 & e alos prinçipes de ser acuçiosos a sus fijos \textbf{ quanto los sus fijos deuen auer mayor sabidina } e mayor bondat ¶ & quanto filii eorum pollere debent \textbf{ maiori prudentia et ampliori bonitate . } Tertia via ad hoc ostendendum sumitur \\\hline
2.2.3 & Et esto puede ser en dos maneras . \textbf{ Ca el que gouierna o entiende el su bien prop̃o } e assi es dicho gouernamiento despotico o suilo & et hoc potest esse dupliciter : \textbf{ quia vel sic regens | intendit bonum proprium , } et sic dicitur regimen despoticum vel dominatiuum : \\\hline
2.2.3 & por el fallesçimiento dellos \textbf{ e por que quiere el su bien ¶ } pues que ssi es paresçe & propter profectus ipsorum , \textbf{ et quia vult eorum bonum . } Constat ergo quod huiusmodi regimen sumit originem ex amore . \\\hline
2.2.3 & enssennorea alos sieruos \textbf{ tor el su bien propo } e non por el bien de los sieruos . & Nam dominus praeest seruis \textbf{ propter bonum proprium , } non propter bonum seruorum : \\\hline
2.2.5 & que la sabiduria de dios \textbf{ e la su auctoridat sobrepiua toda sotileza de engennio humanal . por la qual cosa mas prouechosa cosa es de creer } sinplemente la auctoridat de dios & diuinam prudentiam et eius auctoritatem , \textbf{ omnem perspicaciam humani generis superare . | Quare utilius auctoritati diuinae simpliciter creditur , } quam acquiescatur rationibus \\\hline
2.2.5 & e estas son allegadas al coraçon \textbf{ por que los moços en la su moçedat son acostunbrados en ellas . } Por ende por que firmemente nos alleguemos ala fe & et sunt sic applicabiles animo , \textbf{ si pueri ab infantia assuescant illis } ut firmiter adhaereant fidei christianae , \\\hline
2.2.5 & Et que todos resuçitaremos \textbf{ e estaremos ante la su faz } dando razon de todos nuestros fechos . & et omnes resurgemus , \textbf{ et stabimus ante tribunal eius , } reddituri de factis propriis rationem . \\\hline
2.2.7 & que por el pudiessen razonar e mostrar \textbf{ acabadamente te dos los sus conçibimientos . } Por la qual & ut per ipsum possent omnes \textbf{ suos conceptus sufficienter exprimere . } Quare si hoc idioma est completum , \\\hline
2.2.7 & e mal a pareiados para laber \textbf{ Por que el nuestro conosçimiento comiença enl seso } e en las cosas & uniuersaliter tamen homines male nati sunt ad sciendum : \textbf{ quia nostra cognitio incipit } a sensu et a posterioribus . \\\hline
2.2.7 & si quisieren \textbf{ que los sus fijos departidamente } e derechamente fablen las palabras delas letras & et maxime Reges , et Principes , \textbf{ si volunt suos filios distincte } et recte loqui literales sermones , \\\hline
2.2.11 & si demandat en viandas muy delicadas \textbf{ maque demanda el su estado . } Ca el delicamiento delas viandas & si quaerantur cibaria nimis lauta et delicata \textbf{ ultra quam eius status requirat . } Delicatio enim ciborum accipienda est \\\hline
2.2.11 & Et pues que assi es aquel que quiere \textbf{ mas que la condiçionde la su persona demanda } e mas que conuiene al su estado demanda viandas delicadas peca enllo . & et secundum statum nobilitatis eius . \textbf{ Qui ergo , ultra quam conditio personae exigat , } et ultra quam eius status requirat , \\\hline
2.2.14 & Ca el alma en la mayor parte sigue la conplission del cuerpo . \textbf{ Ca por que el nuestro conosçimiento comiença en el seso } e las colas & Anima enim ut plurimum sequitur complexiones corporis : \textbf{ nam quia nostra cognitio incipit a sensu , } et sensibilia sunt nobis magis nota ; \\\hline
2.2.15 & assi commo son los alimanes \textbf{ e los de nuruega de vannar los sus fijos en los trios muy frios } por que los fagan muy fuertes & quod apud aliquas Barbaras nationes consuetudo est \textbf{ in fluminibus frigidis balneare filios , } ut eos fortiores reddant . \\\hline
2.2.17 & que nol manda algua cosa \textbf{ si non por el su prouecho } ¶ La segunda razon es & quem scit non percipere aliqua \textbf{ nisi ad bonum eius . } Secunda est ; \\\hline
2.2.17 & fincanos de demostrar \textbf{ en qual manera ha de ser alunbrado el su entendimiento . } Mas esto assaz puede ser manifiesto & reliquum est ut ostendatur , \textbf{ quomodo recte illuminandus sit intellectus . } Hoc autem satis esse potest \\\hline
2.2.19 & ca dixiemos de suso \textbf{ que el nuestro conosçimiento comiença enlseso } e mucho mas son a nos conosçidas las cosas senssibles & Dicebatur enim supra , \textbf{ nostram cognitionem incipere a sensu , } et maxime nobis nota esse sensibilia ; \\\hline
2.2.20 & en sus obras propreas \textbf{ e todos aman las sus obras } assi commo dize el philosofo & Quia igitur omnes delectantur in propriis operibus , \textbf{ et omnes diligunt sua opera , } ut vult Philosophus 9 Ethicorum , \\\hline
2.2.20 & luego commo alguno non se da a vsos \textbf{ conueinbł stanto la su uoluntad } andauagando aquende & nescit ociosa esse , \textbf{ statim cum quis non dat se licitis exercitiis , } vagatur eius mens \\\hline
2.3.3 & e tan magnifico \textbf{ commo muestran las sus obras } ca la grandeza delas moradas & et tam magnificum . \textbf{ Magnitudo enim aedificiorum licet } non sit fienda \\\hline
2.3.3 & e alos prinçipes de fazer moradas costosas e nobles \textbf{ assi commo el su estado demanda } por que non sean auidos en menospreçio del pueblo . & facere aedificia magnifica , \textbf{ prout requirit decentia status , } in quo existunt . \\\hline
2.3.4 & Lo primero puede contesçer si la morada \textbf{ segunt la su parte mayor } catare a lorsete del yuierno & si aedificium \textbf{ secundum suam ampliorem partem respiciat oriens hyemale : } tunc enim eo quod in hyeme oppositum sit soli , \\\hline
2.3.7 & que la uida del robar es conuenible \textbf{ e que non solamente los omes deuen tomar et robar las sus cosas alos otros } mas avn deuen tomar a ellos en su perssona & ut quod licitum esset \textbf{ non solum hos depraedari et accipere sua , } sed eos etiam accipere in praeda , \\\hline
2.3.7 & si quieren gouernar conueniblemente en su uida \textbf{ las sus casas propreas } conuiene les de saber quantas son las uidas & et Principes uiuere uita uirtuosa ; \textbf{ si uolunt domus proprias debite gubernare , } decet eos scire \\\hline
2.3.10 & contesçe les de resçebir mas \textbf{ por ellas que en las sus tierras propreas . } por la qual cosa a auentra a & accidit eos plus recipere \textbf{ pro numismatibus illis , | quam in partibus propriis : } propter quod casu campsoria usi sunt . \\\hline
2.3.12 & segunt uida politica de auer cuydado de ganar dineros segunt que requiere \textbf{ e demanda el su estado de cada vno . } Mas alos Reyes e alos prinçipes & habere curam de acquisitione pecuniae , \textbf{ secundum quod exigit suus status : } Apud Reges autem , \\\hline
2.3.12 & de saber las condiconnes particulares del regno \textbf{ e los fechs particulares de los sus anteçessor } ssegunt los quales gana una conueniblemente so dineros . & sciendo particulares conditiones regni , \textbf{ et gesta particularia praedecessorum suorum , } secundum quae licite pecuniam acquirebant . \\\hline
2.3.14 & en las politicas \textbf{ aya algua auentaia sobre el su sieruo . } Et esta auentaia puede ser en dos maneras & ( ut dicitur in Politic’ ) \textbf{ habere aliquem excessum respectu serui . } Huiusmodi autem excessus dupliciter esse potest , \\\hline
2.3.17 & Mas penssada la condiçion delas personas \textbf{ assi se deue partir acadera vno dellos segunt el su estado } por que en esto parezca la sabiduria & nec aeque pulchris indumentis gaudere debent , \textbf{ sed considerata conditione personarum sic secundum suum statum cuilibet sunt talia tribuenda , } ut in hoc appareat prouidentia et industria principantis . \\\hline
2.3.18 & assi que non sea memoria en el pueblo \textbf{ que los sus padres } nin los sus auuelos fueron pobres & ita quod non sit memoria \textbf{ in populo progenitores suos fuisse pauperes , } dicitur habere nobilitates generis , \\\hline
2.3.18 & que los sus padres \textbf{ nin los sus auuelos fueron pobres } e estos tales son dichos & ita quod non sit memoria \textbf{ in populo progenitores suos fuisse pauperes , } dicitur habere nobilitates generis , \\\hline
2.3.18 & de ser ellos mas labios que los otros . \textbf{ Et segunt el su estado conuienel es de ser meiores } que los & eos esse prudentiores aliis , \textbf{ et secundum quem decet } eos esse meliores aliis ; \\\hline
2.3.18 & que fazen obras de uirtudes \textbf{ partiendo los sus bienes alos otros } e esto non lo fazen & Sunt enim multi facientes opera virtutum \textbf{ ut bona sua aliis largientes , } non agentes hoc quia eis placeat expendere ; \\\hline
2.3.19 & que han tan grand poder \textbf{ por que esto puedan escusar el su procurador deue tomar esta carga e esta honrra } e ellos deuen beuir çiuilmente & quod quibusdam est potestas , \textbf{ ut hoc vitent , | procurator accipit hunc honorem : } ipsi vero ciuiliter viuunt , \\\hline
2.3.19 & deuen se mostrar \textbf{ tonprados a los sus seruientes propreos } los quales en conparacion dellos son humillosos e baxos & quos decet esse magnanimos \textbf{ ad proprios ministros , } qui respectu eorum sunt inferiores et humiles , \\\hline
2.3.19 & e de reuerençia \textbf{ ¶ Etrossi non se deue mostrar tantlto el su estado } por que en toda manera parezca cruel & et ut non appareat persona reuerenda : \textbf{ nec debet se sic excellentem ostendere , } ut omnino appareat austerus et onerosus . \\\hline
2.3.20 & o avn los fechos \textbf{ que son de loar de los sus anteçessores } e mayormente de aquellos que scanmente e religiosamente se oueron & si tales sunt in scripto redactae , \textbf{ vel etiam laudabilia gesta praedecessorum suorum , } et maxime eorum \\\hline
3.1.1 & e que non es bueno en ssica \textbf{ commo las nuestras obras nos ordenamos } a algun bien alguas uezes & quod existit bonum . \textbf{ Nam cum opera nostra ordinamus } ad aliquod bonum , \\\hline
3.1.9 & por meior que el otro quarria \textbf{ que segunt la su dignỉdat le diessen mayor gualardo delas cosas comunes } mas esta egualdat non se podria guardar de ligero & se meliorem alio , \textbf{ uellet secundum dignitatem suam | ei fieri retributionem . } Hanc autem aequalitatem non de facili esset possibile reseruari inter ciues \\\hline
3.1.10 & segunt su proporcion \textbf{ e segunt el su estado } e a cada vno es guardado su derecho & quando quilibet se habet \textbf{ secundum proportionem suam , } ut quando ignobiles seruient nobilibus , \\\hline
3.1.10 & si les non fuesse mostrado \textbf{ o el su parentesco } por la qual cosa commo en las personas tan ayuntadas & non prohibebatur eis amor libidinosus et concupiscentia , \textbf{ ex quo non manifestabatur eis parentela illa . } Quare cum in personis tam coniunctis \\\hline
3.1.11 & non los podiendo escusar \textbf{ que si entre los sennoron e los sus sieruos } que han assi subiectos & et diu conuersari cum illis . \textbf{ Quare si inter dominos et famulos quos habent } ita subiectos propter diuturnam conuersationem tot litigia oriuntur , \\\hline
3.1.13 & non es manifiesta \textbf{ assi la su bondat o la su maliçia . } ca quando es persona prouada sin ofiçio o sin dignidat non se puede & quam assumantur homines ad aliquam dignitatem , \textbf{ non sic manifestatur eorum bonitas et malitia . } Nam quia persona priuata \\\hline
3.1.14 & que sintieron los philosofos enel gouernamiento \textbf{ e tractadas las sus opimones } mas claramente paresçra & Viso enim quid circa hoc senserunt Philosophi , \textbf{ et pertractatis eorum opinionibus , } clarius apparebit \\\hline
3.1.14 & que la parte se ponga al peligro \textbf{ por el su todo } assi commo braço luego se pone a peligro & Nam semper bonum commune praeponendum est bono priuato : \textbf{ naturale enim est partem se exponere periculo pro toto , } ut brachium statim exponit se periculo pro defensione corporis , \\\hline
3.1.15 & Si quisieremos entender los dichos de socrates \textbf{ non assi conmo suena las palabras podremos entender la su opinion diziendo } que non es cosa que pueda ser & intelligere dicta Socratica , \textbf{ saluare poterimus positionem eius . } Omnia enim esse ciuibus communia \\\hline
3.1.15 & por quacada vno de los çibdadanos \textbf{ deue enprestar alos otros çibdadanos los sus bienes } assi conmoiudga la razon . & quia quilibet ciuis debet \textbf{ aliis ciuibus } ( ut dictat ratio ) \\\hline
3.1.15 & quanto ala comunidat de los çibdadanos \textbf{ En essa misma manera podemos saluar el su dicho } del quanto ala vnidat dela çibdat & quantum ad communitatem ciuium : \textbf{ sic etiam saluare possumus dictum eius quantum ad unitatem ciuitatis . } Nam cum dixit ciuitatem debere esse maxime unam , \\\hline
3.1.15 & que los uarones desmanparando la çibdat \textbf{ e commo salliessen della fue cometida la çibdat de los sus enemigos dellos } por la qual cosa conuenio alas mugers & et euntibus in exercitium supra ciuitatem aliquam , \textbf{ dum viri etiam in exercitu existerent , | inuasa est eorum ciuitas ab hostibus , } propter quod oportuit \\\hline
3.1.17 & Lo primero quando los pobres se fazen ricos \textbf{ non saben sofrir la su buena uentura } assi commo el philosofo muestra llanamente en el segundo libro de la rectoriça & et iurgia in ciuitate . \textbf{ Primo quia pauperes cum ditantur nesciunt fortunas ferre , } ut plane ostendit \\\hline
3.1.19 & que faze tuerto a otro \textbf{ o le faze tuerto en las sus cosas } enpeesçiendol o dannando gelas o le faze tuerto enla persona & quicunque enim iniustificat in alium , \textbf{ vel iniustificat in res nocendo } et damnificando ipsum : \\\hline
3.1.19 & assi commo paresçe \textbf{ por los sus dichos } que el prinçipe non deue ser fecho & nolebant enim \textbf{ ( ut apparet ex dictis suis ) } principem debere esse per haereditatem , \\\hline
3.1.20 & por que estas cosas tales \textbf{ que ellos dizen algunas uezes abuian el nuestro entendimiento } ca los malos dichos de los otros muchͣs & et discordant ab opinionibus nostris : \textbf{ excitant enim talia aliquando intellectum : } mala quidem aliorum dicta \\\hline
3.1.20 & e por ende contamos la opinion de ipodomio \textbf{ por que el en la su poliçia manifesto muchͣs bueanssmans . } Empo algunas cosas establesçio non conuenible mente . & Hippodami ergo opinionem recitauimus , \textbf{ quia in sua politia multas bonas sententias promulgauit : } aliqua tamen incongrue statuit . \\\hline
3.1.20 & quanto a tres cosas que el puso . \textbf{ Lo primero quanto ala inpossibilidat de los sus establesçimientos¶ } Lo segundo quanto ala manera que establesçio en iudgar . & Hippodamum quantum ad tria . \textbf{ Primo , quantum ad impossibilitatem statutorum . } Secundo quantum ad modum , \\\hline
3.2.4 & que vno tal sea sennor que muchs . \textbf{ ante el su sennor seria muy malo . } Por que assi commo paresçe por las cosas ya dichͣs & talem ergo dominari non esset melius quam plures : \textbf{ immo pessimum esset eius dominium , } eo quod \\\hline
3.2.5 & e non ouieren muy grant cuydado \textbf{ que los sus fijos de su ninnez sean bie doctrinados } e bien acostunbrados & si non nimia cura solicitentur \textbf{ quod filii ab ipsa infantia et disciplina } et bonis moribus sint imbuti ; \\\hline
3.2.6 & por ende el omne bueno e uirtuoso mas procurara el bien comun \textbf{ que el su bien pro ̉o priuado . } Lo terçero conuiene & quam bonum aliquod proprium et priuatum . \textbf{ Tertio expedit eum abundare } in ciuili potentia , ut possit corrigere volentes insurgere , \\\hline
3.2.6 & el que entiende en el bien comun . \textbf{ por la qual cosa si el thirano entiende el su bien propreo siguese } que la su entençion es mala & et quanto honore est dignius intendens commune bonum . \textbf{ Quare si tyrannus intendit | bonum proprium et priuatum , } sequitur quod eius intentio versetur \\\hline
3.2.6 & por la qual cosa si el thirano entiende el su bien propreo siguese \textbf{ que la su entençion es mala } ca non es cerca el bien honrrado e de honira & bonum proprium et priuatum , \textbf{ sequitur quod eius intentio versetur | circa bonum delectabile . } Sed intentio Regis versatur \\\hline
3.2.6 & si non delas sus delectaçonnes propreas . \textbf{ Et por ende la su entençion toda se pone en el auer } o en los dinos creyendo que por ellos puede auer las otras cosas delectables . & nisi de delectationibus propriis , \textbf{ maxime versatur sua intentio circa pecuniam , credens se per eam posse huiusmodi delectabilia obtinere . } Sed regis intentio versatur circa virtutem , \\\hline
3.2.6 & e del bien comun fia muchͣ \textbf{ de aquellos que son en el su regno . } Et por ende se faze guardar & quod videat se maximam curam habere de bono regni et communi , \textbf{ maxime confidit de his | qui sunt in regno . } Ideo facit se custodiri a propriis , \\\hline
3.2.6 & Et por ende se faze guardar \textbf{ de los sus çibdadanos propreos } e non de los estrannos . & qui sunt in regno . \textbf{ Ideo facit se custodiri a propriis , } non ab extraneis . \\\hline
3.2.7 & estonçe es Rey \textbf{ e el su prinçipado es muy bueno . } ca por la uirtud ayuntada en el & nam si monarchia habet intentionem rectam , \textbf{ tunc est Rex et est optimus principatus : } quia propter unitatem virtutes potest \\\hline
3.2.7 & estonçe es tirano \textbf{ et el su prinçipado es muy malo } ca por el su poderio muy grande & si vero monarchia habet intentionem peruersam , \textbf{ tunc est tyrannus et est pessimus , } quia propter suam unitam potentiam potest \\\hline
3.2.7 & et el su prinçipado es muy malo \textbf{ ca por el su poderio muy grande } que es ayuntado en vno puede fazer muchs males & tunc est tyrannus et est pessimus , \textbf{ quia propter suam unitam potentiam potest } multa mala efficere . \\\hline
3.2.7 & ca qtanto peor es el prinçipe \textbf{ quanto peor es el su sennorio } por la qual cosa el Rey deue poner muy grant acuçia & Nam tanto peior est Princeps , \textbf{ quanto peiori dominio principatur : | quare } si omnem diligentiam adhibere debet Rex ne sit pessimus , \\\hline
3.2.8 & que le es a comnedada \textbf{ e si dessea saber qual es el su ofiçio } deue penssar con grant acuçia & Si Rex aut Princeps gentem sibi commissam vult debite gubernare , \textbf{ et scire desiderat | quod sit eius officium : } diligenter considerare debet \\\hline
3.2.8 & Et por ende el rey deue ser muy acuçioso \textbf{ por que en el su regno aya estudio de letris } e por que sean y muchos sabios & Debet igitur Rex solicitari \textbf{ ut in suo regno uigeat studium litterarum , } et ut ibi sint multi sapientes et industres . \\\hline
3.2.8 & e non quisiere \textbf{ que los sus subditos sean sabios } el tal non es Rey mas es tirano . & non promoueat studium , \textbf{ et non uelit sibi subditos esse scientes , } non est Rex , sed tyrannus . \\\hline
3.2.8 & assi las çibdades e los regnos \textbf{ por que los sus subditos abonden en las cosas de fuera } en quanto ellas siruen a bien beuir & Decet ergo Reges et Principes sic regere ciuitates et regna , \textbf{ ut sibi subiecti abundent rebus exterioribus } prout deseruiunt ad bene viuere , \\\hline
3.2.9 & commo enlas fijas \textbf{ e roban les los sus bienes } ¶Lo quinto conuiene alos Reyes & et in filiabus , \textbf{ et rapiunt eorum bona . } Quinto decet Reges \\\hline
3.2.9 & ca commo ellos non entienden \textbf{ si non el su bien propo } assi commo auer dineros o auer delecta connes . & nam cum non intendant \textbf{ nisi commodum proprium } ut bonum pecuniosum et delectabile , \\\hline
3.2.9 & Lo septimo conuiene al uerdadero Rey de conponer \textbf{ e guarnesçer las çibdades e los castiellos que son en el su regno } assi que parezca mas ser procurador del bien comun & Septimo decet verum Regem ornare \textbf{ et munire ciuitates | et castra existentia in regno , } ut appareat magis esse procurator communis boni , \\\hline
3.2.9 & que non serien tan \textbf{ honrradosente los sus çibdadanos propreos } si entre ellos estudiessen . & ut putetur non sic honoratos esse \textbf{ a ciuibus propriis , } si inter ipsos existerent , \\\hline
3.2.9 & todas las cosas son manifiestas \textbf{ e el su poderio aqui non puede ser ninguna cosa contraria legnia } assi conmo cunple a su salut . & cui omnia sunt nota , \textbf{ et eius potentia cui nihil potest resistere , | continget eum } ut expedit suae saluti \\\hline
3.2.9 & Et por ende por la sanidat del Rey dios muchas uezes faze muchs bienes \textbf{ a aquellos que son en el su regno . } Et esto dezimos a postremas de todo notablemente & multotiens Deus \textbf{ multa bona confert existentibus in ipso regno . } Ultimo autem diximus \\\hline
3.2.10 & La primera cautela del tirano es matar los grandes omes e los poderosos . \textbf{ Ca commo el tirano non ame sinon el su bien propreo los poderosos e los nobłs } ̃en el su regno & est excellentes perimere . \textbf{ Cum enim tyrannus non diligat | nisi bonum proprium , } excellentes et nobiles existentes \\\hline
3.2.10 & Ca commo el tirano non ame sinon el su bien propreo los poderosos e los nobłs \textbf{ ̃en el su regno } non quariendo sofrir sus males leuna tanse contra el . & nisi bonum proprium , \textbf{ excellentes et nobiles existentes } in regno non valentes hoc pati , \\\hline
3.2.10 & qua non son sus parientes \textbf{ mas avn los sus hͣmanos proprios } e los sus parientes muy cercanos matan con venino e con poconna o con otra falssedat . & non solum excellentes alios , \textbf{ immo etiam proprios fratres , } et nimia sibi consanguinitate coniunctos venenant , \\\hline
3.2.10 & mas avn los sus hͣmanos proprios \textbf{ e los sus parientes muy cercanos matan con venino e con poconna o con otra falssedat . } la qual cosaes muy mala señal de tirama . & immo etiam proprios fratres , \textbf{ et nimia sibi consanguinitate coniunctos venenant , | et perimunt : } quod signum est tyrannidis pessimae . \\\hline
3.2.10 & non mata los grandes e los nobles \textbf{ e much menos los sus parientes propreos . } por los quales el buen estado del regno se puede guardar mas saluales & qui sunt in regno , excellentes , et nobiles , \textbf{ et multo magis cognatos proprios , } per quos bonus status regni conseruari potest , \\\hline
3.2.10 & que ellos non entienden enl bien comun \textbf{ mas en el su bien propreo . } Por ende querrien que todos los sus subditos fuessen sin sabiduria e nesçios & Vident enim se contra dictamen rectae rationis agere , \textbf{ et non intendere bonum commune sed proprium : } ideo vellent omnes suos subditos \\\hline
3.2.10 & mas en el su bien propreo . \textbf{ Por ende querrien que todos los sus subditos fuessen sin sabiduria e nesçios } por qua non conosçiessen & et non intendere bonum commune sed proprium : \textbf{ ideo vellent omnes suos subditos | esse ignorantes et inscios , } ne cognoscentes eorum nequitiam , \\\hline
3.2.10 & por razon que los sabios conosciendo \textbf{ e sabiendo las sus buenas obras } mueuen & et honorat , \textbf{ eo quod ipsi cognoscentes bona opera ipsius , } populum commouent ad amorem eius . \\\hline
3.2.10 & o por algun guerra derechurera . \textbf{ La nouena caute la del tirano es poner grant guarda en el su cuerpo } non por aquellos que son del regno & vel pro aliquo alio iusto bello . \textbf{ Nona , est custodiam corporis exercere } non per eos \\\hline
3.2.11 & Et pues que assi es contra estas quatto cosas procuran los tyranos de matar los nobles e los grandes \textbf{ por que los sus subditos non sean osados } nin de grandes coraçones . & Contra haec ergo quatuor procurant tyranni perimere excellentes , \textbf{ ne sui subditi sunt magnanimi : } destruere sapientes , \\\hline
3.2.13 & Lo quinto son puestos asechos alos tiranos de algunos non \textbf{ por que ayan el su señorio mas por que paresca alos omes } que fazen algunos omes apartadas & Quinto fiunt insidiae tyrannis ab aliquibus , \textbf{ non ut possideant monarchiam , | sed ut videantur } facere actiones aliquas singulares . \\\hline
3.2.14 & para mostrar que si los rrey e cobdiçian de duar muncho \textbf{ el su señorio es toda manera deuen estudiar } por que se no fagantiranos . & ostendentes quod si reges cupiant suum durare dominium , \textbf{ summo opere studere debent } ne efficiantur tyranni , \\\hline
3.2.14 & por la qual cosa quando algͤse arriedra dela iustiçia apareia en ssi carrera \textbf{ por que sea corrun pido el su sennorio } pues que assi es quanto el gouernamiento & praeparatur via \textbf{ ut corrumpatur principatus ille . } Politia ergo quanto de se magis a iustitia recedit , \\\hline
3.2.14 & fuerca atenprar la tirania \textbf{ ca quanto mas poco tiranzar en tanto mas dura el su sennorio ¶ } Lo segundo la tirania se corronpe & suam tyrannidem pro viribus moderare debent , \textbf{ quia quanto remissius tyrannizabunt , | tanto durabilius principabuntur . } Secundo tyrannis corrumpitur \\\hline
3.2.14 & por que muchͣs vezes vn prinçipe tirano se leunata contra otro prinçipe tirano \textbf{ por que gane el su prinçipado . } Por ende mucho deuen escusar los Reyes & quia multotiens unus monarcha tyrannus insurgit in alium , \textbf{ ut obtineat principatum eius . } Debent ergo cauere Reges et Principes ne tyranizent , \\\hline
3.2.14 & e en tantas \textbf{ manerasse ha de desfazer el su prinçipado } non es bue no de tiranizar mas el regno & et tot modis habet \textbf{ dissolui eius principatus . Regium autem dominium } non tot periculis exponitur , \\\hline
3.2.16 & por que sienpre son en mouimiento \textbf{ e non se pueden mudar los sus mouimientos } por las nuestras obras & sydera enim licet moueantur , \textbf{ tamen quia semper sunt in motu , } nec propter nostra opera \\\hline
3.2.16 & nin taen so consseio . \textbf{ ca non caen so el nuestro conseio } las obras de aquellos omes & puta de thesauri inuentione . \textbf{ Quinto non sunt consiliabilia } etiam omnia humana opera : \\\hline
3.2.16 & assi commo la fin non ca en so nuestro conseio \textbf{ ca el nuestro consseio non es dela fu . } por que conuiene que en el conseio sorongamos la fin & Nam consilium non est de fine , \textbf{ sed de his quae sunt ad finem : } oportet enim in consilio \\\hline
3.2.18 & ca por esto cada vno razon a bien en los consseios \textbf{ e es de creer en los sus dichos } por que cuydan los omes & Nam ex hoc aliquis persuadet in consiliis \textbf{ et creditur dictis eius , } quia existimatur bonus consiliarius esse ad persuadendum . \\\hline
3.2.19 & si tomasse los bienes \textbf{ de aquellos que son en el su regno sin derech } lo segundo ha de tener mient̃s el Rey de non ser engannado enlas sus rentas . & Rursus est attendendum , \textbf{ ne in suis prouentibus defraudetur : } expedit enim regium consilium \\\hline
3.2.19 & que sea puesta en los derechs del Rey \textbf{ e sean acresçentadas las sus rentas ¶ } Lo segundo deueser tomado consseio en fech delas uiandas & vel est diminutus , \textbf{ apponatur et augeatur . } Secundo debet esse consilium de alimento , \\\hline
3.2.21 & mas tornassea mouer el uuez a sana o a aborrençia \textbf{ contra los sus contrarios } e pugnan por lo mouer amanssedunbre e amiscderia contra si mismos . & ad commouendum iudicem \textbf{ ad iram et odium circa aduersarios , } et ad benignitatem et misericordiam erga seipsos . \\\hline
3.2.24 & ommo las leyes sean vnas reglas de derech . \textbf{ por las quales nos somos reglados en las nuestras obras } iudgando por ellas & Cum leges sunt quaedam regulae iuris , \textbf{ per quas in agibilibus regulamur , } diiudicantes per ipsas \\\hline
3.2.25 & Mas si esta inclinaçion \textbf{ siguiere la nuestra natura en quanto auemos conueniençia con las otras aian lias } assi es dich derech natural . & Si vero inclinatio illa sequatur naturam nostram , \textbf{ ut conuenimus cum animalibus aliis : } sic dicitur esse ius naturale . \\\hline
3.2.25 & que la natura enssenno a todas las otras nanlas \textbf{ e que sigue la nuestra inclinaçion natural } en quanto participamos con las otras aianlas & Ius itaque illud quod natura omnia animalia docuit , \textbf{ et quod sequitur inclinationem nostram naturalem } ut communicamus cum animalibus aliis , \\\hline
3.2.25 & tantomas conosçida es a nuestro entendimiento \textbf{ Et primero cae en el nuestro conosçimiento . } Avn este derecho tal que es derecho delas & tanto est intellectui nostro notius , \textbf{ et prius cadit in apprehensione nostra . } Est etiam huiusmodi ius immutabilius , quia regulae iuris \\\hline
3.2.25 & e el mal es aquello que desseamos naturalmente \textbf{ en quanto la nuestra nata conuiene con todas las substançias . } Et assi avn es el derech natural & quod naturaliter appetimus , \textbf{ prout natura nostra conuenit | cum omnibus entibus , } sic est de iure naturali : \\\hline
3.2.26 & que sea entendido el bien comun . \textbf{ Ca sienpre la fin es regla de todas las nuestras obras . } Ca assi commo dize el pueblo comunalmente & intendi commune bonum : \textbf{ nam semper finis est regula omnium nostrorum agibilium . } Nam et vulgo dicitur , \\\hline
3.2.26 & que desacuerda del su todo \textbf{ e que non acuerda con el su todo } si en las leyes es entendido algun bien & quae discordat a toto , \textbf{ et quae suo non congruit uniuerso : } si in legibus intenditur aliquod bonum proprium , \\\hline
3.2.30 & o qual quier otro fazendor de ley \textbf{ si non entendiere fazer a todos los sus çib } dadanos los mas uirtuosos que pudiere ¶as commo ninguno non pueda venir a acatadas uirtudes sim̃o entendiere escusar todos los pecados & alius legislator , \textbf{ nisi intendat suos conciues | facere } quam virtuosiores potest . \\\hline
3.2.32 & e su magnifiçençia \textbf{ e por que les pudiesse dar de los sus bienes non termie todos aquellos bienes en much . } Et pues que assi es la çibdat fue fecha & ut alii suam magnificentiam perciperent , \textbf{ et ut eis sua bona communicare posset , | non multum reputaret illa . } Facta est ergo ciuitas , \\\hline
3.2.33 & que los reyes e los prinçipes ayan cautelas e sabidurias . \textbf{ por que en el su regno sean muchͣs perssonas medianeras } por que los vnos non vengan atan grant pobreza & Decet ergo Reges et Principes adhibere cautelas , \textbf{ ut in regno suo abundent multae personae mediae ; } ut ne aliis ad nimiam paupertatem deuenientibus efficiantur reliqui nimis diuites : \\\hline
3.2.34 & Por la qual cosa si el prinçipe gouernar e derechamente el pueblo qual es acomne dado \textbf{ por que la su entençion es enduzir los otros a uirtud . } Et la uirtud faze al que la ha buenon & Quare si principans recte regat populum sibi commissum , \textbf{ quia intentio eius est | inducere alios ad virtutem , } cum virtus faciat habentem bonum ; \\\hline
3.2.34 & non entendiesse \textbf{ que los sus subditos fuessen bueons } e uirtuosos & Si enim Rex non intenderet \textbf{ quod sibi subiecti essent boni et virtuosi , } iam non esset Rex , sed tyrannus . \\\hline
3.2.34 & ata dela desobediençia del prinçipe \textbf{ e del despreçiamiento de los sus mandamientos } es pues que enssennamos al pueblo & quam sit malum , quod consurgit ex inobedientia Principis , \textbf{ et ex praeuaricatio ne mandatorum eius . } Postquam docuimus populum , \\\hline
3.2.35 & El padir . Et la madre . \textbf{ Et generalmente todos los sus parientes . } Et la muger e los fijos e los sus subditos . & Ad Regem autem pertinere videntur quatuor genera personarum \textbf{ videlicet parentes et uniuersaliter omnes cognati , } uxor filii , et subditi . \\\hline
3.2.35 & Et generalmente todos los sus parientes . \textbf{ Et la muger e los fijos e los sus subditos . } Et por ende parte nesçe alos moradores del regno sinon & videlicet parentes et uniuersaliter omnes cognati , \textbf{ uxor filii , et subditi . } Spectat itaque ad habitatores regni \\\hline
3.2.35 & e obedescerle \textbf{ e non fazer tuerto contra los sus parientes } nin contra sus fijos & obedire ei : \textbf{ non forefacere in cognatos eius , } nec in filios , \\\hline
3.2.36 & la qual cosa puede contesçer \textbf{ quando assi se han los sus uiezes } e los sus mjnos ascondidamente e cautelosamente en dar las penas & quod fieri contingit , \textbf{ cum ad eorum iudices } et praepositos latenter et caute se gerunt \\\hline
3.2.36 & quando assi se han los sus uiezes \textbf{ e los sus mjnos ascondidamente e cautelosamente en dar las penas } e en fazer la iustiçia & cum ad eorum iudices \textbf{ et praepositos latenter et caute se gerunt | in punitionibus exequendis , } et in iustitia facienda : \\\hline
3.3.1 & que al prinçipe \textbf{ por que la su doctrina } e el su ensseñamiento pueda aprouechar a todos los sus subditos & vel plura quam Principem , \textbf{ cuius doctrina omnibus prodesse potest . } Inde est igitur \\\hline
3.3.1 & por que la su doctrina \textbf{ e el su ensseñamiento pueda aprouechar a todos los sus subditos } Et pues que assi es por ende enssennando los Reyes & vel plura quam Principem , \textbf{ cuius doctrina omnibus prodesse potest . } Inde est igitur \\\hline
3.3.2 & que los ferreros e los carpenteros son aprouechables a las obras de la batalla \textbf{ por que por la su arte han los braços acostunbrados e apareiados para ferir . } Avn en essa misma manera son aprouechables los carniceros & utiles sunt ad opera bellica : \textbf{ quia ex arte sua habent brachia apta | et assueta ad percutiendum . } Sic etiam utiles sunt Macellarii : \\\hline
3.3.3 & por que conuiene \textbf{ que los mançebos en el comienço de la su moçedat } e de la su maçebia & Est etiam specialis ratio , \textbf{ quare oporteat iuuenes } ab ipsa iuuentute assuescere \\\hline
3.3.3 & que los mançebos en el comienço de la su moçedat \textbf{ e de la su maçebia } se acostunbren a la arte de lidiar & quare oporteat iuuenes \textbf{ ab ipsa iuuentute assuescere } ad artem bellandi : \\\hline
3.3.5 & que puedan sofrir grandes pesos \textbf{ e grandes trabaios continuados en los sus cuerpos } e que puedan sofrir escasseça de viandas & qui possent sustinere magnitudinem ponderis , \textbf{ continuum laborem membrorum , parcitatem victus , } et incommoditatem iacendi et standi , \\\hline
3.3.6 & Avn espantanse los enemigos desto \textbf{ quando veen los sus enemigos } assi por el canpo saltando . & Terrentur etiam ex hoc aduersarii , \textbf{ quando sic vident hostes per saltum incedere . } Rursus , ipse saltus ratione motus facit \\\hline
3.3.9 & Estonçe el cabdiello de la hueste mesurado \textbf{ e en viso segunt que viere la su hueste } ha conplimiento en estas seys condiçiones & Et tunc dux sobrius , \textbf{ et vigilans prout viderit suum exercitum } in his conditionibus abundare , \\\hline
3.3.10 & por la qual los diez uarones lidiadores \textbf{ de los quales era mayor el dean conosçiessen el su dean propreo . } Et por ende en esta manera proprea o avn en otra tan bien en la az de los caualleros & per quod decem bellatores viri , \textbf{ quibus ipse erat praepositus , | decanum proprium agnoscebant . } Hoc itaque modo , \\\hline
3.3.10 & por la qual cosa commo los peones \textbf{ si quisieren ser buenos lidiadores deuan ser fuertes e rezios en los sus cuerpos } e altos en el estado del cuerpo & Quare cum pedites , \textbf{ si debent boni bellatores existere | debeant esse fortes viribus , } proceri statura , \\\hline
3.3.10 & e que aya uso en todas las armas \textbf{ por que pueda ensseñar todos los sus caualleros a la batalla } por que lidien fuertemente e quel alinpien las armas & habere omnium armorum exercitium , \textbf{ ut possit suos commilitones de pugna erudire , } ut fortiter pugnent , arma tergant , \\\hline
3.3.11 & Ca veyendo los periglos de la mar . \textbf{ por que las sus naues non sufran periglo } pintaron e escriuieron la mapa mundi de la mar & qui videntes maris pericula , \textbf{ ne eorum naues patiantur naufragium , } descripserunt maris mappam \\\hline
3.3.12 & Et quales cautelas ha de auer el señor de la batalla \textbf{ por que la su hueste non sea dañada en el camino . } Et este quanto a la batalla del canpo & et quibus cautelis abundare decet bellorum ducem \textbf{ ne suus exercitus laedatur } in via quantum ad campestrum bellum . \\\hline
3.3.14 & en qual manera los deuen acometer \textbf{ Mas quanto los sus negoçios } e los sus fechos son mas ascondidos & qualiter debeant impugnari : \textbf{ quanto vero eorum negocia sunt magis latentia , } magis ignoratur impugnationis modus . \\\hline
3.3.14 & Mas quanto los sus negoçios \textbf{ e los sus fechos son mas ascondidos } menos es sabida la manera de los acometer . & qualiter debeant impugnari : \textbf{ quanto vero eorum negocia sunt magis latentia , } magis ignoratur impugnationis modus . \\\hline
3.3.14 & estonçe los deuen acometer \textbf{ por que la vista de los sus oios derramada } e dañada por el sol & tunc debet eos inuadere , \textbf{ quia oculis eorum disgregatis a sole , } et offensis per ventum et puluerem , \\\hline
3.3.14 & deue el cabdiello \textbf{ con grant acuçia escudriñar las condiçiones de los sus enemigos } en qual manera andan & Septimo debet diligenter \textbf{ explorare conditiones hostium : | qualiter se gerant , } qualiter se habeant , \\\hline
3.3.15 & deue auer el cabdiello de la batalla . \textbf{ La primera es de parte de la su hueste propria . } Ca si el cabdiello ouiere conseio de non lidiar . & debet dux belli duplicem cautelam habere . \textbf{ Prima est , | quantum ad exercitum proprium . } Nam et si dux consilium habeat \\\hline
3.3.15 & e sean muertos fuyendo de sus enemigos \textbf{ e persegriendo los sus enemigos . } Et pues que assi es en tal manera se deue auer el cabdiello & turpiter fugiant , \textbf{ et ab insequentibus hostibus occidantur . } Taliter itaque dux se habere debet , \\\hline
3.3.15 & si fuere menester \textbf{ que los sus contrarios los ayan a fazer foyr . } p paresçe que todas e las batallas se puenden adozir a quatro maneras . & ad quem posset confugere exercitus , \textbf{ si fugaretur ab hostibus . } Videntur omnia bella \\\hline
3.3.19 & por ençima de la tabla faza el muro \textbf{ e si la su vista sobiere mas alta que el muro } de que quiere tomar la alteza alleguesse & et aspiciat per summitatem illius tabulae : \textbf{ et si visus eius proceditur | magis alte quam sit aedificium } cuius est altitudo sumenda , \\\hline
3.3.21 & que son dichas se podran defender los que estan çercados \textbf{ por que las sus fortalezas non sean tomadas } nin vençidas & resistere poterunt obsessi ; \textbf{ ne eorum munitiones } per pugnam ab obsidentibus deuincantur . \\\hline
3.3.23 & que los Reyes e los prinçipes ayan batalla derecha \textbf{ et los sus enemigos turben la paz } e el bien comun a tuerto non es cosa sin guisa & non est inconueniens docere eos omnia genera bellandi , \textbf{ et omnem modum | per quem possint suos hostes vincere , } quod totum ordinare debent \\\hline

\end{tabular}
