\begin{tabular}{|p{1cm}|p{6.5cm}|p{6.5cm}|}

\hline
1.2.11 & ex parte ipsius regni . \textbf{ Regnum enim et omnis politia est quidam ordo , } et quidam principatus . & La segunda manera para prouar esto mismo se toma de parte del regno . \textbf{ Ca el regno e toda comunidat es vna orden e vn prinçipado . } ¶ pues que assi es como la orden \\\hline
1.2.11 & nec ad Principem . \textbf{ Non ergo ulterius reseruaretur in eis politia , } nec esset ulterius regnum . & nin alas leyes nin al prinçipe . \textbf{ Et pues que assi es non se guardaria dende adelante | en ellos comunidat } ni seria dende adelante regno . \\\hline
1.2.11 & sic non quaelibet Iniustitia \textbf{ corrumpit totaliter regnum , et politiam , } tamen per quamlibet Iniustitiam regnum , & Bien assi cada vna mengua de iustiçia \textbf{ non corronpe del todo el regno e la comunidat . } Enpero que por qual quier menguade iustiçia \\\hline
1.2.11 & tamen per quamlibet Iniustitiam regnum , \textbf{ et politia infirmatur , } et disponitur ad corruptionem . & Enpero que por qual quier menguade iustiçia \textbf{ es enfermo el regno } e la comunidat apareiada acorruy conn¶ \\\hline
1.2.27 & homines fierent iniuriatores aliorum , \textbf{ et politia durare non posset . } Nullus igitur debet irasci per odium , & los omes serian fechos jniuriadores e forçadores de los otros . \textbf{ Et la paliçia | que es ordenança dela çibdat non podria durar } ¶ \\\hline
2.1.3 & ut ordinantur ad communitatem , \textbf{ et ad politiam ciuium . } Intendimus ergo ostendere de domo , & en quanto estas cosas son ordenadas ala comunidat \textbf{ e ala polliçia e ordenamiento de los çibdadanos . } Et pues que assi es nos entendemos de determinar dela casa \\\hline
2.1.14 & ab ipso regnante et principante , \textbf{ sed magis ab ipsa politia et ab ipsis ciuibus . } Dicitur ergo tale regimen politicum vel ciuile . & nin del prinçipante . \textbf{ Mas deue se nonbrar dela çibdat | e de los lus çibdadanos } e es dicho tal gouernamiento politico e çiuil . \\\hline
2.3.13 & ut superius diffusius probabatur , \textbf{ numquam ex pluribus hominibus fieret naturaliter una societas vel una politia , } nisi naturale esset & assi commo es prouado de suso \textbf{ mas conplidamente nunca de mucho omes se faria naturalmente | vna conpannia o vna poliçia } si naturalmente non fuesse a ellos conuenible que algunos fuessen sennores \\\hline
2.3.13 & et corpus obedit : \textbf{ sic in politia bene ordinata sapientes debent dominari , } et insipientes obedire : & e el cuerpo obedesçe assi en la poliçia \textbf{ e enla çibdat bien ordenadlos sabios deuen enssennorear } e los non si bios deuen obedesçer \\\hline
3.1.6 & Secundo ostendetur , \textbf{ quae sit optima politia siue optimum regnum , } et quibus cautelis uti debeant principantes , & en qual manera de una gouernar las çibdades e los regnos ¶ \textbf{ Lo segundo mostraremos qual es la muy buean politica o çibdat o muy vuen regno } e de quales cautelas deuen usar los prinçipes e los . Reyes \\\hline
3.1.9 & ne circa ipsa contingat error . \textbf{ Quare cum in regimine ciuitatis primo sit politia ordinanda , } diu inuestigandum est , & escodrinnar por que çerca ellos non contezca yerro . \textbf{ por la qual cosa commo en el gouernamiento dela çibdat | primeramente se ha de ordenar la poliçia } muy luengamente es de buscar \\\hline
3.1.16 & quod inter caetera quae debet \textbf{ intendere legislator et rector politiae , } est , & que entre todas las cosas \textbf{ aque deue parar mientes el fazedor della es } en qual manera los çibdadanos de una auer las possesipnes igualadas \\\hline
3.1.16 & ex his quae videbat in ciuitatibus aliis : \textbf{ nam in multis politiis bene ordinatis } magna cura fuit legislatoribus de possessionibus ciuium : & que el veya enlas otras çibdades \textbf{ ca veya en muchͣs çibdades bien ordenadas } que auyan grant cura los fazedores dela ley delas possessiones delos çibdadanos . \\\hline
3.1.16 & ideo Phalas forte motus \textbf{ ex his quae videbat in politiis aliis , } statuit potissime curandum esse de possessionibus ciuium , & Et por ende felleas por auentura fue mouido por estas cosas \textbf{ que veya en las otrå sçibdades . } Et por ende establesçio \\\hline
3.1.19 & diuersa genera personarum . \textbf{ Hippodamus autem statuens suam politiam , } primo intromisit se de multitudine & que tannian alguons linages de personas dezimos \textbf{ que y podo mio | establesciendo su poliçia } primero se entremetio dela muchedunbre \\\hline
3.1.20 & hoc posito artifices , \textbf{ et agricolae non haberent partem in politia ; } et bellatores non permitterent eos participare & Puesto que el dizie siguese que los menestrales \textbf{ e los labradores non los } dexanien auer parte en la elecçion del prinçipe . \\\hline
3.1.20 & Unde et Philosophus innuit , \textbf{ quod in quibusdam bonis politiis econtrario statuitur } quam ordinauerit Hippodamus : & Onde el phoda a entender \textbf{ que en algunos buenos ordenamientos de çibdat | el contra no es establesçido } delo que ordeno ypodomio \\\hline
3.2.1 & praemittendo quaedam praeambula ad propositum , \textbf{ et recitando opinionem diuersorum Philosophorum instituentium politiam , } et tradentium artem & ante pomiendo alguons preanbulos al nuestro proponimiento \textbf{ e rezando opiniones de departidos philosofos | que establesçieron poliçias } e dieron arte del gouernamiento dela çibdar \\\hline
3.2.2 & vocat eum Philosophus nomine communi , \textbf{ et dicit ipsum esse Politiam . } Politia enim quasi idem est , & llamalle el philosofo nonbre comun \textbf{ e diz el poliçia } por que poliçia es \\\hline
3.2.2 & eo quod non habeat commune nomen , \textbf{ Politia dicitur . } Nos autem talem principatum appellare possumus gubernationem populi , & Enpero el prinçipado del pueblo si derecho es \textbf{ por que non ha nonbre comun es dich poliçia } e nos podemos llamar atal prinçipado gouernamiento del pueblo \\\hline
3.2.3 & Quia intendimus ostendere \textbf{ quae sit optima politia , } et quis sit optimus principatus . & ir que entendemos mostrar \textbf{ qual es la muy buena poliçia } e que cosa es el muy buen prinçipado \\\hline
3.2.3 & propter abundantiorem potentiam \textbf{ melius posset politiam gubernare . } Tertia via sumitur & Et aquel prinçipe por ma . \textbf{ yor cunplimiento de poderio meior podria gouernar la çibdat } que muchos ¶ \\\hline
3.2.14 & ut corrumpatur principatus ille . \textbf{ Politia ergo quanto de se magis a iustitia recedit , } tanto ex se habet & por que sea corrun pido el su sennorio \textbf{ pues que assi es quanto el gouernamiento | mas se arriedra dela iustiçia } tanto ha dessi \\\hline
3.2.15 & iis \textbf{ qui sunt in politia : } nam corruptiones longe secundum rem , & La terçera cosa que guarda al gouernamiento del regno es meter mie \textbf{ do aquellos que son enla çibdat } e en el regno \\\hline
3.2.15 & et antecessores sui obtinuerunt huiusmodi principatum , tanta cautela non magnam utilitatem habere videtur . \textbf{ Quartum autem quod politiam saluare videtur , } est cauere seditiones et contentiones nobilium ; & que dichͣes non paresçe \textbf{ que puede ser muy prouechosa la quarta cosa | que salua la poliçia } es escusar las discordias \\\hline
3.2.15 & Nihil enim adeo regnum conseruat \textbf{ et politiam saluat , } sicut praeficere homines bonos et virtuosos , & ca ninguna cosa non salua tanto el regno \textbf{ e la poliçia } commo poner los bueons e los uirtuosos en las dignidades \\\hline
3.2.15 & et conferre eis dominia et principatus . \textbf{ Quare maxime saluatiuum politiae est , } regiam maiestatem considerare diligenter & e dar les los señorios e los prinçipados . \textbf{ por la qual | cosalo que mucho salua la poliçia } es que el Rey piensse con grant acuçia \\\hline
3.2.15 & et amorem ad bonum regni , \textbf{ et ad politiam , } in qua principatur . & aya grant amor al bien del regno \textbf{ e al bien dela çibdat } en quien \\\hline
3.2.15 & et periculis imminentibus obuiare . \textbf{ Octauum saluans regnum et politiam , } est habere ciuilem potentiam . & e pueda contradezir alos peligros \textbf{ que pueden acaesçer¶ La . viijn . | cosa que salua el regno } e la poliçia es auer poderio \\\hline
3.2.15 & sic eos bonitate superet : \textbf{ hoc enim maxime saluabit regnum et politiam , } si Rex sit bonus et virtuosus , & assi lieue aun ataia en bondat \textbf{ e esto es lo que much salua el regno | e la poliçia } si el Rey fuere bueno e uirtuoso \\\hline
3.2.19 & quod esset bonus virtuosus \textbf{ et politiam diligeret , } existentes in regno promoueret et honoraret : & tal que sea bueno e uirtuoso \textbf{ e ame la comunidat } et ꝓmueua los q̃ son eñl su regno e los hõ rre . \\\hline
3.2.26 & quod non oportet \textbf{ adaptare politias legibus , } sed leges politiae , & que non conuiene de apropar las comunidades \textbf{ delas çibdades alas leyes . } Mas las leyes alas comunidades \\\hline
3.2.26 & adaptare politias legibus , \textbf{ sed leges politiae , } quas leges oportet diuersas esse & delas çibdades alas leyes . \textbf{ Mas las leyes alas comunidades } de las çibdades las quales leyes conuiene de ser departidas \\\hline
3.2.26 & quas leges oportet diuersas esse \textbf{ secundum diuersitatem politiarum . } Volens ergo leges ferre , & de las çibdades las quales leyes conuiene de ser departidas \textbf{ segunt el departimiento delas comunidades . } Et pues que assi es el que quisiere poner leyes \\\hline
3.2.29 & Adducit autem rationes duas , \textbf{ quod melius sit politiam regni Regi optima lege , } quam optimo Rege . & Mas para esto prouar aduze dos razones \textbf{ que meior es de ser gouernado el regno | por muy buena ley } que por muy buen Rey ¶ \\\hline
3.2.33 & ex quibus sumi possunt quatuor viae , \textbf{ ostendentes meliorem esse politiam , } vel melius esse regnum et ciuitatem , & delas quales se pueden tomar quatro razonnes \textbf{ que muestran que meior es la poliçia } o meior es el regno o la çibdat \\\hline
3.2.35 & et ad obseruandum \textbf{ ea quae requirit politia , } vel regimen regni , & e aguardar aquellas cosas \textbf{ que demanda la poliçia } o el gouernamiento del regno \\\hline
3.2.36 & timentur Reges et Principes ; \textbf{ si in eos qui ultra modum regnum et politiam per turbant , } inexquisitas crudelitates exerceant . & e los prinçipes \textbf{ si mostraren grandescrueldades } e dieren grandes penas \\\hline

\end{tabular}
