\begin{tabular}{|p{1cm}|p{6.5cm}|p{6.5cm}|}

\hline
1.1.1 & sed est de negociis singularibus : \textbf{ quae ( ut declarari habet 2 Ethicorum ) propter sui variabilitatem , } magnam incertitudinem habent . & los quales negoçios segund \textbf{ que demuestra Elphon | enel segundo libro delas ethicas non pueden aver certidunbre de Razon } por la mudaçion \\\hline
1.1.1 & Finis ergo intentus in hac scientia , \textbf{ non est sui negocii cognitio , sed opus : } nec est veritas , sed bonum . & E por ende la fin que se entiende en esta sçiençia \textbf{ non es conosçimjento mas obra njn es | por graçia de buscar verdad delas cosas } mas \\\hline
1.1.1 & et qualiter debeant \textbf{ suis subditis imperare , } oportet doctrinam hanc extendere usque ad populum , & De saber fazer ¶ \textbf{ E si por qual manera Deuen mandar a los sus Subditos } conujene esta doctrina \\\hline
1.1.1 & ut sciat qualiter debeat \textbf{ suis Principibus obedire . } Et quia hoc fieri non potest & e esta sçiençia estender la fasta el pueblo \textbf{ por que Sepa commo ha de obedesçer a sus prinçipes } E por que esto non puede ser \\\hline
1.1.2 & In secundo vero manifestabitur , \textbf{ quomodo debeat suam familiam gubernare . } In tertio autem declarabitur , & E enel segundo mostraremos \textbf{ commo deue el Rey | e Cada vno delos otros gouernar su conpaña } ¶E enel terçero declaremos \\\hline
1.1.2 & quare rationabile est , \textbf{ ut prius determinetur de regimine sui , } quam de regimine familiae , siue regni . & Por la qual cosa de rrazon es \textbf{ que primeramente digamos del gouernamjento de si mism̊ } que del gouernamjento dela conpaña del regno¶ \\\hline
1.1.2 & primo scire se ipsum regere , \textbf{ secundo scire suam familiam gubernare , } tertio scire regere regnum , et ciuitatem . In primo autem libro in quo agetur de regimine sui , & Lo terçero que sepa \textbf{ gouernả su rregno | e sus çibdades ¶ } pues que asy es en el primo libro \\\hline
1.1.2 & secundo scire suam familiam gubernare , \textbf{ tertio scire regere regnum , et ciuitatem . In primo autem libro in quo agetur de regimine sui , } sunt quatuor declaranda . & e sus çibdades ¶ \textbf{ pues que asy es en el primo libro | en el qual tractaremos del gouerna mjeto del omne . } En sy mesmo son quatro cosas de declarar e de demostrͣ \\\hline
1.1.2 & sunt quatuor declaranda . \textbf{ Nam Primo ostendetur in quo regia maiestas debeat suum finem , } et suam felicitatem ponere . & En sy mesmo son quatro cosas de declarar e de demostrͣ \textbf{ Ca primamente demostrͣemos | en que deue la Real magestado el rrey } pon su fin e su bienandança¶ \\\hline
1.1.2 & Nam Primo ostendetur in quo regia maiestas debeat suum finem , \textbf{ et suam felicitatem ponere . } Secundo quas virtutes debeat habere , & en que deue la Real magestado el rrey \textbf{ pon su fin e su bienandança¶ } Lo segundo demostrͣemos quales uertudes deue auer el Rey e el gouernador ¶ \\\hline
1.1.2 & et bonis operibus regulatis ordine rationis : \textbf{ volens tractare de regimine sui , } oportet ipsum notitiam tradere de omnibus his & por orden de Razon \textbf{ el que quiere tractar del gouernaiento } e fablarde sy mesmo conujene de tractar \\\hline
1.1.2 & Senes enim \textbf{ ( ut suo loco ostendetur ) } sunt naturaliter increduli , et auari : & que los que han costunbres de moços \textbf{ Ca los vieios asy commo se mostrͣa en su logar } son natraalmente mal creyentes e auarientos \\\hline
1.1.2 & ut finem sibi praestituant \textbf{ conformem suo habitui . } In primo ergo libro de omnibus his quatuor tractabimus , & por ellos a escoger otros fines concordables \textbf{ alas sus dipo inconnes o alos sus desseos del alma¶ } pues que asy es en el primero libro tractaremos destas quatro cosas \\\hline
1.1.5 & per debitas transmutationes \textbf{ consequitur suam perfectionem et formam , } sic homo per rectas & por sus conueientes e ordenadas \textbf{ t̃ns muta connes viene a rresçebir su forma | e su perfecçion } asi el omne por derechas \\\hline
1.1.5 & et debitas operationes \textbf{ consequitur suam perfectionem et felicitatem . } Cum ergo nunquam contingat recte agere , & e conueni entes obras \textbf{ viene a auer su perfecçion | e su bien andança acabada¶ } pues que asy es com̃ nunca pueda omne bien \\\hline
1.1.5 & expedit volenti \textbf{ consequi suum finem , } vel suam felicitatem , & Conviene a todo omne \textbf{ que quiera alcançar e auer su fin } e la su bien andança de auer \\\hline
1.1.5 & consequi suum finem , \textbf{ vel suam felicitatem , } habere praecognitionem ipsius finis . & que quiera alcançar e auer su fin \textbf{ e la su bien andança de auer } ante algun conosçimjento dela su fin e dela su bien andança . \\\hline
1.1.5 & duplici via venari possumus , \textbf{ quod expedit regi suum finem cognoscere . } Prima est , & e cobrar prouar \textbf{ que conujene al rrey | en toda manera de conosçer la su fin ¶ } La primera rrazon es en quanto el rrey \\\hline
1.1.5 & Prima est , \textbf{ inquantum per sua opera cooperatur , } ut sit finis consecutiuus . & La primera rrazon es en quanto el rrey \textbf{ por las sus obras ayuda asy mesmo } por que aya e alçen la su fin \\\hline
1.1.5 & Nam ad hoc quod aliquis \textbf{ per suas operationes } finem consequatur , & Ca para que cada vno \textbf{ por las sus obras } alcançe la su fin \\\hline
1.1.5 & et delectabiliter , \textbf{ expedit suam felicitatem praecognoscere : } sed maxime hoc expedit regiae maiestati , & connosçerlo su fin e la su bien andança . \textbf{ por que pueda obrar bien e de uoluntad e delectosamente . } Et sy espero conuiene a cada vno mucho \\\hline
1.1.5 & sed maxime hoc expedit regiae maiestati , \textbf{ quia in operibus suis debet } intendere bonum gentis et commune , & mas conuiene al Rey o al prinçipe . \textbf{ por que entondas sus obras } deue entender al bien dela gente \\\hline
1.1.5 & quod maxime decet regiam maiestatem \textbf{ cognoscere suam felicitatem , } ut opera communia , & que muy mas conuiene al Reio al prinçipe conosçer la su fin \textbf{ e la su bien andança | que a otro ninguno . } por que pueda fazer buenas obras e comunes \\\hline
1.1.6 & quando assecutus est id , \textbf{ in quo consistit suum perfectum bonum . } Unde Philosophus 1 Ethicorum & Ca estonçe dezimos \textbf{ que el omne es bien auenturado | quando el alcança aquello en que esta el bien acabado } ¶onde dize el philosofo \\\hline
1.1.6 & et sicut imperfectum \textbf{ ad suam perfectionem , } bona corporis sunt imperfecta & e menguada es ordenada alaꝑfecçion \textbf{ e al su conplimiento . } asi los bienes del cuerpo deuen ser ordenados \\\hline
1.1.6 & quod non decet aliquem hominem \textbf{ suam felicitatem ponere } in delectationibus sensibilibus . & qua non conuiene a ningun omne \textbf{ de poner la feliçidat suia | e la su bien andança } enlas delectaçiones sensibles e dela carne ¶ \\\hline
1.1.6 & Est ergo detestabile cuilibet Homini \textbf{ ponere suam felicitatem in voluptatibus . } Sed maxime hoc est detestabile Regiae maiestati : & ̉ qual quier omne \textbf{ que toda su bien andança pone en delecta connes dela carne } mas mucho mas es de denostar el Rey \\\hline
1.1.7 & Cuilibet ergo Homini detestabile est \textbf{ ponere suam felicitatem in diuitiis , } sed maxime detestabile est regiae maiestati . & ¶pres que assi es mucho es de denostar todo en que pone su feliçidat \textbf{ e su bien andança en las riquezas corporales . } Mas mayor mente es de denostar la Real magestad \\\hline
1.1.7 & sed maxime detestabile est regiae maiestati . \textbf{ Nam si Rex aut Princeps ponat suam felicitatem in diuitiis , } tria maxima mala inde consequuntur . & si en ellas la pone . \textbf{ Ca si el rey o el prinçipe | pone su } feliçadato su bien andança en las riquezas corporales . \\\hline
1.1.7 & et in opinione ponentis \textbf{ suam felicitatem in diuitiis , } diuitiae sunt quid magnum , & e en la opinion de aquel \textbf{ que pone su bien andança en las riquezas . } las riquezas son grant cosa e grant bien . \\\hline
1.1.7 & detestabile est ei \textbf{ suam felicitatem } in talibus ponere . & Et mucho seria de denostar \textbf{ si la su bien andança pusiese en estas riquezas corporales } ¶ \\\hline
1.1.7 & Secundo detestabile est Regi , \textbf{ vel Principi suam felicitatem ponere in diuitiis , } quia hoc facto Tyrannus efficitur . & Lo segundo se declara \textbf{ assi mucho es de denostar el Rei o el prinçipe | que pone su bien andança en las riquezas corporales . } ¶ Ca por esto se fare tirano \\\hline
1.1.7 & Cum ergo finis maxime diligatur , \textbf{ ponens suam felicitatem in numismate , } principaliter intendit reseruare sibi , & a aquel que pone las un feliçidat \textbf{ e la su bien andança en las riquezas | e en los aueres } prinçipalmente entiende de thesaurizar e fazer thesoro e llegar muchos dineros \\\hline
1.1.7 & quo potest , \textbf{ consequi finem suum . } Ponens igitur suam felicitatem in diuitiis , & que pudiere \textbf{ por que pueda alcançar aquella fin | e aquel bien ¶Donde se sigue } que el prinçipe \\\hline
1.1.7 & consequi finem suum . \textbf{ Ponens igitur suam felicitatem in diuitiis , } non erit sibi curae , & que el prinçipe \textbf{ que pone su feliçidat | e su bien andança en las riquezas corporales } non aura cuydado ninguno \\\hline
1.1.7 & omni via qua potest , \textbf{ velle consequi suum finem . } Est igitur Rex Tyrannus , & por ninguna manera \textbf{ que non pueda querer seguir la su fin | ante se trabaia dela alcançar quanto puede ¶ } Pues que assi es el Rei es tirano \\\hline
1.1.7 & et depraedatorem detestabile \textbf{ quoque est suam felicitatem } in diuitiis ponere . & de poner el Rey la su feliçidat \textbf{ e la su bien andança en las riquezas corporales . } ora uentra a muchos biuen uida politica \\\hline
1.1.8 & non in eo cui inclinatio exhibetur . \textbf{ Accidens enim proprie est in suo subiecto , } non autem in obiecto : & e non en aquel a quien se inclina \textbf{ Ca propiamente el accidente es en el su sƀiecto } e non en otro ninguno \\\hline
1.1.8 & Indecens est ergo cuilibet homini \textbf{ ponere suam felicitatem in honoribus , } ut credat se esse felicem , & es que ningun omne . \textbf{ ponga su bien andança en las honrras } assi que crea que es bien andante \\\hline
1.1.8 & quod etiam triplici via venari potest . \textbf{ Si enim Rex suam felicitatem in honoribus ponat , } sequentur ipsum tria mala : & Et ahun esto podemos prouar por tres razones . \textbf{ Ca si el Rey pone su bienandança en las honrras } siguen se le tres males ¶ \\\hline
1.1.8 & erit praesumptuosus , et erit iniustus , et inaequale . \textbf{ Nam si Princeps suam felicitatem in honoribus ponat , } cum sufficiat ad hoc & ¶Lo primero se muestra assi . \textbf{ Ca el prinçipe | si pusiere la su bien andança en honrras } conmoabaste acanda vno \\\hline
1.1.8 & Secundo indecens est Regi , \textbf{ ponere suam felicitatem in honoribus , } quia ex hoc efficietur periclitator Populi , et praesumptuosus : & assi que muy desconueible cosa es al Rey \textbf{ poner su bien andança en las honrras | Ca por esso seria prisuptuoso } e sob̃uio \\\hline
1.1.8 & nam cum finis maxime diligatur , \textbf{ si Princeps suam felicitatem in honoribus ponat , } ut possit honorem consequi , & Ca commo cada vn omne mucho ame la su fin \textbf{ en que pone la su bien andança | si el prinçipe pusiere la su bien andança enlas honrras } por que pueda delo que feziere honrra alcançar \\\hline
1.1.8 & ut possit honorem consequi , \textbf{ praesumet suam gentem exponere omni periculo . } Exemplum huiusmodi habemus & por que pueda delo que feziere honrra alcançar \textbf{ presumira de poner los pueblos a todo peligro } por que pueda alcançar aquella honrra¶ \\\hline
1.1.8 & non expedit ei \textbf{ suam felicitatem in honoribus ponere . } Tertio hoc est indicens ei , & non presuma \textbf{ nin se ensoƀuezca mucho nol conuiene de poner su bien andança en las honrras ¶ } Lo terçero se demuestra \\\hline
1.1.8 & decet enim Principem \textbf{ sua bona distribuere } secundum dignitatem personarum , & Mas conuiene le partir los sus bienes \textbf{ alos sus vassallos } segunt las dignidades delas personas ¶ \\\hline
1.1.8 & ut plura bona decet dignis , et sapientibus , quam indignis , et Histrionibus . \textbf{ Sed si Princeps suam felicitatem in honoribus ponat , } quia finem summo ardore diligit , & nin alos iuglares ¶ \textbf{ Mas si el prinçipe | pusiere su bien andaça en las honrria } por que la fin e la bien andança es muy amada \\\hline
1.1.8 & quo plus honoris consequi possit . \textbf{ Si ergo Rex suam felicitatem in honoribus ponat , } erit malus in se , & Et por ende ençerrado todo lo \textbf{ que dicho es en este capitulo | si el Rei pusiere la su feliçidat } e la su bienandança en las honrras sera malo en si mesmo \\\hline
1.1.8 & et praecipitanter exponere : \textbf{ erit malus in suis rebus , } quia eas non distribuet aequaliter & Ca non fara fuerça de poner el pueblo a grandes peligros presuptuosamente e arrebatadamente . \textbf{ Et sera malo en partir sus aueres . } Ca non los partir a egualmente \\\hline
1.1.9 & Videtur ergo quod maxime Princeps \textbf{ in hoc suam felicitatem ponere debeat , } dicente Philosopho 5 Ethic’ & que los prinçipes deuen poner mayormente la su feliçidat \textbf{ e la su bien andança en la eglesia | e en la } honrra¶por que dize el philosofo \\\hline
1.1.9 & bonitas ergo nostra per se dependet a notitia Dei , \textbf{ tanquam effectus a sua causa . } Rursus circa bonitatem nostram notitia Dei non fallit , & Pues que assi es lanr̃a bondat desçende derechamente del conosçimiento de dios \textbf{ assi commo obra de su obrador | e assi commo cosa fechan de su fazedor } ¶Otro si el \\\hline
1.1.9 & Rursus circa bonitatem nostram notitia Dei non fallit , \textbf{ cum scientia sua falli non possit . } Amplius bonitatem , & conosçimientode dios no puede ser engannado en lanr̃a bondat . \textbf{ Ca lascian de dios non puede resçebir enganno . | Et ahun dezimos mas adelante } que dios mas claramente vee \\\hline
1.1.9 & nisi sit bonus , et beatus , \textbf{ ut exigit status suus . } Licet ergo sic sit de notitia Dei , & si non fuere bueno en uerdat e de fecho \textbf{ assi commo demanda el su estado ¶ } Et pues que assi es maguera \\\hline
1.1.9 & Non est intelligendus textus Philosophi , \textbf{ quod Reges principaliter pro suo merito quaerere debeant gloriam , } et famam Hominum , & assi entender \textbf{ que los Reys prinçipalmente por su meresçimiento deuen demandar } e quere reglesia e fama de los omes . \\\hline
1.1.10 & non decet principem \textbf{ suam felicitatem ponere in ciuili potentia . } Quinto hoc non decet ipsum , & Non conuiene alos prinçipes \textbf{ poner su bien andança en el poderio | çiuil¶ } La quinta razon por que non conuiene al prinçipe \\\hline
1.1.10 & Nam cum felicitas sit finis omnium operatorum , \textbf{ quilibet totam vitam suam , } et omnia opera sua & sea fin de todas las nuestras obras \textbf{ ¶ Cada vno porna toda su uida } e ordenada todas sus obras \\\hline
1.1.10 & ad illud \textbf{ in quo suam felicitatem ponit : } ponens ergo suam felicitatem & en las mas cosas ha aquello en que pone toda su feliçidat \textbf{ e toda su bien andança . } Et por ende aquel que pone su feliçidat \\\hline
1.1.10 & in quo suam felicitatem ponit : \textbf{ ponens ergo suam felicitatem } in ciuili potentia , & e toda su bien andança . \textbf{ Et por ende aquel que pone su feliçidat | e su bien andança } en poderio çiuil \\\hline
1.1.10 & inconueniens est etiam Principem \textbf{ ponere suam felicitatem } in ciuili potentia , & que non es cosa conuenible \textbf{ que el prinçipe ponga su bien andança en poderio çiuil . } Et por que cuyde que es bien \\\hline
1.1.10 & Quod non deceat Regiam maiestatem \textbf{ suam felicitatem ponere in robore corporali , } vel in pulchritudine , & quando pudiere subiugar \textbf{ assi las naçiones . } et las gentes . \\\hline
1.1.11 & videlicet , sanitatem , pulchritudinem , et robur . \textbf{ Immo nonnulli in talibus suam felicitatem ponunt . } Videtur enim omnino esse contrarium & salud e fermosura e fuerça . \textbf{ ¶ Mas alguons fueron | e son que ponen su bien andança en tales cosas } Mas deuedes saber \\\hline
1.1.11 & et sit pulcher in anima , \textbf{ tunc ( ut exigit suus status ) } credat se esse felicem . & sea fermoso en el alma . \textbf{ Estonçe crea el } que es bien auentraado segunt su estado \\\hline
1.1.11 & Dicimus autem \textbf{ ( ut exigit suus status ) } quia plena felicitas in hac vita haberi non potest . & Et dezimos segunt \textbf{ que requiere su estado | por que la feliçidat } e la bien andança conplida \\\hline
1.1.11 & nec aliquem hominem in talibus \textbf{ suam felicitatem ponere , } quae sunt corporalia , & nin a ningun omnen poner la su feliçidat \textbf{ e la su bien andança en tales cosas } por que son corporales \\\hline
1.1.12 & quomodo deceat regiam maiestatem \textbf{ ponere suam felicitatem } in actu prudentiae , & en qual manera conuenga ala Real magestad \textbf{ de poner la primera feliçidat } en las obras de pradençia . \\\hline
1.1.12 & sciendum quod decet Regem maxime \textbf{ suam felicitatem } ponere in ipso Deo , & ¶ Et la segunda commo le conuiene \textbf{ de poner er la su bien andança solamente en dios . } Esto pondemos prouar por tres razones ¶ \\\hline
1.1.12 & et rationem participat , \textbf{ ponere suam felicitatem } in bono maxime uniuersali , & e ha razon e entendimiento \textbf{ de poner la su bien andança en bien muy comun } e muy entelligible \\\hline
1.1.12 & Secundo decet Principem \textbf{ suam felicitatem } ponere in ipso Deo , & e muy alongado de toda materia¶ \textbf{ La segunda razon por que el rey ha de poner la su bien andaça } en dios solo es esta . \\\hline
1.1.12 & Quare si minister , \textbf{ suam mercedem , } et suum praemium debet & e los seruientes del señor \textbf{ deuen poner la su merçed } e el su \\\hline
1.1.12 & suam mercedem , \textbf{ et suum praemium debet } ponere in suo Domino , & deuen poner la su merçed \textbf{ e el su } gualardon en el su señor \\\hline
1.1.12 & et suum praemium debet \textbf{ ponere in suo Domino , } et debet eam expectare ab ipso , & e el su \textbf{ gualardon en el su señor } e deuen la esparar del . \\\hline
1.1.12 & qui est Dei minister , \textbf{ suam felicitatem ponere in ipso Deo , } et suum praemium expectare ab ipso . & que es ofiçial de dios \textbf{ poner la su bien andança en dios que es prinçipal señor } e del solo deue esperar \\\hline
1.1.12 & suam felicitatem ponere in ipso Deo , \textbf{ et suum praemium expectare ab ipso . } Tertio hoc decet Regem , ex eo , & poner la su bien andança en dios que es prinçipal señor \textbf{ e del solo deue esperar | gualardon e merçed } ¶La terçera razon por que el Rey ha de poner su bien andança en dios \\\hline
1.1.12 & In eo ergo debet \textbf{ suam felicitatem ponere , } quod est maxime , & mientesal bien comun de todos . \textbf{ Et por ende deue poner la su feliçidat } e la su bien andança \\\hline
1.1.12 & et tum quia intendit bonum commune , \textbf{ debet suam felicitatem ponere in Deo , } cui seruit , & Et lo otro por que deue te çier mientes al bien comun \textbf{ deue pener la su feliçidat | e la su bien andança en dios } a quien deue seruir . \\\hline
1.1.12 & Si ergo Rex debet in Deo \textbf{ ponere suam felicitatem , } oportet ipsum huiusmodi felicitatem ponere & ¶ Et pues que el Rey deue poner la su feliçidat \textbf{ e la su bien andança en dios . } Conuiene le dela poner en la obra de aquella uirtud \\\hline
1.1.13 & Immo eo ipso quod Rex studet per legem , \textbf{ et prouidentiam suum regnum regere , } quomodo Deus totum uniuersum regit et gubernat , & que el es Rey deue estudiar \textbf{ por que gouierne su regno } por ley e por sabiduria bien commo dios gouienna todo el mundo . \\\hline
1.1.13 & est magna merces Regis , \textbf{ si regnum suum recte regat , quia ex hoc ipsi Deo maxime conformatur . } Secundo , magnum est praemium Regis , & gualardon grande es el su \textbf{ gualardon si gouernare derechamente su regno . | Ca por esto se conforma mucho con dios } ¶ \\\hline
1.1.13 & quia indiscrete agit , \textbf{ suum meritum non augmentatur . } Tertio , magnum est meritum Principis , & enlo que han de fazer \textbf{ e non acresçentarian en su meresçimiento . } ¶ La terçera razon por que es grande el meresçimiento de lons Reyes \\\hline
1.1.13 & ad quem spectat regere non solum se , \textbf{ et suam familiam , } sed etiam totum regnum . & non solamente assi mesmo \textbf{ e asu conpanna } mas ahun a todo el regno . \\\hline
1.1.13 & magnum erit meritum \textbf{ bene regentium regnum suum . } Quinto , magnum erit praemium ipsorum Regum , & e gm̃t gualardon gerad sera el meresçimiento e el gualardon de los Reyes . \textbf{ quando bien gouernar en sus regnos ¶ } La quinta razonn por que es grand el meresçimien todelons reys \\\hline
1.2.1 & primam partem huius primi libri , \textbf{ in quo agitur de regimine sui , } ostendentes in quo Reges et Principes & ues que ya con el ayuda de dios acabamos la primera parte deste libro primero \textbf{ en que se tracta del gouernamiento del omnen en ssi . } Et mostramos en que deuen poner los Reyes \\\hline
1.2.1 & ostendentes in quo Reges et Principes \textbf{ suam felicitatem debeant ponere , } quia non decet & e los prinçipes la su feliçidat \textbf{ e la su bien andança . } Et que non los conuiene poner la su fin en riquezas \\\hline
1.2.1 & quia non decet \textbf{ eos suum finem ponere in diuitiis , } nec in ciuili potentia , & e la su bien andança . \textbf{ Et que non los conuiene poner la su fin en riquezas } nin en poderio çiuil \\\hline
1.2.1 & debent uti tanquam organis ad felicitatem . \textbf{ Suam autem felicitatem ponere debent } in actu prudentiae , & para ganar la feliçidat e la bien andança . \textbf{ Mas la su bien andança deuen poner en obras de pradençia e de sabiduria } segund que tales obras son regladas \\\hline
1.2.1 & imperatus a charitate : \textbf{ nam tunc Reges habent felicitatem suo statui debitam , } et condignam , & e por el amor de dios \textbf{ Ca estonçe han los Reyes | e los prinçipes la su feliçidait } e la su feliçidat bien andança qual deuen auer \\\hline
1.2.1 & Principaliter ergo Regum felicitas ponenda est in ipso Deo , \textbf{ et ex cognitione et dilectione eius studium suum , } et vitam suam & e de los prinçipes es de poner en solo dios . \textbf{ Ca deuen ellos ordenar la su uida } e el su estado a esto \\\hline
1.2.1 & et ostenso in quo Reges ponere debeant \textbf{ suum finem } secundum ordinem superius praetaxatum , & cosahan los Reyes \textbf{ e los prinçipes de poner su fin } segund la orden ya dicha \\\hline
1.2.1 & potentiae naturales variantur \textbf{ in actibus suis , } sed semper secundum modum sibi possibilem & que los poderios del alma sean mouidos \textbf{ nin desuariados en sus obras } Mas segunt su manera \\\hline
1.2.1 & sed semper secundum modum sibi possibilem \textbf{ suas actiones efficiunt . } Tertio , in talibus potentiis & Mas segunt su manera \textbf{ e su poder fazen sienpre sus obras ¶ } La terçera razon por que enlos poderios naturales \\\hline
1.2.1 & vel ex superfluo cibo , \textbf{ nam erat in potestate sua , } ut posset uti potu , & por que comio mucho e beuio mucho . \textbf{ Ca en su poderio era de vsar } tenpradamente del comer e del beuer ¶ \\\hline
1.2.1 & sufficienter determinantur \textbf{ ad actiones suas per suam naturam : } sic et sensus . & senssiblessor determinados a sus oƀras \textbf{ por su naturaleza . } Ca assi commo el fuego tanto escalienta quato puede escalentar \\\hline
1.2.1 & propter quod , \textbf{ quia determinatus est in actione sua , } nec laudatur , nec vituperatur , & Ca assi commo el fuego tanto escalienta quato puede escalentar \textbf{ por que es determimado en la su obra } por esso nies de loar \\\hline
1.2.2 & prudentes tamen esse non possunt , \textbf{ ut suo loco patebit . } Inde est ergo quod dicitur 6 Ethic’ & enpero non pueden ser pradentes ni sabios \textbf{ assi commo lo prouaremos en su logar ¶ } Et por ende dize el philosofo \\\hline
1.2.2 & Nam cum intellectus uniuersaliori modo \textbf{ respiciat suum obiectum quam sensus , } appetitus intellectiuus , & Ca commo el entendimiento sea mas general que el seso . \textbf{ Mas generalmente cata aquello en que ha de obrar que el seso . } ¶ El appetito del entendimiento \\\hline
1.2.3 & vel quod ei detur \textbf{ quod suum est . } Si autem moderat , & o lo quel deuen dar \textbf{ o lo que es suyo . } Mas si mesura enderesça las passiones \\\hline
1.2.4 & de quibus in praecedenti capitulo fecimus mentionem . \textbf{ De omnibus ergo his quatuor suo loco dicemus . } Determinabimus ergo de uirtutibus , & Et las otras de que fiziemos mençion en el capitu lo ant̃ dich̃o ¶ \textbf{ pues que assi es de todas estas quatro disposiconnes diremos } de cada vna en su logar . Ca determinaremos delas uirtudes mostrando . \\\hline
1.2.7 & Secundo studere debet , \textbf{ ne suus principatus in tyrannidem conuertatur . } Tertio studere debet , & Lo segundo deue estudiar el Rey \textbf{ que el su prinçipadgo | e el su sennorio non se torne en tirania } que es señorio malo e desigual \\\hline
1.2.7 & Est enim Regis officium , \textbf{ ut suam gentem regat , } et dirigat in debitum finem . & e de dignidat \textbf{ por que el ofiçio del Rey es que gouierne e guie la su gente . } la qual cosa muestra el nonbre del rey . \\\hline
1.2.7 & et dimissis uirtutibus \textbf{ totum studium suum ponet , } ut affluat diuitiis , & e paresçientes dexa las uirtudes \textbf{ e pone todo su estudio } por que abonde en riquezas \\\hline
1.2.8 & Ergo ratione bonorum , \textbf{ ad quae Rex gentem suam dirigere debet , } expedit ut habeat prouidentiam futurorum , & pues que assi es por razon de aquellos \textbf{ bienesa que el Rey deue guiar su gente e su conpanna . } Conuiene le que aya prouision delas cosas que han de venir \\\hline
1.2.8 & Modus enim , \textbf{ quo Rex suum populum dirigit , } oportet quod sit humanus , & e razon o conuiene le que sea entendido e razonable . \textbf{ Ca la manera por que el Rey guia el su pueblo } Conuiene que sea manera de omne . \\\hline
1.2.8 & Non enim decet Regem \textbf{ in omnibus sequi caput suum , } nec inniti semper solertiae propriae : & aquel que bien le conseia \textbf{ por la qual cosa non le conuiene al Rey de seguir en todas cosas su cabeça } nin atener se sienpre al su engennio propio . \\\hline
1.2.9 & non debent vanitatibus intendere : \textbf{ sed maiorem partem vitae suae debent expendere in cogitando } quae possunt esse regno proficua . & non se deuen dar auanidades \textbf{ mas deuen espender la mayor parte de su uida | en cuydar quales son las cosas } que mas aprouechosas son a su regno . \\\hline
1.2.9 & sed debent eis adeo moderate uti , \textbf{ ut non impediantur in regimine regni sui . } Seipsos ergo poterunt prudentes facere , & Mas deuen usar dellos tenpradamente \textbf{ en tal manera que non sean enbargados en el gouernamiento del regno ¶ } Pues que assi es los Reyes \\\hline
1.2.9 & quid agendum sit in futurum . \textbf{ Nam semper debet suum regimen conformare regimini retroacto , } sub quo regnum tutius , & en lo que ha de venir . \textbf{ Ca sienpre deue el Rey conformar e ordenar el su gouernamiento | segunt el gouernamiento del tp̃o passado } en el qual el su regno meior \\\hline
1.2.10 & Inde est ergo quod haec Iustitia dicitur \textbf{ unicuique suum tribuere , } quia ius in quadam aequalitate consistit : & Et por ende se sigue \textbf{ que esta iustiçia egual manda dar a cada vno su derecho } ca el derechon esta en vna ygualdat \\\hline
1.2.10 & Sic etiam dicitur \textbf{ unicuique tribuere quod suum est : } quia aequum est , & e lo que es igual \textbf{ Et assi es dicha dar a cada vno | lo que es suyo . } Ca cosa igual es \\\hline
1.2.10 & quia aequum est , \textbf{ quemlibet possidere sua . } Si igitur haec Iustitia specialis aequalis dicitur , & lo que es suyo . \textbf{ Ca cosa igual es } que cada vno sea señor de lo suyo ¶ Pues que assi es si esta iustiçia sp̃al es dicha igual por que entiende a egualdat . \\\hline
1.2.11 & in quo ille abundat . \textbf{ Ideo ut quilibet suae indigentiae prouideret , } inuenta fuit commutatiua Iustitia . & enla qual abonda el otro . \textbf{ Et por ende por que cada vno pudiesse proueer a la su mengua } sue fallada la iustiçia mudadora \\\hline
1.2.11 & Nisi igitur membra sic \textbf{ suae indigentiae subuenirent : } ut nisi manus purgaret oculum , & por quel trahe sobre si . \textbf{ pues que assy es si los mienbros en esta manera non se ayudassen los vnos alos otros } assi que la mano non alinpiase el oio \\\hline
1.2.11 & prout habent ordinem ad se inuicem , \textbf{ et sibi inuicem suae indigentiae subueniunt , } est in eis quaedam commutatiua Iustitia , & en quanto han ordenamiento entre si mismos \textbf{ e se acorten a sus menguas los vnos alos otros } o es en ellos vna iustiçia mudadora e acorredora \\\hline
1.2.11 & et sibi inuicem \textbf{ secundum quandam commutationem suis indigentiis satisfaciunt , } est in eis commutatiua Iustitia , & o de vn regno han ordenamiento entre si mismos \textbf{ e se acorren a las sus menguas los vnos alos otros mudando | e dando las vnas cosas por las otras . } es en ellos la iustiçia mudadora \\\hline
1.2.11 & Nam cor singulis membris \textbf{ secundum suam proportionem , } et dignitatem influit eis spiritum vitalem , & Ca el coraço ha todos los mienbros del \textbf{ cuerpoda } e enbia sp̃s de uida e mouimientos \\\hline
1.2.11 & summo opere Rex studere debet , \textbf{ ut in suo Regno , } seruetur Iustitia , & afincadamente deue el rey estudiar \textbf{ por que en los sus regnos sea guardada la iustiçia } non solamente aquellos que nasçieron en el regno \\\hline
1.2.12 & quae est valde pulchra , et clara : \textbf{ et propter sui pulchritudinem , } et venustatem communi nomine & que es muy fermosa e muy clara \textbf{ e por la su fermosura } e por la su claridat es llamada renꝮ \\\hline
1.2.12 & cum potest sibi simile producere , \textbf{ et cum actio sua ad alios se extendit : } ut tunc aliquid est perfecte calidum , & enssi quando puede fazer otra tal commo si . \textbf{ Et quando la su obra se estiende alos otros | assi commo paresçe por la calentura . } Ca estonçe es dicha alguna cosa \\\hline
1.2.12 & quando potest alia calefacere , \textbf{ et quando actio sua ad alia se extendit . } Et tunc est aliquis perfecte sciens , & quando puede calentar alas otras cosas . \textbf{ Et quando la su obra | e la su calentura se estiende alos otros . Et esso mismo } estonçe es dicho el ome \\\hline
1.2.12 & quando potest alios docere , \textbf{ et quando scientia sua ad alios se extendit . } Ideo scribitur 1 Metaphys’ & quando puede ensseñar los otros \textbf{ e quando la su sçiençia se estiende alos otros . } Et por ende dize el philosofo \\\hline
1.2.12 & tunc est aliquis perfecte bonus , \textbf{ quando bonitas sua usque ad alios se extendit . } Inde est ergo , & Et por ende fablando por semeiança podemos dezir que estonce es dicho el omne \textbf{ conplidamente bueon | quando la su bondat se estiende alos otros } Et por ende la bondat acabada de los omes \\\hline
1.2.12 & quia oportet , \textbf{ quod bonitas sua ad alios se extendat , } tunc melius apparet qualis sit , & Mas quando es puesto en algun prinçipado o en algun sennorio \textbf{ por que la su bondat se ha de estender a otros } estonçe meior paresçe quales si es bueno o malo \\\hline
1.2.12 & tunc melius apparet qualis sit , \textbf{ eo quod opera sua ad exteriora se extendant . } Si ergo nobis exteriora magis nota sunt , & estonçe meior paresçe quales si es bueno o malo \textbf{ por que las sus obras le estienden a los otros | ¶ } Et pues que assi es si las cosas \\\hline
1.2.12 & quanto aliquis in maiori principatu constituitur ; \textbf{ quia opera sua ad plura se extendunt , } magis apparet qualis sit . & e en mayor dignidat \textbf{ por que las sus obras se estienden amas } estonçe paresçe meior cada vno quales . \\\hline
1.2.12 & qui non solum est bonus in se , \textbf{ sed etiam bonitas sua se extendit ad alios : } sic peior est , & que non solamente es bueno en si \textbf{ Mas ahun la su bendat se estiende alos otros omes . } Assi peor es el omne \\\hline
1.2.12 & qui non solum malus est in se , \textbf{ sed etiam malitia sua se extendit ad alios : } et quantum ad plures se extendit malitia eius , & que non solamente es malo en ssi \textbf{ mas ahun la su maliçia se estiende alos otros omes } Et quanto amas se estiende la su maliçia \\\hline
1.2.14 & Nam , cum nauigiis , \textbf{ et cum toto suo exercitu transfretaret , } ne aliquis de suo exercito haberet materiam fugiemdi , & Ca commo el estudiese en sus naues \textbf{ e con todas sus naues passase la mar } por que ninguno de sus conpannas non ouiese manera de fuyr \\\hline
1.2.14 & et cum toto suo exercitu transfretaret , \textbf{ ne aliquis de suo exercito haberet materiam fugiemdi , } omnes naues confregit . & e con todas sus naues passase la mar \textbf{ por que ninguno de sus conpannas non ouiese manera de fuyr } quebranto todas las naues . \\\hline
1.2.14 & Milites vero bellorum experti , \textbf{ confidentes de sua experientia , } et cognoscentes bellorum pericula , & que son vsados delas batallas fiando de su praeua \textbf{ e delo que han prouado } e conosçiendo los periglos delas batallas \\\hline
1.2.14 & has maneries fortitudinum scire debeant , \textbf{ ut cognoscant qualiter populus suus fortis est , } et quomodo possunt & e los prinçipes deuen saber estas maneras de fortaleza \textbf{ que son dichas | por que sepan en qual manera han de ser fuertes . } Et en commo pueden lidiar con sus enemigos . \\\hline
1.2.14 & ipsi tamen debent esse fortes fortitudine virtuosa , \textbf{ ut non exponant suam gentem periculis bellicis , } nisi habeant iusta bella , & Enpero ellos deuen ser fuertes de fortaleza uirtuosa \textbf{ por que non pongan la su gente | e el su pueblo a periglos de batallas } si non quando ouieren razon derecha para auer batalla . \\\hline
1.2.15 & ad bonum proprium , \textbf{ ut ad moderationem sui ipsius . } Bonum autem commune & e abien propio \textbf{ assi | conmoare frenança de si mismo . } Et nos sienpre dezimos \\\hline
1.2.16 & ut haberet colloquia \textbf{ cum baronibus regni sui ; } sed omnes collocutiones eius erant & a auer fabla con los Ricos omes \textbf{ e cauałłos de su regno . } mas todas sus fablas eran en las camaras con las mugieres \\\hline
1.2.16 & Dux autem ille assuetus rebus bellicis , \textbf{ videns Regem suum esse totum muliebrem et bestialem , } statim ipsum habuit in contemptum : & veyendo \textbf{ que el su Rey era todo mugeril | e toda su } conuerssaçion era entre mugers e era bestial . \\\hline
1.2.16 & et cum toto thesauro , \textbf{ et omnibus supellectilibus suis , } se combussit . & e con todas sus alfaias \textbf{ e que mosse con todo } ¶ \\\hline
1.2.17 & ut se diligat , \textbf{ et ut sua bona custodiat . } Dare autem propria bona , & Ca cada hun omne es naturalmente inclinado a amar asi mismo \textbf{ e aguardar los sus biens propos } Mas dar los sus biens propios ha alguna guaueza por si . \\\hline
1.2.18 & Tertio huiusmodi virtus dicitur communicabilitas : \textbf{ quia per eam homines communicant sua bona , } per quam communicationem ab aliis potissime diliguntur : & ¶ Lo terçero esta uirtud es dicha franqueza \textbf{ por que por ellas los omes parten los sus bienes } por la qual participaçion se departen estri̊madamente de los otros . \\\hline
1.2.20 & vel quomodo faciat decentes nuptias : \textbf{ sed tota sua intentio est , } quomodo faciat paruos sumptus . & o en qual manera faga sus bodas conuenibles \textbf{ mas toda su entençion es } en qual manera faga peannas espenssas . \\\hline
1.2.20 & quasi sibi incorporata , \textbf{ et quasi pertinentia ad substantiam suam . } Et quia sic afficitur ad ea & assi commo si fuessen ael encorparados \textbf{ e assi commo si parte nesçiessen a su sustançia . } Et tanto ama aquellos biens \\\hline
1.2.20 & quia est ibi quasi quaedam diuisio continui , \textbf{ eo quod paruificus reputat suam pecuniam } quasi sibi incorporatam et continuatam , & assi conmo vn taiamiento del cuerpo continuo \textbf{ por tanto que el parufico cuyda | que el su auer } e los sus desque son \\\hline
1.2.21 & ideo maxime spectat ad magnificum \textbf{ in suis magnificis operibus , } et distributionibus intendere finaliter bonum , & Por ende mucho parte nesçe al magnifico \textbf{ en las sus muy grandes obras } e en las sus parti \\\hline
1.2.23 & Cum autem sic se periculis exponit , \textbf{ adeo debet esse constans in suis negociis , } ut etiam , si viderit expedire , & Et quando assi el mangnanimo se pusiere alos periglos . \textbf{ deue ser ta firme en sus negoçios e en sus obras . } que ahun si viere \\\hline
1.2.23 & casus adeo arduus , \textbf{ quod Rex gentem suam , } vel etiam seipsum debeat & Mas si acaesçiere algun caso tan alto \textbf{ por que al Rey conuenga de poner su gente o avn assi mismo } aperigłsᷤen \\\hline
1.2.24 & Increpamus enim aliquos , \textbf{ dicentes eos non curare de honore suo , } et rursus quia vituperamus ambitiosos & denostamosa alguos \textbf{ deziendo que non curan de su honrra . } Et otrosi denostamos los muy cobdiçiosos de honrra . \\\hline
1.2.24 & et rursus quia vituperamus ambitiosos \textbf{ laudamus non curantes de honore suo . } Curare igitur de proprio honore , & Et por ende lo amos \textbf{ aquellos que non curan de su honrra . } ¶ Pues que assi es auer cuydado ome de su propia honrra \\\hline
1.2.26 & sed etiam per deiectionem . \textbf{ Nam si quis ultra quam suus status requirat , } praeter rationem et notabiliter se deiiceret : & por fallesçimiento . \textbf{ Et si alguno mas que su estado demanda sin razon } e notablemente se despreciasse por que fuesse vil e despreçiado e bestia \\\hline
1.2.26 & deiectus , et bestia , \textbf{ quia suum statum non cognosceret ; } vel esset superbus , et iactator . & e notablemente se despreciasse por que fuesse vil e despreçiado e bestia \textbf{ e que non conosçiesse si estado } o que fuesse sob̃uio e alabador dessi \\\hline
1.2.26 & quod faciunt humiles : \textbf{ quod tamen suam felicitatem } non ponant & e esto es lo que fazen los humildosos . \textbf{ Et otrosi que non pongan la su bien andança } en sobrepuiança de honrra lo que fazen los sobuios \\\hline
1.2.26 & Secundo decet eos esse humiles ratione operum fiendorum . \textbf{ Nam superbus quaerens suam excellentiam ultra quam debeat , } ut plurimum tendit & que han de fazer . \textbf{ Ca el sobra uio demandado | e quariendo su excellençia e sobrepuiamiento } mas que deue \\\hline
1.2.29 & semper est declinandum in minus . \textbf{ Prima sumitur ex parte sui . } Secunda ex parte aliorum . & deuemos declinar alo menos ¶ \textbf{ Lo primera se toma de parte de ssi mismo¶ } La segunda de parte de los otros ¶ \\\hline
1.2.29 & ex parte aliorum . \textbf{ Nam non declinantes in minus sunt laudatores sui , } iactantes se de bonis quae habent . & La segunda razon se toma de parte de los otros . \textbf{ Ca aquellos que non declinan alo menos son alabadores de ssi mismos } e alabando se de aquellos bienes \\\hline
1.2.31 & si pauperes \textbf{ secundum suam facultatem sunt vere , } et perfecte liberales propinquissimum est , & segunt dize el philosofo \textbf{ Empero si los pobres segunt su poder son uerdaderamente } e acabadamente liberales muy cercanos son para ser magnificos \\\hline
1.2.31 & et perficiat habentem , \textbf{ et opus suum bonum reddat : } cum ad bene eligere , & e acaba a aquel que la ha \textbf{ e faga la su obra buena . } Por ende commo havien escoger \\\hline
1.2.31 & Sic etiam si esset auarus , \textbf{ quia finem suum poneret } in habendo pecuniam , & En essa misma manera avn si alguno fuesse auariento \textbf{ por que pusiesse la su fin en auer riquezas e dineros } commo quier \\\hline
1.2.32 & Cum enim qui alios conuiuare volebat , \textbf{ si filius suus domi non erat , } a vicino suo mutuabat filium , & Ca quando alguno quaria conbidar a \textbf{ otrossi el su fiio non era en casa tomaua prestado el fijo de otro su vezino } e aprestaual para fazer el conbit \\\hline
1.2.32 & si filius suus domi non erat , \textbf{ a vicino suo mutuabat filium , } et ipsum parabat in conuiuium , & Ca quando alguno quaria conbidar a \textbf{ otrossi el su fiio non era en casa tomaua prestado el fijo de otro su vezino } e aprestaual para fazer el conbit \\\hline
1.2.32 & spondens quod quando vellet conuiuium facere , \textbf{ ei suum filium tribueret . } Sic etiam multae de Phalaride bestialitates narrantur . & et prometial que quando quisiesse fazer conbit \textbf{ que el qual daria su fijo } que fiziesse conbite del¶ \\\hline
1.2.32 & quod Homerus refert de Hectore , \textbf{ quod Rex Priamus Pater suus dicebat de ipso , } quod erat valde bonus : & e cuenta alos omes de ector \textbf{ que el Rey preamo su padre dezia del que era muy bueno } por la qual cosa non le paresçia \\\hline
1.2.33 & ita quod cuilibet generi bonorum \textbf{ demus suum ordinem virtutum . } Dicemus ergo quod perseuerantes habent virtutes politicas : & Et en essa misma manera podemos departir quatro ordenes de uirtudes \textbf{ assi que a cada vn linage de los bueons demos su orden de uirtudes . } Et por ende diremos \\\hline
1.2.33 & Tantae enim bonitatis debet esse Princeps , \textbf{ ut quilibet suus subditus accipiat } inde formam viuendi , & Ca de tanta bondat deue ser el prinçipe \textbf{ que cada vno de los sus subditos tome del forma e manera de beuir } e conosca can vno su mengua veyendo la uida \\\hline
1.2.33 & inde formam viuendi , \textbf{ et cognoscat defectum suum , } videns vitam et perfectionem principantis . & que cada vno de los sus subditos tome del forma e manera de beuir \textbf{ e conosca can vno su mengua veyendo la uida } e la grant perfeçion del prinçipe e del señor . \\\hline
1.3.1 & quia ostensum est \textbf{ in quo Reges et Principes suum finem ponere debeant , } et quomodo oportet & Ca mostrado es de ssuso \textbf{ enque deuen los Reyes | e los prinçipes poner su fin e su bien andança . } Et otrosi mostrado es en commo les conuiene de ser uirtuosos \\\hline
1.3.2 & et quae prosequendae : \textbf{ et quia in hoc primo libro determinare intendimus de regimine sui , } videndum est quot sunt passiones , & e quales son de leguir \textbf{ por ende en este libro primero | en que entendemos de determinar del gouernamiento del omne en si mismo . } en quanto es omne conuiene de ueer quantas son las passiones \\\hline
1.3.3 & quae ad prudentiam requiruntur , \textbf{ per quam possit melius suum populum regere . } Immo si bonum commune praeponat bono priuato , & que son meester ala pradençia e ala sabiduria . \textbf{ por las quales pue da meior gouernar su pueblo . } Mas si ante pusiere e preçiare mas el bien comun \\\hline
1.3.3 & Erit temperatus ; \textbf{ quia si intentio sua principaliter versetur } circa bonum regni , & e avn sera tenprado \textbf{ por que si la entencion suya prinçipal fuere en trabaiar en el bien del regno despreçiar } a las delectaçiones destenpradas de los sesos \\\hline
1.3.4 & Nam sicut corpora naturalia \textbf{ per suas formas , } ut per grauitatem vel per leuitatem & Ca assi commo los cuerpos naturales \textbf{ por sus formas . } assi commo la piedra \\\hline
1.3.5 & Sperare enim ultra quam sit sperandum , \textbf{ et aggredi opus ultra vires suas , } videtur ex imprudentia procedere , & que deue omne esparar \textbf{ e acometer alguna obra | mas que la su fuerca demanda paresçe } que esto uiene mas de mengua de sabiduria \\\hline
1.3.7 & non sit virtus corporalis , \textbf{ utitur tamen in suo actu corporalibus organis ; } propter quod corpore existente indisposito , & Ca maga el entendimiento non sea uirtud corporal \textbf{ enpero en su obra vsa de entender de organos e de mienbros corporales . } Por la qual cosa el cuerpo non estando bien ordenado \\\hline
1.3.7 & non potest libere \textbf{ uti actu suo . } Quare si in quolibet homine & commo deue non \textbf{ vsa el entendimiento liberalmente de su obra . } Et ponen de si en cada vn omne es de esquiuar \\\hline
1.3.8 & Sed hi omnem delectationem condemnantes , \textbf{ statim suam positionem ostendebant reprehensibilem : } quia ( secundum Philosoph’ ) & Mas todos estos que despreçia un a todas las delectaçiones \textbf{ luego mostra una | que la su posicion era de reprehender . } Ca segunt el philosofo \\\hline
1.3.9 & Primo , ut comparantur ad passiones alias . \textbf{ Secundo , ut comparantur ad sua obiecta , } vel ad materiam circa quam versantur . & que las otras podemos lo prouar \textbf{ por tres maneras ¶pmero en quanto son conparadas alas otras ¶ lo segundo en quanto son conparadas alos sus obiectos } o ala su materia çerca la qual obran ¶ Lo terçero en quanto son conparadas alas potençias del alma \\\hline
1.3.11 & de prosperitatibus malorum , \textbf{ ne indignis distribuant sua bona . } Sic ergo se habere debent & dela bien andança de los malos \textbf{ en quanto ellos non deuen partir los sus bienes alos malos . | nin alos que non son dignos . } Et por ende assi se deuen auer los Reyes alas pasiones sobredichos \\\hline
1.4.1 & quia ostensum est \textbf{ in quo Reges et Principes suum finem ponere debeant : } et quibus virtutibus debeant esse ornati : & Ca mostrado es en que deuen los Reyes \textbf{ e los prinçipes petier su fin e su bien andança . } Otrossi es mostrado de quales utudes deuen ser honrrados \\\hline
1.4.1 & non acquisiuerunt proprio labore . \textbf{ Nam quilibet cum maiori diligentia retinet facultates suas , } quando propter indigentiam passus est aliqua mala , & por su trabaio propreo . \textbf{ Ca cada vno con mayor acuçia guarda las sus riquezas | e el su auer } quando ha sofrido alguons males \\\hline
1.4.1 & immo adeo quilibet delectatur in proprio opere , \textbf{ quod quicquid sua industria acquirit , } charius possidet . & Et avn en tanto se delecta cada vno en su obra propria \textbf{ que qual si quier cosa que gana } por su sabiduria o por su trabaio mas caramente la guarda . \\\hline
1.4.1 & Ideo dicitur secundo Rhetoricorum , \textbf{ quod pueri sua innocentia alios mensurant . } Sicut enim ipsi sunt innocentes , & que los moços mesuran \textbf{ por su ioçençia | e por su sinpleza todos los otros . } Ca assi commo ellos son Innoçentes \\\hline
1.4.1 & si credamus alios indigne pati . \textbf{ Quare si iuuenes sua innocentia alios mensurant , } et credunt alios indigne pati , & que los otros sufren mala tuerto \textbf{ e sin meresçimiento . | Por la qual cosa si los mançebos mesuran los otros } por su inoçençia e por su sinpleza e creen \\\hline
1.4.1 & Cum ergo matres semper moneant \textbf{ suos filios ad honesta ; } quia honestum idem est & Et pues que assi es commo las muger \textbf{ ssienpre amonestan asus fijos a honestad } e a honrrata honestades \\\hline
1.4.2 & Non enim putant alios esse malos , \textbf{ sed sua innocentia alios mensurant . } Cum ergo naturale sit , & que son malos . \textbf{ Mas por la su moçençia | e por la su sinpleza mesuran alos otros . } Et pues que assi es commo natural cosa sea \\\hline
1.4.2 & ut dicunt . \textbf{ Sexto in suis actionibus } non habent modum , & ¶ \textbf{ Lo sexto non han manera en las sus obras } Mas todas las cosas fazen forçadamente \\\hline
1.4.2 & eos non habere modum \textbf{ in actionibus suis : } quia cum alia sint moderanda per mensuram , & non auer manera en las sus obras . \textbf{ Ca commo todas las sus obras } de una ser tenp̃das \\\hline
1.4.3 & et innocentes sunt , \textbf{ sua innocentia alios mensurant , } et omnia referunt in meliorem partem : & e non han fecho muchos males \textbf{ por la su sinpleza et inoçençia iudgan todos los otros . } Et todas las cosas retuerçen ala meior parte \\\hline
1.4.3 & sed solum confidunt de iis quae habent . \textbf{ Ponentes ergo in eis suam spem et confidentiam , } non audent expensas facere . & mas solamente fian de aquellas cosas que han e tienen . \textbf{ Et por ende poniendo en lo que han su esperança } e su fiuza non osan fazer espenssas . \\\hline
1.4.4 & quia sunt amatores amicitiarum , \textbf{ et quia sua innocentia alios mensurant , } existimant omnes bonos esse ; & Et por ende los mançebos pro por que son amadores de amistanças \textbf{ e por que por su sinpleza mesutan los otros cuydan } que todos los otros son bueons \\\hline
1.4.4 & Quarto nihil agunt valde , \textbf{ sed in omnibus operibus suis videntur esse temperati . } Nam sicut iuuenes , & Lo quarto non fazen ninguna cosa con sobrepunaça \textbf{ mas en todas las sus obras quieren paresçer tenprados . } Ca assi conmo los mançebos \\\hline
1.4.4 & ne per hoc iudicentur leues et indiscreti . \textbf{ Quarto in suis actionibus debent habere moderationem et temperamentum : } quia ( ut dictum est ) & e de poco saber ¶ lo quarto los Reyes \textbf{ e los prinçipes deuen auer | en las sus obras mesura e tenpramiento } por que assi commo dicho es ellos \\\hline
1.4.5 & quia ergo nobiles ex antiquo fuerunt praesides , \textbf{ et in suo genere fuerunt multi insignes et diuites , } eleuatur cor nobilium & Et por ende por que los nobles de antiguo tienpo fueron prinçipes \textbf{ e en su linage fueres mucho nobles e ricos leunatase el coraçon de los nobles } por \\\hline
1.4.5 & per creationem filiorum , \textbf{ tanto minus est memoria genitores suos fuisse pauperes ; } ideo semper augmentatur nobilitas , & quanto mas va descendiendo la generaçion de los fijos \textbf{ tanto menos es memoria | que los sus parientes fuessen pobres . } Et por ende sienpre cresçe la nobleza \\\hline
1.4.5 & Vult enim ibidem , \textbf{ quod nobiles ex sua nobilitate incitantur , } ut sint magnanimi , et magnifici . & por que dize alli el philosofo \textbf{ que los nobles | por su nobleza se esfuerçan } a ser magnanimos e magnificos . \\\hline
1.4.5 & Nobiles ergo , \textbf{ quia ex suo genere videntur } esse honorabiles , & Et por ende los nobles \textbf{ por que son honrrados | por el su linage } por ende quieren acresçentar aquella honrra \\\hline
1.4.5 & maiores se reputant , \textbf{ quam patres suos ; } quod ideo contingit , & Ca en la mayor parte los nobles \textbf{ por mayores se tienen que sus padres . } la qual cosa lescontesçe \\\hline
1.4.5 & Esse autem elatum , \textbf{ et despicere suos progenitores , } et nimis esse honoris cupidi , & por que sienpre es mas antigua¶ \textbf{ Mas ser sobrauios e despreçiar los sus engendradores } e ser muy cobdiçiosos de honrra \\\hline
1.4.6 & et pretium omnium aliorum ; \textbf{ quare in cordibus suis efficiuntur superbi et elati , } credentes omnibus excellentiores esse . & por que cuydan que los dineros son dignidat e preçio de todas las otras cosas . \textbf{ Por la qual cosa en los susco raçons se ensoƀueçen | e se leuna tan cuydando } que son meiores \\\hline
1.4.6 & indigere bonis diuitum , \textbf{ contingit ut diuites in suis cordibus eleuentur , } dispicientes alios , & Por que contesçe que los sabios han menester de los bienes de los ricos \textbf{ e por ende los ricos se leunatan en sus coraçones } despreçiando alos otros \\\hline
1.4.6 & nisi fugiant malos mores ipsorum diuitum , \textbf{ et nisi suas diuitias ordinent ad bonum , } et ad opera virtuosa . & si non se arte draten delas malas costunbres de los ricos \textbf{ e si non ordenar en las sus riquezas abien o a obras de uirtud } ¶ \\\hline
1.4.7 & quia est in aliquo principatu , \textbf{ et habet multos sub suo dominio ; } quare cum multos nobiles videamus esse impotentes , & por que es en algun prinçipado \textbf{ e ha muchos so su sennorio . } por la qual cosa commo nos veamos muchos ser nobles \\\hline
1.4.7 & quod tamen est nobilis , \textbf{ et ab antiquo sui progenitores diuites extiterunt , } melius nouit diuitias supportare , & que es noble \textbf{ e de antigo tienpo los sus auuelos fueron ricos meior sabe sofrir las riquezas } e por ellas non se leu nata en so ƀͣuia \\\hline
2.1.1 & et Principes debeant \textbf{ suam felicitatem ponere : } quas virtutes habere : & por que ya mostramos en qual cosa de un a los Reyes \textbf{ e los prinçipes poner su bien andança . } Et quales uirtudes deuen auer . \\\hline
2.1.1 & Natura cum aliquibus animalibus \textbf{ ad sui tuitionem dedit cornua , } ut bubalis et bobus . & Ca por que la natura dio a algunas anmalias \textbf{ para su defendemiento cuernos } assi commo alos bubalos e alos bueyes . \\\hline
2.1.1 & non dedit \textbf{ ad sui tuitionem cornua vel ungues : } sed dedit ei manum , & conmoaianl mas noble que las otras nol dio \textbf{ para su defendemiento cuernos } nin hunnas mas diol mano . \\\hline
2.1.3 & Sic aliqui dicere \textbf{ consueuerunt domos suas hoc operatas esse , } non quia lapides illud egerint , & Bien assi alguons suelen dezir \textbf{ que las sus casas fizieron esto } o aquello non por que las piedras fizieron esto \\\hline
2.1.3 & non quia lapides illud egerint , \textbf{ sed quia sui progenitores fecerunt illud , } quare sicut communicatio ciuium ciuitas nominatur , & o aquello non por que las piedras fizieron esto \textbf{ mas por que los sus padres o los sus fujos lo fizieron . } Por la qual razon \\\hline
2.1.3 & quod decet homines habere habitationes decentes \textbf{ secundum suam possibilem facultatem ; } non tamen spectat & por figera e por exienplo que conuiene alos omes de auer conueibles moradas \textbf{ segunt el su poder e la su riqueza . } Enpero non pertenesçea el de tractar delan casa prinçipalmente \\\hline
2.1.3 & nunquam enim quis debitus rector regni vel ciuitatis efficitur , \textbf{ nisi se et suam familiam } sciat debite gubernare . & Et ela conosçimiento del gouernamento dela casa ¶ Por que nunca ninguno puede ser buen gouernador del regno o dela çibdat \textbf{ si non sopiere bien gouernar | assi e ala su conpana . } Por la qual cosasi espeçialmente pertenesçe alos Reyes \\\hline
2.1.3 & intendere bonum regni : \textbf{ spectat ad unumquemque ciuem scire regere domum suam , } non solum inquantum huiusmodi regimen est bonum proprium , & entenderal bien del regno . \textbf{ Por ende pertenesçe a cada vn | çibdadano saber gouernar su casa } non solamente en quanto este gouernamiento es bien propreo suyo \\\hline
2.1.4 & tam necessaria in vita ciuili , \textbf{ spectat ad quemlibet ciuem scire debite regere suam domum : } tanto tamen magis hoc spectat ad Reges et Principes , & sea tan neçessaria \textbf{ enla uida çiuil pertenesçe a cada vn çibdada | no de laber gouernar conueniblemente lucasa . } Et tanto mas esto parte nesçe alos Reyes \\\hline
2.1.6 & quia non habet felicitatem politicam \textbf{ cum omni sua claritate . } Patet ergo quod ad hoc quod domus habeat esse perfectum , & por que non ha la bien andança \textbf{ çiuil con todas las cosas } que cunplen para esto . \\\hline
2.1.7 & Deinde ostendemus , \textbf{ qualiter viri suas uxores regere debeant } et ad quas virtutes , & e mayormente los Reyes e los prinçipes . \textbf{ Despues mostraremosen qual manera los uarones deuen gouernar sus mugers } e a quales uirtudes \\\hline
2.1.7 & Ponentes ergo propria ad commune , \textbf{ ut cum uxor propria sua ordinat } in bonum uiri uel & Et por ende poniendo ellos las cosas propreas al comun \textbf{ assi commo quando la muger orden a las sus obras propreas al bien de su marido } o al bien de toda la casa . \\\hline
2.1.7 & in bonum totius domus : \textbf{ et vir propria sua ordinat } in bonum domus , & o al bien de toda la casa . \textbf{ Et el marido ordena sus obras propreas al bien de casa } e al bien de su muger \\\hline
2.1.8 & Decet ergo omnes ciues coniungi \textbf{ suis uxoribus indiuisibiliter absque repudiatione , } tanto magis hoc decet reges et principes , & asus muger \textbf{ ssin departimiento ninguno | e sin repoyamiento . } Mas esto \\\hline
2.1.8 & ut sint amici inter se parentes , \textbf{ qui naturaliter diligunt suam prolem , } ex dilectione naturali & Et por ende el padre e la madre \textbf{ por que naturalmente aman a sus fijos } por el amor natural que han con ellos \\\hline
2.1.8 & ad quam ordinatur coniugium , \textbf{ inseparabiliter conuiuere suis uxoribus . } Tanto tamen hoc magis decet Reges et Principes , & ala qual es ordena de el casamiento de beuir con sus mugieres \textbf{ sin ningun departimiento . } Enpero esto tanto pertenesçe \\\hline
2.1.8 & tanto magis decet Reges , et Principes , \textbf{ quam diu suae uxores vixerint , } eis inseparabiliter adhaerere . & mas conuiene alos Reyes \textbf{ e alos prinçipes mientre sus mugers biuieren ayuntar se a ellas sin ningun departimiento . } e algunas sectas non los iudgan contra razon que vn omne aya muchͣs mugers \\\hline
2.1.9 & si hoc indecens est parte uxoris , \textbf{ ne uxor a suo coniuge non debite diligatur . } Nam inter uxorem et virum debet & En essa misma manera esto es desconueinble de parte dela muger \textbf{ por que la muger non sea desamada nin aborresçida de su marido . } Ca entre la mugni e el uaron deue ser grand amor \\\hline
2.1.9 & quia , ne indebite utantur venereis , \textbf{ inter eos et suas coniuges maxime reseruari debet } amor debitus coniugalis . & çomoles non conuiene \textbf{ por que entre ellos e sus mugers sea mucho mas guardado el amor matermoinal . } ¶ La terçera razon para prouar esto mesmo se toma de parte dela \\\hline
2.1.10 & esse subiecta viro ; \textbf{ ex quo totam sui corporis potestatem } uni viro tribuit , & Por la qual cosa si la muger deue ser subiecta al uaron \textbf{ pues que todo el pode rio de su cuerpo es dado avn uaron } contra la orden natraales \\\hline
2.1.10 & Nam cum quilibet moleste ferat , \textbf{ si in usu suae rei delectabilis impeditur ; } absque dissensione et discordia esse non posset , & Ca commo qual si quier sufra \textbf{ guauemente si le enbargaren del vso de aquella cosa | en que se delecta } Esto non puede ser sin dissension o sin discordia . \\\hline
2.1.10 & contra parentes et consanguineos uxoris ad inimicitiam moueretur , \textbf{ eo quod suam coniugem alteri uiro per coniugium subiecerunt . } Ex coniugio igitur , & ante cada vno de aquellos uarones se mouria a enemistad contra los parientes e amigos de aquella muger \textbf{ por que casaron a su muger con otro marido . } ¶ Et pues que assi es del casamiento \\\hline
2.1.10 & Diligenter ergo aduertere debent singulae mulieres , \textbf{ quae suis viris per coniugium copulantur , } quanta diligentia debeant & deuen tener mientes con grand acuçia \textbf{ que las que son ayuntadas a sus maridos } por matmonio con quanta diligençia deuen guardar la su honestad \\\hline
2.1.10 & quanta diligentia debeant \textbf{ ad suam pudicitiam obseruare , } et quanto conatu fidem suis viris obseruent : & que las que son ayuntadas a sus maridos \textbf{ por matmonio con quanta diligençia deuen guardar la su honestad } e con quanto esfuerço deuen guardar la fialdat \\\hline
2.1.10 & ad suam pudicitiam obseruare , \textbf{ et quanto conatu fidem suis viris obseruent : } quia si esse non potest , & por matmonio con quanta diligençia deuen guardar la su honestad \textbf{ e con quanto esfuerço deuen guardar la fialdat | que prometieron a sus maridos } Ca assi commo non puede ser \\\hline
2.1.10 & et detestabile est etiam per coniugium foemina \textbf{ ( viuente viro suo ) } viro alio copulari , & e es cosa desconueible \textbf{ que la fenbra biuiendo su marido sea ayuntada a otro marido } por casamiento \\\hline
2.1.10 & Sed si una foemina pluribus nubat viris , \textbf{ patres de suis filiis certi esse non poterunt , } quare non adhibebunt illam diligentiam & Mas si vna fenbra casare con muchos uarones \textbf{ los padres non podrian ser çiertos de sus fijos . } Et por ende non aurian tan grand cuydado \\\hline
2.1.10 & quam debent \textbf{ ut suis filiis debite in nutrimento } et in haereditate prouideant . & nin tan grand acuçia \textbf{ commo deurien en el nudermiento conuenible de sus fijos } nin en proueer los dela hedat ¶ \\\hline
2.1.10 & quia per hoc magis impeditur certitudo filiorum . \textbf{ Quare si decet omnes ciues certos esse de suis filiis , } ut eis diligenter prouideant in haereditate et in nutrimento : & ca por esto se enbargaria mas la çertidunbre de los fijos . \textbf{ Por la qual cosa sy conuiene a todos los çibdadanos | ser çier tos de lus fios } por que los puedan proueer \\\hline
2.1.11 & dicta naturalis ratio , \textbf{ quod nimis propinqua ex suo genere } non est per coniugium socianda , & que la muger \textbf{ que es mucho allegada | e muy çercana } por su linage a algun uaron \\\hline
2.1.12 & apud Reges , \textbf{ et Principes in sui , } coniugibus quaerenda est nobilitas generis : & conuiene alos Reyes \textbf{ e alos prinçipes de querer } en sus mugi eres nobleza de liuage \\\hline
2.1.12 & quae deseruiunt ad sufficientiam vitae : \textbf{ decet eos in suis coniugibus } principalius quaerere , & que siruen a abastamiento dela uida . \textbf{ Conuiene aellos de demandar en las sus mugers } mas prinçipalmente que ellas sean nobles de linage \\\hline
2.1.12 & Viso quomodo Reges , \textbf{ et Principes in suis coniugibus } debent quaerere exteriora bona : & Visto en qual manera los Reyes \textbf{ e los prinçipes } deuen demandar con sus mugers los bienes de fuera \\\hline
2.1.13 & ut filii polleant magnitudine corporali , \textbf{ quaerere in suis uxoribus magnitudinem corporis : } tanto tamen magis hoc decet Reges et Principes , & e por que los fijos dellos \textbf{ resplandezcan por grandeza de cuepo de demandar en las sus mugers grandeza de cuerpo . } Enpero tanto mas esto conuiene alos Reyes \\\hline
2.1.13 & decet eos \textbf{ in suis uxoribus quaerere magnitudinem , } et pulchritudinem corporalem : & por fiios grandes e fermosos . \textbf{ Conuiene a ellos de demandar en las sus mugieres grandeza e fermosura corporal . } Ca paresçe que la fermosura dela muger \\\hline
2.1.13 & Decet ergo omnes ciues \textbf{ hoc in suis coniugibus quaerere : } tanto tamen hoc decet Reges et Principes , & Et pues que assi es conuiene a todos los çibdadanos \textbf{ de demandar esto en las sus mugers } Empero tanto conuiene esto mas alos Reyes \\\hline
2.1.14 & quomodo eum regere debeat : \textbf{ sed pater secundum suum arbitrium } prout melius viderit filio expedire , & en qual manera lo deua gouernar . \textbf{ Mas el padre segunt su aluedrio } en quanto viere \\\hline
2.1.14 & sicut et Rex gentem sibi subiectam regere debet \textbf{ secundum suum arbitrium , } prout melius viderit illi genti expedire . & que es subiecta a el \textbf{ segunt su aluedo } en quanto viere \\\hline
2.1.14 & et conuentiones \textbf{ quasdam in suo regimine obseruare . } Ex ipso ergo modo regendi , & e las con diconnes \textbf{ que deue el guardar en el su gouernamiento . } Et pues que assi es dela manera del gouernar \\\hline
2.1.15 & Quod autem non debeat \textbf{ uti sua coniuge tanquam serua , } triplici via venari possumus . & mas que el uaron non deua vsar de su mug̃r \textbf{ assi commo de sierua } esto podemos prouar \\\hline
2.1.15 & unde et idem Philosophus ait , \textbf{ quod unumquodque organorum optime perficiet suum opus , } si non multis operibus sit seruiens , sed uni . & Ende en esse logar dize el philosofo \textbf{ que qual si quier de los instrumentos fara conplidamente su obra si non siruiere en muchos obras mas en vna . } Por la qual cosa commo la natura aya ordenada la mugr \\\hline
2.1.16 & ne possit bene speculari , \textbf{ et ne possit libere exequi actiones suas . } Nascentes ergo ex tali coniugio & por que non pueda bien entender \textbf{ e que non pueda faze sus obras libremente . } ¶ Et pues que assi es los que nasçen de tal casamiento \\\hline
2.1.16 & sumitur ex intemperantia mulierum . \textbf{ Nam si in aetate valde iuuenili uxores suis viris copulentur , } non solum filii inde laeduntur , & La segunda razon para prouar esto mesmo se toma dela destenprança delas mugers . \textbf{ Ca si en la hedat de grand moçedat las mugers se ayuntaren a sus maridos } non solamente los fijos resçiben ende danno \\\hline
2.1.17 & propter roborationem caloris materni uteri , \textbf{ magis possunt conseruare suos foetus , } et eos perfectiores faciunt . & calentraa del uientre de la \textbf{ madremas pueden guardar las ceraturas } e fazer las mas fuertes \\\hline
2.1.18 & ( ut superius dicebatur ) sunt miseratiui , \textbf{ quia sua innocentia alios mensurantes credunt } omnes innocentes esse , & asi commo dicho es de suso son mibicordiosos \textbf{ Ca mesuran los otros | por su sinpleza } et creen \\\hline
2.1.19 & Nam sicut est in locutionibus , \textbf{ sic suo modo est in ipsis operibus . } Videmus enim aliquos habere linguas disertas , & Ca assi commo es en las fablas \textbf{ assi en su manera es en las obras propreas . } Ca veemos alguon sauer las lenguas escorrechas \\\hline
2.1.19 & Decet enim coniuges esse castas \textbf{ non solum propter fidem seruandam suis viris , } sed etiam propter procreandam prolem . & Ca conuiene alas mugieres de ser castas \textbf{ non solamente | por guardar fe a sus maridos } mas avn \\\hline
2.1.19 & sed requiritur \textbf{ ut pater sit certus de sua prole . } Cum ergo signa inhonesta , & Mas conuiene que el \textbf{ padresea çierto de su fijo . } Et pues que assi es por que las señales desonestas \\\hline
2.1.19 & quandam suspitionem adgenerent de incontinentia coniugis ; \textbf{ ut pater sit certus de sua prole , } expedit coniuges pudicas esse . & fazen algunan sospecha dela desonestad delas mugers . \textbf{ Para que el padre sea çierto de sus fijos } conuiene que las mugers sean linpias e honestas e guardadas en sus palauras ¶ \\\hline
2.1.19 & et diuitiis deficientes , \textbf{ debent per seipsos suas instruere coniuges , } et debitas cautelas adhibere , & e en riqueza deuen enssennar a sus mugers \textbf{ por si mesmos } e dar les castigos conuenibles \\\hline
2.1.19 & Quare decet omnes ciues \textbf{ sic suas coniuges regere : } et tanto magis hoc decet Reges , et Principes , & non o fallando otras cautellas para esto . \textbf{ por la qual cosa conuiene a todos los çibdadanos de gouernar a sus mugers assi . } Et esto tanto mas conuiene alos Reyes \\\hline
2.1.20 & quomodo Reges et Principes , \textbf{ et uniuersaliter omnes ciues debeant suas coniuges regere , } et ad quas bonitates debeant eas inducere : & en cunple de saber en qual manera los Reyes e los prinçipes \textbf{ e generalmente todos los çibdadanos | y deuen gouernar sus mugers } e aquales bonda deslas de una enduzir e traher linon lo pieren \\\hline
2.1.20 & Tertio debent cum eis debite conuersari . \textbf{ Decet enim eos suis coniugibus } moderate et discrete uti : & ¶ Lo primero se praeua \textbf{ assi que conuien e alos uarones de vsar con sus muger } stenpdamente e sabia mente . \\\hline
2.1.20 & non solum amicitia delectabilis , sed honesta . Viso , \textbf{ quomodo decet viros suis uxoribus moderate uti et discrete : } restat videre , & mas avn amistança honesta . \textbf{ ¶ Visto en qual manera conuiene alos uarons de vlar labiamente } e tenpradamente de sus muger sfinca de ver \\\hline
2.1.20 & secundum possibilem facultatem decet \textbf{ suam uxorem honorifice retinere } in debito apparatu , & segunt su poder \textbf{ e sus riquezas } en apareiamiento conuenible \\\hline
2.1.20 & decet quamlibet \textbf{ secundum suum statum } uxorem propriam honorifice pertractare . & Por ende conuiene a cada vn marido \textbf{ segunt su estado de tractar muy honrradamente a su muger . } ¶ Mostrado en qual manera conuiene alos maridos usen de sus mugers \\\hline
2.1.20 & uxorem propriam honorifice pertractare . \textbf{ Ostenso , quomodo decet viros suis uxoribus moderate et discrete } uti , & segunt su estado de tractar muy honrradamente a su muger . \textbf{ ¶ Mostrado en qual manera conuiene alos maridos usen de sus mugers } sabiamente e tenprada mente . \\\hline
2.1.20 & et inspectis conditionibus personarum , \textbf{ suis uxoribus ostendere debita amicitiae signa , } et eas ( ut expedit ) & e catadas las condiconnes delas perssonas mostrar a sus \textbf{ mugersseñales conueibles de amor } e enssennarlas \\\hline
2.1.21 & quomodo circa ornatum corporis deceat \textbf{ suas coniuges debite se habere . } Nam cum vir suam uxorem regere debeat , & e generalmente a todos los çibdadanos saber en qual manera \textbf{ couiene alas sus mugers | de se auer conueniblemente en el conponimiento e honrramiento de sus cuerpos . } Ca quando el marido gouierna e castiga a su muger \\\hline
2.1.21 & suas coniuges debite se habere . \textbf{ Nam cum vir suam uxorem regere debeat , } eam dirigendo ad actiones honestas , & de se auer conueniblemente en el conponimiento e honrramiento de sus cuerpos . \textbf{ Ca quando el marido gouierna e castiga a su muger } deue la castigar a obras honestas \\\hline
2.1.21 & ait , eos esse infelices secundum dimidium , \textbf{ eo quod uxoribus suis illicita permittebant . } Ne ergo domus principis & e desauentraados en la meytad de su fazienda \textbf{ por que consienten a sus mugieres cosas desconuenibles ¶ Et pues que assi es por que la casa del prinçipe } e avn de cada vn \\\hline
2.1.21 & Decet enim viros \textbf{ secundum suum statum , } suis uxoribus , & si se fizieren commo deuen e commo cunple . \textbf{ Ca conuiene alos maridos de proueer conueniblemente a sus } mugerssegunt sus estados e en vestiduras conuenibles \\\hline
2.1.21 & secundum suum statum , \textbf{ suis uxoribus , } in debitis vestimentis , & Ca conuiene alos maridos de proueer conueniblemente a sus \textbf{ mugerssegunt sus estados e en vestiduras conuenibles } e en los otros conponimientos conueinbles . \\\hline
2.1.21 & Unde et Valerius Maximus ciues Romanos commendat , \textbf{ qui suis uxoribus in pulchris indumentis } et in aliis ornamentis debite prouidebant : & Onde ualerio maximo alaba alos çibdadanos de Roma \textbf{ por que proue en honrradamente a sus mugers | de uestiduras fermosas } e de los otros conponimientos honrrados . \\\hline
2.1.21 & sed agunt \textbf{ ut suis viris placentes , } eos a fornicatione retrahant . & nin se afeytan por vana eglesia \textbf{ mas esto fazen por fazer plazer a sus maridos e por los tirar de forncacion e de luxuria . } Mas estonçe son tenpradas \\\hline
2.1.21 & Tunc vero sunt moderatae , \textbf{ quando considerato suo statu } non superflua vestimenta quaerunt . & Mas estonçe son tenpradas \textbf{ quando catado el su estado non quieren } nin demandan uestiduras sobeias . \\\hline
2.1.21 & si non esset moderata , \textbf{ et ultra quam suus status requireret , } appeteret ornamenta . & Et si \textbf{ dessease los honrramientos e conponimientos del cuerpo | mas que demanda el su estado . } ¶ Lo terçero conuiene alas mugers de ser \\\hline
2.1.21 & non propter vanam gloriam se ornaret , \textbf{ nec ultra suum statum ornamenta appeteret : } posset delinquere , & por vana eglesia \textbf{ nin dessea conponimiento | mas de quanto demanda su estado . } Enpero avn podria pecar \\\hline
2.1.21 & vituperat Laconios , \textbf{ qui infra suum statum vestimenta quaerentes } ex hoc in elationem et iactantiam mouebantur . & que llaman latonios \textbf{ por que quarian vestiduras mas viles | que el su estado demandaua . } Et por esta razon se mouiana so ƀiuia e a alabança . \\\hline
2.1.22 & triplici via ostendere possumus . \textbf{ Nam cum quis erga suam coniugem est nimis zelotypus , } ex nimio zelo quem erga illam gerit , & por tres razones . \textbf{ Ca quando alguno es muy çeloso de su muger } por el grand çelo que ha della sospecha todas las cosas \\\hline
2.1.22 & Decet ergo omnes ciues \textbf{ non esse nimis zelotipos de suis coniugibus : } et tanto magis hoc decet Reges et Principes , & pues que assi es conuiene a todos los çibdadanos \textbf{ de non ser muy çelosos de sus mugieres } Et tanto mas esto conuiene alos Reyes \\\hline
2.1.22 & sumitur ex eo quod uxores incitantur ad malum , \textbf{ si contingat suos viros } esse nimis zelotypos . & que por ende las muger sson mas abiuadas a mal \textbf{ quando sus maridos son muy çelosos dellas . } Ca comunal cosa es sienpre \\\hline
2.1.22 & Non ergo decet uiros \textbf{ de suis coniugibus } esse nimis zelotypos . & uezes varaias e contiendas . \textbf{ ¶ Et pues que assi es non conuiene alos maridos ser muy çelosos de sus mugrs } nin avn les conuiene \\\hline
2.1.22 & Nec etiam decet eos \textbf{ circa suas coniuges nullam habere custodiam } et nullum habere zelum , & nin avn les conuiene \textbf{ de non poner alguna guarda en sus mugers | nin les conuiene avn } de non auer algun çelo dellas \\\hline
2.1.22 & Sic enim decet uirum quemlibet \textbf{ erga suam coniugem ornatum habere zelum , } ut sit inter eos amicitia naturalis delectabilis , et honesta . & e deue auer acuçia conuenible de su casa . \textbf{ Ca assi conuiene a cada vn marido de auer çelo ordenado de su mugni } por que sea entre ellos amistança natural delectable e honesta \\\hline
2.1.23 & omnia minora et debiliora \textbf{ citius veniunt ad suum complementum . } Consilium ergo mulieres , & en el libro delas aian lias las ainalias menores e mas flacas . \textbf{ mas ayna vienen a su conplimienta . } Et por ende el consseio delas mugers \\\hline
2.1.23 & quam consilium virile , \textbf{ citius venit ad suum complementum . } Ceteris ergo paribus & e menos poderoso que el conseio delons \textbf{ uarones̃ mas ayna viene a su conplimiento . } Et por ende estando todas las \\\hline
2.1.23 & cito perducit \textbf{ ipsam ad suum augmentum . } Sic et mulier quantum & por que la natura ha poco cuydado della \textbf{ e ayna la aduze a su conplimiento . } ¶ En essa misma manera la muger \\\hline
2.1.23 & et natura minus de ipso curet , \textbf{ citius venit ad suum complementum , } quam virile . & e la natura ha menor cuydado del mas ayna viene \textbf{ a su conplimiento } que el cuerpo del uaron . \\\hline
2.1.23 & esse muliebre consilium melius quam virile : \textbf{ ut quia illud est citius in suo complemento , } sic oporteret repentino operari , & que del omne \textbf{ por que el consseio de la muger es mas ayna el su conplimiento que deluats . } por que si acaesçiesse de obrar alguna cosa adesora \\\hline
2.1.24 & Nam quia videntur \textbf{ a suis viris diligi ; } si sciant secreta ipsorum ; & alas otras las poridades de sus maridos \textbf{ por que parescan ser amadas de sus maridos } si sopieren las poridades dellos . \\\hline
2.1.24 & si possint se laudari \textbf{ quod a suis maritis diligantur , } appetentes quandam inanem gloriam , & si pueden ser loadas \textbf{ que son amadas de sus maridos . } por ende desseando alguna uana gloria \\\hline
2.1.24 & Ex hoc autem de facili apparet , \textbf{ qualiter viri suis coniugibus debeant } reuelare secreta . & Et por ende de ligero paresçe \textbf{ en qual manera los maridos de una descobrir a sus mugieres los sus secretos . } Ca quando nos dezimos \\\hline
2.1.24 & Viri igitur non debent \textbf{ suis coniugibus secreta aperire , } nisi per diuturna tempora sint experti , & Et pues que assi es los maridos \textbf{ non deuen descobrir a sus mugers las sus poridades } saluo a aquellas de que han prouado de luengot \\\hline
2.1.24 & quomodo Reges et Principes , \textbf{ et uniuersaliter omnes ciues se habere debeant ad suas coniuges , } et quomodo cum eis debeant conuersari , & en qual manera los Reyes e los prinçipes \textbf{ e generalmente todos los çibdad a uos se de una auer alus mugers } e como de una beuir con ellas . \\\hline
2.1.24 & in qua traditum fuit , \textbf{ quo regimine Reges et Principes debeant suas coniuges regere . } SECUNDA PARS Secundi Libri de regimine Principum : & quales obras conuiene que vsen las mugers \textbf{ Mas por que dellas diremos adelante } non quegremos a àmas dellas dezir . \\\hline
2.2.1 & magis incitabuntur parentes \textbf{ ut suos filios bene regant . } Possumus autem triplici via venari , & mas sian mouidos los padres \textbf{ para gouernar bien sus fijos } Et nos podemos mostrar por tres razones \\\hline
2.2.1 & quomodo per debita auxilia influat , \textbf{ et subueniat suis subiectis , } et quomodo eos regulet et conseruet . & por ayudas conueinbles enbie su uirtud \textbf{ e ayude a sus subditos } en qual manera los regle e los guarde . \\\hline
2.2.2 & magis habet solicitudinem circa filios : \textbf{ naturale est enim quemlibet diligere sua opera , } ut Philosophus in Ethicorum & mas ha cuydado de sus fijos \textbf{ Ca natural cosa es que cada vno ame sus obras } assi commo dize el philosofo en las ethicas \\\hline
2.2.2 & unde et patres naturaliter diligunt filios , \textbf{ et poetae sua poemata tanquam proprium opus . } Quando ergo aliquis est intelligentior , & Onde los padres naturalmente aman los fijos \textbf{ e los poetas sus ditados | assi commo obra proprea ¶ } Pues que assi es quando cada vno es mas entendido \\\hline
2.2.3 & perpetuari in seipsis , \textbf{ perpetuantur in suo simili . } Quare si regimen paternum & por que lo qua nonpite de ser durable \textbf{ en ssi sea durable en ssemeiante . } Por la qual cosa si el gonernamiento del padre desto tora a comienco \\\hline
2.2.3 & quando potest sibi simile generare . \textbf{ Quare cum quilibet suam perfectionem diligat , } naturaliter pater diligit filium , & quando ꝑuede engendrar su semeiante . \textbf{ Et commo quier que cada vno ame su perfecçion . } Emperona traalmente el padre ama el fijo \\\hline
2.2.4 & Nam parentes magis sunt certi \textbf{ de sua prole } quam proles de suis parentibus : & La segunda razon para puar esto mismo se toma dela c̀tidunbre de los fueros \textbf{ Ca los padres mas ciertos son de los fiios } que los fijos de los padres . Ca los fijos non pueden ser \\\hline
2.2.4 & de sua prole \textbf{ quam proles de suis parentibus : } proles enim certificari non potest & Ca los padres mas ciertos son de los fiios \textbf{ que los fijos de los padres . Ca los fijos non pueden ser } çiertos quales fueron sus padres \\\hline
2.2.5 & Si enim in aliis legibus parentes statim sunt soliciti erudire proprios filios \textbf{ in iis quae sunt fidei suae , } et in iis quae pertinent ad eorum legem , & Ca si e las otras leyes los padres son acuçiosos de enssennar sus fijos en aquellas cosas \textbf{ que son de su fe } Et en aquellas cosas \\\hline
2.2.7 & ut per ipsum possent omnes \textbf{ suos conceptus sufficienter exprimere . } Quare si hoc idioma est completum , & que por el pudiessen razonar e mostrar \textbf{ acabadamente te dos los sus conçibimientos . } Por la qual \\\hline
2.2.7 & ad intelligendum et ad cognoscendum naturas rerum : \textbf{ homo tamen a sui natiuitate est } male dispositus & e conosçer las uaturas delas cosas . \textbf{ Enpero el ome comneco de su nasçimiento es mal despuesto } e mal ordenado \\\hline
2.2.7 & et maxime Reges , et Principes , \textbf{ si volunt suos filios distincte } et recte loqui literales sermones , & e alos prinçipes \textbf{ si quisieren | que los sus fijos departidamente } e derechamente fablen las palabras delas letras \\\hline
2.2.8 & sub tali enim sermone Philosophi \textbf{ suam scientiam tradiderunt . } Quare si per nosipsos & Ca los philosofos dauna \textbf{ e mostra una su sçiençia } por tal \\\hline
2.2.8 & ut innuit Philosophus \textbf{ in Rhetoricis suis , } quasi quaedam grossa dialectica . & Mas la rectorica \textbf{ assy commo dize el philosofo en la rectorica es } assi commo vna gruessa logica . \\\hline
2.2.8 & nisi studerent literalibus disciplinis , \textbf{ et nisi suis exercitiis } interponerent delectationes musicales , & si non estudiassen en las sçiençias liberales \textbf{ e si non entrepusiessen alguas vezes en sus obras } de alguas delectaçonnes de cantares e de musica \\\hline
2.2.8 & Adhuc quaedam morales scientiae , \textbf{ ut Ethica , quae est de regimine sui , } et Oeconomica , quae est de regimine familiae : & Et avn las sçiençias \textbf{ assi commo la hetica | que es del gouernamiento del omne en ssi mismo . } Et la y conomica \\\hline
2.2.8 & Nam sicut laici et vulgares , \textbf{ quia arguunt et formant rationes suas , } quem modum arguendi docet dialectica , & e los omes del pueblo \textbf{ por que argumentan | e forman sus razones rudamente e sin arte . } La qual manera de argumentar muestra la logica . \\\hline
2.2.8 & et Principes scire idioma literale , \textbf{ ut possint secreta sua alii scribere } et legere absque aliorum scitu . & de saber el lenguage delas letris \textbf{ por que puedan alos otros escuirles sus poridades } e leerlas \\\hline
2.2.8 & ad intelligendum quaecunque proposita : \textbf{ quo facto totum suum ingenium debent exponere , } ut bene intelligant moralia & que les sean propuestas \textbf{ la qual casa fechͣ deuen poner todo su en gennio } porque puedan bien entender las sçiençias m orales \\\hline
2.2.9 & sic decet esse diligentem et cautum , \textbf{ ut proponat suis auditoribus vera } sine admixtione falsorum . & assi commo conuiene al doctor e al maestro en las sçiençias especulatiuas de ser acuçioso e sabio \textbf{ en manera que proponga a sus disçipulos cosas uerdaderas } sin ningun mezclamiento de cosas falssas . \\\hline
2.2.9 & et uniuersaliter omnes ciues valde solicitantur , \textbf{ qualem proponant suis numismatibus , } possessionibus , et rebus inanimatis : & deuen ser muy acuçiosos \textbf{ en catar qual mayordomo deuen poner en sus riquezas } e en sus posessiones e enlas o triscosas \\\hline
2.2.11 & quod vix aut nunquam comedere possunt , \textbf{ quin sua vestimenta deturpent : } turpitudo autem corporalaris licet & Los quales abeso nunca pueden comer \textbf{ que non enlixen sus vestiduras . } Mas la torpedat del cuerpo \\\hline
2.2.12 & ratio nostra \textbf{ quantum ad suos actus , } quia non possumus libere ratione uti . & El qual meollo turbado ciegasse el entendimiento \textbf{ quanto alas sus obras } por que non podemos libremente vsar de razon . \\\hline
2.2.12 & ex inflammatione sanguinis , \textbf{ vinum , quod propter sui caliditatem inflammat sanguinem , } reddit hominem animosum et irascibilem : & por que la sanna seleunata dela inflamaçion dela sangre . \textbf{ Et el vino | por su calentura en flama } e ençiende la sangre \\\hline
2.2.13 & et aliquas deductiones \textbf{ interponere suis curis , } ut ex hoc aliquam requiem recipientes , & conuienel de entreponer alguons trebeios \textbf{ e algunos solazes en sus cuydados . } assi que en esto resçibiendo alguno folgua a puedan mas trabaiar para alcançar su fin . \\\hline
2.2.13 & gestus ordinatos et honestos : \textbf{ cohibent enim sua membra , } ne aliquem motum habeant , & e los buenos han gestos ordenados e honestos \textbf{ por que estos tales costramnen e apetan sus mienbros } por que non ayan algun mouimiento \\\hline
2.2.16 & nec omnia agant valde \textbf{ sed in suis actibus } et sermonibus moderationem accipiant . & nin fagan todas las cosas \textbf{ que fazen mas que deuen mas en todas sus obras } e en todas sus palabras \\\hline
2.2.17 & quando habet tale corpus , \textbf{ quale requirit suum officium : } ut tunc miles habet corpus bene dispositum , & quando ha tal cuerpo \textbf{ qual demanda el su ofiçio } assi con no dezios que el cauallero \\\hline
2.2.17 & sed sint subiecti \textbf{ et obedientes suis patribus et senioribus . } Tangit autem Philosophus & por que non sean orgullosos \textbf{ mas que sean subiectos e obedientes a sus padres et alos uieios . } Mas el pho pone en el vii̊ . \\\hline
2.2.17 & quia tunc quasi peruenerunt \textbf{ omnimode ad suam perfectionem , } debent esse tales , & e del ayo \textbf{ por que endçe vienen del todo a su perfecçion } e deuen ser tałs \\\hline
2.2.17 & sed per ea quae diximus in primo libro \textbf{ qui est de regimine sui , } possunt documenta accipere , & que dixiemos en el primero libro \textbf{ que es del gouerna mi Mait de su milmo pueden tomar enssennamientos } en qual manera reglen assi mismos . \\\hline
2.2.18 & a Regibus et Principibus , \textbf{ et a suis haeredibus } magis est vitanda inertia , & e abueans costunbres pueden ellos \textbf{ e sus herederos } e todos gouernadores meior escusar la peza et el uagar \\\hline
2.2.19 & ne indebite circuant et discurrant : \textbf{ quanto ex impudicitia et lasciuia suarum filiarum potest } maius malum vel periculum imminere . & e salgan fuera \textbf{ quanto dela locania | e dela desuergonança delas sus fijas } puede contesçer mayor mal e mayor periglo . \\\hline
2.2.20 & Quia igitur omnes delectantur in propriis operibus , \textbf{ et omnes diligunt sua opera , } ut vult Philosophus 9 Ethicorum , & en sus obras propreas \textbf{ e todos aman las sus obras } assi commo dize el philosofo \\\hline
2.2.21 & ut magis appareant ornatae et decentes , \textbf{ et ut a viris suis magis diligantur . } Secunda , ne loquantur indebite et incaute . & e mas apuestas en sus faziendas \textbf{ et por que sean mas amadas de sus maridos | ¶ La ijn se toma } por que non \\\hline
2.2.21 & quia si contingat eas postmodum \textbf{ per connubium suis viris copulari , } ab eis , & e non aparlar \textbf{ ca si contesçiere despues que sean ayuntadas a sus maridos } por casamientos \\\hline
2.3.1 & qualiter decet \textbf{ uiros suas coniuges regere , } et qualiter patres suos filios gubernare . & Ca es mostrai ser do \textbf{ en qual manera conuiene a los maridos de gouernar a sus mugers . } Et en qual maneta los padres deuen regir e gouernar a sus fijos . \\\hline
2.3.1 & uiros suas coniuges regere , \textbf{ et qualiter patres suos filios gubernare . } Restat exequi de parte tertia , & en qual manera conuiene a los maridos de gouernar a sus mugers . \textbf{ Et en qual maneta los padres deuen regir e gouernar a sus fijos . } finca de dezer dela terçera ꝑte \\\hline
2.3.1 & eo quod hae materiae sunt connexae , \textbf{ intendimus instruere uolentem suas domus debite gubernare , } non solum quantum ad regimen ministrorum et familiae , & por que estas materias son ayuntadas en vno entendemos de enssennar \textbf{ a aquellos que quisieren | conueinblemente gouernar sus calas } non lo lamente quanto al \\\hline
2.3.1 & Nam sicut ceterae artes , \textbf{ ut ars fabrilis , et textoria , habent sua organa , } per quae perficiunt actiones suas : & Ca assi commo las otras artes \textbf{ assi commo es arte de ferreria | e de texederia han sus estrumentos } por los quales acaban sus obras . \\\hline
2.3.1 & ut ars fabrilis , et textoria , habent sua organa , \textbf{ per quae perficiunt actiones suas : } sic et gubernatio domus requirit sua organa , & e de texederia han sus estrumentos \textbf{ por los quales acaban sus obras . } En essa misma manera el arte del gouernamiento dela casa demanda sus estrumentos \\\hline
2.3.1 & per quae perficiunt actiones suas : \textbf{ sic et gubernatio domus requirit sua organa , } per quae opera sua complere possit . & por los quales acaban sus obras . \textbf{ En essa misma manera el arte del gouernamiento dela casa demanda sus estrumentos } por los quales pueda conplir sus obras . \\\hline
2.3.1 & sic et gubernatio domus requirit sua organa , \textbf{ per quae opera sua complere possit . } Volens ergo tradere notitiam de arte fabrili , & En essa misma manera el arte del gouernamiento dela casa demanda sus estrumentos \textbf{ por los quales pueda conplir sus obras . } Et por ende los que quieren dar conosçimiento dela arte del ferrero \\\hline
2.3.4 & vel ne transeat per metallorum venas : \textbf{ habent enim metalla suas venas subterraneas , } et aqua in locis subterraneis generatur , & por las venas de los metalles \textbf{ por que los metalles han sus careras sotercannas } e el agua se engendra en los logares soterrannos \\\hline
2.3.4 & si aedificium \textbf{ secundum suam ampliorem partem respiciat oriens hyemale : } tunc enim eo quod in hyeme oppositum sit soli , & Lo primero puede contesçer si la morada \textbf{ segunt la su parte mayor | catare a lorsete del yuierno } por que estonçe la morada \\\hline
2.3.5 & quod dominetur istis sensibilibus , \textbf{ et quod possit eis uti in suum obsequium , } et quia hoc est quodammodo possidere ea , & que enssennore e a estas cosas senssibles \textbf{ e que pueda vsar dellas | e resçebir seruiçio dellas } segunt quel fuere uisto \\\hline
2.3.6 & sed quilibet ad quamlibet \textbf{ pro sua voluptate accederet , } esset suprema unitas , & mas cada vno se llegasse \textbf{ a qual quisiesse por su uoluntad } que por esto serie grand ayuntamiento \\\hline
2.3.6 & eo quod nesciret pater \textbf{ quis puer filius suus esset , } sed reputaret quemlibet proprium filium , & sabrie \textbf{ qual moço fuesse su fijo } mas cuydarie \\\hline
2.3.7 & ut quod licitum esset \textbf{ non solum hos depraedari et accipere sua , } sed eos etiam accipere in praeda , & que la uida del robar es conuenible \textbf{ e que non solamente los omes deuen tomar et robar las sus cosas alos otros } mas avn deuen tomar a ellos en su perssona \\\hline
2.3.8 & si gubernator domus non vult contra naturam agere , \textbf{ sed vult suam domum regere } secundum modum et ordinem naturalem , & si el gouernador dela casa quiere fazer contra natura \textbf{ mas si quiere gouernar su casa } segunt manera natural \\\hline
2.3.8 & quot \textbf{ secundum exigentiam sui status } bene sufficiant ad gubernationem domus , & Mas quando ha tantas riquezas \textbf{ qual abastan segunt el mester de su estado deue ser pagado } dellas quanto al gouernamiento de su casa . \\\hline
2.3.8 & et diuitiis , \textbf{ quantas requirit exigentia sui status . } Nam non satiari possessionibus & e de tantas riquezas \textbf{ quantas demanda el menester de su estado } ca non se fartar omne de possessiones \\\hline
2.3.9 & ut cognoscendo , \textbf{ melius sciat suae domui prouidere . } Conuenienter post tractatum de possessionibus & por que sabien do esto \textbf{ sepa meior proueer su casa } onueinblemente depues del tractado de las possessiones \\\hline
2.3.10 & habere aliqua numismata , \textbf{ quae non multum appretiabantur in regione sua , } eo quod non esset propria regioni illi : & que algunos por auentura han alguas monedas \textbf{ que non son muy preçiadas en su regno } por que non son propreas de aqual regno . \\\hline
2.3.11 & uel uenditur ibi usus \textbf{ qui non est suus . } Ad cuius euidentiam sciendum , & uegadaso se vende el uso \textbf{ y que non es suyo dela cosa . } Et para esto entender \\\hline
2.3.11 & usus pertinet ad ipsum . \textbf{ Vendens ergo quod suum est , } et quod pertinet ad ipsum , & a aquel cuya es la sustançia . \textbf{ Et pues que assi es el que vende lo que es suyo } e lo que parte nesçe a el \\\hline
2.3.11 & Quare si de usu pensionem accipiat , \textbf{ uendit quod non est suum , } uel accipit pensionem & dende adelante non parte nesçe a ellos el uso della . \textbf{ Por la qual cosa el que resçibe ganançia del uso del dinero vende lo que non e suyo o tomagat saçia de aquello que non parte nesçe ael } por que dende adelante non pertenesçe ael el uso del diuero \\\hline
2.3.11 & est apparere : \textbf{ multi enim ostendunt denarios suos non ad expendendum , } sed ad apparendum & Vso non prop̃o es paresçer con ellos \textbf{ Ca muchͣs demuestran sus dineros non para despender los } mas para paresçer con ellos \\\hline
2.3.12 & et improperaretur sibi a multis cur philosopharetur , \textbf{ et ad quid valeret Philosophia sua , } cum semper in egestate viueret . & e le denostassen sus amigos \textbf{ diziendol que por que se daua tanto ala ph̃ia } e aquel aprouechaua suph̃ia pues siengͤ biue en pobreza e en mengua . \\\hline
2.3.12 & taxat precium \textbf{ pro suae voluntatis arbitrio : volentem ergo pecuniam acquirere , } oportet haec & commo se el quiere \textbf{ por su uoluntad e por su aluedrio . | Et por ende el que quiere gana rriqueza } conuiene le de tener enla memoria estos fechs particulares e otros semeiantes \\\hline
2.3.12 & Quinta via dicitur esse artifica , \textbf{ quando quis per artem suam aliqua exerceret , } propter quae pecuniam lucratur . & ¶ La quarta manera es dicha artifiçial esto es \textbf{ quando alguno | por su arte faze alguas obras } por que gana dineros . Ca commo quier la fin dela arte dela caualłia sea uictoria \\\hline
2.3.12 & habere curam de acquisitione pecuniae , \textbf{ secundum quod exigit suus status : } Apud Reges autem , & segunt uida politica de auer cuydado de ganar dineros segunt que requiere \textbf{ e demanda el su estado de cada vno . } Mas alos Reyes e alos prinçipes \\\hline
2.3.15 & ex virtute et dilectione ; \textbf{ licet enim dignus sit operarius mercede sua , } et non sit congruum aliquem proprii stipendiis militare . & por uirtud e por amor . \textbf{ Et commo quier que el merçenario sea digno de su merçed } e non sea cosa conuenible \\\hline
2.3.17 & Quia maxime apparet Regis prudentia , \textbf{ si suam familiam debito modo gubernet , } et si ei debite et ordinate necessaria tribuat : & or que mucho paresçe la sabiduria del Rey \textbf{ si gouernare en manera conuenible a su conꝑannappra } e sil diere ordenadamente \\\hline
2.3.17 & ut supra in primo libro diffusius probabatur , \textbf{ decet ipsum erga suos ministros decenter se habere in apparatu debito , } et in debitis indumentis . & assi commo es prouado mas conplidamente en el primero libro \textbf{ conuiene les de auer sus siruientes apareiados | conueniblemente en el parescer de fuera } e en uestiduras conuenibles \\\hline
2.3.17 & tamen ut Reges et Principes conseruent se \textbf{ in statu suo magnifico , } et ne a populis condemnantur , & e de una fazer tales cosas . \textbf{ Enpero por que los Reyes e los prinçipes sean guardados en su estado granado } e por qua non sean despreçiados de los pueblos conuieneles de fazer grandes \\\hline
2.3.17 & nec aeque pulchris indumentis gaudere debent , \textbf{ sed considerata conditione personarum sic secundum suum statum cuilibet sunt talia tribuenda , } ut in hoc appareat prouidentia et industria principantis . & nin deuen gozar egualmente de uestiduras fermosas . \textbf{ Mas penssada la condiçion delas personas | assi se deue partir acadera vno dellos segunt el su estado } por que en esto parezca la sabiduria \\\hline
2.3.18 & ita quod non sit memoria \textbf{ in populo progenitores suos fuisse pauperes , } dicitur habere nobilitates generis , & assi que non sea memoria en el pueblo \textbf{ que los sus padres | nin los sus auuelos fueron pobres } e estos tales son dichos \\\hline
2.3.18 & quod est opus temperantiae . \textbf{ Curiales etiam dicuntur homines se habere erga suos ciues , } si non eis iniuriam inferant in uxoribus , & nin torpemente la qual cosa es obra de tenprança . \textbf{ avn los omes son dicho | que } seancurialmente contra sus çibdadanos sinon les fezieren tuerto en las mugers \\\hline
2.3.18 & et a iustitia legali , et a curialitate : \textbf{ ut si quis suis conciuibus bona sua prompte largitur , } si haec agit , & et de curialidat \textbf{ assi commo si alguno dieres o bienes alos sus çibdadanos liberalmente } si esto faze \\\hline
2.3.18 & Sunt enim multi facientes opera virtutum \textbf{ ut bona sua aliis largientes , } non agentes hoc quia eis placeat expendere ; & que fazen obras de uirtudes \textbf{ partiendo los sus bienes alos otros } e esto non lo fazen \\\hline
2.3.18 & habere mores nobiles et curiales , ministros , \textbf{ quos in bonis decet suos dominos imitari , } oportet curiales esse . & e de ser curiales e nobles \textbf{ assi conuiene alos seruientes dellos | los que quieren semeiar a sus sennors } de ser buenos e mesurados e corteses . \\\hline
2.3.20 & et obseruare ordinem naturalem \textbf{ omnino in suis mensis , } ordinare debent & e guardar la orden natural en toda \textbf{ meranera deuen ordenar en sus mesas } que los que se assentaren \\\hline
3.1.1 & omnia operantur omnes . \textbf{ Si ergo omnes homines ordinant sua opera in id quod videtur bonum , } cum ciuitas sit opus humanum , & que les paresçe buean . \textbf{ ¶ Et pues que assi es si todos los omes ordenan sus obras | a aquello que paresçe bien } o commo la çibdat \\\hline
3.1.1 & cuiusmodi est communitas regni , \textbf{ de qua suo loco dicetur : } ostendemus enim communitatem regni & que ella la qual es comunidat del regno \textbf{ dela qual diremos en su logar | ca mostraremos } que la comunidat del regno es prouechosa en la uida humanal \\\hline
3.1.1 & et esse principaliorem communitate ciuitatis . \textbf{ Videtur enim suo modo communitas regni } se habere & que la comunidat dela çibdat ca paresçe \textbf{ que assi se ha la comunidat del regno } ala comunidat dela çibdat \\\hline
3.1.2 & habere sufficiens esse \textbf{ secundum naturam suam , } quae secundum suam speciem habent esse completum . & auer el ser conplidamente \textbf{ segunt su natura } que han ser conplido \\\hline
3.1.2 & secundum naturam suam , \textbf{ quae secundum suam speciem habent esse completum . } Si enim alicui rei deficiat & segunt su natura \textbf{ que han ser conplido | segunt su natura } e su linage \\\hline
3.1.2 & Si enim alicui rei deficiat \textbf{ aliqua perfectio competens suae speciei , } licet possit habere illa res esse aliquod , & por que si a algunan cosa fallesçiere algun acabamiento \textbf{ que pertenezca ala suspeno ala su semeiança } commo quier que puede auer aquella cosa \\\hline
3.1.2 & Irrationalia ergo et etiam inanimata \textbf{ secundum naturam suam possunt } habere esse sufficiens , & que non han razon e las cosas \textbf{ que non han alma | segunt su natura } pueden auer su ser conplido \\\hline
3.1.4 & et cum tristatur potest \textbf{ alteri cani per suum latratum } significare tristitiam , & e en otra manera \textbf{ quando se trista puede a otro can de mostrar } por su ladrado sutsteza o su delecta conn que ha mas al omne \\\hline
3.1.7 & magnam difficultatem esse , \textbf{ ut narrat Philosophus in Metaphysica sua , } conuertit se ad Moralia , & ueyendo muy grant guaueza cerca la sciençia natural \textbf{ assi commo cuenta el pho | enla su mecha phisica } conuirtiosse alascina moral . \\\hline
3.1.7 & conuertit se ad Moralia , \textbf{ quem Plato suus discipulus in multis secutus est , } propter quod Philoso’ Platonem ipsum & conuirtiosse alascina moral . \textbf{ al qual socrates siguio platon su disçipulo en muchͣs cosas } por la qual cosa el philosofo aristotiles llamo a platon el segundo socrates . \\\hline
3.1.7 & et quod crederent \textbf{ eos esse suos filios , } illi vero opinarentur & por que los antiguos creerian \textbf{ que los moços eran sus fiios } e los mocos cuydarian \\\hline
3.1.7 & illi vero opinarentur \textbf{ eos esse suos patres . } Tertium vero quod senserunt & e los mocos cuydarian \textbf{ que ellos eran sus padres . } ¶ Lo terçero que sintieron los dichs philosofos cerca el gouernamiento dela çibdat . \\\hline
3.1.7 & ut quod fiant subditi , \textbf{ et deponantur a magistratibus suis . } Quintum autem quod dicti Philosophi senserunt statuendum circa ciuitatem , & assi que sean subditos \textbf{ e sean despuestos de sů maestradgos . } Mas lo quinto que los dichs phos sintieron \\\hline
3.1.8 & et ad hoc quod uniuersum \textbf{ secundum suum statum sit maxime perfectum , } oportet ibi dare diuersa secundum speciem . & que son mester en el mundo sean en el mundo \textbf{ e por que el mundo segunt su estado sea muy acabado } conuietie de dar en el \\\hline
3.1.9 & cessarent litigia , \textbf{ quia crederent ciues omnes pueros esse filios suos , } et sic esset in ciuitate maximus amor . & e las contiendas enla çibdat \textbf{ por que cuydarian los çibdadanos | que todos los moços eran sus fijos propreos } e por ende en la çibdat seria muy grant amor Et pues que assi es nos podemos mostrar \\\hline
3.1.10 & propter honestatem \textbf{ et bonitatem morum parentes esse certos de suis filiis , } et quoslibet certificari de eorum consanguineis , & por bondat e honestad de costunbres \textbf{ que los padres sean çiertos de sus fiios } e cada vnos sean çiertos de sus parientes \\\hline
3.1.10 & de facili \textbf{ propter ignorantiam iniurari consanguineis suis . } Secundum malum sic ostenditur . & por esta non sabiduria \textbf{ nin çertidunbre farien tuerto alos parientes e asus padres . } El segundo mal se muestra \\\hline
3.1.10 & quando quilibet se habet \textbf{ secundum proportionem suam , } ut quando ignobiles seruient nobilibus , & quando cada vno se ha \textbf{ segunt su proporcion | e segunt el su estado } e a cada vno es guardado su derecho \\\hline
3.1.10 & nullo modo suspicari posset \textbf{ omnes pueros esse suos filios . } Si ergo omnes diligerent tanquam filios , & si non fuesse loco en ninguna manera non podria sospechͣr \textbf{ que todos los moços fuessen sus fijos . } Et pues que assi es si a todos amassen \\\hline
3.1.15 & de rebus aliorum , \textbf{ ac si essent suae . } In uxoribus autem ex filiis debet & o quanto pudiere delas cosas de los otros \textbf{ assi commo si fuessen suyas . } Mas en las mugers e en los fijos deue ser guardada comunidat \\\hline
3.1.15 & ( ut dictat ratio ) \textbf{ bona sua communicare . } Saluauimus igitur dictum Socraticum & deue enprestar alos otros çibdadanos los sus bienes \textbf{ assi conmoiudga la razon . } Pues que assi es assi commo saluamos el dicho \\\hline
3.1.16 & et scirent se non posse \textbf{ excedere suos conciues in possessionibus , } frustra propter hoc insurgerent lites et placita . & que vno non podia sobrepuiar \textbf{ los otros çibdadanos sus uesnos en possessiones } por esto debalde se leunatarian entre ellos las contiendas e las uaraias \\\hline
3.1.18 & ut quod quilibet \textbf{ quod suum est possideat : } sed multa ordinare decet & en mesurar las possessiones \textbf{ assi que cada vno aya lo que suyo es } Mas conuiene les de ordenar muchͣs cosas \\\hline
3.1.19 & diuersa genera personarum . \textbf{ Hippodamus autem statuens suam politiam , } primo intromisit se de multitudine & que tannian alguons linages de personas dezimos \textbf{ que y podo mio | establesciendo su poliçia } primero se entremetio dela muchedunbre \\\hline
3.1.19 & Dicebat autem quod audita causa quilibet iudex per se cogitaret , \textbf{ et postea in pugillaribus scriptam adduceret suam sententiam : } ut si incusatus simpliciter condemnandus esset , & por si deuia penssar \textbf{ e despues poner sus nina en esc̀pto } assi que si el acusado sin ninguna condiçion fuesse de condepnar el \\\hline
3.1.19 & Ideo ordinauit \textbf{ quod quilibet priuatim sententiam suam scriberet . } Sexto statuit quasdam leges , & tristemiendosse dellos \textbf{ e por ende ordeno que cada vno apareiadamente ordenasse su suina } e la diesse por escpto ¶ \\\hline
3.1.19 & nolebant enim \textbf{ ( ut apparet ex dictis suis ) } principem debere esse per haereditatem , & assi commo paresçe \textbf{ por los sus dichos } que el prinçipe non deue ser fecho \\\hline
3.1.19 & uniuersaliter omnes personas impotentes , \textbf{ non valentes per se ipsas sua iura conquirere . Spectat enim ad Regem et Principem , } qui debet esse custos iusti , & nin podian \textbf{ por si mismas guardar su | derechca parte nesçe al Rey e al } prinçipeque deue ser guardador dela iustiçia de auer cuydado espeçial delas cosas comunes \\\hline
3.1.19 & eo quod talibus alii de facili iniuriantur , \textbf{ cum non possint defendere iura sua . } Multa bona consequimur & por que tales perssonas los otros de ligero les fazen tuerto \textbf{ por que non pueden defender su derecho } uchos bienes se nos siguen delas opiniones de los phos antigos \\\hline
3.1.20 & ex opinionibus antiquorum Philosophorum , \textbf{ eo quod ipsi in suis dictis multa bona et vera dixerunt . } Dato tamen quod nihil veri dixissent , & uchos bienes se nos siguen delas opiniones de los phos antigos \textbf{ por que ellos en sus dichos dixieron muchsbieño | e uerdaderas cosas . } Enpero puesto que non dixiessen algua cosa uerdadera \\\hline
3.1.20 & Hippodami ergo opinionem recitauimus , \textbf{ quia in sua politia multas bonas sententias promulgauit : } aliqua tamen incongrue statuit . & e por ende contamos la opinion de ipodomio \textbf{ por que el en la su poliçia manifesto muchͣs bueanssmans . } Empo algunas cosas establesçio non conuenible mente . \\\hline
3.2.2 & et omnium ciuium \textbf{ secundum suum statum , } sic est aequale et rectum . & e el bien de todos los çibdadanos \textbf{ segunt su estado } assi es sennorio ygual e derech̃ . \\\hline
3.2.2 & mediarum personarum , et diuitum , \textbf{ et omnium secundum suum statum : } et tunc est rectus et aequalis : & e delas perssonas medianeras e de los ricos \textbf{ e de todos comunalmente | segunt su estado } estonçe el prinçipado es derech e ygual . \\\hline
3.2.2 & si rectus sit . \textbf{ Sed si populus sic dominans non intendit bonum omnium secundum suum statum , } sed vult tyrannizare et opprimere diuites , & si derecho es . \textbf{ Mas si el pueblo assi | enssennoreante non entiende a bien de todos } segunt su estado \\\hline
3.2.5 & quanto credit ipsum regnum \textbf{ magis esse bonum suum et bonum proprium : } quare si Rex videat & que el regno es mas su bien \textbf{ e mas su bien propo | que de otro ninguno } Por la qual cosa si el Rey viere \\\hline
3.2.5 & magis reputabit bonum regni \textbf{ esse bonum suum , } et ardentius solicitabitur & mas avn por heredat en sus fijos . \textbf{ mas terna que el bien del regno es su bien propreo } e con mayor \\\hline
3.2.6 & nisi de delectationibus propriis , \textbf{ maxime versatur sua intentio circa pecuniam , credens se per eam posse huiusmodi delectabilia obtinere . } Sed regis intentio versatur circa virtutem , & si non delas sus delectaçonnes propreas . \textbf{ Et por ende la su entençion toda se pone en el auer | o en los dinos creyendo que por ellos puede auer las otras cosas delectables . } Mas la entençion del Rey esta \\\hline
3.2.6 & qui sunt in regno , \textbf{ totam suam custodiam corporis committit extraneis : } sed Rex econuerso eo & de aquellos que son en el regno . \textbf{ Por ende toda la guarda del su cuerpo | acomienda a omnes estrannos . } Mas e Rey faze todo el contrario \\\hline
3.2.7 & tunc est tyrannus et est pessimus , \textbf{ quia propter suam unitam potentiam potest } multa mala efficere . & et el su prinçipado es muy malo \textbf{ ca por el su poderio muy grande } que es ayuntado en vno puede fazer muchs males \\\hline
3.2.8 & quod natura primo dat rebus ea per quae possunt \textbf{ consequi finem suum . } Secundo dat eis ea & que la natura primeramente da a todas las cosas \textbf{ aquello por que pueden alcançar su fin . } ¶ Lo segundo les da aquellas cosas \\\hline
3.2.8 & Tertio per huiusmodi collata naturaliter intendunt \textbf{ in suos fines siue in suos terminos . } Ut natura dat igni leuitatem , & por estas cosas que les da la natura . \textbf{ naturalmente una a sus terminos o a ssus fines } assi commo paresçe por este exenplo \\\hline
3.2.8 & Debet igitur Rex solicitari \textbf{ ut in suo regno uigeat studium litterarum , } et ut ibi sint multi sapientes et industres . & Et por ende el rey deue ser muy acuçioso \textbf{ por que en el su regno aya estudio de letris } e por que sean y muchos sabios \\\hline
3.2.8 & spectat igitur ad rectorem regni ordinare \textbf{ suos subditos ad virtutes . } Tertio ad consequendum finem & Et por ende parte nesçe al gouernador del regno de otdenar sus \textbf{ subditosa uirtudes e a buenas costunbres ¶ } Lo terçero para alcançar la fin \\\hline
3.2.8 & naturaliter appetunt perpetuari \textbf{ in suis filiis siue sint naturales siue adoptiui . } Videtur enim homini quasi post mortem viuere , & dessean naturalmente de durar en sus fijos . \textbf{ si quier sean naturales | siquier por fuados . } por que paresçe alos bueons \\\hline
3.2.8 & si eo decedente \textbf{ secundum suam institutionem } alius in haereditatem succedat . & que biuen despues de su muerte \textbf{ si despues que ellos morieren segunt su } establesçimientero otro aya hedat en sucession \\\hline
3.2.9 & ne contemnantur a populis , \textbf{ non deberent suam intemperantiam ostendere : } laudatur enim sobrietas et temperantia , & por qua non sean menospreçiados de los pueblos \textbf{ non deuen mostrar su destenpramiento alos otros . } Ca sienpte es de alabar la mesura e la tenprança . \\\hline
3.2.9 & Nono decet verum Regem per usurpationem et iniustitiam \textbf{ non dilatare suum dominium . } Nam ut dicitur Polit’ & ¶ Loye conuiene al Rey uerdadero de non enssanchar su regno \textbf{ por tomar lo ageno | por fuerça e sin iustiçia . } Ca assi commo dize el philosofo \\\hline
3.2.9 & Recitat autem Philosophus 5 Polit’ \textbf{ quod cum quidam Rex partem sui regni dimisisset , } quia eam forte iniuste tenebat : & en el quanto libro delas politicas \textbf{ que commo vn Rey dexasse vna parte de su regno . } por que por auentura non la tenie \\\hline
3.2.9 & nam licet semper se simulent iuste agere , \textbf{ tamen in multis suum dominium iniuste ampliant , } et aliorum haereditates sine ratione usurpant . & que fazen las cosas derechamente \textbf{ enpero en muchas cosas | enssancha su regno sin derecho } e toman las hedades de los otros \\\hline
3.2.9 & continget eum \textbf{ ut expedit suae saluti } semper in suis actibus prosperari . & e el su poderio aqui non puede ser ninguna cosa contraria legnia \textbf{ assi conmo cunple a su salut . } Et fazel ser sienpre bien \\\hline
3.2.9 & ut expedit suae saluti \textbf{ semper in suis actibus prosperari . } Immo propter sanctitatem regis , & assi conmo cunple a su salut . \textbf{ Et fazel ser sienpre bien | auentraado en todos sus fechos } Et por ende por la sanidat del Rey dios muchas uezes faze muchs bienes \\\hline
3.2.10 & per quas nititur \textbf{ tyrannus se in suo dominio praeseruare . } Prima cautela tyrannica , & nchas cautelas tanne el philosofo en el quinto libro delas politicas delas quales quanto par tenesçe alo presente podemos tomar diez . \textbf{ por las qualose esfuerça el tiranno de se mantener en su sennorio . } La primera cautela del tirano es matar los grandes omes e los poderosos . \\\hline
3.2.10 & et non intendere bonum commune sed proprium : \textbf{ ideo vellent omnes suos subditos } esse ignorantes et inscios , & mas en el su bien propreo . \textbf{ Por ende querrien que todos los sus subditos fuessen sin sabiduria e nesçios } por qua non conosçiessen \\\hline
3.2.11 & Contra haec ergo quatuor procurant tyranni perimere excellentes , \textbf{ ne sui subditi sunt magnanimi : } destruere sapientes , & Et pues que assi es contra estas quatto cosas procuran los tyranos de matar los nobles e los grandes \textbf{ por que los sus subditos non sean osados | nin de grandes coraçones . } Otrossi procuran de destroyr los sabios \\\hline
3.2.12 & Legitur enim de quodam tyranno , \textbf{ qui cum a fratre suo cotidie increparetur , } quare ipse semper tristis existeret , & ca leemos de vn tirano \textbf{ que cada dia era denostado de vn su hͣrmano } por que sienpre andaua triste \\\hline
3.2.12 & Cum enim populus principatur peruerse , \textbf{ non intendit quodlibet seruare in suo statu , } sed satagit opprimere nobiles , et insignes . & por que quando el pueblo enseñorea malamente non entiende guaedar a \textbf{ njnguno en su estado mas esfuercas } e quanto puede para abaxar los nobles e los altos \\\hline
3.2.14 & volumus alias rationes adducere , \textbf{ ostendentes quod si reges cupiant suum durare dominium , } summo opere studere debent & avn en este cpleo queremos adozjr otras rrazones \textbf{ para mostrar que si los rrey e cobdiçian de duar muncho } el su señorio es toda manera deuen estudiar \\\hline
3.2.14 & Reges ergo et principes \textbf{ si volunt suum durare dominium , } summe cauere debent & e los prinçipes \textbf{ si quieren durar en su sennorio } mucho deuen escusar \\\hline
3.2.14 & et si eos aliquo modo tyrannizare contingat , \textbf{ suam tyrannidem pro viribus moderare debent , } quia quanto remissius tyrannizabunt , & que en alguna manera ayan de tiranizar deue \textbf{ por toda su | fuerca atenprar la tirania } ca quanto mas poco tiranzar en tanto mas dura el su sennorio ¶ \\\hline
3.2.15 & et quae oportet facere Regem ad hoc \textbf{ ut se in suo principatu praeseruet . } Primo est , non permittere in suo regno transgressiones modicas . & las quals conuiene al Rey de fazer \textbf{ para que se pueda man tener en lu prinçipado e en lu lennorio ¶ } La primera es que non consienta en su regno muchos pequanos males \\\hline
3.2.15 & ut se in suo principatu praeseruet . \textbf{ Primo est , non permittere in suo regno transgressiones modicas . } Nam multae modicae transgressiones & para que se pueda man tener en lu prinçipado e en lu lennorio ¶ \textbf{ La primera es que non consienta en su regno muchos pequanos males } ca muchs pequannos males \\\hline
3.2.15 & ex quo ille \textbf{ et antecessores sui obtinuerunt huiusmodi principatum , tanta cautela non magnam utilitatem habere videtur . } Quartum autem quod politiam saluare videtur , & despues que el e sus \textbf{ anteçessoresouieron aquel regnado . | En estos tal es aquella cautela } que dichͣes non paresçe \\\hline
3.2.15 & quod expendunt , et quomodo possunt \textbf{ reddere rationem sui victus : } nam qui huiusmodi rationem non potest reddere , & et commo pueden dar razon de su uida \textbf{ e de comm̃ se mantienen } ca aquel que non puede dar razon desto señal \\\hline
3.2.17 & attamen imprudens est \textbf{ qui solo suo capiti innittitur , } et renuit aliorum audire sententias . & enpero non es sabio aquel \textbf{ que se esfuerça en su cabeça sola } e menospreçia de oyr las suinas de los otros \\\hline
3.2.18 & Ut si boni sint , \textbf{ non mentiantur ratione sui , } quia bonis displicet omne malum , & e sean sabios e sean amigos . \textbf{ assi que si fueren bueons non mintran } por razon dessi \\\hline
3.2.19 & Primo , ne maiestas regia \textbf{ aliquos prouentus iniuste usurpet a suis conciuibus : } probabatur enim supra , & ¶Lo primero couiene que el Rey non tome ningunas rentas \textbf{ sin derecho de sus subditos . } ca prouado es de suso \\\hline
3.2.19 & Rursus est attendendum , \textbf{ ne in suis prouentibus defraudetur : } expedit enim regium consilium & si tomasse los bienes \textbf{ de aquellos que son en el su regno sin derech | lo segundo ha de tener mient̃s el Rey de non ser engannado enlas sus rentas . } ca conuiene que el conseio del Rey sea bue no para saluar \\\hline
3.2.19 & est pretium venditionis , \textbf{ si ( ultra quam debent ) venditores res suas vendere vellent . } Tertio , est consilium adhibendum & que se venden \textbf{ si los vendedores quisieren vender las cosas | mas de quando deuen ¶ } Lo terçero el conseio es de tomar çerca la guarda dela çibdat \\\hline
3.2.19 & et quomodo Rex se debeat \textbf{ in suo dominio praeseruare , } fuit in superioribus patefactum . & e qual deua ser el ofiçio del rey \textbf{ e en qual manera el Rey se deua guardar en su sennorio mostrado fue conplidamente en los dichos de ssuso . } Ca ya por los dichos dessuso \\\hline
3.2.20 & Contingit etiam absque corruptione et morte iudices \textbf{ a suo officio remoueri , } et alio in suum locum succedere . & que sin cornupçion \textbf{ e sin muerte los iuezes son tirados de sus oficlvii i̊ çios } e son prouestos otros en sir logar \\\hline
3.2.20 & a suo officio remoueri , \textbf{ et alio in suum locum succedere . } Igitur saltem per successionem ipsorum oportet & e sin muerte los iuezes son tirados de sus oficlvii i̊ çios \textbf{ e son prouestos otros en sir logar } Et pues que assi es si mas que non por alongamiento detp̃o conuiene \\\hline
3.2.21 & ad maliuolentiam partis aduersae , \textbf{ et ad beniuolentiam sui , } est omnino impertinens ad propositum : & e a mal querençia dela parte contraria \textbf{ e a bien querençia de ssi mismo . } Esto non pertenesçe en ninguna guasa al proposito . \\\hline
3.2.24 & Dicuntur enim iusta naturalia , \textbf{ quae sunt adaequata et proportionata ex natura sua , } vel dicuntur iusta naturaliter & derechsson dichos naturales \textbf{ por que son proporçionados e ygualados por su natura . } O son dichos natales \\\hline
3.2.24 & Iusta vero positiua dicuntur , \textbf{ quae non ex natura sua , } sed ex pacto hominum & e las leyes positiuas son dichos aquellos \textbf{ que non por su natura son tales . } mas por ponimiento de los o en so \\\hline
3.2.26 & quae discordat a toto , \textbf{ et quae suo non congruit uniuerso : } si in legibus intenditur aliquod bonum proprium , & que desacuerda del su todo \textbf{ e que non acuerda con el su todo } si en las leyes es entendido algun bien \\\hline
3.2.28 & et diligenter per se \textbf{ et suos consiliarios discutiant } quae bona sunt praecipienda et praemianda , & asi que con grant acuçia \textbf{ por si e por sus consseieros examun en quales bueans obras son demandar } e de poner so mandamiento . \\\hline
3.2.29 & est tamen supra legem positiuam , \textbf{ quia illam sua auctoritate constituit . } Itaque sicut Rex nunquam recte regit , & Enpero es sobre la ley positiua \textbf{ por que establesçio el aquella ley con su auctoridat . } Et pues que assi es assi commo el rey nunca gouierna derechamente \\\hline
3.2.30 & et rationabiliter fieri possunt , clementia et seueritas simul cum iustitia possunt existere . \textbf{ Fuerunt enim aliqui de suo ingenio praesumentes , } dicentes Theologiam superfluere , & si la theologia es sciençian . \textbf{ Ca fueron muchos presunptuosos | que presumiendo de su engennio dixieron } que la theologia era superflua . \\\hline
3.2.30 & alius legislator , \textbf{ nisi intendat suos conciues } facere & o qual quier otro fazendor de ley \textbf{ si non entendiere fazer a todos los sus çib } dadanos los mas uirtuosos que pudiere ¶as commo ninguno non pueda venir a acatadas uirtudes sim̃o entendiere escusar todos los pecados \\\hline
3.2.31 & statuere leges , \textbf{ ut ciues possent uxores suas vendere . } Sic etiam contingit leges aliquas esse stultas , & establesçer tales leyes \textbf{ por las quales los çibdadanos pudiessen vender so mugers . } assi avn contesçe que algunas leyes son locas \\\hline
3.2.31 & cuiusmodi erat lex illa , \textbf{ quod ciues possent suas uxores vendere , } vel quaecunque aliae leges sic prauae et iniustae , & assi commo emaquella ley que dizie \textbf{ que los çibdadanos podien vender sus mugiers } o otras quales si quier leyes malas \\\hline
3.2.32 & et si non viueret in societate , \textbf{ ut alii suam magnificentiam perciperent , } et ut eis sua bona communicare posset , & Enperosi non visquiesse en conpannia \textbf{ por que los otros sintiessen su conpannia | e su magnifiçençia } e por que les pudiesse dar de los sus bienes non termie todos aquellos bienes en much . \\\hline
3.2.32 & ut alii suam magnificentiam perciperent , \textbf{ et ut eis sua bona communicare posset , } non multum reputaret illa . & e su magnifiçençia \textbf{ e por que les pudiesse dar de los sus bienes non termie todos aquellos bienes en much . } Et pues que assi es la çibdat fue fecha \\\hline
3.2.33 & Decet ergo Reges et Principes adhibere cautelas , \textbf{ ut in regno suo abundent multae personae mediae ; } ut ne aliis ad nimiam paupertatem deuenientibus efficiantur reliqui nimis diuites : & que los reyes e los prinçipes ayan cautelas e sabidurias . \textbf{ por que en el su regno sean muchͣs perssonas medianeras } por que los vnos non vengan atan grant pobreza \\\hline
3.2.35 & Secundo vero , quia ea dirigit \textbf{ in actiones suas . } In capite enim viget sensus et imaginatio , & que los otros mienbros . \textbf{ la segunda que la cabescagnia e enderesça alos otros mienbros a sus obras . } ca en la cabesça estan todos los sesos \\\hline
3.3.2 & utiles sunt ad opera bellica : \textbf{ quia ex arte sua habent brachia apta } et assueta ad percutiendum . & que los ferreros e los carpenteros son aprouechables a las obras de la batalla \textbf{ por que por la su arte han los braços acostunbrados e apareiados para ferir . } Avn en essa misma manera son aprouechables los carniceros \\\hline
3.3.6 & ut gradatim pergant ita , \textbf{ ut quilibet se in suo ordine teneat . } Nam si acies siue peditum & e generalmente todos los lidiadores se deuen vsar a andar ordenadamente e a passo en la batalla . \textbf{ por que cada vno tenga su orden | e vaya en su lugar . } Ca si el az si quier de peones \\\hline
3.3.6 & impedietur ad percutiendum . \textbf{ Nam cum bellator a suo consocio nimis comprimitur , } sua impediuntur brachia , & e muy espessa enbargan se los vnos a los otros para ferir . \textbf{ Ca quando el lidiador esta muy apretado de su conpañon } enbargansele los braços para ferir \\\hline
3.3.6 & Nam cum bellator a suo consocio nimis comprimitur , \textbf{ sua impediuntur brachia , } ne possit hostibus plagas infligere . & Ca quando el lidiador esta muy apretado de su conpañon \textbf{ enbargansele los braços para ferir } e no puede dar colpes en los enemigos . \\\hline
3.3.8 & adeo efficiuntur timidi , \textbf{ quod contra suos victores vix aut nunquam audent bella committere . } Quare si in bellis omnino est superabundandum cautelis , & En tal manera que fuyendo fazense temerosos en manera que non pueden ser vençedores de sus enemigos \textbf{ e apenas o nunca osan acometer batalla contra ellos . } Por la qual cosa si en las batallas auemos de auer muchas cautellas \\\hline
3.3.8 & superuenientibus hostibus fugit exercitus debellatus . \textbf{ Igitur postquam exercitus suam dietam compleuit , } alicubi vult pernoctare , & e es vençida la hueste . \textbf{ Pues que assi es despues | que la hueste ha conplido su iornada } e quiere folgar de noche en algun logar \\\hline
3.3.9 & Et tunc dux sobrius , \textbf{ et vigilans prout viderit suum exercitum } in his conditionibus abundare , & Estonçe el cabdiello de la hueste mesurado \textbf{ e en viso segunt que viere la su hueste } ha conplimiento en estas seys condiçiones \\\hline
3.3.10 & habere omnium armorum exercitium , \textbf{ ut possit suos commilitones de pugna erudire , } ut fortiter pugnent , arma tergant , & e que aya uso en todas las armas \textbf{ por que pueda ensseñar todos los sus caualleros a la batalla } por que lidien fuertemente e quel alinpien las armas \\\hline
3.3.11 & qualiter exercitus deberet pergere , \textbf{ tutius posset suum exercitum ducere . } Sic etiam marinarii faciunt , & assi que por vista de los oios catasse en qual manera la hueste pudiesse andar . \textbf{ Mas seguramente podria guiar su hueste } por que assi lo fazen los marineros . \\\hline
3.3.12 & et quibus cautelis abundare decet bellorum ducem \textbf{ ne suus exercitus laedatur } in via quantum ad campestrum bellum . & Et quales cautelas ha de auer el señor de la batalla \textbf{ por que la su hueste non sea dañada en el camino . } Et este quanto a la batalla del canpo \\\hline
3.3.12 & Seruare autem debitum ordinem in acie \textbf{ ut equites et pedites suam aciem seruent , } non sine magno exercitio fieri potest . & mas guardar orden conuenible en la az \textbf{ e que los caualleros e los peones guarden su az } non se puede fazer \\\hline
3.3.13 & hoc potissime attendendum : \textbf{ ut pugnantes absque nimia fatigatione sui possint } nimis aduersarios laedere . & que los lidiadores \textbf{ sin grand canssamiento de sus mienbros puedan ferir mucho a sus enemigos e a sus contrarios . } Ca si los lidiadores canssaren mucho de guisa \\\hline
3.3.14 & quomodo et qualiter bellantes \textbf{ suos hostes inuadere debeant . } Nam cum septem modis enumeratis hostes fortiores existant ; & para se defender de ligero puede paresçer commo \textbf{ e en qual manera los lidiadores deuen acometer sus enemigos . } Ca commo en las siete maneras contadas \\\hline
3.3.14 & ut tali hora faciat \textbf{ suos commilitones cibum capere , } et requiescere : & Lo quarto el señor de la hueste se deue tenprar \textbf{ assi que en tal ora faga tomar la vianda a los caualleros } e folgar e dar çeuada a los cauallos \\\hline
3.3.14 & Sexto ( secundum Vegetium ) debet dux belli \textbf{ inter suos hostes et inimicos , } vel per se , & segunt dize vegecio el señor de la batalla deue poner \textbf{ por sio | por otros discordias entre los enemigos } e boluer contiendas \\\hline
3.3.21 & abscissis crinibus \textbf{ eos suis maritis tradiderunt : } per quos machinis reparatis & las mugeres de roma cortaron se los cabellos \textbf{ e dieron los a sus maridos } de los quales cabellos fezieron sogas \\\hline
3.3.23 & Tertio est circa marinum bellum attendendum , \textbf{ ut semper pugnantes nauem suam faciant } circa profundum aquarum , & conuiene de tener mientes \textbf{ que los que lidian sobre mar | sienpre pongan su naue } a la fondeza de las agüso en la mar mas alta \\\hline
3.3.23 & et omnem modum \textbf{ per quem possint suos hostes vincere , } quod totum ordinare debent & que los Reyes e los prinçipes ayan batalla derecha \textbf{ et los sus enemigos turben la paz } e el bien comun a tuerto non es cosa sin guisa \\\hline
3.3.23 & in qua est suprema requies : \textbf{ quam Deus ipse suis promisit fidelibus , } qui est benedictus in saecula saeculorum . & para saber todas las maneras de lidiar e toda manera . \textbf{ por que puedan vençer sus enemigos . }  \\\hline

\end{tabular}
