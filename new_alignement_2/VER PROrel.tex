\begin{tabular}{|p{1cm}|p{6.5cm}|p{6.5cm}|}

\hline
1.1.3 & quod si hac careat , \textbf{ dato quod per ciuilem potentiam principetur , } magis tamen est dignus & non es digno de ser prinçipe \textbf{ njn de gouernar a ninguno et puesto que enssennore | e por poderio çiuił toda uja } es mas digno de ser subdito \\\hline
1.1.4 & est naturaliter animal sociale , ciuile , et politicum , \textbf{ sequitur quod regatur secundum prudentiam , } et viuat vita politica . & e ordenado \textbf{ asy commo prueua el philosofo en esse mjsmo libro siguese que el omne deue ser gouernado | segunt sabiduria e rrazon derecha } Et de beuir vida politica e ordenada \\\hline
1.1.6 & nisi delectationes sensibiles : \textbf{ et inde est quod communi nomine } ( ut communiter ponitur ) & si non las delectaçiones sensibles . \textbf{ Et por ende es que communalmente los omes las delecta connes } que son mas sensibles \\\hline
1.1.6 & si felicitas ponitur \textbf{ esse perfectum bonum , oportet quod sit bonum } secundum intellectum , et rationem : & Pues si la bien auentraança es bien acabado e conplido . \textbf{ Conuiene que sea bien segunt el en tedimiento } e segunt Razon \\\hline
1.1.6 & vel ( quod idem est ) \textbf{ oportet quod sit tale bonum , } quale recta ratio prosequendi iudicet . & e segunt Razon \textbf{ O conuiene que sea tal bien qual iudga la razon derecha } que nos auemos de segnir \\\hline
1.1.6 & Sequens enim delectationes sensibiles , \textbf{ dato quod sit Senex tempore , } quia est Puer moribus , & Ca el que sigue las delectaçonnes carnales \textbf{ puesto que sea uieio entp̃o e en hedat } por que es moço en costunbres \\\hline
1.1.7 & ex rebus naturalibus producuntur , \textbf{ cuiusmodi sunt quae accipiuntur ex agris , } vel ex arboribus , & que uienen n atraalmente de cosas natraales \textbf{ asi commo son aquellas } que se toman delas tierras o de los arboles \\\hline
1.1.8 & si plene manifestare vult ipsum signatum , \textbf{ oportet quod sit } quid notum et manifestum : & si conplida mente quiere demostrar le que significa \textbf{ conuiene que sea conosçida cosa e magnifiesta . } Mas las cosas que son de dentro del alma \\\hline
1.1.8 & filium sic praesumptuosum occidit , \textbf{ non obstante quod dictus filius } victoriam obtinuerat ab hoste . & e que non fuesen cobdiçon sos de honrra mato a su fijo presunptuoso \textbf{ e soƀuio commo quier } que aquel su fijo ouiese auido uictoria de los sus enemigos ¶ \\\hline
1.1.9 & inter gloriam , et famam : \textbf{ diceremus quod fama oritur ex gloria : } erit ergo hic ordo , & Enpero si quisieremos fazer depart ineto entre la fama e la eglesia \textbf{ diremos | que la fama nasçe de la eglesia } Pues que assi es esta es la orden destas cosas \\\hline
1.1.9 & Sicut enim ad hoc quod aliquis honoretur , \textbf{ sufficit quod exterius bonus appareat : } sic ad hoc quod aliquis sit & Ca asi commo para alguno sea honrrado abasta \textbf{ que aparescabueon } assi ꝑan \\\hline
1.1.12 & in actu prudentiae , \textbf{ sciendum quod decet Regem maxime } suam felicitatem & en las obras de pradençia . \textbf{ ¶ Et la segunda commo le conuiene } de poner er la su bien andança solamente en dios . \\\hline
1.1.12 & et perfecte solus Deus , \textbf{ oportet quod quicunque principatur , } siue regnat , & e de gouernar prinçipalmente e acabadamente . \textbf{ Conuiene que qual se quier } prinçipeo Rey \\\hline
1.1.12 & esse dilectiuus alterius , \textbf{ si agat quae ipse vult : } si Princeps est felix diligendo Deum , & otrosi faze \textbf{ aquello que el su | amigoquiere¶ } Si el prinçipe es bien auenturado \\\hline
1.1.13 & si quis periculo se exponat , \textbf{ dato quod non transgrediatur , } quia indiscrete agit , & por que se ponen a peligro han mayor meresçiminto . \textbf{ Ca puesto que non passen la ley | si non se pusiessen a peligro } por el bien comun menguarian \\\hline
1.2.2 & dicitur quaedam virtus moralis , \textbf{ dicere possumus quod } secundum has quatuor potentias animae & en quanto la pradençia es dicha vna uirtud moral \textbf{ assi podemos dezir } que segund estos quatro poderios del alma \\\hline
1.2.3 & et superficialiter pertransire , \textbf{ dicamus quod passiones , } vel surgunt ex bono , vel ex malo . & superfiçialmente digamos \textbf{ que las passiones del alma } o se leuantan de bien o de mal si de mal o de mal de futuro o de mal de presente . \\\hline
1.2.5 & si virtuosus esse debet , \textbf{ oportet quod fiat prudenter , iuste , fortiter , et temperate : } ideo hae quatuor uirtutes , & conuiene que se faga sabiamente \textbf{ e iusta mente . | fuerte mente . e tenprada mente . } Et por ende estas quatro uirtudes son dichas \\\hline
1.2.5 & Secundo determinabimus de ipsa Iustitia , \textbf{ ostendentes quod decet Reges , } et Principes esse iustos . & Lo segundo diremos dela iustiçia \textbf{ mostrando que conuiene alos Reyes } e alos prinçipes de ser iustos e derechureros . \\\hline
1.2.5 & Deinde determinabimus de Fortitudine , et Temperantia , \textbf{ declarantes quod contingit Reges } et Principes esse fortes et temperatos . & mostrando e declarando \textbf{ que conuiene alos Reyes e alos prinçipes de seer fuertes e tenprados . } Mas enpos esto todo \\\hline
1.2.7 & Viso quid est prudentia , \textbf{ et ostenso quod per prudentiam recte dirigimur } in bonum finem , & isto que cosa es la prudençia \textbf{ e mostrado que por la pradençia somos enderesçados e guiados derechamente } a buena fin a la qual nos inclinan las uirtudes morales . \\\hline
1.2.8 & ratione propriae personae \textbf{ quae alios est dirigens , oportet quod sit solers , et docilis : } ratione vero gentis quam dirigit , & por razon de la su propia persona \textbf{ que ha de guiar los otros . | Conuiene le de ser sotil e doctrinable ¶ } Mas por razon dela gente \\\hline
1.2.8 & ratione vero gentis quam dirigit , \textbf{ congruit quod sit expertus et cautus . } Si enim Rex debet & a quien ha de gouernar \textbf{ Conuiene le de ser prouado | e cauto ete sogedor de bien ¶ } Ca si el Rey ha a guiar la su gente \\\hline
1.2.8 & gentem aliquam ad bonum dirigere , \textbf{ oportet quod habeat memoriam praeteritorum , } et prouidentiam futurorum . & e la su conpanna a alguons bienes . \textbf{ Conuiene que aya memoria de las cosas passadas . } Et que aya prouision delas cosas passadas \\\hline
1.2.8 & per quem dirigit , \textbf{ oportet quod habeat intellectum et rationem , } siue oportet & por la qual deue guiar el Rey . \textbf{ Conuiene le que aya entendimiento } e razon o conuiene le que sea entendido e razonable . \\\hline
1.2.8 & quo Rex suum populum dirigit , \textbf{ oportet quod sit humanus , } quia Rex ipse homo est . & Ca la manera por que el Rey guia el su pueblo \textbf{ Conuiene que sea manera de omne . } Ca el Rey omne es \\\hline
1.2.8 & volens alios dirigere , \textbf{ oportet quod sit intelligens , } cognoscendo principia , & El que quiere alos otros guiar \textbf{ conuiene le que sea entendido } conosciendo los prinçipios e las razones . \\\hline
1.2.8 & ex illis praemissis cunclusiones intentas . \textbf{ Vel oportet quod sit intelligens , } sciendo leges , & e las razones que quiere ençerrar ¶ \textbf{ Et otrosi conuiene al Rey | que sea entendido } e sabio sabiendo las leys \\\hline
1.2.8 & quae est alios dirigens , \textbf{ oportet quod sit solers , et docilis . } Nam qui in tanto culmine est positus , & que es tal que ha de gouernar los otros . \textbf{ Conuiene le de sor sotil e doctrinable . } Ca aquel que esta en tanta alteza de dignidat \\\hline
1.2.8 & ut tantam gentem regere habeat , \textbf{ oportet quod sit industris , et solers , } ut sciat ex se inuenire bona gentis sibi commissae . & que es puesto para gouernar tanta gente e tanto pueblo . \textbf{ Conuiene le que sea engennoso e sotil | por que sepa } por si buscar e fallar aquellos bienes \\\hline
1.2.10 & quod est aequum , \textbf{ idest quod sibi debetur . } Differentia autem harum Iustitiarum & lo que conuiene \textbf{ e aquello que es suyo . } Mas en otra manera se puede tomar la diferençia destas dos iustiçias . \\\hline
1.2.10 & vel ordine ad Ciuitatem : \textbf{ non obstante quod Temperatia , } et Fortitudo , & assi commo en orden al prinçipe o en orden ala çibdat . \textbf{ como quier que la tenperança e la fortaleza e las otras uirtudes prinçipales acaben } aque que las ha segunt si¶ \\\hline
1.2.10 & Sic etiam dicitur \textbf{ unicuique tribuere quod suum est : } quia aequum est , & e lo que es igual \textbf{ Et assi es dicha dar a cada vno | lo que es suyo . } Ca cosa igual es \\\hline
1.2.11 & est in eis quaedam commutatiua Iustitia , \textbf{ sine qua corpus naturale durare non posset : } sic prout ciues eiusdem ciuitatis , & o es en ellos vna iustiçia mudadora e acorredora \textbf{ por sabiduria natural | sin la qual el cuerpo nal non podria durar . } Dien assi en quanto los çibdadanos de vna çibdat \\\hline
1.2.11 & est in eis commutatiua Iustitia , \textbf{ sine qua ciuitas , } vel regnum non posset subsistere . & es en ellos la iustiçia mudadora \textbf{ sin la qual la çibdat o el regno non podria estar ¶ } Lo segundo en los mienbros del cuerpo hay vna iustiçia partidora en quanto han ordenamiento \\\hline
1.2.12 & quadruplici via venari , \textbf{ secundum quatuor quae tanguntur de Iustitia in 5 Ethicorum . } Prima via sumitur & en quatro maneras \textbf{ segunt quatro cosas | que tanne el philosofo dela iustiçia en el quimo libro delas ethicas ¶ } La primera manera se toma de parte dela persona del Rey ¶ \\\hline
1.2.16 & est quasi extra se , \textbf{ nec voluntarie et deliberate agit quod agit . } Tolerabilius est igitur peccare per timorem , & Mas quando alguno esta acometido \textbf{ e esta fuera de ssi non faz aquello que faze por uoluntad ñcon delibramiento . } Et por ende mas de foyr \\\hline
1.2.16 & animo est exprobrabile , \textbf{ patet quod est exprobrabilius } ipsum esse intemperatum , & e non fuer firme en el coraçon es de deno star por ello . \textbf{ Et es mas de denostar si fuer deste prado } e segnidor de passiones . \\\hline
1.2.16 & Rex ille volens complacere illi Duci , \textbf{ praecepit quod duceretur ad ipsum . } Dux autem ille assuetus rebus bellicis , & Et el Rey que tiendo fazer plazer a aquel \textbf{ prinçipe mando qual pusiessen dentro ante si . } Mas aquel prinçipe por que era acostunbrado delas batallas \\\hline
1.2.18 & vel plus , quam debeat . \textbf{ Secundo debet respicere quibus det , } ut non det quibus non oportet . & delo que deue dar ¶ \textbf{ Lo segundo deue catar aqui lo da } por que non de \\\hline
1.2.18 & Secundo debet respicere quibus det , \textbf{ ut non det quibus non oportet . } Tertio videndum est cuius gratia det , & Lo segundo deue catar aqui lo da \textbf{ por que non de | aquien non deue dar ¶ } Lo terçero deue veer \\\hline
1.2.19 & ut sunt decentes operibus , \textbf{ non respicit quaecunque opera : } quia non est difficile facere decentes sumptus & Mas la magnificençia que cata alas espenssas en quanto son conuenbles alas obras non cata \textbf{ nin tiene oio quales sean las obras . } Ca non es guaue cosa de fazer conuenibles espenssas \\\hline
1.2.20 & et subterfugit quantum potest : \textbf{ sic dato quod paruificum oporteat } expensas facere , & por que se non taiasse . \textbf{ Bien alłi puesto que el paruifico } e al escasso sea dado de fazer grandes espenssas sienpre tarda \\\hline
1.2.25 & ne trahamur ratione difficultatis , \textbf{ oportet quod ei sit annexa humilitas , } ne ultra quam ratio dictet & por razon dela graueza . \textbf{ Et por ende conuiene | que aella sea ayuntada la humildat } por que non pueda passar allende \\\hline
1.2.26 & circa moderationem deiectionis : \textbf{ restat videre quod decet Reges } et Principes esse humiles , & e despues desto cerca la tenprança del despreçiamiento e del abaxamiento . \textbf{ finca de ver | que conuiene alos Reyes } e alos prinçipes ser humildosos \\\hline
1.2.29 & a veritate recedit ratione abundantiae : \textbf{ qui vero quae sunt negat , } et minora confitetur , & en razon de sobrepuiamiento . \textbf{ Mas aquel que mengua aque llascolas | que lon en el } e confiessa \\\hline
1.2.29 & Sciendum ergo quod licet \textbf{ affirmare in se esse quod non est , } vel negare quod est , & Et pues que assi es conuiene saber \textbf{ que maguera firmar cada vno de ser en ssi aquello que non es o negar } aquello que es en ssi sea mentira \\\hline
1.2.29 & affirmare in se esse quod non est , \textbf{ vel negare quod est , } sit mentiri : & Et pues que assi es conuiene saber \textbf{ que maguera firmar cada vno de ser en ssi aquello que non es o negar } aquello que es en ssi sea mentira \\\hline
1.2.29 & cognoscere seipsum , \textbf{ et sciri quod propria bona } semper aestimantur maiora quam sint . & Ca muy grand pradençia \textbf{ e grant sabiduria es conosçer assi mismo . omne e saber que los sus bienes propreos } sienpreles son vistos mayores que son ¶ \\\hline
1.2.32 & et ipsum parabat in conuiuium , \textbf{ spondens quod quando vellet conuiuium facere , } ei suum filium tribueret . & e aprestaual para fazer el conbit \textbf{ et prometial que quando quisiesse fazer conbit } que el qual daria su fijo \\\hline
1.2.32 & et principari desiderant , \textbf{ oportet quod habeant virtutem illam , } quae est dominans & e enssenorear alos otros . \textbf{ Conuienele que aya aquella uirtud } que es sennora e prinçipante a todas las otras uirtudes . \\\hline
1.3.1 & sicut dicebamus esse duodecim virtutes , \textbf{ sic dicere possumus quod sunt duodecim passiones : } videlicet , amor , odium , desiderium , abominatio , delectatio , tristitia , spes , desperatio , timor , audacia , ira , et mansuetudo . & que eran doze uirtudes \textbf{ assi podemos dezinr que las passiones son doze | conuiene saber amor e mal querençia e desseo . } e aborrençia er delectacion . \\\hline
1.3.2 & Quia nullus bene seipsum regere potest , \textbf{ nisi sciat quae passiones sunt fugiendae , } et quae prosequendae : & orque niguno non puede bien gor̉inar \textbf{ assi mismo | si non sopiere quals passiones son de fuyr } e quales son de leguir \\\hline
1.3.3 & in appetitu sensitiuo et intellectiuo : \textbf{ dicere possumus quod semper obiectum amoris est bonum . } Ubi ergo reperitur & e en el apetito intellectiuo \textbf{ que es la uoluntad podemos dezir | que la razon del amor es sienpre algun bien . } Et pues que assi es en aquel loguat do \\\hline
1.3.3 & non dubitabit etiam personam exponere , \textbf{ si viderit quod expediat regno . } Erit temperatus ; & e avn non dubdara de poner la persona a muerte siuiere \textbf{ que sea cosa | que conuenga al regno } e avn sera tenprado \\\hline
1.3.5 & Cum determinauimus de ordine passionum animae , \textbf{ diximus quod amor et odium erant passiones primae , } desiderium vero et abominatio erant passiones secundae : & uando determinamos dela ança orden delas passiones del alma dixiemos \textbf{ que el amor e la malqreçia eran las primeras passiones } Et el desseo e la aborrençia eran las segundas passiones . \\\hline
1.3.7 & non posset tantum habere de malo , \textbf{ quin vellemus quod haberet plus . } Sed ira quae est appetitus poenae , & que nos non quisiesemos \textbf{ que ouiesse mas | Mas la saña } que es appetito de prinar non sinplemente \\\hline
1.3.7 & et mansuetudinem : \textbf{ sciendum quod ira aliquando rationem praecedit , } et tunc est inordinata et cauenda , & e los prinçipesse de una auer çerca la sanna \textbf{ e cerca la mansedunbre conuiene de saber que la sanna } algunans vezes va ante la razon \\\hline
1.3.8 & ei omnem delectationem fugere ; \textbf{ sequitur quod fugiens omnem delectationem , } sequatur aliquam delectationem . & si non fuere a el delectable de foyr toda delectaçion siguese \textbf{ que aquel que fuye toda delectaçion } sigue algunan delectaçion . \\\hline
1.4.2 & cum maxima diligentia cogitare debent , \textbf{ qui sunt qui loquuntur , } utrum sint sapientes vel ignorantes , & commo les fabla cada vno \textbf{ o que les son aquellos que les fablan } si sen sabios o non sabios \\\hline
1.4.3 & quicunque est naturaliter frigidus , \textbf{ sequitur quod sit naturaliter formidolosus . } Sequitur ergo senes esse naturaliter timidos , & Et qual si quier que naturalmente es frio naturalmente es temeroso \textbf{ Et por ende siguese } que los uieios son naturalmente temerosos . Ca fallesçe enellos la calentura natural \\\hline
1.4.3 & ad huiusmodi se debeant habere . \textbf{ Nam constat quod licet Reges et Principes } non debeant esse & Ca cierta cosa es \textbf{ que commo quier que los Reyes non de una ser } en todas las cosas creedores de ligero \\\hline
1.4.4 & Primo , si ultra quam ratio dictet , \textbf{ teneat quod habet . } Secundo , si praeter rationem concupiscat habere & La primera se retiene lo que han \textbf{ mas } de quanto demanda la razon¶ \\\hline
1.4.4 & Senes magis peccant \textbf{ per illiberalitatem in retinendo quae habent , } quam in concupiscendo indebite & auer lo que non han . \textbf{ Por ende los uieios mas pecan por la escasseza en reteniendo lo que han que en } desseando \\\hline
1.4.4 & quia senes vixerunt multis annis , \textbf{ et viderunt quod saepe sunt decepti : } non audent pertinaciter aliquid asserere , & por que los uieios visquieron muchos años \textbf{ e vieron que fueron muchͣs uezes engannados non osan afirmar ningunan cosa } afincandamente temiendo \\\hline
2.1.1 & naturalia sunt ea , \textbf{ sine quibus non potest } bene conseruari in esse . & que naturalmente es fecha todas aquellas cosas le son naturales \textbf{ sin las quales non se puede bien guardar en su ser . } Ca la nata en vano faria las cosas \\\hline
2.1.1 & quae faciunt ad bene viuere , \textbf{ et sine quibus non potest } sibi in vita sufficere , & que fazen a bien beuir \textbf{ e sin las quales non puede el omne abondar } assi mesmo en la uida \\\hline
2.1.1 & prouidere videtur in victu : \textbf{ sic videtur quod eis sufficienter prouideat in vestitu . } Bestiae enim , & a inalias en la uida assi paresçe \textbf{ que les prouee conplidamente en la uestidura . } Ca las bestias e las aues veemos \\\hline
2.1.1 & sine societate alterius , \textbf{ sequitur quod homo naturalem impetum habeat } ut sit animal sociale ; & e ninguno non abaste assi mismo sin conpannia de otro \textbf{ para estas cosas siguese | que el o omne ha natural inclinamiento } para ser conp̃anero e ainal aconpannable . \\\hline
2.1.3 & cum de domo loquimur , \textbf{ sciendum quod domus nominari potest } aedificium constitutum & or que non trabaiemos en vano fablando dela casa \textbf{ conuiene de saber que la casa algunas uezes } puede ser dicha costruymiento fech̃o de paredes e de techo e de \\\hline
2.1.3 & ut sciant domum propriam gubernare , \textbf{ et ut cognoscant quae et qualis est communitas domus } ut se habet ad regnum et ciuitatem , & e que conoscan que cosa \textbf{ e qual es la comunidat dela casa | ca es comunidat en alguna manera natural } Et en algunan manera esta comunidat se ha al regno \\\hline
2.1.4 & ubi distinguentur omnes partes domus , \textbf{ et probabitur quod quaelibet } talis pars est aliquid naturale . & e departiremos todas las ꝑtes dela casa \textbf{ e prouaremos que cada vna ꝑtetal dela casa es cosa natural . } Pues que assi es finça de declarar en la difiniçion sobredichͣ \\\hline
2.1.5 & Ex illis enim dicitur domus constare , \textbf{ sine quibus congrue esse non potest . } Quod vero sine viro et uxore , & ca de aquallas comuindades deue ser establesçida la casa \textbf{ sin las quales non puede ser conueniblemente la casa } mas que sin varon e sin mugnỉ e sin sennor e sin sieruo . \\\hline
2.1.6 & Cum enim primo homo est , \textbf{ oportet quod sit genitus : } et natura statim est solicita de salute eius ; & Ca quando el ome es primero \textbf{ conuiene que sea engendrado . } Et la natura luego es acuçiosa de su salud . \\\hline
2.1.6 & quare si est impotens ad agendum , \textbf{ sequitur quod ei deficiat aliqua forma } vel aliqua perfectio , & si non ha poderio de obrar \textbf{ siguese qual mengua alguna forma o alguna perfection e conplimiento } que es comienço de obra \\\hline
2.1.7 & quid naturale , \textbf{ sequitur quod fornicatio , } quae contrariatur coniugio , & mas si el ma termonio es cosa natural siguese \textbf{ que la fornicaçion } que es contraria al mater moino \\\hline
2.1.7 & de societate politica , \textbf{ videlicet quod eligens solitudinem , } et nolens ciuiliter viuere , & conmodiziemos dela uida politica e de çibdat . \textbf{ Conuiene a saber que el que escoge beuir solo } e non quiere beuir \\\hline
2.1.8 & et ad hoc quod inter uxorem et virum sit amicitia naturalis , \textbf{ oportet quod sibi inuicem seruent fidem , } ita quod ab inuicem non discedant . & que sea segunt natura \textbf{ e para que entre el uaron e la muger sea amistança natural conuiene que guarden vno a otro fe e lealtad } assi que non se puedan partir vno de otro . \\\hline
2.1.8 & semper enim de ratione communis , \textbf{ est quod contineat , uniat , et coniungat , } sicut de ratione proprii , & por que son bien comunal dellos . \textbf{ Ca sienpre es de la razon del bien comun que tenga e ayunte amistança } assi commo dela razon del bien propreo es que ayunte e desayunte el vno del otro . \\\hline
2.1.8 & sicut de ratione proprii , \textbf{ est quod diuidat et distinguat . } Hanc autem rationem tangit Philosophus 8 Ethic’ dicens , & Ca sienpre es de la razon del bien comun que tenga e ayunte amistança \textbf{ assi commo dela razon del bien propreo es que ayunte e desayunte el vno del otro . } Et esta razon pone el philosofo en el viii̊ libro delas ethicas \\\hline
2.1.9 & ut vult Philosophus 9 Ethicor’ , \textbf{ indecens est quoscunque ciues plures habere uxores : } quia eas non tanta amicitia diligerent , & assi conmo dize el philosofo en elix̊ . \textbf{ delas ethicas cosa desconuenible es | a quales si quier çibdadanos } e a quales se quier uatones de auer muchͣs mugieres \\\hline
2.1.9 & et unus uni adhaeret ; \textbf{ sequitur quod in hominibus } per totam vitam coniuges simul conuiuant , & e la fenbra biuen en vno \textbf{ e vno se ayunta a vna siguese que en los omes el marido e la muger } por toda su uida biuan en vno \\\hline
2.1.11 & Secunda , ex bono quod ex coniugio consurgit . \textbf{ Tertia , ex malo quod inde vitatur . } Prima via sic patet . & La segunda del bien que se leunata del mater moion \textbf{ ¶La terçera del mal | que se deude escusa¶ } La primera razon se declara assi . \\\hline
2.1.11 & Tertia via ad inuestigandum hoc idem , \textbf{ sumitur ex malo quod per coniugium vitatur . } Per coniugium enim & ¶La terçera razon para prouar esto \textbf{ mesmo se toma del mal | que se puede escusar } por el casamiento . \\\hline
2.1.16 & quia regimen coniugale est aliud a paternali et seruili : \textbf{ et ostendere quod aliter debet se habere vir } tam erga uxorem , & matermoian les otro que el paternal \textbf{ e que el suil e mostrar } que en otra manera se deua auer el uaron cerca la mugni \\\hline
2.1.16 & si illud sit imperfecte calidum , \textbf{ sequitur quod imperfecte calefaciat . } Sic etiam quia ad hoc quod aliquid calefaciat , & Ca assi commo para escalentar es menester calentura \textbf{ si aquella calentura non es calentura acabada siguese que non es caliente acabada mente . } En essa misma nanera avn pero \\\hline
2.1.16 & Sic etiam quia ad hoc quod aliquid calefaciat , \textbf{ requiritur quod sit dispositum } ad susceptionem caloris ; & que algunan cosa sea escalençada es mester \textbf{ que sea apareiada } para resçebir aquella calentura . \\\hline
2.1.16 & ad huiusmodi susceptionem , \textbf{ sequitur quod imperfecte calorem suscipiat . } Quare si coniunctio uxoris et viri requiritur & para resçebir esta calentura siguese \textbf{ que non resçibrie esta calentura acabada mente . } Por la qual cosa si el ayuntamiento del uaron \\\hline
2.1.16 & si ex tali coniunctione nascantur pueri , \textbf{ sequitur quod producantur imperfecti } et debiles corpore ; & e non vinieres a conplimiento conuenible \textbf{ si de tal ayuntamiento nasçieren moços siguese que non nasçeran acabados } e seran flacos de cuerpo \\\hline
2.1.16 & Unde in Politicis , \textbf{ dicitur quod masculorum corpora laeduntur , } si tempore augmenti & Onde en las politicas dize el philosofo \textbf{ que los cuerpos de los mas los resçiben | danno } si en el t pon del cresçer \\\hline
2.1.18 & Viso , quae sunt laudabilia in foeminis : \textbf{ restat narrare quae sunt vituperabilia in eis . } Possumus autem narrare tria & Visto quales cosas son de alabar en las muger \textbf{ s finca de dezir que cosas son de denostar en ellas . } Et podemos dez que tres cosas son de denostar en ellas ¶ \\\hline
2.1.21 & debeant se habere , \textbf{ aduertendum quod } circa ornamentum vestimentorum & mas espeçialmente en qual manera se deuen auer \textbf{ conueiblemente en sus uestiduras e en los otros conponimientos del cuerpo . } C suiene de saber que çerca los conponimientos delas uestiduras podemos pecar en dos maneras ¶ \\\hline
2.1.21 & quod infirmior magis gloriatur , \textbf{ quia credit quod in cum plures aspiciant , } et sperat se plures eleemosynas accepturum : & contesçe que el mas enfermose eglesia \textbf{ mas por que cree que muchos catan ael } e es para que resçibra mas helemosinas que los otros . \\\hline
2.1.23 & in quo est suprema prudentia ; \textbf{ oportet quod agat ordinate et prudenter . } Prudentis est enim cito se expedire , & que faze la natura \textbf{ que las faga ordenadamente | e con sabiduria } Ca al sabio pertenesçe \\\hline
2.2.2 & si debeant naturaliter dominari , \textbf{ oportet quod polleant prudentia et intellectu : } tanto decet Reges et Principes & e generalmente todos los señores sy de una naturalmente ensseñorear \textbf{ conuiene les que ayan sabidia e entendimiento . } Et tanto mas conuiene alos Reyes \\\hline
2.2.3 & cum amare aliquod , \textbf{ idem sit quod velle ei bonum , } pater debet & e por el bien dollos commo amar a alguno sea esso mismo \textbf{ que querer qual bien . } Et el padre deue \\\hline
2.2.3 & Viso , quod paternum regimen ex amore nascitur , \textbf{ patet quod filiis debet } praeesse pater propter bonum filiorum . & Visto que el gouernamiento del padre nasçe de amor paresçe \textbf{ que el padre deue } enssennorear alos fiios \\\hline
2.2.4 & Tamen ut magis specialiter appareat intentum , \textbf{ sciendum quod licet parentes } magis afficiantur circa filios , & Empero por que mas spanlmente paresca la su entençion \textbf{ deuedes saber | que commo quier que los padres } mas sear inclinados alas fijos \\\hline
2.2.4 & cum diligere aliquem , \textbf{ idem sit quod velle ei bonum , } distinguendum est de ipso bono . & commo amara alguno sea essa misma cosa \textbf{ que querer bien } para el deuemos departir deste bien . \\\hline
2.2.5 & pueriscire non possunt : \textbf{ sufficit quod talia proponantur grosse , } et in quadam summa : & non pueden auer los legos \textbf{ e menos los moços . | Abasta que tałs cosas sotiles } que parte nesçen ala fe \\\hline
2.2.8 & quare si debent eis aliqua delectabilia concedi , \textbf{ dignum est quod ordinentur } ad delectationes innocuas : & que los moços non pueden sofrir ninguna cosa de tristeza . \textbf{ Por la qual cosa si les son otorgadas algunas cosas delectabłs conuiene } que les otorguen cosas \\\hline
2.2.8 & ut ex hoc subtiliores fiant \textbf{ ad intelligendum quaecunque proposita : } quo facto totum suum ingenium debent exponere , & por que por esto sean mas sotiles \textbf{ para entender quales si quier cosas | que les sean propuestas } la qual casa fechͣ deuen poner todo su en gennio \\\hline
2.2.9 & tam de inuentis quam de intellectis . \textbf{ Nam et dato quod filii nobilium , } et maxime Regum , & e buen iudgador tan bien delas cosas entendidas \textbf{ commo delas falladas . | Ca puesto que los fijas delos nobles } e mayormente de los Reyes \\\hline
2.2.9 & sed etiam operibus et exemplis ; \textbf{ requiritur quod huiusmodi doctor sit } in vita bonus et honestus . & e por ende conuiene \textbf{ que este doctor e maestro sea enssi bueno e honesto en su uida . } Ca por que la he dar de los moços \\\hline
2.2.10 & Secundo adhibenda est cautela in iuuenibus , \textbf{ ut instruantur quod palpebras oculorum } cum maturitate eleuent , & ¶ Lo segundo deuemos dar cautella alos moços \textbf{ que sean enssennados | quanto ala manera de ver } assi que alçen las palpebras de los oios con grand madureza \\\hline
2.2.10 & prohibendi sunt iuuenes , \textbf{ ne audiant quodcunque turpium : } quia audire , est prope ad ipsum facere . & deue dar alos mançebos \textbf{ que non oyan cosas torpes } por que el oyr es muy çerca del obrar ¶ \\\hline
2.2.11 & quod cibus diu in ore existat , \textbf{ sed cupiunt quod cito perueniat ad guttur . } Ideo tales & sauianda non se delectan mucho \textbf{ por que este la uianda luengo tienpo en la boca mas cobdiçian } que luego vaya ala garganta . \\\hline
2.2.13 & 8 Poli’ \textbf{ est necessarius in vita quod } ( quantum ad praesens spectat ) & en el viij̊ libro delas politicas \textbf{ es neçessario enla vida } humanal la qual cosa podemos declarar \\\hline
2.2.15 & et facit ad bonam dispositionem corporis , \textbf{ sequitur quod sit } quoddam proficuum ad augmentum . & e faze abuean disposiçion del cuerpo \textbf{ Et por ende se sigue } que sea aprouechoso alacresçentamiento del cuerpo . \\\hline
2.3.1 & propter quod declarata est prima pars capituli , \textbf{ ubi dicebatur quod specta : } ad gubernationem domus considerare & Por la qual cosa es declarada la primera parte del capitulo \textbf{ do es dicho queꝑ tenesçe al gouernador dela casa } de auer cuydado de los siruientos \\\hline
2.3.3 & Nam ubi multae sunt diuitiae , \textbf{ multi sunt qui comedunt illas . } In domibus ergo Regum et Principum & riquesasy son muchs \textbf{ que las coman e las despiendan } e pues que assi es en las casas de los Reyes \\\hline
2.3.4 & Tertium , quod considerandum est in aquis , \textbf{ est quod sit coloris perspicui . } Nam ipsa infectio coloris , & ¶Lo terçero que es de penssar en las aguas es \textbf{ que sean de color claro de gnisa } que passe el oio de vna parte a otra \\\hline
2.3.6 & et esse pauperes , \textbf{ non obstante quod ciues possunt } gaudere possessionibus propriis , & que ninguon non quarrie trabaiar \textbf{ por ellas ca agora en la çibdat | lon muchs pobres avn que non contradigamos } que los çibdada nos puedan auer possessiones proprias \\\hline
2.3.6 & quilibet ministrorum retrahitur , \textbf{ ne faciat quod mandatur , } sperans alium implere & e se tira \textbf{ que non faga | aquello qual es mandado } elperando que el otro cunplira aquello que a el es mandado . \\\hline
2.3.8 & ut melius homines possint \textbf{ explere quod volunt , } diuitiis et possessionibus non satiantur . & assi que por ellas cada vno cuyda \textbf{ que podra alcancar aquello que dessea . } por ende los omes non se fartan de riquezas nin de possessions \\\hline
2.3.8 & et aliter ea quae sunt ad finem . Nam finis appetitur in infinitum : \textbf{ ea vero quae sunt ad finem , } secundum modum et mensuram ipsius finis . & e sin fin mas aquellas cosas \textbf{ que son ordenadas ala fin son desseadas segunt manera } e segunt mesura de aquella fin \\\hline
2.3.9 & oportuit introduci , \textbf{ sciendum quod si non esset } nisi communitas domus & que estas tales muda connes fuessen puestas en la \textbf{ tiecra deuedes saber } que si non fuesse sinon la comiundat dela casa \\\hline
2.3.10 & quod facit pecuniatiua usuraria , ut plane patet , \textbf{ vult quod denarii illi pariant et generent : } recte ergo usura vocata est & assi con \textbf{ moclaramente paresçe que quiere | que aquellos e paran e engendren dineros . } Et por end ex derecha la usura es llamada assi con e mo parto de diueros . \\\hline
2.3.11 & ut quod decem post lapsum temporis fiant viginti , \textbf{ vult quod artificialia seipsa multiplicent : } et quia hoc est contra naturam artificialium , & despues que passare algun tienpo \textbf{ que se faganveite quiere | que las cosas artifiçiales crezcan } e se amuchiguen en simiłmos \\\hline
2.3.11 & potest inde accipi pensio , \textbf{ dato quod res illa in nullo deterioraretur . } Sed si non potest & ally en aquella cosa se puede tomar loguer o alquiler della puesto \textbf{ que aquella cosa se enpeor } e por aquel uso \\\hline
2.3.11 & Quare si de usu pensionem accipiat , \textbf{ uendit quod non est suum , } uel accipit pensionem & dende adelante non parte nesçe a ellos el uso della . \textbf{ Por la qual cosa el que resçibe ganançia del uso del dinero vende lo que non e suyo o tomagat saçia de aquello que non parte nesçe ael } por que dende adelante non pertenesçe ael el uso del diuero \\\hline
2.3.11 & quod non est proprius usus eius , \textbf{ dato quod non statim pecuniam acciperet , } si propter usum domus vellet & canbiasse la qual cosa non es uso ppreo dela casa \textbf{ puesto que non resçebiesse luego los dineros mas por el uso dela casa quisiesse tomar dineros acometrie usura } por que ya el uso dela casa non parte nesçrie a el \\\hline
2.3.12 & esse expertum circa possessiones , \textbf{ sciendo quae sunt magis fructiferae , } et ex quibus potest melius subueniri indigentiae corporali domesticae siue gubernationi domus . & derca las possessiones \textbf{ sabien | do quales son de mayor fructo } e de quales puede meior acorrer ala mengua dela cala . \\\hline
2.3.12 & Hoc autem fieri contingit , \textbf{ si sciatur quae in quibus partibus abundant , } ut quis illis animalibus abundet , & Et esto le puede fazer \textbf{ si lopieren quales aianlias | e en quales partes dela terrra abondan } e quien son aquellos que han aquellas aianlias \\\hline
2.3.12 & lucratus est pecuniam multam , \textbf{ et ostendit quod facile erat Philosophis ditari . } Secundum particulare gestum , & pusol preçio qual queso e gano muy grand auer . \textbf{ Et en esto mostro que ligera cosaes alos philosofos de se } enrriquesçer quando quisieren . \\\hline
2.3.16 & nam saepe quilibet ministrantium huiusmodi ministerium negligit , \textbf{ credens quod alius exequatur illud : } ubicunque enim est multitudo , & Ca muchͣs uezes cada vno de aquellos seruientes \textbf{ menospreçia aquel seruiçio cuydando que el otro lo fara . } Por que do quier que ay muchedunbre alli es confusion \\\hline
2.3.17 & videndum est qualiter sunt exhibenda indumenta ministris . \textbf{ Ad cuius euidentiam sciendum quod circa hoc } ( quantum ad praesens spectat ) & e departir las vestiduras alos seruientes . \textbf{ e para conosçimiento desto | conuieneles de saber } quanto pertenesçe alo presente \\\hline
2.3.18 & secundum opinionem , \textbf{ alia vero quae fundantur in magnis bonis } secundum existentiam et veritatem . & segunt opinion de los omes . \textbf{ Et otra que se funda en grandes bienes segunt uerdat } e esta es nobleza uerdadera \\\hline
2.3.19 & ex vili genere sunt assumpti , \textbf{ dato quod in aliquibus paruis magistratibus videantur } prudenter et fideliter se gessisse , & son tomados de villoguar \textbf{ puesto que en algunos pequanos maestradgos | e ofiçios parezçan sabios } e que se han sabia mente \\\hline
2.3.20 & ab ipso deo et intelligentiis ordinata : \textbf{ dato quod natura faciat } idem organum ad duo opera , & e de los angeles . \textbf{ puesto que la natura faga vn estrumento para dos obras } enpero por que non sea confusion en las obras \\\hline
3.1.3 & eligunt perfectiorem vitam . \textbf{ Nam licet qui nubit , } et qui ciuiliter viuit , & mas refusan el casamiento \textbf{ e la çibdat e escogenuida mas acabada } assi commo son los religiosos \\\hline
3.1.3 & Est ergo homo naturaliter animal ciuile , \textbf{ non obstante quod contingat } aliquos non ciuiliter viuere : & e por ende es el omne natalmente aian lçiuil \textbf{ puesto que contezca que alguons non bi una çiuilmente } ca qualquier que non biue cuulmente estol contesçe \\\hline
3.1.6 & in secundo libro fecimus mentionem , \textbf{ ubi diximus quod propter excrescentiam filiorum collectaneorum } et nepotum domus potest in vicum , & ¶La primera manera es aquella dela qual en el segundo libro feziemos mençion desuso do dixiemos \textbf{ que por las cresçençias de los fijos e de los nietos e de los bisnietos e de los parientes } La casa podria cresçet enuarrio \\\hline
3.1.7 & eos esse suos patres . \textbf{ Tertium vero quod senserunt } dicti Philosophi & que ellos eran sus padres . \textbf{ ¶ Lo terçero que sintieron los dichs philosofos cerca el gouernamiento dela çibdat . } es que dixieron \\\hline
3.1.8 & quasi sex rationes , \textbf{ probantes quod non oportet ciuitatem } esse maxime unitam , & que prue una \textbf{ que non conuiene ala çibdat } de ser muy vna \\\hline
3.1.10 & non habebitur eorum cura debita : \textbf{ sequitur quod supposita communitate , } quam ordinauerat Socrates , & non se puede auer \textbf{ er cuydado conuenible de los moços | e dende se sigue } que puesta tal comunidat commo ordeno soctateᷤ \\\hline
3.1.10 & et non iudicabantur eis proprii parentes , \textbf{ dato quod prohiberetur filio actus venereus circa matrem , } et patri circa filiam , & e non les fuessen mostrados sus padres proprios \textbf{ puesto que los prinçipes defendiessen alos fijos | que non fiziessen lururia con sus madres } e los padres con sus fiias \\\hline
3.1.15 & et de unitate ciuium , \textbf{ verum est quod ipse opinabatur , } quod in ciuitate esset maxima pax , & e dela vnidat de los çibdadanos \textbf{ uerdat es lo que el cuydaua e ymaginaua } que en la çibdat seria grant paz \\\hline
3.1.16 & frustra propter hoc insurgerent lites et placita . \textbf{ Nam dato quod alter litigantium causam obtineret , } non multum ex hoc gaudere posset & por esto debalde se leunatarian entre ellos las contiendas e las uaraias \textbf{ ca puesto que vno de los contendores uençiesse el pleito } non podia much gozar \\\hline
3.1.17 & esse liberales et temperatos : \textbf{ non ergo bene dictum est quod ad bonum regimen ciuitatis sufficit ciues habere possessiones aequatas , } nisi aliquid determinetur & que los çibdadanos sean liberales e francos \textbf{ e por ende non es bien dicho | que a buen gouernamiento dela çibdat } cunple de ser las possessiones egualadas \\\hline
3.1.19 & et distinctione ciuium , \textbf{ dicens quod optima quantitas ciuium est circa decem millia virorum . } Hanc autem quantitatem distinxit in tres partes , & e del deꝑtimiento de los çibdadanos \textbf{ e dizia que sia muy buena quantidat de çibdadanos | si fuessen fasta diez minl uarones } e esta quantidat departia en tres partes \\\hline
3.1.19 & si iudices iuramento essent astricti \textbf{ ut dicerent quod sentirent , } forte degenerarent & philosofoda la razon por que y podo mio assi lo establesçio ca creye que los iiezes eran estrennidos por iuramento \textbf{ que diessen lo que sentiessen e entendiessen } ca por auentura negarien de dezer lo que sienten ante los o \\\hline
3.1.19 & forte degenerarent \textbf{ timendo coram aliis dicere quod sentiunt . } Ideo ordinauit & ca por auentura negarien de dezer lo que sienten ante los o \textbf{ tristemiendosse dellos } e por ende ordeno que cada vno apareiadamente ordenasse su suina \\\hline
3.1.20 & quam si loquantur publice in praetorio : \textbf{ et si iurauerunt dicere quod sentiunt , } citius degenerabunt & que si fablassen en publico en audiençia \textbf{ e sy iuraren de dezer lo que sienten mas ayna proui raran } por esta manera priuada \\\hline
3.2.1 & Videtur autem Philos’ 3 Polit’ tangere , \textbf{ quatuor quae consideranda sunt } in regimine ciuitatis . & Mas el philosofo en el terçero libro delas politicas \textbf{ tanne quatro cosas | que son de penssar } enł gouernamiento del regno et dela çibdat \\\hline
3.2.1 & siue ad totum populum . \textbf{ Quare si considerentur quae requiruntur ad hoc quod tempore pacis per leges bene gubernetur ciuitas , } oportet in huiusmodi regimine & eston pertenesce a todos los çibdadanos o a todo el pueblo \textbf{ por la qual cosa si fueren penssadas las posas que son meester | para esto que la çibdat sea gouernada } entp̃o dela paz \\\hline
3.2.4 & Sed ut soluantur obiectiones praetactae , \textbf{ sciendum quod quia plura cognoscunt plures quam unus , } et citius corrumpitur unus quam plures , & e las obiectiones \textbf{ sobredichͣs deuedes saber | que la razon que dizia que muchs conosçen mas que vno¶ } Et la segunda que dizia que mas ayna se coronpe vno que muchs . \\\hline
3.2.5 & ait , quod hoc est difficile , \textbf{ videlicet quod sic patres possint } tradere regimen regni propriis filiis . & que esto es muy \textbf{ guaue conuene saber | que los padres pueden dar } assi el gouierno del regno asus fijos propreos . \\\hline
3.2.5 & sic etiam litigia oriuntur , \textbf{ si non determinetur quae persona } in illa prosapia debeat principari . & assi avn nasçen discordias \textbf{ e lides si non fuere determinada } qual perssona en qual linage deua ser prinçipe \\\hline
3.2.5 & sciens ipsum peruenire ad filium plus dilectum . \textbf{ Et si dicatur quod contingit aliquando magis diligere minores . } Talibus obiectionibus de facili respondetur : & que el regno parte nesçe al su fiio mas amado \textbf{ Et si dixiere alguno | que contesçe algunas uezes } que los padres mas aman alos menores \\\hline
3.2.5 & Quod vero superius tangebatur , \textbf{ videlicet quod ire per haereditatem , dignitatem regiam , } est exponere fortunae , & Mas aqual lo que dessuso fue dich \textbf{ conuiene saber | que quando va el regno } por hedat \\\hline
3.2.5 & cum bonum commune et totius regni in hoc consistat . \textbf{ Nec sufficit quod quia solus primogenitus regnare debet , } ut de eo solo cura habeatur diligens : & por que el bien de todo el regno esta en esto . \textbf{ Et non cunple | que por el primogenito deue regnar } que del solo deue ser tomada acuçia \\\hline
3.2.6 & et optat eos habere in dominos . \textbf{ Inde est quod antiquitus plures sic praeficiebantur in Reges . } Nam si aliquis fuerat primo beneficus , & por sennores \textbf{ e por ende antiguamente los mas de los sennores fueron tomados en Reyes . } por que si alguno era atal que feziera bien al pueblo \\\hline
3.2.6 & Ex hac autem differentia prima sequitur secunda : \textbf{ videlicet quod tyrannus intendit bonum delectabile : } Rex vero bonum honorificum . & e deste departimiento primero se sigue el segundo . \textbf{ Conuiene a saber que el thiranno entiende en el bien delectable . } Mas el rey en tiede el bien de honrraca \\\hline
3.2.6 & bonum proprium et priuatum , \textbf{ sequitur quod eius intentio versetur } circa bonum delectabile . & por la qual cosa si el thirano entiende el su bien propreo siguese \textbf{ que la su entençion es mala } ca non es cerca el bien honrrado e de honira \\\hline
3.2.6 & Ex hac autem secunda differentia sequitur tertia ; \textbf{ videlicet quod intentio tyrannica est circa pecuniam . } Tyrannus quia spreto communi bono non curat & Conuiene de saber \textbf{ que la entencion del tiran no es en auer riquezas o dineros | Ca el tirano } por que despreçia el bien comun \\\hline
3.2.6 & Ex hac autem differentia tertia \textbf{ sequitur quarta videlicet quod tyrannus non curat custodiri a ciuibus , } sed ab extraneis : & Et deste departimiento terçero se sigue el quarto . \textbf{ Conuiene de saber | que el tiranno non ha cuydado } de ser guardado de los çibdadanos mas de los estrannos . \\\hline
3.2.7 & vel si dominetur totus populus , \textbf{ dato quod sic dominantes } non intenderent & enssennoreare todo el pueblo \textbf{ puesto que los que assi enssenno rean non entiendan } si non el bien propo enpero non se arriedran del todo dela entençion del bien comun . \\\hline
3.2.8 & in finem dirigere . \textbf{ Ea vero quae deseruiunt } ut populus possit & e su pueblo a su fin \textbf{ Mas aquellas cosas que siruen aesto } por que el pueblo pueda alcançar su fin \\\hline
3.2.8 & et fons scripturarum , \textbf{ oportet quod inde totus populus } aliquam eruditionem accipiat : & e la fuente delas esc̀ yturas . \textbf{ conuiene que de ally tome todo el pueblo algun enssennamiento } e de prinda alguna sabiduria . \\\hline
3.2.10 & et conseruat , \textbf{ videns quod per ipsum , bonum commune , } et bonus status regni , & e mantiene le ueyendo \textbf{ que por el bien comun } e el buen estado del regno \\\hline
3.2.10 & Decima cautela tyrannica , \textbf{ est quod postquam procurauit diuisiones } et partes in regno , & La dezena cautela del tirano es que del \textbf{ pues que ha puesto vandos e departimientos en el regno } que con vn unado atormente al otro assi que con vn clauo atenaçe el otro . \\\hline
3.2.14 & volumus alias rationes adducere , \textbf{ ostendentes quod si reges cupiant suum durare dominium , } summo opere studere debent & avn en este cpleo queremos adozjr otras rrazones \textbf{ para mostrar que si los rrey e cobdiçian de duar muncho } el su señorio es toda manera deuen estudiar \\\hline
3.2.15 & sed etiam principatus ex hoc durabilior redditur , \textbf{ dato quod in ipso sit aliquid obliquitatis ad mixtum . } Tertium est , & se faze mas durable \textbf{ puesto que en el sea alguna cosa meztlada de maldat ¶ } La terçera cosa que guarda al gouernamiento del regno es meter mie \\\hline
3.2.15 & nam qui huiusmodi rationem non potest reddere , \textbf{ signum est quod ex furto } vel ex male ablato viuat . & es \textbf{ que biue de furto o de rapina . } ca assi fazie do podra guardar la iustiçia \\\hline
3.2.16 & quantum spectat ad praesens negocium , \textbf{ sufficienter tractauimus quae circa Principem sunt dicenda . } Restat ergo de consilio pertransire & Et quanto parte nesçe a este negoçio presente \textbf{ conplidamente dixiemos aquellas cosas | que eran de dezer } para enformaçion del prinçipe . \\\hline
3.2.16 & Secundo etiam consiliabilia \textbf{ non sunt quaecunque mobilia , } si semper uniformiter moueantur . & nin toma consseio de ninguna otra cosa que seño puede mudar ¶ \textbf{ Lo segundo non caen so consseio aquellas cosas } que se mueuen sienpre de vna manera \\\hline
3.2.17 & Priamum in consiliis esse secretarium et veracem , commendans eum dicebat , \textbf{ Iste est qui consuluit . } Ac si diceret , & e muy uerdadero \textbf{ alabandolo dize del este es aquel que conseia } assi commo si diriesse \\\hline
3.2.18 & non oportet ipsum esse existenter talem , \textbf{ sed sufficit quod videatur } vel appareat talis esse : & que el sea tal fechmas cunple \textbf{ que parezca tal cael o en iudga las cosas que paresçen de fuera por las cosas que vee } e por ende conplidamente es dada fe al omnen \\\hline
3.2.18 & Sed ad hoc quod bonus consiliator existat , \textbf{ non sufficit quod sit apparenter talis , } sed requiritur existenter talem esse . & del que es buen consseiero . \textbf{ mas para que el sea buen cosseiero non cunple } que parezra tal mas es meester \\\hline
3.2.18 & ideo ad hoc quod aliquis ex rebus \textbf{ de quibus loquitur fidem faciat , vel oportet quod sit prudens } vel quod credatur esse prudens . & para fazer tales cosas \textbf{ Morende para que alguno faga fe delas cosas de que fabla o conuiene que sea sabio } o que sea tenido por sabio . \\\hline
3.2.18 & et adhibetur fides , \textbf{ vel oportet quod sit bonus , } vel quod amicus , & a cuyos dichos creen los omes \textbf{ e es dada feo conuiene que sea bueono } que sea amigo \\\hline
3.2.18 & debet habere apparenter , \textbf{ oportet quod bonus consiliator habeat existenter : } satis apparet quales consiliatores deceat & deue auer en el \textbf{ e paresçer todas aquellas cosas | que ha todo buen conseiero en ssi de fecho . } Et por ende assaz parelçe quales conseieros deue auer el rey \\\hline
3.2.20 & aut per utrunque priusquam ostendamus qualiter sit iudicandum , \textbf{ declarare volumus quod quantum possibile est } sunt omnia legibus determinanda , et quam pauciora possunt & queremos declarar \textbf{ que quanto puede ser todas las cosas son de determinar | por las leyes } e las menores cosas \\\hline
3.2.20 & in uniuersali et de futuris , \textbf{ dicentes quicunque sic egerit , } sic puniatur , ignorantes an amicus , & e delas cosas que auien de venir diziendo \textbf{ que qual quier que tal cosa fiziere tal pena aura } non sabiendo si serie amigo o enemigo \\\hline
3.2.20 & et debeat illam subire sententiam . \textbf{ Nam si scirent quod amicus , } forte obliquerentur in iudicando , & e deuie passar por tal suina . \textbf{ ca si por auentra asopiessen ellos | que su amigo auie de fazer aquella cosa } por \\\hline
3.2.23 & erga delinquentes in ipsos ; \textbf{ ait quod magis debent } recordari bonorum & contra los que yerran contra ellos dize \textbf{ que mas se deuen acordar de los biens } que resçibieron del que \\\hline
3.2.24 & quarto modo ius distinxerunt , \textbf{ dicentes quod est quoddam ius naturale , } et quoddam ius gentium , & Mas los iuristas departieron en la quarta manera el derecho \textbf{ diziendo } que es algun derechn atal . Et alguno es derecho delas gentes \\\hline
3.2.24 & et dare quintam distinctionem iuris , \textbf{ dicendo quod quadruplex est ius , } videlicet naturale , animalium , gentium , et ciuile . & e el quinto departimiento del derech̃ . \textbf{ diziendo que en quatro maneras se departe el derech . | Conuiene a saber ende recħ natural } e en derecho delas ainalias \\\hline
3.2.24 & et edicta principum non sunt eadem apud omnes . \textbf{ Inde est quod ius naturale dicitur } differre a positiuo : & Et por ende se sigue \textbf{ que el derecho natural es departido del derechpo sitiuo . } ca el derech natural \\\hline
3.2.24 & sed ad placitum . \textbf{ Inde est quod omnes homines loquuntur , } non tamen omnes proferunt idem idioma . & lenguare o otro esto non es natraal mas es a uoluntad . \textbf{ Et por ende es que todos los omes fablan } enpero non fablan todos vn lenguaie . \\\hline
3.2.24 & hoc praesupponens ius positiuum procedit ulterius , \textbf{ determinans qua poena sint talia punienda . } Hoc viso quantum & positiuo prisu pone esto va adelante \textbf{ determinando de qual pena de una ser | tales cosas castigadas o condep̃nadas . } ¶ Esto uisto quanto pertenesçe alo presente podemos mostrar dos departimientos \\\hline
3.2.25 & ut ius animalium . \textbf{ Ad cuius euidentiam sciendum quod homo } ut est homo et secudum propriam rationem consideratus differt & que es derecho delas aian lias . \textbf{ Et para declaraçion desto conuiene de saber | que el omne } en quanto es omne e penssando segunt su razon proprea \\\hline
3.2.25 & quod ius naturale , \textbf{ est quod natura omnia animalia docuit . } Huiusmodi autem ius & dize el derecho natural \textbf{ enssenna a todas las ainalias . } Ca este derechotal assi commo alli dize \\\hline
3.2.25 & ut emptio , venditio , locatio , conductio et cetera talia , \textbf{ sine quibus societas humana } non bene sufficit sibi ad vitam . & e al qual es \textbf{ et otras cosas tales | sin las quales la conpanna de los omes non ha conplidamente } aquello que ha mester para la uida . \\\hline
3.2.26 & ad legem naturae , \textbf{ oportet quod sit iusta : } ut comparatur ad bonum commune , & en quanto es conparada ala ley de natura \textbf{ conuiene que sea derechͣ . } Et en quanto es conparada al bien comun \\\hline
3.2.26 & ut comparatur ad bonum commune , \textbf{ necesse est quod sit utilis : } sed ut refertur ad populum & Et en quanto es conparada al bien comun \textbf{ conuiene que sea aprouechosa . } Mas en quanto es conparada al pueblo \\\hline
3.2.26 & et debet regulari per huiusmodi legem , \textbf{ oportet quod sit competens } et compossibilis consuetudini patriae et tempori : & el qual pueblo deueser reglado por aquella ley . \textbf{ conuiene que sea conuenible } e que conuenga con el uso \\\hline
3.2.27 & sunt leges et regulae agibilium , \textbf{ sequitur quod non a bono priuato et domestico sed a bono } quod intenditur in regno et ciuitate sumendae & e las reglas delas nuestras obras \textbf{ del bien propreo nin dela casa . | Mas del bien comun } que es entendido en el regno \\\hline
3.2.28 & quodammodo indifferentes . \textbf{ Multa enim sunt quae de se sunt quasi indifferentia , } licet forte ex intentione operantium possint & que nin son malas nin buenas . \textbf{ ca muchas cosas son dessi tales | que nin son malas nin buenas } maguer que por auentura \\\hline
3.2.28 & His itaque sic pertractatis , \textbf{ dicamus quod decet Reges et Principes , } quorum interest solicitari & Et por ende estas cosas \textbf{ assi tractadas digamos | que pertenesçe alos Reyes } e alos prinçipes \\\hline
3.2.31 & quaedam gens statuit \textbf{ videlicet quod si aliquis ciuis esset occisus , } et aliquis consanguineus mortui inuaderet aliquem ciuem , & assi commo dize el pho establesçio vna gente \textbf{ que si algun çibdada no matasse a otro en la çibdat } e algun pariente uiniesse acometiendo contra algun çibdadano . \\\hline
3.2.31 & de quae sito , \textbf{ sciendum quod lex positiua si recta sit , } oportet quod innitatur legi naturali , & e qual es la soluçion della . \textbf{ Conuiene de saber que la ley politica sitiua | si fuere derecha conuiene que se raygͤ } e se funde enla ley natural . \\\hline
3.2.31 & sciendum quod lex positiua si recta sit , \textbf{ oportet quod innitatur legi naturali , } et quod determinet gesta particularia hominum . & si fuere derecha conuiene que se raygͤ \textbf{ e se funde enla ley natural . } Et conuiene que determine las obras \\\hline
3.2.31 & ut patet ex Institutis de iure naturali , \textbf{ ubi dicitur quod leges humanae contrariae sunt iuri naturali ; } quia iure naturali & assi commo paresçe en la \textbf{ institutado dize que las leyes humanales contrarias son al derecho natraal . } Ca de comienço todos los omes \\\hline
3.2.31 & quia non complete determinant particularia agibilia , \textbf{ dato quod occurrant leges meliores et magis sufficientes , } non est assuescendum innouare leges . & por que non determinan conplidamente los fechos particula respuesto \textbf{ que sean falladas leyes meiores e mas conplidas . } Enpero non nos auemos a acostunbrar a renouar las leyes . \\\hline
3.2.32 & accipienda est eius notitia , \textbf{ benedictum est quod dicitur 3 Polit’ } quod unitas loci , communicatio connubiorum , compugnationis gratia , commutatio rerum , & e el conosçimiento de aquella cosa . \textbf{ Bien diches lo que dize el philosofo | en el terçero libro delas politicas . } que la morada del lo gar \\\hline
3.2.32 & et cetera talia sunt ea , \textbf{ sine quibus non habet esse ciuitas . } Principaliter tamen est ciuitas constituta & Et o tristales cosas son aquellas \textbf{ sin las quales non puede ser la çibdat . } Enpero prinçipalmente es establesçida la çibdat \\\hline
3.2.33 & et de facili rationi obediunt . \textbf{ Hoc est quod dicitur 4 Politicorum } quod quoniam mediocre est optimum , & e de ligero los omes obedescran ala razon . \textbf{ Et esto es lo que dize el philosofo en el quarto libro delas politicas } que por que lo medianero es muy bueno la possesion medianera es muy buena . \\\hline
3.2.34 & non solum Regibus recte regentibus , \textbf{ sed etiam dato quod in aliquo tyrannizarent , } studeret populus obedire illis . & mas avn alos malos . \textbf{ Ca avn puesto | que en alguna cosa tira } nizen deue estu diar \\\hline
3.3.1 & pugnare pro iustitia et pro iuribus , \textbf{ remouere quaecunque } impedire possunt commune bonum . & e con todo su poder tirar e arredrar \textbf{ quales se quier cosas } que enbarguen el bien comun . \\\hline
3.3.2 & et etiam potentes ad tolerandum labores : \textbf{ dicere possumus quod Fabriferrarii , et carpentarii } utiles sunt ad opera bellica : & e deuan ser poderosos para sofrir los trabaios . \textbf{ Podemos dezir } que los ferreros e los carpenteros son aprouechables a las obras de la batalla \\\hline
3.3.6 & quod se bene didicisse confidit . \textbf{ Inde est quod tantum valet armorum exercitatio , } quod in bellorum certamine paucitas exercitata & que lo sabe bien . \textbf{ Et dende viene | que tanto vale el vso de las armas } que en la contienda de las batallas pocos omnes bien usados son apareiados \\\hline
3.3.7 & non grauabantur in percutiendo cum claua , \textbf{ vel in sustinendo quoscunque labores bellicos . } Tertio exercitandi sunt bellatores & quando despues venien a la batalla non resçibien trabaio en ferir con la maca \textbf{ nin en sofrir quales quier otros trabaios de la batalla . } Lo terçero son de usar los lidiadores \\\hline
3.3.7 & utile est eos sagittis impugnare : \textbf{ immo dato quod pugnantes } se cum hostibus possint coniungere , & prouechosa cosa es lançar las saetas \textbf{ mas puesto que los lidiadores se puedan ayuntar con los enemigos } ante que se apunte con ellos \\\hline
3.3.7 & contingit multos periclitatos esse . \textbf{ Inde est quod apud Romanos antiquitus consuetudo erat , } quod iuuenes futuri bellatores & caen en muchos periglos . \textbf{ Et por ende era costunbre antiguamente entre los romanos } que los mançebos \\\hline
3.3.10 & qui debet esse eorum caput et eorum directiuum . \textbf{ Inde est quod antiquitus } ne accideret & mayorales que sean cabeças dellos e guiadores en la hueste . \textbf{ Et por ende antiguamente } por que non acaesçiesse con fondimiento \\\hline
3.3.11 & Sic enim dicendo , \textbf{ dato quod accideret } aliquis repentinus insultus , & puedan defender se de los acometedores . \textbf{ Ca assi diziendo puesto } que contesçiesse algun rebate a desora \\\hline
3.3.12 & et ex quibus regionibus sunt meliores pugnantes \textbf{ et est quibus artibus sunt meliores bellicosi : } declarauimus etiam qualiter in exercitu & e de quales tierras son los meiores lidiadores \textbf{ e de quales artes son de escoger los mayores lidiadores . } Et avn declaramos en qual manera en la hueste \\\hline
3.3.13 & difficilius itur ad carnem . \textbf{ Inde est quod bellorum experti dicunt pugnantes } semper debere habere loricas amplas ita , & por el detenemiento de las armas . \textbf{ Et dende viene | que los que son prouados en las batallas . } dizen que los lidiadores sienpre deuen auer las lorigas anchas . \\\hline
3.3.15 & innititur alii pedi non moto : \textbf{ oportet quod pars dextra innitatur parti sinistrae quiescenti . } Cum igitur pes sinister anteponitur , & En essa manera conuiene \textbf{ que quando se mueue la parte diestra | que se afirme sobre la siniestra } que se non mueue . \\\hline
3.3.15 & in eos qui contra pugnant . \textbf{ Inde est quod laudatur Scipionis sententia , dicentis : } Nunquam sic esse claudendos hostes , & pueden acometer muchas malas cosas contra aquellos que lidian contra ellos . \textbf{ Et por ende es alabada la sentençia de çipion } por la qual dizia que nunca eran de encerrar los enemigos \\\hline
3.3.16 & plus affligit fames quam gladius . \textbf{ Inde est quod multotiens obsidentes } volentes citius opprimere munitiones , & mas que la espada nin el cuchiello . \textbf{ Et por ende contesçe que muchas uegadas } los que çercan queriendo \\\hline
3.3.17 & Ideo nisi sint muniti , \textbf{ contingit quod existentes in castris } ( cum fuerint occupati obsidentes somno , & Et por ende si non estudieren guarnesçidos puede les contesçer \textbf{ que los que estan en los castiellos | o en las çibdades cercadas } quando los que çercan durmieren o comieren o estudieren de vagar o fueren derramados \\\hline
3.3.19 & ad impugnandum munitionem aliquam , \textbf{ dato quod quis non possit pertingere usque ad muros eius . } Nam quia huiusmodi trabs habens caput sic ferratum retrahitur et impingitur , & Et uale este artifiçio para acometer alguna fortaleza . \textbf{ puesto que non puedan llegar a los muros della . } Ca por que esta viga ha la cabeça \\\hline
3.3.19 & poterit percuti murus ipsius munitionis obsessae , \textbf{ dato quod textura illa } sub qua sunt homines impingentes trabem non pertingat usque ad muros . & assi que puede de lueñe ferir en los muros de la fortaleza cercada \textbf{ puesto que el techo so que estan los omnes } que mueuen el \\\hline
3.3.20 & ne deuincatur per machinas lapidarias . \textbf{ Nam dato quod per huiusmodi machinas } totus murus exterior rueret , & por las piedras de los engeñios . \textbf{ Ca puesto que todo el muro de fuera fuesse destroydo | por las piedras de los engeñios en } pero el muro que esta fecho de tierra \\\hline
3.3.20 & portas munitionis succendere , \textbf{ cataracta quae est ante portam prohibebit eos . } Rursus supra cataractam debet & Ca si los que cercan quisieren llegar a quemar las puertas de la fortaleza . \textbf{ esta tal puerta que esta | ante las otras puertas gelo defendra . } Otrossi sobre las puertas de la trayçion \\\hline

\end{tabular}
