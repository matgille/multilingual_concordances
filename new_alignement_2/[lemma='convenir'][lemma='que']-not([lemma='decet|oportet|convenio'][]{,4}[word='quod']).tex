\begin{tabular}{|p{1cm}|p{6.5cm}|p{6.5cm}|}

\hline
1.1.5 & en el segundo libro delas ethicas \textbf{ que conuiene que las obras } por las quales nos alcançamos la fin & ut dicitur 2 Ethic . ) \textbf{ oportet operationes , } per quas finem consequimur , \\\hline
1.1.13 & Et commo el amor sienpre sean los semeiables e acordables con el . \textbf{ Conuiene que aquel que es para de ser } gualardonado de dios & cum semper amor sit ad similes , et conformes , \textbf{ oportet esse similem , } et conformem Deo , \\\hline
1.2.2 & Et el otro es para cobdiçiar \textbf{ ¶ Conuiene que toda uirtud moral } o sea en el entendimiento o en la uoluntad o en el appetito enssannador & irascibilis scilicet , et concupiscibilis : \textbf{ oportet omnem virtutem moralem , } vel esse in intellectu , \\\hline
1.2.5 & Ca toda obra si deue ser uirtuosa . \textbf{ conuiene que se faga sabiamente } e iusta mente . & Nam omnis actus , \textbf{ si virtuosus esse debet , } oportet quod fiat prudenter , iuste , fortiter , et temperate : \\\hline
1.2.6 & en las cosas singulares . \textbf{ Conuiene que la pradençia sea cerca las cosas singulares } e particulares & et agibilia sint singularia , \textbf{ oportet prudentiam esse circa particularia , } applicando uniuersales regulas \\\hline
1.2.7 & e los mançebos \textbf{ conuiene que naturalmente sean subiectos de los mas antigos } por que non son espiertos & Hoc etiam modo iuuenes naturaliter decet \textbf{ antiquioribus esse subiectos , } quia inexperti agibilium \\\hline
1.2.7 & por que naturalmente sea sennor \textbf{ conuiene que florescaen sabiduria e en entendimiento } or que nunca conplidamente se pueda auer el todo & Ut igitur Rex naturaliter dominetur oportet \textbf{ quod polleat prudentia , et intellectu . } Quot , et quae oporteat habere Regem , \\\hline
1.2.31 & commo los p̃h̃osacuerdan en esta sentençia \textbf{ que conuiene que todas las uirtudes sean ayinntadas la vna con la vna con la otra . } Ca dixieron que aquel que ha vna uirtud & in hanc sententiam conuenerunt , \textbf{ quod oportet virtutes connexas esse . } Dixerunt enim \\\hline
1.3.4 & Por la qual cosa si el desseo deue tomar mesura del amor \textbf{ Conuiene que los Reyes e los prinçipes desse en prinçipalmente el buen estado del regno } assi que todos quantos son en el regno & mensuram sumere \textbf{ ex amore , | principaliter Reges et Principes debent desiderare bonum statum regni : } ut quod qui in regno sunt , \\\hline
1.3.9 & terca la qual es el dolor e la tristeza . \textbf{ Conuiene que la delectaçion e la tristeza } sean prinçipales passiones & circa quod est dolor et tristitia \textbf{ oportet delectationem et tristitiam } esse principales passiones respectu concupiscibilis . \\\hline
1.4.7 & e non sepa sotrir la buena uentra a \textbf{ conuiene que con las riquezas sea aconpanada la nobleza } porque el rico e el noble de antiguedat & et nesciat fortunas ferre , \textbf{ expedit ut diuitias concomitetur nobilitas . } Diues enim nobilis et ab antiquo , \\\hline
1.4.7 & li dellos contamos algunas malas costunbres \textbf{ ca non conuiene que todos seantales . } Mas abasta que aquellas costunbres sean falladas en muchos por que non & aliquos malos mores : \textbf{ quia non oportet omnes esse tales , } sed sufficit reperiri illud in pluribus : \\\hline
2.1.2 & e por si vale a conplimiento dela uida . \textbf{ Conuiene que la comunidat dela casa sea mas neçessaria } Et pues que assi es los Reyes e los prinçipes & ad per se sufficientiam vitae , \textbf{ oportet communitatem domus necessariam esse . } Reges ergo et Principes , \\\hline
2.1.4 & nin conpannia de vno \textbf{ assi commo si queremos saluar la comuidat dela casa conuiene que ella sea establesçida de muchͣs perssonas } mas assi commo adelanţe paresçra & cum non sit proprie communitas nec societas ad seipsum , \textbf{ si in domo communitatem saluare volumus , | oportet eam } ex pluribus constare personis ; \\\hline
2.1.5 & e enl entendemiento para se saluar \textbf{ conuiene que obedezca } e que sirua a aquel que ha sabiduria e entendemiento . & et deficit scientia , et cognitione : \textbf{ si saluari debet , | expedit ei ut obtemperet , } et seruiat vigenti prudentia , et intellectu . Expedit ergo seruo , \\\hline
2.1.5 & e aministdores que los siruna . \textbf{ Conuiene que ayan algunan cosa en logar de sieruo } assi commo es bue̊ o asno o alguna otra ainalia & et ministrum rationalem , \textbf{ est eis hos vel aliquod aliud animal pro seruo , } et habent ministrum irrationalem ; \\\hline
2.1.6 & luego que es fecha fazer otra semeiante \textbf{ assi mas conuiene que ella primeramente sea acabada } enssi & potest sibi simile producere , \textbf{ sed oportet prius ipsam esse perfectam . } statim enim , \\\hline
2.1.6 & luego otro su semeinante \textbf{ mas conuiene que primeramente el sea acabado . } Et pues que assi es engendrar su semeiante non pertenesçe a cosa natural tomada en qual quier manera mas pertenesçe a cosa natural en quanto ella es acabada . & nec statim potest sibi simile producere , \textbf{ sed oportet prius ipsum esse perfectum : | producere ergo sibi similem , } non est de ratione rei naturalis \\\hline
2.1.6 & Et por ende paresçe que para que la casa sea acabada \textbf{ que conuiene que sean enlla tres comuundades . } ¶ La vna del uaron e dela muger ¶ & Patet ergo quod ad hoc quod domus habeat esse perfectum , \textbf{ oportet ibi esse tres communitates : } unam viri et uxoris , aliam domini et serui , \\\hline
2.1.6 & e tres gouernamientos de ligero puede paresçer \textbf{ que conuiene que sean y . } quatro linages de perssonas & de leui patere potest , \textbf{ quod ibi oportet } esse quatuor genera personarum . \\\hline
2.1.6 & Ca commo en la casa acabada sean tres gouernamientos . \textbf{ Ca conuiene que este libro sea partido en tres partes . } ¶ En la primera delas quales tractaremos del gouernamiento mater moianl . & Nam cum in domo perfecta sint tria regimina , \textbf{ oportet hunc librum tres habere partes ; } in quarum prima tractetur primo de regimine coniugali : \\\hline
2.1.9 & e segunt ordenna traal . \textbf{ Conuiene que todos los çibdadanos sean pagados } cada vno de vna sola mugier . & et ordinem naturalem , \textbf{ decet omnes ciues } una sola uxore esse contentos . \\\hline
2.1.10 & assi commo si algun çibdadano es subiecto al preuoste e al Rey . \textbf{ Conuiene que el } prinoste sea ordenado al Rey e sea so el . & ut si quis subiicitur Proposito et Regi , \textbf{ oportet Propositum illum ad Regem ordinari , } et esse sub ipso repugnat \\\hline
2.1.10 & ala generacion de los fijos \textbf{ conuiene que las mugers de todos los çibdadanos } por que non sea enbargado el & ad bonum prolis , \textbf{ decet coniuges omnium ciuium , } ne impediatur earum foecunditas , \\\hline
2.1.15 & por la natura \textbf{ conuiene que sea muy ordenado . } Ca aquel gnia la natura de que viene todo ordenamiento & et quicquid natura praeparatur , \textbf{ oportet ordinatissimum esse : } quia ille naturam dirigit , \\\hline
2.1.15 & para la generaçino de los fijos \textbf{ non conuiene que sea ordenada a seruir . } ¶ Et pues que assi es non es orden natural & coniugem ad generationem , \textbf{ indecens est | quod ordinetur ad seruiendum . } Non est ergo naturalis ordo , \\\hline
2.1.15 & por que non sabe gniar assi mismo \textbf{ e conuiene que sea gado de otro este } tal es naturalmente barbaro e sieruo & quia nescit seipsum dirigere , \textbf{ et expedit ei quod ab aliquo alio dirigatur , } idem est esse natura barbarum et seruum . \\\hline
2.1.19 & de aquel que non es su padre . \textbf{ Mas conuiene que el } padresea çierto de su fijo . & ut alius filius non succedat in haereditatem , \textbf{ sed requiritur } ut pater sit certus de sua prole . \\\hline
2.1.19 & Para que el padre sea çierto de sus fijos \textbf{ conuiene que las mugers sean linpias e honestas e guardadas en sus palauras ¶ } Lo terçero conuiene a ellas de ser abstinentes & ut pater sit certus de sua prole , \textbf{ expedit coniuges pudicas esse . } Tertio oportet eas esse abstinentes , \\\hline
2.1.23 & e de dios \textbf{ en que es conplimiento de sabiduria . Et por ende conuiene que aquellas cosas } que faze la natura & Natura enim cum moueatur ab intelligentiis , et a Deo , \textbf{ in quo est suprema prudentia ; } oportet quod agat ordinate et prudenter . \\\hline
2.1.24 & suso deuemos mostrar \textbf{ quales obras conuiene que vsen las mugers } Mas por que dellas diremos adelante & Sed quia de eis infra dicetur , \textbf{ volumus ea hic silentio praeterire . Primae partis secundi libri de regimine Principum finis , | in qua traditum fuit , } quo regimine Reges et Principes debeant suas coniuges regere . \\\hline
2.2.1 & assi commo se praeua en el viij delas . ethicas . \textbf{ Conuiene que los padres por amor natural } que han alos fijos sean cuy dadosos della & ut probatur 8 Ethicorum , \textbf{ decet patres ex ipso amore naturali , } quem habent ad filios , \\\hline
2.2.6 & Ca quando alguno es inclinado a alguacosa . \textbf{ Conuiene que el vse mucho en el contrario } por que non sea inclinado a aquella cosa . & Nam cum aliquis est pronus ad aliquid , \textbf{ oportet ipsum multum assuescere in contrarium , } ne inclinetur ad illud : \\\hline
2.2.7 & Lo primero paresce assi . \textbf{ Ca conuiene que los que quieren aprinder sçiençia de letris } que aprendan pronunçiar departidamente las palabras delas letras ¶ & quae est ex scientia acquirenda . \textbf{ Decet enim volentes literas discere , } literales sermones scire distincte proferre . \\\hline
2.2.9 & Et pues que assi es non abasta al conplimiento dela sçiençia \textbf{ que alguno sea inuentiuo e fallador dessi mas conuiene que sea agudo } e que entienda los dichͣs de los otros . & Non ergo sufficit ad perfectionem scientiae \textbf{ quod aliquis sit inuentiuus ex se , | sed quod sit perspicax } et intelligens aliorum dicta . \\\hline
2.2.9 & e que entienda los dichͣs de los otros . \textbf{ ¶ Lo terçero conuiene que sea iudgador } e que aya razon para iudgar . & et intelligens aliorum dicta . \textbf{ Tertio oportet ipsum esse iudicatiuum : } nam perfectio scientiae potissime \\\hline
2.2.11 & Ca si la vianda se ouiere bien a cozer \textbf{ conuiene que sea bien proporçionada ala calentura natural } Por la qual cosa si en tan grand quantia se & Si enim cibus digeri debeat , \textbf{ oportet | ipsum esse proportionatum calori naturali . } Quare si in tanta quantitate sumatur , \\\hline
2.3.13 & que alguna suidunbre es dichͣ natural \textbf{ e que conuiene que alg ssean subietos naturalmente a algunos otros } la qual cosa praeua el philosofo & Ostendemus enim primo seruitutem aliquam naturalem esse , \textbf{ et quod naturaliter expedit aliquibus aliis esse subiectos : } quod probat Philosophus primo Polit’ quadruplici via , \\\hline
2.3.14 & la primera razon paresçe assi . \textbf{ Ca conuiene que el sennor } segunt & Prima congruitas sic patet : \textbf{ oportet enim dominans } ( ut dicitur in Politic’ ) \\\hline
2.3.15 & Mas aquellos que non son poderosos \textbf{ conuiene que sean ministros e siruientes por ley } assi commo si algunos fallesciessen enl poderio & Impotentes vero contingit \textbf{ esse ministros ex lege : } ut si qui in potentia deficientes ; \\\hline
2.3.15 & que dende aurian de resçebir \textbf{ Et conuiene que ouiessen algunos otros } seruientesamadores e uirtuosos & intuitu mercedis , \textbf{ et aliquos dilectiuos virtuosos ministrantes } ex amore boni . \\\hline
2.3.15 & e el amor de bien los inclina asuir . \textbf{ Conuiene que los prinçipes se ayan çerca ellos } assi commo cerca de fijos . & quos virtus et amor boni inclinat ad seruiendum , \textbf{ decet principantes se habere quasi ad filios , } et decet eos regere non regimine seruili , \\\hline
2.3.16 & En essa misma manera cada vna muchedunbre si bien ordenada es \textbf{ conuiene que sea aduchͣa vn ordenador } de quien ella sea ordenada . & si debet esse ordinata , \textbf{ oportet reduci in unum aliquem , } a quo ordinetur . \\\hline
2.3.17 & Lo terçero çerca la prouision delas uestidas es de penssar la condiçion delas personas \textbf{ por que non conuiene que todos sean uestidos } de eguales uestiduras caenta & consideranda est conditio personarum . \textbf{ Nam non omnes decet } habere aequalia indumenta . \\\hline
3.1.1 & e todo cunplimiento ha de ser \textbf{ or que toda çibdat conuiene que sea alguna comunindat } commo toda comunidat sea por graçia de algun bien . & in quo declaratur quomodo maiestas regia praeesse debeat ciuitati et regno . \textbf{ Quoniam omnem ciuitatem contingit | esse communitatem quandam , } cum omnis communitas fit \\\hline
3.1.1 & commo toda comunidat sea por graçia de algun bien . \textbf{ Conuiene que la çibdat sea establesçida por algun bien } Ca pruena el pho & gratia alicuius boni , \textbf{ oportet ciuitatem ipsam constitutam esse propter aliquod bonum . } Probat autem Philosophus primo Polit’ duplici via , \\\hline
3.1.1 & que establescen la çibdat . \textbf{ Conuiene que la çibdat sea establesçida } por grande alguna cosa & ex parte hominum constituentium ciuitatem oportet \textbf{ ipsam constitutam esse gratia eius } quod videtur bonum ; \\\hline
3.1.6 & niguno non abasta assi mismo en fallar algunan arte \textbf{ mas conuiene que sea ayuda de } por ayuda de los que passaronante & in inueniendo artem aliquam , \textbf{ sed oportet ad hoc iuuari } per auxilium praecedentium \\\hline
3.1.8 & por que nos auemos me estermuchͣs cosas departidas para abastamiento dela uida \textbf{ conuiene que enla çibdat sea algun departimiento . } La tercera razon que declara e manifiesta las razones & Quia ergo diuersis indigemus ad vitam , \textbf{ oportet in ciuitate diuersitatem esse . } Tertia via declarans \\\hline
3.1.11 & que si fuessen comunes \textbf{ mas conuiene que las cosas sean comunes } segunt uirtud de franqueza & debitam diligentiam circa illa . \textbf{ Expedit autem talia esse communia secundum liberalitatem : } quia cives inter se debent liberales esse , \\\hline
3.1.12 & que son meester para la batalla \textbf{ ca los omes lidiadores conuiene que sean cuerdos } por entendimiento e sabios & secundum tria quae requiruntur ad bellum . \textbf{ Homines enim bellatores decet } esse mente cautos et prouidos : \\\hline
3.1.15 & assi commo los nobles \textbf{ Por ende conuiene que estos nobles de una } prinçipalmente defender la tierra entre los otros & ut nobiles : \textbf{ hi videlicet nobiles potissime debent defendere patriam , } et eorum maxime est vacare \\\hline
3.2.5 & que el gouernamiento real \textbf{ que conuiene que xaya por suçession } e por heredat alos fijos & arguere possumus , \textbf{ quod expediat regale regimen } in filios per haereditatem succedere . \\\hline
3.2.6 & quando es muy escalentada \textbf{ e muy enraleçida conuiene que la raledat e la calentura mas acabadamente sea fallada en el fuego } despues que fuere engendrado e ençendido . & cum calefit et rarefit , \textbf{ oportet raritatem et calorem perfectius reperiri } in igne iam generato \\\hline
3.2.8 & Lo primero que en tal manera sea el pueblo apareiado e ordenado por que pue da alcançar su fin que entiende . \textbf{ Lo segundo conuiene que sean arredradas todas aquellas cosas } que enbargan de alcançar aquella fin & ut possit \textbf{ consequi finem intentum . | Secundo , ut remoueantur prohibentia et deuiantia } ab huiusmodi fine . \\\hline
3.2.16 & ca el nuestro consseio non es dela fu . \textbf{ por que conuiene que en el conseio sorongamos la fin } e que non tomemos consseio della & sed de his quae sunt ad finem : \textbf{ oportet enim in consilio | praesupponere finaliter intentum , } et non consiliari de ipso , \\\hline
3.2.19 & e enssenados los conseieros e los sabidores dellas . \textbf{ Lo primero conuiene que el conseio del Rey } sea cerca las sus rentas & consiliatores esse instructos . \textbf{ Primo enim contingit esse Regis consilium circa prouentus , } in quo duo sunt attendenda . \\\hline
3.2.19 & lo segundo ha de tener mient̃s el Rey de non ser engannado enlas sus rentas . \textbf{ ca conuiene que el conseio del Rey sea bue no para saluar } por todo su ponder los derechs del Rey . & ne in suis prouentibus defraudetur : \textbf{ expedit enim regium consilium } pro viribus saluare iura Regis , \\\hline
3.2.19 & e a buen estado del Rey e del pueblo \textbf{ Et pues que assi es conuiene que los } consseierossepan las entradas e las sallidas del regno & et ad bonum statum eius : \textbf{ decet ergo consiliarios } scire introitus \\\hline
3.2.19 & e de los otros derechos . \textbf{ Et conuiene que sepan las rentas del regno } las que han de venir al Rey quales e quantas son & scire introitus \textbf{ et prouentus regni , } quos oportet peruenire ad regem , \\\hline
3.2.20 & esto mismo se puede assi declarar . \textbf{ ca assi commo paresçra adelançe conuiene que algunas cosas sean puestas en aluedrio } e en poder de los iuezes & sic declarari potest . \textbf{ Nam ( ut infra patebit ) } oportet \\\hline
3.2.22 & por abortençia o por mal querençia . \textbf{ conuiene que el uiez judgue mal e desigual mente . } Ca entonçe el uuzio non salle de zelo de iustiçia & ab alia vero recedit per odium , \textbf{ oportet ipsum iudicare inique : } quia tunc iudicium non procedit \\\hline
3.2.23 & mayoraspeza e mayor dureza de quanta deue . \textbf{ Conuiene que por el entendimiento piadoso sea atenprada la guaueza dela pena } e esto es lo que dize el pho & sunt amplioris seueritatis contentiua , \textbf{ decet ut per pium intellectum moderetur supplicii magnitudo , } hoc est ergo quod dicitur 1 Rhetor’ \\\hline
3.2.26 & Et en quanto es conparada al bien comun \textbf{ conuiene que sea aprouechosa . } Mas en quanto es conparada al pueblo & ut comparatur ad bonum commune , \textbf{ necesse est quod sit utilis : } sed ut refertur ad populum \\\hline
3.2.26 & Ca segunt que estas tales cosas se departen \textbf{ conuiene que enlas leyes sea algun departimiento . } ues que assi es . & quia secundum quod talia diuersificantur , \textbf{ oportet in ipsis legibus } aliquam diuersitatem existere . \\\hline
3.2.26 & ues que assi es . \textbf{ Lo primero conuiene que la ley humanal o positiua sea derecha } en quanto es conparada ala razon natural o ala ley de natura . & aliquam diuersitatem existere . \textbf{ Primo igitur oportet legem humanam | siue positiuam esse iustam } ut comparatur ad rationem naturalem \\\hline
3.2.26 & en la qual es entendido el bien propreo . \textbf{ Ca conuiene que enlas leyes } si derechͣs fueren & in qua intenditur priuatum bonum ; \textbf{ oportet enim in legibus } ( si rectae sint ) \\\hline
3.2.26 & e esto queremos alcançar \textbf{ conuiene que fagamos estas cosas . } Et pues que assi estales deuen ser las leyes & et hoc sequi volumus , \textbf{ oportet hoc agere . } Tales ergo debent esse leges , \\\hline
3.2.26 & porque el bien propra o es ordenado al bien comun . \textbf{ Conuiene que las leyes tales sean non } quales demanda el bien propre & et bonum priuatum ordinetur ad ipsum , \textbf{ oportet tales leges fieri } non quales requirit bonum priuatum , \\\hline
3.2.26 & por las palabras solas . \textbf{ Et por ende conuenia que por estas tales fuessen establesçidas las leyes . } las quales assi commo dize al philosofo & nec per solos sermones corriguntur : \textbf{ oportuit igitur saltem | propter tales statuere leges , } quae ( ut dicitur 10 Ethicorum ) \\\hline
3.2.27 & Poque la ley aya uirtud e fuerça de obligar \textbf{ conuiene que sea publicada e pregonada . } Mas commo otra sea la ley natural e otra la positiua en vna manera se deue publicar la vna & ad hoc quod lex habeat vim obligandi , \textbf{ oportet eam promulgatam esse . } Sed cum alia sit lex naturalis , \\\hline
3.2.29 & generalmente aquello que non es general mente \textbf{ Por que conuiene que las leyes humanales } commo quier que sean examinadas de fallesçer en algun caso . & quod non est uniuersaliter : \textbf{ oportet enim humanas leges } quantumcunque sint exquisitae \\\hline
3.2.29 & que por buena ley que por la ley non puede determinar todas los casos particulates . \textbf{ Por ende conuiene que el Rey o otro prinçipe } por razon derechͣo por ley natural . & eo quod lex particularia determinare non potest . \textbf{ Ideo expedit Regem | aut alium principantem per rationem rectam , } aut per legem naturalem , \\\hline
3.2.29 & quela ley demanda o que la ley nidga . \textbf{ Et algunas vezes conuiene que la regla se encorue } ala parte contraria & quam lex dictat : \textbf{ aliquando etiam oportet eam plicare ad partem oppositam , } et rigidius punire peccantem , \\\hline
3.2.30 & comunalmente non puede alcançar forma de beuir en punto . \textbf{ Por ende conuiene que } dessemeie alguons pecados & attingere punctalem formam viuendi , \textbf{ ideo oportet aliqua peccata dissimulare } et non punire lege humana , \\\hline
3.2.30 & por ley humanal . \textbf{ Et por ende conuiene que sin la ley humanal fuesse } dada otra ley diuinal & quae lege humana puniri non possunt . \textbf{ Oportuit igitur praeter legem humanam } dari aliquam legem , \\\hline
3.2.30 & por la qual somos ordenados a aquel bien . \textbf{ Et por ende conuiene que los Reyes e los prinçipes } alos quales parte nesçe ser & per quam ordinamur ad illud bonum . \textbf{ Decet ergo reges et principes , } quos competit esse quasi semideos , \\\hline
3.2.31 & Conuiene de saber que la ley politica sitiua \textbf{ si fuere derecha conuiene que se raygͤ } e se funde enla ley natural . & de quae sito , \textbf{ sciendum quod lex positiua si recta sit , } oportet quod innitatur legi naturali , \\\hline
3.2.31 & e se funde enla ley natural . \textbf{ Et conuiene que determine las obras } e los fechos particulares de los omes . & oportet quod innitatur legi naturali , \textbf{ et quod determinet gesta particularia hominum . } Dupliciter ergo potest \\\hline
3.2.33 & uenta el philosofo en el quarto libro delas politicas \textbf{ que conuiene que sean tres partes dela çibdat . } Ca alguons son muy ricos . & Quarto Politicorum ait Philosophus , \textbf{ quod tres oportet | esse partes ciuitatis . } Nam alii quidem sunt opulenti valde , \\\hline
3.2.34 & Et la uirtud faze al que la ha buenon \textbf{ Esta buean obra conuiene que sea en el gouernamiento derech } que el buen çibdada no sea buen omne . & cum virtus faciat habentem bonum ; \textbf{ et opus bonum , | oportet in recto regimine , } quod bonus ciuis sit bonus homo ; \\\hline
3.3.1 & por la qual cada vno sabe gouernar la casa e la conpaña . \textbf{ Conuiene que sea otra e departida de la sabiduria } por la qual cada vno sabe gouernar a ssi mismo . & per quam quis scit regere domum et familiam , \textbf{ oportet esse aliam a prudentia , } qua quis nouit seipsum regere . \\\hline
3.3.1 & Et todas estas tres sabidurias \textbf{ conuiene que aya el Rey . } Conuiene a saber . & et gubernare ciues . \textbf{ Omnes autem tres prudentias decet habere Regem , } videlicet particularem , oeconomicam et regnatiuam . \\\hline
3.3.6 & si quieren ser buenos lidiadores \textbf{ conuiene que en su mançebia sean usados a saltar } por que puedan & si contingat eos pedestres esse , si volunt boni bellatores existere , \textbf{ sic ab ipsa iuuentute exercitandi sunt ad saliendum , } ut possint per saltum foueas , \\\hline
3.3.9 & ca los que estan en las huestes \textbf{ conuiene que sufran muchos males . } por la qual cosa si fueren y algunos muelles e mugerilles & erga necessitates corporis . \textbf{ Nam existentes in exercitu oportet multa incommoda tolerare : } quare si sint ibi aliqui molles , \\\hline
3.3.10 & por que se non desordenasse la hueste . \textbf{ Otrossi conuiene que en la hueste } establesçiessen cabdiellos & ne confunderetur exercitus . \textbf{ Rursus constituere expediebat duces , centuriones , decanos , } et alios praepositos belli . \\\hline
3.3.23 & Ca assi commo en la batalla de la tierra . \textbf{ conuiene que los lidiadores sean bien armados } e avn que se sepan bien cobrir e guardar de los colpes & Nam sicut terrestri pugna oportet \textbf{ pugnantes bene armatos esse , } et bene se scire \\\hline
3.3.23 & en la batalla de la naue . \textbf{ Ante conuiene que en esta batalla de la } naue sean los omnes . & sic et haec requiruntur in bello nauali . \textbf{ Immo in huiusmodi pugna oportet } homines melius esse armatos , quam in terrestri : \\\hline
3.3.23 & et non los dexan foyr . \textbf{ Lo . viij̇° . es de tomar esta cautela en la batalla dela naue que conuiene que se fincan muchas cantaras de cal poluorizada . } Et quando fuere la batalla & ut non permittant eos discedere . \textbf{ Octauo in nauali bello est haec cautela attendenda : | ut de calce alba puluerizata habeant multa vasa plena , } quae ex alto sunt proiicienda in naues hostium , \\\hline

\end{tabular}
