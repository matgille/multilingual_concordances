\begin{tabular}{|p{1cm}|p{6.5cm}|p{6.5cm}|}

\hline
1.1.7 & e si robare el pueblo \textbf{ e desfiziere la comunidat solamente } que el pueda allegar riquezas e dineros . & si opprimat viduas , et pupillos , \textbf{ si depraedetur populum et Rem publicam , } dum tamen possit pecuniam congregare , \\\hline
1.2.10 & e en los partires . \textbf{ Ca alas vezes algunos trabaian mas por la comunidat } e resçiben menos ante esta iustiçia & Rursus contingit inaequalitas in distributionibus , \textbf{ quia aliquando aliqui plus laborantes pro Republica , | minus accipiunt : } immo haec Iustitia quodammodo deprauata est , \\\hline
1.2.11 & La segunda manera para prouar esto mismo se toma de parte del regno . \textbf{ Ca el regno e toda comunidat es vna orden e vn prinçipado . } ¶ pues que assi es como la orden & ex parte ipsius regni . \textbf{ Regnum enim et omnis politia est quidam ordo , | et quidam principatus . } Cum igitur ordo , \\\hline
1.2.11 & Et pues que assi es non se guardaria dende adelante \textbf{ en ellos comunidat } ni seria dende adelante regno . & nec ad Principem . \textbf{ Non ergo ulterius reseruaretur in eis politia , } nec esset ulterius regnum . \\\hline
1.2.11 & Ca cada vno de los regnos \textbf{ e canda vna comunidat semeia avn cuerpo natural . } Ca assi commo veemos & Quodlibet enim regnum , \textbf{ et quaelibet congregatio assimilatur cuidam corpori naturali . } Sicut enim videmus corpus animalis constare \\\hline
1.2.11 & assi cada vno dellos regnos \textbf{ e cada vna delas comunidades es conpuesta de de pattidas personas ayuntadas e ordenadas a vna cosa . } Et por ende pues nos conuiene de fablar & sic quodlibet regnum , \textbf{ et quaelibet congregatio constat | ex diuersis personis connexis , } et ordinatis ad unum aliquid . \\\hline
1.2.11 & Bien assi cada vna mengua de iustiçia \textbf{ non corronpe del todo el regno e la comunidat . } Enpero que por qual quier menguade iustiçia & sic non quaelibet Iniustitia \textbf{ corrumpit totaliter regnum , et politiam , } tamen per quamlibet Iniustitiam regnum , \\\hline
1.2.11 & es enfermo el regno \textbf{ e la comunidat apareiada acorruy conn¶ } pues que assi es paresçe & et politia infirmatur , \textbf{ et disponitur ad corruptionem . } Patet igitur quod prout ciues habent ordinem ad se inuicem , \\\hline
1.2.27 & et por zelo de iustiçia \textbf{ o por amor de la comunidat } por que sin ella la comunidat non podrie durar . & et zelum iustitiae , \textbf{ vel propter amorem Reipublicae } quia sine ea Respublica durare non posset . \\\hline
1.2.27 & o por amor de la comunidat \textbf{ por que sin ella la comunidat non podrie durar . } por la qual cosa si el Rey o el prinçipe & vel propter amorem Reipublicae \textbf{ quia sine ea Respublica durare non posset . } Quare si quis in tantum esset mitis , \\\hline
1.2.27 & quanto mas pertenesçe a ellos de seer guardadores dela iustiçia \textbf{ et mantenedores dela comunidat . } Por la qual cosa si alos Reyes non conuiene de seer sannudos & esse custodes iustitiae , \textbf{ et conseruatores Reipublicae . } Quare si Reges non debent esse iracundi , \\\hline
1.3.3 & sienpredeuemos ante poner el bien comun \textbf{ e dela comunidat al bien propio e personal de cada vno . } Ca nos natraalmente veemos & includitur bonum priuatum , \textbf{ semper bono priuato praeponendum est commune bonum . } Naturaliter enim videmus \\\hline
2.1.2 & que sea comunidat çiuil de casa mas paresce \textbf{ que sea comunidat çiuil e de çibdat . } Ca segunt dize el philosofo & non videtur esse communitas domestica , \textbf{ sed ciuilis : } quia secundum Philosophum 1 Politicorum , \\\hline
2.1.2 & Comuidat dela casa \textbf{ Et comunidat de uarrio . } Et comunidat de çibdat . & ø \\\hline
2.1.2 & Et comunidat de uarrio . \textbf{ Et comunidat de çibdat . } Et comunidat de regno . & apparebit quadruplicem esse communitatem ; \textbf{ videlicet , domus , vici , ciuitatis , et regni . } Nam sicut ex pluribus personis fit domus , \\\hline
2.1.2 & Et comunidat de çibdat . \textbf{ Et comunidat de regno . } Ca assi commo de muchas perssonas se faz la comunidat dela casa & apparebit quadruplicem esse communitatem ; \textbf{ videlicet , domus , vici , ciuitatis , et regni . } Nam sicut ex pluribus personis fit domus , \\\hline
2.1.2 & Et comunidat de regno . \textbf{ Ca assi commo de muchas perssonas se faz la comunidat dela casa } assi de muchas casas se faz la comunidat de vn uarrio & videlicet , domus , vici , ciuitatis , et regni . \textbf{ Nam sicut ex pluribus personis fit domus , } sic ex multis domibus fit vicus , \\\hline
2.1.2 & Ca assi commo de muchas perssonas se faz la comunidat dela casa \textbf{ assi de muchas casas se faz la comunidat de vn uarrio } e de muchos uarrios se faz comunidat de çibdat & Nam sicut ex pluribus personis fit domus , \textbf{ sic ex multis domibus fit vicus , } et ex multis vicis ciuitas , \\\hline
2.1.2 & assi de muchas casas se faz la comunidat de vn uarrio \textbf{ e de muchos uarrios se faz comunidat de çibdat } e de muchas çibdades se faz comunidat de vn regno & sic ex multis domibus fit vicus , \textbf{ et ex multis vicis ciuitas , } et ex multis ciuitatibus regnum ; \\\hline
2.1.2 & e de muchos uarrios se faz comunidat de çibdat \textbf{ e de muchas çibdades se faz comunidat de vn regno } por la qual cosa & et ex multis vicis ciuitas , \textbf{ et ex multis ciuitatibus regnum ; } quare sicut singulares personae sunt partes domus , \\\hline
2.1.2 & sea alas trͣs comunidades \textbf{ en tal manera que todas las otras comunidades ençierren } e ante ponen en ssi la comuidat dela casa & ad communitates alias : \textbf{ quia omnes aliae ipsam praesupponunt : } et ipsa est quodammodo pars omnium aliarum . \\\hline
2.1.2 & e ante ponen en ssi la comuidat dela casa \textbf{ por que ella es en alguna manera parte de todas las otras comunidades . } Ca assi commo paresçe por el philosofo & quia omnes aliae ipsam praesupponunt : \textbf{ et ipsa est quodammodo pars omnium aliarum . | Naturalis enim origo ciuitatis } ut patet per Philosophum 1 Politicorum , \\\hline
2.1.3 & e la çibdat \textbf{ ante ponen la comunidat dela casa } assi el gouernamiento del regno & quia sicut regnum vel ciuitas praesupponunt esse domum , \textbf{ sic regimen regni et ciuitatis } praesupponit \\\hline
2.1.3 & ca es comunidat en alguna manera natural \textbf{ Et en algunan manera esta comunidat se ha al regno } e ala çibdat & et ut cognoscant quae et qualis est communitas domus \textbf{ ut se habet ad regnum et ciuitatem , } ut est in praesenti capitulo declaratum : \\\hline
2.1.4 & politicasasse declara \textbf{ e difine la comunidat dela casa } diziendo & ø \\\hline
2.1.4 & non solamente es menester la comunidat dela casa \textbf{ mas ahun la comunidat del uarrio e dela çibdat e del regno } mas si por otras razones & expediens communitas domus , \textbf{ sed et vici , ciuitatis , et regni . } Utrum autem propter alias causas , \\\hline
2.1.5 & por que sin ellas non puede ser la cosa conueniblemente . \textbf{ Mas que la comunidat del uaron } e dela muger sea & quia sine eis domus congrue esse non valet . \textbf{ Quod autem communicatio viri et uxoris sit propter generationem , } videre non habet dubium : \\\hline
2.1.5 & que assi conmo para el establesçemiento dela casa \textbf{ es men ester la comunidat del uaron } e dela mugni & quia sicut ad constitutionem domus \textbf{ requiritur } communitas viri et uxoris propter generationem , \\\hline
2.1.5 & Visto en qual manera \textbf{ alo menos estas dos comunidades } son menester para la casa . & Viso , quomodo saltem hae duae communitates requiruntur ad domum , \textbf{ quia secundum Philosophum ex eis constare } dicitur domus prima : \\\hline
2.1.6 & e dela muger que es para la generaçion \textbf{ e la comunidat del sennor e del sieruo } que es para la saluaçion & quae est propter generationem ; \textbf{ et domini et serui , } quae est propter saluationem , \\\hline
2.2.18 & Mas avn es neçessario para el bien \textbf{ e para el defendimiento dela comunidat } por ende alos que quieren beuir & non solum aliquando est licitus , \textbf{ sed etiam necessarius pro bono Reipublicae , } volentibus politice viuere , \\\hline
2.2.20 & quales si quier obras \textbf{ nin c̃ca los gouernamientos dela comunidat nin dela çibdat } Si la uoluntad del omne & et non vacant ciuilibus operibus , \textbf{ nec regiminibus reipublicae ; } si mens humana \\\hline
2.3.9 & llgua manera abastasse la mudaçion delas cosas alas cosas \textbf{ enpero ala comunidat } que hades es en todo el regno & commutatio rerum ad res : \textbf{ tamen ad communicationem } quod habetur in toto regno , \\\hline
3.1.1 & La segunda se toma de parte dela çibdat establesçida \textbf{ por conparaçion alas otras comunidades ¶ } La primera razon se declara & ex parte hominum constituentium ciuitatem . \textbf{ Secunda ex parte ciuitatis constitutae . } Prima via sic patet . \\\hline
3.1.1 & Ca commo quier que toda comun dar natural sea ordenada a bien \textbf{ enpero mayormente es ordenada aquel bien la comunidat } que es mas prinçipal & Nam licet omnis communitas naturalis ordinetur ad bonum , \textbf{ maxime tamen ordinatur } ad ipsum communitas principalissima : \\\hline
3.1.4 & que la comunidat dela casa \textbf{ e la comunidat dela çibdat sean cosas naturales } ca si la natura dio al omne palabra natural aquella comunidat & oportet communitatem domesticam \textbf{ et ciuilem esse quid naturale . } Nam si natura dedit homini sermonem , \\\hline
3.1.9 & commo cuydaua socrat̃s \textbf{ ca muchͣ comunidat mas aduze de amor } si fuere cierta e conosçida & ut opinabatur Socrates : \textbf{ nam modica coniunctio plus inducit de amore , } si sit certa et nota , \\\hline
3.1.10 & tire la cercidunbre de los fijos e el conosçimiento del parentesco \textbf{ non es de ella bartal comunidat } ca della se sigue & et notitiam consanguineitatis , \textbf{ non est commendanda : } quia ex hoc consequitur aliquos \\\hline
3.1.10 & e dende se sigue \textbf{ que puesta tal comunidat commo ordeno soctateᷤ } non se puede auer cuydado conuenible & sequitur quod supposita communitate , \textbf{ quam ordinauerat Socrates , } non habeatur debita cura , \\\hline
3.1.10 & por ende much era de reprehender la \textbf{ opimon de socrates dela comunidat } que puso delas mugeres e de los fijos . & et amor libidinosus , \textbf{ reprehensibilis erat opinio Socratis } de communitate uxorum et filiorum . \\\hline
3.2.17 & que se dixieren en los consseios . \textbf{ ca esto fue lo que enssalço la comunidat de Roma } fieldat de buenos consseieros & quae ibi sunt tradita . \textbf{ Hoc enim fuit , } quod apud Romanam Rempublicam exaltauit fidelitas consiliantium : \\\hline
3.2.17 & e muy alto era conssisto \textbf{ no secreto dela comunidat de roma } alos quel guardananca era guaruido de grant fialdat . & ait , quod fidum et altum erat \textbf{ secretum consistorium reipublicae , } silentique salubritate munitum : \\\hline
3.2.19 & tal que sea bueno e uirtuoso \textbf{ e ame la comunidat } et ꝓmueua los q̃ son eñl su regno e los hõ rre . & quod esset bonus virtuosus \textbf{ et politiam diligeret , } existentes in regno promoueret et honoraret : \\\hline
3.2.26 & en el quarto libro delas politicas \textbf{ que non conuiene de apropar las comunidades } delas çibdades alas leyes . & Ideo dicitur 4 Politicorum \textbf{ quod non oportet } adaptare politias legibus , \\\hline
3.2.26 & delas çibdades alas leyes . \textbf{ Mas las leyes alas comunidades } de las çibdades las quales leyes conuiene de ser departidas & adaptare politias legibus , \textbf{ sed leges politiae , } quas leges oportet diuersas esse \\\hline
3.2.26 & de las çibdades las quales leyes conuiene de ser departidas \textbf{ segunt el departimiento delas comunidades . } Et pues que assi es el que quisiere poner leyes & quas leges oportet diuersas esse \textbf{ secundum diuersitatem politiarum . } Volens ergo leges ferre , \\\hline
3.2.32 & que cosa es la çibdat podemos responder \textbf{ e dez que es comunidat de çibdadanos } por beuir bien e uirtuosamente & Dici debet \textbf{ quod est communicatio ciuium propter bene , } et virtuose viuere ; \\\hline

\end{tabular}
