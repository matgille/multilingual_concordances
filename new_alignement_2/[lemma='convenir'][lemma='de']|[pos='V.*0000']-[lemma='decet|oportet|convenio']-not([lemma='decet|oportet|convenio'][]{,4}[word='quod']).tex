\begin{tabular}{|p{1cm}|p{6.5cm}|p{6.5cm}|}

\hline
1.1.2 & el que quiere tractar del gouernaiento \textbf{ e fablarde sy mesmo conujene de tractar } e de dar conosçimiento de todas aquellas cosas & volens tractare de regimine sui , \textbf{ oportet ipsum notitiam tradere de omnibus his } quae diuersificant mores et actiones . \\\hline
1.1.3 & njn guardar \textbf{ syn la gera de dios conviene de cada vn omne } e mayormente prinçipe o Rey & absque diuina gratia obseruari non possunt , \textbf{ decet quemlibet hominem , } et maxime regiam maiestatem \\\hline
1.1.9 & e muestre algua bondat de fuera \textbf{ por la qual cosa commo al Rey conuenga ser todo diuinal e semeiante a dios } si non es cosa conuenible & quod exterius bona praetendat . \textbf{ Quare cum Regem deceat } esse totum diuinum , \\\hline
1.2.1 & e la su bien andança . \textbf{ Et que non los conuiene poner la su fin en riquezas } nin en poderio çiuil & suam felicitatem debeant ponere , \textbf{ quia non decet | eos suum finem ponere in diuitiis , } nec in ciuili potentia , \\\hline
1.2.5 & Ca commo contesca de razonar derechamente \textbf{ e non derechamente conuiene de dar alguna uirtud } que sea razon derecha . & ratiocinari recte et non recte , \textbf{ oportet dare virtutem aliquam , } quae sit recta ratio , \\\hline
1.2.5 & assi commo son las passiones dela saña . \textbf{ Conuiene dar alguna uirtud en las passiones } por la qual las passiones non nos pueden mouer & ut passiones irascibiles : \textbf{ circa passiones oportet | dare virtutem aliquam , } ne passiones nos impellant \\\hline
1.2.5 & nin inclinar a aquelo que uieda la razon ¶ \textbf{ Et otrosi nos conuiene de dar otra uirtud } por la qual las passiones non nos pueden arredrar & ad id quod ratio vetat : \textbf{ et oportet dare virtutem aliam , } ne passiones retrahant nos ab eo , \\\hline
1.2.8 & aque ha de guiar el Rey su pueblo \textbf{ le conuiene de ser acordable e prouisor . } Assi por razon de la manera & ad quae dirigit , \textbf{ oportet Regem esse memorem , | et prouidum : } sic ratione modi per quem dirigit , \\\hline
1.2.8 & por la qual ha de guiar el pueblo \textbf{ le conuiene de ser entendido e razonable . } Mas por razon dela su persona propia & sic ratione modi per quem dirigit , \textbf{ oportet ipsum esse intelligentem , et rationalem . } Sed ratione propriae personae \\\hline
1.2.10 & assi conmosi quisiere auer mas de aquellos bienes \textbf{ de quanto le conuiene auer } por esta razon viene danno alos otros çibdadanos & ut quod velit habere plus de iis , \textbf{ quam eum deceat : } ex hoc infertur nocumentum aliis ciuibus : \\\hline
1.2.13 & e pecar en obrando . \textbf{ Conuiene de dar e de ponetur algua uirtud } por la qual seamos reglados en las obras & et bene agere , \textbf{ oportet dare virtutem aliquam , } per quam regulentur in agendo . \\\hline
1.2.17 & escontra regla derecha de razon e de entendimiento . \textbf{ Conuiene de dar alguna uirtud medianera } entre la auariçia e el gastamiento . & contra rectam regulam rationis , \textbf{ oportet dare virtutem aliquam mediam } inter auaritiam , et prodigalitatem : \\\hline
1.2.18 & que menos dan de quanto les conuiene ᷤ dar \textbf{ Et menos fazen de quanto les conuiene de fazer . } Et desto puede bien paresçer & Semper ergo cogitare debent , \textbf{ quod minora faciunt , | quam deceat . } Ex hoc autem apparere potest \\\hline
1.2.18 & los que son en el su regno \textbf{ mucho les conuiene de ser liberales e francos } Mas par tenesce al libal e alstan ço de catar tres cosas ¶ & qui sunt in Regno , \textbf{ maxime decet eos liberales esse . } Spectat autem ad liberalem primo \\\hline
1.2.18 & que si espendiere \textbf{ do non le conuiene espender } Mas los Reyes e los prinçipes de suranse & ubi oportet , \textbf{ quam si expendat | ubi non oportet . } Deuiant autem a liberalitate Reges , \\\hline
1.2.18 & e a otros semeiantes \textbf{ a quien non conuiene de dar . } por que aquestos tales mas les conuiene de ser pobres & vel aliis , \textbf{ quibus non oportet dare : } quia magis deceret \\\hline
1.2.18 & a quien non conuiene de dar . \textbf{ por que aquestos tales mas les conuiene de ser pobres } que non ser ricos . & quibus non oportet dare : \textbf{ quia magis deceret | eos esse pauperes , } quam diuites . \\\hline
1.2.24 & e alos prinçipes de seer magnificos e liberales \textbf{ assi en essa misma manera les conuiene de seer magranimos } e amado res de honrra . & et Principes esse magnificos , et liberales : \textbf{ sic decet eos esse magnanimos , } et honoris amatiuos . Reges enim et Principes decet honores diligere modo quo dictum est ; \\\hline
1.2.25 & e nos allega a aquello que la razon manda o uieda . \textbf{ Conuiene de dar en aquella cosa dos uirtudes ¶ La vna que nos allegue . } Et la otra qua nos arriedre dello . & si unum et idem aliter et aliter acceptum nos retrahit et impellit , \textbf{ oportebit circa illud dare duas virtutes , | unam impellentem , } et aliam retrahentem . \\\hline
1.2.27 & e uenganças del contesçe de sobrepiuar e de fallesçer . \textbf{ Conuiene de dar y alguna uirtud } que reprima las sobrepuianças & contingit superabundare et deficere : \textbf{ oportet ibi dare virtutem } aliquam reprimentem superabundantias , \\\hline
1.2.28 & cerca la qual contesçe de sobrepuiar e de fallesçer . \textbf{ Conuiene de dar uirtud alguna } que reprima las sobrepuianças & circa quam contingit abundare et deficere , \textbf{ oportet dare uirtutem } aliquam reprimentem superabundantias , \\\hline
1.2.29 & que quiere dezir escarnidores e despreçiadores dessi mismos . \textbf{ Et pues que assi es conuiene de dar alguna uirtud medianera } por la qual sean tenpradas las cosas menguadas & idest irrisores , et despectores . \textbf{ Oportet ergo dare aliquam virtutem mediam , } per quam moderentur diminuta , \\\hline
1.2.31 & e non conplidas \textbf{ non conuiene de ser conexas } nin ayuntadas vna a otra & Virtutes autem naturales et imperfectae , \textbf{ non oportet esse connexas . } Videmus enim aliquos naturaliter habere \\\hline
1.3.1 & e los prinçipes poner su fin e su bien andança . \textbf{ Et otrosi mostrado es en commo les conuiene de ser uirtuosos } ¶ Agora finca de dezir dela tercera parte deste libro & in quo Reges et Principes suum finem ponere debeant , \textbf{ et quomodo oportet | eos virtuosos esse . } Restat exequi de tertia parte huius primi libri , \\\hline
1.3.5 & e alos prinçipes de entender en el bien \textbf{ Mas avn les conuiene de entender } en bien alto e grande e guaue de fazer De mas desto & et Principes tendere in bonum , \textbf{ sed etiam decet eos tendere in bonum arduum . } Amplius quanto maior est communitas , \\\hline
1.3.6 & por ende en este primero libro \textbf{ conuiene de tractar delas costunbres de lons Reyes } uniuersalmente & ut dicitur 1 Physicorum , \textbf{ deo in hoc primo de moribus Regum oportet } pertransire uniuersaliter typo : \\\hline
1.3.6 & en el primero libͤ de los grandes morales . \textbf{ Et pues que assi es conuiene deuer } en qual manera conuiene alos Reyes de sertemosos & non est fortis , sed satuus . \textbf{ Oportet ergo videre } quo modo eos esse deceat timidos , et audaces . \\\hline
1.4.4 & e confonden el entendemiento . \textbf{ Otrosy les conuiene de ser misconiosos non por fallesçimiento nin por llaqueza de } coraçon quales en los vieios . & rationem percutiunt . \textbf{ Decet etiam eos esse miseratiuos , | non propter defectum , } vel propter imbecillitatem : \\\hline
1.4.5 & Onde el philosofo dize en el quarto libro de la rectorica \textbf{ que conuiene de ser los nobles magranimos } e de grandes coraçones e magnificos & Unde Philos’ 4 Eth’ ait , \textbf{ quod magnanimos et magnificos decet } esse nobiles et gloriosos . \\\hline
1.4.5 & e escodrinnadores sotilmente de todo aquello \textbf{ que les conuiene de fazer } por que las sus obras & subtiliter inuestigantes \textbf{ quid decet eos facere , } ne opera eorum , \\\hline
1.4.7 & Mas los poderosos et los prinçipes \textbf{ por que les conuiene de entender } e auer cuydados de muchͣs cosas retrahen se & Potentes vero et principantes , \textbf{ quia oportet eos intendere exterioribus curis , } retrahuntur , \\\hline
2.1.4 & non cunplie la comunidat de vna casa \textbf{ mas conuiene de dar comunidat de varrio . } Por que commo el uarrio sea fech̃ de muchas casas & non sufficiebat communitas domestica , \textbf{ sed oportuit dare communitatem vici , } ita quod cum vicus constet \\\hline
2.1.4 & todas las cosas neçessarias ala uida \textbf{ conuiene de dar comunidat ala çibdat } sobre la comunidat deluarrio . & omnia necessaria ad vitam , \textbf{ praeter communitatem vici | oportuit } dare communitatem ciuitatis . \\\hline
2.1.4 & non solamente la casa es vna comiundat \textbf{ mas en la casa conuiene de dar muchͣs comunidades } la qual cosa non puede ser sin muchͣs perssonas . & non solum domus est communitas quaedam , \textbf{ sed in domo oportet | dare plures communitates : } quod sine pluralitate personarum \\\hline
2.1.7 & en la comunidat dela casa \textbf{ primeramente conuiene de ayuntar el uaron con la mugni } e esta orden es muy con razon . & in communitate domestica , \textbf{ primum oportet | congregare marem , et foeminam . } Est autem hic ordo rationabilis . \\\hline
2.1.10 & por que en el casamiento \textbf{ dellos conuiene de guardar la orden natural mas que en otro ninguno . } ¶ Lo segundo esso mismo pue de ser mostrada & coniuges Regum et Principum , \textbf{ quia in eorum coniugio magis quam in alio decet | naturalem ordinem conseruare . } Secundo hoc idem inuestigari potest \\\hline
2.1.16 & e en qual manera de una vsar del . \textbf{ Et pues que assi es conuiene de desçender } mas en particular & et quomodo utendum sit eo . \textbf{ Oportet ergo magis in particulari descendere , } qualiter omnes ciues \\\hline
2.1.19 & e then a sus maridos a mayor amor . \textbf{ Et por ende les conuiene de ser calladas } e en essa misma manera avn les conuiene de ser estables e firmes & et ad maiorem amorem viros inducunt : \textbf{ decet ergo eas esse taciturnas . } Sic etiam decet esse stabiles : \\\hline
2.1.19 & Et por ende les conuiene de ser calladas \textbf{ e en essa misma manera avn les conuiene de ser estables e firmes } que quanto la mug̃res mas firme e mas estable & decet ergo eas esse taciturnas . \textbf{ Sic etiam decet esse stabiles : } quia quanto uxor est magis constans , \\\hline
2.2.2 & e en algun sennorio \textbf{ en quel conuiene gouernar los otros } mucho les conuiene de ser sabios e buenos . & et in aliquo dominio , \textbf{ in quo oportet eos alios gubernare ; } maxime decet eos esse prudentes et bonos . \\\hline
2.2.2 & en quel conuiene gouernar los otros \textbf{ mucho les conuiene de ser sabios e buenos . } Mas commo los fijos bengan a mayor bondat e a mayor sabiduria & in quo oportet eos alios gubernare ; \textbf{ maxime decet eos esse prudentes et bonos . } Et cum filii perueniunt \\\hline
2.2.7 & e en las sçiençias liberales \textbf{ quanto mas les conuiene de ser mas entendudos } e mas sabios que los otros & insudare literalibus disciplinis , \textbf{ quanto decet eos intelligentiores et prudentiores esse , } ut possint naturaliter dominari . \\\hline
2.2.8 & i anifestamos nr̃a uoluntad e nr̃a entençion . \textbf{ Et por ende conuiene de fallar algua sçiençia } que nos mostrasse & et per debitas rationes manifestemus propositum . \textbf{ Oportuit ergo inuenire aliquam scientiam docentem modum , } quo formanda sunt argumenta , et rationes . \\\hline
2.2.10 & ¶ La primera quanto alas cosas uisibles \textbf{ que assi commo non les conuiene de fablar cosas torpes } Et la razon desto pone el philosofo en łvij̊ libro delas ethicas & Quantum ad visibilia quidem , \textbf{ quia sicut non decet | eos turpia sequi : } sic indecens est eos turpia videre . \\\hline
2.2.12 & non solamente que se non fagan gollosos por el comͣ \textbf{ mas avn les conuiene de ser mesurados } que non se fagan beodos & ex sumptione cibi : \textbf{ sed etiam decet eos esse sobrios , } ut non efficiantur ebrii \\\hline
2.2.18 & nin por otra uentra a qual si quier non osen tomar armas . \textbf{ Enpero por que mas conuiene de ser sabios } que lidiadores alos fijos de los Reyes & nec pro alio casu audeant arma assumere ; \textbf{ attamen quia decet | eos esse magis prudentes quam bellatores , } filii Regum et Principum \\\hline
2.2.20 & si non nos delectaremos en alguas cosas \textbf{ conuiene de tomar alguas obras conuenibles e honestas } çerca las quales entendamos & nisi in aliquibus delectemur : \textbf{ decet nos assumere | aliqua opera licita et honesta , } circa quae vacantes , \\\hline
2.2.20 & Mas si alguno demandare \textbf{ de que se deuen trabaiar las mugers conuiene de fablar en tales cosas departidamente } segunt el departimiento delas perssonas & Si autem quaeratur \textbf{ circa qualia opera solicitari debent : | oportet in talibus differenter loqui } secundum diuersitatem personarum . \\\hline
2.2.21 & ostrado que non conuiene alas moças de andar uagarosas a quande e allende \textbf{ nin les conuiene de beuir ociosas } finca que agora lo terçero mostremos & quod non decet puellas esse vagabundas , \textbf{ nec decet eas viuere otiose : } restat ut nunc tertio ostendamus , \\\hline
2.3.3 & que anings de los otros nobles \textbf{ ca ellos conuiene de ser nobles prinçipalmente } e magnificos en todas sus cosas . & qui debent esse nobiles et praeclari , \textbf{ potissime decet esse magnificos . } Alii enim moderatas possessiones habentes , \\\hline
2.3.9 & que hades es en todo el regno \textbf{ conuiene de poner m̃udaçion delas cosas alos dineros } e de los diueros alas cosas & quod habetur in toto regno , \textbf{ oportuit introduci commutationem rerum ad denarios , } et econuerso . \\\hline
2.3.9 & e de departidas prouinçias \textbf{ conuiene de poner non sola mente mudaçion delas cosas alas cosas } o delas cosas alos dineros & et prouinciarum , \textbf{ oportuit introduci | non solum commutationem rerum ad res , } vel rerum ad denarios ; \\\hline
2.3.9 & non las poderemos leuar conueniblemente a luengas tierras . \textbf{ Et pues que assi es conuiene de fablar alguna cosa } que se podiesse leuar & commode ad partes longinquas portari non possunt . \textbf{ Oportuit ergo inuenire aliquid } quod esset portabile , \\\hline
2.3.15 & tenporal esto deue ser despues de aquel bien que entiende . \textbf{ Mas conuiene de dar a ministraçion de alquiler e de amor sin la ministt̃ion natural et segunt ley . } Ca por que en nos es el appetito corrupto & hoc debet esse ex consequenti . \textbf{ Oportuit autem dare ministrationem conductam et dilectiuam | praeter ministrationem naturalem } et secundum legem : \\\hline
2.3.16 & en el gouernamiento delas casas de los Reyes \textbf{ En las quales por la grandeza de los offiçios conuiene de } acomne dar vn ofiçio a muchos seruientes & in gubernatione domorum regalium , \textbf{ ubi propter magnitudinem officiorum oportet } idem ministerium committi ministris multis , \\\hline
2.3.18 & e de los no nobles omes alos \textbf{ que les conuienne de ser dadores } e cobidadores & sed quia volunt retinere mores curiae et nobilium , \textbf{ quos decet datiuos esse ; } propter quod tales curiales dici debent . \\\hline
2.3.19 & Lo quinto \textbf{ e lo postrimero conuiene de saber } en qual manera los señores les han de fazer bien & et aperienda sunt secreta . \textbf{ Quinto et ultimo oportet cognoscere , } qualiter sunt beneficiandi , \\\hline
2.3.19 & e los prinçipes \textbf{ alos quales conuiene de ser magn animos } deuen se mostrar & Reges ergo et Principes , \textbf{ quos decet esse magnanimos } ad proprios ministros , \\\hline
2.3.20 & Mas podemos mostrar por dos razones \textbf{ que non conuiene de fablar mucho en las mesas de los Reyes } nin de los prinçipes & Possumus autem duplici via ostendere , \textbf{ quod non decet | in mensis Regum et Principum } et uniuersaliter omnium nobilium \\\hline
2.3.20 & Et pues que assi es los Reyes \textbf{ e los prinçipeᷤ alos quales conuiene ser muy tenprados } e guardar la orden natural en toda & Reges ergo et Principes , \textbf{ quos decet maxime temperatos esse , } et obseruare ordinem naturalem \\\hline
3.1.8 & quales si acaesçiere logar nos podremos dellas fazer mençion . \textbf{ on conuiene de demandar en todas las cosas } grant egualdat cosas fuessen & Maximam unitatem et aequalitatem \textbf{ non oportet | quaerere in omnibus rebus . } Nam si omnia essent aequalia , \\\hline
3.1.8 & nin en vna semeiança \textbf{ conuiene de dar } y deꝑ tidas espeçies & reseruari in una specie , \textbf{ oportet ibi dare species diuersas ; } ut in pluribus speciebus entium reseruetur maior perfectio , \\\hline
3.1.8 & para que aya ser acabada \textbf{ conuiene de dar ay algun departimiento } nin conuiene de ser & esse perfectum , \textbf{ oportet dare diuersitatem aliquam , nec oportet ibi esse } omnimodam conformitatem et aequalitatem , \\\hline
3.1.8 & conuiene de dar ay algun departimiento \textbf{ nin conuiene de ser } y en toda manera confirmada egualdat & esse perfectum , \textbf{ oportet dare diuersitatem aliquam , nec oportet ibi esse } omnimodam conformitatem et aequalitatem , \\\hline
3.1.8 & assi commo de andar e de tanner e de oyr e deuer . \textbf{ por ende conuiene de dar . } y departidos mienbros & ut ambulatione , tactu , visione , \textbf{ et auditus ideo oportet } ibi dare diuersa membra exercentia diuersos actus : \\\hline
3.1.8 & auemos mester casas e uestid̃as e viandas e otras cosas tales \textbf{ por ende conuiene de dar algun departimiento en la çibdat por que en ella sean falladas todas las cosas } que cunplen ala uida . & et aliis huiusmodi ; \textbf{ oportet in ciuitate | dare diuersitatem aliqua , } ut in ea reperiatur sufficientia ad vitam . \\\hline
3.1.8 & delos çibdadanos a algun prinçipe o algun sennor \textbf{ e commo en la çibdat conuenga de dar alguons ofiçioso } alguons maestradgos o algunas alcaldias & ad aliquem principantem vel dominantem , \textbf{ ut cum in ciuitate oporteat | dare aliquos magistratus , } et aliquas praeposituras , \\\hline
3.1.8 & por ende commo estas cosas demanden departimiento \textbf{ conuiene de dar en la çibdat algun departimiento . } La quanta razon se toma & Quare cum hoc diuersitatem requirat , \textbf{ oportet in ciuitate | dare diuersitatem aliquam . } Quinta uia sumitur \\\hline
3.1.8 & si non sopiere en qual manera es establesçida la çibdat \textbf{ e si non sopiere en qual manera conuiene de auer en ella departimiento de ofiçios e de ofiçiales } l sermon en los comienços deueser luengo & nisi sciuerit qualiter constituitur ; \textbf{ et nisi cognoscat | quod oportet in ea diuersitatem esse . } Sermo in principiis debet esse longus , \\\hline
3.1.9 & e de escodrinnar \textbf{ en qual manera la çibdat conuiene de ser vna } e qual departimiento deue auer enlła & diu inuestigandum est , \textbf{ qualiter ciuitatem oportet esse unam , } et quam diuersitatem habere debet , \\\hline
3.1.11 & e nos enssannamos contra ellos muchͣs uezes \textbf{ por que nos conuiene de fablar muchͣs uezes con ellos } e de beuir conellos & et indignamur erga illos , \textbf{ quia oportet nos habere | ad illos multa colloquia , } et diu conuersari cum illis . \\\hline
3.1.12 & Et por ende por que los lidiadores non se enflaquezcan en las batallas \textbf{ conuiene de echar dela batalla } e dela fazienda alos de flaco & ne igitur reddantur bellantes pusillanimes , \textbf{ quos constat esse timidos oportet } ab exercitu expelli . \\\hline
3.1.14 & que sienpre los çibdadanos \textbf{ non les conuenga de lidiar } por defendimiento de su tierra & ab artificibus et ab aliis ciuibus , \textbf{ quod ciues alii pro defensione patriae bellare non oporteat } melius est ergo dicere in ciuitate \\\hline
3.1.17 & por que podrian los çibdadanos auer tan pocas possessiones \textbf{ que les conuenia de beuir } assi es casamente & possent enim ciues adeo modicas possessiones habere , \textbf{ quod oporteret eos ita parce viuere } quod opera liberalitatis de facili exercere non valerent . \\\hline
3.2.1 & entp̃o dela paz \textbf{ por las leyes conuiene de fazer tractado destas quatro cosas sobredichͣs en este gouernamiento ¶ } La segunda razon para prouar & Quare si considerentur quae requiruntur ad hoc quod tempore pacis per leges bene gubernetur ciuitas , \textbf{ oportet in huiusmodi regimine | de praedictis quatuor considerationem facere . } Secunda via ad inuestigandum hoc idem sumitur ex fine \\\hline
3.2.5 & linage donde ha de ser tomado el sennor . \textbf{ Mas avn conuiene de determinar la perssona . } Ca assi commo nasçen discordias & ex qua praeficiendus est dominus , \textbf{ sed etiam oportet determinare personam . } Nam sicut oriuntur dissentiones et lites , \\\hline
3.2.13 & segund que dize el philosofo \textbf{ ca conujene de dar a entender } que estos tales non han cuydado de saluar su vida ¶ & ( ut ait Philos’ ) \textbf{ sunt paucissimi numero , | supponi oportet } eos nihil curare , \\\hline
3.2.16 & Reyr \textbf{ quales cosas le conuiene de fazer } para que derechamente gouierne el pueblo qual es acomendado . & manifestauimus item quod sit Regis officium , \textbf{ et quae oporteat ipsum facere } ut recte regat populum sibi commissum : \\\hline
3.2.17 & por la quel cosa commo muchs mas cosas ayan prouadas \textbf{ que vno solo conuiene de llamar otros } para los negoçios . por que por el conseio dellos pueda ser escogida la meior carrera & Quare cum plures plura experti sint , \textbf{ quam unus solus : | decet ad huiusmodi negocia alios aduocare , } ut per eorum consilium possit \\\hline
3.2.17 & mas obramos en poco tienpo \textbf{ e luego e que conuiene de touiar conseio prolongadamente } mas conuiene de fazerl cosas conseiadas mucho ayna . & operamur autem prompte : \textbf{ et quod oportet consiliari tarde , } sed facere consiliata velociter . \\\hline
3.2.21 & La primera seqma par aquello que tales palabras han de to terçeres desegualar eliez \textbf{ el qual conuiene de ser } assi commo regla derecha en & obligare habent iudicem , \textbf{ quem esse oportet } quasi regulam in iudicando . \\\hline
3.2.22 & demos contar quatro cosas \textbf{ que conuiene de auer alos iuezes } para que den uerdaderos iuisios & Possumus autem quatuor enumerare , \textbf{ quae oportet habere iudices , } ut vera iudicia proferant , \\\hline
3.2.24 & por que non se pierda dela memoria de los omes \textbf{ conuiene de ser escerpto en algun libro . } Enpero cada vno destos dos derechs tan bien el natural commo el positiuo se puede escͥuir en algun libro & ne a memoria recederet , \textbf{ oportuit ipsum scribi | in aliqua exteriori substantia . } Potest itaque \\\hline
3.2.26 & en el quarto libro delas politicas \textbf{ que non conuiene de apropar las comunidades } delas çibdades alas leyes . & Ideo dicitur 4 Politicorum \textbf{ quod non oportet } adaptare politias legibus , \\\hline
3.2.26 & Mas las leyes alas comunidades \textbf{ de las çibdades las quales leyes conuiene de ser departidas } segunt el departimiento delas comunidades . & sed leges politiae , \textbf{ quas leges oportet diuersas esse } secundum diuersitatem politiarum . \\\hline
3.2.30 & assi commo paresçra adelante . \textbf{ Et por ende conuiene de dar ley diuinal } e e un agłical segunt la qual fuessen vedados los pecados todos . & ut in prosequendo patebit : \textbf{ oportuit igitur dare legem euangelicam et diuinam , } secundum quam prohiberentur \\\hline
3.2.36 & La terçera cosa para que los Reyes sean amados del pueblo \textbf{ es quales conuiene de ser derechureros e eguales . } Ca el pueblo mayormente se le una taria a mal querençia del Rey & ut Reges diligantur a populo , \textbf{ decet eos esse iustos , et aequales . } Nam maxime prouocatur populus ad odium Regis , \\\hline
3.3.2 & o en quales tierras son meiores lidiadores . \textbf{ Conuiene de tener mientes en estas dos cosas sobredichas . } Et pues que asy es en las partes & in quibus regionibus meliores sunt bellatores , \textbf{ oportet attendere circa praedicta duo . } In partibus igitur nimis propinquis soli , \\\hline
3.3.8 & e fazer muy apriessa . \textbf{ Mas conuiene de poner algunos maestros } para costruyr los castiellos & debet celeriter castra construere . \textbf{ Oportet autem semper construendis castris , } et faciendis fossis aliquos magistros praestitui , \\\hline
3.3.8 & e son de fazer mas anchas carcauas . \textbf{ mas solamente quieren y estar vna noche o por poco tienpo non conuiene de fazer tantas guarniçiones . } Mas la manera e la quantidat de las carcauas pone la vegeçio & aut ibi debet \textbf{ per modicum tempus existere , | non oportet tantas munitiones expetere . } Modum autem , \\\hline
3.3.10 & para guiar los lidiadores . \textbf{ Mas conuiene de dar otras seña les manifiestas . por que cada vno viendo aquellas señales } se sepa tener ordenadamente en su az & non sufficiunt ad dirigendum bellantes , \textbf{ sed oportet dare euidentia signa ; | ut quilibet solo intuitu sciat } se tenere ordinate in acie , \\\hline
3.3.13 & por que quanto aquellos aniellos mas son ayuntados . \textbf{ tanto conuiene de cortar mas dellos } para que los colpes enpeescan . & quia quanto illi annuli magis sunt compacti , \textbf{ tanto oportet plures ex eis frangere } ut vulnera noceant . \\\hline
3.3.13 & Mas en feriendo cortando . \textbf{ por que conuiene de fazer grand mouimiento de los braços } ante que se de el colpe el enemigo & In percutiendo autem caesim , \textbf{ quia oportet fieri magnum brachiorum motum prius quam infligatur plaga , } aduersarius ex longinquo potest prouidere vulnus , \\\hline
3.3.14 & en qual manera deuen lidiar . \textbf{ Por la qual cosa les conuiene de foyr . } Lo quarto el señor de la hueste se deue tenprar & qualiter debeant dimicare : \textbf{ propter quod oportebit eos fugam eligere . } Quarto dux exercitus sic se temperare debet : \\\hline
3.3.16 & Et avn algunas vezes contesçe que algunos otros çercan sus villas o sus castiellos . \textbf{ Por la qual cosa les conuiene de vsar de batalla defenssiua para se defender . } Otrossi contesçe que en el prinçipado & inuadere aliquas munitiones eorum ; \textbf{ propter quod eos oportet | uti pugna defensiua . } Amplius in principatu et regno contingit \\\hline
3.3.16 & Ca contesçe algunas vegadas que los cercados non han agua . \textbf{ e por ende o les conuiene de peresçer o de morir } de sedo de dar las fortalezas . & Contingit enim aliquando obsessos carere aqua : \textbf{ ideo vel oportet eos siti perire , } vel munitiones reddere . \\\hline
3.3.18 & fasta las menas del castiello o de la çibdat cercada . \textbf{ Conuiene de vsar de tales armadijas o de tales armamientos } por que puedan ganar el logar & vel ciuitatis obsessae , \textbf{ oportet talibus uti argumentis } ut habeatur intentum . \\\hline
3.3.22 & la qual tierra cauada \textbf{ conuiene de apoyar bien el castiello o la çerca } por que se non funda & qua suffossa , et castro demerso in ipsam propter magnitudinem ponderis , \textbf{ oportet castrum iterum construi , } eo quod non possit \\\hline
3.3.23 & de la batalla de las naues . \textbf{ enpero non conuiene de nos } de tener çerca esto tanto . & volumus aliqua de nauali bello : \textbf{ non tamen oportet } circa hoc tantum insistere , \\\hline

\end{tabular}
