\begin{tabular}{|p{1cm}|p{6.5cm}|p{6.5cm}|}

\hline
1.3.3 & Dilectatio enim quam habebant Romani ad Rempublicam fecit Romam esse principantem \textbf{ et monarcham . } Hoc ergo modo quoslibet homines decet esse amatiuos , & Ca el amor que auian los romanos al bien comun e publicofizo a Roma ser sennora \textbf{ e auer sennorio en todo el mundo . } Pues que assi es que esto conuiene a todos los omes \\\hline
3.2.14 & ut tyrannis populi contrariatur tyrannidi monarchiae : et una monarchia tyrannica contrariatur alii . \textbf{ Cum enim aliquis monarcha vel aliquis unus Princeps tyrannizet in populum , } gens illa oppressa non valens sustinere tyrannidem Principis , insurgit & escontrana ala tirama del mal prinçipado Et vn prinçipado tiranico es contrario a otro prinçipado tiranico e malo \textbf{ ca quando algun enparadoro algun prinçipe vno titaniza en el pueblo | aquella gente apremiada } non podie do sofrir su tira maleunatasse \\\hline
3.2.14 & et tyrannis populi corrumpit tyrannidem monarchiam . Sic etiam una tyrannis monarchia corrumpit aliam : \textbf{ quia multotiens unus monarcha tyrannus insurgit in alium , } ut obtineat principatum eius . Debent ergo cauere Reges et Principes & e destruye la tirana del prinçipe bien assi avn vna tirama de sennorio corronpe a otra \textbf{ por que muchͣs vezes vn prinçipe tirano se leunata contra otro prinçipe tirano } por que gane el su prinçipado . Por ende mucho deuen escusar los Reyes e los prinçipes \\\hline

\end{tabular}
