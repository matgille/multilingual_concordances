\begin{tabular}{|p{1cm}|p{6.5cm}|p{6.5cm}|}

\hline
1.1.1 & Immo quia ( secundum Philosophum in Politicis ) \textbf{ quae oportet dominum scire praecipere , } haec oportet subditum scire facere : & Ca Segund dize el philosopho enlas politicas \textbf{ que aquellas cosas | que conujene al Senonr } de saber mandar essas mesmas \\\hline
1.1.5 & non debetur eis corona . \textbf{ Oportet igitur actu bene agere , } ut per opera nostra mereamur & Enpero si non lidiaren de fech̃o non les es deuida corona ¶ \textbf{ pues que asi es conviene bien fazer de fecho } por que por las nr̃as obras merescamos de auer buena fino buena ventura \\\hline
1.1.6 & Dictum est enim \textbf{ quod decet Principem esse supra Hominem , } et totaliter diuinum . & Ca dicho n auemos ya desuso \textbf{ que conuiene al Rei | e al prinçipe sier } mas que omne e ser del todo diuinal . \\\hline
1.1.9 & honor tamen eos consequitur , \textbf{ et decet eos acceptare honorem sibi exhibitum , } non habentibus Hominibus aliquid maius , & enpero la honrra les parte nesçe a ellos . \textbf{ Et conuiene les alos Reys de resçebir la honrra | que les fazenn los omes } por que los omes non les pueden dar mayor cosa que honrra \\\hline
1.1.13 & cum semper amor sit ad similes , et conformes , \textbf{ oportet esse similem , } et conformem Deo , & Et commo el amor sienpre sean los semeiables e acordables con el . \textbf{ Conuiene que aquel que es para de ser } gualardonado de dios \\\hline
1.2.1 & Virtutes autem quaedam sunt quidam ornatus , \textbf{ et quaedam perfectiones animae . Oportet ergo prius ostendere , } quot sunt potentiae animae , & Ca las uirtudes son vnos hornamentos e conponimientos e hunas perfectiones \textbf{ que fazen acabada el alma . | Pues que assi es conuiene } mostrͣ primero quantos son los poderios del alma \\\hline
1.2.1 & et appetitus intellectiuus , et sensitiuus , \textbf{ oportet esse in talibus virtutes morales . } Quomodo distinguuntur virtutes : & que es la uoluntad Et el appetito sen setiuo \textbf{ que sigue al seso conuiene } por fuerça \\\hline
1.2.2 & virtutem esse aliquid secundum rationem : \textbf{ oportet ergo esse rationalem potentiam , } in qua potest esse virtus . & segunt razon \textbf{ e por ende conuiene | que sea poderio razonable } aquel en que esta la uirtud ¶ \\\hline
1.2.3 & de quibus omnibus quid sunt , \textbf{ et quomodo decet eas Reges habere , } et quas partes habent , & con estas dichas diez \textbf{ son doze las uirtudes morales delas quales todas que cosas son e en qual manera . | Conuiene a los Reyes } e alos prinçipes delos auer \\\hline
1.2.6 & et agibilia sint singularia , \textbf{ oportet prudentiam esse circa particularia , } applicando uniuersales regulas & en las cosas singulares . \textbf{ Conuiene que la pradençia sea cerca las cosas singulares | e particulares } allegando las reglas generales alos negoçios singulares \\\hline
1.2.7 & Triplici ergo via inuestigare possumus , \textbf{ quod decet Regem esse prudentem . Primo , quia sine prudentia non est } Rex & Et pues que assi es podemos prouar en tres maneras \textbf{ que conuiene al rey de seer sabio ¶ | La primera es que sin la pradençia non puede seer Rey segunt } uerdatmas solamente lo serie segunt el nonbre \\\hline
1.2.8 & quia nulli agenti hoc est possibile , \textbf{ sed decet Regem habere praeteritorum memoriam , } ut possit ex praeteritis cognoscere , & Ca esto ninguno non lo pie de fazer . \textbf{ Mas conuiene al Rey de auer memoria delans cosas passadas | por que pue da } por las cosas passadas conosçer e tomar \\\hline
1.2.8 & nec inniti semper solertiae propriae : \textbf{ sed oportet ipsum esse docilem , } ut sit habilis ad capescendam doctrinam aliorum , & nin atener se sienpre al su engennio propio . \textbf{ Mas conuiene le de ser doctrinable | por que sea ido neo } para tomar doctrina de los otros tom̃ado conseio de buenos \\\hline
1.2.8 & Sed ratione gentis quam dirigit , \textbf{ oportet ipsum esse expertum , } et cautum . & e del pueblo \textbf{ a que ha de gouernar . | Conuiene al Rey } que sea j muy prouado et muy aꝑçebido \\\hline
1.2.8 & ut possit eam melius in debitum finem dirigere . \textbf{ Ultimo oportet ipsum esse cautum . } Nam sicut in speculabilibus falsa aliquando admiscentur veris , & e traher los ala fin que deue¶ \textbf{ Lo postrimero conuiene al Rey | que sea muy aꝑçebido . } Ca assi commo en las sçiençias especulatuias algunas cosas falsas \\\hline
1.2.8 & sed apparent bona . \textbf{ Oportet igitur Regem esse cautum , } respuendo apparenter bona , & maguera que lo non sean . \textbf{ commo quier que paresçan buenas Et pues que assi es conuiene al Rey sea aꝑcebido } para desechar e despreçiar aquellas cosas \\\hline
1.2.9 & quae superius diximus , \textbf{ oportet ipsos esse bonos , } et non habere voluntatem deprauatam : & que dixiemos de suso \textbf{ conuieneles | que sean buenos } e que non ayan uoluntad mala nin desordenada \\\hline
1.2.12 & quae est quaedam clarissima virtus , probari potest , \textbf{ quod decet eos obseruare Iustitiam . } Tertio hoc probari potest & que se puede prouar \textbf{ que conuiene alos Reyes | de guardar la iustiçia . } lo terçero esso mismo se puede prouar \\\hline
1.2.12 & quam ex aliis virtutibus moralibus . \textbf{ Decet ergo Reges et Principes esse iustos , } tum quia debent esse regula agendorum , & que a ninguno de los otros . \textbf{ Et pues que assi es conuiene alos Reyes | e alos prinçipes de ser iustolo vno } por que deue ser regla de todas las cosas \\\hline
1.2.13 & et non recte , \textbf{ oportet dare virtutem aliquam } circa timores , et audacias . & e en las osadias \textbf{ por la qual sea el omne reglado en ellos . | por que contesce que algunos remen algunas cosas } que han de temer e alas uegadas temen alguas cosas \\\hline
1.2.15 & Nam si volumus nutrimentales delectationes reprimere , \textbf{ oportet nos temperari a potu , et cibo . } Temperando nos a potu , & delectaçonnes nutermentales \textbf{ con que se cera el cuerpo . | Conuiene nos que seamos tenprados en comer e en beuer } ca tenprando nos en el beuer seremos mesurados \\\hline
1.2.18 & quae continet . \textbf{ Cum ergo tanto deceat fontem habere os largius , } quanto ex eo plures participare debent : & Ca ha . manera daua so ancho e largo e da conplidamente lo que tiene \textbf{ ¶pues que assi es conmo tanto conuenga ala fuente auer la boca | mas ancha } quanto della deuen \\\hline
1.2.23 & ideo decet magnanimum esse paucorum operatiuum . \textbf{ Quarto decet magnanimum esse apertum , } ut sit veridicus , & Et por ende conuiene al magnanimo ser de pocas obras ¶ \textbf{ La quarta propiedat es que conuiene al maguanimo ser magnifiesto } assi que sea uerdadero \\\hline
1.2.23 & habere decet Reges , et Principes . \textbf{ Decet enim Reges se non exponere } pro quibuscunque periculis , & e alos prinçipes delas auer \textbf{ Por que conuiene alos Reyes non se poner } a quales se quier periglos \\\hline
1.2.23 & ubi agetur de regimine Regni , \textbf{ non decet eos esse plangitiuos , } vel deprecatiuos pro exterioribus bonis . & o determinaremos del gouernamiento del regno \textbf{ que non conuiene alos Reyes de seer lloradores nin rogadores } por los bienes de fuera . \\\hline
1.2.24 & bene dictum est , \textbf{ quod decet eos esse magnanimos , } et honoris amatiuos . & Bien dicho es \textbf{ que a ellos pertenesce seer magnanimos } e amadores de honrra \\\hline
1.2.25 & si unum et idem aliter et aliter acceptum nos retrahit et impellit , \textbf{ oportebit circa illud dare duas virtutes , } unam impellentem , & e nos allega a aquello que la razon manda o uieda . \textbf{ Conuiene de dar en aquella cosa dos uirtudes ¶ La vna que nos allegue . } Et la otra qua nos arriedre dello . \\\hline
1.2.27 & mansuetudo nominat temperamentum irae . \textbf{ Quod autem deceat Reges et Principes esse mansuetos , } ostendere non est difficile . & La manssedunbre nonbra tenpramiento de sana . \textbf{ mas mostrar que conuiene alos Reyes | e alos prinçipes de ser manssos } esto non es cosa guaue mas ligera . \\\hline
1.2.27 & virtuosus non esset . \textbf{ Tanto ergo magis decet Reges et Principes moueri } ad punitionem faciendam , & non seria uirtuoso . \textbf{ Et pues que assi es tanto | mas conuiene alos Reyes } e alos prinçipes de se mouer a dar penas . \\\hline
1.2.29 & idest irrisores , et despectores . \textbf{ Oportet ergo dare aliquam virtutem mediam , } per quam moderentur diminuta , & que quiere dezir escarnidores e despreçiadores dessi mismos . \textbf{ Et pues que assi es conuiene de dar alguna uirtud medianera } por la qual sean tenpradas las cosas menguadas \\\hline
1.2.31 & in hanc sententiam conuenerunt , \textbf{ quod oportet virtutes connexas esse . } Dixerunt enim & commo los p̃h̃osacuerdan en esta sentençia \textbf{ que conuiene que todas las uirtudes sean ayinntadas la vna con la vna con la otra . } Ca dixieron que aquel que ha vna uirtud \\\hline
1.2.33 & et Principes alios excellere debent , \textbf{ tanto ardentius decet eos diuinam gratiam postulare . } In hoc ergo eliditur Philosophorum elatio , & e los prinçipes deuen sobrepular los otros en bondat tanto mas cobdiçiosamente \textbf{ e con mayor desseo deuen demandar la gera de dios . | ¶ Et pues que assi es por esto se puede tirar } e quebrantar el orgullo \\\hline
1.3.3 & ad Rempublicam fecit Romam esse principantem et monarcham . \textbf{ Hoc ergo modo quoslibet homines decet esse amatiuos , } ut primo et principaliter diligant & e auer sennorio en todo el mundo . \textbf{ Pues que assi es que esto conuiene a todos los omes de ser amadores } assi que primero e prinçipalmente amen el bien diuinal \\\hline
1.3.5 & Possumus autem quadrupliciter ostendere , \textbf{ quod deces Reges et Principes decenter se habere circa spem , } et sperare speranda , & Mas nos podemos mostrar en quatro maneras \textbf{ que conuiene alos Reyes | e alos prinçipes de se auer } conueniblemente cerca la esperança \\\hline
1.3.5 & propter quae arguere possumus , \textbf{ quod decet Reges et Principes esse bene sperantes . } Spes enim primo est de bono : & por las quales podemos mostrar \textbf{ que conuiene alos Reyes e alos prinçipes de ser bien esperantes ¶ } Lo primero que la espança es de algun bien \\\hline
1.3.5 & Possumus autem duplici via inuestigare , \textbf{ quod decet Reges et Principes aliquid aggredi ultra vires , } et sperare ultra quam sit sperandum . & por dos maneras \textbf{ que non conuiene alos Reyes | e alos prinçipes de acometer ninguna cosa } mas que la fuerca suya \\\hline
1.3.6 & non est fortis , sed satuus . \textbf{ Oportet ergo videre } quo modo eos esse deceat timidos , et audaces . & en el primero libͤ de los grandes morales . \textbf{ Et pues que assi es conuiene deuer } en qual manera conuiene alos Reyes de sertemosos \\\hline
1.3.7 & a Regibus , et Princibus , \textbf{ quia eos maxime decet sequi imperium rationis . } Cauenda est ergo ira inordinata , & e alos prinçipes \textbf{ por que mucho mas conuiene aellos | de segnir el iuyzio dela razon e del entendimiento . } Et pues que assi es paresçe \\\hline
1.3.11 & imitari debent . \textbf{ Decet enim eos esse gratiosos , et misericordes . } Ipsi enim maxime esse debent & en quanto son passiones de loar \textbf{ Por que conuiene a ellos de ser guaçiosos e mis cordiosos . } Ca ellos deuen ser muy conuenibles partidores de los bienes \\\hline
1.3.11 & quia non existimamus \textbf{ ipsum oportere operari , } in quibus est verecundia . & Ca non cuydamos \textbf{ que conuenga aellos de obrar ninguna cosa } en que caya uerguença . \\\hline
1.4.1 & quin committant aliqua turpia , \textbf{ de quibus decet eos uerecundari : } Reges tamen et Principes , & que non acometan algunas cosas torpes \textbf{ delas quales les conuiene | que tomne uerguença . } Enpero esto non es de loar en los uieios nin en los Reyes . \\\hline
1.4.1 & adaptare possumus Regibus et Principibus : \textbf{ quia decet eos esse liberales , } bonae spei , & e alos prinçipes \textbf{ por que conuiene aellos de ser liberales } e de buena esperança . \\\hline
1.4.1 & maxime magnanimitas competit Regibus et Principibus , \textbf{ quia eos maxime magna decet operari , } et in ardua tendere . & e alos prinçipes la magernimidat \textbf{ que alos otros | por que mucho conuiene a ellos de obrar grandes cosas } e entender cerca las cosas altas \\\hline
1.4.2 & ergo quia aliorum debent esse regula et mensura , \textbf{ potissime eos decet moderatos esse . } Enumeratis moribus iuuenum , & Et pues que assi es que los Reyes deuen ser regla \textbf{ e mesura de todos los otros mucho les conuiene aellos de ser mas mesurados que los otros . } ontadas las costunbres de los mançebos \\\hline
1.4.5 & subtiliter inuestigantes \textbf{ quid decet eos facere , } ne opera eorum , & e escodrinnadores sotilmente de todo aquello \textbf{ que les conuiene de fazer } por que las sus obras \\\hline
1.4.7 & Potentes vero et principantes , \textbf{ quia oportet eos intendere exterioribus curis , } retrahuntur , & Mas los poderosos et los prinçipes \textbf{ por que les conuiene de entender | e auer cuydados de muchͣs cosas retrahen se } e tiran se \\\hline
1.4.7 & aliquos malos mores : \textbf{ quia non oportet omnes esse tales , } sed sufficit reperiri illud in pluribus : & li dellos contamos algunas malas costunbres \textbf{ ca non conuiene que todos seantales . } Mas abasta que aquellas costunbres sean falladas en muchos por que non \\\hline
2.1.2 & ad per se sufficientiam vitae , \textbf{ oportet communitatem domus necessariam esse . } Reges ergo et Principes , & e por si vale a conplimiento dela uida . \textbf{ Conuiene que la comunidat dela casa sea mas neçessaria } Et pues que assi es los Reyes e los prinçipes \\\hline
2.1.3 & et typo ostendere , \textbf{ quod decet homines habere habitationes decentes } secundum suam possibilem facultatem ; & por que ael parte nesçe generalmente demostrar \textbf{ por figera e por exienplo que conuiene alos omes de auer conueibles moradas } segunt el su poder e la su riqueza . \\\hline
2.1.5 & et serui ad conseruationem . Quare si generatio et conseruatio est quid naturale , \textbf{ oportet domum quid naturale esse . } Amplius , quia generatio et conseruatio & conseruaçique por la qual cosa si la generaçion e la conseruaçion es cosa natural \textbf{ conuiene | que la casa sea cosa natural ¶ } Otrossi por que la generaçion e la conseruaçion non pueden ser apartadas la vna dela otra \\\hline
2.1.6 & potest sibi simile producere , \textbf{ sed oportet prius ipsam esse perfectam . } statim enim , & luego que es fecha fazer otra semeiante \textbf{ assi mas conuiene que ella primeramente sea acabada } enssi \\\hline
2.1.6 & nec statim potest sibi simile producere , \textbf{ sed oportet prius ipsum esse perfectum : } producere ergo sibi similem , & luego otro su semeinante \textbf{ mas conuiene que primeramente el sea acabado . } Et pues que assi es engendrar su semeiante non pertenesçe a cosa natural tomada en qual quier manera mas pertenesçe a cosa natural en quanto ella es acabada . \\\hline
2.1.6 & Patet ergo quod ad hoc quod domus habeat esse perfectum , \textbf{ oportet ibi esse tres communitates : } unam viri et uxoris , aliam domini et serui , & Et por ende paresçe que para que la casa sea acabada \textbf{ que conuiene que sean enlla tres comuundades . } ¶ La vna del uaron e dela muger ¶ \\\hline
2.1.6 & Nam cum in domo perfecta sint tria regimina , \textbf{ oportet hunc librum tres habere partes ; } in quarum prima tractetur primo de regimine coniugali : & Ca commo en la casa acabada sean tres gouernamientos . \textbf{ Ca conuiene que este libro sea partido en tres partes . } ¶ En la primera delas quales tractaremos del gouernamiento mater moianl . \\\hline
2.1.7 & tanto magis decet fugere Reges et Principes , \textbf{ quanto decet eos meliores et virtuosiores esse . } His visis , & mas conuiene alos Reyes \textbf{ e alos prinçipes delo esquiuar quanto mas conuiene aellos de ser meiores | e mas uirtuosos que los otros . } ¶ Estas cosas dichͣs \\\hline
2.1.8 & Probant autem Philosophi , \textbf{ quod decet coniugia indiuisibilia esse . } Ad quod ostendendum adducere possumus duas vias , & os philosofos prue una que los con̊uiene \textbf{ que los casamientos sean sin departimiento ninguno } e que non le puedan partir . \\\hline
2.1.8 & et inrepudiatum . \textbf{ Decet ergo omnes ciues coniungi } suis uxoribus indiuisibiliter absque repudiatione , & por la qual el casamiento non deue ser partido nin repoyado . \textbf{ Et pues que assy es conuiene a todos los çibdadanos de se ayuntar | asus muger } ssin departimiento ninguno \\\hline
2.1.9 & Et tanto magis hoc decet Reges et Principes , \textbf{ quanto decet eos meliores esse aliis , } et magis sequi ordinem naturalem . & e alos prinçipes de segnir mas orden natural \textbf{ quanto mas conuiene aellos de ser meiores | que todos los otros . } Et pues que assi es paresçe \\\hline
2.1.10 & ut si quis subiicitur Proposito et Regi , \textbf{ oportet Propositum illum ad Regem ordinari , } et esse sub ipso repugnat & assi commo si algun çibdadano es subiecto al preuoste e al Rey . \textbf{ Conuiene que el | prinoste sea ordenado al Rey e sea so el . } Et por ende contradize ala orden natural \\\hline
2.1.10 & simul viris pluribus detestabilius esse debet . \textbf{ Decet ergo coniuges omnium ciuium uno viro esse contentas : } multo magis tamen hoc decet & que vna muger case en vno con mugons varones . \textbf{ ¶ Et pues que assi es conuiene | quelas mugers de todos los çibdadanos sean pagadas de vn uaron . } Enpero mucho mas conuiene esto alas mugers de los Reyes \\\hline
2.1.11 & tanta multiplicaretur dilectio , \textbf{ quod oporteret eos nimium vacare venereis . } Decet ergo omnes ciues non inire connubia & tanto se acrescentarie el amor entre ellos \textbf{ que les conuernie de enteder | e de darse mucho alas obras de lux̉ia . } Et pues que assi es conuiene a todos los çibdadanos \\\hline
2.1.11 & quod oporteret eos nimium vacare venereis . \textbf{ Decet ergo omnes ciues non inire connubia } cum personis nimia consanguinitate coniunctis ; & e de darse mucho alas obras de lux̉ia . \textbf{ Et pues que assi es conuiene a todos los çibdadanos | de non fazer casamientos entre perssonas } que son muy ayuntadas en parentesço \\\hline
2.1.13 & quam intemperantia coniugum aliorum . \textbf{ Decet ergo coniuges temperatas esse . } Decet eas etiam amare operositatem : & dellos puede fazer mayor danno e enpeçemiento que la destenprança delas mugers de los otros . \textbf{ Et pues que al sy es conuiene | que las muger ssean tenpradas . } Et avn les conuiene aellas de amar fazer buenas obras . \\\hline
2.1.14 & Patet ergo , \textbf{ quomodo decet omnes ciues alio regimine praeesse uxoribus , } et alio filiis : & Et pues que assi es paresçe \textbf{ en qual manera conuiene a todos los çibdadanos | que enssennore en } por otro gouernamiento alos mugers \\\hline
2.1.15 & et quicquid natura praeparatur , \textbf{ oportet ordinatissimum esse : } quia ille naturam dirigit , & por la natura \textbf{ conuiene que sea muy ordenado . } Ca aquel gnia la natura de que viene todo ordenamiento \\\hline
2.1.16 & et quomodo utendum sit eo . \textbf{ Oportet ergo magis in particulari descendere , } qualiter omnes ciues & e en qual manera de una vsar del . \textbf{ Et pues que assi es conuiene de desçender | mas en particular } mostrando en qual manera todos los çibdadanos et mayormente los Reyes \\\hline
2.1.19 & Nam quicunque vult aliquid bene regere , \textbf{ oportet ipsum speciales habere cautelas ad ea , } circa quae videt ipsum magis deficere . & conuiene \textbf{ que el aya algunas cautelas espeçiales | para aquellas cosas } en las quales vee \\\hline
2.1.19 & in loquela dirigere , \textbf{ oporteret eos instruere , } ut specialem pugnam & que qualiesse enderesçar los tartamudos en la fabla conuenir le ha \textbf{ que los ensseñasse que tomassen espeçial batalla } e espeçial esfuerço cerca aquellas palauras \\\hline
2.1.19 & vel dissensio oriri . \textbf{ Secundo decet eas esse pudicas } et honestas . & e mayor discordia \textbf{ que lo segundo couiene a el de los otros . } las de ser linpias e honestas \\\hline
2.1.20 & insurgere nocumentum proli , \textbf{ decet abstinere a tali copula : } est ergo obseruandum tempus debitum . & conuiene les de guardar se \textbf{ de allegar se mucho alas mugers . Et pues que assi es conuiene les alos casados } de guardar tienpo conuenible \\\hline
2.1.23 & Quia ergo sic est , \textbf{ oportet foeminas deficere a ratione , } et habere consilium inualidum . & que las mugers \textbf{ que fallezcan de vso de razon e que ayan el conseio flaço . } Ca quando el cuerpo es meior conplissionado tanto \\\hline
2.1.23 & ut quia illud est citius in suo complemento , \textbf{ sic oporteret repentino operari , } forte elegibilius esset huiusmodi consilium . & por que el consseio de la muger es mas ayna el su conplimiento que deluats . \textbf{ por que si acaesçiesse de obrar alguna cosa adesora } por auentra a seria \\\hline
2.2.1 & Possumus autem triplici via venari , \textbf{ quod decet huiusmodi solicitudinem habere parentes . } Prima via sumitur & Et nos podemos mostrar por tres razones \textbf{ que conuienen a todos los padres | de auer grand cuydado de sus fijos } ¶ \\\hline
2.2.2 & solicitari circa eos . \textbf{ Licet omnes patres deceat solicitari } circa proprios filios , & que han alos fijos sean cuy dadosos della \textbf{ aguer que todos los padres de una } auer cuydado de sus fijos \\\hline
2.2.2 & eo quod principantis sit alios regere et gubernare : \textbf{ tanto ergo magis decet Reges et Principes solicitari } circa proprios filios , & por que alos prinçipes parte nesçe de gouernar e de garalo sots . \textbf{ pues que assi es tanto mas conuiene alos Reyes } e alos prinçipes de ser acuçiosos de sus fijos \\\hline
2.2.6 & Nam cum aliquis est pronus ad aliquid , \textbf{ oportet ipsum multum assuescere in contrarium , } ne inclinetur ad illud : & Ca quando alguno es inclinado a alguacosa . \textbf{ Conuiene que el vse mucho en el contrario } por que non sea inclinado a aquella cosa . \\\hline
2.2.6 & a lasciuiis retrahantur . \textbf{ Decet ergo omnes ciues solicitari erga filios , } ut ab ipsa infantia instruentur ad bonos mores . & e por bueons castigos sean tirados delas loçanias . \textbf{ Et pues que assi es | conuieneque todos los çibdadanos ayan grand cuydado de sus fijos } assi que luego en su moçedat \\\hline
2.2.7 & in regno maius periculum imminere . \textbf{ Licet deceret omnes homines cognoscere literas ; } ut per eas prudentiores effecti , & Et quanto dela maliçia dellos vernie mayor periglo a todo el regno \textbf{ omo quier que conuenga a todos los omes de saber letras } por que por ellas puedan ser mas sabios \\\hline
2.2.7 & quae est ex scientia acquirenda . \textbf{ Decet enim volentes literas discere , } literales sermones scire distincte proferre . & que omne aprende ¶ \textbf{ Lo primero paresce assi . | Ca conuiene que los que quieren aprinder sçiençia de letris } que aprendan pronunçiar departidamente las palabras delas letras ¶ \\\hline
2.2.7 & insudare literalibus disciplinis , \textbf{ quanto decet eos intelligentiores et prudentiores esse , } ut possint naturaliter dominari . & e en las sçiençias liberales \textbf{ quanto mas les conuiene de ser mas entendudos | e mas sabios que los otros } por que puedan \\\hline
2.2.8 & et per debitas rationes manifestemus propositum . \textbf{ Oportuit ergo inuenire aliquam scientiam docentem modum , } quo formanda sunt argumenta , et rationes . & i anifestamos nr̃a uoluntad e nr̃a entençion . \textbf{ Et por ende conuiene de fallar algua sçiençia | que nos mostrasse } en qual manera son de enformar los argumentos e las razones . \\\hline
2.2.8 & inquantum deseruiunt morali negocio . \textbf{ Decet igitur eos scire grammaticam , } ut intelligant idioma literale : & en quanto siruen ala ph̃ia moral . \textbf{ Et pues que assi es conuiene les a ellos de saber la guamatica } por que entiendan el lenguage delas letras \\\hline
2.2.9 & quam doctor : \textbf{ decet igitur ipsum esse inuentiuum . } Secundo decet ipsum esse intelligentem et perspicacem . & este tal mas es rezador que doctor ¶ \textbf{ Et pues que assi es conuiene al maestro | que non tan solamente sea fallador delas cosas } mas que sea entendido e sotil . Ca assi commo ninguon non puede abastar \\\hline
2.2.9 & decet igitur ipsum esse inuentiuum . \textbf{ Secundo decet ipsum esse intelligentem et perspicacem . } Nam sicut nullus bene et perfecte & que non tan solamente sea fallador delas cosas \textbf{ mas que sea entendido e sotil . Ca assi commo ninguon non puede abastar } asi en la uida bien \\\hline
2.2.9 & et intelligens aliorum dicta . \textbf{ Tertio oportet ipsum esse iudicatiuum : } nam perfectio scientiae potissime & e que entienda los dichͣs de los otros . \textbf{ ¶ Lo terçero conuiene que sea iudgador } e que aya razon para iudgar . \\\hline
2.2.9 & quomodo obliquata essent . \textbf{ Decet igitur aliorum directorem memorem esse praeteritorum . } Secundo decet & Et pues que assi es conuiene \textbf{ que el que ha degniar los otros | que sea acordado delas cosas passadas } ¶ \\\hline
2.2.9 & Quantum vero ad prudentiam agibilium , \textbf{ decet ipsum esse memorem , } prouidum , cautum , et circumspectum . & que son de fazer \textbf{ conuiene le que el doctor sea menbrado e prouado e sabio e acatado . } Mas quanto ala uida deue ser honesto e bueno . \\\hline
2.2.17 & Sufficiat autem ad praesens scire , \textbf{ quod decet patres sic solicitari erga regimen filiorum , } ut habeant sic bene dispositum corpus , & mas quanto alo presente abonda de saber \textbf{ que conuiene alos padres ser assi | acuçiossos çerca el gouernamiento delos fijos } por que ayan el cuerpo bien ordenado \\\hline
2.2.17 & tres breues rationes , \textbf{ quare decet filios esse subiectos , } et obedire senioribus , et patribus . & razono breues \textbf{ por que conuiene alos fijos de ser lubiectos } e obedientes a lus padres e alos vieios . \\\hline
2.2.20 & circa qualia opera solicitari debent : \textbf{ oportet in talibus differenter loqui } secundum diuersitatem personarum . & Mas si alguno demandare \textbf{ de que se deuen trabaiar las mugers conuiene de fablar en tales cosas departidamente } segunt el departimiento delas perssonas \\\hline
2.2.20 & infra declarandum esse , \textbf{ circa quae opera deceat foeminas esse intentas . } Ostenso , & casamien toca y dixiemos que adelante serie de declarar cerca quales obras conuenia \textbf{ que las mugers fuesen acuçiosas . } ostrado que non conuiene alas moças de andar uagarosas a quande e allende \\\hline
2.2.21 & Ostenso , \textbf{ quod non decet puellas esse vagabundas , } nec decet eas viuere otiose : & que las mugers fuesen acuçiosas . \textbf{ ostrado que non conuiene alas moças de andar uagarosas a quande e allende } nin les conuiene de beuir ociosas \\\hline
2.2.21 & restat ut nunc tertio ostendamus , \textbf{ quod decet eas taciturnas esse , } quod triplici via venari possumus . & finca que agora lo terçero mostremos \textbf{ que deuen ser callantias | e non parleras } la qual cosa podemos mostrar \\\hline
2.2.21 & cautos proferre sermones , \textbf{ decet eas non esse loquaces : } sed oportet ipsas esse debite taciturnitas , & tomadesto \textbf{ que las mugrͣ̃s non sean prestas avaraias e apeleas } ca commo las muger se mayormente las mocas \\\hline
2.3.2 & et mensuram ex illis : \textbf{ oportet organa domus ordinata esse , } et organa inferiora , & Et resçiben dellas manera de seruiçio e mesura . \textbf{ Conuiene avn que los estrumentos dela casa sean ordenados } e que los instrumentos mas baxos \\\hline
2.3.3 & nam secundum Philosophum 4 Ethicorum capitulo de Magnificentia , \textbf{ maxime gloriosos et nobiles decet esse magnificos : } Reges ergo et Principes , & en el quarto libro delas ethicas \textbf{ enł capitulo dela magnifiçençia | que much mas conuiene alos Reyes } e alos prinçipes \\\hline
2.3.3 & In domibus ergo Regum et Principum \textbf{ oportet multos abundare ministros , } ut ergo non solum personas Regis et Principis , & e de los prinçipes conuiene \textbf{ que ayan muchos ofiçiales | e much ssiruient s̃ Et pues que assi es } por que non solamente la persona del Rey o del prinçipe mas avn \\\hline
2.3.3 & in aedificiis constructis , \textbf{ oportet ipsa esse magnifica . } Viso , qualia debent esse aedificia , & que ellos fazen a \textbf{ conuiene | que ellos sean muy grandes e muy costosas } ¶ Visto quales deuen ser las moradas \\\hline
2.3.9 & Ut ergo sciamus quomodo huiusmodi commutationes \textbf{ oportuit introduci , } sciendum quod si non esset & Et pues que assi es por que sepamos en qual manera conuiene \textbf{ que estas tales muda connes fuessen puestas en la } tiecra deuedes saber \\\hline
2.3.9 & quibus non abundant frigidae et econuerso . \textbf{ Propter quod non solum oportet communicare } et conuersari & Et por el cotrario en alguas abonda friura \textbf{ que non abonda ca lentura } Por la qual cosa non solamente conuiene alos omes morar e conuerssar los vnos con los otros \\\hline
2.3.9 & commode ad partes longinquas portari non possunt . \textbf{ Oportuit ergo inuenire aliquid } quod esset portabile , & non las poderemos leuar conueniblemente a luengas tierras . \textbf{ Et pues que assi es conuiene de fablar alguna cosa } que se podiesse leuar \\\hline
2.3.9 & et rerum ad numismata , \textbf{ oportuit inuenire commutationem numismatum ad numismata . } Patet ergo quot sunt commutationes , & e delas cosas alos \textbf{ diueros otra mudaçiones | que es de monedas alas monedas . } Et pues que assi es paresçe \\\hline
2.3.13 & ut si plures voces efficiunt aliquam harmoniam , \textbf{ oportet ibi dare aliquam vocem praedominantem , } secundum quam tota harmonia diiudicatur . & Assi commo si muchas uozes fiziess en alguna armonia o concordança de canto . \textbf{ Conuerna de dar y alguna bos | que enssennoreasse sobre las otras } segunt la qual serie iudgada toda aquella concordança delas uozes delas otras avn en essa misma manera \\\hline
2.3.15 & quos virtus et amor boni inclinat ad seruiendum , \textbf{ decet principantes se habere quasi ad filios , } et decet eos regere non regimine seruili , & e el amor de bien los inclina asuir . \textbf{ Conuiene que los prinçipes se ayan çerca ellos | assi commo cerca de fijos . } Et conuiene les alos prinçipes delos gouernar non \\\hline
2.3.16 & si debet esse ordinata , \textbf{ oportet reduci in unum aliquem , } a quo ordinetur . & En essa misma manera cada vna muchedunbre si bien ordenada es \textbf{ conuiene que sea aduchͣa vn ordenador } de quien ella sea ordenada . \\\hline
2.3.18 & sed quia volunt retinere mores curiae et nobilium , \textbf{ quos decet datiuos esse ; } propter quod tales curiales dici debent . & e de los no nobles omes alos \textbf{ que les conuienne de ser dadores } e cobidadores \\\hline
2.3.18 & de leui patet \textbf{ quod decet ministros Regum et Principum curiales esse . } Nam si decet Reges et Principes & Visto que cosa es la curialidat et la cortesia de ligero puede paresçer \textbf{ que los seruientes de los Reyes | e delos prinçipes deuen ser curiales } e cortesesca \\\hline
2.3.18 & habere mores nobiles et curiales , ministros , \textbf{ quos in bonis decet suos dominos imitari , } oportet curiales esse . & e de ser curiales e nobles \textbf{ assi conuiene alos seruientes dellos | los que quieren semeiar a sus sennors } de ser buenos e mesurados e corteses . \\\hline
3.1.1 & gratia alicuius boni , \textbf{ oportet ciuitatem ipsam constitutam esse propter aliquod bonum . } Probat autem Philosophus primo Polit’ duplici via , & commo toda comunidat sea por graçia de algun bien . \textbf{ Conuiene que la çibdat sea establesçida por algun bien | Ca pruena el pho } enl primero libro delas politicas \\\hline
3.1.4 & ut natura non deficiat in necessariis , \textbf{ oportet quid naturale esse } quicquid secundum se deseruit & conuiene \textbf{ que sea cosa natural todo aquello } que sirue a conplimiento de uida \\\hline
3.1.4 & quam communitates illae , \textbf{ oportet eam esse secundum naturam . } Secunda uia ad inuestigandum hoc idem , & que estas dos comuidades \textbf{ por ende conuiene | que la çibdat lea comuidat natraal ¶ } La segunda razon para prouar \\\hline
3.1.6 & in inueniendo artem aliquam , \textbf{ sed oportet ad hoc iuuari } per auxilium praecedentium & niguno non abasta assi mismo en fallar algunan arte \textbf{ mas conuiene que sea ayuda de } por ayuda de los que passaronante \\\hline
3.1.8 & Quia ergo diuersis indigemus ad vitam , \textbf{ oportet in ciuitate diuersitatem esse . } Tertia via declarans & por que nos auemos me estermuchͣs cosas departidas para abastamiento dela uida \textbf{ conuiene que enla çibdat sea algun departimiento . } La tercera razon que declara e manifiesta las razones \\\hline
3.1.8 & nisi sit ibi diuersitas officiorum . \textbf{ Decet ergo hoc Reges , et Principes cognoscere , } quod nunquam quis bene nouit regere ciuitatem , & e de los ofiçiales . \textbf{ Et pues que assi es conuiene alos Reyes | e alos prinçipes de sabesto } por que munca ninguno sopo bien gouernar çibdat \\\hline
3.1.11 & ex parte ipsorum communicantium in haereditate communi , \textbf{ eo quod oporteat eos valde ad inuicem conuersari , } ostenditur ut plurimum homines habere lites et iurgia & los que han la heredat en comun paresçe \textbf{ por que han de beuir en vno } que por la mayor parte han contiendas e uaraias por la qual cosa dize el philosofo \\\hline
3.1.11 & et indignamur erga illos , \textbf{ quia oportet nos habere } ad illos multa colloquia , & e nos enssannamos contra ellos muchͣs uezes \textbf{ por que nos conuiene de fablar muchͣs uezes con ellos } e de beuir conellos \\\hline
3.1.14 & et onerosius et quasi omnino importabile esset sustentare sic quinque milia : \textbf{ oporteret enim ciuitatem illam habere possessiones quasi ad votum , } ut posset ex communibus sumptibus & ca conuerne \textbf{ que aquella çibdat ouiesse tantas possessiones | quantas quisiesse a ssu uoluntad } por que pudiesse de las rentas comunes abondar atanta muchedunbre \\\hline
3.1.17 & possumus triplici via venari , \textbf{ quod non oportet possessiones aequatas esse , } ut Phaleas statuebat . & por tres razones podemos prouar \textbf{ que non conuiene | que las possessiones sean egualadas } en aquella manera \\\hline
3.1.17 & sumitur ex parte virtutum \textbf{ quas decet habere ciues : } decet enim ipsos & non es conuenible se toma de parte delas uirtudes \textbf{ que deuen auer los çibdadanos } por que conuiene \\\hline
3.2.5 & per haereditatem transferatur ad posteros , \textbf{ oportet eam transferre in filios , } quia secundum lineam consanguinitatis filii parentibus maxime sunt coniuncti : & por hedamiento conuiene alos pueblos \textbf{ que tomne alos fijos } ca segunt el linage del patente \\\hline
3.2.5 & quia ( ut ait Philosophus in Politiis ) \textbf{ decet iuniores senioribus obedire . } Immo quia patres plus communiter primogenitos diligunt ; & conuiene \textbf{ que los mas mançebos obedescan alos mas uieios e avn } por que los padres comunalmente \\\hline
3.2.6 & cum calefit et rarefit , \textbf{ oportet raritatem et calorem perfectius reperiri } in igne iam generato & por que la materia estonçe es puesta e tornada en fuego \textbf{ quando es muy escalentada | e muy enraleçida conuiene que la raledat e la calentura mas acabadamente sea fallada en el fuego } despues que fuere engendrado e ençendido . \\\hline
3.2.6 & in ipsa monarchia perfectius reperiri . \textbf{ Decet enim ipsum regem volentem recte regere } ( quantum ad praesens spectat ) & que ante ca conuiene \textbf{ que el Rey | que quiere bien gouernar su regno } quanto parte nesçe alo presente \\\hline
3.2.10 & quibus indigent , \textbf{ ut non vacet eis aliquid machinari contra ipsos , nec oporteat ipsos habere aliquam custodiam propter illos . } Verus autem Rex & en que han de de beuir de cada dia \textbf{ por que no les uague de fazer ayuntamiento contra ellos | nin los tiranos non ayan menester ninguna guarda } por temor dellos . \\\hline
3.2.15 & et epiikis idest super iustus : \textbf{ decet enim talem esse quasi semideum , } ut sicut alios dignitate et potentia excellit , & ca conuiene \textbf{ que el tal que sea | assi commo dios } assi que commo lieua auna taia de los otros en dignidat e en poderio \\\hline
3.2.15 & et saluant . \textbf{ Decet ergo Regem frequenter meditari et habere memoriam praeteritorum } quae contigerunt in regno , & e qual cosa lo salua . \textbf{ Et pues que assi es conuiene al Rey de penssar mucha menudo | e muchͣs uezes delas cosas que passaron . } Et conuiene le de auer memoria de los fecho passados \\\hline
3.2.16 & utrum ciues inter se pacem debeant habere , \textbf{ et utrum regnum oporteat esse in bono statu : } sed haec accipit tanquam certa et nota , & si los çibdadanos deuen auer entre ssi paz . \textbf{ Et si conuiene | que el regno se en buen estado . } mas esto sopone \\\hline
3.2.17 & quam unus solus : \textbf{ decet ad huiusmodi negocia alios aduocare , } ut per eorum consilium possit & por la quel cosa commo muchs mas cosas ayan prouadas \textbf{ que vno solo conuiene de llamar otros } para los negoçios . por que por el conseio dellos pueda ser escogida la meior carrera \\\hline
3.2.17 & operamur autem prompte : \textbf{ et quod oportet consiliari tarde , } sed facere consiliata velociter . & mas obramos en poco tienpo \textbf{ e luego e que conuiene de touiar conseio prolongadamente } mas conuiene de fazerl cosas conseiadas mucho ayna . \\\hline
3.2.18 & Sed ad hoc quod aliquis sit bene creditiuus , \textbf{ non oportet ipsum esse existenter talem , } sed sufficit quod videatur & mas para que alguno sea bien de creer \textbf{ non conuiene | que el sea tal fechmas cunple } que parezca tal cael o en iudga las cosas que paresçen de fuera por las cosas que vee \\\hline
3.2.19 & et prouentus regni , \textbf{ quos oportet peruenire ad regem , } qui et quanti sunt : & Et conuiene que sepan las rentas del regno \textbf{ las que han de venir al Rey quales e quantas son } por que si alguͣ cosa es superflua \\\hline
3.2.22 & Possumus autem quatuor enumerare , \textbf{ quae oportet habere iudices , } ut vera iudicia proferant , & demos contar quatro cosas \textbf{ que conuiene de auer alos iuezes } para que den uerdaderos iuisios \\\hline
3.2.22 & ab alia vero recedit per odium , \textbf{ oportet ipsum iudicare inique : } quia tunc iudicium non procedit & por abortençia o por mal querençia . \textbf{ conuiene que el uiez judgue mal e desigual mente . } Ca entonçe el uuzio non salle de zelo de iustiçia \\\hline
3.2.23 & quibus congruit ampliori bonitate pollere . \textbf{ Decet itaque eos esse clementes et benignos , } non quia iustitiam deserant , & por mayor bondat . \textbf{ Et pues que assi es conuiene a ellos de ser piadosos e benignos non } por que dexen la iustiçiaca sin ella la paz del regno \\\hline
3.2.26 & et hoc sequi volumus , \textbf{ oportet hoc agere . } Tales ergo debent esse leges , & e esto queremos alcançar \textbf{ conuiene que fagamos estas cosas . } Et pues que assi estales deuen ser las leyes \\\hline
3.2.26 & et bonum priuatum ordinetur ad ipsum , \textbf{ oportet tales leges fieri } non quales requirit bonum priuatum , & porque el bien propra o es ordenado al bien comun . \textbf{ Conuiene que las leyes tales sean non } quales demanda el bien propre \\\hline
3.2.27 & ad hoc quod lex habeat vim obligandi , \textbf{ oportet eam promulgatam esse . } Sed cum alia sit lex naturalis , & Poque la ley aya uirtud e fuerça de obligar \textbf{ conuiene que sea publicada e pregonada . } Mas commo otra sea la ley natural e otra la positiua en vna manera se deue publicar la vna \\\hline
3.2.29 & quae sit applicabilis humanis actibus . \textbf{ Oportet igitur aliquando legem plicare ad partem unam , } et agere mitius cum delinquente , & e allegar alas obras delos omes . \textbf{ Et por ende conuiene quela ley que se ençorue } e se allegue algunas vezes ala vna parte e que obre mas manssamente con el que peca \\\hline
3.2.29 & quam lex dictat : \textbf{ aliquando etiam oportet eam plicare ad partem oppositam , } et rigidius punire peccantem , & quela ley demanda o que la ley nidga . \textbf{ Et algunas vezes conuiene que la regla se encorue | ala parte contraria } e que mas reziamente de pena \\\hline
3.2.30 & attingere punctalem formam viuendi , \textbf{ ideo oportet aliqua peccata dissimulare } et non punire lege humana , & comunalmente non puede alcançar forma de beuir en punto . \textbf{ Por ende conuiene que | dessemeie alguons pecados } e que les de passada \\\hline
3.2.32 & in ciuitate et regno , \textbf{ oportet esse talem , } quod viuat bene et virtuose . & que es en el regno e enla çibdat . \textbf{ conuiene que sea atal que biuna bien e uirtuosamente . } Et por ende assi conmo dize el philosofo en el terçero libro delas politicas \\\hline
3.2.33 & ex personis mediis . \textbf{ Decet ergo Reges et Principes adhibere cautelas , } ut in regno suo abundent multae personae mediae ; & establesçidas de perssonas medianeras . \textbf{ Et pues que assi es conuiene | que los reyes e los prinçipes ayan cautelas e sabidurias . } por que en el su regno sean muchͣs perssonas medianeras \\\hline
3.2.36 & quiescant male agere : \textbf{ oportuit ergo aliquos inducere ad bonum , } et retrahere a malo timore poenae . & Por la qual cosa conuiene \textbf{ que alguon s | enduxiessemosa bien } e arredrassemos del mal \\\hline
3.3.1 & per quam quis scit regere domum et familiam , \textbf{ oportet esse aliam a prudentia , } qua quis nouit seipsum regere . & por la qual cada vno sabe gouernar la casa e la conpaña . \textbf{ Conuiene que sea otra e departida de la sabiduria } por la qual cada vno sabe gouernar a ssi mismo . \\\hline
3.3.1 & et gubernare ciues . \textbf{ Omnes autem tres prudentias decet habere Regem , } videlicet particularem , oeconomicam et regnatiuam . & e en quanto ha de poner leyes e gouernar los çibdadanos . \textbf{ Et todas estas tres sabidurias | conuiene que aya el Rey . } Conuiene a saber . \\\hline
3.3.1 & Hanc autem prudentiam videlicet militarem , \textbf{ maxime decet habere Regem . } Nam licet executio bellorum , et remouere impedimenta ipsius communis boni , & Et esta sabiduria de caualleria \textbf{ mas pertenesçe al rey que a otro ninguno . | Ca commo quier que pertenezca a los caualleros } la essecuçion de las batallas \\\hline
3.3.3 & esse videtur armorum industria . \textbf{ Nam siue equitem siue peditem oportet esse bellantem , } quasi fortuito videtur & nin ligera arte auer sabiduria de las armas . \textbf{ Ca si quier sea cauallero si quier peon el que ha de lidiar paresçe } que por uentura alcaça uictoria \\\hline
3.3.4 & infra patebit . \textbf{ Octauo decet bellatores verecundari , } et erubescere turpem fugam . & e otras cosas que se ayuntan a estas adelante paresçra Lo . viij° \textbf{ que pertenesçe a los lidiadores | es de auer uerguença } e de guardar se \\\hline
3.3.5 & quid sit de quaesito tenendum , \textbf{ oportet aduertere , } quod secundum diuersitatem pugnarum & Et pues que assi es para saber la uerdat \textbf{ que auemos de tener } desta question de tener mientes \\\hline
3.3.8 & oporteat exercitum constringi et constipari . \textbf{ Quarto si oporteat in loco illo exercitum moram contrahere , } et adsit possibilitas est eligenda & nin avn sea tomado tan pequeno espaçio por que la hueste este a mayor estrechura que deue . \textbf{ Lo quarto si conueniere que aquella hueste aya de fazer | en aquel logar alguna tardança } e fuere cosa que se puede fazer \\\hline
3.3.9 & erga necessitates corporis . \textbf{ Nam existentes in exercitu oportet multa incommoda tolerare : } quare si sint ibi aliqui molles , & penssada la sufrençia en las neçessidades del cuerpo . \textbf{ ca los que estan en las huestes | conuiene que sufran muchos males . } por la qual cosa si fueren y algunos muelles e mugerilles \\\hline
3.3.10 & non sufficiunt ad dirigendum bellantes , \textbf{ sed oportet dare euidentia signa ; } ut quilibet solo intuitu sciat & para guiar los lidiadores . \textbf{ Mas conuiene de dar otras seña les manifiestas . por que cada vno viendo aquellas señales } se sepa tener ordenadamente en su az \\\hline
3.3.13 & In percutiendo autem caesim , \textbf{ quia oportet fieri magnum brachiorum motum prius quam infligatur plaga , } aduersarius ex longinquo potest prouidere vulnus , & Mas en feriendo cortando . \textbf{ por que conuiene de fazer grand mouimiento de los braços | ante que se de el colpe el enemigo } o el contrario de \\\hline
3.3.14 & difficilius se defendere poterunt : \textbf{ quia oportet eos sparsim incedere . Quare sicut locus ineptus defensioni , } si in eo hostes inueniantur , & e con mayor trabaio . \textbf{ Ca conuieneles que anden esparzidos . | Por la qual cosa } assi commo el logar malo \\\hline

\end{tabular}
