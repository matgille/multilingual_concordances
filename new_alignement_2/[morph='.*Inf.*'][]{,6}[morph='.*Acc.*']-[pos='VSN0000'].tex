\begin{tabular}{|p{1cm}|p{6.5cm}|p{6.5cm}|}

\hline
1.1.3 & Media , sunt bona interiora , \textbf{ quae possunt esse communia bonis , } et malis : & ¶ Los bienes medianeros son ençerrados de dentro del alma \textbf{ e ño paresçen los quales pueden ser comunales alos buenos e alos malos } Et estos son sotileza del entendemiento \\\hline
1.1.6 & Secundo , quia eum contemptibilem reddunt . \textbf{ Tertio , quia esse ipsum indignum principari faciunt . } Dicebatur enim supra , & le fazen ser menospreçiado de los omes \textbf{ ¶la terçera es que las plazenterias dela carne le fazen | que non sea digno de ser prinçipe¶ } Ca asi commo ya dixiemos de suso \\\hline
1.1.6 & Dictum est enim \textbf{ quod decet Principem esse supra Hominem , } et totaliter diuinum . & Ca dicho n auemos ya desuso \textbf{ que conuiene al Rei | e al prinçipe sier } mas que omne e ser del todo diuinal . \\\hline
1.1.6 & quod maxime expedit Principibus , \textbf{ esse moderatos } in delectationibus corporibus . & dize \textbf{ que mucho conuiene alos prinçipes ser mesurados } e tenprados en las delecta con nes corpora les . \\\hline
1.1.7 & nunquam potest esse magnificus , \textbf{ cuius est facere magnos sumptus : } nec etiam potest esse Magnanimus , & nin granado \textbf{ el qual magnifico ha de fazer grandes espensas para ser granado } nin ahun puede ser mager fico \\\hline
1.1.7 & diuitiae sunt quid magnum , \textbf{ impossibile est talem esse Magnanimum . } Si ergo Magnificentia , & las riquezas son grant cosa e grant bien . \textbf{ Et por ende non puede ser ninguon tal magnanimo | nin de alto coraçon . } pues que asi es ssy la magnifiçençia \\\hline
1.1.7 & et maxime decet regiam maiestatem \textbf{ esse ornatam talibus virtutibus , } detestabile est ei & mucho conuiene al Rei \textbf{ e ala Real magestad de ser conpuesta e ennobleçida de tales uirtudes } commo estas ¶ \\\hline
1.1.8 & Si ergo maxime decet \textbf{ Regem esse bonum existentem , } maxime indecens est & que estos son infintos e superfiçiales . \textbf{ Et pues que assi es si mucho conuiene al Rey de ser bueno uerdaderamente } e non solamente de paresçer bueno . \\\hline
1.1.9 & qua decet Reges , \textbf{ et Principes esse magnificos , et magnanimos . } Magnanimi autem licet & por que conuiene alos Reys alos prinçipes \textbf{ de ser manificos e grandes e magnanimos e de grandes coraçones . } Ca commo quier los de altos coraçones \\\hline
1.1.10 & ( secundum eundem Vegetium ) \textbf{ hoc esse debet principalissimum in intentione Principis , } quod abundet in ciuili potentia , & e todas las gentes . \textbf{ ¶ Por la qual cosa segunt este philosofo vegeçio esto deue ser la cosa mas prinçipal | en la entençion del prinçipe } que abonde en poderio ciuil que es poderio de çibdat e de regno \\\hline
1.1.10 & aliquem putare \textbf{ esse felicem , } si abiiciat bene viuere . & que cosa de escarnio es cuydar \textbf{ que alguno puede seer bien } auentraado si despreçia el bien beuir . \\\hline
1.1.11 & et exercere operationes virtutum : \textbf{ decet enim Regem esse magnificum , } beneficiare personas dignas : & e fazer obras de uertudes . \textbf{ E conuiene al Rey de seer magnifico e largo } por que pueda bien fazera las personas dignas \\\hline
1.1.13 & cum semper amor sit ad similes , et conformes , \textbf{ oportet esse similem , } et conformem Deo , & Et commo el amor sienpre sean los semeiables e acordables con el . \textbf{ Conuiene que aquel que es para de ser } gualardonado de dios \\\hline
1.2.2 & quod impossibile est prudentem \textbf{ esse non existentem bonum . } Dimissis ergo virtutibus huiusmodi intellectualibus , & en el seyto libro delas ethicas . ca non puede ser el omne pradente e sabio \textbf{ e non ser bueno . } Pues que assi es dexando aparte las uirtudes intellectuales \\\hline
1.2.4 & ostendentes , \textbf{ quomodo Reges et Principes debent habere uirtutes . } Determinabimus etiam de adminiculantibus uirtuti , & de cada vna en su logar . Ca determinaremos delas uirtudes mostrando . \textbf{ en qual manera los Reyes e los prinçipes . | han de auer uirtudes e seer uirtuosos . } Et ahun determinaremos delas otras \\\hline
1.2.4 & quomodo Reges et Principes oportet \textbf{ talibus esse ornatos . } Sed primo de ipsis uirtutibus & que son sobre todas las otras uirtudes mostrando en qual manera los Reyes e las prinçipes \textbf{ han de ser conpuestos e honrrados } de tales uertu des diuinales . \\\hline
1.2.5 & in ratiocinando de agibilibus : \textbf{ sic est dare virtutem , } per quam dirigimur in operanda ipsa agibilia . & ¶Otrosi ahun por que nos \textbf{ contesçe de ser passionados } e resçebidores de passiones \\\hline
1.2.5 & ostendentes quod decet Reges , \textbf{ et Principes esse iustos . } Deinde determinabimus de Fortitudine , et Temperantia , & mostrando que conuiene alos Reyes \textbf{ e alos prinçipes de ser iustos e derechureros . } Et desi diremos dela fortaleza e dela tenprança \\\hline
1.2.5 & declarantes quod contingit Reges \textbf{ et Principes esse fortes et temperatos . } Consequenter vero tractabimus de Magnitudine , et Magnificentia , & mostrando e declarando \textbf{ que conuiene alos Reyes e alos prinçipes de seer fuertes e tenprados . } Mas enpos esto todo \\\hline
1.2.5 & quomodo decet Reges \textbf{ et Principes talibus uirtutibus esse perfectos . } Prudentia autem , & e declarando en qual manera conuiene alos Reyes \textbf{ e alos prinçipes de ser acabados e conplidos destas tales uirtudes . } la pradençia e la sabiduria de que primeramente auemos de fablaͬ \\\hline
1.2.7 & Triplici ergo via inuestigare possumus , \textbf{ quod decet Regem esse prudentem . Primo , quia sine prudentia non est } Rex & Et pues que assi es podemos prouar en tres maneras \textbf{ que conuiene al rey de seer sabio ¶ | La primera es que sin la pradençia non puede seer Rey segunt } uerdatmas solamente lo serie segunt el nonbre \\\hline
1.2.8 & ad quae dirigit , \textbf{ oportet Regem esse memorem , et prouidum : } propter modum & aque guiasa pradençia . \textbf{ Conuiene al Rey de ser acordable e prouisor ¶ } Et por razon dela manera segunt laquel guia . \\\hline
1.2.8 & secundum quem dirigit , \textbf{ oportet ipsum esse intelligentem , } et rationabilem : & Et por razon dela manera segunt laquel guia . \textbf{ Conuiene al Rey de ser entendido e razonable ¶ } por razon de la su propia persona \\\hline
1.2.8 & sic ratione modi per quem dirigit , \textbf{ oportet ipsum esse intelligentem , et rationalem . } Sed ratione propriae personae & por la qual ha de guiar el pueblo \textbf{ le conuiene de ser entendido e razonable . } Mas por razon dela su persona propia \\\hline
1.2.8 & cum hoc quod Regem expedit \textbf{ esse solertem ex se , } quae bona sunt regno utilia excogitando , & que son aprouechables a todo el regno . \textbf{ Enpero con esto que conuiene al Rey de ser sotil e agudo de si penssando los bienes } que son aprouechables a su regno \\\hline
1.2.8 & quae bona sunt regno utilia excogitando , \textbf{ oportet ipsum esse docilem , } aliorum consiliis acquiescendo . & que son aprouechables a su regno \textbf{ ahun conuiene le de ser doctrinable resçebiendo e tomando coseio de bueons } quel han bien de conseiar . \\\hline
1.2.8 & nec inniti semper solertiae propriae : \textbf{ sed oportet ipsum esse docilem , } ut sit habilis ad capescendam doctrinam aliorum , & nin atener se sienpre al su engennio propio . \textbf{ Mas conuiene le de ser doctrinable | por que sea ido neo } para tomar doctrina de los otros tom̃ado conseio de buenos \\\hline
1.2.8 & quae est alios dirigens , \textbf{ oportet Regem esse solertem , et docilem . } Sed ratione gentis quam dirigit , & que es gouernandor de los otros . \textbf{ Conuiene al rey de ser engennioso e doctrinable . } Mas por razon dela gente \\\hline
1.2.9 & eliciendo ex eis debitas conclusiones agibilium . \textbf{ Non enim sufficit esse intelligentem , } habendo cognitionem legum , & para todas las cosas \textbf{ que ha de fazer . | Ca non abasta seer entendido } sabiendo las leyes e las costunbres \\\hline
1.2.10 & determinat sibi specialem materiam , \textbf{ et habet esse circa haec bona exteriora , } in quibus ciues communicant . & çibdadano viene danno a otro la iustiçia spanl determinasse a materia sp̃al . \textbf{ Et a de ser cerca estos bienes de fuera } en los quales los çibdadanos partiçipa \\\hline
1.2.11 & et prohibet omne malum : \textbf{ implere legem , } est esse perfecte virtuosum . & que le saga todo bien \textbf{ e uieda todo mal cunplir la les es seer omne uertuoso acabadamente . } Et por ende dize el philosofo en el primero cabło dela \\\hline
1.2.11 & implere legem , \textbf{ est esse perfecte virtuosum . } Ideo primo Magnorum moralium , & que le saga todo bien \textbf{ e uieda todo mal cunplir la les es seer omne uertuoso acabadamente . } Et por ende dize el philosofo en el primero cabło dela \\\hline
1.2.11 & in aliquo legalem Iustitiam , \textbf{ est eos esse integre et perfecte malos . } Sed ( ut dicitur Ethic’ 4 cap’ de mansuetudine ) & nin toma ninguna parte dela iustiçia legal . \textbf{ esto es lo que los faze ser enteramente e coplidamente malos . } Mas assi como dize el philosofo enl de . x . qunto libro delas ethicas enl capitulo dela manssedunbre . \\\hline
1.2.12 & quod maxime decet Reges , \textbf{ et Principes esse iustos . } Possumus autem hanc veritatem & e alos prinçipes \textbf{ de ser iustos | et de guardar iustiçia . } Mas esta uerdat podemos prouar \\\hline
1.2.12 & ut sint Reges . \textbf{ Cum enim deceat regulam esse rectam et aequalem , } Rex quia est quaedam animata lex , & enpero non son dignos de seer Reyes . \textbf{ Ca commo conuenga ala regla de ser derecha } e egual e el Rey sea vna ley animada e vna regla . \\\hline
1.2.12 & quam ex aliis virtutibus moralibus . \textbf{ Decet ergo Reges et Principes esse iustos , } tum quia debent esse regula agendorum , & que a ninguno de los otros . \textbf{ Et pues que assi es conuiene alos Reyes | e alos prinçipes de ser iustolo vno } por que deue ser regla de todas las cosas \\\hline
1.2.16 & 3 Ethic’ quatuor rationibus , \textbf{ detestabilius esse intemperatum , } quam timidum . & por quatro razones \textbf{ que mas de deno star es el omne | por non ser tenpra } et do que por ser temeroso \\\hline
1.2.16 & quare exprobrabilius est \textbf{ nos esse intemperatos , } quam non esse fortes . & por la qual cosa mas de denostar son los omes \textbf{ por non ser tenprados } que por non ser fuertes . \\\hline
1.2.16 & cuius est aliis dominari , \textbf{ esse bestialem et seruilem : } indecens est ipsum esse intemperatum . & que ha de \textbf{ enssennorear alos otros de ser bestial e sieruo } e non es cosa conuenible \\\hline
1.2.16 & Si ergo indecens est Regem \textbf{ esse puerum moribus , } et non sequi rationem , sed passionem : & Et por ende si non es cosa conueniente \textbf{ de ser el Rey moço en costunbres } e de non segnir razon \\\hline
1.2.18 & quod indecens sit \textbf{ eos esse auaros . } Nam si regnum alicuius debet & quanto son de denostar los Reyes \textbf{ e los prinçipessi fueten auarientos | e que non les conuiene en ninguna manera delo ser . } Ca si el regno de cada vno dios Reyes deue ser natural \\\hline
1.2.18 & Nam si regnum alicuius debet \textbf{ esse naturale , } proportionari debet his quae videmus in natura . & Ca si el regno de cada vno dios Reyes deue ser natural \textbf{ e bien ordenado | deue ser proporçionado } e conparado a aquellas cosas \\\hline
1.2.18 & quod si possent esse prodigi , \textbf{ melius esset eos esse prodigos quam auaros . } Primo enim , quia melius est & que si pueden ser gastadores \textbf{ meiores ser gastadores | que seer auarientos } ¶La primera razon por que es esta \\\hline
1.2.18 & et quod melius esset \textbf{ ipsum esse prodigum , quam auarum . } Secundo hoc idem patet : & que meior les es \textbf{ de ser gastadores que ser auarientos . } ¶ La segunda razon paresçeassi . \\\hline
1.2.18 & Viso quod quasi impossibile est \textbf{ Reges esse prodigos , } et quod omnino detestabile est & Et por ende muy de depostar es el Rey si fuer auariento ¶ visto \textbf{ que los Reyes non pueden ser gastadores } e que muchon son de denostar \\\hline
1.2.18 & quod deceat \textbf{ eos esse largos , liberales , et communicatiuos . } Haec enim virtus , & que conuiene alos Reyes \textbf{ de ser largos liberales e dadores . } Ca esta uirtud que es en las espenssas \\\hline
1.2.18 & Quare si indecens est Reges , \textbf{ et Principes esse seruos , } conueniens est eos esse liberales . & por la qual razo si non conuiene alos Reyes \textbf{ e alos prinçipes de ser sieruos } conuiene les de ser liberales e francos . \\\hline
1.2.19 & Dicebatur supra , \textbf{ circa sumptus duplicem esse virtutem . } Unam , quae respicit moderatos sumptus , & ssi commo es dichon de suso \textbf{ çerca las espenssas han de ser dos uirudes } ¶La vna que cata alas espenssas mesuradas \\\hline
1.2.19 & Dicebatur enim supra \textbf{ non esse liberalitatem } in multitudine datorum . & Ca ya dixiemos de ssuso \textbf{ que la liberalidat non ha de ser en muchedunbre de donadios ¶ } Pues que assi es el liberal \\\hline
1.2.20 & omnino detestabile est \textbf{ Regem esse paruificum . } Quod autem deceat & en todas maneras es de denostar \textbf{ que el Rey sea periufico mas que conuengaal Rey de ser magnifico } e de fazer grandes espenssas \\\hline
1.2.21 & quod magnificus est excellenter liberalis . \textbf{ Idem est enim esse magnificum , } quod esse abundanter liberalem . & que el magnifico es conplidamente liberal \textbf{ por que essa misma cosa es ser magnifico } que ser conplidamente libal e franço . \\\hline
1.2.21 & Idem est enim esse magnificum , \textbf{ quod esse abundanter liberalem . } Nam ( ut infra patebit ) & por que essa misma cosa es ser magnifico \textbf{ que ser conplidamente libal e franço . } Ca assi commo paresçera adelante la \\\hline
1.2.21 & si facere decentes sumptus est \textbf{ esse liberalem , } facere maximos decentes sumptus , & que faze conuenibles espenssas enlas grandes obras \textbf{ si faz conuenibles espenssas faz omne ser liberal fazer muy grandes } e muy conuenibles espenssas \\\hline
1.2.21 & quos facit magnificus , \textbf{ est esse maxime liberalem . } Sexta proprietas magnifici , & e muy conuenibles espenssas \textbf{ lo que faze el magnifico es ser mucho mas liberal ¶ } La sexta propiedat es del magnifico \\\hline
1.2.23 & Bene autem se habere circa ea , \textbf{ est non esse amatorem periculorum , } neque se exponere & Mas auer se bien çerca los periglos \textbf{ es non ser amador de los periglos } nin esponer su cuerpo a periglos \\\hline
1.2.23 & et multum appreciatur opera virtutum . \textbf{ Propter quod , quia esse plurimum retributiuum , } est agere opera virtutum , & e mucho preçia las obras delas uirtudes \textbf{ Por la qual cosa seer mucho partidor | e dador de los galardones } es fazer obras de uirtudes . \\\hline
1.2.23 & Tertio spectat \textbf{ ad ipsum esse paucorum operatiuum . } Dictum est enim magnanimum esse circa magna , & almagnanimo ser mucho partidor e dador de gualardones ¶ \textbf{ La tercera propiedat es que pertenesçe al magnanimo ser de pocasobras . } por que dicho es de suso que el magnanimo ha de seer \\\hline
1.2.23 & talia autem non multotiens occurrunt , \textbf{ ideo decet magnanimum esse paucorum operatiuum . } Quarto decet magnanimum esse apertum , & Et tales cosas commo estas non contesçen muchas vezes . \textbf{ Et por ende conuiene al magnanimo ser de pocas obras ¶ } La quarta propiedat es que conuiene al maguanimo ser magnifiesto \\\hline
1.2.23 & ideo decet magnanimum esse paucorum operatiuum . \textbf{ Quarto decet magnanimum esse apertum , } ut sit veridicus , & Et por ende conuiene al magnanimo ser de pocas obras ¶ \textbf{ La quarta propiedat es que conuiene al maguanimo ser magnifiesto } assi que sea uerdadero \\\hline
1.2.23 & Secundo decet Reges , \textbf{ et Principes esse plurimum retributiuos : } quia quanto sunt in altiori gradu quam alii , & ¶Lo segundo conuiene alos Reyes \textbf{ e alos prinçipes de ser | partidore ser muy gualardonadores } por que i quanto mas en alto grado estan que los otros \\\hline
1.2.23 & et in retributionibus debent alios superare . \textbf{ Tertio decet eos esse paucorum operatiuos : } negocia enim ardua pauca sunt respectu aliorum . & e enpartimientos de gualardones ¶ \textbf{ Lo terçero çonuiene alos Reyes de seer de pocas obras } A por que los negoçios muy altos son pocos \\\hline
1.2.23 & quantumcumque modica expedire per seipsos , \textbf{ nec decet eos omnium esse operatiuos ; } sed ut possit liberius intendere expeditioni negociorum magnorum & mayormente los que son pequa nons \textbf{ nin conuiene aellos de seer obradores de todas las cosas } mas por que puedan mas liberalmente entender a desenbargar los grandes negoçios que son pocos deuena comne dar los otros negoçios \\\hline
1.2.23 & Quarto decet esse apertos , \textbf{ ut esse veridicos ; } cum sint regula aliorum , & Lo quarto conuiene alos Reyes \textbf{ de seer manifiestos e claros e seer uerdaderos } por que son regla de los otros \\\hline
1.2.23 & et falsificari non debet . \textbf{ Decet etiam eos esse manifestos oditores , et amatores , } ut manifeste odiant vitia , & La qual regla non se deue torcer nin falssar \textbf{ Et ahun conuiene les de seer manifiestos aborresçedores e amadores } por que manifiesta miente \\\hline
1.2.23 & ubi agetur de regimine Regni , \textbf{ non decet eos esse plangitiuos , } vel deprecatiuos pro exterioribus bonis . & o determinaremos del gouernamiento del regno \textbf{ que non conuiene alos Reyes de seer lloradores nin rogadores } por los bienes de fuera . \\\hline
1.2.23 & Omnes ergo assignatae proprietates competere debent Regibus et Principibus . \textbf{ Quare decet eos esse magnanimos . } Amatores honorum aliquando vituperantur , & deuen part enesçer alos Reyes \textbf{ e alos prinçipes . | por la qual cosa les conuienea ellos de seer magnanimos . ¶ } euedes saber que los amadores de las honrras \\\hline
1.2.24 & Sicut ergo decet Reges \textbf{ et Principes esse magnificos , et liberales : } sic decet eos esse magnanimos , & Et pues que assi es assi commo conuiene alos Reyes \textbf{ e alos prinçipes de seer magnificos e liberales } assi en essa misma manera les conuiene de seer magranimos \\\hline
1.2.24 & et Principes esse magnificos , et liberales : \textbf{ sic decet eos esse magnanimos , } et honoris amatiuos . Reges enim et Principes decet honores diligere modo quo dictum est ; & e alos prinçipes de seer magnificos e liberales \textbf{ assi en essa misma manera les conuiene de seer magranimos } e amado res de honrra . \\\hline
1.2.24 & bene dictum est , \textbf{ quod decet eos esse magnanimos , } et honoris amatiuos . & Bien dicho es \textbf{ que a ellos pertenesce seer magnanimos } e amadores de honrra \\\hline
1.2.25 & et huiusmodi virtutes morales , \textbf{ esse habent circa animi passiones . } In tantum enim dictae virtutes & e las otras uirtudes morales \textbf{ han de seer cerca las passiones del alma . } Ca en tanto las uirtudes son dichas bien \\\hline
1.2.26 & quare si decet Reges \textbf{ et Principes esse magnanimos , } decet eos esse humiles . & por la qual cosa se conuiene alos Reyes \textbf{ e alos prinçipes de seer magranimos } conuiene les de ser humildosos . \\\hline
1.2.27 & sed etiam quodammodo naturale est nobis \textbf{ appetere punitionem ultra condignum . } Nam quia malum nobis illatum videtur nobis maius esse , & a aquellos que nos fazen alguons males . \textbf{ Mas avn en alguna manera natural cosa esa nos de dessear de ser vengados dellos mas que deuemos . } Ca por que el mal que ellos nos fazen paresçe a nos \\\hline
1.2.27 & mansuetudo nominat temperamentum irae . \textbf{ Quod autem deceat Reges et Principes esse mansuetos , } ostendere non est difficile . & La manssedunbre nonbra tenpramiento de sana . \textbf{ mas mostrar que conuiene alos Reyes | e alos prinçipes de ser manssos } esto non es cosa guaue mas ligera . \\\hline
1.2.27 & Nam cum ira peruertat iudicium rationis , \textbf{ non decet Reges et Principes esse iracundos , } cum in eis maxime vigere debeat ratio et intellectus . Sicut enim videmus & tristorna el iuyzio dela razon \textbf{ e del entendimiento non conuiene alos Reyes | et alos prinçipes de seer sannudos } por que en ellos mayormente deue seer apoderada la razon e el entendemiento \\\hline
1.2.27 & quanto magis spectat ad ipsos \textbf{ esse custodes iustitiae , } et conseruatores Reipublicae . & e fazer uengancas \textbf{ quanto mas pertenesçe a ellos de seer guardadores dela iustiçia } et mantenedores dela comunidat . \\\hline
1.2.29 & Quid enim aliud est mentiri , \textbf{ nisi non esse apertum , } et non ostendere se talem , & muestranabiercamente tales quales son estos son mucho de reprehender \textbf{ por que vn mentires non ser el omne manifiesto } nin se mostrar tal qual es . \\\hline
1.2.29 & Sciendum ergo quod licet \textbf{ affirmare in se esse quod non est , } vel negare quod est , & Et pues que assi es conuiene saber \textbf{ que maguera firmar cada vno de ser en ssi aquello que non es o negar } aquello que es en ssi sea mentira \\\hline
1.2.29 & sit mentiri : \textbf{ tamen non dicere totum quod est , } absque mendacio fieri potest . & aquello que es en ssi sea mentira \textbf{ Empero non dezir todo aquello que es en el puede ser sin mentira . } Pues que assi es el que quiere ser uerdadero \\\hline
1.2.29 & nec debet concedere \textbf{ in se esse malitiam , } quam sibi inesse non credit . & nin deue \textbf{ ototgarde ser en ssi maliçia } la qual non ha \\\hline
1.2.29 & aut nihil opere implent . \textbf{ Tales autem quia patet esse iactatores et albos , } non apertos et veraces , & delas quales cosas muy poco o ninguna cosa non cunplen por obra . \textbf{ Et tales paresçen seer alabadores e encubridores } e non manifiestos nin uerdaderos . \\\hline
1.2.29 & sed gratiosos et amabiles : \textbf{ decet eos non esse iactatores vel derisores , } sed apertos et veraces , & nin guaues mas guatiosos e amables . \textbf{ Conuiene a ellos de non seer alabadores de ssi mismos } nin escarnidores dessi mas manifiestos \\\hline
1.2.30 & et Principes decet \textbf{ esse iocundos . } Ut enim ex habitis patere potest , & e alos prinçipes \textbf{ de ser alegres e iugadores . } Ca assi commo puede paresçer delas cosas \\\hline
1.2.31 & quia ( ut plane patet ) \textbf{ oportet eos esse prudentes et iustos : } omnino manifestum esse debet , & Ca assi commo \textbf{ parescellanamente conuiene a ellos de seer pradentes e sabios e iustos . } Et por ende en todo en todo deue ser manifiesto \\\hline
1.2.31 & Virtutes autem naturales et imperfectae , \textbf{ non oportet esse connexas . } Videmus enim aliquos naturaliter habere & e non conplidas \textbf{ non conuiene de ser conexas | nin ayuntadas vna a otra } por que veemos alguons naturalmente auer alguna n industria \\\hline
1.2.31 & si non debite tendamus in bonum finem . \textbf{ Volunt enim aliqui esse distributores bonorum , } et proponunt sibi , & si non entendieremos conueniblemente en buena fin . \textbf{ por que algunos quieren ser partidores de los bienes } e ponen assi en logar de fin obras de largueza e de franqueza . \\\hline
1.2.31 & Quare sic decet Reges , \textbf{ et Principes esse quasi semideos , } et habere virtutes perfectas : & sin todas las otras . \textbf{ Por la qual cosa si conuiene alos Reyes e alos prinçipes de ser } assi commo medios dioses \\\hline
1.2.32 & In hoc ergo gradu debent esse Reges et Principes . \textbf{ Et si in tale gradu esse decet bonos et perfectos Principes seculares } quales esse debeant & e alos prinçipes de ser bueons . \textbf{ Et si en tal grado de buenos | conuiene alos prinçipes seglares de ser buenos } e acabados \\\hline
1.2.33 & continentes , et temperatos : \textbf{ sed decet eos quodammodo esse diuinos . } Virtutes ergo competentes eis , & e continentes \textbf{ e tenprados mas conuienel es de ser | en algunan manera diuinales } Et por ende las uirtudes que parten esten aellos \\\hline
1.2.33 & quia ipsi aliorum debent \textbf{ esse regula et exemplar . } Tantae enim bonitatis debet esse Princeps , & pueden ser dichas exenplares \textbf{ por que ellos deuen ser regla e exenplario de los otros } Ca de tanta bondat deue ser el prinçipe \\\hline
1.3.2 & et odium , \textbf{ esse habent desiderium , et abominatio . } Nam desiderium immediate innititur amori , & por que ama o por que aborresçe . Mas despues del amor o dela mal querençia \textbf{ han de ser el desseo e la aborrençia . } Ca el desseo sin ningun medio se ayunta luego al amor . \\\hline
1.3.3 & quomodo deceat Reges \textbf{ et Principes esse amatiuos , et oditiuos . } Et ut ostendamus nomen amoris , & en qual manera conuiene alos Reyes \textbf{ et alos prinçipes de ser amadores e de ser mal queredores . } Et por que entendamos la fuerca del amor \\\hline
1.3.3 & Dilectatio enim quam habebant Romani \textbf{ ad Rempublicam fecit Romam esse principantem et monarcham . } Hoc ergo modo quoslibet homines decet esse amatiuos , & Ca el amor que auian los romanos al bien comun \textbf{ e publicofizo a Roma ser sennora | e auer sennorio en todo el mundo . } Pues que assi es que esto conuiene a todos los omes de ser amadores \\\hline
1.3.3 & ad Rempublicam fecit Romam esse principantem et monarcham . \textbf{ Hoc ergo modo quoslibet homines decet esse amatiuos , } ut primo et principaliter diligant & e auer sennorio en todo el mundo . \textbf{ Pues que assi es que esto conuiene a todos los omes de ser amadores } assi que primero e prinçipalmente amen el bien diuinal \\\hline
1.3.3 & si considerentur virtutes , \textbf{ quibus decet Reges esse ornatos . } Sicut enim detestabilius est & ¶Lo segundo esto mesmo se praeua assi si pensaremos las uirtudes \textbf{ por las quales conuiene alos reyes de ser honrrados . } Ca assi commo es \\\hline
1.3.3 & quia secundum Philosophum 4 Ethic’ magnificentia potissime habet \textbf{ esse circa diuina , et communia . } Erit fortis ; quia cum bonum cumune proponat bono priuato , & en el quarto libro delas ethicas \textbf{ la magnificençia | prinçipalmente ha de ser çerca los bienes diuinales e comunes . } Otrosi sera fuerte por que ante pone el bien comunal bien propreo \\\hline
1.3.3 & quibus decet Reges \textbf{ et Principes esse ornatos , } et principaliter debent diligere & Et pues que assi espenssando las uirtudes \textbf{ por las quales deuen ser los Reyes honrrados } prinçipalmente deuen ellos amar el bien diuinal \\\hline
1.3.5 & Dicebatur enim supra , \textbf{ quod eos esse decet humiles et magnanimos : } cum ergo humilitas moderet spem , & Ca dixiemos de suso \textbf{ que conuenia alos Reyes de ser humildosos e de ser mag̃nimos . } Et por ende por que la humildat tienpra la esperança \\\hline
1.3.5 & ut supra ostendimus , \textbf{ debet esse bonum diuinum et commune , } cum talia sint bona excellentia et ardua , & e del rey \textbf{ assi commo mostramos de suso deue ser el bien diuinal e comunal } por que tales bienes son bienes mas sobrepiunates \\\hline
1.3.6 & Videtur autem forte aliquibus Reges , \textbf{ et Principes in nullo debere esse timidos , } quia talia regiae maiestati derogare dicuntur . & Mas por auenta parescrie a alguno que los reyes \textbf{ e los prinçipes en ninguna cosa non deuen ser temerosos } por que tales cosas \\\hline
1.3.6 & Oportet ergo videre \textbf{ quo modo eos esse deceat timidos , et audaces . } Timor autem si moderatus sit , & Et pues que assi es conuiene deuer \textbf{ en qual manera conuiene alos Reyes de sertemosos | e de ser osados } por que el temor si fuere tenprado es conuenible alos Reyes e alos prinçipes . \\\hline
1.3.6 & Ergo si inconueniens est \textbf{ Regem esse tremulentum , } qui debet esse virilis et constans , & e viene les luego el tremer ¶ \textbf{ Et pues que assi es si cosa desconuenible es al Rey de ser tremuliento } la qual cosa deue seruaron costante e firme desconuenible cosa es ael de temer \\\hline
1.3.7 & Sed odium sine tristitia esse potest : \textbf{ nam odium esse valet ad aliquid in communi . } Odire enim possumus & Mas la mal querençia puede ser sintsteza \textbf{ por que la mal querençia puede ser | contra alguna cosa en comun . } Ca nos podemos natanlmente querer mal a todos los ladrones . \\\hline
1.3.8 & quia quod ab omnibus appetitur \textbf{ maxime videtur esse bonum } et eligibile : & e fazia esta razon que aquella cola que es desseada de todos \textbf{ paresçe mucho mas ser buean et escogible que ninguna otra . } Et todas las cosas dessean de se delectar \\\hline
1.3.9 & spes et timor sunt principales passiones respectu irascibilis . \textbf{ Sed cum ex passionibus diuersificari habeant opera nostra , } decet nos diligenter intendere , & en conparacion del appetito enssannador . \textbf{ Mas commo las nr̃as obras ayan de ser departidas | por estas passiones . } Cconuiene a nos de acuçiosamente entender \\\hline
1.3.11 & imitari debent . \textbf{ Decet enim eos esse gratiosos , et misericordes . } Ipsi enim maxime esse debent & en quanto son passiones de loar \textbf{ Por que conuiene a ellos de ser guaçiosos e mis cordiosos . } Ca ellos deuen ser muy conuenibles partidores de los bienes \\\hline
1.3.11 & eius autem non est praua operari . \textbf{ Quare si decet Reges esse studiosos , } et esse senes moribus , & por que la uerguença es delas cosas malas . Mas al estudioso non conuiene obrar ningunas cosas malas . \textbf{ Por la qual cosa si conuiene alos Reyes | e alos prinçipes de ser estudiosos } e de ser uieios en constunbres \\\hline
1.4.1 & et bonae spei . \textbf{ Tertio contingit eos esse magnanimos , } cuius causa ex praecedentibus assignatur . & contesce aellos de ser de grand coraçon e de grant esperança \textbf{ ¶Lo terçero contesçe | a ellos de ser de grand coraçon } por la razon que ya es dicha e mostrada de ssuso . \\\hline
1.4.1 & non tamen est laudabile in homine . \textbf{ Sic , licet uerecundari sit laudabile in iuuenibus , } quia ratione aetatis se continere non possunt & Empero non es de loar en el omne \textbf{ En essa misma manera maguer ser | uergonçoso sea de loar en los mançebos } por razon delan hedat \\\hline
1.4.1 & Sextum autem , \textbf{ videlicet , esse verecundos , } non decet simpliciter competere Regibus et Principibus . Decet enim Reges et Principes esse liberales : & e de ser mibicordiosos . \textbf{ Mas la sexta condicion conuiene a saber ser uergoncosos . } Esta non conuiene nin pertenesçe \\\hline
1.4.1 & Decet etiam eos \textbf{ tanto magis esse bonae spei quam alios , } quanto facta communia & ¶avn conuiene alos Reyes en \textbf{ tantod ser de buean esperança | que los otros en quanto los fechos comunes } çerca los quales ellos se deuen trabaiar son mas dignos que los otros ¶ \\\hline
1.4.2 & Ideo de omnibus respondent , \textbf{ et volunt videri omnia scire , } et omnia affirmant & Et por ende de todas las cosas responden \textbf{ e quieren ser uistos | que saben todas las cosas } e todas las cosas afirman \\\hline
1.4.2 & Indecens enim est Reges et Principes \textbf{ esse passionum insecutores , } et venereorum habere concupiscentias vehementes : & que deuen lercabesca e regla de todos los otros . \textbf{ Et por ende cosa desconuenible es alos Reyes de ser segnidores delas passiones } e de auer cobdiçias afincadas de lux̉ia \\\hline
1.4.2 & immo decet \textbf{ eos esse firmos et stabiles . } Tertio indecens est & e se tristor nen . \textbf{ Mas conuiene aellos de ser firmes e estables ¶ } Lo terçero cosa desconuenible es alos Reyes \\\hline
1.4.3 & et non curant reputari : \textbf{ quare contingit eos esse inuerecundos . } Posset autem una causa assignari , & e non han cuydado \textbf{ que los otros les tengan en much̃ . | Et por esta razon aceesçe alos uieios de ser desuergoncados } Mas puede aqui ser fallada vna razon que es a comun a todas estas cosas de suso dichͣs . \\\hline
1.4.3 & non tamen decet \textbf{ eos esse incredulos , } sed consideratis conditionibus personarum , & assi conmo los moços . \textbf{ Enpero non les conuiene aellos de ser incredulos } e non creyentes del todo . \\\hline
1.4.3 & Secundo non decet \textbf{ eos esse suspitiosos , } ut omnia referant & e segunt iuzio de entendemiento ¶ \textbf{ Lo segundo non conuiene alos Reyes de ser sospechosos } assi que to das las cosas retuercan ala peor parte \\\hline
1.4.3 & in deteriorem partem : \textbf{ quia ex hoc contingeret eos esse seueros et immiseratiuos , } et incurrerent maliuolentiam subditorum . & assi que to das las cosas retuercan ala peor parte \textbf{ Ca desto se les leunataria a ellos ser crueles | e non misi cordiosos } e caerien en malquerençia de los sus subditos ¶ \\\hline
1.4.3 & et incurrerent maliuolentiam subditorum . \textbf{ Tertio non decet eos esse timidos et pusillanimes , } immo fortes et magnanimos : & e caerien en malquerençia de los sus subditos ¶ \textbf{ Lo terçero non conuiene a ellos de ser tem̃osos e de flacos coraçones } mas conuiene les de ser fuertes e de grandes coraçones \\\hline
1.4.3 & sint magna et ardua , \textbf{ oportet eos esse fortes et magnanimos . } Quarto detestabile est & e los prinçipes se deuen trabaiar son grandes e altos \textbf{ por ende conuiene les aellos de ser fuertes | e de grandes coraçones } ¶ \\\hline
1.4.3 & Sexto non decet \textbf{ eos esse inuerecundos } eo modo quo senes existunt : & e assi el regno serie en periglo ¶ \textbf{ Lo sexto non conuiene aellos de ser desuergonçados } en aquella manera \\\hline
1.4.4 & Nam totum regnum ad dilectionem Regis prouocatur , \textbf{ si viderit ipsum esse clementem , et misericordem . } Tertio , non decet Reges , & Ca todo el regno se mueue a amor del Rey \textbf{ sil | vieren ser piadoso e miscderioso . } ¶ Lo terçero non conuiene alos Reyes \\\hline
1.4.5 & affectant magna , \textbf{ et contingit eos esse magnanimos . } Secundo nobiles non solum sunt magnanimi & por que lemeien a sus parientes dessean grandes cosas \textbf{ e acahesçe les ser de grandes coraçones } ¶Lo segundo son los nobles non solamente magnanimos e de grand coraçon . \\\hline
1.4.5 & ut faciant magna , \textbf{ sequitur nobiles esse magnificos , } etiam magis quam parentes : & para fazer grandes cosas siguese \textbf{ que los nobles han de ser magnificos } e ahun mas los fijos que los padres \\\hline
1.4.5 & ut vult Philos’ 2 de Anima : \textbf{ contingit nobiles habere mentem aptam , } et esse dociles et industres , & en el segundo del alma contesçe \textbf{ alos nobles de auer el alma mas apareiada e de ser ellos mas enssennados e mas engennosos que los otros } por que en ellos es la buean \\\hline
1.4.5 & contingit nobiles habere mentem aptam , \textbf{ et esse dociles et industres , } quia in eis viget carnis mollicies , & en el segundo del alma contesçe \textbf{ alos nobles de auer el alma mas apareiada e de ser ellos mas enssennados e mas engennosos que los otros } por que en ellos es la buean \\\hline
1.4.5 & esse magna societas , \textbf{ conuenit eos esse politicos et sociales , } quia ut plurimum in societate vixerunt . & Ca porque en la mayor parte en las cortes delos nobles \textbf{ es acostunbrado de auer grandes con p̃anas acahesçeles de ser corteses e aconpanables } por que en la mayor parte visquieron en conpanina de buenos . \\\hline
1.4.5 & quomodo decet Reges , \textbf{ et Principes esse magnanimos , quomodo magnificos , } quomodo prudentes et dociles , & en qual manera conuiene alos Reyes de ser magranimos \textbf{ e en qual mauera de ser magnificos } e en qual manera sabios e ensseñados \\\hline
1.4.5 & Esse autem elatum , \textbf{ et despicere suos progenitores , } et nimis esse honoris cupidi , & por que sienpre es mas antigua¶ \textbf{ Mas ser sobrauios e despreçiar los sus engendradores } e ser muy cobdiçiosos de honrra \\\hline
1.4.6 & Videtur enim eis , \textbf{ diuitias esse tantum bonum , } quod quicunque hoc bono affluit , & que las aya \textbf{ conplidamente es diguno de ser sennor e prinçipe . } Et pues que assi es todas estas costunbres \\\hline
1.4.6 & eo quod decipiantur circa diuitias , \textbf{ credentes eas esse maius bonum quam sint . } Decet ergo Reges , & por tanto que son engannados en las riquezas \textbf{ por que creen | que ellos son el mayor bien que puede ser . } Et por ende conuiene alos Reyes \\\hline
1.4.6 & Nec propter hoc credere debent , \textbf{ se esse dignos principari . } Nam dignitas principatus principaliter & nin por esso non deuen creer \textbf{ que son dignos de ser sennores . } Ca la dignidat del prinçipado e del sennorio \\\hline
1.4.7 & Differunt ergo esse nobilem , \textbf{ et esse diuitem . } Differt etiam esse nobilem , & Pues que assi es diferençia ay \textbf{ entre ser noble e ser rico } e ahun disferençia ay entre ler noble e rico \\\hline
1.4.7 & et esse diuitem , \textbf{ ab esse potentem . } Nam dicitur aliquis esse potens , & e ahun disferençia ay entre ler noble e rico \textbf{ e entre ser poderoso . } Ca por tanto es dicho alguno poderoso \\\hline
1.4.7 & et non posse principari , \textbf{ non est idem esse nobilem , } et esse potentem . & e non pueden ser prinçipes \textbf{ por ende non es vna cosa ser noble e ser poderoso . } Otrossi veemos otros algunos \\\hline
1.4.7 & non est idem esse nobilem , \textbf{ et esse potentem . } Rursus videmus aliquos abundare argento et auro , & e non pueden ser prinçipes \textbf{ por ende non es vna cosa ser noble e ser poderoso . } Otrossi veemos otros algunos \\\hline
1.4.7 & quia nulli reguntur sub eius imperio ; \textbf{ quare non est idem esse diuitem , } et esse non potentem . & ca ninguons non se gouiernan so su sennorio . \textbf{ Por la qual cosa non es vna cosa ser el omne rico e ser toderoso¶ } Visto quales son las costunbres delons nobłs \\\hline
1.4.7 & quare non est idem esse diuitem , \textbf{ et esse non potentem . } Viso ergo & ca ninguons non se gouiernan so su sennorio . \textbf{ Por la qual cosa non es vna cosa ser el omne rico e ser toderoso¶ } Visto quales son las costunbres delons nobłs \\\hline
1.4.7 & circa opera potentatus , \textbf{ quae debent esse bona et studiosa . } Secundo potentes sunt & por que son costrintados de entender çerca las obras del poderio \textbf{ que deuen ser buenas e estudiosas ¶ } Lo segundo los poderosos son mas tenprados que los ricos \\\hline
1.4.7 & videlicet , diuitiis , nobilitate , potentia : \textbf{ decet eos esse eruditos , et temperatos . } Nam per nobilitatem & que son rriquezas nobleza e poderio . \textbf{ Conuiene les de ser enssennados e tenprados } ca por la nobleza \\\hline
1.4.7 & ( ut superius dicebatur ) \textbf{ debent esse exemplar , } et regula aliorum , & Ca ellos segunt diches de suso \textbf{ deuen ser exienplo } e forma e regla de bien beuir a todos los otros \\\hline
2.1.1 & si de domo determinare volumus , \textbf{ videndum est quomodo se habeat homo adesse communicatiuum , et sociale . } Sciendum igitur , & conuiene nos de ver \textbf{ commo se deue auer el omne | para ser comunal con todos e conpanenro . } Et por ende conuiene de saber \\\hline
2.1.3 & ex pariete , tecto , et fundamento : \textbf{ vel nominare potest familiam in ea contentam . } Sicut et ciuitas aliquando muros , & paredesçimientos \textbf{ o en otra manera puede ser dichͣ la conpanna | que mora en aquella casa . } Ca bien commo la çibdat algunas uezes nonbralos muros \\\hline
2.1.5 & in esse conseruari . \textbf{ Hoc ergo modo hae duae communitates faciunt domum esse quid naturale : } quia communitas viri et uxoris ordinatur ad generationem , & si en ninguna manera non pudiesse ser conseruada en su ser . \textbf{ Et pues que assi es en esta manera estas dos comuindades | fazen ser la casa cosa natural . } Ca la comunidat del uaron e dela mugnies ordenada ala generacion . \\\hline
2.1.6 & quod oportet \textbf{ in domo perfecta esse tria regimina . } Nam nunquam est dare communitatem & Et destas cosas puede paresçer \textbf{ que en la casa acabada | deuen ser tres gouernamientos . } Ca nunca podemos dar comunindat ninguna bien ordenada \\\hline
2.1.6 & secundum quod dominus praeest seruis . \textbf{ Viso , in domo perfecta esse communitates tres , } et tria regimina : & segunt que el sennor deue ser antepuesto alos sieruos \textbf{ Visto commo en la casa acabada | ian de ser tres comuindades } e tres gouernamientos de ligero puede paresçer \\\hline
2.1.6 & ibi debere \textbf{ esse sex genera personarum , } ita quod prima persona sit & Empero podrie paresçer a alguno por auentura que deuen y ser seys linages de perssonas \textbf{ alssi que la primera perssona deue ser ꝑ el uaron¶ } La segunda dela muger ¶ \\\hline
2.1.8 & vel ex parte amicitiae naturalis , \textbf{ quae debet esse inter virum et uxorem . } Secunda vero ex parte prolis . & o de parte dela amistança natural \textbf{ que deue ser entre el marido e la muger . } ¶ La segunda razon se toma de parte dela generaçion de los fijos \\\hline
2.1.9 & Sed quod recta ratio dictat quoslibet ciues , \textbf{ et maxime Reges et Principes unica coniuge debere esse contentos , } triplici via venari possumus . & que cada vno de los çibdadauos \textbf{ e mayormente los Reyes | e los prinçipes deuen ser pagados de vna muger sola } e esto podemos prouar \\\hline
2.1.9 & quod sit dexter , \textbf{ licet contingat aliquos esse sinistros . } Sic quia ut in pluribus sola foemina & que natural cosa es al omne de ser diestro \textbf{ commo quier que contezca a algunos de ser esquierdos en essa manera } por que en la mayor parte vna sola fenbra \\\hline
2.1.10 & lis et discordia oriretur . \textbf{ Quare decet coniuges omnium ciuium uno uiro esse contentas : } tanto tamen hoc magis decet & e amistança nasçria contienda e discordia . \textbf{ por la qual cosa conuiene alas mugieres de todos los çibdadanos de ser pagadas de vn solo uaron . } Empero tanto o mas conuiene esto \\\hline
2.1.10 & ut eis diligenter prouideant in haereditate et in nutrimento : \textbf{ decet coniuges omnium ciuium uno viro esse contentas ; } tanto tamen hoc magis decet & con grand acuçia en las hedades e en el nudrimiento . \textbf{ Conuiene alas mugers de todos los çibdadanos | de ser pagadas de vn solo uaron . } Et esto tanto conuiene mas alas mugers de los Reyes \\\hline
2.1.12 & Probabatur enim supra , \textbf{ coniugium esse secundum naturam , } eo quod homo naturaliter esse animal sociale : & Ca prouado es de suso \textbf{ que el casamiento deue ser segunt natura } por que el omne naturalmente es aian la conpannable ama termoino \\\hline
2.1.12 & consurgunt dissensiones et bella : \textbf{ quare sicut ad esse sanum requiritur , } quod quis habeat naturam fortem , & entre si sele una tan discordias e peleas . \textbf{ por la qual cosa asi commo para ser el omne sano } conuiene \\\hline
2.1.13 & Ostendebatur supra , \textbf{ quale debet esse coniugium , } quia est quid naturale , & assi que entre los casados sea guardada alguna egualdat \textbf{ ostrado es de sus qual deue ser el casamiento . } Ca es cosa natural \\\hline
2.1.13 & et est unius ad unam . \textbf{ Ostendimus etiam inter quas personas debet esse coniugium , } quia non inter nimia propinquitate coniunctos . & Et que es de vno avna . \textbf{ Et avn mostratemos entre quales perssonas deue ser el casamiento | ca non de ue ser } entre los que lon ayuitados \\\hline
2.1.15 & esse naturaliter barbarum et seruum ; \textbf{ esse enim barbarum ab aliquo , } hoc est , esse extraneum ab eo , & mas vn omne era naturalmente barbaro e sieruo . \textbf{ Ca ser barbaro de alguno este es ser estranno del } porque non es entendido del . \\\hline
2.1.15 & quanto detestabilius est \textbf{ eos esse Barbaros , } et carere ratione et intellectu . & quanto mas de denostares \textbf{ a ellos de ser barbaros } e de ser menguados de razon e de encendemiento . \\\hline
2.1.18 & quicquid tamen sit de eius causis , \textbf{ laudabile est in ipsis esse uerecundas : } quia propter uerecundiam multa turpia dimittunt & Enpero que quier que sea destas razones \textbf{ mucho es de alabar en ellas ser uergon cosas } ca por la uerguença dexan de fazer muchs cosas torpes \\\hline
2.1.18 & Hoc autem tertium \textbf{ et si in bonis potest esse laudabile , } in malis vero est vituperabile . & que pudiessen fazer cosas tan torpes \textbf{ Mas esta terçera cosa maguera pueda ser alabada en los buenos . } Enpero enlos otros es muchos de denostar ¶ \\\hline
2.1.19 & sic suo modo est in ipsis operibus . \textbf{ Videmus enim aliquos habere linguas disertas , } aliquos vero balbutientes esse : & assi en su manera es en las obras propreas . \textbf{ Ca veemos alguon sauer las lenguas escorrechas | Et ueemos algunos ser tartamudos } e los que son tartamudos non son de vna \\\hline
2.1.19 & quando sunt castae , honestae , abstinentes , et sobriae . \textbf{ Decet enim coniuges esse castas } non solum propter fidem seruandam suis viris , & quando son castas e honestas e abstinents e mesuradas . \textbf{ Ca conuiene alas mugieres de ser castas } non solamente \\\hline
2.1.19 & Decet ergo coniuges \textbf{ omnium ciuium esse castas : } et tanto magis hoc decet & Et pues que assi es conuiene a todas las mugers \textbf{ de todos los çibdadanos de ser castas } e tanto mas esto conuiene alas mugers de los Reyes \\\hline
2.1.19 & et cauere sibi ab operibus illicitis : \textbf{ sed oportet eas esse pudicas et honestas , } ut sibi caueant a signis et a verbis , & e se guarden de malas obras . \textbf{ mas conuiene a ellas de ser linpias e honestas } assi que se guarden de señales \\\hline
2.1.19 & Nam cibi superfluitas ad incontinentiam inclinat . \textbf{ Quarto decet eas esse sobrias , } ut caueant sibi a superfluitate potus . & por que la demasia dela uianda inclina mucho ala lux̉ia ¶ \textbf{ Lo quarto conuiene aellas de ser mesuradas } por que se guarden de sobrepuiança de vino \\\hline
2.1.19 & et ad maiorem amorem viros inducunt : \textbf{ decet ergo eas esse taciturnas . } Sic etiam decet esse stabiles : & e then a sus maridos a mayor amor . \textbf{ Et por ende les conuiene de ser calladas } e en essa misma manera avn les conuiene de ser estables e firmes \\\hline
2.1.21 & Decet enim uxorem militis \textbf{ magis esse ornatam vestibus , } quam uxorem ciuis simplicis . & nin demandan uestiduras sobeias . \textbf{ Ca conuiene ala muger del cauallero de ser mas honrrada de uestiduras } que ala muger del çibdadano sinple . \\\hline
2.1.22 & Decet ergo omnes ciues \textbf{ non esse nimis zelotipos de suis coniugibus : } et tanto magis hoc decet Reges et Principes , & pues que assi es conuiene a todos los çibdadanos \textbf{ de non ser muy çelosos de sus mugieres } Et tanto mas esto conuiene alos Reyes \\\hline
2.1.24 & si enim indecens est \textbf{ eas esse ociosas , } ut superius dicebatur , & quales obras son ordenadas las casadas . \textbf{ por que si es cosa desconuenible a ellas de ser uagarosas } assy commo dicho es de \\\hline
2.2.1 & quod decet omnes patres \textbf{ circa proprios filios esse solicitos . } Nam cognita solicitudine , & primeramente queremos mostrar \textbf{ que conuiene a todos los padres de ser muy cuydadosos de los fiios . } Ca penssada la acuçia \\\hline
2.2.1 & decet patres habere curam filiorum , \textbf{ et solicitari erga eos , } ut inueniant eis illa , & e los fijos naturalmente han el ser de los padres . Conuiene alos padres de auer cuydado de los fijos \textbf{ e ser cuydadosos dellos } por que les busquan aquellas cosas \\\hline
2.2.2 & in quo oportet eos alios gubernare ; \textbf{ maxime decet eos esse prudentes et bonos . } Et cum filii perueniunt & en quel conuiene gouernar los otros \textbf{ mucho les conuiene de ser sabios e buenos . } Mas commo los fijos bengan a mayor bondat e a mayor sabiduria \\\hline
2.2.4 & et congregare aliis bona , \textbf{ et solicitari circa eorum vitam , } sit eos regere et gubernare , & Commo allegar les los bienes \textbf{ e ser cuydadosos | çerca la uida } dellos sea gouernar les \\\hline
2.2.4 & in honore et reuerentia : \textbf{ cum honorari et reuereri alium sit } quodammodo subiici illi ; & Commo honrrar \textbf{ e auer reuerençia a otro sea en alguna manera ser subiecto a el . } Por ende assi commo por el amor que han los padres alos fijos los deuen gouernar ben \\\hline
2.2.5 & qui vero mala , in ignem aeternum . \textbf{ Debent ergo omnes ciues solicitari circa proprios filios , } ut ab infantia instruantur in hac fide ; & assi que aquellos que bien fezieron yran ala uida perdurable . \textbf{ Mas aquellos que fizieron mal yran al fuegon del infierno ¶ Et pues que assi es todos los çibdadanos deuen ser acuçiosos de sus fiios } por que enla moçedat sean enssennados en esta fe \\\hline
2.2.6 & sunt instruendi ad bonos mores , \textbf{ et debent eis fieri monitiones debitae . } Secunda via ad inuestigandum hoc idem , & estonçe son de enssennar en bueans costunbres \textbf{ e deuen les ser fechos amonestamientos conuenibles . } ¶ La segunda razon para prouar esto mismo se toma del fallesçimiento dela razon \\\hline
2.2.6 & hoc modo docet \textbf{ nos dirigere ad bonos mores , } quo dirigitur virga tortuosa . & Onde el philosofo çerca la fin del segundo libro delas ethicas \textbf{ en esta manera nos muestra ser endereçados a bueans costunbres } en la qual manera se enderesça la piertega tuerta . \\\hline
2.2.7 & literales sermones scire distincte proferre . \textbf{ Decet etiam eos esse attentos , } et studiosos circa ea , & que aprendan pronunçiar departidamente las palabras delas letras ¶ \textbf{ Et avn conuiene les de ser acuçiosos } e estudiosos çerca aquellas cosas \\\hline
2.2.8 & Nam cum oporteat \textbf{ eos esse quasi semideos , } et debite et absque negligentia & e mayormente delos Reyes e de los prinçipes . \textbf{ Ca commo les conuenga a ellos de ser } assi commo medios dioses e de entender conueinblemente \\\hline
2.2.8 & ipsos bene se habere circa diuina , \textbf{ et esse instructos et firmos in fide , } et illas scientias scire , & mucho les conuiene aellos de se auer bien cerca las cosas diuinales \textbf{ e ser enssennados e firmes en la fe } e conuiene les de saber aquellas sçiençias \\\hline
2.2.9 & Quare sicut doctor est in speculabilibus , \textbf{ sic decet esse diligentem et cautum , } ut proponat suis auditoribus vera & Por la qual cosa \textbf{ assi commo conuiene al doctor e al maestro en las sçiençias especulatiuas de ser acuçioso e sabio } en manera que proponga a sus disçipulos cosas uerdaderas \\\hline
2.2.9 & Quantum autem ad vitam , \textbf{ decet ipsum esse honestum , et bonum . } Si ergo Reges et Principes & conuiene le que el doctor sea menbrado e prouado e sabio e acatado . \textbf{ Mas quanto ala uida deue ser honesto e bueno . } ¶ Et pues que assi es si los Reyes e los prinçipes \\\hline
2.2.12 & ex sumptione cibi : \textbf{ sed etiam decet eos esse sobrios , } ut non efficiantur ebrii & non solamente que se non fagan gollosos por el comͣ \textbf{ mas avn les conuiene de ser mesurados } que non se fagan beodos \\\hline
2.2.12 & sit contra rationis dictamen , \textbf{ quia decet patrem sic solicitari erga filios , } ut sint virtuosi , & sean contra ordenamiento de \textbf{ razon conuiene al padre de ser acuçioso çerca de sus tijos } por que non puedan ser malos nin viçiosos . \\\hline
2.2.13 & quilibet motus membrorum , \textbf{ ex quibus iudicari possunt motus animae . } Videmus enim prudentes et bonos habere & Los gestos son dichos quals si quier mouimientos de los mienbros \textbf{ por los quales pueden ser iudgados los mouimientos del alma } Ca veemos que los sabios \\\hline
2.2.16 & usque ad septem annos sic esse regendos : \textbf{ a septimo usque ad decimumquartum sic esse instruendos , } huiusmodi septennia sunt abbreuianda et elonganda & fasta los que torze \textbf{ assi deuian ser enssennados . } Et estos setenarios son de encortar o de al ougar \\\hline
2.2.17 & tres breues rationes , \textbf{ quare decet filios esse subiectos , } et obedire senioribus , et patribus . & razono breues \textbf{ por que conuiene alos fijos de ser lubiectos } e obedientes a lus padres e alos vieios . \\\hline
2.2.18 & attamen quia decet \textbf{ eos esse magis prudentes quam bellatores , } filii Regum et Principum & nin por otra uentra a qual si quier non osen tomar armas . \textbf{ Enpero por que mas conuiene de ser sabios } que lidiadores alos fijos de los Reyes \\\hline
2.2.20 & nisi ex aliquibus exercitiis sensibilibus delectationem sumant . \textbf{ Qualia autem debent esse opera , } circa quae mulieres insudare decet , & çonnen alg̃s obras senssibło \textbf{ mas quales deuen ser las obras } cerca las quales conuiene alas mugers de \\\hline
2.2.21 & propter quod iudicentur litigiosae , et discolae : \textbf{ quare decet ipsas esse taciturnas , } ne in verba litigiosa prorumpant . & e por desacordadas \textbf{ por la qual razon les | conuiene a ellas de ser callantias } e que non se entremetan \\\hline
2.3.3 & qui debent esse nobiles et praeclari , \textbf{ potissime decet esse magnificos . } Alii enim moderatas possessiones habentes , & que anings de los otros nobles \textbf{ ca ellos conuiene de ser nobles prinçipalmente | e magnificos en todas sus cosas . } ca si los otros nobles \\\hline
2.3.3 & oportet ipsa esse magnifica . \textbf{ Viso , qualia debent esse aedificia , } quantum ad magnificentiam et industriam operis : & que ellos sean muy grandes e muy costosas \textbf{ ¶ Visto quales deuen ser las moradas } quanto ala grandeza \\\hline
2.3.3 & Dicit enim salubritatem aeris primo \textbf{ declarare loca a vallibus infimis libera . Si enim in vallibus infimis aedificia construantur , } quia aer est ibi grossus , & e esta sanidat esta lo primero \textbf{ en non ser asentada la morada en los valles muy baxos | por que si en los valłs baxos fuessen fechͣs } el ayre seria \\\hline
2.3.5 & eo enim quod bestiae naturaliter homini debent esse subiectae , \textbf{ et debent ordinari in obsequium eius : } si ipsae refugiant obsequium hominis , & por que las bestias deuen ser suiebtas del omne \textbf{ e deuen ser ordenadas a seruiçio del omne } e si ellas fuyen del omne seruiçio del omne \\\hline
2.3.7 & Propter quod contra tales recusantes subiici , \textbf{ quos dignum est esse subiectos , } secundum sententiam Philosophi sic intellectam , & que refusan ser subiectos \textbf{ los quales son dignos de ser subiectos } segunt la sentençia del philosofo \\\hline
2.3.8 & et multo magis Reges et Principes \textbf{ esse contentos tantis possessionibus , } et diuitiis , & e alos prinçipes \textbf{ de ser pagados de tantas possessiones } e de tantas riquezas \\\hline
2.3.10 & tamen et principibus \textbf{ quod decet esse quasi semideos , } exercere non congruit . & commo quier que le conlientan alos mercadores e algunos otros . \textbf{ Enpero alos Reyes e alos prinçipes los quales deuen ser medios dioses } non los conuiene de usar dellas \\\hline
2.3.12 & vidit per astronomiam , \textbf{ futuram esse magnam copiam oliuarum : } et ab omnibus incolis regionis illius emit tantum oleum , & que aquel anero \textbf{ que auie de uenir | que auie de ser grant cunplimiento de oliuas e de olio . } Et el por ende conpro todo el olio \\\hline
2.3.13 & subiici prudentibus \textbf{ expedit enim eis sic esse subiectos , } ut per eorum industriam dirigantur & de ser suiebtos alos sabios . \textbf{ Et por ende les conuiene de ser assi subietos } por que por la sabiduria de los sabios sean enderescados e sean sabios \\\hline
2.3.14 & Unde et Philosophus ait , \textbf{ Iustius esse diffiniri dominium } secundum bona animae , & Onde el philosofo dize \textbf{ que mas e iusta cosa es sentir | que el señorio deue ser } segunt los bienes del alma \\\hline
2.3.15 & ex ipsa dignitate ministrantium patet \textbf{ huiusmodi ministros a principante esse magis honorandos et praemiandos . } Secunda via ad ostendendum hoc idem , & por la dignndat de los sirinentes \textbf{ que estos tales deue ser mas honrrados | e mas gualardonados del prinçipe que los otros ¶ } La segundi razon paramos \\\hline
2.3.17 & et congruentia temporum . \textbf{ Cum enim deceat Regem esse magnificum , } ut supra in primo libro diffusius probabatur , & La conueniençia de los tiepos . \textbf{ Ca commo conuenga alos Reyes | e alos prinçipes ser magnificos } assi commo es prouado mas conplidamente en el primero libro \\\hline
2.3.18 & et quod communiter dicitur , \textbf{ impossibile est esse falsum secundum totum , } ut videtur velle Philosophus 7 Ethicorum , & e lo que lo sons dizen comunalmente \textbf{ non puede ser falso del todo } assi commo dize el philosofo en el viij̊ . libro delas politicas . \\\hline
2.3.18 & qui creduntur et aestimantur boni , \textbf{ esse tales secundum veritatem , } decens est nobiles genere esse nobiles secundum mores . & que los que son contados e estimados \textbf{ por buenos de ser tales segunt uerdat . | Et por ende cosa conueinble es } que los nobles \\\hline
2.3.18 & eo quod sunt in maximo nobilitatis gradu , \textbf{ habere mores nobiles et curiales , ministros , } quos in bonis decet suos dominos imitari , & por que son en muy grand grado de nobleza \textbf{ auer buenas costunbres | e de ser curiales e nobles } assi conuiene alos seruientes dellos \\\hline
2.3.19 & Ostensum est , \textbf{ quales debent esse ministri Regum et Principum , } quia debent habere nobiles et curiales . & ¶ \textbf{ ostrado es quales deuen ser los seruientes de los Reyes } e delos prinçipes \\\hline
2.3.19 & ut debitam politiam seruent \textbf{ esse iustos legales , } sic decet ministros dominorum & para guardar su poliçia conueniblemente \textbf{ assi conuiene alos sermient | s̃ de los sennores de ser curiales } por que guarden el estado e la honrra dela corte \\\hline
2.3.19 & ut seruent decentiam curiae \textbf{ et honoris statum curiales esse quare si scimus quales oportet esse ministros , } restat ostendere qualiter Reges et Principes & por que guarden el estado e la honrra dela corte \textbf{ conueiblemente | por ende si sabemos quales deuen ser los seruientes } fincanos de demostrar \\\hline
2.3.19 & Reges ergo et Principes , \textbf{ quos decet esse magnanimos } ad proprios ministros , & e los prinçipes \textbf{ alos quales conuiene de ser magn animos | deuen se mostrar } tonprados a los sus seruientes propreos \\\hline
3.1.2 & si habeant perfectiones competentes propriae speciei : \textbf{ esse tamen virtuosum habere non possunt , } quia nequeunt participare virtute . & que parte nesçen ala su semeiançaprop̃a . \textbf{ Enpero non puede auer el ser uirtuoso } por que non puede partiçipar la uirtud ¶ Et pues que assi es en aquella manera \\\hline
3.1.3 & quia ignoratur , \textbf{ quomodo naturale est homini esse animal ciuile . } Non enim hoc est sic homini naturale , & por que non saben los \textbf{ que en esto dubdan | en qual maneran atra al cosa es al omne de ser } aianlçiuil \\\hline
3.1.3 & ut licet naturale sit \textbf{ homini esse dextrum , } multi tamen ex aliquo impedimento & assi commo dezimos \textbf{ que maguera natural cosa sea al omne de ser diestro } enpero much sson fallados simestris \\\hline
3.1.4 & ad sufficientiam vitae , \textbf{ sed etiam quia generatio eius habet esse naturale : } fit enim vicus naturaliter & non solamente por que sirue a conplimiento dela uida \textbf{ mas avn por que la generaçion del uarro ha de ser cosa natal } ca fazesse el uarion a tal monte de acresçentamiento de fijos e de metos e de parientes e de vezinos \\\hline
3.1.6 & quorum quilibet dici potest naturalis : \textbf{ possumus addere modum tertium , } qui quasi est simpliciter violentus . & e del tegno delas \textbf{ quales cada vna puede ser dichͣ natural Podemos eñader la terçera manera que es sinplemente } assi commo manera forcada \\\hline
3.1.7 & et regnum ciuile . \textbf{ Primum est , quia dixerunt ciuitatem debere esse maxime unam . } Quod forte ideo hoc opinati sunt , & e el gouernamiento çiuil . \textbf{ ¶ Lo primo que dixieron es | que la çibdat deuia ser muy vna } e muy ayuͣtada los quales por auentura por esto lo cuydaron por \\\hline
3.1.7 & quod circa ciuitatem statuerunt dicti Philosophi , \textbf{ est , quia dixerunt ciuibus omnia debere esse communia : } ut quod haberent communes possessiones communes uxores , & çerca la çibdat es \textbf{ que dixieton | que todas las cosas de uun ser comunes a cada vno } assi que ouiessen las possessiones comunes e las mug̃es comunes \\\hline
3.1.7 & est , quia dixerunt mulieres \textbf{ instruendas esse ad opera bellica , } et debere bellare & es que dixieron \textbf{ que las mugers deuian ser enssennadas | alas obras dela batalla } e que deuian batallar e guerrear \\\hline
3.1.7 & ut existentes in minoribus , \textbf{ non debent conuerti in venam ferri , } ut quod fiant subditi , & que estan en menores poderios \textbf{ e por ende non deuen ser conuertidos en venas de fierro } assi que sean subditos \\\hline
3.1.8 & esse perfectum , \textbf{ oportet dare diuersitatem aliquam , nec oportet ibi esse } omnimodam conformitatem et aequalitatem , & para que aya ser acabada \textbf{ conuiene de dar ay algun departimiento | nin conuiene de ser } y en toda manera confirmada egualdat \\\hline
3.1.8 & probantes quod non oportet ciuitatem \textbf{ esse maxime unitam , } et maxime conformem . & que non conuiene ala çibdat \textbf{ de ser muy vna } nin muy ayuntada . \\\hline
3.1.8 & Dicere ergo in ciuitate \textbf{ vel in regno esse debere omnem unitatem , } est dicere ciuitatem & e si se estendiere a mayor vnidat paresçra el ser dela çibdat . \textbf{ Et pues que assi es dezer que enla çibdat o en el regno deua ser tan grant vnidat } commo dizian socrates e platones dezer que la çibdat non sea çibdat \\\hline
3.1.9 & diu inuestigandum est , \textbf{ qualiter ciuitatem oportet esse unam , } et quam diuersitatem habere debet , & muy luengamente es de buscar \textbf{ e de escodrinnar | en qual manera la çibdat conuiene de ser vna } e qual departimiento deue auer enlła \\\hline
3.1.9 & quibus communicare debent , \textbf{ utrum omnia deberent esse communia , } et quid debeant habere proprium . & e en quales cosas deuen partiçipar \textbf{ e si deuen todas las cosas ser comunes a todos } e qual cosa de una auer proprea \\\hline
3.1.9 & Unde et Philosophus ait in Politicis ciues \textbf{ non esse applicandos legibus , } sed leges ciuibus . & Onde el pho dize en el terçero libro delas politicas \textbf{ que los çibdadanos non deuen ser llegados alas leyes } mas las leyes alos çibdadanos \\\hline
3.1.9 & sed leges ciuibus . \textbf{ Tales enim debent esse leges , } et talis debet esse ordinatio ciuitatis , & mas las leyes alos çibdadanos \textbf{ ca tales deuen ser las leyes } e tal deue ser el ordenamiento dela çibdat \\\hline
3.1.10 & et personarum vilium . \textbf{ Nam esse non potest uxores et filios communes , } nisi aequalis cura geratur de filiis nobilium , & e delas uiles perssonas . \textbf{ poues que assi es non puede ser de razon | que las mugers et los fijos sean comunes } si non fuere puesta igual cura \\\hline
3.1.10 & nam ut dicit Philosophus 8 Ethic’ \textbf{ multis esse amicum } secundum perfectam amicitiam non contingit , & e de los fijos se sigua iniuria et tuerto de los fijos \textbf{ assi commo dize el philosofo en el viij libro delas ethicas non puede ser amistança acabada de vno a muchos } nin puede vno amar mucha muchas personas en vno \\\hline
3.1.10 & per multitudinem foeminarum difficile est \textbf{ et quasi impossibile est esse temperatum . } Quintum autem malum sic manifestari potest . & quando esta muy despartado e muy abiuado por muchedunbre de mugenses . \textbf{ cosa muy guaue e non puede ser | que sea el o entenprado ¶ } El quinto mal se puede \\\hline
3.1.11 & ait enim possessiones \textbf{ et res civium debere esse proprias , et communes . } Proprias quidem quantum ad dominum , communes vero & e el estado de los omes es mas conuenible ala çibdat \textbf{ ca dize que las possessiones e las cosas de los çibdadanos deuen ser propreas } e comunes propraas quanto al sennorio . \\\hline
3.1.12 & Ut in praecedentibus dicebatur , \textbf{ Socrates statuit mulieres instruendas esse ad opera bellica , } et debere bellare , & ssi commo fue dicho dessuso socrates \textbf{ ordeno que las mugers deuian ser enssennadas | alas obras dlas batallas } en manera que pudiessen lidiar . \\\hline
3.1.14 & Volebat enim bellantes \textbf{ esse partem ciuitatis distinctam ab aliis ciuibus . } Numerum autem bellantium statuebat , & Et los labradores . \textbf{ Ca dizia que los lidiadores deuen ser parte dela çibdat apartada de los otros çibdadanos } e establesçie cuento de los lidiadores \\\hline
3.1.14 & secundum tria quae de ipsis bellantibus Socrates statuebat . \textbf{ Primo enim dicebat eos esse alios et distinctos ab aliis ciuibus . } Secundo statuebat magnam multitudinem bellatorum . & que el establesçia en los lidiadores ¶ \textbf{ Ca primeramente dize | que los lidiadores deuian ser apartados de los otros çibdadanos } Lo segundo establesçie grant muchedunbre de lidiadores \\\hline
3.1.14 & melius est ergo dicere in ciuitate \textbf{ tot esse bellatores et defensores patriae , } quot sunt ibi ciues valentes portare arma , & e por ende meior es dez \textbf{ que enla çibdat tantos deuen ser los lidiadores | e los defenssores dela tierra } quantos son y çibdadanos \\\hline
3.1.15 & quod Socrates et discipulus eius Plato dixerunt , \textbf{ ciuitatem sic esse regendam et gubernandam , } ut ciuibus communes essent uxores , et filii , et possessiones . & ontado es de suso que socrates e su disçipnlo platon dixieron \textbf{ que la çibdat deuia | assi ser gouernada } que todos los çibdadanos deuian auer las possessionos e las mugers \\\hline
3.1.15 & intelligere dicta Socratica , \textbf{ saluare poterimus positionem eius . } Omnia enim esse ciuibus communia & Si quisieremos entender los dichos de socrates \textbf{ non assi conmo suena las palabras podremos entender la su opinion diziendo | que non es cosa que pueda ser } nin es cosa aprouechable \\\hline
3.1.15 & sicut et proprias . \textbf{ Hoc ergo modo ciuibus omnia debent esse communia , } ut quilibet intendat bonum commune et bonum omnium , & assi commo las suyas propreas . \textbf{ Et pues que assi es en esta manera deuen ser todas las cosas comunes alos çibdadanos } por que cada vno entienda el bien comun \\\hline
3.1.15 & In uxoribus autem ex filiis debet \textbf{ reseruari communitas quantum ad amorem : } sed in possessionibus non solum debet & assi commo si fuessen suyas . \textbf{ Mas en las mugers e en los fijos deue ser guardada comunidat } non solamente quanto al amor \\\hline
3.1.15 & sic etiam saluare possumus dictum eius quantum ad unitatem ciuitatis . \textbf{ Nam cum dixit ciuitatem debere esse maxime unam , } forte non intellexit de unitate habitationis , & del quanto ala vnidat dela çibdat \textbf{ ca quando dixo | que deuia ser la çibdat much vna } por auentura non entendio de vnidat dela morada \\\hline
3.1.15 & Volebat ergo Socrates politiam \textbf{ aliquam non debere nominari ciuitatem , } nisi saltem contineret mille nobiles , & esquariasocrates \textbf{ que algua poliçia non deuiesse ser llamada çibdat } si a lo de menos non ouiesse en ella minłłomes nobles \\\hline
3.1.18 & ut ciues possessiones aequatas habeant ; \textbf{ nec intentio legislatoris principaliter esse debet circa possessiones , } sed principalius debet & por que los cibdadanos ayan las possessiones eguales \textbf{ nin la entençion prinçipal | del que faze la ley deue ser çerca delas possessiones } mas mayormente deue entender en la reprehension delas cobdiçias \\\hline
3.1.19 & Quarto intromisit se de distinctione iudicantium . \textbf{ Dicebat enim debere esse duo genera iudicantium , } et duplex praetorium : & del departimiento \textbf{ de aquellos que iudgan | ca dize que dos deuian ser los linages de los iudgadores } e delos alcalłs e dias audiençias . \\\hline
3.1.19 & ( ut apparet ex dictis suis ) \textbf{ principem debere esse per haereditatem , } sed per electionem , & por los sus dichos \textbf{ que el prinçipe non deue ser fecho } por heredamiento mas por el ectiuo la qual elecçion daua a todo el pueblo ¶ \\\hline
3.1.20 & tangentes diuersa genera personatum . \textbf{ Primo enim dictus Phil’ deferre fecit statuendo impossibilia . } Nam statutum de distinctione ciuium stare & tanniendo departidos linages de perssonas . \textbf{ Ca lo primero el dicho philosofo fallesçio | establesçien do establesçimientos que non podian ser nin estar en vno . } ca el establesçimiento del departimiento de los çibdadanos \\\hline
3.2.5 & Nesciunt enim tales fortunas ferre , \textbf{ nuper enim esse exaltatum in Regem , } est quasi quaedam ineruditio regiae dignitatis & ca estos tales non saben sofrir las uenturas \textbf{ por que ser leunatado en rey del otro dia es } assi commo non saber qual es la maiestad nin la dignidat Real \\\hline
3.2.6 & nisi sensibilia bona , \textbf{ quos videt esse liberales et beneficos , } nimis ardenter mouetur in eorum amorem , & si non las cosas que siente \textbf{ por ende aquellos que vee ser liberales } e bien fechores mueuense con grant ardor alos amar \\\hline
3.2.7 & magis est inuoluntarium , \textbf{ magis debet dici in naturale : } tyrannis igitur est pessima , & por que quanto el sennorio de alguon ses mas contra uoluntad de los omes \textbf{ tanto mas deue ser dich des natural . } Et por ende la tirania es muy mala \\\hline
3.2.8 & spectat ad Reges \textbf{ et Principes valde esse solicitos de studio litterarum . } Immo ( ut infra patebit ) & por ende pertenesçe alos Reyes \textbf{ e alos prinçipes | de ser muy acuçiosos del estudio delas letris e delas sçiençias . } Mas assi commo paresçra adelante mas . \\\hline
3.2.14 & Una ergo tyrannis potest contrariari alii , \textbf{ et corrumpere ipsam ; } ut tyrannis populi contrariatur tyrannidi monarchiae : & e el mal senorio es contrario al malo . \textbf{ Et por ende vna tirama puede ser contraria a otra } e corronper la assi cotio la tirania del pueblo \\\hline
3.2.19 & probabatur enim supra , \textbf{ Regem debere esse talem , } quod esset bonus virtuosus & ca prouado es de suso \textbf{ que el Rey deue ser } tal que sea bueno e uirtuoso \\\hline
3.2.19 & existentes in regno promoueret et honoraret : \textbf{ quod esse non posset , si bona eorum quae sunt in regno usurparet iniuste . } Rursus est attendendum , & et ꝓmueua los q̃ son eñl su regno e los hõ rre . \textbf{ la qual cosa non podria ser } si tomasse los bienes \\\hline
3.2.19 & quam alia : \textbf{ debet ergo adhiberi consilium , } ut circa talia maior custodia praebeatur . & para fazer mal que los otros . \textbf{ e por ende deue ser tomado consseio } por que en tales sea puesta mayor guarda . \\\hline
3.2.19 & sed sunt diligenter custodienda et munienda . \textbf{ Quarto habet esse consilium } circa pacem et bellum , & mas son con grant acuçia de guardat e de guamesçer . \textbf{ Lo quarto deue ser } tomadon consseio dela paz e dela guerra \\\hline
3.2.19 & et in hoc non est quaestio nec consilium . \textbf{ Sed utrum cum extraneis debeamus habere pacem } vel bella dubitabile esse potest , & questiuo nin consseio . \textbf{ Mas si deuemos auer paz con los estrannos o guerra pue de ser cosa dubdosa } e çerca esto pueden ser tomados consseios çerca la qual \\\hline
3.2.19 & Quis sit autem optimus modus principandi , \textbf{ et quale debeat esse regis officium , } et quomodo Rex se debeat & Mas qual es la manera muy buena de prinçipar o de enssennorear . \textbf{ e qual deua ser el ofiçio del rey } e en qual manera el Rey se deua guardar en su sennorio mostrado fue conplidamente en los dichos de ssuso . \\\hline
3.2.20 & et quae et quot sunt illa \textbf{ circa quae habet esse consilium . } Restat secundum ordinem praetaxatum & e quales e quantas cosas son aquellas \textbf{ en que han de ser tomados los consseios fincan nos segunt la orden sobredichͣ } que digamos del alcalłia o del iuyzio mostrando \\\hline
3.2.22 & facient iudicium iniquum : \textbf{ nam ut patet ex habitis iudex debet esse quasi regula recta media inter utrasque partes : } quare si huiusmodi regula a medio deuiat , & Ca assi commo paresçe \textbf{ por lo que dich̉es el iues deue ser | assi commo regla derecha medianera entre amas las partes } por la qual cosa \\\hline
3.2.23 & quibus congruit ampliori bonitate pollere . \textbf{ Decet itaque eos esse clementes et benignos , } non quia iustitiam deserant , & por mayor bondat . \textbf{ Et pues que assi es conuiene a ellos de ser piadosos e benignos non } por que dexen la iustiçiaca sin ella la paz del regno \\\hline
3.2.24 & vel dicuntur iusta naturaliter \textbf{ quae dictat esse talia ratio naturalis , } vel ad quae habemus naturalem impetum et inclinationem . & O son dichos natales \textbf{ por que la razon natural los muestra de ser tales } o por que auemos natural apetito o natural inclinaçion \\\hline
3.2.24 & idem apud omnes , \textbf{ ideo dicitur esse ius commune : } sed ius positiuum diuersificatur apud diuersas ciuitates , & que el derecho narurales vno a todo los omes . \textbf{ Et por ende es dicho ser derecho comunal . } Mas el derecho positiuo es departido en departidas çibdades \\\hline
3.2.26 & oportet hoc agere . \textbf{ Tales ergo debent esse leges , } quae sunt regulae agibilium , & conuiene que fagamos estas cosas . \textbf{ Et pues que assi estales deuen ser las leyes } que sean reglas delas nuestras obras \\\hline
3.2.28 & licet forte ex intentione operantium possint \textbf{ esse bona et laudabilia , } vel mala et vituperabilia . & por la entençion \textbf{ del que obra pueden ser buenas e de loar o malas e de denostar assi } conmoleunatar la paia de tierra \\\hline
3.2.28 & quod de se est quasi indifferens , \textbf{ esse potest virtuosum et laudabile . } Secundum igitur haec tria genera fiendorum , & que dessi non es buena nin mala \textbf{ enpero puede ser uirtuosa e de loar . } pues que assi es segunt estas tres maneras delas obras que son de fazer podemos a podar \\\hline
3.2.29 & et si non quantum ad esse naturae , \textbf{ tamen quantum ad esse optimum : } quia cum optimus homo incipit furire & quanto al ser dela natura \textbf{ enpero matase quanto al ser muy bueno . } por que quando el muy bueno en se en comiença de enssennar e de cobdiçiar las cosas malas \\\hline
3.2.29 & et si non interimitur quantum ad esse simpliciter , \textbf{ interimitur tamen quantum ad esse optimum , } quia non est ulterius optimus . & quantoal ser sinple mente . \textbf{ Enpero matasse | quanto asser muy bueon . } Ca de ally adelante non es muy bueno . \\\hline
3.2.29 & et quemlibet principantem \textbf{ esse medium } inter legem naturalem et positiuam ; & conuiene de saber que el rey \textbf{ e qual se quier sennor deue ser medianero entre la ley natural e la ley positiua . } Ca ninguno non iudga derechamente nin \\\hline
3.2.29 & et decet Regem regere alios , \textbf{ et esse regulam aliorum , } oportet Regem in regendo alios & Conuiene al rey de gouernar los otros \textbf{ e de ser regla de los otros . | Et por ende conuiene } que el Rey en gouernando los otros sigua razon de rechͣ . \\\hline
3.2.29 & sicut lex naturalis est supra . Et si dicatur legem \textbf{ aliquam positiuam esse supra principantem , } hoc non est ut positiua est , & por que la non puede mudar . \textbf{ Et si dixiere alguno | que alguna ley positiua deue ser sobre el sennor . } esto non es en quanto es ley positua \\\hline
3.2.30 & Decet ergo reges et principes , \textbf{ quos competit esse quasi semideos , } et esse intellectum sine concupiscentia , & Et por ende conuiene que los Reyes e los prinçipes \textbf{ alos quales parte nesçe ser | assi commo medios dioses } e de auer entendimiento sin cobdiçia \\\hline
3.2.30 & et esse intellectum sine concupiscentia , \textbf{ et esse formam viuendi , } et regulam agibilium , & e de auer entendimiento sin cobdiçia \textbf{ e de ser forma de beuir e regla de todas las obras . } assi se auer ala leyna traal e diuinal e humanal \\\hline
3.2.31 & dato etiam quod occurrant leges aliquae \textbf{ quae videantur esse magis proficuae et meliores . } Adducit autem Phil’ & puesto avn que algunas leyes fuessen falladas \textbf{ que paresçiessen ser mas aprouechosas e meiores . } Mas el philosofo enel quarto libro delas politicas pone quatro razones \\\hline
3.2.31 & nam esse vestitum videtur contrariari \textbf{ ei quod est esse nudum . } Secundum hunc modum loquendi loquuntur Iuristae , & e la uestidura es contra a natura \textbf{ Ca ser el omne uestido paresçe contrario a aquello que es ser de sudo . } Et segunt esta manera de fablar fablan los iuristas del derecho natural \\\hline
3.2.34 & et secundum dictum Homeri , \textbf{ est esse magis bestiam , } quam hominem . & e segunt el dicho comun de los omes \textbf{ es mas ser bestia que omne . } Por la qual cosa los que non guardan las leyes \\\hline
3.2.36 & ut Reges diligantur a populo , \textbf{ decet eos esse iustos , et aequales . } Nam maxime prouocatur populus ad odium Regis , & La terçera cosa para que los Reyes sean amados del pueblo \textbf{ es quales conuiene de ser derechureros e eguales . } Ca el pueblo mayormente se le una taria a mal querençia del Rey \\\hline
3.3.1 & hunc totalem librum diuisimus in tres libros . \textbf{ Quia in primo libro docuimus Regem esse prudentem } prout Rex aut Princeps est quaedam persona in se , & e los prinçipes partimos este libro todo en tres libros \textbf{ Ca en el primero libro mostramos al Rey ser sabio } en quanto el Rey \\\hline
3.3.1 & Quare sicut nullus efficiendus est magister in aliis scientiis , \textbf{ nisi constet ipsum esse doctum in arte illa : } sic nullus assumendus est & si non fuere çierto \textbf{ que el es ensseñado en aquella sçiençia | de la qual ha de ser maestro . } En essa misma manera ninguno non deue ser tomado \\\hline
3.3.3 & Viso in qua aetate assuescendi sunt \textbf{ qui debent effici bellatores } ad actiones bellicas : & Visto en qual hedat son de acostunbrar a obras de batalla \textbf{ aquellos que se deuen fazir caualleros e ser lidiadores . } finca de ver por quales señales se han de conosçer los buenos lidiadores . \\\hline
3.3.3 & homines similiores animalibus bellicosis , \textbf{ utiliores videntur esse ad bellum . } Tribus igitur generibus signorum & que son semeiantes masa las animalias lidiadoras \textbf{ paresçe ser mas prouechosos para la batalla . } Et pues que asy es \\\hline
3.3.4 & propter quod tales debent \textbf{ esse homines pugnatiui , } ut diu tolerare possint & para dar las feridas . \textbf{ Por la qual cosa tales deuen ser los omnes lidiadores } que prolongadamente puedan sofrir el andar \\\hline
3.3.5 & diuersi eligendi sunt bellatores . \textbf{ Potest enim esse pugna pedestris , et equestris . } In pedestri itaque certamine & son de escoger departidos lidiadores . \textbf{ Ca la batalla puede ser de omnes de pie o de omnes de cauallo . } Et por ende en la batalla de los peones \\\hline
3.3.6 & propter particularium inexperientiam \textbf{ esse imprudentem in aliquo . } Unde multotiens contingit , & por non auer prueua de las cosas particulares \textbf{ de non ser sabio en otra cosa . } Onde muchas vezes \\\hline
3.3.6 & et pedites , et etiam milites , \textbf{ si contingat eos pedestres esse , si volunt boni bellatores existere , } sic ab ipsa iuuentute exercitandi sunt ad saliendum , & Et los peones \textbf{ e avn los caualleros si contesçiere que ellos esten de pie | si quieren ser buenos lidiadores } conuiene que en su mançebia sean usados a saltar \\\hline
3.3.8 & munitiones et fossas fiendas circa exercitum debere \textbf{ habere formam quadrilateram oblongam . } Attamen quia figura circularis est capacissima , & que son de fazer çerca de la hueste \textbf{ deuen ser quadradas e luengas . } Enpero por que la forma redonda conprehende \\\hline
3.3.8 & quod si non immineat magna vis hostium , \textbf{ fossa debet esse lata pedes nouem , alta septem . } Sed si aduersariorum vis acrior imminet , & diziendo que si paresciere grant fuerça de los enemigos . \textbf{ la carcaua deue ser muy ancha de nueue pies e alta de siete . } Mas si la fuerça de los enemigos paresciere mas fuerte conuiene de fazer las carcauas mas anchas et mas fondas \\\hline
3.3.9 & Considerato enim bello in uniuersali , \textbf{ omnes volunt esse boni bellatores , } sed postquam veniunt & Ca penssada la batalla en general \textbf{ todos quieren ser buenos lidiadores } mas despues que vienen a la prueua de los fechos particulares \\\hline
3.3.16 & quia sicut contingit \textbf{ esse pugnam in terra , } sic contingit eam esse in aquis . & naual de las naues e de la mar . \textbf{ Ca assi commo contesçe de ser batalla en la tierra } assi contesçe de ser en las aguas . \\\hline
3.3.16 & contingat obtineri et deuinci munitiones et urbanitates : \textbf{ restat dicere quot modis talia deuinci possunt . } Est autem triplex modus obtinendi & e los castiellos e fortalezas . \textbf{ fincanos de dezir | en quantas maneras tales fortalezas pueden ser vençidas . } Et conuiene de saber \\\hline
3.3.20 & Rursus supra cataractam debet \textbf{ esse murus perforatus recipiens ipsam , } per quem locum poterunt proiici lapides , & Otrossi sobre las puertas de la trayçion \textbf{ deue ser el muro foradado de guisa | que la puedan leuantar arriba et baxarla cada que quisieren . } Et por aquel logar pueden lançar piedras . \\\hline
3.3.23 & cum quo teruntur muri ciuitatis obsessae . \textbf{ Debet autem sic ordinari lignum illud , } ut ligamentum retinens & que sea tal commo el carnero con el qual suelen quebrar los muros de la çibdat çercada . \textbf{ Et deue este madero | assi ser ordenado } que con el atadura \\\hline

\end{tabular}
