\begin{tabular}{|p{1cm}|p{6.5cm}|p{6.5cm}|}

\hline
1.1.4 & ¶ Enpero del an vida politica e ordenada la qual los theologos llaman vida actiua \textbf{ que quiere dezir vida de bien obrar } Et dela vida contenplatian e intellectual non sintieron la uerdat conplidamente & de vita tamen politica , quam Theologi vocant vitam actiuam , \textbf{ et de vita contemplatiua } non usquequaque vera senserunt : crediderunt enim , \\\hline
1.1.5 & Onde dize el philosofo en el primero libro delas ethicas que en olimpiedes \textbf{ que quiere dezer en aquellas faziendas } o es aquellas batallas non son coronados los muy fuertes & Unde Philosophus 1 Ethic’ ait , quod in Olimpidiadibus , \textbf{ idest in illis bellis } et agonibus non coronantur fortissimi , \\\hline
1.1.7 & en el quarto libro delas ethicas en el capitulo dela maganimidat \textbf{ que quiere dezir grandeza de coraçon } Ca en la opinion del auariento e en la opinion de aquel & ut vult Philosophus 4 Ethicorum cap’ de Magnanimitate ) \textbf{ quia in opinione auari , } et in opinione ponentis suam felicitatem in diuitiis , \\\hline
1.1.7 & que el pueda allegar riquezas e dineros . Ca aquel que bien entiende \textbf{ que quiere dezir e quanto lieua este nonbre fin } e bien andança non se le puede esconder por ninguna manera & dum tamen possit pecuniam congregare , qui enim bene intelligit , \textbf{ quid importatur nomine finis , } non potest eum latere quemlibet , omni via qua potest , \\\hline
1.1.10 & que es con uirtud es muy meior que ensennorear despotice \textbf{ que quiere dezir enssennorear . } seruilmente e sobre los sieruos ¶ La quarta razon es & est melior , quam principari despotice , \textbf{ idest dominaliter . } Quarto non est ponenda felicitas in ciuili potentia : \\\hline
1.2.3 & la otra es bien fablança . La otra es eutropolia \textbf{ que quiere dezir buena conuerssaçion o buena manera de beuir . } Mas la uerdat assi conma aqui fablamos de uerdat non en quanto es uirtud & Erit ergo triplex virtus ; videlicet , Veritas , Affabilitas , et Eutrapelia , \textbf{ quae potest dici bona versio . } Est autem Veritas ( ut hic de ea loquimur , \\\hline
1.2.3 & nin del conpuesto en sus palauras . mas es bien fablante e curial¶ \textbf{ Mas entropolia que quiere dezir buena conpanma } o buena manera de beuir en conpanna es quando alguno se sabe bien auer en los trebeios & non est discolus , sed est affabilis , et curialis . \textbf{ Eutrapelia vero siue bona versio , } est , quando aliquis sic se habet in ludis , ut non sit histrio , \\\hline
1.2.6 & ala qual uirtud llama el philosofo en el sexto libro delas ethicas . \textbf{ Eubullia que quiere dezer uirtud para bien coseiar . } la otra es por la qual iudgamos bien delas cosas falladas . & et bene confiliemur , quam Philosophus Ethic’ 6 appellat eubuliam , \textbf{ idest bene consiliatiua . } Alia vero per quam bene iudicamus de inuentis , quam Philosophus appellat synesin , \\\hline
1.2.6 & iudgamos bien delas cosas falladas . la qual llama el philosofo sinesis . \textbf{ que quiere dezir uirtud de bien iudgar } ¶ la terçera es uirtud por la qual mandamos que se fagan las obras todas segunt las cosas falladas e iudgadas & Alia vero per quam bene iudicamus de inuentis , quam Philosophus appellat synesin , \textbf{ idest bene iudicatiuam . } Tertia , per quam praecipiamus ut fiant opera \\\hline
1.2.12 & e por la su claridat es llamada renꝮ por nonbre comunal \textbf{ que quiere dezir cosa clara e cosa apuesta . } Et esta estrella algunas vezes nasçe ante del sol e estonçe paresçe en la mananna & et propter sui pulchritudinem , et venustatem communi nomine \textbf{ appellatur Venus . } Haec autem stella aliquando praecedit solem , et tunc apparet in mane , \\\hline
1.2.17 & Et por esso el philosofo en el quarto libro delas ethicas llama a estos tales non liberales \textbf{ que quiere dezir non francos } assi commo son los logreros e los garçons e los que biuen de alcauteria & Usurpans enim bona utilia , et non accipiens ea sicut debet , \textbf{ nimis videtur auidus pecuniae . Propter quod Philosophus 4 Ethic’ usurarios , lenones , } idest viuentes de meretricio , expoliatores mortuorum , et aleatores , \\\hline
1.2.19 & La otra cata alas grandes espenssas la qual laman magnifiçençia \textbf{ que quiere dezir grandeza en despender . } Mas commo en cada cosa mas e menos non fagan departimiento en la naturaleza & quam liberalitatem vocant . Aliam , quae respicit sumptus magnos , \textbf{ quam magnificentiam nominant . } Sed cum magis , et minus non videantur \\\hline
1.2.19 & a estos tales llama los han asos \textbf{ que quiere dezir fuegos e fornos } por que estos tałs assi commo fuego & et tales vocantur consumptores . Philosophus vero vocat eos chaunos \textbf{ idest ignes et fornaces , } quia tales sicut fornax omnia consumunt . Quidam vero in magnis operibus \\\hline
1.2.21 & por menudo e non da delectablemente sin detenimiento \textbf{ aquello que ha de dar non es dicho magnifico mas pariufico que quiere dezir de pequena fazienda . } Et por ende dize el philosofo . en el quarto libro delas etris & et non delectabiliter , et prompte largitur \textbf{ quae largiti debet , | non est magnificus , sed paruificus . } Ideo dicitur 4 Ethic’ quod diligentia ratiocinii est paruifica . \\\hline
1.2.22 & assic̃omo conuiene . Et estos son dichos magnanimos \textbf{ que quiere dezir omes de grand coraçon ca nos ueemos algunos } que dessi son aptos e apareiados para fazer grandes cosas . & ut decet , ut magnanimi . \textbf{ Videmus enim aliquos de se aptos ad magna , } potentes magna et ardua exercere : quadam tamen pusillanimitate ducti , \\\hline
1.2.22 & que digna mientre non las pueden conplir . Et estos tałs ̃ llama el philosofo caymos \textbf{ que quiere dezir fumosos e ventosos } mas nos podemos los llamar prasunptuosos Pues que assi es el magnanimo es medianero entre el pusillanimo que es flaco de coraçon & quae digne complere non possunt , quos Philosophus vocat chaunos , \textbf{ idest fimosos et ventosos . } Nos autem eos praesumptuosos vocare possumus . Magnanimus vero medius est \\\hline
1.2.26 & ethicas llama vna gente gniega que dizen latun meses sobuios e alabadores de ssi \textbf{ que quiere dezir alabadores } que se alaban por que se vestian de villes pannos & Unde Philosophus 4 Ethic’ quandam gentem Graecam , Lacedaemones scilicet , \textbf{ iactatores et superbos appellat : } quia ultra quam eorum status requireret , vilius induebantur : \\\hline
1.2.28 & conueniblemente nos auemos con los otros honrrando los e resçibiendo los \textbf{ assi commo deuemos somos amigables e afabiles que quiere dezir amigos bien fablantes . } Pues que assi es non es otra cosa amistanȩ & honorando eos , et recipiendo ipsos \textbf{ ut debemus , | sumus amicabiles , et affabiles . } Nihil est ergo aliud amicabilitas , siue affabilitas , de qua hic determinare intendimus , \\\hline
1.2.28 & Et otrosi alegera conuenible la qual el philosofo llama heutropeliam \textbf{ que quiere dezir buena conpanina . } Et pues que assi es si quisieremos bien couerssar partiçipando con los otros deuemos seer alegres conueniblemente & veritas , quae apertio nuncupatur : et debita iocunditas , \textbf{ quam eutrapeliam vocat . } Communicando igitur cum aliis , si bene conuersari volumus , \\\hline
1.2.29 & que en ellos son . Et estos llama el philosofo yrones \textbf{ que quiere dezir escarnidores e despreçiadores dessi mismos . } Et pues que assi es conuiene de dar alguna uirtud medianera por la qual sean tenpradas las cosas menguadas & quae in ipsis non sunt , quos Philosophus vocat irones , \textbf{ idest irrisores , et despectores . } Oportet ergo dare aliquam virtutem mediam , per quam moderentur diminuta , \\\hline
1.2.32 & sobre la manera comunal de los omes es llamada del philosofo eroyca \textbf{ que quiere dezir prinçipante e sennor ante } por que es señora delas otras uirtudes ¶ Et pues que assi es por esto paresçe manifiestamente que los Reyes & esse bonus ultra modum humanum , appellatur a Philosopho heroica \textbf{ idest principans , et dominatiuat . } Ex hoc ergo manifeste patet , quod Reges , et Principes \\\hline
1.3.1 & Mas el appetito senssitiuo del seso assi commo mas largamente dixiemos de suso partese en apetito iraçibile \textbf{ que quiere dezir enssannador e concupiçible } que quiere dezir desseador . Et por ende las sobredichas passiones & in appetitu sensitiuo . Sensitiuus autem appetitus \textbf{ ( ut supra diffusius diximus ) diuiditur in irascibilem , et concupiscibilem . } Praedictae ergo passiones sic distinguuntur , quia primae sex videlicet , \\\hline
1.3.1 & partese en apetito iraçibile que quiere dezir enssannador e concupiçible \textbf{ que quiere dezir desseador . } Et por ende las sobredichas passiones assi se departen . & in appetitu sensitiuo . Sensitiuus autem appetitus \textbf{ ( ut supra diffusius diximus ) diuiditur in irascibilem , et concupiscibilem . } Praedictae ergo passiones sic distinguuntur , quia primae sex videlicet , \\\hline
1.3.10 & rectorica cuenta otras seys passiones . Conuiene saber Relo . \textbf{ gera Njemesim que quiere dezir tanto } commo indignacion dela buena andança de los malos . Misericordia e jnuidia . & 2 Rhetor’ sex alias passiones enumerare videtur , \textbf{ videlicet , zelum , gratiam , nemesin | ( quod idem est } quod indignatio de prosperitatibus malorum ) misericordiam , inuidiam , et erubescentiam siue verecundia . \\\hline
1.3.10 & e la honrra ha nonbre espeçial e es dicha uerguença o herubesçençia \textbf{ que quiere dezir en bermegecimiento . } Et pues que assi es la uergunença es temor espeçial e reduzesse al tenmor general & habet speciale nomen , et dicitur verecundia , vel erubescentia . \textbf{ Verecundia ergo est quidam timor , } et reducitur ad timorem . Viso , quomodo zelus et gratia reducuntur ad amorem : \\\hline
1.3.10 & mas en quanto el non meresçe de auer aquel bien assi es dicha enemessis o indignaçion \textbf{ que quiere dezir desden . } Ca segunt el philosofo en el segundo libro dela rectorica . Nemessis o indignaçiones auer tristeza de aquel & sed ut indigne habetur ab eo : sic est nemesis , vel indignatio . \textbf{ Nam ( secundum Philosophum 2 Rhetoricorum ) } nemesis vel indignatio , est tristari de eo \\\hline
2.1.3 & Et por ende pertenesçe al ph̃smoral assi commo alinconicos nico \textbf{ que quiere dezir ordenador de casa } de deter minar de los heditiçios delas calas por que ael parte nesçe generalmente demostrar & spectat enim ad moralem Philosophum , ut ad oeconomicum , \textbf{ determinare de aedificiis domorum : } quia spectat ad ipsum uniuersaliter et typo ostendere , \\\hline
2.1.7 & que el omne es naturalmente aina l aconpannable e comun incatiuo \textbf{ que quiere dezir ꝑtiçipante con otro } Mas la comunidat en la uida humanal assi commo dicho es dessuso & Tertia ex parte operum . Probabatur enim in primo capitulo huius secundi libri , \textbf{ hominem esse naturaliter animal sociale et communicatiuum . } Communitas autem in vita humana ( ut supra tangebatur ) \\\hline
2.3.10 & Natural . Et canssoria de canbio . \textbf{ Obolostica que quiere dezer maunera } de tornar los dineros en pasta . Et talzes que es husura¶ & in Poli’ quatuor species pecuniatiuae : \textbf{ videlicet naturalem , campsoriam , obolostaticam , } et tacos siue usuram : his enim quatuor modis possideri \\\hline
2.3.10 & e por esso el dinero deue ser dicho comienço e fin ¶La terçera manera del arte pecuniatiua de dineros es obolostica \textbf{ que quiere dezer arte de peso sobrepuiante que por auentura fue fallada assi . } Ca assi commo la massa del metal es partida en los dineros e en cada vn dinero es puesta senal publica . & et principium dici debet . Tertia species pecuniatiuae est obolostatica , \textbf{ vel ponderis excessiua : | quae forte sic inuenta fuit . } Nam sicut massa metalli in denarios diuiditur , \\\hline
2.3.12 & entre todas las cosas que acresçientan las riquezas es fazer monopolia \textbf{ que quiere dezer vendiconn de vno solo . } Ca quando vno solo uende taxa el preçio commo se el quiere & ( secundum Philos’ ) est facere monopoliam , \textbf{ idest facere vendationem unius : | nam quia unus solus vendit , } taxat precium pro suae voluntatis arbitrio : volentem ergo pecuniam acquirere , \\\hline
3.2.2 & e los trsson malos . ca el regno e la aristo carçia \textbf{ que quiere dezer sennorio de buenos } e la poliçia que quiere dezer pueblo bien & quorum tres sunt boni , et tres sunt mali . \textbf{ Nam regnum aristocratia , } et politia sunt principatus boni : tyrannides , oligarchia , et democratia sunt mali . \\\hline
3.2.2 & que quiere dezer sennorio de buenos e la poliçia \textbf{ que quiere dezer pueblo bien } enssenoreante son bueons prinçipados . La thirama que quiere dezer sennorio malo & et tres sunt mali . Nam regnum aristocratia , \textbf{ et politia sunt principatus boni : } tyrannides , oligarchia , et democratia sunt mali . Docet enim idem ibidem \\\hline
3.2.2 & que quiere dezer pueblo bien enssenoreante son bueons prinçipados . \textbf{ La thirama que quiere dezer sennorio malo } e la obligaçia que quiere dezer sennorio duro . Et la democraçia & Nam regnum aristocratia , et politia sunt principatus boni : \textbf{ tyrannides , oligarchia , et democratia sunt mali . } Docet enim idem ibidem discernere \\\hline
3.2.2 & enssenoreante son bueons prinçipados . La thirama que quiere dezer sennorio malo \textbf{ e la obligaçia que quiere dezer sennorio duro . } Et la democraçia que quiere dez maldat del pueblo & et politia sunt principatus boni : tyrannides , oligarchia , et democratia sunt mali . \textbf{ Docet enim idem ibidem } discernere bonum principatum a malo . \\\hline
3.2.2 & que entienden al bien comun e tal prinçipado es dicħa ristrocaçia \textbf{ que quiere dezer prinçipado de buenos omes } e uir̉tuosos e dende vienen que los mayores en el pueblo & et tunc talis principatus dicitur Aristocratia , \textbf{ quod idem est | quod principatus bonorum et virtuosorum . } Inde autem venit ut maiores in populo , \\\hline
3.2.2 & que los mayores en el pueblo e los que deuen gouernar el pueblo son dich sobtimates \textbf{ que quiere dezir muy buenos } ca muy buenos deuen ser aquellos que dessean ser mayores que los otros . & et qui debent populum regere vocati sunt optimates , \textbf{ quia optimi debent esse } qui aliis praeesse desiderant . Sed si illi pauci non sunt virtuosi , \\\hline
3.2.2 & e entienden a ganançia proprea este prinçiado tal es dich obligartia \textbf{ que quiere dezir prançipado de ricos . } Et pues que assi es dos prinçipados se leuna tan del sennorio de po cos vno derecho & intendunt proprium lucrum huiusmodi principatus Oligarchia dicitur , quod idem est \textbf{ quod principatus diuitum . } Consurgit igitur duplex principatus ex dominio paucorum : \\\hline
3.2.7 & en el quinto libro delas politicas do dize que la tirnia es la postrimera obligarçia \textbf{ que quiere dezer muy mala obligacion } por que es muy enpesçedera alos subditos ¶ La quarta razon se toma & Hanc autem rationem tangit Philosophus quinto Politicorum ubi ait , tyrannidem esse oligarchiam \textbf{ extremam idest pessimam : } quia est maxime nociua subditis . Quarta via sumitur \\\hline
3.2.12 & por que son buenos e uirtuosos este sennorio es derech e es llamado anstrocraçia \textbf{ que quiere dezer señorio de buenos . } Mas si enssennorear en pocos non por que son buenos & est rectus principatus , et vocatur aristocratia \textbf{ siue principatus bonorum . } Si vero dominentur non quia boni , sed quia diuites , \\\hline
3.2.12 & por que son buenos mas por que son ricos es llamado obligarçia \textbf{ que quiere dezer señorio tuerto . } Mas quando enssennore a todo el pueblo si entienda al bien comun de todos tan bien de los nobles conmo de los otros es senorio derech & Si vero dominentur non quia boni , sed quia diuites , \textbf{ est peruersus | et vocatur oligarchia . } Sed si dominatur totus populus et intendat bonum omnium \\\hline
3.2.27 & El primero es que las leyes sean bien establesçidas ¶ Lo segundo que sean bien guardadas \textbf{ que quiere dezer tanto commo } que atales leyes assi establesçidas obedes tan bien los omes & primo ut leges bene instituantur : secundo , ut bene custodiantur , \textbf{ vel ( quod idem est ) } ut legibus sic institutis bene obediatur . Ostendimus in praecedentibus capitulis , \\\hline
3.3.12 & que fagan az quadrada . e desende que establezcan vn triangulo \textbf{ que quiere dezir forma de tres linnas } e esto se faz ligeramente . Ca despues que el az esta quadrada & Quo facto statim debet praecipere dux belli , ut aciem quadratam faciant , \textbf{ et deinde , ut constituant trigonum : } quod faciliter fit . Nam acie quadruplicata \\\hline
3.3.16 & de de tener nos mas luengamente . Enpero diremos de la batalla osse ssiua \textbf{ e que quiere dezir batalla de cercamiento . } Et por ende uisto quantas son las maneras de las batallas . Et dicho que despues de la lid canpal del canpo primero & quis debeat se habere , non oportet circa alia bellorum genera diutius immorari . \textbf{ Primo tamen dicemus de bello obsessiuo . } Viso ergo quot sunt bellorum genera , et dicto quod post castrum campestre primo dicendum est \\\hline

\end{tabular}
