\begin{tabular}{|p{1cm}|p{6.5cm}|p{6.5cm}|}

\hline
1.1.6 & Pues si la bien auentraança es bien acabado e conplido . \textbf{ Conuiene que sea bien segunt el en tedimiento } e segunt Razon & si felicitas ponitur \textbf{ esse perfectum bonum , oportet quod sit bonum } secundum intellectum , et rationem : \\\hline
1.1.6 & e segunt Razon \textbf{ O conuiene que sea tal bien qual iudga la razon derecha } que nos auemos de segnir & secundum intellectum , et rationem : \textbf{ vel ( quod idem est ) | oportet quod sit tale bonum , } quale recta ratio prosequendi iudicet . \\\hline
1.1.8 & si conplida mente quiere demostrar le que significa \textbf{ conuiene que sea conosçida cosa e magnifiesta . } Mas las cosas que son de dentro del alma & si plene manifestare vult ipsum signatum , \textbf{ oportet quod sit | quid notum et manifestum : } intrinseca autem non sunt nobis nota , \\\hline
1.1.12 & e de gouernar prinçipalmente e acabadamente . \textbf{ Conuiene que qual se quier } prinçipeo Rey & et perfecte solus Deus , \textbf{ oportet quod quicunque principatur , } siue regnat , \\\hline
1.2.8 & e la su conpanna a alguons bienes . \textbf{ Conuiene que aya memoria de las cosas passadas . } Et que aya prouision delas cosas passadas & gentem aliquam ad bonum dirigere , \textbf{ oportet quod habeat memoriam praeteritorum , } et prouidentiam futurorum . \\\hline
1.2.8 & Ca la manera por que el Rey guia el su pueblo \textbf{ Conuiene que sea manera de omne . } Ca el Rey omne es & quo Rex suum populum dirigit , \textbf{ oportet quod sit humanus , } quia Rex ipse homo est . \\\hline
2.1.6 & Ca quando el ome es primero \textbf{ conuiene que sea engendrado . } Et la natura luego es acuçiosa de su salud . & Cum enim primo homo est , \textbf{ oportet quod sit genitus : } et natura statim est solicita de salute eius ; \\\hline
2.1.6 & Ca para ser la cosa acabada \textbf{ non conuiene que faga sienpre } e engendre su semeiante & Ad hoc enim quod aliquid sit perfectum , \textbf{ non oportet | quod sibi simile producat , } sed quod possit sibi simile producere : \\\hline
2.1.8 & que sea segunt natura \textbf{ e para que entre el uaron e la muger sea amistança natural conuiene que guarden vno a otro fe e lealtad } assi que non se puedan partir vno de otro . & ad hoc quod coniugium sit secundum naturam , \textbf{ et ad hoc quod inter uxorem et virum sit amicitia naturalis , | oportet quod sibi inuicem seruent fidem , } ita quod ab inuicem non discedant . \\\hline
3.2.8 & e la fuente delas esc̀ yturas . \textbf{ conuiene que de ally tome todo el pueblo algun enssennamiento } e de prinda alguna sabiduria . & et fons scripturarum , \textbf{ oportet quod inde totus populus } aliquam eruditionem accipiat : \\\hline
3.2.18 & para fazer tales cosas \textbf{ Morende para que alguno faga fe delas cosas de que fabla o conuiene que sea sabio } o que sea tenido por sabio . & existimantur talia facere : \textbf{ ideo ad hoc quod aliquis ex rebus | de quibus loquitur fidem faciat , vel oportet quod sit prudens } vel quod credatur esse prudens . \\\hline
3.2.18 & a cuyos dichos creen los omes \textbf{ e es dada feo conuiene que sea bueono } que sea amigo & vel omnis ille cuius dictis creditur \textbf{ et adhibetur fides , | vel oportet quod sit bonus , } vel quod amicus , \\\hline
3.2.26 & en quanto es conparada ala ley de natura \textbf{ conuiene que sea derechͣ . } Et en quanto es conparada al bien comun & ad legem naturae , \textbf{ oportet quod sit iusta : } ut comparatur ad bonum commune , \\\hline
3.2.26 & el qual pueblo deueser reglado por aquella ley . \textbf{ conuiene que sea conuenible } e que conuenga con el uso & et debet regulari per huiusmodi legem , \textbf{ oportet quod sit competens } et compossibilis consuetudini patriae et tempori : \\\hline
3.2.29 & assi commo por regla de fierro . \textbf{ Mas conuiene que se reglen con regla de plommo } que se pueda en coruar & ut puta ferrea : \textbf{ sed oportet | quod mensurentur regula plumbea , } quae sit applicabilis humanis actibus . \\\hline

\end{tabular}
