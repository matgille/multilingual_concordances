\begin{tabular}{|p{1cm}|p{6.5cm}|p{6.5cm}|}

\hline
1.1.5 & et agonibus \textbf{ non coronantur fortissimi , } sed agonizantes : & o es aquellas batallas \textbf{ non son coronados los muy fuertes } mas los bien lidiantes \\\hline
1.1.5 & sed agonizantes : \textbf{ qui enim fortissimi sunt , } potentes agonixare , & mas los bien lidiantes \textbf{ ca los que son muy fuertes pueden lidiar . } Enpero si non lidiaren de fech̃o non les es deuida corona ¶ \\\hline
1.2.12 & Unde 5 Ethic’ dicitur , \textbf{ quod praeclarissima virtutum videtur esse Iustitia : } et neque Hesperus , & Et por ende dize el philosofo \textbf{ en el quinto libro de las ethicas que paresçe que la iustiçia es muy mayor | e mas respladeciente entre todas . } las uirtudes . \\\hline
1.2.12 & Si ergo decet Reges et Principes \textbf{ habere clarissimas virtutes } ex parte ipsius Iustitiae , & Et pues que assi es si conuiene alos Reyes \textbf{ e alos prinçipes de auer | muy claras } uirtudes paresçe de parte dela iustiçia \\\hline
1.2.12 & ex parte ipsius Iustitiae , \textbf{ quae est quaedam clarissima virtus , probari potest , } quod decet eos obseruare Iustitiam . & uirtudes paresçe de parte dela iustiçia \textbf{ que es muy clara uirtud | que se puede prouar } que conuiene alos Reyes \\\hline
1.2.14 & unde ait Philosophus , \textbf{ quod secundum hanc Fortitudinem fortissimi videntur esse } apud illas gentes , & Onde dize el philosofo \textbf{ que segunt esta manera de fortaleza | aquellos son dichos muy fuertes } que quieren gauar honrra entre aquellas gentes . \\\hline
1.2.21 & opus optimum , \textbf{ et decentissimum , } quam qualiter , & en qual manera faga obra muy buena \textbf{ e muy conuenible que entender en qual manera } e quanta despenssa fara en aquella obra . \\\hline
1.2.31 & secundum suam facultatem sunt vere , \textbf{ et perfecte liberales propinquissimum est , } ut sint magnifici : & Empero si los pobres segunt su poder son uerdaderamente \textbf{ e acabadamente liberales muy cercanos son para ser magnificos } por que si abondassen en los bienes de fuera \\\hline
2.1.15 & et dirigitur \textbf{ a sapientissimos artifice } ut a Deo , nihil agit superfluum , & por que es mouida \textbf{ e gada de maestro muy sabio } assi commo de dios \\\hline
2.1.15 & et quicquid natura praeparatur , \textbf{ oportet ordinatissimum esse : } quia ille naturam dirigit , & por la natura \textbf{ conuiene que sea muy ordenado . } Ca aquel gnia la natura de que viene todo ordenamiento \\\hline
2.2.21 & in locutionem incautam , \textbf{ haec videtur esse potissima , } ut nullum sermonem proferat , & para fablar la cosa \textbf{ que non ha penssada esta paresçe muy grande } que ninguno non diga \\\hline
3.1.2 & et aliud virtuose viuere . \textbf{ Nam esse latissimum est , } ut dicitur in libro de Causis : & e otra cosa es beuir uirtuosamente \textbf{ por que el seres cosa muy general | e muy ancha } assi commo es dicho en el libro de causis \\\hline
3.1.7 & arguitur esse summe bonus . \textbf{ Videtur ergo ciuitas esse potissime bona , } si sit potissime una ; & que dios es muy bueno \textbf{ e por ende paresçe que la çibdat es muy | buenasi fuere muy vna . } Et pues que assi es quanto mas se allega a vnidat \\\hline
3.2.4 & cum ipse pluries dicat in eisdem politicis , \textbf{ regnum esse dignissimum principatum : } inter principatus enim rectos , & en esse mismo libro delas politicas \textbf{ que el regno es prinçipado muy digno } por que entre los prinçipados derechs el prinçipado de vno \\\hline
3.2.4 & Censendum est igitur , \textbf{ regnum esse dignissimum principatum , } et secundum rectum dominium melius est dominari unum , & Et pues̃ que assi es deuemos otorgar \textbf{ que el regno es prinçipado muy digno } e segut derech \\\hline
3.2.7 & Secunda , ex eo quod est maxime innaturale . \textbf{ Tertia , ex eo quod est efficacissimum ad nocendum . } Quarta , ex eo quod impedire habet & muchodes natural ¶ \textbf{ La terçera se toma | por razon que tal prinçipado es muy afincado por enpesçer . } la quarta por razon que tal prinçipado ha de enbargar \\\hline
3.2.7 & quod sicut Regnum est optima \textbf{ et dignissima politia , } sic tyrannis est pessima : & Et esta razon tanne el philosofo çerca el comienço del quarto libro delas politicas . \textbf{ do dize que assi conmo el regno es muy buena et muy digna poliçia . } assi la tirania es muy mala \\\hline
3.2.7 & ex eo quod talis principatus \textbf{ est efficacissimus ad nocendum . } Nam sicut principatus Regis & ¶La terçera razon se toma \textbf{ por que tal prinçipado es muy afincado para enpeesçer . } casi commo el prinçipado del Rey \\\hline
3.2.7 & eo quod sit maxime unitus , \textbf{ est efficacissimus ad proficiendum : } sic tyrannis efficacissima ad nocendum . & casi commo el prinçipado del Rey \textbf{ por que es muy vno es muy afincado para aprouechar . } Assi la tirania es muy afincada para enpees çer \\\hline
3.2.7 & est efficacissimus ad proficiendum : \textbf{ sic tyrannis efficacissima ad nocendum . } Monarchia enim quia ibi dominatur unus , & por que es muy vno es muy afincado para aprouechar . \textbf{ Assi la tirania es muy afincada para enpees çer } ca el senñorio de vno \\\hline
3.2.8 & et Rex regum , \textbf{ a quo rectissime regitur } uniuersa tota natura : & e Rey de los Reyes \textbf{ por el qual es gouernada muy derechamente toda la natura del mundo . } Por ende del gouernamiento \\\hline
3.2.12 & et ligari : \textbf{ et supra caput eius acutissimum gladium } pendentem tenuissimo filo apponi fecit : & e fizola tar \textbf{ e fizol colgar una espada sobre su cabesça muy aguda de vn filo muy delgado } e fizo \\\hline
3.2.12 & et supra caput eius acutissimum gladium \textbf{ pendentem tenuissimo filo apponi fecit : } circa ipsum quosdam homines cum ballistis , sagittis appositis , & e fizola tar \textbf{ e fizol colgar una espada sobre su cabesça muy aguda de vn filo muy delgado } e fizo \\\hline
3.2.19 & ut eligens optimum modum principandi , \textbf{ ferat leges iustissimas , } secundum quas saluari habet principatus ille . & por que escogiendo la meior manera de prinçipar o de \textbf{ enssennorear | ponga leyes muy derechos . } segunt las quales aquel prinçipado se ha de saluar . \\\hline
3.2.20 & per legem omnia determinari , \textbf{ et quam paucissima arbitrio iudicum committere . } Secunda via sic ostenditur , & por leyes \textbf{ e muy pocas sean acomnedadas al aluedrio de los uiezes } ¶ \\\hline
3.2.20 & omnia lege determinare , \textbf{ et quam paucissima arbitrio iudicum committere . } Has autem tres rationes tangit Philosophus 1 Rhetoricorum dicens & por las leyes \textbf{ e muy pocas cosas sean dexadas en poder | e en aluedrio de los iuezes . } Et estas tres cosas tanne elpho \\\hline
3.2.20 & quaecunque possibile est determinare : \textbf{ et quam paucissima committere iudicantibus . } Primum quidem quia facilius est habere & quanto pueden ser todas las cosas \textbf{ et que muy pocas cosas sean dexadas alos iuezes ¶ } Lo primero por que mas ligeramente pueden auer los omes vn sabio o pocos que muchos . \\\hline
3.3.4 & Nam , ut dicitur 3 Ethic’ \textbf{ apud illos sunt viri fortissimi , } apud quos honorantur fortes . & en el tercero libro de las Ethicas \textbf{ entre aquellos son los varones muy fuertes entre los quales los fuertes son muy honrrados . } Mas entre todas las cosas \\\hline
3.3.11 & quae requiruntur ad bellum . \textbf{ Mors est quid terribilissimum , } et finis omnium terribilium , & que son menester para la batalla . \textbf{ l lA muerte es cosa muy espantable } e fin de todas cosas \\\hline
3.3.11 & et in qualibet acie \textbf{ habere aliquos equites fidelissimos et strenuissimos , } habentes equos veloces et fortes ; & que deue el señor de la hueste en cada conpaña \textbf{ e en cada vna az auer vnos caualleros muy fieles | e muy estremados } que ayan cauallos muy ligeros \\\hline
3.3.11 & Itaque cum pericula visa minus noceant , \textbf{ per velocissimos equites sunt detegendae insidiae , } ne exercitus circa aliquam partem ex improuiso patiatur molestias . & Et por ende por que los periglos que son ante vistos menos enpeesçen . \textbf{ por caualleros muy ligeros son de descobrir las çeladas } por que la hueste non aya de resçebir a desora en alguna parte algunos daños . \\\hline
3.3.18 & munitiones aliquas obsessas \textbf{ super lapides fortissimos esse constructas , } vel esse aquis circumdatas , & e assi podran ganar aquellas fortalezas . \textbf{ m muchas uegadas contesçe que algunas fortalezas çercadas son fundadas sobre pennas muy fuertes } o son cercadas de agua \\\hline
3.3.18 & vel esse aquis circumdatas , \textbf{ vel habere profundissimas foueas , } vel aliquo alio modo esse munitas : & o son cercadas de agua \textbf{ o han carcauas muy fondas } o son \\\hline
3.3.19 & ideo appellatur aries , \textbf{ quia ratione ferri ibi appositi durissimam habet } frontem ad percutiendum . & Ca por razon del fierro \textbf{ que ponen y . | ha muy fuerte et muy . } dura fruente para ferir \\\hline
3.3.21 & ( ut ait Vegetius ) \textbf{ illae pudicissimae foeminae } cum maritis conuiuere deformato capite , & Ca dize vegeçio \textbf{ que mas quisieron aquellas buenas mugeres muy castas beuir con sus maridos trasquiladas } que non yr con sus enemigos con cabellos . \\\hline
3.3.22 & Contra hoc autem constituitur \textbf{ quoddam ferrum curuum dentatum dentibus fortissimis , } et acutis , et ligatum funibus , & et contra este \textbf{ carnero se puede fazer vn fierro | coruo dentado de dientes muy fuertes e muy agudos } e atado con fuertes cuerdas \\\hline

\end{tabular}
