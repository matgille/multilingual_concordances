\begin{tabular}{|p{1cm}|p{6.5cm}|p{6.5cm}|}

\hline
1.1.5 & consequi suum finem , \textbf{ vel suam felicitatem , } habere praecognitionem ipsius finis . & que quiera alcançar e auer su fin \textbf{ e la su bien andança de auer } ante algun conosçimjento dela su fin e dela su bien andança . \\\hline
1.1.5 & duplici via venari possumus , \textbf{ quod expedit regi suum finem cognoscere . } Prima est , & e cobrar prouar \textbf{ que conujene al rrey | en toda manera de conosçer la su fin ¶ } La primera rrazon es en quanto el rrey \\\hline
1.1.5 & Prima est , \textbf{ inquantum per sua opera cooperatur , } ut sit finis consecutiuus . & La primera rrazon es en quanto el rrey \textbf{ por las sus obras ayuda asy mesmo } por que aya e alçen la su fin \\\hline
1.1.5 & quod maxime decet regiam maiestatem \textbf{ cognoscere suam felicitatem , } ut opera communia , & que muy mas conuiene al Reio al prinçipe conosçer la su fin \textbf{ e la su bien andança | que a otro ninguno . } por que pueda fazer buenas obras e comunes \\\hline
1.1.6 & quod non decet aliquem hominem \textbf{ suam felicitatem ponere } in delectationibus sensibilibus . & qua non conuiene a ningun omne \textbf{ de poner la feliçidat suia | e la su bien andança } enlas delectaçiones sensibles e dela carne ¶ \\\hline
1.1.7 & Cum ergo finis maxime diligatur , \textbf{ ponens suam felicitatem in numismate , } principaliter intendit reseruare sibi , & a aquel que pone las un feliçidat \textbf{ e la su bien andança en las riquezas | e en los aueres } prinçipalmente entiende de thesaurizar e fazer thesoro e llegar muchos dineros \\\hline
1.1.7 & omni via qua potest , \textbf{ velle consequi suum finem . } Est igitur Rex Tyrannus , & por ninguna manera \textbf{ que non pueda querer seguir la su fin | ante se trabaia dela alcançar quanto puede ¶ } Pues que assi es el Rei es tirano \\\hline
1.1.8 & erit praesumptuosus , et erit iniustus , et inaequale . \textbf{ Nam si Princeps suam felicitatem in honoribus ponat , } cum sufficiat ad hoc & ¶Lo primero se muestra assi . \textbf{ Ca el prinçipe | si pusiere la su bien andança en honrras } conmoabaste acanda vno \\\hline
1.1.8 & nam cum finis maxime diligatur , \textbf{ si Princeps suam felicitatem in honoribus ponat , } ut possit honorem consequi , & Ca commo cada vn omne mucho ame la su fin \textbf{ en que pone la su bien andança | si el prinçipe pusiere la su bien andança enlas honrras } por que pueda delo que feziere honrra alcançar \\\hline
1.1.8 & quo plus honoris consequi possit . \textbf{ Si ergo Rex suam felicitatem in honoribus ponat , } erit malus in se , & Et por ende ençerrado todo lo \textbf{ que dicho es en este capitulo | si el Rei pusiere la su feliçidat } e la su bienandança en las honrras sera malo en si mesmo \\\hline
1.1.9 & Videtur ergo quod maxime Princeps \textbf{ in hoc suam felicitatem ponere debeat , } dicente Philosopho 5 Ethic’ & que los prinçipes deuen poner mayormente la su feliçidat \textbf{ e la su bien andança en la eglesia | e en la } honrra¶por que dize el philosofo \\\hline
1.1.11 & nec aliquem hominem in talibus \textbf{ suam felicitatem ponere , } quae sunt corporalia , & nin a ningun omnen poner la su feliçidat \textbf{ e la su bien andança en tales cosas } por que son corporales \\\hline
1.1.12 & Quare si minister , \textbf{ suam mercedem , } et suum praemium debet & e los seruientes del señor \textbf{ deuen poner la su merçed } e el su \\\hline
1.1.12 & qui est Dei minister , \textbf{ suam felicitatem ponere in ipso Deo , } et suum praemium expectare ab ipso . & que es ofiçial de dios \textbf{ poner la su bien andança en dios que es prinçipal señor } e del solo deue esperar \\\hline
1.1.12 & In eo ergo debet \textbf{ suam felicitatem ponere , } quod est maxime , & mientesal bien comun de todos . \textbf{ Et por ende deue poner la su feliçidat } e la su bien andança \\\hline
1.1.12 & et tum quia intendit bonum commune , \textbf{ debet suam felicitatem ponere in Deo , } cui seruit , & Et lo otro por que deue te çier mientes al bien comun \textbf{ deue pener la su feliçidat | e la su bien andança en dios } a quien deue seruir . \\\hline
1.1.12 & Si ergo Rex debet in Deo \textbf{ ponere suam felicitatem , } oportet ipsum huiusmodi felicitatem ponere & ¶ Et pues que el Rey deue poner la su feliçidat \textbf{ e la su bien andança en dios . } Conuiene le dela poner en la obra de aquella uirtud \\\hline
1.2.1 & ostendentes in quo Reges et Principes \textbf{ suam felicitatem debeant ponere , } quia non decet & e los prinçipes la su feliçidat \textbf{ e la su bien andança . } Et que non los conuiene poner la su fin en riquezas \\\hline
1.2.1 & quia non decet \textbf{ eos suum finem ponere in diuitiis , } nec in ciuili potentia , & e la su bien andança . \textbf{ Et que non los conuiene poner la su fin en riquezas } nin en poderio çiuil \\\hline
1.2.7 & Secundo studere debet , \textbf{ ne suus principatus in tyrannidem conuertatur . } Tertio studere debet , & Lo segundo deue estudiar el Rey \textbf{ que el su prinçipadgo | e el su sennorio non se torne en tirania } que es señorio malo e desigual \\\hline
1.2.7 & Est enim Regis officium , \textbf{ ut suam gentem regat , } et dirigat in debitum finem . & e de dignidat \textbf{ por que el ofiçio del Rey es que gouierne e guie la su gente . } la qual cosa muestra el nonbre del rey . \\\hline
1.2.8 & Modus enim , \textbf{ quo Rex suum populum dirigit , } oportet quod sit humanus , & e razon o conuiene le que sea entendido e razonable . \textbf{ Ca la manera por que el Rey guia el su pueblo } Conuiene que sea manera de omne . \\\hline
1.2.9 & quid agendum sit in futurum . \textbf{ Nam semper debet suum regimen conformare regimini retroacto , } sub quo regnum tutius , & en lo que ha de venir . \textbf{ Ca sienpre deue el Rey conformar e ordenar el su gouernamiento | segunt el gouernamiento del tp̃o passado } en el qual el su regno meior \\\hline
1.2.11 & in quo ille abundat . \textbf{ Ideo ut quilibet suae indigentiae prouideret , } inuenta fuit commutatiua Iustitia . & enla qual abonda el otro . \textbf{ Et por ende por que cada vno pudiesse proueer a la su mengua } sue fallada la iustiçia mudadora \\\hline
1.2.11 & et sibi inuicem \textbf{ secundum quandam commutationem suis indigentiis satisfaciunt , } est in eis commutatiua Iustitia , & o de vn regno han ordenamiento entre si mismos \textbf{ e se acorren a las sus menguas los vnos alos otros mudando | e dando las vnas cosas por las otras . } es en ellos la iustiçia mudadora \\\hline
1.2.12 & quae est valde pulchra , et clara : \textbf{ et propter sui pulchritudinem , } et venustatem communi nomine & que es muy fermosa e muy clara \textbf{ e por la su fermosura } e por la su claridat es llamada renꝮ \\\hline
1.2.14 & ipsi tamen debent esse fortes fortitudine virtuosa , \textbf{ ut non exponant suam gentem periculis bellicis , } nisi habeant iusta bella , & Enpero ellos deuen ser fuertes de fortaleza uirtuosa \textbf{ por que non pongan la su gente | e el su pueblo a periglos de batallas } si non quando ouieren razon derecha para auer batalla . \\\hline
1.2.17 & ut se diligat , \textbf{ et ut sua bona custodiat . } Dare autem propria bona , & Ca cada hun omne es naturalmente inclinado a amar asi mismo \textbf{ e aguardar los sus biens propos } Mas dar los sus biens propios ha alguna guaueza por si . \\\hline
1.2.18 & Tertio huiusmodi virtus dicitur communicabilitas : \textbf{ quia per eam homines communicant sua bona , } per quam communicationem ab aliis potissime diliguntur : & ¶ Lo terçero esta uirtud es dicha franqueza \textbf{ por que por ellas los omes parten los sus bienes } por la qual participaçion se departen estri̊madamente de los otros . \\\hline
1.2.31 & et perficiat habentem , \textbf{ et opus suum bonum reddat : } cum ad bene eligere , & e acaba a aquel que la ha \textbf{ e faga la su obra buena . } Por ende commo havien escoger \\\hline
1.2.32 & Cum enim qui alios conuiuare volebat , \textbf{ si filius suus domi non erat , } a vicino suo mutuabat filium , & Ca quando alguno quaria conbidar a \textbf{ otrossi el su fiio non era en casa tomaua prestado el fijo de otro su vezino } e aprestaual para fazer el conbit \\\hline
1.3.8 & Sed hi omnem delectationem condemnantes , \textbf{ statim suam positionem ostendebant reprehensibilem : } quia ( secundum Philosoph’ ) & Mas todos estos que despreçia un a todas las delectaçiones \textbf{ luego mostra una | que la su posicion era de reprehender . } Ca segunt el philosofo \\\hline
1.3.11 & de prosperitatibus malorum , \textbf{ ne indignis distribuant sua bona . } Sic ergo se habere debent & dela bien andança de los malos \textbf{ en quanto ellos non deuen partir los sus bienes alos malos . | nin alos que non son dignos . } Et por ende assi se deuen auer los Reyes alas pasiones sobredichos \\\hline
1.4.2 & Non enim putant alios esse malos , \textbf{ sed sua innocentia alios mensurant . } Cum ergo naturale sit , & que son malos . \textbf{ Mas por la su moçençia | e por la su sinpleza mesuran alos otros . } Et pues que assi es commo natural cosa sea \\\hline
1.4.3 & et innocentes sunt , \textbf{ sua innocentia alios mensurant , } et omnia referunt in meliorem partem : & e non han fecho muchos males \textbf{ por la su sinpleza et inoçençia iudgan todos los otros . } Et todas las cosas retuerçen ala meior parte \\\hline
1.4.4 & ne per hoc iudicentur leues et indiscreti . \textbf{ Quarto in suis actionibus debent habere moderationem et temperamentum : } quia ( ut dictum est ) & e de poco saber ¶ lo quarto los Reyes \textbf{ e los prinçipes deuen auer | en las sus obras mesura e tenpramiento } por que assi commo dicho es ellos \\\hline
1.4.5 & Nobiles ergo , \textbf{ quia ex suo genere videntur } esse honorabiles , & Et por ende los nobles \textbf{ por que son honrrados | por el su linage } por ende quieren acresçentar aquella honrra \\\hline
1.4.5 & Esse autem elatum , \textbf{ et despicere suos progenitores , } et nimis esse honoris cupidi , & por que sienpre es mas antigua¶ \textbf{ Mas ser sobrauios e despreçiar los sus engendradores } e ser muy cobdiçiosos de honrra \\\hline
1.4.6 & nisi fugiant malos mores ipsorum diuitum , \textbf{ et nisi suas diuitias ordinent ad bonum , } et ad opera virtuosa . & si non se arte draten delas malas costunbres de los ricos \textbf{ e si non ordenar en las sus riquezas abien o a obras de uirtud } ¶ \\\hline
1.4.7 & quod tamen est nobilis , \textbf{ et ab antiquo sui progenitores diuites extiterunt , } melius nouit diuitias supportare , & que es noble \textbf{ e de antigo tienpo los sus auuelos fueron ricos meior sabe sofrir las riquezas } e por ellas non se leu nata en so ƀͣuia \\\hline
2.1.3 & non quia lapides illud egerint , \textbf{ sed quia sui progenitores fecerunt illud , } quare sicut communicatio ciuium ciuitas nominatur , & o aquello non por que las piedras fizieron esto \textbf{ mas por que los sus padres o los sus fujos lo fizieron . } Por la qual razon \\\hline
2.1.10 & quanta diligentia debeant \textbf{ ad suam pudicitiam obseruare , } et quanto conatu fidem suis viris obseruent : & que las que son ayuntadas a sus maridos \textbf{ por matmonio con quanta diligençia deuen guardar la su honestad } e con quanto esfuerço deuen guardar la fialdat \\\hline
2.1.13 & ut filii polleant magnitudine corporali , \textbf{ quaerere in suis uxoribus magnitudinem corporis : } tanto tamen magis hoc decet Reges et Principes , & e por que los fijos dellos \textbf{ resplandezcan por grandeza de cuepo de demandar en las sus mugers grandeza de cuerpo . } Enpero tanto mas esto conuiene alos Reyes \\\hline
2.1.13 & decet eos \textbf{ in suis uxoribus quaerere magnitudinem , } et pulchritudinem corporalem : & por fiios grandes e fermosos . \textbf{ Conuiene a ellos de demandar en las sus mugieres grandeza e fermosura corporal . } Ca paresçe que la fermosura dela muger \\\hline
2.1.13 & Decet ergo omnes ciues \textbf{ hoc in suis coniugibus quaerere : } tanto tamen hoc decet Reges et Principes , & Et pues que assi es conuiene a todos los çibdadanos \textbf{ de demandar esto en las sus mugers } Empero tanto conuiene esto mas alos Reyes \\\hline
2.1.14 & et conuentiones \textbf{ quasdam in suo regimine obseruare . } Ex ipso ergo modo regendi , & e las con diconnes \textbf{ que deue el guardar en el su gouernamiento . } Et pues que assi es dela manera del gouernar \\\hline
2.1.21 & si non esset moderata , \textbf{ et ultra quam suus status requireret , } appeteret ornamenta . & Et si \textbf{ dessease los honrramientos e conponimientos del cuerpo | mas que demanda el su estado . } ¶ Lo terçero conuiene alas mugers de ser \\\hline
2.1.21 & vituperat Laconios , \textbf{ qui infra suum statum vestimenta quaerentes } ex hoc in elationem et iactantiam mouebantur . & que llaman latonios \textbf{ por que quarian vestiduras mas viles | que el su estado demandaua . } Et por esta razon se mouiana so ƀiuia e a alabança . \\\hline
2.1.23 & esse muliebre consilium melius quam virile : \textbf{ ut quia illud est citius in suo complemento , } sic oporteret repentino operari , & que del omne \textbf{ por que el consseio de la muger es mas ayna el su conplimiento que deluats . } por que si acaesçiesse de obrar alguna cosa adesora \\\hline
2.1.24 & Ex hoc autem de facili apparet , \textbf{ qualiter viri suis coniugibus debeant } reuelare secreta . & Et por ende de ligero paresçe \textbf{ en qual manera los maridos de una descobrir a sus mugieres los sus secretos . } Ca quando nos dezimos \\\hline
2.1.24 & Viri igitur non debent \textbf{ suis coniugibus secreta aperire , } nisi per diuturna tempora sint experti , & Et pues que assi es los maridos \textbf{ non deuen descobrir a sus mugers las sus poridades } saluo a aquellas de que han prouado de luengot \\\hline
2.2.7 & et maxime Reges , et Principes , \textbf{ si volunt suos filios distincte } et recte loqui literales sermones , & e alos prinçipes \textbf{ si quisieren | que los sus fijos departidamente } e derechamente fablen las palabras delas letras \\\hline
2.2.20 & Quia igitur omnes delectantur in propriis operibus , \textbf{ et omnes diligunt sua opera , } ut vult Philosophus 9 Ethicorum , & en sus obras propreas \textbf{ e todos aman las sus obras } assi commo dize el philosofo \\\hline
2.3.12 & habere curam de acquisitione pecuniae , \textbf{ secundum quod exigit suus status : } Apud Reges autem , & segunt uida politica de auer cuydado de ganar dineros segunt que requiere \textbf{ e demanda el su estado de cada vno . } Mas alos Reyes e alos prinçipes \\\hline
2.3.17 & nec aeque pulchris indumentis gaudere debent , \textbf{ sed considerata conditione personarum sic secundum suum statum cuilibet sunt talia tribuenda , } ut in hoc appareat prouidentia et industria principantis . & nin deuen gozar egualmente de uestiduras fermosas . \textbf{ Mas penssada la condiçion delas personas | assi se deue partir acadera vno dellos segunt el su estado } por que en esto parezca la sabiduria \\\hline
3.1.20 & Hippodami ergo opinionem recitauimus , \textbf{ quia in sua politia multas bonas sententias promulgauit : } aliqua tamen incongrue statuit . & e por ende contamos la opinion de ipodomio \textbf{ por que el en la su poliçia manifesto muchͣs bueanssmans . } Empo algunas cosas establesçio non conuenible mente . \\\hline
3.2.6 & nisi de delectationibus propriis , \textbf{ maxime versatur sua intentio circa pecuniam , credens se per eam posse huiusmodi delectabilia obtinere . } Sed regis intentio versatur circa virtutem , & si non delas sus delectaçonnes propreas . \textbf{ Et por ende la su entençion toda se pone en el auer | o en los dinos creyendo que por ellos puede auer las otras cosas delectables . } Mas la entençion del Rey esta \\\hline
3.2.8 & Debet igitur Rex solicitari \textbf{ ut in suo regno uigeat studium litterarum , } et ut ibi sint multi sapientes et industres . & Et por ende el rey deue ser muy acuçioso \textbf{ por que en el su regno aya estudio de letris } e por que sean y muchos sabios \\\hline
3.2.19 & Rursus est attendendum , \textbf{ ne in suis prouentibus defraudetur : } expedit enim regium consilium & si tomasse los bienes \textbf{ de aquellos que son en el su regno sin derech | lo segundo ha de tener mient̃s el Rey de non ser engannado enlas sus rentas . } ca conuiene que el conseio del Rey sea bue no para saluar \\\hline
3.2.32 & ut alii suam magnificentiam perciperent , \textbf{ et ut eis sua bona communicare posset , } non multum reputaret illa . & e su magnifiçençia \textbf{ e por que les pudiesse dar de los sus bienes non termie todos aquellos bienes en much . } Et pues que assi es la çibdat fue fecha \\\hline
3.3.10 & habere omnium armorum exercitium , \textbf{ ut possit suos commilitones de pugna erudire , } ut fortiter pugnent , arma tergant , & e que aya uso en todas las armas \textbf{ por que pueda ensseñar todos los sus caualleros a la batalla } por que lidien fuertemente e quel alinpien las armas \\\hline
3.3.12 & et quibus cautelis abundare decet bellorum ducem \textbf{ ne suus exercitus laedatur } in via quantum ad campestrum bellum . & Et quales cautelas ha de auer el señor de la batalla \textbf{ por que la su hueste non sea dañada en el camino . } Et este quanto a la batalla del canpo \\\hline
3.3.23 & et omnem modum \textbf{ per quem possint suos hostes vincere , } quod totum ordinare debent & que los Reyes e los prinçipes ayan batalla derecha \textbf{ et los sus enemigos turben la paz } e el bien comun a tuerto non es cosa sin guisa \\\hline

\end{tabular}
