\begin{tabular}{|p{1cm}|p{6.5cm}|p{6.5cm}|}

\hline
1.1.2 & et aliis habitibus , praestituimus nobis alios et alios fines . Nam ( ut dicitur 3 Ethic’ ) qualis unusquisque est , talis finis sibi videtur : \textbf{ ut in intemperato videtur quod totus finis , } et magna felicitas , & quales cada vno tal fin escoge \textbf{ asy commo aquel que es destenprado en los desseos dela carne paresçe } que toda su bien andança es usar de delectaçiones carnałs bien asy ahun los que han otras disposiconnes departidas son inclinados por ellos a escoger otros fines concordables \\\hline
1.1.3 & quod si hac careat , \textbf{ dato quod per ciuilem potentiam principetur , } magis tamen est dignus subiici quam principari , & njn de gouernar a ninguno \textbf{ et puesto que enssennore | e por poderio çiuił toda uja es } mas digno de ser subdito que prinçipe ¶ \\\hline
1.1.4 & ( ut ibi probatur ) est naturaliter animal sociale , ciuile , \textbf{ et politicum , sequitur quod regatur secundum prudentiam , et viuat vita politica . } Si autem est bestia , & asy commo prueua el philosofo en esse mjsmo libro siguese que el omne deue ser gouernado \textbf{ segunt sabiduria e rrazon derecha | Et de beuir vida politica e ordenada } Mas sy el omne es bestia e peor que omne \\\hline
1.1.5 & habere praecognitionem ipsius finis . \textbf{ Possumus autem dicere quod ( ut ad praesens spectat ) } duplici via venari possumus , & e dela su bien andança . \textbf{ Mas podemosdezer | quanto pertenesçe alo presente } que en dos maneras \\\hline
1.1.5 & et si non apprehenderemus aliquid , \textbf{ ut bonum ultimatum quod voluntatem moueret , } nullum aliud bonum voluntatem mouere posset ; & Et sy nos non conosçiesemos alguna cosa \textbf{ asi commo bien postrimero | e fin postrima } que mouiese lanr̃a uoluntad \\\hline
1.1.6 & nisi delectationes sensibiles : \textbf{ et inde est quod communi nomine } ( ut communiter ponitur ) & si non las delectaçiones sensibles . \textbf{ Et por ende es que communalmente los omes las delecta connes } que son mas sensibles \\\hline
1.1.6 & et ut nunc , \textbf{ si felicitas ponitur esse perfectum bonum , oportet quod sit bonum } secundum intellectum , et rationem : & e de poco tp̃o ¶ \textbf{ Pues si la bien auentraança es bien acabado e conplido . } Conuiene que sea bien segunt el en tedimiento \\\hline
1.1.6 & Sequens enim delectationes sensibiles , \textbf{ dato quod sit Senex tempore , } quia est Puer moribus , & Ca el que sigue las delectaçonnes carnales \textbf{ puesto que sea uieio entp̃o e en hedat } por que es moço en costunbres \\\hline
1.1.8 & si plene manifestare vult ipsum signatum , \textbf{ oportet quod sit quid notum et manifestum : } intrinseca autem non sunt nobis nota , & si conplida mente quiere demostrar le que significa conuiene \textbf{ que sea conosçida cosa e magnifiesta . } Mas las cosas que son de dentro del alma \\\hline
1.1.8 & filium sic praesumptuosum occidit , \textbf{ non obstante quod dictus filius victoriam obtinuerat ab hoste . } Ne ergo Princeps se praecipitet , et ne nimis praesumat , & e que non fuesen cobdiçon sos de honrra mato a su fijo presunptuoso \textbf{ e soƀuio commo quier que aquel su fijo ouiese auido uictoria de los sus enemigos ¶ } Pues que assi es el prinçipe \\\hline
1.1.9 & distinguere \textbf{ inter gloriam , et famam : diceremus quod fama oritur ex gloria : } erit ergo hic ordo , & Ca la fama es vn claro conosçimiento con loor . \textbf{ Enpero si quisieremos fazer depart ineto entre la fama e la eglesia diremos que la fama nasçe de la eglesia } Pues que assi es esta es la orden destas cosas \\\hline
1.1.9 & Sicut enim ad hoc quod aliquis honoretur , \textbf{ sufficit quod exterius bonus appareat : } sic ad hoc quod aliquis sit apud homines in fama , & para alguno sea honrrado abasta \textbf{ que aparescabueon assi ꝑan } que alguno sea en fama e en eglesia entre los omes abasta \\\hline
1.1.10 & Hoc enim ( secundum ipsum ) \textbf{ est quod Romam maxime exaltauit , } quia maxime dederunt operam rebus bellicis , & que el dize \textbf{ que mucho exalço la çibdat de Roma } por que sobre todo lo al se dieron alas obras delas batallas \\\hline
1.1.10 & quando eis libere , \textbf{ et voluntarie principatur quod contingit } si populus libere , & quando ha señorio sobrellos liberalmente \textbf{ e por su uoluntad dellos . | la qual cosa contesçe } si el pueblo liberalmente \\\hline
1.1.12 & suam felicitatem in actu prudentiae , \textbf{ sciendum quod decet Regem maxime suam felicitatem ponere in ipso Deo , } quod triplici via venari possumus . & en qual manera conuenga ala Real magestad \textbf{ de poner la primera feliçidat en las obras de pradençia . ¶ Et la segunda commo le conuiene de poner er la su bien andança solamente en dios . } Esto pondemos prouar por tres razones ¶ \\\hline
1.1.12 & et perfecte solus Deus , \textbf{ oportet quod quicunque principatur , } siue regnat , & e de gouernar prinçipalmente e acabadamente . \textbf{ Conuiene que qual se quier prinçipeo Rey } que ha de gouernar sea instrumento de dios \\\hline
1.1.13 & si quis periculo se exponat , \textbf{ dato quod non transgrediatur , } quia indiscrete agit , suum meritum non augmentatur . Tertio , & por que se ponen a peligro han mayor meresçiminto . \textbf{ Ca puesto que non passen la ley | si non se pusiessen a peligro } por el bien comun menguarian \\\hline
1.1.13 & secundum naturam esse videtur , \textbf{ quod pars exponat se pro toto . Videmus quod quia periclitato capite , } periclitatur & que la parte se ponga \textbf{ por el todo . | Ca veemos que quando ha peligro enla cabesça } ay periglo entondo el cuerpo . \\\hline
1.2.3 & facit enim , \textbf{ quod cuilibet tribuatur quod decet , } vel quod ei detur quod suum est . & lo qual \textbf{ e conuiene } o lo quel deuen dar \\\hline
1.2.3 & quod cuilibet tribuatur quod decet , \textbf{ vel quod ei detur quod suum est . } Si autem moderat , et rectificat passiones , & e conuiene \textbf{ o lo quel deuen dar | o lo que es suyo . } Mas si mesura enderesça las passiones e los mouemientos \\\hline
1.2.3 & Ut ergo liceat typo , et superficialiter pertransire , \textbf{ dicamus quod passiones , } vel surgunt ex bono , & e superfiçialmente digamos \textbf{ que las passiones del alma } o se leuantan de bien o de mal si de mal o de mal de futuro \\\hline
1.2.5 & si virtuosus esse debet , \textbf{ oportet quod fiat prudenter , } iuste , & Ca toda obra si deue ser uirtuosa . \textbf{ conuiene que se faga sabiamente } e iusta mente . fuerte mente . \\\hline
1.2.7 & Viso quid est prudentia , \textbf{ et ostenso quod per prudentiam } recte dirigimur in bonum finem , & isto que cosa es la prudençia \textbf{ e mostrado | que por la pradençia somos enderesçados } e guiados derechamente a buena fin \\\hline
1.2.8 & secundum quem dirigit , oportet ipsum esse intelligentem , et rationabilem : ratione propriae personae \textbf{ quae alios est dirigens , oportet quod sit solers , et docilis : } ratione vero gentis quam dirigit , & por razon de la su propia persona \textbf{ que ha de guiar los otros . | Conuiene le de ser sotil e doctrinable ¶ } Mas por razon dela gente a quien ha de gouernar \\\hline
1.2.8 & ratione vero gentis quam dirigit , \textbf{ congruit quod sit expertus et cautus . } Si enim Rex debet gentem aliquam ad bonum dirigere , & Mas por razon dela gente a quien ha de gouernar \textbf{ Conuiene le de ser prouado | e cauto ete sogedor de bien ¶ } Ca si el Rey ha a guiar la su gente \\\hline
1.2.8 & Si enim Rex debet gentem aliquam ad bonum dirigere , \textbf{ oportet quod habeat memoriam praeteritorum , } et prouidentiam futurorum . & e la su conpanna a alguons bienes . \textbf{ Conuiene que aya memoria de las cosas passadas . } Et que aya prouision delas cosas passadas \\\hline
1.2.8 & et ut habeat memoriam praeteritorum , \textbf{ ut ex actis praeteritis sciat quid agere debeat in futurum . Ratione vero modi per quem dirigit , oportet quod habeat intellectum et rationem , } siue oportet quod sit intelligens & que aya memoria delas cosas passadas \textbf{ por que delas cosas passadas sepa lo que ha de fazer | en lo que ha de venir . } Mas por razon dela manera \\\hline
1.2.8 & ut ex actis praeteritis sciat quid agere debeat in futurum . Ratione vero modi per quem dirigit , oportet quod habeat intellectum et rationem , \textbf{ siue oportet quod sit intelligens } et rationale . Modus enim , & en lo que ha de venir . \textbf{ Mas por razon dela manera | por la qual deue guiar el Rey . } Conuiene le \\\hline
1.2.8 & quo Rex suum populum dirigit , \textbf{ oportet quod sit humanus , } quia Rex ipse homo est . & por que el Rey guia el su pueblo \textbf{ Conuiene que sea manera de omne . } Ca el Rey omne es \\\hline
1.2.8 & qui est inditus hominibus , \textbf{ volens alios dirigere , oportet quod sit intelligens , cognoscendo principia , et praemissa , et rationalis , } ratiocinando , et eliciendo ex illis praemissis cunclusiones intentas . Vel oportet quod sit intelligens , & conuiene le que sea entendido \textbf{ conosciendo los prinçipios e las razones . | Et que sea razonable razonando } e escogiendo de aquellos principios \\\hline
1.2.8 & volens alios dirigere , oportet quod sit intelligens , cognoscendo principia , et praemissa , et rationalis , \textbf{ ratiocinando , et eliciendo ex illis praemissis cunclusiones intentas . Vel oportet quod sit intelligens , } sciendo leges , & Et que sea razonable razonando \textbf{ e escogiendo de aquellos principios | e daquellas razones las conclusiones e las razones } que quiere ençerrar ¶ \\\hline
1.2.8 & Sed ratione propriae personae quae est alios dirigens , \textbf{ oportet quod sit solers , et docilis . } Nam qui in tanto culmine est positus , & que es tal que ha de gouernar los otros . \textbf{ Conuiene le de sor sotil e doctrinable . } Ca aquel que esta en tanta alteza de dignidat \\\hline
1.2.8 & Nam qui in tanto culmine est positus , \textbf{ ut tantam gentem regere habeat , oportet quod sit industris , et solers , } ut sciat ex se inuenire & Ca aquel que esta en tanta alteza de dignidat \textbf{ que es puesto para gouernar tanta gente e tanto pueblo . | Conuiene le que sea engennoso e sotil } por que sepa por si buscar e fallar aquellos bienes \\\hline
1.2.10 & quod est aequum , \textbf{ idest quod sibi debetur . Differentia autem harum Iustitiarum sic potest accipi . } Nam & lo que conuiene \textbf{ e aquello que es suyo . | Mas en otra manera se puede tomar la diferençia destas dos iustiçias . } Ca assi commo dize el philosofo en el quinto libro delas ethicas . \\\hline
1.2.10 & vel ordine ad Ciuitatem : \textbf{ non obstante quod Temperatia , } et Fortitudo , & assi commo en orden al prinçipe o en orden ala çibdat . \textbf{ como quier que la tenperança e la fortaleza } e las otras uirtudes prinçipales acaben \\\hline
1.2.10 & quia potissime innittitur aequalitati , \textbf{ ut quod unusquisque in huiusmodi exterioribus bonis habeat quod aequum est . Inde est ergo quod haec Iustitia dicitur unicuique suum tribuere , } quia ius in quadam aequalitate consistit : & por que prinçipalmente entiende a ygualdat de los çibdadanos \textbf{ assi que cada vno de los çibdadanos aya | lo que es ygual destos bienes de fuera . } Et por ende se sigue \\\hline
1.2.10 & vel aequum est . \textbf{ Sic etiam dicitur unicuique tribuere quod suum est : } quia aequum est , & lo que es e suyo \textbf{ e lo que es igual | Et assi es dicha dar a cada vno } lo que es suyo . \\\hline
1.2.16 & quia insequi voluntates intemperatas , est delectabile : fugere autem et timere , est tristabile . \textbf{ Magis quis voluntarie agit quod facit cum delectatione , } quam quod facit cum tristitia . Peccans igitur per intemperantiam , & Et mas de uoluntad faze \textbf{ cada vno | lo que faze con delectaçion } que lo que faze con tͥsteza . \\\hline
1.2.16 & et non esse constantem animo est exprobrabile , \textbf{ patet quod est exprobrabilius ipsum esse intemperatum , } et insecutorem passionum . Possumus & e non fuer firme en el coraçon es de deno star \textbf{ por ello . | Et es mas de denostar } si fuer deste prado e segnidor de passiones . \\\hline
1.2.16 & cum quidam Dux exercitus diu ei seruiuisset , et fideliter , Rex ille volens complacere illi Duci , \textbf{ praecepit quod duceretur ad ipsum . Dux autem ille assuetus rebus bellicis , } videns Regem suum esse totum muliebrem & mando qual pusiessen dentro ante si . \textbf{ Mas aquel prinçipe | por que era acostunbrado delas batallas } veyendo que el su Rey era todo mugeril \\\hline
1.2.18 & quia etiam sibiipsi nequam est : prodigus autem multis prodest . Omnino ergo detestabile est , Regem esse auarum . \textbf{ Viso quod quasi impossibile est Reges esse prodigos , } et quod omnino detestabile est eos esse auaros : & Et el gastadora muchos aprouecha dando . \textbf{ Et por ende muy de depostar es el Rey | si fuer auariento ¶ visto que los Reyes non pueden ser gastadores } e que muchon son de denostar \\\hline
1.2.20 & et subterfugit quantum potest : \textbf{ sic dato quod paruificum oporteat expensas facere , } tamen illos sumptus tardat , & por que se non taiasse . \textbf{ Bien alłi puesto que el paruifico | e al escasso sea dado de fazer grandes espenssas } sienpre tarda e fuye \\\hline
1.2.25 & ne trahamur ratione difficultatis , \textbf{ oportet quod ei sit annexa humilitas , } ne ultra quam ratio dictet tendat in ea ratione bonitatis , & por razon dela graueza . \textbf{ Et por ende conuiene | que aella sea ayuntada la humildat } por que non pueda passar allende de quanto la razon manda \\\hline
1.2.27 & cum in eis maxime vigere debeat ratio \textbf{ et intellectus . Sicut enim videmus quod lingua infecta per coleram , } vel per alios humores , & que en otros ningunos Ca \textbf{ assi commo veemos | que la lengua enpoçonnada e dessaborada } por colera \\\hline
1.2.27 & et qui debet esse regula agendorum , \textbf{ inconueniens est quod sit iracundus , } ne per iram peruertatur & et que deue seer regla de todas las cosas fazederas . \textbf{ non es cosa conuenible al Rey de ser sañudo . } por que por la saña non sea tristornado nin torçido . \\\hline
1.2.29 & restat videre circa quae habet esse . \textbf{ Sciendum ergo quod licet affirmare in se esse quod non est , } vel negare quod est , & traca quales cosas ha de seer . \textbf{ Et pues que assi es conuiene saber | que maguera firmar cada vno de ser } en ssi aquello que non es o negar aquello que es \\\hline
1.2.29 & Sciendum ergo quod licet affirmare in se esse quod non est , \textbf{ vel negare quod est , } sit mentiri : & que maguera firmar cada vno de ser \textbf{ en ssi aquello que non es o negar aquello que es } en ssi sea mentira \\\hline
1.2.29 & quam sibi inesse non credit . \textbf{ Non tamen oportet quod de se dicat totam bonitatem , } quam sibi inesse cognoscit : & nin conosce \textbf{ que es en ssi . Enpero non conuiene | que dessi mismo daga toda la bondat } que conosçe \\\hline
1.2.29 & cognoscere seipsum , \textbf{ et sciri quod propria bona semper aestimantur maiora quam sint . } Secunda ratio sumitur ex parte aliorum . & e grant sabiduria es conosçer \textbf{ assi mismo . omne e saber | que los sus bienes propreos sienpreles son vistos mayores que son ¶ } La segunda razon se toma de parte de los otros . \\\hline
1.2.31 & et econuerso : \textbf{ non tamen oportet quod habens perfecte unam virtutem moralem , } habeat omnes virtutes morales . Potest enim quis habere perfecte temperantiam , & e el contrario todo sabio es bueno . \textbf{ Enpero non conuiene } que aquel que ha vna uirtud moral aya todas las uirtudes morales \\\hline
1.2.32 & et principari desiderant , \textbf{ oportet quod habeant virtutem illam , } quae est dominans et principans respectu aliarum , & por que aquel que dessea de prinçipar e enssenorear alos otros . \textbf{ Conuienele que aya aquella uirtud } que es sennora \\\hline
1.3.3 & etiam personam exponere , \textbf{ si viderit quod expediat regno . } Erit temperatus ; & e avn non dubdara de poner la persona a muerte \textbf{ siuiere que sea cosa | que conuenga al regno } e avn sera tenprado \\\hline
1.3.5 & de ordine passionum animae , \textbf{ diximus quod amor } et odium erant passiones primae , & uando determinamos dela ança orden delas passiones del alma dixiemos \textbf{ que el amor e la malqreçia eran las primeras passiones } Et el desseo e la aborrençia eran las segundas passiones . \\\hline
1.3.7 & nam si odium est appetitus mali simpliciter , \textbf{ ille quem odimus , non posset tantum habere de malo , quin vellemus quod haberet plus . } Sed ira quae est appetitus poenae , & Por que si la mal querençia es apetito de mal sinplemente \textbf{ aquel que nos mal queremos | non podtia tanto auer de mal } que nos non quisiesemos \\\hline
1.3.7 & et Principes se habere debeant circa iram , et mansuetudinem : \textbf{ sciendum quod ira aliquando rationem praecedit , } et tunc est inordinata et cauenda , & Et pues que assi es \textbf{ por que paresca en qual manera los Reyes e los prinçipesse de una auer çerca la sanna } e cerca la mansedunbre conuiene de saber \\\hline
1.3.8 & nisi delectabile sit ei omnem delectationem fugere ; \textbf{ sequitur quod fugiens omnem delectationem , } sequatur aliquam delectationem . & si non fuere a el delectable de foyr toda delectaçion \textbf{ siguese que aquel que fuye toda delectaçion sigue algunan delectaçion . } Entre estas dos maneras \\\hline
1.3.10 & videlicet , zelum , gratiam , nemesin \textbf{ ( quod idem est quod indignatio de prosperitatibus malorum ) misericordiam , inuidiam , et erubescentiam siue verecundia . } Sed omnes hae passiones reducuntur ad aliquas passiones praedictarum : & gera Njemesim \textbf{ que quiere dezir tanto commo indignacion dela buena andança de los malos . | Misericordia e jnuidia . } e erubesçençia o uerguença . \\\hline
1.4.1 & Cum ergo matres semper moneant suos filios ad honesta ; \textbf{ quia honestum idem est quod honoris status , } iuuenes multum affectant ea quae importare uidentur honoris statum , & por los quales los mocos sson ensseñados e doctrinados . \textbf{ Et pues que assi es commo las muger ssienpre amonestan asus fijos a honestad e a honrrata honestades } al si non estado de honira . \\\hline
1.4.4 & si ultra quam ratio dictet , \textbf{ teneat quod habet . Secundo , } si praeter rationem concupiscat habere quod non habet . & mas de quanto demanda la razon¶ \textbf{ La segunda si contra razon dessean auer } lo que non han . \\\hline
1.4.4 & nec sunt in tot decepti ut senes , \textbf{ nec omnibus credunt quod faciunt iuuenes propter inexperientiam , } nec omnino discredunt & nin son en tantas cosas engannados \textbf{ commo los uieios non creen todas las cosas | assi commo los moços } por que las non han prouado \\\hline
1.4.5 & Nam nobilitas ( ut dicitur 2 Rhetoricorum ) \textbf{ idem est quod virtus generis . } Ex hoc enim aliqui dicuntur esse nobiles , & assi commo dize el philosofo en el primero e en el segundo libͤdelan \textbf{ rectorica es esso mismo } que honrra de linage . \\\hline
1.4.6 & se Deo dona largiri , \textbf{ sed magis cogitaret quod ei reddunt } quod ab ipso accipiunt . & Et por ende non creyrien que ellos dan grandes dones a dios \textbf{ mas cuydarian quel dan aquello } que del resçebie con . \\\hline
2.1.1 & quibus tegimur . \textbf{ Nam sicut natura sufficienter aliis animalibus prouidere videtur in victu : sic videtur quod eis sufficienter prouideat in vestitu . Bestiae enim , et aues , } quasi naturale indumentum , habere videntur lanam , vel pennas . Homini autem non sufficienter prouidet natura in vestitu : & de quanos cobrimos . \textbf{ Ca assi commo la nata conplidamente prouee a todas las otras a inalias en la uida assi paresçe | que les prouee conplidamente en la uestidura . } Ca las bestias e las aues veemos \\\hline
2.1.3 & cum de domo loquimur , \textbf{ sciendum quod domus nominari potest aedificium constitutum ex pariete , } tecto , & or que non trabaiemos en vano fablando dela casa \textbf{ conuiene de saber | que la casa algunas uezes puede ser dicha costruymiento fech̃o de paredes e de techo e de paredesçimientos } o en otra manera puede ser dichͣ la conpanna \\\hline
2.1.3 & et primum in via perfectionis \textbf{ et complementi . Videmus quod ea , quae sunt ad finem , praecedunt finem in opere et in executione : } sed finis praecedit & e primero en carrera de perfectiuo e de conplimiento . \textbf{ Ca veemos que aquellas colas | que son ordenadas a alguna fin } son primero en la obra \\\hline
2.1.3 & nam per hoc magnam viam habebunt ad inuestigandum regimen ciuitatis \textbf{ et regni . Est tamen diligenter notandum quod licet quodam speciali } et excellenti modo & Enpero deuedes saber con grant acuçia \textbf{ que commo quier que en alguna manera espeçial | e alta part enesça alos Reyes } e alos prinçipes de entender \\\hline
2.1.6 & Cum enim primo homo est , \textbf{ oportet quod sit genitus : } et natura statim est solicita de salute eius ; & Ca quando el ome es primero conuiene \textbf{ que sea engendrado . } Et la natura luego es acuçiosa de su salud . \\\hline
2.1.6 & quare si est impotens ad agendum , \textbf{ sequitur quod ei deficiat aliqua forma vel aliqua perfectio , } quae sit principium actionis . & por la qual cosa si non ha poderio de obrar \textbf{ siguese qual mengua alguna forma o alguna perfection e conplimiento } que es comienço de obra \\\hline
2.1.7 & Sed si coniugium est \textbf{ quid naturale , sequitur quod fornicatio , } quae contrariatur coniugio , sit uniuersaliter a ciuibus vitanda , & mas si el ma termonio es cosa natural siguese \textbf{ que la fornicaçion } que es contraria al mater moino \\\hline
2.1.7 & Quare sicut dicebamus de societate politica , \textbf{ videlicet quod eligens solitudinem , } et nolens ciuiliter viuere , & assi conmodiziemos dela uida politica \textbf{ e de çibdat . | Conuiene a saber que el que escoge beuir solo } e non quiere beuir \\\hline
2.1.8 & et ad hoc quod inter uxorem et virum sit amicitia naturalis , \textbf{ oportet quod sibi inuicem seruent fidem , } ita quod ab inuicem non discedant . & que sea segunt natura \textbf{ e para que entre el uaron e la muger sea amistança natural conuiene que guarden vno a otro fe e lealtad } assi que non se puedan partir vno de otro . \\\hline
2.1.8 & semper enim de ratione communis , \textbf{ est quod contineat , uniat , et coniungat , } sicut de ratione proprii , & por que son bien comunal dellos . \textbf{ Ca sienpre es de la razon del bien comun que tenga e ayunte amistança } assi commo dela razon del bien propreo es que ayunte e desayunte el vno del otro . \\\hline
2.1.8 & sicut de ratione proprii , \textbf{ est quod diuidat et distinguat . Hanc autem rationem tangit Philosophus 8 Ethic’ dicens , } quod quia commune continet et coniungit , & assi commo dela razon del bien propreo es que ayunte e desayunte el vno del otro . \textbf{ Et esta razon pone el philosofo en el viii̊ libro delas ethicas } do dize que por que el bien comunal contiene \\\hline
2.1.11 & Tertia via ad inuestigandum hoc idem , \textbf{ sumitur ex malo quod per coniugium vitatur . } Per coniugium enim non solum producitur & para prouar esto mesmo se toma del mal \textbf{ que se puede escusar | por el casamiento . } Ca por el casamiento non solamente viene el bien de los fijos . \\\hline
2.1.15 & sed uni . \textbf{ Quare cum natura ordinauerit coniugem ad generationem , indecens est quod ordinetur ad seruiendum . } Non est ergo naturalis ordo , virum praeesse uxori eo regimine & mas en vna . \textbf{ Por la qual cosa commo la natura aya ordenada la mugr | para la generaçino de los fijos } non conuiene \\\hline
2.1.16 & quia regimen coniugale est aliud a paternali et seruili : \textbf{ et ostendere quod aliter debet se habere vir tam erga uxorem , } quam erga filios , & e que el suil \textbf{ e mostrar | que en otra manera se deua auer el uaron cerca la mugni } que çerca los fijos e cerca los sieruos \\\hline
2.1.16 & si illud sit imperfecte calidum , \textbf{ sequitur quod imperfecte calefaciat . Sic } etiam quia & Ca assi commo para escalentar es menester calentura \textbf{ si aquella calentura non es calentura acabada siguese que non es caliente acabada mente . En essa misma nanera avn pero que algunan cosa sea escalençada es mester } que sea apareiada \\\hline
2.1.16 & ad hoc quod aliquid calefaciat , \textbf{ requiritur quod sit dispositum ad susceptionem caloris ; } si aliquid sit imperfecte dispositum ad huiusmodi susceptionem , & que sea apareiada \textbf{ para resçebir aquella calentura . } Mas si alguna cosa fuesse non apareiada acabadamente \\\hline
2.1.16 & si aliquid sit imperfecte dispositum ad huiusmodi susceptionem , \textbf{ sequitur quod imperfecte calorem suscipiat . } Quare si coniunctio uxoris & para resçebir esta calentura siguese \textbf{ que non resçibrie esta calentura acabada mente . } Por la qual cosa si el ayuntamiento del uaron \\\hline
2.1.16 & si ex tali coniunctione nascantur pueri , \textbf{ sequitur quod producantur imperfecti et debiles corpore ; } quare quantum ad corpora ex coniunctione nimis iuuenili , & si de tal ayuntamiento nasçieren moços siguese que non nasçeran acabados \textbf{ e seran flacos de cuerpo | por la qual cosa quanto al cuerpo } por ayuntamiento dela hedat muy de moços seleunata danno alos fijos . \\\hline
2.1.16 & Unde in Politicis , \textbf{ dicitur quod masculorum corpora laeduntur , } si tempore augmenti et crescente corpore utantur venereis . & Onde en las politicas dize el philosofo \textbf{ que los cuerpos de los | mas los resçiben danno } si en el t pon del cresçer \\\hline
2.1.21 & Sed ut uxores ipsas magis specialiter instruamus qualiter circa indumenta \textbf{ et circa alia corporis ornamenta debeant se habere , aduertendum quod circa ornamentum vestimentorum contingit dupliciter peccare . } Primo ex superabundantia . Secundo ex defectu . Superabundantia vero & mas espeçialmente \textbf{ en qual manera se deuen auer conueiblemente en sus uestiduras e en los otros conponimientos del cuerpo . | C suiene de saber } que çerca los conponimientos delas uestiduras podemos pecar en dos maneras ¶ \\\hline
2.1.21 & et in infirmis existentibus apud ianuas ecclesiarum , \textbf{ ut in plurimum accidit quod infirmior magis gloriatur , } quia credit quod in cum plures aspiciant , & que estan ante las puertas de las eglesias enla mayor parte \textbf{ contesçe | que el mas enfermose eglesia } mas por que cree \\\hline
2.1.21 & ut in plurimum accidit quod infirmior magis gloriatur , \textbf{ quia credit quod in cum plures aspiciant , } et sperat se plures eleemosynas accepturum : & que el mas enfermose eglesia \textbf{ mas por que cree | que muchos catan ael } e es \\\hline
2.1.22 & eo quod tunc magis deficit eis uita . \textbf{ Semper ergo magis concupiscitur quod abest , } et quod uidemus nobis deficere . & por que estonçe les fallesçe \textbf{ mas la vida . ¶ Et pues que assi es mas desseada es la cosa | que omne non ha } e aquello que veemos \\\hline
2.1.24 & cum conglorientur , \textbf{ si possint se laudari quod a suis maritis diligantur , } appetentes quandam inanem gloriam , & Et por ende se glorian \textbf{ si pueden ser loadas | que son amadas de sus maridos . } por ende desseando alguna uana gloria \\\hline
2.2.2 & et quanto maiori intelligentia vigent . \textbf{ Sed cum habitum sit quod Reges , } et Principes & e quanto mayor entendimiento ha en ellos . \textbf{ Mas conmo sea dicho } que los Reyes e los prinçipes \\\hline
2.2.2 & si debeant naturaliter dominari , \textbf{ oportet quod polleant prudentia et intellectu : } tanto decet Reges et Principes magis solicitari circa proprios filios quam ceteri , & e generalmente todos los señores sy de una naturalmente ensseñorear conuiene les \textbf{ que ayan sabidia e entendimiento . } Et tanto mas conuiene alos Reyes \\\hline
2.2.3 & sed sumit originem ex amore tamen , \textbf{ ut manifestius appareat quod dicitur , } possumus duplici via venare , & que el gouernamiento del padre non es tal commo el gouernamiento a uereruo ca toma comienço de amor . \textbf{ Enpero por que mas mani fiestamente paresca } lo que dezimos todemos prouar pardas rasones \\\hline
2.2.3 & quod paternum regimen ex amore nascitur , \textbf{ patet quod filiis debet } praeesse pater propter bonum filiorum . & Visto que el gouernamiento del padre nasçe de amor paresçe \textbf{ que el padre deue enssennorear alos fiios } por el bien de los fijos . \\\hline
2.2.4 & Tamen ut magis specialiter appareat intentum , \textbf{ sciendum quod licet parentes magis afficiantur circa filios , et magis intense velint bonum filiorum quam econuerso : } non est tamen inconueniens quantum ad aliquod bonum filios magis diligere quam econuerso . & que commo quier que los padres mas sear inclinados alas fijos \textbf{ e mas les pertenesca de querer el bien de los fiios | que los fijos de los padres . } Enpero non es cosa desconuenible \\\hline
2.2.4 & cum diligere aliquem , \textbf{ idem sit quod velle ei bonum , } distinguendum est de ipso bono . & que los fijos alos padres \textbf{ commo amara alguno sea essa misma cosa | que querer bien } para el deuemos departir deste bien . \\\hline
2.2.5 & ait , \textbf{ quantam vero vim habeat quod consuetum est , } leges ostendunt , & fuerça dize assi . \textbf{ Tu puedes ver | quanto faze la costunbre } cuydando en las leyes de los moros o de los x̉anos \\\hline
2.2.5 & et maxime pueriscire non possunt : \textbf{ sufficit quod talia proponantur grosse , et in quadam summa : } ut quod eis dicatur , & que parte nesçen ala fe sean dichͣs a los legos \textbf{ e alos mocos gruessamente | e en suma } assi que les sea dicho \\\hline
2.2.8 & quare si debent eis aliqua delectabilia concedi , \textbf{ dignum est quod ordinentur ad delectationes innocuas : } quare ( secundum eundem Philosophum ) musica est consentanea naturae iuuenum , & Por la qual cosa \textbf{ si les son otorgadas algunas cosas delectabłs conuiene que les otorguen cosas delectabło sin daño . } por ende segunt que dize este mismo philosofo la musica es conueinble ala naturaleza de los mançebos por que les muestra delectaçicen sin danno ¶ \\\hline
2.2.9 & tam de inuentis quam de intellectis . \textbf{ Nam et dato quod filii nobilium , } et maxime Regum , & commo delas falladas . \textbf{ Ca puesto que los fijas delos nobles } e mayormente de los Reyes \\\hline
2.2.9 & sed etiam operibus et exemplis ; \textbf{ requiritur quod huiusmodi doctor sit in vita bonus } et honestus . & e por ende conuiene \textbf{ que este doctor e maestro sea enssi bueno e honesto en su uida . } Ca por que la he dar de los moços es muy inclinada a destenpramiento e a loçania \\\hline
2.2.10 & Secundo adhibenda est cautela in iuuenibus , \textbf{ ut instruantur quod palpebras oculorum cum maturitate eleuent , } ut non habeant oculos vagabundos . Inclinatur enim aetas illa ( eo quod omnia respiciat tanquam noua ) & que sean enssennados \textbf{ quanto ala manera de ver | assi que alçen las palpebras de los oios con grand madureza } e que non echen los oios a cada parte con locura . \\\hline
2.2.13 & et circa vestitum debeant se habere . Ludus autem , \textbf{ ut probat Philosophus 8 Poli’ est necessarius in vita quod ( quantum ad praesens spectat ) duplici via declarari potest . Primo , } ex vitatione illicitae solicitudinis . Secundo , & assi commo prueua el philosofo \textbf{ en el viij̊ libro delas politicas | es neçessario enla vida humanal } la qual cosa podemos declarar \\\hline
2.2.15 & et facit ad bonam dispositionem corporis , \textbf{ sequitur quod sit quoddam proficuum ad augmentum . } Nam & e faze abuean disposiçion del cuerpo \textbf{ Et por ende se sigue que sea aprouechoso alacresçentamiento del cuerpo . } Ca commo el acresçentamiento se faga del nudrimiento aquellas cosas \\\hline
2.2.17 & quod si aduenerit tempus \textbf{ et congruitas quod respublica defensione indigeat , habeant corpus sic dispositum , } ut possint tales subire labores , & assi que si viniere tienpo \textbf{ en que la tierra aya meester defendimiento | ayan el cuerpo o bien ordenado } por que puedan tomar trabaios \\\hline
2.2.19 & si adsit nobis commoditas delinquendi , \textbf{ unde et Philosophus in Rheto’ vult quod homines } ut plurimum male faciant , & en la mayor parte pecamos en tales cosas si nos fuere dada manera de pecar \textbf{ Onde el philosofo en la rectorica dize } que los omes en la mayor parte fazen mal quando pueden . \\\hline
2.3.1 & propter quod declarata est prima pars capituli , \textbf{ ubi dicebatur quod specta : } ad gubernationem domus considerare de ministris & Por la qual cosa es declarada la primera parte del capitulo \textbf{ do es dicho queꝑ tenesçe al gouernador dela casa } de auer cuydado de los siruientos \\\hline
2.3.2 & quod declarat Philosophus 1 Polit’ per simile in aliis artibus , \textbf{ ubi innuit quod plectra non per se cytharizant , } et pectines non per se ipsos pectinant . Ideo ad cytharizandum plectrum indiget ministro mouente , & por cosa semeiable en las otras artes \textbf{ do da a entender que el instrumento dela itulero non es | por si çitulador } nin tanne la citula nin el penne \\\hline
2.3.4 & quod considerandum est in aquis , \textbf{ est quod sit coloris perspicui . } Nam ipsa infectio coloris , & que es de penssar en las aguas \textbf{ es que sean de color claro de gnisa } que passe el oio de vna parte a otra por ella \\\hline
2.3.6 & Fuit opinio Socratis et Platonis , \textbf{ ut recitat Philosophus 2 Polit’ quod esset utile } et expediens ciuitati quod ciues propriis possessionibus non gauderent , & ne opinion de socrates e de platon \textbf{ assi commo cuenta el philosofo enł segundo libro delas politicas | que cola aprouechosa } e conueniente serie ala çibdat \\\hline
2.3.6 & quod in ciuitate contingit multos egere et esse pauperes , \textbf{ non obstante quod ciues possunt gaudere possessionibus propriis , } et quod solicitantur circa ea tanquam circa propria bona . & que ninguon non quarrie trabaiar por ellas ca agora en la çibdat lon muchs pobres \textbf{ avn que non contradigamos | que los çibdada nos puedan auer possessiones proprias } e que sean acuçiosos çerca dellas \\\hline
2.3.6 & quilibet ministrorum retrahitur , \textbf{ ne faciat quod mandatur , } sperans alium implere quod iubetur ; & e se tira \textbf{ que non faga aquello qual es mandado elperando } que el otro cunplira aquello que a el es mandado . \\\hline
2.3.6 & ne faciat quod mandatur , \textbf{ sperans alium implere quod iubetur ; } propter quod oportet rem illam vel non produci ad effectum , & que non faga aquello qual es mandado elperando \textbf{ que el otro cunplira aquello que a el es mandado . } Por la qual cosa conuiene \\\hline
2.3.9 & uel numismata regionis unius commutantur in numismata regionis alterius . \textbf{ Ut ergo sciamus quomodo huiusmodi commutationes oportuit introduci , sciendum quod si non esset } nisi communitas domus quae est communitas prima , & Et pues que assi es \textbf{ por que sepamos en qual manera conuiene | que estas tales muda connes fuessen puestas en la tiecra } deuedes saber \\\hline
2.3.10 & Si quis ergo ex decem denariis post aliquod tempus vult habere duodecim quod facit pecuniatiua usuraria , ut plane patet , \textbf{ vult quod denarii illi pariant } et generent : & assi con moclaramente paresçe \textbf{ que quiere que aquellos e paran e engendren dineros . } Et por end ex derecha la usura es llamada assi con \\\hline
2.3.11 & ut quod decem post lapsum temporis fiant viginti , \textbf{ vult quod artificialia seipsa multiplicent : } et quia hoc est contra naturam artificialium , & que passare algun tienpo \textbf{ que se faganveite quiere | que las cosas artifiçiales crezcan } e se amuchiguen en simiłmos \\\hline
2.3.11 & potest inde accipi pensio , \textbf{ dato quod res illa in nullo deterioraretur . } Sed si non potest concedi usus absque concessione substantiae , & ally en aquella cosa se puede tomar loguer o alquiler della \textbf{ puesto que aquella cosa se enpeor | e por aquel uso } assi commo paresçe en las bestias e en las casas . \\\hline
2.3.11 & Quare si de usu pensionem accipiat , \textbf{ uendit quod non est suum , } uel accipit pensionem de eo quod non spectat ad ipsum , & non parte nesçe a ellos el uso della . \textbf{ Por la qual cosa } el que resçibe ganançia del uso del dinero vende lo que non \\\hline
2.3.11 & quod non est proprius usus eius , \textbf{ dato quod non statim pecuniam acciperet , } si propter usum domus vellet ulteriorem pecuniam accipere , & la qual cosa non es uso ppreo dela casa \textbf{ puesto que non resçebiesse luego los dineros mas por el uso dela casa quisiesse tomar dineros acometrie usura } por que ya el uso dela casa non parte nesçrie a el \\\hline
2.3.16 & nam saepe quilibet ministrantium huiusmodi ministerium negligit , \textbf{ credens quod alius exequatur illud : } ubicunque enim est multitudo , & Ca muchͣs uezes cada vno de aquellos seruientes menospreçia aquel seruiçio cuydando \textbf{ que el otro lo fara . } Por que do quier que ay muchedunbre alli es confusion \\\hline
2.3.16 & ex iis quae dicuntur 4 Polit’ . \textbf{ Debemus enim sic imaginari quod sicut se habet magna ciuitas paruam , } sic se habet magna domus ad paruam . & que son dichͣs en el quarto libro delas politicas . \textbf{ Ca deuemoos assi ymaginar que comm̃o se ha la guand çibdat ala pequeña . } Assi se ha la grand casa ala pequana . \\\hline
2.3.17 & videndum est qualiter sunt exhibenda indumenta ministris . \textbf{ Ad cuius euidentiam sciendum quod circa hoc ( quantum ad praesens spectat ) quinque sunt attendenda , } videlicet regis magnificentia , & en qual manera son de dar e departir las vestiduras alos seruientes . \textbf{ e para conosçimiento desto conuieneles de saber | quanto pertenesçe alo presente } que cinco cosas son de penssar en esto . Conuiene a saber la magnificençia et gran dia del Rey . \\\hline
2.3.17 & nimis afficimur ad illa . \textbf{ Ideo dicitur circa finem 7 Polit’ quod semper prima magis amamus . Videmus enim communiter homines adeo affici ad patrias consuetudines , et ad conuersationes regionis propriae , } ut etiam si peiores & mas de buenamente la veemos \textbf{ e por ende los ons | en tanto lon mas inclinados alas costunbres propreas dela su tierra } e alas conuersaconnes de su regno \\\hline
2.3.18 & ubi multi conuersantur , \textbf{ et ubi communiter abundant exteriora bona , potissime largitas quantum ad sumptus , et affabilitas quantum ad conuersationem requiruntur in ipsis . Viso quid est curialitas , de leui patet quod decet ministros Regum et Principum curiales esse . } Nam si decet Reges et Principes & e do abondan los omes \textbf{ comualmente | en los bien es de fuera es neçessaria a ellos prinçipalmente la largueza } quanto alas despenssas \\\hline
2.3.19 & ex vili genere sunt assumpti , \textbf{ dato quod in aliquibus paruis magistratibus videantur prudenter } et fideliter se gessisse , & por mayores en los ofiçios son tomados de villoguar \textbf{ puesto que en algunos pequanos maestradgos e ofiçios parezçan sabios e que se han sabia mente e fielmente } enpero non se deue luego tomar \\\hline
2.3.20 & cum opera eius sint ab ipso deo et intelligentiis ordinata : \textbf{ dato quod natura faciat idem organum ad duo opera , } ne sit in operibus confusio , & por que las obras dela natura son ordenadas de dios e de los angeles . \textbf{ puesto que la natura faga vn estrumento para dos obras enpero } por que non sea confusion en las obras \\\hline
3.1.3 & et lapidi deorsum tendere : \textbf{ quia talia sic eis naturaliter competunt quod ad contrarium assuefieri non possunt , } et competunt eis semper & e ala piedra de desçender ayuso \textbf{ por que estas cosas tales | assi parte nesçen naturalmente al fuego } e ala piedra \\\hline
3.1.3 & ut liberius contemplationi vacet , melius facit . Est ergo homo naturaliter animal ciuile , \textbf{ non obstante quod contingat aliquos non ciuiliter viuere : } nam quicunque non ciuiliter viuit & e entienda a contenplaçion \textbf{ e por ende es el omne natalmente aian lçiuil | puesto que contezca } que alguons non bi una çiuilmente \\\hline
3.1.6 & Primus est ille de quo supra in secundo libro fecimus mentionem , \textbf{ ubi diximus quod propter excrescentiam filiorum collectaneorum } et nepotum domus potest in vicum , & dela qual en el segundo libro feziemos mençion desuso \textbf{ do dixiemos | que por las cresçençias de los fijos } e de los nietos \\\hline
3.1.7 & et quod crederent eos esse suos filios , \textbf{ illi vero opinarentur eos esse suos patres . Tertium vero quod senserunt dicti Philosophi circa regimen ciuitatis , est , } quia dixerunt mulieres instruendas esse ad opera bellica , & que ellos eran sus padres . \textbf{ ¶ Lo terçero | que sintieron los dichs philosofos cerca el gouernamiento dela çibdat . } es que dixieron \\\hline
3.1.7 & ut dicitur , sunt omnes foeminae , quarum masculi sunt viliores eis . \textbf{ Quare si in aliis animalibus hoc videmus quod non solum bellant mares } sed foeminae , & comunalmente son meiores las fenbras \textbf{ que los mas los | e por ende los mas los valen menos } que las fenbras . \\\hline
3.1.7 & et in quibus abundat vena argenti similiter se habere . \textbf{ Si ergo natura sic agit quod venam auri non conuertit in venam argenti , } vel in venam ferri , & sienpre hayuena de plata . \textbf{ ¶ Et pues que assi es si la nata assi faze | que la vena del oro non se conuierte en vena de plata } nin en vena de fierro \\\hline
3.1.8 & nisi sciuerit qualiter constituitur ; \textbf{ et nisi cognoscat quod oportet in ea diuersitatem esse . } Sermo in principiis debet esse longus , & si non sopiere en qual manera es establesçida la çibdat \textbf{ e si non sopiere en qual manera conuiene de auer en ella departimiento de ofiçios e de ofiçiales } l sermon en los comienços deueser luengo e bien examinado \\\hline
3.1.10 & non habebitur eorum cura debita : \textbf{ sequitur quod supposita communitate , } quam ordinauerat Socrates , & e dende se sigue \textbf{ que puesta tal comunidat commo ordeno soctateᷤ } non se puede auer cuydado conuenible \\\hline
3.1.10 & et non iudicabantur eis proprii parentes , \textbf{ dato quod prohiberetur filio actus venereus circa matrem , } et patri circa filiam , & e non les fuessen mostrados sus padres proprios \textbf{ puesto que los prinçipes defendiessen alos fijos que non fiziessen lururia con sus madres } e los padres con sus fiias \\\hline
3.1.15 & et de unitate ciuium , \textbf{ verum est quod ipse opinabatur , } quod in ciuitate esset maxima pax , et non orirentur ibi litigia . & assi es poniendo la entençio de socrates dela comunidat delas cosas \textbf{ e dela vnidat de los çibdadanos uerdat es | lo que el cuydaua e ymaginaua } que en la çibdat seria grant paz \\\hline
3.1.16 & Statuit enim Phaleas ciuitatis rectorem \textbf{ hoc modo reducere hanc inaequalitatem ad aequalitatem mediantibus dotibus statuendo quod pauperes contrahant } cum diuitibus : & que el rector dela çibdat \textbf{ en esta manera aduxiesse esta desegualdat a egualdat | Conuiene a saber } por las arras \\\hline
3.1.19 & quare Hippodanus sic statuit . Credebat quidem quod si iudices iuramento essent astricti \textbf{ ut dicerent quod sentirent , } forte degenerarent timendo coram aliis dicere quod sentiunt . & ca creye que los iiezes eran estrennidos por iuramento \textbf{ que diessen lo que sentiessen e entendiessen } ca por auentura negarien de dezer \\\hline
3.1.19 & ut dicerent quod sentirent , \textbf{ forte degenerarent timendo coram aliis dicere quod sentiunt . } Ideo ordinauit quod quilibet priuatim sententiam suam scriberet . & ca por auentura negarien de dezer \textbf{ lo que sienten | ante los o tristemiendosse dellos } e por ende ordeno \\\hline
3.1.19 & forte degenerarent timendo coram aliis dicere quod sentiunt . \textbf{ Ideo ordinauit quod quilibet priuatim sententiam suam scriberet . } Sexto statuit quasdam leges , & ante los o tristemiendosse dellos \textbf{ e por ende ordeno | que cada vno apareiadamente ordenasse su suina } e la diesse por escpto ¶ \\\hline
3.2.2 & et tunc talis principatus dicitur Aristocratia , \textbf{ quod idem est quod principatus bonorum et virtuosorum . Inde autem venit } ut maiores in populo , & e tal prinçipado es dicħa ristrocaçia \textbf{ que quiere dezer prinçipado de buenos omes e uir̉tuosos | e dende vienen } que los mayores en el pueblo \\\hline
3.2.2 & et opprimentes alios intendunt proprium lucrum huiusmodi principatus Oligarchia dicitur , \textbf{ quod idem est quod principatus diuitum . } Consurgit igitur duplex principatus ex dominio paucorum : & tal es dich obligartia \textbf{ que quiere dezir prançipado de ricos . } Et pues que assi es dos prinçipados se leuna \\\hline
3.2.4 & tanto peior principatus : \textbf{ sed si dominentur multi dato quod intendant bonum proprium ; } quia bonum multorum est & Et pues que assi es \textbf{ quanto menos entiede en el bien comun } tanto peoras el prinçipe e el prinçipado \\\hline
3.2.4 & Sed ut soluantur obiectiones praetactae , \textbf{ sciendum quod quia plura cognoscunt plures quam unus , } et citius corrumpitur unus quam plures , & e las obiectiones sobredichͣs deuedes saber \textbf{ que la razon que dizia | que muchs conosçen mas que vno¶ } Et la segunda que dizia \\\hline
3.2.5 & filii ex hoc non inflantur nec efficiuntur elati , quia non reputant magnum , \textbf{ si illud habeant quod patres possederant : } quare ex parte filiorum debentium succedere in haereditatem paternam , & por ello \textbf{ nin se fazen chiranos nin inchados . | ca non tienen por desaguisado } si heredar en aquello que heredaron sus padres \\\hline
3.2.5 & Quod vero superius tangebatur , \textbf{ videlicet quod ire per haereditatem , } dignitatem regiam , est exponere fortunae , & Mas aqual lo que dessuso fue dich conuiene saber \textbf{ que quando va el regno | por hedat } que la dignidat real se espone a ocasion e auentura \\\hline
3.2.5 & et totius regni in hoc consistat . \textbf{ Nec sufficit quod quia solus primogenitus regnare debet , } ut de eo solo cura habeatur diligens : & por que el bien de todo el regno esta en esto . \textbf{ Et non cunple | que por el primogenito deue regnar } que del solo deue ser tomada acuçia \\\hline
3.2.6 & nimis ardenter mouetur in eorum amorem , \textbf{ et optat eos habere in dominos . Inde est quod antiquitus plures sic praeficiebantur in Reges . } Nam si aliquis fuerat primo beneficus , & e bien fechores mueuense con grant ardor alos amar \textbf{ e dessean de los auer por sennores | e por ende antiguamente los mas de los sennores fueron tomados en Reyes . } por que si alguno era atal que feziera bien al pueblo aquella gente inclinada a el \\\hline
3.2.6 & et priuatum , \textbf{ sequitur quod eius intentio versetur circa bonum delectabile . } Sed intentio Regis versatur & si el thirano entiende el su bien propreo \textbf{ siguese que la su entençion es mala } ca non es cerca el bien honrrado e de honira \\\hline
3.2.6 & Ex hac autem secunda differentia sequitur tertia ; \textbf{ videlicet quod intentio tyrannica est circa pecuniam . } Tyrannus & veste departimiento segundo se sigue el terçero . \textbf{ Conuiene de saber | que la entencion del tiran no es en auer riquezas o dineros } Ca el tirano \\\hline
3.2.6 & Ex hac autem differentia \textbf{ tertia sequitur quarta videlicet quod tyrannus non curat custodiri a ciuibus , } sed ab extraneis : & Conuiene de saber \textbf{ que el tiranno non ha cuydado de ser guardado de los çibdadanos } mas de los estrannos . \\\hline
3.2.7 & vel si dominetur totus populus , \textbf{ dato quod sic dominantes non intenderent } nisi bonum proprium , & o si enssennoreare todo el pueblo \textbf{ puesto que los que assi enssenno rean non entiendan } si non el bien propo enpero non se arriedran del todo dela entençion del bien comun . \\\hline
3.2.8 & licet per praecedentia sit aliqualiter manifestum , \textbf{ clarius tamen infra dicetur . Viso quod spectat ad Regis officium solicitari circa ea per quae possit populus consequi finem intentum : } restat ostendere , & Visto que parte nesçe al ofiçio del Rey \textbf{ de ser acuçioso çerca aquellas cosas | por las quales el pueblo puede alcançar su fin } que entiende finca de demostrar \\\hline
3.2.10 & et conseruat , \textbf{ videns quod per ipsum , } bonum commune , & e mantiene le ueyendo \textbf{ que por el bien comun } e el buen estado del regno \\\hline
3.2.10 & Decima cautela tyrannica , \textbf{ est quod postquam procurauit diuisiones } et partes in regno , & La dezena cautela del tirano \textbf{ es que del pues } que ha puesto vandos e departimientos en el regno \\\hline
3.2.12 & et vocatur Democratia , \textbf{ quod idem est quod quasi peruersio et corruptio populi . Tyrannis vero corruptus principatus diuitum , } et iniquum dominium populi , sunt regimina peruersa . Tyrannis & e tal sennorio es llamado de mocraçia \textbf{ que tanto quiere dezer commo corrupçion e maldat del pueblo . Et pues que assi es la tirania | e el sennorio corrupto de los ricos . } Et el sennorio malo del pueblo son tres señorios muy malos . \\\hline
3.2.15 & sed etiam principatus ex hoc durabilior redditur , \textbf{ dato quod in ipso sit aliquid obliquitatis ad mixtum . Tertium est , } incutere timorem iis qui sunt in politia : & Mas avn por esta razon el prinçipado se faze mas durable \textbf{ puesto que en el sea alguna cosa meztlada de maldat ¶ | La terçera cosa } que guarda al gouernamiento del regno es meter mie do aquellos que son enla çibdat e en el regno \\\hline
3.2.15 & nam qui huiusmodi rationem non potest reddere , \textbf{ signum est quod ex furto } vel ex male ablato viuat . & ca aquel que non puede dar razon desto \textbf{ señal | es que biue de furto o de rapina . } ca assi fazie \\\hline
3.2.17 & quod consilium \textbf{ quasi idem est quod considium : } dictum est autem consilium , & por descobrir se los conseios \textbf{ e por ende por auentura el conseio dende tomo nonbre } ca dizen alguons \\\hline
3.2.18 & non oportet ipsum esse existenter talem , \textbf{ sed sufficit quod videatur vel appareat talis esse : } nam homo iudicat quae foris patent , & que el sea tal fechmas \textbf{ cunple que parezca tal cael o en iudga las cosas que paresçen de fuera } por las cosas que vee \\\hline
3.2.18 & Et quia prudentes sciunt facere , \textbf{ et qui existimantur prudentes , existimantur talia facere : ideo ad hoc quod aliquis ex rebus de quibus loquitur fidem faciat , vel oportet quod sit prudens } vel quod credatur esse prudens . & e aquellos que son tenidos por sabios son contados \textbf{ para fazer tales cosas Morende | para que alguno faga fe delas cosas } de que fabla o conuiene \\\hline
3.2.18 & vel omnis ille cuius dictis creditur et adhibetur fides , \textbf{ vel oportet quod sit bonus , } vel quod amicus , & a cuyos dichos creen los omes \textbf{ e es dada feo conuiene | que sea bueono } que sea amigo o que sea sabio \\\hline
3.2.18 & debet habere apparenter , \textbf{ oportet quod bonus consiliator habeat existenter : } satis apparet quales consiliatores deceat quaerere regiam maiestatem ; & e paresçer todas aquellas cosas \textbf{ que ha todo buen conseiero en ssi de fecho . } Et por ende assaz parelçe \\\hline
3.2.20 & sit illa facturus , et debeat illam subire sententiam . \textbf{ Nam si scirent quod amicus , } forte obliquerentur in iudicando , et poenam palliarent : & por tal suina . \textbf{ ca si por auentra asopiessen ellos | que su amigo auie de fazer aquella cosa } por auentraase torçerian en iudgando \\\hline
3.2.21 & Prima via sic patet . \textbf{ Scire enim debemus quod iudex in iudicando de litigiis , } ut recte iudicet , & La primera razon paresçe assiça deuedessaber \textbf{ que el iuez en iudgando de los pleitos } para que derechamente iudgue \\\hline
3.2.24 & et dare quintam distinctionem iuris , \textbf{ dicendo quod quadruplex est ius , } videlicet naturale , animalium , gentium , & e el quinto departimiento del derech̃ . \textbf{ diziendo que en quatro maneras se departe el derech . | Conuiene a saber ende recħ natural } e en derecho delas ainalias \\\hline
3.2.24 & sed ad placitum . \textbf{ Inde est quod omnes homines loquuntur , } non tamen omnes proferunt & mas es a uoluntad . \textbf{ Et por ende es que todos los omes fablan } enpero non fablan todos vn lenguaie . \\\hline
3.2.25 & ut ius animalium . \textbf{ Ad cuius euidentiam sciendum quod homo } ut est homo et secudum & que es derecho delas aian lias . \textbf{ Et para declaraçion desto conuiene de saber } que el omne en quanto es omne e penssando segunt su razon proprea \\\hline
3.2.25 & quod ius naturale , \textbf{ est quod natura omnia animalia docuit . } Huiusmodi autem ius & do estas cosas son puestas \textbf{ dize el derecho natural enssenna a todas las ainalias . } Ca este derechotal \\\hline
3.2.25 & quod est proprium soli humano generi . \textbf{ Ex hoc igitur manifeste patet quod ius gentium non dicitur ita ius naturale , } sicut ius quod nam omnia animalia docuit : & el qual derecho esppreo solamente al linage humanal . \textbf{ Et pues que assi es desto paresçe manifiestamiente | que assi commo el derecho delas gentes non es dicho } assi derecho natural commo el derecho \\\hline
3.2.26 & Nam ut lex humana comparatur ad legem naturae , \textbf{ oportet quod sit iusta : } ut comparatur ad bonum commune , & en quanto es conparada ala ley de natura \textbf{ conuiene que sea derechͣ . } Et en quanto es conparada al bien comun \\\hline
3.2.27 & Quare si a bono perfectiori quod magis habet rationem finis sumendae sunt leges et regulae agibilium , \textbf{ sequitur quod non a bono priuato } et domestico & que ha mas razon de fin \textbf{ e de bien | deuen ser tomadas las leyes } e las reglas delas nuestras obras \\\hline
3.2.27 & si tota huiusmodi multitudo principetur . \textbf{ Viso quod non est cuiuslibet leges condere , } de leui potest patere legem non habere vim obligandi , & si tal muchedunbre enssennoreare ¶ \textbf{ Visto que non parte nesçe a cada vno establesçer las leyes de ligo puede paresçer } que la ley non ha uirtud de obligar a \\\hline
3.2.28 & His itaque sic pertractatis , \textbf{ dicamus quod decet Reges et Principes , } quorum interest solicitari circa bonum commune : & assi tractadas digamos \textbf{ que pertenesçe alos Reyes e alos prinçipes alos quales conuiene ser muy acuçiosos çerca del bien comun } e çerca del gouernamiento del regno \\\hline
3.2.29 & ostendit ibi Philosophus in eodem 3 . \textbf{ Nam ( ut ait ) lex uniuersaliter dicit quod non est uniuersaliter : oportet enim humanas leges quantumcunque sint exquisitae in aliquo casu deficere : melius est igitur regnum Regi Rege , } quam lege , & aquello que non es general mente \textbf{ Por que conuiene que las leyes humanales | commo quier que sean examinadas de fallesçer en algun caso . } Et por ende meior es \\\hline
3.2.31 & et antiquas propter meliores leges nouiter inuentas . Videntur \textbf{ itaque hae rationes probare quod quotiescunque occurrit aliquid melius , } sunt leges paternae immutandae . & Et por ende paresçe \textbf{ que estas razones sobredichas prueuna que cada que acahesçiere algua cosa meior las leyes dela tierra son de mudar . } Mas afirmar esto sinplemente es muy perigloso ala çibdat e altegno . \\\hline
3.2.31 & quid tenendum sit de quae sito , \textbf{ sciendum quod lex positiua si recta sit , } oportet quod innitatur legi naturali , & Conuiene de saber \textbf{ que la ley politica sitiua | si fuere derecha conuiene } que se raygͤ \\\hline
3.2.31 & sciendum quod lex positiua si recta sit , \textbf{ oportet quod innitatur legi naturali , } et quod determinet gesta particularia hominum . Dupliciter ergo potest & si fuere derecha conuiene \textbf{ que se raygͤ | e se funde enla ley natural . } Et conuiene que determine las obras e los fechos particulares de los omes . \\\hline
3.2.31 & Secundum hunc modum loquendi loquuntur Iuristae , ut patet ex Institutis de iure naturali , \textbf{ ubi dicitur quod leges humanae contrariae sunt iuri naturali ; } quia iure naturali ab initio homines liberi nascebantur . Seruitus ergo est contra naturam , & assi commo paresçe en la institutado dize \textbf{ que las leyes humanales contrarias son al derecho natraal . } Ca de comienço todos los omes nascian forros e libres . \\\hline
3.2.31 & quia non complete determinant particularia agibilia , \textbf{ dato quod occurrant leges meliores } et magis sufficientes , non est assuescendum innouare leges . Primo , quia aliquando contingit circa talia decipi , & por que non determinan conplidamente los fechos particula respuesto \textbf{ que sean falladas leyes meiores | e mas conplidas . } Enpero non nos auemos a acostunbrar a renouar las leyes . \\\hline
3.2.33 & et de facili rationi obediunt . \textbf{ Hoc est quod dicitur 4 Politicorum } quod quoniam mediocre est optimum , & e de ligero los omes obedescran ala razon . \textbf{ Et esto es lo que dize el philosofo en el quarto libro delas politicas } que por que lo medianero es muy bueno \\\hline
3.2.34 & sit bonus vir . \textbf{ Si enim Rex non intenderet quod sibi subiecti essent boni } et virtuosi , & Et aquel que bien obedesçe al Rey es buen uaron \textbf{ casi el Rey non entendiesse } que los sus subditos fuessen bueons e uirtuosos \\\hline
3.2.34 & non solum Regibus recte regentibus , \textbf{ sed etiam dato quod in aliquo tyrannizarent , studeret populus obedire illis . } Nam magis est tolerabilis aliqualis tyrannis principantis , quam sit malum , quod consurgit ex inobedientia Principis , & mas avn alos malos . \textbf{ Ca avn puesto que en alguna cosa tira nizen deue estu diar avn el pueblo de obedesçer los . } Ca mas sofridera es alguna tirama o desordenamiento del prinçipe \\\hline
3.3.1 & quam collocauimus in prima specie . \textbf{ Nam aliud est quod sciat se regere } ut est aliquid in se , & la qual pusiemos en la primera manera de sabiduria . \textbf{ Ca otra cosa es | que el omne se sepa gouernar } en quanto es alguna cosa en ssi . \\\hline
3.3.6 & quod se bene didicisse confidit . \textbf{ Inde est quod tantum valet armorum exercitatio , } quod in bellorum certamine paucitas exercitata plus valet & que lo sabe bien . \textbf{ Et dende viene | que tanto vale el vso de las armas } que en la contienda de las batallas pocos omnes bien usados son apareiados \\\hline
3.3.7 & vel cum ballistis . \textbf{ Nam quia contingit quod ipsos hostes non possumus immediate attingere , } utile est eos sagittis impugnare : & a alançar saetas con arcos e con ballestas . \textbf{ ca quando contesçe | que non podemos de tan çerca llegar a los enemigos } para ferirlos . \\\hline
3.3.7 & utile est eos sagittis impugnare : \textbf{ immo dato quod pugnantes se cum hostibus possint coniungere , } antequam coniungantur proficuum est eos arcubus & prouechosa cosa es lançar las saetas \textbf{ mas puesto que los lidiadores se puedan ayuntar con los enemigos } ante que se apunte con ellos \\\hline
3.3.7 & contingit multos periclitatos esse . \textbf{ Inde est quod apud Romanos antiquitus consuetudo erat , } quod iuuenes futuri bellatores & que por non saber nadar caen en muchos periglos . \textbf{ Et por ende era costunbre antiguamente entre los romanos | que los mançebos } que auian de ser lidiadores \\\hline
3.3.8 & ut sit lata pedes duodecim , \textbf{ et alta nouem . Est tamen aduertendum quod si fossa sit alta pedum nouem , propter terram eiectam supra fossam crescit } quasi pedes quatuor : & assi que sea la carcaua ancha de doze pies \textbf{ e alta de nueue . | Enpero conuiene de saber } que si la carcaua fuere fonda de nueue pies echando la tierra a la parte de la hueste fazese la carcaua mas alta de quatro pies \\\hline
3.3.11 & possent inuadentibus resistere . Sic enim dicendo , \textbf{ dato quod accideret aliquis repentinus insultus , } esset quasi prouisus , & Ca assi diziendo \textbf{ puesto que contesçiesse algun rebate a desora menos les podria enpeesçer } por que estauan aperçebidos . \\\hline
3.3.15 & possunt multa mala committere in eos \textbf{ qui contra pugnant . Inde est quod laudatur Scipionis sententia , dicentis : } Nunquam sic esse claudendos hostes , & si non la muerte pueden acometer muchas malas cosas contra aquellos que lidian contra ellos . \textbf{ Et por ende es alabada la sentençia de çipion | por la qual dizia } que nunca eran de encerrar los enemigos \\\hline
3.3.15 & qua recedente , equites postea melius possunt vitare hostium percussiones . \textbf{ Est etiam aduertendum quod quando sic declinatur pugna , } nunquam acies se debent diuidere : & Et ellos ydos los caualleros pueden meior despues escusar los colpes de los enemigos . \textbf{ Avn conuiene de saber | que quando se assi escusa la batalla nunca la az se deue departir . } ca podrie contesçer \\\hline
3.3.16 & quam gladius . \textbf{ Inde est quod multotiens obsidentes volentes citius opprimere munitiones , } si contingat eos capere aliquos de obsessis , & nin el cuchiello . \textbf{ Et por ende contesçe que muchas uegadas | los que çercan queriendo mas ayna ganar las fortalezas } si contezca \\\hline
3.3.17 & non est possibile obsidentes semper esse paratos aeque . Ideo nisi sint muniti , \textbf{ contingit quod existentes in castris } ( cum fuerint occupati obsidentes somno , & Et por ende si non estudieren guarnesçidos puede les contesçer \textbf{ que los que estan en los castiellos o en las çibdades cercadas } quando los que çercan durmieren \\\hline
3.3.18 & vel est aliquod praedictorum , vel potest originem sumere ex praedictis . \textbf{ Est etiam aduertendum quod die } et nocte per lapidarias machinas impugnari possunt munitiones obsessae . & o tomar puede rayz o comienço de aquellas sobredichas . \textbf{ Et avn conuiene de saber } que tan bien de noche \\\hline
3.3.20 & per machinas lapidarias . \textbf{ Nam dato quod per huiusmodi machinas } totus murus exterior rueret , & por las piedras de los engeñios . \textbf{ Ca puesto que todo el muro de fuera fuesse destroydo } por las piedras de los engeñios \\\hline

\end{tabular}
