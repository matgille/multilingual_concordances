\begin{tabular}{|p{1cm}|p{6.5cm}|p{6.5cm}|}

\hline
1.2.1 & e la su bien andança . \textbf{ Et que non los conuiene poner la su fin en riquezas } nin en poderio çiuil & suam felicitatem debeant ponere , \textbf{ quia non decet | eos suum finem ponere in diuitiis , } nec in ciuili potentia , \\\hline
1.2.18 & los que son en el su regno \textbf{ mucho les conuiene de ser liberales e francos } Mas par tenesce al libal e alstan ço de catar tres cosas ¶ & qui sunt in Regno , \textbf{ maxime decet eos liberales esse . } Spectat autem ad liberalem primo \\\hline
1.2.18 & a quien non conuiene de dar . \textbf{ por que aquestos tales mas les conuiene de ser pobres } que non ser ricos . & quibus non oportet dare : \textbf{ quia magis deceret | eos esse pauperes , } quam diuites . \\\hline
1.2.24 & e alos prinçipes de seer magnificos e liberales \textbf{ assi en essa misma manera les conuiene de seer magranimos } e amado res de honrra . & et Principes esse magnificos , et liberales : \textbf{ sic decet eos esse magnanimos , } et honoris amatiuos . Reges enim et Principes decet honores diligere modo quo dictum est ; \\\hline
1.3.5 & e alos prinçipes de entender en el bien \textbf{ Mas avn les conuiene de entender } en bien alto e grande e guaue de fazer De mas desto & et Principes tendere in bonum , \textbf{ sed etiam decet eos tendere in bonum arduum . } Amplius quanto maior est communitas , \\\hline
1.4.4 & e confonden el entendemiento . \textbf{ Otrosy les conuiene de ser misconiosos non por fallesçimiento nin por llaqueza de } coraçon quales en los vieios . & rationem percutiunt . \textbf{ Decet etiam eos esse miseratiuos , | non propter defectum , } vel propter imbecillitatem : \\\hline
1.4.5 & e escodrinnadores sotilmente de todo aquello \textbf{ que les conuiene de fazer } por que las sus obras & subtiliter inuestigantes \textbf{ quid decet eos facere , } ne opera eorum , \\\hline
2.1.10 & por que en el casamiento \textbf{ dellos conuiene de guardar la orden natural mas que en otro ninguno . } ¶ Lo segundo esso mismo pue de ser mostrada & coniuges Regum et Principum , \textbf{ quia in eorum coniugio magis quam in alio decet | naturalem ordinem conseruare . } Secundo hoc idem inuestigari potest \\\hline
2.1.19 & e then a sus maridos a mayor amor . \textbf{ Et por ende les conuiene de ser calladas } e en essa misma manera avn les conuiene de ser estables e firmes & et ad maiorem amorem viros inducunt : \textbf{ decet ergo eas esse taciturnas . } Sic etiam decet esse stabiles : \\\hline
2.1.19 & Et por ende les conuiene de ser calladas \textbf{ e en essa misma manera avn les conuiene de ser estables e firmes } que quanto la mug̃res mas firme e mas estable & decet ergo eas esse taciturnas . \textbf{ Sic etiam decet esse stabiles : } quia quanto uxor est magis constans , \\\hline
2.2.2 & en quel conuiene gouernar los otros \textbf{ mucho les conuiene de ser sabios e buenos . } Mas commo los fijos bengan a mayor bondat e a mayor sabiduria & in quo oportet eos alios gubernare ; \textbf{ maxime decet eos esse prudentes et bonos . } Et cum filii perueniunt \\\hline
2.2.7 & e en las sçiençias liberales \textbf{ quanto mas les conuiene de ser mas entendudos } e mas sabios que los otros & insudare literalibus disciplinis , \textbf{ quanto decet eos intelligentiores et prudentiores esse , } ut possint naturaliter dominari . \\\hline
2.2.10 & ¶ La primera quanto alas cosas uisibles \textbf{ que assi commo non les conuiene de fablar cosas torpes } Et la razon desto pone el philosofo en łvij̊ libro delas ethicas & Quantum ad visibilia quidem , \textbf{ quia sicut non decet | eos turpia sequi : } sic indecens est eos turpia videre . \\\hline
2.2.12 & non solamente que se non fagan gollosos por el comͣ \textbf{ mas avn les conuiene de ser mesurados } que non se fagan beodos & ex sumptione cibi : \textbf{ sed etiam decet eos esse sobrios , } ut non efficiantur ebrii \\\hline
2.2.18 & nin por otra uentra a qual si quier non osen tomar armas . \textbf{ Enpero por que mas conuiene de ser sabios } que lidiadores alos fijos de los Reyes & nec pro alio casu audeant arma assumere ; \textbf{ attamen quia decet | eos esse magis prudentes quam bellatores , } filii Regum et Principum \\\hline
2.2.20 & si non nos delectaremos en alguas cosas \textbf{ conuiene de tomar alguas obras conuenibles e honestas } çerca las quales entendamos & nisi in aliquibus delectemur : \textbf{ decet nos assumere | aliqua opera licita et honesta , } circa quae vacantes , \\\hline
2.3.3 & que anings de los otros nobles \textbf{ ca ellos conuiene de ser nobles prinçipalmente } e magnificos en todas sus cosas . & qui debent esse nobiles et praeclari , \textbf{ potissime decet esse magnificos . } Alii enim moderatas possessiones habentes , \\\hline
2.3.18 & e de los no nobles omes alos \textbf{ que les conuienne de ser dadores } e cobidadores & sed quia volunt retinere mores curiae et nobilium , \textbf{ quos decet datiuos esse ; } propter quod tales curiales dici debent . \\\hline
2.3.19 & e los prinçipes \textbf{ alos quales conuiene de ser magn animos } deuen se mostrar & Reges ergo et Principes , \textbf{ quos decet esse magnanimos } ad proprios ministros , \\\hline
2.3.20 & Et pues que assi es los Reyes \textbf{ e los prinçipeᷤ alos quales conuiene ser muy tenprados } e guardar la orden natural en toda & Reges ergo et Principes , \textbf{ quos decet maxime temperatos esse , } et obseruare ordinem naturalem \\\hline
3.2.17 & por la quel cosa commo muchs mas cosas ayan prouadas \textbf{ que vno solo conuiene de llamar otros } para los negoçios . por que por el conseio dellos pueda ser escogida la meior carrera & Quare cum plures plura experti sint , \textbf{ quam unus solus : | decet ad huiusmodi negocia alios aduocare , } ut per eorum consilium possit \\\hline
3.2.36 & La terçera cosa para que los Reyes sean amados del pueblo \textbf{ es quales conuiene de ser derechureros e eguales . } Ca el pueblo mayormente se le una taria a mal querençia del Rey & ut Reges diligantur a populo , \textbf{ decet eos esse iustos , et aequales . } Nam maxime prouocatur populus ad odium Regis , \\\hline

\end{tabular}
