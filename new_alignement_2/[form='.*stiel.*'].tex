\begin{tabular}{|p{1cm}|p{6.5cm}|p{6.5cm}|}

\hline
1.2.6 & escodrinnar maneras e sotilezas \textbf{ por que podamos tomar aquel castiello ¶ } Lo segundo deuemos iudgar de aquellas carreras e maneras que fablamos & et cogitandi essent modi , \textbf{ per quos castrum istud capi posset . } Secundo iudicandum esset \\\hline
1.3.6 & temen \textbf{ luego fuyen alos castiellos e alas torres . } En essa misma manera & in campis timent , \textbf{ statim confugiunt ad castrum , vel ad arcem : } sic cum quis timet , \\\hline
3.1.6 & assi commo sy muchͣs çibdades \textbf{ o muchs castiellos se ayuntassen en vno } por amistança & fieri constitutio regni , \textbf{ ut si multae ciuitates et castra simul confoederarentur et concordarent , } ut sub uno rege existerent , \\\hline
3.2.9 & Lo septimo conuiene al uerdadero Rey de conponer \textbf{ e guarnesçer las çibdades e los castiellos que son en el su regno } assi que parezca mas ser procurador del bien comun & Septimo decet verum Regem ornare \textbf{ et munire ciuitates | et castra existentia in regno , } ut appareat magis esse procurator communis boni , \\\hline
3.3.8 & que estando la hueste sin carcauas \textbf{ e sin castiellos o otros defendimientos non cuydando } que sus enemigos estan çerca vienen a desora los & Contingit autem pluries diuino et nocturno tempore , \textbf{ quod , exercitu absque fossis et castris existente , } et non credentes hostes esse propinquos , \\\hline
3.3.8 & que lieuan consigo assi commo vna çibdat guarnida . \textbf{ Visto commo es cosa prouechable a la hueste fazer carcauas e costruir guarniçiones e castiellos . } finca de demostrar en qual manera las tales guarniciones & Viso utile esse \textbf{ circa exercitum facere fossas | et construere castra : } restat ostendere , \\\hline
3.3.8 & finca de demostrar en qual manera las tales guarniciones \textbf{ et los tales castiellos se deuen fazer . } Ca si los enemigos non estudieren cerca de ligero pueden fazer carcauas çerca de la hueste & quomodo huiusmodi monitiones \textbf{ et castra sunt construenda . } Nam si hostes sunt absentes \\\hline
3.3.8 & e leuantar \textbf{ guarnicoñes e fazer castiellos . } Mas si los enemigos fueren cerca & facile est fossas circa exercitum fodere , \textbf{ munitiones erigere et castra construere . } Sed si aduersarii praesentes adsint , \\\hline
3.3.8 & graue cosa es de guarnescer la hueste \textbf{ e de fazer castiellos . } Ca en tal caso commo este dos cosas son menester . & Sed si aduersarii praesentes adsint , \textbf{ difficilius est castra munire . } Sunt enim in tali casu duo necessaria , \\\hline
3.3.8 & Lo primero estar e lidiar contra los enemigos \textbf{ Lo segundo fazer los castiellos . } Pues que assi es en tal auenemiento conmo este & videlicet hostibus resistere , \textbf{ et castra construere . } In tali ergo euentu \\\hline
3.3.8 & e manden a cada vno qual cosa deua fazer . \textbf{ Mostrado que prouechosa cosa es de fazer los castiellos . } avn en qual manera los enemigos presentes son de fazer los castiellos & quod ipsum oporteat facere . \textbf{ Ostenso utile esse castra construere , } et qualiter etiam praesentibus hostibus construenda sint castra : \\\hline
3.3.8 & quales cosas son de penssar \textbf{ en el fazimiento de los castiellos . } Por que en el fazimiento de las carcauas & quae sunt attendenda \textbf{ in constructione castrorum . } In faciendis enim fossis , \\\hline
3.3.8 & quales cosas deuen ser \textbf{ penssadas çerca de los assentamientos de los castiellos . } Conuiene de declarar & Declarato ergo quae sunt attendenda \textbf{ circa situm castrorum : } declarandum est , \\\hline
3.3.8 & por do ha de salir la hueste . \textbf{ Avn deuen se poner en los castiellos pendones algunos o algunas senales } para espantar los enemigos & vel circa quam protectus est exercitus . \textbf{ Sunt etiam in castris ponenda insignia ad terrendum hostes : } et etiam ad hoc , \\\hline
3.3.8 & para espantar los enemigos \textbf{ e avn si contesçiere que algunos se ayan de alongar de la hueste de los castiellos . } vistas aquellas señales & et etiam ad hoc , \textbf{ ut si contingat aliquos | de exercitu elongare a castris , } visis insignis melius sciant ad castra redire . \\\hline
3.3.8 & vistas aquellas señales \textbf{ sepan meior tornar a la hueste o a los castiellos . } Et estas cosas assi dichas finca de uer & de exercitu elongare a castris , \textbf{ visis insignis melius sciant ad castra redire . } His itaque pertractatis superest videre \\\hline
3.3.8 & en qual manera de guarnimiento es de catar \textbf{ en el fazer de los castiellos . } Ca si la hueste mucho ouiere de morar & quis munitionis modus attendendus sit \textbf{ in constructione castrorum . } Nam si exercitus diu \\\hline
3.3.8 & Et pues que assi es si assi fueren fechas las carcauas \textbf{ e los castiellos podra la hueste estar segura . } a assi commo paresçe par las cosas ya dichas & sic fossis factis , \textbf{ poterit exercitus morari securus . } Ut patet per habita , \\\hline
3.3.16 & e de las çibdades \textbf{ e de los castiellos a lidiar al canpo . } Mas ellos acometen aquellas villas & quod hostes de munitionibus exeuntes vadant \textbf{ bellare ad campum , } sed ipsi munitiones inuadunt \\\hline
3.3.16 & Mas ellos acometen aquellas villas \textbf{ e aquellos castiellos e çercanlos . } e tal manera de batalla & sed ipsi munitiones inuadunt \textbf{ et obsident illas . } Tale genus pugnae communi nomine \\\hline
3.3.16 & Mas ellos acometen las villas \textbf{ e los castiellos o las fortalezas e los çercan . } assi contesçe que alguno lidiadores son tan pocos et tan flacos & ut non expectant hostes exire ad campum , \textbf{ sed ipsas munitiones obsideant et inuadant : } sic contingit aliquos esse adeo paucos et tam debiles , \\\hline
3.3.16 & Ca si en toda batalla es algun acometemiento en alguna manera . \textbf{ Enpero quando alguno o algunos çercan villas o castiellos o fortallezas } mas dezimos que aquellos acometen que se defienden . & Nam etsi in omni pugna est aliquo modo inuasio et defensio : \textbf{ attamen cum quis obsedit munitiones , et castra , } magis dicitur alios inuadere \\\hline
3.3.16 & mas es acometemiento que defendimiento . \textbf{ Mas en la batalla en que se los omnes defienden en villas o en castiellos o en las fortalezas . } mas es ay defendemiento que acometemiento . & In bello vero \textbf{ quo quis se tuetur | in munitionibus et castris , } magis est ibi defensio , \\\hline
3.3.16 & Ca algunas uezes acometen batalla canpal e en el canpo . \textbf{ Et algunas vezes çerca villas o castiellos o fortalezas . } Et avn algunas vezes contesçe que algunos otros çercan sus villas o sus castiellos . & Nam aliquando committunt campestre bellum . \textbf{ Aliquando vero obsident munitiones et castra . } Contingit etiam aliquando aliquos \\\hline
3.3.16 & Et algunas vezes çerca villas o castiellos o fortalezas . \textbf{ Et avn algunas vezes contesçe que algunos otros çercan sus villas o sus castiellos . } Por la qual cosa les conuiene de vsar de batalla defenssiua para se defender . & Aliquando vero obsident munitiones et castra . \textbf{ Contingit etiam aliquando aliquos | inuadere aliquas munitiones eorum ; } propter quod eos oportet \\\hline
3.3.16 & contesçe tomar e vençer las villas \textbf{ e los castiellos e fortalezas . } fincanos de dezir & cum per huiusmodi pugnam \textbf{ contingat obtineri et deuinci munitiones et urbanitates : } restat dicere quot modis talia deuinci possunt . \\\hline
3.3.16 & Et conuiene de saber \textbf{ que son tres maneras de ganar las fortalezas e los castiellos . } Conuiene saber . & restat dicere quot modis talia deuinci possunt . \textbf{ Est autem triplex modus obtinendi | munitiones et castra , } videlicet , per sitim , famem , et pugnam . \\\hline
3.3.16 & las fortalezas cerradas \textbf{ finca de demostrar en que tienpo es meior de çercar las çibdades e las castiellos . } Et por ende conuiene de saber & restat ostendere , \textbf{ quo tempore melius est | obsidere ciuitates et castra . } Sciendum itaque quod tempore aestiuo \\\hline
3.3.17 & deuen fincar las tiendas \textbf{ e el real alueñe de la çibdat o del castiello cercado } quanto podrie lançar la vallesta o el dardo & longe a munitione obsessa saltem \textbf{ per ictum teli } vel iaculi debent castrametari , \\\hline
3.3.17 & e a los mas fuertes adarues del \textbf{ castielloo de la çibdat çercada . } Et por cueuas deuen venir & ad maiores munitiones \textbf{ et ad maiora moenia castri , vel ciuitatis obsessae , } et per similes vias subterraneas est similiter faciendum circa ea , \\\hline
3.3.17 & por las cueuas soterrañas \textbf{ assi que por ellas puedan entrar a la çibdat o al castiello . } Et estas cosas todas deuense fazer muy encubiertamente & diuertendo vias subterraneas , \textbf{ ut per eas possit | haberi ingressus ad ciuitatem et castrum : } quae omnia latenter fieri possunt \\\hline
3.3.17 & Et por aquellas cosas soterrañas \textbf{ ayan entrada al castiello o a la çibdat } e por la entrada que se faze & et munitiones suffossae : \textbf{ et per vias subterraneas fiat ingressus ad castrum , vel ad ciuitatem : } et per aditum factum ex muris cadentibus reliqui obsidentes ingrediantur castrum , \\\hline
3.3.17 & en la çibdato \textbf{ en el castiello çercado } e assi podran ganar aquellas fortalezas . & et per aditum factum ex muris cadentibus reliqui obsidentes ingrediantur castrum , \textbf{ vel ciuitatem obsessam : } et sic poterunt obtinere illam . \\\hline
3.3.18 & que lançan piedras \textbf{ o por castiellos que se pueden enpuxar } fasta las menas del castiello o de la çibdat cercada . & sicut per machinas lapidarias , \textbf{ vel per aedificia propulsa usque ad moenia castri , } vel ciuitatis obsessae , \\\hline
3.3.18 & o por castiellos que se pueden enpuxar \textbf{ fasta las menas del castiello o de la çibdat cercada . } Conuiene de vsar de tales armadijas o de tales armamientos & vel per aedificia propulsa usque ad moenia castri , \textbf{ vel ciuitatis obsessae , } oportet talibus uti argumentis \\\hline
3.3.18 & Et pues que assi es aquel \textbf{ que çerca algun castiello o alguna çibdat } si la quiere tomar & Illae igitur \textbf{ qui obsidet castrum aut ciuitatem aliquam , } si vult eam impugnare \\\hline
3.3.18 & o en todas aquellas maneras de lançar \textbf{ que dichas son o en algunas o en alguna dellas podra acomter el castiello o la çibdat cercada . } Ca si conplidamente sopiere todas estas maneras de engennios & vel omnibus praefatis modis proiiciendi , \textbf{ vel aliquibus | sine aliqua praedictarum machinarum , castrum , vel ciuitatem obsessam poterit impugnare . } Si enim plena notitia habeatur de machinis , \\\hline
3.3.19 & por los artifiçios de madera enpuxados \textbf{ e allegados a los muros del castiello o de la çibdat . } Et estos artifiçios pueden se adozir a quatro maneras . & per aedificia impulsa ad muros , \textbf{ vel ad moenia castri , | vel ciuitatis obsessae . } Huiusmodi autem aedificia \\\hline
3.3.19 & que y estan cauan los muros de la fortaleza \textbf{ e en castiella a estos artifiçios llaman gatas . } Et es este artifizo muy prouechoso & et impellendum usque ad muros munitionis obsessae ; \textbf{ sub quo homines existentes fodiunt muros illos . } Est autem hoc aedificium utile , \\\hline
3.3.19 & commo este . \textbf{ la terçera manerar de artificio es torres o castiellos . } Ca si las fortalezas cercadas non se pueden tomar par los carneros & tale aedificium impelli . \textbf{ Tertium genus aedificiorum sunt turres vel castra . } Nam si nec per arietes , \\\hline
3.3.19 & e segunt aquella mesma o avn segunt mas asta medida son de fazer las torres \textbf{ o los castiellos de madera } los quales castiellos deuen ser cubiertos de cueros crudos & et secundum huiusmodi mensuram , \textbf{ vel etiam secundum altiorem construendae sunt ligneae turres vel castra , } quae tegenda sunt crudis coriis , \\\hline
3.3.19 & o los castiellos de madera \textbf{ los quales castiellos deuen ser cubiertos de cueros crudos } por que non gelos quemen con fuego . & vel etiam secundum altiorem construendae sunt ligneae turres vel castra , \textbf{ quae tegenda sunt crudis coriis , } ne succendantur ab igne . \\\hline
3.3.19 & por que non gelos quemen con fuego . \textbf{ Et con estos castiellos de madera } en dos maneras conbaten las fortalezas cercadas . & ne succendantur ab igne . \textbf{ Cum his quidem ligneis castris } dupliciter impugnantur munitiones obsessae . \\\hline
3.3.19 & e assy los puede entrar \textbf{ Otrossi en estos castielos se ordenan puentes } que echan sobre los muros & qui sunt in basso , vel in terra . \textbf{ Rursus in huiusmodi castris } ordinantur pontes cadentes , \\\hline
3.3.19 & so los quales se encubren los omnes \textbf{ que traen o enpuxan los castiellos de madera } fasta los muros de la fortaleza cercada & sub quibus teguntur homines trahentes , \textbf{ vel impellentes castra } usque ad moenia munitionis obsessae . \\\hline
3.3.19 & en tres maneras puede acometer la fortaleza . \textbf{ ca que el castiello assi fecho } para conbatir la fortaleza & tripliciter impugnanda est munitio . \textbf{ Nam in castro sic aedificato } ad munitionem impugnandam , \\\hline
3.3.19 & so los quales estan los omnes \textbf{ que traen o enpuxan los castiellos de madera . } Et quando aquel castiello o castiellos se llegaren & a quibus sunt homines trahentes , \textbf{ vel impellentes castrum . } Cum ergo castrum illud appropinquauit \\\hline
3.3.19 & que traen o enpuxan los castiellos de madera . \textbf{ Et quando aquel castiello o castiellos se llegaren } quanto deuen a los muros de la fortaleza los cercados & vel impellentes castrum . \textbf{ Cum ergo castrum illud appropinquauit } quantum debuit \\\hline
3.3.20 & e saber en qual manera son de construyr \textbf{ e de fazer los castiellos et las cibdades e las otras fortalezas } por que non se puedan conbatir ligeramente . & qualiter aedificanda sunt castra , \textbf{ et ciuitates , | et munitiones ceterae , } ne faciliter impugnentur . \\\hline
3.3.21 & entre todas las otras cosas de que deuen basteçer la fortaleza \textbf{ e el castiello deuenla basteçer mayormente de mijo . Ca el mijo menos se podresçe } e mas dura que tedos los otros granos . & vel castrum obsessum milio : \textbf{ nam milium inter cetera minus putrefit , | et plus durare perhibetur . } Copia etiam carnium salitarum non est praetermittenda . \\\hline
3.3.21 & e presta en la fortaleza çercada . \textbf{ Lo segundo en basteciendo el castiello o la cibdat } que teme de ser cercada & eo quod ad multa sit utilis . \textbf{ Secundo in muniendo castrum } vel ciuitatem aliquam obsidendam , \\\hline
3.3.22 & E la terçera manera por artifiçios . \textbf{ assi commo castiellos e gatas de madera enpuxados } fasta los muros del & Et tertius per aedificia impulsa \textbf{ usque ad moenia castri , } vel ciuitatis obsessae . \\\hline
3.3.22 & fasta los muros del \textbf{ castielloo de la çibdat cercada . } Por la qual cosa si ya ensseñamos & usque ad moenia castri , \textbf{ vel ciuitatis obsessae . } Quare si docuimus per praefatos modos \\\hline
3.3.22 & que se puede de ligero cauar e estonçe es de \textbf{ enfortaleçer el castiello o la çibdat } afondando mucho las carcauas & quae de facili fodi potest : \textbf{ et tunc per profundas foueas est fortificanda munitio , } ne per cuniculos deuincatur . \\\hline
3.3.22 & a este fierro llamaron lobo por que con sus dientes prende el carnero . \textbf{ Mas contra los castiellos de madera } valen mucho los pedaços de fierro ençendidos . & Unde et bellatores antiqui huiusmodi ferrum vocauerunt Lupum , \textbf{ eo quod acutis dentibus arietem caperet . } Contra castra vero multum valent ferra ignita : \\\hline
3.3.22 & por que puedan passar \textbf{ allende del castiello o de la villa çercada . } la qual tierra cauada & unde debet transire castrum ; \textbf{ qua suffossa , et castro demerso in ipsam propter magnitudinem ponderis , } oportet castrum iterum construi , \\\hline
3.3.22 & la qual tierra cauada \textbf{ conuiene de apoyar bien el castiello o la çerca } por que se non funda & qua suffossa , et castro demerso in ipsam propter magnitudinem ponderis , \textbf{ oportet castrum iterum construi , } eo quod non possit \\\hline
3.3.22 & quando desto dubdaren ante que los muros sean foradados \textbf{ çerca de aquellos muros deuen algunos castiellos de madera } o si pueden deuen fazer muros de piedra & antequam hoc fiat , \textbf{ iuxta illos muros erigantur aedificia lignea , } vel ( si sit possibile ) \\\hline

\end{tabular}
