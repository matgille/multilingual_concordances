\begin{tabular}{|p{1cm}|p{6.5cm}|p{6.5cm}|}

\hline
1.1.1 & Amen . FINIS \textbf{ Oportet ut latitudo sermonis } in unaquaque re sit & por que puedan vençer sus enemigos . \textbf{ onuiene que la largueza delos sermones } e delas palabras en cada \\\hline
1.1.3 & absque diuina gratia obseruari non possunt , \textbf{ decet quemlibet hominem , } et maxime regiam maiestatem & njn guardar \textbf{ syn la gera de dios conviene de cada vn omne } e mayormente prinçipe o Rey \\\hline
1.1.4 & His peractis , dicamus , \textbf{ quod decet regiam maiestatem } hos modos viuendi cognoscere , & pues que asy es estas cosas acabadas digamos \textbf{ que conujene ala rreal magestad } e a todo rrey saber \\\hline
1.1.5 & ut dicitur 2 Ethic . ) \textbf{ oportet operationes , } per quas finem consequimur , & en el segundo libro delas ethicas \textbf{ que conuiene que las obras } por las quales nos alcançamos la fin \\\hline
1.1.5 & Tertio agere \textbf{ oportet delectabiliter : } nam quanto quis in aliquo opere magis delectatur , & Lo terçero conuiene \textbf{ que no sobremos con delectaçion e con plazer } por que qua quanto cada vno mas se delecta en la obra \\\hline
1.1.5 & tamen quia non faciunt ea bene et delectabiliter , \textbf{ non oportet per huiusmodi opera } eos & fazen bien njn \textbf{ delectosamente non les conuiene } que por aquellas obras alcançen buena fin \\\hline
1.1.5 & Ut ergo agamus , \textbf{ oportet nobis } praestituere aliquem finem : & ø \\\hline
1.1.6 & si felicitas ponitur \textbf{ esse perfectum bonum , oportet quod sit bonum } secundum intellectum , et rationem : & Pues si la bien auentraança es bien acabado e conplido . \textbf{ Conuiene que sea bien segunt el en tedimiento } e segunt Razon \\\hline
1.1.6 & vel ( quod idem est ) \textbf{ oportet quod sit tale bonum , } quale recta ratio prosequendi iudicet . & e segunt Razon \textbf{ O conuiene que sea tal bien qual iudga la razon derecha } que nos auemos de segnir \\\hline
1.1.6 & Ad praesens autem scire sufficiat , \textbf{ quod non decet aliquem hominem } suam felicitatem ponere & Mas quanto a esto presente cunple de saber \textbf{ qua non conuiene a ningun omne } de poner la feliçidat suia \\\hline
1.1.8 & si plene manifestare vult ipsum signatum , \textbf{ oportet quod sit } quid notum et manifestum : & si conplida mente quiere demostrar le que significa \textbf{ conuiene que sea conosçida cosa e magnifiesta . } Mas las cosas que son de dentro del alma \\\hline
1.1.8 & secundum dignitatem personarum , \textbf{ ut plura bona decet dignis , et sapientibus , quam indignis , et Histrionibus . } Sed si Princeps suam felicitatem in honoribus ponat , & segunt las dignidades delas personas ¶ \textbf{ assi que de mayores bienes alos que son dignos e sabios | que non alos que non son dignos } nin alos iuglares ¶ \\\hline
1.1.8 & secundum personarum dignitatem . \textbf{ Quod non decet regiam maiestatem , } suam ponere felicitatem in gloria , & ø \\\hline
1.1.9 & sed hoc pro tanto dictum est , \textbf{ qua decet Reges , } et Principes esse magnificos , et magnanimos . & mas por tanto dixo esto el philosofo \textbf{ por que conuiene alos Reys alos prinçipes } de ser manificos e grandes e magnanimos e de grandes coraçones . \\\hline
1.1.9 & nam ex hoc efficerentur populi praedatores . \textbf{ Quod non decet regiam maiestatem suam } Vegetius in libro De re militari , & non serie buen prinçipe mas serie tirano ¶ \textbf{ Et por ende serie dicho Robador del pueblo . } ize vegeçio enł primero libro \\\hline
1.1.10 & quam aliquod bonum singulare , \textbf{ non decet principem } suam felicitatem ponere in ciuili potentia . & que el bien singular et personal . \textbf{ Non conuiene alos prinçipes } poner su bien andança en el poderio \\\hline
1.1.10 & suam felicitatem ponere in ciuili potentia . \textbf{ Quinto hoc non decet ipsum , } quia huiusmodi principatus infert & çiuil¶ \textbf{ La quinta razon por que non conuiene al prinçipe } poner la su bien andança en el poderio çiuiles \\\hline
1.1.10 & si possit sibi subiicere nationes multas . \textbf{ Quod non deceat Regiam maiestatem } suam felicitatem ponere in robore corporali , & quando pudiere subiugar \textbf{ assi las naçiones . } et las gentes . \\\hline
1.1.11 & et pulchritudo . \textbf{ Non decet ergo Regem , } nec aliquem hominem in talibus & en que es guardada la color e la fermosura ¶ \textbf{ Et pues que assi es non conuiene al rey } nin a ningun omnen poner la su feliçidat \\\hline
1.1.11 & Dicamus ergo ad intelligentiam omnium dictorum , \textbf{ quod non decet } Principem felicitatem ponere & Pues que assi es para entender todas las cosas sobredichas digamos \textbf{ que non conuiene al Rey } nin al prinçipe poner su feliçidat \\\hline
1.1.12 & Tamen , ut appareat , \textbf{ quomodo deceat regiam maiestatem } ponere suam felicitatem & Enpero por que paresca lo primero \textbf{ en qual manera conuenga ala Real magestad } de poner la primera feliçidat \\\hline
1.1.12 & in actu prudentiae , \textbf{ sciendum quod decet Regem maxime } suam felicitatem & en las obras de pradençia . \textbf{ ¶ Et la segunda commo le conuiene } de poner er la su bien andança solamente en dios . \\\hline
1.1.12 & sed bonum rationis sit bonum uniuersale , et intelligibile , \textbf{ decet Regiam maiestatem } eo ipso quod homo est , & e delan razon es bien comun e entelligible \textbf{ Conuiene ala Real magestad } en quanto es omne \\\hline
1.1.12 & et perfecte solus Deus , \textbf{ oportet quod quicunque principatur , } siue regnat , & e de gouernar prinçipalmente e acabadamente . \textbf{ Conuiene que qual se quier } prinçipeo Rey \\\hline
1.1.12 & et debet eam expectare ab ipso , \textbf{ decet Regem , } qui est Dei minister , & e deuen la esparar del . \textbf{ Conuiene al Rey } que es ofiçial de dios \\\hline
1.2.1 & Virtutes autem quaedam sunt quidam ornatus , \textbf{ et quaedam perfectiones animae . Oportet ergo prius ostendere , } quot sunt potentiae animae , & Ca las uirtudes son vnos hornamentos e conponimientos e hunas perfectiones \textbf{ que fazen acabada el alma . | Pues que assi es conuiene } mostrͣ primero quantos son los poderios del alma \\\hline
1.2.1 & quae sit unaquaeque illarum virtutum , \textbf{ et quomodo decet Reges , } et Principes tales virtutes habere . & qual es cada vna de estas uirtudes . \textbf{ Et en qual manera conuiene alos Reyes } e alos prinçipes de auer estas uirtudes ¶ \\\hline
1.2.1 & et appetitus intellectiuus , et sensitiuus , \textbf{ oportet esse in talibus virtutes morales . } Quomodo distinguuntur virtutes : & que es la uoluntad Et el appetito sen setiuo \textbf{ que sigue al seso conuiene } por fuerça \\\hline
1.2.2 & virtutem esse aliquid secundum rationem : \textbf{ oportet ergo esse rationalem potentiam , } in qua potest esse virtus . & segunt razon \textbf{ e por ende conuiene | que sea poderio razonable } aquel en que esta la uirtud ¶ \\\hline
1.2.2 & irascibilis scilicet , et concupiscibilis : \textbf{ oportet omnem virtutem moralem , } vel esse in intellectu , & Et el otro es para cobdiçiar \textbf{ ¶ Conuiene que toda uirtud moral } o sea en el entendimiento o en la uoluntad o en el appetito enssannador \\\hline
1.2.3 & de quibus omnibus quid sunt , \textbf{ et quomodo decet eas Reges habere , } et quas partes habent , & con estas dichas diez \textbf{ son doze las uirtudes morales delas quales todas que cosas son e en qual manera . | Conuiene a los Reyes } e alos prinçipes delos auer \\\hline
1.2.3 & nisi circa ea quae sunt in potestate nostra , \textbf{ in quibus decet nos ponere medietatem , } vel aequalitatem , siue rectitudinem : & que son en nuestro poder . \textbf{ En las quales cosas nos conuiene } de poner meatado ygualdat o derechura . \\\hline
1.2.4 & ostendentes , \textbf{ quomodo Reges et Principes oportet } talibus esse ornatos . & Et en cabo de todo esto de aquellas uirtudes \textbf{ que son sobre todas las otras uirtudes mostrando en qual manera los Reyes e las prinçipes } han de ser conpuestos e honrrados \\\hline
1.2.5 & si virtuosus esse debet , \textbf{ oportet quod fiat prudenter , iuste , fortiter , et temperate : } ideo hae quatuor uirtutes , & conuiene que se faga sabiamente \textbf{ e iusta mente . | fuerte mente . e tenprada mente . } Et por ende estas quatro uirtudes son dichas \\\hline
1.2.5 & quid est prudentia , \textbf{ et quod decet Reges , } et Principes esse prudentes . & Ca primeramente diremos que cosa es la pradença . \textbf{ Et que conuiene alos Reyes } e alos prinçipes de ser pradentes e sabios \\\hline
1.2.5 & Secundo determinabimus de ipsa Iustitia , \textbf{ ostendentes quod decet Reges , } et Principes esse iustos . & Lo segundo diremos dela iustiçia \textbf{ mostrando que conuiene alos Reyes } e alos prinçipes de ser iustos e derechureros . \\\hline
1.2.5 & et aliis uirtutibus , manifestantes , \textbf{ quomodo decet Reges } et Principes talibus uirtutibus esse perfectos . & e delas otras uertudes mostrando \textbf{ e declarando en qual manera conuiene alos Reyes } e alos prinçipes de ser acabados e conplidos destas tales uirtudes . \\\hline
1.2.6 & et agibilia sint singularia , \textbf{ oportet prudentiam esse circa particularia , } applicando uniuersales regulas & en las cosas singulares . \textbf{ Conuiene que la pradençia sea cerca las cosas singulares | e particulares } allegando las reglas generales alos negoçios singulares \\\hline
1.2.7 & eo quod naturaliter deficiat a viri prudentia . \textbf{ Hoc etiam modo iuuenes naturaliter decet } antiquioribus esse subiectos , & por que naturalmente fallesçe dela sabiduria del omne ¶ahun \textbf{ en esta misma gusa las moços } e los mançebos \\\hline
1.2.7 & qui naturaliter seruus existit . \textbf{ Ut igitur Rex naturaliter dominetur oportet } quod polleat prudentia , et intellectu . & por cuya mengua el sieruo esta naturalmente en su seruidunbre . \textbf{ Pues que assi es el Rey | por que naturalmente sea sennor } conuiene que florescaen sabiduria e en entendimiento \\\hline
1.2.7 & quod polleat prudentia , et intellectu . \textbf{ Quot , et quae oporteat habere Regem , } si & ø \\\hline
1.2.8 & gentem aliquam ad bonum dirigere , \textbf{ oportet quod habeat memoriam praeteritorum , } et prouidentiam futurorum . & e la su conpanna a alguons bienes . \textbf{ Conuiene que aya memoria de las cosas passadas . } Et que aya prouision delas cosas passadas \\\hline
1.2.8 & per quem dirigit , \textbf{ oportet quod habeat intellectum et rationem , } siue oportet & por la qual deue guiar el Rey . \textbf{ Conuiene le que aya entendimiento } e razon o conuiene le que sea entendido e razonable . \\\hline
1.2.8 & oportet quod habeat intellectum et rationem , \textbf{ siue oportet } quod sit intelligens et rationale . & por la qual deue guiar el Rey . \textbf{ Conuiene le que aya entendimiento } e razon o conuiene le que sea entendido e razonable . \\\hline
1.2.8 & quo Rex suum populum dirigit , \textbf{ oportet quod sit humanus , } quia Rex ipse homo est . & Ca la manera por que el Rey guia el su pueblo \textbf{ Conuiene que sea manera de omne . } Ca el Rey omne es \\\hline
1.2.8 & volens alios dirigere , \textbf{ oportet quod sit intelligens , } cognoscendo principia , & El que quiere alos otros guiar \textbf{ conuiene le que sea entendido } conosciendo los prinçipios e las razones . \\\hline
1.2.8 & ex illis praemissis cunclusiones intentas . \textbf{ Vel oportet quod sit intelligens , } sciendo leges , & e las razones que quiere ençerrar ¶ \textbf{ Et otrosi conuiene al Rey | que sea entendido } e sabio sabiendo las leys \\\hline
1.2.8 & et regulae agendorum . \textbf{ Oportet autem quod sit rationalis , } speculando ex illis regulis & e reglas para lo que ha de fazer ¶ \textbf{ Et otrosi conuiene al Rey | que sea razonable conosçiendo e entendiendo } por aquellas reglas \\\hline
1.2.8 & quae est alios dirigens , \textbf{ oportet quod sit solers , et docilis . } Nam qui in tanto culmine est positus , & que es tal que ha de gouernar los otros . \textbf{ Conuiene le de sor sotil e doctrinable . } Ca aquel que esta en tanta alteza de dignidat \\\hline
1.2.8 & ut tantam gentem regere habeat , \textbf{ oportet quod sit industris , et solers , } ut sciat ex se inuenire bona gentis sibi commissae . & que es puesto para gouernar tanta gente e tanto pueblo . \textbf{ Conuiene le que sea engennoso e sotil | por que sepa } por si buscar e fallar aquellos bienes \\\hline
1.2.8 & quod dicitur de Magnanimo 4 Ethicorum , \textbf{ quod non decet } ipsum fugere commouentem . & e del que ha grant coraçon en el deçimo libro delas ethicas \textbf{ do dize que non conuiene al magnanimo } menospreçiara \\\hline
1.2.8 & et alia exquirenda . \textbf{ Oportet igitur Principem respectu gentis } cui praeest , & por ende conuiene al Rey \textbf{ e al prinçipe en conparacion de su gente } e de su pueblo \\\hline
1.2.8 & ut possit eam melius in debitum finem dirigere . \textbf{ Ultimo oportet ipsum esse cautum . } Nam sicut in speculabilibus falsa aliquando admiscentur veris , & e traher los ala fin que deue¶ \textbf{ Lo postrimero conuiene al Rey | que sea muy aꝑçebido . } Ca assi commo en las sçiençias especulatuias algunas cosas falsas \\\hline
1.2.8 & sed apparent bona . \textbf{ Oportet igitur Regem esse cautum , } respuendo apparenter bona , & maguera que lo non sean . \textbf{ commo quier que paresçan buenas Et pues que assi es conuiene al Rey sea aꝑcebido } para desechar e despreçiar aquellas cosas \\\hline
1.2.9 & quae superius diximus , \textbf{ oportet ipsos esse bonos , } et non habere voluntatem deprauatam : & que dixiemos de suso \textbf{ conuieneles | que sean buenos } e que non ayan uoluntad mala nin desordenada \\\hline
1.2.12 & quibus ostendi poterit , \textbf{ quod maxime decet Reges , } et Principes esse iustos . & por las quales podremos mostrar \textbf{ que mucho conuiene alos reyes | e alos prinçipes } de ser iustos \\\hline
1.2.12 & ex parte ipsius personae regiae \textbf{ maxime decet } ipsum seruare Iustitiam . & animada de todo lo que le ha de fazer . Paresçe de parte dela persona del Rey \textbf{ que conuiene mucho al Rey } de guardar la iustiçia¶ \\\hline
1.2.12 & est claritate stellarum . \textbf{ Si ergo decet Reges et Principes } habere clarissimas virtutes & que la claridat delas estrellas \textbf{ Et pues que assi es si conuiene alos Reyes } e alos prinçipes de auer \\\hline
1.2.13 & et non recte , \textbf{ oportet dare virtutem aliquam } circa timores , et audacias . & e en las osadias \textbf{ por la qual sea el omne reglado en ellos . | por que contesce que algunos remen algunas cosas } que han de temer e alas uegadas temen alguas cosas \\\hline
1.2.15 & quam de bono alterius . \textbf{ Non igitur oportuit } tam vehementes delectationes ponere in nutrimento , & que del bien de otre . \textbf{ Et por ende non conuiene } quela natura pusiese tan grandes delecta connes en la uianda \\\hline
1.2.15 & Sed castitas , et pudicitia venereas delectationes refraenant . \textbf{ Oportet enim vere temperatum } non exercere opera venerea , neque gestus . & ca Mas la castidat \textbf{ e la linpieza refreña } e abaxan las delecta connes \\\hline
1.2.16 & ex quibus tres rationes sumi possunt , \textbf{ quod maxime decet } Reges & para prouar \textbf{ que mucho conuiene alos Reyes } et alos prinçipes de ser tenprados . \\\hline
1.2.16 & et non sequitur rationem , sed passionem . \textbf{ Quare si decet } personam regiam ostendere & nin entendimiento mas passion \textbf{ e delecta conn | por la qual cosa } si pertenesçe ala persona del Rey \\\hline
1.2.18 & restat ostendere , \textbf{ quod deceat } eos esse largos , liberales , et communicatiuos . & si fueren auarientos fincanos de demostrar \textbf{ que conuiene alos Reyes } de ser largos liberales e dadores . \\\hline
1.2.18 & nam liberales sunt potissime amabiles . \textbf{ Quare si maxime decet Reges et Principes , } ut sint dilecti & Ca los liberales son estrimadamente amables e de amar \textbf{ por la qual razon si mucho conuiene alos Reyes } e alos prinçipes de ser amados de todos \\\hline
1.2.18 & Imo si contingat liberalem \textbf{ dare plus quam deceat , } ut vult Philosophus , & apenas puede sobrepuiar ala muchedunbre de las sus rentas . Por ende si contesçe algunas uegadas al liberal de dar \textbf{ mas de quanto deue legunt } dize el philosofo tomara tristeza tenpradamente \\\hline
1.2.18 & vel si non expendat \textbf{ ubi oportet , } quam si expendat & o si non espendiere do deua \textbf{ que si espendiere } do non le conuiene espender \\\hline
1.2.18 & et Principes in dando \textbf{ quibus non oportet , } vel cuius gratia non oportet . & Mas los Reyes e los prinçipes de suranse \textbf{ e arriedran se dela liberalidat en dando aqui } e non deuen o non \\\hline
1.2.18 & quibus non oportet , \textbf{ vel cuius gratia non oportet . } Dant enim & e arriedran se dela liberalidat en dando aqui \textbf{ e non deuen o non } por la razon que deuen . \\\hline
1.2.18 & et ut liberales sint , \textbf{ oportet eos beneficiare bonos , } et boni gratia . & e por razon devien non \textbf{ por otra razon ninguen } ssi commo es dichon de suso \\\hline
1.2.19 & et militiae . \textbf{ Decet enim magnificum } ( ut dicitur 4 Ethicor’ ) & assi commo son los casamientos e las caualłias . \textbf{ Ca conuiene al magnifico } assi commo dize aristotiles \\\hline
1.2.20 & et subterfugit quantum potest : \textbf{ sic dato quod paruificum oporteat } expensas facere , & por que se non taiasse . \textbf{ Bien alłi puesto que el paruifico } e al escasso sea dado de fazer grandes espenssas sienpre tarda \\\hline
1.2.20 & distribuere bona regni , \textbf{ maxime decet ipsum esse magnificum . Nam quia est caput regni , } et gerit in hoc Dei vestigium , & e a el pertenesca de partir los bienes del regno mucho le conuiene a el de ser magnifico . \textbf{ Ca porque es cabeça del regno } e ha en esto semeiança de dios \\\hline
1.2.21 & qui sumptus \textbf{ quibus operibus deceant . } Ad eos autem maxime spectat & por que ellos mucho mas deuen ser sabios \textbf{ e conosçedores quales despenssas a quales obras conuienen . } Et aellos otrosi mucho mas pertenesçe de fazer grandes donaconnes \\\hline
1.2.21 & et prompte expendere . \textbf{ Decet etiam Reges , } et Principes magis intendere , & e mas espender delectable ment en sin detenimiento \textbf{ Et otrosi avn conuiene alos Reyes } e alos prinçipes de entender e cuydar \\\hline
1.2.21 & Omnes igitur proprietates magnifici per amplius , \textbf{ et perfectius decet } ipsos Reges habere . Unde et Philos’ 4 Ethic’ vult , & conuiene auer a los Reyes \textbf{ e ales prinçipes mas conplidamente e cabada mente . } Onde el philosofo en el quarto libro delas ethicas dize \\\hline
1.2.21 & Sed , ut ibidem dicitur , \textbf{ tales oportet esse nobiles et gloriosos . } Quare quanto est nobilior aliis , & por que non puede cada vno fazer grandes espenssas \textbf{ Mas assi commo alli dize el philosofo tales son los nobles e los głiosos . } por la quel cosa en quanto el Rey es mas noble \\\hline
1.2.22 & Quidam autem se habent , \textbf{ ut decet , } ut magnanimi . & Mas otros son que se han cerca las grandes honrras \textbf{ assic̃omo conuiene . } Et estos son dichos magnanimos \\\hline
1.2.23 & ut pro bono diuino et communi , paratus sit vitam exponere . \textbf{ Secundo decet Reges , } et Principes esse plurimum retributiuos : & e ponersea la muerte \textbf{ ¶Lo segundo conuiene alos Reyes } e alos prinçipes de ser \\\hline
1.2.23 & negocia enim ardua pauca sunt respectu aliorum . \textbf{ Non autem decet Reges } et Principes & e en conpara connde los otros . \textbf{ Ca non conuiene alos Reyes } nin alos prinçipes desenbargar \\\hline
1.2.23 & quae sunt multa . \textbf{ Quarto decet esse apertos , } ut esse veridicos ; & que son muchos alos otros ¶ \textbf{ Lo quarto conuiene alos Reyes } de seer manifiestos e claros e seer uerdaderos \\\hline
1.2.23 & ut sit iustus , et virtuosus . \textbf{ Quinto decet Reges , } et Principes non curare , & e uirtu oso¶ \textbf{ Lo quinto conuiene alos Reyes } e alos prinçipes \\\hline
1.2.24 & quae dicitur honoris amatiua . \textbf{ Sicut ergo decet Reges } et Principes esse magnificos , et liberales : & que es dicha amadora de h̃orra . \textbf{ Et pues que assi es assi commo conuiene alos Reyes } e alos prinçipes de seer magnificos e liberales \\\hline
1.2.25 & ne trahamur ratione difficultatis , \textbf{ oportet quod ei sit annexa humilitas , } ne ultra quam ratio dictet & por razon dela graueza . \textbf{ Et por ende conuiene | que aella sea ayuntada la humildat } por que non pueda passar allende \\\hline
1.2.26 & Nam aliqui ex hoc quaerunt excellentiam et iactantiam , \textbf{ deiiciendo se ultra quam deceat . } Unde Philosophus 4 Ethic’ quandam gentem Graecam , & Por que alguons por esto demandan ser enxalçados e alabados despreçiando se \textbf{ mas que les conuiene . } Et por ende el philosofo en el quarto libͤdelas \\\hline
1.2.26 & et quod semper magnanimitati est annexa humilitas : \textbf{ quare si decet Reges } et Principes esse magnanimos , & es ayuntada la humildat . \textbf{ por la qual cosa se conuiene alos Reyes } e alos prinçipes de seer magranimos \\\hline
1.2.26 & non ultra quam ratio dictet , \textbf{ sed ut decet eorum statum , } quod faciunt humiles : & e el entendimiento lo dize . \textbf{ mas assi comma conuiene al su estado dellos . } e esto es lo que fazen los humildosos . \\\hline
1.2.27 & de virtutibus respicientibus exteriora bona , \textbf{ et ostendimus quomodo Reges et Principes decet ornari virtutibus illis , } restat dicere de mansuetudine , & que catan alos bienes de fuera . \textbf{ Et mostramos en qual mana conuiene alos Reyes e alos | prinçipesser honrrados destas uirtudes . } fincanos agora de dezir dela \\\hline
1.2.27 & et deficere , \textbf{ oportet ibi dare virtutem aliquam , } per quam dirigamur ad bene agendum , & e tal sesçer conuiene de dar y . \textbf{ alguna uirtud por la qual seamos enderesçados } abien obrar \\\hline
1.2.27 & virtuosus non esset . \textbf{ Tanto ergo magis decet Reges et Principes moueri } ad punitionem faciendam , & non seria uirtuoso . \textbf{ Et pues que assi es tanto | mas conuiene alos Reyes } e alos prinçipes de se mouer a dar penas . \\\hline
1.2.28 & ut probari habet 1 Politicorum , \textbf{ oportet circa uerba , } et opera , & en el primero libro delas politicas . \textbf{ Conuiene cerca las palauras } e çerca las obras \\\hline
1.2.29 & quam sibi inesse non credit . \textbf{ Non tamen oportet } quod de se dicat totam bonitatem , & nin conosce que es en ssi . \textbf{ Enpero non conuiene } que dessi mismo daga toda la bondat \\\hline
1.2.29 & laudantes seipsos , \textbf{ ideo etiam ex parte aliorum decet } in talibus declinare in minus , & a aquellos que alaban assi mismos . \textbf{ Por ende avn de parte de los otros } conuiene en tales cosas declinar a lo menos \\\hline
1.2.29 & videntur esse iactatores , et onerosi . \textbf{ Quare si Reges decet } esse non contemptibiles , & estos paresçen alabadores e pesados . \textbf{ Por la qual cosa si conuiene alos Reyes } de non ser despreçiados \\\hline
1.2.30 & et bene facere , \textbf{ oportet } circa ipsos iocos & e de bien fazer \textbf{ conuiene erca tales iuegos } e cerca tales delecta connes deuiegos dar alguna uirtud . \\\hline
1.2.30 & quomodo Reges , \textbf{ et Principes decet } esse iocundos . & en qual manera conuiene alos Reyes \textbf{ e alos prinçipes } de ser alegres e iugadores . \\\hline
1.2.30 & et econuerso . \textbf{ Si igitur decet homines reprimere } superfluitates ludorum , & que es sobeia a otro . \textbf{ Et por ende si conuiene aton dos los omes } de repremir las sobeianias de los iuegos mucho \\\hline
1.2.30 & habere annexam . \textbf{ Tanto igitur decet Reges et Principes moderate } uti delectationibus ludorum , & e honesto pareste \textbf{ que aya en el alguna moçedat ayuntada Et pues que assi es en tanto conuiene alos Reyes } e alos prinçipes de vsar \\\hline
1.2.31 & in hanc sententiam conuenerunt , \textbf{ quod oportet virtutes connexas esse . } Dixerunt enim & commo los p̃h̃osacuerdan en esta sentençia \textbf{ que conuiene que todas las uirtudes sean ayinntadas la vna con la vna con la otra . } Ca dixieron que aquel que ha vna uirtud \\\hline
1.2.31 & Cum ergo omnino manifestum sit , \textbf{ quod decet Reges } et Principes & Et pues que assi es commo en todo en todo es manifiesto e prouado \textbf{ que conuiene alos Reyes } e alos prinçipes de auer alguas uirtudes . \\\hline
1.2.31 & omnes habet , \textbf{ nec oportet } si aliquae virtutes sint necessariae regibus , & non las ha todas . \textbf{ nin otrossi non conuiene } que si alguas uirtude sson necessarias alos Reyes \\\hline
1.2.31 & ita quod omnis bonus est prudens , et econuerso : \textbf{ non tamen oportet } quod habens perfecte unam virtutem moralem , & e el contrario todo sabio es bueno . \textbf{ Enpero non conuiene } que aquel que ha vna uirtud moral \\\hline
1.2.32 & et principari desiderant , \textbf{ oportet quod habeant virtutem illam , } quae est dominans & e enssenorear alos otros . \textbf{ Conuienele que aya aquella uirtud } que es sennora e prinçipante a todas las otras uirtudes . \\\hline
1.3.2 & et quomodo vitandae passiones praedictae . \textbf{ Quod scire tanto magis decet Reges et Principes , } quanto per passiones ipsorum maius valet induci malum , & ¶ La qual cosa sobre tanto \textbf{ mas parte nesçe alos Reyes | e alos prinçipes } en quanto por las passiones \\\hline
1.3.3 & prius videndum est , \textbf{ quomodo deceat Reges } et Principes esse amatiuos , et oditiuos . & primero deuemos ver \textbf{ en qual manera conuiene alos Reyes } et alos prinçipes de ser amadores e de ser mal queredores . \\\hline
1.3.3 & ex consequenti vero bonum proprium et priuatum : \textbf{ maxime tamen hoc decet Reges et Principes , } quod triplici via declarare possumus . & Et despues desto amen el bien propio e personal . \textbf{ Et commo quier que esto conuiene a todos los omes Empero mucho mas conuiene alos Reyes | e a los prinçipes } la qual cosa podemos declarar \\\hline
1.3.3 & ut dicitur 2 Rhetoricorum , \textbf{ decet Reges } et Principes amare Iustitiam , & assi commo dize el philosofo en el segundo libro dela rectorica . \textbf{ Conuiene alos Reyes } e alos prinçipes \\\hline
1.3.3 & quae ei contrariantur : \textbf{ tanto magis decet Reges et Principes , } quanto magis sunt persona publica , et communis . & que son contrarias al bien comun \textbf{ Tanto mas conuiene alos Reyes | e alos prinçipes } quanto mas son perssonas publicas e comunes que los otros . \\\hline
1.3.4 & Huiusmodi autem abominationem et desiderium \textbf{ tanto magis decet Reges et Principes , quanto magis eos decet } habere curam de bono regni et communi . & e desseo tanto \textbf{ mas conuiene alos Reyes e alos prinçipes de auer | quanto mas conuiene a ellos } de auer cuydado del regno e del bien comun . \\\hline
1.3.5 & nec etiam cadunt sub prouidentia . \textbf{ Decet ergo Reges et Principes } considerare bona non solum & nin avn ca en so prouidençia ¶ \textbf{ Et pues que assi es conuiene alos Reyes } e alos prinçipes de penssar los bienes \\\hline
1.3.5 & et debeant prouidere bona futura possibilia ipsi regno : \textbf{ decet eos esse bene sperantes per magnanimitatem , } quia habent omnia & que han de venir e los bienes que pueden acahesçer a su regno . \textbf{ Por ende conuiene a ellos de serbine esparautes | por la magnanimidat } que han en ssi \\\hline
1.3.5 & quae ad spem debitam requiruntur . \textbf{ Viso , quomodo decet Reges } et Principes & tenesçen ala espança conuenible \textbf{ ¶ visto conmolos Reyes } e los prinçipes se deuen bien auer \\\hline
1.3.5 & et non passionatum immoderata passione , \textbf{ decet Reges et Principes } non aggredi aliquid ultra vires , & por passion destenprada . \textbf{ Conuiene alos Reyes e alos prin çipes } de non acometer ninguna cosa \\\hline
1.3.5 & et non sperare aliqua non speranda . \textbf{ Secundo hoc decet } eos & que non son de esparar ¶ \textbf{ Lo segundo esto les conuiene alos Reyes } e alos prinçipes de parte del pueblo \\\hline
1.3.6 & Attamen immoderate timere \textbf{ nullo modo decet eos . } Immoderatus enim timor quatuor habere videtur , & Enpero temer destenpradamente en ninguna manera \textbf{ non conuiene alos Reyes . } t te paresçe que el temor \\\hline
1.3.7 & vel irae immittitur modo debito , \textbf{ et ut decet , } quando ira est organum virtutis , et rationis . & ala sanna en manera conue nible \textbf{ e assi commo conuiene } quando la lanna es organo \\\hline
1.3.7 & tanto magis decet Reges et Principes \textbf{ quanto magis decet } eos non impediri in usu rationis , & e alos prinçipes de se auer \textbf{ assi commo dicho es | quanto mas conuiene aellos } de non ser enbargados en el vso dela razon . \\\hline
1.3.8 & impediant operationes virtuosas , \textbf{ tanto magis decet Reges , } et Principes tales tristitias moderare , & enbarguen las obras de uirtudes tanto \textbf{ mas conuiene alos Reyes } e alos prinçipes de tenprar tales tristezas \\\hline
1.3.9 & circa quod est dolor et tristitia \textbf{ oportet delectationem et tristitiam } esse principales passiones respectu concupiscibilis . & terca la qual es el dolor e la tristeza . \textbf{ Conuiene que la delectaçion e la tristeza } sean prinçipales passiones \\\hline
1.3.9 & et quae timeamus . \textbf{ Sed hoc tanto magis decet Reges et Principes , } quanto eorum opera sunt digniora , & e quales cosas deuemos temer . \textbf{ Mas esto tanto mas conuiene alos Reyes | et alos prinçipes } en quanto las sus obras \\\hline
1.3.10 & decet nos omnes eas cognoscere ; \textbf{ et tanto magis hoc decet Reges et Principes , } quanto habere debent operationes maxime excellentes . & todasnr̃as obras conuiene a nos delas cognosçer todas . \textbf{ Et tanto mas esta conuiene alos Reyes | e alos prinçipes } quantomas deuen auer las obras mas altas e mas nobles . \\\hline
1.4.1 & quin committant aliqua turpia , \textbf{ de quibus decet eos uerecundari : } Reges tamen et Principes , & que non acometan algunas cosas torpes \textbf{ delas quales les conuiene | que tomne uerguença . } Enpero esto non es de loar en los uieios nin en los Reyes . \\\hline
1.4.1 & Reges tamen et Principes , \textbf{ quos decet esse quasi semideos , } non solum quod turpia committant , & por que los Reyes e los prinçipes alos quales conuiene de ser \textbf{ assi commo medios dioses } non solamente non les conuiene de fazer cosas torpes \\\hline
1.4.1 & quod debeant uerecundari . \textbf{ Non ergo decet eos uerecundari , } nisi ex suppositione : & que de una auer uerguença saluo \textbf{ por alguna razon } assi commo si contesçiesse \\\hline
1.4.1 & ut supra in tractatu de liberalitate sufficienter tetigimus . \textbf{ Decet etiam eos } tanto magis esse bonae spei quam alios , & e tractamos suficientemente en el terctado dela liberalidat \textbf{ ¶avn conuiene alos Reyes en } tantod ser de buean esperança \\\hline
1.4.1 & sunt digniora quam alia . \textbf{ Rursus decet eos esse magnanimos : } quia ( ut dicebatur & e alos prinçipes de ser magranimos \textbf{ e de grand coraçon } Ca assi commo es dicho dessuso \\\hline
1.4.1 & et esse vastatores gentium . \textbf{ Sic etiam decet } eos esse miseratiuos . & que los Reyes serien tiranos e destruydores delas gentes \textbf{ Et otrossi conuiene alos Reyes } e alos prinçipes ser mibicordiosos . \\\hline
1.4.1 & Esse autem verecundos \textbf{ non decet eos simpliciter , } ut superius dicebatur . & por si mas por la mala obra \textbf{ si la fezieren } assi commo es dicho de ssuso . \\\hline
1.4.2 & quod de facili peruertantur , \textbf{ immo decet } eos esse firmos et stabiles . & que de ligero se trasmuden \textbf{ e se tristor nen . } Mas conuiene aellos de ser firmes e estables ¶ \\\hline
1.4.2 & quia cum alia sint moderanda per mensuram , \textbf{ maxime decet mensuram } et regulam moderatam & por me lura \textbf{ e por regla mucho } mas conuiene ala regla \\\hline
1.4.3 & sufficienter ostensum fuit , \textbf{ quod non solum decet } Reges et Principes esse liberales , & Ca assi commo dicho es de suso quando tractamos delas uirtudes suficientemente fue mostrado \textbf{ que non solamente conuiene alos Reyes } e alos prinçipes de ser francos faziendo espenssas medianas \\\hline
1.4.4 & si viderit ipsum esse clementem , et misericordem . \textbf{ Tertio , non decet Reges , } et Principes dubia pertinaciter & vieren ser piadoso e miscderioso . \textbf{ ¶ Lo terçero non conuiene alos Reyes } e alos prinçipes de afirmar \\\hline
1.4.4 & et intellectui , \textbf{ decet Reges , et Principes , } qui aliis dominantur , & enssenorear por razon \textbf{ e por entendemiento Conuiene alos Reyes e alos prinçipes } que son senores de los otros segnir costunbres bueans e de loar \\\hline
1.4.5 & qui possunt esse boni et laudabiles , \textbf{ quomodo habere oporteat Reges , } et Principes , & Mas estas costunbres las quales pueden ser bueans e de loar \textbf{ e en qual manera conuiene alos Reyes } e alos prinçipes delas auer \\\hline
1.4.6 & credentes eas esse maius bonum quam sint . \textbf{ Decet ergo Reges , } et Principes hos malos mores fugere . & que ellos son el mayor bien que puede ser . \textbf{ Et por ende conuiene alos Reyes } e alos prinçipes de foyr \\\hline
1.4.6 & Viso qui sunt mali mores diuitum , \textbf{ et quod decet Reges , } et Principes fugere tales mores : & Visto quales son las malas costunbres de los ricos \textbf{ e que conuiene alos Reyes e alos } prinçipesarredrar se de tales costunbrs finca de ueer \\\hline
1.4.6 & videlicet bene se habere circa diuina , \textbf{ tanto magis decet Reges , et Principes , } quanto summo Deo iudici de pluribus debent reddere rationem . & que es auer se el omne bien terca las cosas diuinales tanto \textbf{ mas conuiene alos Reyes | e alos prinçipes } quanto ellos han de dar cuenta de mas cosas a dios \\\hline
1.4.7 & aliquos malos mores : \textbf{ quia non oportet omnes esse tales , } sed sufficit reperiri illud in pluribus : & li dellos contamos algunas malas costunbres \textbf{ ca non conuiene que todos seantales . } Mas abasta que aquellas costunbres sean falladas en muchos por que non \\\hline
1.4.7 & et secundum diuersos status : \textbf{ decet omnes homines } sequi mores laudabiles , & segunt departidos estados e departidas edades . \textbf{ Conuiene a todos los omes } de segnir las costunbres \\\hline
1.4.7 & et fugere vituperabiles . \textbf{ Sed tanto magis hoc decet Reges , et Principes , } quanto in altiori gradu existunt . & que son de denostar . \textbf{ Et tanto mas conuiene esto | alos Reyes } e alos prinçipes \\\hline
2.1.1 & et tunc sunt quasi diuini . \textbf{ Decet ergo omnes homines , } et maxime Reges & ¶ \textbf{ Et pues que assi es conuiene a todos los omes e mayormente alos Reyes } e alos prinçipes parar mientes muy acuçiosamente \\\hline
2.1.2 & ad per se sufficientiam vitae , \textbf{ oportet communitatem domus necessariam esse . } Reges ergo et Principes , & e por si vale a conplimiento dela uida . \textbf{ Conuiene que la comunidat dela casa sea mas neçessaria } Et pues que assi es los Reyes e los prinçipes \\\hline
2.1.3 & et typo ostendere , \textbf{ quod decet homines habere habitationes decentes } secundum suam possibilem facultatem ; & por que ael parte nesçe generalmente demostrar \textbf{ por figera e por exienplo que conuiene alos omes de auer conueibles moradas } segunt el su poder e la su riqueza . \\\hline
2.1.3 & non proprie quid artificiale erit , \textbf{ sed oportet illud } ( secundum quod huiusmodi est ) naturale esse . & e antepuesta non puede ser propreamente cosa artifiçial \textbf{ mas conuiene que aquello en quanto es tal sea cosa natural } Por la qual cosa si el omne es naturalmente \\\hline
2.1.3 & praesupponat communitatem domus , \textbf{ oportet communitatem domesticam siue domum } quid naturale esse . & e ante pongan la comunidat dela casa \textbf{ conuiene quala comunidat dela casa o la casa sea cosa natural . } Et por ende conuiene alos Reyes e alos prinçipes \\\hline
2.1.3 & quid naturale esse . \textbf{ Reges ergo et Principes decet } scire gubernare domestica , & conuiene quala comunidat dela casa o la casa sea cosa natural . \textbf{ Et por ende conuiene alos Reyes e alos prinçipes } de saber gouernar las cosas dela casa \\\hline
2.1.5 & et serui ad conseruationem . Quare si generatio et conseruatio est quid naturale , \textbf{ oportet domum quid naturale esse . } Amplius , quia generatio et conseruatio & conseruaçique por la qual cosa si la generaçion e la conseruaçion es cosa natural \textbf{ conuiene | que la casa sea cosa natural ¶ } Otrossi por que la generaçion e la conseruaçion non pueden ser apartadas la vna dela otra \\\hline
2.1.5 & quia eis ignoratis ignorabitur regimen ipsius domus ; \textbf{ multo magis tamen hoc decet Reges et Principes . } Quia ergo cognitio partium domus , & non sabran bien gouernar la casa . \textbf{ Empero esto much̃ mas conuiene alos Reyes e alos prinçipes . } Ca saber las partes dela casa \\\hline
2.1.6 & si domus debet esse perfecta , \textbf{ oportet ibi dare communitatem tertiam , } scilicet patris et filii . & Emposi la casa fuere acabada conuiene de dar \textbf{ y la terçera comunidat } que es de padre e de fijo . \\\hline
2.1.6 & potest sibi simile producere , \textbf{ sed oportet prius ipsam esse perfectam . } statim enim , & luego que es fecha fazer otra semeiante \textbf{ assi mas conuiene que ella primeramente sea acabada } enssi \\\hline
2.1.6 & nec statim potest sibi simile producere , \textbf{ sed oportet prius ipsum esse perfectum : } producere ergo sibi similem , & luego otro su semeinante \textbf{ mas conuiene que primeramente el sea acabado . } Et pues que assi es engendrar su semeiante non pertenesçe a cosa natural tomada en qual quier manera mas pertenesçe a cosa natural en quanto ella es acabada . \\\hline
2.1.6 & Cum enim primo homo est , \textbf{ oportet quod sit genitus : } et natura statim est solicita de salute eius ; & Ca quando el ome es primero \textbf{ conuiene que sea engendrado . } Et la natura luego es acuçiosa de su salud . \\\hline
2.1.6 & Ad hoc enim quod aliquid sit perfectum , \textbf{ non oportet } quod sibi simile producat , & Ca para ser la cosa acabada \textbf{ non conuiene que faga sienpre } e engendre su semeiante \\\hline
2.1.6 & Patet ergo quod ad hoc quod domus habeat esse perfectum , \textbf{ oportet ibi esse tres communitates : } unam viri et uxoris , aliam domini et serui , & Et por ende paresçe que para que la casa sea acabada \textbf{ que conuiene que sean enlla tres comuundades . } ¶ La vna del uaron e dela muger ¶ \\\hline
2.1.6 & de leui patere potest , \textbf{ quod ibi oportet } esse quatuor genera personarum . & e tres gouernamientos de ligero puede paresçer \textbf{ que conuiene que sean y . } quatro linages de perssonas \\\hline
2.1.6 & Nam cum in domo perfecta sint tria regimina , \textbf{ oportet hunc librum tres habere partes ; } in quarum prima tractetur primo de regimine coniugali : & Ca commo en la casa acabada sean tres gouernamientos . \textbf{ Ca conuiene que este libro sea partido en tres partes . } ¶ En la primera delas quales tractaremos del gouernamiento mater moianl . \\\hline
2.1.6 & Haec autem tria regimina bene cognoscere \textbf{ maxime decet Reges et Principes ; } quia eis diligenter inspectis , & ¶ En la terçera del señorio . \textbf{ ¶ Empero mucho conuiene alos Reyes | e alos prinçipes } de conosçer bien estos tres gouernamientos \\\hline
2.1.7 & et maxime Reges \textbf{ et Principes deceat sumere . } Deinde ostendemus , & quales si quier çibdadanos \textbf{ e mayormente los Reyes e los prinçipes . } Despues mostraremosen qual manera los uarones deuen gouernar sus mugers \\\hline
2.1.7 & et uniuersaliter omnem venereorum usum illicitum , \textbf{ tanto magis decet fugere Reges et Principes , } quanto decet eos meliores et virtuosiores esse . & La qual fornicaçion e general mente todo vso de luxuria non conueinble tanto \textbf{ mas conuiene alos Reyes } e alos prinçipes delo esquiuar quanto mas conuiene aellos de ser meiores \\\hline
2.1.8 & Probant autem Philosophi , \textbf{ quod decet coniugia indiuisibilia esse . } Ad quod ostendendum adducere possumus duas vias , & os philosofos prue una que los con̊uiene \textbf{ que los casamientos sean sin departimiento ninguno } e que non le puedan partir . \\\hline
2.1.8 & ne sit ibi violatio fidei , \textbf{ oportet virum indiuisibiliter suae } adhaerere uxori , & por que non le ay destroymiento de buean fe . \textbf{ Conuiene al marido | que se ayunte a su muger } sin ningun departimiento \\\hline
2.1.8 & et ad hoc quod inter uxorem et virum sit amicitia naturalis , \textbf{ oportet quod sibi inuicem seruent fidem , } ita quod ab inuicem non discedant . & que sea segunt natura \textbf{ e para que entre el uaron e la muger sea amistança natural conuiene que guarden vno a otro fe e lealtad } assi que non se puedan partir vno de otro . \\\hline
2.1.8 & suis uxoribus indiuisibiliter absque repudiatione , \textbf{ tanto magis hoc decet reges et principes , } quanto magis in eis relucere debet fidelitas , et ceterae bonitates . & e sin repoyamiento . \textbf{ Mas esto | tantomas parte nesçe alos Reyes e alos prinçipes } quanto mas deue en ellos reluzir la fialdat \\\hline
2.1.8 & in quo coniungitur vir et uxor , \textbf{ ratione ipsius prolis decet } uxorem viro , & en el qual se ay unta el marido e la muger . \textbf{ Conuiene por razon de los fiios } que la muger se ayunte al uaton \\\hline
2.1.8 & Patet ergo , \textbf{ quod decet omnes ciues , } non solum propter bonum fidei coniugalis , & sin departimiento ninguno . \textbf{ Et pues que assi es conuiene a todos los çibda danos } que non solamente \\\hline
2.1.8 & inseparabiliter conuiuere suis uxoribus . \textbf{ Tanto tamen hoc magis decet Reges et Principes , } quanto de prole suscepta & sin ningun departimiento . \textbf{ Enpero esto tanto pertenesçe | mas alos Reyes } e alos prinçipes \\\hline
2.1.8 & quam circa alias , \textbf{ tanto magis decet Reges , et Principes , } quam diu suae uxores vixerint , & que de los fiios de los otros tanto \textbf{ mas conuiene alos Reyes } e alos prinçipes mientre sus mugers biuieren ayuntar se a ellas sin ningun departimiento . \\\hline
2.1.9 & et ordinem naturalem , \textbf{ decet omnes ciues } una sola uxore esse contentos . & e segunt ordenna traal . \textbf{ Conuiene que todos los çibdadanos sean pagados } cada vno de vna sola mugier . \\\hline
2.1.10 & ut si quis subiicitur Proposito et Regi , \textbf{ oportet Propositum illum ad Regem ordinari , } et esse sub ipso repugnat & assi commo si algun çibdadano es subiecto al preuoste e al Rey . \textbf{ Conuiene que el | prinoste sea ordenado al Rey e sea so el . } Et por ende contradize ala orden natural \\\hline
2.1.10 & simul viris pluribus detestabilius esse debet . \textbf{ Decet ergo coniuges omnium ciuium uno viro esse contentas : } multo magis tamen hoc decet & que vna muger case en vno con mugons varones . \textbf{ ¶ Et pues que assi es conuiene | quelas mugers de todos los çibdadanos sean pagadas de vn uaron . } Enpero mucho mas conuiene esto alas mugers de los Reyes \\\hline
2.1.10 & Decet ergo coniuges omnium ciuium uno viro esse contentas : \textbf{ multo magis tamen hoc decet } coniuges Regum et Principum , & quelas mugers de todos los çibdadanos sean pagadas de vn uaron . \textbf{ Enpero mucho mas conuiene esto alas mugers de los Reyes } e delos prinçipes \\\hline
2.1.10 & Quare decet coniuges omnium ciuium uno uiro esse contentas : \textbf{ tanto tamen hoc magis decet } coniuges Regum et Principum , & por la qual cosa conuiene alas mugieres de todos los çibdadanos de ser pagadas de vn solo uaron . \textbf{ Empero tanto o mas conuiene esto } alas mugers de los Reyes \\\hline
2.1.10 & ad bonum prolis , \textbf{ decet coniuges omnium ciuium , } ne impediatur earum foecunditas , & ala generacion de los fijos \textbf{ conuiene que las mugers de todos los çibdadanos } por que non sea enbargado el \\\hline
2.1.10 & ne impediatur earum foecunditas , \textbf{ uno viro esse contentas . Tanto tamen hoc magis decet Regum , } et Principum coniuges , & por que non sea enbargado el \textbf{ sunconçebemiento sean paragadas de vn marido solo . | Enpero tanto mas conuiene esto } alas mugers de los Reyes e de los prinçipes \\\hline
2.1.10 & decet coniuges omnium ciuium uno viro esse contentas ; \textbf{ tanto tamen hoc magis decet } coniuges Regum , et Principum , & de ser pagadas de vn solo uaron . \textbf{ Et esto tanto conuiene mas alas mugers de los Reyes } e de los prinçipes \\\hline
2.1.11 & cum quibuscunque personis ; \textbf{ tanto tamen hoc magis decet Reges , et Principes , } quanto magis eos obseruare & con quales quier perssonas . \textbf{ Enpero tanto mas esto conuiene alos Reyes e alos prinçipes } quanto mas conuiene a ellos de guardar la orden natural \\\hline
2.1.11 & coniungat contractio copulae coniugalis . \textbf{ Decet ergo omnes ciues } non contrahere cum personis & e por amistança que los ayunte el sacͣmento del matermonio . \textbf{ Et pues que assi es conuiene a todos } losçibdadanos \\\hline
2.1.11 & nimia consanguineitate coniunctis : \textbf{ magis tamen hoc decet Reges , et Principes , } quia quanto sunt in maiori statu & entre perssonas muy ayuntadas por parentesço . \textbf{ Empero esto tanto mas parte nesçe alos Reyes e alos prinçipes . } Ca quanto son en mayor estado \\\hline
2.1.11 & et a ciuilibus operibus . \textbf{ Tanto hoc ergo magis decet Reges , et Principes , } quanto ipsi plus vigere debent prudentia et intellectu : & e delas obras çiuiles dando se mucho a obras lux̉iosas . \textbf{ pues que assi es tanto mas esto conuiene alos Re yes | e alos prinçipes } quanto ellos mas deuen ser conplidos de razon e de entendemiento \\\hline
2.1.12 & Nam si Reges , \textbf{ et Principes decet } accipere nobiles coniuges , & Ca si conuiene alos Reyes \textbf{ e alos prinçipes } de tomar mugieres nobles \\\hline
2.1.13 & filii utplurimum magni nascuntur . \textbf{ Decet omnes ciues propter bonum prolis , } ut filii polleant magnitudine corporali , & en la mayor parte nasçen grandes de cuerpos . \textbf{ Et pues que assi es conuiene a todos los çibdadanos } por el bien dela generaçion de los fijos \\\hline
2.1.13 & quaerere in suis uxoribus magnitudinem corporis : \textbf{ tanto tamen magis hoc decet Reges et Principes , } quanto ipsi circa proprios filios , & resplandezcan por grandeza de cuepo de demandar en las sus mugers grandeza de cuerpo . \textbf{ Enpero tanto mas esto conuiene alos Reyes | e alos prinçipes } quanto ellos deuen auer mayor cuydado de sus fijos propreos \\\hline
2.1.13 & sic et ex pulchris nascuntur pulchri . \textbf{ Quare si decet omnes ciues , } et maxime Reges et Principes solicitari , & Assi de los fermosos nasçen los fermosos . \textbf{ Por la qual cosa si conuiene a todos los çibdadanos } e mayormente alos Reyes \\\hline
2.1.13 & ut polleant filiis pulchris et magnis ; \textbf{ decet eos } in suis uxoribus quaerere magnitudinem , & e alos prinçipes de ser cuydadosos que resplandez cau \textbf{ por fiios grandes e fermosos . } Conuiene a ellos de demandar en las sus mugieres grandeza e fermosura corporal . \\\hline
2.1.13 & licet singulis virtutibus \textbf{ secundum modum eis congruum foeminas pollere deceat , } tamen cum tradenda est aliqua nuptui , & resplandesçer \textbf{ entondas las uirtudes | segunt la manera que les conuiene . } Enpero quando la fenbra es de dar a algun marido mayormente deuemos tener \\\hline
2.1.13 & eo quod ad intemperantiam foeminae maxime incitentur . \textbf{ Decet ergo omnes ciues } hoc in suis coniugibus quaerere : & e a cosas destenpradas . \textbf{ Et pues que assi es conuiene a todos los çibdadanos } de demandar esto en las sus mugers \\\hline
2.1.13 & hoc in suis coniugibus quaerere : \textbf{ tanto tamen hoc decet Reges et Principes , } quanto intemperantia coniugum ipsorum plus nocumenti inferre potest , & de demandar esto en las sus mugers \textbf{ Empero tanto conuiene esto mas alos Reyes | e alos prinçipes } quanto la destenprança delas mugers \\\hline
2.1.13 & quam intemperantia coniugum aliorum . \textbf{ Decet ergo coniuges temperatas esse . } Decet eas etiam amare operositatem : & dellos puede fazer mayor danno e enpeçemiento que la destenprança delas mugers de los otros . \textbf{ Et pues que al sy es conuiene | que las muger ssean tenpradas . } Et avn les conuiene aellas de amar fazer buenas obras . \\\hline
2.1.14 & Patet ergo , \textbf{ quomodo decet omnes ciues alio regimine praeesse uxoribus , } et alio filiis : & Et pues que assi es paresçe \textbf{ en qual manera conuiene a todos los çibdadanos | que enssennore en } por otro gouernamiento alos mugers \\\hline
2.1.14 & et alio filiis : \textbf{ tanto tamen hoc magis decet Reges et Principes , } quanto ipsi plus obseruare debent & e por otro alos fijos \textbf{ Enpero tanto mas esto pertenesçe alos Reyes | e alos prinçipes } quanto mas ellos deuen guardar aquellas cosas \\\hline
2.1.15 & et quicquid natura praeparatur , \textbf{ oportet ordinatissimum esse : } quia ille naturam dirigit , & por la natura \textbf{ conuiene que sea muy ordenado . } Ca aquel gnia la natura de que viene todo ordenamiento \\\hline
2.1.16 & requiritur aliquid , \textbf{ si illud sit imperfectum oportet } effectum imperfectum esse : & que quando para fazimiento de alguna nobra es menester alguna cosa \textbf{ si aquella cosa non es acabada | conuiene } que aquella obra non sea acabada . \\\hline
2.1.16 & sed etiam mente . \textbf{ Decet ergo omnes ciues } non uti coniugio in aetate nimis iuuenili ; & mas avn son menguados en el alma . \textbf{ ¶ Et pues que assi es conuiene a todos los çibdadanos } de non vsar de casamiento \\\hline
2.1.16 & non uti coniugio in aetate nimis iuuenili ; \textbf{ hoc tamen tanto magis decet Reges et Principes , } quanto ipsi plus debent esse soliciti , & Et esto \textbf{ tantomas conuiene alos Reyes | e alos prinçipes } quanto ellos deuen ser mas cuydadosos \\\hline
2.1.16 & magis vitanda sunt in Princibus et Regibus quam in aliis , \textbf{ potissime non decet } eos uti coniugio in nimia iuuentute . & e en los prinçipes \textbf{ que en los otros | Et por ende mayormente conuiene aellos } de non vsar de casamiento en grand moçedat . \\\hline
2.1.17 & quo sit melior procreatio filiorum : \textbf{ tanto tamen hoc magis decet Reges et Principes , } quanto decet eos elegantiores habere filios . & en elt pon que es meior la generaçion de los fijos . \textbf{ Enpero esto tanto mas conuiene alos Reyes | e alos prinçipes } quanto mas les conuiene aellos de auer los fijos grandes e esforcados de cuerpo \\\hline
2.1.19 & quo regimine regendae sunt coniuges : \textbf{ ideo oportet in speciali } dicere aliqua de regimine coniugum . & Suil si non dixieremos \textbf{ rtractaremos en espeçial } por qual gouernamiento se han de gouernar las mugers . Por ende deuemos dezir en espeçial algunas cosas del gouernamiento del casamiento . \\\hline
2.1.19 & Nam quicunque vult aliquid bene regere , \textbf{ oportet ipsum speciales habere cautelas ad ea , } circa quae videt ipsum magis deficere . & conuiene \textbf{ que el aya algunas cautelas espeçiales | para aquellas cosas } en las quales vee \\\hline
2.1.19 & in loquela dirigere , \textbf{ oporteret eos instruere , } ut specialem pugnam & que qualiesse enderesçar los tartamudos en la fabla conuenir le ha \textbf{ que los ensseñasse que tomassen espeçial batalla } e espeçial esfuerço cerca aquellas palauras \\\hline
2.1.19 & in haereditatem patris . \textbf{ Decet ergo coniuges } omnium ciuium esse castas : & que non seria su padre . \textbf{ Et pues que assi es conuiene a todas las mugers } de todos los çibdadanos de ser castas \\\hline
2.1.19 & omnium ciuium esse castas : \textbf{ et tanto magis hoc decet } coniuges Regum et Principum , & de todos los çibdadanos de ser castas \textbf{ e tanto mas esto conuiene alas mugers de los Reyes } e de los prinçipes \\\hline
2.1.19 & vel dissensio oriri . \textbf{ Secundo decet eas esse pudicas } et honestas . & e mayor discordia \textbf{ que lo segundo couiene a el de los otros . } las de ser linpias e honestas \\\hline
2.1.19 & sic suas coniuges regere : \textbf{ et tanto magis hoc decet Reges , et Principes , } quanto ex eorum indebito regimine potest & por la qual cosa conuiene a todos los çibdadanos de gouernar a sus mugers assi . \textbf{ Et esto tanto mas conuiene alos Reyes | e alos prinçipes } quanto por el gouernamiento desconuenible \\\hline
2.1.20 & uti moderate coniugali copula , \textbf{ et tanto magis hoc decet Reges , et Principes , } quanto indecentius est eos propter huiusmodi actus habere corpus debilitatum , & tenpradamente e mesuradamente del ayuntamiento matermoinal . \textbf{ Et tanto mas conuiene esto alos Reyes | e alos prinçipes } quanto mas desconuenible es aellos \\\hline
2.1.20 & Quemlibet enim virum \textbf{ secundum possibilem facultatem decet } suam uxorem honorifice retinere & si quier uaron de tener su muger \textbf{ honrradamente | segunt su poder } e sus riquezas \\\hline
2.1.20 & sed quasi sociam , \textbf{ decet quamlibet } secundum suum statum & assi commo sierua mas assi commo conpanera . \textbf{ Por ende conuiene a cada vn marido } segunt su estado de tractar muy honrradamente a su muger . \\\hline
2.1.20 & uxorem propriam honorifice pertractare . \textbf{ Ostenso , quomodo decet viros suis uxoribus moderate et discrete } uti , & segunt su estado de tractar muy honrradamente a su muger . \textbf{ ¶ Mostrado en qual manera conuiene alos maridos usen de sus mugers } sabiamente e tenprada mente . \\\hline
2.1.20 & fatuis vero est asperior increpatio adhibenda . \textbf{ Decet ergo quoslibet viros , } considerato proprio statu , & que ayan algun miedo ¶ \textbf{ Et pues que assi es conuiene a cada vno de los uarones } penssando el su estado propreo \\\hline
2.1.21 & per debitas monitiones instruere . \textbf{ Decet Reges , et Principes , } et uniuersaliter omnes ciues scire , & por conuenibłs castigos . \textbf{ onuiene alos Reyes e alos prinçipes } e generalmente a todos los çibdadanos saber en qual manera \\\hline
2.1.21 & circa ornatum corporis ; \textbf{ quare decet viros cognoscere } quis mulierum ornatus sit licitus , & mayormente pecan en el conponimiento de los cuerpos . \textbf{ Por la qual cosa conuiene alos uarones } de saber qual conponimiento es conuenible alas mugieres \\\hline
2.1.21 & Superabundantia vero \textbf{ ( ut videtur ) oportet } ibi triplicem virtutem concurrere , & Lo segundo por desfallesçimi ento . \textbf{ Mas alli commo parelçe en el sobrepiuamiento } conuiene de ser tres uirtudes \\\hline
2.1.21 & Adhuc etiam uxorem Principis , \textbf{ vel etiam Regis decet } magis ornatam esse . & que ala muger del çibdadano sinple . \textbf{ Et avn ala mugni del Rey o del prinçipe } conuiene de ser mas honrrada \\\hline
2.1.22 & et a ciuilibus operibus . \textbf{ Decet ergo omnes ciues } non esse nimis zelotipos de suis coniugibus : & e avn de ser enbargados enlas obras çiuiles ¶ \textbf{ pues que assi es conuiene a todos los çibdadanos } de non ser muy çelosos de sus mugieres \\\hline
2.1.22 & non esse nimis zelotipos de suis coniugibus : \textbf{ et tanto magis hoc decet Reges et Principes , } quanto maius praeiudicium potest & de non ser muy çelosos de sus mugieres \textbf{ Et tanto mas esto conuiene alos Reyes | e alos prinçipes } quanto mayor mal e mayor preiizio se puede leunatar al regno \\\hline
2.1.22 & esse nimis zelotypos . \textbf{ Nec etiam decet eos } circa suas coniuges nullam habere custodiam & ¶ Et pues que assi es non conuiene alos maridos ser muy çelosos de sus mugrs \textbf{ nin avn les conuiene } de non poner alguna guarda en sus mugers \\\hline
2.1.23 & Quia ergo sic est , \textbf{ oportet foeminas deficere a ratione , } et habere consilium inualidum . & que las mugers \textbf{ que fallezcan de vso de razon e que ayan el conseio flaço . } Ca quando el cuerpo es meior conplissionado tanto \\\hline
2.1.23 & in quo est suprema prudentia ; \textbf{ oportet quod agat ordinate et prudenter . } Prudentis est enim cito se expedire , & que faze la natura \textbf{ que las faga ordenadamente | e con sabiduria } Ca al sabio pertenesçe \\\hline
2.2.1 & et ea regulent et conseruent : \textbf{ decet quemlibet dominantem , } et praeeminentem solicitari , & por que los reglen e guarden en su ser . \textbf{ Conuiene a cada vn sennor } que ha sennorio sobre otro de ser muy cuydadoso en qual manera \\\hline
2.2.1 & ut probatur 8 Ethicorum , \textbf{ decet patres ex ipso amore naturali , } quem habent ad filios , & assi commo se praeua en el viij delas . ethicas . \textbf{ Conuiene que los padres por amor natural } que han alos fijos sean cuy dadosos della \\\hline
2.2.2 & si debeant naturaliter dominari , \textbf{ oportet quod polleant prudentia et intellectu : } tanto decet Reges et Principes & e generalmente todos los señores sy de una naturalmente ensseñorear \textbf{ conuiene les que ayan sabidia e entendimiento . } Et tanto mas conuiene alos Reyes \\\hline
2.2.2 & oportet quod polleant prudentia et intellectu : \textbf{ tanto decet Reges et Principes } magis solicitari & conuiene les que ayan sabidia e entendimiento . \textbf{ Et tanto mas conuiene alos Reyes | e alos prinçipes } de auer mayor cuydado de sus fijos \\\hline
2.2.2 & sumitur ex bonitate filiorum . \textbf{ Decet enim filios Regum et Principum } maiori bonitate pollere quam alios : & La segunda razon para propuar esto mismo se toma dela bondat de los fijos . \textbf{ Ca conuiene alos fijos de los Reyes | e de los prinçipes } de auer mayor bondat \\\hline
2.2.2 & quam circa ipsos se habeant negligenter , \textbf{ tanto decet Reges et Principes } magis solicitari & que si se ouieren negligenter te çerca dellos tanto \textbf{ mas conuiene alo Reyes } e alos prinçipes de ser acuçiosos a sus fijos \\\hline
2.2.2 & eo quod principantis sit alios regere et gubernare : \textbf{ tanto ergo magis decet Reges et Principes solicitari } circa proprios filios , & por que alos prinçipes parte nesçe de gouernar e de garalo sots . \textbf{ pues que assi es tanto mas conuiene alos Reyes } e alos prinçipes de ser acuçiosos de sus fijos \\\hline
2.2.5 & ut ab infantia instruantur in hac fide ; \textbf{ tanto tamen hoc magis decet Reges , et Principes , } quanto ex feruore fidei ipsorum potest & por que enla moçedat sean enssennados en esta fe \textbf{ Empero tanto mas esto conuiene alos reyes | e alos prinçipes } quanto por el feruor \\\hline
2.2.6 & Nam cum aliquis est pronus ad aliquid , \textbf{ oportet ipsum multum assuescere in contrarium , } ne inclinetur ad illud : & Ca quando alguno es inclinado a alguacosa . \textbf{ Conuiene que el vse mucho en el contrario } por que non sea inclinado a aquella cosa . \\\hline
2.2.6 & a lasciuiis retrahantur . \textbf{ Decet ergo omnes ciues solicitari erga filios , } ut ab ipsa infantia instruentur ad bonos mores . & e por bueons castigos sean tirados delas loçanias . \textbf{ Et pues que assi es | conuieneque todos los çibdadanos ayan grand cuydado de sus fijos } assi que luego en su moçedat \\\hline
2.2.6 & ut ab ipsa infantia instruentur ad bonos mores . \textbf{ Tanto tamen hoc magis decet Reges et Principes , } quanto bonitas filiorum est utilior ipsi regno , & sean acuçiosos en en poter los en bueans costunbres . \textbf{ Enpero esto tanto mas conuiene alos Reyes | e alos prinçipes } quanto la bondat de sus fijos es mas prouechosa al regno . \\\hline
2.2.7 & quae in literis traduntur . \textbf{ Tertio decet ipsos , } secundum modum sibi possibilem , & que son dadas a muchos ¶ \textbf{ Lo terçero les conuiene } quanto ellos pudieren de venir ala perfecçion \\\hline
2.2.7 & et quasi ab ipsis cunabulis inchoare . \textbf{ Decet ergo omnes nobiles , } et maxime Reges , et Principes , & assi commo quando nos tirassen delas amas . \textbf{ Et pues que assi es conuiene a todos los nobles } e mayormente alos Reyes \\\hline
2.2.7 & et magis viget prudentia et intellectu : \textbf{ tanto magis decet filios Regum , } et Principum & e por entendimiento tanto \textbf{ mas conuiene alos fijos de los Reyes } luego en su moçedat de trabaiar se en las letris \\\hline
2.2.8 & per quas ostendi posset , \textbf{ quod filios nobilium decet } addiscere musicam . & por las quales se podrie mostrar \textbf{ que conuienea los fijos de los nobles } de aprender la musica \\\hline
2.2.8 & Huiusmodi autem sunt scientiae morales . \textbf{ Alias ergo scientias in tantum decet } eos scire , & Et estas sçiençias son las sçiençias i trorales . \textbf{ Et pues que assi es en tanto les conuiene aellos de sablas o tris sçiençias } en quanto siruen ala ph̃ia moral . \\\hline
2.2.8 & Imo si nunquam grammatica deseruiret negocio morali , \textbf{ decet Reges , } et Principes scire idioma literale , & ante digo que si nunca fuessen las otras sçiençias la guamatica siruiria ala sçiençia moral . \textbf{ Et por ende conuiene alos Reyes } e a los prinçipes \\\hline
2.2.9 & quam doctor : \textbf{ decet igitur ipsum esse inuentiuum . } Secundo decet ipsum esse intelligentem et perspicacem . & este tal mas es rezador que doctor ¶ \textbf{ Et pues que assi es conuiene al maestro | que non tan solamente sea fallador delas cosas } mas que sea entendido e sotil . Ca assi commo ninguon non puede abastar \\\hline
2.2.9 & ad hoc quod habeamus sufficientiam in vita , \textbf{ oportet nos in societate viuere , } ita ut per auxilia aliorum & e conplidamente mas para que aya conplimiento en la uida \textbf{ conuienele de beuir en conpannia } assi que por las ayudas de los otros sea acorrido enla mengua dela uida . \\\hline
2.2.9 & et intelligens aliorum dicta . \textbf{ Tertio oportet ipsum esse iudicatiuum : } nam perfectio scientiae potissime & e que entienda los dichͣs de los otros . \textbf{ ¶ Lo terçero conuiene que sea iudgador } e que aya razon para iudgar . \\\hline
2.2.9 & Decet igitur aliorum directorem memorem esse praeteritorum . \textbf{ Secundo decet } ipsum esse prouidum futurorum . & que sea acordado delas cosas passadas \textbf{ ¶ | Lo segundo le conuiene } que sea prouiso en las cosas \\\hline
2.2.9 & Quantum vero ad prudentiam agibilium , \textbf{ decet ipsum esse memorem , } prouidum , cautum , et circumspectum . & que son de fazer \textbf{ conuiene le que el doctor sea menbrado e prouado e sabio e acatado . } Mas quanto ala uida deue ser honesto e bueno . \\\hline
2.2.10 & quod omnia prima amamus magis : \textbf{ propter quod oportet } ab ipsis iuuenibus extranea & que amamos todas las cosas primeras \textbf{ mas que conuiene Et desto nos viene } de fazer cosas estrannas \\\hline
2.2.10 & et ad passiones insequendas , \textbf{ non oportet ipsam per visionem turpium } ad ulteriorem prouocare . & e aseguir sus passiones . \textbf{ Et por ende non conuiene en aquella hedat } por uision de cosas torpes \\\hline
2.2.11 & Si enim cibus digeri debeat , \textbf{ oportet } ipsum esse proportionatum calori naturali . & Ca si la vianda se ouiere bien a cozer \textbf{ conuiene que sea bien proporçionada ala calentura natural } Por la qual cosa si en tan grand quantia se \\\hline
2.2.12 & sed etiam potus . \textbf{ Decet ergo iuuenes } non solum esse abstinentes , & Cerca el beuer . \textbf{ Et çerca el vso dela luxia . } Ca non solamente la uianda tomada \\\hline
2.2.12 & quia facilius adhaeremus iis , ad quae ab infantia assueti sumus , \textbf{ decet omnes patres et maxime reges et Principes solicitari } circa regimen filiorum , & aque somos acostunbrados en la moçedat . \textbf{ Conuiene a todos los padres | e mayormente alos Reyes } e alos prinçipes de auer grand cuydado çerca el \\\hline
2.2.14 & est virtus organica siue corporalis . \textbf{ Quare oportet talem appetitum sumere modum , } et mensuram ex ipso corpore . & mas el appetito de los sesos es uirtud organica o corporal . \textbf{ Por la qual cosa conuiene } que tal desseo tome manera e mesura del cuerpo . \\\hline
2.2.18 & et agibilia cognoscere . \textbf{ Decet ergo eos , } qui debent alios regere , & por que puedan conosçer las costunbres de los omes et las obras dellos . \textbf{ ¶ Et pues que assi es conuiene aquellos } que deuen gouernar los otros de escusar la ꝑeza \\\hline
2.2.19 & qualis cura gerenda sit circa filias . \textbf{ Nam sicut decet coniuges esse continentes , } pudicas , abstinentes , et sobrias : & çerca delas fijas \textbf{ ca assi commo conuiene alas madres } de ser continentes e castas e guardadas e mesuradas en essa misma manera conuiene alas fijas de ser tales \\\hline
2.2.20 & Possumus autem triplici via venari , \textbf{ quod decet omnes ciues , } et multo magis nobiles , & por tres razones \textbf{ que conuiene a todos los çibdadanos } e mayormente alos Reyes \\\hline
2.2.20 & Qualia autem debent esse opera , \textbf{ circa quae mulieres insudare decet , } in prosequendo patebit . & mas quales deuen ser las obras \textbf{ cerca las quales conuiene alas mugers de } trabaiatadelante paresçra \\\hline
2.2.20 & infra declarandum esse , \textbf{ circa quae opera deceat foeminas esse intentas . } Ostenso , & casamien toca y dixiemos que adelante serie de declarar cerca quales obras conuenia \textbf{ que las mugers fuesen acuçiosas . } ostrado que non conuiene alas moças de andar uagarosas a quande e allende \\\hline
2.2.21 & quod non decet puellas esse vagabundas , \textbf{ nec decet eas viuere otiose : } restat ut nunc tertio ostendamus , & ostrado que non conuiene alas moças de andar uagarosas a quande e allende \textbf{ nin les conuiene de beuir ociosas } finca que agora lo terçero mostremos \\\hline
2.2.21 & cautos proferre sermones , \textbf{ decet eas non esse loquaces : } sed oportet ipsas esse debite taciturnitas , & tomadesto \textbf{ que las mugrͣ̃s non sean prestas avaraias e apeleas } ca commo las muger se mayormente las mocas \\\hline
2.2.21 & decet eas non esse loquaces : \textbf{ sed oportet ipsas esse debite taciturnitas , } ut possint omnem sermonem prolatum diligenter excutere . & que las mugrͣ̃s non sean prestas avaraias e apeleas \textbf{ ca commo las muger se mayormente las mocas } fallezcan de vso de razon \\\hline
2.3.1 & Quod autem omnia haec considerare \textbf{ deceat prudentem patremfamilias , } vel doctum gubernatorem familiae : & Mas que conuenga de cuydar todas estas cosas \textbf{ al sabio padre familias | que es padre dela conpanna } o al sabio gouernador dela conpanna . \\\hline
2.3.2 & quomodo ad inuicem comparantur . \textbf{ Oportet enim } ( secundum Philosophum ) & en qual manera son conparados los vnos alos otros . \textbf{ Ca conuiene segunt el pho } que estos tales instrumentos sean ordenados los vnos alos otros \\\hline
2.3.2 & et mensuram ex illis : \textbf{ oportet organa domus ordinata esse , } et organa inferiora , & Et resçiben dellas manera de seruiçio e mesura . \textbf{ Conuiene avn que los estrumentos dela casa sean ordenados } e que los instrumentos mas baxos \\\hline
2.3.3 & nam secundum Philosophum 4 Ethicorum capitulo de Magnificentia , \textbf{ maxime gloriosos et nobiles decet esse magnificos : } Reges ergo et Principes , & en el quarto libro delas ethicas \textbf{ enł capitulo dela magnifiçençia | que much mas conuiene alos Reyes } e alos prinçipes \\\hline
2.3.3 & ubi ait , \textbf{ quod Principes decet } sic magnifica facere , & do dize que alos Reyes \textbf{ e alos prinçipes parte nesçe } de fazer tan grandes cosas \\\hline
2.3.3 & ad ostentationem et inanem gloriam : \textbf{ decet tamen Reges et Principes , } ne in contemptum habeantur a populo , & nin a grandezaauana eglesia \textbf{ enpero conuiene alos Reyes } e alos prinçipes de fazer moradas costosas e nobles \\\hline
2.3.3 & In domibus ergo Regum et Principum \textbf{ oportet multos abundare ministros , } ut ergo non solum personas Regis et Principis , & e de los prinçipes conuiene \textbf{ que ayan muchos ofiçiales | e much ssiruient s̃ Et pues que assi es } por que non solamente la persona del Rey o del prinçipe mas avn \\\hline
2.3.3 & in aedificiis constructis , \textbf{ oportet ipsa esse magnifica . } Viso , qualia debent esse aedificia , & que ellos fazen a \textbf{ conuiene | que ellos sean muy grandes e muy costosas } ¶ Visto quales deuen ser las moradas \\\hline
2.3.5 & ut in prima parte huius secundi libri diffusius probabatur : \textbf{ oportet aliquomodo naturalia esse } quae sunt necessaria in vita politica ; & mas conplidamente en la primera parte deste segundo libro . \textbf{ conuiene en algunan manera } que las cosas naturales sean neçessarias enla uida politica . \\\hline
2.3.6 & quod iubetur ; \textbf{ propter quod oportet rem illam } vel non produci ad effectum , & elperando que el otro cunplira aquello que a el es mandado . \textbf{ Por la qual cosa conuiene } que aquella cosa non sea aduchͣobra \\\hline
2.3.7 & non de usurpatione alieni . \textbf{ Quia igitur decet ipsos ciues , } et multo magis Reges , & e non deue tomar lo ageno¶ Et \textbf{ pues que assi es por que conuiene alos çibdadanos } e mucho mas alos Reyes \\\hline
2.3.8 & quaerere infinitas possessiones . \textbf{ Decet igitur omnes ciues } et multo magis Reges et Principes & Et pues que assi es nin el arte del gouernamiento dela casa non deue demandar possessiones et riquezas sin mesura e sin fin . \textbf{ ¶ Et por ende conuiene a todos los çibdadanos } e mayormente alos Reyes \\\hline
2.3.9 & Ut ergo sciamus quomodo huiusmodi commutationes \textbf{ oportuit introduci , } sciendum quod si non esset & Et pues que assi es por que sepamos en qual manera conuiene \textbf{ que estas tales muda connes fuessen puestas en la } tiecra deuedes saber \\\hline
2.3.9 & in toto regno uno , \textbf{ propter quod oportet } discurrere aliquos & non son falladas en todo vn regno \textbf{ por la qual cosa conuiene a algunos omes } que corran et vayan \\\hline
2.3.9 & quibus non abundant frigidae et econuerso . \textbf{ Propter quod non solum oportet communicare } et conuersari & Et por el cotrario en alguas abonda friura \textbf{ que non abonda ca lentura } Por la qual cosa non solamente conuiene alos omes morar e conuerssar los vnos con los otros \\\hline
2.3.9 & ne nimis grauarentur homines commorantes in ipso , \textbf{ cum ex una parte regni oportebat } eos accedere ad aliam , & Et pues que assi es por que non fuessen muy agua uiados los omes \textbf{ que moran en vn regno | e que estan en vn logar del regno } commo contesçe alas vezes \\\hline
2.3.9 & et rerum ad numismata , \textbf{ oportuit inuenire commutationem numismatum ad numismata . } Patet ergo quot sunt commutationes , & e delas cosas alos \textbf{ diueros otra mudaçiones | que es de monedas alas monedas . } Et pues que assi es paresçe \\\hline
2.3.9 & et quae fuit necessitas inuenire denarios . \textbf{ Decet ergo prudentem patremfamilias , } et doctum gubernatorem cognoscere , & para fallar los des \textbf{ e por ende conuiene al sabio padre familias } e al sabio gouernador dela casa saber \\\hline
2.3.10 & Nam primam speciem pecuniatiuae , \textbf{ quae est oeconomica et quasi naturalis , decet . } Decet enim ipsos abundare & saluo dela primera manera pecumatiua \textbf{ que es i conomica | e assi commo natural } ca conuiene alos Rey \\\hline
2.3.11 & contra naturam est . \textbf{ Decet ergo Reges et Principes , } si volunt naturaliter Dominari , & que non pertenesçe al uendedores contra natura . \textbf{ Et pues que assi es conuiene alos Reyes e a los prinçipeᷤ } si quisieren ser señors natalmente \\\hline
2.3.12 & cum stipendiarii fiunt , pecuniam intendunt . \textbf{ Decet ergo quemlibet } secundum vitam politicam volentem prouidere indigentiae domesticae , & todos entienden admeros . \textbf{ Et pues que assi es conuiene a cada vno } que quiere proueer ala mengua dela casa \\\hline
2.3.12 & videlicet possessoria , et experimentalis . \textbf{ Decet enim eos , } vel per se , & Et la espirimental que es delas praeuas . \textbf{ por que conuiene a ellos } que por si o por otros ayan prouada \\\hline
2.3.12 & quam etiam in volucribus . \textbf{ Hoc enim decet Reges , } et Principes , & commo en las aues . \textbf{ Et esto conuiene alos Reyes e alos prinçipes } por que den exienplo alos otros que son en el regno en \\\hline
2.3.12 & quam ut ciuis . \textbf{ Quare decet Reges , } et Principes habere homines industres & por su dinero paresçe que biue mas commo auenedigo e peleger no que non commo çibdadano . \textbf{ Por la qual cosa conuiene alos Reyes e alos principes } de auer omes acuçiosos \\\hline
2.3.13 & ad constitutionem eiusdem corporis mixti , \textbf{ oportet aliquod elementum praedominans , } secundum quod illi mixto competat & conuiene de dar \textbf{ ay algun helemento | que ssennoree sobre los otros } segunt el qual conuiene aquel cuerpo mesclado mouimiento con ueinble o asentamiento con ueible \\\hline
2.3.14 & Prima congruitas sic patet : \textbf{ oportet enim dominans } ( ut dicitur in Politic’ ) & la primera razon paresçe assi . \textbf{ Ca conuiene que el sennor | segunt } que dize el philosofo \\\hline
2.3.14 & ut superantes in bello dominentur superatis . \textbf{ Propter quod decet virtuosos et sapientes et bonos } non resistere ordinationi legali : & enssenuoreassen e fuessen sennors de los vençidos \textbf{ Por la qual cosa conuiene | que los uertuosos e los sabios } e los bueons \\\hline
2.3.15 & ne ergo tales omnino priuentur ministris , \textbf{ ad supplendum indigentiam domesticam oportuit } esse aliquos ministros conductos seruientes & Et pues que assi es por que tales del todo non sean priuados de seruientes \textbf{ conuiene para cunplimiento dela mengua dela casa } que ouiessen algunos seruientes alquilados \\\hline
2.3.15 & quos virtus et amor boni inclinat ad seruiendum , \textbf{ decet principantes se habere quasi ad filios , } et decet eos regere non regimine seruili , & e el amor de bien los inclina asuir . \textbf{ Conuiene que los prinçipes se ayan çerca ellos | assi commo cerca de fijos . } Et conuiene les alos prinçipes delos gouernar non \\\hline
2.3.16 & si debet esse ordinata , \textbf{ oportet reduci in unum aliquem , } a quo ordinetur . & En essa misma manera cada vna muchedunbre si bien ordenada es \textbf{ conuiene que sea aduchͣa vn ordenador } de quien ella sea ordenada . \\\hline
2.3.16 & in gubernatione domorum regalium , \textbf{ ubi propter magnitudinem officiorum oportet } idem ministerium committi ministris multis , & en el gouernamiento delas casas de los Reyes \textbf{ En las quales por la grandeza de los offiçios conuiene de } acomne dar vn ofiçio a muchos seruientes \\\hline
2.3.17 & consideranda est conditio personarum . \textbf{ Nam non omnes decet } habere aequalia indumenta . & Lo terçero çerca la prouision delas uestidas es de penssar la condiçion delas personas \textbf{ por que non conuiene que todos sean uestidos } de eguales uestiduras caenta \\\hline
2.3.18 & nisi domus nobilium et magnorum : \textbf{ et quia decet nobiles } et magnos esse nobiles secundum mores , & si non casa de grandes e de nobles \textbf{ e por que conuiene } que los nobles e los guaades sean nobles en costunbres \\\hline
2.3.18 & si vero id agat \textbf{ quia hoc decet mores curiae et mores nobilium , } curialis est . & mas si esto faze \textbf{ por que esto conuiene alas costunbres dela corte | e alas costunbres delos } noblsomes este es dichcurial . \\\hline
2.3.18 & quod decet ministros Regum et Principum curiales esse . \textbf{ Nam si decet Reges et Principes } eo quod sunt in maximo nobilitatis gradu , & e cortesesca \textbf{ assi conuienea los Reyes | e alos prinçipes } por que son en muy grand grado de nobleza \\\hline
2.3.19 & esse iustos legales , \textbf{ sic decet ministros dominorum } ut seruent decentiam curiae & s̃ de los sennores de ser curiales \textbf{ por que guarden el estado e la honrra dela corte } conueiblemente \\\hline
2.3.19 & et solicitudinem de ministris , \textbf{ non decet magnos dominos , } nec Reges et Principes . & que les son acomnedados \textbf{ ca non conuiene a grandes sennores } nin a Reyes \\\hline
2.3.19 & aut philosophantur . \textbf{ Reges ergo et Principes quos decet } esse & e darse a grandes cosas o a sabiduria . \textbf{ ¶ Et pues que assi es alos Reyes } e alos prinçipes alos quales couiene de auer altos coraçones \\\hline
2.3.19 & magnanimos decet operari pauca et magna , \textbf{ ut decet ipsos solicitari } circa ea quae directe spectant & conuiene les de obrar pocas cosas \textbf{ e grandes ca les conuiene } de ser acuçiosos \\\hline
2.3.19 & et velle se de quibuscumque inimicis intrommittere , \textbf{ nullatenus decet ipsos . Hoc viso restat } ostendere tertium , & de quales quier ofiçiales \textbf{ nin de sus ofiçios | ca esto ꝑtenesçe alos menores . } ¶ Esto iusto finça de demostrar lo terçero \\\hline
2.3.19 & 4 Ethi’ ubi vult , \textbf{ quod ad humiles decet } magnanimos se habere moderate , & la qual cosa muestra el philosofo en alguna manera \textbf{ en el quarto libro delas ethicas do dize que conuiene alos magnanimos } e de altos coraçones de se auer \\\hline
2.3.19 & magnanimos se habere moderate , \textbf{ sed ad eos qui sunt in dignitatibus decet } magnanimos ostendere se magnos . & tenpradamente alos hunul lodos \textbf{ mas a aquellos que son en grandes dignidades } los magnanimos se deuen mostrar grandes . \\\hline
2.3.19 & apparere seuera , sed reuerenda . \textbf{ Non ergo decet Principem } tam familiarem se exhibere ministris , & non deue paresçer cruel mas honrrada e mesurada . \textbf{ Et pues que assi es non conuiene al prinçipe } de se fazer tan familiar alos sus siruientes \\\hline
2.3.19 & quae non esset laudabilis in Rege : \textbf{ omnino enim decet Reges et Principes } minus se exhibere quam caeteros , & e en el prinçipe \textbf{ por que en toda manera conuiene alos Reyes | e alos prinçipes } de se faz menos familiares \\\hline
2.3.20 & ut recumbentes in multiloquia non prorumpant \textbf{ decet } etiam hoc uniuersaliter nobiles et omnes ciues , & non se estiendan a fablar mucho . \textbf{ Et avn esto mismo conuiene generalmente a todos los çibdadanos } por que avn cosa conuenible es aellos de auer uirtudes e bueans costunbres . \\\hline
2.3.20 & et ne intemperati appareant , \textbf{ decet in mensis vitare sermonum multitudinem , } decet etiam hoc ipsos ministrantes , & e por que non parezcan destenprados \textbf{ assi commo dicho es . } Avn esto mismo conuiene alos seruientes por que la orden e la manera del seruir \\\hline
3.1.1 & gratia alicuius boni , \textbf{ oportet ciuitatem ipsam constitutam esse propter aliquod bonum . } Probat autem Philosophus primo Polit’ duplici via , & commo toda comunidat sea por graçia de algun bien . \textbf{ Conuiene que la çibdat sea establesçida por algun bien | Ca pruena el pho } enl primero libro delas politicas \\\hline
3.1.1 & cum ciuitas sit opus humanum , \textbf{ ex parte hominum constituentium ciuitatem oportet } ipsam constitutam esse gratia eius & sea obra de los omes \textbf{ de parte de los omes | que establescen la çibdat . } Conuiene que la çibdat sea establesçida \\\hline
3.1.4 & ut natura non deficiat in necessariis , \textbf{ oportet quid naturale esse } quicquid secundum se deseruit & conuiene \textbf{ que sea cosa natural todo aquello } que sirue a conplimiento de uida \\\hline
3.1.4 & quam communitates illae , \textbf{ oportet eam esse secundum naturam . } Secunda uia ad inuestigandum hoc idem , & que estas dos comuidades \textbf{ por ende conuiene | que la çibdat lea comuidat natraal ¶ } La segunda razon para prouar \\\hline
3.1.4 & et ad fugiendum iniustum , \textbf{ oportet communitatem domesticam } et ciuilem esse quid naturale . & delo que non es iusto \textbf{ conuiene | que la comunidat dela casa } e la comunidat dela çibdat sean cosas naturales \\\hline
3.1.4 & sit secundum naturam , \textbf{ oportet ciuitatem } esse quid naturale , & mas commo aquello a que auemos inclinaçion natural sea cosa natural \textbf{ conuiene quela çibdat sea cosa natural } e sea alguna cosa segunt natura \\\hline
3.1.5 & Non sic intelligendum est , \textbf{ quod semper oporteat ciuitatem } ex propriis possessionibus habere omnia quae requiruntur ad vitam : & que ꝑtenesçen ala uida esto non es assi de entender que sienpre conuenga \textbf{ que la çibdat abonde de sus propas possessiones | para aquellas cosas } que son menester ala uida \\\hline
3.1.5 & in quo contingit ciuitates alias abundare ; \textbf{ oportet ciuitatem } unam indigere auxilio alterius . & entgo en el qual otras çibdades abondan \textbf{ por ende conuiene } que la vna çibdat aya acerto del otra \\\hline
3.1.5 & excellunt exteriora bona . \textbf{ Oportet ergo rectores ciuitatis } habere ciuilem potentiam , & Pues que assi es conuiene \textbf{ que los gouernadores de la çibdat ayan poderio çiuil } por que puedan costrennir e fazer iustiçia \\\hline
3.1.6 & in inueniendo artem aliquam , \textbf{ sed oportet ad hoc iuuari } per auxilium praecedentium & niguno non abasta assi mismo en fallar algunan arte \textbf{ mas conuiene que sea ayuda de } por ayuda de los que passaronante \\\hline
3.1.8 & quasi sex rationes , \textbf{ probantes quod non oportet ciuitatem } esse maxime unitam , & que prue una \textbf{ que non conuiene ala çibdat } de ser muy vna \\\hline
3.1.8 & Quia ergo diuersis indigemus ad vitam , \textbf{ oportet in ciuitate diuersitatem esse . } Tertia via declarans & por que nos auemos me estermuchͣs cosas departidas para abastamiento dela uida \textbf{ conuiene que enla çibdat sea algun departimiento . } La tercera razon que declara e manifiesta las razones \\\hline
3.1.8 & si sint omnes voces aequales : \textbf{ sed ad rectam consonantiam oportet } ibi dare diuersitatem tonorum . & si non fueren y todas las bozes eguales \textbf{ mas ala derecha consonançia delas bozes } conuiene de dar y departimiento de los tonos \\\hline
3.1.8 & nisi sit ibi diuersitas officiorum . \textbf{ Decet ergo hoc Reges , et Principes cognoscere , } quod nunquam quis bene nouit regere ciuitatem , & e de los ofiçiales . \textbf{ Et pues que assi es conuiene alos Reyes | e alos prinçipes de sabesto } por que munca ninguno sopo bien gouernar çibdat \\\hline
3.1.9 & nutriat corpus alterius . \textbf{ Existentibus ergo possessionibus communibus oporteret } cuilibet distribui & dano realmente crie el cuerpo del otro . \textbf{ Et pues que assi es estando las possessiones comunes conuernia a cada vno } de departir aquellas cosas \\\hline
3.1.9 & nam si omnia sic essent communia , \textbf{ non oporteret ciues } omnes pueros reputare filios proprios . & assi ca si todas las cosas fuessen \textbf{ assi comunes non conuernia } que los çibdadanos cuydassen \\\hline
3.1.9 & non est expediens ciuitati , \textbf{ Decet autem hoc Reges , } et Principes diligenter aduertere , & que ponian platon e socrates non era conueinble ala çibdat . \textbf{ Mas conuiene alos Reyes } e alos \\\hline
3.1.10 & Primum sic patet . \textbf{ Nam oportet in ciuitate } consurgere lites , vulnerationes , et contumelias , & El quinto es abusion de los parientes ¶ Lo primero paresçe assi ca conuiene \textbf{ que en la çibdat se leunatenlides e feridas e deniestos } las quales cosas tanto son \\\hline
3.1.10 & de communitate uxorum et filiorum . \textbf{ Decet ergo Reges et Principes } sic ordinare ciuitatem , & que puso delas mugeres e de los fijos . \textbf{ Et pues que assi es conuiene alos Reyes } e alos prinçipes de ordenar assi la çibdat \\\hline
3.1.11 & Sed ut dicitur secundo Politicorum \textbf{ in actibus particularibus oportet } ad experientiam recurrere : experti enim sumus & en el segundo libro delas politicas \textbf{ en los fechos particulares } conuiene de venir ala prueua \\\hline
3.1.11 & si haberent haereditatem communem , \textbf{ et si oporteret eos valde ad inuicem conuersari . } Tertia via sumitur , & los quales non quarrien ser subiectos el vno al otrosi ouiessen la heredat en comun \textbf{ e ouiessen de beuir plongadamente en vno . } ¶ La terçera razon se toma \\\hline
3.1.12 & secundum tria quae requiruntur ad bellum . \textbf{ Homines enim bellatores decet } esse mente cautos et prouidos : & que son meester para la batalla \textbf{ ca los omes lidiadores conuiene que sean cuerdos } por entendimiento e sabios \\\hline
3.1.12 & ex parte fortitudinis corporalis . \textbf{ Nam cum bellantes oporteat } diu & que son temerosos ¶ \textbf{ La terçera razon se toma de parte dela fortaleza corporal } ca commo los lidiadores ayan de sofrir el peso delas armas \\\hline
3.1.13 & vel ad aliquem magistratum assumitur . \textbf{ Quare cum deceat regia maiestatem } et uniuersaliter omnem ciuem , & commo se conosçe despues que esle un atada en alguna dignidat o en algun maestradgo o en algun poderio \textbf{ por la qual razon commo venga ala real magestad } e generalmente a qual quier que ha de dar \\\hline
3.1.14 & et onerosius et quasi omnino importabile esset sustentare sic quinque milia : \textbf{ oporteret enim ciuitatem illam habere possessiones quasi ad votum , } ut posset ex communibus sumptibus & ca conuerne \textbf{ que aquella çibdat ouiesse tantas possessiones | quantas quisiesse a ssu uoluntad } por que pudiesse de las rentas comunes abondar atanta muchedunbre \\\hline
3.1.14 & Propter quod Philosophus 2 Politicorum reprehendens Socratem de huiusmodi ordine ciuitatis , \textbf{ ait , quod oportet ciuitatem illam sic institutam esse in regione Babilonica , } ubi forte propter magnitudinem desertorum & que aquella çibdat assi establesçida deuie ser \textbf{ tamannera | commo babilonia en la qual } por auentura \\\hline
3.1.15 & inuasa est eorum ciuitas ab hostibus , \textbf{ propter quod oportuit } et mulieres & e commo salliessen della fue cometida la çibdat de los sus enemigos dellos \textbf{ por la qual cosa conuenio alas mugers } por mengua de los çibdadanos de defender la çibdat mas lo que enandio adelante diziendo \\\hline
3.1.15 & Quod autem ulterius addebat , \textbf{ quod semper oportet } eosdem in magistratibus praeferri , & por mengua de los çibdadanos de defender la çibdat mas lo que enandio adelante diziendo \textbf{ que sienpre conuenia } que vnos mismos fuessen adelantados en los maestradgos \\\hline
3.1.16 & non multum ex hoc gaudere posset \textbf{ quia oporteret ipsum } per dationem dotium aequari & por esto \textbf{ por quel conuernia } por las donaconnes \\\hline
3.1.17 & possumus triplici via venari , \textbf{ quod non oportet possessiones aequatas esse , } ut Phaleas statuebat . & por tres razones podemos prouar \textbf{ que non conuiene | que las possessiones sean egualadas } en aquella manera \\\hline
3.1.17 & quas decet habere ciues : \textbf{ decet enim ipsos } esse liberales et temperatos : & que deuen auer los çibdadanos \textbf{ por que conuiene } que los çibdadanos sean liberales e francos \\\hline
3.1.17 & possent enim ciues adeo modicas possessiones habere , \textbf{ quod oporteret eos ita parce viuere } quod opera liberalitatis de facili exercere non valerent . & por que podrian los çibdadanos auer tan pocas possessiones \textbf{ que les conuenia de beuir | assi es casamente } que non podrien ser lo ƀales \\\hline
3.1.18 & propter delectationem , et intemperantiam . \textbf{ Decet ergo Reges , } et Principes non omnes leges ferre & e por los destenpramientos . \textbf{ Et pues que assi es conuiene alos Reyes } e alos prinçipes \\\hline
3.1.20 & et patriam ab hostibus defendere : \textbf{ oportebat bellatores } habere maiorem potentiam , & Et segunt esto conuiene \textbf{ que los lidiadores ouiessen mayor poderio } que los menestrales nin los labradores todos en vno . \\\hline
3.1.20 & Rursus deficit dictus modus , \textbf{ quia in aliquibus iudiciis oportet } collationem habere ad inuicem ; & Otrossi fallesçe la dichͣ manera \textbf{ por que en alguons iuyzios conuiene } de auer fablas los iiezes vno con otro \\\hline
3.2.4 & quae requiritur in principante : \textbf{ decet enim Principem } esse regulam rectam et stabilem , ut per iram et concupiscentias & que deue ser en el prinçipe \textbf{ por que conuiene } que el prinçipe sea regla derecha e firme et estable \\\hline
3.2.5 & per haereditatem transferatur ad posteros , \textbf{ oportet eam transferre in filios , } quia secundum lineam consanguinitatis filii parentibus maxime sunt coniuncti : & por hedamiento conuiene alos pueblos \textbf{ que tomne alos fijos } ca segunt el linage del patente \\\hline
3.2.5 & quia secundum lineam consanguinitatis filii parentibus maxime sunt coniuncti : \textbf{ oportet autem talem dignitatem } magis transferre & e los tuos son muy apuntados alos padres \textbf{ e que conuiene | que esta dignidat real masspasse alos } mas los que alas fenbras \\\hline
3.2.5 & passionum minus insecutor . \textbf{ Inter masculos autem oportet talem dignitatem } tribuere magis primogenito quam aliis : & e menos sigue las possessiones que la fenbra . \textbf{ Mas entre los mas los conuiene | que la dignidat real } mas sea dada al primogenito \\\hline
3.2.5 & quia ( ut ait Philosophus in Politiis ) \textbf{ decet iuniores senioribus obedire . } Immo quia patres plus communiter primogenitos diligunt ; & conuiene \textbf{ que los mas mançebos obedescan alos mas uieios e avn } por que los padres comunalmente \\\hline
3.2.6 & si recta sit generatio eius , \textbf{ oportet huiusmodi excessum } peramplius & conuiene \textbf{ que estas aunataas sean mas conplidamente } e mas acabadamente en el Rey \\\hline
3.2.6 & cum calefit et rarefit , \textbf{ oportet raritatem et calorem perfectius reperiri } in igne iam generato & por que la materia estonçe es puesta e tornada en fuego \textbf{ quando es muy escalentada | e muy enraleçida conuiene que la raledat e la calentura mas acabadamente sea fallada en el fuego } despues que fuere engendrado e ençendido . \\\hline
3.2.8 & et fons scripturarum , \textbf{ oportet quod inde totus populus } aliquam eruditionem accipiat : & e la fuente delas esc̀ yturas . \textbf{ conuiene que de ally tome todo el pueblo algun enssennamiento } e de prinda alguna sabiduria . \\\hline
3.2.9 & et iura regni non obseruant . \textbf{ Tertio decet Regem , } et Principem non ostendere se nimis terribilem et seuerum , & nin los derechos del regno . \textbf{ ¶ Lo terçeto conuiene al Rey et al prinçipe de non mostrarsse muy espantable nin muy cruel . } nin le conuiene otrosi de se fazer muy familiar alos omnes \\\hline
3.2.9 & et rapiunt eorum bona . \textbf{ Quinto decet Reges } et Principes & e roban les los sus bienes \textbf{ ¶Lo quinto conuiene alos Reyes } e alos prinçipes \\\hline
3.2.9 & secundum veritatem caret illis . \textbf{ Sexto decet verum Regem } esse moderatum in cibis , & e non las hauer dada mente . \textbf{ Lo seeto conuiene audadero Rey } de ser muy mesura \\\hline
3.2.9 & quam tyrannus quaerens utilitatem propriam . \textbf{ Octauo decet verum Regem } ( ut ait Philosophus ) & que tirano que quiere sienpre supro . \textbf{ Lo octauo conuiene al uerdadero . } Rey assi commo dize el philosofo de honrrar alos sabios \\\hline
3.2.9 & et aliorum haereditates sine ratione usurpant . \textbf{ Decimo et ultimo decet } veros reges & sin razon ¶Lo de zeno \textbf{ e lo postrimero conuiene alos uerdaderos } Reyes de se auer bien çerca las cosa de dios . \\\hline
3.2.9 & Ultimo autem diximus \textbf{ quod decet regiam maiestatem } bene se habere circa diuina : & Et esto dezimos a postremas de todo notablemente \textbf{ que conuiene ala Real maiestad } de se auer bien çerca aquellas cosas \\\hline
3.2.15 & sed custos ciuitatis \textbf{ et custos regni totius oportet } quod sit virtuosus & mas el guardador de la çibdat \textbf{ e del regno conuiene } que sea omne uirtuoso e uisto \\\hline
3.2.15 & et epiikis idest super iustus : \textbf{ decet enim talem esse quasi semideum , } ut sicut alios dignitate et potentia excellit , & ca conuiene \textbf{ que el tal que sea | assi commo dios } assi que commo lieua auna taia de los otros en dignidat e en poderio \\\hline
3.2.16 & ut recte regat populum sibi commissum : \textbf{ probauimus etiam multis viis decere } Regem vigilem curam assumere , & para que derechamente gouierne el pueblo qual es acomendado . \textbf{ e prouamos por nuchͣs razones } que conuenia al Rey de ser acuçioso \\\hline
3.2.16 & sed de his quae sunt ad finem : \textbf{ oportet enim in consilio } praesupponere finaliter intentum , & ca el nuestro consseio non es dela fu . \textbf{ por que conuiene que en el conseio sorongamos la fin } e que non tomemos consseio della \\\hline
3.2.16 & utrum ciues inter se pacem debeant habere , \textbf{ et utrum regnum oporteat esse in bono statu : } sed haec accipit tanquam certa et nota , & si los çibdadanos deuen auer entre ssi paz . \textbf{ Et si conuiene | que el regno se en buen estado . } mas esto sopone \\\hline
3.2.18 & Sed ad hoc quod aliquis sit bene creditiuus , \textbf{ non oportet ipsum esse existenter talem , } sed sufficit quod videatur & mas para que alguno sea bien de creer \textbf{ non conuiene | que el sea tal fechmas cunple } que parezca tal cael o en iudga las cosas que paresçen de fuera por las cosas que vee \\\hline
3.2.18 & ideo ad hoc quod aliquis ex rebus \textbf{ de quibus loquitur fidem faciat , vel oportet quod sit prudens } vel quod credatur esse prudens . & para fazer tales cosas \textbf{ Morende para que alguno faga fe delas cosas de que fabla o conuiene que sea sabio } o que sea tenido por sabio . \\\hline
3.2.18 & et adhibetur fides , \textbf{ vel oportet quod sit bonus , } vel quod amicus , & a cuyos dichos creen los omes \textbf{ e es dada feo conuiene que sea bueono } que sea amigo \\\hline
3.2.19 & et ad bonum statum eius : \textbf{ decet ergo consiliarios } scire introitus & e a buen estado del Rey e del pueblo \textbf{ Et pues que assi es conuiene que los | consseierossepan las entradas e las sallidas del regno } por razon de los peaes \\\hline
3.2.19 & quia absque iustitia nequeunt regna subsistere . \textbf{ Decet autem scire Regem } quot sunt genera dominorum , & nin mucho durar sin iustiçia . \textbf{ Et por ende conuiene al Rey } de saber quantas son las . \\\hline
3.2.20 & secundum leges iam conditas . \textbf{ Nam in qualibet ciuitate oporteret } esse aliquod praetorium ordinarium & y a puestas \textbf{ ca en cada vna çibdat conuiene } que aya vna alcalłia otdinaria \\\hline
3.2.20 & et alio in suum locum succedere . \textbf{ Igitur saltem per successionem ipsorum oportet } in eadem ciuitate & e son prouestos otros en sir logar \textbf{ Et pues que assi es si mas que non por alongamiento detp̃o conuiene } que en vna çibdat sean much suiezes \\\hline
3.2.20 & eo quod ipsi quasi continue innouantur : \textbf{ non tamen sic oportet } multiplicare legum conditores , & por que los iuezes continuadamente se remueuna . \textbf{ Enpero non conuiene } que assi se amuchiguen los fazedores delas leyes . \\\hline
3.2.20 & si illud multo tempore diligenti consideratione discutiatur , \textbf{ quam si oporteat } statim iudicatiuam sententiam proferre . & por mucho tienpo \textbf{ e con grant diligençia fueren escodrinnados } que si luego man ama no fuesse dadas m̃a difinitiua . \\\hline
3.2.20 & Nam ( ut infra patebit ) \textbf{ oportet } aliqua committere arbitrio iudicum , & ca assi commo paresçra adelançe conuiene que algunas cosas sean puestas en aluedrio \textbf{ e en poder de los iuezes } por que los fecho \\\hline
3.2.21 & inter litigantes non declinans ad alteram partem , \textbf{ quasi regula recta decet } iustum esse iustum & que contienden non se enclinando a ninguna delas partes es \textbf{ assi commo regla derecha diziendo } e mostrando lo que es derecho \\\hline
3.2.21 & per legis enim latorem , \textbf{ vel per leges ab eo conditas oportet } apparere iudici & ca por el ponedor dela ley \textbf{ o por las leyes puestas } por el deue paresçer \\\hline
3.2.22 & ab alia vero recedit per odium , \textbf{ oportet ipsum iudicare inique : } quia tunc iudicium non procedit & por abortençia o por mal querençia . \textbf{ conuiene que el uiez judgue mal e desigual mente . } Ca entonçe el uuzio non salle de zelo de iustiçia \\\hline
3.2.23 & quae videtur tangere Philos’ 1 Rhet’ \textbf{ ad quae decet } respicere iudicem , & en el primero libro \textbf{ uiene que tenga el iuez sienpre mientes } para que perdone alas obras de los omes \\\hline
3.2.23 & sunt amplioris seueritatis contentiua , \textbf{ decet ut per pium intellectum moderetur supplicii magnitudo , } hoc est ergo quod dicitur 1 Rhetor’ & mayoraspeza e mayor dureza de quanta deue . \textbf{ Conuiene que por el entendimiento piadoso sea atenprada la guaueza dela pena } e esto es lo que dize el pho \\\hline
3.2.23 & etiam canes manifestant non mordentes eos qui resident . \textbf{ Patet ergo quomodo decet } iudices esse magis clementes quam seueros : & que yazen homillosamente ante ellos \textbf{ Et pues que assi es peiresçe | en qual manera conuiene } que los miezes sean mas piadosos que crueles \\\hline
3.2.23 & iudices esse magis clementes quam seueros : \textbf{ et si hoc decet iudices , } multo magis decet Reges et Principes , & que los miezes sean mas piadosos que crueles \textbf{ Et si esto conuiene alos iuezes mucho } mas conuiene alos Reyes \\\hline
3.2.23 & et si hoc decet iudices , \textbf{ multo magis decet Reges et Principes , } quibus congruit ampliori bonitate pollere . & Et si esto conuiene alos iuezes mucho \textbf{ mas conuiene alos Reyes } e alos prinçipes alos quales conuiene de resplandesçer \\\hline
3.2.26 & ad legem naturae , \textbf{ oportet quod sit iusta : } ut comparatur ad bonum commune , & en quanto es conparada ala ley de natura \textbf{ conuiene que sea derechͣ . } Et en quanto es conparada al bien comun \\\hline
3.2.26 & et debet regulari per huiusmodi legem , \textbf{ oportet quod sit competens } et compossibilis consuetudini patriae et tempori : & el qual pueblo deueser reglado por aquella ley . \textbf{ conuiene que sea conuenible } e que conuenga con el uso \\\hline
3.2.26 & quia secundum quod talia diuersificantur , \textbf{ oportet in ipsis legibus } aliquam diuersitatem existere . & Ca segunt que estas tales cosas se departen \textbf{ conuiene que enlas leyes sea algun departimiento . } ues que assi es . \\\hline
3.2.26 & aliquam diuersitatem existere . \textbf{ Primo igitur oportet legem humanam } siue positiuam esse iustam & ues que assi es . \textbf{ Lo primero conuiene que la ley humanal o positiua sea derecha } en quanto es conparada ala razon natural o ala ley de natura . \\\hline
3.2.26 & in qua intenditur priuatum bonum ; \textbf{ oportet enim in legibus } ( si rectae sint ) & en la qual es entendido el bien propreo . \textbf{ Ca conuiene que enlas leyes } si derechͣs fueren \\\hline
3.2.26 & et hoc sequi volumus , \textbf{ oportet hoc agere . } Tales ergo debent esse leges , & e esto queremos alcançar \textbf{ conuiene que fagamos estas cosas . } Et pues que assi estales deuen ser las leyes \\\hline
3.2.26 & et bonum priuatum ordinetur ad ipsum , \textbf{ oportet tales leges fieri } non quales requirit bonum priuatum , & porque el bien propra o es ordenado al bien comun . \textbf{ Conuiene que las leyes tales sean non } quales demanda el bien propre \\\hline
3.2.26 & tales debet eis leges imponere . \textbf{ Viso quales leges Reges et Principes deceat ponere , } quia condendae sunt leges , & tales leyes les deue poner . \textbf{ Voisto quales leyes los Reyes } e los prinçipes deuen poner Ca deuen poner buenas e aprouechables \\\hline
3.2.26 & nec per solos sermones corriguntur : \textbf{ oportuit igitur saltem } propter tales statuere leges , & por las palabras solas . \textbf{ Et por ende conuenia que por estas tales fuessen establesçidas las leyes . } las quales assi commo dize al philosofo \\\hline
3.2.27 & ad hoc quod lex habeat vim obligandi , \textbf{ oportet eam promulgatam esse . } Sed cum alia sit lex naturalis , & Poque la ley aya uirtud e fuerça de obligar \textbf{ conuiene que sea publicada e pregonada . } Mas commo otra sea la ley natural e otra la positiua en vna manera se deue publicar la vna \\\hline
3.2.27 & consistat salus regni et ciuitatis ; \textbf{ decet Reges et Principes } non modicum solicitari & este la salut del regno e dela çibdat . \textbf{ Conuiene alos Reyes e alos prinçipes } que sean muy acuçiosos \\\hline
3.2.28 & His itaque sic pertractatis , \textbf{ dicamus quod decet Reges et Principes , } quorum interest solicitari & Et por ende estas cosas \textbf{ assi tractadas digamos | que pertenesçe alos Reyes } e alos prinçipes \\\hline
3.2.29 & quod non est uniuersaliter : \textbf{ oportet enim humanas leges } quantumcunque sint exquisitae & generalmente aquello que non es general mente \textbf{ Por que conuiene que las leyes humanales } commo quier que sean examinadas de fallesçer en algun caso . \\\hline
3.2.29 & et esse regulam aliorum , \textbf{ oportet Regem in regendo alios } sequi rectam rationem , & Et por ende conuiene \textbf{ que el Rey en gouernando los otros sigua razon de rechͣ . | Et assi se sigue } que sigua la ley natural \\\hline
3.2.29 & ut puta ferrea : \textbf{ sed oportet } quod mensurentur regula plumbea , & assi commo por regla de fierro . \textbf{ Mas conuiene que se reglen con regla de plommo } que se pueda en coruar \\\hline
3.2.29 & quae sit applicabilis humanis actibus . \textbf{ Oportet igitur aliquando legem plicare ad partem unam , } et agere mitius cum delinquente , & e allegar alas obras delos omes . \textbf{ Et por ende conuiene quela ley que se ençorue } e se allegue algunas vezes ala vna parte e que obre mas manssamente con el que peca \\\hline
3.2.29 & quam lex dictat : \textbf{ aliquando etiam oportet eam plicare ad partem oppositam , } et rigidius punire peccantem , & quela ley demanda o que la ley nidga . \textbf{ Et algunas vezes conuiene que la regla se encorue | ala parte contraria } e que mas reziamente de pena \\\hline
3.2.30 & attingere punctalem formam viuendi , \textbf{ ideo oportet aliqua peccata dissimulare } et non punire lege humana , & comunalmente non puede alcançar forma de beuir en punto . \textbf{ Por ende conuiene que | dessemeie alguons pecados } e que les de passada \\\hline
3.2.30 & quae lege humana puniri non possunt . \textbf{ Oportuit igitur praeter legem humanam } dari aliquam legem , & por ley humanal . \textbf{ Et por ende conuiene que sin la ley humanal fuesse } dada otra ley diuinal \\\hline
3.2.30 & per quam ordinamur ad illud bonum . \textbf{ Decet ergo reges et principes , } quos competit esse quasi semideos , & por la qual somos ordenados a aquel bien . \textbf{ Et por ende conuiene que los Reyes e los prinçipes } alos quales parte nesçe ser \\\hline
3.2.31 & sciendum quod lex positiua si recta sit , \textbf{ oportet quod innitatur legi naturali , } et quod determinet gesta particularia hominum . & si fuere derecha conuiene que se raygͤ \textbf{ e se funde enla ley natural . } Et conuiene que determine las obras \\\hline
3.2.32 & sic debet eos excedere in bonitate et virtute : \textbf{ decet nobiles et ingenuos } esse magis bonos et virtuosos & por ende conuiene \textbf{ que los nobles e los altos } e los mas fijos dalgo sean mas buenos \\\hline
3.2.32 & propter quod regem ipsum tanquam omnibus excellentiorem \textbf{ decet esse optimum , } et quasi semideum . & assi commo aquel que sobrepula todos los otros en dignidat \textbf{ e en pero de rio sea muy bueno } e sea assi commo medio dios . \\\hline
3.2.32 & in ciuitate et regno , \textbf{ oportet esse talem , } quod viuat bene et virtuose . & que es en el regno e enla çibdat . \textbf{ conuiene que sea atal que biuna bien e uirtuosamente . } Et por ende assi conmo dize el philosofo en el terçero libro delas politicas \\\hline
3.2.33 & Quarto Politicorum ait Philosophus , \textbf{ quod tres oportet } esse partes ciuitatis . & uenta el philosofo en el quarto libro delas politicas \textbf{ que conuiene que sean tres partes dela çibdat . } Ca alguons son muy ricos . \\\hline
3.2.33 & ex personis mediis . \textbf{ Decet ergo Reges et Principes adhibere cautelas , } ut in regno suo abundent multae personae mediae ; & establesçidas de perssonas medianeras . \textbf{ Et pues que assi es conuiene | que los reyes e los prinçipes ayan cautelas e sabidurias . } por que en el su regno sean muchͣs perssonas medianeras \\\hline
3.2.34 & et opus bonum , \textbf{ oportet in recto regimine , } quod bonus ciuis sit bonus homo ; & Et la uirtud faze al que la ha buenon \textbf{ Esta buean obra conuiene que sea en el gouernamiento derech } que el buen çibdada no sea buen omne . \\\hline
3.2.35 & quae ad regem spectant . \textbf{ Decet ergo omnes ciues } et uniuersaliter omnes habitatores regni & que pertenesçen al rey ¶ Et \textbf{ pues que assi es conuiene alos çibdadanos } e generalmente a todos los moradores del regno \\\hline
3.2.36 & debent esse fortes et magnanimi , \textbf{ ponentes ( si oporteat ) } seipsos pro bono communi , & e de grandes coraçones esponiendo se assi \textbf{ e avn alos otros } si fuere menester a toda cosa \\\hline
3.2.36 & Imo , ut vult Philos’ 7 Polit’ \textbf{ ut decet Reges } magis timeantur , & Ante assi conmo dize el philosofo \textbf{ en el septimo libro delas roliticas conuiene alos Reyes e alos prinçipeᷤ } por que sean mas temidos \\\hline
3.2.36 & quiescant male agere : \textbf{ oportuit ergo aliquos inducere ad bonum , } et retrahere a malo timore poenae . & Por la qual cosa conuiene \textbf{ que alguon s | enduxiessemosa bien } e arredrassemos del mal \\\hline
3.3.1 & per quam quis scit regere domum et familiam , \textbf{ oportet esse aliam a prudentia , } qua quis nouit seipsum regere . & por la qual cada vno sabe gouernar la casa e la conpaña . \textbf{ Conuiene que sea otra e departida de la sabiduria } por la qual cada vno sabe gouernar a ssi mismo . \\\hline
3.3.1 & tanto prudentia quae requiritur \textbf{ in Rege oportet } excedere prudentiam patrisfamilias , & en tanto la sabiduria \textbf{ que pertenesçe al Rey deue } sobrepuiar la sabiduria \\\hline
3.3.1 & quod ait Vegetius in primo libro de re militari , \textbf{ quod neque quemquam magis decet } vel meliora scire , & en el primer libro del Fecho de la caualleria \textbf{ que non conuiene a ninguno saber mas cosas nin meiores } que al prinçipe \\\hline
3.3.1 & et gubernare ciues . \textbf{ Omnes autem tres prudentias decet habere Regem , } videlicet particularem , oeconomicam et regnatiuam . & e en quanto ha de poner leyes e gouernar los çibdadanos . \textbf{ Et todas estas tres sabidurias | conuiene que aya el Rey . } Conuiene a saber . \\\hline
3.3.1 & Hanc autem prudentiam videlicet militarem , \textbf{ maxime decet habere Regem . } Nam licet executio bellorum , et remouere impedimenta ipsius communis boni , & Et esta sabiduria de caualleria \textbf{ mas pertenesçe al rey que a otro ninguno . | Ca commo quier que pertenezca a los caualleros } la essecuçion de las batallas \\\hline
3.3.3 & Est etiam specialis ratio , \textbf{ quare oporteat iuuenes } ab ipsa iuuentute assuescere & que es passada avn ay razon special \textbf{ por que conuiene | que los mançebos en el comienço de la su moçedat } e de la su maçebia \\\hline
3.3.4 & Rex aut Princeps eligere . \textbf{ Primo enim oportet pugnatiuos homines posse } sustinere magnitudinem ponderis . & e quales deuen escoger el Rey o el principe para la batalla \textbf{ Lo primero conuiene } que los omnes lidiadores puedan sofrir grandes pesos . \\\hline
3.3.4 & assiduum membrorum motum . \textbf{ Tertio homines pugnatiuos decet } non curare de parcitate victus . & e el mouimiento continuado de los mienbros . \textbf{ Lo terçero los omnes lidiadores } non deue auer cuydado de escassa uianda . \\\hline
3.3.4 & ut laborem pugnandi melius tolerare possent . \textbf{ Quarto decet eos } non curare de incommoditate iacendi et standi . & por que pueda meior sofrir el trabaio de la batalla \textbf{ Lo quarto conuiene a los lidiadores } non auer cuydado de mal yazer \\\hline
3.3.4 & eis commodum aut requies . \textbf{ Quinto decet ipsos propter iustitiam } et commune bonum & non deuen auer cuydado de folgura . \textbf{ Lo quinto conuiene a los lidadores } de non preçiar la vianda corporal \\\hline
3.3.6 & In alia vero arcta et stricta , \textbf{ plus quam oporteat . } Primo ergo ex eo quod & e en la otra sera mas apretada \textbf{ e mas estrecha quel conuiene . } Pues que assi es lo primero \\\hline
3.3.8 & ut ultra quam debeat , \textbf{ oporteat exercitum constringi et constipari . } Quarto si oporteat in loco illo exercitum moram contrahere , & ø \\\hline
3.3.9 & erga necessitates corporis . \textbf{ Nam existentes in exercitu oportet multa incommoda tolerare : } quare si sint ibi aliqui molles , & penssada la sufrençia en las neçessidades del cuerpo . \textbf{ ca los que estan en las huestes | conuiene que sufran muchos males . } por la qual cosa si fueren y algunos muelles e mugerilles \\\hline
3.3.10 & quam in pedestri pugna . \textbf{ Oportet igitur praepositum } et ducem militaris belli & que en la batalla de los peones . \textbf{ Et pues que assi es conuiene } que el que es antepuesto es cabdiello de la caualleria en la batalla \\\hline
3.3.13 & ut vulnera noceant . \textbf{ Sic quia percutientes caesim oportet } plus de armis incidere , & para que los colpes enpeescan . \textbf{ Bien assi los que fieren taiando conuiene } que mas corten de las armas \\\hline
3.3.13 & quam caesim . \textbf{ Percutiendo enim caesim oportet } eleuare brachium dextrum : & por ende es meior ferir de punta que taiando . \textbf{ por que firiendo taiando } conuiene de leuantar el braço derecho e diestro . \\\hline
3.3.14 & difficilius se defendere poterunt : \textbf{ quia oportet eos sparsim incedere . Quare sicut locus ineptus defensioni , } si in eo hostes inueniantur , & e con mayor trabaio . \textbf{ Ca conuieneles que anden esparzidos . | Por la qual cosa } assi commo el logar malo \\\hline
3.3.15 & innititur alii pedi non moto : \textbf{ oportet quod pars dextra innitatur parti sinistrae quiescenti . } Cum igitur pes sinister anteponitur , & En essa manera conuiene \textbf{ que quando se mueue la parte diestra | que se afirme sobre la siniestra } que se non mueue . \\\hline
3.3.16 & per quam pergit aqua ad obsessos , \textbf{ oportebit ipsos pati aquarum penuriam . } Rursus , aliquando munitiones sunt altae , & por do viene el agua a los cercados han de auer los cercados \textbf{ por fuerça mengua de agua . } Otrossi algunas uegadas las fortalezas son altas \\\hline
3.3.18 & sicut praedicta tria genera machinarum : \textbf{ tamen non oportet tantum tempus apponere } ad proportionandum huiusmodi machinam , & commo los tres engeñios sobredichos . \textbf{ Enpero non es menester tanto tienpo } para armar este engeñio commo en los otros tres sobredichos . \\\hline
3.3.23 & volumus aliqua de nauali bello : \textbf{ non tamen oportet } circa hoc tantum insistere , & de la batalla de las naues . \textbf{ enpero non conuiene de nos } de tener çerca esto tanto . \\\hline
3.3.23 & cum ipsa pugna terrestri . \textbf{ Nam sicut terrestri pugna oportet } pugnantes bene armatos esse , & a semeiança de lidiar en algunas cosas con la batalla de la tierra . \textbf{ Ca assi commo en la batalla de la tierra . } conuiene que los lidiadores sean bien armados \\\hline
3.3.23 & sic et haec requiruntur in bello nauali . \textbf{ Immo in huiusmodi pugna oportet } homines melius esse armatos , quam in terrestri : & fazen menester \textbf{ en la batalla de la naue . | Ante conuiene que en esta batalla de la } naue sean los omnes . \\\hline

\end{tabular}
