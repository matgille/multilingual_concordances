\begin{tabular}{|p{1cm}|p{6.5cm}|p{6.5cm}|}

\hline
1.1.4 & Enpero del an vida politica e ordenada la qual los theologos llaman vida actiua \textbf{ que quiere dezir vida de bien obrar } Et dela vida contenplatian e intellectual & quam Theologi vocant vitam actiuam , \textbf{ et de vita contemplatiua } non usquequaque vera senserunt : \\\hline
1.1.5 & que en olimpiedes \textbf{ que quiere dezer en aquellas faziendas } o es aquellas batallas & quod in Olimpidiadibus , \textbf{ idest in illis bellis } et agonibus \\\hline
1.1.7 & en el capitulo dela maganimidat \textbf{ que quiere dezir grandeza de coraçon } Ca en la opinion del auariento & 4 Ethicorum cap’ de Magnanimitate ) \textbf{ quia in opinione auari , } et in opinione ponentis \\\hline
1.1.7 & Ca aquel que bien entiende \textbf{ que quiere dezir e quanto lieua este nonbre fin } e bien andança non se le puede esconder & qui enim bene intelligit , \textbf{ quid importatur nomine finis , } non potest eum latere quemlibet , \\\hline
1.1.10 & que ensennorear despotice \textbf{ que quiere dezir enssennorear . } seruilmente e sobre los sieruos ¶ & quam principari despotice , \textbf{ idest dominaliter . } Quarto non est ponenda felicitas \\\hline
1.2.3 & La otra es eutropolia \textbf{ que quiere dezir buena conuerssaçion o buena manera de beuir . } Mas la uerdat assi conma aqui fablamos de uerdat & videlicet , Veritas , Affabilitas , et Eutrapelia , \textbf{ quae potest dici bona versio . } Est autem Veritas \\\hline
1.2.3 & mas es bien fablante e curial¶ \textbf{ Mas entropolia que quiere dezir buena conpanma } o buena manera de beuir en conpanna es & sed est affabilis , et curialis . \textbf{ Eutrapelia vero siue bona versio , } est , quando aliquis sic se habet in ludis , \\\hline
1.2.6 & en el sexto libro delas ethicas . \textbf{ Eubullia que quiere dezer uirtud para bien coseiar . } la otra es por la qual & quam Philosophus Ethic’ 6 appellat eubuliam , \textbf{ idest bene consiliatiua . } Alia vero per quam bene iudicamus de inuentis , \\\hline
1.2.6 & la qual llama el philosofo sinesis . \textbf{ que quiere dezir uirtud de bien iudgar } ¶ la terçera es uirtud & quam Philosophus appellat synesin , \textbf{ idest bene iudicatiuam . } Tertia , per quam praecipiamus \\\hline
1.2.12 & por nonbre comunal \textbf{ que quiere dezir cosa clara e cosa apuesta . } Et esta estrella algunas vezes nasçe ante del sol & et venustatem communi nomine \textbf{ appellatur Venus . } Haec autem stella aliquando praecedit solem , \\\hline
1.2.17 & en el quarto libro delas ethicas llama a estos tales non liberales \textbf{ que quiere dezir non francos } assi commo son los logreros e los garçons & et non accipiens ea sicut debet , \textbf{ nimis videtur auidus pecuniae . Propter quod Philosophus 4 Ethic’ usurarios , lenones , } idest viuentes de meretricio , \\\hline
1.2.19 & la qual laman magnifiçençia \textbf{ que quiere dezir grandeza en despender . } Mas commo en cada cosa & Aliam , quae respicit sumptus magnos , \textbf{ quam magnificentiam nominant . } Sed cum magis , \\\hline
1.2.19 & llama los han asos \textbf{ que quiere dezir fuegos e fornos } por que estos tałs & Philosophus vero vocat eos chaunos \textbf{ idest ignes et fornaces , } quia tales sicut fornax omnia consumunt . \\\hline
1.2.21 & e non da delectablemente sin detenimiento \textbf{ aquello que ha de dar non es dicho magnifico mas pariufico que quiere dezir de pequena fazienda . } Et por ende dize el philosofo . & et prompte largitur \textbf{ quae largiti debet , | non est magnificus , sed paruificus . } Ideo dicitur 4 Ethic’ \\\hline
1.2.22 & Et estos son dichos magnanimos \textbf{ que quiere dezir omes de grand coraçon ca nos ueemos algunos } que dessi son aptos e apareiados & ut magnanimi . \textbf{ Videmus enim aliquos de se aptos ad magna , } potentes magna et ardua exercere : \\\hline
1.2.22 & Et estos tałs ̃ llama el philosofo caymos \textbf{ que quiere dezir fumosos e ventosos } mas nos podemos los llamar prasunptuosos & quos Philosophus vocat chaunos , \textbf{ idest fimosos et ventosos . } Nos autem eos praesumptuosos vocare possumus . \\\hline
1.2.26 & que dizen latun meses sobuios e alabadores de ssi \textbf{ que quiere dezir alabadores } que se alaban & Lacedaemones scilicet , \textbf{ iactatores et superbos appellat : } quia ultra quam eorum status requireret , \\\hline
1.2.28 & e resçibiendo los \textbf{ assi commo deuemos somos amigables e afabiles que quiere dezir amigos bien fablantes . } Pues que assi es non es otra cosa & et recipiendo ipsos \textbf{ ut debemus , | sumus amicabiles , et affabiles . } Nihil est ergo aliud amicabilitas , siue affabilitas , \\\hline
1.2.28 & la qual el philosofo llama heutropeliam \textbf{ que quiere dezir buena conpanina . } Et pues que assi es si quisieremos bien couerssar partiçipando con los otros & et debita iocunditas , \textbf{ quam eutrapeliam vocat . } Communicando igitur cum aliis , \\\hline
1.2.29 & Et estos llama el philosofo yrones \textbf{ que quiere dezir escarnidores e despreçiadores dessi mismos . } Et pues que assi es conuiene de dar alguna uirtud medianera & quos Philosophus vocat irones , \textbf{ idest irrisores , et despectores . } Oportet ergo dare aliquam virtutem mediam , \\\hline
1.2.32 & es llamada del philosofo eroyca \textbf{ que quiere dezir prinçipante e sennor ante } por que es señora delas otras uirtudes & appellatur a Philosopho heroica \textbf{ idest principans , et dominatiuat . } Ex hoc ergo manifeste patet , \\\hline
1.3.1 & partese en apetito iraçibile \textbf{ que quiere dezir enssannador e concupiçible } que quiere dezir desseador . & Sensitiuus autem appetitus \textbf{ ( ut supra diffusius diximus ) diuiditur in irascibilem , et concupiscibilem . } Praedictae ergo passiones sic distinguuntur , \\\hline
1.3.1 & que quiere dezir enssannador e concupiçible \textbf{ que quiere dezir desseador . } Et por ende las sobredichas passiones & Sensitiuus autem appetitus \textbf{ ( ut supra diffusius diximus ) diuiditur in irascibilem , et concupiscibilem . } Praedictae ergo passiones sic distinguuntur , \\\hline
1.3.10 & Conuiene saber Relo . \textbf{ gera Njemesim que quiere dezir tanto } commo indignacion dela buena andança de los malos . & sex alias passiones enumerare videtur , \textbf{ videlicet , zelum , gratiam , nemesin | ( quod idem est } quod indignatio de prosperitatibus malorum ) \\\hline
1.3.10 & e es dicha uerguença o herubesçençia \textbf{ que quiere dezir en bermegecimiento . } Et pues que assi es la uergunença es temor espeçial & et dicitur verecundia , vel erubescentia . \textbf{ Verecundia ergo est quidam timor , } et reducitur ad timorem . \\\hline
1.3.10 & assi es dicha enemessis o indignaçion \textbf{ que quiere dezir desden . } Ca segunt el philosofo & sic est nemesis , vel indignatio . \textbf{ Nam ( secundum Philosophum 2 Rhetoricorum ) } nemesis vel indignatio , \\\hline
2.1.3 & assi commo alinconicos nico \textbf{ que quiere dezir ordenador de casa } de deter minar de los heditiçios delas calas & ut ad oeconomicum , \textbf{ determinare de aedificiis domorum : } quia spectat ad ipsum uniuersaliter \\\hline
2.1.7 & aina l aconpannable e comun incatiuo \textbf{ que quiere dezir ꝑtiçipante con otro } Mas la comunidat en la uida humanal & Probabatur enim in primo capitulo huius secundi libri , \textbf{ hominem esse naturaliter animal sociale et communicatiuum . } Communitas autem in vita humana \\\hline
2.3.10 & Et canssoria de canbio . \textbf{ Obolostica que quiere dezer maunera } de tornar los dineros en pasta . & quatuor species pecuniatiuae : \textbf{ videlicet naturalem , campsoriam , obolostaticam , } et tacos siue usuram : \\\hline
2.3.10 & ¶La terçera manera del arte pecuniatiua de dineros es obolostica \textbf{ que quiere dezer arte de peso sobrepuiante que por auentura fue fallada assi . } Ca assi commo la massa del metal es partida en los dineros & Tertia species pecuniatiuae est obolostatica , \textbf{ vel ponderis excessiua : | quae forte sic inuenta fuit . } Nam sicut massa metalli \\\hline
2.3.12 & que acresçientan las riquezas es fazer monopolia \textbf{ que quiere dezer vendiconn de vno solo . } Ca quando vno solo uende taxa el preçio & est facere monopoliam , \textbf{ idest facere vendationem unius : | nam quia unus solus vendit , } taxat precium \\\hline
3.2.2 & ca el regno e la aristo carçia \textbf{ que quiere dezer sennorio de buenos } e la poliçia & et tres sunt mali . \textbf{ Nam regnum aristocratia , } et politia sunt principatus boni : \\\hline
3.2.2 & e la poliçia \textbf{ que quiere dezer pueblo bien } enssenoreante son bueons prinçipados . & Nam regnum aristocratia , \textbf{ et politia sunt principatus boni : } tyrannides , oligarchia , et democratia sunt mali . \\\hline
3.2.2 & enssenoreante son bueons prinçipados . \textbf{ La thirama que quiere dezer sennorio malo } e la obligaçia que quiere dezer sennorio duro . & et politia sunt principatus boni : \textbf{ tyrannides , oligarchia , et democratia sunt mali . } Docet enim idem ibidem \\\hline
3.2.2 & La thirama que quiere dezer sennorio malo \textbf{ e la obligaçia que quiere dezer sennorio duro . } Et la democraçia & tyrannides , oligarchia , et democratia sunt mali . \textbf{ Docet enim idem ibidem } discernere \\\hline
3.2.2 & e tal prinçipado es dicħa ristrocaçia \textbf{ que quiere dezer prinçipado de buenos omes } e uir̉tuosos e dende vienen & dicitur Aristocratia , \textbf{ quod idem est | quod principatus bonorum et virtuosorum . } Inde autem venit \\\hline
3.2.2 & e los que deuen gouernar el pueblo son dich sobtimates \textbf{ que quiere dezir muy buenos } ca muy buenos deuen ser aquellos & vocati sunt optimates , \textbf{ quia optimi debent esse } qui aliis praeesse desiderant . \\\hline
3.2.2 & tal es dich obligartia \textbf{ que quiere dezir prançipado de ricos . } Et pues que assi es dos prinçipados & quod idem est \textbf{ quod principatus diuitum . } Consurgit igitur duplex principatus \\\hline
3.2.7 & do dize que la tirnia es la postrimera obligarçia \textbf{ que quiere dezer muy mala obligacion } por que es muy enpesçedera alos subditos ¶ & tyrannidem esse oligarchiam \textbf{ extremam idest pessimam : } quia est maxime nociua subditis . \\\hline
3.2.12 & e es llamado anstrocraçia \textbf{ que quiere dezer señorio de buenos . } Mas si enssennorear en pocos non & et vocatur aristocratia \textbf{ siue principatus bonorum . } Si vero dominentur non quia boni , \\\hline
3.2.12 & mas por que son ricos es llamado obligarçia \textbf{ que quiere dezer señorio tuerto . } Mas quando enssennore a todo el pueblo & sed quia diuites , \textbf{ est peruersus | et vocatur oligarchia . } Sed si dominatur totus populus \\\hline
3.2.27 & Lo segundo que sean bien guardadas \textbf{ que quiere dezer tanto commo } que atales leyes & secundo , ut bene custodiantur , \textbf{ vel ( quod idem est ) } ut legibus sic institutis bene obediatur . \\\hline
3.3.12 & e desende que establezcan vn triangulo \textbf{ que quiere dezir forma de tres linnas } e esto se faz ligeramente . & ut aciem quadratam faciant , \textbf{ et deinde , ut constituant trigonum : } quod faciliter fit . \\\hline
3.3.16 & Enpero diremos de la batalla osse ssiua \textbf{ e que quiere dezir batalla de cercamiento . } Et por ende uisto quantas son las maneras de las batallas . & non oportet circa alia bellorum genera diutius immorari . \textbf{ Primo tamen dicemus de bello obsessiuo . } Viso ergo quot sunt bellorum genera , \\\hline

\end{tabular}
