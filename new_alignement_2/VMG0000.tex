\begin{tabular}{|p{1cm}|p{6.5cm}|p{6.5cm}|}

\hline
1.1.1 & en el terçero libro de la rethorica \textbf{ que quanto mayor es el pueblo tendo menores } e mas alongado el entendimjento onde se sigue & propter quod dicitur 3 Rhetoricorum , \textbf{ quod quanto maior est populus , } remotior est intellectus . Auditor ergo moralis negocii est simplex et grossus , \\\hline
1.1.2 & asi commo premero cada vn omne ha sçiençia \textbf{ e conosçiendo menguado } e despues alo mas conplido ¶ & ut prius quis habeat scientiam , \textbf{ et cognitionem imperfectam , } et postea habeat eam perfectiorem : \\\hline
1.1.2 & e despues alo mas conplido ¶ \textbf{ E asy dandose a aprender } e a especulaçion & et cognitionem imperfectam , \textbf{ et postea habeat eam perfectiorem : } et sic dando se speculationi , \\\hline
1.1.3 & e engannosa \textbf{ contando la orden de las cosas que aqui auemos de dezir ¶finca que en este terçero capitulo } fagamos la Real magestad atenta e acuçiosa & et in secundo reddidimus eam docilem , \textbf{ narrando ordinem dicendorum : | restat ut in hoc capitulo tertio } reddamus eam attentam , \\\hline
1.1.3 & fagamos la Real magestad atenta e acuçiosa \textbf{ declarandol quanto es el prouecho delas cosas } que aqui auemos a dezir . & reddamus eam attentam , \textbf{ declarando quanta sit utilitas in dicendis . } Nam quia communiter homines \\\hline
1.1.3 & e ençierran en sy bondat de tondos los bienes prouechosos ¶ \textbf{ Pues que asy es commo en este libro ente damos demostrͣ } commo la magesad Real aya de ser uertuosa & et includunt bonitatem utilium bonorum . \textbf{ Cum ergo in hoc libro intendatur , } quomodo maiestas regia fiat virtuosa , \\\hline
1.1.3 & e gana asy mesmo¶ \textbf{ Lo terçero el prinçipe ganando asy mesmo ganara asy mesmo los otros } por que qual quier & homo seipsum lucratur . \textbf{ Tertio lucratur alios : } nam ex hoc quod aliquis recte regit seipsum , \\\hline
1.1.4 & e la bien andança \textbf{ diziendo que non es en elła . } La qual cosa njegan los theologos en essa mjsma guisa & negauerunt esse felicitatem , \textbf{ quod et Theologi negant : } posuerunt enim felicitatem politicam , et contemplatiuam : \\\hline
1.1.4 & çiudadanamente quando es bien auenturͣado \textbf{ asy commo omne aujendo ensy pradençia e sabiduria } que es Razon derecha de todas las cosas & quando est felix ut homo , \textbf{ habendo in se prudentiam , } quae est recta ratio agibilium : \\\hline
1.1.4 & entienden las obras \textbf{ faziendo cosas granadas } e honrra das et gouernando derechamente los sus . & non enim vitam contemplatiuam \textbf{ ponimus in pura speculatione , } ut Philosophi sentiebant . \\\hline
1.1.4 & faziendo cosas granadas \textbf{ e honrra das et gouernando derechamente los sus . } subditos ¶ Et por la vida & ponimus in pura speculatione , \textbf{ ut Philosophi sentiebant . | Unde si in speculatione diuinorum } vita contemplatiua consistit , \\\hline
1.1.5 & electiuo non las fazemos de uoluntad ¶ \textbf{ pues que asy es fablando propiamente de tales obras qua non son de uoluntad fechas segunt que tales son non se nos sigue buena fin } njn buena ventura de los . & ergo per se loquendo \textbf{ ex talibus operibus | ( secundum quod huiusmodi sunt ) } non consequimur beatitudinem , et felicitatem . \\\hline
1.1.5 & e postrimera mente ¶ \textbf{ pues que asy es non conosçiendo alguna cosa } so Razon de fin nos non podemos bien obrar . & quod vult finaliter et ultimate . \textbf{ Non apprehenso ergo aliquo } sub ratione finis , \\\hline
1.1.5 & mas es por auentra \textbf{ a¶ Onde el philosofo quariendo mostrar en el primero libro delas ethicas } que es neçesario de connosçer ante la fin & sed à fortuna . \textbf{ Unde Philosophus 1 Ethicor’ | volens ostendere } necessariam esse praecognitionem finis , ait , \\\hline
1.1.6 & llaman tan solamente plazenterias e delecta connes . \textbf{ Enpero fablando mas altamente las delectaçonnes del entendimiento } e las spuerales sin conparaçion & et delectationes , \textbf{ cum tamen ( simpliciter loquendo ) | delectationes intelligibiles , } et spirituales \\\hline
1.1.6 & en el primero libro delas ethicas \textbf{ quariendo mostrar } que cosa es la fe liçidat & Unde Philosophus 1 Ethicorum \textbf{ describens felicitatem , } ait , \\\hline
1.1.6 & asi commo cada vno prueba en si mesmo . \textbf{ ¶ Onde el philosofo en el terçero libro delas ethicas fablando de tales delectaçonnes dize } que el apetito delectable de los sesos & ut in seipso quilibet experiri potest . \textbf{ unde Philosophus 3 Ethi’ loquens de talibus delectationibus ait , } quod insatiabilis est delectabilis appetitus . \\\hline
1.1.7 & e uerdaderamente cunpliesen las menguas del cuerpo ¶ \textbf{ Et pues que asi es concludendo todo lo que } dichones podemos dezir & vel cetera numismata , \textbf{ vere essent diuitiae , } et vere satisfacerent indigentiae corporali . \\\hline
1.1.7 & nin de grant coraçon . \textbf{ Ca temiendo deꝑder los des e las riquezas nunca acometra grandes cosas } Et la razon es esta & nec etiam potest esse Magnanimus , \textbf{ quia metuens pecuniam perdere , | nihil magnum attentabit . } Immo ( cum ille sit Magnanimus , \\\hline
1.1.7 & ¶Lo terçero se declara \textbf{ asi que poniendo el prinçipe la su feliçidat } e la su bien andança en las riquezas corporales & sed Tyrannus , \textbf{ cum non intendat principaliter bonum publicum , sed priuatum . } Tertio hoc posito sequitur \\\hline
1.1.8 & Et esso mesmo paresçe manifiestamente al seso . \textbf{ Por que sy alguno inclinando se faze reuerençia a otro o le honrra . } çierto es que aquella inclinaçion & Apparet autem hoc esse sensibiliter verum : \textbf{ nam si aliquis inclinat se reuerenter ad alium , | vel honorat ipsum , } constat inclinationem illam proprie esse in inclinante , \\\hline
1.1.9 & mas la fama nasçe dela honrra e de la eglesia \textbf{ Por que quando alguno es en eglesia e en honrra es auido en fama e en preçio tomando asi la gloria e la fama largamente } por vna cosa ¶ & fama autem oritur ex honore et gloria , \textbf{ quia ex hoc quod aliquis est | in honore , } et gloria , \\\hline
1.1.9 & por ende dura \textbf{ por que deles auido entre los omes algun loando e claro conosçimiento¶ } Como el nuestro connosçimiento non sea aque|p{1cm}|p{6.5cm}|p{6.5cm}|la cosa de que es & quia apud homines de ipso habetur \textbf{ quaedam laudabilis , | et clara notitia , } cum scientia nostra \\\hline
1.1.9 & mas delos omes malos . \textbf{ Ca por ꝑ muchas uegadas somos engannados en iudgando } e contesçe muchas uegandas maldat grande & sed etiam de hominibus prauis : \textbf{ quia enim multoties | in iudicando decipimur , } contingit ut plurimum illa detestanda peruersitas ; \\\hline
1.1.9 & gualardon egual e digno al su mesçimiento ¶ \textbf{ La primera manera es pensando } que cosa es la honrra en si¶ & vel ratione ipsius honoris in se , \textbf{ vel ut procedit } ex affectione dantium . \\\hline
1.1.9 & que cosa es la honrra en si¶ \textbf{ La otra manera es teniendo mientes al talante de aquellos que la fazen . } Ca la honrra en si non es & vel ut procedit \textbf{ ex affectione dantium . } Honor autem in se , \\\hline
1.1.9 & gualardon ygual nin digno al su meresçimiento . \textbf{ Mas enpero teniendo mientes ala honrra } en quanto salle de buean uoluntad & ut patet , non est condigna retributio in vita . \textbf{ Attamen ut huiusmodi honor procedit } ex affectione dantium , \\\hline
1.1.10 & Enpero non sabra beuir bien entp̃o de paz . \textbf{ Ca commo en la mayor parte non aya estudiando } si non en vsos de armas e de batallas . & tempore tamen pacis nesciet bene viuere . \textbf{ Nam , cum ut plurimum studuerit , } nisi in exercitiis bellicis , \\\hline
1.1.10 & Et por la qual los omes son ordenados \textbf{ a menores bienes posponiendo los mayores . } Et en aquello que en las mas cosas . & et per quod ordinantur ciues \textbf{ ad minora bona , } et ut plurimum infert nocumentum : \\\hline
1.1.11 & en el terçero libro dela \textbf{ seth̃s fablando delas cosas } que omne ha de escoger & ø \\\hline
1.1.11 & muchos regnos ouieron grand departimiento e grandes escandalos . \textbf{ Ca muriendo los Reys sin fijos legitimos } muchos se le una tan para seer señores & passa sunt diuisionem , et scandala : \textbf{ decedentibus enim Principibus absque liberis , } plures insurgunt , \\\hline
1.1.12 & Si el prinçipe es bien auenturado \textbf{ amando a dios deue creer } que es bien auenturado si obra & si Princeps est felix diligendo Deum , \textbf{ debet credere se esse felicem operando } quae Deus vult . \\\hline
1.1.13 & por que ellos pueden passar los mandamientos \textbf{ e non los passan mas penssando el bien comun } por que se ponen a peligro han mayor meresçiminto . & ex hoc quod transgredi possent , \textbf{ quia non considerato communi bono , } si quis periculo se exponat , \\\hline
1.1.13 & e de los prinçipes \textbf{ es esta penssando en las obras por las quales han de auer su } gualardon bueno o mal . & considerato actu , \textbf{ per quem tale meritum habet esse . } Nam ex hoc actus est vitiosus , \\\hline
1.2.1 & e por amor de dios conasçion dol \textbf{ e amandol assi conmo oficiales uerdaderos suyos } enderesçen e guyen el pueblo & ut cognoscentes , et diligentes Deum , \textbf{ tanquam veri ministri eius | secundum ordinem rationis } dirigant Populum sibi commissum . \\\hline
1.2.1 & Et el appetito que sigue al entendimiento \textbf{ es llamando uoluntad . } Segund essa manera de fablar las bestias han senssualidat et appetito sensitiuo . & nominari sensualitas : \textbf{ sequens intellectum nominatur voluntas : } secundum quem modum loquendi bruta habent sensualitatem , \\\hline
1.2.2 & e non ser bueno . \textbf{ Pues que assi es dexando aparte las uirtudes intellectuales } e las scians especulatias & esse non existentem bonum . \textbf{ Dimissis ergo virtutibus huiusmodi intellectualibus , } ut dimissis scientiis speculatiuis , de prudentia , \\\hline
1.2.2 & e por la liuiandat sube arriba a su logar propio et a su folgura . \textbf{ Et por la calentura obra contra sus contrarios destruyendo los e gastando los . } Pues que assi es por que el fuego non fuese enbargando & et leuis per leuitatem autem tendit in locum proprium , \textbf{ et in quietem sibi conuenientem : | per caliditatem vero agit in contraria : } ne ergo ignis \\\hline
1.2.2 & Et por la calentura obra contra sus contrarios destruyendo los e gastando los . \textbf{ Pues que assi es por que el fuego non fuese enbargando } por los sus contrarios & per caliditatem vero agit in contraria : \textbf{ ne ergo ignis } per quaecunque contraria agentia impediretur , \\\hline
1.2.2 & que ha por aquello en que obra los bienes e los males \textbf{ que sienten tomando los segunt } que son en si ¶ & et bona sensibilia \textbf{ secundum se considerata . } Irascibilis vero , \\\hline
1.2.2 & segund el qual ha de cometer e defenderse de los sue contrarios va alos bienes e alos males senssibles \textbf{ que siente non tornando los bienes } e los males segt̃ & et resistimus prohibentibus , \textbf{ et nociuis , | tendit in bona , } et mala sensibilia , \\\hline
1.2.2 & o en el appetito cobdiçiador \textbf{ pues que assi es tomando largamente la uirtud moral } en quanto la pradençia es dicha vna uirtud moral & vel in concupiscibili . \textbf{ Accipiendo ergo virtutem moralem large } prout ipsa prudentia \\\hline
1.2.3 & a seer buen conpannon de todos ¶ \textbf{ pues que assi es contando la iustiçia e la pradençia } con estas dichas diez & ø \\\hline
1.2.3 & que se sigue en \textbf{ quanto tenprando las passiones mesuradamente } e tenpradamente fazemos las obras de fuera . Mas la iustiçia se ha & hoc est ex consequenti , \textbf{ inquantum moderatis passionibus , } moderate ferimur in operationes exteriores . \\\hline
1.2.4 & si non estas uirtudes que ya dixiemos . \textbf{ propusiemos de mostrar que fablando delans buenas disposiconnes del alma delas quales fablaron los philosofos . } Ca delas otras non entendemos aqui fablar . & quod bonarum dispositionum \textbf{ ( loquendo de bonis dispositionibus , | de quibus locuti sunt Philosophi , } quia de aliis ad praesens non intendimus tractatum constituere ) \\\hline
1.2.4 & Et algunas sobre las uirtudes \textbf{ Ca largamente tomando la uirtud } todas estas buenas disposiconnes se pueden llamar uirtudes . & quaedam supra uirtutem . \textbf{ Large enim accipiendo uirtutem , } omnes huiusmodi bonae dispositiones , \\\hline
1.2.4 & pues que assi es de todas estas quatro disposiconnes diremos \textbf{ de cada vna en su logar . Ca determinaremos delas uirtudes mostrando . } en qual manera los Reyes e los prinçipes . & De omnibus ergo his quatuor suo loco dicemus . \textbf{ Determinabimus ergo de uirtutibus , | ostendentes , } quomodo Reges et Principes debent habere uirtutes . \\\hline
1.2.4 & Et en cabo de todo esto de aquellas uirtudes \textbf{ que son sobre todas las otras uirtudes mostrando en qual manera los Reyes e las prinçipes } han de ser conpuestos e honrrados & et de his quae sunt supra uirtutem , \textbf{ ostendentes , | quomodo Reges et Principes oportet } talibus esse ornatos . \\\hline
1.2.5 & assi commo auemos a dar uirtud . \textbf{ Por la qual cosa somos endereçados en razonando delans obras en essa mis ma guas a auemos de dar uirtud } por la qual seamos endereçados en obrando las nr̃as obras & sic ut est dare virtutem , \textbf{ per quam dirigimur } in ratiocinando de agibilibus : \\\hline
1.2.5 & Por la qual cosa somos endereçados en razonando delans obras en essa mis ma guas a auemos de dar uirtud \textbf{ por la qual seamos endereçados en obrando las nr̃as obras } ¶Otrosi ahun por que nos & per quam dirigimur \textbf{ in ratiocinando de agibilibus : } sic est dare virtutem , \\\hline
1.2.5 & Et pues que assi es commo en el entendimiento pratico la mas prinçipal uirtud son la pradençia . \textbf{ Et en la uoluntad la prinçipal uirtud sea la iustiçia fablando delas uirtudes } que nos ganamos por nuestras obras & principalior virtus sit prudentia , in voluntate \textbf{ ( loquendo de virtutibus acquisitis ) | principalior sit Iustitia , } in irascibili vero \\\hline
1.2.5 & Lo segundo diremos dela iustiçia \textbf{ mostrando que conuiene alos Reyes } e alos prinçipes de ser iustos e derechureros . & Secundo determinabimus de ipsa Iustitia , \textbf{ ostendentes quod decet Reges , } et Principes esse iustos . \\\hline
1.2.5 & Et desi diremos dela fortaleza e dela tenprança \textbf{ mostrando e declarando } que conuiene alos Reyes e alos prinçipes de seer fuertes e tenprados . & et Principes esse iustos . \textbf{ Deinde determinabimus de Fortitudine , et Temperantia , } declarantes quod contingit Reges \\\hline
1.2.5 & tractaremos dela magnanimidat e dela magnfiçençia \textbf{ e delas otras uertudes mostrando } e declarando en qual manera conuiene alos Reyes & Consequenter vero tractabimus de Magnitudine , et Magnificentia , \textbf{ et aliis uirtutibus , manifestantes , } quomodo decet Reges \\\hline
1.2.5 & e delas otras uertudes mostrando \textbf{ e declarando en qual manera conuiene alos Reyes } e alos prinçipes de ser acabados e conplidos destas tales uirtudes . & et aliis uirtutibus , manifestantes , \textbf{ quomodo decet Reges } et Principes talibus uirtutibus esse perfectos . \\\hline
1.2.6 & alas uirtudes morales \textbf{ puede se declarar assi diziendo . } que la pradençia es perfeçion del entendimiento & ut comparatur ad virtutes morales , \textbf{ sic diffiniri potest , } quod est perfectio intellectus , \\\hline
1.2.6 & por la qual es el omne reglado \textbf{ e ordenando ala fin delas uirtudes morales } ¶ lo segundo la pradençia se puede conparar alas uirtudes intellectuales & siue quod est bona qualitas mentis , \textbf{ directiua in finem virtutum moralium . } Secundo potest comparari Prudentia \\\hline
1.2.6 & puedese \textbf{ asi difinir e declarar diziendo . } que la pradençia es mandadora delas cosas falladas e iudgadas ¶ & quae sunt virtutes intellectuales , \textbf{ sic diffiniri potest , } quod est praeceptiua inuentorum et iudicatorum . \\\hline
1.2.6 & e particulares \textbf{ allegando las reglas generales alos negoçios singulares } e particulares & oportet prudentiam esse circa particularia , \textbf{ applicando uniuersales regulas | ad singularia negocia , } ut Ethic’ 6 declarari habet . \\\hline
1.2.6 & assi commo lo de clara el philosofo en el sexto libro delas ethicas ¶ \textbf{ pues que assi es conparando la pradençia ala materia } en que obra puede se & ut Ethic’ 6 declarari habet . \textbf{ Comparatiua ergo prudentia ad materiam } circa quam versatur , \\\hline
1.2.6 & assi difinir e declarar \textbf{ diziendo que la pradençia es uirtud } que iudga alos negoçios particulares seg̃t las reglas vniuerssales . & sic describi potest , \textbf{ quod prudentia est virtus } secundum uniuersales maximas particularia facta concernens . \\\hline
1.2.6 & assi commo dicho es \textbf{ puede se assi difinir e declarar diziendo . } que la pradençia es razon derecha de todas las obras & a qua distinguitur , \textbf{ sic diffiniri potest , } quam est recta ratio agibilium , \\\hline
1.2.6 & difiniçonn o declaraçion comun dela pradençia . \textbf{ diziendo que la pradençia es uirtud intellectual } enderascadora e regladora delas uirtudes morales . & dicendo , \textbf{ quod Prudentia est virtus intellectualis , } directiua virtutum moralium , praeceptiua secundum inuenta , \\\hline
1.2.8 & Ca el Rey omne es \textbf{ e el omne entiende razonando e examinando lo meior . } Ca toma algunos prinçipios & quia Rex ipse homo est . \textbf{ Homo enim intelligit ratiocinando et discurrendo . } Accipit enim aliqua principia et aliquas praemissas , \\\hline
1.2.8 & conuiene le que sea entendido \textbf{ conosciendo los prinçipios e las razones . } Et que sea razonable razonando e escogiendo de aquellos principios & oportet quod sit intelligens , \textbf{ cognoscendo principia , } et praemissa , et rationalis , ratiocinando , \\\hline
1.2.8 & conosciendo los prinçipios e las razones . \textbf{ Et que sea razonable razonando e escogiendo de aquellos principios } e daquellas razones las conclusiones & cognoscendo principia , \textbf{ et praemissa , et rationalis , ratiocinando , | et eliciendo } ex illis praemissis cunclusiones intentas . \\\hline
1.2.8 & que sea entendido \textbf{ e sabio sabiendo las leys } e las costunbres buenas & Vel oportet quod sit intelligens , \textbf{ sciendo leges , } et consuetudines bonas , \\\hline
1.2.8 & Et otrosi conuiene al Rey \textbf{ que sea razonable conosçiendo e entendiendo } por aquellas reglas & et regulae agendorum . \textbf{ Oportet autem quod sit rationalis , } speculando ex illis regulis \\\hline
1.2.8 & que son aprouechables a todo el regno . \textbf{ Enpero con esto que conuiene al Rey de ser sotil e agudo de si penssando los bienes } que son aprouechables a su regno & esse utilia toti regno , \textbf{ cum hoc quod Regem expedit | esse solertem ex se , } quae bona sunt regno utilia excogitando , \\\hline
1.2.8 & que son aprouechables a su regno \textbf{ ahun conuiene le de ser doctrinable resçebiendo e tomando coseio de bueons } quel han bien de conseiar . & quae bona sunt regno utilia excogitando , \textbf{ oportet ipsum esse docilem , | aliorum consiliis acquiescendo . } Possumus enim dicere de Rege , \\\hline
1.2.8 & e de su pueblo \textbf{ de ser my prouado conosçiendo las condiconnes particulares de su gente } e de su pueblo & esse expertum , \textbf{ cognoscendo particulares conditiones gentis sibi commissae , } ut possit eam melius in debitum finem dirigere . \\\hline
1.2.9 & El alma en seyendo \textbf{ e en estudiando } e folgando se faze sabio ¶ & Anima in sedendo , \textbf{ et quiescendo sit prudens , } ut vult Philosophus 7 Physicorum . \\\hline
1.2.9 & e en estudiando \textbf{ e folgando se faze sabio ¶ } pues que assi es los Reyes & Anima in sedendo , \textbf{ et quiescendo sit prudens , } ut vult Philosophus 7 Physicorum . \\\hline
1.2.9 & e mas en paz fue gouernado . \textbf{ Ca assi commo en las scinas especulatians entendiendo } e estudiando & et melius regebatur . \textbf{ Nam sicut in speculando , } et cogitando \\\hline
1.2.9 & Ca assi commo en las scinas especulatians entendiendo \textbf{ e estudiando } lo que dixieron & Nam sicut in speculando , \textbf{ et cogitando } quae antiqui Philosophi conscripserunt , \\\hline
1.2.9 & en aquellas scienças especulatinas . \textbf{ Assi los Reyes los prinçipes penssando } e acordandasse de los bue nos fecho de los sus & sumus sapientiores in speculabilibus : \textbf{ sic Reges et Principes cogitando acta suorum praedecessorum , } fiunt magis prudentes in agibilibus . \\\hline
1.2.9 & puede bien gouernar su regno \textbf{ tomando delas razones conuenibles conclusiones } para todas las cosas & et consuetudines debite regnum regat , \textbf{ eliciendo ex eis debitas conclusiones agibilium . } Non enim sufficit esse intelligentem , \\\hline
1.2.9 & Ca non abasta seer entendido \textbf{ sabiendo las leyes e las costunbres } que son comiencos & Non enim sufficit esse intelligentem , \textbf{ habendo cognitionem legum , | et consuetudinem , } quae sunt principia agibilium : \\\hline
1.2.9 & si el non fuere razonable \textbf{ sacando de aquellas leyes } e de aquellas costunbres razones e conclusiones conuenibles & nisi quis sit rationalis , \textbf{ ex illis legibus , } et consuetudinibus debitas \\\hline
1.2.9 & do diremos del gouernamiento del regno¶ \textbf{ pues que assi es los Reyes dando acuçia } en aquellas ocho cosas & ubi agetur de regimine regni , \textbf{ plenius ostendemus . | Reges igitur , } dando operam illis octo , \\\hline
1.2.9 & Et que non yerren en el iuyzio \textbf{ Judgando } que han de fazer aquello & ne propter malitiam appetitus , imprudenter agant , \textbf{ et iudicent esse agenda , } quae sunt fugienda . \\\hline
1.2.10 & por otra entençion en \textbf{ quanto faziendo sus obras cunplen sa ley . } Et desta diferenços se sigue la segunda . & hoc est ex consequenti , \textbf{ prout agendo talia opera , | legem implet . } Ex ista autem differentia sequitur secunda : \\\hline
1.2.10 & Mas el iustolegal en quanto se deleita en las obras delas otras uirtudes \textbf{ cunpliendo la ley la iustiçia legal } non faze al omne acabado segunt si . & quia delectatur in operibus illis , \textbf{ prout implet legem , } Iustitia legalis non perficit hominem secundum se , \\\hline
1.2.10 & o en los logreros de casas o de otras cosas . \textbf{ Ca conteçe que en vendiendo e enconprando e enlogando } e en enprestando alguno da poco e reçibe mucho . & ut in emptionibus , venditionis , mutationibus , et locationibus . \textbf{ Contingit enim in vendendo , } emendo , locando , mutuando , dare parum , \\\hline
1.2.10 & Ca conteçe que en vendiendo e enconprando e enlogando \textbf{ e en enprestando alguno da poco e reçibe mucho . } Et alas vezes el contrario & Contingit enim in vendendo , \textbf{ emendo , locando , mutuando , dare parum , | et accipere multum , } vel etiam e conuerso , \\\hline
1.2.10 & ca todas las materias morales e de costunbres \textbf{ e cerca todas las obras delas uirtudes non tomando las segunt si . } Mas en quanto por ellas es fecho conplimiento de ley¶ & circa totam materiam moralem , \textbf{ et circa omnia opera virtutum : | non secundum se accepta , } sed prout per ea est impletio legis . \\\hline
1.2.11 & Et por ende dize el philosofo en el primero cabło dela \textbf{ guatph̃ia moral fablando desta iustiçia . } que la iustiçia legales entera & Ideo primo Magnorum moralium , \textbf{ capitulo de Iustitia , | dicitur , } quod legalis Iustitia est perfecta virtus . \\\hline
1.2.11 & e acabada uirtud Et \textbf{ pues que assi es entediendo lo } por el contrario . & quod legalis Iustitia est perfecta virtus . \textbf{ ergo per locum ab opposito , } legalis Iniustitia est integra , \\\hline
1.2.11 & hay dos maneras de iustiçia vna mudadora e otra partidora \textbf{ Ca la iustiçia tomando la } propiamente non es sinon en conparacion a otro & commutatiua , et distributiua . \textbf{ Iustitia enim propria sumpta , } non est nisi ad alterum , \\\hline
1.2.11 & e non es sinon de departidas personas . \textbf{ Enpero tomando la en semeiança podemos dezir } que la iustiçia es de vno assi mismo & et non est nisi diuersarum personarum . \textbf{ Metaphorice tamen , | et per quandam similitudinem , } est Iustitia eiusdem ad se ipsum , \\\hline
1.2.11 & por que por ella pudiesen los omes traher a egualdat este \textbf{ abondamieto e esta mengua dando vna cosa por otra . } Ca vno dades en los que ha abondo & Per commutatiuam ergo iustitiam huiusmodi superabundantia , \textbf{ et defectus reducitur ad aequalitatem : | quia unus dat pecuniam } qua abundat , \\\hline
1.2.11 & o de vn regno han ordenamiento entre si mismos \textbf{ e se acorren a las sus menguas los vnos alos otros mudando } e dando las vnas cosas por las otras . & et sibi inuicem \textbf{ secundum quandam commutationem suis indigentiis satisfaciunt , } est in eis commutatiua Iustitia , \\\hline
1.2.11 & e se acorren a las sus menguas los vnos alos otros mudando \textbf{ e dando las vnas cosas por las otras . } es en ellos la iustiçia mudadora & et sibi inuicem \textbf{ secundum quandam commutationem suis indigentiis satisfaciunt , } est in eis commutatiua Iustitia , \\\hline
1.2.12 & quando puede enssennar los otros . \textbf{ Et por ende fablando por semeiança podemos dezir que estonce es dicho el omne } conplidamente bueon & est posse docere . \textbf{ Ergo a simili , | tunc est aliquis perfecte bonus , } quando bonitas sua usque ad alios se extendit . \\\hline
1.2.12 & nasçe acabada bondat \textbf{ Ca sacando la uirtud } que es dicha pradençia & innotescit perfecta bonitas . \textbf{ Nam excepta Prudentia , } quae aliis virtutibus perfectior est , \\\hline
1.2.13 & que quando en alguas cosas podemos bien obrar \textbf{ e pecar en obrando . } Conuiene de dar e de ponetur algua uirtud & Quia circa quodcunque contingit peccare , \textbf{ et bene agere , } oportet dare virtutem aliquam , \\\hline
1.2.13 & que andan \textbf{ por casa gruniendo e corriendo . } Otrosi algunos en tanto son osados & Sunt enim aliqui adeo pauidi , \textbf{ quod circunstantes muros timent . } Aliqui vero adeo fatue audent , \\\hline
1.2.13 & que de restenar las osadias . \textbf{ Otrosi por que en auiendo osadia } non es tan fuerte nin tan graue cosa acometer la batalla & quam moderare audacias : \textbf{ rursus quia in audendo } non tam difficile est aggredi pugnam , \\\hline
1.2.13 & assi commo ahun mas prinçipalmente esta esta uirtud \textbf{ en sufriendo los lidiadores } que en acometiendo los . & huiusmodi virtus \textbf{ in sustinendo pugnantes , } quam in aggrediendo eos . \\\hline
1.2.13 & en sufriendo los lidiadores \textbf{ que en acometiendo los . } Mas que los periglos delas batallas sean mas fuertes & in sustinendo pugnantes , \textbf{ quam in aggrediendo eos . } Quod autem pericula bellica sint \\\hline
1.2.13 & Otrosi maguera que la fortaleza sea cerca los periglos delas batallas \textbf{ repremiendo los temerosos } e refrenando las osadias . Enpero mas prinçipalmente es cerca aquellas cosas & reprimendo timores , \textbf{ et moderando audacias : } principalius tamen est \\\hline
1.2.13 & repremiendo los temerosos \textbf{ e refrenando las osadias . Enpero mas prinçipalmente es cerca aquellas cosas } que repremen los temores & et moderando audacias : \textbf{ principalius tamen est } circa repressionem timorum , \\\hline
1.2.13 & que cerca el refrenamiento delas osadias ¶ \textbf{ Mas que la fortaleza sea mas prinçipalmente en sofriendo } assi commo todos dizen comunal mente esto podemos prouar por tres razones ¶ & quam circa moderantiam audaciarum . \textbf{ Quod autem Fortitudo principalius sit in sustinendo , | quam in aggrediendo , } ut communiter ponitur , \\\hline
1.2.13 & Et por ende mas guaue cosa es auerse ome fuertemente \textbf{ e firmemente en sufriendo las batallas } la qual cosa requiere & Difficilius est autem habere se fortiter , \textbf{ et constanter in sustinendo bella , } quod requirit durabilitatem et tempus , \\\hline
1.2.13 & luengotp̃o que en \textbf{ acometiendo lo que se puede fazera desora . } Onde el philosofo en el terçero libro de las ethicas & quam in aggrediendo , \textbf{ quod subito fieri potest . } Unde Philosophus 3 Ethicorum cap’ de Fortitudine ait , \\\hline
1.2.13 & dize \textbf{ que la fortaleza es cerca los temores repremiendo los } e cerca las osadias res renando las & circa timores , et audacias . \textbf{ Magis tamen est | circa timores reprimendo eos , } quam circa audacias moderando ipsas \\\hline
1.2.13 & que la fortaleza es cerca los temores repremiendo los \textbf{ e cerca las osadias res renando las } Et adelante dize & circa timores reprimendo eos , \textbf{ quam circa audacias moderando ipsas } ( et subdit ) \\\hline
1.2.13 & assi commo la largueza es contraria alga stamiento \textbf{ que sobrepuia en espendiendo . } Et otrosi ala auaricia que fallesçe en mengua e en espendiendo . & ut largitas opponitur prodigalitati , \textbf{ quae superabundat in expendendo : } et auaritiae , quae deficit . \\\hline
1.2.13 & que sobrepuia en espendiendo . \textbf{ Et otrosi ala auaricia que fallesçe en mengua e en espendiendo . } assi la fortaleza es contraria ala osadia & quae superabundat in expendendo : \textbf{ et auaritiae , quae deficit . } Et Fortitudo quae opponitur audaciae , \\\hline
1.2.13 & Ca prinçipalmente es en las batallas \textbf{ e es en reprimiendo los temores . } Et en sufriendo & circa pericula bellica , \textbf{ et in reprimendo timores , } et in sustinendo pugnam . \\\hline
1.2.13 & e es en reprimiendo los temores . \textbf{ Et en sufriendo } e perseuerando en las batallas . & et in reprimendo timores , \textbf{ et in sustinendo pugnam . } Ex consequenti autem , \\\hline
1.2.13 & Et en sufriendo \textbf{ e perseuerando en las batallas . } Et en pos esto la fortaleza es cerca los otros periglos & et in reprimendo timores , \textbf{ et in sustinendo pugnam . } Ex consequenti autem , \\\hline
1.2.13 & que pueden conteçer . \textbf{ Et ahun es en refrenando las osadias } e en acometiendo los lidiadores . & est circa pericula alia , \textbf{ et in moderando audacias , } et in aggrediendo pugnantes . \\\hline
1.2.13 & Et ahun es en refrenando las osadias \textbf{ e en acometiendo los lidiadores . } ¶ Lo terçero ya declaramos & et in moderando audacias , \textbf{ et in aggrediendo pugnantes . } Tertio declaratum fuit , \\\hline
1.2.14 & ¶La fortaleza çeuiles \textbf{ quando alguno temiendo uerguença } e quariendo ganar honrra & Fortitudo enim ciuilis est , \textbf{ quando aliquis timens verecundiam , } et volens honorem adipisci , \\\hline
1.2.14 & quando alguno temiendo uerguença \textbf{ e quariendo ganar honrra } acomete alguna cosa fuerte e espantable . & quando aliquis timens verecundiam , \textbf{ et volens honorem adipisci , } aggreditur aliquod terribile , \\\hline
1.2.14 & para le dezir muchos denuestos . \textbf{ Et por ende temiendo qual denostaria su contrario era fuerte . } Et ahun pone otro & qui erat ex parte aduersa , \textbf{ primum sibi increpationes imponeret . } Sic etiam \\\hline
1.2.14 & Ca dize que si non lidiase reziamente su contrario ector \textbf{ alabandose entre los troyanos } dirie que diomedes era flaco & si non strenue bellaret , \textbf{ Hector laudans se in Troianis diceret } Diomedem ab eo deuictum esse . \\\hline
1.2.14 & por alguna neçesidat acomete la fazienda . \textbf{ Et en esta manera muchos de los troyanos temiendo a ector } que era cabdiello dela caualłia eran fuertes . & vel aliqua necessitate ductus aggreditur pugnam . \textbf{ Hoc modo multi Troianorum timentes } Hectorem fortes erant . \\\hline
1.2.14 & Et a esta manera de fortaleza \textbf{ enduzen los pueblos los caudiellos dela hueste estableciendo } e ordenando pena alos que fuyen & quod non esset sufficiens fugare canes . \textbf{ Ad hanc Fortitudinem inducunt populum Duces exercitus , } statuentes poenam fugientibus , \\\hline
1.2.14 & enduzen los pueblos los caudiellos dela hueste estableciendo \textbf{ e ordenando pena alos que fuyen } e faziendo cauas e cercas . & Ad hanc Fortitudinem inducunt populum Duces exercitus , \textbf{ statuentes poenam fugientibus , } faciendo foueas , \\\hline
1.2.14 & e ordenando pena alos que fuyen \textbf{ e faziendo cauas e cercas . } por que non puedan foyr las conpannas & statuentes poenam fugientibus , \textbf{ faciendo foueas , } ne possit exercitus fugere , \\\hline
1.2.14 & quando oyen solamente vn rroydo delas armas fuyen non \textbf{ sabiendo departir nin conosçer } quales son los periglos delas batallas & Videmus enim aliquos audientes \textbf{ solum strepitum armorum fugiunt , nescientes discernere } quae sunt periculosa in bellis , \\\hline
1.2.14 & Ca los cauałłos \textbf{ que son vsados delas batallas fiando de su praeua } e delo que han prouado & et quae non . \textbf{ Milites vero bellorum experti , } confidentes de sua experientia , \\\hline
1.2.14 & e delo que han prouado \textbf{ e conosçiendo los periglos delas batallas } acometen mas reçiamente las cosas guaues & confidentes de sua experientia , \textbf{ et cognoscentes bellorum pericula , } aggrediuntur aliqua terribilia , \\\hline
1.2.14 & Pues que assi es tales semeian fuertes \textbf{ por que acometen la batalla esperando dela victoria } e non creyendo & Tales ergo fortes esse videntur , \textbf{ quia aggrediuntur pugnam , | sperantes de victoria , } et non credentes \\\hline
1.2.14 & por que acometen la batalla esperando dela victoria \textbf{ e non creyendo } que sufran alguna cosa de mal & sperantes de victoria , \textbf{ et non credentes } aliquid mali pati : \\\hline
1.2.14 & La sexta fortaleza es testial e de bestia \textbf{ assi commo quando alguno comiença de lidiar non sabiendo } nin conosçiendo la fortaleza del su contrario . & Sexta fortitudo dicitur esse bestialis , \textbf{ ut cum aliquis ignorans fortitudinem aduersarii , bellatur . } Ut puta si habitantes in septentrione sunt fortes , et audaces , \\\hline
1.2.14 & assi commo quando alguno comiença de lidiar non sabiendo \textbf{ nin conosçiendo la fortaleza del su contrario . } Enxient lo desto . & Sexta fortitudo dicitur esse bestialis , \textbf{ ut cum aliquis ignorans fortitudinem aduersarii , bellatur . } Ut puta si habitantes in septentrione sunt fortes , et audaces , \\\hline
1.2.14 & nin conosçiendo la fortaleza del su contrario . \textbf{ Enxient lo desto . } assi commo si algunos morasen en setenturon & ut cum aliquis ignorans fortitudinem aduersarii , bellatur . \textbf{ Ut puta si habitantes in septentrione sunt fortes , et audaces , } in meridiano vero sunt debiles , \\\hline
1.2.14 & Et si alguons acometiesen la batalla con los setenteronales \textbf{ que son fuertes cuydando } que son meridionales & aggredientes pugnam cum septentrionalibus , \textbf{ credentes eos esse meridionales , } habent Fortitudinem bestialem . \\\hline
1.2.14 & por que aquellos acometen batalla \textbf{ non conosçiendo el pederio de los contrarios } e de los enemigos & Sunt enim tales quasi bestiae insensatae aggredientes bellum , \textbf{ non cognoscentes aduersariorum potentiam . } Septima Fortitudo dicitur virtuosa : \\\hline
1.2.14 & que se sigue dela batalla \textbf{ e escogiendo aquello por meior . } Et por que lo dize la razon e el entendimiento . & sed propter bonum , \textbf{ et ex electione . } Reges ergo et Principes licet \\\hline
1.2.15 & Ca contesçe alos omes de pecar \textbf{ non solamente signiendo las delecta connes corporales } mas ahun fuyendo dellas . & contingit autem peccare \textbf{ non solum delectationes sensibiles prosequendo , } sed etiam eas fugiendo . \\\hline
1.2.15 & non solamente signiendo las delecta connes corporales \textbf{ mas ahun fuyendo dellas . } Ca aquel que del todo faze abstinençia del comer e del beuer & non solum delectationes sensibiles prosequendo , \textbf{ sed etiam eas fugiendo . } Nam qui adeo abstineret \\\hline
1.2.15 & pues que assi es la tenpranca \textbf{ por si e derechamente ha de seer cerca las delectaçonnes del tannimiento e del gusto refrenando las } Mas si ella ha de seer çerca las & circa delectationes gustus , et tactus . \textbf{ Si autem est circa delectationes aliorum sensuum , } hoc est per accidens : \\\hline
1.2.15 & Et si alguas vezes se gozan con el olor delas liebres \textbf{ esto es cuydando que se fartaran dellas . } Et el leon non se goza con la bos del buen & Si autem gaudent odore eorum , \textbf{ hoc est , quia credunt se cibari ex illis . } Ideo non gaudet voce bonis , \\\hline
1.2.15 & Enpero prinçipalmente esderca las delecta connes del tannimiento \textbf{ restenando las } por que los omes en estas delecta conn es se delecta con mayor desseo . & circa delectationes tactus , et gustus : \textbf{ quia in illis homines ardentius delectantur ; } quod ( ut videtur ) rationabiliter accidit . \\\hline
1.2.15 & delectaconnes delas viandas mas nos \textbf{ delectamostanniendo que gostando . } Ca comiendo e beuiendo & Immo in ipsis delectationibus nutrimentalibus \textbf{ magis delectamur in tactu , | quam in gustu . } Comedendo enim et bibendo \\\hline
1.2.15 & delectamostanniendo que gostando . \textbf{ Ca comiendo e beuiendo } assi commo praeua cada vno en si mismo & quam in gustu . \textbf{ Comedendo enim et bibendo } ( ut unusquisque in seipso experitur ) \\\hline
1.2.15 & Mas rogo que la su garganta fuese mas luenga que garganta de grulla \textbf{ por que comiendo e beuiendose delectase mas prolongadamente por el tannimiento ¶ pues que assi es paresçe ya } por lo que dicho es & sed ut guttur eius esset longius gutture gruis , \textbf{ ut comedendo , | et bibendo diutius } delectaretur per tactum . \\\hline
1.2.15 & Conuiene nos que seamos tenprados en comer e en beuer \textbf{ ca tenprando nos en el beuer seremos mesurados } assi commo aquellos que sobrepuian en el beuer son beodos & oportet nos temperari a potu , et cibo . \textbf{ Temperando nos a potu , | sumus sobrii : } sicut qui excedunt in potando , \\\hline
1.2.15 & assi commo aquellos que sobrepuian en el beuer son beodos \textbf{ Mas nos tenprando nos en el comer lo mos astinentes . } Et pues que assi es la astinençia & sunt ebrii . \textbf{ Temperando vero nos a cibo , } sumus abstinentes . Abstinentia ergo , \\\hline
1.2.16 & Ca assi commo mostramos en el capitulo sobredicho \textbf{ tenpranca ganamos retrayendonos } e tirando nos delas delecta connes senssibles . & Nam ( ut in praecedenti capitulo dicebatur ) \textbf{ temperantiam acquirimus , | abstinendo , } et retrahendo nos \\\hline
1.2.16 & tenpranca ganamos retrayendonos \textbf{ e tirando nos delas delecta connes senssibles . } Mas la fortaleza podemos ganar & abstinendo , \textbf{ et retrahendo nos | a delectationibus sensibilibus : } fortitudinem vero acquirere possumus , \\\hline
1.2.16 & Mas la fortaleza podemos ganar \textbf{ acometiendo las cosas muy espantables } e prounado la batalla & fortitudinem vero acquirere possumus , \textbf{ aggrediendo terribilia , } et experiendo pugnam : \\\hline
1.2.16 & Mas aquel prinçipe por que era acostunbrado delas batallas \textbf{ veyendo } que el su Rey era todo mugeril & praecepit quod duceretur ad ipsum . \textbf{ Dux autem ille assuetus rebus bellicis , } videns Regem suum esse totum muliebrem et bestialem , \\\hline
1.2.16 & e quaso yr contra el para lo matar . \textbf{ Et el Rey temiendo lo fuyo . } Et por que creya que non podia foyr delas manos & voluit eum inuadere . \textbf{ Rex autem timens , fugit : } et quia credebat se non posse fugere manus illius Ducis , \\\hline
1.2.17 & Et esto en qual manera se deue entender adelante lo mostrͣemos¶ \textbf{ pues que assi es en faziendo espenssas } contesçe alas vezes de fallesçer & in prosequendo patebit . \textbf{ Si igitur in faciendo sumptus conuenit deficere , } quod facit auaritia : \\\hline
1.2.17 & por que lo es . es dicha mastal . \textbf{ Et por ende si alguno es liberal e franco en guardando las sus rentas propreas } e tomando onde deue esto & et illud magis . \textbf{ Si enim liberalis conseruans proprios redditus , } et accipiens unde debet , \\\hline
1.2.17 & Et por ende si alguno es liberal e franco en guardando las sus rentas propreas \textbf{ e tomando onde deue esto } por tanto lo faze por que pueda fazer espessas quales deuede sus rentas prop̃as & Si enim liberalis conseruans proprios redditus , \textbf{ et accipiens unde debet , } hoc ideo facit , \\\hline
1.2.17 & por que en aquello esta mas la uirtud delo qual se leunata mayor loor e mayor honrra . \textbf{ Et mayor loor e mayor honrra se le unata en bien espendiendo } e en bien faziendo alos otros & circa quod consurgit maior laus . \textbf{ Maior autem laus consurgit | in bene expendendo , } et aliis benefaciendo , quam in custodiendo propria , \\\hline
1.2.17 & Et mayor loor e mayor honrra se le unata en bien espendiendo \textbf{ e en bien faziendo alos otros } que en guardando los bienes propios & in bene expendendo , \textbf{ et aliis benefaciendo , quam in custodiendo propria , } vel in non usurpando aliena . \\\hline
1.2.17 & e en bien faziendo alos otros \textbf{ que en guardando los bienes propios } o en non usurpando & in bene expendendo , \textbf{ et aliis benefaciendo , quam in custodiendo propria , } vel in non usurpando aliena . \\\hline
1.2.17 & que en guardando los bienes propios \textbf{ o en non usurpando } nin tomando los agenos ¶ & et aliis benefaciendo , quam in custodiendo propria , \textbf{ vel in non usurpando aliena . } Liberalitas ergo principalius consistit \\\hline
1.2.17 & o en non usurpando \textbf{ nin tomando los agenos ¶ } Et pues que assi es la franqueza & et aliis benefaciendo , quam in custodiendo propria , \textbf{ vel in non usurpando aliena . } Liberalitas ergo principalius consistit \\\hline
1.2.17 & e los aueres que tienen . que cuydan que los algos son encorporados enllos . \textbf{ Ca semeia les que quindo les alguno toma los dineros } o el algo que les toma alguna cosa proprea del su cuerpo ¶ & reputent aliquid incorporatum sibi . \textbf{ Videtur enim eis , | quando accipitur pecunia ab eis , } quod accipiatur aliquid de proprio corpore . \\\hline
1.2.17 & Et por ende la franqueza esta prinçipalmente en aquello \textbf{ por que cada vno es muchon amando . } mas non es mas amado el omne & circa illud maxime consistit liberalitas , \textbf{ quod quis agendo maxime amatur . } Non autem maxime amatur aliquis , \\\hline
1.2.17 & prinçipalmente en bien espendiendolo suyo \textbf{ e en partiendo e en dando los sus bienes alos ctros . } visto que cosa es la franqueza . & ø \\\hline
1.2.17 & comtradize al miedo que ala osadia . \textbf{ Et nos fazemos a nos mismos fuertes declinando ala osadia } assi que seamos mas osados que temerosos . & timori quam audaciae , \textbf{ facimus nosipsos fortes , | declinando ad audaciam ; } ita quod potius plus audeamus , \\\hline
1.2.17 & Et mas deuemos sobrepuiar en \textbf{ dando que fallesçer en reci } egunt que dize el philosofo & et magis debemus superabundare in dando , \textbf{ quam deficere . } Vult Philosophus 4 Ethicorum liberalitatem \\\hline
1.2.18 & por que sebrepuian en muchedunbre de possesiones \textbf{ e de rentas non solamente non pueden ser gastadores dando } mas apenas pueden alcançar a que sean francos dando e espendiendo . & quia multitudine possessionum superabundant , \textbf{ non solum non possunt esse prodigi , } sed vix possunt attingere \\\hline
1.2.18 & e de rentas non solamente non pueden ser gastadores dando \textbf{ mas apenas pueden alcançar a que sean francos dando e espendiendo . } Ca sienpre deuen penssar & non solum non possunt esse prodigi , \textbf{ sed vix possunt attingere | ut sint liberales . } Semper ergo cogitare debent , \\\hline
1.2.18 & por que las sus espenssas son muy mayores que las sus rentas . \textbf{ Et por ende veyendo la mengua } en que cayen guaresçen dela enfermedat & quia expensae superabundant redditibus . \textbf{ Experiendo ergo indigentiam , } inducuntur \\\hline
1.2.18 & que avn assi mesmo es malo . \textbf{ Et el gastadora muchos aprouecha dando . } Et por ende muy de depostar es el Rey si fuer auariento ¶ visto & quia etiam sibiipsi nequam est : \textbf{ prodigus autem multis prodest . } Omnino ergo detestabile est , \\\hline
1.2.18 & e non por otra razon ninguna \textbf{ Mas los Reyes e los prinçipes apenas pueden desuiar se dela liberalidat en dando mucho } por que la grandeza delas espenssas & non propter aliquam aliam causam . \textbf{ Reges enim et Principes vix possunt | deuiare a liberalitate in dando plus , } quia magnitudo expensarum vix potest \\\hline
1.2.18 & Mas los Reyes e los prinçipes de suranse \textbf{ e arriedran se dela liberalidat en dando aqui } e non deuen o non & et Principes in dando \textbf{ quibus non oportet , } vel cuius gratia non oportet . \\\hline
1.2.19 & Pues que assi es nos faremos a nos mismos magnificos \textbf{ si ouieremos de que declinando mas al gastar e al destroyr } que ala parui ficençia & faciemus magnificos , \textbf{ declinando ad consumptionem , } ut etiam in magnis operibus potius superabundent sumptus , \\\hline
1.2.19 & en tal manera \textbf{ que enlas grandes obras las espenssas sobra pugen mayormente en despendiendo } que non en mengunado & declinando ad consumptionem , \textbf{ ut etiam in magnis operibus potius superabundent sumptus , } quam deficiant . \\\hline
1.2.19 & que non en mengunado \textbf{ e fallesciendo de despender . } l philosofo en el quarto libro delas ethicas & ut etiam in magnis operibus potius superabundent sumptus , \textbf{ quam deficiant . } Tangit autem Philosophus 4 Ethicor’ \\\hline
1.2.20 & que faze el paruifico \textbf{ sienpre las faze tardando . } Ca paresçe leal paruifico & quod quaecunque facit paruificus , \textbf{ semper facit tardans . } Videtur enim ei , \\\hline
1.2.20 & e cerca todas aquellas cosas \textbf{ que pertenesçen a todo el regno guardando las } e proprouechando las mucho ¶ & se habere circa bona communia , \textbf{ et circa ea quae respiciunt regnum totum . } Rursus quia ad ipsum maxime spectat \\\hline
1.2.20 & que pertenesçen a todo el regno guardando las \textbf{ e proprouechando las mucho ¶ } Otrosi por que a el parte nesce & et circa ea quae respiciunt regnum totum . \textbf{ Rursus quia ad ipsum maxime spectat } distribuere bona regni , \\\hline
1.2.20 & que son ayuntadas a el assi commo es la muger \textbf{ e los fijos auiendo moradas honrradas } e faziendo bodas conuenibles e honrradas & ut erga uxorem et filios , \textbf{ habendo habitationes honorabiles , } faciendo nuptias decentes , \\\hline
1.2.20 & e los fijos auiendo moradas honrradas \textbf{ e faziendo bodas conuenibles e honrradas } e vsando de caualłias marauillosał & habendo habitationes honorabiles , \textbf{ faciendo nuptias decentes , } exercendo militias admirabiles . \\\hline
1.2.20 & e faziendo bodas conuenibles e honrradas \textbf{ e vsando de caualłias marauillosał } e mucho honrradas . & faciendo nuptias decentes , \textbf{ exercendo militias admirabiles . } Philosophus 4 Ethicorum capitulo De magnificentia , \\\hline
1.2.22 & Mas despues desto es çerca las riquezas e los prinçipados e çerca los otros bienes de fuera \textbf{ en tal manera que el magnanimo conueniblemente se ha enauiendo las riquezas e los prinçipados . } Et sufriendo las cosas uenturadas e non venturadas . & et circa alia bona exteriora , \textbf{ ita quod magnanimus decenter se habet | in possidendo diuitias , } et principatus , \\\hline
1.2.22 & en tal manera que el magnanimo conueniblemente se ha enauiendo las riquezas e los prinçipados . \textbf{ Et sufriendo las cosas uenturadas e non venturadas . } Ca el magnanimo es dicho & in possidendo diuitias , \textbf{ et principatus , | et in tolerando fortunia , et infortunia . } Magnanimitas enim est \\\hline
1.2.23 & La segunda propiedat que parte nesçe al magnanimo es \textbf{ auer se bien çerca las particones delos dones dando a cada vno } gualardon commo lo meresçe . & ø \\\hline
1.2.23 & quanto mas han de lisongeros \textbf{ los quales loando los se } esfuercana los tris tornar & quanto plures habent adulatores , \textbf{ qui eos laudando conantur } ipsos peruertere . \\\hline
1.2.24 & denostamosa alguos \textbf{ deziendo que non curan de su honrra . } Et otrosi denostamos los muy cobdiçiosos de honrra . & Increpamus enim aliquos , \textbf{ dicentes eos non curare de honore suo , } et rursus quia vituperamus ambitiosos \\\hline
1.2.25 & que ha en reuerençia alos otros \textbf{ por que cuydando en los sus desfallesçimientos propios en las cosas conuenibles e honestas . } faze reuerençia alos otros ¶ & Ideo humilis dicitur alios reuereri , \textbf{ quia considerans proprios defectus , | in rebus licitis et honestis alios reueretur . } Secundo differt haec ab illa , \\\hline
1.2.25 & Mas la humildat prinçipalmente tienpra la esꝑanca \textbf{ por que alguno auiendo grand esperança del bien } non vaya en pos grandes honrras & Humilitas vero principaliter moderat ipsam spem , \textbf{ ne aliquis nimis sperans de ipso bono , } ultra rationem prosequatur magnos honores . \\\hline
1.2.26 & e nos esfuerca a cosas grandes \textbf{ que non retrayendo nos dellas . } Et pues que assi es mas prinçipalmente la magnanimidat repremela deses paracion & virtus impellens in magna , \textbf{ quam retrahens nos ab illis . } Principalius ergo magnanimitas reprimit desperationem , \\\hline
1.2.26 & e al menospreçiamiento \textbf{ Ca en quariendo omne obrar obras } que son dignas de grant honrra & ex consequenti vero contrariatur deiectioni . \textbf{ Inquirendo enim opera honore digna , } non solum contingit peccare per superbiam , \\\hline
1.2.26 & o que fuesse sob̃uio e alabador dessi \textbf{ Por que alguons por esto demandan ser enxalçados e alabados despreçiando se } mas que les conuiene . & vel esset superbus , et iactator . \textbf{ Nam aliqui ex hoc quaerunt excellentiam et iactantiam , } deiiciendo se ultra quam deceat . \\\hline
1.2.26 & por que se vestian de villes pannos \textbf{ mas que el su estado demandaua creyendo } por esto & vilius induebantur : \textbf{ credentes ex hoc in quendam honorem , } et in quandam excellentiam consurgere . \\\hline
1.2.26 & Ca el sobra uio demandado \textbf{ e quariendo su excellençia e sobrepuiamiento } mas que deue & Secundo decet eos esse humiles ratione operum fiendorum . \textbf{ Nam superbus quaerens suam excellentiam ultra quam debeat , } ut plurimum tendit \\\hline
1.2.26 & Et por ende conuiene alos omes de ser humildosos \textbf{ por que cuydando el en su fallescimiento propio } o el su poder non vayan a cosas mas altas que deuen . & Ideo decet homines esse humiles , \textbf{ ut considerato proprio defectu } vel propria facultate , \\\hline
1.2.27 & e tienpre las superfluidades . \textbf{ Mas envengando los males e los tuertos que son fechos } por los otros algunos sobrepuian . & et moderantem superfluitates . \textbf{ In vindicando autem exteriora mala } ab aliis facta , \\\hline
1.2.27 & e sobrepuian en \textbf{ desseando vengança . } Et otros son que en la sanna encubren mucho & Quidam ergo sunt irascibiles \textbf{ et superabundant in cupiendo vindictam . } Quidam autem sunt irascibiles , \\\hline
1.2.27 & si en las passiones dela saña e en \textbf{ desseando penas al su contrario } e uenganças del contesçe de sobrepiuar e de fallesçer . & Quare si in passionibus irae , \textbf{ et in appetendo punitiones et vindictas , } contingit superabundare et deficere : \\\hline
1.2.27 & por que muchos pecan en \textbf{ desseando mayor vengança } e pocos pecan en & ø \\\hline
1.2.27 & e pocos pecan en \textbf{ desseando menor vengança . } Por la qual cosa si la uirtud es çerca bien e çerca colaguaue . & ultra quam dictet ratio . Plures ergo peccant in appetendo plus : \textbf{ pauci vero delinquunt in appetendo minus . } Propter quod si virtus est \\\hline
1.2.28 & Ca en quanto por las palauras e por las obras \textbf{ conueniblemente nos auemos con los otros honrrando los } e resçibiendo los & et operibus debite \textbf{ conuersamur cum aliis , | honorando eos , } et recipiendo ipsos \\\hline
1.2.28 & conueniblemente nos auemos con los otros honrrando los \textbf{ e resçibiendo los } assi commo deuemos somos amigables e afabiles que quiere dezir amigos bien fablantes . & honorando eos , \textbf{ et recipiendo ipsos } ut debemus , \\\hline
1.2.28 & que quiere dezir buena conpanina . \textbf{ Et pues que assi es si quisieremos bien couerssar partiçipando con los otros } deuemos seer alegres conueniblemente & quam eutrapeliam vocat . \textbf{ Communicando igitur cum aliis , | si bene conuersari volumus , } debemus esse debite iocundi , \\\hline
1.2.28 & delas quales cosas todas auemos aqui de dozir \textbf{ mas primero diremos dela amistan ca por que ueemos que en partiçipando } e en conuerssando con los otros & de quibus omnibus est dicendum . \textbf{ Sed primo de amicabilitate . } Videmus enim quod conuersando cum aliis , \\\hline
1.2.28 & mas primero diremos dela amistan ca por que ueemos que en partiçipando \textbf{ e en conuerssando con los otros } algunos sobrepuian por que se muestran mucho amigables & Sed primo de amicabilitate . \textbf{ Videmus enim quod conuersando cum aliis , } aliqui superabundant , \\\hline
1.2.28 & Onde el philosofo enel quarto libro delas politicas \textbf{ dando captelas alos Reyes } e alos prinçipes dize & Unde Philosophus 5 Politicorum \textbf{ dando cautelas Regum et Principum , } ait , quod decet Reges et Principes \\\hline
1.2.29 & por la qual alguno se faze uerdadero e manifiesto . \textbf{ algunos se desuian por sobrepuiança mostrando de ssi mismos } por palauras o por fechos mayores cosas que sean en ellos & per quam quis reddit se veracem et manifestum , \textbf{ aliqui deniant per superabundantiam , | ostendentes de se verbis } aut factis maiora quam sint , \\\hline
1.2.29 & mas otros ay que se desuian desta uirtud \textbf{ por falles çemiento deziendo } e segerendo de ssi mismos algunas cosas villes & et tales vocantur iactatores . \textbf{ Aliqui vero ab hac veritate declinant per defectum , } fingentes de se aliqua vilia \\\hline
1.2.29 & por falles çemiento deziendo \textbf{ e segerendo de ssi mismos algunas cosas villes } que en ellos non sono negando de ssi mismos & Aliqui vero ab hac veritate declinant per defectum , \textbf{ fingentes de se aliqua vilia } quae in ipsis non sunt , \\\hline
1.2.29 & e segerendo de ssi mismos algunas cosas villes \textbf{ que en ellos non sono negando de ssi mismos } alguas cosas & Aliqui vero ab hac veritate declinant per defectum , \textbf{ fingentes de se aliqua vilia } quae in ipsis non sunt , \\\hline
1.2.29 & mas deuemos de elinar en tales cosas alo menos . . \textbf{ deziendo cada vno dessi menores cosas } que sean en el que sobrepiuaren mas afirmando & quia ( ut dictum est ) \textbf{ in talibus magis est declinandum in minus dicendo de se minora quam sint , } quam sit excedendum \\\hline
1.2.29 & deziendo cada vno dessi menores cosas \textbf{ que sean en el que sobrepiuaren mas afirmando } e deziendo de ssi mayores cosas que son en el . & in talibus magis est declinandum in minus dicendo de se minora quam sint , \textbf{ quam sit excedendum } in plus asserendo de se maiora . \\\hline
1.2.29 & que sean en el que sobrepiuaren mas afirmando \textbf{ e deziendo de ssi mayores cosas que son en el . } Mas nos podemos mostrͣ & quam sit excedendum \textbf{ in plus asserendo de se maiora . } Possumus autem duplicem causam assignare \\\hline
1.2.29 & Et por ende comunalmente los omes son engannados de ssi mismos . \textbf{ cuydando que valen mas de quantovalen . } Et por ende en contando cada vno los sus propreos bienes & Ideo communiter homines decipiuntur de se ipsis , \textbf{ plus credentes se plus valere , | quam valeant . } In narrando ergo propria bona , \\\hline
1.2.29 & cuydando que valen mas de quantovalen . \textbf{ Et por ende en contando cada vno los sus propreos bienes } deue se sienpre inclinar alo menos . & quam valeant . \textbf{ In narrando ergo propria bona , } semper declinandum est in minus : \\\hline
1.2.29 & Ca aquellos que non declinan alo menos son alabadores de ssi mismos \textbf{ e alabando se de aquellos bienes } que han son alos otros en carga & Nam non declinantes in minus sunt laudatores sui , \textbf{ iactantes se de bonis quae habent . } Et quia homines communiter horrent , \\\hline
1.2.29 & por que aquellos que se inclinan notablemente \textbf{ alo menos diziendo } de ssi menores cosas & et Principes esse veraces . \textbf{ Nam declinantes notabiliter in minus , } et dicentes de se notabiliter minora et viliora , \\\hline
1.2.29 & nin escarnidores dessi mas manifiestos \textbf{ e uerdaderos non mostrando } nin alabando dessi mayores cosas que son & sed apertos et veraces , \textbf{ non ostendendo , } vel iactando de se maiora quam sint , \\\hline
1.2.29 & e uerdaderos non mostrando \textbf{ nin alabando dessi mayores cosas que son } nin prometiendo alos otros mayores cosas que faran . & non ostendendo , \textbf{ vel iactando de se maiora quam sint , } vel promittendo aliis maiora quam faciant . \\\hline
1.2.29 & nin alabando dessi mayores cosas que son \textbf{ nin prometiendo alos otros mayores cosas que faran . } Mas por tanto conuiene alos Reyes & vel iactando de se maiora quam sint , \textbf{ vel promittendo aliis maiora quam faciant . } Immo tanto magis decet Reges et Principes cauere iactantiam , \\\hline
1.2.30 & assi commo el veer \textbf{ e el oyr t̃baian en sintiendo las cosas senssibles . } Et por ende la natura ordeno el su enno & ut visus , et auditus , \textbf{ quia laborant in sentiendo , } natura ordinauit somnum propter eorum requiem , \\\hline
1.2.30 & En essa misma manera \textbf{ por que en estudiando } e trabaiando en los negoçios del regno & et est necessarius somnus in vita . \textbf{ Sic quia studendo , } vel negociis regni insistendo , \\\hline
1.2.30 & por que en estudiando \textbf{ e trabaiando en los negoçios del regno } e faziendo otras cosas & Sic quia studendo , \textbf{ vel negociis regni insistendo , } vel alia faciendo , \\\hline
1.2.30 & e trabaiando en los negoçios del regno \textbf{ e faziendo otras cosas } muchas trabaiamos continuada mente . & vel negociis regni insistendo , \textbf{ vel alia faciendo , } continue laboramus , \\\hline
1.2.30 & Et por que en estos iuegos algunos sobre punan \textbf{ desseando auer riso } dellos & quidam superabundant desiderantes \textbf{ omnino risum facere , } de quibus 4 Ethicorum dicitur , \\\hline
1.2.30 & Por que prinçipalmente ha de seer en \textbf{ repremiendo las superfluydades de luego } Et despues desto ha de ser & quia principalius est in reprimendo \textbf{ superfluitates ludi , } ex consequenti circa moderationem defectuum . \\\hline
1.2.30 & Et despues desto ha de ser \textbf{ entenprando los fallescemientos . } Et pues que assi es finca deuer & superfluitates ludi , \textbf{ ex consequenti circa moderationem defectuum . } Restat ergo videre , \\\hline
1.2.31 & Et tales commo estos algunas vezes toman por furto \textbf{ e algunas vegadas por manifiesta o prasion apremiando los otros } e algunas uegadas por robo & et aliquando per furtum , \textbf{ aliquando per manifestam oppressionem alios depraedantur , } ut exerceant opera largitatis . \\\hline
1.2.31 & que son ordenadas ala fin . \textbf{ Ca nos proponiendonos } e ordenando nos a buena fin & quae sunt ad finem . \textbf{ Nam proponentes nobis bonum finem } per virtutes morales , \\\hline
1.2.31 & Ca nos proponiendonos \textbf{ e ordenando nos a buena fin } por las uirtudes morales . & quae sunt ad finem . \textbf{ Nam proponentes nobis bonum finem } per virtutes morales , \\\hline
1.2.31 & que son ordenadas a aquella fin . \textbf{ Et pues que assi es fablando delas uirtudes dezimos } que prinçipalmente et primeramente la uirtud moral rectifica & ea quae sunt ad finem . \textbf{ Loquendo ergo principaliter et primo , } virtus moralis rectificat terminum : \\\hline
1.2.31 & et endustrioso el qual \textbf{ proponiendo qual li quier fin falla carreras e caminos } por que mas ayna alcançe aquella fin . & Ille enim dicitur Denos , et industris , \textbf{ qui proposito quocunque fine inueniat vias , } ut citius consequatur finem illum : \\\hline
1.2.32 & Et algunos son tenprados . Et algunos son diuinales \textbf{ Mas tornando al pri . } mero grado de los malos ditemos & quidam temperati , \textbf{ quidam vero diuini . } Molles autem dicuntur illi , \\\hline
1.2.32 & delas ethicas el delicamiento es vna molleza . \textbf{ Et por ende estos tales non quariendo sofrir ninguna cosa guaue } luego que padelçen o son passionados por alguna passion & delicia quaedam mollicies est . \textbf{ Tales ergo nihil difficile sustinere volentes , } statim cum passionantur , \\\hline
1.2.32 & sufren tentaçion e peua . \textbf{ mas en sofriendo la fallesçen . Et por esto son dichos non continentes } que se non contienen & Incontinentes vero pugnam sustinent , \textbf{ sed in sustinendo deficiunt . } Continere enim , \\\hline
1.2.32 & e obran o coriiençan a obrar conueniblemente \textbf{ por que ellos estando fuera delas passiones } por las quales son tentados tan bien los non continentes & et aliud agunt existentes \textbf{ enim extra passiones tam incontinentes , } quam molles , \\\hline
1.2.32 & e por pequana tentaçion . \textbf{ Mas fablando dela persseuerança } assi commo el philosofo dize non es otra cosa & et non modica tentatione cadunt . \textbf{ Loquendo ergo de Perseuerantia , } ut Philosophus loquitur , \\\hline
1.2.33 & por vso de buenas obras \textbf{ ¶Et pues que assi es siguiendo el camino } et la carrera de los philosofos & dicebant esse acquisitam . \textbf{ Sectando ergo Philosophorum viam , } dicere possumus , \\\hline
1.2.33 & por que estas tales uirtudes son muy pequanas entre las otras uirtudes \textbf{ Et los ꝑseuerantes fablando dela persseuerança } ally commo el philosofo fabla della tienen mas baroguado en el linage de los buenos . & Sunt autem huiusmodi virtutes minime inter virtutes alias : \textbf{ et perseuerantes | ( loquendo de Perseuerantia , } ut de ea Philosophus loquitur ) \\\hline
1.2.33 & que cada vno de los sus subditos tome del forma e manera de beuir \textbf{ e conosca can vno su mengua veyendo la uida } e la grant perfeçion del prinçipe e del señor . & inde formam viuendi , \textbf{ et cognoscat defectum suum , | videns vitam et perfectionem principantis . } Quare apud Reges et Principes \\\hline
1.2.34 & e podriemos ganar bondat acabada . \textbf{ ssi commo es dicho dsuso commo quier que largamente tomando las uirtudes toda buean dispoliçion del alma } puede ser dicha alguna uirtud . & et perfectam bonitatem acquirere . \textbf{ Dicebatur enim supra , | quod licet largo modo accipiendo virtutes , } omnis bona dispositio mentis possit \\\hline
1.2.34 & e sin esis que es uirtud que iudgalo conseiado . \textbf{ Estas uirtudes tomando largamente la uirtud . } Onde el philosofo en el sesto libro delas ethicas llama a estas uirtudes . & et synesis siue consiliatiua , \textbf{ et iudicatiua , virtutes sunt , | accipiendo virtutem large . } Unde et Philosophus 6 Ethicorum , \\\hline
1.2.34 & Onde el philosofo en el sesto libro delas ethicas llama a estas uirtudes . \textbf{ Enpero tomando la uirtud } assi commo della fablamos aqui en algua manera & eas virtutes appellat . \textbf{ Accipiendo virtutem tamen , } ut hic de virtute loquimur , \\\hline
1.2.34 & assi commo es la perseuerança e la continençia . \textbf{ Ca la continençia propiamente fablando non es uirtud } por que al uirtuoso es cosa delectable & ut perseuerantia , et continentia . \textbf{ Continentia enim non proprie est virtus , } quia virtuoso delectabile est benefacere : \\\hline
1.2.34 & non ha acabado uso de razon nin de uirtud . \textbf{ Enpero venciendo aquellas uirtudes passiones } es despuesto para ser uirtuoso & non habet perfectum usum rationis et virtutis : \textbf{ attamen vincendo passiones illas , } disponitur ut sit virtuosus . \\\hline
1.2.34 & Mas commo la continençia sea meior \textbf{ e mayor que la perseuerança tomando la perseueraça } assy como de la fabla aqui & Sed cum continentia potior sit , \textbf{ quam perseuerantia | ( accipiendo perseuerantiam , } ut de ea loquitur Philosophus ) \\\hline
1.2.34 & abasta de saber \textbf{ que fablando dela perseuerança } assi commo fablamos de sieso & Sufficit autem ad praesens scire , \textbf{ quod loquendo de perseuerantia } ut superius dicebatur , \\\hline
1.2.34 & que uirtud \textbf{ ¶ Onde el philosofo en el septimo libro delas etl sfablando desta uirtud dize } que non es uirtud & quam sit virtus . \textbf{ Unde Philosophus 7 Ethicorum loquens de hac virtute , } ait , quod non est virtus , \\\hline
1.3.1 & ¶ Agora finca de dezir dela tercera parte deste libro \textbf{ e mostrando quales passiones } e quales mouimientos de coraçon & Restat exequi de tertia parte huius primi libri , \textbf{ ostendendo quas passiones , } et quos motus animi Reges et Principes debeant imitari . \\\hline
1.3.1 & por ende primeramente tractaremos de estas passiones \textbf{ Mas tomando el cuento delas passiones } assi commo dixiemos de suso & ideo de his primo tractabimus . \textbf{ Accipiendo autem numerum passionum , } sicut dicebamus esse duodecim virtutes , \\\hline
1.3.2 & O singu larmente cada vna por lli . \textbf{ O conparandalas e ayuntando las vna con otra¶ } pues que assi estomando esta orden & Ordo autem earum dupliciter potest accipi : \textbf{ vel singulariter , vel per combinationem . } Accipiendo ergo huiusmodi ordinem secundum combinationem , \\\hline
1.3.2 & O conparandalas e ayuntando las vna con otra¶ \textbf{ pues que assi estomando esta orden } tal segunt conbinaçion & vel singulariter , vel per combinationem . \textbf{ Accipiendo ergo huiusmodi ordinem secundum combinationem , } dicere possumus primas passiones esse , amor , et odium . \\\hline
1.3.2 & O si la ouieremos \textbf{ desseamos la de guardar en auiendo la . } Mas la aborrençia sin ningun medio se ayunta ala mal querençia & uel si ipsum habemus , \textbf{ desideramus conseruari in habendo ipsum . } Abominatio uero immediate innititur odio : \\\hline
1.3.3 & por que el Rey prinçipalmente entiende el bien comun de todos . \textbf{ Et entendiendo en el bien comun } entiende en el su bien proprio & quia Rex principaliter intendit bonum commune : \textbf{ et intendendo bonum commune , } intendit bonum proprium : \\\hline
1.3.3 & e a cada vna dellas . \textbf{ Et pues que assi espenssando las uirtudes } por las quales deuen ser los Reyes honrrados & ad virtutes singulas . \textbf{ Considerando ergo virtutes , } quibus decet Reges \\\hline
1.3.3 & nin puede en otra manera durar el bien comun \textbf{ si non destruiendo } e matando los omes malos & nec potest aliter durare commune bonum , \textbf{ nisi exterminando maleficos homines , } extirpandi sunt tales , \\\hline
1.3.3 & si non destruiendo \textbf{ e matando los omes malos } que fazen mal . & nec potest aliter durare commune bonum , \textbf{ nisi exterminando maleficos homines , } extirpandi sunt tales , \\\hline
1.3.5 & en el terçero libro lo mostraremos mas conplidamente \textbf{ uando determinamos dela ança orden delas passiones del alma dixiemos } que el amor e la malqreçia eran las primeras passiones & in tertio libro diffusius ostendetur . \textbf{ Cum determinauimus de ordine passionum animae , } diximus quod amor et odium erant passiones primae , \\\hline
1.3.5 & Et por ende por que la humildat tienpra la esperança \textbf{ ca los humildosos conosçiendo su propre o fallesçemiento non esperan } mas de aquello que deuen esparar . & cum ergo humilitas moderet spem , \textbf{ quia humiles cognoscentes defectum proprium , } non sperant ultra quam debeant : magnanimitas vero reprimat desperationem , \\\hline
1.3.7 & luego ladran \textbf{ non departiendo nin conosçiendo si aquel que viene es amigo o enemigo . } Bien assi faze la saña . & latrant , non distinguentes , \textbf{ an veniens sit amicus , vel inimicus . } Sic et ira facit : \\\hline
1.3.7 & por que sea fecha uengança \textbf{ non espando sobresto iuyzio delanrazon e del entendimiento } en qual manera esta uenganca deue ser fecha . & ut vindictam exequatur , \textbf{ non expectans super hoc iudicium rationis , } qualiter vindicta illa fieri debeat . \\\hline
1.3.7 & oscuresçe la razon e el entendimiento \textbf{ Ca el cuerpo non estando en tenpramiento conuenible somos enbargados en el vso dela razon . } Por la qual cosa commo por la saña se ençienda la sangre cerca el coraçon tornasse el cuerpo destenprado & quia rationem obnubilat . \textbf{ Nam corpore non existente indebito temperamento , | impedimur ab usu rationis , } quare cum per iram accendatur sanguis circa cor , \\\hline
1.3.7 & enpero en su obra vsa de entender de organos e de mienbros corporales . \textbf{ Por la qual cosa el cuerpo non estando bien ordenado } commo deue non & utitur tamen in suo actu corporalibus organis ; \textbf{ propter quod corpore existente indisposito , } non potest libere \\\hline
1.3.8 & Mas otros dizian todo el contrario desto \textbf{ diziendo } que toda delectaçion era mala de foyr e de esquiuar & Alii autem econtrario , \textbf{ dicebant omnem delectationem esse fugiendam . } Sed hi omnem delectationem condemnantes , \\\hline
1.3.8 & Ca assy commo la fabla non puede ser negada sinon por la fabla . \textbf{ Ca nengando la fabla fabla el omne en fablando otorga e pone la fabla . } Et por ende negando la fabla otorga & negans loquelam , \textbf{ loquitur : | loquendo autem , } concedit loquelam : \\\hline
1.3.8 & Ca nengando la fabla fabla el omne en fablando otorga e pone la fabla . \textbf{ Et por ende negando la fabla otorga } e pone la fabla . & loquendo autem , \textbf{ concedit loquelam : | quare negando loquelam , } concedit loquelam . \\\hline
1.3.9 & que son cotadas de ssuso \textbf{ signiendo la doctrina de nuestros anteçessores podemos dezer que son las quatro prinçipales . } assi commo la esperança Et el temor . & sequendo praedecessorum doctrinam , \textbf{ dicere possumus , | quod sunt quatuor principales ; } ut spes , timor , gaudium , et tristitia . \\\hline
1.3.9 & de algun mal comiença en la mal querençia \textbf{ e vayendo para la foyr } e para lo aborrescer & Respectu vero mali incipit ab odio , \textbf{ et procedit in fugam , } vel abominationem , \\\hline
1.3.10 & e declarar el zelo \textbf{ en conparaçion destas cosas diziendo } assi que el zelo es amor muy grande & ø \\\hline
1.3.10 & en el segundo libro dela rectorica \textbf{ diziendo que es tristeza destos bienes tales } non por que son en otro & a Philosopho 2 Rheto’ \textbf{ quod est tristitia | de huiusmodi bonis , } non quia insint alteri , \\\hline
1.4.1 & que fizieron pocas cosas \textbf{ Mas mucho se delectan en cuydando aquellas cosas } que han de fazer & quia memorantur se modica fecisse : \textbf{ sed multum delectantur in cogitando , } quae facturi sunt . \\\hline
1.4.2 & Lo quarto son peleadores ¶ \textbf{ Lo quanto son mintrosos afirmando } porfiosamente todas las cosas & Quinto sunt mendaces , \textbf{ omnia quodammodo pertinaciter asserentes . } Sexto in actionibus non habent modum , \\\hline
1.4.2 & et aquel que cuyda que non fabla en enganno \textbf{ Por ende los mancebos creyendo que todos son sinples } e sin maliçia de ligero creen a todos . & ex malitia loqui : \textbf{ iuuenes credendo omnes esse innocentes , } omnibus de facili credunt . Rursus hoc idem contingit ex inexperientia . \\\hline
1.4.2 & Mas despues que han mentido \textbf{ desseando eglesia } son porfiosos et afincados en la mentira & Postquam autem mentiti fuerunt , \textbf{ appetentes gloriam , } sunt pertinaces in mendatio cogitant enim \\\hline
1.4.2 & Ca commo ellos ayan muchos lisongeros \textbf{ e muchos les estenruyendo alas oreias } deuen penssar con grand acuçia & Nam cum multos habeant adulatores , \textbf{ et plurimi sint in eorum auribus susurrantes , } cum maxima diligentia cogitare debent , \\\hline
1.4.3 & e temerosos e son de flaco coraçon \textbf{ por que les va fallesçiendo la uida . } Ca por el mucho beuir son encoruados e fallesçen . & Pusillanimes enim sunt , \textbf{ quia sunt humiliati a vita : } propter enim multum viuere sunt humiliati , \\\hline
1.4.3 & que todas las cosas les fallescan . \textbf{ Et por ende temiendo que uern que amengua son escassos } e non osan espender Ante & quod omnia eis deficiant . \textbf{ Timentes ergo defectum pati , | sunt illiberales , } et non audent expendere : \\\hline
1.4.3 & e non osan espender Ante \textbf{ ueyendo se assi fallesçer en los cuerpos } non fian de su fuerça pprea & et non audent expendere : \textbf{ immo videntes sic se deficere , non confidunt de propriis viribus , } sed solum confidunt de iis quae habent . \\\hline
1.4.3 & mas solamente fian de aquellas cosas que han e tienen . \textbf{ Et por ende poniendo en lo que han su esperança } e su fiuza non osan fazer espenssas . & sed solum confidunt de iis quae habent . \textbf{ Ponentes ergo in eis suam spem et confidentiam , } non audent expensas facere . \\\hline
1.4.3 & que ellos sufrieron muchͣs menguas . \textbf{ Et por ende temiendo que auran adelante menguas son escassos . } aun contesçe a ellos de ser escassos e non francos & credibile est eos fuisse passos indigentias multas . \textbf{ Timentes ergo indigentiam pati , | illiberales fiunt . } Contingit etiam eos illiberales esse , \\\hline
1.4.3 & ca non biuen \textbf{ nin se delectan en esperando } mas en auiendo memoria & non enim viuunt , \textbf{ nec delectantur in sperando , } sed in memorando . \\\hline
1.4.3 & nin se delectan en esperando \textbf{ mas en auiendo memoria } delo que han fech̃o . & nec delectantur in sperando , \textbf{ sed in memorando . } Huiusmodi autem signum est : \\\hline
1.4.3 & mas non se delectan en \textbf{ contando las cosas } que son de fazer . & quas fecerunt ; \textbf{ non autem delectantur in recitando res fiendas , } quas sunt facturi , \\\hline
1.4.3 & Et por ende son de poca esperança \textbf{ por que en esperando fallesçen } e esperan de fazer pocas cosas & Sunt ergo difficilis spei , \textbf{ quia in sperando deficiunt , } et pauca se facere sperant . \\\hline
1.4.3 & e el frio tondas las cosas estrinne e aprieta e costͥmedolas \textbf{ e apretandolas torna las colas mas pesadas } e faz las dessear el logar mas bayo . & et constringit : \textbf{ et constringendo ea , | reddit ipsa grauiora , } et facit ea appetere inferiorem locum . \\\hline
1.4.3 & que non solamente conuiene alos Reyes \textbf{ e alos prinçipes de ser francos faziendo espenssas medianas } mas ahun les conuiene de sern magnificos & quod non solum decet \textbf{ Reges et Principes esse liberales , | faciendo mediocres sumptus : } sed etiam congruit eos esse magnificos , \\\hline
1.4.4 & auer lo que non han . \textbf{ Por ende los uieios mas pecan por la escasseza en reteniendo lo que han que en } desseando & quod non habet . \textbf{ Senes magis peccant | per illiberalitatem in retinendo quae habent , } quam in concupiscendo indebite \\\hline
1.4.4 & Por ende los uieios mas pecan por la escasseza en reteniendo lo que han que en \textbf{ desseando } commo non deuen lo que non han . & per illiberalitatem in retinendo quae habent , \textbf{ quam in concupiscendo indebite } quae non habent : \\\hline
1.4.4 & e faze se mudamiento en las obras del alma . \textbf{ Ca esfriando el cuerpo el alma es inclinada } por el appetito & et fit variatio actionum eius ; \textbf{ corpore igitur infrigidato , } anima per appetitum inclinatur \\\hline
1.4.4 & e vieron que fueron muchͣs uezes engannados non osan afirmar ningunan cosa \textbf{ afincandamente temiendo } que el fecho saldra de otra guisa & non audent pertinaciter aliquid asserere , \textbf{ timentes , } ne ita res se habeat , \\\hline
1.4.4 & Et todas las cosas fazen forcadamente \textbf{ e con sobeiama non teniendo manera } Et esso mismo los uieios & et omnia agunt valde : \textbf{ sic senes , } quia habent passiones , \\\hline
1.4.4 & Por ende se \textbf{ siguet̃yendo lo todo a vno } que todas las cosas que son de loar en los uieios & et temperati cum virilitate . \textbf{ Ut ergo sit ad unum dicere , } quicquid laudabilitatis est in senibus , \\\hline
1.4.5 & por que los fijos son fechuras de los padron natural cosa es que los fuos semeien alos paradres . \textbf{ Et por ende los nobles teniendo mientes } que en el su linage fueron muchos nobles & naturale est filios imitari parentes . \textbf{ Nobiles ergo aduertentes } quod in eorum genere fuerunt multi insignes , \\\hline
1.4.5 & en que los padres de alguons comne caron de se enrriqueçer \textbf{ quanto mas va descendiendo la generaçion de los fijos } tanto menos es memoria & in quo genitores alicuius ditari inceperunt : \textbf{ quanto magis proceditur | per creationem filiorum , } tanto minus est memoria genitores suos fuisse pauperes ; \\\hline
1.4.6 & La quarta es que son alabadores dessi mismos . \textbf{ despreçiando los otros ¶ } La quinta es que se tienen & Quarto sunt iactatores , \textbf{ alios despicientes . } Quinto autem reputant se dignos principari . \\\hline
1.4.6 & por tanto son assi despuestos \textbf{ por que auyendo las riquezas } cuydan & ideo sic disponuntur , \textbf{ quia habendo diuitias aliquas , } credunt se acquisiuisse omnia bona . \\\hline
1.4.6 & Por la qual cosa en los susco raçons se ensoƀueçen \textbf{ e se leuna tan cuydando } que son meiores & et pretium omnium aliorum ; \textbf{ quare in cordibus suis efficiuntur superbi et elati , } credentes omnibus excellentiores esse . \\\hline
1.4.6 & e por ende los ricos se leunatan en sus coraçones \textbf{ despreçiando alos otros } e cuydando que son mayores que ellos & contingit ut diuites in suis cordibus eleuentur , \textbf{ dispicientes alios , } et credentes se esse super eos , \\\hline
1.4.6 & despreçiando alos otros \textbf{ e cuydando que son mayores que ellos } por que veen & dispicientes alios , \textbf{ et credentes se esse super eos , } eo quod videant illos indigere bonis eorum . \\\hline
1.4.6 & alosricos que se han bien certa las cosas diuinales \textbf{ creyendo en alguna manera } que por las fadas & quia bene se habent circa diuina , \textbf{ tredentes aliqualiter per fata , } idest , \\\hline
1.4.7 & que si los poderosos fazen tuerto a \textbf{ alguons non ge lo fazen en pequanas cosas mas en grandes . Ca los poderosos estando en gerad sennorio } por que son en logar digno de grand honrra & si potentes iniuriantur , \textbf{ non iniuriantur in paruis , | sed in magnis . } Potentes enim existentes in Principatu , \\\hline
2.1.1 & de todas las otras uiandas \textbf{ por que nunca el omne estando señero puede conplir assy mismo } para auer viandas conuenibles & intelligendum est de cibariis aliis . \textbf{ numquam cum homo existens | solus sufficit sibi } ad habendum congrua cibaria , \\\hline
2.1.2 & mas pertenesçe al terçero libro \textbf{ en el qual diremos del gouernamiento dela çibdat . paresçe que auemos trispassado los terminos desta arte determinando en el capitulo passado algunas cosas } que pertenesçen ala comunidat dela çibdat . & sed ad tertium , \textbf{ ubi agitur de regimine ciuitatis : | videmur transgressi fuisse limites huius artis , } determinando in praecedenti capitulo aliqua pertinentia ad communitatem ciuitatis . \\\hline
2.1.2 & Et pues que assi es en el capitulo sobredicho auemos determinado dela conpannia humanal \textbf{ mostrando que es neçessario a lanr̃auida } por que por esta razon se muestra manifiestamente & In praecedenti ergo capitulo determinauimus de societate humana , \textbf{ ostendentes eam esse necessariam ad vitam nostram : } quia per hoc manifeste ostenditur \\\hline
2.1.2 & mas las casas son dichas partes de la çibdat e del regno \textbf{ por que faziendo uarrio siguese } que pueden fazer çibdat e regno . & ciuitatis vero domus partes esse dicuntur , \textbf{ quia constituendo vicum , } ex consequenti constituere possunt ciuitatem , et regnum . \\\hline
2.1.2 & Ca primeramente fue fecha vna casa \textbf{ e despues cresçiendo los fijos e las fijas } e por que non podieron por muchedunbre dellos morar todos en vna cala & quia primo facta fuit una aliqua domus : \textbf{ sed crescentibus filiis et filiabus , } et non valentibus praemultitudine habitare in domo illa , \\\hline
2.1.2 & de muchedunbre de mietos e de fijos . \textbf{ Ca assi commo dicho es de suso cresçiendo los nietos e los fijos de los fijos } por que non podien todos morar en vna casa & Nam , ut tangebatur , \textbf{ crescentibus collectaneis | idest nepotibus , et filiis , et filiorum filiis : } et non valentibus habitare in una domo , \\\hline
2.1.2 & e fizieron vn uarrio . \textbf{ Et assi yendo } actesçentandose el linage dellos & et constituere vicum . \textbf{ Sic procedente generatione ipsorum , } et ulterius augmentata multitudine , \\\hline
2.1.3 & e los regnos siruen al conplimiento dela uida del omne . \textbf{ or que non trabaiemos en vano fablando dela casa } conuiene de saber que la casa algunas uezes & quomodo domus et ciuitates deseruiunt ad sufficientiam humanae vitae . \textbf{ Ne laboremus in aequiuoco , | cum de domo loquimur , } sciendum quod domus nominari potest \\\hline
2.1.3 & e queremos aquellas cosas \textbf{ que son ordenadas ala fin entendiendo } e quariendo la fin . & Nam intendimus et volumus \textbf{ ea quae sunt ad finem , } intendendo et volendo finem , \\\hline
2.1.3 & que son ordenadas ala fin entendiendo \textbf{ e quariendo la fin . } Assi que la fin es primero quarida e entendida . & ea quae sunt ad finem , \textbf{ intendendo et volendo finem , } ita quod finis est primo volitus et intentus : \\\hline
2.1.3 & Assi que la fin es primero quarida e entendida . \textbf{ Mas en obrando e en execuçion dela obra es la manera contraria } Ca por la obra alcançamos la fin obrando aquellas cosas & ita quod finis est primo volitus et intentus : \textbf{ sed in operando , | et exequendo est econtrario . } Nam per opus consequitur finem , \\\hline
2.1.3 & Mas en obrando e en execuçion dela obra es la manera contraria \textbf{ Ca por la obra alcançamos la fin obrando aquellas cosas } que son ordenadas ala fin & et exequendo est econtrario . \textbf{ Nam per opus consequitur finem , } operando ea quae sunt ad finem , \\\hline
2.1.3 & Onde el philosofo en el primero libro delas politicas \textbf{ conparando la çibdat aluarrio } e ala casa dize & Unde et Philosophus 1 Politicorum \textbf{ comparans ciuitatem } ad vicum et domum , \\\hline
2.1.4 & e difine la comunidat dela casa \textbf{ diziendo } que la casa es comunidat & ø \\\hline
2.1.4 & e prouaremos que cada vna ꝑtetal dela casa es cosa natural . \textbf{ Pues que assi es finça de declarar en la difiniçion sobredichͣ } en qual manera la casa sea comunidat establesçida para cada dia . & talis pars est aliquid naturale . \textbf{ Restat ergo declarare | in descriptione praedicta , } quomodo domus sit communitas constituta in omnem diem . \\\hline
2.1.5 & Ca por mengua dela fuerça corporal \textbf{ ensegniendo las cosas neçessarias ala uida non pueden abastar } assimesmos & quia propter defectum fortitudinis corporalis , \textbf{ in exequendo necessaria ad vitam , | sibi ipsis non possunt sufficere . } Quare si dominus saluatur propter seruum , \\\hline
2.1.6 & por suçession e generaçion de los fiios el padre \textbf{ engendrando el fuo el fijo otro fijo } e assi biue por sienpre & sed quodammodo perpetuatur humana vita \textbf{ per successionem filiorum : } domus ubi est carentia prolis , \\\hline
2.1.7 & que cata la comunindat del padre et del fiio . \textbf{ Mas avn en determinando del gouernamiento coniugal } terminemos esta orden . & quod respicit communitatem patris et filii . \textbf{ In determinando autem de regimine coniugali , } hunc tenebimus ordinem : quia primo dicemus , \\\hline
2.1.7 & e aquales obras las de una ordenar . \textbf{ Mas en demostrando quales el ayuntamiento del uaron e dela muger } pmeramente nos conuiene de declarar en qual manera el mater moino es alguna cosa segunt natura . & et ad quae opera eas debeant ordinare . \textbf{ In ostendendo quidem quale sit ipsum coniugium , } primo declarandum occurrit , \\\hline
2.1.7 & Et pues que assi es que el philosofo en el octauo delas ethicas \textbf{ quariendo mostrar } qual es el amistança del uaron & Sciendum ergo quod Philosophus 8 Ethic’ volens \textbf{ ostendere } qualis amicitia sit viri ad uxorem , \\\hline
2.1.7 & que son de fazer fuera de casa . \textbf{ Mas las obras dela muger son en guardando las alfaias dela casa } e en obrando algunas cosas & quae sunt fienda extra domum : \textbf{ opera uero uxoris in conseruando suppellectilia , } uel in operando aliqua intra domum . \\\hline
2.1.7 & Mas las obras dela muger son en guardando las alfaias dela casa \textbf{ e en obrando algunas cosas } que son de dentro de casa . & opera uero uxoris in conseruando suppellectilia , \textbf{ uel in operando aliqua intra domum . } Ponentes ergo propria ad commune , \\\hline
2.1.7 & que son de dentro de casa . \textbf{ Et por ende poniendo ellos las cosas propreas al comun } assi commo quando la muger orden a las sus obras propreas al bien de su marido & uel in operando aliqua intra domum . \textbf{ Ponentes ergo propria ad commune , } ut cum uxor propria sua ordinat \\\hline
2.1.10 & e es cosa desconueible \textbf{ que la fenbra biuiendo su marido sea ayuntada a otro marido } por casamiento & et detestabile est etiam per coniugium foemina \textbf{ ( viuente viro suo ) | viro alio copulari , } magis detestabile est \\\hline
2.1.11 & ¶ \textbf{ ora uentra a cuydarie alguno guardando } que vna fenbra sea ayuntada avn uaron & quam incuria aliorum . \textbf{ Crederet forte aliquis } ( dum tamen una foemina per coniugium uni copuletur viro ) \\\hline
2.1.11 & Onde el philosofo en las politicas \textbf{ mouiendo se con razon natural saca algunas perssonas } que non son conuenibles a mater momo . & Unde et Philosophus 2 Polit’ \textbf{ sola ratione naturali ductus | exceptuat personas aliquas a contractione connubii : } nunquam enim fuit licitum alicui , \\\hline
2.1.11 & por que non sean muy destenpdos e sueltos \textbf{ mezclandose a quales quier muger sconuiene les de auer sus casamientos } por que sean pagados cada vno de su muger & ne sint nimis intemperati , \textbf{ quibuslibet foeminis se miscendo ; | expedit eis inire connubia , } ut una uxore contenti \\\hline
2.1.11 & e se ayan de tirar de los cuydados conuenibles \textbf{ e delas obras çiuiles dando se mucho a obras lux̉iosas . } pues que assi es tanto mas esto conuiene alos Re yes & et retrahantur a curis debitis \textbf{ et a ciuilibus operibus . } Tanto hoc ergo magis decet Reges , et Principes , \\\hline
2.1.12 & Ca mas deuen entender enla \textbf{ mugniassi commo paresçra en signiendo esta materia } adelante ala nobleza del linage & Magis autem attendendum est in coniuge \textbf{ ( ut in prosequendo patebit ) } honorabilitas generis , \\\hline
2.1.13 & que el philosofo en el primero libro de la rectorica \textbf{ contando los bienes delas iugers } dize & quod Philosophus 1 Rhetoricorum \textbf{ enumerando bona foeminarum , } ait , quod bona corporis foeminarum \\\hline
2.1.13 & las quales son de demandar enlas mugt̃s \textbf{ adelante parescra en signỉendo esta materia . } Mas que estos bienes del cuerpo & quae quaerenda sunt in coniugibus , \textbf{ multa in prosequendo patebit . } Quod autem haec bona corporis , \\\hline
2.1.13 & es que los casados guarden fialdat assi mesmos . \textbf{ La qual fialdat guardando escusan la fornicaçion } e avn el bien dela generaçion de los fijos . & ( ut quod coniuges sibi fidem seruent , \textbf{ quam seruando fornicationem vitant ) } et bonum prolis magis directe pertinere videntur ad coniugium , \\\hline
2.1.14 & Et por ende es dicho tal gouernamiento politico e çiuil por que es semeiado a aquel gouernamiento \textbf{ en el qual los çibdadanos llamando a su señor muestran le los pleitos } e las con diconnes & quia assimilatur illi regimini , \textbf{ quo ciues vocantes dominum , | ostendunt ei pacta } et conuentiones \\\hline
2.1.15 & tal es naturalmente barbaro e sieruo \textbf{ Por la qual cosa sient los barbaros han vna orden la muger } e el sieruo esto es por fallesçemiento de razon e de entendimiento & idem est esse natura barbarum et seruum . \textbf{ Quare si apud Barbaros eundem habent ordinem uxor et seruus , } hoc est propter rationis defectum , \\\hline
2.1.16 & mas en particular \textbf{ mostrando en qual manera todos los çibdadanos et mayormente los Reyes } e los prinçipes de uenular & Oportet ergo magis in particulari descendere , \textbf{ qualiter omnes ciues | et maxime Reges et Principes debent } uti copula coniugali . \\\hline
2.1.16 & si en el t pon del cresçer \textbf{ e cresçiendo el cuerpo vsaren de lux̉ia . } Por la qual cosa si estos males & si tempore augmenti \textbf{ et crescente corpore utantur venereis . } Quare si haec mala , \\\hline
2.1.17 & el abrego aduze muchedunbre de luuias \textbf{ e el estando puro } meiora se la conplision & Auster pluuiarum multitudinis adductiuus . \textbf{ Aere autem existente puro } melioratur complexio existentium in eo , \\\hline
2.1.18 & Mas sin freno varaian e parlan . \textbf{ Ca nos veemos que las mugers mas perseueran en varaiando } e en parlando que los uarones & nesciunt se moderare , \textbf{ sed sine fraeno garriunt et litigant . Videmus enim foeminas plus perseuerare } in litigando et garriendo , quam viri ; \\\hline
2.1.18 & Ca nos veemos que las mugers mas perseueran en varaiando \textbf{ e en parlando que los uarones } por que mas fallesçen en razon & sed sine fraeno garriunt et litigant . Videmus enim foeminas plus perseuerare \textbf{ in litigando et garriendo , quam viri ; } eo quod magis a ratione deficiant quam ipsi . \\\hline
2.1.19 & o por madronas de buen testimo \textbf{ non o fallando otras cautellas para esto . } por la qual cosa conuiene a todos los çibdadanos de gouernar a sus mugers assi . & vel per matronas boni testimonii , \textbf{ vel per cautelas alias adhibendo . } Quare decet omnes ciues \\\hline
2.1.20 & en apareiamiento conuenible \textbf{ dandol conueniblemente lo que ha menester . } Ca commo la mug̃r sea perssona . & in debito apparatu , \textbf{ ei necessaria debita tribuendo . } Nam cum uxor sit \\\hline
2.1.20 & ¶ Otrossi en tal manera deuemos beuir con ella \textbf{ sperando mientes } que en otra manera son de enssennar las sabias & Rursus sic conuersandum est cum eis , \textbf{ quod aliter instruendae sunt prudentes , } aliter fatuae . \\\hline
2.1.20 & Et pues que assi es conuiene a cada vno de los uarones \textbf{ penssando el su estado propreo } e catadas las condiconnes delas perssonas mostrar a sus & Decet ergo quoslibet viros , \textbf{ considerato proprio statu , } et inspectis conditionibus personarum , \\\hline
2.1.21 & en el primero libro de las rectorica \textbf{ do dize fablando de los laçedemonios } que son gente de greçia & ostendit Philosophus 1 Rhet’ \textbf{ qui loquens de Lacedaemoniis , } ait , eos esse infelices secundum dimidium , \\\hline
2.1.21 & Et en los otros conponimientos del cuerpo \textbf{ los quales conponimientos pensando el estado propio } e las condiconnes delas perssonas son conueinbles e honestas & et in aliis ornamentis , \textbf{ quae si considerato proprio statu } et conditionibus personarum \\\hline
2.1.21 & por que no fallezcan por ꝑeza notablemente en su estado \textbf{ dellas non tomando conponimiento } qual conuiene a su cuerpo & Quinto , ne sint negligentes , \textbf{ ne totaliter infra eorum statum propter pigritiam deficiant erga ornamentum corporis . } Sexto , ne ex vilitate habitus sint superstitiosae , \\\hline
2.1.23 & uarones̃ mas ayna viene a su conplimiento . \textbf{ Et por ende estando todas las } otrascondiconnes eguales del omne et dela muger & citius venit ad suum complementum . \textbf{ Ceteris ergo paribus } si quis statim operari deberet , \\\hline
2.1.24 & que son amadas de sus maridos . \textbf{ por ende desseando alguna uana gloria } e alguna alabança de ligero descubren los secretos de sus maridos & quod a suis maritis diligantur , \textbf{ appetentes quandam inanem gloriam , } et quandam laudem , \\\hline
2.2.1 & assi commo de aquello de que deuemos auer mayor cuydado . \textbf{ Et en determinando del gouernamiento de los fijos } primeramente queremos mostrar & tanquam de eo circa quod esse debet amplior cura . \textbf{ In determinando quidem de regimine filiorum , } primo ostendere uolumus , \\\hline
2.2.4 & por amor \textbf{ mas quando van cresçiendo } e passando el tp̃o pueden departir & non statim per amorem efficiuntur ad illos ; \textbf{ sed per processum temporis , } quando possunt discernere parentes ab aliis , \\\hline
2.2.4 & mas quando van cresçiendo \textbf{ e passando el tp̃o pueden departir } e conosçer sus padres entre los otros & sed per processum temporis , \textbf{ quando possunt discernere parentes ab aliis , } incipiunt eos diligere . \\\hline
2.2.5 & Onde el philosofo en el segundo libro dela \textbf{ methafisica quariendo prouar } que la costunbre es de grand fuerça dize assi . & Unde et Philosophus 2 Meta’ \textbf{ volens probare } consuetudinem esse magnae efficaciae , \\\hline
2.2.5 & Tu puedes ver \textbf{ quanto faze la costunbre cuydando en las leyes de los moros o de los x̉anos o de otras gentes } quales quiera que sean & quod consuetum est , \textbf{ leges ostendunt , } in quibus fabulas et apologos \\\hline
2.2.5 & e estaremos ante la su faz \textbf{ dando razon de todos nuestros fechos . } assi que aquellos que bien fezieron yran ala uida perdurable . & et stabimus ante tribunal eius , \textbf{ reddituri de factis propriis rationem . } Ita quod qui bona egerunt , \\\hline
2.2.6 & por que la podamos fazer uenir al medio . \textbf{ En essa misma manera nos en fuyendo delas cosas } delectabłs deuemos tris passar & ut possit ad medium redire : \textbf{ sic et nos in fugiendo delectabilia , } debemus ultra medium nos facere , \\\hline
2.2.6 & por ende deuemos comneçar luego enla moçedat \textbf{ por que dexando las } locanias siguamos bueans costunbres . & est ab ipsa infantia inchoandum , \textbf{ ut relinquentes lasciuias sequamur bonos mores , } nec est ulterius differendum . \\\hline
2.2.8 & Ca si nos non sopiessemos la manera de argumentar e de razonar \textbf{ podriemos pecar en argumentando e en razonando . } Et por ende podriamos ser engannados & Nam nisi modum arguendi sciremus , \textbf{ possemus in arguendo peccare , } et per consequens decipi . \\\hline
2.2.8 & que les otorguen cosas \textbf{ delectabło sin daño . } por ende segunt que dize este mismo philosofo & dignum est quod ordinentur \textbf{ ad delectationes innocuas : } quare ( secundum eundem Philosophum ) \\\hline
2.2.8 & e de los nobles \textbf{ que non se entremetiendo de lauores } nin de otras artes mecanicas estarian oçiosos e uagarosos & Maxime autem hoc decens est filiis liberorum et nobilium , \textbf{ qui non vacantes moechanicis artibus , } remanent ociosi , \\\hline
2.2.8 & mas assi commo paresçra en \textbf{ signiendo esta materia los fijos de los nobles } e mayormente los fijos de los Reyes & et necessariae filiis liberorum et nobilium . \textbf{ Immo ( ut in prosequendo patebit ) filii nobilium , } et maxime filii Regum et Principum , \\\hline
2.2.8 & por que aquellas cosas delas quales es la sçiençia politica \textbf{ digen los legistas contando } e non demostrando las & sic Legistae , quia ea de quibus est politica , \textbf{ dicunt narratiue et sine ratione , } appellari possunt idiotae politici . \\\hline
2.2.8 & digen los legistas contando \textbf{ e non demostrando las } por razon por ende pueden ser llamados nesçios politicos . & sic Legistae , quia ea de quibus est politica , \textbf{ dicunt narratiue et sine ratione , } appellari possunt idiotae politici . \\\hline
2.2.8 & que los que fablan \textbf{ non dando razon de sus dichos tanto estos tales son mas honrrados que los otros . } Et avn desto puede paresçer & honorabiliores sunt loquentibus \textbf{ et non reddentibus causam dicti : | tanto tales honorobiliores sunt illis . } Ex hoc autem apparere potest , \\\hline
2.2.9 & mas assi commo paresçe por el pho en el segundo dela \textbf{ methafisica ninguon non abasta assi en estudiando enlas sçiençias speculatians } mas sienpre los philosofos postmos ouieron ayudas & sic et forte multo magis , \textbf{ ut patet per Philosophum 2 Metaphysicae , nullus sibi sufficit in speculando , } sed semper posteriores Philosophi \\\hline
2.2.9 & Ca assi commo dixiemos en el primer libro \textbf{ assi commo en conosçiendo } e en & quia ut in primo libro tetigimus , \textbf{ sicut in cognoscendo et speculando } est adhibenda cautela , \\\hline
2.2.9 & e en \textbf{ contenplando es de tomar cautela } por que las cosas falssas non sean mezcladas alas uerdaderas . & sicut in cognoscendo et speculando \textbf{ est adhibenda cautela , } ne falsa admisceantur veris : \\\hline
2.2.9 & por si \textbf{ e catando ala paresçençia de fuera paresçen bueans . } assi algunas cosas paresçen uerdaderas & quaedam enim \textbf{ secundum se mala superficietenus considerata , | apparent bona : } sicut quaedam falsa apparent vera . \\\hline
2.2.9 & deue ser sabio \textbf{ aponiendo les los bienes } sin mesclamiento de ninguons males & Sic qui vult iuuenes dirigere debet esse , \textbf{ cautus proponens eis bona } sine admixtione malorum . \\\hline
2.2.10 & e conosçen poco \textbf{ catando alas pocas cosas } que saben ayna pronunçian & quia ergo pauca cognoscunt , \textbf{ ad pauca respicientes enunciant facile , } idest enunciant cito et debiliter , \\\hline
2.2.10 & Enpero si se acostunbraten a responder con penssamiento \textbf{ e cuydando ante en las palabras } que han de dezer & ut praemeditati respondeant , \textbf{ et ut prae cogitent | in sermonibus proferendis , } per successionem temporis disponentur \\\hline
2.2.12 & es çegamiento de la razon e del entendimiento . \textbf{ Ca subiendo las fumosidades del vino } ala cabesça turbasse el meollo . & est depressio rationis . \textbf{ Nam ascendentibus fumositatibus vini ad caput , } turbatur cerebrum : \\\hline
2.2.13 & e mesurados enel beuer \textbf{ e tenpdos en la lux̉ia en tomando su casamiento en hedat conuenible } e commo se deuen auer & sobrii in potu , \textbf{ temperati in venereis , | contrahendo coniugium in aetate debita , } et modeste se habere \\\hline
2.2.13 & e non entiende en algunas delectaçiones conuenibles \textbf{ luego comiença a andar vagando cuydando enlas cosas desconueibles . } Onde el philosofo enłviiij libro delas politicas dize & et non intendit aliquibus delectationibus licitis , \textbf{ statim incipit vagari cogitando de illicitis : } unde Philosophus 8 Polit’ ait , \\\hline
2.2.13 & e algunos solazes en sus cuydados . \textbf{ assi que en esto resçibiendo alguno folgua a puedan mas trabaiar para alcançar su fin . } Onde el philosofo enłviij̊ delas politicas dize & interponere suis curis , \textbf{ ut ex hoc aliquam requiem recipientes , | magis possint laborare in consecutione finis . } Unde et Philosophus 8 Politicorum ait , \\\hline
2.2.13 & la qual cosa se puede ver \textbf{ departiendo entre las conplissiones e los tienpos e las hedades . } Ca los que han las conplissiones espessas & Quod videri habet , \textbf{ distinguendo inter complexiones , | tempora , et aetates . } Nam habentes complexiones magis depressas et minus porosas , \\\hline
2.2.16 & que fazen fazen las mucho \textbf{ mas que deuen asi que quando aman am̃a much̃ . Etrͣndo } comiençan de trebeiar trebeian much̃ . & et omnia faciunt valde , \textbf{ ita quod cum amant nimis amant , } cum incipiunt ludere nimis ludunt , \\\hline
2.2.17 & Et despues enł . xviij ̊anero \textbf{ estando enssennados los mocos en trabaio dela lucha e del caualgar } e en los trabaios & similem exercitationi bellicae ; \textbf{ ut postea in quartodecimo anno instructi in luctatiua et in equitativa , } et in aliis quae ad militiam requiruntur , \\\hline
2.2.17 & ¶ \textbf{ Lo segundo pecan en signiendo los desseos de luxia } por que estonçe comiençan mas & et dedignantur aliis esse subiecti . \textbf{ Secundo delinquunt in prosequendo venerea , } quia tunc incipiunt ardentius \\\hline
2.2.18 & e generalmente todo sennor del pueblo \textbf{ commo quier que en lidiando } e en tomando armas non vala mas & et uniuersaliter omnis dominator populi , \textbf{ licet in bellando } et in assumendo arma \\\hline
2.2.18 & commo quier que en lidiando \textbf{ e en tomando armas non vala mas } que vn ome e alguas uegadas menos que vno otro omne . & licet in bellando \textbf{ et in assumendo arma } quasi non plus valeat quam unus homo , \\\hline
2.2.18 & que deuen gouernar los otros de escusar la ꝑeza \textbf{ e el cuydado desconueinble estudiando enlas sçiençias morales } cuydando mucho a menudo enlas bueans costunbres del regno & vitare inertiam et solicitudinem illicitam , \textbf{ vacando moralibus scientiis , } recogitando frequenter bonas consuetudines regni , \\\hline
2.2.18 & e el cuydado desconueinble estudiando enlas sçiençias morales \textbf{ cuydando mucho a menudo enlas bueans costunbres del regno } e oyendo mucħa menudo los fechos & vacando moralibus scientiis , \textbf{ recogitando frequenter bonas consuetudines regni , } audiendo saepius acta praedecessorum \\\hline
2.2.18 & cuydando mucho a menudo enlas bueans costunbres del regno \textbf{ e oyendo mucħa menudo los fechos } de los que ante passaron & recogitando frequenter bonas consuetudines regni , \textbf{ audiendo saepius acta praedecessorum } bene regentium regnum . \\\hline
2.2.18 & de los que ante passaron \textbf{ e bien gouernaron el regno . Et pues que assi es dando se los reyes a sabiduria } e abueans costunbres pueden ellos & audiendo saepius acta praedecessorum \textbf{ bene regentium regnum . | Hoc ergo modo , } videlicet , vacando actibus prudentiae , \\\hline
2.2.19 & que non anden por las placas \textbf{ nin anden paresçiendo nin corriendo ¶ } La segunda razon para mostrar & et prohibendae sunt \textbf{ a circuitu et discursu . } Secunda via ad inuestigandum \\\hline
2.2.19 & nin entre las gentes . \textbf{ por la qual cosa las moças uagando } e andando por la tierra & est non assuescere eas inter gentes . \textbf{ Quare cum puellae circemeundo , } et vagando per patriam \\\hline
2.2.19 & por la qual cosa las moças uagando \textbf{ e andando por la tierra } acostunbran se auer los omes & Quare cum puellae circemeundo , \textbf{ et vagando per patriam } assuescant virorum aspectibus , \\\hline
2.2.19 & que non corran \textbf{ nin anden uagando allende } nin a quande por que non se fagan desuergoncadas & Decens ergo est cohibere puellas \textbf{ a discursu et euagatione , } ne fiant inuerecundae , \\\hline
2.2.20 & conueinbł stanto la su uoluntad \textbf{ andauagando aquende } e allende çerca obras & statim cum quis non dat se licitis exercitiis , \textbf{ vagatur eius mens } circa illicitas occupationes , \\\hline
2.2.20 & que los omes \textbf{ e quanto pensando las cosas } que les non conuienen & quanto molliores sunt illis , \textbf{ et quanto cogitando illicita facilius trahuntur , } ut ea ( si adsit commoditas ) opere compleant . \\\hline
2.2.20 & non estudiesse baldia nin oçiosa mas muchͣs \textbf{ e muchos uegadas tomando el libro s } e trabaiasse en rezar sus leçiones o sus salmos o sus oronnes . & non vacaret ociose , \textbf{ sed saepe saepius librum arripiens , } se lectionibus occuparet . \\\hline
2.3.1 & Et pues que assi es el arte del gouernamiento dela casa maguera . \textbf{ largamente fablando puede ser dichͣ arte } mas propreamente fablando deue ser dicha sabiduria . & Ars ergo gubernationionis domus licet \textbf{ largo modo possit dici ars , } proprie tamen prudentia dici debet . \\\hline
2.3.1 & largamente fablando puede ser dichͣ arte \textbf{ mas propreamente fablando deue ser dicha sabiduria . } Et pues que assi ese ponestas e prouadas las dos razones & largo modo possit dici ars , \textbf{ proprie tamen prudentia dici debet . } Praemissis ergo duabus rationibus , \\\hline
2.3.3 & que ayan muchos ofiçiales \textbf{ e much ssiruient s̃ Et pues que assi es } por que non solamente la persona del Rey o del prinçipe mas avn & In domibus ergo Regum et Principum \textbf{ oportet multos abundare ministros , } ut ergo non solum personas Regis et Principis , \\\hline
2.3.5 & assi commo son las aues la natura \textbf{ assi ordeno poniendo en los hueuos blanco e bermeio } assi que del blanco se engendra el aue & sic natura ordinauit , \textbf{ ponens in ipsis ouis album et rubeum , } ita quod ex albo generatur auis , \\\hline
2.3.6 & que diz boesçio es mas fermoso e reluze \textbf{ mas claramente enpero las cosas estando } assi como agoraes tan cosa aprouechosa es ala çibdat que los çibdadanos . & pulchrius elucescit . \textbf{ Rebus tamen stantibus ut nunc , } utile est ciuitati ciues gaudere possessionibus propriis , \\\hline
2.3.6 & sinplemente es proprouechosa en algun caso . \textbf{ Et pues que assi es estando las cosas } assi commo dicho es pro prouechosa cosa es ala çibdat & quod est utile in casu . \textbf{ In rebus ergo sic se habentibus , } utile est ciuitati ciues \\\hline
2.3.6 & possessionspropreas \textbf{ por que non auiendo cuy dado çerca las cosas comunes dela casa } uernien los omes amengua & habere possessiones proprias , \textbf{ ne propter ignauiam circa communia , } domus ciuium patiantur inopiam . \\\hline
2.3.6 & aquello qual es mandado \textbf{ elperando que el otro cunplira aquello que a el es mandado . } Por la qual cosa conuiene & ne faciat quod mandatur , \textbf{ sperans alium implere | quod iubetur ; } propter quod oportet rem illam \\\hline
2.3.7 & por quela natura engendro tales cosas \textbf{ ordenando la sal vso del omne } ca sienpre las cosas & quia natura talia produxit , \textbf{ ordinans ea ad usum hominis : } semper enim imperfecta \\\hline
2.3.7 & e alas o trisaian lias \textbf{ ca fablando omne uerdaderamente } por si los omes pueden prender & et ad alia animalia est iustum bellum ; \textbf{ per se enim loquendo homines iuste possunt } talia facere , \\\hline
2.3.7 & fazen o fizieron algua cosa desagnisada contra ellos . \textbf{ ¶ Et pues que assi es el ome peca en faziendo mal al omne } enpero fablando sinplemente & vel forefecerunt in ipsos . \textbf{ Delinquit ergo homo offendendo hominem : } per se tamen loquendo , \\\hline
2.3.7 & ¶ Et pues que assi es el ome peca en faziendo mal al omne \textbf{ enpero fablando sinplemente } e por ssi non pecaen faziendo mal alas bestias & Delinquit ergo homo offendendo hominem : \textbf{ per se tamen loquendo , } non delinquit offendendo bestias . \\\hline
2.3.7 & enpero fablando sinplemente \textbf{ e por ssi non pecaen faziendo mal alas bestias } Et si en faziendo mal alas bestias es pecado & per se tamen loquendo , \textbf{ non delinquit offendendo bestias . } Si autem in offensione bestiarum est delictum , \\\hline
2.3.7 & e por ssi non pecaen faziendo mal alas bestias \textbf{ Et si en faziendo mal alas bestias es pecado } esto es & non delinquit offendendo bestias . \textbf{ Si autem in offensione bestiarum est delictum , } hoc est quasi per accidens , \\\hline
2.3.9 & o alguna otra señal publica \textbf{ por que catando aquella señal fuesse sabido de quanto peso } e de quanto ualor era aquel metal & vel aliquod aliud signum publicum , \textbf{ per cuius inspectionem sciretur quanti ponderis , } et quanti valoris esset metallum illud . Hoc ergo modo inuentus fuit denarius et numisma , \\\hline
2.3.10 & Mas despues por espiençia fue fechͣ arti fiçial \textbf{ por que cada moneda fablando propriamente } mas uale en su regno & sed deinde per experientiam iam est artificialis effecta ; \textbf{ quodlibet enim numisma | ( per se loquendo ) } plus valet in propria regione . \\\hline
2.3.10 & que cuestan vna meaia . \textbf{ Et tornando los en } massavalen despues tres o quatro meaias & ø \\\hline
2.3.11 & del qual nobre el philosofo la rephede en el primero libro delas politicas \textbf{ diziendo que es contra natura } ca parir e engendrar e amuchiguar se las cosas en si mismas & ex quo nomine arguit Philosophus 1 Polit’ \textbf{ eam contra naturam esse . } Nam parere , et generare , \\\hline
2.3.11 & enpero en alguas cosas nunca se puede otorgar el uso dellas sinon \textbf{ otorgandose la sustançia } ca non se puede partir el uso dela sustançia . & separari potest \textbf{ a concessione substantiae . } In quibuscunque igitur potest \\\hline
2.3.11 & Et por ende en quales se quier cosas \textbf{ en que se puede otorgar el uso dela cosa non se otorgando la sustançia della } ally en aquella cosa se puede tomar loguer o alquiler della puesto & In quibuscunque igitur potest \textbf{ concedi usus rei | absque eo quod concedatur eius substantia , } potest inde accipi pensio , \\\hline
2.3.11 & conueiblemente lo pue de fazer \textbf{ non faziendo tuerto a ninguno . } Et por ende assi faziendo non robanada & licite potest , \textbf{ et nulli iniuriatur . } Nihil ergo rapit , \\\hline
2.3.11 & non faziendo tuerto a ninguno . \textbf{ Et por ende assi faziendo non robanada } nin toma uso ageno si retiniendo en ssi el señorio dela casa vede la morada & et nulli iniuriatur . \textbf{ Nihil ergo rapit , | et nihil usurpat , } si retinens sibi dominium domus , \\\hline
2.3.11 & Et por ende assi faziendo non robanada \textbf{ nin toma uso ageno si retiniendo en ssi el señorio dela casa vede la morada } e el uso della mas en los dineros non es & et nihil usurpat , \textbf{ si retinens sibi dominium domus , | uendit inhabitationem , } et usum eius . \\\hline
2.3.11 & otorgan la sustançia dela \textbf{ otorgando la sustançia } dende adelante non parte nesçe a ellos el uso della . & concedit substantiam eius : \textbf{ concedendo uero substantiam , } non ulterius spectat \\\hline
2.3.12 & Et qual cuydado deuen tomar çerca los arboles \textbf{ que plantan propusiemos de pasar todas estas cosas en silençio e callando } por que paresçe & et qualis cura circa arbores sit gerenda , \textbf{ disposuimus silentio pertransire , } eo quod alii de talibus sufficienter tradidisse videntur . \\\hline
2.3.12 & e le denostassen sus amigos \textbf{ diziendol que por que se daua tanto ala ph̃ia } e aquel aprouechaua suph̃ia pues siengͤ biue en pobreza e en mengua . & et improperaretur sibi a multis cur philosopharetur , \textbf{ et ad quid valeret Philosophia sua , } cum semper in egestate viueret . \\\hline
2.3.12 & que son de loar segunt las quales acresçenta una sus rentas \textbf{ conueiblemente non tomando los bienes de los otros por fuerça . } Et pues que assi es el cuydado de los fechos particulares & secundum quas capiant licitos redditus , \textbf{ non usurpando aliorum bona . } Consideratio ergo gestorum particularium , \\\hline
2.3.16 & Ca muchͣs uezes cada vno de aquellos seruientes \textbf{ menospreçia aquel seruiçio cuydando que el otro lo fara . } Por que do quier que ay muchedunbre alli es confusion & nam saepe quilibet ministrantium huiusmodi ministerium negligit , \textbf{ credens quod alius exequatur illud : } ubicunque enim est multitudo , \\\hline
2.3.16 & que dinemos alli este \textbf{ tal deue ser tendo por sabio } e esto segunt & et alia quae ibi diximus , \textbf{ prudens est reputandus : } et secundum magis et minus , \\\hline
2.3.18 & donde vino la curialidat e la cortesia \textbf{ ca propreamente fablando non es dichͣ corte } si non casa de grandes e de nobles & Ex hoc ergo curialitas venisse videtur . \textbf{ Nam curia proprie non dicitur } nisi domus nobilium et magnorum : \\\hline
2.3.18 & que fazen obras de uirtudes \textbf{ partiendo los sus bienes alos otros } e esto non lo fazen & Sunt enim multi facientes opera virtutum \textbf{ ut bona sua aliis largientes , } non agentes hoc quia eis placeat expendere ; \\\hline
2.3.19 & para guardar su poliçia conueniblemente \textbf{ assi conuiene alos sermient } s̃ de los sennores de ser curiales & ut debitam politiam seruent \textbf{ esse iustos legales , } sic decet ministros dominorum \\\hline
2.3.19 & e atal non deuen descobrir sus poridades . \textbf{ Mas fablando elpho destas cosas } cerca la fin del prim̃o libro & et ab usu rationis deficiat . \textbf{ De his autem loquens Philosophus } circa finem primi Politicorum ait , \\\hline
2.3.20 & e so cuento propusiemos de passar las en silençio \textbf{ e callando las } e poniendo fin a este segundo libro & tamen quia non omnia particularia sub narratione cadunt , \textbf{ proponimus ea silentio pertransire , } imponentes finem huic secundo Libro , \\\hline
2.3.20 & e callando las \textbf{ e poniendo fin a este segundo libro } en que diemos arte & proponimus ea silentio pertransire , \textbf{ imponentes finem huic secundo Libro , } in quo de regimine domestico \\\hline
2.3.20 & segunt la manera \textbf{ delanr̃a sçiençia del gouernamiento dela casa es fortando nos } en la ayuda de aquel & a quo omnis bonitas , \textbf{ et sufficientia habet esse . | Secundi libri de regimine Principum , } in quo tractatur de regimine domus , \\\hline
3.1.2 & en quanto es omne . \textbf{ Et por ende estendiendo el beuir politico } segunt alguas leys de loar & ut homo est . \textbf{ Ostendendo ergo viuere politicum secundum aliquas leges } et secundum aliquas laudabiles ordinationes , \\\hline
3.1.2 & assi commo se prueua adelante . \textbf{ Et por ende fablando del beuir } assi commo de beuir & ut infra patebit . \textbf{ Ipsum ergo viuere } ( loquendo de viuere ut homo ) \\\hline
3.1.2 & e auer abastamiento en la uida . \textbf{ Et veyendo que por grant cuydado } que ouiessen non podrian abastar assi mesmos en la uida & et habere sufficientia in vita . \textbf{ Videntes autem quod solitarie } non poterant sibi in vita sufficere , \\\hline
3.1.2 & e los omes cataron \textbf{ e cuydaron mas sotilmente vevendo } que non era asaz auer cunplimiento enla uida & Constituta autem iam ciuitate , \textbf{ et homines perspicatius intuentes et videntes } quod non satis est \\\hline
3.1.5 & que por mercadores \textbf{ e por acarreo trayendo cargas } e lo que es mester & sed sufficit ciuitatem sic esse sitam , quod per mercationes , \textbf{ et ponderis portatiuam , } et per humanam industriam faciliter habere possit sufficientia vitae . \\\hline
3.1.6 & e esta bescien para ssi algua çibdat \textbf{ en la qual morando en vno podiessen auer } mas conplidamente aquellas cosas & constituentes sibi ciuitatem aliqua , \textbf{ in qua simul morantes habere possent sufficientius } quae requiruntur ad indigentiam vitae . \\\hline
3.1.6 & que estudiesse so vn Rey \textbf{ temiendo el poderio de los enemigos } e cada vna desta dos maneras esnatraal & ut sub uno rege existerent , \textbf{ timentes inimicorum potentiam . } Uterque autem horum modorum est naturalis : \\\hline
3.1.6 & a establesçer çibdat e regno \textbf{ por que si por generaçion cresçiendo los fijos } e los mietos en vna casa & quia homines naturalem habent impetum ad constituendam ciuitatem et regnum . \textbf{ Si enim per generationem } ex crescentibus filiis et nepotibus in eadem domo , \\\hline
3.1.6 & e los mietos en vna casa \textbf{ e non podiendo morar en vno fagan para ssi muchos casas } e establescan vn uarrio . & ex crescentibus filiis et nepotibus in eadem domo , \textbf{ et non valentibus simul habitare , | faciant sibi plures domos , } et constituant sibi vicum ; \\\hline
3.1.6 & e establescan vn uarrio . \textbf{ Et despues mas adelante cresçiendo } e non podiendo morar en vn uarrio fagan mas adelante & et constituant sibi vicum ; \textbf{ et ulterius excrescentibus } et non valentibus habitare in uno vico , \\\hline
3.1.6 & Et despues mas adelante cresçiendo \textbf{ e non podiendo morar en vn uarrio fagan mas adelante } para si much suarrios & et ulterius excrescentibus \textbf{ et non valentibus habitare in uno vico , } faciant sibi vicos plures , \\\hline
3.1.6 & e establescandellos çibdat . \textbf{ Et despues mas adelante cresçiendo } e noo podiendo morar en vna çibdat & et constituant ciuitatem : \textbf{ amplius autem ipsis excrescentibus } et non valentibus habitare in una ciuitate , \\\hline
3.1.6 & Et despues mas adelante cresçiendo \textbf{ e noo podiendo morar en vna çibdat } fagan para si muchͣs çibdades & amplius autem ipsis excrescentibus \textbf{ et non valentibus habitare in una ciuitate , } fabricent sibi ciuitates plures , \\\hline
3.1.7 & as socrates commo ouiesse phophado luengo tienpo çerca las naturas delas cosas \textbf{ ueyendo muy grant guaueza cerca la sciençia natural } assi commo cuenta el pho & circa naturas rerum , \textbf{ videns circa naturalem scientiam | magnam difficultatem esse , } ut narrat Philosophus in Metaphysica sua , \\\hline
3.1.9 & dano realmente crie el cuerpo del otro . \textbf{ Et pues que assi es estando las possessiones comunes conuernia a cada vno } de departir aquellas cosas & nutriat corpus alterius . \textbf{ Existentibus ergo possessionibus communibus oporteret | cuilibet distribui } quae requiruntur \\\hline
3.1.10 & que aquella muchedunbre sea plazible \textbf{ e amada non estando el amost de los çibdadanos alos moços } non se puede auer & non posset reddere placibilem et dilectam . \textbf{ Sed non existente dilectione ciuium ad pueros , } non habebitur eorum cura debita : \\\hline
3.1.10 & e los padres con sus fijas . \textbf{ Empero socrates quariendo escusar este mal dix̉o } que al prinçipe dela çibdat pertenesçia de auer cuydado e acuçia & et patres filias . \textbf{ Socrates volens hoc inconueniens vitare , | dixit , } quod spectabat ad Principem ciuitatis \\\hline
3.1.11 & e de beuir conellos \textbf{ non los podiendo escusar } que si entre los sennoron e los sus sieruos & ad illos multa colloquia , \textbf{ et diu conuersari cum illis . } Quare si inter dominos et famulos quos habent \\\hline
3.1.11 & segunt uirtud de franqueza \textbf{ por que los çibdadanos entre ssi deuen ser francos ꝑtiendo sus bienes entre ssi . } Onde entre los guaegos & Expedit autem talia esse communia secundum liberalitatem : \textbf{ quia cives inter se debent liberales esse , | communicando sibi invicem propria bona . } Unde et apud Lacedaemones , \\\hline
3.1.12 & e de grandes coraçones temen \textbf{ quando veen que los temerosos van temiendo . } Et por ende por que los lidiadores non se enflaquezcan en las batallas & viriles etiam et animosi trepidant \textbf{ videntes timidos trepidare : } ne igitur reddantur bellantes pusillanimes , \\\hline
3.1.13 & e mudado los maestradgos e los prinçipados \textbf{ e partiendo los a departidas } ꝑsonas faze a buen estado e paçifico dela çibdat & Mutare autem aliquando magistratus et principatus , \textbf{ et distribuere eos diuersis personis , } ut innuit Philosophus \\\hline
3.1.13 & en el segundo libro delas politicas \textbf{ ca si menospreçiando alos vnos sienpre los otros fueren puestos en los ofiçios } e en las dignidades los otros viendo se menospreçiados le una tan se en vandos & et pacificum statum ciuium . \textbf{ Nam si spretis aliis semper iidem in magistratibus et praeposituris praeficiantur , } alii videntes se esse despectos \\\hline
3.1.13 & ca si menospreçiando alos vnos sienpre los otros fueren puestos en los ofiçios \textbf{ e en las dignidades los otros viendo se menospreçiados le una tan se en vandos } e en peleas & Nam si spretis aliis semper iidem in magistratibus et praeposituris praeficiantur , \textbf{ alii videntes se esse despectos } ad seditionem consurgunt , \\\hline
3.1.15 & Si quisieremos entender los dichos de socrates \textbf{ non assi conmo suena las palabras podremos entender la su opinion diziendo } que non es cosa que pueda ser & intelligere dicta Socratica , \textbf{ saluare poterimus positionem eius . } Omnia enim esse ciuibus communia \\\hline
3.1.15 & assi que estonçe seria la çibdat muy buena \textbf{ quando los çibdadanos amandose } e quariendose muy bien fuessen & ut quod tunc esset ciuitas optima , \textbf{ quando ciues se amando et diligendo maxime unirentur . } Sic ergo exposita mente Socratis \\\hline
3.1.15 & Et pues que assi es \textbf{ assi es poniendo la entençio de socrates dela comunidat delas cosas } e dela vnidat de los çibdadanos & quando ciues se amando et diligendo maxime unirentur . \textbf{ Sic ergo exposita mente Socratis | de communitate rerum } et de unitate ciuium , \\\hline
3.1.15 & que deuian ser ordena a obras de batalla \textbf{ puedese saluar non entendiendo esto sinplemente } mas en algun caso . & quod ordinandae essent ad opera bellica . \textbf{ Saluari potest non intelligendo hoc simpliciter , } sed in casu . \\\hline
3.1.15 & çerca las partes de ytalia \textbf{ que los uarones desmanparando la çibdat } e commo salliessen della fue cometida la çibdat de los sus enemigos dellos & Multotiens autem circa partes Italiae hoc contigit , \textbf{ quod viris deserentibus ciuitatem , } et euntibus in exercitium supra ciuitatem aliquam , \\\hline
3.1.15 & por la qual cosa conuenio alas mugers \textbf{ por mengua de los çibdadanos de defender la çibdat mas lo que enandio adelante diziendo } que sienpre conuenia & et mulieres \textbf{ propter penuriam ciuium defendere ciuitatem . | Quod autem ulterius addebat , } quod semper oportet \\\hline
3.1.15 & en menestrales e en labradores \textbf{ e en batalladores quariendo } que alo menos la çibdat ouiesse mil ł batalladores & Quod autem ciuitatem diuidebat \textbf{ in agricolas , artifices , et bellatores , } volens ciuitatem \\\hline
3.1.16 & que se entremi tio del ordenamiento dela çibdat \textbf{ establesçiendo en qual manera se podria ordenas muy bien la poliçia e la çibdat } ca dizia & 2 Politicorum intromisit se de ordine ciuitatis , \textbf{ statuens quomodo posset | optime politia ordinari . } Dicebat autem , \\\hline
3.1.16 & mas la çibdat ya establesçida los çibdadanos \textbf{ auiendo ya sus possessiones } deseguales mayor trabaio delas traer despues a ygualdat & Sed ciuitate iam constituta , \textbf{ et ciuibus iam habentibus possessiones inaequales , } difficilius erat hoc adaequalitatem reducere . \\\hline
3.1.16 & que se dan en los casamientos \textbf{ ca establesçiendo que los pobrescasen con las ricas } e en casamiento resçiban grandes arras & ad aequalitatem mediantibus dotibus statuendo \textbf{ quod pauperes contrahant cum diuitibus : } et in contrahendo accipiant dotes , \\\hline
3.1.16 & e ellos que non den nada de lo suyo . \textbf{ Por ende los pobres resçibiendo grandes artas de los ricos } podria se ygualar alos ricos en las possessiones . & et in contrahendo accipiant dotes , \textbf{ et non dent pauperes ergo accipiendo magnas dotes } a diuitibus poterunt \\\hline
3.1.16 & por que los çibdadanos de buenamente toman las polsessiones \textbf{ diziendo esto es mio . } Et por ende non solamente se le una tan lides e pleitos & dicente , \textbf{ Hoc est meum , } non solum insurgunt lites et placita ; \\\hline
3.1.17 & yguales \textbf{ si non poniendo } que tantos fijos aya el vn çibdadano commo el otro & possessiones aequatas , \textbf{ nisi totidem filii procedant ab uno ciuium , } quot procedunt ab alio : \\\hline
3.1.17 & yguales alos ricos enlas possessiones \textbf{ tomando grandes dones } e arras los pobres de los ricos & pauperes aequari diuitibus in possessionibus , \textbf{ accipiendo magnas dotes ab eis , } et non dando dotes illis , contingit multotiens diuites fieri pauperes , \\\hline
3.1.17 & e arras los pobres de los ricos \textbf{ e non las dando } assy commo dize felleas contesçeria muchos uegadas & accipiendo magnas dotes ab eis , \textbf{ et non dando dotes illis , contingit multotiens diuites fieri pauperes , } et econuerso . \\\hline
3.1.17 & nin fazer ligeramente obras de largueza \textbf{ Otrossi auiendo las possessiones ygualadas } podrian assi abondar en ellas & quod opera liberalitatis de facili exercere non valerent . \textbf{ Rursus habendo possessiones aequatas possent } ita abundare in eis , \\\hline
3.1.19 & que se entremetio del gouernamiento dela çibdat \textbf{ establesçiendo muchas cosas } que parte n esçiençia al gouernamiento de los çibdadanos & intromittens se de regimine ciuitatis , \textbf{ statuens multa pertinentia ad regimen ciuium . } Videntur autem quasi ad sex reduci \\\hline
3.1.19 & que y podo mio \textbf{ establesciendo su poliçia } primero se entremetio dela muchedunbre & diuersa genera personarum . \textbf{ Hippodamus autem statuens suam politiam , } primo intromisit se de multitudine \\\hline
3.1.19 & entremetiose e determino del departimiento delas possessiones \textbf{ partiendo todo el regno } o todo el terretorio de la çibdat en tres partes . & et determinauit de distinctione possessionum , \textbf{ diuidens totam regionem idest } totum territorium ciuitatis \\\hline
3.1.19 & o le faze tuerto en las sus cosas \textbf{ enpeesçiendol o dannando gelas o le faze tuerto enla persona } e esto en dos maneras & vel iniustificat in res nocendo \textbf{ et damnificando ipsum : | vel in personam , } et hoc dupliciter , \\\hline
3.1.19 & o de so nuestra dol en la persona \textbf{ dandol feridas } e faziendo le llagas en el cuerpo & vel dehonorando eam , \textbf{ faciendo ei opprobria et vituperia : } vel offendendo ipsam , \\\hline
3.1.19 & dandol feridas \textbf{ e faziendo le llagas en el cuerpo } mas el tuerto & vel dehonorando eam , \textbf{ faciendo ei opprobria et vituperia : } vel offendendo ipsam , \\\hline
3.1.20 & Et por ende podemos quanto parte nesçe a lo presente \textbf{ signiendo los dichos del philosofo } enel segundo libro delas politicas rephender a & Possumus autem quantum ad praesens spectat , \textbf{ sequendo dicta Philos’ 2 Pol’ increpare } Hippodamum quantum ad tria . \\\hline
3.1.20 & que el establesçio \textbf{ tanniendo departidos linages de perssonas . } Ca lo primero el dicho philosofo fallesçio & quem statuit , \textbf{ tangentes diuersa genera personatum . } Primo enim dictus Phil’ deferre fecit statuendo impossibilia . \\\hline
3.1.20 & Lo segundo fallesçia i podo mio \textbf{ quanto ala manera que establesçia en iudgando } ca quarie que los iuezes non deuian auer acuerdo & Secundo deficiebat Hippodamus \textbf{ quantum ad modum quem statuit in iudicando ; } volebat enim iudices \\\hline
3.1.20 & quanto ala ley \textbf{ que establesçio alos sabios diziendo } que si qual quier sabio & tangentes diuersa personarum genera , \textbf{ et specialiter quantum ad legem quam statuit erga sapientes . } Nam si quicunque sapiens inueniens aliquid expediens ciuitati , \\\hline
3.1.20 & que en esta materia son de dezer adelante lo tractaremos mas conplidamente . \textbf{ Enpero quanto alo presente c̃ple de auer tranniendo esto } que diches delas opimones de los philosofos & Et in hoc terminetur prima pars huius tertii libri , \textbf{ in quo agitur de regimine ciuitatis et regni . | Primae partis , } in qua dictum fuit , \\\hline
3.2.1 & es pues que con el ayuda de dios cunpliemos la primera parte deste terçero libro \textbf{ ante pomiendo alguons preanbulos al nuestro proponimiento } e rezando opiniones de departidos philosofos & compleuimus primam partem huius tertii libri , \textbf{ praemittendo quaedam praeambula ad propositum , } et recitando opinionem diuersorum Philosophorum instituentium politiam , \\\hline
3.2.1 & ante pomiendo alguons preanbulos al nuestro proponimiento \textbf{ e rezando opiniones de departidos philosofos } que establesçieron poliçias & praemittendo quaedam praeambula ad propositum , \textbf{ et recitando opinionem diuersorum Philosophorum instituentium politiam , } et tradentium artem \\\hline
3.2.4 & que el pho pone en el terçero libro delas politicas dudado \textbf{ e poniendo muchͣs razones para esto } que meior es que much & dubitando , \textbf{ assignans rationes multas , } quod melius sit dominari multitudinem : \\\hline
3.2.5 & e de auer la dignidat real . \textbf{ Et por ende paresça e a alguons que fablando sueltamente meiores } que el prinçipe sea establesçido & habere regiam dignitatem . \textbf{ Absolute ergo loquendo , } melius est Principem praestituendum esse per electionem , \\\hline
3.2.5 & por que el padre con mayor acuçia aya cuydado del bien del regno \textbf{ sabiendo } que el regno parte nesçe al su fiio mas amado & ut pater ampliori solicitudine curet de bono regni , \textbf{ sciens ipsum peruenire ad filium plus dilectum . } Et si dicatur quod contingit aliquando magis diligere minores . \\\hline
3.2.6 & malamente e desigualmente \textbf{ enssennorea aquel que despreçiando el bien comun } entiende el bien propreo & diuinius bono unius , \textbf{ peruerse dominatur | qui spreto bono communi } intendit bonum proprium . \\\hline
3.2.6 & Et por ende la su entençion toda se pone en el auer \textbf{ o en los dinos creyendo que por ellos puede auer las otras cosas delectables . } Mas la entençion del Rey esta & nisi de delectationibus propriis , \textbf{ maxime versatur sua intentio circa pecuniam , credens se per eam posse huiusmodi delectabilia obtinere . } Sed regis intentio versatur circa virtutem , \\\hline
3.2.6 & si non delas sus delecta connes proprias . \textbf{ veyendo se en carga e en aborresçimiento } de aquellos que son en el regno . & nisi de delectationibus propriis , \textbf{ videns se esse onerosum et tediosum } ab iis \\\hline
3.2.9 & si tal muchedunbre de poderio çiuil fuere ganada \textbf{ tomando sennorio ageno } e por fuerça e sin iustiçia . & Quod maxime verum est \textbf{ si huiusmodi multitudo ciuilis potentiae acquisita sit } per usurpationem et iniustitiam . \\\hline
3.2.9 & por que por auentura non la tenie \textbf{ commo deuie sue denostado de su muger diziendo que grand uerguença le era } que dexaua menor regno a sus fijos & quia eam forte iniuste tenebat : \textbf{ increpatus ab uxore dicente , | quod verecundari deberet , } quia minus regnum dimitteret filiis , \\\hline
3.2.9 & Enpero nos podemos traher otra \textbf{ razon meiora esto diziendo } que si el Rey ha a dios por amigo . & et nihil iniquum exercere . \textbf{ Possumus tamen ad hoc aliam meliorem rationem adducere dicentes } quod si Rex habeat \\\hline
3.2.10 & ̃en el su regno \textbf{ non quariendo sofrir sus males leuna tanse contra el . } Et el tirano de que conosçe & excellentes et nobiles existentes \textbf{ in regno non valentes hoc pati , | insurgunt contra ipsum : } tyrannus autem ex quo talem se esse cognoscit , \\\hline
3.2.10 & Mas el uerdadero Rey faze todo el contrario \textbf{ entendiendo en el bien comun } e conosçiendo que el es amado de todos & quod signum est tyrannidis pessimae . \textbf{ Verus autem Rex econuerso intendens commune bonum , } et cognoscens se diligi ab ipsis \\\hline
3.2.10 & entendiendo en el bien comun \textbf{ e conosçiendo que el es amado de todos } los que son en su regno & Verus autem Rex econuerso intendens commune bonum , \textbf{ et cognoscens se diligi ab ipsis } qui sunt in regno , excellentes , et nobiles , \\\hline
3.2.10 & es destroyr los sabios . \textbf{ Ca ueyendo que aquello que fazen es } contra razon derech̃tu eyendo & est sapientes destruere . \textbf{ Vident enim se contra dictamen rectae rationis agere , } et non intendere bonum commune sed proprium : \\\hline
3.2.10 & Ca ueyendo que aquello que fazen es \textbf{ contra razon derech̃tu eyendo } que ellos non entienden enl bien comun & est sapientes destruere . \textbf{ Vident enim se contra dictamen rectae rationis agere , } et non intendere bonum commune sed proprium : \\\hline
3.2.10 & nin sopiessen su maldat \textbf{ ca conosçiendo la mourien el pueblo contra ellos . } Ca sienpre el que mal faze & ne cognoscentes eorum nequitiam , \textbf{ incitent populum contra ipsos : | semper enim } qui male agit , odit lucem , \\\hline
3.2.10 & e promueuelos e honrralos \textbf{ por razon que los sabios conosciendo } e sabiendo las sus buenas obras & et honorat , \textbf{ eo quod ipsi cognoscentes bona opera ipsius , } populum commouent ad amorem eius . \\\hline
3.2.10 & por razon que los sabios conosciendo \textbf{ e sabiendo las sus buenas obras } mueuen & et honorat , \textbf{ eo quod ipsi cognoscentes bona opera ipsius , } populum commouent ad amorem eius . \\\hline
3.2.10 & que promueue el estudio \textbf{ e mantiene le ueyendo } que por el bien comun & Verus autem Rex econtrario studium promouet , \textbf{ et conseruat , } videns quod per ipsum , bonum commune , \\\hline
3.2.10 & Mas esto non quiere el tirano . \textbf{ ca teme que los çibdadanos fiando vnos de otros } se leuaten contra el & hoc autem tyrannus non diligit : \textbf{ timet quidem ne ciues de se confidentes } propter iniurias \\\hline
3.2.10 & que non les vague de seleunatar contra el tirano . \textbf{ Mas el uerdadero rey non entiende de atormentar los subditos mouiendo les } e procurado les guerras & non vacet eis aliquid machinari contra tyrannum . \textbf{ Verus autem Rex non intendit affligere subditos , } suscitando et procurando bella , \\\hline
3.2.11 & que las obras del Rey son muy bueans \textbf{ mostrando que las obras del tirano son muy malas } e desto paresçe manifiesta miente & opera regia esse optima , \textbf{ ostendendo tyrannica esse pessima . } Ex hoc autem manifeste patet , \\\hline
3.2.12 & enssennore antes mayormente han acuçia eti guarda de su cuerpo . \textbf{ Ca ninguno despreciando el bien comun } non entiende a riquezas & circa custodiam corporis . \textbf{ Nam nullus spreto communi bono intendit } ad pecuniam et voluptates corporis , \\\hline
3.2.12 & nin aplazentias corporales \textbf{ si non aguauiando el pueblo en muchͣs cosas . } ca tal commo este & ad pecuniam et voluptates corporis , \textbf{ nisi in multis offendat populum : } nam talis ut pecuniam habeat \\\hline
3.2.12 & e faze much stuertos alos çibdadanos e enlas mugres e en las fijnas . \textbf{ Et por ende veyendo se aborresçido del pueblo } non fia dela muchedunbre de los çibdadanos & quantum ad uxores et filias . \textbf{ Videt ergo se esse odiosum populo , } ideo non credit se multitudini , \\\hline
3.2.12 & que nunca mostraua la cara alegte \textbf{ e aquel tirano quariendo dar razon desto fizo despoiar a su hͣmano } e fizola tar & et quare nunquam hylarem vultum ostenderet . \textbf{ Tyrannus ille volens reddere causam quaesiti , | eum expoliari fecit , } et ligari : \\\hline
3.2.13 & que son de flaco coraçon \textbf{ quando temen muncto non creyendo } que pueden esca par & cum nimis timent , \textbf{ et non credunt se posse euadere , } quasi desperantes inuadunt alios , \\\hline
3.2.13 & munchas vezes los subditos asechanal tirano \textbf{ e matan lo temjendo } qua resçibran daño & hoc ergo modo multotiens subditi insidiantur tyranno , \textbf{ et perimunt ipsum tyrannum , } timentes se offendi ab eo . \\\hline
3.2.13 & que han del \textbf{ algunas vezes la acometen quariendo vengar aquellos tuertos } e aquellas miurias que rresçibieron ¶ & et propter vehementem iram aliquando inuadunt \textbf{ ipsum volentes latas iniurias vindicare . } Tertio insidiantur aliqui tyranno , \\\hline
3.2.13 & e por ende vn omne \textbf{ que auja nonbre dion viendol } que sienpre estaua enbriago & eo quod quasi semper esset ebrius : \textbf{ quidam enim nomine Dion videns ipsum } quasi semper esse ebrium , \\\hline
3.2.13 & por despechon quel auje \textbf{ e despreçiandol } leunatose contra el e matol¶ & quasi semper esse ebrium , \textbf{ propter despectionem insurrexit in ipsum . } Quarto hoc fieri contingit \\\hline
3.2.13 & entre los bienes \textbf{ que paresçen alos omes de fuera munchos veyendo } que el tirano non ha ciudado & inter bona exteriora sint bonum maximum , \textbf{ multi videntes tyrannum } non quaerere \\\hline
3.2.13 & njn quiere el bien comun \textbf{ quariendo algunos alcançar la gloriar la honrra } que veen enel tirano acometen ler matanle , & et non quaerere commune bonum , \textbf{ volentes adipisci honorem } et gloriam quam conspiciunt in tyranno , \\\hline
3.2.13 & por que munchos cuda \textbf{ que el auer es muy grand bien veyendo } que el tirano non entiende & et perimunt ipsum . \textbf{ Sic etiam quia multi reputant pecuniam esse maximum bonum , } videntes tyrarannum non intendere \\\hline
3.2.13 & enel quinto libro delas politicas \textbf{ que algunos veyendo las grandes ganançias } e las grandes honrras & Unde dicitur 5 Polit’ \textbf{ quod quidam tyrannos inuadunt , | videntes lucra magna , } et honores magnos existentes in ipsis . \\\hline
3.2.14 & prinçipe vno titaniza en el pueblo aquella gente apremiada non podie do sofrir su tira \textbf{ maleunatasse e tiraniza contrael prinçipe matandol o echandol del prinçipado . } Et por ende todo el pueblo es fech & sustinere tyrannidem Principis , \textbf{ insurgit et tyrannizat in ipsum , | eum perimens vel expellens . } Totus ergo populus efficitur \\\hline
3.2.15 & e bien fazer aquellos que son en el regno \textbf{ poniendo los en alguons prinçipados } e honrrando los & est bene uti iis qui sunt in regno , \textbf{ introducendo eos ad aliquos principatus , } honorando eos , \\\hline
3.2.15 & poniendo los en alguons prinçipados \textbf{ e honrrando los } e non les faziendo tuerto . & introducendo eos ad aliquos principatus , \textbf{ honorando eos , } et non iniuriando eis . \\\hline
3.2.15 & e honrrando los \textbf{ e non les faziendo tuerto . } Ca assi commo dize el philosofo & honorando eos , \textbf{ et non iniuriando eis . } Nam ut innuit Philosophus in Poli’ \\\hline
3.2.15 & e las contiendas delos nobles \textbf{ e esto poniendo les leyes } ca non pueden de ligero contradezir alas leyes & est cauere seditiones et contentiones nobilium ; \textbf{ et hoc ponendo eis leges , } quia legibus non de facili contradicitur . \\\hline
3.2.15 & saluat se ha el regno \textbf{ ca temiendo que } cotezccan alguas cosas contrarias en el regno & Quare si Rex bonum regni diligat , saluabitur regnum ; \textbf{ quia timens ne in regno aduersa contingant , } adhibebit multa consilia qualiter possit \\\hline
3.2.15 & Mas la iustiçia non se puede guardar en el regno \textbf{ sinon dando pena poderio çiuil } aquellos que traspassan la iustiçia . & Sed iustitia in regno conseruari non potest , \textbf{ nisi per potentiam ciuilem } puniantur transgressores iusti . \\\hline
3.2.15 & e si asi ere dar pena alos malos \textbf{ que trasgre en passando la iustiçia } auer much sassechadores e muchs pesquiridores & si vult seruare iustitiam \textbf{ et vult punire transgressores iusti , } habere multos exploratores , \\\hline
3.2.17 & si quier mal demanda alguna cosa \textbf{ et razon a demandando alguna cosa . } por la qual cosa qual se quier que demanda conseio demanda alguna cosa & consilians siue bene siue male consiliatur , \textbf{ quaerit aliquid , et ratiocinatur . } Propter quod quicunque consiliatur , \\\hline
3.2.17 & que nos tomamos consłeios \textbf{ en las grandes cosas desfunzando de nos mismos } assi commo si non fuessemos sufiçientes para lo conosçer & Ideo dicitur 3 Ethicorum consiliatores \textbf{ assumimus in magna discernentes , } nobis ipsis velut non sufficientibus dignoscere . \\\hline
3.2.17 & mas por auentura meior po demos \textbf{ dezir que cosseio sea dicha conssilendo } que quiere tanto dezir commo cosa que se deue callar entre muchs camuches eston de guardar en los consseios & quod consilium dictum sit \textbf{ a Con et Sileo | ut illud dicatur esse Consilium , } quod simul aliqui plures silent et tacent . \\\hline
3.2.17 & do se tractan los fechos e las negoçios comunes del regno . \textbf{ por que cada vno de los consseieros tirando de ssi el } amortan solamente tenga mientes al bien comun & ubi tractantur negocia communia et facta regni : \textbf{ ut unusquisque consiliarius } adiecta dilectione priuati boni , \\\hline
3.2.17 & establesçimientos antigos \textbf{ alabando alas consseieros de roma } dize que de grant fe & capitulo de Institutis antiquis , \textbf{ commendans Romanos consiliatores , } ait , quod fidum et altum erat \\\hline
3.2.17 & en el qual conssistorio \textbf{ quando ellos entra una tirando dessi el amor propreo } assi se reuistien de amor comun e pubłico & silentique salubritate munitum : \textbf{ cuius limen intrantes abiecta priuata dilectione } ita dilectionem publicam inducebant , \\\hline
3.2.17 & que non fablen y cosas plazenteras mas uerdaderas \textbf{ ca los lisongeros estudiando de fazer plaza los prinçipes callan la uerdat } e dizen las cosas & ut non loquantur ibi placentia , sed vera . \textbf{ Adulatores enim | dum Principi placere student , } vera silentes , \\\hline
3.2.17 & ¶ La otra que non sean plazenteros \textbf{ assi que parezcan lisongeros auiendo mayor cuydado de fablar cosas plazenteras que uerdaderas . } En essa misma manera abn segunt dize el pho & ut quod essent adulatores , \textbf{ plus curantes loqui placentia , | quam vera . } Sic etiam ut recitat Philosophus \\\hline
3.2.17 & que vn poeta \textbf{ que auie nonbre alixandre veyendo } que primero era muy guardado enlos conseios & 2 Rhetor’ \textbf{ quidam poeta nomine Alexander videns } Priamum in consiliis esse secretarium et veracem , commendans eum dicebat , \\\hline
3.2.17 & e muy uerdadero \textbf{ alabandolo dize del este es aquel que conseia } assi commo si diriesse & Priamum in consiliis esse secretarium et veracem , commendans eum dicebat , \textbf{ Iste est qui consuluit . } Ac si diceret , \\\hline
3.2.18 & çerca nuestros amigos \textbf{ assi commo çercanos mismos creyendo dellos } que valen mas de quanto ualen & sicut circa nos ipsos , \textbf{ ut credamus eos plus valere quam valeant , } et esse meliores quam sint . \\\hline
3.2.19 & enssennorea se ha de saluar o corronper . \textbf{ por que escogiendo la meior manera de prinçipar o de } enssennorear & secundum quem dominatur habet saluari et corrumpi : \textbf{ ut eligens optimum modum principandi , } ferat leges iustissimas , \\\hline
3.2.19 & Ca ya por los dichos dessuso \textbf{ ayudando nos la esperiençia } e la prueua de los fechos particulares & Quare per superius iam dicta \textbf{ coadiuuante experientia gestorum particularium } quae continue occurrunt \\\hline
3.2.20 & e det̃minamos del conseio \textbf{ declarado qual esstrando qual deue ser el prinçipe deuen ser los conseieros . } e quales e quantas cosas son aquellas & et determinauimus de consilio , \textbf{ declarando quales debent esse consiliarii : } et quae et quot sunt illa \\\hline
3.2.20 & en que han de ser tomados los consseios fincan nos segunt la orden sobredichͣ \textbf{ que digamos del alcalłia o del iuyzio mostrando } en qual manera deuemos iudgar & Restat secundum ordinem praetaxatum \textbf{ ut exequamur de praetorio , | siue de iudicio , } inuestigando qualiter iudicandum sit , \\\hline
3.2.20 & ca los fazedores de las leyes fozieron leyes en general \textbf{ e delas cosas que auien de venir diziendo } que qual quier que tal cosa fiziere tal pena aura & nam conditores legum leges ferunt \textbf{ in uniuersali et de futuris , } dicentes quicunque sic egerit , \\\hline
3.2.20 & que qual quier que tal cosa fiziere tal pena aura \textbf{ non sabiendo si serie amigo o enemigo } aquel que auie de fazer aquella cosa & dicentes quicunque sic egerit , \textbf{ sic puniatur , ignorantes an amicus , } vel inimicus sit illa facturus , \\\hline
3.2.20 & por \textbf{ auentraase torçerian en iudgando } e encobririen la pena con algun color . & Nam si scirent quod amicus , \textbf{ forte obliquerentur in iudicando , } et poenam palliarent : \\\hline
3.2.20 & e non laben que cola contesçra en lo elpeçial . \textbf{ non se tuerçen en iudgando } nin en faziendo las leyes enclinados se & et nesciunt quid in particularibus sit futurum , \textbf{ non peruertuntur } in iudicando amore , \\\hline
3.2.20 & non se tuerçen en iudgando \textbf{ nin en faziendo las leyes enclinados se } por amor o por mal querençia . & non peruertuntur \textbf{ in iudicando amore , } vel odio inclinati . \\\hline
3.2.20 & en el primero libro de la rectorica \textbf{ diziendo } que mucho conuiene & Has autem tres rationes tangit Philosophus 1 Rhetoricorum dicens \textbf{ quod maxime quidem contingit } recte positas leges , \\\hline
3.2.20 & muchos vezes se ayunta algun pro para si mesmos \textbf{ et por ende se pueden torçer en iudgando . } la quarta razon para mostrar & et quibus proprium commodum annexum est saepe . \textbf{ Quarta via ad ostendendum hoc idem , } sic declarari potest . \\\hline
3.2.20 & e por la qual cosa commo el iues \textbf{ iudgando los culpados } segunt las leyes & finem executione iudiciorum . \textbf{ Quare cum Iudex iudicando reos } secundum leges \\\hline
3.2.21 & ca muchos de los que contienden en iuyzio \textbf{ sabiendo que tienen mal pleito } non cuentan lo que es fecho & in iudicio prohibeantur : \textbf{ multi enim litigantium cognoscentes se habere malam causam , } non narrant quid factum et quid non factum , \\\hline
3.2.21 & assi commo regla derecha en \textbf{ iudgando la segunda razon se toma } por que tales palabras tiran la orden del iuizio ¶ & quasi regulam in iudicando . \textbf{ Secunda vero , } quia praedicti sermones tollunt ordinem iudicandi . \\\hline
3.2.21 & La primera razon paresçe assiça \textbf{ deuedessaber que el iuez en iudgando de los pleitos } para que derechamente iudgue & Prima via sic patet . \textbf{ Scire enim debemus | quod iudex in iudicando de litigiis , } ut recte iudicet , \\\hline
3.2.21 & quando non esta desigualada \textbf{ por algun humor iudga derechͣmente diziendo } que lo amargo es amargo & ut quamdiu lingua non est \textbf{ infecto aliquo humore , recte iudicat , } dicens amarum esse amarum , \\\hline
3.2.21 & mas esta trayda a \textbf{ algundelas partes contrarias iudga mal diziendo } que lo dulçe es amargo & tunc non quasi existens in medio , \textbf{ sed contracta ad alterum contrariorum , } peruersae iudicat , \\\hline
3.2.21 & entre las partes \textbf{ que contienden non se enclinando a ninguna delas partes es } assi commo regla derecha diziendo & quando est medius \textbf{ inter litigantes non declinans ad alteram partem , } quasi regula recta decet \\\hline
3.2.21 & que contienden non se enclinando a ninguna delas partes es \textbf{ assi commo regla derecha diziendo } e mostrando lo que es derecho & inter litigantes non declinans ad alteram partem , \textbf{ quasi regula recta decet } iustum esse iustum \\\hline
3.2.21 & assi commo regla derecha diziendo \textbf{ e mostrando lo que es derecho } que es derech & quasi regula recta decet \textbf{ iustum esse iustum } et iniustum iniustum . \\\hline
3.2.21 & qual es derecho que ha de iudgar en las obras delos omes . \textbf{ por las quales leyes se regla el iues en iudgando . } Et por ende el iiez del ponedor dela ley & quid iustum in agibilibus humanis \textbf{ per cuius leges regulatur in iudicando , } a legislatore ergo discit iudex quid iustum . \\\hline
3.2.21 & y la orden de iudgar . \textbf{ Ca las partes mouiendo el iuez } assi fazen paresçer & Peruertitur ibi talis ordo , \textbf{ quia partes passionando iudicem , } ei faciunt apparere aliquid iustum vel iniustum , \\\hline
3.2.21 & Mas enduzir al iiez \textbf{ por palabras contando le las miurias . } las quales la parte contraria fizo al iuez & passionare autem iudicem , \textbf{ aut narrare iniurias } quas pars aduersa iudici intulit , \\\hline
3.2.21 & las quales la parte contraria fizo al iuez \textbf{ o contando le los bienes } que ellos fizieron a liiez . & quas pars aduersa iudici intulit , \textbf{ vel narrare bona } quae ipsi iudici contulerunt , \\\hline
3.2.22 & que sean omildosos \textbf{ non tomando auctoridat } nin poder que les non es acomnedado . & qui sint humiles , \textbf{ non excedentes autoritatem sibi commissam ; } sint prudentes in legibus scientes \\\hline
3.2.23 & Et por ende estas esta razon que inclina al iues a piedat \textbf{ catando alongamiento de tp̃o non es vna } nin es essa misma & Istud itaque sextum inclinatiuum ad pietatem \textbf{ respiciens diuturnitatem temporis , } non est idem cum quinto , \\\hline
3.2.23 & por ende el pho en el primero libro de la \textbf{ rectorica quariendo enduzir los iuezes a misericordia } contra los que yerran contra ellos dize & Ideo Philos’ 1 Rhet’ \textbf{ volens iudicantes | ad misericordiam adducere } erga delinquentes in ipsos ; \\\hline
3.2.24 & por las quales nos somos reglados en las nuestras obras \textbf{ iudgando por ellas } que cosa es iusta & per quas in agibilibus regulamur , \textbf{ diiudicantes per ipsas } quid iustum \\\hline
3.2.24 & Mas los iuristas departieron en la quarta manera el derecho \textbf{ diziendo } que es algun derechn atal . Et alguno es derecho delas gentes & quarto modo ius distinxerunt , \textbf{ dicentes quod est quoddam ius naturale , } et quoddam ius gentium , \\\hline
3.2.24 & e el quinto departimiento del derech̃ . \textbf{ diziendo que en quatro maneras se departe el derech . } Conuiene a saber ende recħ natural & et dare quintam distinctionem iuris , \textbf{ dicendo quod quadruplex est ius , } videlicet naturale , animalium , gentium , et ciuile . \\\hline
3.2.24 & positiuo prisu pone esto va adelante \textbf{ determinando de qual pena de una ser } tales cosas castigadas o condep̃nadas . & hoc praesupponens ius positiuum procedit ulterius , \textbf{ determinans qua poena sint talia punienda . } Hoc viso quantum \\\hline
3.2.25 & que el omne \textbf{ en quanto es omne e penssando segunt su razon proprea } que es auer entendimiento & Ad cuius euidentiam sciendum quod homo \textbf{ ut est homo et secudum propriam rationem consideratus differt } ab animalibus aliis , \\\hline
3.2.25 & e razon departesse delas otras aina las que non han entendimiento . \textbf{ Mas tomando el omne } segunt razon comun & ab animalibus aliis , \textbf{ sed ut animal est } et secundum rationem communem \\\hline
3.2.27 & por que non han ningun poderio para costrennir . \textbf{ mas estendiendo e alargando el nonbre dela ley } quales si quier mandamientos & quia nihil habent coactiuum . \textbf{ Extendendo autem nomen legis , } quaelibet mandata , \\\hline
3.2.28 & non solamente conssentir aquellas cosas \textbf{ que nin son bueans nin malas non las defendiendo } nin dando pena por ellas . & non solum permittere \textbf{ indifferentia non prohibendo ea , } nec puniendo , \\\hline
3.2.28 & que nin son bueans nin malas non las defendiendo \textbf{ nin dando pena por ellas . } Mas avn parte nesçe al ponedor dela ley conssentir aquellas cosas & indifferentia non prohibendo ea , \textbf{ nec puniendo , } sed etiam spectat ad ipsum permittere \\\hline
3.2.29 & Et por ende conuiene \textbf{ que el Rey en gouernando los otros sigua razon de rechͣ . } Et assi se sigue & et esse regulam aliorum , \textbf{ oportet Regem in regendo alios } sequi rectam rationem , \\\hline
3.2.29 & la qual se leunata de razon derecha e de entendumento derecho . \textbf{ Et por ende el rey en gouernando es a } quande dela ley natural & sequi rectam rationem , \textbf{ et per consequens sequi naturalem legem , } quia in tantum recte regit , \\\hline
3.2.29 & que es mas prinçipal en \textbf{ gouernando la ley natural } que el rey & Si loquamur de lege naturali , \textbf{ patet hanc principaliorem esse in regendo , } quam sit ipse Rex : \\\hline
3.2.29 & enssennorea diose e el entendimiento \textbf{ quando alguno en gouernando los otros non se parte } nin se arriedra de derecha razon & Tunc vero principatur Deus , \textbf{ quando quis in regendo alios } non deuiat \\\hline
3.2.29 & Et pues que assi es dende viene \textbf{ que en iudgando algunas cosas } son dichas ser de egualdat . & Inde est ergo quod in iudicando , \textbf{ aliqua dicuntur esse de aequalitate , } aliqua de rigore , \\\hline
3.2.30 & e la piedat pueden estar en vno con la iustiçia \textbf{ e mandan algunos presunptuosos presumiendo de su engennio . } si la theologia es sciençian . & ø \\\hline
3.2.30 & Ca fueron muchos presunptuosos \textbf{ que presumiendo de su engennio dixieron } que la theologia era superflua . & et rationabiliter fieri possunt , clementia et seueritas simul cum iustitia possunt existere . \textbf{ Fuerunt enim aliqui de suo ingenio praesumentes , } dicentes Theologiam superfluere , \\\hline
3.2.31 & para fallar costunbres nueuas \textbf{ diziendo que aquellas eran prouechosas ala çibdat . } Et en esto desfazien e destruyen las leyes antiguas dela tr̃ra . & ut inuenientes consuetudines nouas , \textbf{ dicentes eas esse utiles et proficuas ciuitati , } soluerent leges patrias et antiquas . \\\hline
3.2.31 & que si algun çibdada no matasse a otro en la çibdat \textbf{ e algun pariente uiniesse acometiendo contra algun çibdadano . } Et aquel presentes alguons çibdadanos fuyesse del & videlicet quod si aliquis ciuis esset occisus , \textbf{ et aliquis consanguineus mortui inuaderet aliquem ciuem , } et ille praesentibus aliquibus fugeret \\\hline
3.2.31 & çerca tales cosas \textbf{ creyendo que algunas leyes son meiores } que son peores e creyendo que cunplen mas & circa talia decipi , \textbf{ quia creduntur meliores | quae sunt peiores , } et creduntur magis sufficientes \\\hline
3.2.31 & creyendo que algunas leyes son meiores \textbf{ que son peores e creyendo que cunplen mas } e cunplen menos & quae sunt peiores , \textbf{ et creduntur magis sufficientes } quae sunt minus . \\\hline
3.2.31 & Ca en quanto dela vna parte a alguno prouecha \textbf{ dando leyess mas suficiente en tanto dela otra enpeesçe tolliendo las costunbres dela tierra } que son por alongamiento del tienpo & quia quanto ex una parte quis perficit , \textbf{ dando sufficientiorem legem , | tanto ex alia parte nocet , } tollendo consuetudinem et diuturnitatem temporis , \\\hline
3.2.32 & Et cuenta el philosofo enel terçero libro delas politicas \textbf{ quariendo de el arar } que cosa es la çibdat seys bienes & quod bonorum illorum sit potius . \textbf{ Narrat quidem Philosophus 3 Politic’ volens diffinire } quid sit ciuitas , \\\hline
3.2.32 & Et por ende fue fechͣ la çibdat \textbf{ por que el omne estando solo } non se podria defender de los enemigos . & constituta fuit ciuitas , \textbf{ ut homo qui solitarius se non potest tueri } ab hostibus existens pars multitudinis , \\\hline
3.2.32 & las quales cosas todas podian auer meior \textbf{ los omes biuiendo en vno } que si biuiessen apartados & quae omnia quia facilius fiunt hominibus \textbf{ simul conuiuentibus , } constituta fuit ciuitas , \\\hline
3.2.32 & es abenemiento de casamientos . \textbf{ Ca los omes biuiendo en vno } ganan amistança los vnos con los otros & est communicatio connubiorum . \textbf{ Nam homines simul conuiuentes } ad inuicem amicitiam contrahunt , \\\hline
3.2.32 & e acostunbran se a fazer buenas obras . \textbf{ la qual cosa faziendo los omes } ordenansse para ser bueons e uirtuosos ¶ & et assuescunt ad operationes bonas : \textbf{ quod faciendo , disponuntur , } et fiunt boni , et virtuosi . \\\hline
3.2.32 & assi declarar e demostrar \textbf{ diziendo } que el regno es grand muchedunbre & Potest ergo sic diffiniri regnum , \textbf{ quod est multitudo magna , } in qua sunt multi nobiles et ingenui , \\\hline
3.2.33 & e los otros que fueren pobres seran muy malos \textbf{ assechando ascondidamente } e con faldrimiento a los ricos . & fient nequi , \textbf{ valde astuti et latenter insidiantes diuitibus . } Secunda via sumitur ex mutuo amore , \\\hline
3.2.33 & e poner so pie alos otros \textbf{ e los pobres contradiziendo alos ricos faran discordia en la çibdat } e si contezca que los pobres venzcan & et suppeditare alios . \textbf{ Alii vero contra nitentes dissensionem faciunt , } et si contingat pauperes obtinere , \\\hline
3.2.33 & nin donde aya enuidia al otro \textbf{ ueyendo que es su egual e veyendo } qua non ay grant auna taia entre el vno e el otro . & nec unde ei inuideat \textbf{ videns se ei quasi aequalem existere , } et non esse magnum excessum inter ipsos . \\\hline
3.2.33 & nin ouiere cada vno liçençia de conprar qualsquier possessiones . \textbf{ Ca auiendo diligençia e acuçia conuenible en las conpras } e enlas uendidas de los canpos e delas tierras & quascunque possessiones emere , \textbf{ adhibita enim debita diligentia circa emptionem , } et venditionem agrorum et terrarum , \\\hline
3.2.35 & La segunda sy non le fueren subiectos e obedientes \textbf{ guardando sus leyes e sus mandamientos . } Ca quando estas dos cosas . & Secundo , si non sint ei subiecti et obedientes , \textbf{ obseruando eius leges et mandata . } Cum enim haec duo , \\\hline
3.2.35 & Visto en qual manera los moradores del regno \textbf{ non deuen mouer el Rey a saña errando contra el } e non le faziendo & Viso quomodo habitatores regni non debent \textbf{ prouocare Regem ad iram , | forefaciendo in ipsum , } non exhibere ei debitum honorem \\\hline
3.2.35 & non deuen mouer el Rey a saña errando contra el \textbf{ e non le faziendo } obediençia qual deuen e honrra conuenble . & forefaciendo in ipsum , \textbf{ non exhibere ei debitum honorem } et obedientiam condignam . \\\hline
3.2.35 & finca de uer \textbf{ en qual manera non le deuen mouer a sanna al Rey errando en aquellas cosas } que son delo veniendo contra aquellas cosa & Restat videre , \textbf{ quomodo non debent ipsum prouocare , } forefaciendo in eos \\\hline
3.2.35 & en qual manera non le deuen mouer a sanna al Rey errando en aquellas cosas \textbf{ que son delo veniendo contra aquellas cosa } que parte nesçen a el . & quomodo non debent ipsum prouocare , \textbf{ forefaciendo in eos | qui sunt eius , } et qui pertinent ad ipsum . \\\hline
3.2.36 & para que los Reyes sean ama dos del pueblo es que deuon ser fuertes \textbf{ e de grandes coraçones esponiendo se assi } e avn alos otros & Secundo ut Reges amentur in populo , \textbf{ debent esse fortes et magnanimi , } ponentes ( si oporteat ) \\\hline
3.2.36 & quando veen que mal fazen . \textbf{ Et pues que assi es cada vno del pueblo teme de mal fazer cuydando } que non podra escapar dela pena . & si viderint eos forefacere . \textbf{ Timet igitur tunc quilibet ex populo forefacere , } cogitans se non posse punitionem effugere . \\\hline
3.3.1 & e el su ensseñamiento pueda aprouechar a todos los sus subditos \textbf{ Et pues que assi es por ende enssennando los Reyes } e los prinçipes partimos este libro todo en tres libros & cuius doctrina omnibus prodesse potest . \textbf{ Inde est igitur | quod in erudiendo Reges et Principes , } hunc totalem librum diuisimus in tres libros . \\\hline
3.3.4 & non se podrie ferir tan de ligero del uallestero \textbf{ assi el ome andando } e volviendosse de vna parte a otra & non sic de facili percuteretur ab arcu , \textbf{ sic homo se circumuoluens , } non sic de facili vulneratur ab hoste . \\\hline
3.3.4 & de tener las armas de dia e de noche . \textbf{ Et por ende nin estando nin yaziendo } non deuen auer cuydado de folgura . & die noctuque esse in armis : \textbf{ propter quod nec in stando , } nec in iacendo est \\\hline
3.3.5 & Et desta opinion fue \textbf{ vegeçio diziendo assi . } Creo que ninguno nunca pudo dubdar & Huiusmodi autem opinionis visus est \textbf{ esse Vegetius , dicens : } Numquam credo potuisse dubitari \\\hline
3.3.5 & e de los aldeanos paresçe mas cruel . \textbf{ Et pues que assi es parando mientes a estas cosas podemos iudgar } que los rusticos & ruralium gens videtur esse crudelius . \textbf{ Ad haec igitur intendentibus videtur } censendum esse meliores bellatores esse rurales . \\\hline
3.3.6 & ordonadamentedos males se siguen dende . \textbf{ Ca non guardando la orden } que deuen guardar en la vna parte & duo mala inde consequuntur . \textbf{ Nam non seruato debito ordine , } in una parte erit acies \\\hline
3.3.6 & La qual cosa non se puede fazer \textbf{ non guardando grado conuenible } e orden qual deuen en andando en el az . & et non impediatur ad percutiendum . \textbf{ Quod , nisi seruato debito gradu , } et debito ordine in incessu , \\\hline
3.3.6 & non guardando grado conuenible \textbf{ e orden qual deuen en andando en el az . } Et pues que assi es los lidiadores tan bien peones & Quod , nisi seruato debito gradu , \textbf{ et debito ordine in incessu , } fieri non potest . \\\hline
3.3.6 & lo terçero son de vsar los lidiadores al salto \textbf{ por que sepan andar saltando e por saltos . } la qual cosa es prouechosa a tres cosas & Tertio exercitandi sunt bellatores ad saltum , \textbf{ ut sciant saltim , | vel per saltum incedere . } Quod etiam ad tria est utile . \\\hline
3.3.6 & quando veen los sus enemigos \textbf{ assi por el canpo saltando . } Otrossi el salto & Terrentur etiam ex hoc aduersarii , \textbf{ quando sic vident hostes per saltum incedere . } Rursus , ipse saltus ratione motus facit \\\hline
3.3.7 & a las vezes çerca de tierra a las vezes en medio . \textbf{ Et assi yua rezio feriendo contra aquel palo } commo si fuesse contra su enemigo . & nunc in imo , nunc medio , \textbf{ et contra palum illum sic impetuose } se gerebat percutiendo ipsum , et alia faciendo \\\hline
3.3.7 & commo si fuesse contra su enemigo . \textbf{ e assi se yua cobriendo del escudo } e faziendo todas las otras cosas & et contra palum illum sic impetuose \textbf{ se gerebat percutiendo ipsum , et alia faciendo } quae requiruntur ad bellum , \\\hline
3.3.7 & e assi se yua cobriendo del escudo \textbf{ e faziendo todas las otras cosas } que son menester & et contra palum illum sic impetuose \textbf{ se gerebat percutiendo ipsum , et alia faciendo } quae requiruntur ad bellum , \\\hline
3.3.7 & e despues enbiarlo reziamente . \textbf{ ca esgrimiendo el dardo } por el mayor mouimineto & et postea fortiter impellendum : \textbf{ vibrato enim telo } propter maiorem motum \\\hline
3.3.7 & por el pueblo de roma non cuydaua vençer en otra manera a los enemigos \textbf{ si non poniendo arqueros } e ballesteros mucho escogidos en todas las azes . & non aliter contra hostes se obtinere credebat , \textbf{ nisi in omnibus aciebus electos sagittarios miscuisset . } Quinto , \\\hline
3.3.7 & assi que contra departidos enemigos de \textbf{ partidamente lidien firiendo . } Lo vij° . & ut contra alios et alios hostes , \textbf{ aliter et aliter percutiendo , | dimicent . } Septimo , \\\hline
3.3.8 & nin sabiamente luego mueren o fuyen . \textbf{ En tal manera que fuyendo fazense temerosos en manera que non pueden ser vençedores de sus enemigos } e apenas o nunca osan acometer batalla contra ellos . & vel in fugam versi \textbf{ adeo efficiuntur timidi , } quod contra suos victores vix aut nunquam audent bella committere . \\\hline
3.3.8 & por que contesçe muchas vegadas tan bien de dia commo de noche \textbf{ que estando la hueste sin carcauas } e sin castiellos o otros defendimientos non cuydando & Contingit autem pluries diuino et nocturno tempore , \textbf{ quod , exercitu absque fossis et castris existente , } et non credentes hostes esse propinquos , \\\hline
3.3.8 & que estando la hueste sin carcauas \textbf{ e sin castiellos o otros defendimientos non cuydando } que sus enemigos estan çerca vienen a desora los & Contingit autem pluries diuino et nocturno tempore , \textbf{ quod , exercitu absque fossis et castris existente , } et non credentes hostes esse propinquos , \\\hline
3.3.8 & Mas la manera e la quantidat de las carcauas pone la vegeçio \textbf{ diziendo que si paresciere grant fuerça de los enemigos . } la carcaua deue ser muy ancha de nueue pies e alta de siete . & et quantitatem fossarum tradit Vegetius dicens , \textbf{ quod si non immineat magna vis hostium , } fossa debet esse lata pedes nouem , alta septem . \\\hline
3.3.8 & que si la carcaua fuere fonda de \textbf{ nueue pies echando la tierra a la parte de la hueste } fazese la carcaua mas alta de quatro pies & quod si fossa sit alta pedum nouem , \textbf{ propter terram eiectam } supra fossam crescit \\\hline
3.3.9 & Lo primero es el cuento de los lidiadores . \textbf{ ca do son mas lidiadores las otras cosas estando eguales } segunt razon deuen auer uictoria . & Primum est , numerus bellantium . \textbf{ Nam ubi plures sunt bellantes | ( caeteris paribus aliis ) } secundum quod huiusmodi sunt \\\hline
3.3.9 & e non han los mienbros usados a la batalla \textbf{ estos fallesçen en sufriendo la batalla . } Ca la costunbre es & et membra inexercitata ad bellandum , \textbf{ deficiunt in sustinendo pugnam : } est enim consuetudo quasi altera natura , \\\hline
3.3.10 & e çenturiones \textbf{ que son señores de çient caualleros } e deanes & Rursus constituere expediebat duces , centuriones , decanos , \textbf{ et alios praepositos belli . } Nam totus exercitus habet se \\\hline
3.3.10 & Et so este cabdiello eran los çenturiones \textbf{ que eran señores de çient caualleros . } Et so el centurion eran los deanes & Sub hoc autem duce erant centuriones . \textbf{ Sub centurione vero decani . } Dicitur enim decanus a decem . \\\hline
3.3.10 & algunas o alguna señal manifiesta . \textbf{ a la qual catando los deanes conosçian al su senora propreo } e sabian aqual dean seguir & vel signum aliquod euidens ; \textbf{ quod respicientes decani | agnoscebant centurionem proprium , } et sciebant quem sequi debebant . \\\hline
3.3.10 & para guiar los lidiadores . \textbf{ Mas conuiene de dar otras seña les manifiestas . por que cada vno viendo aquellas señales } se sepa tener ordenadamente en su az & non sufficiunt ad dirigendum bellantes , \textbf{ sed oportet dare euidentia signa ; | ut quilibet solo intuitu sciat } se tenere ordinate in acie , \\\hline
3.3.10 & que leuaua la seña fizo falsedat \textbf{ encubriendo la seña } e escondiendo la . & eo quod vexillifer fraudem committens \textbf{ velauit vexillum } et abscondit ipsum : \\\hline
3.3.10 & encubriendo la seña \textbf{ e escondiendo la . } Por la qual cosa fueron confondidos los lidiadores assi que non auian cabesça & velauit vexillum \textbf{ et abscondit ipsum : } quare confundebantur bellatores , \\\hline
3.3.10 & o en aquel que deue ser antel puesto a los peones lidiadores . \textbf{ Et que assi es concluyendo } e ençerrando razones dezimos & qui est supra pugnatores pedites praeponendus . \textbf{ Debet ergo } qui in pugna pedicibus praeponitur \\\hline
3.3.10 & Et que assi es concluyendo \textbf{ e ençerrando razones dezimos } que al que deue ser & qui est supra pugnatores pedites praeponendus . \textbf{ Debet ergo } qui in pugna pedicibus praeponitur \\\hline
3.3.11 & por que assi lo fazen los marineros . \textbf{ Ca veyendo los periglos de la mar . } por que las sus naues non sufran periglo & Sic etiam marinarii faciunt , \textbf{ qui videntes maris pericula , } ne eorum naues patiantur naufragium , \\\hline
3.3.11 & que son con razon pintadas \textbf{ las quales catando las los marineros } luego entienden & et cetera talia proportionaliter sunt descripta , \textbf{ qui marinarii intuentes , } statim percipiunt qualiter debeant pergere , \\\hline
3.3.11 & Otrossi deue ser el señor de la hueste \textbf{ e el que es señor de çient caualleros e de diez deanes } e los otros que eran ante puestos a las obras de la batalla & quasi negligentem et dormientem . \textbf{ Debet etiam dux exercitus centuriones , et decani , et alii , } qui operibus bellicis praeponuntur , \\\hline
3.3.11 & puedan defender se de los acometedores . \textbf{ Ca assi diziendo puesto } que contesçiesse algun rebate a desora & possent inuadentibus resistere . \textbf{ Sic enim dicendo , | dato quod accideret } aliquis repentinus insultus , \\\hline
3.3.13 & e de ordenar las azes fincanos de mostrar en qual manera los lidiadores deuen ferir \textbf{ e si es meior de ferir cortando o ferir de punta o estocando . } Mas podemos mostrar por çinco razones & qualiter pugnantes percutere debeant , \textbf{ utrum eligibilius est percutere caesim vel punctim . } Possumus autem quinque viis ostendere , \\\hline
3.3.13 & que son de estrannar \textbf{ e de escarnesçer los que fieren cortando . } Et que mas de escoger es ferir de punta . & Possumus autem quinque viis ostendere , \textbf{ quod deridendi sunt percutientes caesim , } et eligibilius est percutere punctim . \\\hline
3.3.13 & para que los colpes enpeescan . \textbf{ Bien assi los que fieren taiando conuiene } que mas corten de las armas & ut vulnera noceant . \textbf{ Sic quia percutientes caesim oportet } plus de armis incidere , \\\hline
3.3.13 & Et por ende mas de escoger es ferir de punta \textbf{ que ferir taiando . } Ca el colpe mas ayna viene a la carne & magis est eligibile percutere punctim , \textbf{ quam caesim . } Modica autem armorum incisio sufficit \\\hline
3.3.13 & por que pequeno cortamiento de las armas abasta \textbf{ para ferir en la carne feriendo de punta } el qual non abastarie & Modica autem armorum incisio sufficit \textbf{ ad laedendum carnem percutiendo punctim , } quae non sufficeret \\\hline
3.3.13 & el qual non abastarie \textbf{ si feriesse cortando . } La segunda razon para puar esto se toma del defendimiento de los huessos & quae non sufficeret \textbf{ si percuteretur caesim . } Secunda via ad inuestigandum hoc idem , \\\hline
3.3.13 & ca si alguno avn que estudiesse desarmado en la ferida \textbf{ que se faze cortando fuesse ferido } ante que el colpe veniesse al coraçon o a los mienbros de vida & Nam et si quis quasi inermis existeret , \textbf{ in percussione caesim priusquam perueniretur } ad cor vel ad membra vitalia , \\\hline
3.3.13 & e de cortar muchos huessos . \textbf{ Mas feriendo de punta pequeno colpe mata al omen . } ca dos onças de sangre abastan & et multa ossa incidere : \textbf{ sed percutiendo punctim } duae unciae sufficiunt ad hoc \\\hline
3.3.13 & meior es ferir de punta . \textbf{ por que feriendo } assi mas ayna se faze llaga mortal . & percutiendum est punctim , \textbf{ quia sic feriendo citius infligitur plaga mortifera . } Tertia via sumitur \\\hline
3.3.13 & fazen que los que non son vistos . \textbf{ Mas en feriendo cortando . } por que conuiene de fazer grand mouimiento de los braços & quia iacula praeuisa minus laedunt . \textbf{ In percutiendo autem caesim , } quia oportet fieri magnum brachiorum motum prius quam infligatur plaga , \\\hline
3.3.13 & Ca los romanos escarnesçien de todos los caualleros \textbf{ que ferien cortando } por que ellos & Deridebant enim Romani milites , \textbf{ omnes percutientes caesim , } quia et ipsi semper volebant percutere punctim . \\\hline
3.3.13 & e tornarsse han a foyr . \textbf{ Por la qual cosa commo feriendo cortando } por el grand mouimiento de los braços & et conuertuntur in fugam . \textbf{ Quare cum percutiendo caesim } propter magnum motum brachiorum insurgat \\\hline
3.3.13 & leuantasse ende grant trabaio . \textbf{ Mas feriendo de punta el canssamiento es muy pequeno . Por ende es meior de ferir de punta } que cortando & punctim uero feriendo \textbf{ modica fatigatio sufficiat , | elegibilius est percutere punctim , } quam caesim . \\\hline
3.3.13 & Mas feriendo de punta el canssamiento es muy pequeno . Por ende es meior de ferir de punta \textbf{ que cortando } por que la ferida de taio & elegibilius est percutere punctim , \textbf{ quam caesim . } Caesa enim percussio quouis impetu veniat \\\hline
3.3.13 & segunt la qual el que fiere se descubre menos . \textbf{ por que assi feriendo menor daño le puede contesçer . } Por la qual cosa commo feriendo de punta & secundum quem seriens minus discooperitur et detegitur ; \textbf{ quia sic feriendo , | minor laesio ei potest accidere . } Quare cum percutiendo punctim \\\hline
3.3.13 & por que assi feriendo menor daño le puede contesçer . \textbf{ Por la qual cosa commo feriendo de punta } avn que este el cuerpo cubierto & minor laesio ei potest accidere . \textbf{ Quare cum percutiendo punctim } etiam tecto corpore possit \\\hline
3.3.13 & puede resçebir grand daño el enemigo . \textbf{ por ende es meior ferir de punta que taiando . } por que firiendo taiando & nimis aduersarius laedi , \textbf{ melius est percutere punctim | quam caesim . } Percutiendo enim caesim oportet \\\hline
3.3.13 & por ende es meior ferir de punta que taiando . \textbf{ por que firiendo taiando } conuiene de leuantar el braço derecho e diestro . & quam caesim . \textbf{ Percutiendo enim caesim oportet } eleuare brachium dextrum : \\\hline
3.3.13 & conuiene de leuantar el braço derecho e diestro . \textbf{ Et leuantando el braço derecho paresçe descubierto el costado derecho } Et da manera al enemigo & eleuare brachium dextrum : \textbf{ quo eleuato dextrum latus nudatur et discooperitur , } et datur hosti via , \\\hline
3.3.14 & Ca quando los enemigos son canssados \textbf{ e han mucho trabaiado velando o en algunas otras malandanças } si estonçe los acometieren seran mas ayna vençidos & Nam quanto hostes sunt \textbf{ lassati laboribus , vigiliis , et incommoditatibus aliis : } si tunc inuaduntur , \\\hline
3.3.14 & assi que los acometa \textbf{ quando estudieren comiendo } o quando durmieren & ut eos inuadant \textbf{ quando cibum capiunt , } vel quando dormiunt , \\\hline
3.3.14 & assi que non fien dessi mismos . \textbf{ Et esto fecho si los acometieren non auiendo fiuza enssi mismos de ligero se a foyr . } Mas esta cautela commo quier que la ponga vegeçio & si eos inuadat , \textbf{ non habentes fiduciam de se inuicem , | de facili conuertentur in fugam . } Sed haec cautela licet \\\hline
3.3.15 & que meior es de ferir de punta \textbf{ que non taiando } en el qual mostramos a los caualleros & Diximus in quodam capitulo praecedenti , \textbf{ percutiendum esse punctim non caesim : } in quo docuimus milites , \\\hline
3.3.15 & assi conmo quando fieren los enemigos \textbf{ lançando piedras e dardos e saetas . } Et otra manera ay & ut cum iaciendo iacula , \textbf{ vel missilia aduersarios feriunt . } Alius autem cum adeo appropinquant , \\\hline
3.3.15 & Et en otra manera quando se fieren de çerca . \textbf{ Ca lançando dardos de lueñe } deuen tener los pies esquierdos & et aliter cum ex propinquo se feriunt . \textbf{ Nam iaciendo iacula a remotis , } debent habere ipsos pedes sinistros ante , \\\hline
3.3.15 & e la esquierda para folgura . \textbf{ Et por ende feriendo de lueñe } deuemos folgar sobre el pie esquierdo puesto delante & et sinistra ad quiescendum . \textbf{ Ideo percutientes a remotis debemus } quiescere \\\hline
3.3.15 & Mas quando venimos a las manos \textbf{ lidiando con las espadas } o con los cuchiellos & et vibrare iaculum . \textbf{ Sed quando manu ad manum pugnatur gladio : } debemus e contrario nos habere , \\\hline
3.3.15 & deuense arredrar con el pie derecho . \textbf{ Et pues que assi es temiendo } assi el pie esquierto firme & cum eodem pede debent se retrahere ; \textbf{ sic itaque tenendo pedem sinistrum immobilem , } et cum dextro se mouendo , \\\hline
3.3.15 & assi el pie esquierto firme \textbf{ e mouiendo se con el esquierdo pie } podrian mas fuertemente ferir los enemigos e mas ligeramente foyr los colpes dellos . & sic itaque tenendo pedem sinistrum immobilem , \textbf{ et cum dextro se mouendo , } poterunt fortius hostes percutere , \\\hline
3.3.15 & assi commo costreñidos por fuerça fazen se mas osados \textbf{ veyendo que non les finca si non la muerte . } Ca veyendo que les non finca & ø \\\hline
3.3.15 & veyendo que non les finca si non la muerte . \textbf{ Ca veyendo que les non finca } si non la muerte & quia desperantes quasi necessitate compulsi efficiuntur audaces , \textbf{ videntes enim se necessario moriendos , } possunt multa mala committere \\\hline
3.3.15 & assi que les non fincasse logar para foyr . \textbf{ Ca yendo los enemigos non ay ningun periglo } e en la fuyda mucho resçiben periglo sin enpesçimiento & quod non pateat eis aditus fugiendi . \textbf{ Nam fugientibus hostibus nullum est periculum , } et in fuga periclitantur multi absque nocumento persequentium ; \\\hline
3.3.15 & por que non ayan de foyr malamente con temor . \textbf{ e sean muertos fuyendo de sus enemigos } e persegriendo los sus enemigos . & turpiter fugiant , \textbf{ et ab insequentibus hostibus occidantur . } Taliter itaque dux se habere debet , \\\hline
3.3.15 & e sean muertos fuyendo de sus enemigos \textbf{ e persegriendo los sus enemigos . } Et pues que assi es en tal manera se deue auer el cabdiello & turpiter fugiant , \textbf{ et ab insequentibus hostibus occidantur . } Taliter itaque dux se habere debet , \\\hline
3.3.15 & Por la qual cosa la batalla de los peones \textbf{ encubiertamente se escusa yendo se los peones . } Et ellos ydos los caualleros pueden meior despues escusar los colpes de los enemigos . & ne pedites videre possint : \textbf{ propter quod pedestris pugna latenter recedit , } qua recedente , equites postea melius possunt \\\hline
3.3.15 & e mayor daño \textbf{ e enpeesçimiento les faryan fuyendo } que si se tornassen e lidiassen . & plures occiderent ; \textbf{ et maius nocumentum eis infertur a fugientibus , } quam si se verterent et bellarent . \\\hline
3.3.16 & Et por ende contesçe que muchas uegadas \textbf{ los que çercan queriendo } mas ayna ganar las fortalezas & Inde est quod multotiens obsidentes \textbf{ volentes citius opprimere munitiones , } si contingat eos capere aliquos de obsessis , \\\hline
3.3.17 & e por piedras lançadas con las manos o con fondas \textbf{ e avn poniendo escaleras . } Ca avn que non sean los cercadores muy sabidores en la batalla & et per lapides emissos manibus vel fundis , \textbf{ et etiam per appositiones scalarum . } Multum enim industres in pugna , \\\hline
3.3.17 & los que estan cercados de comiençan a cauar . \textbf{ Ca cauando ally } e faziendo cauas soterrañas & ubi incipiant fodere : \textbf{ ibi enim faciendo vias subterraneas } sicut faciunt fodientes argentum \\\hline
3.3.17 & Ca cauando ally \textbf{ e faziendo cauas soterrañas } assi commo cauan los que buscan la plata & ubi incipiant fodere : \textbf{ ibi enim faciendo vias subterraneas } sicut faciunt fodientes argentum \\\hline
3.3.17 & por aquellas carreras soterrañas \textbf{ faziendolas toda via mayores } e mas anchas e mas fondas que las carcauas de la çibdat o de la fortaleza & debent per vias illas , \textbf{ faciendo eas profundiores , } quam sint fossae munitionis deuincendae , \\\hline
3.3.17 & si vieren los \textbf{ que çercan que cayendo los muros } pueden luego tomar el logar & vel maximam partem murorum sic suffosserunt et subpunctauerunt , \textbf{ si viderint obsidentes } quod per solum casum murorum possint munitionem obtinere , \\\hline
3.3.17 & Mas quando cuydan que non pueden entrar el logar \textbf{ estando socauados los muros } e sopuestos non deuen luego poner el fuego mas deuen yr so tierra & Sed cum hoc creditur non sufficere , \textbf{ muris existentibus subfossis et subpunctatis , } nondum apponendus est ignis , \\\hline
3.3.17 & Et por cueuas deuen venir \textbf{ fasta que entiendan que poniendo fuego pueden caer las fortalezas } assi commo dicho es de los muros . & et ad maiora moenia castri , vel ciuitatis obsessae , \textbf{ et per similes vias subterraneas est similiter faciendum circa ea , } quod factum est circa muros . \\\hline
3.3.17 & assi commo dicho es de los muros . \textbf{ Otrossi deuen yr so a tierra partiendose a muchas partes } por las cueuas soterrañas & quod factum est circa muros . \textbf{ Rursus procedendum est } diuertendo vias subterraneas , \\\hline
3.3.18 & por cueuas soterrañas . \textbf{ enpero los çercados veyendo que los pueden entrar } socauando los enbargan el cauar & per vias subterraneas capi possit , \textbf{ obsessi tamen prouidentes fossionem impediunt eam , | ne per ipsam fraudulenter } et per insidias deuincantur . \\\hline
3.3.18 & enpero los çercados veyendo que los pueden entrar \textbf{ socauando los enbargan el cauar } assi que por tales cueuas non se pueda tomar la fortaleza & ne per ipsam fraudulenter \textbf{ et per insidias deuincantur . } Quod quomodo fieri habeat , \\\hline
3.3.18 & Mas avn leuantanle con cuerdas el partegal del engeñio \textbf{ guindandol el qual leuantando } arroian las piedras . & sed ulterius cum funibus eleuatur virga machinae , \textbf{ qua eleuata iaciuntur lapides . } Si ergo per solum contrapondus fit huiusmodi proiectio : \\\hline
3.3.18 & deue penssar con grand acuçia \textbf{ si puede meior conbatir aquella fortaleza lançando derechamente } o lançando mas alueñe o en manera medianera entre estas dos & diligenter considerare debet , \textbf{ utrum magis possit | munitionem illam impugnare proiiciendo rectius vel longius , } vel medio modo inter utrunque vel etiam magis posset obsessos offendere \\\hline
3.3.18 & si puede meior conbatir aquella fortaleza lançando derechamente \textbf{ o lançando mas alueñe o en manera medianera entre estas dos } o deue avn penssar & munitionem illam impugnare proiiciendo rectius vel longius , \textbf{ vel medio modo inter utrunque vel etiam magis posset obsessos offendere } proiiciendo spissius et frequentius . \\\hline
3.3.18 & si mas puede confonder e dannarlos \textbf{ cercandos lançando amenudo e muchas vezes . } Ca assi commo viere que mas conuiene en todas aquellas maneras sobredichas de engeñios . & vel medio modo inter utrunque vel etiam magis posset obsessos offendere \textbf{ proiiciendo spissius et frequentius . } Nam prout viderit \\\hline
3.3.19 & en dos maneras conbaten las fortalezas cercadas . \textbf{ Lo primero lançando piedras . } Ca si el alteza de los castiellos & dupliciter impugnantur munitiones obsessae . \textbf{ Primo iaciendo lapides . } Nam si altitudo castrorum excedit altitudinem murorum , \\\hline
3.3.19 & mas con la tabla en tal manera \textbf{ que catando } por ençima de la tabla vea la mas alta parte del muro & si vero visus protendatur magis basse , cum tabula sit existente ad pedes , \textbf{ et sic iacens in terra , | elonget se ab aedificio praedicto , } donec per summitatem tabulae punctaliter \\\hline
3.3.20 & que se fazia \textbf{ por los que cercauan mostrando } que los que çercan las fortalezas & Postquam diximus de bello campestri , \textbf{ et determinauimus de bello obsessiuo , } docentes ipsos obsidentes munitiones et castra qualiter debeant ea obsidere , et debellare . \\\hline
3.3.20 & Et si non han uagar de fundar las fortalezas de nueuo \textbf{ e algunos teniendo la yra de los señores } o temiendo el señorio & Vel si non vacat munitiones de nouo aedificare , \textbf{ et aliqui timentes iram dominorum , } aut Domini metuentes furorum populi , \\\hline
3.3.20 & e algunos teniendo la yra de los señores \textbf{ o temiendo el señorio } e la sanera del pueblo quieren se defender en algunan fortaleza & Vel si non vacat munitiones de nouo aedificare , \textbf{ et aliqui timentes iram dominorum , } aut Domini metuentes furorum populi , \\\hline
3.3.21 & e presta en la fortaleza çercada . \textbf{ Lo segundo en basteciendo el castiello o la cibdat } que teme de ser cercada & eo quod ad multa sit utilis . \textbf{ Secundo in muniendo castrum } vel ciuitatem aliquam obsidendam , \\\hline
3.3.21 & por los enemigos que la tienen cercada . \textbf{ Estonçe pueden el agua salado fazer dulçe colando la } por la çera & ad quam capiendam prohibent obsidentes : \textbf{ tunc mediante caera poterit dulcificari . } Nam secundum philosophum in Meteoris : \\\hline
3.3.21 & por la çera \textbf{ e faziendo pellas hueças } e echando las en el agua salada . & tunc mediante caera poterit dulcificari . \textbf{ Nam secundum philosophum in Meteoris : } Quicquid ex aqua salita \\\hline
3.3.21 & e faziendo pellas hueças \textbf{ e echando las en el agua salada . } el agua que entra en ellas es dulçe . & Nam secundum philosophum in Meteoris : \textbf{ Quicquid ex aqua salita | per poros cerae pertransit , } totum in dulce conuertitur . \\\hline
3.3.21 & que teme ser cercada \textbf{ por que beuiendo agua sola los lidiadores enflaquesçerse yan en tanto que non podrian defenderse de los enemigos . } Mostrado quales remedios se deuen tomar & ne ex potu solius equae bellatores adeo debilitentur , \textbf{ quod non possint viriliter resistere obsidentibus . } Ostenso quomodo sunt remedia adhibenda contra famem , \\\hline
3.3.22 & Et podemos contra esto poner dos remedios . \textbf{ El vno es afondando mucho las carcauas } e finchiendolas de agua & duo remedia assignare . \textbf{ Unum est per profunditatem fossarum repletarum aquis . } Nam si circa munitionem obsessam sint \\\hline
3.3.22 & enfortaleçer el castiello o la çibdat \textbf{ afondando mucho las carcauas } por que non puedan passar & quae de facili fodi potest : \textbf{ et tunc per profundas foueas est fortificanda munitio , } ne per cuniculos deuincatur . \\\hline
3.3.23 & e de lançar a las naues de los enemigos . \textbf{ Et lançandolos assi en las naues quebrantan se los cantaros } e aquel fuego fuerte ençiende & et proiicienda ad nauem hostium . \textbf{ Ex qua proiectione vas frangitur , } et illud incendiarium comburitur \\\hline
3.3.23 & assi commo en la de la tierra . \textbf{ Ca acometiendo los de cada parte } los que estan en las naues & quod in bello nauali est valde periculosum , \textbf{ quia ex omni parte bellantes } in tali bello vident \\\hline
3.3.23 & e foraden la pordiuso . \textbf{ Et faziendo muchos forados } los quales non podran los enemigos çercar & et eam in profundo perforare , \textbf{ faciendo ibi plura foramina , } quae foramina ab hostibus reperiri non poterunt , \\\hline
3.3.23 & e en qual manera auemos de lidiar en la mar . \textbf{ finca nos de mostra concluyendo e ençerrado razones } aque son ordenadas & et quomodo bellandum est in nauali bello . \textbf{ Reliquum est , } ut declaremus , \\\hline

\end{tabular}
