\begin{tabular}{|p{1cm}|p{6.5cm}|p{6.5cm}|}

\hline
1.1.2 & In secundo vero manifestabitur , \textbf{ quomodo debeat suam familiam gubernare . } In tertio autem declarabitur , & E enel segundo mostraremos \textbf{ commo deue el Rey | e Cada vno delos otros gouernar su conpaña } ¶E enel terçero declaremos \\\hline
1.1.2 & primo scire se ipsum regere , \textbf{ secundo scire suam familiam gubernare , } tertio scire regere regnum , et ciuitatem . In primo autem libro in quo agetur de regimine sui , & Lo terçero que sepa \textbf{ gouernả su rregno | e sus çibdades ¶ } pues que asy es en el primo libro \\\hline
1.1.2 & Nam Primo ostendetur in quo regia maiestas debeat suum finem , \textbf{ et suam felicitatem ponere . } Secundo quas virtutes debeat habere , & en que deue la Real magestado el rrey \textbf{ pon su fin e su bienandança¶ } Lo segundo demostrͣemos quales uertudes deue auer el Rey e el gouernador ¶ \\\hline
1.1.2 & Senes enim \textbf{ ( ut suo loco ostendetur ) } sunt naturaliter increduli , et auari : & que los que han costunbres de moços \textbf{ Ca los vieios asy commo se mostrͣa en su logar } son natraalmente mal creyentes e auarientos \\\hline
1.1.5 & per debitas transmutationes \textbf{ consequitur suam perfectionem et formam , } sic homo per rectas & por sus conueientes e ordenadas \textbf{ t̃ns muta connes viene a rresçebir su forma | e su perfecçion } asi el omne por derechas \\\hline
1.1.5 & et debitas operationes \textbf{ consequitur suam perfectionem et felicitatem . } Cum ergo nunquam contingat recte agere , & e conueni entes obras \textbf{ viene a auer su perfecçion | e su bien andança acabada¶ } pues que asy es com̃ nunca pueda omne bien \\\hline
1.1.5 & expedit volenti \textbf{ consequi suum finem , } vel suam felicitatem , & Conviene a todo omne \textbf{ que quiera alcançar e auer su fin } e la su bien andança de auer \\\hline
1.1.6 & et sicut imperfectum \textbf{ ad suam perfectionem , } bona corporis sunt imperfecta & e menguada es ordenada alaꝑfecçion \textbf{ e al su conplimiento . } asi los bienes del cuerpo deuen ser ordenados \\\hline
1.1.6 & Est ergo detestabile cuilibet Homini \textbf{ ponere suam felicitatem in voluptatibus . } Sed maxime hoc est detestabile Regiae maiestati : & ̉ qual quier omne \textbf{ que toda su bien andança pone en delecta connes dela carne } mas mucho mas es de denostar el Rey \\\hline
1.1.7 & Cuilibet ergo Homini detestabile est \textbf{ ponere suam felicitatem in diuitiis , } sed maxime detestabile est regiae maiestati . & ¶pres que assi es mucho es de denostar todo en que pone su feliçidat \textbf{ e su bien andança en las riquezas corporales . } Mas mayor mente es de denostar la Real magestad \\\hline
1.1.7 & et in opinione ponentis \textbf{ suam felicitatem in diuitiis , } diuitiae sunt quid magnum , & e en la opinion de aquel \textbf{ que pone su bien andança en las riquezas . } las riquezas son grant cosa e grant bien . \\\hline
1.1.7 & Secundo detestabile est Regi , \textbf{ vel Principi suam felicitatem ponere in diuitiis , } quia hoc facto Tyrannus efficitur . & Lo segundo se declara \textbf{ assi mucho es de denostar el Rei o el prinçipe | que pone su bien andança en las riquezas corporales . } ¶ Ca por esto se fare tirano \\\hline
1.1.7 & consequi finem suum . \textbf{ Ponens igitur suam felicitatem in diuitiis , } non erit sibi curae , & que el prinçipe \textbf{ que pone su feliçidat | e su bien andança en las riquezas corporales } non aura cuydado ninguno \\\hline
1.1.8 & Indecens est ergo cuilibet homini \textbf{ ponere suam felicitatem in honoribus , } ut credat se esse felicem , & es que ningun omne . \textbf{ ponga su bien andança en las honrras } assi que crea que es bien andante \\\hline
1.1.8 & quod etiam triplici via venari potest . \textbf{ Si enim Rex suam felicitatem in honoribus ponat , } sequentur ipsum tria mala : & Et ahun esto podemos prouar por tres razones . \textbf{ Ca si el Rey pone su bienandança en las honrras } siguen se le tres males ¶ \\\hline
1.1.8 & Secundo indecens est Regi , \textbf{ ponere suam felicitatem in honoribus , } quia ex hoc efficietur periclitator Populi , et praesumptuosus : & assi que muy desconueible cosa es al Rey \textbf{ poner su bien andança en las honrras | Ca por esso seria prisuptuoso } e sob̃uio \\\hline
1.1.8 & non expedit ei \textbf{ suam felicitatem in honoribus ponere . } Tertio hoc est indicens ei , & non presuma \textbf{ nin se ensoƀuezca mucho nol conuiene de poner su bien andança en las honrras ¶ } Lo terçero se demuestra \\\hline
1.1.8 & decet enim Principem \textbf{ sua bona distribuere } secundum dignitatem personarum , & Mas conuiene le partir los sus bienes \textbf{ alos sus vassallos } segunt las dignidades delas personas ¶ \\\hline
1.1.8 & ut plura bona decet dignis , et sapientibus , quam indignis , et Histrionibus . \textbf{ Sed si Princeps suam felicitatem in honoribus ponat , } quia finem summo ardore diligit , & nin alos iuglares ¶ \textbf{ Mas si el prinçipe | pusiere su bien andaça en las honrria } por que la fin e la bien andança es muy amada \\\hline
1.1.8 & et praecipitanter exponere : \textbf{ erit malus in suis rebus , } quia eas non distribuet aequaliter & Ca non fara fuerça de poner el pueblo a grandes peligros presuptuosamente e arrebatadamente . \textbf{ Et sera malo en partir sus aueres . } Ca non los partir a egualmente \\\hline
1.1.9 & bonitas ergo nostra per se dependet a notitia Dei , \textbf{ tanquam effectus a sua causa . } Rursus circa bonitatem nostram notitia Dei non fallit , & Pues que assi es lanr̃a bondat desçende derechamente del conosçimiento de dios \textbf{ assi commo obra de su obrador | e assi commo cosa fechan de su fazedor } ¶Otro si el \\\hline
1.1.10 & non decet principem \textbf{ suam felicitatem ponere in ciuili potentia . } Quinto hoc non decet ipsum , & Non conuiene alos prinçipes \textbf{ poner su bien andança en el poderio | çiuil¶ } La quinta razon por que non conuiene al prinçipe \\\hline
1.1.10 & ad illud \textbf{ in quo suam felicitatem ponit : } ponens ergo suam felicitatem & en las mas cosas ha aquello en que pone toda su feliçidat \textbf{ e toda su bien andança . } Et por ende aquel que pone su feliçidat \\\hline
1.1.11 & videlicet , sanitatem , pulchritudinem , et robur . \textbf{ Immo nonnulli in talibus suam felicitatem ponunt . } Videtur enim omnino esse contrarium & salud e fermosura e fuerça . \textbf{ ¶ Mas alguons fueron | e son que ponen su bien andança en tales cosas } Mas deuedes saber \\\hline
1.1.11 & Dicimus autem \textbf{ ( ut exigit suus status ) } quia plena felicitas in hac vita haberi non potest . & Et dezimos segunt \textbf{ que requiere su estado | por que la feliçidat } e la bien andança conplida \\\hline
1.1.13 & Immo eo ipso quod Rex studet per legem , \textbf{ et prouidentiam suum regnum regere , } quomodo Deus totum uniuersum regit et gubernat , & que el es Rey deue estudiar \textbf{ por que gouierne su regno } por ley e por sabiduria bien commo dios gouienna todo el mundo . \\\hline
1.2.1 & sed semper secundum modum sibi possibilem \textbf{ suas actiones efficiunt . } Tertio , in talibus potentiis & Mas segunt su manera \textbf{ e su poder fazen sienpre sus obras ¶ } La terçera razon por que enlos poderios naturales \\\hline
1.2.1 & sufficienter determinantur \textbf{ ad actiones suas per suam naturam : } sic et sensus . & senssiblessor determinados a sus oƀras \textbf{ por su naturaleza . } Ca assi commo el fuego tanto escalienta quato puede escalentar \\\hline
1.2.2 & prudentes tamen esse non possunt , \textbf{ ut suo loco patebit . } Inde est ergo quod dicitur 6 Ethic’ & enpero non pueden ser pradentes ni sabios \textbf{ assi commo lo prouaremos en su logar ¶ } Et por ende dize el philosofo \\\hline
1.2.11 & prout habent ordinem ad se inuicem , \textbf{ et sibi inuicem suae indigentiae subueniunt , } est in eis quaedam commutatiua Iustitia , & en quanto han ordenamiento entre si mismos \textbf{ e se acorten a sus menguas los vnos alos otros } o es en ellos vna iustiçia mudadora e acorredora \\\hline
1.2.14 & Nam , cum nauigiis , \textbf{ et cum toto suo exercitu transfretaret , } ne aliquis de suo exercito haberet materiam fugiemdi , & Ca commo el estudiese en sus naues \textbf{ e con todas sus naues passase la mar } por que ninguno de sus conpannas non ouiese manera de fuyr \\\hline
1.2.20 & vel quomodo faciat decentes nuptias : \textbf{ sed tota sua intentio est , } quomodo faciat paruos sumptus . & o en qual manera faga sus bodas conuenibles \textbf{ mas toda su entençion es } en qual manera faga peannas espenssas . \\\hline
1.2.23 & Cum autem sic se periculis exponit , \textbf{ adeo debet esse constans in suis negociis , } ut etiam , si viderit expedire , & Et quando assi el mangnanimo se pusiere alos periglos . \textbf{ deue ser ta firme en sus negoçios e en sus obras . } que ahun si viere \\\hline
1.2.26 & sed etiam per deiectionem . \textbf{ Nam si quis ultra quam suus status requirat , } praeter rationem et notabiliter se deiiceret : & por fallesçimiento . \textbf{ Et si alguno mas que su estado demanda sin razon } e notablemente se despreciasse por que fuesse vil e despreçiado e bestia \\\hline
1.2.26 & Secundo decet eos esse humiles ratione operum fiendorum . \textbf{ Nam superbus quaerens suam excellentiam ultra quam debeat , } ut plurimum tendit & que han de fazer . \textbf{ Ca el sobra uio demandado | e quariendo su excellençia e sobrepuiamiento } mas que deue \\\hline
1.2.31 & si pauperes \textbf{ secundum suam facultatem sunt vere , } et perfecte liberales propinquissimum est , & segunt dize el philosofo \textbf{ Empero si los pobres segunt su poder son uerdaderamente } e acabadamente liberales muy cercanos son para ser magnificos \\\hline
1.2.32 & spondens quod quando vellet conuiuium facere , \textbf{ ei suum filium tribueret . } Sic etiam multae de Phalaride bestialitates narrantur . & et prometial que quando quisiesse fazer conbit \textbf{ que el qual daria su fijo } que fiziesse conbite del¶ \\\hline
1.2.33 & ita quod cuilibet generi bonorum \textbf{ demus suum ordinem virtutum . } Dicemus ergo quod perseuerantes habent virtutes politicas : & Et en essa misma manera podemos departir quatro ordenes de uirtudes \textbf{ assi que a cada vn linage de los bueons demos su orden de uirtudes . } Et por ende diremos \\\hline
1.3.1 & quia ostensum est \textbf{ in quo Reges et Principes suum finem ponere debeant , } et quomodo oportet & Ca mostrado es de ssuso \textbf{ enque deuen los Reyes | e los prinçipes poner su fin e su bien andança . } Et otrosi mostrado es en commo les conuiene de ser uirtuosos \\\hline
1.3.3 & quae ad prudentiam requiruntur , \textbf{ per quam possit melius suum populum regere . } Immo si bonum commune praeponat bono priuato , & que son meester ala pradençia e ala sabiduria . \textbf{ por las quales pue da meior gouernar su pueblo . } Mas si ante pusiere e preçiare mas el bien comun \\\hline
1.3.4 & Nam sicut corpora naturalia \textbf{ per suas formas , } ut per grauitatem vel per leuitatem & Ca assi commo los cuerpos naturales \textbf{ por sus formas . } assi commo la piedra \\\hline
1.3.7 & non sit virtus corporalis , \textbf{ utitur tamen in suo actu corporalibus organis ; } propter quod corpore existente indisposito , & Ca maga el entendimiento non sea uirtud corporal \textbf{ enpero en su obra vsa de entender de organos e de mienbros corporales . } Por la qual cosa el cuerpo non estando bien ordenado \\\hline
1.4.1 & quia ostensum est \textbf{ in quo Reges et Principes suum finem ponere debeant : } et quibus virtutibus debeant esse ornati : & Ca mostrado es en que deuen los Reyes \textbf{ e los prinçipes petier su fin e su bien andança . } Otrossi es mostrado de quales utudes deuen ser honrrados \\\hline
1.4.1 & Ideo dicitur secundo Rhetoricorum , \textbf{ quod pueri sua innocentia alios mensurant . } Sicut enim ipsi sunt innocentes , & que los moços mesuran \textbf{ por su ioçençia | e por su sinpleza todos los otros . } Ca assi commo ellos son Innoçentes \\\hline
1.4.3 & sed solum confidunt de iis quae habent . \textbf{ Ponentes ergo in eis suam spem et confidentiam , } non audent expensas facere . & mas solamente fian de aquellas cosas que han e tienen . \textbf{ Et por ende poniendo en lo que han su esperança } e su fiuza non osan fazer espenssas . \\\hline
1.4.4 & quia sunt amatores amicitiarum , \textbf{ et quia sua innocentia alios mensurant , } existimant omnes bonos esse ; & Et por ende los mançebos pro por que son amadores de amistanças \textbf{ e por que por su sinpleza mesutan los otros cuydan } que todos los otros son bueons \\\hline
1.4.5 & quia ergo nobiles ex antiquo fuerunt praesides , \textbf{ et in suo genere fuerunt multi insignes et diuites , } eleuatur cor nobilium & Et por ende por que los nobles de antiguo tienpo fueron prinçipes \textbf{ e en su linage fueres mucho nobles e ricos leunatase el coraçon de los nobles } por \\\hline
1.4.5 & Vult enim ibidem , \textbf{ quod nobiles ex sua nobilitate incitantur , } ut sint magnanimi , et magnifici . & por que dize alli el philosofo \textbf{ que los nobles | por su nobleza se esfuerçan } a ser magnanimos e magnificos . \\\hline
1.4.6 & indigere bonis diuitum , \textbf{ contingit ut diuites in suis cordibus eleuentur , } dispicientes alios , & Por que contesçe que los sabios han menester de los bienes de los ricos \textbf{ e por ende los ricos se leunatan en sus coraçones } despreçiando alos otros \\\hline
1.4.7 & quia est in aliquo principatu , \textbf{ et habet multos sub suo dominio ; } quare cum multos nobiles videamus esse impotentes , & por que es en algun prinçipado \textbf{ e ha muchos so su sennorio . } por la qual cosa commo nos veamos muchos ser nobles \\\hline
2.1.1 & et Principes debeant \textbf{ suam felicitatem ponere : } quas virtutes habere : & por que ya mostramos en qual cosa de un a los Reyes \textbf{ e los prinçipes poner su bien andança . } Et quales uirtudes deuen auer . \\\hline
2.1.1 & Natura cum aliquibus animalibus \textbf{ ad sui tuitionem dedit cornua , } ut bubalis et bobus . & Ca por que la natura dio a algunas anmalias \textbf{ para su defendemiento cuernos } assi commo alos bubalos e alos bueyes . \\\hline
2.1.1 & non dedit \textbf{ ad sui tuitionem cornua vel ungues : } sed dedit ei manum , & conmoaianl mas noble que las otras nol dio \textbf{ para su defendemiento cuernos } nin hunnas mas diol mano . \\\hline
2.1.7 & Deinde ostendemus , \textbf{ qualiter viri suas uxores regere debeant } et ad quas virtutes , & e mayormente los Reyes e los prinçipes . \textbf{ Despues mostraremosen qual manera los uarones deuen gouernar sus mugers } e a quales uirtudes \\\hline
2.1.8 & ut sint amici inter se parentes , \textbf{ qui naturaliter diligunt suam prolem , } ex dilectione naturali & Et por ende el padre e la madre \textbf{ por que naturalmente aman a sus fijos } por el amor natural que han con ellos \\\hline
2.1.8 & tanto magis decet Reges , et Principes , \textbf{ quam diu suae uxores vixerint , } eis inseparabiliter adhaerere . & mas conuiene alos Reyes \textbf{ e alos prinçipes mientre sus mugers biuieren ayuntar se a ellas sin ningun departimiento . } e algunas sectas non los iudgan contra razon que vn omne aya muchͣs mugers \\\hline
2.1.9 & si hoc indecens est parte uxoris , \textbf{ ne uxor a suo coniuge non debite diligatur . } Nam inter uxorem et virum debet & En essa misma manera esto es desconueinble de parte dela muger \textbf{ por que la muger non sea desamada nin aborresçida de su marido . } Ca entre la mugni e el uaron deue ser grand amor \\\hline
2.1.9 & quia , ne indebite utantur venereis , \textbf{ inter eos et suas coniuges maxime reseruari debet } amor debitus coniugalis . & çomoles non conuiene \textbf{ por que entre ellos e sus mugers sea mucho mas guardado el amor matermoinal . } ¶ La terçera razon para prouar esto mesmo se toma de parte dela \\\hline
2.1.10 & esse subiecta viro ; \textbf{ ex quo totam sui corporis potestatem } uni viro tribuit , & Por la qual cosa si la muger deue ser subiecta al uaron \textbf{ pues que todo el pode rio de su cuerpo es dado avn uaron } contra la orden natraales \\\hline
2.1.10 & contra parentes et consanguineos uxoris ad inimicitiam moueretur , \textbf{ eo quod suam coniugem alteri uiro per coniugium subiecerunt . } Ex coniugio igitur , & ante cada vno de aquellos uarones se mouria a enemistad contra los parientes e amigos de aquella muger \textbf{ por que casaron a su muger con otro marido . } ¶ Et pues que assi es del casamiento \\\hline
2.1.10 & Sed si una foemina pluribus nubat viris , \textbf{ patres de suis filiis certi esse non poterunt , } quare non adhibebunt illam diligentiam & Mas si vna fenbra casare con muchos uarones \textbf{ los padres non podrian ser çiertos de sus fijos . } Et por ende non aurian tan grand cuydado \\\hline
2.1.10 & quam debent \textbf{ ut suis filiis debite in nutrimento } et in haereditate prouideant . & nin tan grand acuçia \textbf{ commo deurien en el nudermiento conuenible de sus fijos } nin en proueer los dela hedat ¶ \\\hline
2.1.14 & sicut et Rex gentem sibi subiectam regere debet \textbf{ secundum suum arbitrium , } prout melius viderit illi genti expedire . & que es subiecta a el \textbf{ segunt su aluedo } en quanto viere \\\hline
2.1.15 & unde et idem Philosophus ait , \textbf{ quod unumquodque organorum optime perficiet suum opus , } si non multis operibus sit seruiens , sed uni . & Ende en esse logar dize el philosofo \textbf{ que qual si quier de los instrumentos fara conplidamente su obra si non siruiere en muchos obras mas en vna . } Por la qual cosa commo la natura aya ordenada la mugr \\\hline
2.1.18 & ( ut superius dicebatur ) sunt miseratiui , \textbf{ quia sua innocentia alios mensurantes credunt } omnes innocentes esse , & asi commo dicho es de suso son mibicordiosos \textbf{ Ca mesuran los otros | por su sinpleza } et creen \\\hline
2.1.19 & sed requiritur \textbf{ ut pater sit certus de sua prole . } Cum ergo signa inhonesta , & Mas conuiene que el \textbf{ padresea çierto de su fijo . } Et pues que assi es por que las señales desonestas \\\hline
2.1.19 & quandam suspitionem adgenerent de incontinentia coniugis ; \textbf{ ut pater sit certus de sua prole , } expedit coniuges pudicas esse . & fazen algunan sospecha dela desonestad delas mugers . \textbf{ Para que el padre sea çierto de sus fijos } conuiene que las mugers sean linpias e honestas e guardadas en sus palauras ¶ \\\hline
2.1.19 & Quare decet omnes ciues \textbf{ sic suas coniuges regere : } et tanto magis hoc decet Reges , et Principes , & non o fallando otras cautellas para esto . \textbf{ por la qual cosa conuiene a todos los çibdadanos de gouernar a sus mugers assi . } Et esto tanto mas conuiene alos Reyes \\\hline
2.1.20 & quomodo Reges et Principes , \textbf{ et uniuersaliter omnes ciues debeant suas coniuges regere , } et ad quas bonitates debeant eas inducere : & en cunple de saber en qual manera los Reyes e los prinçipes \textbf{ e generalmente todos los çibdadanos | y deuen gouernar sus mugers } e aquales bonda deslas de una enduzir e traher linon lo pieren \\\hline
2.1.20 & secundum possibilem facultatem decet \textbf{ suam uxorem honorifice retinere } in debito apparatu , & segunt su poder \textbf{ e sus riquezas } en apareiamiento conuenible \\\hline
2.1.20 & uxorem propriam honorifice pertractare . \textbf{ Ostenso , quomodo decet viros suis uxoribus moderate et discrete } uti , & segunt su estado de tractar muy honrradamente a su muger . \textbf{ ¶ Mostrado en qual manera conuiene alos maridos usen de sus mugers } sabiamente e tenprada mente . \\\hline
2.1.21 & quomodo circa ornatum corporis deceat \textbf{ suas coniuges debite se habere . } Nam cum vir suam uxorem regere debeat , & e generalmente a todos los çibdadanos saber en qual manera \textbf{ couiene alas sus mugers | de se auer conueniblemente en el conponimiento e honrramiento de sus cuerpos . } Ca quando el marido gouierna e castiga a su muger \\\hline
2.1.21 & suas coniuges debite se habere . \textbf{ Nam cum vir suam uxorem regere debeat , } eam dirigendo ad actiones honestas , & de se auer conueniblemente en el conponimiento e honrramiento de sus cuerpos . \textbf{ Ca quando el marido gouierna e castiga a su muger } deue la castigar a obras honestas \\\hline
2.1.21 & secundum suum statum , \textbf{ suis uxoribus , } in debitis vestimentis , & Ca conuiene alos maridos de proueer conueniblemente a sus \textbf{ mugerssegunt sus estados e en vestiduras conuenibles } e en los otros conponimientos conueinbles . \\\hline
2.1.21 & Unde et Valerius Maximus ciues Romanos commendat , \textbf{ qui suis uxoribus in pulchris indumentis } et in aliis ornamentis debite prouidebant : & Onde ualerio maximo alaba alos çibdadanos de Roma \textbf{ por que proue en honrradamente a sus mugers | de uestiduras fermosas } e de los otros conponimientos honrrados . \\\hline
2.1.21 & non propter vanam gloriam se ornaret , \textbf{ nec ultra suum statum ornamenta appeteret : } posset delinquere , & por vana eglesia \textbf{ nin dessea conponimiento | mas de quanto demanda su estado . } Enpero avn podria pecar \\\hline
2.1.22 & triplici via ostendere possumus . \textbf{ Nam cum quis erga suam coniugem est nimis zelotypus , } ex nimio zelo quem erga illam gerit , & por tres razones . \textbf{ Ca quando alguno es muy çeloso de su muger } por el grand çelo que ha della sospecha todas las cosas \\\hline
2.1.22 & Decet ergo omnes ciues \textbf{ non esse nimis zelotipos de suis coniugibus : } et tanto magis hoc decet Reges et Principes , & pues que assi es conuiene a todos los çibdadanos \textbf{ de non ser muy çelosos de sus mugieres } Et tanto mas esto conuiene alos Reyes \\\hline
2.1.22 & Nec etiam decet eos \textbf{ circa suas coniuges nullam habere custodiam } et nullum habere zelum , & nin avn les conuiene \textbf{ de non poner alguna guarda en sus mugers | nin les conuiene avn } de non auer algun çelo dellas \\\hline
2.1.22 & Sic enim decet uirum quemlibet \textbf{ erga suam coniugem ornatum habere zelum , } ut sit inter eos amicitia naturalis delectabilis , et honesta . & e deue auer acuçia conuenible de su casa . \textbf{ Ca assi conuiene a cada vn marido de auer çelo ordenado de su mugni } por que sea entre ellos amistança natural delectable e honesta \\\hline
2.1.23 & omnia minora et debiliora \textbf{ citius veniunt ad suum complementum . } Consilium ergo mulieres , & en el libro delas aian lias las ainalias menores e mas flacas . \textbf{ mas ayna vienen a su conplimienta . } Et por ende el consseio delas mugers \\\hline
2.1.23 & quam consilium virile , \textbf{ citius venit ad suum complementum . } Ceteris ergo paribus & e menos poderoso que el conseio delons \textbf{ uarones̃ mas ayna viene a su conplimiento . } Et por ende estando todas las \\\hline
2.1.23 & cito perducit \textbf{ ipsam ad suum augmentum . } Sic et mulier quantum & por que la natura ha poco cuydado della \textbf{ e ayna la aduze a su conplimiento . } ¶ En essa misma manera la muger \\\hline
2.1.23 & et natura minus de ipso curet , \textbf{ citius venit ad suum complementum , } quam virile . & e la natura ha menor cuydado del mas ayna viene \textbf{ a su conplimiento } que el cuerpo del uaron . \\\hline
2.1.24 & si possint se laudari \textbf{ quod a suis maritis diligantur , } appetentes quandam inanem gloriam , & si pueden ser loadas \textbf{ que son amadas de sus maridos . } por ende desseando alguna uana gloria \\\hline
2.2.1 & magis incitabuntur parentes \textbf{ ut suos filios bene regant . } Possumus autem triplici via venari , & mas sian mouidos los padres \textbf{ para gouernar bien sus fijos } Et nos podemos mostrar por tres razones \\\hline
2.2.2 & magis habet solicitudinem circa filios : \textbf{ naturale est enim quemlibet diligere sua opera , } ut Philosophus in Ethicorum & mas ha cuydado de sus fijos \textbf{ Ca natural cosa es que cada vno ame sus obras } assi commo dize el philosofo en las ethicas \\\hline
2.2.2 & unde et patres naturaliter diligunt filios , \textbf{ et poetae sua poemata tanquam proprium opus . } Quando ergo aliquis est intelligentior , & Onde los padres naturalmente aman los fijos \textbf{ e los poetas sus ditados | assi commo obra proprea ¶ } Pues que assi es quando cada vno es mas entendido \\\hline
2.2.3 & quando potest sibi simile generare . \textbf{ Quare cum quilibet suam perfectionem diligat , } naturaliter pater diligit filium , & quando ꝑuede engendrar su semeiante . \textbf{ Et commo quier que cada vno ame su perfecçion . } Emperona traalmente el padre ama el fijo \\\hline
2.2.7 & ad intelligendum et ad cognoscendum naturas rerum : \textbf{ homo tamen a sui natiuitate est } male dispositus & e conosçer las uaturas delas cosas . \textbf{ Enpero el ome comneco de su nasçimiento es mal despuesto } e mal ordenado \\\hline
2.2.8 & sub tali enim sermone Philosophi \textbf{ suam scientiam tradiderunt . } Quare si per nosipsos & Ca los philosofos dauna \textbf{ e mostra una su sçiençia } por tal \\\hline
2.2.8 & ad intelligendum quaecunque proposita : \textbf{ quo facto totum suum ingenium debent exponere , } ut bene intelligant moralia & que les sean propuestas \textbf{ la qual casa fechͣ deuen poner todo su en gennio } porque puedan bien entender las sçiençias m orales \\\hline
2.2.9 & sic decet esse diligentem et cautum , \textbf{ ut proponat suis auditoribus vera } sine admixtione falsorum . & assi commo conuiene al doctor e al maestro en las sçiençias especulatiuas de ser acuçioso e sabio \textbf{ en manera que proponga a sus disçipulos cosas uerdaderas } sin ningun mezclamiento de cosas falssas . \\\hline
2.2.9 & et uniuersaliter omnes ciues valde solicitantur , \textbf{ qualem proponant suis numismatibus , } possessionibus , et rebus inanimatis : & deuen ser muy acuçiosos \textbf{ en catar qual mayordomo deuen poner en sus riquezas } e en sus posessiones e enlas o triscosas \\\hline
2.2.11 & quod vix aut nunquam comedere possunt , \textbf{ quin sua vestimenta deturpent : } turpitudo autem corporalaris licet & Los quales abeso nunca pueden comer \textbf{ que non enlixen sus vestiduras . } Mas la torpedat del cuerpo \\\hline
2.2.12 & ex inflammatione sanguinis , \textbf{ vinum , quod propter sui caliditatem inflammat sanguinem , } reddit hominem animosum et irascibilem : & por que la sanna seleunata dela inflamaçion dela sangre . \textbf{ Et el vino | por su calentura en flama } e ençiende la sangre \\\hline
2.2.13 & et aliquas deductiones \textbf{ interponere suis curis , } ut ex hoc aliquam requiem recipientes , & conuienel de entreponer alguons trebeios \textbf{ e algunos solazes en sus cuydados . } assi que en esto resçibiendo alguno folgua a puedan mas trabaiar para alcançar su fin . \\\hline
2.2.13 & gestus ordinatos et honestos : \textbf{ cohibent enim sua membra , } ne aliquem motum habeant , & e los buenos han gestos ordenados e honestos \textbf{ por que estos tales costramnen e apetan sus mienbros } por que non ayan algun mouimiento \\\hline
2.2.17 & quando habet tale corpus , \textbf{ quale requirit suum officium : } ut tunc miles habet corpus bene dispositum , & quando ha tal cuerpo \textbf{ qual demanda el su ofiçio } assi con no dezios que el cauallero \\\hline
2.2.17 & sed sint subiecti \textbf{ et obedientes suis patribus et senioribus . } Tangit autem Philosophus & por que non sean orgullosos \textbf{ mas que sean subiectos e obedientes a sus padres et alos uieios . } Mas el pho pone en el vii̊ . \\\hline
2.2.17 & quia tunc quasi peruenerunt \textbf{ omnimode ad suam perfectionem , } debent esse tales , & e del ayo \textbf{ por que endçe vienen del todo a su perfecçion } e deuen ser tałs \\\hline
2.2.19 & ne indebite circuant et discurrant : \textbf{ quanto ex impudicitia et lasciuia suarum filiarum potest } maius malum vel periculum imminere . & e salgan fuera \textbf{ quanto dela locania | e dela desuergonança delas sus fijas } puede contesçer mayor mal e mayor periglo . \\\hline
2.3.1 & qualiter decet \textbf{ uiros suas coniuges regere , } et qualiter patres suos filios gubernare . & Ca es mostrai ser do \textbf{ en qual manera conuiene a los maridos de gouernar a sus mugers . } Et en qual maneta los padres deuen regir e gouernar a sus fijos . \\\hline
2.3.1 & uiros suas coniuges regere , \textbf{ et qualiter patres suos filios gubernare . } Restat exequi de parte tertia , & en qual manera conuiene a los maridos de gouernar a sus mugers . \textbf{ Et en qual maneta los padres deuen regir e gouernar a sus fijos . } finca de dezer dela terçera ꝑte \\\hline
2.3.1 & eo quod hae materiae sunt connexae , \textbf{ intendimus instruere uolentem suas domus debite gubernare , } non solum quantum ad regimen ministrorum et familiae , & por que estas materias son ayuntadas en vno entendemos de enssennar \textbf{ a aquellos que quisieren | conueinblemente gouernar sus calas } non lo lamente quanto al \\\hline
2.3.1 & Nam sicut ceterae artes , \textbf{ ut ars fabrilis , et textoria , habent sua organa , } per quae perficiunt actiones suas : & Ca assi commo las otras artes \textbf{ assi commo es arte de ferreria | e de texederia han sus estrumentos } por los quales acaban sus obras . \\\hline
2.3.1 & per quae perficiunt actiones suas : \textbf{ sic et gubernatio domus requirit sua organa , } per quae opera sua complere possit . & por los quales acaban sus obras . \textbf{ En essa misma manera el arte del gouernamiento dela casa demanda sus estrumentos } por los quales pueda conplir sus obras . \\\hline
2.3.6 & sed quilibet ad quamlibet \textbf{ pro sua voluptate accederet , } esset suprema unitas , & mas cada vno se llegasse \textbf{ a qual quisiesse por su uoluntad } que por esto serie grand ayuntamiento \\\hline
2.3.8 & si gubernator domus non vult contra naturam agere , \textbf{ sed vult suam domum regere } secundum modum et ordinem naturalem , & si el gouernador dela casa quiere fazer contra natura \textbf{ mas si quiere gouernar su casa } segunt manera natural \\\hline
2.3.8 & et diuitiis , \textbf{ quantas requirit exigentia sui status . } Nam non satiari possessionibus & e de tantas riquezas \textbf{ quantas demanda el menester de su estado } ca non se fartar omne de possessiones \\\hline
2.3.9 & ut cognoscendo , \textbf{ melius sciat suae domui prouidere . } Conuenienter post tractatum de possessionibus & por que sabien do esto \textbf{ sepa meior proueer su casa } onueinblemente depues del tractado de las possessiones \\\hline
2.3.12 & taxat precium \textbf{ pro suae voluntatis arbitrio : volentem ergo pecuniam acquirere , } oportet haec & commo se el quiere \textbf{ por su uoluntad e por su aluedrio . | Et por ende el que quiere gana rriqueza } conuiene le de tener enla memoria estos fechs particulares e otros semeiantes \\\hline
2.3.17 & Quia maxime apparet Regis prudentia , \textbf{ si suam familiam debito modo gubernet , } et si ei debite et ordinate necessaria tribuat : & or que mucho paresçe la sabiduria del Rey \textbf{ si gouernare en manera conuenible a su conꝑannappra } e sil diere ordenadamente \\\hline
2.3.17 & ut supra in primo libro diffusius probabatur , \textbf{ decet ipsum erga suos ministros decenter se habere in apparatu debito , } et in debitis indumentis . & assi commo es prouado mas conplidamente en el primero libro \textbf{ conuiene les de auer sus siruientes apareiados | conueniblemente en el parescer de fuera } e en uestiduras conuenibles \\\hline
2.3.18 & et a iustitia legali , et a curialitate : \textbf{ ut si quis suis conciuibus bona sua prompte largitur , } si haec agit , & et de curialidat \textbf{ assi commo si alguno dieres o bienes alos sus çibdadanos liberalmente } si esto faze \\\hline
2.3.18 & habere mores nobiles et curiales , ministros , \textbf{ quos in bonis decet suos dominos imitari , } oportet curiales esse . & e de ser curiales e nobles \textbf{ assi conuiene alos seruientes dellos | los que quieren semeiar a sus sennors } de ser buenos e mesurados e corteses . \\\hline
2.3.20 & et obseruare ordinem naturalem \textbf{ omnino in suis mensis , } ordinare debent & e guardar la orden natural en toda \textbf{ meranera deuen ordenar en sus mesas } que los que se assentaren \\\hline
3.1.1 & omnia operantur omnes . \textbf{ Si ergo omnes homines ordinant sua opera in id quod videtur bonum , } cum ciuitas sit opus humanum , & que les paresçe buean . \textbf{ ¶ Et pues que assi es si todos los omes ordenan sus obras | a aquello que paresçe bien } o commo la çibdat \\\hline
3.1.1 & cuiusmodi est communitas regni , \textbf{ de qua suo loco dicetur : } ostendemus enim communitatem regni & que ella la qual es comunidat del regno \textbf{ dela qual diremos en su logar | ca mostraremos } que la comunidat del regno es prouechosa en la uida humanal \\\hline
3.1.2 & secundum naturam suam , \textbf{ quae secundum suam speciem habent esse completum . } Si enim alicui rei deficiat & segunt su natura \textbf{ que han ser conplido | segunt su natura } e su linage \\\hline
3.1.7 & conuertit se ad Moralia , \textbf{ quem Plato suus discipulus in multis secutus est , } propter quod Philoso’ Platonem ipsum & conuirtiosse alascina moral . \textbf{ al qual socrates siguio platon su disçipulo en muchͣs cosas } por la qual cosa el philosofo aristotiles llamo a platon el segundo socrates . \\\hline
3.1.7 & et quod crederent \textbf{ eos esse suos filios , } illi vero opinarentur & por que los antiguos creerian \textbf{ que los moços eran sus fiios } e los mocos cuydarian \\\hline
3.1.7 & illi vero opinarentur \textbf{ eos esse suos patres . } Tertium vero quod senserunt & e los mocos cuydarian \textbf{ que ellos eran sus padres . } ¶ Lo terçero que sintieron los dichs philosofos cerca el gouernamiento dela çibdat . \\\hline
3.1.8 & et ad hoc quod uniuersum \textbf{ secundum suum statum sit maxime perfectum , } oportet ibi dare diuersa secundum speciem . & que son mester en el mundo sean en el mundo \textbf{ e por que el mundo segunt su estado sea muy acabado } conuietie de dar en el \\\hline
3.1.10 & propter honestatem \textbf{ et bonitatem morum parentes esse certos de suis filiis , } et quoslibet certificari de eorum consanguineis , & por bondat e honestad de costunbres \textbf{ que los padres sean çiertos de sus fiios } e cada vnos sean çiertos de sus parientes \\\hline
3.1.10 & nullo modo suspicari posset \textbf{ omnes pueros esse suos filios . } Si ergo omnes diligerent tanquam filios , & si non fuesse loco en ninguna manera non podria sospechͣr \textbf{ que todos los moços fuessen sus fijos . } Et pues que assi es si a todos amassen \\\hline
3.1.19 & diuersa genera personarum . \textbf{ Hippodamus autem statuens suam politiam , } primo intromisit se de multitudine & que tannian alguons linages de personas dezimos \textbf{ que y podo mio | establesciendo su poliçia } primero se entremetio dela muchedunbre \\\hline
3.1.19 & Dicebat autem quod audita causa quilibet iudex per se cogitaret , \textbf{ et postea in pugillaribus scriptam adduceret suam sententiam : } ut si incusatus simpliciter condemnandus esset , & por si deuia penssar \textbf{ e despues poner sus nina en esc̀pto } assi que si el acusado sin ninguna condiçion fuesse de condepnar el \\\hline
3.1.19 & uniuersaliter omnes personas impotentes , \textbf{ non valentes per se ipsas sua iura conquirere . Spectat enim ad Regem et Principem , } qui debet esse custos iusti , & nin podian \textbf{ por si mismas guardar su | derechca parte nesçe al Rey e al } prinçipeque deue ser guardador dela iustiçia de auer cuydado espeçial delas cosas comunes \\\hline
3.2.2 & et omnium ciuium \textbf{ secundum suum statum , } sic est aequale et rectum . & e el bien de todos los çibdadanos \textbf{ segunt su estado } assi es sennorio ygual e derech̃ . \\\hline
3.2.2 & mediarum personarum , et diuitum , \textbf{ et omnium secundum suum statum : } et tunc est rectus et aequalis : & e delas perssonas medianeras e de los ricos \textbf{ e de todos comunalmente | segunt su estado } estonçe el prinçipado es derech e ygual . \\\hline
3.2.8 & Tertio per huiusmodi collata naturaliter intendunt \textbf{ in suos fines siue in suos terminos . } Ut natura dat igni leuitatem , & por estas cosas que les da la natura . \textbf{ naturalmente una a sus terminos o a ssus fines } assi commo paresçe por este exenplo \\\hline
3.2.9 & ne contemnantur a populis , \textbf{ non deberent suam intemperantiam ostendere : } laudatur enim sobrietas et temperantia , & por qua non sean menospreçiados de los pueblos \textbf{ non deuen mostrar su destenpramiento alos otros . } Ca sienpte es de alabar la mesura e la tenprança . \\\hline
3.2.9 & Recitat autem Philosophus 5 Polit’ \textbf{ quod cum quidam Rex partem sui regni dimisisset , } quia eam forte iniuste tenebat : & en el quanto libro delas politicas \textbf{ que commo vn Rey dexasse vna parte de su regno . } por que por auentura non la tenie \\\hline
3.2.9 & nam licet semper se simulent iuste agere , \textbf{ tamen in multis suum dominium iniuste ampliant , } et aliorum haereditates sine ratione usurpant . & que fazen las cosas derechamente \textbf{ enpero en muchas cosas | enssancha su regno sin derecho } e toman las hedades de los otros \\\hline
3.2.9 & ut expedit suae saluti \textbf{ semper in suis actibus prosperari . } Immo propter sanctitatem regis , & assi conmo cunple a su salut . \textbf{ Et fazel ser sienpre bien | auentraado en todos sus fechos } Et por ende por la sanidat del Rey dios muchas uezes faze muchs bienes \\\hline
3.2.10 & per quas nititur \textbf{ tyrannus se in suo dominio praeseruare . } Prima cautela tyrannica , & nchas cautelas tanne el philosofo en el quinto libro delas politicas delas quales quanto par tenesçe alo presente podemos tomar diez . \textbf{ por las qualose esfuerça el tiranno de se mantener en su sennorio . } La primera cautela del tirano es matar los grandes omes e los poderosos . \\\hline
3.2.12 & Cum enim populus principatur peruerse , \textbf{ non intendit quodlibet seruare in suo statu , } sed satagit opprimere nobiles , et insignes . & por que quando el pueblo enseñorea malamente non entiende guaedar a \textbf{ njnguno en su estado mas esfuercas } e quanto puede para abaxar los nobles e los altos \\\hline
3.2.14 & et si eos aliquo modo tyrannizare contingat , \textbf{ suam tyrannidem pro viribus moderare debent , } quia quanto remissius tyrannizabunt , & que en alguna manera ayan de tiranizar deue \textbf{ por toda su | fuerca atenprar la tirania } ca quanto mas poco tiranzar en tanto mas dura el su sennorio ¶ \\\hline
3.2.15 & ut se in suo principatu praeseruet . \textbf{ Primo est , non permittere in suo regno transgressiones modicas . } Nam multae modicae transgressiones & para que se pueda man tener en lu prinçipado e en lu lennorio ¶ \textbf{ La primera es que non consienta en su regno muchos pequanos males } ca muchs pequannos males \\\hline
3.2.17 & attamen imprudens est \textbf{ qui solo suo capiti innittitur , } et renuit aliorum audire sententias . & enpero non es sabio aquel \textbf{ que se esfuerça en su cabeça sola } e menospreçia de oyr las suinas de los otros \\\hline
3.2.19 & Primo , ne maiestas regia \textbf{ aliquos prouentus iniuste usurpet a suis conciuibus : } probabatur enim supra , & ¶Lo primero couiene que el Rey non tome ningunas rentas \textbf{ sin derecho de sus subditos . } ca prouado es de suso \\\hline
3.2.19 & et quomodo Rex se debeat \textbf{ in suo dominio praeseruare , } fuit in superioribus patefactum . & e qual deua ser el ofiçio del rey \textbf{ e en qual manera el Rey se deua guardar en su sennorio mostrado fue conplidamente en los dichos de ssuso . } Ca ya por los dichos dessuso \\\hline
3.2.20 & Contingit etiam absque corruptione et morte iudices \textbf{ a suo officio remoueri , } et alio in suum locum succedere . & que sin cornupçion \textbf{ e sin muerte los iuezes son tirados de sus oficlvii i̊ çios } e son prouestos otros en sir logar \\\hline
3.2.28 & et diligenter per se \textbf{ et suos consiliarios discutiant } quae bona sunt praecipienda et praemianda , & asi que con grant acuçia \textbf{ por si e por sus consseieros examun en quales bueans obras son demandar } e de poner so mandamiento . \\\hline
3.2.29 & est tamen supra legem positiuam , \textbf{ quia illam sua auctoritate constituit . } Itaque sicut Rex nunquam recte regit , & Enpero es sobre la ley positiua \textbf{ por que establesçio el aquella ley con su auctoridat . } Et pues que assi es assi commo el rey nunca gouierna derechamente \\\hline
3.2.30 & et rationabiliter fieri possunt , clementia et seueritas simul cum iustitia possunt existere . \textbf{ Fuerunt enim aliqui de suo ingenio praesumentes , } dicentes Theologiam superfluere , & si la theologia es sciençian . \textbf{ Ca fueron muchos presunptuosos | que presumiendo de su engennio dixieron } que la theologia era superflua . \\\hline
3.2.31 & cuiusmodi erat lex illa , \textbf{ quod ciues possent suas uxores vendere , } vel quaecunque aliae leges sic prauae et iniustae , & assi commo emaquella ley que dizie \textbf{ que los çibdadanos podien vender sus mugiers } o otras quales si quier leyes malas \\\hline
3.2.32 & et si non viueret in societate , \textbf{ ut alii suam magnificentiam perciperent , } et ut eis sua bona communicare posset , & Enperosi non visquiesse en conpannia \textbf{ por que los otros sintiessen su conpannia | e su magnifiçençia } e por que les pudiesse dar de los sus bienes non termie todos aquellos bienes en much . \\\hline
3.3.6 & ut gradatim pergant ita , \textbf{ ut quilibet se in suo ordine teneat . } Nam si acies siue peditum & e generalmente todos los lidiadores se deuen vsar a andar ordenadamente e a passo en la batalla . \textbf{ por que cada vno tenga su orden | e vaya en su lugar . } Ca si el az si quier de peones \\\hline
3.3.11 & qualiter exercitus deberet pergere , \textbf{ tutius posset suum exercitum ducere . } Sic etiam marinarii faciunt , & assi que por vista de los oios catasse en qual manera la hueste pudiesse andar . \textbf{ Mas seguramente podria guiar su hueste } por que assi lo fazen los marineros . \\\hline
3.3.12 & Seruare autem debitum ordinem in acie \textbf{ ut equites et pedites suam aciem seruent , } non sine magno exercitio fieri potest . & mas guardar orden conuenible en la az \textbf{ e que los caualleros e los peones guarden su az } non se puede fazer \\\hline
3.3.14 & quomodo et qualiter bellantes \textbf{ suos hostes inuadere debeant . } Nam cum septem modis enumeratis hostes fortiores existant ; & para se defender de ligero puede paresçer commo \textbf{ e en qual manera los lidiadores deuen acometer sus enemigos . } Ca commo en las siete maneras contadas \\\hline
3.3.21 & abscissis crinibus \textbf{ eos suis maritis tradiderunt : } per quos machinis reparatis & las mugeres de roma cortaron se los cabellos \textbf{ e dieron los a sus maridos } de los quales cabellos fezieron sogas \\\hline

\end{tabular}
