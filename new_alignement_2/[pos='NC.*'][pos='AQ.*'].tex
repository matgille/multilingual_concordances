\begin{tabular}{|p{1cm}|p{6.5cm}|p{6.5cm}|}

\hline
1.1.1 & que asi commo los fechons \textbf{ e las obras singulares e personales } que son manera desta obra dela moral phina & Quia ergo sic est , \textbf{ ipsa acta singularia , } quae sunt materia huius operis , \\\hline
1.1.1 & Onde dize el philosofo mas adelante \textbf{ que de omne sabio es } en tanto demandar çertidunbre de cada cosa & secundum subiectam materiam . \textbf{ Unde subdit , quod disciplinati est , } intantum certitudinem inquirere \\\hline
1.1.1 & Ca semeja la naturaleza \textbf{ Dela sçiençia moral del todo ser contraria ala sçiençia matematica } Ca las demostraçiones mathematicas son çiertas & inquantum natura rei recipit . \textbf{ Videtur enim natura rei moralis | omnino esse opposita negocio mathematico . } Nam demonstrationes mathematicae sunt certae in primo gradu certitudinis , \\\hline
1.1.1 & asi commo Dicho nes \textbf{ Si non por Razones superfiçiales e sensibles } Conuie ne & Et quia hoc fieri non potest \textbf{ ( ut tactum est ) nisi per rationes superficiales et sensibiles : } oportet modum procedendi in hoc opere , \\\hline
1.1.2 & que todo este libro entendemos partir \textbf{ en tres libros particulares ¶ } En el primero libro delos quales demostraremos & quod hunc totalem librum intendimus \textbf{ in tres partiales libros diuidere . } In quorum primo ostendetur , \\\hline
1.1.2 & en el octauo libro delas ethicas \textbf{ que las cosas amigables e buenas } que pertenesçen alos amigos vienen & Unde 9 Ethic’ scribitur , \textbf{ quod amicabilia quae sunt ad amicos , } videntur venisse ex iis , \\\hline
1.1.2 & E delas sçieçias praticas \textbf{ Ca bien asi commo enlas scinas especulatiuas el conoscimjento menguado natanl mete viene ante del conplido } asi enlas siençias praticas & veritatem habet de ipsis operabilibus . \textbf{ Nam sicut in speculabilibus | cognitio imperfecta naturaliter praecedit perfectam : } sic in operabilibus perfectam industriam praecedit astutia imperfecta . \\\hline
1.1.2 & para gouerna mj̊ de çibdado den rregno \textbf{ Conuiene Segund orden natural ala rreal magestad } primeramente que el Ruy sepa gouernar asy mesmo ¶ & quanta in gubernatione ciuitatis et regni : \textbf{ ordine naturali decet regiam maiestatem } primo scire se ipsum regere , \\\hline
1.1.2 & asy mesmo non pue da ser \textbf{ sy non que se de el omne abunos fechos e abunas obras rregladas } por orden de Razon & esse non possit , \textbf{ nisi quis se det bonis actibus , } et bonis operibus regulatis ordine rationis : \\\hline
1.1.2 & Mas las nr̃as obras \textbf{ quanto alo prisente parte nesçe de quatro gujsas } e de quatro maneras las veemos nasçer e departir ¶ & quomodo faciendum sit eas . \textbf{ Operationes autem nostrae ex quatuor } ( quantum ad praesens spectat ) \\\hline
1.1.2 & Ca segunt que dize el philosofo \textbf{ en el segundo libro delas ethicas señal dela } disponiconno delascina engendoͣda & Rursus quia \textbf{ ( ut dicitur 2 Ethicorum ) signum generati habitus , } est delectationem , \\\hline
1.1.2 & en el segundo libro delas ethicas señal dela \textbf{ disponiconno delascina engendoͣda } en el alma es auer & ( ut dicitur 2 Ethicorum ) signum generati habitus , \textbf{ est delectationem , } et tristitiam fieri in opere , \\\hline
1.1.2 & en el alma es auer \textbf{ en la obra deleta çion o tristeza } por que segunt que auemos departidas dispoçones & est delectationem , \textbf{ et tristitiam fieri in opere , } secundum quod alios et alios habitus habemus , \\\hline
1.1.3 & para preguntar actento e acuçioso para rretener e tomar Et \textbf{ pues que ya en el primero capitulo fizimos la magestad Real begniuola } e uolunterosa mostradol aquellas cosas & nisi sit beniuolus , docilis , et attentus : \textbf{ postquam in primo capitulo reddidimus | regiam maiestatem beniuolam , } ostendendo , quae dicenda sunt , \\\hline
1.1.3 & contando la orden de las cosas que aqui auemos de dezir ¶finca que en este terçero capitulo \textbf{ fagamos la Real magestad atenta e acuçiosa } declarandol quanto es el prouecho delas cosas & restat ut in hoc capitulo tertio \textbf{ reddamus eam attentam , } declarando quanta sit utilitas in dicendis . \\\hline
1.1.3 & vegadas los mas de los oydores son begniuolos e uolunterosos \textbf{ a aquellos que les dizen sermones ligeros e superfiçiales } Ca magera que en las . & ut plurimum auditores \textbf{ sunt beniuoli proferentibus sermones faciles , et superficiales . } Quod si in aliis scientiis hoc est corruptio appetitus , \\\hline
1.1.3 & e guaesa non es corrupçion del apetito ñj del desseo del omne \textbf{ Mas es orden derecha e deujda } e atal arte ¶ & non est corruptio appetitus , \textbf{ sed magis est ordo rectus et debitus . } In huiusmodi ergo arte \\\hline
1.1.3 & Mas por el prouecho delas cosas \textbf{ que son de dezer es fecho el oydor atento e acuçioso } para aprender & ex utilitate autem dicendorum \textbf{ redditur auditor attentus , } nam quilibet attente audit , \\\hline
1.1.3 & por que cada vno acuçiosamente oye sy es \textbf{ para quel diran cosas prouechosas . } ¶ pues que asy es delas cosas & nam quilibet attente audit , \textbf{ si sperat se utilia auditurum . } Ex dicendis autem , \\\hline
1.1.3 & que aura a dios \textbf{ e ganaran głoia perdurable ¶ } Et pues que asy es los bienes departen se en vna manera & ut habeat ipsum Deum , \textbf{ et felicitatem aeternam . } Distinguuntur quidem \\\hline
1.1.3 & e nos para resçende fuera . \textbf{ ¶ Los bienes medianeros son ençerrados de dentro del alma } e ño paresçen los quales pueden ser comunales alos buenos e alos malos & Minima , sunt bona exteriora . \textbf{ Media , sunt bona interiora , } quae possunt esse communia bonis , \\\hline
1.1.3 & Et algunos bienes son honestos \textbf{ Mas los bienes honestos son bienes de grant aunataja } Ca enestos bienes honestos & quaedam utilia , quaedam honesta . \textbf{ Bona autem honesta , | sunt bona per excellentiam : } nam in his bonis \\\hline
1.1.3 & e an en sy grant deletaçion \textbf{ e ençierran en sy bondat de tondos los bienes prouechosos ¶ } Pues que asy es commo en este libro ente damos demostrͣ & habent in se magnam delectationem , \textbf{ et includunt bonitatem utilium bonorum . } Cum ergo in hoc libro intendatur , \\\hline
1.1.3 & Pues que asy es commo en este libro ente damos demostrͣ \textbf{ commo la magesad Real aya de ser uertuosa } e commo conuiene alos rreys & Cum ergo in hoc libro intendatur , \textbf{ quomodo maiestas regia fiat virtuosa , } et quomodo eos , \\\hline
1.1.3 & non obedesçieren al rrey o al cabdiello \textbf{ asy qual quier omne singular non puede auer asy mesmo } sy el apetito discordare dela Razon e del entendemjento & qui non obediant regi , vel duci : \textbf{ sic homo aliquis singularis dicitur non habere seipsum , } si appetitus dissentiat rationi , \\\hline
1.1.3 & la qual cosa conteçe \textbf{ en aquellos que han el alma mala e desordenada } Estonçe el omne non es ayuntado consigo . & ( quod contingit \textbf{ in habentibus animam peruersam , ) } tunc homo non est unitus , \\\hline
1.1.3 & Mas aquellos que son conplidos de uirtudes \textbf{ e abondados en bienes honestos por que siguen la rregla dela rrazon } e siguen los bienes & Sed pollentes virtutibus , \textbf{ et abundantes in bonis honestis , | quia consequuntur regulam rationis , } et consequuntur bona simpliciter , \\\hline
1.1.3 & e ganaremos a dios \textbf{ e por el ganaremos la eglesia perdurable ¶ } Mas por que estas cosas de que auemos de dar doctrina & et habebimus ipsum Deum , \textbf{ et per consequens felicitatem aeternam . } Verum quia ea , \\\hline
1.1.3 & que demande mucho afincadomente la gera de dios \textbf{ Ca quanto la magestad rreal esta en logar } mas alto tanto ha & implorare diuinam gratiam . \textbf{ Nam quanto maiestas regia in loco altiori consistit , } tanto magis indiget diuina gratia , \\\hline
1.1.4 & e traher los sus subienctos a uirtudes . \textbf{ puestos ya vnos preanbulos neçesarios al proposito } Ca en rrelaçion desta obra & et quomodo in \textbf{ Praemissis quibusdam praeambulis necessariis ad propositum , } quia respectu sequentis operis \\\hline
1.1.4 & Ca en rrelaçion desta obra \textbf{ que se sigue feziemos la Real magestad begniuola et uolunterosa para oyr } e a prinder & Praemissis quibusdam praeambulis necessariis ad propositum , \textbf{ quia respectu sequentis operis } ex facilitate modi tradendi \\\hline
1.1.4 & por ende auemos a comneçar en la fin \textbf{ e en la bien andança de todos los bienes obrantes . } ¶ pues que Asy es commo departidos omes & ut dicebatur supra , \textbf{ ideo a fine et felicitate inchoandum est . } Cum ergo secundum diuersos modos viuendi \\\hline
1.1.4 & por el primero libro delas ethicas ¶ \textbf{ Conuien de saber ujda delectosa et plazentera ¶ } vida politica e çiuil¶ & ( ut patet ex 1 Ethic’ ) \textbf{ triplicem vitam , } videlicet , voluptuosam , politicam , et contemplatiuam . \\\hline
1.1.4 & vida politica e çiuil¶ \textbf{ Et uida contenplatina e acabada¶ } Ca veyen los philosofos & triplicem vitam , \textbf{ videlicet , voluptuosam , politicam , et contemplatiuam . } Videbant enim hominem esse medium inter superiora , et inferiora : \\\hline
1.1.4 & Ca quisieron alguons philosofos \textbf{ que al omne conuiene la uida delectosa en } quanto partiçipara con las bestias & a Philosophis praedictae \textbf{ tres vitae voluerunt enim quod homini } ut communicat cum brutis , \\\hline
1.1.4 & delectosa biue commo bestia \textbf{ segunt vida çiuil e ordenada } biue commo omne . & viuit ut bestia : \textbf{ secundum ciuilem , } viuit ut homo : \\\hline
1.1.4 & asy commo omne aujendo ensy pradençia e sabiduria \textbf{ que es Razon derecha de todas las cosas } que ha de obrar e de fazer ¶ & habendo in se prudentiam , \textbf{ quae est recta ratio agibilium : } dicatur felix contemplatiue , \\\hline
1.1.4 & en que es algua cosa diujnal \textbf{ e algua cosa mejor que omne } pues que asy es llaman al omne acabado en las obras & sed ut est in eo aliquid diuinum , \textbf{ et aliquid melius homine . } Perfectum igitur in agibilibus , \\\hline
1.1.4 & por Razon njn por sabiduria mas gouiernase por pasion \textbf{ e biue vida delectosa e bestial . } Mas sy es omne diuinal e mejor que omne . & sed regitur passione , \textbf{ et viuit vita voluptuosa . } Sed si sit diuinus , \\\hline
1.1.4 & e biue vida delectosa e bestial . \textbf{ Mas sy es omne diuinal e mejor que omne . } Estonçe escudrinan por sabiduria & et viuit vita voluptuosa . \textbf{ Sed si sit diuinus , | et homine melior , } tunc speculatur per sapientiam , \\\hline
1.1.4 & e por entendimjento las cosas \textbf{ e biue vida contenplatina e çelestial } pues que asy es tanto es el d partimjento & tunc speculatur per sapientiam , \textbf{ et viuit vita contemplatiua . } Tanta est ergo differentia \\\hline
1.1.4 & que ha de fazer \textbf{ Et el acabado ente las sçiençias especulatians quanta es } entre el que biue vida humanal e vida politica & inter prudentem in agibilibus , \textbf{ et perfectum in speculabilibus , } quanta est inter viuentem vita humana et politica , \\\hline
1.1.4 & Ca mager que dissiese nudat \textbf{ que en la vida seliçonsa non es de poner bien andança } asy como adelante lo mostraremos mas claramente & Nam licet vere dixerunt \textbf{ quod in vita voluptuosa | non est quaerenda felicitas , } ut infra clarius ostendetur : \\\hline
1.1.4 & e beuir acabadamente \textbf{ segunt vida actiua e de obrar } Et segunt vida contenplatian e entendimjento & et perfecte viuere \textbf{ secundum vitam actiuam , } vel contemplatiuam . \\\hline
1.1.4 & Et avn pusieron los philosofos \textbf{ que la vida contenplatiua estapuramente en el entender } la qual cosa es falsa & Posuerunt etiam vitam contemplatiuam \textbf{ esse in pura speculatione . } Quod est falsum . \\\hline
1.1.4 & Et conviene le de foyr \textbf{ e de arredrarse dela vida delectosa e carnal } por qua non sea peor que omne . & hos modos viuendi cognoscere , \textbf{ et vitam voluptuosam fugere , } ne sit homine peior : \\\hline
1.1.4 & e cada vn prinçipe es dicho avn \textbf{ en sy vida actiua e de obrar ¶ Et contenplatian e de entender . } por que por la ujda actiua & sibi subditos recte regendo . \textbf{ Per vitam contemplatiuam vacet sibi per internam deuotionem } et Dei dilectionem , \\\hline
1.1.5 & asi commo si el primero mouedor non fuese ninguno otro non seria mouedor \textbf{ asi si la fin postrimera non fuese } Et sy nos non conosçiesemos alguna cosa & nullum agens ageret : \textbf{ sic si non esset finis ultimus , } et si non apprehenderemus aliquid , \\\hline
1.1.5 & dize \textbf{ que para lanr̃auida grant acresçentamiento } faze connosçer ante la fin & necessariam esse praecognitionem finis , ait , \textbf{ quod cognitio finis | ad vitam nostram magnum habet incrementum : } consequemur enim \\\hline
1.1.5 & e la buena uentraa \textbf{ que omne ha por las buean s obras las obras guaues e fuertes de fazer se } fazen muy delectables e plazenteras ¶ & sed etiam delectabiliter . \textbf{ Nam grauia efficiuntur delectabilia et dulcia , } considerata beatitudine , et felicitate , \\\hline
1.1.5 & que ningun bien singłar njn personal ¶ \textbf{ pues que asy co asaz paresçe } que muy mas conuiene al Reio al prinçipe conosçer la su fin & quam bonum aliquod singulare . \textbf{ Patet ergo , } quod maxime decet regiam maiestatem \\\hline
1.1.6 & comunalmente non siente \textbf{ si non las delectaçiones sensibles . } Et por ende es que communalmente los omes las delecta connes & Vulgus communiter non percipit , \textbf{ nisi delectationes sensibiles : } et inde est quod communi nomine \\\hline
1.1.6 & e mal altas \textbf{ que las delectaçonnes sensibles de los sesos ¶ } Mas por tres Razones podemos nos prouar & sint potiores , et excellentiores , \textbf{ quam voluptates sensibiles . } In huiusmodi autem voluptatibus sensibilibus \\\hline
1.1.6 & Mas por tres Razones podemos nos prouar \textbf{ que en estas delecta çonnes sensibles de los sesos . } non es de poner la feliçidat e la bien andança . & quam voluptates sensibiles . \textbf{ In huiusmodi autem voluptatibus sensibilibus } non esse felicitatem ponendam , \\\hline
1.1.6 & nin son bien segunt el alma \textbf{ nin segunt Razon assaz paresçe } que en las tales delectaçiones & nec bonum secundum animam , \textbf{ et secundum rationem , } constat in talibus \\\hline
1.1.6 & ¶ Onde el philosofo en el terçero libro delas ethicas fablando de tales delectaçonnes dize \textbf{ que el apetito delectable de los sesos } non se puede fartar delas delectaçones & unde Philosophus 3 Ethi’ loquens de talibus delectationibus ait , \textbf{ quod insatiabilis est delectabilis appetitus . } Secundo in talibus non est ponenda felicitas , \\\hline
1.1.6 & Por que quantomayores son los bienes \textbf{ tantom fazen la razon libre e desenbargada } Ca lo que es segunt razon non conuiene & quanto maiora sunt , \textbf{ tanto magis reddunt rationem liberam , et expeditam . } Nam quod secundum rationem existit , \\\hline
1.1.6 & que enbargue ala Razon . \textbf{ mas estas delectaçonnes sensibles e carnales oscureçen e enbargan la razon } quando son grandes e fuertes . & quod rationem impediat : \textbf{ sed huiusmodi delectationes sensibiles } si vehementes sint , \\\hline
1.1.6 & en el terçero libro delas ethicas \textbf{ que si las tales delecta connes carnales fueren grandes e afincadas } çiegan la razon e el entendimiento & iuxta illud 3 Ethicorum . \textbf{ Si tales delectationes magnae , | et vehementes sint , } cognitionem idest rationem percutiunt . \\\hline
1.1.6 & que maguera que alguas delecta connes sean conuenibles e honestas \textbf{ por que las obras uirtuosas fazen al omnen bueno e uirtuoso e de grant delectaçion . } Enpero njnguna delectaçion non es feliçadat & quod licet sint delectationes aliquae licitae , et honestae , \textbf{ eo quod ipsa opera virtutum Homini bono , | et virtuoso magnam delectationem faciant : } nulla tamen delectatio est essentialiter ipsa felicitas , \\\hline
1.1.6 & ala feliçidat e ala bien andança . \textbf{ Mas declarar esto non parte nesçe a esta arte presente . } Enpero que por auentra a adelante diremos alguna cosa desto . & licet possit esse aliquid felicitatem consequens , \textbf{ sed hoc declarare non est praesentis negocii . } Forte tamen de hoc aliquid infra dicetur . \\\hline
1.1.6 & e la su bien andança \textbf{ enlas delectaçiones sensibles e dela carne ¶ } Pues que asi es mucho es de denosta & suam felicitatem ponere \textbf{ in delectationibus sensibilibus . } Est ergo detestabile cuilibet Homini \\\hline
1.1.6 & por que non sea menospreçiado de su pueblo¶ \textbf{ La terçera razon por que el prinçipe non deue poner su bien andança en las delecta çonnes corporales es esta } e conuiene al & ne contemptibilis uideatur . \textbf{ Tertio decet | Principem talia detestari , ne principari efficiatur indignus , } nam nullus eligit Iuuenes in Duces , \\\hline
1.1.6 & por la qual cosa asy commo sy \textbf{ fuese el omne uieio en tp̃o } e moço en las costunbres non seria digno de ser prinçipe asy si fuer moço en hedat & propter quod sicut si sit Senex tempore , \textbf{ et Puer moribus , } indigne principatur ; \\\hline
1.1.6 & fuese el omne uieio en tp̃o \textbf{ e moço en las costunbres non seria digno de ser prinçipe asy si fuer moço en hedat } e vieio en costunbres & et Puer moribus , \textbf{ indigne principatur ; | sic si sit Puer aetatae , } et Senex moribus , \\\hline
1.1.7 & e alguons son artifiçiales \textbf{ ¶Riquezas natraales son dichas aquellas } que uienen n atraalmente de cosas natraales & quaedam sunt artificiales . \textbf{ Diuitiae naturales dicuntur esse , } quae naturaliter \\\hline
1.1.7 & e para el uestir \textbf{ son contadas entre las riquezas natraales ¶ } Mas riquezas artisiçiales son aquellas & et vestitum deseruiunt , \textbf{ inter naturales diuitias computantur . } Artificiales autem diuitiae sunt , \\\hline
1.1.7 & son contadas entre las riquezas natraales ¶ \textbf{ Mas riquezas artisiçiales son aquellas } que por arte e sabiduria de los omes son falladas e comuertidas e camiadas & inter naturales diuitias computantur . \textbf{ Artificiales autem diuitiae sunt , } quae per artem , \\\hline
1.1.7 & que luego por sy non cunplen \textbf{ nin satisfazen alas menguas corporales . } Mas por comutaçion e por camio siruen alas menguas corporales . & cuiusmodi sunt aurum , et argentum , et uniuersaliter omne numisma , \textbf{ quae immediate indigentias corporales non supplent , } sed per commutationem deseruiunt indigentiae corporali . \\\hline
1.1.7 & nin satisfazen alas menguas corporales . \textbf{ Mas por comutaçion e por camio siruen alas menguas corporales . } Ca el oro e la plata mager & quae immediate indigentias corporales non supplent , \textbf{ sed per commutationem deseruiunt indigentiae corporali . } Aurum enim , \\\hline
1.1.7 & por las quales nos pondemos prouar \textbf{ que la feliçidat e la bien andança non es de poner en les riquezas artifiçiales ¶ } La primera razon es por que las riquezas artifiçiales son orderandas alas riquezas natraales ¶la segunda & propter quae venari possumus , \textbf{ in artificialibus diuitiis | felicitatem non esse ponendam . } Primo , quia artificiales diuitiae \\\hline
1.1.7 & que la feliçidat e la bien andança non es de poner en les riquezas artifiçiales ¶ \textbf{ La primera razon es por que las riquezas artifiçiales son orderandas alas riquezas natraales ¶la segunda } por que las Rianzas ar tifiçiales han que sean & felicitatem non esse ponendam . \textbf{ Primo , quia artificiales diuitiae } ad naturales ordinantur . \\\hline
1.1.7 & por el ordenamiento de los omes \textbf{ ¶ Onde el philosofo enl primero libro delans politicas dize que trismudados los usadores } e trismudado el ordenamiento de los usadores & nisi ex ordinatione Hominum . \textbf{ Unde 1 Politicorum dicitur , | transmutatis utentibus , } idest transmutata dispositione utentium , \\\hline
1.1.7 & trismudada la orden dela disposiçon de los omes \textbf{ Estas riquezas artifiçiales non han ninguna dignidat ni ningun prouech̃o . } Ca que tanto de oro o tanta moneda & et dispositione hominum , \textbf{ huiusmodi diuitiae nullam habent dignitatem , | neque utilitatem , } quod autem tantum auri , \\\hline
1.1.7 & que non es de poner la feliçidat \textbf{ e la bien andança en las riquezas artifiçiales ¶ } Lo primero por que los riquezas e las monedas artifiçiales & vere essent diuitiae , \textbf{ et vere satisfacerent indigentiae corporali . } Tum ergo quia numismata sunt diuitiae \\\hline
1.1.7 & son riquezas ordenandas a otra cosa . \textbf{ conuiene saber a los riquezas natraales ¶ } Lo segundo que por que estas Riquezas son riquezas & in ordine ad aliud , \textbf{ tum quia sunt diuitiae } ex institutione Hominum , tum quia cum sint corporalia , \\\hline
1.1.7 & nin conplir \textbf{ por si alas menguas corporales . } Et por ende non es de poner la bien andança enellas . & per se non sufficiunt , \textbf{ in eis non est ponenda felicitas . } Quod autem in naturalibus diuitiis , \\\hline
1.1.7 & ¶pres que assi es mucho es de denostar todo en que pone su feliçidat \textbf{ e su bien andança en las riquezas corporales . } Mas mayor mente es de denostar la Real magestad & Cuilibet ergo Homini detestabile est \textbf{ ponere suam felicitatem in diuitiis , } sed maxime detestabile est regiae maiestati . \\\hline
1.1.7 & pone su \textbf{ feliçadato su bien andança en las riquezas corporales . } tres males muy grandes se le siguen & Nam si Rex aut Princeps ponat suam felicitatem in diuitiis , \textbf{ tria maxima mala inde consequuntur . } Primo , quia amittit maxima bona . \\\hline
1.1.7 & tres males muy grandes se le siguen \textbf{ dende ¶ El pmero mal es que pierde muy grandes bienes ¶ } El segundo es que se faze por ende tirano quiere dezer le unadoro . & tria maxima mala inde consequuntur . \textbf{ Primo , quia amittit maxima bona . } Secundo , quia efficitur Tyrannus . \\\hline
1.1.7 & nin de grant coraçon . \textbf{ Ca temiendo deꝑder los des e las riquezas nunca acometra grandes cosas } Et la razon es esta & nec etiam potest esse Magnanimus , \textbf{ quia metuens pecuniam perdere , | nihil magnum attentabit . } Immo ( cum ille sit Magnanimus , \\\hline
1.1.7 & assi mucho es de denostar el Rei o el prinçipe \textbf{ que pone su bien andança en las riquezas corporales . } ¶ Ca por esto se fare tirano & Secundo detestabile est Regi , \textbf{ vel Principi suam felicitatem ponere in diuitiis , } quia hoc facto Tyrannus efficitur . \\\hline
1.1.7 & de poner el Rey la su feliçidat \textbf{ e la su bien andança en las riquezas corporales . } ora uentra a muchos biuen uida politica & et depraedatorem detestabile \textbf{ quoque est suam felicitatem | in diuitiis ponere . } Forte multi viuentes vita politica credunt \\\hline
1.1.8 & que fazen unos omes a otros en testimoino de uirtud \textbf{ por que son omes uirtuosos ¶ } Pues que assi es quando alguon demanda & in testimonium virtutis . \textbf{ Causa ergo , } quare maxime homines volunt honorari , \\\hline
1.1.8 & por que la reuerençia delos tales \textbf{ non es testimonio conuenible de sabiduria nin de uirtud } Por la qual razon si la honrra es bien & Ideo secundum Philosophum nullus turat honorari a Pueris , \textbf{ quia reuerentia talium non est debitum testimonium Sapientiae , } et virtutis \\\hline
1.1.8 & mas es de poner en los bienes del alma \textbf{ que son bienes mayores . } Siguese que la bien andança non se deue poner en las honrras & sed in intrinsecis , \textbf{ quae sunt bona maiora , } in honoribus felicitas poni non debet . \\\hline
1.1.8 & e la su bienandança en las honrras sera malo en si mesmo \textbf{ e non fara fuerça de ser bueno mas de paresçer bueno . } Et sera malo al pueblo & erit malus in se , \textbf{ quia non curabit esse | nisi superficialiter bonus : } erit malus in populo sibi commisso , \\\hline
1.1.9 & Ca la alabaça propriamente non es si non \textbf{ por sseñales uocales e de palabras } Mas la honrra a de ser quales por se quier sseñales & nam laus proprie non est \textbf{ nisi per signa vocalia ; } sed honor esse habet \\\hline
1.1.9 & non solamente asauida de los bueons \textbf{ mas delos omes malos . } Ca por ꝑ muchas uegadas somos engannados en iudgando & non solum habetur de bonis , \textbf{ sed etiam de hominibus prauis : } quia enim multoties \\\hline
1.1.9 & que en la bondat . \textbf{ Et non es cosa conuenible de poner la bien andança } en aquello que puede auer los malos njn & inconueniens est felicitatem ponere in eo , \textbf{ quod est signum bonitatis , | quam sit bonitas , } et quod ipse praui participare possunt , \\\hline
1.1.9 & mas sera bien auentra ado \textbf{ si fuer głioso ante dios } Ca dicho nauemos . & attamen beatus est , \textbf{ si sit in gloria apud Deum . } Dictum enim est , \\\hline
1.1.9 & ya desuso que la eglesia \textbf{ e la fama es vn connosçimiento claro ton alabança } mas quanto parte nesçe alo presente en tres cosas & Dictum enim est , \textbf{ quod gloria et fama , } est quaedam clara cum laude notitia . \\\hline
1.1.9 & Et segunt los estrologos \textbf{ cada vna delas estrellas fixas notables } ala uista del omes & et secundum Astronomos \textbf{ quaelibet stella fixa visu notabilis } sit maior tota terra , \\\hline
1.1.9 & delos omes es muy breue e non duradera . \textbf{ Ca todo el tpon dela uida presente de los omeses } assi commo vn momneto & Rursus homini fama est breuis non diuturna , \textbf{ cum totum tempus vitae praesentis sit } quasi punctale respectu Dei aeternitatis : \\\hline
1.1.9 & assi commo vn momneto \textbf{ e vn punto en conparaçion del anuida perdurable . } Ca commo el omne segunt el alma & cum totum tempus vitae praesentis sit \textbf{ quasi punctale respectu Dei aeternitatis : } cum enim Homo \\\hline
1.1.9 & Ca non por que la honrra sea \textbf{ gualardon igual nin digno alos meresçimientos } assi commo ally dize el philosofo ¶ & non quod honor fit \textbf{ condigna retributio eis , } ut ibidem dicitur . \\\hline
1.1.9 & e \textbf{ gualardon egual e digno al su mesçimiento ¶ } La primera manera es pensando & potest intelligi dupliciter , \textbf{ vel ratione ipsius honoris in se , } vel ut procedit \\\hline
1.1.9 & Et por ende non es \textbf{ gualardon ygual nin digno al su meresçimiento . } Mas enpero teniendo mientes ala honrra & et quoddam testimonium bonitatis , \textbf{ ut patet , non est condigna retributio in vita . } Attamen ut huiusmodi honor procedit \\\hline
1.1.10 & en la entençion del prinçipe \textbf{ que abonde en poderio ciuil que es poderio de çibdat e de regno } e que por este poder puereda subiugar & hoc esse debet principalissimum in intentione Principis , \textbf{ quod abundet in ciuili potentia , } et quod per eam sibi subiiciat nationes et gentes . \\\hline
1.1.10 & e las gentes \textbf{ por poderio çiuil esto esquerer } enssen onrear por fuerça & velle sibi subiicere nationes , \textbf{ hoc est , velle dominari per violentiam . } Violentia autem perpetuitatem nescit . \\\hline
1.1.10 & Et por ende la feliçidat \textbf{ e la bien andaça non es de poner en ninguna cosa passadera nin tenporal . } Mas es de poner en aquello & si per violentiam , \textbf{ et per ciuilem potentiam dominetur : } quia tale dominium \\\hline
1.1.10 & si aquella cosa non fuere muy blanca . \textbf{ Mas que el poderio çiuil pueda seer } en alguno sin bondat deuida paresçe & nisi ille fit intense albus . \textbf{ Quod autem ciuilis potentia possit } inesse alicui \\\hline
1.1.10 & assi commo dize el philosofo en las politicas \textbf{ que dionisio siracusano o dionisios çiçiliano ouo muy grant poderio çiuil } empero fue muy mal tirano & ut recitat Philosophus in politicis , \textbf{ maxime abundauit in ciuili potentia , } et tamen pessimus Tyrannus erat . \\\hline
1.1.10 & Masnero e cesar que fueron prinçipes Ro manos \textbf{ e abondaron mucho en poderio çiuil . } Empero viuian muy mal & qui fuerunt Romani Principes , \textbf{ maxime abundauerunt in ciuili potentia ; } pessime tamen viuebant . \\\hline
1.1.10 & que la feliçidat \textbf{ et la bien andança non es de poner en este poderio çiuil . } Por que este sennorio non es muy bueno nin muy digno . & Tertio in huiusmodi potentia \textbf{ non est ponenda felicitas , } quia huiusmodi Principatus non est optimus , \\\hline
1.1.10 & muy bueno e muy digno . \textbf{ Mas ensennorear por poderio çiuiles ensseñorear alos sieruos } e non alos libres que han libertad . & ponenda est in Principatu optimo , et digno . \textbf{ Principari autem per ciuilem potentiam , | est principari seruis , } non liberis : \\\hline
1.1.10 & e mas digno que ensseñorear a los sieruos . \textbf{ Et por que el señorio por fuerça e por poderio çiuil commo non sea delons libres } mas de los sieruos non puede ser muy bueno nin digno¶ & Principatus ergo per coactionem , \textbf{ et ciuilem potentiam , | cum non sit liberorum , } sed seruorum , \\\hline
1.1.10 & e la bien andança \textbf{ non se deue poner en poderio çiuil . } Por que si el prinçipe o el Rey crea & Quarto non est ponenda felicitas \textbf{ in ciuili potentia : } quia si Princeps se crederet \\\hline
1.1.10 & que es bien auen traado \textbf{ por que abonda en poderio çiuil non ordenara los çibdadanos } si non a exerçiçio o auso de armas . & esse felicem , \textbf{ si abundet in ciuili potentia , | non ordinabit ciues , } nisi ad exercitum armorum , \\\hline
1.1.10 & deno stando alos griegos \textbf{ por que ponian la su bien andança en el poderio çiuil . } Et dize assi que torpe cosa es & ponentes felicitatem \textbf{ in ciuili potentia , } ait , turpe esse , \\\hline
1.1.10 & por las quales cosas ya dichas \textbf{ si non es cosa conuenible de poner la bien andança en alguna cosa } que non sea duradera & Quare si inconueniens est \textbf{ ponere felicitatem } in aliquo non diuturno , \\\hline
1.1.10 & que non es cosa conuenible \textbf{ que el prinçipe ponga su bien andança en poderio çiuil . } Et por que cuyde que es bien & inconueniens est etiam Principem \textbf{ ponere suam felicitatem | in ciuili potentia , } et quod credat se esse felicem , \\\hline
1.1.11 & ¶ \textbf{ Ca primeramente que omne goste e siente } que le son estos bienes corporales paresçen le mayores de quanto son & magis desiderantur : \textbf{ prius enim , | quam talia bona praegustentur , } creduntur esse maiora , \\\hline
1.1.11 & Ca primeramente que omne goste e siente \textbf{ que le son estos bienes corporales paresçen le mayores de quanto son } Mas despues que los ha auido paresçen meno res de quanto el cuydaua . & quam talia bona praegustentur , \textbf{ creduntur esse maiora , | quam sint : } eis autem adeptis , \\\hline
1.1.11 & Mas las uertudes del alma \textbf{ e los bienes intellectuales han contraria manera . } Ca quando los omes los han paresçen les mayores de quanto cuydaua & Virtutes autem , \textbf{ et bona intellectualia modum conuersum tenent : } nam eis adeptis , \\\hline
1.1.11 & que omne ha de escoger \textbf{ que la salut es egualamiento conuenible de los humores . } Et la fermosura es mesuramiento conuenible de los mienbros . & Nam ( ut vult Philosop’ 3 De eligendis ) \textbf{ sanitas est debita adaequatio humorum . } Pulchritudo est debita commensuratio membrorum . \\\hline
1.1.11 & que la salut es egualamiento conuenible de los humores . \textbf{ Et la fermosura es mesuramiento conuenible de los mienbros . } Et la fortaleza conuenible proporçion de los huessos & sanitas est debita adaequatio humorum . \textbf{ Pulchritudo est debita commensuratio membrorum . } Robur debita proportio ossium et neruorum . \\\hline
1.1.11 & Et la fermosura es mesuramiento conuenible de los mienbros . \textbf{ Et la fortaleza conuenible proporçion de los huessos } e delons nieruos . & Pulchritudo est debita commensuratio membrorum . \textbf{ Robur debita proportio ossium et neruorum . } Quare cum humores , membra , nerui , \\\hline
1.1.11 & Por la qual cosa commo los humores e los mienbros e los neruios \textbf{ e los huessos sean cosas corporales . } la sanidat e la formosura & Quare cum humores , membra , nerui , \textbf{ et ossa sint corporalia , } sanitas , pulchritudo , \\\hline
1.1.11 & la sanidat e la formosura \textbf{ e la fuerça del cuerpo son cosas corporales . } Et por ende en tales bienes & sanitas , pulchritudo , \textbf{ et robur corporalia esse dicuntur ; } non ergo in eis est ponenda felicitas . \\\hline
1.1.11 & por que son bienes de fuera . \textbf{ Mas en los bienes solos de dentro del alma } es propiamente de poner la feliçidat & felicitas poni non debet . \textbf{ Quod autem in bonis interioribus sit proprie felicitas , } patet per Philosophum 7 Politicorum dicentem , \\\hline
1.1.11 & Mas perdida la sanidat del cuerpo \textbf{ pierde se la fuerça corporal . } Ca esflaqueçidos los . mienbros & Sed amissa sanitate , \textbf{ amittitur robur corporis , } quia laxantur membra , \\\hline
1.1.11 & e delons huessos \textbf{ en los quales esta la fortaleza corporal . } Et en essa misma gnisa ahun se tira la fermosura deł cuepo . & in quibus habet \textbf{ esse fortitudo corporalis . } Sic etiam tollitur pulchritudo , \\\hline
1.1.11 & en los quales esta la fortaleza corporal . \textbf{ Et en essa misma gnisa ahun se tira la fermosura deł cuepo . } Ca perdida la sanidat & esse fortitudo corporalis . \textbf{ Sic etiam tollitur pulchritudo , } quia amissa sanitate , \\\hline
1.1.11 & Ca perdida la sanidat \textbf{ fazen se los mienbros magros e flacos } e non finca en ellos color natraal & quia amissa sanitate , \textbf{ membra fiunt macilenta , } non remanet in eis color debitus : \\\hline
1.1.11 & e que ligeramente se pueden mudar e perder¶ \textbf{ Pues que assi es para entender todas las cosas sobredichas digamos } que non conuiene al Rey & et de facili mutabilia . \textbf{ Dicamus ergo ad intelligentiam omnium dictorum , } quod non decet \\\hline
1.1.11 & ¶ \textbf{ Ahun en essa misma manera es el Rey digno de honrra } e deue seer honrrado & quod sine diuitiis fieri non potest . \textbf{ Sic etiam , } ne vilipendatur maiestas regia , \\\hline
1.1.11 & por que non sea menospreçiada la Real magestad . \textbf{ Et por ende le conuiene de auer poderio çeuil . } Ca por el & est Rex dignus honore , \textbf{ et expedit ei habere ciuilem potentiam : } nam propter paruipensionem Principis , \\\hline
1.1.11 & Ca por el \textbf{ menospreçiamientodel prinçipe muchͣs vezes contesçe que alguons fazen e obran malas cosas } e fazen muchos tuertos a otros & et expedit ei habere ciuilem potentiam : \textbf{ nam propter paruipensionem Principis , | ut plurimum aliqui operantur mala , } et offendunt alios , \\\hline
1.1.12 & Et la uirtud acabada en la \textbf{ uidaçon tenplatiua es sapiençia o methaphisica } segund esse mismo philosofo ¶ Et & secundum virtutem perfectam . \textbf{ Cum igitur perfecta virtus } secundum Philosophum \\\hline
1.1.12 & e non lo ha conplidamente es instrumento de aquel que la ha naturalmente e conplidamente . \textbf{ Et pues que assi es commo dios solo aya poderio de regnar } e de gouernar prinçipalmente e acabadamente . & est instrumentum et organum eius , \textbf{ quod habet illud essentialiter et perfecte , | quia ergo vim regitiuam , } et potentiam regendi habet principaliter , \\\hline
1.1.13 & por que gouierne su regno \textbf{ por ley e por sabiduria bien commo dios gouienna todo el mundo . } Et assi el rey es mas conformado a . dios . & et prouidentiam suum regnum regere , \textbf{ quomodo Deus totum uniuersum regit et gubernat , } Rex maxime conformatur ei : \\\hline
1.1.13 & que otro ninguon Et por ende en conparaçon de dios de quien es para \textbf{ gualardon grande es el su } gualardon si gouernare derechamente su regno . & a quo praemium expectatur , \textbf{ est magna merces Regis , } si regnum suum recte regat , quia ex hoc ipsi Deo maxime conformatur . \\\hline
1.1.13 & es esta penssando en las obras por las quales han de auer su \textbf{ gualardon bueno o mal . } Ca por tanto es dicha la obra uiçiosa e mala & considerato actu , \textbf{ per quem tale meritum habet esse . } Nam ex hoc actus est vitiosus , \\\hline
1.1.13 & gualardon bueno o mal . \textbf{ Ca por tanto es dicha la obra uiçiosa e mala } por quanto es contra la natura & per quem tale meritum habet esse . \textbf{ Nam ex hoc actus est vitiosus , } inquantum est contra naturam , \\\hline
1.1.13 & en que obra el rey \textbf{ la qual materia es la gente e la muchedunbre delons pueblos esta muestra } que el su gualardon es muy grande¶ & circa quam operatur Rex , \textbf{ quae est gens , | et multitudo , } indicat eius praemium esse magnum . \\\hline
1.2.1 & e por amor de dios conasçion dol \textbf{ e amandol assi conmo oficiales uerdaderos suyos } enderesçen e guyen el pueblo & ut cognoscentes , et diligentes Deum , \textbf{ tanquam veri ministri eius | secundum ordinem rationis } dirigant Populum sibi commissum . \\\hline
1.2.1 & ca alguons destos poderios del alma son naturales \textbf{ e algunos son poderios senssitiuos conosçedores . } Et algunos desseadores & quaedam sunt naturales , \textbf{ quaedam cognitiuae sensitiuae , } quaedam appetitiuae , \\\hline
1.2.1 & e alas plantas commo anos ¶ \textbf{ Mas los pode rios conosçedores de los sesos son } assi commo el viso & et talia quae etiam ipsis arboribus competunt . \textbf{ Potentiae vero cognitiuae sensitiuae , } sunt visus , gustus , auditus , \\\hline
1.2.1 & Et estos tales son comunes tan bien alas bestias commo anos ¶ \textbf{ Mas los poderios desseadores se departen en esta gusa . } Ca alguno dellos es propio al omne & in quibus communicamus cum brutis . \textbf{ Appetitiuae vero distinguuntur : } nam quidam appetitus est in homine , \\\hline
1.2.1 & es llamando uoluntad . \textbf{ Segund essa manera de fablar las bestias han senssualidat et appetito sensitiuo . } Mas non han uoluntad nin appetito & sequens intellectum nominatur voluntas : \textbf{ secundum quem modum loquendi bruta habent sensualitatem , | et appetitum sensitiuum , } sed non habent voluntatem , et intellectum . \\\hline
1.2.1 & o en todas estos o en alguons dellos . \textbf{ Mas que en los poderios naturales non pueden ser las uirtudes podemos lo prouar } por tres razons ¶ & vel in aliquibus horum . \textbf{ In potentiis autem naturalibus | esse non possunt , } quod tripliciter patet . \\\hline
1.2.1 & ¶ Et pues que assi es commo la natura sea determimada a vna cosa \textbf{ e los poderios naturales sean determinandos conplidamente para obrar } segund su natura & cum ergo natura sit determinata ad unum , \textbf{ et potentiae naturales sufficienter determinentur ad agendum , } ex natura sua \\\hline
1.2.1 & por los conosçimientos de los sesos \textbf{ nin por los poderios senssituos . } Ca assi commo ningun omne non es alabado & sic non laudamur , \textbf{ nec vituperamur ex sensibus . } Nam sicut nullus laudatur \\\hline
1.2.1 & La segunda razon es por que las uirtudes morales \textbf{ non deuen ser puestas enlos poderios senssibles . } porque assi commo los poderios naturales & et cibo moderate . \textbf{ Secundo in sensibus non est ponenda virtus moralis , } quia sicut potentiae naturales \\\hline
1.2.1 & mas claramente nin menos . \textbf{ Por la qual razon si las uirtudes morales son regladas } segunt razon ellas non deuen ser puestas en los sesos ¶ la terçera razon por que las uirtudes morałs & vel minus clare : \textbf{ quare si virtus moralis est aliquid secundum rationem , | huiusmodi virtus } in sensibus poni non debet . Tertio in talibus non est moralis virtus , \\\hline
1.2.1 & non deuen ser puestas en los sesos es esta \textbf{ Ca assi commo los poderios naturales son conplidamente determinados a sus obras por naturaleza . } Bien assi los sesos et los poderios & in sensibus poni non debet . Tertio in talibus non est moralis virtus , \textbf{ quia sicut potentiae naturales } sufficienter determinantur \\\hline
1.2.1 & por que escalienta \textbf{ nin es en el uirtud moral . } Bien assi por que el poderio de moler la uianda & nec laudatur , nec vituperatur , \textbf{ nec est in eo virtus moralis : } sic quia potentia digestiua \\\hline
1.2.1 & non son en los poderios naturales \textbf{ nin en los sesos naturales . } Commo sin estos poderios naturales & Si ergo nec in potentiis naturalibus , \textbf{ nec in sensibus est virtus moralis , } cum praeter potentias naturales , \\\hline
1.2.2 & Et algunas medianeras entre las intellectuales e las morales ¶ \textbf{ Las uirtudes intellectuales son dichas aquellas } que son en el entendimiento especulatiuo & inter intellectuales et morales . \textbf{ Virtutes intellectuales dicuntur illae } quae sunt in intellectu speculatiuo , \\\hline
1.2.2 & que esta enl entendimiento \textbf{ assi commo la ph̃ia natural e la geometria e la methaphisica } e las otras sçiençias tałs¶ & cuiusmodi sunt scientiae speculatiuae , \textbf{ ut naturalis Philosophia , Geometria , Metaphysica , } et caetera talia . \\\hline
1.2.2 & en este libͤespeçialmente de cada vna dellas ¶ \textbf{ Mas las uirtudes medianeras entre los intellectuales } e las morales & Virtutes autem mediae \textbf{ inter intellectuales et morales , } sunt virtutes existentes \\\hline
1.2.2 & e las morales \textbf{ son aquellas que estan en el entendimiento pratico e obrador } assi commo es la pradençia & inter intellectuales et morales , \textbf{ sunt virtutes existentes | in intellectu practico , } ut Prudentia , \\\hline
1.2.2 & assi commo es la pradençia \textbf{ que es regla derecha para bien obrar . } Et las otras uirtudes & in intellectu practico , \textbf{ ut Prudentia , } et aliae virtutes sibi annexae . \\\hline
1.2.2 & Et por ende dize el philosofo \textbf{ en el seyto libro delas ethicas . ca non puede ser el omne pradente e sabio } e non ser bueno . & Inde est ergo quod dicitur 6 Ethic’ \textbf{ quod impossibile est prudentem } esse non existentem bonum . \\\hline
1.2.2 & e de las uirtudes morales \textbf{ que son en el entendimiento pratico o en el apetito . } Ca assi commo muchas vezes auemos dicho tomamos esta obra presente & ut dimissis scientiis speculatiuis , de prudentia , \textbf{ et de virtutibus moralibus est tractandum . } Suscepimus enim \\\hline
1.2.2 & Mas deuedes saber \textbf{ que poderio razonable es en dos maneras } segunt dize el philosofo ¶ & in qua potest esse virtus . \textbf{ Rationale autem duplex est } secundum Philosophum , \\\hline
1.2.2 & Et otro del seso . \textbf{ Ca assi commo el appetito natural sigue a su forma naturalmente auida . } Assi el appetito conoscedor sigue la forma resçebidao aujda & In nobis autem duplex est appetitus , intellectiuus , et sensitiuus . \textbf{ Nam sicut appetitus naturalis | sequitur formam naturaliter adeptam , } sic appetitus cognitiuus \\\hline
1.2.2 & Ca assi commo el appetito natural sigue a su forma naturalmente auida . \textbf{ Assi el appetito conoscedor sigue la forma resçebidao aujda } por el conosçimiento ¶ Verbigera ¶ & sequitur formam naturaliter adeptam , \textbf{ sic appetitus cognitiuus } sequitur formam per cognitionem apprehensam . \\\hline
1.2.2 & e connosçida por entendimiento ¶ \textbf{ Mas el appetito senssitiuo es en dos maneras } ca commo las aina las sean sobre las cosas & qui sequitur formam apprehensam per intellectum . \textbf{ Appetitus autem sensitiuus duplex est . } Nam cum animalia sint supra inanimata , \\\hline
1.2.2 & Ca ueemos nos que el fuego naturalmente es caliente e liuiano \textbf{ e por la liuiandat sube arriba a su logar propio et a su folgura . } Et por la calentura obra contra sus contrarios destruyendo los e gastando los . & Videmus enim quod ignis naturaliter est calidus , \textbf{ et leuis per leuitatem autem tendit in locum proprium , } et in quietem sibi conuenientem : \\\hline
1.2.2 & e su propia delectaçion \textbf{ assi commo es el appetito desseador . } ¶ Et dioles otro appetito & et propriam delectationem , \textbf{ ut concupiscibilem : } et alium per quem resistunt , \\\hline
1.2.2 & assi commo liuiandat \textbf{ e assi commo pesadura sean en las cosas liuianas e pesadas . } Mas el appetito enssannador sea & sicut leuitas , \textbf{ vel grauitas , in leuibus , et grauibus . } Irascibilis vero se habet \\\hline
1.2.2 & por su desseo cobdiçiador fuyen los males dela tristeza . \textbf{ Et siguen los bienes delectabłs en que se delectan . } Mas por el desseo enssannador & fugiunt mala tristia , \textbf{ et prosequuntur bona delectabilia . } Per irascibilem vero aggrediuntur contraria , \\\hline
1.2.2 & e por el mal \textbf{ en quanto han razon de cosa guaue e fuerte . } Ca commo el bien & irascibilis vero respicit bonum , \textbf{ et malum inquantum habent rationem difficilis , et ardui . } Nam cum bonum secundum se dicat prosequendum , \\\hline
1.2.2 & lo fiziera la natura \textbf{ si diera alas aian lias appetito cobdiçiador } por el qual signiessen el bien & Imperfecto ergo egisset natura , \textbf{ si dedisset animalibus concupiscibilem , } per quam vitarent mala , \\\hline
1.2.2 & aian las \textbf{ segund su manera conuenible se pueden delectar conplidamente } por el appetito & secundum modum sibi conuenientem , \textbf{ prout bene possint delectari per concupiscibilem , } data est eis irascibilis , \\\hline
1.2.2 & Pues que assi es dos son los . appetitos \textbf{ e los desseos senssibles . } El . vno enssañador . & Duplex est ergo appetitus sensitiuus , \textbf{ irascibilis , et concupiscibilis . } Appetitus autem intellectiuus , \\\hline
1.2.2 & Et otro segund el qual \textbf{ segnimos las cosas delectables del seso } e acometemos las cosas espantables . & et alius , \textbf{ secundum quem prosequimur delectabilia , } et secundum quem aggredimur terribilia : \\\hline
1.2.2 & segnimos las cosas delectables del seso \textbf{ e acometemos las cosas espantables . } Enpero el appetito del entendimiento seyendo vno & secundum quem prosequimur delectabilia , \textbf{ et secundum quem aggredimur terribilia : } appetitus tamen intellectiuus unus , \\\hline
1.2.2 & nin departido el appetito del entendimiento \textbf{ segund el qual segnimos los bienes delectables por el entendimiento . } Et acometemos los bienes fuertes de alcançar & Non ergo est alius appetitus intellectiuus , \textbf{ secundum quem prosequimur bona delectabilia per intellectum , } et aggredimur bona ardua : sicut est alius appetitus sensitiuus , \\\hline
1.2.2 & segund el qual segnimos los bienes delectables por el entendimiento . \textbf{ Et acometemos los bienes fuertes de alcançar } assi commo es otro & secundum quem prosequimur bona delectabilia per intellectum , \textbf{ et aggredimur bona ardua : sicut est alius appetitus sensitiuus , } secundum quem prosequimur bona sensibilia delectabilia , \\\hline
1.2.2 & e departido el appetito senssitiuo \textbf{ segund el qual segnimos los bienes senssibles e delectables de aquel } por que acometemos las cosas espantables e tstables¶ & et aggredimur bona ardua : sicut est alius appetitus sensitiuus , \textbf{ secundum quem prosequimur bona sensibilia delectabilia , } et aggredimur terribilia . \\\hline
1.2.2 & segund el qual segnimos los bienes senssibles e delectables de aquel \textbf{ por que acometemos las cosas espantables e tstables¶ } pues que assi es si toda uirtud moral & secundum quem prosequimur bona sensibilia delectabilia , \textbf{ et aggredimur terribilia . } Quare si omnis virtus moralis , \\\hline
1.2.2 & en los quales ha de seer \textbf{ la uirtud podemos tomar e entender quatro uirtudes cardinales e generales delas quales ¶ } La vna es la pradençia e la otra la iustiçia & in quibus habet esse virtus , \textbf{ sumptae sunt quatuor Virtutes Cardinales ; } videlicet , Prudentia , Iustitia , Fortitudo , et Temperantia . \\\hline
1.2.3 & lphilosofo cerca la fin del segundo libro delas ethicas \textbf{ sin la pradençia et la iustiçia cuenta diez uirtudes morales delas quales ¶ } la primera es fortaleza¶ & circa finem V Ethicorum \textbf{ praeter Prudentiam , et iustitiam , | enumera 10 virtutes morales , } videlicet , Fortitudinem , Temperantiam , \\\hline
1.2.3 & La octaua uerdat ¶ \textbf{ La nona famil yaridat } ¶La . x̊ . . eutropolia & Igitur computata Iustitia , \textbf{ et Prudentia duodecim sunt virtutes morales ; } de quibus omnibus quid sunt , \\\hline
1.2.3 & con estas dichas diez \textbf{ son doze las uirtudes morales delas quales todas que cosas son e en qual manera . } Conuiene a los Reyes & de quibus omnibus quid sunt , \textbf{ et quomodo decet eas Reges habere , } et quas partes habent , \\\hline
1.2.3 & assi se puede tomar . \textbf{ Ca commo el subiecto delas uirtudes sea o el entendimiento o la uoluntad o el appetito senssitiuo . } toda uirtud moral o es en el entendimiento & Numerus autem earum sic potest accipi . \textbf{ Nam cum subiectum virtutis sit , | vel intellectus , vel voluntas , } vel appetitus sensitiuus : \\\hline
1.2.3 & Ca commo el subiecto delas uirtudes sea o el entendimiento o la uoluntad o el appetito senssitiuo . \textbf{ toda uirtud moral o es en el entendimiento } assi commo la pradençia o enla uoluntad & vel appetitus sensitiuus : \textbf{ omnis virtus moralis , | vel est in intellectu , } ut Prudentia : \\\hline
1.2.3 & diuision e departimiento de las uirtudes . \textbf{ Ca toda uirtud moral va a bien de razon . } Mas yra bien de razon puede ser en tres maneras . & haec eadem diuisio sumi potest . \textbf{ Nam omnis virtus moralis | tendit } in bonum rationis . \\\hline
1.2.3 & temoͬ quando fuymos del mal futur . \textbf{ Osadia quando acometemos algun mal futuro ¶ } Mas entre estas dos passiones se toma vna uirtud medianera & timor cum ab eo refugimus , \textbf{ audacia cum illud aggredimur . } Inter has autem duas passiones sumitur \\\hline
1.2.3 & Enpero la manssedunbre non dize propiamente \textbf{ uirtud medianera entre abaxamiento e ira . } mas dize vna passion & Mansuetudo tamen non proprie nominat virtutem mediam \textbf{ inter mitiditatem , | et iram , } immo magis nominat ipsam passionem , \\\hline
1.2.3 & son tomadas çerca delas passiones \textbf{ que nasçen del mal presente . } Assi commo la fortaleza es çerca delas passiones & sumuntur circa passiones , \textbf{ quae oriuntur ex malo , } ut fortitudo est \\\hline
1.2.3 & Et la manssedunbre es çerca delas passiones \textbf{ que nasçendel mal presente . } Mas si fueren tomadas las uirtudes cerca delas passiones & circa passiones ortas ex malo futuro : \textbf{ mansuetudo circa passiones ortas ex malo praesenti . } Si autem sumantur virtutes circa passiones , \\\hline
1.2.3 & assi commo es tenperança \textbf{ que es en el appetito desseador . } Cerca delas passiones & ut Temperantia , \textbf{ quae est in concupiscibili . } Circa passiones vero ortas \\\hline
1.2.3 & Pues que assi es la liberalidat e franqueza \textbf{ sera en el appetito desseador . } Mas la manifiçençia e la grandeza & quod magnificens : \textbf{ erit igitur liberalitas in concupiscibili , } magnificentia vero ratione arduitatis erit in irascibili . \\\hline
1.2.3 & Et la magnanimidat et grandeza de coraçon sera en el appetito enssannador . \textbf{ Et daqui paresçe que por que los bienes delectables non pueden auer } assi manera de guaueza & Ex quo patet , \textbf{ quod quia bona delectabilia | non sic possunt } habere rationem ardui sicut utilia , et honesta , \\\hline
1.2.3 & Et la otra en el enssannador . \textbf{ Enpero de los bienes delectables non se toma } si non vna uirtud & alia in irascibili : \textbf{ ex bonis tamen delectabilibus non sumitur , } nisi una virtus , \\\hline
1.2.3 & non en quanto es uirtud \textbf{ que esta en el entendimiento especulatiuo . } mas en quanto se toma aqui & quae potest dici bona versio . \textbf{ Est autem Veritas } ( ut hic de ea loquimur , \\\hline
1.2.3 & e non es ypocrita \textbf{ nin alabador de ssi mesma . } Mas es manifiesto e claro en sus palauras e en sus fechos & quando aliquis non est hypocrita nec iactator , \textbf{ sed est apertus , } et verbis et factis ostendit se talem , \\\hline
1.2.3 & por que non son çerca cosa \textbf{ guaue son en el appetito desseador ¶ } Pues que assi es paresçe & circa aliquid arduum , \textbf{ sunt in concupiscibili . } Patet ergo quod cum quatuor potentiae animae sint \\\hline
1.2.3 & En el entendimiento es la pradençia \textbf{ que es razon derecha para bien obrar . } Et en la uoluntad es la iustiçia & In intellectu est prudentia . \textbf{ In voluntate iustitia . } In irascibili est Fortitudo , \\\hline
1.2.3 & para dar a cada vno lo suyo . \textbf{ Et en el appetito enssannadores la fortaleza } para acometer grandes cosas . & In voluntate iustitia . \textbf{ In irascibili est Fortitudo , } Mansuetudo , Magnanimitas , et Magnificentia , \\\hline
1.2.3 & ¶ La manssedunbre cerca delas passiones \textbf{ que nasçen en el coraçon de los males presentes . } Mas la magnifiçençia e la magnanimidat & Mansuetudo circa passiones ortas \textbf{ ex malis praesentibus . } Magnificentia vero , \\\hline
1.2.3 & que son grandeza e alteza de coraçon \textbf{ son çerca de los bienes guaues e fuertes de alcançar . } Enpero de departidas maneras . & Magnificentia vero , \textbf{ et Magnanimitas sunt circa bona ardua , aliter et aliter : } quia Magnificentia est \\\hline
1.2.3 & Enpero de departidas maneras . \textbf{ Ca la magnifiçençia es çerca de los bienes grandes e prouechosos } assi commo en fazer grandes espenssas . & et Magnanimitas sunt circa bona ardua , aliter et aliter : \textbf{ quia Magnificentia est | circa magna bona utilia , } ut circa magnos sumptus : \\\hline
1.2.4 & si non estas uirtudes que ya dixiemos . \textbf{ propusiemos de mostrar que fablando delans buenas disposiconnes del alma delas quales fablaron los philosofos . } Ca delas otras non entendemos aqui fablar . & quod bonarum dispositionum \textbf{ ( loquendo de bonis dispositionibus , | de quibus locuti sunt Philosophi , } quia de aliis ad praesens non intendimus tractatum constituere ) \\\hline
1.2.4 & lonsomesassi alguons otros omes son diuinales \textbf{ e son buenos sobre la manera comunal . } de lons omes . & sic aliqui sunt quasi diuini , \textbf{ et sunt boni supra modum } propter quod tales , \\\hline
1.2.4 & que son mas uirtuosos \textbf{ que son los omes comunal mente . } Mas esta uirtud diuinal & propter quod tales , \textbf{ superuirtuosi dici possunt . } Huiusmodi autem uirtutem diuinam , \\\hline
1.2.5 & e menos prinçipales \textbf{ e ayuntables alas prinçipales . } Mas las cardinales son quatro delas quales . & quia quaedam sunt Cardinales et principales , \textbf{ quaedam vero annexae . } Cardinales autem sunt quatuor , \\\hline
1.2.5 & e non derechamente conuiene de dar alguna uirtud \textbf{ que sea razon derecha . } Por la qual de todas las obras & oportet dare virtutem aliquam , \textbf{ quae sit recta ratio , } per quam de ipsis agibilibus rectas rationes faciamus . \\\hline
1.2.5 & o en la uoluntad o en el appe tito enssannador . \textbf{ en el appetito desseador ¶ } Et pues que assi es commo en el entendimiento pratico la mas prinçipal uirtud son la pradençia . & esse in quatuor potentiis animae , \textbf{ videlicet , in intellectu , in voluntate , in irascibili , in concupiscibili : } cum ergo in intellectu practico \\\hline
1.2.5 & en el appetito desseador ¶ \textbf{ Et pues que assi es commo en el entendimiento pratico la mas prinçipal uirtud son la pradençia . } Et en la uoluntad la prinçipal uirtud sea la iustiçia fablando delas uirtudes & videlicet , in intellectu , in voluntate , in irascibili , in concupiscibili : \textbf{ cum ergo in intellectu practico | principalior virtus sit prudentia , in voluntate } ( loquendo de virtutibus acquisitis ) \\\hline
1.2.5 & Et por ende estas quatro uirtudes son dichas \textbf{ sinplemente uirtudes prinçipales e cardinales } ¶ & ideo hae quatuor uirtutes , \textbf{ principales et cardinales esse dicuntur . } In aliis ergo uirtutibus potest \\\hline
1.2.6 & en la qual ha de obrar \textbf{ ¶L quarto alascina ¶ } Et lo quinto ala arte delas quales dos cosas se departe ¶ & ad materiam , circa quam versatur : \textbf{ ad scientiam , et ad artem , | a quibus distinguitur . } Est enim prudentia virtutum moralium directiua : \\\hline
1.2.6 & Lo primero dezimos que la prerudençia es uirtud \textbf{ que endereça e regla todas las uirtudes morales . } por que las uirtudes morales dessi inclinan al omne & Est enim prudentia virtutum moralium directiua : \textbf{ nam virtutes morales } de se inclinant \\\hline
1.2.6 & que endereça e regla todas las uirtudes morales . \textbf{ por que las uirtudes morales dessi inclinan al omne } ala fin que les conuenible & nam virtutes morales \textbf{ de se inclinant } in finem sibi conuenientem , \\\hline
1.2.6 & mas non abasta a omne ser inclinado \textbf{ afin conuenible de tenprança } o a fin conuenible delas otras uirtudes morales & sed non sufficit inclinari \textbf{ in finem debitum temperantiae , } vel in finem aliarum virtutum moralium , \\\hline
1.2.6 & afin conuenible de tenprança \textbf{ o a fin conuenible delas otras uirtudes morales } si non sopiere & in finem debitum temperantiae , \textbf{ vel in finem aliarum virtutum moralium , } nisi sciamus , \\\hline
1.2.6 & e esto ha de saber \textbf{ por la pradençia ¶ Et pues que assi es por las uirtudes morales somos ordenados } a nuestros fines buenos e conuenibles . & quod fit per prudentiam . \textbf{ Per virtutes ergo morales } praestituimus nobis debitos fines : \\\hline
1.2.6 & por la pradençia ¶ Et pues que assi es por las uirtudes morales somos ordenados \textbf{ a nuestros fines buenos e conuenibles . } Mas por la pradençia somos reglados & Per virtutes ergo morales \textbf{ praestituimus nobis debitos fines : } sed per prudentiam \\\hline
1.2.6 & que la pradençia es perfeçion del entendimiento \textbf{ que as buena calidat endereçadora e regladora del alma . } por la qual es el omne reglado & quod est perfectio intellectus , \textbf{ siue quod est bona qualitas mentis , } directiua in finem virtutum moralium . \\\hline
1.2.6 & por que las manda luego fazer \textbf{ que la uirtud buscadora e falladora . } Et pues que assi es commo en las uirtudes morales las obras & eo quod praecipiat illa fieri , \textbf{ quam faciat virtus inuentiua et iudicatiua . } Cum ergo in moralibus actus \\\hline
1.2.6 & que la uirtud buscadora e falladora . \textbf{ Et pues que assi es commo en las uirtudes morales las obras } e los fechos sean dichos mayores e meiors . & quam faciat virtus inuentiua et iudicatiua . \textbf{ Cum ergo in moralibus actus } et opera dicantur esse potiora , \\\hline
1.2.6 & Et las obras ayan de ser \textbf{ en las cosas singulares . } Conuiene que la pradençia sea cerca las cosas singulares & Cum enim Prudentia sit circa agibilia , \textbf{ et agibilia sint singularia , } oportet prudentiam esse circa particularia , \\\hline
1.2.6 & e particulares \textbf{ allegando las reglas generales alos negoçios singulares } e particulares & oportet prudentiam esse circa particularia , \textbf{ applicando uniuersales regulas | ad singularia negocia , } ut Ethic’ 6 declarari habet . \\\hline
1.2.6 & diziendo que la pradençia es uirtud \textbf{ que iudga alos negoçios particulares seg̃t las reglas vniuerssales . } las quales reglas generales son buenas leyes e buenas costunbres . & quod prudentia est virtus \textbf{ secundum uniuersales maximas particularia facta concernens . | Huiusmodi autem uniuersales regulae } sunt bonae leges , \\\hline
1.2.6 & que iudga alos negoçios particulares seg̃t las reglas vniuerssales . \textbf{ las quales reglas generales son buenas leyes e buenas costunbres . } Et otras cosas tales & Huiusmodi autem uniuersales regulae \textbf{ sunt bonae leges , | debitae consuetudines , } et alia per quae regulari possumus in agendis . \\\hline
1.2.6 & lo quarto puede se conparar la pradençia ala sçiençia dela qual d se departe en esta manera . \textbf{ Porque lasçina es propiamente delas cosas neçesarias e duraderas . Onde dize boeçio en el primero libro dela arismetica . } que la scina es de aquellas cosas & Quarto comparari habet prudentia \textbf{ ad ipsam scientiam , | a qua distinguitur : } nam scientia proprie , \\\hline
1.2.6 & Et aquella sabiduria es dicha pradençia \textbf{ por la qual cosa seg̃t que la pradençia ha departimiento dela sçina } pue dese & quae sunt in potestate nostra . \textbf{ Quinto comparari potest prudentia ad artem , } a qua etiam distingui habet . \\\hline
1.2.6 & que aquel que peca contra uoluntad . \textbf{ Et pues que assi es en las obras mecanicas que . } siruen al arte meior es pecar de uoluntad & est peior . \textbf{ In operibus ergo deseruientibus arti , melius est peccare voluntarie , } quam inuoluntarie : \\\hline
1.2.6 & Et pues que assi es en las obras mecanicas que . \textbf{ siruen al arte meior es pecar de uoluntad } que sin uoluntad . & est peior . \textbf{ In operibus ergo deseruientibus arti , melius est peccare voluntarie , } quam inuoluntarie : \\\hline
1.2.6 & si non fuere bueno \textbf{ e non ouiere uoluntad derecha e reglada . } Et por ende la pradençia en quanto es conparada & nisi sit bonus \textbf{ et habeat voluntatem rectam , | non est ars , sed virtus . } Prout ergo prudentia comparatur ad artem \\\hline
1.2.6 & puede se assi difinir e declarar diziendo . \textbf{ que la pradençia es razon derecha de todas las obras } que auemos de fazer & sic diffiniri potest , \textbf{ quam est recta ratio agibilium , } praesupponens rectitudinem voluntatis . \\\hline
1.2.6 & que requiere e demanda reglamiento de uoluntad ¶ \textbf{ Et pues que assi es de todas estas cosas sobredichas podemos tomar vna } difiniçonn o declaraçion comun dela pradençia . & praesupponens rectitudinem voluntatis . \textbf{ Ex omnibus ergo his , | de ipsa prudentia unam communem descriptionem formare possumus , } dicendo , \\\hline
1.2.6 & Et pues que assi es de todas estas cosas sobredichas podemos tomar vna \textbf{ difiniçonn o declaraçion comun dela pradençia . } diziendo que la pradençia es uirtud intellectual & de ipsa prudentia unam communem descriptionem formare possumus , \textbf{ dicendo , } quod Prudentia est virtus intellectualis , \\\hline
1.2.6 & diziendo que la pradençia es uirtud intellectual \textbf{ enderascadora e regladora delas uirtudes morales . } Mandadora . e sennora delas cosas falladas & quod Prudentia est virtus intellectualis , \textbf{ directiua virtutum moralium , praeceptiua secundum inuenta , } et iudicata secundum uniuersales maximas , \\\hline
1.2.7 & e mostrado que por la pradençia somos enderesçados e guiados derechamente \textbf{ a buena fin a la qual nos inclinan las uirtudes morales . } finca de demostrar & et ostenso quod per prudentiam recte dirigimur \textbf{ in bonum finem , | in quem inclinant virtutes morales : } restat ostendere , \\\hline
1.2.7 & que muchon deue tener mientes el Rey \textbf{ ¶la primera parte nesçe al Rey de tener . } mucho mientes que sea Rey en uerdat & tria quae maxime Rex attendere debet . \textbf{ Primo enim spectat | ad ipsum summe intendere , } ut sit Rex \\\hline
1.2.7 & Mas gouernar alos otros \textbf{ e guiar los en su fin conuenible . } Esto ha de ser por la pradençia & regere autem alios , \textbf{ et dirigere ipsos in finem debitum , } sit per prudentiam . \\\hline
1.2.7 & Pues que assi es la pradençia es vn oio con \textbf{ que catamos el bien e la fin conuenible . } Et el que non ha este oio non puede conplidamente ueer el bien & Prudentia ergo est quidam oculus , \textbf{ quo bonus et debitus finis conspicitur . } Qui ergo hoc oculo caret , \\\hline
1.2.7 & Ca dicho es ya que por la sabiduria somos guiandos e endereçados a buena fin \textbf{ ala qual nos inclinan las uirtudes morales . } Ca de omne sabio es proueer buenas cosas . & in bonum finem , \textbf{ in quem inclinant virtutes morales . } Est enim prudentis , \\\hline
1.2.7 & ala qual nos inclinan las uirtudes morales . \textbf{ Ca de omne sabio es proueer buenas cosas . } assi e alos otros e de guiar & in quem inclinant virtutes morales . \textbf{ Est enim prudentis , | prouidere bona sibi et aliis , } et dirigere se et alios in optimum finem . \\\hline
1.2.7 & por que abonde en riquezas \textbf{ e en estos bienes senssibles e menguados . } Et por ende faze se tomador e robador del pueblo & ut affluat diuitiis , \textbf{ et sensibilibus bonis . } Efficietur ergo depraedator populi , \\\hline
1.2.7 & por que es menguado de entendimiento \textbf{ e non sabe gouernar a ssi mismo . } Et por ende es dicho alguno naturalmente señor & quia deficit intellectu , \textbf{ et nescit seipsum regere . } Ex hoc autem naturaliter est Dominus , \\\hline
1.2.7 & mas ahun alabanla \textbf{ e confirman la todos los gouernamientos naturales . } Ca uehemos que los omes son naturalmente & non solum approbant physica dicta , \textbf{ sed etiam confirmant singula regimina naturalia . } Videmus enim naturaliter homines dominari bestiis , \\\hline
1.2.8 & si non se ouieren todas las sus partes \textbf{ si alguno ouiere aser sabio conplida mente . } Conuienel e de auer todas aquellas cosas & nisi habeantur partes eius : \textbf{ si debeat aliquis esse perfecte prudens , } oportet ipsum habere omnia \\\hline
1.2.8 & Ca esto ninguno non lo pie de fazer . \textbf{ Mas conuiene al Rey de auer memoria delans cosas passadas } por que pue da & quia nulli agenti hoc est possibile , \textbf{ sed decet Regem habere praeteritorum memoriam , } ut possit ex praeteritis cognoscere , \\\hline
1.2.8 & que es tal que ha de gouernar los otros . \textbf{ Conuiene le de sor sotil e doctrinable . } Ca aquel que esta en tanta alteza de dignidat & quae est alios dirigens , \textbf{ oportet quod sit solers , et docilis . } Nam qui in tanto culmine est positus , \\\hline
1.2.8 & por la qual cosa non le conuiene al Rey de seguir en todas cosas su cabeça \textbf{ nin atener se sienpre al su engennio propio . } Mas conuiene le de ser doctrinable & in omnibus sequi caput suum , \textbf{ nec inniti semper solertiae propriae : } sed oportet ipsum esse docilem , \\\hline
1.2.8 & por que la praeua es delas cosas particulares \textbf{ e por que las cosas particulares son muchas e muy departidas } e en quanto alguon ha de gouernar departidas gentes son le mester departidas cosas . & Experientia enim est rerum particularium . \textbf{ Prout igitur sunt alia , | et alia particularia , } et prout aliquis negociatur circa aliam , \\\hline
1.2.8 & e de su pueblo \textbf{ de ser my prouado conosçiendo las condiconnes particulares de su gente } e de su pueblo & esse expertum , \textbf{ cognoscendo particulares conditiones gentis sibi commissae , } ut possit eam melius in debitum finem dirigere . \\\hline
1.2.9 & assi que ellos non de una auer \textbf{ alguas vezes algunos solazes corporales e honestos . } Mas deuen usar dellos tenpradamente & Quod non sic intelligendum est , \textbf{ ut nullas recreationes corporales habere debeant , } sed debent eis adeo moderate uti , \\\hline
1.2.9 & Et para esto deuen leer las coronicas \textbf{ e los fechos antigos de los buenos Reyes } por que ayan memoria de los fechos que passaron & sub quibus temporibus \textbf{ regnum melius regebatur , } propter quod habeant memoriam praeteritorum , \\\hline
1.2.9 & que pue den ser dannosos al su regno \textbf{ por que por esta manera aur̃a sabiduria delas cosas } que han de venir & et mala , quae possunt esse nociua . \textbf{ Nam ex hoc habebunt prouidentiam futurorum , } ut possint mala expeditius vitare , \\\hline
1.2.9 & puede bien gouernar su regno \textbf{ tomando delas razones conuenibles conclusiones } para todas las cosas & et consuetudines debite regnum regat , \textbf{ eliciendo ex eis debitas conclusiones agibilium . } Non enim sufficit esse intelligentem , \\\hline
1.2.9 & e ala sabiduria \textbf{ delas quales fablamos ya en el capitulo soƀ dicho podran fazer assi mismos sabios } Mas por que la malicia es & quae ad prudentiam requiruntur , \textbf{ de quibus in praecedenti capitulo fecimus mentionem , | poterunt seipsos prudentes facere . } Verum quia malitia est corruptiua principii . \\\hline
1.2.9 & por maliçia es ciego en el entendimiento \textbf{ e en la razon por que iudge mal en lo que ha de fazer } Ca alas vezes & et deprauatam voluntatem , excoecatur in intellectu , \textbf{ ut male iudicet de agibilibus : } iudicat enim esse agendum \\\hline
1.2.9 & por la qual cosa si los Reyes e los prinçipes quieren ser sabios con esto \textbf{ que deuen ser acordables prouisores engennosos e doctrinables } e auer las otras cosas & et Principes volunt esse prudentes , \textbf{ cum hoc quod debent esse memores , prouidi , solertes , et dociles , } et alia , \\\hline
1.2.9 & que sean buenos \textbf{ e que non ayan uoluntad mala nin desordenada } por que por la maliçia dela uoluntad fagan las cosas sin razon & oportet ipsos esse bonos , \textbf{ et non habere voluntatem deprauatam : } ne propter malitiam appetitus , imprudenter agant , \\\hline
1.2.10 & que es delas leyes \textbf{ e enn iustiçia ygual . } Mas la iustiçia legales uirtud general & duplicem Iustitiam , \textbf{ legalem , et aequalem . } Legalis enim Iustitia est quid generale , \\\hline
1.2.10 & e enn iustiçia ygual . \textbf{ Mas la iustiçia legales uirtud general } e es en algua manera toda uirtud . & legalem , et aequalem . \textbf{ Legalis enim Iustitia est quid generale , } et quodammodo omnis virtus . \\\hline
1.2.10 & Ca todas las uirtudes se ençierran enlla . \textbf{ Mas la uirtud iguales uirtud espeçial } Et es vna uirtud singular . & et quodammodo omnis virtus . \textbf{ Iustitia vero aequalis , | est quid speciale , } et est quaedam particularis virtus . \\\hline
1.2.10 & Mas la uirtud iguales uirtud espeçial \textbf{ Et es vna uirtud singular . } Ca por esso es alguno dicho & est quid speciale , \textbf{ et est quaedam particularis virtus . } Nam ex hoc est quis iustus legalis , \\\hline
1.2.10 & Mas assi commo dize el philosofo \textbf{ en el primero libro dela grand ph̃ia moral . } La ley manda fazer las obras de todas las uirtudes . & quia adimplet praecepta legis . \textbf{ Sed ( ut dicitur primo Magnorum Moralium ) } lex praecipit actus omnium virtutum . \\\hline
1.2.10 & La ley manda fazer las obras de todas las uirtudes . \textbf{ Ca manda la ley obrar obras fuertes e obras tenpradas . } Et generalmente todas las obras & lex praecipit actus omnium virtutum . \textbf{ Praecipit enim lex operari fortia et temperata , } et uniuersaliter omnia \\\hline
1.2.10 & que es conplimiento delas leyes es en alguna manera toda uirtud \textbf{ Mas la iustiçia yguales vna uirtud espeçial } por la qual es dado a cada vno & est quodammodo omnis virtus . \textbf{ Iustitia autem aequalis , | est quaedam virtus specialis , } per quam redditur cuilibet \\\hline
1.2.10 & e al bien comun e al fazedor dela ley . \textbf{ al Rey en tanto ha de seer en ellos estas dos iustiçias legal e ygual . } Ca enlos çibdadanos en quanto han esta orden a otro & et ad legislatores seu ad ipsum Regem , \textbf{ habet esse in eis Iustitia legalis , et aequalis . } Nam in ipsis ciuibus , \\\hline
1.2.10 & Ca si quieren bien comun \textbf{ assi es en ello la iustiçia legal . } Mas si quieren algun bien espeçial & Si quaeritur commune bonum : \textbf{ sic est in eis Iustitia legalis . } Si autem quaeritur \\\hline
1.2.10 & Mas si quieren algun bien espeçial \textbf{ e proprio assi es enllos iustiçia ygual . Ca el bien comun nasçe de todos los bienes de los çibdadanos . } Ca el bien comun e toda la çibdat es meior & in ipsis aliquod bonum priuatum : \textbf{ erit in eis Iustitia aequalis . | Bonum enim commune resultat } ex omni bono ciuium : \\\hline
1.2.10 & sienpre en el bien comun Manda \textbf{ e ordenan toda manera de bondat Et por ende seer el omne iusto segunt la ley } e conplir la iustiçia legales & quia intendunt commune bonum , \textbf{ praecipiunt omnem modum bonitatis . | Esse igitur Iustum secundum legem , } et implere legalem Iustitiam , \\\hline
1.2.10 & Enpero conuiene de saber \textbf{ que la iustiçia legal non es dicha toda uirtud } que lea apartada de cada vna delas uirtudes . & quodammodo omnis virtus , quia exercet opera omnium virtutum . \textbf{ Non est autem simpliciter legalis Iustitia omnis virtus , } quia est virtus distincta \\\hline
1.2.10 & Mas es dicha toda uirtud en alguna manera \textbf{ en quanto non se determina a materia espeçial . } Mas esta iustiçia legal departese de cada vna dela sotras uirtudes en dos cosas . & Sed dicitur esse quodammodo omnis virtus , \textbf{ quia non determinat sibi specialem Iustitiam . } Differt autem huiusmodi Iustitia \\\hline
1.2.10 & en quanto non se determina a materia espeçial . \textbf{ Mas esta iustiçia legal departese de cada vna dela sotras uirtudes en dos cosas . } Ca commo quier que el iusto legal faga essas mismas obras & quia non determinat sibi specialem Iustitiam . \textbf{ Differt autem huiusmodi Iustitia | a qualibet virtute in duobus : } nam licet eadem opera agat Iustus legalis , \\\hline
1.2.10 & Mas esta iustiçia legal departese de cada vna dela sotras uirtudes en dos cosas . \textbf{ Ca commo quier que el iusto legal faga essas mismas obras } que faze el fuerte e el tenprado . & a qualibet virtute in duobus : \textbf{ nam licet eadem opera agat Iustus legalis , } quae agit fortis , et temperatus : \\\hline
1.2.10 & mas en quanto las manda fazer la ley \textbf{ e el quiere conplir la ley es dicho iusto legal . } Et pues que assi es el iusto legal & sed quia ea lex praecipit , \textbf{ et vult implere legem , | iustus legalis est . } Iustus ergo legalis , \\\hline
1.2.10 & deleytase en conplir la ley . \textbf{ Mas si el iusto legal se deleyta en las obras de cada vna delas otras uirtudes } esto es por accidente e por otra entençion . & delectatur in impletione legis . \textbf{ Si autem delectatur | in operibus singularium virtutum , } hoc est ex consequenti , \\\hline
1.2.10 & assi commo en orden al prinçipe o en orden ala çibdat . \textbf{ como quier que la tenperança e la fortaleza e las otras uirtudes prinçipales acaben } aque que las ha segunt si¶ & vel ordine ad Ciuitatem : \textbf{ non obstante quod Temperatia , | et Fortitudo , } et aliae virtutes huiusmodi perficiant habentem secundum se . \\\hline
1.2.10 & Pues que assi es paresçe \textbf{ commo el iusto se gales en alguna manera uirtuoso en tonda uirtud } Et non se determina a ninguna manera espeçial & quomodo Iustitia legalis \textbf{ est | quodammodo omnis virtus , } et quod non determinat sibi specialem Iustitiam , sed agit opera specialium virtutum . \\\hline
1.2.10 & Et non se determina a ninguna manera espeçial \textbf{ mas faze las obras de cada vna delas uirtudes sp̃ales ¶ } Mas la iustiçia igual non es toda uirtud & quodammodo omnis virtus , \textbf{ et quod non determinat sibi specialem Iustitiam , sed agit opera specialium virtutum . } Iustitia vero aequalis \\\hline
1.2.10 & mas faze las obras de cada vna delas uirtudes sp̃ales ¶ \textbf{ Mas la iustiçia igual non es toda uirtud } nin faze las obras de cada vna delas uirtudes singulares & et quod non determinat sibi specialem Iustitiam , sed agit opera specialium virtutum . \textbf{ Iustitia vero aequalis | non est omnis virtus , } nec agit opera singularum virtutum , \\\hline
1.2.10 & nin faze las obras de cada vna delas uirtudes singulares \textbf{ mas determina se amateria sp̃al en quanto en ella entiende algun bien span l . } Ca maguera el çibdadano sea bueno & nec agit opera singularum virtutum , \textbf{ sed determinat sibi specialem Iustitiam , | eo quod in ea intenditur speciale bonum . } Nam licet \\\hline
1.2.10 & et en qual quier manera que el çibdadano sea malo \textbf{ por ende se a la çibdat peor . } Enpero por qual quier bondat de vn çibdadano & et qualitercunque malus sit , \textbf{ inde peior ciuitas : } non tamen ex quacunque bonitate unius ciuis per se , \\\hline
1.2.10 & pues que assi es la iustiçia spanl \textbf{ por la qual es esquiuado el mal spanl de alguno } en quanto dela maliçia de vn & Iustitia ergo specialis , \textbf{ per quam vitatur malum aliquod speciale , } prout ex malitia unius ciuis \\\hline
1.2.10 & en quanto dela maliçia de vn \textbf{ çibdadano viene danno a otro la iustiçia spanl determinasse a materia sp̃al . } Et a de ser cerca estos bienes de fuera & prout ex malitia unius ciuis \textbf{ infertur nocumentum alteri , | determinat sibi specialem materiam , } et habet esse circa haec bona exteriora , \\\hline
1.2.10 & Pues que assi es la iusti \textbf{ çia ygual es dicha iustiçia sp̃al } por que prinçipalmente entiende a ygualdat de los çibdadanos & in quibus ciues communicant . \textbf{ Dicta est igitur haec Iustitia aequalis , } quia potissime innittitur aequalitati , \\\hline
1.2.10 & Et por ende se sigue \textbf{ que esta iustiçia egual manda dar a cada vno su derecho } ca el derechon esta en vna ygualdat & quod aequum est . \textbf{ Inde est ergo quod haec Iustitia dicitur | unicuique suum tribuere , } quia ius in quadam aequalitate consistit : \\\hline
1.2.10 & ca el derechon esta en vna ygualdat \textbf{ Mas esta iustiçia egual manda a cada vno dar lo que es e suyo } e lo que es igual & quia ius in quadam aequalitate consistit : \textbf{ haec autem unicuique tribuit | quod iustum vel aequum est . } Sic etiam dicitur \\\hline
1.2.10 & lo que es suyo . \textbf{ Ca cosa igual es } que cada vno sea señor de lo suyo ¶ Pues que assi es si esta iustiçia sp̃al es dicha igual por que entiende a egualdat . & unicuique tribuere quod suum est : \textbf{ quia aequum est , | quemlibet possidere sua . } Si igitur haec Iustitia specialis aequalis dicitur , \\\hline
1.2.10 & Ca cosa igual es \textbf{ que cada vno sea señor de lo suyo ¶ Pues que assi es si esta iustiçia sp̃al es dicha igual por que entiende a egualdat . } Como los çibdadanos pue dan estos biens de fuera partiçipar en dos maneras & quemlibet possidere sua . \textbf{ Si igitur haec Iustitia specialis aequalis dicitur , | et aequalitati intendit : } cum bona exteriora \\\hline
1.2.10 & e tiranlos alos que los meresçen donde se sigue \textbf{ que son dos las iustiçias spanles ¶ } La vna es conmutatiua & et repellentes dignos . \textbf{ Erit igitur dupliciter specialis Iustitia , } commutatiua , et distributiua . \\\hline
1.2.10 & en ordena otro . \textbf{ Empero la iustiçia comutatiua e camiadora } e la iustiçia distrabutian & semper perficit habentem in ordine ad alium . \textbf{ Magis tamen Iustitia commutatiua , } et distributiua perficit habentem in ordine ad alium , \\\hline
1.2.10 & que las ha en orden a otro \textbf{ que la iustiçia legal . } Mas por auentura adelante aura logar de fablar desto & et distributiua perficit habentem in ordine ad alium , \textbf{ quam legalis . } Sed de hoc forte alibi erit locus . \\\hline
1.2.10 & por que es dela ley . \textbf{ Et otra es la iustiçia spanl que es dicha ygual } por que ha de ygualar las cosas & quae dicitur legalis : \textbf{ et quaedam specialis , } quae dicitur aequalis . \\\hline
1.2.10 & cerca quales cosas ha de seer la iustiçia . \textbf{ Ca la iustiçia legal ha de ser } en & circa quae habet esse Iustitia , \textbf{ quia Iustitia Legalis habet } esse \\\hline
1.2.10 & en \textbf{ ca todas las materias morales e de costunbres } e cerca todas las obras delas uirtudes non tomando las segunt si . & esse \textbf{ circa totam materiam moralem , } et circa omnia opera virtutum : \\\hline
1.2.11 & La primera manera assi se declara . Ca assi como es dicho en el capitulo \textbf{ sobredicho la iustiçia legales en alguna meranera toda uirtud } Ca auer esta iustiçia es conplir la ley ¶ & nam ( ut in praecedenti capitulo dicebatur ) \textbf{ Legalis Iustitia est quodammodo omnis virtus . } Habere enim huiusmodi Iustitiam , \\\hline
1.2.11 & que le saga todo bien \textbf{ e uieda todo mal cunplir la les es seer omne uertuoso acabadamente . } Et por ende dize el philosofo en el primero cabło dela & et prohibet omne malum : \textbf{ implere legem , | est esse perfecte virtuosum . } Ideo primo Magnorum moralium , \\\hline
1.2.11 & Et por ende dize el philosofo en el primero cabło dela \textbf{ guatph̃ia moral fablando desta iustiçia . } que la iustiçia legales entera & Ideo primo Magnorum moralium , \textbf{ capitulo de Iustitia , | dicitur , } quod legalis Iustitia est perfecta virtus . \\\hline
1.2.11 & guatph̃ia moral fablando desta iustiçia . \textbf{ que la iustiçia legales entera } e acabada uirtud Et & dicitur , \textbf{ quod legalis Iustitia est perfecta virtus . } ergo per locum ab opposito , \\\hline
1.2.11 & que es dichaim iustiçia . \textbf{ es maliçia entera e acabada¶ pues que assi es quando los çibdadanos } ennigua cosa non guardan las leyes & legalis Iniustitia est integra , \textbf{ et perfecta malitia . } In nullo ergo obseruare leges , \\\hline
1.2.11 & ennigua cosa non guardan las leyes \textbf{ nin toma ninguna parte dela iustiçia legal . } esto es lo que los faze ser enteramente e coplidamente malos . & In nullo ergo obseruare leges , \textbf{ et ciues non participare | in aliquo legalem Iustitiam , } est eos esse integre et perfecte malos . \\\hline
1.2.11 & esto es lo que los faze ser enteramente e coplidamente malos . \textbf{ Mas assi como dize el philosofo enl de . x . qunto libro delas ethicas enl capitulo dela manssedunbre . } El mal destruye assi mismo & est eos esse integre et perfecte malos . \textbf{ Sed ( ut dicitur Ethic’ 4 cap’ de mansuetudine ) } malum se ipsum destruit : \\\hline
1.2.11 & ni la su çibdat non podrie mucho durar . \textbf{ Et por ende de parte dela iustiçia legal . } que es acabada uirtud . & nec vellent in aliquo participare legalem Iustitiam . \textbf{ Ex parte igitur ipsius legalis Iustitiae , } quae est perfecta virtus , \\\hline
1.2.11 & assi commo lo muestra el philosofo enlas polititas . \textbf{ Si los çibdadanos non guardaren la iustiçia legal . } non seria en ellos guardada orden . & ut declarari habet in Politicis : \textbf{ si ciues non participarent legalem Iustitiam , } non reseruaretur in eis ordo \\\hline
1.2.11 & non pueden los regnos estar nin durar . \textbf{ Mas ahun que sin la iustiçia sp̃al que se parte en iustiçias comutatiua e distributiua } non pueden estar nin durar esto & durare non possunt . \textbf{ Sed quod absque Iustitia speciali , | quae diuiditur } in Iustitiam commutatiuam , \\\hline
1.2.11 & Ca cada vno de los regnos \textbf{ e canda vna comunidat semeia avn cuerpo natural . } Ca assi commo veemos & Quodlibet enim regnum , \textbf{ et quaelibet congregatio assimilatur cuidam corpori naturali . } Sicut enim videmus corpus animalis constare \\\hline
1.2.11 & Ca assi commo veemos \textbf{ que el cuerpo de cada vna delas ainalias es conpuesto de departidos mienbros ayuntados e ordenados en ssi mismos . } assi cada vno dellos regnos & Sicut enim videmus corpus animalis constare \textbf{ ex diuersis membris connexis , | et ordinatis ad se inuicem : } sic quodlibet regnum , \\\hline
1.2.11 & pues que assi es en quanto los mienbros han orden entre si mismos \textbf{ e es en ellos en algua manera iuftiçia mudadora . } Mas en quanto se ordenan al coraçon & Ut ergo membra habent ordinem ad se inuicem , \textbf{ est in eis quodammodo Iustitia commutatiua . } Sed ut ordinantur ad cor , \\\hline
1.2.11 & e resçibe dineros . \textbf{ pueᷤ que assi es la iustiçia mudadora es } en quanto el vno ordena los sus bienes aprouecho del otro . Et el otro los sus bienes & et suscipit pecuniam \textbf{ qua caret . | Est igitur commutatiua Iustitia , } prout unus ordinat bona sua \\\hline
1.2.11 & e se acorten a sus menguas los vnos alos otros \textbf{ o es en ellos vna iustiçia mudadora e acorredora } por sabiduria natural & et sibi inuicem suae indigentiae subueniunt , \textbf{ est in eis quaedam commutatiua Iustitia , } sine qua corpus naturale durare non posset : \\\hline
1.2.11 & por sabiduria natural \textbf{ sin la qual el cuerpo nal non podria durar . } Dien assi en quanto los çibdadanos de vna çibdat & est in eis quaedam commutatiua Iustitia , \textbf{ sine qua corpus naturale durare non posset : } sic prout ciues eiusdem ciuitatis , \\\hline
1.2.11 & sin la qual la çibdat o el regno non podria estar ¶ \textbf{ Lo segundo en los mienbros del cuerpo hay vna iustiçia partidora en quanto han ordenamiento } e son ordenados al coraçon . & vel regnum non posset subsistere . \textbf{ Secundo in membris est quaedam distributiua Iustitia , } prout habent ordinem ad ipsum cor . \\\hline
1.2.11 & e el comienço del mouimiento es en el coraçon assi commo por auentura el comienço del sentires en el meollo bien \textbf{ assi en los çibdadanos ha iustiçia partidora en } quanto son ordenados a vna cosa & sicut forte principium sentiendi est in cerebro . \textbf{ Sic et in ciuibus est distributiua Iustitia , } prout ordinantur ad unum aliquid , \\\hline
1.2.11 & no podria estar \textbf{ si en el non fuesse guardada vna iustiçia partidora engsa } que los mienbros partiessen los vnos con los otros & Sicut ergo corpus naturale non subsisteret , \textbf{ nisi in eo reseruaretur | quaedam distributiua Iustitia , } ut quod cor membris debite influere : \\\hline
1.2.11 & que los çibdadanos \textbf{ en quanto han ordenamiento bueno entre si mismos } es en ellos la iustiçia mudadora . & et disponitur ad corruptionem . \textbf{ Patet igitur quod prout ciues habent ordinem ad se inuicem , } est in eis Iustitia commutatiua . \\\hline
1.2.11 & en quanto han ordenamiento bueno entre si mismos \textbf{ es en ellos la iustiçia mudadora . } mas en quanto han orden & Patet igitur quod prout ciues habent ordinem ad se inuicem , \textbf{ est in eis Iustitia commutatiua . } Prout vero habent ordinem ad Regem , \\\hline
1.2.11 & e conpara conn al Rey o al prinçipe \textbf{ es en ellos guardada la iustiçia partidora . } Mas en quanto los çibdadanos cunplen las leyes & vel ad Principem , \textbf{ reseruatur in eis Iustitia distributiua . } Prout vero adimplent leges , \\\hline
1.2.11 & Mas en quanto los çibdadanos cunplen las leyes \textbf{ es en ellos iustiçia legal . } Empero en qual quier manera que se tome la iustiçia & Prout vero adimplent leges , \textbf{ est in illis Iustitia legalis . } Quocunque tamen modo sumatur Iustitia , \\\hline
1.2.11 & en el capitulo dela iustiçia \textbf{ que cosa iusta es vna cosa bien ygualada e bien ordenada } e mantiene en si las çibdades e los regnos . & cap’ de iustitia , \textbf{ quod iustum est | quoddam proportionabile , } et continet urbanitates . \\\hline
1.2.12 & pues que assi es quanto la cosa \textbf{ que ha alma sobrepiua ala que non ha alma . } tanto el Rey o el prinçipe deue sobrepuiar la ley . & Princeps vero est quaedam animata lex . \textbf{ Quantum ergo animatum inanimatum superat , } tantum Rex siue Princeps debet superare legem . \\\hline
1.2.12 & Ca la mengua dela iustiçia \textbf{ e la desigualdat les tira la dignidat e la magestad real . } Ca los Reyes sin iustiçia & quia eorum iniustitia , \textbf{ et inaequalitas tollit ab eis regiam dignitatem . } Nam reges iniusti \\\hline
1.2.12 & por nonbre comunal \textbf{ que quiere dezir cosa clara e cosa apuesta . } Et esta estrella algunas vezes nasçe ante del sol & et venustatem communi nomine \textbf{ appellatur Venus . } Haec autem stella aliquando praecedit solem , \\\hline
1.2.12 & Et por ende la entençion del philosofo es dezir que esta estrella \textbf{ que dizen venꝮ que es ran fermosa e tan apuesta que algunas vezes la llaman luzero e algunas vezes uespero . } non es tan resplandeçiente & quod Venus , \textbf{ quae est tam pulcherrima stella , | et quae aliquando dicitur Lucifer , } aliquando Hesperus , \\\hline
1.2.12 & Et las estrellas resplandesçen por fermosura corporal \textbf{ e nos alunbran con luz corporal . } Mas la iustiçia resplandeçe por fermosura de honestad & Stellae enim pollent pulchritudine corporali , \textbf{ et illuminant nos lumine corporali : } sed Iustitia pollet \\\hline
1.2.12 & e por fermosura spunal \textbf{ e honrra nos de perfecçion uirtual . } Et por ende quanto la fermosura spunal & pulchritudine honesta et spirituali , \textbf{ et ornat nos virtuali perfectione . } Quanto igitur pulchritudo spiritualis \\\hline
1.2.12 & en el primers libro dela methaphisica \textbf{ que señal manifiesta es de omne sabio } quando puede enssennar los otros . & Ideo scribitur 1 Metaphys’ \textbf{ quod signum omnino scientis , } est posse docere . \\\hline
1.2.12 & por que las ha de guiar la iustiçia es la mas acabada . \textbf{ Et por ende todas las otras uirtudes morałs̃ que acaban el omne en ssi deuen auer la iustiçia } assi commo reina e sennora & quia est earum directiua , \textbf{ omnes aliae virtutes morales , | quae perficiunt hominem in se , } se videntur habere ad Iustitiam , \\\hline
1.2.12 & por la iustiçia \textbf{ que por las otras uirtudes morales . } assi conuiene mas al Rey & quam in aliis : \textbf{ sic ex Iustitia } magis manifestatur perfectio bonitatis , \\\hline
1.2.13 & que auemos de fazer ¶ \textbf{ pues que assi es commo los omes alguas vezes puedan } e les contezca de se auer derechamente & per quam regulentur in agendo . \textbf{ Cum igitur circa timores , } et audacias contingat \\\hline
1.2.13 & e non derechamente en los temores e en las osadias . \textbf{ Conuiene de dar alguna uirtud medianera en los temores } e en las osadias & et audacias contingat \textbf{ aliquem se habere recte , } et non recte , \\\hline
1.2.13 & nin ninguno non es dicho osado \textbf{ si non quando acomete alguna cosa espantable e peligrosa . } Mas deuedes saber & nec omnis dicitur audax , \textbf{ nisi aggrediatur aliquod terribile , et periculosum . } Periculorum autem quaedam sunt bellica , \\\hline
1.2.13 & sea çerca algun bien \textbf{ e cerca alguna cosa fuerte e graue } por que los periglos de las batallas son mas fuertes & circa bonum , \textbf{ et difficile ( quia difficiliora , } et terribiliora sunt pericula bellica , \\\hline
1.2.13 & aquel es fuerte que non es temeroso . \textbf{ Et por ende al fuerte parte nesçe non temer qual si quier periglo que la razon o el entendimiento iudga sinplemente } que non son de temer . & Sed adhuc et in mari , \textbf{ et in aegritudinibus intimidus est , } qui est fortis . \\\hline
1.2.13 & Et naturalmente cada vno fuye dela tristoza \textbf{ assi commo naturalmente sigue las cosas delectables . } Et pues que assi es commo nos natural mente fuyamos dela tristeza & Tristia autem naturaliter \textbf{ quilibet fugit , sicut naturaliter sequitur delectabilia . } Cum ergo naturaliter tristia fugiamus , \\\hline
1.2.13 & assi commo todos dizen comunal mente esto podemos prouar por tres razones ¶ \textbf{ La primera por que acometer parte nesçe al mas fuerte . } Mas sufrir parte nesçe al mas fiaco & triplici via venari potest . \textbf{ Primo , quia aggrediendum , | est fortioris : } sustinere autem , \\\hline
1.2.13 & La primera por que acometer parte nesçe al mas fuerte . \textbf{ Mas sufrir parte nesçe al mas fiaco } por que el que acomete es conparado a aquellos a quien acomete & est fortioris : \textbf{ sustinere autem , } debilioris est . \\\hline
1.2.13 & Et adelante dize \textbf{ que la fortaleza es en sofrir las cosas tristes . } Et por ende ya declarado es cerca quales cosas ha de seer la fortaleza . & ( et subdit ) \textbf{ Fortitudinem esse in sustinendo tristia . } Declaratum est igitur , \\\hline
1.2.13 & Pues que assi es fincanos de declarar \textbf{ en qual manera podemos fazer anos mismos fuertes } Pues que assi es deuen dos notar e entender que commo quier que la uirtud sea contraria . & restat ergo declarandum , \textbf{ quomodo possumus facere nos ipsos fortes . } Notandum ergo , \\\hline
1.2.14 & que el philosofo en el terçer \textbf{ libdelas ethicas en el capitulo dela fortaleza } de parte siere espeçias o . & Distinguit Philosophus 3 Ethicorum cap’ \textbf{ de fortitudine septem species , } seu septem maneries Fortitudinis . \\\hline
1.2.14 & La segunda seruil ¶ \textbf{ La terçera caualleril ¶ } La quarta os fortaleza rrauiosa & Secunda seruilis . \textbf{ Tertia , militaris . } Quarta , furiosa . \\\hline
1.2.14 & La serta es bestial e de bestra¶ \textbf{ Lisertima es fortaleza uirtuosa e de uirtud } ¶La fortaleza çeuiles & Sexta , bestialis . \textbf{ Et septima , est Fortitudo virtuosa . } Fortitudo enim ciuilis est , \\\hline
1.2.14 & e quariendo ganar honrra \textbf{ acomete alguna cosa fuerte e espantable . } Onde dize el philosofo & et volens honorem adipisci , \textbf{ aggreditur aliquod terribile , } unde ait Philosophus , \\\hline
1.2.14 & Enpero tales non son fuertes propiamente . \textbf{ Ca la fortaleza udadera acomete la batalla } segunt que dize el philosofo non por & non tamen proprie tales fortes existunt , \textbf{ quia fortitudo vera aggreditur pugnam } ( ut Philosophus innuit ) non propter furorem , \\\hline
1.2.14 & que son meridionales \textbf{ e flacos aurian fortaleza bestial e de bestia . } ca tales serien semeiantes alas bestias & credentes eos esse meridionales , \textbf{ habent Fortitudinem bestialem . } Sunt enim tales quasi bestiae insensatae aggredientes bellum , \\\hline
1.2.14 & e el su pueblo a periglos de batallas \textbf{ si non quando ouieren razon derecha para auer batalla . } Et si non vieren & ut non exponant suam gentem periculis bellicis , \textbf{ nisi habeant iusta bella , } et nisi videant magnum bonum patriae vel regni , \\\hline
1.2.15 & euedes saber \textbf{ que la tenprança entre las quatro uirtudes cardinales tiene el postrimer grado . } Calapdençia e la iustiçia son mas prinçipales & Temperantia \textbf{ inter virtutes cardinales ultimum gradum tenet . } Prudentia enim et Iustitia principaliores \\\hline
1.2.15 & Calapdençia e la iustiçia son mas prinçipales \textbf{ que las otras uirtudes morales . } Por que la pradençia es en el entendimiento & Prudentia enim et Iustitia principaliores \textbf{ sunt virtutibus moralibus ; } quia Prudentia est in intellectu , \\\hline
1.2.15 & e la iustiçia en la uoluntad . \textbf{ las otras uirtudes morales son en el appetito sensitiuo } asi commo en el appetito enssannador & Iustitia in voluntate . \textbf{ Virtutes vero morales sunt | in appetitu sensitiuo , } ut in irascibili , et concupiscibili . \\\hline
1.2.15 & asi commo en el appetito enssannador \textbf{ e en el appetito desseador . } Ca el appetito sensitiuo fallesçe & in appetitu sensitiuo , \textbf{ ut in irascibili , et concupiscibili . } Appetitus autem sensitiuus deficit a voluntate , et intellectu . \\\hline
1.2.15 & e en el appetito desseador . \textbf{ Ca el appetito sensitiuo fallesçe } e es menor que la uoluntad e el entendimiento . & ut in irascibili , et concupiscibili . \textbf{ Appetitus autem sensitiuus deficit a voluntate , et intellectu . } Ideo Prudentia et Iustitia sunt virtutes principaliores . \\\hline
1.2.15 & Por ende la pradençia e la iustiçia son mas prinçipales . \textbf{ Ca la pradençia es guiadora de todas las otras uirtudes . Et la iustiçia regladora . } mas la fortaleza e la tenprança maguer & Ideo Prudentia et Iustitia sunt virtutes principaliores . \textbf{ Est autem Prudentia principalior Iustitia ; | quia ipsa est directiua omnium aliarum virtutum . } Fortitudo autem , et Temperantia , \\\hline
1.2.15 & por razon que las uirtudes \textbf{ que tienpran las passiones en dos maneras sieruen ala razon e al entendimiento } segunt que las passions en dos manera & tamen inter virtutes principales computantur . \textbf{ Nam virtutes moderantes passionem , } dupliciter rationi deseruiunt , \\\hline
1.2.15 & e la tenperança \textbf{ deuen ser todas entre las uirtudes prinçipales . } Et por ende la fortaleza es mas prinçipal & Fortitudo ergo et Temperantia inter principales virtutes computari debent . \textbf{ Est autem Fortitudo principalior Temperantia , } quia Fortitudo magis ordinatur \\\hline
1.2.15 & nos de dezir dela tenperança \textbf{ que entre las uirtudes prinçipales tiene el postrimero grado . } Et pues que assi es deuedes saber & Restat dicere de ipsa Temperantia , \textbf{ quae inter virtutes principales | ultimum gradum tenet . } Sciendum ergo , \\\hline
1.2.15 & que alguas delas \textbf{ delectaconnes corporales son fuertes } e algunan sson flacas . & Delectationum autem sensibilium \textbf{ quaedam sunt fortes , } quaedam sunt debiles , \\\hline
1.2.15 & e nos contezça de tomar \textbf{ delectaconn en las cosas senssibles de todos los sesos . } Empero las mas fuertes delecta connes son en el gusto & Licet enim sint quinque sensus , \textbf{ et circa sensibilia omnium sensuum } contingat nos delectari : \\\hline
1.2.15 & e con mayor feruor nos delectamos enellas . \textbf{ Et por que mas nos ayuntamos alas cosas delectables del gusto } e del tan nimiento & et feruentius delectamur . \textbf{ Magis autem unimur delectationibus gustus , } et tactus , \\\hline
1.2.15 & es \textbf{ por que las cosas senssibles del gusto } e del tannimiento & Secundo hoc idem patet : \textbf{ quia sensibilia gustus , } et tactus magis directe \\\hline
1.2.15 & Por la qual razon si la uirtud ha de seer cerca el bien \textbf{ e cerca cosa guaue prinçipalmente es de poner la tenpranca çerca } aquellas delectaçonnes delas quales nos arredramos con mayor g̃ueza & circa bonum et delectabile , \textbf{ ponenda est principaliter Temperantia } circa delectationes illas , \\\hline
1.2.15 & Et assi podemos tomar de ligero las maneras dela tenpranca . \textbf{ Ca si parte nesçe ala tenpranca de repremirlas } delecta connes matermentales con que seca el cuerpo nos & Species autem temperantiae de leui sumere possumus . \textbf{ Nam si spectat ad Temperantiam reprimere } delectationes nutrimentales , et venereas : \\\hline
1.2.15 & fuermos mesurados e abstinentes en el comer e en el beuer . \textbf{ Mas las delecta connes carnales abaxaremos e reprimetemos si fuermos castos e linpios . } Et por ende quatro son las partes dela & si fuerimus sobrii , et abstinentes : \textbf{ venereas vero , | si fuerimus casti , et pudici : } quatuor erunt partes Temperantiae , \\\hline
1.2.15 & assi commo aquellos que sobrepuian en el beuer son beodos \textbf{ Mas nos tenprando nos en el comer lo mos astinentes . } Et pues que assi es la astinençia & sunt ebrii . \textbf{ Temperando vero nos a cibo , } sumus abstinentes . Abstinentia ergo , \\\hline
1.2.15 & assi commo \textbf{ en manera contraria la vna dela otra . } Ca la fortaleza es en acometer las cosas espantables & Nam Temperantia , \textbf{ et Fortitudo quasi e contrario se habent . } Fortitudo enim est \\\hline
1.2.15 & que cosa es la tenpranca . \textbf{ Ca es uirtud que repreme las delectaçonnes sensibles de los cinco sesos } e tienpra los non sentimientos & quid est Temperantia : \textbf{ quia est virtus reprimens | delectationes sensibiles , } et moderans insensibilitates . \\\hline
1.2.15 & Lo terçero auemos mostrado \textbf{ quales e quantas lon las maneras dela tenprança . } Ca son quatro . & Tertio autem manifestabatur , quae , \textbf{ et quot sunt species eius : } quia quatuor , \\\hline
1.2.15 & Ca esto podemos fazer mayormente \textbf{ si nos guardaremos e arredraremos de las cosas delectables de los sesos . } Ca segunt el philosofo & quia hoc maxime faciemus \textbf{ a delectactionibus abstinendo . | Debemus enim } secundum Philosophum Ethicorum 2 hoc pati , \\\hline
1.2.16 & tenpranca ganamos retrayendonos \textbf{ e tirando nos delas delecta connes senssibles . } Mas la fortaleza podemos ganar & abstinendo , \textbf{ et retrahendo nos | a delectationibus sensibilibus : } fortitudinem vero acquirere possumus , \\\hline
1.2.16 & por que muchas cosas tales prouamos en nr̃auida . \textbf{ Ca muchas vezes contesçe que vienen a nos muchas cosas delectables delas quales si nos nos guardatemos e apareiaremos para ser tenprados . } mas non es assi dela fortaleza . & quia multa talia experimur in vita , \textbf{ multotiens enim occurrunt nobis delectabilia , | a quibus abstinendo disponimur , } ut simus temperati . \\\hline
1.2.16 & Et por ende si el Rey non fuere fuerte \textbf{ e non fuer firme en el coraçon es de deno star por ello . } Et es mas de denostar si fuer deste prado & Si ergo Regem non esse virilem , \textbf{ et non esse constantem | animo est exprobrabile , } patet quod est exprobrabilius \\\hline
1.2.16 & Mas que sea pecado muy bestial paresçe por que assi commo es dicho de suso la tenpranca ha de ser \textbf{ cora las cosas delectables del tannemiento e del gusto } segunt las quales cosas es cosa comun anos & Nam ( ut supra dicebatur ) \textbf{ temperantia et intemperantia fieri habent | circa delectabilia tactus , et gustus , } secundum quae , \\\hline
1.2.16 & cora las cosas delectables del tannemiento e del gusto \textbf{ segunt las quales cosas es cosa comun anos } e alas bestias de nos delectar . & circa delectabilia tactus , et gustus , \textbf{ secundum quae , } delectari commune est nobis , et brutis . \\\hline
1.2.16 & Pues que assi es \textbf{ si non es cosa conuenible al rey } que ha de & naturaliter est quid seruile . \textbf{ Si ergo indecens est Regem , } cuius est aliis dominari , \\\hline
1.2.16 & Et por ende si non es cosa conueniente \textbf{ de ser el Rey moço en costunbres } e de non segnir razon & Si ergo indecens est Regem \textbf{ esse puerum moribus , } et non sequi rationem , sed passionem : \\\hline
1.2.17 & Onde conuiene saber \textbf{ que esta s ochon uirtudes o catan alos bienes tenporales de fuera } o alos males tenporales de fuera . & vel respiciunt \textbf{ exteriora bona , } vel exteriora mala . \\\hline
1.2.17 & que esta s ochon uirtudes o catan alos bienes tenporales de fuera \textbf{ o alos males tenporales de fuera . } Mas aquellos bienes tenporales de fuera & exteriora bona , \textbf{ vel exteriora mala . } Si exteriora bona , \\\hline
1.2.17 & o alos males tenporales de fuera . \textbf{ Mas aquellos bienes tenporales de fuera } o son bienes & vel exteriora mala . \textbf{ Si exteriora bona , } illa vel sunt bona secundum se , \\\hline
1.2.17 & Et pues que assi es primeramente diremos delas uirtudes \textbf{ que catan alos bienes tenporales de fuera segunt si . } Et despues diremos delas à tu & Primo ergo dicemus \textbf{ de virtutibus respicientibus } exteriora bona secundum se . \\\hline
1.2.17 & Et despues diremos delas à tu \textbf{ desque catan alos males tenporales de fuera . } Et otrossi diremos delas uirtudes & de virtutibus respicientibus \textbf{ exteriora bona secundum se . } Postea determinabimus \\\hline
1.2.17 & Et otrossi diremos delas uirtudes \textbf{ que catan alos bienes tenporales de fuera } en quanto son ordenados a otra cola . & Postea determinabimus \textbf{ de virtutibus respicientibus exteriora bona in ordine ad aliud . } Bona autem exteriora \\\hline
1.2.17 & Et deuedes laber \textbf{ que los bienes tenporales de fuera o son aprouechosos } assi conmo los dineros o las riquezas & de virtutibus respicientibus exteriora bona in ordine ad aliud . \textbf{ Bona autem exteriora } vel sunt utilia , \\\hline
1.2.17 & e despues diremos delas uirtudes \textbf{ que catan alos bienes honestos . } por que el nuestro conosçimiento conmienca en los sesos & de virtutibus respicientibus bona utilia : \textbf{ et postea de respicientibus bona honesta . } Incipit enim nostra cognitio a sensu . \\\hline
1.2.17 & Et por ende commo los bienes aprouechosos \textbf{ sintamos nos mas que los bienes honestos . } primera mendiremos delas & Cum ergo bona utilia \textbf{ sensibiliora sint honestis , } prius determinandum est \\\hline
1.2.17 & primera mendiremos delas \textbf{ uirtudesque catan dos bienes prouechosos . } Mas assi commo dize el philosofo & prius determinandum est \textbf{ de virtutibus respicientibus bona utilia . } Circa autem bona utilia \\\hline
1.2.17 & Et por que cada vna destas cosas \textbf{ escontra regla derecha de razon e de entendimiento . } Conuiene de dar alguna uirtud medianera & quia utrunque est \textbf{ contra rectam regulam rationis , } oportet dare virtutem aliquam mediam \\\hline
1.2.17 & espenssas quales deue fazer \textbf{ non deue las suᷤ propias rentas esparzer } nin espender vanamente & tamen ut possit debitos sumptus facere , \textbf{ non debet proprios redditus inaniter dispergere . } Ergo non usurpare redditus alienos , \\\hline
1.2.17 & nin espender vanamente \textbf{ nin deue tomar las rentas agenas por fuerca } mas deue auer cuydado de su fazienda & non debet proprios redditus inaniter dispergere . \textbf{ Ergo non usurpare redditus alienos , } habere debitam curam de propriis , \\\hline
1.2.17 & mas deue auer cuydado de su fazienda \textbf{ e delas sus rentas propias e fazer dellas sus espenssas quales conuiene ¶ } Estas tres cosas son aquellas en que ha de ser la franqueza & habere debitam curam de propriis , \textbf{ et ex eis debitos sumptus facere : } sunt illa tria circa quae videtur esse liberalitas . \\\hline
1.2.17 & espenssas quales deuees la franqueza prinçipalmente e primero . \textbf{ Mas despues desto es en guardar las tentas propias . } Et despues es en non tomar nin forcar los bienes agenos . & est liberalitas principaliter , et primo . \textbf{ Circa autem proprios redditus custodire , } et circa non accipere alienos , \\\hline
1.2.17 & Mas despues desto es en guardar las tentas propias . \textbf{ Et despues es en non tomar nin forcar los bienes agenos . } Ca aquel que vsurpa e toma los bienes prouechosos agenos malamente commo non deue . & Circa autem proprios redditus custodire , \textbf{ et circa non accipere alienos , | est ex consequenti . } Usurpans enim bona utilia , \\\hline
1.2.17 & Et despues es en non tomar nin forcar los bienes agenos . \textbf{ Ca aquel que vsurpa e toma los bienes prouechosos agenos malamente commo non deue . } este paresçe que es muy cobdicioso de auer ¶ & est ex consequenti . \textbf{ Usurpans enim bona utilia , } et non accipiens ea sicut debet , \\\hline
1.2.17 & Ca por qual si quier cosa que el omne es dich̃o tal aquella cosa \textbf{ por que lo es . es dicha mastal . } Et por ende si alguno es liberal e franco en guardando las sus rentas propreas & Nam propter quod unumquodque tale , \textbf{ et illud magis . } Si enim liberalis conseruans proprios redditus , \\\hline
1.2.17 & assi commo dicho es ha de seer \textbf{ en uso conueinble de espender del auer . } Ca husar del auer es en espender lo & esse debet \textbf{ circa debitum usum pecuniae . } Uti autem pecunia , \\\hline
1.2.17 & Lo segundo esto mismo se praeua assi por que ala uirtud \textbf{ mas prinçipal parte nesçe de fazer mayor bien . } Et mayor bien es en bien fazer & Secundo hoc idem patet , \textbf{ quia ad virtutem principalius spectat | facere maius bonum . } Maius autem bonum est benefacere , \\\hline
1.2.17 & Et mayor bien es en bien obrar \textbf{ que en non obrar cosas torpes . } Et aquel que despiende commo deue & et bene operari , \textbf{ quam turpia non operari . } Qui autem debite expendit \\\hline
1.2.17 & que tiene este faze bien \textbf{ Mas el que guarda las rentas propias e lo suyo propio } e toma el auer delas sus possesiones propias & et aliis dona largitur , benefacit . \textbf{ Custodiens vero proprios redditus , } et accipiens pecuniam a propriis possessionibus , \\\hline
1.2.17 & que guardar las sus rentas propias \textbf{ o que non tomar los bienes agenos . } Ca guardar omne lo suyo propio & quam proprios redditus custodire , \textbf{ vel quam aliena non surripere . } Nam custodire propria \\\hline
1.2.17 & Ca guardar omne lo suyo propio \textbf{ non es cosa fuerte por si . } Ca cada hun omne es naturalmente inclinado a amar asi mismo & Nam custodire propria \textbf{ secundum se non est difficile : } quia unusquisque naturaliter inclinatur \\\hline
1.2.17 & e aguardar los sus biens propos \textbf{ Mas dar los sus biens propios ha alguna guaueza por si . } Ca los bienes propios son & et ut sua bona custodiat . \textbf{ Dare autem propria bona , | secundum se difficultatem habet : } quia propria bona sunt aliquid ad nos pertinens , \\\hline
1.2.17 & Mas dar los sus biens propios ha alguna guaueza por si . \textbf{ Ca los bienes propios son } cosaque parte nesçen a nos mismos & secundum se difficultatem habet : \textbf{ quia propria bona sunt aliquid ad nos pertinens , } et naturaliter afficimur ad illa . Immo auari adeo afficiuntur \\\hline
1.2.17 & e naturalmente amamos lo que pertenesçe anos . \textbf{ Mas los auarientos tanto aman los bienes tenporales de fuera } assi commo las riquezas & et naturaliter afficimur ad illa . Immo auari adeo afficiuntur \textbf{ ad ista exteriora bona , } ut pecuniam , \\\hline
1.2.18 & mas esta en disposiconn \textbf{ e en ordenamiento firme de alma } e en voluntad del que ha de dar & non esse in multitudine datorum , \textbf{ sed in habitu , } idest in facultate et voluntate dantis . \\\hline
1.2.18 & Pues que assi es nin en el gouernamiento de los omes \textbf{ non deue ser ninguna cosa ociosa nin baldia . } por la qual razon commo la naturaleza de los omes se tenga & Ergo nec in regimine vitae humanae \textbf{ aliquid ociosum esse debet . } Quare cum natura humana modicis contenta sit , \\\hline
1.2.18 & desbien commo el libal non lo es de ligero se puede fazer \textbf{ qual quier gastador liberal e franco ¶ } pues que assi es si es conueinble al Rey & de leui \textbf{ quis cum sit prodigus , | fieri poterit liberalis . } Si ergo omnino decens est \\\hline
1.2.18 & que conuiene alos Reyes \textbf{ de ser largos liberales e dadores . } Ca esta uirtud que es en las espenssas & quod deceat \textbf{ eos esse largos , liberales , et communicatiuos . } Haec enim virtus , \\\hline
1.2.18 & por ende el que es abondado en espenssas \textbf{ e en donadios es dichon largo . } Ca ha . manera daua so ancho e largo e da conplidamente lo que tiene & Abundans ergo in sumptibus , et dationibus , \textbf{ dicitur largus : } quia ad modum largi vasis abunde emittit \\\hline
1.2.18 & Ca non lo dan por razon de bien \textbf{ mas dan lo por vana eglesia o por que sean loados o por algunan otra razon vana ¶ Et pues que assi es conuiene alos Reyes } e alos prinçipes de ser liƀͣales . & Non enim dant boni gratia , \textbf{ sed magis dant | ut laudentur , } et propter inanem gloriam \\\hline
1.2.19 & mas espeçial razon de bondat e de guaueza \textbf{ que en las espenssas medianas e mesuradas . } Conuiene de dezir que la magnifiçençia & specialis ratio bonitatis et difficultatis , \textbf{ quae non reperitur | in mediocribus sumptibus , } dici potest magnanimitatem \\\hline
1.2.19 & sienpre sera cerca las espenssas mesuradas . \textbf{ Ca assi commo las espenssas pequanas son mesuradas } e son conuenibles alos pobres . & circa mediocres sumptus . \textbf{ Nam sicut parui sumptus sunt mediocres } et proportionati pauperibus : \\\hline
1.2.19 & Mas deuedes saber \textbf{ que enlas grandes obras contesçe a algimos de fallesçer . } Ca non entienden & a magnis operibus sumpsit nomen . \textbf{ In magnis autem operibus contingit aliquos deficere , } quia non intendunt \\\hline
1.2.19 & Et assi commo la libalidat faze espenssas \textbf{ e da dones conuenibles a las riquezas } assi la magnificençia faze espenssas conuenibles alas grandes obras ¶ & Et sicut liberalitas est faciens sumptus , \textbf{ et dationes proportionatas facultatibus : } sic magnificentia est faciens sumptus decentes magnis operibus . \\\hline
1.2.19 & e da dones conuenibles a las riquezas \textbf{ assi la magnificençia faze espenssas conuenibles alas grandes obras ¶ } visto que cosa es la magnificençia & et dationes proportionatas facultatibus : \textbf{ sic magnificentia est faciens sumptus decentes magnis operibus . } Viso quid est magnificentia : \\\hline
1.2.19 & ¶L segundo a toda la comunidat¶ \textbf{ Lo terçero a algunas personas espeçiales ¶ } Lo quarto assi mismo . & ad totam communitatem , \textbf{ ad aliquas personas , et ad seipsum . } Magnificus enim circa omnia \\\hline
1.2.19 & si \textbf{ ouiereconplidamente las riquezas fazer espenssas conuenibles en toda la comunidat } por que los bienes comunes lon en alguna manera diuinales & ( si adsit facultas ) \textbf{ facere decentes sumptus | circa totam communitatem . } Nam ipsa bona communia \\\hline
1.2.19 & ouiereconplidamente las riquezas fazer espenssas conuenibles en toda la comunidat \textbf{ por que los bienes comunes lon en alguna manera diuinales } e de dios . & circa totam communitatem . \textbf{ Nam ipsa bona communia | quodammodo diuina sunt . } Unde Philosophus ait , \\\hline
1.2.19 & que los dones comuns han alguna semeiança \textbf{ alos dones sagrados e de dios . } Ca el bien diuinal muy flacamente es representado en vna persona singular . & quod dona communia habent aliquid \textbf{ simile donis Deo sacratis . } Bonum enim diuinum \\\hline
1.2.19 & alos dones sagrados e de dios . \textbf{ Ca el bien diuinal muy flacamente es representado en vna persona singular . } Mas el bien diuinal reluze e & simile donis Deo sacratis . \textbf{ Bonum enim diuinum } valde debiliter repraesentatur \\\hline
1.2.19 & Et pues que assi es paresçe cerca qual cosa es la magnificençia . \textbf{ Ca ha de ser çerca espenssas conuenibles en la grandes obras . } Empero prinçipalmente et primero es cerca las grandes espenssas & Patet ergo circa quid est magnificentia : \textbf{ quia est circa sumptus decentes magnis operibus . } Principaliter tamen est \\\hline
1.2.19 & fechas cerca las personas dignas \textbf{ e çerca la su persona miłma ¶ } Mostrado que cosa es la magnificençia & circa personas dignas , \textbf{ et circa seipsum . Ostenso quid est magnificentia , } et circa quae habet esse : \\\hline
1.2.20 & por vna dinarada de pimienta . \textbf{ Ca quando non fazen espenssas conuenibles en el grant conbite } esto non es si non por que quiere bazer pequanas el penssas & pro denariato piperis . \textbf{ Dum enim circa magnum conuiuium | non faciunt decentes sumptus , } volentes parcere modicae expensae , \\\hline
1.2.20 & en qual manera faga granada obra \textbf{ e en qual manera faga sus dones granados e conuenibles } o en qual manera faga sus bodas conuenibles & qualiter faciat magnum opus , \textbf{ ut qualiter faciat debitas largitiones , } vel quomodo faciat decentes nuptias : \\\hline
1.2.20 & mas toda su entençion es \textbf{ en qual manera faga peannas espenssas . } por que la propiedat del paruifico es & sed tota sua intentio est , \textbf{ quomodo faciat paruos sumptus . } Est enim proprietas paruifici , \\\hline
1.2.20 & assi conmo el mienbro non se podria partir del cuerpo lindolor . \textbf{ Et en esto el partu fico partiçipa con el auariento } por que todos los paruificos son auarientos . & sine tristitia , et dolore . \textbf{ Participat enim in hoc paruificus cum auaro : } quia omnis paruificus auarus est , \\\hline
1.2.20 & parufico dar vna cosa muy pequana dela cosa que mucho ama . \textbf{ puesto avn que mucho resciba dela cosa vil nol parezca a el que da mas que deue dar . } assi en essa misma manera non puede el & ita modicum de caro , \textbf{ dato etiam quod multum recipiat de vili , | quin videatur ei plus dare , } quam debeat : \\\hline
1.2.20 & diuinalez \textbf{ çerca los bienes comunes . } Et despues desto es çerca las personas dignas & circa opera diuina , \textbf{ et circa communia : } ex consequenti vero \\\hline
1.2.20 & que es cabesça e prinçipe de todo el mundo . \textbf{ mucho parte nesçe a el de se auer granadamente } e honrradamente cerca las eglesias saguadas & qui est caput et Princeps uniuersi , \textbf{ maxime spectat ad ipsum magnifice se habere } circa templa sacra , \\\hline
1.2.20 & e honrradamente cerca las eglesias saguadas \textbf{ e cerca los apareiamientos diuinales e dela santa eglesia . } Mas por el que es persona publica e comuna & maxime spectat ad ipsum magnifice se habere \textbf{ circa templa sacra , | et erga praeparationes diuinorum . } Quia vero est persona publica , \\\hline
1.2.20 & e todo el regno es ordenado \textbf{ mucho parte nesçe a el de se auer granadamente } e honrradamente çerca los bien es comuns & et totum regnum , \textbf{ maxime spectat ad eum magnifice } se habere circa bona communia , \\\hline
1.2.20 & e los fijos auiendo moradas honrradas \textbf{ e faziendo bodas conuenibles e honrradas } e vsando de caualłias marauillosał & habendo habitationes honorabiles , \textbf{ faciendo nuptias decentes , } exercendo militias admirabiles . \\\hline
1.2.21 & e en el bien comun \textbf{ e en las personas dignas de honrra . } Por ende aquel paresçegłioso & et erga rempublicam , \textbf{ et circa personas dignas , } maxime quis apparet gloriosus , \\\hline
1.2.21 & Et por quela uirtud es cerca bien e cerca la cosaguaue . \textbf{ Por ende mucho parte nesçe al magnifico } en las sus muy grandes obras & circa bonum et difficile , \textbf{ ideo maxime spectat ad magnificum } in suis magnificis operibus , \\\hline
1.2.21 & pequana cosa pierden lo mucho . \textbf{ Et pues que assi es el magnifico aqui parte nesçe de non auer cuydado de contar lo que despiende de despenssa egual faze mas granada obra e mas mognifica por que non perdona alas despenssas conuenibles . } Mas contesçe alas vegadas que el auariento despiende mas en alguna obra que el libal & ab aequali sumptu , \textbf{ facit opus magis magnificum , | quia non parcit decentibus sumptibus . } Contingit enim aliquando auarum \\\hline
1.2.21 & Por ende la obra en que despiende mucho non la faze commo conuiene . \textbf{ Por la qual cosa dela espenssa egual o alguas vegadas dela } mas pequeña el liberal o el magnifico fazemas conueniblemente su obra que el paruifico e el auariento . & ubi multum expendit , \textbf{ indecenter facit . Quare ab aequale sumptu , | vel etiam aliquando a minori , liberalis , } vel magnificus facit decentius opus , \\\hline
1.2.21 & las riquezas son oçiosas \textbf{ e en vano si dellas non fueren fechas donaçonnes conuenibles e espenssas conuenibles . } Et pues que assi es en quanto los Reyes e los prinçipes & ociosae sunt diuitiae , \textbf{ nisi ex eis fiant debitae largitiones , | et decentes sumptus . } Quanto igitur Reges , \\\hline
1.2.22 & assy commo son las honrras . \textbf{ Por enl de assi commo çerca los bienes aprouechosos son dos uirtudes . } La vna que cata alas grandes espenssas & cuiusmodi sunt honores . \textbf{ Sicut igitur circa ipsa bona utilia est duplex virtus } una respiciens magnos sumptus , \\\hline
1.2.22 & En essa misma manera \textbf{ çerca los bienes honestos son dos uirtudes ¶ } La vna que cata las grandes horras & ut Liberalitas : \textbf{ sic circa ipsa bona honesta est duplex virtus , } una quae respicit magnos honores , \\\hline
1.2.22 & Et otra es que cata las honrras medianas \textbf{ que comunalmente se puede dezir uirtud amadora de honrra } Mas en tres maneras se puede cada vno auer en las grandes honrras & et alia quae respicit mediocres , \textbf{ quae communi nomine dici potest honoris amatiua . } In magnis autem honoribus tripliciter \\\hline
1.2.22 & para fazer grandes cosas . \textbf{ Et poderosos para vsar de cosas grandes e altas . } Enpero por fiaqueza de coraçon & Videmus enim aliquos de se aptos ad magna , \textbf{ potentes magna et ardua exercere : } quadam tamen pusillanimitate ducti , \\\hline
1.2.22 & e non cerca quales se quier honrras \textbf{ mas es cerca las honrras grandes . } Et despues desto es çerca las riquezas & et non circa honores \textbf{ quoscunque sed honores magnos . } Ex consequenti autem est \\\hline
1.2.23 & la quinta propriedat es \textbf{ que parte nesçe al } magnamn moqua non aya cuydado de seer alabado & Quinto spectat ad magnanimum , \textbf{ ut non sit ei curae quod laudetur , } nec quod alii vituperentur . \\\hline
1.2.23 & La sexta propreedat es \textbf{ que parte nesçe al magnanimo non ser llorador nin rogador } Et mayormente non deue ser tal en conpara raçonn de los bienes de fuera & Sexto spectat ad ipsum non esse plangitiuum , \textbf{ neque deprecatiuum , | et maxime non debet } esse \\\hline
1.2.24 & Ca non auer cuydado dela honrra \textbf{ por que non quiere fazer obras dignas de honrra esto es de denostar } mas auer cuydado de honrra & Nam non curare de honore , \textbf{ quia non vult agere | opera honore digna , } vituperabile est . \\\hline
1.2.24 & mas auer cuydado de honrra \textbf{ en quanto quiere fazer obras dignas de honrra esto es de loar . } Et por ende nos deuemos auer cuydado de honrra & opera honore digna , \textbf{ vituperabile est . } Nobis igitur debet \\\hline
1.2.24 & nin por que pongamos nr̃a fin e nra bien andança en honrras . \textbf{ mas por que fagos obras dignas de honrra . } Mas las obras dignas de honrra & nec quod finem nostrum ponamus in honoribus , \textbf{ sed quod agamus opera honore digna . } Opera autem honore digna \\\hline
1.2.24 & mas por que fagos obras dignas de honrra . \textbf{ Mas las obras dignas de honrra } se puede entender en dos maneras . & sed quod agamus opera honore digna . \textbf{ Opera autem honore digna } dupliciter considerari possunt . \\\hline
1.2.24 & Ca esto es obra de pradençia \textbf{ nin parte nesçe de fazer cosas non iustas } o que non conuienen ala iustiçia & quod est actus prudentiae , \textbf{ nec facere iniusta , } quod pertinet ad iniustitiam . \\\hline
1.2.24 & e en quanto son ordenadas \textbf{ alas honrras medianeras pertenescen ala uirtud } que es dicha amadora de h̃orra . & et ut ordinantur ad mediocres honores , \textbf{ pertinent ad virtutem , } quae dicitur honoris amatiua . \\\hline
1.2.24 & assi commo la fermosura cortoral se ha \textbf{ ala apostura grande de todo el cuerpo . } Ca los pequa nons omes & Videtur enim honoris amatiua se habere ad magnanimitatem , \textbf{ sicut formositas corporis | se habet ad pulchritudinem . } Parui enim si habent membra bene proportionata , \\\hline
1.2.24 & si han los mienbros bien proporçionados \textbf{ e bien formados son dichos fermosos . } Enpero non son dichos muy apuestos & Parui enim si habent membra bene proportionata , \textbf{ et conformia , formosi sunt : } non tamen pulchri , \\\hline
1.2.24 & Et en essa misma manera \textbf{ los que fazen obras dignas de honrra medianera } son dichos amadores de henrra & nisi in magno corpore . \textbf{ Sic facientes opera mediocri honore digna , } dicuntur honoris amatiui : \\\hline
1.2.24 & son dichos amadores de henrra \textbf{ Mas propreamente quando fazen obras dignas de grant honrra } estonçe son dichos magranimos & dicuntur honoris amatiui : \textbf{ tamen tunc proprie sunt magnanimi , } quando agunt opera magno honore digna . \\\hline
1.2.24 & non fagan alguna cosa rorpe \textbf{ mas por que sienpre fagan obras dignas de honrra . } Bien dicho es & nec in mediocribus faciant aliquid turpe ; \textbf{ sed ut semper agant opera honore digna : } bene dictum est , \\\hline
1.2.25 & dela qual falla el philosofo \textbf{ e llamala a esta uirtud amadora de honrra } por que entiende alas honrras medianeras & et quandam humilitatem animi importat , \textbf{ quam Philosophus appellat honoris amatiuam , } quia tendit in mediocres honores , \\\hline
1.2.25 & humildoso non auriemos dado de suso \textbf{ docłna conuenible alos prinçipes } o les mostramos & quod nisi omnis magnanimus esset humilis , \textbf{ non dedissemus principibus congruum documentum , } quia docuissemus eos esse sine virtutibus , \\\hline
1.2.25 & que la magranimidat ha de seer çerca las grandes honrras . \textbf{ Mas la grand honrra es en alguna manera grant bien . } Et el grand bien & Magnus autem honor est \textbf{ quodammodo magnum bonum . Magnum autem bonum , } ratione qua magnum est , \\\hline
1.2.25 & Et pues que assi es cerca las grandeshonrras \textbf{ e cerca los grandes bienes contesçe de pecar en dos maneras ¶ } Lo primero si mas que la razon e el & Ergo circa magnos honores , \textbf{ et circa magna bona dupliciter contingit peccare . } Primo , si ultra quam ratio dictet \\\hline
1.2.25 & por razon dela bondat \textbf{ dellos somos dichͣ̃s humildosos . } Pues que assi es ay diferençia & ultra quam ratio dictat , \textbf{ ratione bonitatis sumus humiles . } Differt ergo magnanimitas \\\hline
1.2.25 & que ha en reuerençia alos otros \textbf{ por que cuydando en los sus desfallesçimientos propios en las cosas conuenibles e honestas . } faze reuerençia alos otros ¶ & Ideo humilis dicitur alios reuereri , \textbf{ quia considerans proprios defectus , | in rebus licitis et honestis alios reueretur . } Secundo differt haec ab illa , \\\hline
1.2.25 & por razon dela guaueza \textbf{ que non pueda alcançar obras dignas de grand sonrra . } Mas la humildat prinçipalmente tienpra la esꝑanca & ne aliquis ratione difficultatis desperet , \textbf{ ne tendat in opera magno honore digna . } Humilitas vero principaliter moderat ipsam spem , \\\hline
1.2.25 & por que alguno auiendo grand esperança del bien \textbf{ non vaya en pos grandes honrras } mas de quanto la razon manda & ne aliquis nimis sperans de ipso bono , \textbf{ ultra rationem prosequatur magnos honores . } Erit igitur omnis magnanimus \\\hline
1.2.25 & Mas si la humildat es essa misma cosa \textbf{ sinplemente que amar las honrras medianeras . } O si es essa misma cosa & Utrum autem humilitas sit idem simpliciter \textbf{ quod diligere mediocres honores , } vel utrum sit idem simpliciter \\\hline
1.2.25 & mostrariamos que la uirtud de que fabla el philosofo \textbf{ non es en toda manera vna cosa misma con la humildat } por que aquella de que fabla el philosofo & de qua loquitur Philosophus , \textbf{ non esse per omnem modum idem cum humilitate : } quia illa de qua Philosophus loquitur , \\\hline
1.2.25 & que assi commo dicho es \textbf{ es cerca las honrras medianeras . } Mas es en el appetito senssitiuo . & quae ( ut dictum est ) \textbf{ tendit in mediocres honores , magis est in appetitu sensitiuo : humilitas vero , } magis est in intellectiuo . \\\hline
1.2.25 & es cerca las honrras medianeras . \textbf{ Mas es en el appetito senssitiuo . } Et la humildat es mas en el appetito intellectiuo & quae ( ut dictum est ) \textbf{ tendit in mediocres honores , magis est in appetitu sensitiuo : humilitas vero , } magis est in intellectiuo . \\\hline
1.2.26 & Ca commo almagranimo pertenesca de yr \textbf{ e entender en cosas grandes la magranimidat } mas es uirtud & Non tamen aequae principaliter operatur utrunque : \textbf{ nam cum magnanimi sit tendere in magnum , } magnanimitas magis est \\\hline
1.2.26 & por que non podamos rethernos \textbf{ e tirarnos de las cosas altas . } Et despues desto tienpra la esperanca & Principalius ergo magnanimitas reprimit desperationem , \textbf{ ne retrahamur ab arduis : } et ex consequenti moderat spem , \\\hline
1.2.26 & por que non seamos retenidos \textbf{ qua non siguamos las obras dignas de grant honrra } por razon dela guaueza dellas . & ne ratione difficultatis retrahamur , \textbf{ ut non prosequamur opera magno honore digna . } Quia igitur in hoc aliqui superabundant , ut praesumptuosi , \\\hline
1.2.26 & assi commo los presunptuosos \textbf{ que siguen las cosas altas mas que deuen . } Otrosi alguons fallesçen & Quia igitur in hoc aliqui superabundant , ut praesumptuosi , \textbf{ prosequentes ardua magis quam debeant : } quidam vero deficiunt , \\\hline
1.2.26 & assi commo aquellos que non demandan honrras \textbf{ nin demandan obras dgnas de honrra } la qual cosa non es obra de humildat & qui nec quaerunt honores , \textbf{ nec quaerunt opera honore digna : } quod non est humilitatis , \\\hline
1.2.26 & ethicas llama vna gente gniega \textbf{ que dizen latun meses sobuios e alabadores de ssi } que quiere dezir alabadores & Unde Philosophus 4 Ethic’ quandam gentem Graecam , \textbf{ Lacedaemones scilicet , } iactatores et superbos appellat : \\\hline
1.2.26 & et alos prinçipes \textbf{ assi de madar las obras dignas de honrra } que non sean mas que la razon & Debent enim Reges \textbf{ sic quaerere opera honore digna , } non ultra quam ratio dictet , \\\hline
1.2.26 & en sobrepuiança de honrra lo que fazen los sobuios \textbf{ por que deuen fazer los Reyes bueans obras e dignas de honrra } non por alabança & quod faciunt superbi . \textbf{ Debent enim agere bona opera } et honore digna boni gratia , \\\hline
1.2.27 & Et mostramos en qual mana conuiene alos Reyes e alos \textbf{ prinçipesser honrrados destas uirtudes . } fincanos agora de dezir dela & de virtutibus respicientibus exteriora bona , \textbf{ et ostendimus quomodo Reges et Principes decet ornari virtutibus illis , } restat dicere de mansuetudine , \\\hline
1.2.27 & entre los miedos e las osadias \textbf{ assi en essa misma manera es la manssedunbre medianera entre la sanna } por la qual & quia sicut fortitudo est media inter timores et audacias , \textbf{ sic mansuetudo est media inter iram , } per quam cupimus vindictam , \\\hline
1.2.27 & e prinçipalmente entiende repremir las sañas \textbf{ mas despues desto entiende tenprar las passiones contrarias dela sana } que es nunca se enssanar & Mansuetudo enim principaliter \textbf{ et primo intendit reprimere iras , } ex consequenti autem intendit moderare passiones oppositas irae . \\\hline
1.2.27 & propreo non puede ser \textbf{ assi pequano que non paresca anos grande . } Et non solamente naturalmente somos inclinados & Rursus , quia malum proprium vix potest \textbf{ esse ita modicum , | quin videatur nobis multum , } non solum naturaliter inclinamur , \\\hline
1.2.27 & a aquellos que nos fazen alguons males . \textbf{ Mas avn en alguna manera natural cosa esa nos de dessear de ser vengados dellos mas que deuemos . } Ca por que el mal que ellos nos fazen paresçe a nos & inferentes nobis aliqua mala , \textbf{ sed etiam quodammodo naturale est nobis | appetere punitionem ultra condignum . } Nam quia malum nobis illatum videtur nobis maius esse , \\\hline
1.2.27 & e alos prinçipes de ser manssos \textbf{ esto non es cosa guaue mas ligera . } Ca por que la yr a & Quod autem deceat Reges et Principes esse mansuetos , \textbf{ ostendere non est difficile . } Nam cum ira peruertat iudicium rationis , \\\hline
1.2.27 & e forma de beuir alos otros \textbf{ et que deue seer regla de todas las cosas fazederas . } non es cosa conuenible al Rey de ser sañudo . & quasi speculum et forma viuendi , \textbf{ et qui debet esse regula agendorum , } inconueniens est \\\hline
1.2.27 & et que deue seer regla de todas las cosas fazederas . \textbf{ non es cosa conuenible al Rey de ser sañudo . } por que por la saña non sea tristornado nin torçido . & et qui debet esse regula agendorum , \textbf{ inconueniens est | quod sit iracundus , } ne per iram peruertatur et obliquatur . \\\hline
1.2.27 & por que por la saña non sea tristornado nin torçido . \textbf{ Et avn en essa misma manera non es cosa conuenible al Rey de nunca se enssanar } e de nunca se mouer a dar pena . & ne per iram peruertatur et obliquatur . \textbf{ Sic etiam , | si nullo modo esset irascibilis , } et nullo modo commoueretur \\\hline
1.2.28 & algunos sobrepuian por que se muestran mucho amigables \textbf{ quales son los que fablan palauras blandas e plazenteras } Ca estos en tanto se muestran conpanneros & quia se ostendunt nimis amicabiles , \textbf{ cuiusmodi sunt blanditores et placidi . } Hi enim adeo se ostendunt communicabiles et sociales , \\\hline
1.2.28 & Et pues que assi es commo la uirtud sea medianera \textbf{ entre la cosa sobrepuiante e la . } mengunate en la conuerssa conn de los omes & Cum igitur uirtus sit \textbf{ quid medium inter superfluum et diminutum , } in conuersatione hominum , \\\hline
1.2.28 & por bien fablar \textbf{ Empero non deuen todos en vna manera seramigables e bien fablantes . } por que la grant familiaridat pare e faze despreçiamiento . & debeant esse amicabiles et affabiles , \textbf{ non tamen omnes eodem modo amicabiles debent esse . } Nam quia nimia familiaritas contemptum parit , \\\hline
1.2.28 & por que sean auidos en Reuerençia \textbf{ e la dignidat Real non sea abiltada nin menospreçiada } mas cuerdamente se deuen auer & ut in reuerentia habeantur , \textbf{ et ne dignitas regia vilescat , } maturius se habere debent , \\\hline
1.2.28 & e es dicho por ende amigable \textbf{ que si alguna otra perssona comun non mostrasse mayor familiaridat } de quanta muestra el Rey seer le ye contado a menos preçio & et dicitur ex hoc amicabilis esse : \textbf{ quia si aliqua una communis persona | non plus de familiaritate participaret , } reputaretur ei ad vitium , non ad virtutem : \\\hline
1.2.29 & algunos se desuian por sobrepuiança mostrando de ssi mismos \textbf{ por palauras o por fechos mayores cosas que sean en ellos } e estos tal sson llamados alabadores dessi mismos . & ostendentes de se verbis \textbf{ aut factis maiora quam sint , } et tales vocantur iactatores . \\\hline
1.2.29 & por falles çemiento deziendo \textbf{ e segerendo de ssi mismos algunas cosas villes } que en ellos non sono negando de ssi mismos & Aliqui vero ab hac veritate declinant per defectum , \textbf{ fingentes de se aliqua vilia } quae in ipsis non sunt , \\\hline
1.2.29 & que sean es obra de sabio . \textbf{ Pues que assi es parte nesce al uerdadero } non dezir dessi mayores cosas & est opus prudentis . \textbf{ Spectat igitur ad veracem nullo modo dicere de se maiora , } quam sint , \\\hline
1.2.29 & la qual cosa fazen muchos . \textbf{ Ca otorgan alguons de ssi mismos grandes bondades } commo quier que en ellos nen sean & quod multi faciunt . \textbf{ Concedunt enim de se aliqui magnas bonitates , } cum tamen illis careant : \\\hline
1.2.29 & e de repremir los alabamientos . \textbf{ Enpero mas prinçipalmente parte nesçe ala uerdat } de repremir los alabamientos que tenprar los escarnesçimientos por que assi commo dicho es & et reprimere iactantias . \textbf{ Principalius tamen spectat | ad ipsum iactantias reprimere , } quam derisiones moderare : \\\hline
1.2.29 & en tal manera que sienpre creade ssi mismo mas de aquello que es . \textbf{ Et por ende comunalmente los omes son engannados de ssi mismos . } cuydando que valen mas de quantovalen . & esse quam sit . \textbf{ Ideo communiter homines decipiuntur de se ipsis , } plus credentes se plus valere , \\\hline
1.2.29 & por que los omes non sean alos otros pesados e guaues . \textbf{ Et esta razon misma tanne el philosofo en esse mismo quarto libro delas ethicas } o dize que deuemos & ne homines sint aliis onerosi . \textbf{ Hanc autem rationem tangit Philosophus in eodem 4 Ethicorum dicens , } declinandum esse in minus \\\hline
1.2.29 & e mas viles de quanto son . \textbf{ paresçe que son escarnidores e despreçiadores de ssi mismos . } mas aquellos que sobrepuian en lo mas & quam sint , \textbf{ videntur esse derisores , et contemptibiles . } Excedentes vero in plus , \\\hline
1.2.30 & Enpero non es assi ca cosa baldia et vana es aquella \textbf{ que non alcanca su fin conuenible . } Et pues que assi es qual si quier cosa & Nam ociosum est illud , \textbf{ quod caret debito fine ; } quicquid ergo de se est ordinabile \\\hline
1.2.30 & Et pues que assi es qual si quier cosa \textbf{ que dessi es ordenada en fin conuenible non esbaldia nin vana } Et por ende el trebeio & quicquid ergo de se est ordinabile \textbf{ ad debitum finem , | non est ociosum . } Ludus autem si sit liberalis , \\\hline
1.2.30 & e tenprado hase de ordenara buena fin . \textbf{ por que es en alguna manera necessario ala uida del omne . } Et por ende dize el philosofo & ordinari habet in bonum finem : \textbf{ quia est quodammodo necessarius in vita . } Ideo dicitur quarto Ethic’ \\\hline
1.2.30 & que paresce que la folgura \textbf{ e el trebeio es vna cosa necessaria en la uida de los omes . } Et pues que assi es & quod videtur requies \textbf{ et ludus esse aliquid necessarium in vita . } Sicut ergo sensus corporales , \\\hline
1.2.30 & Et pues que assi es \textbf{ assi commo los sesos corporales . } assi commo el veer & et ludus esse aliquid necessarium in vita . \textbf{ Sicut ergo sensus corporales , } ut visus , et auditus , \\\hline
1.2.30 & assi commo el veer \textbf{ e el oyr t̃baian en sintiendo las cosas senssibles . } Et por ende la natura ordeno el su enno & ut visus , et auditus , \textbf{ quia laborant in sentiendo , } natura ordinauit somnum propter eorum requiem , \\\hline
1.2.30 & para lu tolgura dellos . \textbf{ Et assi el suenno es cosa necessaria en la uida . } En essa misma manera & natura ordinauit somnum propter eorum requiem , \textbf{ et est necessarius somnus in vita . } Sic quia studendo , \\\hline
1.2.30 & conuiene erca tales iuegos \textbf{ e cerca tales delecta connes deuiegos dar alguna uirtud . } por la qual conueniblemente nos ayamos alos iuegos e alos trabaios . & circa ipsos iocos \textbf{ dare virtutem aliquam , } per quam debite nos habeamus ad ludos . \\\hline
1.2.30 & fazerriso \textbf{ que de dezir cosas fermosas . } Et estos tales son los iuiglares & quod magis conantur risum facere , \textbf{ quam decora dicere . } Huiusmodi autem sunt histriones , \\\hline
1.2.30 & mas esto conuiene alos Reyes e alos prinçipes en tanto vsar tenpradamente delas delecta connes delos iuegos \textbf{ que si esto feziesen algunas ottas personas comunes paresçeria } que serian montesinos e siluestres . & uti iocosis delectationibus , \textbf{ quod si hoc facerent personae communes , } viderentur esse durae et agrestes . \\\hline
1.2.31 & Mas por que pueda soluer estas razones \textbf{ e estas abusiones sobredichas dize } que las uirtudes pueden se tomar en dos maneras . & Sed ut soluat huiusmodi obiectiones , \textbf{ ait , quod virtutes dupliciter considerari possunt : } vel ut sunt naturales , et imperfectae : \\\hline
1.2.31 & Enpero estos tales non son castos nin son liberales \textbf{ nin han las otras uirtudes morales . } En essa misma manera avn ueemos algunos & nec liberales , \textbf{ nec habent virtutet morales alias . } Sic etiam ex ipsa pueritia \\\hline
1.2.31 & si mucho es llena de flema \textbf{ dulçe paresçen le todas las cosas dulçes . } En essa misma manera quales somos segunt nr̃a uoluntad & si vero sit multum infecta phlegmate dulci , \textbf{ videtur participare quandam dulcedinem . } Sic quales sumus \\\hline
1.2.31 & propone \textbf{ assi en logar de fin las delectaçiones carnales . } Mas el tenprado propone & ut intemperatus proponit sibi , \textbf{ ut finem venerea ; } temperatus , casta ; \\\hline
1.2.31 & Mas el tenprado propone \textbf{ assi por fin las cosas castas . } Et el liberal e el franco propone & ut finem venerea ; \textbf{ temperatus , casta ; } liberalis , expendere ; \\\hline
1.2.31 & por que aquellos que han el appetito bueno \textbf{ por estas uirtudes morales proponen assi buean fin } Mas la pradençia e la sabiduria & quia habentes appetitum bonum \textbf{ per virtutes illas , | proponimus nobis bonum finem . } Prudentia vero , per se , \\\hline
1.2.31 & e ordenando nos a buena fin \textbf{ por las uirtudes morales . } Enpero espeçialmente por la pradençia & Nam proponentes nobis bonum finem \textbf{ per virtutes morales , } per prudentiam bene ratiocinamur \\\hline
1.2.31 & en el sexto libro delas ethicas \textbf{ que las uirtudes morales rectifican } e enderesçan al omne ala fin . & Ideo dicitur 6 Ethic’ \textbf{ quod virtutes morales rectificant finem , } Prudentia vero facit operari recte \\\hline
1.2.31 & Et pues que assi es fablando delas uirtudes dezimos \textbf{ que prinçipalmente et primeramente la uirtud moral rectifica } e enderesça el termino & ea quae sunt ad finem . \textbf{ Loquendo ergo principaliter et primo , } virtus moralis rectificat terminum : \\\hline
1.2.31 & si non fuere prudente e sabio \textbf{ Ca commo la uirtud moral sea habito e disposiçion firme de alma e buena escogedora } e acaba a aquel que la ha & nisi sit prudens . \textbf{ Nam cum virtus moralis sit | habitus bonus , } et electiuus , \\\hline
1.2.31 & e acaba a aquel que la ha \textbf{ e faga la su obra buena . } Por ende commo havien escoger & et perficiat habentem , \textbf{ et opus suum bonum reddat : } cum ad bene eligere , \\\hline
1.2.31 & En essa misma manera avn la pradençia non puede ser \textbf{ sin uirtud moral . } mas deuedes sabra e notar & non potest \textbf{ sine virtute morali . } Differt enim prudentia , et industria , \\\hline
1.2.31 & e la industria moral \textbf{ que es aꝑcebimiento natural . } la qual industria el philosofo llama de motica & Differt enim prudentia , et industria , \textbf{ quam Philosophus appellat Denoteta . } Ille enim dicitur Denos , et industris , \\\hline
1.2.31 & s carreras a buenan fin \textbf{ sin las uirtudes morales . } por las quals proponemos & ad bonum finem \textbf{ sine virtutibus moralibus , } per quas nobis proponimus bonum finem . \\\hline
1.2.31 & mas alguno diria o podria dezir \textbf{ que la uirtud moral non podria ser } sin la pradençia et sabiduria & per quas nobis proponimus bonum finem . \textbf{ Sed dicet aliquis virtutem moralem } non posse esse sine prudentia , \\\hline
1.2.31 & que en ninguna manera fuera de \textbf{ razon non obra cosas delectables segunt la carne . } Por la qual cosa si alguno fuere & Nam ille est temperatus perfecte , \textbf{ qui nullo modo praeter rationem operaretur venerea . } Quare si ille sit timidus , \\\hline
1.2.32 & o por pequeña tentaçion caen . \textbf{ Ca el uocabulo mesmo del muelle lo muestra assi . } Ca segunt el philosofo & vel parua tentatione ruunt , \textbf{ quod ipsum nomen designat . } Nam secundum Philosophum in Meteora , \\\hline
1.2.32 & Ca segunt el philosofo \textbf{ en el libro dela generaçion muelle es dicho aquel } que da logar a otro & quod ipsum nomen designat . \textbf{ Nam secundum Philosophum in Meteora , } molle est illud , \\\hline
1.2.32 & e es uençido por las tentaçiones \textbf{ mas non le es cosa delectable de mal fazer quando cae . } Et pues que assi es los non continentes & et per tentationes deiicitur , \textbf{ sed delectabile est ei malefacere . } Incontinentes ergo \\\hline
1.2.32 & e firmados en el mal \textbf{ que les es cosa delectable de mal fazer¶ } Mas en el quarto guado de los malos son los omes bestiales . & quia sunt adeo habituati in malo , \textbf{ quod delectabile est eis malafacere . } In quarto gradu autem sunt bestiales . \\\hline
1.2.32 & que les es cosa delectable de mal fazer¶ \textbf{ Mas en el quarto guado de los malos son los omes bestiales . } Et estos peores son que los destenprados & quod delectabile est eis malafacere . \textbf{ In quarto gradu autem sunt bestiales . } Hi autem peiores sunt , quam intemperati . \\\hline
1.2.32 & que fazen mal fuera de razon \textbf{ e dela manera comunal de los omes } ¶ dicho de los quatroguados & Illi ergo sunt bestiales , \textbf{ qui ultra modum hominum male agunt . } Dicto de quatuor generibus malorum , \\\hline
1.2.32 & ¶ dicho de los quatroguados \textbf{ delons malos finca de dezir delos quatro linages de los buenos } Ca assi conmo algunos muelles son malos & Dicto de quatuor generibus malorum , \textbf{ restat dicere de quatuor generibus bonorum . } Nam sicut quidam mali sunt molles , \\\hline
1.2.32 & que non sienten aquella batalla \textbf{ nin aquella tentaçion mas es los a ellos cosa delectable de bien fazer } ¶ Et pues que assi es & quod quasi pugnam non sentiunt , \textbf{ et delectabile est eis benefacere . } Sicut ergo perseuerantes opponuntur mollibus , \\\hline
1.2.32 & ¶ En el quartoguado \textbf{ e mas alto de los bueons son los omes diuinales . } Ca assi commo algunos omes son bestiales & In quarto et in supremo gradu bonorum , \textbf{ sunt homines diuini . } Nam sicut aliqui homines sunt bestiales , \\\hline
1.2.32 & Mas aquella uirtud por la qual alguno es dich̃o bueno \textbf{ sobre la manera comunal de los omes } es llamada del philosofo eroyca & per quam quis debet \textbf{ esse bonus ultra modum humanum , } appellatur a Philosopho heroica \\\hline
1.2.32 & la qual quando la ouieren seran buenos \textbf{ sobre la manera comunal de los otros } e seran assi commo omes diuinales de dios . & et principans respectu aliarum , \textbf{ et sint boni ultra modum aliorum , } et sint quasi homines diuini . \\\hline
1.2.32 & sobre la manera comunal de los otros \textbf{ e seran assi commo omes diuinales de dios . } ¶ Et pues que assi es en este grado conuiene alos Reyes & et sint boni ultra modum aliorum , \textbf{ et sint quasi homines diuini . } In hoc ergo gradu debent esse Reges et Principes . \\\hline
1.2.32 & Et si en tal grado de buenos \textbf{ conuiene alos prinçipes seglares de ser buenos } e acabados & In hoc ergo gradu debent esse Reges et Principes . \textbf{ Et si in tale gradu esse decet bonos et perfectos Principes seculares } quales esse debeant \\\hline
1.2.32 & e acabados \textbf{ quanto mas esto pertenesca alos prinçipes eccłiasticos . } Esto sea dexado en el iuyzio del sabio & quales esse debeant \textbf{ ecclesiastice principantes , } prudentis iudicio relinquatur . \\\hline
1.2.33 & que enuia dios en el alma del omne \textbf{ por las quales se ha cada vno bien alas cosas diuinales . } Mas cerca las cosas diuinales & esse virtutes infusas , \textbf{ per quis quis bene se habet ad diuina . } Circa diuina autem aiunt est duplex gradus . \\\hline
1.2.33 & e tales son dichos auer uirtudes de pgado coraçon . \textbf{ Mas commo quier que estos digan cosas uerdaderas dessi } enpero non paresçe que se allegan ala entençion & et tales habere dicuntur virtutes purgati animi . \textbf{ Sed hi licet | secundum se vera dicant , } non tamen videntur accedere ad intentionem eorum , \\\hline
1.2.33 & si non delas uirtudes ganadas por vso de buean sobras \textbf{ Et estos philosofos non pone uirtudes infusas del çielo . } Et por ende toda uirtud & nisi de virtutibus acquisitis . \textbf{ Virtutes autem infusas non posuerunt . } Omnem ergo virtutem , \\\hline
1.2.33 & Mas los tenpdos han uirtud de coraçon purgado . \textbf{ Et los diuinales han uirtudes exenplares . } Por la qual cosa & temperati vero habent virtutes purgati animi : \textbf{ sed diuini habent virtutes exemplares . } Propter quod sicut diuini \\\hline
1.2.33 & e los tenprados meiores que los continentes . \textbf{ Et los continentes meiores que los persseuerantes . } En essa misma manera las uirtudes exenplares son mas altas que las uirtudes del coraçon pragado . & temperati continentibus , \textbf{ continentes perseuerantibus : } sic virtutes exemplares excellunt virtutes purgati animi : \\\hline
1.2.33 & Et los continentes meiores que los persseuerantes . \textbf{ En essa misma manera las uirtudes exenplares son mas altas que las uirtudes del coraçon pragado . } Et las uirtudes del coraçon pragado & continentes perseuerantibus : \textbf{ sic virtutes exemplares excellunt virtutes purgati animi : } virtutes vero purgati animi \\\hline
1.2.33 & que son exenplares \textbf{ serie muy mala cosa deñobrar ninguna cosa torpe en ellas } por la qual cosa bien dicho es & scilicet exemplaribus , \textbf{ nefas est turpe aliquod nominari . } Quare bene dictum est , \\\hline
1.2.33 & por que se tiene \textbf{ e se traye delas delecta connes sensibles . } ca los continentes son omes & quia tenet se , \textbf{ et abstinet se | a delectationibus sensibilibus . } Continentes enim , \\\hline
1.2.33 & por que tales en tanto deuen ser acabados \textbf{ que sean regla derecha e exenplario de los otros ¶ } Et & tales enim perfecti adeo debent esse , \textbf{ ut sint aliorum data regula , et exemplar . } Declaratum est ergo primum , \\\hline
1.2.33 & segunt las quales uirtudes \textbf{ assi commo dize plotino aquel philosofo grand pecado es de nonbrar cosas torpes . } Et commo ninguno non pueda ser & quibus secundum Plotinum \textbf{ nefas est turpia nominari : } sed cum tantae bonitatis nullus esse possit \\\hline
1.2.33 & que dixieron \textbf{ que por prinçipios puros naturales } podriemos escusar todos los males & violentium \textbf{ quod ex puris naturalibus possemus omnia mala vitare , } et perfectam bonitatem acquirere . \\\hline
1.2.34 & por razon dela lid \textbf{ que siente non es a el cosa delectable de bien fazer Et } pues que assi es en quanto alguno es continente & ratione pugnae quam sentit , \textbf{ non est ei delectabile benefacere . } Quandiu ergo aliquis est continens , \\\hline
1.2.34 & Mas declarar en qual manera la perseuerança es disposicion ala uirtud . \textbf{ Et en qual manera es condicion segniente ala uirtud } non es deste presente negoçio & est dispositio ad virtutem , \textbf{ et quomodo est conditio sequens virtutem , } non est praesentis speculationis . \\\hline
1.2.34 & non es deste presente negoçio \textbf{ nin parte nesçe anos de tractar dello } mas quanto pertenesçe a lo presente & et quomodo est conditio sequens virtutem , \textbf{ non est praesentis speculationis . } Sufficit autem ad praesens scire , \\\hline
1.3.1 & en el quarto libro delas ethicas . \textbf{ Ca la manssedunbre nonbra propriamente passion contraria ala saña . } Mas por que ha de ser alguna uirtud entre la sanna e la mansedunbre & ut ait Philosophus 4 Ethic’ . \textbf{ Mansuetudo enim proprie nominare videtur | passionem oppositam irae . } Sed cum sit quaedam virtus \\\hline
1.3.1 & e quoco ala uirtud \textbf{ e ala passion contraria dela sana . } Mas si alguno quisi esse trabaiar de enponer o fallar & Erit mansuetudo aequiuocum ad virtutem , \textbf{ et ad passionem oppositam irae . } Si quis autem laborare vellet , \\\hline
1.3.1 & Ca las passiones propiamente non han de ser \textbf{ si non en el appetito senssitiuo . } Ca el appetito intellectiuo que es la uoluntad & quia passiones proprie esse non habent \textbf{ nisi in appetitu sensitiuo . } Appetitus enim intellectiuus \\\hline
1.3.1 & si non en el appetito senssitiuo . \textbf{ Ca el appetito intellectiuo que es la uoluntad } por que non es uirtud en carne & nisi in appetitu sensitiuo . \textbf{ Appetitus enim intellectiuus } quia non est virtus in corpore , \\\hline
1.3.1 & assi commo fablamos aqui de passion \textbf{ en el appetiuo senssitiuo del seso . } Mas el appetito senssitiuo del seso assi commo mas largamente dixiemos de suso & ( ut hic de passione loquimur ) \textbf{ in appetitu sensitiuo . } Sensitiuus autem appetitus \\\hline
1.3.1 & en el appetiuo senssitiuo del seso . \textbf{ Mas el appetito senssitiuo del seso assi commo mas largamente dixiemos de suso } partese en apetito iraçibile & in appetitu sensitiuo . \textbf{ Sensitiuus autem appetitus } ( ut supra diffusius diximus ) diuiditur in irascibilem , et concupiscibilem . \\\hline
1.3.1 & Et la delectaçion ¶ \textbf{ Et la tristeza pertenesçen al appetito desseador . } Mas las otras seys passiones pertenesçen al appetito enssannador & reliquae vero sex \textbf{ ad irascibilem spectant . } Numerus autem passionum concupiscibilis sic accipi potest : \\\hline
1.3.1 & e todo mouimiento del alma \textbf{ que pertenesçe al appetito desseador o se torna en conparaçion de algun bien } o en conparaçion de algun mal & et omnis motus animae pertinens ad concupiscibilem , \textbf{ vel sumitur respectu boni , } vel respectu mali . \\\hline
1.3.2 & que la sanna e la manssedunbre que son tomadas \textbf{ por razon de mal presente ¶ } Et pues que assi es llanamente & et mansuetudinem , \textbf{ quae sumuntur respectu mali praesentis . } Plane ergo patet , \\\hline
1.3.2 & ¶ La qual cosa sobre tanto \textbf{ mas parte nesçe alos Reyes } e alos prinçipes & et quomodo vitandae passiones praedictae . \textbf{ Quod scire tanto magis decet Reges et Principes , } quanto per passiones ipsorum maius valet induci malum , \\\hline
1.3.3 & e en lanr̃a uida \textbf{ por ende escoła neçesaria de mostrar } en qual manera nos deuemos auer a aquellas passiones & quia diuersificant regnum et vitam nostram , \textbf{ ideo necessarium est ostendere } quomodo nos habere debeamus ad illas . \\\hline
1.3.3 & e mas grande el amor . \textbf{ Mas en los bienes diuinales de dios } e en los bienes comunes & et magis intensus amor . \textbf{ In bonis autem diuinis , } et in bonis communibus , \\\hline
1.3.3 & mas es fallada razon de bondat \textbf{ que en los bienes personales e propios . } Et pues que assi es la manera & magis reperitur ratio bonitatis , \textbf{ quam in bono priuato . } Modus ergo , \\\hline
1.3.3 & assi mismo fazer bueno o guardar \textbf{ assymismo en bondat . La razon natural muestra } que mas deue amar el omne el bien diuinal que assi mismo . & vel se in bonitate conseruare , \textbf{ dictat naturalis ratio } ut magit diligat Deum quam seipsum : \\\hline
1.3.3 & por que non sean los mienbros llagados \textbf{ enlos quales esta la salud comun de todo el cuerpo prinçipal mente . } Et esto por inclinaçion natural pone el braço a periglo & ne vulnerentur membra \textbf{ a quibus principaliter dependet salus corporis , } et ne totum corpus pereat , \\\hline
1.3.3 & Et por ende el sera sabio . \textbf{ Et avn en ella milma maneral era iusto } por que el bien comun es guardado mayormente por la iustiçia . & Est ergo prudens , \textbf{ si etiam erit iustus : } quia bonum commune potissime \\\hline
1.3.3 & e sera magnanimo \textbf{ por que los bienes comunes son muy mas altos } e mas dignos de grand honrra & Erit magnanimus ; \textbf{ quia bona communia maxime sunt ardua } et magno honore digna , \\\hline
1.3.3 & la magnificençia \textbf{ prinçipalmente ha de ser çerca los bienes diuinales e comunes . } Otrosi sera fuerte por que ante pone el bien comunal bien propreo & quia secundum Philosophum 4 Ethic’ magnificentia potissime habet \textbf{ esse circa diuina , et communia . } Erit fortis ; quia cum bonum cumune proponat bono priuato , \\\hline
1.3.3 & a las delectaçiones destenpradas de los sesos \textbf{ por que por ellas non se pueda enbargar la cura conuenible del regno . } Et pues que assi es por que lo podamos todo traer en vna sentençia & spernet delectationes sensibiles immoderatas , \textbf{ ne per eas impediatur debita cura regni . } Ut ergo sit ad unum dicere , \\\hline
1.3.3 & que son contrarios ala uerdat \textbf{ e por que ama la uida corporal teme el cuchiello } que tuelle la uida ¶ Et & qui contrariantur veritati : \textbf{ et qui diligit corporalem vitam , | timet gladium } qui eam tollit . \\\hline
1.3.3 & que los Reyes et los prinçipes \textbf{ por alguna manera especial sobre todos los otros deuen amar el bien diuinal } e el bien comunal en alguna manera espeçial sobre todos los otros deuen aborresçer todas aquellas cosas & Ostenso ergo quomodo Reges et Principes \textbf{ quodam speciali modo prae aliis debent | diligere bonum diuinum et commune , } et quodam speciali modo \\\hline
1.3.3 & por alguna manera especial sobre todos los otros deuen amar el bien diuinal \textbf{ e el bien comunal en alguna manera espeçial sobre todos los otros deuen aborresçer todas aquellas cosas } que son contrarias al bien diuinal e comunal . & diligere bonum diuinum et commune , \textbf{ et quodam speciali modo | prae alios odire debent } quae contrariantur bono diuino et communi : \\\hline
1.3.3 & que son contrarias al bien diuinal e comunal . \textbf{ Et estas tales cosas son obras desiguales e tortizas e obras de denuesto . } Et generalmente todos males e todos pecados & quae contrariantur bono diuino et communi : \textbf{ huiusmodi autem sunt opera iniusta et contumeliosa , } et uniuersaliter omnia vitia . \\\hline
1.3.4 & son semeiables \textbf{ en alguna manera alas cosas naturales . } Ca assi commo los cuerpos naturales & Nam gesta moralia \textbf{ quodammodo rebus naturalibus sunt similia . } Nam sicut corpora naturalia \\\hline
1.3.4 & al bien qual es a el conueniente e proportionado . \textbf{ Mas en los cuerpos pesados e liuianos deuemos pensar tres cosas ¶ } Lo primero la forma del cuerpo pesado o liuiano & quis in bonum sibi proportionatum et conueniens . \textbf{ In grauibus ergo et leuibus est tria considerare . } Primo formam grauis vel leuis , \\\hline
1.3.4 & assi commo en el arte dela fisica \textbf{ prinçipalmente es entendida la salud del cuerpo natural . } Conuiene alos Reyes e alos prinçipes entender e amar & sicut in arte medicandi principaliter \textbf{ intenditur sanitas corporis : } naturaliter decet Reges et Principes \\\hline
1.3.4 & assi que todos quantos son en el regno \textbf{ se ayan bien alas cosas diuinales e de dios } e que fagan obras uirtuosas & ut quod qui in regno sunt , \textbf{ bene se habeant ad diuina , } quod agant opera virtuosa , \\\hline
1.3.4 & e dar las penas \textbf{ por las cosas desiguales e malas . } Et fazer o tris cosas tales delas quales nasçe e cuelga la salud del regno & cohercere malos , \textbf{ punire iniusta , } et facere talia , \\\hline
1.3.5 & ¶lo terçero la esperança ha de ser cerca el bien futuro que ha de uenir . \textbf{ Por que de los bienes presentes non es esperanca } commo quir que pueda ser dellos gozo e delectaçion & circa bonum futurum : \textbf{ de praesentibus enim bonis non est spes , } licet possit \\\hline
1.3.5 & e alos prinçipes de poner las leyes . \textbf{ Et parte nesçe a ellos de es par algun bien . } Otrosi por que la prinçipal entençion del fazedor delas leyes & ad Reges et Principes leges ponere , \textbf{ spectat ad eos sperare bonum . } Rursus quia principale intentum \\\hline
1.3.5 & e mas altos que los otros \textbf{ Por ende non solamente parte nesçe alos Reyes } e alos prinçipes de entender en el bien & cum talia sint bona excellentia et ardua , \textbf{ non solum spectat ad Reges } et Principes tendere in bonum , \\\hline
1.3.5 & e grandes meresçen perdon \textbf{ por que el poderio ciuil e las riquezas } e la nobleza & videntur mereri indulgentiam , \textbf{ quia ciuilis potentia , diuitiae , } et nobilitas non adminiculantur eis , \\\hline
1.3.5 & sirue la nobleza de linage \textbf{ e el poderio çiuil e abondança de riquezas } non se pueden escusar & quibus consequitur nobilitas generis , \textbf{ potentia ciuilis , | abundantia diuitiarum , } inexcusabiles esse videntur , \\\hline
1.3.5 & e los prinçipes \textbf{ de una entender alos bienes altos e grandes } e de una proueer los biens & Quare cum Reges et Principes \textbf{ tendere debeant in bona ardua , } et debeant prouidere bona futura possibilia ipsi regno : \\\hline
1.3.5 & deuen ser apareiados \textbf{ para acometer las cosas altas e guaues e esparar las cosas } que son de esparar . & Nam sicut per magnanimitatem debent esse prompti , \textbf{ ut aggrediantur ardua , } et sperent speranda : \\\hline
1.3.5 & por que los que non son prouados \textbf{ enlos fechͣs non pueden saber las cosas altas e grandes abiertamente } por que non sopieron la guaueza & quod contingit ex ignorantia : \textbf{ inexperti enim non possunt } cognoscere arduitatem operis . \\\hline
1.3.5 & que non pueden acabar nin alcançar ¶ \textbf{ En elsa misma manera avn podemos dezir } que los beddos mas esperan & quae consumate non possunt . \textbf{ Sic etiam dicere possemus , } quod ebriosi plus sperant quam debent : \\\hline
1.3.5 & e mayor que la su fuerça demanda¶ \textbf{ Et por ende si cosa desconuenible es poner toda la gente } e todo el regno a peligros deuen los Reyes & qui aggreditur aliquid ultra vires . \textbf{ Si ergo inconueniens est totam gentem } et totum regnum periculis exponere , \\\hline
1.3.5 & e los prinçipes con grand diligençia \textbf{ e con conseio grande e prolongado cuydar qual cosa de una acometer } por que non acometan cosa mas alta & et totum regnum periculis exponere , \textbf{ diuturno consilio et magna diligentia excogitare debent Reges et Principes } quid aggrediantur , \\\hline
1.3.6 & en el terçero libro delas ethicas çerca las costunbres ¶ \textbf{ Los penssamientos singulares de los omes son mas prouechosos que los generales . } Commo quier que las & in 4 Ethicorum circa mores , \textbf{ singulares considerationes magis proficiunt . } Verum quia sunt nobis nota confusa magis , \\\hline
1.3.6 & e espeçialmente en el terçero \textbf{ descendremos mas alas cercunstançias particulares de cada vno . } Empero conuiene de tractar primero estas cosas uniuersales et generales & quia in secundo , \textbf{ et maxime in tertio plus descendemus ad particulares circumstantias . } Expedit tamen haec uniuersalia pertransire , \\\hline
1.3.6 & descendremos mas alas cercunstançias particulares de cada vno . \textbf{ Empero conuiene de tractar primero estas cosas uniuersales et generales } por que el conosçimiento dellas faze mucho al conosçimiento delas cosas que se siguen . & et maxime in tertio plus descendemus ad particulares circumstantias . \textbf{ Expedit tamen haec uniuersalia pertransire , } quia horum cognitio faciet \\\hline
1.3.6 & o en este libro \textbf{ que tracta delas constunbres nos dieremos a ellos alguas reglas generales con la praeua } e con la esperiençia & in morali negocio eis tradantur , \textbf{ suffragante experientia } quam habent de moralibus gestis , \\\hline
1.3.6 & que ha de venir . \textbf{ En essa misma manera seg̃t essa misma sciençia los podemos ensseñar } en qual manera se de una auer çerca la osadia & quae respiciunt futurum bonum ; \textbf{ sic secundum eandem methodum | eos instruere possumus , } quomodo se habere debeant \\\hline
1.3.6 & Mas el \textbf{ temortenprado non solamente faze alos Reyes tomadores de conseio mas faze avn que fagan las obras mas acuçiosa mente . } Ca si nos viniere algun temor tenprado mas & non solum consiliatiuos facit , \textbf{ sed etiam agit | ut opera diligentius operemur . } Nam si moderatus adsit timor , \\\hline
1.3.6 & que non puede obrar \textbf{ por que quando alguno teme la calentura natural tornasse alos mienbros de dentro . } Ca segunt la manera que nos veemos en todas las cosas podemos & Quarto facit eum inoperatiuum . \textbf{ Cum enim quis timet , | calor ad interiora progreditur ; } modum enim , \\\hline
1.3.6 & Ca segunt la manera que nos veemos en todas las cosas podemos \textbf{ lo ueer en la calentura del cuerpo natural . } Ca quando algunos omes que estan en los canpos & quem videmus in hominibus , \textbf{ aspicere possumus | in calore corporis naturalis . } Cum enim homines existentes \\\hline
1.3.6 & e que el Rey sea sin conseio . \textbf{ Cosa desconuenible es mucho al regno de temer } por temor & et ut Rex sit inconsiliatiuus ; \textbf{ indecens est ipsum timere immoderato timore . } Tertio immoderatus timor reddit hominem tremulentum . \\\hline
1.3.6 & e viene les luego el tremer ¶ \textbf{ Et pues que assi es si cosa desconuenible es al Rey de ser tremuliento } la qual cosa deue seruaron costante e firme desconuenible cosa es ael de temer & quare accidit ei tremor . \textbf{ Ergo si inconueniens est | Regem esse tremulentum , } qui debet esse virilis et constans , \\\hline
1.3.7 & Et desta diferençia \textbf{ prinçipalmente la san na e la mal querençia le toman ocho disterençias } entre ellas las quales pone el philosofo & Ex hac autem differentia principali \textbf{ inter iram et odium , | sumuntur octo differentiae , } quas assignat Philos’ 2’ Rhetor’ . \\\hline
1.3.7 & que alguno es ladron podemos le mal querer \textbf{ si quiera aya fecho mal a nos o a } otros¶ La segunda diferençia es & possumus ipsum odire , \textbf{ siue fore fecerit in nos , | siue in alios . } Secunda differentia est , \\\hline
1.3.7 & otros¶ La segunda diferençia es \textbf{ que la sanna sienpre es en vna cosa singular . } mas la mal querençia puede ser en comun & Secunda differentia est , \textbf{ quia ira semper est in singulari : } odire potest esse in communi . \\\hline
1.3.7 & mas la mal querençia puede ser en comun \textbf{ por que alguno puede mal queter todo ladron o todo retrahendor de mal comunal mente . } Mas enssannar se non puede & odire potest esse in communi . \textbf{ Odire autem potest | aliquis communiter omnem furem , et detractorem : } sed irasci non potest \\\hline
1.3.7 & mas sienpre el tuerto o la imiuria es acometida \textbf{ por algun omne espeçial . } Et por ende podemos querer mal generalmente alos ladrones & sed semper committatur iniuria \textbf{ per aliquem hominem specialem : } licet odire possumus fures uniuersaliter ; \\\hline
1.3.7 & Et por ende podemos querer mal generalmente alos ladrones \textbf{ enpero non nos enssannamos si non a alguna perssona singular . } ¶ La terçera differençia es & licet odire possumus fures uniuersaliter ; \textbf{ non tamen irascimur , | nisi alicui singulari . } Tertia differentia est , \\\hline
1.3.7 & que el otro padesca mal \textbf{ fasta que sea fecha uengança conuenible . } Mas la mal querençia mata & quod alter contra patiatur , \textbf{ donec fiat condigna ultio . } Sed odium exterminat , \\\hline
1.3.7 & en el septimo libro delas ethicas es assemeiada alos canes \textbf{ o es assemeiada a los sieruos ligeros . } Ca los sieruos ligeros & 7 Ethicor’ assimilatur canibus , \textbf{ vel assimilatur seruis velocibus . } Serui enim veloces statim \\\hline
1.3.7 & oscuresçe la razon e el entendimiento \textbf{ Ca el cuerpo non estando en tenpramiento conuenible somos enbargados en el vso dela razon . } Por la qual cosa commo por la saña se ençienda la sangre cerca el coraçon tornasse el cuerpo destenprado & quia rationem obnubilat . \textbf{ Nam corpore non existente indebito temperamento , | impedimur ab usu rationis , } quare cum per iram accendatur sanguis circa cor , \\\hline
1.3.7 & Ca maga el entendimiento non sea uirtud corporal \textbf{ enpero en su obra vsa de entender de organos e de mienbros corporales . } Por la qual cosa el cuerpo non estando bien ordenado & non sit virtus corporalis , \textbf{ utitur tamen in suo actu corporalibus organis ; } propter quod corpore existente indisposito , \\\hline
1.3.7 & et estrumento de la razon e del entendimiento e obra \textbf{ segunt el señorio de razon seguimos mas esforçadamente las obras uirtuosas . } Onde el philosofo en el terçero libro delas ethicas & et agit secundum imperium rationis , \textbf{ virilius exequimur opera virtuosa . } Unde 3 Ethic’ approbatur dictum hominis dicentis \\\hline
1.3.7 & aquello que la razon \textbf{ e el entendemiento iudgua . } ¶ & et viriliter exequi , \textbf{ quae ratio iudicabit . } Dicebatur enim supra , delectationes , \\\hline
1.3.8 & Et algunos lo han bien ordenado assi commo . \textbf{ los omes bueons e uirtuosos ¶ Et pues que assi es assi commo non son de dezir } uerdaderamente cosas dulçes & aliqui habent ipsum bene dispositum , \textbf{ ut homines boni , et virtuosi . | Sicut ergo non sunt dicenda vere dulcia , } quae videntur dulcia infirmis , \\\hline
1.3.8 & aquellas que paresçen dulçes alos enfermos \textbf{ e aquellos que han el gusto corrupto . } Mas aquellas cosas son uerdaderamente dulçes & quae videntur dulcia infirmis , \textbf{ et habentibus gustum infectum : } sed quae videntur dulcia sanis , \\\hline
1.3.8 & aquellas que son delectables alos uiçiosos e alos malos \textbf{ e aquellos que han el appetito corrupto . } Mas aquellas cosas son de dezir & quae sunt delectabilia vitiosis , \textbf{ et habentibus appetitum infectum : } sed quae sunt delectabilia bonis , \\\hline
1.3.8 & que son delectables alos bueons \textbf{ e aquellos que han la uoluntad derecha e buena ¶ } Et pues que assi es algunas cosas seran delectables & sed quae sunt delectabilia bonis , \textbf{ et habentibus voluntatem rectam . } Erunt ergo aliqua delectabilia vere et simpliciter , \\\hline
1.3.8 & e segunt alguna parte . \textbf{ Otrossi por que la delectacion se faze por ayuntamiento dela cosa conuenible con cosa conuenible } Por ende commo algunas cosas conuengan alas bestias & et secundum quid . \textbf{ Rursus quia delectatio contingit | ex coniunctione conuenientis cum conuenienti : } cum ergo alia conueniant bestiis , alia hominibus : \\\hline
1.3.8 & e algunas seran conuenientes alos omes . \textbf{ Ca las delecta con nes intelligibles e del entendimiento } e las uirtu osas son conuenientes alos omes . & aliquae vero hominibus . \textbf{ Delectationes autem intelligibiles et virtuosae sunt conuenientes hominibus : } sed delectationes venereae \\\hline
1.3.8 & connes han orden \textbf{ et siruen alas obras uirtuosas . } Pues que assi es paresçe en qual manera nos deuemos auer a estas delecta connes . & sed prout habent ordinem , \textbf{ et prout deseruiunt actionibus virtuosis . } Patet ergo quomodo nos habere debemus \\\hline
1.3.8 & que sea uiçi olo e malo \textbf{ e que aya costunbres bestiales siguese } que pertenesçe a cada vno de seguir & cuilibet quod sit vitiosus , \textbf{ et quod mores habeat bestiales , } spectat ad quemlibet \\\hline
1.3.8 & que son delectables alas bestias \textbf{ e alos omes uiciosos e malos . } Mas aquellas cosas que son delectables & sequi non quae sunt delectabilia bestiis , \textbf{ et hominibus vitiosis : } sed quae sunt delectabilia rationabilibus , \\\hline
1.3.8 & e de buen entendimiento \textbf{ e alos omes uirtuosos . } Et pues que assi es toda delectaciones buean & sed quae sunt delectabilia rationabilibus , \textbf{ et hominibus virtuosis . } Omnis igitur delectatio bona est , \\\hline
1.3.8 & e faze la obra mas desenbargada e mas conuenible . Et por ende si los Reyes e los prinçipes se delectaren en las obras dela pradençia \textbf{ e en las obras uirtuosas mas desenbargadamente } e mas acabadamente podrian fazer estas tales obras . & et Principes delectabuntur \textbf{ in actibus prudentiae , | et in operibus virtuosis , } expeditiori modo et magis perfecte efficere poterunt huiusmodi opera . Nam quanto quis vehementiori modo delectatur \\\hline
1.3.8 & e por si mas deuen dellas vsar tenpradamente \textbf{ e en quanto son ordenadas alas obras uirtuosas . } ca si tales delecta conn & sed uti debent eis moderate , \textbf{ et prout habent ordinem | ad opera virtuosa . } Nam si tales delectationes vehementes sint , \\\hline
1.3.8 & si non fuere \textbf{ por alguna cosa torpe triste . } Ca si alguno vee & nec est laudabilis , \textbf{ nisi supposito aliquo turpi . } Si quis enim videt se in aliquo turpia egisse , \\\hline
1.3.8 & que el fizo en algua manera \textbf{ cosas torpes deue se doler } et entsteçer se dello . & Si quis enim videt se in aliquo turpia egisse , \textbf{ debet dolere et tristari . } De turpibus igitur est tristandum , \\\hline
1.3.8 & et entsteçer se dello . \textbf{ ¶ pues que assi es delas cosas torpes se deuen los omes entristeçer } e auer tristeza & debet dolere et tristari . \textbf{ De turpibus igitur est tristandum , } sed omnis alia tristitia est moderanda , et vitanda ; \\\hline
1.3.8 & Ca los malos son enemigos assi mismos \textbf{ e en ssi mismos han discordia } assi conmo en aquel logar da a entender el philosofo . & Nam mali sibi ipsis inimicantur , \textbf{ et in seipsis dissentiunt , } ut ibidem innuitur . \\\hline
1.3.8 & e lo al fazen despues por la obra \textbf{ Por la qual cosa commo ellos non ayan paz en si mismos non gozan de ssi mismos . } Et por ende grand remedio es anos & et aliud passione agunt . \textbf{ Quare cum in seipsis pacem non habeant , | de seipsis non gaudent . } Magnum ergo remedium , \\\hline
1.3.8 & non nos dolemos dellos sinon por auentura por accidente alguno en \textbf{ quanto por perdimiento de aquellos bienes somos enbargados delas obras uirtuosas . } Et pues que assi es paresçe & nisi forte per accidens , \textbf{ inquantum per amissionem | eorum impedimur } ab operibus virtuosis . Patet ergo non esse dolendum , \\\hline
1.3.8 & sinon delas cosas torpes \textbf{ e delans obras uiçiosas e malas . } Mas si el dolor fuere & nisi de turpibus , \textbf{ et non de operibus virtuosis . } Si autem propter alia dolor , \\\hline
1.3.8 & Mas avn suel en dar otro remedio quarto a esto . \textbf{ Conuien e saber remedios corporales . } assi conmo el suenno & Consueuit etiam ad hoc dari quartum subsidium , \textbf{ videlicet , remedia corporalia , } ut somnus , balneum , \\\hline
1.3.8 & e alos prinçipes de tenprar tales tristezas \textbf{ quanto mas cosa conueniente es a ellos } e ala su magnificençia Real & et Principes tales tristitias moderare , \textbf{ quanto decentius est } eos excellere in operibus virtuosis . \\\hline
1.3.9 & e terminasse en temor \textbf{ si aquel mal fuerefuturo que ha de uenir . } mas apostremas terminasse en tristeza & et terminatur in timorem , \textbf{ si malum illud sit futurum : } ultimo autem terminatur in tristitiam , \\\hline
1.3.10 & Et tristeza . \textbf{ Esperança mal querençia . } Desseo . & Enumerabantur supra duodecim passiones , \textbf{ videlicet , amor , odium , desiderium , abominatio , delectatio , tristitia , spes , desperatio , timor , audacia , ira , et mansuetudo . } Sed praeter omnes has passiones Philosop’ \\\hline
1.3.10 & que non puede sofrir conpanma ninguna enla cosa que el ama . \textbf{ Et por ende dende sallio el vso que algunos son dichos çelosos de alguna } personasi non quieren auer alguna conparia en ella ¶ & quod est amor intensus non patiens consortium in amato . \textbf{ Inde ergo inoleuit , | quod aliqui dicuntur Zelotypi de persona aliqua , } si noluerint in ea habere aliquod consortium . \\\hline
1.3.10 & personasi non quieren auer alguna conparia en ella ¶ \textbf{ Et pues que assi es el amor grande de las colas corporales parelçe } que es preuad amor & si noluerint in ea habere aliquod consortium . \textbf{ Intensus ergo amor corporalium videtur esse amor priuatus , et reprehensibilis , } et non patiens consortium in amato . \\\hline
1.3.10 & nin quiere conpania en la cosa que ama . \textbf{ Mas en conparaçion de los bienes intellectuales e en conpaçion delas uirtudes } si fuere el amor grande es de loar & et non patiens consortium in amato . \textbf{ Sed respectu bonorum intellectualium , | et respectu virtutum , } si sit intensus amor , \\\hline
1.3.10 & Mas en conparaçion de los bienes intellectuales e en conpaçion delas uirtudes \textbf{ si fuere el amor grande es de loar } e es assi conmocomun . & et respectu virtutum , \textbf{ si sit intensus amor , } est laudabilis , et quasi communis . \\\hline
1.3.10 & e amor en conparaçion \textbf{ delons bienes honrrables es difinido } e declarado assi por el philosofo & Huiusmodi ergo zelus respectu bonorum honorabilium \textbf{ diffinitur } a Philosopho 2 Rheto’ \\\hline
1.3.10 & Et pues que assi es el zelo \textbf{ que es amor grande es aducho al amor . En essa misma manera } lagera es aducha al amor . & zelus ergo reducitur ad amorem . \textbf{ Sic etiam gratia reducitur ad amorem : } quia ex amore efficitur \\\hline
1.3.10 & que corronpen es dicho temeroso . \textbf{ Mas aquel e se espanta de las cosas torpes e desonrradas es dicho uergonçoso . } Et poͬende non es otra cosa uerguença & Sed ille \textbf{ qui expauescit turpia et inhonoratiua , | dicitur verecundus . } Nihil est aliud verecundia , \\\hline
1.3.10 & Onde la uerguença que es b̃me iura en la cara \textbf{ suele se nonbrar uerguença paresçida en el rostre̊ . } Ca los uergonosos comunalmente se tornan bmeios & unde verecundia erubescentia nominari consueuit , \textbf{ quia verecundantes communiter erubescunt , } sicut timentes pallescunt . \\\hline
1.3.10 & la sangre corre alos mienbros de fuera \textbf{ e aparesçe el rostro bermeio . } Et por ende corre la sangre estonçe alos mienbros de fuera & sanguis fluit ad exteriora , \textbf{ et facies apparet rubea ; } currit ergo sanguis tunc ad exteriora , \\\hline
1.3.10 & assi commo es dicho en el segundo libro de la rectorica non es otra cosa \textbf{ si non vna tristeza dela obra aparesçiente de algunos bienes } que han algunos semeiabł & ( ut dicitur 2 Rhetor’ ) \textbf{ nihil est aliud quam tristitia quaedam | super appetenti } actione aliquorum bonorum circa similes , \\\hline
1.3.11 & quantomas deuen auer las obras mas altas e mas nobles . \textbf{ lgunas delas passiones sobredichas paresçen ser de loar } assi conmo la miscderia e la uerguença & quanto habere debent operationes maxime excellentes . \textbf{ Praedictarum passionum | quaedam videntur esse laudabiles : } ut misericordia , et verecundia . \\\hline
1.3.11 & que es cosa de denostar \textbf{ si non fuer mal querençia de los pecados . } Et o tris passiones & vituperabile est odium , \textbf{ nisi sit vitiorum . } Aliae autem passiones videntur se habere ad utrunque , \\\hline
1.3.11 & por que son estremidades . \textbf{ Mas aquel que se ha en manera medianera assi commo el que ha uerguença do deue } e non ha uerguença do non deue & nullus autem horum est laudabilis . \textbf{ Qui vero medio modo se habet , | ut qui verecundatur } ut debet , \\\hline
1.3.11 & e en quanto son de loar tienen el medio . \textbf{ ¶ Estas cosas iustas conuiene de veer } en qual manera los Reyes & ut sunt laudabiles , tenent medium . \textbf{ His visis , videndum est , } quomodo Reges et Principes \\\hline
1.3.11 & uirgon cosos . \textbf{ por que la uerguença es delas cosas malas . Mas al estudioso non conuiene obrar ningunas cosas malas . } Por la qual cosa si conuiene alos Reyes & quod studiosi non est verecundari , \textbf{ quia verecundia est in prauis : | eius autem non est praua operari . } Quare si decet Reges esse studiosos , \\\hline
1.3.11 & assi commo son las uirtudes . \textbf{ Mas por auentra a pueden auer bienes medianeros o bienes muy pequanos } los quales son bienes de fuera . & cuiusmodi sunt virtutes : \textbf{ sed forte possidere possunt bona media , | vel bona minima , } cuiusmodi sunt bona exteriora . \\\hline
1.3.11 & por que non nos reteryamos \textbf{ nin nos tiremos de los bienes grandes e altos } por guaueza de lons ganar . & Nam magnanimitas reprimit desperationem , \textbf{ ne retrahantur a bonis arduis propter difficultatem . } Humilitas vero moderabit spem , \\\hline
1.4.1 & o quando ha ganado sus riquezas \textbf{ por su sabiduria propia o por su trabaio proprao . } Ca aquella cosa que es ganada con trabaio & vel quando facultates illas \textbf{ acquisiuit propria industria , | et proprio labore . } Nam quod cum labore acquiritur , \\\hline
1.4.1 & e avn son de buena esꝑança \textbf{ por que en ellos abonda mucho calentura natural . } Et por ende el coraçon & Sunt etiam bonae spei , \textbf{ quia in eis multum abundat calor : } corde ergo et aliis membris inflammatis \\\hline
1.4.1 & Otrossi los mançebos biuieron poco en el tienpo passado \textbf{ e segunt cursso natural deuen much beuir en el tienpo } que ha de uenir . & Rursus iuuenes parum vixerunt in praeterito , \textbf{ et secundum cursum naturalem debent multum viuere in futuro . } Cum ergo memoria sit respectu praeteritorum , \\\hline
1.4.1 & por ende son animolos e de grand esperança . \textbf{ Et avn podemos aesto adozir otra razon espeçial . } Ca por que los mançebos son muy calientes & quin sint magnanimi . \textbf{ Posset etiam ad hoc specialis ratio assignari . } Nam cum iuuenes sint percalidi , \\\hline
1.4.1 & Et pues que assi es por que los mançebos \textbf{ por la calentura natural desse an } sobrepuiar los otros temen & Cum ergo iuuenes , \textbf{ qui percalidi nimis affectent excellere , } timent inglorificari , \\\hline
1.4.1 & que ellos han \textbf{ e por las quales los preçian en las non espender en vsos o en obras conuenibles e piadosas } assi commo dessuso dixiemos & si multitudinem diuitiarum qua pollent , \textbf{ non multiplicarent in debitos et pios usus , } ut supra in tractatu de liberalitate sufficienter tetigimus . \\\hline
1.4.1 & delons omes demanda perdon \textbf{ por las cosas mal fechos o malobradas . } Por la qual cosa conuiene alos Reyes & ipsa ergo humana fragilitas \textbf{ veniam postulat pro delictis : } quare Reges et Principes , \\\hline
1.4.2 & que deuen lercabesca e regla de todos los otros . \textbf{ Et por ende cosa desconuenible es alos Reyes de ser segnidores delas passiones } e de auer cobdiçias afincadas de lux̉ia & qui debent esse caput et regula aliorum . \textbf{ Indecens enim est Reges et Principes | esse passionum insecutores , } et venereorum habere concupiscentias vehementes : \\\hline
1.4.2 & Mas aquellos que assi non biuen non biuen por razon mas biuen por passion \textbf{ Otrossi cosa desconuenible es } alos reys de ser mudables e trastornables . & non viuunt ratione , sed passione . \textbf{ Rursus detestabile est } eos esse permutabiles , et vertibiles . \\\hline
1.4.2 & alos reys de ser mudables e trastornables . \textbf{ Porque cosa desconuenible es que la regla sea tuerta } e ellos son commo regla . & eos esse permutabiles , et vertibiles . \textbf{ Nam cum inconueniens sit } regulam esse obliquam , \\\hline
1.4.2 & e regla de beuir a todos los otros \textbf{ cosa desconuenible es aellos } que de ligero se trasmuden & et regula aliorum , \textbf{ inconueniens est } quod de facili peruertantur , \\\hline
1.4.2 & Mas conuiene aellos de ser firmes e estables ¶ \textbf{ Lo terçero cosa desconuenible es alos Reyes } e alos prinçipes de çreer de ligero . & eos esse firmos et stabiles . \textbf{ Tertio indecens est } eos esse nimis creditiuos . \\\hline
1.4.2 & o que les son aquellos que les fablan \textbf{ si sen sabios o non sabios } otrossi si son uirtuosos o uiçiosos e pecadores & qui sunt qui loquuntur , \textbf{ utrum sint sapientes vel ignorantes , } utrum uirtuosi uel uitiosi : \\\hline
1.4.2 & que alos locos e malos ¶ \textbf{ Lo quarto non es cosa conueniente aellos de ser tortizeros } e deno stadores & quam insipientibus et malis . \textbf{ Quarto indecens est } eos esse iniuriatores et contumeliosos . Nam poenas inferre debent , \\\hline
1.4.3 & Mas cuydan que todos los otros \textbf{ sanmint tosos e engannadores . } Por ende dize el philosofo & non de facili fit eis fides , \textbf{ sed credunt omnes alios esse deceptores . } Ideo dicitur 2 Rhetoricorum , \\\hline
1.4.3 & Et mas estudian al prouecho \textbf{ que a auer honrra o estado honrado . } Et assi lo dize el philosofo & Magis enim student ad utilitatem , \textbf{ quam ad ea quae requirit honoris status , } ut vult Philosophus 2 Rhetoricorum . \\\hline
1.4.3 & que los otros les tengan en much̃ . \textbf{ Et por esta razon aceesçe alos uieios de ser desuergoncados } Mas puede aqui ser fallada vna razon que es a comun a todas estas cosas de suso dichͣs . & et non curant reputari : \textbf{ quare contingit eos esse inuerecundos . } Posset autem una causa assignari , \\\hline
1.4.3 & nin de ser tenidos en muchͣ \textbf{ por que la cosa fria non ha de querer logar alto mas baxo . } visto quales son las costunbres de los uieios & nec curent reputari : \textbf{ quia frigidi non est | appetere locum superiorem , sed inferiorem . } Viso qui sunt mores senum vituperabiles ; \\\hline
1.4.3 & que commo quier que los Reyes non de una ser \textbf{ en todas las cosas creedores de ligero } assi conmo los moços . & non debeant esse \textbf{ in omnibus de facili creditiui , } ut pueri : \\\hline
1.4.4 & que pueden ser de loar ¶ \textbf{ La primera es que los uieios han las cobdiçias botas e tenpradas e flacas ¶ } La segunda es que son mibicordiosos ¶ & qui possunt esse laudabiles . \textbf{ Primo enim senes habent | concupiscentias remissas , et temperatas . } Secundo sunt miseratiuis . \\\hline
1.4.4 & e van a ella . \textbf{ Et esto es contra la razon natural del frio . } Ca el frio en quanto esfrio non se estiende propreamente ala luxia mas & in alia se extendit . \textbf{ Hoc autem est contra rationem frigidi . } Nam frigidi ( secundum quod huiusmodi ) \\\hline
1.4.4 & en la qual non vse el alma \textbf{ en alguna manera de instrumento e de mienbro corporal . } assi commo paresçe en la obra artifiçial & in qua non utatur \textbf{ aliquo modo organo corporeo : } sicut in opere artificiali , \\\hline
1.4.4 & faze se mudamiento \textbf{ en la obra artifiçial Vien } assi es en las obras del alma & variato organo fit variatio operis ; \textbf{ sic et in actionibus animae , } corpore transmutato , \\\hline
1.4.4 & ¶ \textbf{ Lo terçero los uieios non afirman ningunan cosa dubdosa afinçada mente . } Ca assi conmo dize el philosofo en el segundo libro dela rectorica & et miserentur aliis . \textbf{ Tertio nihil dubium pertinaciter affirmant . } Nam ( ut ait Philosoph’ 2 Rheto’ ) \\\hline
1.4.4 & dubdosamente sentençian de todas las cosas \textbf{ e sienpre ponen en sus razones alguna condiçion dubdosa . } assi conmo es sio & quod senes omnia dubie sententiant , \textbf{ et semper apponunt ibi Quasi , } vel Forte , \\\hline
1.4.4 & todas son falladas \textbf{ en los que son en el estado medianero . } E todas las colas que son de denostar & totum reperitur in iis \textbf{ qui sunt in statu . } Et quicquid vituperabilitatis est \\\hline
1.4.4 & mançebos todo aquello es fallado mas enteramente \textbf{ e mas acabadamente en los que son en el estado medianero ¶ } Otrossi porque qual si quier cosa dela estremidat & totum peramplius et perfectius reperitur \textbf{ in iis qui sunt in statu : } quicquid laudabilitatis est in eis , \\\hline
1.4.4 & e en los uieios es alongado \textbf{ de aquellos que son en estado medianero . } Ca qual si quier cosa que es de denostar & Rursus quia quicquid extremitatis est in eis , \textbf{ remouetur ab eis | qui sunt in statu : } quicquid vituperabilitatis est in illis , \\\hline
1.4.4 & Ca qual si quier cosa que es de denostar \textbf{ en ellos deuemos lo alongar e tirar de aquellos que son en estado medianero . } Et pues que assi es en tal manera deuemos fablar delas costunbres delons omes . & quicquid vituperabilitatis est in illis , \textbf{ remouetur ab iis | qui sunt in statu . } Sic ergo loquendum est de moribus hominum . \\\hline
1.4.4 & en ellos deuemos lo alongar e tirar de aquellos que son en estado medianero . \textbf{ Et pues que assi es en tal manera deuemos fablar delas costunbres delons omes . } Empero non se deue entender & qui sunt in statu . \textbf{ Sic ergo loquendum est de moribus hominum . } Non tamen intelligenda sunt \\\hline
1.4.4 & que veemos en las cosas los mançebos e los vieios \textbf{ e aquellos que son en estado medianero han alguna inclinacion natural alas costunbres } que les conuienen & iuuenes , et senes , \textbf{ et illi qui sunt in statu , | quandam pronitatem , } et inclinationem habent \\\hline
1.4.4 & e alos prinçipes de afirmar \textbf{ afincadamente las cosas dubdosas ala vna parte } por que por esto non sean iudgados liuianos & et Principes dubia pertinaciter \textbf{ in alteram partem asserere , } ne per hoc iudicentur leues et indiscreti . \\\hline
1.4.4 & por que assi commo los vieios e los mançebos maguera ayan disposiconn \textbf{ e indinacion natural ha costunbres malas e de deno star . } Empero pueden fazer contra aquella disposiconn & et iuuenes habent \textbf{ quandam pronitatem naturalem , | et inclinationem ad mores vituperabiles : } possunt tamen contra illam pronitatem facere \\\hline
1.4.4 & e segnir bueans costunbrs e de loar . \textbf{ Vien assi ahun aquellos que son en estado medianero maguera de ssi ayan disposiconn } e indinaçion a costunbres bueans e de loar & Sic et illi \textbf{ qui sunt in statu , | et si de se pronitatem habent } ad mores laudabiles , \\\hline
1.4.4 & Vien assi ahun aquellos que son en estado medianero maguera de ssi ayan disposiconn \textbf{ e indinaçion a costunbres bueans e de loar } enpero pueden uenir e fazer contra esta disposiçion natural & et si de se pronitatem habent \textbf{ ad mores laudabiles , } possunt tamen contra istam pronitatem facere , \\\hline
1.4.4 & pueden seguir malas costunbres e de denostar . \textbf{ Por la qual cosa sinoble cosa es } e muy digna de & vituperabiles mores . \textbf{ Quare si dignum est dominari rationi , } et intellectui , \\\hline
1.4.4 & e por entendemiento Conuiene alos Reyes e alos prinçipes \textbf{ que son senores de los otros segnir costunbres bueans e de loar } segunt & decet Reges , et Principes , \textbf{ qui aliis dominantur , | sequi mores laudabiles } secundum dictamen , \\\hline
1.4.5 & Ca por esso algunos son dichos nobles \textbf{ por que desçenden de linage honrrado . } Mas el linage es dicho honrrado & Ex hoc enim aliqui dicuntur esse nobiles , \textbf{ quia processerunt ex genere honorabili . } Genus autem honorabile dicitur \\\hline
1.4.5 & la qual el philosofo llama nobleza non es otra cosa \textbf{ si non ser de algunt linage alto o de alguna sangre escogida } enla qual antigua miente fueron muchos prinçipes e muchos nobles . & nihil est aliud quam esse \textbf{ ex aliquo genere , | vel ex aliqua prosapia , } in qua etiam ab antiquo fuere multi principantes , \\\hline
1.4.5 & Et pues que assi es por que la nobleza es assi iudgada \textbf{ segunt la opinion comun de los omes } la nobleza non es otra cosa & nobilitas \textbf{ secundum communem acceptionem hominum , } nihil est aliud \\\hline
1.4.5 & que sienpre la fechura quiera semeiar a su fazedor \textbf{ por que los fijos son fechuras de los padron natural cosa es que los fuos semeien alos paradres . } Et por ende los nobles teniendo mientes & cum filii sint \textbf{ quidam effectus parentum , | naturale est filios imitari parentes . } Nobiles ergo aduertentes \\\hline
1.4.5 & si temieren de ser reprehendidos \textbf{ e si temieren de fazer cosas reprehenssibles e si cuydaren sotilmente todo lo que han de fazer } ¶La quatta condicion de los nobles es & si timentes reprehensibilia facere , \textbf{ diligenter considerent quid agendum . } Quarto nobiles contingit esse politicos , et affabiles . \\\hline
1.4.5 & e si temieren de fazer cosas reprehenssibles e si cuydaren sotilmente todo lo que han de fazer \textbf{ ¶La quatta condicion de los nobles es } que son corteses e amigables . & diligenter considerent quid agendum . \textbf{ Quarto nobiles contingit esse politicos , et affabiles . } Nam quia ut plurimum in curiis nobilium consueuit \\\hline
1.4.5 & e alos prinçipes delas auer \textbf{ conplidamente es dich̃o en las cosas sobredichͣs . } Ca ya dixiemos de ssuso & et Principes , \textbf{ in antehabitis sufficienter est dictum . } Diximus enim supra , \\\hline
1.4.5 & e en qual mauera de ser magnificos \textbf{ e en qual manera sabios e ensseñados } Et en qual manera bien razonados e & et Principes esse magnanimos , quomodo magnificos , \textbf{ quomodo prudentes et dociles , } et quomodo affabiles et sociales . \\\hline
1.4.6 & uenta el philosofo en el segundo libro de la rectorica \textbf{ que son çinço malas costunbres de los ricos ¶ } la primera es que los ricos son sobra uos ¶ & 2 Rhetoricor’ \textbf{ quinque malos mores ipsorum diuitum . } Diuites enim primos sunt elati . \\\hline
1.4.6 & por que paresçe \textbf{ segunt la opinion comun de los pueblos } que todas las cosas han de ser mesuradas & Videtur enim esse \textbf{ in communi opinione vulgarium } omnia mensurari numismate , \\\hline
1.4.6 & que non conosçen \textbf{ si non los bienes senssibles cuydan } que este es el mayor bien & a vulgaribus , \textbf{ qui non cognoscunt nisi bona sensibilia , } reputatur bonum excellens , \\\hline
1.4.6 & por que el omne es digno de ser señor . \textbf{ Ca creen que las riquezas son dignidat de sennorio e de prinçipado e semeiales a ellos que las riquezas lon tan grand bien } que qualquier omne & id quo quis efficitur dignus principari : \textbf{ credunt enim , } quod dignitas principatus \\\hline
1.4.6 & Mas si las riquezas fueren ordenadas a alabança o a pelea o a destenprança o a otrasma las obras \textbf{ estonçe mas fazen al omne mal andante que bien andante . } Et por ende ca vno o cuyde quesa digno de ser prinçipe & vel ad intemperantiam , \textbf{ vel ad alia opera vitiosa : | tunc magis reddunt hominem infelicem , } quam felicem . \\\hline
1.4.6 & e por los ordenamientos de dios \textbf{ han estos bienes tenporales e estas riquezas . } Et por ende este dicho tan sotil del philosofo & per ordinationem diuinam \textbf{ habere huiusmodi bona . } Hoc autem dictum Philosophicum \\\hline
1.4.6 & mientesa esto \textbf{ farien grandes cosas çerca las cosas diuinales . } Et por ende non creyrien que ellos dan grandes dones a dios & quam in propriam industriam : \textbf{ quod si hoc bene attenderent diuites faciendo magnifica circa diuina , } non crederent se Deo dona largiri , \\\hline
1.4.6 & Mas estabuean costunbre \textbf{ que es auer se el omne bien terca las cosas diuinales tanto } mas conuiene alos Reyes & Hunc autem bonum morem , \textbf{ videlicet bene se habere circa diuina , } tanto magis decet Reges , et Principes , \\\hline
1.4.7 & Ca contesçe alas vezes que algunos son nobles \textbf{ por que desçendende linage honrrado . } Et pero non son ricos . & Contingit enim aliquos esse nobiles , \textbf{ quia processerunt ex aliquo nobili genere , } qui tamen non sunt diuites . \\\hline
1.4.7 & Et pero non son ricos . \textbf{ Avn en essa misma manera contesçe que algunos son ricos } los quales non son nobles & qui tamen non sunt diuites . \textbf{ Sic etiam contingit aliquos esse diuites } qui non sunt nobiles , \\\hline
1.4.7 & ca ninguons non se gouiernan so su sennorio . \textbf{ Por la qual cosa non es vna cosa ser el omne rico e ser toderoso¶ } Visto quales son las costunbres delons nobłs & quia nulli reguntur sub eius imperio ; \textbf{ quare non est idem esse diuitem , | et esse non potentem . } Viso ergo \\\hline
1.4.7 & Por la qual cosa non es vna cosa ser el omne rico e ser toderoso¶ \textbf{ Visto quales son las costunbres delons nobłs } e de los ricos finça de & et esse non potentem . \textbf{ Viso ergo | qui sunt mores nobilium , } et qui diuitum restat videre , \\\hline
1.4.7 & e tiran se \textbf{ porque non pueden del todo entender alas obras luyiosas . } por la qual cosa contesçe & retrahuntur , \textbf{ ut non omnino possint vacare venereis . } Quare contingit potentes \\\hline
1.4.7 & alguons non ge lo fazen en pequanas cosas mas en grandes . Ca los poderosos estando en gerad sennorio \textbf{ por que son en logar digno de grand honrra } non entienden si non en grandes cosas e altas . & sed in magnis . \textbf{ Potentes enim existentes in Principatu , | quia sunt in loco magno honore digno , } non tendunt nisi in magna et in ardua . \\\hline
1.4.7 & por que non curan de fazer \textbf{ pequano tuerto nin pequeno danno . } Mas o en ninguna cosa non faran danno alos otros o les faran grand danno . & Non enim curabunt \textbf{ facere paruam offensam , } sed vel in nullo damnificabunt alios , \\\hline
1.4.7 & Ca el rico si non fuere noble \textbf{ e non desçendiere de linage antigo e honrrado } mas si fuere enrriqueçido del otro dia & Diues enim si non sit nobilis , et non processerit \textbf{ ex quodam genere antiquo et honorabili , } sed sit nuper ditatus , \\\hline
1.4.7 & mas si fuere enrriqueçido del otro dia \textbf{ aca es dicho bien auentra adolin lelo . } por que non sabe husar delas riquezas & sed sit nuper ditatus , \textbf{ dicitur esse insensatus felix , } quia aescit uti diuitiis , \\\hline
1.4.7 & e de antigo tienpo los sus auuelos fueron ricos meior sabe sofrir las riquezas \textbf{ e por ellas non se leu nata en so ƀͣuia } por que los bienes senssibles paresçen del todo ser contrarios alas sciençias e alas uirtudes . & melius nouit diuitias supportare , \textbf{ et propter eas non tantum extollitur . } Videntur enim bona sensibilia \\\hline
1.4.7 & e por ellas non se leu nata en so ƀͣuia \textbf{ por que los bienes senssibles paresçen del todo ser contrarios alas sciençias e alas uirtudes . } Ca las riquezas e los bienes senssibles & et propter eas non tantum extollitur . \textbf{ Videntur enim bona sensibilia | omnino esse contraria scientiis , et virtutibus . } Nam diuitiae et sensibilia \\\hline
1.4.7 & e han grand disposiçion \textbf{ para segnir las costunbres sobredichͣs . } Et por ende non se deuen enssonnar los mançebos & et magnam pronitatem habent , \textbf{ ut sequantur praedictos mores . } Iuuenes ergo et senes non indignentur , \\\hline
2.1.1 & Ca la nata en vano faria las cosas \textbf{ si las cosas naturales en ninguna manera non se pudiessen guardar en si mesmas e en su ser . } Et en vano & Frustra enim natura ageret , \textbf{ si res naturales nullo modo conseruarentur in esse , } sed statim , \\\hline
2.1.1 & assi mesmo en la uida \textbf{ son colas naturales al omne } Mas entre las o triscosas & sibi in vita sufficere , \textbf{ sunt homini naturalia : } inter alia autem , \\\hline
2.1.1 & Et por ende naturalmente el omne \textbf{ esaianlia aconpanable e conpanera Mas que la conpannia mucho faga a } conplimien to deuida de omne & est societas , \textbf{ naturaliter ergo homo est animal sociabile . | Quod autem societas maxime faciat } ad sufficientiam vitae humanae , \\\hline
2.1.1 & Ca el omne entre todas las aian lias ha meior conplission . \textbf{ Et por ende entre todas las ainalias ha mester meior uianda e meior apareiada . } Ca la natura a todas las & et meliorem complexionem . \textbf{ Ideo inter omnia animalia indiget cibo diligenter , | et artificialiter praeparato . } Natura enim animalibus aliis \\\hline
2.1.1 & por natura \textbf{ maguera sea vianda sufiçiente a todas las otras } aian lias & quod a natura producitur , \textbf{ et si esset sufficiens cibus animalibus aliis : } homini autem non est sufficiens cibus , \\\hline
2.1.1 & e fazen ende pan e lo cuezen \textbf{ por que se auianda conuenible al omne ¶ Et } por que para todas estas colas & et coquitur , \textbf{ ut sit hominum congruus cibus . } Et quia ad haec omnia una sola persona non bene sufficit , \\\hline
2.1.1 & que de parte dela uianda que auemos mester \textbf{ El omne es naturalmente con panno e ainalia aconpannable ¶ } La segunda manera para prouar esto mesmo se toma & quo indigemus , \textbf{ homo est naturaliter animal sociale . } Secunda via ad inuestigandum hoc idem , \\\hline
2.1.1 & que el o omne ha natural inclinamiento \textbf{ para ser conp̃anero e ainal aconpannable . } Et pues que assi es neçessaria fue & sequitur quod homo naturalem impetum habeat \textbf{ ut sit animal sociale ; } necessaria ergo fuit communitas domus , \\\hline
2.1.1 & siguese que beuir en conpannia \textbf{ e en comunidat es en alguna manera natural alos omes } ¶ & Sed si haec sunt necessaria ad conseruandam hominis naturalem vitam , \textbf{ viuere in communitate et in societate est quodammodo homini naturale . } Tertia via ad inuestigandum hoc idem , \\\hline
2.1.1 & que biue solo non abaste assi mismo \textbf{ para auer uianda conueinble nin uestidura } nin para fazer para si armas e jnstrumentos & cum homo solitarius non sufficiat sibi \textbf{ ad habendum congruum victum et vestitum , } et ad fabricandum sibi arma et organa , \\\hline
2.1.1 & Ca tondas las otras aian lias se inclinan conplidamente alas obras \textbf{ que deuen por inclinaçion natural sin ninguno otro ensseñamiento primero } assi commo el aranna & ad opera sibi debita \textbf{ ex instinctu naturae | absque introductione aliqua praecedente : } ut aranea ex instinctu naturae \\\hline
2.1.1 & por la qual cosa \textbf{ si assi es cosa natural al omne de ser aian la conpanable } los que fuyen la conpannia & quam animalia cetera . \textbf{ Quare si sic naturale est , | hominem esse animal sociale : } recusantes societatem , \\\hline
2.1.1 & Et pues que assi es esto contesçe alos omes \textbf{ por que fallesçen de la manera de beuir comunal de los omes } e en entonsçe son assi commo bestias & vel ergo hoc contingit eis , \textbf{ quia deficiunt a modo humano , } et tunc sunt quasi bestiae ; \\\hline
2.1.2 & por la qual abastamos a nos en la uianda e en el uestido \textbf{ e en las otras cosas neçessarias ala uida non paresçe } que sea comunidat çiuil de casa mas paresce & per quam nobis sufficimus in victu et vestitu , \textbf{ et in aliis necessariis ad vitam , } non videtur esse communitas domestica , \\\hline
2.1.2 & e en las otras cosas neçessarias ala uida non paresçe \textbf{ que sea comunidat çiuil de casa mas paresce } que sea comunidat çiuil e de çibdat . & et in aliis necessariis ad vitam , \textbf{ non videtur esse communitas domestica , } sed ciuilis : \\\hline
2.1.2 & que sea comunidat çiuil de casa mas paresce \textbf{ que sea comunidat çiuil e de çibdat . } Ca segunt dize el philosofo & non videtur esse communitas domestica , \textbf{ sed ciuilis : } quia secundum Philosophum 1 Politicorum , \\\hline
2.1.2 & que la comunidat dela casa es neçessaria a esta uida . \textbf{ Et pues que assi es en el capitulo sobredicho auemos determinado dela conpannia humanal } mostrando que es neçessario a lanr̃auida & ad huiusmodi vitam necessariam esse . \textbf{ In praecedenti ergo capitulo determinauimus de societate humana , } ostendentes eam esse necessariam ad vitam nostram : \\\hline
2.1.3 & que mora en aquella casa . \textbf{ Ca bien commo la çibdat algunas uezes nonbralos muros } e la cerca dela çibdat & vel nominare potest familiam in ea contentam . \textbf{ Sicut et ciuitas aliquando muros , } et ambitum ciuitatis : \\\hline
2.1.3 & por que ael parte nesçe generalmente demostrar \textbf{ por figera e por exienplo que conuiene alos omes de auer conueibles moradas } segunt el su poder e la su riqueza . & et typo ostendere , \textbf{ quod decet homines habere habitationes decentes } secundum suam possibilem facultatem ; \\\hline
2.1.3 & es fin dela comunidat del uarrio . \textbf{ Mas la comunidat del regno es fin de todas las otras comuindades sobredichas . } Por la qual cosa conmola cosa non acabada sea primero & communitas ciuitatis communitatis vici : \textbf{ sed communitas regni est finis omnium praedictorum . } Quare cum imperfectum \\\hline
2.1.3 & ante pone la natura . \textbf{ Qual si quier cosa que dela cosa naturales presunpuesta } e antepuesta non puede ser propreamente cosa artifiçial & sed ars naturam : \textbf{ quicquid arte naturali supponitur , } non proprie quid artificiale erit , \\\hline
2.1.3 & e ante pongan la comunidat dela casa \textbf{ conuiene quala comunidat dela casa o la casa sea cosa natural . } Et por ende conuiene alos Reyes e alos prinçipes & praesupponat communitatem domus , \textbf{ oportet communitatem domesticam siue domum | quid naturale esse . } Reges ergo et Principes decet \\\hline
2.1.3 & e gouernar la conpanna dela casa \textbf{ non solamente en quanto deuen ser uarones aconpannables e bien acostunbrados . } Ca en esta manera saber gouernar la casa & et regere familiam siue domum , \textbf{ non solum inquantum esse debent viri sociales et politici , } quia sic scire gubernationem domus pertinet ad omnes ciues : \\\hline
2.1.4 & que el omne es naturalmente ainalia domestica e de casa \textbf{ e quela comunidat dela casa es en alguna manera natural . } Empero por que esto non auemos & quod homo est naturaliter animal domesticum , \textbf{ et quod communitas domus est quodammodo naturalis . } Attamen quia per hoc non sufficienter habetur \\\hline
2.1.4 & e departiremos todas las ꝑtes dela casa \textbf{ e prouaremos que cada vna ꝑtetal dela casa es cosa natural . } Pues que assi es finça de declarar en la difiniçion sobredichͣ & ubi distinguentur omnes partes domus , \textbf{ et probabitur quod quaelibet | talis pars est aliquid naturale . } Restat ergo declarare \\\hline
2.1.4 & Mas porque en cada vna casa non son falladas \textbf{ todas las cosas neçessarias para la uida } non cunplie la comunidat de vna casa & quibus quotidie indigemus . \textbf{ Verum quia in una domo non reperiuntur omnia necessaria ad vitam , } non sufficiebat communitas domestica , \\\hline
2.1.4 & Mas ahun por que en vnuarrio non son falladas \textbf{ todas las cosas neçessarias ala uida } conuiene de dar comunidat ala çibdat & Verum quia etiam in uno vico non reperiuntur \textbf{ omnia necessaria ad vitam , } praeter communitatem vici \\\hline
2.1.4 & que enbarguna \textbf{ e corronpen las cosas neçessarias ala uida . } Et para tirar e arredrar estas cosas & sed etiam ad remouendum prohibentia corruptiua ; \textbf{ ad quae remouendum una ciuitas } non potest plene sufficere , \\\hline
2.1.4 & e de cada dia \textbf{ mas non es cosa fuerte de veer } que la casa sea establesçida de muchas perssonas . & propter opera diurnalia et quotidiana . \textbf{ Quod autem oporteat domum | ex pluribus constare personis , videre non est difficile . } Nam cum domus \\\hline
2.1.4 & e del regno esto se mostrara mas conplidamente en el terçero libro . \textbf{ Mas quantoalo presente abasta de dezir en } tantodel regno e dela çibdat & in tertio libro plenius ostendetur . \textbf{ Ad praesens autem sufficiat } in tantum tangere de regno et ciuitate , \\\hline
2.1.4 & sea tan neçessaria \textbf{ enla uida çiuil pertenesçe a cada vn çibdada } no de laber gouernar conueniblemente lucasa . & tam necessaria in vita ciuili , \textbf{ spectat ad quemlibet ciuem scire debite regere suam domum : } tanto tamen magis hoc spectat ad Reges et Principes , \\\hline
2.1.5 & e assi commo mas conplidamente se declara \textbf{ mas adelante es cosa natural . } Ca prinçipalmente cosa natural es la generaçion delas cosas & Nam domus ( ut superius dicebatur , \textbf{ et ut in prosequendo melius declarabitur ) est quid naturale . } Maxime autem quid naturale esse videtur , \\\hline
2.1.5 & mas adelante es cosa natural . \textbf{ Ca prinçipalmente cosa natural es la generaçion delas cosas } e la conseruaçiondellas & et ut in prosequendo melius declarabitur ) est quid naturale . \textbf{ Maxime autem quid naturale esse videtur , | rerum generatio , } et earum conseruatio . \\\hline
2.1.5 & commo la generaçion sea camino e carrera en la natura \textbf{ et commo las cosas naturales resçiban su naturaleza } propra a por la generaçion & sit via in naturam , \textbf{ et cum res naturales } per generationem propriam naturam accipiant , \\\hline
2.1.5 & conuiene \textbf{ que la casa sea cosa natural ¶ } Otrossi por que la generaçion e la conseruaçion non pueden ser apartadas la vna dela otra & et serui ad conseruationem . Quare si generatio et conseruatio est quid naturale , \textbf{ oportet domum quid naturale esse . } Amplius , quia generatio et conseruatio \\\hline
2.1.5 & quanto mas es esforçado en las uirtudes del cuerpo \textbf{ tanto mas ha mester a alguno que lo enderesçe para su salud propria . } Ca assi commo veemos & quanto magis pollet viribus corporalibus , \textbf{ tanto propter salutem propriam | magis indiget dirigente . } Sicut enim caecus corporaliter , \\\hline
2.1.5 & Ca por mengua dela fuerça corporal \textbf{ ensegniendo las cosas neçessarias ala uida non pueden abastar } assimesmos & quia propter defectum fortitudinis corporalis , \textbf{ in exequendo necessaria ad vitam , | sibi ipsis non possunt sufficere . } Quare si dominus saluatur propter seruum , \\\hline
2.1.6 & Ca ueemos en las cosas naturales \textbf{ que luego que son engendradas las cosas dla natura } luego es acuçiosa dela salud & Videmus enim in naturalibus rebus \textbf{ quod statim quum generatae sunt , } natura est solicita \\\hline
2.1.6 & engendrada la natura es acuçiosa çerca de su salud . \textbf{ Empero faze engendrar cosa semeiante de ssi non es } assi conpado alas cosas natraales & solicitatur natura circa salutem eius ; \textbf{ producere tamen sibi simile , } non sic comparatur ad res naturales : \\\hline
2.1.6 & mas conuiene que primeramente el sea acabado . \textbf{ Et pues que assi es engendrar su semeiante non pertenesçe a cosa natural tomada en qual quier manera mas pertenesçe a cosa natural en quanto ella es acabada . } Et pues que assi es si la casa es cosa natural & producere ergo sibi similem , \textbf{ non est de ratione rei naturalis | quocumque modo , sumptae ; sed est de ratione eius , } ut habet esse perfectum . \\\hline
2.1.6 & e las cosas que veemos en la casa queremos traer \textbf{ a razones naturales diremos que las dos comuidades } que son de varon e de muger e de señor e de sieruo & et ea quae videmus in domo , \textbf{ reducere volumus in naturales causas , | dicemus duas communitates , } videlicet , viri et uxoris , et domini et serui , \\\hline
2.1.6 & que estosson del fazimiento dela primera casa \textbf{ ca sin ellas non puede ser la primera casa conuenible mente . } Mas la terçera comunidat & esse de ratione domus primae , \textbf{ quia sine eis domus congrue esse non potest : } sed tertiam etiam communitatem , \\\hline
2.1.6 & en que non ay engendraçion de fijos . \textbf{ ¶ La segunda razon para prouar esto mesmo se toma de acabamiento de comunindat natural duradera por sienpre . } Ca commo los omes non pueden & ubi non est pollulatio filiorum . \textbf{ Secunda via ad inuestigandum hoc idem , | sumitur ex parte naturalis perpetuitatis . } Nam cum homines non possunt \\\hline
2.1.6 & e viene nueuamente otra conpanna en aquella casa . \textbf{ Et pues que assi es la morada natural dela casa } non puede durar naturalmente para sienpre & noua familia inhabitet domum illam . \textbf{ Naturalis ergo habitatio domestica naturaliter perpetuari non potest , } nisi per generationem , \\\hline
2.1.6 & de parte dela bien andança \textbf{ ca los fijos e el poderio çiuil e las otras } cosastales commo quier que non sean essençiales ala bien andança . & ex parte ipsius felicitatis . \textbf{ Nam filii , et ciuilis potentia , | et cetera talia , } licet non sint essentialia felicitati : \\\hline
2.1.6 & Enpero fazen algunan mostrança \textbf{ e algunan nobleza dela bien andança çiuil . } Onde el philosofo en el primero libra delas ethicas & licet non sint essentialia felicitati : \textbf{ faciunt tamen ad quandam claritatem felicitatis politicae ; } unde Philosophus 1 Ethic’ ait , \\\hline
2.1.6 & et dela fenbra el uaron deua ser prinçipal et ordenador \textbf{ e la fenbra obediente . } Mas en la comunidat del padre & mas debet esse principans , \textbf{ et foemina obsequens : } in communitate vero patris et filii , \\\hline
2.1.7 & que el omne es naturalmente \textbf{ aina l aconpannable e comun incatiuo } que quiere dezir ꝑtiçipante con otro & Probabatur enim in primo capitulo huius secundi libri , \textbf{ hominem esse naturaliter animal sociale et communicatiuum . } Communitas autem in vita humana \\\hline
2.1.7 & Por quela casa es ordenada al nodrimiento e gouernamiento \textbf{ que mucho es cosa neçessaria a cada vn omne singular } e que conuiene mucho al bien propra o de cada vno . & Ordinatur enim domus ad nutritionem , \textbf{ quae maxime est necessaria indiuiduo , } et quae maxime expedit bono proprio : \\\hline
2.1.7 & e que mas es ayuntable \textbf{ por comunidat coniugable e de matermoino } que por comunidat de barrio & quod homo magis sit animal coniugale \textbf{ quod politicum ; et quod magis sit communicatiuum communitate coniugali , } quam communitate vici ciuitatis , et regni : \\\hline
2.1.7 & nin de çibdat nin de regno . \textbf{ por que la casa cuyo gouernamiento primero es gouernamiento coniugal del uaron } e dela muger es primero & quam communitate vici ciuitatis , et regni : \textbf{ quia domus , | cuius regimen primum est coniugale , } est prior vico , regno , et ciuitate . \\\hline
2.1.7 & abastamientode uida . \textbf{ Por la qual cosa si natural cosa es al omne de auer inclinaçion e appetito al abastamiento dela uida natural cosa es a el de querer ser a i al conuigable e ayuntable a su muger } mas si el ma termonio es cosa natural siguese & habere impetum ad sufficientiam vitae : \textbf{ naturale est ei , | quod velit esse animal coniugale . } Sed si coniugium est \\\hline
2.1.7 & Por la qual cosa si natural cosa es al omne de auer inclinaçion e appetito al abastamiento dela uida natural cosa es a el de querer ser a i al conuigable e ayuntable a su muger \textbf{ mas si el ma termonio es cosa natural siguese } que la fornicaçion & quod velit esse animal coniugale . \textbf{ Sed si coniugium est | quid naturale , } sequitur quod fornicatio , \\\hline
2.1.7 & es generalmente de esquiuar alos çibdadanos \textbf{ assi conmo aquella que es contraria ala cosa natural . } La qual fornicaçion e general mente todo vso de luxuria non conueinble tanto & sit uniuersaliter a ciuibus vitanda , \textbf{ tanquam aliquid contrarium rei naturali : } quam videlicet fornicationem , \\\hline
2.1.7 & ¶ Estas cosas dichͣs \textbf{ assi paresçe nasçer vna dubda delas cosas sobredichas . } Ca si el casamiento es al omne natural & His visis , \textbf{ quaedam dubitatio videtur | ex dictis oriri . } Nam si coniugium est homini naturale , \\\hline
2.1.7 & e quiere conteuersse \textbf{ para dar se a contenplaçion e a sçiençia e a obras diuinales . } O non biue commo omne & et vult continere \textbf{ ut vacet contemplationi veritatis et operibus diuinis . } Vel non viuit ut homo , \\\hline
2.1.7 & Et por ende es assy commo bestia . \textbf{ O esto es por que se quiere dar a sçiençias o a obras diuinales . } Et por ende escoge para si uida sobre omne & uel hoc est , \textbf{ quia uult se dare speculationi ueritatis , | et diuinis operibus : } quare eligit sibi uitam supra hominem , \\\hline
2.1.7 & Et pues que assi es los que non quieren casar \textbf{ e se dana mayores bienes } que son los bienes del casamiento & Non nubentes ergo , \textbf{ si dent se potioribus bonis } quam sint bona coniugii , \\\hline
2.1.8 & que sea segunt natura \textbf{ e para que entre el uaron e la muger sea amistança natural conuiene que guarden vno a otro fe e lealtad } assi que non se puedan partir vno de otro . & ad hoc quod coniugium sit secundum naturam , \textbf{ et ad hoc quod inter uxorem et virum sit amicitia naturalis , | oportet quod sibi inuicem seruent fidem , } ita quod ab inuicem non discedant . \\\hline
2.1.8 & Mas esto \textbf{ tantomas parte nesçe alos Reyes e alos prinçipes } quanto mas deue en ellos reluzir la fialdat & suis uxoribus indiuisibiliter absque repudiatione , \textbf{ tanto magis hoc decet reges et principes , } quanto magis in eis relucere debet fidelitas , et ceterae bonitates . \\\hline
2.1.8 & por que naturalmente aman a sus fijos \textbf{ por el amor natural que han con ellos } acresçientase entre ellos amorio natural & qui naturaliter diligunt suam prolem , \textbf{ ex dilectione naturali | quam habent ad ipsam , } augmentatur eorum amicitia naturalis . \\\hline
2.1.9 & e algunas sectas non los iudgan contra razon que vn omne aya muchͣs mugers \textbf{ mas lo que dizela razon derecha es } que cada vno de los çibdadauos & contra dictamen rationis , \textbf{ quod unius et eiusdem viri | simul plures existant uxores . } Sed quod recta ratio dictat quoslibet ciues , \\\hline
2.1.9 & entenebrezcan la uoluntad e çiegun e la razon e el entendemiento \textbf{ si non es cosa conuenible a todos los çibdadanos de dar se } mucho alos deleytes de lux̉ia & et rationem percutiant ; \textbf{ si indecens est omnibus ciuibus } nimis vacare venereis , \\\hline
2.1.9 & e delas obras ciuiles \textbf{ non es cosa conuenible a ellos de auer muchͣs mugieres . } Enpero tanto esto es mas desconuenible alos Reyes & et ab operibus ciuilibus , \textbf{ indecens est eos plures habere coniuges . } Tamen tanto hoc indecens est magis Regibus , et Principibus , \\\hline
2.1.9 & Ca assi commo de parte del uaron \textbf{ es cosa desconuenible de auer muchͣs mugiers } por que por el & ex parte ipsius uxoris . \textbf{ Nam sicut ex parte viri indecens est uxorum pluralitas , } ne propter nimiam operam venereorum \\\hline
2.1.9 & assi conmo dize el philosofo en elix̊ . \textbf{ delas ethicas cosa desconuenible es } a quales si quier çibdadanos & ut vult Philosophus 9 Ethicor’ , \textbf{ indecens est quoscunque ciues plures habere uxores : } quia eas non tanta amicitia diligerent , \\\hline
2.1.9 & çomoles non conuiene \textbf{ por que entre ellos e sus mugers sea mucho mas guardado el amor matermoinal . } ¶ La terçera razon para prouar esto mesmo se toma de parte dela & quia , ne indebite utantur venereis , \textbf{ inter eos et suas coniuges maxime reseruari debet | amor debitus coniugalis . } Tertia via ad inuestigandum hoc idem , \\\hline
2.1.9 & Mas en aquellas aina lias \textbf{ en las quales vna fenbra sola non abasta } para dar conuenible nudͣmiento alos fijos vn & Sed in illis , \textbf{ in quibus sola foemina | non sufficit } ad praestandum filiis debitum nutrimentum , \\\hline
2.1.9 & non puede sofrir las cargas del matermonio \textbf{ nin abonda para dar todas las cosas neçessarias alos fijos } e el nudermiento conuenible . & portare onera matrimonii , \textbf{ nec sufficit ad praestandum filiis omnia necessaria } et debitum nutrimentum : \\\hline
2.1.9 & nin abonda para dar todas las cosas neçessarias alos fijos \textbf{ e el nudermiento conuenible . } Por ende commo quier que por auentura algunas muger & nec sufficit ad praestandum filiis omnia necessaria \textbf{ et debitum nutrimentum : } licet ergo forte aliquae mulieres , \\\hline
2.1.9 & Mas commo los omes en todo tienpo de su iuda ayan menester ayuda de los aueres del padre e dela madre \textbf{ por que a ellos la uianda conuenible non es apareiada conplidamente por natura . } assi se deuen auer el maslo e la fenbraen los omes & et facultatibus parentum , \textbf{ quia eis cibus congruus non sufficienter a natura paratur , } sic se debent habere mas et foemina \\\hline
2.1.10 & non deue estoruar el mandamento \textbf{ nin la ley comun . } Ca segunt el vso comun de razon & vel ex aliqua rationabili causa permissum , \textbf{ communem legem turbare non debet . } Secundum enim commune dictamen rationis detestabile est \\\hline
2.1.10 & nin la ley comun . \textbf{ Ca segunt el vso comun de razon } e de entendemiento cosa de denostares & communem legem turbare non debet . \textbf{ Secundum enim commune dictamen rationis detestabile est } unum virum simul plures habere uxores : \\\hline
2.1.10 & Ca en el mater moino \textbf{ primeramente es guardada la orden natural . } Canatra al cosa es que la fenbrasea subiecta aluaron & esse pluribus viris . \textbf{ In coniugio enim primo reseruatur ordo naturalis : } nam naturale est foeminam \\\hline
2.1.10 & e aguarda dela orden natural \textbf{ e apaz conueinble mas avn es ordenado a generacion de los fijos . } ¶ Lo quarto & ad conseruationem ordinis naturalis , \textbf{ et ad debitam pacem , | sed etiam ordinatur ad procreationem filiorum . } Quarto , sicut ordinatur coniugium \\\hline
2.1.10 & assi commo el casamiento es ordenado a generaçion de los fijos \textbf{ assi es ordenado anudermiento conuenible dellos . } Et por ende non es cosa conueniente & ad filiorum procreationem : \textbf{ sic ordinatur ad eorum debitam nutritionem . } Inconueniens est ergo unam foeminam , \\\hline
2.1.10 & nin seria y conuenible generaçion de los fijos \textbf{ nin les seria dado alos fijos conuenible nudermiento } Et que por esto se tire la orden natural & non erit debita ibi procreatio filiorum , \textbf{ non tribuetur filiis debitum nutrimentum . } Quod autem ex hoc tollatur naturalis ordo , \\\hline
2.1.10 & Enpero que vno obedezca a muchos prinçipantes \textbf{ segunt que son muchos non puede ser segunt orden natural . } Por la qual cosa si cosa de denostares & ut plures sunt , \textbf{ secundum naturalem ordinem esse non potest . Quare et si detestabile est } plures foeminas coniuges esse unius viri , \\\hline
2.1.10 & ¶ Et pues que assi es conuiene \textbf{ quelas mugers de todos los çibdadanos sean pagadas de vn uaron . } Enpero mucho mas conuiene esto alas mugers de los Reyes & simul viris pluribus detestabilius esse debet . \textbf{ Decet ergo coniuges omnium ciuium uno viro esse contentas : } multo magis tamen hoc decet \\\hline
2.1.10 & por que en el casamiento \textbf{ dellos conuiene de guardar la orden natural mas que en otro ninguno . } ¶ Lo segundo esso mismo pue de ser mostrada & coniuges Regum et Principum , \textbf{ quia in eorum coniugio magis quam in alio decet | naturalem ordinem conseruare . } Secundo hoc idem inuestigari potest \\\hline
2.1.10 & por que non sea enbargado el \textbf{ sunconçebemiento sean paragadas de vn marido solo . } Enpero tanto mas conuiene esto & ne impediatur earum foecunditas , \textbf{ uno viro esse contentas . Tanto tamen hoc magis decet Regum , } et Principum coniuges , \\\hline
2.1.10 & por casamiento \textbf{ assi cosa desconuenible es } a qual si quier mug̃r de ser ayuntada a otro uar̃o por fornicaçion ¶ & viro alio copulari , \textbf{ magis detestabile est } alicui viro fornicarie commiscere . \\\hline
2.1.10 & Lo quarto podemos mostrar esso mismo \textbf{ por el nudermiento conueinble de los fuos . } Ca por esso el padre e la madre son soliçitos e acuçiosos & Quarto hoc inuestigare possumus \textbf{ ex filiorum debito nutrimento . } Nam ex hoc parentes solicitantur circa pueros , \\\hline
2.1.10 & que non los prouean diligentemente en la hedat \textbf{ nin en el nudrimiento conuenible . } Mas si vna fenbra casare con muchos uarones & eis prouideant in haereditate \textbf{ et in debito nutrimento . } Sed si una foemina pluribus nubat viris , \\\hline
2.1.10 & nin tan grand acuçia \textbf{ commo deurien en el nudermiento conuenible de sus fijos } nin en proueer los dela hedat ¶ & quare non adhibebunt illam diligentiam \textbf{ quam debent | ut suis filiis debite in nutrimento } et in haereditate prouideant . \\\hline
2.1.11 & La primera razon se declara assi . \textbf{ Ca commo por la orden natural deuamos auer } subiectiuo al padre e ala madre & Prima via sic patet . \textbf{ Nam cum ex naturali ordine debeamus parentibus debitam subiectionem , } et consanguineis debitam reuerentiam , \\\hline
2.1.11 & subiectiuo al padre e ala madre \textbf{ e reuerençia conueible alos parientes } e commo esta reuerençia conueinble non sea guardada & Nam cum ex naturali ordine debeamus parentibus debitam subiectionem , \textbf{ et consanguineis debitam reuerentiam , } cum huiusmodi reuerentia debita non reseruetur \\\hline
2.1.11 & e reuerençia conueible alos parientes \textbf{ e commo esta reuerençia conueinble non sea guardada } entre la muger e el uaron & et consanguineis debitam reuerentiam , \textbf{ cum huiusmodi reuerentia debita non reseruetur } inter virum et uxorem propter ea quae inter eos mutuo sunt agenda , \\\hline
2.1.11 & Onde el philosofo en las politicas \textbf{ mouiendo se con razon natural saca algunas perssonas } que non son conuenibles a mater momo . & Unde et Philosophus 2 Polit’ \textbf{ sola ratione naturali ductus | exceptuat personas aliquas a contractione connubii : } nunquam enim fuit licitum alicui , \\\hline
2.1.11 & parentescoparesca de ser amistança grande \textbf{ Por ende la razon natural dize } que los matermonios non son de fazer & ex ipsa proximitate carnis sufficiens amicitia esse videatur , \textbf{ dictat naturalis } ratio coniugia contrahenda esse inter illos \\\hline
2.1.11 & entre perssonas muy ayuntadas por parentesço . \textbf{ Empero esto tanto mas parte nesçe alos Reyes e alos prinçipes . } Ca quanto son en mayor estado & nimia consanguineitate coniunctis : \textbf{ magis tamen hoc decet Reges , et Principes , } quia quanto sunt in maiori statu \\\hline
2.1.11 & e se ayan de tirar de los cuydados conuenibles \textbf{ e delas obras çiuiles dando se mucho a obras lux̉iosas . } pues que assi es tanto mas esto conuiene alos Re yes & et retrahantur a curis debitis \textbf{ et a ciuilibus operibus . } Tanto hoc ergo magis decet Reges , et Principes , \\\hline
2.1.11 & e enel bien del regno \textbf{ e enlas obras çiuiles . } Et por ende non se deue fazer mater moion & circa salutem regni \textbf{ et circa ciuilia opera non diligenter intendant . } In nimis ergo propinquo gradu consanguineitatis \\\hline
2.1.12 & assi commo paresçe por el philosofo enlas politicas es del maslo e dela fenbra e del uaron e dela mugni . \textbf{ Mas esto non si asi el casamiento non fuesen ordenado a algua conpanna conuenible e natural . } ¶ Et pues que assi es commo deuidamente & est maris et foeminae , viri , et uxoris . \textbf{ Hoc autem non esset , | nisi coniugium ordinaretur } in quandam societatem debitam et naturalem . \\\hline
2.1.12 & por que entre ellos sea paz \textbf{ e conpannia digna en todo casamiento } deue ser esquiuada la grand desigualeza del maridor dela muger . & ut inter eos sit pax et digna societas ; \textbf{ in omni coniugio } nimia imparitas videtur esse vitanda . \\\hline
2.1.13 & mugier fermosura e grandeza . \textbf{ Mas quanto los bienes del alma mayor ment en paresçe } que deue ser demandada en la fenbra tenprança & et magnitudo : \textbf{ sed quantum ad bona animae , } maxime videtur esse quaerendum in foemina \\\hline
2.1.13 & sobredicho sea ordenado \textbf{ aconpania conuenible e abien de paz e aconplimiento deuida . } Enpero avn es ordenado a generaçion conuenible de los fijos & ut in praecedenti capitulo dicebatur , \textbf{ ordinetur ad societatem debitam , et ad esse pacificum , | et ad sufficientiam vitae : } ordinatur etiam nihilominus \\\hline
2.1.13 & aconpania conuenible e abien de paz e aconplimiento deuida . \textbf{ Enpero avn es ordenado a generaçion conuenible de los fijos } e a esquiua la fornicaçion . & et ad sufficientiam vitae : \textbf{ ordinatur etiam nihilominus | ad debitam prolis productionem , } et ad fornicationem vitandam . \\\hline
2.1.13 & que aquellas cosas \textbf{ que dixiemos en el capitulo sobredich̃o . } ¶ Et pues que assi es todas aquellas cosas & et bonum prolis magis directe pertinere videntur ad coniugium , \textbf{ quam ea quae in praecedenti capitulo diximus . } omnia ergo illa , \\\hline
2.1.13 & e alos prinçipes de ser cuydadosos que resplandez cau \textbf{ por fiios grandes e fermosos . } Conuiene a ellos de demandar en las sus mugieres grandeza e fermosura corporal . & et maxime Reges et Principes solicitari , \textbf{ ut polleant filiis pulchris et magnis ; | decet eos } in suis uxoribus quaerere magnitudinem , \\\hline
2.1.13 & por fiios grandes e fermosos . \textbf{ Conuiene a ellos de demandar en las sus mugieres grandeza e fermosura corporal . } Ca paresçe que la fermosura dela muger & decet eos \textbf{ in suis uxoribus quaerere magnitudinem , | et pulchritudinem corporalem : } videtur enim pulchritudo coniugis \\\hline
2.1.13 & en los bienes de fuera \textbf{ e resplandescer en los bienes corporales si non fueren } y los bienes dela uoluntad e del alma deuen demandar en qual manera las mugers & affluere bonis exterioribus , \textbf{ et pollere corporalibus bonis , | nisi adsint } ibi bona mentis et animae , \\\hline
2.1.14 & despues que fuere tomada \textbf{ por ende a estos capitulos sobredichos con razon deuemos ayuntar este capitulo } que se sigue & sciat debite se habere . \textbf{ Ideo praemissis capitulis rationabiliter hoc annectitur , } ut sciamus quomodo regimen nuptiale , \\\hline
2.1.14 & toda la manera del gouernamiento del mundo es fallada en vn omne . \textbf{ Et por ende los pp̃os llaman al omne menor mundo . } Ca assi conmo todo el mundo es gouernado e gado por vn prinçipe & modus regiminis uniuersi reseruatur in uno homine : \textbf{ unde et ab eis homo appellatur minor mundus . } Nam sicut totum uniuersum dirigitur uno Principe , \\\hline
2.1.14 & que plato saluo toda la orden de los çielos \textbf{ en cada vna alma razonable . } Et pues que assi es si el gouernamiento de todo el mundo semeia al gouernamiento & quod Plato totum ordinem caelorum saluauit \textbf{ in qualibet anima rationali . } Si ergo regimen totius uniuersi assimilatur regimini \\\hline
2.1.14 & Conuiene a saber por gouernamientoçiuil . \textbf{ Et por gouernamiento real . } Mas alguon es dicho ser adelantado en sennorio real & duplici regimine regi potest , \textbf{ politico scilicet et regali . } Dicitur autem quis praeesse regali dominio , \\\hline
2.1.14 & que el mismo establesçio \textbf{ mas estonçe es dicho adelantado segunt gouernamiento çiuil quando non es adelantado segunt aluedrio } nin segunt las leyes & quas ipse instistuit . \textbf{ Sed tunc praeest regimine politico , | quando non praeest secundum arbitrium , } nec secundum leges \\\hline
2.1.14 & de aquel que regna \textbf{ e es dicho gouernamiento real . } as quando las leyes non son establesçidas & regimen illud ab illo regnante nomen sumit , \textbf{ et dicitur regale . } Sed cum leges non instituuntur \\\hline
2.1.14 & que es del marido ala muger . \textbf{ Ca el gouernamiento paternal del padre es semeiante al real . } Et el mater moianles semeiante al çiuil . & duo regimina domus , paternale , et coniugale . \textbf{ Nam regimen paternale assimilatur regali : } coniugale vero , politico . \\\hline
2.1.14 & Ca el gouernamiento paternal del padre es semeiante al real . \textbf{ Et el mater moianles semeiante al çiuil . } Ca el uaron es dicho & Nam regimen paternale assimilatur regali : \textbf{ coniugale vero , politico . } Debet enim vir praeesse uxori regimine politico , \\\hline
2.1.14 & con razon ser sennor dela muger \textbf{ por gouernamiento çiuil . } Ca deueen ssennorear a ella & coniugale vero , politico . \textbf{ Debet enim vir praeesse uxori regimine politico , } quia debet ei praeesse \\\hline
2.1.14 & Ca deueen ssennorear a ella \textbf{ segunt leyes çiertas e segunt leyes de matermoion } e segunt las condiçiones e los pleitos del matermoino & quia debet ei praeesse \textbf{ secundum certas leges , | et secundum leges matrimonii , } et secundum conuentiones et pacta . \\\hline
2.1.14 & s segunt aluedrio \textbf{ e segunt gouernamiento real . } Ca entre el padre e el fijo & secundum arbitrium , \textbf{ et secundum regimen regale . } Inter patrem enim et filium \\\hline
2.1.14 & e otro es el real \textbf{ departese el gouernamiento paternal del gouernamiento matermoni al . } Ca veemos que el gouernamiento real es mayor entado e mas natural . & aliud regale , \textbf{ differt regimen coniugale a regimine paternali . } Videmus enim quod regimen regale est magis totale et naturale : \\\hline
2.1.14 & departese el gouernamiento paternal del gouernamiento matermoni al . \textbf{ Ca veemos que el gouernamiento real es mayor entado e mas natural . } Mas el gouernamiento çiuiles mas particular e por election . & differt regimen coniugale a regimine paternali . \textbf{ Videmus enim quod regimen regale est magis totale et naturale : } regimen vero politicum est magis paternale et ex electione . \\\hline
2.1.14 & Ca veemos que el gouernamiento real es mayor entado e mas natural . \textbf{ Mas el gouernamiento çiuiles mas particular e por election . } Ca enssennorear realmente es & Videmus enim quod regimen regale est magis totale et naturale : \textbf{ regimen vero politicum est magis paternale et ex electione . } Nam praeesse regaliter est \\\hline
2.1.14 & Et por ende el sennorio del padre es dicho mas segina natura \textbf{ que el sennorio matrimoinal . } Ca commo quier que el omne sean atal mente & plus secundum naturam , \textbf{ quam coniugale : } quia licet sit \\\hline
2.1.14 & e por elecçion Et \textbf{ pues que assi es el aruernamiento matermonial soes } assi natil commo el del padre al fijo . & hoc est secundum placitum et ex electione . \textbf{ Coniugale ergo regimen non est sic naturale , } ut paternum : \\\hline
2.1.14 & Visto en qual manera se departe el \textbf{ gouernamientoma termoianl del paternal } por la manera de el gouernar & Viso , quomodo differt regimen coniugale \textbf{ a paternali ex modo regendi , } quia unum est magis simpliciter et naturale ; \\\hline
2.1.14 & alas obras de caualleria \textbf{ e alas obras çiuiles alas quales deuen entender } quando fueren criados & Nam filii instruendi sunt ad opera militaria , \textbf{ vel ciuilia , | quibus vacare debeant } cum sint adulti : \\\hline
2.1.14 & que la orden \textbf{ e la razon natural muestra . } a dixiemos de suso que en la casa ay tres gouernamientos departidos & quanto ipsi plus obseruare debent \textbf{ quae dictat ordo et ratio naturalis . } Dicebatur superius \\\hline
2.1.15 & assi comm̃ los çentonicos son barbaros alos italicos e los italicos son barbaros a los ingleses . \textbf{ Enpero aqueles barbaro sinplemente que es estranno assi mesmo } e non entiende assi mismo . & Sicut Theutonici sunt barbari Italicis , et Italici Anglicis . \textbf{ Ille tamen est barbarus simpliciter , | qui seipso est extraneus , } et seipsum non intelligit : \\\hline
2.1.15 & e de ser menguados de razon e de encendemiento . \textbf{ Et pues que assi es de parte de la orden natural paresçe que otra cosa es el gouernamiento del marido ala mug̃r } que del señor al sieruo . & et carere ratione et intellectu . \textbf{ Ex parte igitur ordinis naturalis | patet aliud esse regimen coniugale quam seruile : } et non esse utendum uxoribus tanquam seruis . \\\hline
2.1.16 & assy commo dixiemos en el primero \textbf{ libromenos aprouecha en el neqocio moral e de costunbres . } Et pues que assi es determinar del casamiento & ( ut in primo libro diximus ) \textbf{ circa morale negocium minus proficiunt : } determinare ergo de coniugio , \\\hline
2.1.16 & Et por ende alas palauras \textbf{ e alos ymones generales deuemos añader los sermones particulares . } Ca commo el negoçio moral & quia ignorantia uniuersalium saepe facit particularia ignorare : \textbf{ ipsis tamen uniuersalibus sermonibus sunt particularia addenda , } quia cum negocium morale circa particularia consistat \\\hline
2.1.16 & en tales cosas \textbf{ los smones particulares mas proprouecha } que los generales . Et por ende deuemos determiuar particulariente & ( secundum doctrinam Philosophi 2 Ethicorum ) \textbf{ in talibus particulares sermones plus proficiunt . } Determinandum est ergo particulariter , \\\hline
2.1.16 & dize fue costunbre entre los gentiles \textbf{ de fazer logar speçial de oracion } por el parto delas moças & fuit consuetudo apud gentiles \textbf{ speciale oraculum facere } pro partu iuuencularum , \\\hline
2.1.16 & dize ocho años ante que case . \textbf{ Mas en el uaron ha menester mayor tienpo . } Ca si por todo elt podel cresçer es muy enpesçible a los alos & esse decem et octo annorum . \textbf{ In viro vero plus temporis requiritur . } Nam si per totum tempus augmenti nociuum est masculis uti coniugio , \\\hline
2.1.17 & por ellos el humor natraal \textbf{ por la qual cosa fincan los cuerpos secos . } Otrossi los poros abiertos & Nam tempore calido aperiuntur pori corporis , exalat inde humidum : \textbf{ quare remanent corpora sicca . } Rursus , apertis poris \\\hline
2.1.17 & Otrossi los poros abiertos \textbf{ salle la calentura natural . } La qual sallida fincan los cuerpos de dentro frios & Rursus , apertis poris \textbf{ exalat naturalis calor , } quo exalante corpora intrinsecus \\\hline
2.1.17 & en que vienta el cierço meior muelle la uianda \textbf{ por la calentura natural se ençierran de dentro } por el frio qual çerca de fuera & quod tempore frigido flante borea melius digerit , \textbf{ quia calor eius interius } propter frigus circunstans \\\hline
2.1.17 & mesmo se toma del danno delos uarones \textbf{ ca los uarones mayor danno resçibe } si vsan del ayuntamiento delas mugieres & sumitur ex laesione filiorum . \textbf{ Nam viri magis laeduntur , } si utantur coniugali copula \\\hline
2.1.17 & que en el tienpo frio quando vienta çierco \textbf{ Ca en el tienpo frio del çierço por̉ la } calenturanatraales mas guardada e tornada & Tempore enim boreali et frigido \textbf{ quia calor naturalis magis reseruatur interius , } plus possumus conuertere de alimento . \\\hline
2.1.17 & e alos prinçipes \textbf{ quanto mas les conuiene aellos de auer los fijos grandes e esforcados de cuerpo } euedes saber & tanto tamen hoc magis decet Reges et Principes , \textbf{ quanto decet eos elegantiores habere filios . } Mulierum autem mores \\\hline
2.1.18 & en general \textbf{ non ouiemos cuydado de fazer capitulo espeçial delas costunbres delas mugers } mas diemos lo a entender & ubi uniuersaliter tractabamus de moribus , \textbf{ non curauimus speciale capitulum facere de moribus mulierum : } sed supposuimus coniecturandum esse de huiusmodi moribus \\\hline
2.1.18 & que tales son las costunbres delas mugers \textbf{ commo las costunbres delons moços . } Enpero por que en este segundo libro el gouernamiento de los casados & sed supposuimus coniecturandum esse de huiusmodi moribus \textbf{ ex moribus puerorum . } Verumtamen quia in hoc secundo libro de regimine coniugum specialem requirit tractatum , \\\hline
2.1.18 & assi commo la grandeza e la fermosura \textbf{ e las otras cosas tales son bienes menguadᷤ Et pues que assi es las mugers } en la mayor parte & et caetera talia , \textbf{ imperfecta bona sunt . | Mulieres ergo } ut plurimum uel \\\hline
2.1.18 & e por ende luego que veen a algunos \textbf{ sofrir cosas duras han piadat sobre ellos ¶ } Lo terçero deuemos penssar en las mugers & ideo statim miserentur , \textbf{ cum vident aliquos dura pati . } Tertio considerandum est in mulieribus , \\\hline
2.1.18 & sigunan las conplisiones del cuerto \textbf{ assy commo las mugers han el cuerpo blando e mouible } e non estable & ut plurimum sequatur complexiones corporis : \textbf{ sicut mulieres habent corpus molle et instabile , } sic voluntate et desiderio sunt instabiles et mobiles . \\\hline
2.1.19 & e dezir \textbf{ assi en general del gouernamiento matermonial del marido ala mugni } que es departido del gouernamiento paternal & Sed quia non sufficit sic \textbf{ in uniuersali tractare de regimine coniugali , } dicendo ipsum esse differens \\\hline
2.1.19 & e fallesçer mas ligera mente . \textbf{ por la qual cosa assi commo es dicho de suso en el capitulo sobredicho conmolas mugers } comunalmente sean destenpradas e parleras & circa ea in quibus esse contingit facilior casus . \textbf{ Quare cum mulieres | ( ut in praecedenti capitulo dicebatur ) } communiter sint intemperatae , garrulae , et instabiles ; \\\hline
2.1.19 & que van e demuestran desonestad . \textbf{ Ca non abasta que el fijo ageno non he de la h̃edat } de aquel que non es su padre . & quae videntur inhonestatem protendere : \textbf{ non enim sufficit } ut alius filius non succedat in haereditatem , \\\hline
2.1.19 & e dar les castigos conuenibles \textbf{ por que puedan resplandesçer en las bondades sobredichͣ̃s . } Mas los que abondan en nobleza & et debitas cautelas adhibere , \textbf{ ut polleant bonitatibus supradictis . } Abundantes vero nobilitate , \\\hline
2.1.19 & e la iudgan \textbf{ por amonestamientos conueninbles a las uirtudes e bondades sobredich̃ͣs¶ } Visto en qual manera se deua gouernar la muger & et inducentes eam \textbf{ per monitiones debitas | ad bonitates praehabitas . } Viso , coniugem sic regendam esse , \\\hline
2.1.19 & deuen ser gouernadas las mugieres \textbf{ por que resplandescan por las seys bondades sobredichͣs . } Conuiene a saber & Tali ergo regimine regendae sunt coniuges , \textbf{ ut polleant praedictis sex bonitatibus , } videlicet , \\\hline
2.1.20 & si se diere mucho al ayuntamiento dela muger . \textbf{ por que la uirtud engendradora es muy corrupta } assi commo dize el philosofo & si nimis det operam copulae coniugali , \textbf{ quia vis generatiua est nimis corrupta , } et ( ut vult Philos’ 3 Ethic’ ) \\\hline
2.1.20 & en el terçer libro delas ethicas \textbf{ que el desseo dela cobdiçia carnal es tal que se non farta . } Por la qual cosa en la may & et ( ut vult Philos’ 3 Ethic’ ) \textbf{ insatiabilis est concupiscentiae appetitus : } quare ut plurimum , \\\hline
2.1.20 & Ende el meollo e la uista \textbf{ e los otros mienbros nobles se ensiaqueçen } por sobeiedunbre de ayuntamiento del uaron con la mug & et visus , \textbf{ et alia membra nobilia debilitantur } ex superflua copula . \\\hline
2.1.20 & por sobeiedunbre de ayuntamiento del uaron con la mug \textbf{ ¶Lo segundo el vso sobeio del ayuntamiento del uaron } con la & ex superflua copula . \textbf{ Secundo superfluus usus } non solum corpus debilitat , \\\hline
2.1.20 & Onde es prouado de suso por la autoridat del philosofo \textbf{ que tales cosas commo estas sin razon çiega el alma } e non sienten & unde et supra per auctoritatem Philosophi probabatur , \textbf{ quod talia , } si vehementia sint rationem percutiunt . \\\hline
2.1.20 & Ca qual si quier vso de \textbf{ ayuntamientocarna la biua al ome adelante a mayo ruso . } Et quanto el ome mas vsa dela & Nam quilibet usus carnalis copulae \textbf{ incitat ad ulteriorem usum : } et quanto quis plus ea utitur , \\\hline
2.1.20 & Pues que assi es conuiene a todos los çibdadanos de vsar \textbf{ tenpradamente e mesuradamente del ayuntamiento matermoinal . } Et tanto mas conuiene esto alos Reyes & Decet ergo omnes ciues \textbf{ uti moderate coniugali copula , } et tanto magis hoc decet Reges , et Principes , \\\hline
2.1.20 & e en logar conuenible \textbf{ e en manera conueinble . } Ca son algunos t pos desconueinbles para las obras sobredichͣs & ut fiant tempore debito , \textbf{ loco conuenienti , et modo congruo . } Sunt enim aliqua tempora incongrua praedictis operibus . \\\hline
2.1.20 & de guardar tienpo conuenible \textbf{ e avn assi les conuiene de guardar logar conuenible en manera conuenible } por que sea entre los casados & Sic etiam obseruandus est locus congruus \textbf{ et modus conueniens , } ut sit inter coniuges \\\hline
2.1.20 & non solamente amestança de delectaçion \textbf{ mas avn amistança honesta . } ¶ Visto en qual manera conuiene alos uarons de vlar labiamente & ut sit inter coniuges \textbf{ non solum amicitia delectabilis , sed honesta . Viso , } quomodo decet viros suis uxoribus moderate uti et discrete : \\\hline
2.1.20 & muy ayuntada a su marido \textbf{ la honrra queda el marido a su muger } tornase en la perssona del marido . & persona valde coniuncta , \textbf{ honor , qui uxori exhibetur , } redundat in persona ipsius viri . \\\hline
2.1.20 & mas estonçe es dicha la conuersaçion \textbf{ e la uida conuenible e buena entre el marido e la muger } si se mostraren sennales conuenibles de amistança e de amor . & qualiter cum eis debeant conuersari . \textbf{ Tunc autem viri ad uxorem est conuersatio congrua , } si ei ostendat debita signa amicitiae , \\\hline
2.1.20 & e la uida conuenible e buena entre el marido e la muger \textbf{ si se mostraren sennales conuenibles de amistança e de amor . } Et si el marido enssencare ala muger & Tunc autem viri ad uxorem est conuersatio congrua , \textbf{ si ei ostendat debita signa amicitiae , } et si eas per debitas monitiones instruat . \\\hline
2.1.20 & por conuenibles castigos . \textbf{ Mas declarar quales son las señales conueinbles dela mistança } e quales son las moniçonnes e castigos conuenibles & et si eas per debitas monitiones instruat . \textbf{ Declarare autem quae sunt signa amicitiae debita , } et quae sunt monitiones congruae , \\\hline
2.1.20 & Ca assi deuemos beuir con las mugers \textbf{ que les deuemos mostrar muchͣs señales de amistança } si fueren humildosas & Nam sic conuersandum est cum uxoribus , \textbf{ quod plura signa amicitiae ostendenda sunt eis , } si sint humiles , \\\hline
2.1.20 & Et para correction e castigamiento delas sabias \textbf{ abastan palauras amorosas e blandas . } Mas alas locas es menester de dar denuesto mas apero de & Nam prudentibus ad correptionem leuia verba , \textbf{ et blanda sufficiunt : } fatuis vero est asperior increpatio adhibenda . \\\hline
2.1.20 & e catadas las condiconnes delas perssonas mostrar a sus \textbf{ mugersseñales conueibles de amor } e enssennarlas & et inspectis conditionibus personarum , \textbf{ suis uxoribus ostendere debita amicitiae signa , } et eas ( ut expedit ) \\\hline
2.1.21 & deue la castigar a obras honestas \textbf{ e a fechos uirtuosos . } Conuiene avn que todos los uarones tengan mientes & eam dirigendo ad actiones honestas , \textbf{ et ad opera virtuosa : } expedit quoslibet viros in iis , \\\hline
2.1.21 & e qual desconueinble ca las casadas nunca son uirtuosas \textbf{ nin dessean cosas conuenibles . } Mas quanto mal uenga ala çibdat & Nam nunquam uxores virtuosae existunt , \textbf{ nisi appetant illicita fugere . } Quantum autem malum incurrat ciuitati et regno \\\hline
2.1.21 & e desauentraados en la meytad de su fazienda \textbf{ por que consienten a sus mugieres cosas desconuenibles ¶ Et pues que assi es por que la casa del prinçipe } e avn de cada vn & ait , eos esse infelices secundum dimidium , \textbf{ eo quod uxoribus suis illicita permittebant . | Ne ergo domus principis } vel cuiuscunque ciuis \\\hline
2.1.21 & quanto ꝑtenesçe alo presente esta en dos cosas \textbf{ ca el vn conponimiento delas mugerses enfinto para paresçer . } El otro non es & in duobus consistere , \textbf{ quorum unus potest dici fictitius , } alius non fictitius . \\\hline
2.1.21 & Mas otro conponimiento ay que non es infinto \textbf{ e este esta en uestiduras conueinbles . } Et en los otros conponimientos del cuerpo & Alius autem est ornatus non fictitius , \textbf{ qui consistit in debitis indumentis , } et in aliis ornamentis , \\\hline
2.1.21 & mugerssegunt sus estados e en vestiduras conuenibles \textbf{ e en los otros conponimientos conueinbles . } Onde ualerio maximo alaba alos çibdadanos de Roma & in debitis vestimentis , \textbf{ et in aliis ornamentis , | debite prouidere . } Unde et Valerius Maximus ciues Romanos commendat , \\\hline
2.1.21 & e en los otros conponimientos conueinbles . \textbf{ Onde ualerio maximo alaba alos çibdadanos de Roma } por que proue en honrradamente a sus mugers & debite prouidere . \textbf{ Unde et Valerius Maximus ciues Romanos commendat , } qui suis uxoribus in pulchris indumentis \\\hline
2.1.21 & de uestiduras fermosas \textbf{ e de los otros conponimientos honrrados . } ¶ Et pues que assi es en esta manera deuemos sentir & qui suis uxoribus in pulchris indumentis \textbf{ et in aliis ornamentis debite prouidebant : } sic ergo censendum est de ornatu . \\\hline
2.1.21 & Ca conuiene ala muger del cauallero de ser mas honrrada de uestiduras \textbf{ que ala muger del çibdadano sinple . } Et avn ala mugni del Rey o del prinçipe & magis esse ornatam vestibus , \textbf{ quam uxorem ciuis simplicis . } Adhuc etiam uxorem Principis , \\\hline
2.1.21 & por que non demanden \textbf{ con grand acuçia mayores conponimientos del su cuerpo } de quanto les conuiene . & circa ornatum corporis esse simplices , \textbf{ ut non nimia solicitudine ornamenta requirant . } Nam et si foemina \\\hline
2.1.21 & por que por vileza delas uestidas \textbf{ se leunaten en orgullo e en alabança de ssi mismas . } assi que por el fallesçimiento de uestiduras & Sexto , ne ex vilitate habitus sint superstitiosae , \textbf{ ut ex ipso defectu vestium laudem et gloriam cupiant . } Multi virorum in hoc videntur delinquere , \\\hline
2.1.22 & que han de auer de su casa \textbf{ e avn de ser enbargados enlas obras çiuiles ¶ } pues que assi es conuiene a todos los çibdadanos & retrahi a debitis curis , \textbf{ et a ciuilibus operibus . } Decet ergo omnes ciues \\\hline
2.1.22 & si los Reyes fueren en grand angostura de su coraçon \textbf{ e si fueren enbargados en la cura conueinble del regno ¶ } La segunda razon para prouar esto mismo se toma desto & si Reges sint in anxietate cordis , \textbf{ et retrahantur a debita cura regni . } Secunda via ad inuestigandum hoc idem , \\\hline
2.1.22 & que sus maridos se acallonan sin razon \textbf{ e que sin su culpa sospecha los maridos mal dellas } la qual cosa fazen los maridos muy çelosos . & Nam cum uidetur uxoribus \textbf{ quod sine causa calumnientur , et quod earum uiri sine culpa suspicentur de ipsis mala , } quod faciunt uiri zelotypi ; \\\hline
2.1.22 & ca da vno deue auer cura \textbf{ e cuydado conuenible de su muger } e deue auer acuçia conuenible de su casa . & debet debitam curam , \textbf{ et debitam diligentiam adhibere . } Sic enim decet uirum quemlibet \\\hline
2.1.22 & e cuydado conuenible de su muger \textbf{ e deue auer acuçia conuenible de su casa . } Ca assi conuiene a cada vn marido de auer çelo ordenado de su mugni & debet debitam curam , \textbf{ et debitam diligentiam adhibere . } Sic enim decet uirum quemlibet \\\hline
2.1.22 & Ca assi conuiene a cada vn marido de auer çelo ordenado de su mugni \textbf{ por que sea entre ellos amistança natural delectable e honesta } L consseio delas mugers & erga suam coniugem ornatum habere zelum , \textbf{ ut sit inter eos amicitia naturalis delectabilis , et honesta . } Consilium mulierum , \\\hline
2.1.23 & fallesçedel conplimiento de uaron \textbf{ assi la muger ha consseio flaco por que } fallesçe de ualençia de uaron . & quia deficit a perfectione viri : \textbf{ sic etiam foemina habet inualidum consilium , | quia habet complexionem inualidam } et deficit a valitudine viri . \\\hline
2.1.23 & mas es proporçionado e egualado al alma . \textbf{ Et el alma que esta en tal cuerpo meior vsa de sus obras propiçias } e mas & propter quod magis obsequitur ei : \textbf{ et anima existens in tali corpore , | liberius utitur operibus propriis , } et expeditus habet rationis usum . \\\hline
2.1.23 & Ca assi conmo dize el philosofo \textbf{ en el libro delas aian lias las ainalias menores e mas flacas . } mas ayna vienen a su conplimienta . & Nam , ut vult Philosophus in de Animalibus , \textbf{ omnia minora et debiliora } citius veniunt ad suum complementum . \\\hline
2.1.23 & Et por ende estando todas las \textbf{ otrascondiconnes eguales del omne et dela muger } si alguno quisiesse obrar adesora & citius venit ad suum complementum . \textbf{ Ceteris ergo paribus } si quis statim operari deberet , \\\hline
2.1.23 & Et por ende dize el prouerbio \textbf{ que yerba mala mas ayna cresçe } por que la natura ha poco cuydado della & unde et prouerbialiter dicitur , \textbf{ quod mala herba cito crescit } quia natura de ea modicum curans , \\\hline
2.1.23 & Por la qual cosa \textbf{ commo el alma sigua ala conplission del cuerpo . } A assi commo el cuerpo dela muzer & ad corpus cuius habet \textbf{ esse perfectum quam vir . Quare cum anima sequatur complexionem corporis , } sicut ipsum corpus muliebre \\\hline
2.1.24 & esto mismo se toma dela cobdiçia del loor . \textbf{ Ca en los capitulos sobredichos es dicho } que las mugieres mucho & sumitur ex appetitu laudis . \textbf{ Dicebatur enim in praecedentibus , } quod mulieres nimis appetunt laudem et gloriam . \\\hline
2.1.24 & si quisieren ser constantes e firmes \textbf{ e vençer estos appetitos natraales e estas iclinaçiones . } Ca conmoquier que sea cosa & esse constantes , \textbf{ et vincere huiusmodi impetus et inclinationes . } Nam licet sit difficile \\\hline
2.1.24 & quales obras son ordenadas las casadas . \textbf{ por que si es cosa desconuenible a ellas de ser uagarosas } assy commo dicho es de & ad quae opera sint coniuges ordinandae : \textbf{ si enim indecens est | eas esse ociosas , } ut superius dicebatur , \\\hline
2.2.1 & Enpero assi commo dize el philosofo en el primero delas politicas en el gouernamiento dela \textbf{ casa mayor deue ser el cuydado de los omes } que non delas possessiones & ut dicitur primo Politicorum , \textbf{ oeconomiae amplior est solicitudo } circa homines \\\hline
2.2.1 & por aquello que conuiene entre ellos \textbf{ de ser amistança natural ¶ } La primera razon se praeua assi . & ex eo quod inter eos debet \textbf{ esse amicitia naturalis . } Prima via sic patet . \\\hline
2.2.1 & que ha sennorio sobre otro de ser muy cuydadoso en qual manera \textbf{ por ayudas conueinbles enbie su uirtud } e ayude a sus subditos & et praeeminentem solicitari , \textbf{ quomodo per debita auxilia influat , } et subueniat suis subiectis , \\\hline
2.2.2 & de aquellos que son prinçipes en el regno . \textbf{ Ca bien commo la sanidat na tra᷑al del cuerpo } desçende dela sanidat de todos los mienbros e mayormente dela samdat del coraçon & ex bonitate principantium in ipso . \textbf{ Nam sicut sanitas corporis naturalis dependet | ex sanitate omnium membrorum , } et maxime ex sanitate cordis et membrorum principalium , \\\hline
2.2.2 & pues que assi es prouechosa cosa es a todo el regno \textbf{ de auer bueon sçibdadanos . } Mas mas prouechosa cosa es de auer bueons prinçipes & et dominantur in regno . \textbf{ Utile est ergo toti regno habere bonos ciues , } sed utilius est habere bonos principantes , \\\hline
2.2.2 & Mas mas prouechosa cosa es de auer bueons prinçipes \textbf{ por que alos prinçipes parte nesçe de gouernar e de garalo sots . } pues que assi es tanto mas conuiene alos Reyes & sed utilius est habere bonos principantes , \textbf{ eo quod principantis sit alios regere et gubernare : } tanto ergo magis decet Reges et Principes solicitari \\\hline
2.2.3 & e por que en pos el tractado del gouernamiento dela muger auemos \textbf{ atrattardel gouernamiento paternal . } Conuienne de uer onde toma comienço el gouernamiento paternal . & Quia post tractatum de regimine coniugis , \textbf{ determinandum est de regimine paternali : } videndum est , \\\hline
2.2.3 & atrattardel gouernamiento paternal . \textbf{ Conuienne de uer onde toma comienço el gouernamiento paternal . } Et por qual gduernamiento son de gouernar los fijos ¶ & determinandum est de regimine paternali : \textbf{ videndum est , | unde sumit originem regimen paternum , } et quo regimine regendi sunt filii . \\\hline
2.2.3 & que demandan las leyes del casamiento \textbf{ e del matermoino o assi commo demandan los abenemientos conueibles e honestos } que son puestas entre el uaron e la muger ¶ & ut requirunt leges matrimonii , \textbf{ et ut potius per pacta debita et honesta , } quae interueniunt inter virum et uxorem . \\\hline
2.2.3 & que son puestas entre el uaron e la muger ¶ \textbf{ Mas el gouernamiento del padre es semeiante al gouerna mieto real . } Ca el padre & quae interueniunt inter virum et uxorem . \textbf{ Regimen vero paternale | assimilatur regimini regali . } Nam filiis praeest pater ex arbitrio , \\\hline
2.2.3 & sea por aluedrio e sea por el bien de los fijos . \textbf{ Este gouernamiento non es semeiante al gouernamiento çiuil mas al Real . } Onde el philosofo en el primo libro delas politicas & et sit propter bonum ipsorum filiorum ; \textbf{ huiusmodi regimen non assimilatur regimini politico , sed regali . } Unde et Philosophus 1 Politicorum ait , \\\hline
2.2.3 & por que se gouierna toda la conpanna de la casa . \textbf{ Este es semeiante al gouernamiento enssennoreador . } Mas estas cosas assi mostradas & vel regimen quo regitur familia caetera , \textbf{ assimilatur regimini dominatiuo . } His ergo sic ostensis , \\\hline
2.2.4 & assi commo de sieruos \textbf{ ssi commo es dicho en el capitulo sobredich̃ . } El gouernamiento patrinal toma comienço del amor Et & tanquam seruis . \textbf{ Dicebatur in praecedenti capitulo , } paternale regimen sumere originem ex amore . \\\hline
2.2.4 & ssi commo es dicho en el capitulo sobredich̃ . \textbf{ El gouernamiento patrinal toma comienço del amor Et } pues que assi es deuemos uer & Dicebatur in praecedenti capitulo , \textbf{ paternale regimen sumere originem ex amore . } Videndum est igitur quantus sit amor patrum ad filios , \\\hline
2.2.4 & que pueda conosçer \textbf{ de qual madre salio o de qual padre } Mas el padre e la madre & ut possit cognoscere \textbf{ a qua matre procedit , | vel a quo patre . } Parentes tamen statim cognitionem habent de ipsa prole : \\\hline
2.2.4 & conplidamente \textbf{ que parte nesçe alos padres de ser muy cuydadosos del gouernamiento de sus fijos . } Ca cada vno deue ser cuy dados o de aquellas cosas & sufficienter arguere possumus , \textbf{ quod ad parentes spectat solicitari | circa regimen filiorum , } quia quilibet solicitus esse debet \\\hline
2.2.4 & e son subiectas alas cosas de suso \textbf{ et las cosas baxas alas cosas altas } e non las de suso & Inferiora vero reuerentur , \textbf{ et sunt subiecta superioribus , } non econuerso . \\\hline
2.2.5 & que son de fe non se pueden prouar \textbf{ por razon prouechosa cosa es } que en aquella hedat & et ea quae sunt fidei \textbf{ ratione comprehendi non possunt : } utile est ut in illa aetate proponantur \\\hline
2.2.5 & que la sabiduria de dios \textbf{ e la su auctoridat sobrepiua toda sotileza de engennio humanal . por la qual cosa mas prouechosa cosa es de creer } sinplemente la auctoridat de dios & diuinam prudentiam et eius auctoritatem , \textbf{ omnem perspicaciam humani generis superare . | Quare utilius auctoritati diuinae simpliciter creditur , } quam acquiescatur rationibus \\\hline
2.2.5 & que todas las otras leyes . \textbf{ Ca la ley yana sola es quita de todo ensuziamiento de herror . } Mas en todas las otras leyes & caeteras leges . \textbf{ Sola enim christiana lex est immunis | ab omni errorum contagio : } in caeteris aliis legibus \\\hline
2.2.5 & que contienen en si muchͣs fabliellas \textbf{ enlas quales leyes son muchͣs cosas falssas e de escarneçer } e estas son allegadas al coraçon & et apologos \textbf{ idest multa fabulatoria et derisoria , | plus possunt propter consuetudinem , } et sunt sic applicabiles animo , \\\hline
2.2.5 & assi que les sea dicho \textbf{ que es vn dios poderoso todo criador de todas las cosas } e que es padre e fijo e spunsanto & ut quod eis dicatur , \textbf{ quod unus est Deus omnipotens creator omnium , } qui est pater et filius et spiritus sanctus . \\\hline
2.2.5 & dando razon de todos nuestros fechos . \textbf{ assi que aquellos que bien fezieron yran ala uida perdurable . } Mas aquellos que fizieron mal yran al fuegon del infierno ¶ Et pues que assi es todos los çibdadanos deuen ser acuçiosos de sus fiios & reddituri de factis propriis rationem . \textbf{ Ita quod qui bona egerunt , | ibunt in vitam aeternam : } qui vero mala , in ignem aeternum . \\\hline
2.2.6 & Mas luego en su ninnez son de enssenñar los moços \textbf{ por que dexen la locania e siguna bueans costunbres } Et nos podemos esto mostrar & sed ab ipsa infantia instruendi sunt pueri , \textbf{ ut relinquentes lasciuiam sequantur bonos mores . } Possumus autem quadruplici via venari , \\\hline
2.2.6 & estonçe son de enssennar en bueans costunbres \textbf{ e deuen les ser fechos amonestamientos conuenibles . } ¶ La segunda razon para prouar esto mismo se toma del fallesçimiento dela razon & sunt instruendi ad bonos mores , \textbf{ et debent eis fieri monitiones debitae . } Secunda via ad inuestigandum hoc idem , \\\hline
2.2.6 & rectorica en la hedat dela mançebia \textbf{ mayormente son los omes orgullosos et loçanos } e siguen sus passiones e sus desseos . & in iuuenili aetate \textbf{ maxime sunt homines lasciui } et passionum insecutores : \\\hline
2.2.6 & en esta manera nos muestra ser endereçados a bueans costunbres \textbf{ en la qual manera se enderesça la piertega tuerta . } Ca aquel que quiere endereçar la pierte & nos dirigere ad bonos mores , \textbf{ quo dirigitur virga tortuosa . } Volens enim virgam tortuosam rectificare , \\\hline
2.2.6 & inclina la mucho \textbf{ ala parte contraria la qual assi inclinada torna al medio e aser derecha . } En essa misma manera & inclinat eam \textbf{ ad partem contrariam valde , | quae sic inclinata redit } ad medium et ad rectitudinem . \\\hline
2.2.7 & por que por ellas puedan ser mas sabios \textbf{ e pueda mas las cosas desconueinbles . } Empero paresçe que alguons pueden auer escusacion conueinble & ut per eas prudentiores effecti , \textbf{ magis possent illicita praecauere : } videntur tamen aliqui \\\hline
2.2.7 & Et estos tales son los pobres \textbf{ que non han las cosas neçessarias para la uida } e estos son tirados delas sciençias libales & Huiusmodi autem sunt pauperes , \textbf{ non habentes necessaria vitae : } qui si retrahantur a liberalibus disciplinis , \\\hline
2.2.7 & que en su moçedat \textbf{ luego sean puestos alas siete artes liberales . } Mas que luego enla moçedat de una trabaiar & ut etiam ab ipsa infantia \textbf{ tradantur liberalibus disciplinis . } Quod autem studium literarum sit \\\hline
2.2.7 & lenageiaie si non fuere acostunbrado ael de su moçedat . \textbf{ Ca aquellos que se mudan en hedat acabada a tierras luengas do los legunaies son departidos del } lenguaie de su padre e de su made avn & nisi sit in eo in ipsa infantia assuefactus ; \textbf{ qui enim in aetate perfecta transfert se | ad partes longinquas } ubi idiomata differunt a materno , \\\hline
2.2.7 & alas quales somos acostunbrados en \textbf{ nr̃que nin uero ennr̃a moçedat . } por que cada vno es mas cuydadoso e mas acuçioso çerca aquellas cosas quel plazen & magis placeant illa opera , \textbf{ ad quae sumus ab ipsa infantia assueti : } quia quilibet est magis intentus , \\\hline
2.2.7 & por que nos podamos ser mas cuydadosos e mas acuçiosos \textbf{ cerca el estudio delas letris deuemos trabaiar en el tpon dela moçedat en las sciençias liberales . } ¶ La terçera razon se toma de parte dela perfecçion & et feruentes \textbf{ circa studium literale , | ab infantia insudandum est literalibus disciplinis . } Tertia via sumitur \\\hline
2.2.7 & e conosçer las uaturas delas cosas . \textbf{ Enpero el ome comneco de su nasçimiento es mal despuesto } e mal ordenado & ad intelligendum et ad cognoscendum naturas rerum : \textbf{ homo tamen a sui natiuitate est } male dispositus \\\hline
2.2.7 & dize \textbf{ que el alma mayor tienpo pone } enł non saber & Unde et Philosophus in primo de anima vult , \textbf{ quod anima plus temporis apponat in ignorantia , } quam in scientia . Per multum enim temporis \\\hline
2.2.7 & deuenlos luego poner en su moçedat alas letros \textbf{ e alas sçiençias liberales . } Ca assi conmodicho es de suso ninguno non es dich̃ sennor naturalmente & et peruenire ad aliquam perfectionem scientiae , \textbf{ ab ipsa infantia eos tradere literalibus disciplinis . } Nam ( ut superius dicebatur ) \\\hline
2.2.7 & e por entendimiento de ligero se tornaran en tirannos \textbf{ por que non auran cuydado delas obras uirtuosas . } Mas preciarian mucho las riquezas & de facili conuertitur in tyrannum : \textbf{ quia non curabit de operibus virtutum , } sed appretiabitur nummismata , \\\hline
2.2.8 & ¶ \textbf{ a auctoridat antigua prueua } e muestra & Septem scientias esse famosas apud antiquos , \textbf{ antiqua auctoritas protestatur . } Huiusmodi autem sunt , \\\hline
2.2.8 & e delos nobles eran enssennados en ella ¶ \textbf{ La segunda sciencia liberal es dicha logica . } la qual sçiençia & quia filii liberorum et nobilium instruebantur in illa . \textbf{ Secunda liberalis scientia } dicitur esse dialectica , \\\hline
2.2.8 & que non erremos en argumentar e en razonar ¶ \textbf{ La tercera sçiençia liberales dicha rectorica . } Mas la rectorica & ne erretur in arguendo . \textbf{ Tertia scientia liberalis | dicitur esse Rhetorica . } Est autem Rhetorica , \\\hline
2.2.8 & en las sçiençias especulatinas \textbf{ assi son de fazer razones gruessas en las sciençias morales } que tractan delas obras & in scientiis naturalibus \textbf{ et in aliis scientiis speculabilibus , sic fiendae sunt rationes grossae | in scientiis moralibus , } quae tractant de agibilibus . \\\hline
2.2.8 & que es assi commo vna logica gruessa \textbf{ que nos mostrasse manera gruessa e figural } para argumentar gruessa mente . & docens modum arguendi \textbf{ grossum et figuralem . } Haec autem necessaria est filiis liberorum et nobilium , \\\hline
2.2.8 & e mayormente alos fijos de los Reyes e de los prinçipes . \textbf{ Ca a estos buenos parte nesçe de beuir entre las gentes } e de enssennorear al pueblo el qual pueblo non puede entender & et maxime Regum , et Principum : \textbf{ quia horum est conuersari } inter gentes et dominari populo , \\\hline
2.2.8 & que los moços non pueden sofrir ninguna cosa de tristeza . \textbf{ Por la qual cosa si les son otorgadas algunas cosas delectabłs conuiene } que les otorguen cosas & quia pueri nihil tristabile sustinere possunt : \textbf{ quare si debent eis aliqua delectabilia concedi , | dignum est quod ordinentur } ad delectationes innocuas : \\\hline
2.2.8 & que son \textbf{ delectaconnes conuenibles e sin danno . } Et mayormente esto conuiene alos fijos delos rreys & inter delectationes musicales , \textbf{ quae sunt licitae et innocuae . } Maxime autem hoc decens est filiis liberorum et nobilium , \\\hline
2.2.8 & que non se entremetiendo de lauores \textbf{ nin de otras artes mecanicas estarian oçiosos e uagarosos } si non estudiassen en las sçiençias liberales & qui non vacantes moechanicis artibus , \textbf{ remanent ociosi , } nisi studerent literalibus disciplinis , \\\hline
2.2.8 & assi commo la hetica \textbf{ que es del gouernamiento del omne en ssi mismo . } Et la y conomica & Adhuc quaedam morales scientiae , \textbf{ ut Ethica , quae est de regimine sui , } et Oeconomica , quae est de regimine familiae : \\\hline
2.2.8 & e quasieren gouernar los otro \textbf{ o mayormente se deuen trabaiar çerca destas sçiençias morales . } Et avn son o trissçiençias subal ternadas e subiectas destas & et velint alios regere et gubernare , \textbf{ maxime circa has debent insistere . } Sunt autem et aliae scientiae subalternatae et suppositae istis : \\\hline
2.2.8 & assi commo la ꝑ̊ perspectiua \textbf{ que es dela linna visual e del iuso . } La qual sçiençia es sola geometera . & ut perspectiua , \textbf{ quae est de visu , } est sub Geometria . \\\hline
2.2.8 & La qual sçiençia es sola geometera . \textbf{ Et la sçiençia dela fisica es sola ph̃ia natural . } Et las leyes e los derechos & est sub Geometria . \textbf{ Medicina vero est | sub naturali Philosophia . } Leges et iura , \\\hline
2.2.8 & que todos los legistas son \textbf{ assi commo vnos nesçios politicos . } Ca assi commo los legos & omnes legistae sunt \textbf{ quasi quidam idiotae politici . } Nam sicut laici et vulgares , \\\hline
2.2.8 & e non demostrando las \textbf{ por razon por ende pueden ser llamados nesçios politicos . } Et desto puede parescer en qual manera & dicunt narratiue et sine ratione , \textbf{ appellari possunt idiotae politici . } Ex hoc autem patere potest \\\hline
2.2.8 & Et pues que assi es en tanto les conuiene aellos de sablas o tris sçiençias \textbf{ en quanto siruen ala ph̃ia moral . } Et pues que assi es conuiene les a ellos de saber la guamatica & eos scire , \textbf{ inquantum deseruiunt morali negocio . } Decet igitur eos scire grammaticam , \\\hline
2.2.8 & conmolas otras se entienden mas liga mente . \textbf{ ante digo que si nunca fuessen las otras sçiençias la guamatica siruiria ala sçiençia moral . } Et por ende conuiene alos Reyes & quam aliae facilius traduntur . \textbf{ Imo si nunquam grammatica deseruiret negocio morali , } decet Reges , \\\hline
2.2.9 & razon qual enderesçe e qual tegle . \textbf{ assi el moço ha menester maestro e ayo . } Ca assi commo la razon & ratione dirigente et regulante : \textbf{ sic puer indiget magistro et paedagogo . } Quare sicut ratio semper deprecatur ad optima , \\\hline
2.2.9 & por exienplo e por bondat de uida . \textbf{ Et lo segundo por palaura e por castigos conuenibles . } Mas quanto parte nesçe alo presente el ma estro de los moços nobles & exemplo per bonitatem vitae , \textbf{ et verbo per monitiones debitas . } Quantum ad praesens spectat , \\\hline
2.2.9 & e deue ser bueno en uida \textbf{ mas para que alguno sea sabio en las sciençias speculatiuas tres cosas son menester } Lo primero que sea fallador dessi¶ & Et bonus in vita . \textbf{ Ad hoc autem quod sit sciens in speculabilibus , | requiruntur tria . } Quod sit inuentiuus ex se . \\\hline
2.2.9 & e querer maestro e doctor muy sabio \textbf{ porque la sçiençia de los bueons maestros et sabios es ligera } ca aquel que claramente entiende claramente fabla . & nihilominus tamen doctorem bene scientem debent inquirere , \textbf{ eo quod doctrina prudentium facilis , } et qui clare intelligit , \\\hline
2.2.9 & contenplando es de tomar cautela \textbf{ por que las cosas falssas non sean mezcladas alas uerdaderas . } Assi en las obras que son de fazer & est adhibenda cautela , \textbf{ ne falsa admisceantur veris : } sic in agendis conuenit hominem esse cautum , \\\hline
2.2.9 & Por la qual cosa \textbf{ assi commo conuiene al doctor e al maestro en las sçiençias especulatiuas de ser acuçioso e sabio } en manera que proponga a sus disçipulos cosas uerdaderas & sicut quaedam falsa apparent vera . \textbf{ Quare sicut doctor est in speculabilibus , | sic decet esse diligentem et cautum , } ut proponat suis auditoribus vera \\\hline
2.2.9 & en manera que proponga a sus disçipulos cosas uerdaderas \textbf{ sin ningun mezclamiento de cosas falssas . } En essa misma manera el que quiere enderesçar e enformar los moços & ut proponat suis auditoribus vera \textbf{ sine admixtione falsorum . } Sic qui vult iuuenes dirigere debet esse , \\\hline
2.2.9 & Ca aquel que prueua las cosas \textbf{ es mas çierto en conosçer las cosas particulares e speçiales . } Et este dector tal deue conosçer las condiçiones speçiales delos moços & esse circumspectus , vel expertus . \textbf{ Nam experti , est particularia cognoscere : } sic et huiusmodi doctor debet \\\hline
2.2.9 & es mas çierto en conosçer las cosas particulares e speçiales . \textbf{ Et este dector tal deue conosçer las condiçiones speçiales delos moços } a que ha de castigar e de enssennar . & Nam experti , est particularia cognoscere : \textbf{ sic et huiusmodi doctor debet | cognoscere particulares conditiones illorum iuuenum , } quos debet dirigere . \\\hline
2.2.9 & commo delas entendidas . \textbf{ Enpero quanto ala sabiduria moral delas obras } que son de fazer & tam inuentorum quam intellectorum . \textbf{ Quantum vero ad prudentiam agibilium , } decet ipsum esse memorem , \\\hline
2.2.10 & ¶ \textbf{ Lo primero por que de ligero fablan palauras orgullosas . } ¶ Lo segundo porque de ligero fablan cosas falłas & tripliciter peccare videntur . \textbf{ Primo , quia de facili loquuntur lasciua . } Secundo , quia de leui loquuntur falsa . \\\hline
2.2.10 & la fabla orgullosa \textbf{ e las palauras torpes . } Et son mucho de denostar & a locutione lasciua , \textbf{ et a sermonibus turpibus : } et sunt increpandi et etiam corrigendi , \\\hline
2.2.10 & es esta segunt el philosofo \textbf{ por que por tales palabras de ligero son inclinados a obras torpes . } Ca la fabla de las cosas torpes & est secundum Philosophum , \textbf{ quia ex talibus locutionibus | de facili ad opera turpia inclinantur : } ipsa enim locutio turpem facit \\\hline
2.2.10 & Ca la fabla de las cosas torpes \textbf{ faze en nos memoria delas cosas delectables e non conuenibles . } Ca los fechos acresçientan la cobdiçia & ipsa enim locutio turpem facit \textbf{ in nobis memoriam delectabilium illicitorum : } qua facta , augetur concupiscentia circa illa : \\\hline
2.2.10 & mas ligeramente se inclinan los omes \textbf{ alas cosas delectables e torpes . } Et por ende es bien dicho & concupiscentia vero augmenta \textbf{ facilius inclinamur ad ipsa : } propter quod bene dictum est , \\\hline
2.2.10 & e los mançebos \textbf{ que non fablen cosas falssas . } Ca assi commo dixiemos dessuso los moços & Secundo sunt cohibendi et corrigendi ; \textbf{ ne loquantur falsa . } Nam ( ut superius diximus ) \\\hline
2.2.10 & co¶la segunda \textbf{ quanto ala manera dt veret conuiene de saber } quanto alas cosas iusibło sas & quare magis sumus attenti circa illa , \textbf{ et per consequens ea magis memoriter retinemus . } Iuuenes igitur , \\\hline
2.2.10 & quanto ala manera dt veret conuiene de saber \textbf{ quanto alas cosas iusibło sas } que son primeras & et per consequens ea magis memoriter retinemus . \textbf{ Iuuenes igitur , } quia eis quasi omnia sunt noua , \\\hline
2.2.10 & ca aquella es hedat \textbf{ por que assaz es inclinada de ssi a cosas locauas e orgullosas } e aseguir sus passiones . & non essent iuuenibus ostendendae . \textbf{ Nam quia satis illa aetas de se prouocatur } ad lasciuiam \\\hline
2.2.10 & se guarda cautella en los mançebos \textbf{ si les fuer defendido de oyr cosas torpes . } Ca segunt el philosofo en el vi̊ libro delas politicas & cautela quantum ad iuuenes , \textbf{ si prohibeantur ab auditione turpium . } Nam secundum philosophum vii Polit’ \\\hline
2.2.10 & quanto a aquellos que oyen . \textbf{ Ca assi commo es cosa conuenible a ellos de oyr } cosas honestas e fermosas & ad eos quos audiunt : \textbf{ quia sicut decens est | audire eos honesta , } et pulchra , \\\hline
2.2.10 & Ca assi commo es cosa conuenible a ellos de oyr \textbf{ cosas honestas e fermosas } e desconuenible de oyr cosas torpes & audire eos honesta , \textbf{ et pulchra , } et indecens audire turpia : \\\hline
2.2.11 & Onde por que dixiemos de suso \textbf{ que las palauras generales en esta sçiençia moral menos aprouechan que las spanles } por ende conuiene nos de dezir & verum quia ( ut supra dicebatur ) \textbf{ in morali negocio sermones uniuersaliores minus proficiunt , } oportet specialiter tradere , \\\hline
2.2.11 & por que con ellos mascassen bien la uianda \textbf{ por que mas liga mente passasse la calentura natural ala bianda } e dende sen signiria & Ordinauit enim natura animalibus dentes , ut per eos cibus debite tritus , \textbf{ facilius pateretur a calore naturali , } et per consequens facilius conuerteretur in nutrimentum : \\\hline
2.2.11 & e se conuierte en fuego . \textbf{ Mas esta orden natural en la mayor parte non la guardan } los que toman la uianda muy & et conuertuntur in ignem . \textbf{ Hunc autem ordinem naturalem , | ut plurimum non obseruant } sumentes cibum auide . \\\hline
2.2.11 & sauianda non se delectan mucho \textbf{ por que este la uianda luengo tienpo en la boca mas cobdiçian } que luego vaya ala garganta . & non multum delectantur , \textbf{ quod cibus diu in ore existat , | sed cupiunt quod cito perueniat ad guttur . } Ideo tales \\\hline
2.2.11 & Et enpeesçe otrosi al cuerpo \textbf{ por que enbarga el cozimiento conuenible . } Ca si la vianda se ouiere bien a cozer & et etiam nocet corpori , \textbf{ quia impedit digestionem debitam . } Si enim cibus digeri debeat , \\\hline
2.2.11 & Por la qual cosa si en tan grand quantia se \textbf{ tomaque la calentura natural non pueda } enssennorar sobrella non se puede bien moler nin cozer . & Quare si in tanta quantitate sumatur , \textbf{ quod calor naturalis } ei dominari non possit , non bene digeritur , \\\hline
2.2.11 & por desordenamiento del alma . \textbf{ Por la qual cosa commo la manera torpe de resçebir la vianda } sea señal de golosina & si contingat ex inordinatione animae . \textbf{ Quare cum turpis modus sumendi cibum signum sit cuiusdam gulositatis , } vel inordinationis mentis \\\hline
2.2.11 & mas avn \textbf{ por sanidat del cuerpo deuemos guardar ora conuenible en que deuemos rescebir la vianda . } ¶ Lo quinto pecan los omes & sed etiam propter sanitatem corporis , \textbf{ obseruanda est in sumptione cibi . } Quinto peccatur circa sumptionem cibi , \\\hline
2.2.11 & mas que la condiçionde la su persona demanda \textbf{ e mas que conuiene al su estado demanda viandas delicadas peca enllo . } Por que esto viene & Qui ergo , ultra quam conditio personae exigat , \textbf{ et ultra quam eius status requirat , | delicata cibaria quaerat , delinquit : } quia hoc ex aliqua intemperantia , \\\hline
2.2.12 & en qual manera pecan los omes çerca del comer \textbf{ finca nons de dezer } en qual manera pecan çerca el beuer . & qualiter delinquitur circa cibum . \textbf{ Restat dicere , } quomodo delinquitur circa potum . \\\hline
2.2.12 & ¶ El primer mal es que abiua el omne asa lux̉ia . \textbf{ Ca escalentado el cuerpo faze se enł omne mayor inclinaçion alas obras de luxia . } Onde el vino tomado & Primo , quia venerea prouocat . \textbf{ Cum enim corpore calefacto maior fiat | incitatio ad actus venereos , } vinum quod maxime calorem efficit immoderate sumptum , \\\hline
2.2.12 & Onde el vino tomado \textbf{ destenpradamente faze en el omne grand calentraa } e abiualo a destenprança de lux̉ia . & incitatio ad actus venereos , \textbf{ vinum quod maxime calorem efficit immoderate sumptum , } incitat ad incontinentiam nimiam . \\\hline
2.2.12 & que en la veiedat \textbf{ en quanto aquella hedat esmos inclinada a loçana que la vegez¶ } El segundo mal que viene del tomar mucho el vino & tanto magis in aetate iuuenili quam senili cauenda est , \textbf{ quanto illa aetas pronior est | ad lasciuiam quam alia . } Secundum malum , \\\hline
2.2.12 & de dize ocho a nons e el vaton de veynte e dos \textbf{ porque en tal hedat se engendran los fijos ma acabados } segunt que dize el philosofo . & in masculo sex et triginta : \textbf{ in tali enim aetate } ( secundum ipsum ) \\\hline
2.2.12 & que en todo tienpo dela cresçençia de los omnes \textbf{ en qual tp̃o dura comunalmente } fasta xxvn año deuen se guardar los mocos & sufficeret toto tempore augmenti , \textbf{ quod durat communiter } usque ad vigesimum primum annum , \\\hline
2.2.13 & por dos razones . \textbf{ ¶ Lo pripreo por escusar cuydado desconneible ¶ } Lo segundo por alcançar fin conuenible ¶ & duplici via declarari potest . \textbf{ Primo , ex vitatione illicitae solicitudinis . } Secundo , ex adeptione finis intenti . \\\hline
2.2.13 & ¶ Lo pripreo por escusar cuydado desconneible ¶ \textbf{ Lo segundo por alcançar fin conuenible ¶ } La primera razon paresçe assi . & Primo , ex vitatione illicitae solicitudinis . \textbf{ Secundo , ex adeptione finis intenti . } Prima via sic patet . \\\hline
2.2.13 & e non entiende en algunas delectaçiones conuenibles \textbf{ luego comiença a andar vagando cuydando enlas cosas desconueibles . } Onde el philosofo enłviiij libro delas politicas dize & et non intendit aliquibus delectationibus licitis , \textbf{ statim incipit vagari cogitando de illicitis : } unde Philosophus 8 Polit’ ait , \\\hline
2.2.13 & Conuiene a nos algunas uegadas de auer algunos trebeios \textbf{ e algunos solazes conuenibles e honestos . } Mas quales son estos trebeios & expedit aliquando habere aliquos ludos , \textbf{ et habere aliquas deductiones | licitas et honestas . } Qui sunt autem illi ludi , \\\hline
2.2.13 & o destenpratça del appetito . \textbf{ Et por ende es al omne neçessario el castigo en los gestos } Et por que el omne que es acuçioso çerca la razon & vel insipientiam mentis , vel intemperantiam appetitus . \textbf{ Est enim homini necessaria disciplina in gestibus . } Nam quia ipse intentus est \\\hline
2.2.13 & e el entendimiento non siente \textbf{ assi los mouimientos naturales ni obra } assi por natra al inclinaçion & non sic percipit naturales impetus , \textbf{ nec sic agit ex naturali instinctu , } ut aues et bestiae . \\\hline
2.2.13 & por bien delectable \textbf{ deuemos las querer delicadas e muelles . } Et si las queremos & propter bonum delectabile : \textbf{ sic quaeruntur delicata , et mollia . } Si propter utile : \\\hline
2.2.13 & assi las deuemos querer fermosas e apuestas . \textbf{ Ca cosa desconuenible es al omne ser muy cuy dado lo çerca la blandura delas vestiduras } e çerca toda delectaçion que es en ellas . & sic quaeruntur pulchra , et decora . \textbf{ Indecens est autem nimis solicitari | circa molliciem vestium , } et circa delectationem in ipsis : \\\hline
2.2.13 & por ende son de enssennar \textbf{ que non se delecten mucho en las vestiduras muelles . } ¶ Vicho en qual maneras & instruendi sunt , \textbf{ ut non nimis delectentur in mollicie vestium . } Dicto , quomodo se habere debeant \\\hline
2.2.13 & nin del frio \textbf{ commo los que han las conplissiones ralas . } Onde las mugers & non sic laeduntur a calore et frigore , \textbf{ sicut habentes complexiones raras . } Unde et mulieres , \\\hline
2.2.13 & ca la hedat dela vegez \textbf{ por que ha menos de calentura natural de dentro } mas ha menester calentraa de fuera . & quia senilis aetas , \textbf{ eo quod magis caret calore naturali intrinseco , } magis indiget de calore exteriori . \\\hline
2.2.14 & e las colas \textbf{ senllibls lon anos manitieltas mas . } Por ende en la mayor parte los omes siguen el appetito & nam quia nostra cognitio incipit a sensu , \textbf{ et sensibilia sunt nobis magis nota ; } ideo ut plurimum homines sequuntur \\\hline
2.2.14 & senssitiuo de los sesos \textbf{ mas el appetito de los sesos es uirtud organica o corporal . } Por la qual cosa conuiene & appetitum sensitiuum . \textbf{ Appetitus autem sensitiuus | est virtus organica siue corporalis . } Quare oportet talem appetitum sumere modum , \\\hline
2.2.14 & Ca la delectaçiones \textbf{ por ayuntamiento de cosa conuenible con cosa qual conuiene } Et por ende ninguon non goza de beuir en conpanna & Delectatio enim est \textbf{ ex coniunctione conuenientis cum conuenienti . } Nullus ergo gaudet in societate viuere , \\\hline
2.2.14 & por que ciean \textbf{ que los bienes senssibłs son de segnir } mas que los otros & et de facili persuadetur eis , \textbf{ ut credant bona sensibilia esse sequenda . } Quia sermones particulares valde videntur \\\hline
2.2.15 & mas que los otros \textbf{ or que los sermones particulares son muy aprouechosos ala sçiençia de costunbres . } Por ende desçenpteno particularmente alos tp̃os departidos & ut credant bona sensibilia esse sequenda . \textbf{ Quia sermones particulares valde videntur | esse proficui morali negocio , } ideo particulariter descendendo ad diuersa tempora , \\\hline
2.2.15 & Et pues que assi es¶ \textbf{ Lo primero mostraremos qual cuydado auemos de tomar de los fijos fasta los siete años } ¶ lo segundo qual cuydado deuemos tomas dellos del septimo anero fasta el catorzeno & Primo enim declarabimus \textbf{ qualis cura habenda sit | de filiis usque ad septem annos . } Secundo qualis a septimo usque ad decimumquartum annum . \\\hline
2.2.15 & Lo primero mostraremos qual cuydado auemos de tomar de los fijos fasta los siete años \textbf{ ¶ lo segundo qual cuydado deuemos tomas dellos del septimo anero fasta el catorzeno } Et despues qual de los xiiij̊ & de filiis usque ad septem annos . \textbf{ Secundo qualis a septimo usque ad decimumquartum annum . } Postea a decimoquarto et deinceps . \\\hline
2.2.15 & ¶ la primera es que deuen ser criados \textbf{ fasta los siete años de cosas humidas . } Enpero assy que en el comienço dela nasçençia & quae seruanda sunt in aetate primitiua . \textbf{ Primum est , quia ad septimum debent pasci mollibus , } ita tamen quod a principio sint alendi lacte . \\\hline
2.2.15 & La terçera es que los deuen acostunbrar alos frios \textbf{ ¶la quarta es que son de acostunbrar a mouimientos conuenibles e tenpdos . } Et esto es prouechoso en todas las hedades ¶ & Tertium , sunt assuescendi ad frigora . \textbf{ Quartum , sunt assuescendi ad conuenientes et temperatos motus , } quod in omni aetate videtur esse proficuum . \\\hline
2.2.15 & La quinta es que lon de recrear \textbf{ por trebeios conuenibles . } Et deuen rezar ante ellos algunas bueans estorias . & quod in omni aetate videtur esse proficuum . \textbf{ Quintum , sunt recreandi per debitos ludos , } et sunt eis recitandae aliquae historiae , \\\hline
2.2.15 & Et pues que assi es los moços \textbf{ fasta el septimo año deuen ser cerados de cosas muelles e blandas . } Enpero assi que en el comienço dela su nasçençia & Sextum , a ploratu sunt cohibendi . \textbf{ Iuuenes ergo usque ad septennium alendi sunt mollibus ; } ita tamen quod a principio maxime alendi sunt lacte : \\\hline
2.2.15 & Lo quarto los mocos deuen ser acostunbrados \textbf{ a mouimientos conuenibles e tenprados } por que segunt el philosofo el mouimiento tenprado & Quarto pueri sunt assuescendi ad conuenientes , \textbf{ et temperatos motus . } Nam secundum Philosophum , \\\hline
2.2.15 & tenprados los mienbros del su cuerpo seran mas firmes \textbf{ e mas fu ertes . } Et pues que assi es los moços & membra corporis eius solidantur , \textbf{ et fiunt fortiora . } Pueri ergo quia nimis habent tenera membra , \\\hline
2.2.15 & Et avn deuen les dezir algunos cantos \textbf{ ca los cantos honestos son de cantar alos moços } por que los moços non pueden sostir ninguna cosa triste . & percipere significationes verborum . \textbf{ Vel etiam aliqui cantus honesti | sunt eis cantandi . } Nam ipsi nihil tristes sustinere possunt : \\\hline
2.2.15 & ca los cantos honestos son de cantar alos moços \textbf{ por que los moços non pueden sostir ninguna cosa triste . } Por ende es bien de los acostunbrara algs trebeios tenprados & sunt eis cantandi . \textbf{ Nam ipsi nihil tristes sustinere possunt : } ideo bonum est , eos assuescere ad aliquos moderatos ludos , \\\hline
2.2.16 & segunt que a el paresçiere que les conuiene . \textbf{ Mas en este tp̃o que es del septimo año fasta el año xiiij̊ . } son de penssar tres cosas enl gouernamiento de los fijos . & ut ei videbitur expedire . \textbf{ In hoc autem tempore , | quod est a septimo usque ad decimumquartum annum , tria sunt consideranda } circa regimen filiorum . \\\hline
2.2.16 & Mas en este tp̃o que es del septimo año fasta el año xiiij̊ . \textbf{ son de penssar tres cosas enl gouernamiento de los fijos . } Ca el omne enł primero departimiento & quod est a septimo usque ad decimumquartum annum , tria sunt consideranda \textbf{ circa regimen filiorum . } Nam homo prima diuisione diuiditur in animam , et corpus . \\\hline
2.2.16 & e bien ordenado son de vsar \textbf{ por bsos e por mouimientos conuenibles . } Mas por que ayan la uoluntad bien dispuesta & exercitandi sunt per debita exercitia , \textbf{ et per debitos motus . } Ut habeant voluntatem bene ordinatam , \\\hline
2.2.16 & deuen se enduzir a uertudes conuenibles \textbf{ e a obras uirtuosas . } Mas por que ayan el entendemiento acabado & inducendi sunt ad debitas virtutes , \textbf{ et ad virtutum opera . } Sed ut habeant intellectum perfectum , \\\hline
2.2.16 & deuen ser enssennados \textbf{ en sciençias conuenibles . } Et pues que assi es la sçiençia e la uirtud & Sed ut habeant intellectum perfectum , \textbf{ instruendi sunt in debitis scientiis . Scientia ergo , virtus , et exercitium attendenda sunt } in regimine filiorum . \\\hline
2.2.16 & deuen los acostunbrar enl comienço dela su nasçençia \textbf{ a algunos mouimientos conuenibles . } Mas quando ouieren conplido el vii̊ año fasta el xiiij̊ . & assuescendi sunt pueri \textbf{ ad aliquos motus . } Sed cum impleuerunt septennium \\\hline
2.2.16 & segunt el philosofo \textbf{ paresçe que son vsos e mouimientos conuenibles enlos moços . } Et enł segundo setenario & vel luctatio secundum Philosophum videntur \textbf{ esse debita exercitia in iuuenibus . } In secundo tamen septennio sunt ulteriora exercitia assumenda \\\hline
2.2.16 & por que tal hedat es muy tierna \textbf{ non deuen tomar obras de caualłia nin otras obras fuertes . } Et por ende el pho enł viij̊ libro delas politicas dize & eo quod nimis sit tenera , \textbf{ non sunt assumenda opera militaria nec opera ardua . } Unde Philosophus 8 Polit’ ait , \\\hline
2.2.16 & e de non enduzir los mocos a uirtudes \textbf{ e aguardar las leyes bueans e aprouechosas . } Ca el philosofo enłvij̊ libͤ & non instruere pueros ad virtutem , \textbf{ et ad obseruantiam legum utilium . } Inquirit enim Philosophus 8 Polit’ \\\hline
2.2.16 & por que ayan appetito \textbf{ e desseo conuenible o por que ayan entendimiento acabado . } Mas el ph̃ prueua que pmeramente deuemos auer cuydado de la ordenaçion dela & ut habeant debitum appetitum , \textbf{ vel ut habeant perfectum intellectum . } Probat autem prius esse curandum \\\hline
2.2.16 & en qual manera ayan la uoluntad bien ordenada \textbf{ que en qual manera ayan el entendimiento sabio . } Mas ninguno non ha ordenada uoluntad & quomodo habeant voluntatem bene dispositam , \textbf{ quam quomodo scientificum intellectum . } Nullus autem habet bene ordinatam voluntatem , \\\hline
2.2.17 & aquelt p̃o esto solo es los lenguages delos omes . \textbf{ Mas desde los siete años fasta los xiiij̊ . años } por que ya comiencan a auer algunas cobdiçias desordenadas . & hoc est idiomata vulgaria . \textbf{ Sed a septimo usque ad quartumdecimum } quia iam incipiunt habere concupiscentias aliquas illicitas , \\\hline
2.2.17 & Mas enł segundo se tenario \textbf{ que es de los siete años fasta los . } xiiij̊ deuen entender prinçipalmente çercados cosas & Sed in secundo septennio , \textbf{ ut a duodecimo anno usque ad quartumdecimum , } est quasi intendendum principaliter circa duo , ut circa dispositionem corporis , \\\hline
2.2.17 & aquellos que quieren beuir uida politica e de çibdat es esta que del . xiiij ̊ . \textbf{ año adelante t omne trabaios mayores que tomaron ante . } Ca todos los trabaios que tomaron en los xiiiij años passados & a quartodecimo anno et deinceps , \textbf{ est ut assumant fortiores labores quam hactenus . } Nam sicut a septimo anno usque ad 14 \\\hline
2.2.17 & año adelante t omne trabaios mayores que tomaron ante . \textbf{ Ca todos los trabaios que tomaron en los xiiiij años passados } deuen ser liuianos e ligeros . & est ut assumant fortiores labores quam hactenus . \textbf{ Nam sicut a septimo anno usque ad 14 | assumendi sunt fortiores labores } quam in septem praecedentibus annis : \\\hline
2.2.17 & Et en tanto que segunt el philosofo desde los . \textbf{ xiiij̊ años se deuen acostunbrar los mocos a trabaios fuertes . } Assi commo al vso dela lucha o a otro trabaio & assuescendi sunt pueri \textbf{ ad labores fortes , | ut ad exercitationem luctatiuam , } vel ad aliquam aliam exercitationem \\\hline
2.2.17 & por que ayan el cuerpo bien ordenado \textbf{ por que puedan tomar trabaios conuenibles la qual cosa } mayormente se pue de fazer & ut habeant sic bene dispositum corpus , \textbf{ ut possint debitos subire labores , } quod maxime fieri contingit , \\\hline
2.2.17 & si vsaren los fijos a ex̉çiçios \textbf{ e amouimientos conuenibles ¶ } Visto en qual manera del . xiiij . año adelante deuen los padres auer cuydado de los fijos & quod maxime fieri contingit , \textbf{ si ad debita exercitia assuescant . } Viso , quomodo a quartodecimo anno ultra solicitari debent patres erga filios , \\\hline
2.2.17 & assi e alos otros . \textbf{ Mas el mançebo non es oydor conuenible delas sçiençias morales . } Et por ende en la hedat muy de moço & quomodo se et alios debeant gubernare : \textbf{ moralium autem scientiarum iuuenis et insecutor passionum | non est sufficiens auditor . } In aetate ergo nimis puerili , \\\hline
2.2.18 & mucho alas sçiençias speculatinas \textbf{ nin alas delectaçiones spunales conuiene les para escusar la ꝑeza } e para escusar & nam quia tales non multum vacant inquisitioni veritatis , \textbf{ nec in spiritualibus delectationibus ; } expedit eis \\\hline
2.2.18 & e para escusar \textbf{ cuydado desconuenible de se dar } a algunos trabaios corporales & expedit eis \textbf{ ut vitent inertiam , } et ut vitent solicitudinem illicitam , \\\hline
2.2.18 & enłbso delas armas \textbf{ por que el mouimiento conuenible del cuerpo faze el cuerpo mas fuerte } e mas rezio para que pueda sofrir & circa armorum usum . \textbf{ Exercitatio enim corporalis debita | reddit corpus robustius , } ut facilius duriciem armorum sustinere possit . \\\hline
2.2.18 & por la qual somos despuestos \textbf{ para estudiar enlas sçiençias especulatiuas . } Mas por el trabaio & Nam per sessionem et quietem redditur caro mollis , \textbf{ per quam sumus apti ad speculandum ; } per laborem vero et motum efficitur caro dura , \\\hline
2.2.18 & que deuen gouernar los otros de escusar la ꝑeza \textbf{ e el cuydado desconueinble estudiando enlas sçiençias morales } cuydando mucho a menudo enlas bueans costunbres del regno & vitare inertiam et solicitudinem illicitam , \textbf{ vacando moralibus scientiis , } recogitando frequenter bonas consuetudines regni , \\\hline
2.2.18 & e sus herederos \textbf{ e todos gouernadores meior escusar la peza et el uagar } que non por el vso corporal & et a suis haeredibus \textbf{ magis est vitanda inertia , } quam per laborem , \\\hline
2.2.19 & que dixiemos qual cuydado deuen auer los padres \textbf{ cerca los fiios fincanos de dez qual cuydado deuen auer los padres çerca las fijas } mas esto ha menester muy pequano tractado & qualis cura gerenda est circa filios , \textbf{ restat dicere , | qualis gerenda sit circa filias . } Sed hoc breui tractatu indiget : \\\hline
2.2.19 & Enpero algunas cosas anedremos a los dichos de çima \textbf{ para el gouernamiento conuenible delas fiias } entre las quales cosas primero dezimos & Aliqua tamen superaddemus dictis \textbf{ propter debitum regimen filiarum : } inter quae primo dicemus filias cohibendas esse a circuitu , et discursu , \\\hline
2.2.19 & en los quales es la razon \textbf{ e el entendimiento mayor es grant peligro } de non escusar las azinas de los pecados much mas es esto de escusar en las mugers . & Si ergo in viris , \textbf{ in quibus est ratio praestantior , } est magnum periculum non vitare commoditates delictorum : \\\hline
2.2.19 & fallesçe la razon por la qual a cada vno es defendido \textbf{ que non sigua las cosas desconuenibles . } Et pues que assi es muy grant freno delas fenbras & propter quam quis prohibetur , \textbf{ ne prosequatur illicita maximum fraenum foeminarum , } et potissime puellarum , \\\hline
2.2.19 & e mayormente delas moças es la uerguença \textbf{ por que non puedan sallir a fazer cosas torpes . } Et pues que assi es cosa conuenible es de defender es alas mocas & et potissime puellarum , \textbf{ ne prorumpant in turpia , } videtur esse verecundia . \\\hline
2.2.19 & por que non puedan sallir a fazer cosas torpes . \textbf{ Et pues que assi es cosa conuenible es de defender es alas mocas } que non corran & ne prorumpant in turpia , \textbf{ videtur esse verecundia . | Decens ergo est cohibere puellas } a discursu et euagatione , \\\hline
2.2.20 & e estudiemos \textbf{ por que sean en nos algunas delectaçonnes conuenibles . } Et pues que assi es assi commo los uarones & circa quae vacantes , \textbf{ insunt nobis aliquae delectationes licitae . } Sicut ergo viri , \\\hline
2.2.20 & o cerca el gouernamiento dela casa \textbf{ o cerca otros vsos conuenibles en essa misma manera avn las mugers } por que non buian en vagar & vel circa regimen domus , \textbf{ vel circa aliqua alia exercitia licita . | Sic et mulieres , } ne ociose viuant , \\\hline
2.2.20 & disposiconn dela uoluntad de dentro . \textbf{ Et pues que assi es deuen auer los padres grant acuçia } e auer grant cuydado & vel aliqua bona dispositio interioris mentis . \textbf{ Debet ergo diligentia et cura adhiberi , } ut mulieres sint bonae et virtuosae , \\\hline
2.2.20 & por que las mugers non biuna en uagar \textbf{ mas que se trabaien çerca alguas cosas conueibles e honestas } por que dende salga fructo e prouecho & ne foeminae ociose viuant , \textbf{ sed exercitent se | circa aliqua opera licita , et honesta , } ut ex hoc resultet fructus , \\\hline
2.2.20 & entn el alto grado \textbf{ e que non fuesse costunbre cła tierra } que se trabaiasse en tales obras & quod non esset dignum \textbf{ vel non esset consuetum | secundum morem patriae , } ut se circa talia exercitaret : \\\hline
2.2.21 & Et pues que assi es por que a cada vno paresçe \textbf{ de ser cosa fermosa e conuenible aquello que ama } Si las mugers quacallan mas son amadas & quia ergo cuilibet videtur \textbf{ esse pulchrum et decens | quod amati } si mulieres taciturnae magis amantur , \\\hline
2.3.1 & e a cunplimiento dela uida \textbf{ las qual sson conplummientos conueinbles delas moradas } e muchedunbre de dineros & et ad sufficientiam uitae : \textbf{ cuiusmodi sunt decentia aedificiorum , } multitudo numismatum , \\\hline
2.3.1 & e a conplimiento dela uida \textbf{ e que paresçen que cunplen la mengua corporal . } Et por ende determinaremos en esta terçera parte deste segundo libro & vel ad sufficientiam vitae , \textbf{ quae supplere videntur indigentiam corporalem . } Determinabimus igitur \\\hline
2.3.1 & e delas otras cosas \textbf{ que cunplen la mengua corporal ¶ } La primera razon se toma del cunplimiento dela uida¶ & et de supellectilibus , et de aliis , \textbf{ quae supplent indigentiam corporalem . } Prima ratio sumitur \\\hline
2.3.1 & Mas finca perfeccion enłalma \textbf{ mas los estrumentos delas artes mecanicas son factiuos e obradores } por que por ellos finca algua cosa fecha de fuera & quia organa gubernationis sunt actiua , \textbf{ organa vero moechanicorum sunt factiua . } Nam in moechanicis regulamur per artem , \\\hline
2.3.1 & delos estrumentos del arte del gouernamiento dela casa ¶avn ay departimiento entre esta arte e la otra \textbf{ por que el arte mecanica es derecha razon delas cosas } que ha de fazer de fuera . & ab organis gubernationis . \textbf{ Differt autem prudentia ab arte , } quia ars est recta ratio factibilium et per artem resultat aliquid factum in materia extra : \\\hline
2.3.1 & que son de fazer \textbf{ e por ella non sale propreamente ninguna cosa fechͣ de fuera } Massale alguna & sed prudentia est recta ratio agibilium , \textbf{ et per eam non proprie | resultat aliquid factum extra : } sed magis resultat aliqua actio , \\\hline
2.3.1 & Mas por que aquellas mismas razones sen podia prouar \textbf{ que parte nesçe al gouernamiento dela casa saber se auer } conueinblemente çerca los ofiçiales & Per illas autem easdem rationes probari posset , \textbf{ quod spectat ad gubernatorem domus | scire debite se habere } circa ministros et seruos : \\\hline
2.3.2 & enł primero libro delas politicas \textbf{ por cosa semeiable en las otras artes } do da a entender & quod declarat Philosophus 1 Polit’ \textbf{ per simile in aliis artibus , } ubi innuit \\\hline
2.3.2 & por si non peỹna ante el instrumento del çitolero \textbf{ que es el tocador ha menester maestro quel mueua . } Et el penne para peyndar ha menester algun mouedor & Ideo ad cytharizandum plectrum \textbf{ indiget ministro mouente , } et pecten ad pectinandum indiget mouente ipsum . \\\hline
2.3.2 & si los tocadores por si çitolassen e los peỹnes por si \textbf{ peinnassen non serien mester maestros nin mouedores } nin los sieruos sennors non aurian mester & si plectra per se cytharizarent , et pectines per se pectinarent , \textbf{ nihil opus esset architectoribus ministrorum , } nec dominis seruorum . \\\hline
2.3.2 & peinnassen non serien mester maestros nin mouedores \textbf{ nin los sieruos sennors non aurian mester } Ca por ende los maestras & nihil opus esset architectoribus ministrorum , \textbf{ nec dominis seruorum . } Ideo enim architectores , \\\hline
2.3.2 & en manera de fabli ella \textbf{ que por si cunplie su obra conuenible . } Et do tales fuessen los rastrumentos dela casa & de qua Philosophus fabulose recitat in Politicis ; \textbf{ quod per se implebat opus debitum : } nulla indigentia esset seruorum , \\\hline
2.3.2 & Et las cosas mas baxas son instrumentos sin alma . \textbf{ Et las cosas medianeras son } assi commo los siruientes e los sieruos & esse videntur architectores et domini : \textbf{ infima vero sunt organa inanimata : } media sunt ministri et serui , \\\hline
2.3.2 & que vsen de o tristales cosas . \textbf{ mas cosaco nueible es } que fagan estas cosas & aut aliqua talia exercere : \textbf{ sed congruentius est haec } per medios ministros efficere . \\\hline
2.3.2 & que fagan estas cosas \textbf{ por los siruientes medianeros . } Et pues que assi es en la casa conplida & sed congruentius est haec \textbf{ per medios ministros efficere . } In domo ergo completa \\\hline
2.3.3 & Lo otro es el tenpramiento del ayre \textbf{ por que la morada sea asentada en ayre conuenible . } Mas que los Reyes e los prinçipes de una auer & et temperamentum aeris , \textbf{ ut sit in debito aere collocatum . } Quod autem Reges et Principes debeant \\\hline
2.3.3 & esto prueua el philosofo por dos \textbf{ razonesl delas quales la primera se toma de parte dela grandeza real . } ¶ La segunda de parte del pueblo & probat Philos’ duplici ratione . \textbf{ Quarum prima sumitur | ex parte magnificentiae regiae . } Secunda ex parte populi . \\\hline
2.3.3 & por quela magnifiçençia ha de ser çerca grandes despenssas \textbf{ mas assi commo es dicho en esse mismo quarto libro delas ethicas al magnifico parte nesçe de apareiar morada conuenible } mas non es conuenible morada al magnifico & Sed magnifici , \textbf{ ut dicitur in eodem 4 Ethicor’ } est \\\hline
2.3.3 & por la grand marauilla \textbf{ que veran ca el puenblo menoͤ se leunata } contra el prinçipe & quasi sit mente suspensus propter vehementem admirationem . \textbf{ Nam populus minus insurgit contra principem , } videns ipsum sic magnificum : \\\hline
2.3.3 & enpero conuiene alos Reyes \textbf{ e alos prinçipes de fazer moradas costosas e nobles } assi commo el su estado demanda & decet tamen Reges et Principes , \textbf{ ne in contemptum habeantur a populo , | facere aedificia magnifica , } prout requirit decentia status , \\\hline
2.3.3 & en non ser asentada la morada en los valles muy baxos \textbf{ por que si en los valłs baxos fuessen fechͣs } el ayre seria & Dicit enim salubritatem aeris primo \textbf{ declarare loca a vallibus infimis libera . Si enim in vallibus infimis aedificia construantur , } quia aer est ibi grossus , \\\hline
2.3.3 & e por todas estas cosas iudgamos la linpiedat e bondat del ayre \textbf{ e por las cosas contrarias iudgamos que el ayre es enfermo } ca si los moradores de aquel logar non han color sano mas amariello & Per omnia haec indicatur bonitas aeris : \textbf{ et per contraria indicatur | aer esse infirmus . } Si enim habitatores in ipso non habeant \\\hline
2.3.3 & e por las cosas contrarias iudgamos que el ayre es enfermo \textbf{ ca si los moradores de aquel logar non han color sano mas amariello } e de ligero padesçen dolor en la cabeça & aer esse infirmus . \textbf{ Si enim habitatores in ipso non habeant | colorem sanum sed croceum , } de facili in capite patiantur , \\\hline
2.3.4 & assi comm̃es la sanidat del agua \textbf{ e departimiento conuenible del mundo } por que el agua segunt el philosofo es muy comun a todos & ut aquae salubritas , \textbf{ et debita dispositio uniuersi . | Aqua enim } ( secundum Philosophum ) \\\hline
2.3.4 & que la morada sea assentada en tal logar \textbf{ por que aya abastamiento de agua buena e sana } por que los moradores de aquellas moradas & ut sic aedificium situetur , \textbf{ ut ei sit aquae salubris copia : } ne habitatores eius \\\hline
2.3.4 & e el agua se engendra en los logares soterrannos \textbf{ e passan por las venas sola tierra } por la qual cosa & et aqua in locis subterraneis generatur , \textbf{ et per venas subterraneas transit ; } quare si locus generationis aquarum , \\\hline
2.3.4 & ¶Lo terçero que es de penssar en las aguas es \textbf{ que sean de color claro de gnisa } que passe el oio de vna parte a otra & Tertium , quod considerandum est in aquis , \textbf{ est quod sit coloris perspicui . } Nam ipsa infectio coloris , \\\hline
2.3.4 & e non este algun limo \textbf{ ca la tierra limosa e lodosa } por que es corrupta non puede ser sana . & ne aquis illis aliquis insideat simus : \textbf{ nam terra limosa et lutosa , } eo quod infecta sit , \\\hline
2.3.4 & e non podamos auer \textbf{ y abastamiento de agua sana deuemos } segunt manda aquel philosofo & Quod si tamen aedificandi necessitas urgeat , \textbf{ nec tamen ibi aquae salubris sit copia , } est ibi ( secundum Palladium ) construenda cisterna , \\\hline
2.3.4 & por que por el nadamiento de los peces \textbf{ el agua estante semeie en lignieza al agua que corre ¶ Visto en qual manera son de fazer las moradas } quanto ala salud delas agunas finca de ver & ut horum natatu aqua \textbf{ stans agilitatem currentis imitetur . | Viso , qualiter est aedificium construendum quantum } ad salubritatem aquae : \\\hline
2.3.4 & ¶ Avn en essa misma manera se pueden departir o \textbf{ triscondiconnes particulares delas moradas . } mas por que tales cosas son & et longe a stabulis , fimo , et sterquiliniis . \textbf{ Sic etiam aliae particulares conditiones aedificiorum distingui possent . } Sed quia talia nimis particularia sunt , \\\hline
2.3.5 & assi casi los omes biuen naturalmente \textbf{ e la conpanna dela çibdat es en alguna manera natural al omne } assi commo es prouado & Si enim homines naturaliter viuunt , \textbf{ et societas politica est quodammodo homini naturalis , } ut in prima parte huius secundi libri diffusius probabatur : \\\hline
2.3.5 & conuiene en algunan manera \textbf{ que las cosas naturales sean neçessarias enla uida politica . } mas segunt el philosofo & oportet aliquomodo naturalia esse \textbf{ quae sunt necessaria in vita politica ; } sed secundum Philosophum primo Polit’ \\\hline
2.3.5 & en \textbf{ conparaconn de las cosas corporales e sensibles } por ende han señorio natural sobrellas . & eo enim ipso quod homo respectu corporalium \textbf{ et sensibilium est creaturarum dignissima , } habet naturale dominium super ipsa : \\\hline
2.3.5 & conparaconn de las cosas corporales e sensibles \textbf{ por ende han señorio natural sobrellas . } por la qual cosa natural cosa es al ome & et sensibilium est creaturarum dignissima , \textbf{ habet naturale dominium super ipsa : } quare naturale est homini \\\hline
2.3.5 & por ende han señorio natural sobrellas . \textbf{ por la qual cosa natural cosa es al ome } que enssennore e a estas cosas senssibles & habet naturale dominium super ipsa : \textbf{ quare naturale est homini } quod dominetur istis sensibilibus , \\\hline
2.3.5 & que la possession de tales cosas es natural . \textbf{ dize que naturalmente es batalla derecha de los omes contra las bestias } por que las bestias deuen ser suiebtas del omne & ubi probat possessionem talium naturalem esse , \textbf{ ait , quod hominum ad bestias naturaliter est iustum bellum : } eo enim quod bestiae naturaliter homini debent esse subiectae , \\\hline
2.3.5 & non fallesçe alas ainalias \textbf{ que non son acabadas mas naturalmente apareia el nudmiento conuenible a ellas mas conueible cosa es } que les non fallesca & non deficit animalibus , \textbf{ sed naturaliter praeparat eis debitum nutrimentum ; | congruum est } ut non deficiat eis iam perfectis . \\\hline
2.3.5 & ca si las cosas acabadas son mas diguas que las non acabadas \textbf{ e la natra a apareia alas aianlas non acabadas nud̀miento conuenible . } mucho mas deue apareiar alas ainalias ya acabadas & Si enim digniora sunt perfecta imperfectis , \textbf{ et ipsis imperfectis animalibus | praeparatur debitum alimentum a natura , } multo magis hoc praeparabitur \\\hline
2.3.5 & assi commo son las aues la natura \textbf{ assi ordeno poniendo en los hueuos blanco e bermeio } assi que del blanco se engendra el aue & sic natura ordinauit , \textbf{ ponens in ipsis ouis album et rubeum , } ita quod ex albo generatur auis , \\\hline
2.3.5 & por la qual cofa si la natura engendrada anudmiento \textbf{ e cosas neçessarias ala uida alas aianlias non acabadas much } mas esto faze e deue fazer alas aianlias acabadas . & Quare si animalibus imperfectis natura \textbf{ praeparat nutrimentum et necessaria vitae , } multo magis hoc facit animalibus perfectis . \\\hline
2.3.5 & Et pues que assi es natal cosa es a nos de auer las cosas de fuera \textbf{ e por ende el sennorio delas cosas de fuera es en algua manera natural al omne . } por que la natan engendro & habere res exteriores . \textbf{ Habere ergo dominium rerum exteriorum est quodammodo homini naturale : } quia natura produxit \\\hline
2.3.5 & por que la natan engendro \textbf{ e fizo estas cosas senssibles para el omne . } ca assi commo es dicho & quia natura produxit \textbf{ huiusmodi sensibilia propter hominem . } Sumus enim quodammodo nos finis omnium , \\\hline
2.3.5 & mas escogen \textbf{ para si uida çelestial e uida sobre omne } que es mas alta & absque dominio exteriorum rerum , non proponit viuere ut homo , \textbf{ sed eligit sibi vitam caelestem , } et supra hominem . \\\hline
2.3.6 & mas claramente enpero las cosas estando \textbf{ assi como agoraes tan cosa aprouechosa es ala çibdat que los çibdadanos . } ayan sus possessiones proprias & Rebus tamen stantibus ut nunc , \textbf{ utile est ciuitati ciues gaudere possessionibus propriis , } eo quod vita ciuilis sit \\\hline
2.3.6 & por ellas ca agora en la çibdat \textbf{ lon muchs pobres avn que non contradigamos } que los çibdada nos puedan auer possessiones proprias & et esse pauperes , \textbf{ non obstante quod ciues possunt } gaudere possessionibus propriis , \\\hline
2.3.6 & Et pues que assi es estando las cosas \textbf{ assi commo dicho es pro prouechosa cosa es ala çibdat } que los çibdadanos ayan & In rebus ergo sic se habentibus , \textbf{ utile est ciuitati ciues } habere possessiones proprias , \\\hline
2.3.6 & possessionspropreas \textbf{ por que non auiendo cuy dado çerca las cosas comunes dela casa } uernien los omes amengua & habere possessiones proprias , \textbf{ ne propter ignauiam circa communia , } domus ciuium patiantur inopiam . \\\hline
2.3.6 & ca por la mayor parte se le una tan contiendas e uaraias \textbf{ entre los que han cosas comunes algunas . } caueemos que los hͣrmaros fijos de vn padre entre los quales seg̃tel philosofo enł . viij̊ & lites et bella \textbf{ inter participantes aliquid commune : } videmus enim ipsos fratres ex eodem patre natos , \\\hline
2.3.6 & que conuienen ala çibdat \textbf{ cosa aprouechosa es que los çibdadanos ayan possessiones propreas } por que con mayor cuy dado & circa expedientia ciuitati , \textbf{ utile est ciues gaudere possessionibus propriis , } ut magis solicite , \\\hline
2.3.7 & e las ordeno a vso e añro sennorio . \textbf{ ¶ Et pues que assi es cosa conuenible es } de tomar nudermiento delos canpos & ordinauit enim ea ad usum et dominium nostrum ; \textbf{ licitum est ergo sumere nutrimentum ex agris , } et animalibus domesticis \\\hline
2.3.7 & lias batalla derecha \textbf{ por la qual cosa el philosofo enl ꝑ̀mo libro delas politicas } dize & habet contra talia iustum bellum \textbf{ propter quod Philosophus 1 Politic’ vult venatiuam et piscatiuam } esse vitas licitas . \\\hline
2.3.7 & que epho quiere \textbf{ que non tan solamente fue batalla derecha del ome alas bestias } mas avn & Videtur tamen velle Philosophus , \textbf{ quod non solum hominis ad bestias , } sed etiam hominis ad barbaros sit iustum bellum : \\\hline
2.3.7 & mas avn \textbf{ que es batalla derecha de los omes alos barbaros } ca los omes barbaros e siluesttes e montesmos & quod non solum hominis ad bestias , \textbf{ sed etiam hominis ad barbaros sit iustum bellum : } homines enim barbari syluestres , \\\hline
2.3.7 & que es batalla derecha de los omes alos barbaros \textbf{ ca los omes barbaros e siluesttes e montesmos } por que fallesçen de uso de razon naturalmente & sed etiam hominis ad barbaros sit iustum bellum : \textbf{ homines enim barbari syluestres , } quia ab usu rationis deficiunt , \\\hline
2.3.7 & e por entendimiento \textbf{ han batalla derecha contra los rusticos } e contralos aldeanos & qui magis uigent prudentia et intellectu , \textbf{ iustum habent bellum contra rusticos , } si recusent subiici illis . \\\hline
2.3.8 & para bien beuir \textbf{ demandan dellas usos corporales con mesura } e non sin mesura & quicunque autem propter ipsum bene viuere diuitias volunt , \textbf{ fruitiones corporales non infinitas quaerunt : } bene enim viuere , \\\hline
2.3.8 & enł primero libro delas politicaͤ \textbf{ e assi commo dixiemos dessuso enl primero } libromas conplidamente de otra gnisa las cosas que son ordenadas & Nam ut distinguit Philosophus primo Polit’ \textbf{ et ut supra in primo libro diffusius diximus , } aliter appetitur finis , \\\hline
2.3.8 & La primera se toma dela semeiança \textbf{ que ha la yconomica e la arte gouernadora de la casa } e m la natura ¶ & Prima sumitur ex similitudine \textbf{ quam habent oeconomica | et ars gubernatiua domus ad naturam . } Secunda , ex similitudine , \\\hline
2.3.8 & La prima razon se declara assi . \textbf{ Ca ueemos que lan atraa non es cuydados a çerca el nudermiento sin mesura } mas tanto pone el nudmiento & Prima via sic patet . \textbf{ Videmus enim naturam non solicitari | circa nutrimentum infinitum , } sed tantum apponit de nutrimento \\\hline
2.3.8 & Bien assi si la natura faze el aialia \textbf{ e la ceratura dela sangre mestrual dela fenbra } e la cera dela leche & Sic etiam si natura \textbf{ ex menstruo facit animal , et ex lacte nutrit ipsum , } non apponit in uberibus infinitum lac , \\\hline
2.3.9 & assi commo del trigo al uino o al ordio o alas otras cosas \textbf{ que cunplen la mengua corporal . } Otra es muda conn delas cosas & uel ad ordeum , \textbf{ uel ad alia supplentia indigentiam corporalem . } Alia est commutatio rerum ad numismata . \\\hline
2.3.9 & e canbio de diueros a dineros \textbf{ ca en el tienpo antiguo los oens } assi commo da a entender el philosofo & vel rerum ad denarios ; \textbf{ sed etiam denariorum ad denarios . Antiquitus enim homines } ( ut satis innuit Philosophus primo Politicorum ) \\\hline
2.3.9 & assi commo si vn omne abondasse en vino \textbf{ e otro abondasse en trigo muda una el vino } entgo & ut si unus abundabat in vino , \textbf{ et alius in frumento , | commutabant vinum ad frumentum , } et per huiusmodi commutationem \\\hline
2.3.9 & assi commo plazia de establesçer en aquel tienpo alos pueblos e alos Reyes . \textbf{ Mas por que era cosa guaue en toda conpra } e en toda uendida pesar si enfͤlos metales & ut placebat tunc temporis populis et regibus instituere . \textbf{ Sed quia difficile erat } in omni emptione vel venditione , \\\hline
2.3.9 & por ende fueron puestas \textbf{ alquas señales en los metales } assi commo ymagen de prinçipe & absoluerentur ab huiusmodi pondere , \textbf{ in ipsis metallis | sculptum fuit signum aliquod , } ut imago principis , \\\hline
2.3.10 & La primera manera dela pecunia es dichͣ \textbf{ por aquello que las cosas naturales se mudan en dineros . } Et por ende si alguno abondasse en vino & Prima ergo species ipsius pecuniatiuae \textbf{ dicitur esse quasi naturalis : | quae fit ex eo quod res naturales commutantur in pecuniam . } Si quis ergo abundans in vino et frumento , \\\hline
2.3.10 & assi commo natural \textbf{ por razon que ha comienco delas cosas naturales . } ¶ la segunda manera de los dineros es dichͣ camiadora . & talis pecuniatiua quasi naturalis diceretur , \textbf{ quia a rebus naturalibus inciperet . } Secunda species pecuniatiuae \\\hline
2.3.10 & por razon de ganer ardiños \textbf{ e Mas esta arte pecumatiua de dineros non deue ser dicha natural } por que non com . & esset causa lucrandi pecuniam . \textbf{ Haec enim pecuniatiua , | naturalis dici non debet ; } quia nec a rebus naturalibus incipit , \\\hline
2.3.10 & en el primero libro delas politicas \textbf{ e el dinero es comienço e fin por que esta e arte comiencaen erl dinero } que omne da & ( secundum Philosophum primo Politicorum ) \textbf{ denarius est elementum et terminus , | idest principium et finis . } Incipit enim haec ars a denario , \\\hline
2.3.10 & e por esso el dinero deue ser dicho comienço e fin \textbf{ ¶La terçera manera del arte pecuniatiua de dineros es obolostica } que quiere dezer arte de peso sobrepuiante que por auentura fue fallada assi . & et principium dici debet . \textbf{ Tertia species pecuniatiuae est obolostatica , } vel ponderis excessiua : \\\hline
2.3.10 & ¶La terçera manera del arte pecuniatiua de dineros es obolostica \textbf{ que quiere dezer arte de peso sobrepuiante que por auentura fue fallada assi . } Ca assi commo la massa del metal es partida en los dineros & Tertia species pecuniatiuae est obolostatica , \textbf{ vel ponderis excessiua : | quae forte sic inuenta fuit . } Nam sicut massa metalli \\\hline
2.3.10 & por que pare por que paresçe que esta pare e engendradinos la qual arte nos \textbf{ por nonbre comunal llamamos usura } por que niguas cosas nunca cresçen en ssi mismas & et generare denarios , \textbf{ quam nos communi nomine appellamus usuram : } nunquam enim aliqua crescunt \\\hline
2.3.10 & e tienpo quissiesse auer doze \textbf{ la qual cosa faze el arte pecumatiua dela usura } assi con & post aliquod tempus vult habere duodecim \textbf{ quod facit pecuniatiua usuraria , ut plane patet , } vult quod denarii illi pariant et generent : \\\hline
2.3.11 & e para otras cosas necessarias \textbf{ a quarta manera de arte pecumatiua de diueros } la qual llama el philosofo & in possessionibus et in redditibus , ex quorum fructu pro defensione regni et aliis necessariis possunt abundare pecunia . \textbf{ Quarta species pecuniatiuae , } quam Philosophus appellat tachos , \\\hline
2.3.11 & que se faganveite quiere \textbf{ que las cosas artifiçiales crezcan } e se amuchiguen en simiłmos & ut quod decem post lapsum temporis fiant viginti , \textbf{ vult quod artificialia seipsa multiplicent : } et quia hoc est contra naturam artificialium , \\\hline
2.3.11 & Lo segundo podemos mostrar \textbf{ que esta manera de arte pecumatiua es de denostar } por el otro nonbre & Secundo huiusmodi pecuniatiuam possumus ostendere \textbf{ detestabilem esse } ex alio nomine quo nominatur , \\\hline
2.3.11 & Et por ende assi faziendo non robanada \textbf{ nin toma uso ageno si retiniendo en ssi el señorio dela casa vede la morada } e el uso della mas en los dineros non es & et nihil usurpat , \textbf{ si retinens sibi dominium domus , | uendit inhabitationem , } et usum eius . \\\hline
2.3.11 & para paresçer con ellas . \textbf{ La qual cosa fazen los rảcadores muchͣs uezes } por que parescan ricos & sed ad apparendum : \textbf{ quod forte multotiens mercatores faciunt , } qui ut appareant diuites , \\\hline
2.3.12 & La vna es dichͣ possessoria e de possessions ¶ \textbf{ La segunda mercatiua e de mercadurias ¶ } La terçera merçenaria & Quarum una dicitur possessoria . \textbf{ Secunda mercatiua . } Tertia mercenaria vel conducta . \\\hline
2.3.12 & ¶ la quarta espunental e de praeua¶ \textbf{ La quinta artifiçial e por artifiçio ¶ } Por la primera manera de possessiones se ganan los aueres & Quarta experimentalis . \textbf{ Quinta artificia . } Via autem possessoria acquiritur pecunia , \\\hline
2.3.12 & en que gana algo ¶ \textbf{ la quarta manera es dicho esperimental de praeua . por quela praeua es delas cosas particulares . } Et por ende quando alguno conosçelos fechs particulares de algunos omes & vel precio conductus aliqua operatur . \textbf{ Quarta via dicitur : experimentalis : | experimentum enim particularium est . } cum ergo quis nouit particularia facta aliquorum , \\\hline
2.3.12 & la quarta manera es dicho esperimental de praeua . por quela praeua es delas cosas particulares . \textbf{ Et por ende quando alguno conosçelos fechs particulares de algunos omes } por los quales fechos ganaron alguas riquezas & experimentum enim particularium est . \textbf{ cum ergo quis nouit particularia facta aliquorum , } quibus pecuniam sunt lucrati , \\\hline
2.3.12 & Et por ende el que quiere gana rriqueza \textbf{ conuiene le de tener enla memoria estos fechs particulares e otros semeiantes } por los quales algunos ganaron grandes algos & pro suae voluntatis arbitrio : volentem ergo pecuniam acquirere , \textbf{ oportet haec | et similia particularia gesta , } per quae aliqui pecuniam sunt lucrati , \\\hline
2.3.12 & que por si o por otros ayan prouada \textbf{ de saber las condiconnes particulares del regno } e los fechs particulares de los sus anteçessor & vel per alios esse expertos , \textbf{ sciendo particulares conditiones regni , } et gesta particularia praedecessorum suorum , \\\hline
2.3.12 & de saber las condiconnes particulares del regno \textbf{ e los fechs particulares de los sus anteçessor } ssegunt los quales gana una conueniblemente so dineros . & sciendo particulares conditiones regni , \textbf{ et gesta particularia praedecessorum suorum , } secundum quae licite pecuniam acquirebant . \\\hline
2.3.12 & e las otras cosas tales \textbf{ mas avn en las posessiones muebles . } Ca conuiene a ellos de resplandesçer & et caetera huiusmodi : \textbf{ sed etiam in possessionibus mobilibus . } Decet enim ipsos pollere multitudine bestiarum , \\\hline
2.3.13 & e en conpara conn delas otras . \textbf{ Assi commo si muchas uozes fiziess en alguna armonia o concordança de canto . } Conuerna de dar y alguna bos & nisi sit ibi aliquid praedominans respectu aliorum : \textbf{ ut si plures voces efficiunt aliquam harmoniam , } oportet ibi dare aliquam vocem praedominantem , \\\hline
2.3.13 & Por que cada vna aianlia es departida en cuerpo e en alma \textbf{ assi commo en partes essençiales . } En la qual particion el alma es & quodlibet enim animal tanquam \textbf{ in partes naturales diuiditur in corpus et animam : } ubi anima est quasi dominans , \\\hline
2.3.13 & e de ser subietos a otros . \textbf{ Et pues que assi es assi commo en el ome uirtuatso e bien dispuesto . } e bien ordena de el alma en ssennorea & et ut aliis sint subiecti . \textbf{ Sicut ergo in homine virtuoso , | et bene disposito , } anima dominatur , \\\hline
2.3.13 & por desordenança dela poliçia e dela çibdat . \textbf{ Ca assi commo en el omne pestilençial e malo } e que ha el almo desordenada e el cuerpo & ex peruersitate politiae : \textbf{ nam sicut in homine pestilente , } et habente animam peruersam ; \\\hline
2.3.13 & assi enlas poliçias \textbf{ delas çibdades pestilençiosas e desordenadas } mas & magis quam anima vel ratio : \textbf{ sic in politiis pestilentibus , } et corruptis magis dominantur ignorantes , \\\hline
2.3.13 & de seruir alos omes \textbf{ assi cosa naturales alos non sabios } de ser suiebtos alos sabios . & sicut naturale est bestias seruire hominibus , \textbf{ sic naturale est ignorantes } subiici prudentibus \\\hline
2.3.13 & ¶La quarta razon se toma \textbf{ por departimiento de los linages masculino e femenino } o por conparaçion delas fenbras a los uatones . Ca ueemos que por que el uaton es mas ennoblesçido en razon & Quarta via sumitur ex diuersitate sexum , \textbf{ vel ex compassione virorum ad foeminas . } Videmus enim virum , \\\hline
2.3.14 & e sin sabiduria deuen puir a los sabios . \textbf{ es de dar serudunbre legal de ley puesta por los omes } segunt la qual los flacos e los vençidos & secundum quam ignorantes debent seruire sapientibus , \textbf{ esset dare seruitutem legalem , | et quasi positiuam , } secundum quam debiles et victi \\\hline
2.3.14 & deuen seruir alos fuertes e alos beçedores . \textbf{ Ca decho eslegal segunt } que dize el philosofo & seruirent victoribus et potentibus . \textbf{ Est enim iustum legale , } ut recitat Philosophus 1 Politicorum superatos \\\hline
2.3.14 & segunt los bienes del alma \textbf{ puede ser dichͣa uentaia sinple mente . } Mas el auentaia que es & Excessus autem secundum bona animae , \textbf{ est quasi excessus simpliciter , | eo quod illa bona simpliciter dici possunt . } Excessus vero secundum bona corporis respectu illius , \\\hline
2.3.14 & e sinplemente e sin condiconn . \textbf{ Ca cosa digna es } e con razon que los sabios & reddunt dominium naturale et simpliciter : \textbf{ est enim dignum naturaliter } et simpliciter sapientes \\\hline
2.3.14 & e bienes de fuera \textbf{ non fazen sennorio natural sinplemente . } Mas mas fazen sennorio legal & quae sunt bona corporalia et exteriora , \textbf{ non faciunt dominium simpliciter naturale , } sed magis faciunt ipsum legale et positiuum . \\\hline
2.3.14 & Ca aquel que ha auentaia en los bienes del cuerpo \textbf{ assi commo en fortaleza o en poderio ciuil deue } enssennorear & Quod enim superans in bonis corporis , \textbf{ ut in fortitudine , | vel in ciuili potentia , } dominetur iis quos debellauit , \\\hline
2.3.14 & por que la ley diesse iuyzio fuerte \textbf{ e de cosa çierta . } Visto fue alos establesçedores delas leyes & quam animae et interiora : \textbf{ ut lex daret iudicium de aliquo certo , } visum fuit legum latoribus , \\\hline
2.3.14 & La terçera razon se toma dela salud de los vençidos . \textbf{ Ca por esta ley muchͣs vezes los vençidos enla batalla son saluos } por que los otros oms vençedores & ex salute debellatorum : \textbf{ nam propter hanc legem multotiens superati in bello saluantur : } homines enim alios debellantes proniores essent ad homicidium , \\\hline
2.3.15 & tenporal esto deue ser despues de aquel bien que entiende . \textbf{ Mas conuiene de dar a ministraçion de alquiler e de amor sin la ministt̃ion natural et segunt ley . } Ca por que en nos es el appetito corrupto & hoc debet esse ex consequenti . \textbf{ Oportuit autem dare ministrationem conductam et dilectiuam | praeter ministrationem naturalem } et secundum legem : \\\hline
2.3.15 & non guardamos \textbf{ sienpre la orden natural . } Et por ende en la mayor parte los prinçipados e los sennorios son malos & nam quia est in nobis corruptio appetitus , \textbf{ et non semper reseruamus ordinem naturalem , } ut plurimum principatus sunt peruersi : \\\hline
2.3.15 & por gouernamiento seruil \textbf{ mas por gouernamiento paternal e real } Esto podemos mostrar & et decet eos regere non regimine seruili , \textbf{ sed magis quasi paternali et regali . } Possumus autem duplici via ostendere , \\\hline
2.3.15 & que han estos sirmentes tales al señor . \textbf{ Ca paresçe que cosa digna es } que las partes que son mas çeranas dela fuerte mas abonden & quam habent huiusmodi ministri ad principantem . \textbf{ Dignum est enim } ut partes propinquiores fonti plus profundantur aqua : \\\hline
2.3.15 & segunt voluntat al prinçipe que los otrs . \textbf{ por la qual cosa digna cosa es } que ellos reçiban & quam alii . \textbf{ Quare dignum est ipsos } plus de influentia recipere , \\\hline
2.3.16 & assi acomnedados alos seruientes \textbf{ por que sea y guardada la orden conuenible del seruiçio } la qual cosa se pue de bien conplir & sic esse committenda ministris , \textbf{ ut reseruetur ibi debitus ordo ministrandi : } quod maxime fieri contingit , \\\hline
2.3.16 & en el segundo libro delas politicas \textbf{ do dize que alguas vezes peor siruen los muchs seruidores que los pocos . } Ca quando vn seruiçio es a comnedado a muchs & ubi dicitur , \textbf{ quod aliquando deterius seruiunt | multi ministrantes quam pauci . } Nam cum multis idem ministerium committitur ; \\\hline
2.3.16 & Et pues que assi es aquello que es dicho dela çibdat pequanan \textbf{ e grande deue se entender avn dela casa pequana e grande } por que en la casa pequeña & de ciuitate parua et magna , \textbf{ intelligendum est de parua et magna domo . } In domo enim parua , \\\hline
2.3.16 & son en toda manera los ofiçios de partir \textbf{ e de dar a much s ofiçiales . } Et non son muchs ofiçios & et ubi officia maximam curam habent , \textbf{ sunt omnino officia particulanda , et distinguenda , } et non sunt plura committenda eidem , \\\hline
2.3.16 & que les son acomnedados \textbf{ por que son engannadores e menguadores de los derechos reales . } as otros ay que fazen mal el ofiçio & nam aliqui male exequuntur opus iniunctum , \textbf{ quia sunt decipientes , | et defraudantes iura legalia : } aliqui vero male consequuntur ipsum , \\\hline
2.3.17 & or que mucho paresçe la sabiduria del Rey \textbf{ si gouernare en manera conuenible a su conꝑannappra } e sil diere ordenadamente & Quia maxime apparet Regis prudentia , \textbf{ si suam familiam debito modo gubernet , } et si ei debite et ordinate necessaria tribuat : \\\hline
2.3.17 & e conuenble mente las colas nesçessarias \textbf{ e por que prouision conueible delas uestiduras } mayormente parte nesçe a estado de onrra & et si ei debite et ordinate necessaria tribuat : \textbf{ et quia debita prouisio maxime videtur } facere ad honoris statum , \\\hline
2.3.17 & e por que prouision conueible delas uestiduras \textbf{ mayormente parte nesçe a estado de onrra } por ende por que los Reyes e los prinçipes sean ensennados & et quia debita prouisio maxime videtur \textbf{ facere ad honoris statum , } ut instruantur Reges , et Principes , \\\hline
2.3.17 & e entre todos estos \textbf{ algsson mayores e algs menores } e por ende conuiene a ellos de ser honrrados de uestiduras en departidas maneras & et inter utrosque quidam superiores , \textbf{ et quidam inferiores : } propter quod decet \\\hline
2.3.17 & e por ende non deuen los seruientes gozar egualmente de vn apareiamiento fermoso \textbf{ nin deuen gozar egualmente de uestiduras fermosas . } Mas penssada la condiçion delas personas & non omnes ministrantes aequae pulchro apparatu , \textbf{ nec aeque pulchris indumentis gaudere debent , } sed considerata conditione personarum sic secundum suum statum cuilibet sunt talia tribuenda , \\\hline
2.3.17 & ¶ Lo quinto que es de penssar çerca desto \textbf{ que son las cosas conuemientes alos tienpos . } Ca commo estos cuerpos de aqui deyuso sean gouernados & nisi consuetudines illae sint penitus corruptelae . \textbf{ Quinto circa hoc considerandum occurrit congruentia temporum . } Nam cum haec inferiora corpora per super caelestia regantur \\\hline
2.3.18 & Et otrossi por que muchos oios catan aellos comunalmente mas uerguença toman que los otros \textbf{ e mas desdennan de obrar cosas torpes que los otros Et pues que assi es por que los nobles omes } segunt linageson en tal estado en que paresçe que esrtov cosa prouable & et dedignantur operati turpia , \textbf{ quam alii . | Quia ergo nobiles homines } secundum genus sunt in statu , \\\hline
2.3.18 & que los \textbf{ otrossi cosa prouable es } que de los bueons nasçen buenos & et secundum quem decet \textbf{ eos esse meliores aliis ; } si probabile est ex bonis bonos , \\\hline
2.3.18 & assi commo paresçe por lo que dicho es \textbf{ Et por ende cosa prouable es } que ellos son sabios e buenons . & et tales ( ut patet per praehabita ) \textbf{ probabile est esse prudentes , et bonos . } Huic autem probabilitati aliquando subest falsitas , \\\hline
2.3.18 & que ellos son sabios e buenons . \textbf{ Enpero en esta opinion prouable algunas uezes } ay falssedat & probabile est esse prudentes , et bonos . \textbf{ Huic autem probabilitati aliquando subest falsitas , } quia ( ut dicitur in Politicis ) \\\hline
2.3.18 & mas fallesçe por algun enbargo \textbf{ por que alguons nobles de linage de su naian se } e desdizense de uerdadera nobleza & sed deficit . \textbf{ Quidam enim nobiles genere degenerant } a naturae nobilitate , \\\hline
2.3.18 & por buenos de ser tales segunt uerdat . \textbf{ Et por ende cosa conueinble es } que los nobles & qui creduntur et aestimantur boni , \textbf{ esse tales secundum veritatem , } decens est nobiles genere esse nobiles secundum mores . \\\hline
2.3.18 & por conparacion a la nobleza delas costunbres \textbf{ ca assi la commo la iustiçia legales toda uirtud } por conparaçion al cunplimiento dela ley & per comparationem ad nobilitatem morum , \textbf{ sicut legalis iustitia est tota virtus } per respectum ad impletionem legis . \\\hline
2.3.18 & por que quiera cunplir la ley \textbf{ que lo manda la qual cosa faze el iusto legal . } Mas por que el quiere retener las costunbres dela corte & implere legem hoc precipientem , \textbf{ quod facit iustus legalis : } sed quia volunt retinere mores curiae et nobilium , \\\hline
2.3.19 & e generalmente todos los sennores se de una auer cerca los seruientes \textbf{ mas la manera conuenible de se auer cerca ellos paresçe que esta en çinco cosas . } ¶ Lo primero que les sean acomnedados los ofiçios conueiblemente ¶ & erga eos debeant se habere . \textbf{ Debitus autem modus se habendi | circa ipsos } quasi in quinque videtur consistere . \\\hline
2.3.19 & por qua non sea engannado . \textbf{ por que quanto mas es omne çierto de su fieldat } e de su sabiduria & prudens ne defraudetur : \textbf{ quanto plus constat } de eius fidelitate et prudentia , \\\hline
2.3.19 & ca algunos son sieruos naturalmente \textbf{ e algersson seruientes por ley } e alg ssiruen prinçipalmente & quia quidam sunt serui naturaliter , \textbf{ quidam ex lege , } quidam vero seruiunt principaliter pro mercede , \\\hline
2.3.20 & que tal cosa commo esta \textbf{ contradize ala orden natural . } ¶ La segunda de aquello que contradize ala bondat delas buenas costunbres . & Prima via sumitur \textbf{ ex eo quod hoc repugnat ordini naturali . } Secunda ex eo quod contradicit bonitati morum . \\\hline
2.3.20 & assi conmoparagostar e para fablar . \textbf{ Et por ende contra orden natural es } que quando nos usamos deste estrumento & ut in gustum , \textbf{ et locutionem contra naturalem ordinem est , } cum huiusmodi organum exercemus \\\hline
2.3.20 & que es el fablar \textbf{ contradize ala orden natural . } Lo segundo contradize ala bondat delas costunbres . & quod est loqui , \textbf{ repugnat ergo hoc ordini naturali . } Secundo repugnat bonitati morum . \\\hline
2.3.20 & Lo segundo contradize ala bondat delas costunbres . \textbf{ Ca si los que se assientan enla mesa sedana mucho fablar } por que paresçe que el vino acresçienta las fablas & Secundo repugnat bonitati morum . \textbf{ Nam si recumbentes | in mensa erga multiloquium insistant , } quia vinum videtur \\\hline
2.3.20 & e los prinçipeᷤ alos quales conuiene ser muy tenprados \textbf{ e guardar la orden natural en toda } meranera deuen ordenar en sus mesas & quos decet maxime temperatos esse , \textbf{ et obseruare ordinem naturalem } omnino in suis mensis , \\\hline
2.3.20 & Et avn esto mismo conuiene generalmente a todos los çibdadanos \textbf{ por que avn cosa conuenible es aellos de auer uirtudes e bueans costunbres . } Mas alos que son assentados en las mesas & etiam hoc uniuersaliter nobiles et omnes ciues , \textbf{ quia congruum est } etiam \\\hline
2.3.20 & Et pues que assi es esto \textbf{ e o triscosas aprouechosas deuen leer alas mesas de los Reyes } e de los prinçipes dela trra & Haec ergo , \textbf{ vel alia utilia tradita } secundum vulgare idioma , \\\hline
2.3.20 & commo quier que al gunas o trisco las particulares se podiessen tractar . \textbf{ Enpero por que todas las cosas ꝑticulares non caen sonairaçion } e so cuento propusiemos de passar las en silençio & licet quaedam alia particularia tractari possent , \textbf{ tamen quia non omnia particularia sub narratione cadunt , } proponimus ea silentio pertransire , \\\hline
3.1.1 & a algun bien alguas uezes \textbf{ a inclinaçion e amouemiento naturala aquel bien . } Et alguas uezes somos inclina dos a aquel bien & ad aliquod bonum , \textbf{ aliquando ad bonum illud habemus impetum a natura , } aliquando quasi ex corruptione naturae . \\\hline
3.1.1 & que es llamada \textbf{ por nonbre comunal çibdat . } Enpero conuiene de saber & haec autem est communitas politica , \textbf{ quae communi nomine vocatur ciuitas . } Aduertendum tamen , \\\hline
3.1.2 & dela qual fablamos aqui \textbf{ que es disposicion moral de alma } para escoger entre el bien e el mal & de qua hic loqui intendimus , \textbf{ participare non possunt } nisi rationalia et habentia intellectum . \\\hline
3.1.2 & e el beuir conplidamente \textbf{ e el benir uirtuosa mente . } Ca en qual si quier manera & sufficienter viuere , \textbf{ et virtuose viuere . } Qualitercumque etiam homo habeat esse , viuit : \\\hline
3.1.2 & que abastan \textbf{ conueinblemente a conplir la mengua corporal . } Et por ende otra cosa es beuir & nisi habeat ea quae congrue sufficiunt \textbf{ ad supplendam indigentiam corporalem , } aliud est ergo viuere aliud sufficienter viuere . \\\hline
3.1.3 & e çiuil \textbf{ e escogen uida solitaria e montanensa } assi commo los hermitaons . . & Videmus autem multos societatem politicam retinentes , \textbf{ eligere solitariam vitam , et campestrem . } Sed hae et aliae dubitationes \\\hline
3.1.3 & ca non es esto assi natural al omne \textbf{ conmoes cosa natural al fuego de escalentar } e ala piedra de desçender ayuso & Non enim hoc est sic homini naturale , \textbf{ sicut est naturale igni calefacere , } et lapidi deorsum tendere : \\\hline
3.1.3 & por algun enbargo o por algua otra cosa bien \textbf{ assi maguera natural cosa sea al omne de beuir çiuil mente . } Enpero muchos son fallados canpesinos e montanneses & vel ex aliqua causa reperiuntur abdextri : \textbf{ sic licet naturale sit homini viuere ciuiliter , } reperiuntur tamen multi campestre viuentes . \\\hline
3.1.4 & que fazen uida contenplatiua \textbf{ el capitulo sobredicho soluiemos las obiecconnes contrarias } por las quales se prouaua & vel est deus . \textbf{ Remouebantur in praecedenti capitulo obiectiones contrariae , } per quas probari uidebatur , \\\hline
3.1.4 & assi ca fue prouado dessuso \textbf{ que beuir es cosa natural al omne } e por que la nafa non pueda fallesçer en las cosas neçessarias & Probabatur enim supra , \textbf{ quod uiuere erat homini secundum naturam ; } ut natura non deficiat in necessariis , \\\hline
3.1.4 & conuiene \textbf{ que sea cosa natural todo aquello } que sirue a conplimiento de uida & ut natura non deficiat in necessariis , \textbf{ oportet quid naturale esse } quicquid secundum se deseruit \\\hline
3.1.4 & por ende conuiene \textbf{ que la çibdat lea comuidat natraal ¶ } La segunda razon para prouar & quam communitates illae , \textbf{ oportet eam esse secundum naturam . } Secunda uia ad inuestigandum hoc idem , \\\hline
3.1.4 & lo que es fin dela generaçion \textbf{ delas cosas naturales es cosa natural } e es nata delas cosas engendradas & quod est finis generationis naturalium , \textbf{ est quid naturale , } et est natura ipsorum generatorum : \\\hline
3.1.4 & assi commo lo que es fin dela generaçio del omne \textbf{ es cosa natural al omne } e es nata del omne & ut quod est finis generationis hominis \textbf{ est quid naturale , } et est natura eius . \\\hline
3.1.4 & es la forma de cada cosa \textbf{ la qual forma por sobrepuiança es cosa natraal e es essa misma natura . } mas ueemos que la comuidat dela casa es cosa natural & quae per antonomasiam est \textbf{ quid naturale , | et est ipsa natura . } Videmus autem quod communitas domus est quid naturale , \\\hline
3.1.4 & mas avn por que aquellas comuindades \textbf{ que acaban la casa son cosa natural . } ca la casa se faze de comuidat de omne e de su muger e de sennor e de sieruo e de padre e de fijos . Et cada vna destas comuidades es cosa segunt natura bien & sed etiam quia communitates illae , \textbf{ quae perficiunt domum , | sunt quid naturale : } constat enim domus \\\hline
3.1.4 & Et pues que assi es iusto \textbf{ que la çibdat es cosa natural . } finca de demostrar & tanquam ad finem et complementum , \textbf{ ordinatur ad ciuitatem . Viso , ciuitatem esse aliquid secundum naturam : } reliquum est ostendere , \\\hline
3.1.4 & e de cosa tristable \textbf{ por que ayan seso delectable e tristable } ca estas se cosas demuestran las aialas vna a otra & eis signum delectabilis et tristabilis , \textbf{ ut habeant sensum delectabilem et tristabilem : } hoc enim sibi inuicem per vocem significant . \\\hline
3.1.4 & que sea natural \textbf{ ca la cosa iusta e la cosa } que non es iusta & quae sunt apta nata exprimi per sermonem : \textbf{ iustum enim et iniustum non proprie habet } esse in communitate domestica , \\\hline
3.1.4 & e para fazer çibdat \textbf{ mas commo aquello a que auemos inclinaçion natural sea cosa natural } conuiene quela çibdat sea cosa natural & et ad constituendum ciuitatem . \textbf{ Sed cum id , | ad quod habemus impetum naturalem , } sit secundum naturam , \\\hline
3.1.5 & que sin la comunidat dela çibdat \textbf{ cosa aprouechosa fue ala uida humanal } de establesçer comunidat de regno ¶ & quod praeter communitatem ciuitatis , \textbf{ utile est humanae vitae } statuere communitatem regni . \\\hline
3.1.5 & assi commo el uanio \textbf{ en que se usa el arte de car pentena o de ferteria } ha mester otro uarrio & Sicut ergo vicus , \textbf{ in quo exercetur ars fabrilis , } indiget vico alio , \\\hline
3.1.5 & por la qual cosa \textbf{ assi commo cosa apuechosa es ala uida humanal } que en vna çibdat sean ayuntados muchos uarrios & unam indigere auxilio alterius . \textbf{ Quare sicut utile est | vitae humanae } in eadem ciuitate \\\hline
3.1.5 & e el vn mienbro ha mester seruiçio del otro por la qual cosa . \textbf{ cosa aprouechosa es alos mienbros de ser ayuntados en vn cuerpo } por que se acorran los bnos alos otros bien & propter quod utile est \textbf{ ipsis membris congregari in uno corpore , } ut sibi inuicem subueniant : \\\hline
3.1.5 & so vn Rey aqui pertenesçe de defender a cada vna parte del regno \textbf{ e ordener el poderio çiuil delas otras çibdades } a defendimiento de cada vna delas çibdades del regno & cuius est quemlibet partem regni defendere , \textbf{ et ordinare ciuilem potentiam aliarum ciuitatum } ad defensionem cuiuslibet ciuitatis regni ; \\\hline
3.1.5 & por mas ligero defendemiento \textbf{ e mas seguro cosa aprouechosa fue } que de muchͣs comuidades publicas se ssziese vna comuidat de regno & si contingat eam ab extraneis impugnari , \textbf{ propter faciliorem defensionem et tuitionem utile fuit } ex pluribus communitatibus politicis constituere communitatem unam regni . \\\hline
3.1.6 & la qual cosa se faze \textbf{ por la conpannia çiuil . } En essa misma manera han inclinacion natraal & et ut sufficiant sibi in vita , \textbf{ quod fit per ciuilem societatem : } sic naturalem habent impetum \\\hline
3.1.6 & senorear alguna çibdat \textbf{ por tirania e por poderio ciuil apremiasse las otras çibdades } e se feziesse rey sobre ellos . & ut si quis dominans alicui ciuitati per tyrannidem \textbf{ et per ciuilem potentiam | opprimat ciuitates alias , } et faciat se Regem constitui super illas . \\\hline
3.1.6 & e del regno de dezer e de contar \textbf{ que sintieron los philosofos antiguos de tal gouernamiento } e en qual manera eran las çibdades gouernadas en el tp̃o antiguo & propter sufficientiam artis regiminis ciuitatis et regni citare \textbf{ quid senserunt Philosophi de huiusmodi regimine , } et quomodo antiquitus regebantur ciuitates . \\\hline
3.1.7 & enla su mecha phisica \textbf{ conuirtiosse alascina moral . } al qual socrates siguio platon su disçipulo en muchͣs cosas & ut narrat Philosophus in Metaphysica sua , \textbf{ conuertit se ad Moralia , } quem Plato suus discipulus in multis secutus est , \\\hline
3.1.7 & por la qual cosa el philosofo aristotiles llamo a platon el segundo socrates . \textbf{ Mas determinado socrates e platon delas cosas morales paresçe } que pusieron e dixieron cerca çinco cosas ser la çibdat & secundum Socratem appellauit . \textbf{ Determinando autem Socrates et Plato de moralibus , } quinque tetigisse videntur \\\hline
3.1.7 & que pusieron e dixieron cerca çinco cosas ser la çibdat \textbf{ e el gouernamiento çiuil . } ¶ Lo primo que dixieron es & circa ciuitatem \textbf{ et regnum ciuile . } Primum est , quia dixerunt ciuitatem debere esse maxime unam . \\\hline
3.1.7 & e sienpre la hunidat \textbf{ paresçeser meior que la muchedunbre . } Ande dizen soctates e platon que la primera cosa & et semper unitas videtur \textbf{ esse potior multitudine } unde prima causa Deus ipse quia est summe unus , \\\hline
3.1.7 & que todas las cosas de uun ser comunes a cada vno \textbf{ assi que ouiessen las possessiones comunes e las mug̃es comunes } en tal manera que cada vno se allegasse & est , quia dixerunt ciuibus omnia debere esse communia : \textbf{ ut quod haberent communes possessiones communes uxores , } ut quod quilibet accederet ad quamlibet , \\\hline
3.1.7 & e assi se signiria \textbf{ que ouiessen los fijos comunes . } ca por el uso comun delas fenbras los padres non ian ètificados de sus fijos & ut quod quilibet accederet ad quamlibet , \textbf{ et per consequens haberent communes filios : } nam quia propter communem usum foeminarum patres \\\hline
3.1.7 & que ouiessen los fijos comunes . \textbf{ ca por el uso comun delas fenbras los padres non ian ètificados de sus fijos } mas cuydarian de cada vno moço & et per consequens haberent communes filios : \textbf{ nam quia propter communem usum foeminarum patres | non certificarentur } de propriis filiis \\\hline
3.1.7 & que ellos eran sus padres . \textbf{ ¶ Lo terçero que sintieron los dichs philosofos cerca el gouernamiento dela çibdat . } es que dixieron & eos esse suos patres . \textbf{ Tertium vero quod senserunt | dicti Philosophi } circa regimen ciuitatis , \\\hline
3.1.7 & que los mallos calicuydaremos en las aues \textbf{ que biuen de rapina mayores son de cuerpo } e mas osadas de coraçon & nam si consideramus aues ipsas viuentes ex raptu , \textbf{ maiores corpore et audaciores corde } et praestantiores viribus sunt foeminae quam masculi : \\\hline
3.1.7 & assi comm̃los prinçipantes de mayor prinçipado \textbf{ e semeia alos que estan en mayores senno rios } e la uena dela plata es & vena auri ut principantes maiori principatu , \textbf{ vel existentes in maioribus principatibus ; } et vena argenti \\\hline
3.1.7 & assi que sean subditos \textbf{ e sean despuestos de sů maestradgos . } Mas lo quinto que los dichs phos sintieron & ut quod fiant subditi , \textbf{ et deponantur a magistratibus suis . } Quintum autem quod dicti Philosophi senserunt statuendum circa ciuitatem , \\\hline
3.1.8 & ca en muchos espeçies \textbf{ e semeianças delas cosas se salua mayor perfectiuo } que en vna tan sola mente . & oportet ibi dare species diuersas ; \textbf{ ut in pluribus speciebus entium reseruetur maior perfectio , } quam in una tantum . \\\hline
3.1.8 & La primera se toma \textbf{ quanto al ser dela çibdat misma . } ¶ La segunda por conparacion ala hueste . & Prima sumitur quantum \textbf{ ad esse ipsius ciuitatis . } Secunda per comparationem ad exercitium . \\\hline
3.1.8 & que morassen en ella se feziessen vna perssona \textbf{ ya non seria casa mas serie vn omne singular . } En essa misma manera si el uarrio en tanto se feziesse vno que se feziesse vna casa & quod omnes habitantes in ipsa fieret una persona , \textbf{ iam non remaneret domus , | sed fieret homo unus aliquis singularis : } sic si tantum uniretur vicus , \\\hline
3.1.8 & ca este defendimiento es prouechable \textbf{ seg̃t la quantidat mayor de los omes } commo quier & nam hoc quidem scilicet compugnatio utilis est \textbf{ secundum quantitatem , } quamuis sit idem specie : \\\hline
3.1.8 & e commo en la çibdat conuenga de dar alguons ofiçioso \textbf{ alguons maestradgos o algunas alcaldias } la qual cosa non seria & dare aliquos magistratus , \textbf{ et aliquas praeposituras , } quod non esset , \\\hline
3.1.8 & assi commo es dicho \textbf{ para abastamiento deuida son meester muchͣs cosas departidas } por ende conuiene enla çibdat de auer en ssi algun departimiento & sed ut dictum est \textbf{ ad sufficientiam vitae | requiruntur diuersa , } ideo oportet ciuitatem \\\hline
3.1.9 & dano realmente crie el cuerpo del otro . \textbf{ Et pues que assi es estando las possessiones comunes conuernia a cada vno } de departir aquellas cosas & nutriat corpus alterius . \textbf{ Existentibus ergo possessionibus communibus oporteret | cuilibet distribui } quae requiruntur \\\hline
3.1.9 & que segunt ella \textbf{ losoens puedan beuir comunal mente . } Onde el pho dize en el terçero libro delas politicas & et toti populo debet esse talis , \textbf{ secundum quam homines communiter possint uiuere . } Unde et Philosophus ait in Politicis ciues \\\hline
3.1.9 & que pertenezca a todo el pueblo \textbf{ e ala uida comunal del pueblo } e non fue tal ordenamiento de socrates & et talis debet esse ordinatio ciuitatis , \textbf{ quae competat omni populo et communi uitae ; } cuiusmodi non fuit ordinatio Socratis , \\\hline
3.1.9 & assi conmo sobrino \textbf{ e el pho llama primos alos } que son fijos de dos hͣmanos . & alius tanquam fratruelem \textbf{ ( appellat autem Philosophus fratrueles } qui sunt ex duobus fratribus nati ) \\\hline
3.1.10 & El terçero mal se declara \textbf{ assi ca conmo de la comunidat sobredichͣ delas mugiets } e de los fijos se sigua iniuria et tuerto de los fijos & Tertium malum sic declaratur . \textbf{ Nam supposita praedicta communitate , } sequeretur in curia filiorum \\\hline
3.1.10 & Mas esto reprahende el philosofo enel segundo libro delas politicas \textbf{ ca por dos o por tres o por pocos mocos querera mar grant muchedunbre de moços } assi conmo a fijos propreos & propter duos vel tres \textbf{ vel propter paucos pueros velle magnam multitudinem diligere puerorum tanquam proprios filios , } hoc est ponere parum de melle in multa aqua . \\\hline
3.1.10 & non se puede auer \textbf{ er cuydado conuenible de los moços } e dende se sigue & non habebitur eorum cura debita : \textbf{ sequitur quod supposita communitate , } quam ordinauerat Socrates , \\\hline
3.1.10 & e la cobdiçia del appetito del omne estan sin fartura \textbf{ que avn que el o en non ouiesse si non vna mugnia vn seria cosa guaue que el se ouiesse conueinblemente } e tenpradamente en vsar della & et sic est insatiabilis concupiscentiae appetitus , \textbf{ quod non habente viro nisi unam uxorem , | adhuc est valde difficile debite } et temperate se habere erga illam . \\\hline
3.1.10 & quando esta muy de pierta \textbf{ e muy golola por grant muchedunbre de uiandas es cosa guaue de fazer al omne astinençia . } assi el appetito de luxuria & per multitudinem ciborum \textbf{ difficile est esse abstinentes , } sic prouocaris venereis \\\hline
3.1.11 & quando el vno enbarga al otro del uso \textbf{ e del fructo de aquella cosa comun de alli se leuna talid e discordia entre ellos caveemos } que los hͣrmanos & propter alium impeditur , \textbf{ consurgit lis et discordia inter eos . } Videmus enim fratres uterinos \\\hline
3.1.11 & de auer las cosas \textbf{ e las possessiones prop̃as quanto al sennorio . } ca cada vn sennor de sus bienes propreos aura mayor acuçia de aquellos bienes & expedit cuilibet habere res \textbf{ et possessiones proprias quantum ad dominum : } nam quilibet dominans bonis propriis \\\hline
3.1.12 & e tomaua enxienplo delas bestias \textbf{ entre las quales las fenbras lidianta bien commo los mas los . } Mas que los Reyes e los prinçipes & assumens exemplum ex bestiis , \textbf{ ex quibus tam mares quam foeminae bellant . } Sed quod Reges et Principes , \\\hline
3.1.12 & de ordenar la çibdat \textbf{ non deuen ordenar las muger sa obras de batallas } nin a lidiar mas los omes & et uniuersaliter illi quorum est ciuitatem disponere , \textbf{ non debeant ordinare mulieres | ad opera bellica , } sed viros , \\\hline
3.1.12 & ca assi commo dize el pho en el terçero libro delas ethicas fin \textbf{ e acabamiento de todas las cosas espantables es la muerte } e por ende las obras dela batalla demandan s omes sin miedo & Nam ut dicitur 3 Ethic’ finis \textbf{ et terminus omnium terribilium , | est mors : } opera ergo bellica requirunt hominem \\\hline
3.1.12 & commo dela frialdat dela conplission \textbf{ ca assi commo es dicho de suso la frialdat apareia carrera al temor } por que el frio a restrennir e apretar & quam ex frigiditate complexionis : \textbf{ nam ( ut dicebatur supra ) | frigiditas viam timori praeparat ; } frigidi enim est costringere et retrahere ; \\\hline
3.1.12 & por que han las carnes muelles \textbf{ e non han fortaleza corporal non deuen ser puestas alas obras delas batallas . } Mas la razon que monio a socrates & mulieres igitur eo quod habent carnes molles \textbf{ et deficiunt a fortitudine corporali , | ad opera bellica ordinari non debent . } Ratio autem quae mouit Socratem \\\hline
3.1.12 & e la çibdat \textbf{ segunt ordenamiento conueinble en aquellas cosas } que las bestias fazen & et secundum debitam dispensationem ordinare domum et ciuitatem . \textbf{ Quare in iis , } in quibus bestiae praeter rationem agunt , eas sequi non debent . \\\hline
3.1.14 & por la quel cosa commo sea manifiesto \textbf{ por las cosas dichͣs de suso } que non conuiene ala çibdat & quid circa huiusmodi regimen sit censendum . \textbf{ Quare cum patefactum sit in praecedentibus , } non expedire ciuitati possessiones , \\\hline
3.1.14 & en cada vna delas çibdades \textbf{ assi que sean çinco miłl omin lesto serie muy graue } e de grant carga alos çibdadanos & in qualibet ciuitate \textbf{ ut quinque milia , } vel etiam mille , \\\hline
3.1.14 & çibdat mantener mill caualleros \textbf{ de las rentas comunes de vna } çibdat los quales caualleros non ouiessen otro ofiçio ninguno & esse valde difficile et onerosum ipsis ciuibus . \textbf{ Onerosum enim et difficile esset ciuibus unius ciuitatis sustentare mille viros in stipendiis communibus , } quorum nullum esset aliud officium , \\\hline
3.1.14 & quantas quisiesse a ssu uoluntad \textbf{ por que pudiesse de las rentas comunes abondar atanta muchedunbre } por la qual cosa el philosofo & oporteret enim ciuitatem illam habere possessiones quasi ad votum , \textbf{ ut posset ex communibus sumptibus } tantam multitudinem pascere . \\\hline
3.1.14 & que uiuiessen de los fructos \textbf{ e de los bienes comunes dela çibdat a tres cosas deuia tener mientes conuiene a saber . } a los çibdadanos & ex communibus fructibus ciuitatis , \textbf{ ad tria deberet respicere . } Primo ad ciues : \\\hline
3.1.14 & e para se defender \textbf{ aurian meester mayor conpanna delidiadores } que los pudiessen ayudar & et inutiles ad bellum , \textbf{ indigeret maiori copia bellatorum sic se habentium . } Secundo inspiciendum esset ad regionem : \\\hline
3.1.14 & Lo terçero deuen tener mientes alos logares \textbf{ quel son uezinos assi commo si aquella çibdat ouiesse çerca de ssi uezinos amigos o enemigos flacos de coraçon o homillosos } Ca departidas las condiçonnes de los uezinos & Tertio aspiciendum esset ad loca vicina , \textbf{ ut utrum ciuitas illa haberet | circa se vicinos amicos vel inimicos , pusillanimes vel viriles . } Nam variatis conditionibus vicinorum , \\\hline
3.1.14 & por la qual cosa la arte e la sçiençia non pueden ser \textbf{ cerca las cosas particulares e senñaladas . } El que quiere dar arte e sçiençia de gouernamiento dela çibdat & Quare cum ars et scientia non possit \textbf{ esse circa particularia signata , } volens tradere artem de regimine ciuitatum , \\\hline
3.1.15 & assi commo las suyas propreas . \textbf{ Et pues que assi es en esta manera deuen ser todas las cosas comunes alos çibdadanos } por que cada vno entienda el bien comun & sicut et proprias . \textbf{ Hoc ergo modo ciuibus omnia debent esse communia , } ut quilibet intendat bonum commune et bonum omnium , \\\hline
3.1.15 & que son demandadas para conplimiento dela uida \textbf{ assi que non fuessen en la çibdat muchͣs casas } e departidas & ad sufficientiam vitae , \textbf{ ut quod non essent in ciuitate plura } et diuersa \\\hline
3.1.16 & ca para ganar possessiones \textbf{ fazen los çibdadanos tuerto vno a otro en sus ppreas personas . } An fazense en la çibdat furtos & Nam pro acquirendis possessionibus \textbf{ inferunt sibi ciues iniurias | et contumelias in personis propriis ; } fiunt autem in ciuitate furta , \\\hline
3.1.17 & destruyr se ya la ley de de felleas \textbf{ que non aurien las possessiones yguales . } Pues que assi es commo quanto alguna cosa es partida en mas partes & succedentibus filiis in haereditates parentum soluta erit lex Phaleae . \textbf{ Quare non habebunt possessiones aequales . } Cum ergo quanto aliquid \\\hline
3.1.18 & en el segundo libro delas politicas era ley \textbf{ que para que las suertes antiguas fuessen guardadas } sin dan no non conuerma a ninguon de uender sus possessiones & ut recitat Philosophus 2 Politic’ lex erat , \textbf{ quod ad hoc ut antiquae sortes seruarentur illesae , } nulli licebat possessiones vendere , \\\hline
3.1.18 & si non podiessen mostrar \textbf{ conplidamente qual contesçiera alguna grant ocasion o algua mala uentura } por que son muchͣs cosas de determinar en las leyes çerca las possession s & nisi posset sufficienter ei ostendere \textbf{ aliquod magnum infortunium accidisse . } Sunt enim multa determinanda \\\hline
3.1.18 & que las sustançias de fuera ¶ La terçera razon se toma de parte delas rstezas las quales los omes fuyen por su poder . \textbf{ ca lo s ons quieren gozar de delecta çonnes sin tristezas } e por ende fazen tuertos alos otros non & ex parte tristitiarum , \textbf{ quas homines pro viribus fugiunt . | Volunt enim homines gaudere delectationibus absque tristitiis , } ideo iniuriantur aliis in se \\\hline
3.1.19 & Et para estas tres cosas \textbf{ siruen los tres linages sobredichos de los uarones } ca quanto ala uianda cunplen los labradores & quae possunt in ciuitate contingere . \textbf{ Ad haec enim tria deseruiunt praedicta tria genera virorum . } Quia quantum ad victum \\\hline
3.1.19 & en parte sagrada \textbf{ e en parte comun e en parte proprea . } La parte sagrada daua ala honrra de dios . & videlicet in partem sacram , \textbf{ communem , et propriam . } Partem sacram attribuebat cultui diuino , \\\hline
3.1.19 & e en parte comun e en parte proprea . \textbf{ La parte sagrada daua ala honrra de dios . } ¶ Et la comuna los lidiadores . & communem , et propriam . \textbf{ Partem sacram attribuebat cultui diuino , } communem bellatoribus , propriam agricolis . \\\hline
3.1.19 & assi que si los pleitos fueren mal iudgados \textbf{ en la audiençia ordinaria aqueͣllos } uieios e sabios & ut si causae male iudicatae fuerunt \textbf{ in praetorio ordinario , } senes illi electi \\\hline
3.1.19 & e por bien dela çibdat \textbf{ que estos tales ouiessen uianda e mantenençia de los bienes comunes lo terçero } quanto a todo el pueblo establesçio & et pro bono ciuitatis , \textbf{ acciperent cibum de aerario publico . } Tertio quantum ad totum populum statuit , \\\hline
3.1.19 & por si mismas guardar su \textbf{ derechca parte nesçe al Rey e al } prinçipeque deue ser guardador dela iustiçia de auer cuydado espeçial delas cosas comunes & uniuersaliter omnes personas impotentes , \textbf{ non valentes per se ipsas sua iura conquirere . Spectat enim ad Regem et Principem , } qui debet esse custos iusti , \\\hline
3.1.19 & derechca parte nesçe al Rey e al \textbf{ prinçipeque deue ser guardador dela iustiçia de auer cuydado espeçial delas cosas comunes } e delos pelegninos e delas perssonas & non valentes per se ipsas sua iura conquirere . Spectat enim ad Regem et Principem , \textbf{ qui debet esse custos iusti , | de rebus communibus , } et etiam de peregrinis , \\\hline
3.1.20 & Et segunt la suia del philosofo \textbf{ en el segundo libro dela mecha phisica deuemos dar grans aquellos } que se desuian de la uerdat & et secundum sententiam Philosophi 2 Metaphysicae \textbf{ debemus gratias reddere eis } qui a veritate deuiant , \\\hline
3.1.20 & e por ende contamos la opinion de ipodomio \textbf{ por que el en la su poliçia manifesto muchͣs bueanssmans . } Empo algunas cosas establesçio non conuenible mente . & Hippodami ergo opinionem recitauimus , \textbf{ quia in sua politia multas bonas sententias promulgauit : } aliqua tamen incongrue statuit . \\\hline
3.2.1 & e bien puestas por sabiduria ¶ \textbf{ Lo segundo que por el poderio ciuil sean bien guardadas } Lo terçero que por las leyes falladas & Primo ut leges per sapientiam sint bene inuentae . \textbf{ Secundo ut per ciuilem potentiam sint bene custoditae . } Tertio ut per leges inuentas \\\hline
3.2.1 & Mas guardar bien las leyes \textbf{ por el poderio çiuil esto parte nesçe alos prinçipes } onde el philosofo enłqnto libbo delas ethicas dize & Bene autem custodire leges \textbf{ per ciuilem potentiam | spectat ad principem . } Unde et Philosophus 5 Ethicor’ ait , \\\hline
3.2.1 & qual deue ser el prinçipe \textbf{ aquien parte nesçe de pouer las leyes } e delas guardar & qualis debeat esse princeps \textbf{ cuius est leges ferre et custodire , } sed etiam quales debeant esse consiliarii \\\hline
3.2.2 & de los quales tro son buenos \textbf{ e los trsson malos . } ca el regno e la aristo carçia & quorum tres sunt boni , \textbf{ et tres sunt mali . } Nam regnum aristocratia , \\\hline
3.2.2 & La thirama que quiere dezer sennorio malo \textbf{ e la obligaçia que quiere dezer sennorio duro . } Et la democraçia & tyrannides , oligarchia , et democratia sunt mali . \textbf{ Docet enim idem ibidem } discernere \\\hline
3.2.2 & segunt su estado \textbf{ assi es sennorio ygual e derech̃ . } mas si es entendido & secundum suum statum , \textbf{ sic est aequale et rectum . } Sed si intenditur ibi bonum aliquorum \\\hline
3.2.2 & enssennoreante non entiende el bien comun \textbf{ Mas entiende por poderio çiuil apremiar los otros } e todas las cosas ordena al su bien propio & non intendit commune bonum , \textbf{ sed per ciuilem potentiam opprimens alios , } omnia ordinat in bonum proprium et priuatum , \\\hline
3.2.2 & que quando fallesçie el ssenador en aquel tp̃o en̊ medianero \textbf{ ante que otro senador fuesse escogido todo el pueblo romano era gouernado } por algers pocos uarones & tempore illo intermedio , \textbf{ antequam alius senator eligeretur , | regebatur totus Romanus populus } quibusdam paucis viris : \\\hline
3.2.2 & o es entendido el bien de los pobres \textbf{ e delas perssonas medianeras e de los ricos } e de todos comunalmente & vel intenditur bonum commune egenorum , \textbf{ mediarum personarum , et diuitum , } et omnium secundum suum statum : \\\hline
3.2.2 & por que poliçia es \textbf{ assi commo ordenamiento bueno de çibdat } quanto a todos los prinçipados & Politia enim quasi idem est , \textbf{ quod ordinatio ciuitatis quantum } ad omnes principatus \\\hline
3.2.2 & Enpero el prinçipado del pueblo si derecho es \textbf{ por que non ha nonbre comun es dich poliçia } e nos podemos llamar atal prinçipado gouernamiento del pueblo & Principatus tamen populi si rectus sit , \textbf{ eo quod non habeat commune nomen , | Politia dicitur . } Nos autem talem principatum appellare possumus gubernationem populi , \\\hline
3.2.3 & e seria mas uirtuoso para traer la \textbf{ naueque muchs por la qual razon si todo el pode rio çiuil } que es en muchs prinçipantes & virtuosior esset in trahendo . \textbf{ Quare si tota ciuilis potentia , } quae est in pluribus principantibus , \\\hline
3.2.3 & Et aquel prinçipe por ma . \textbf{ yor cunplimiento de poderio meior podria gouernar la çibdat } que muchos ¶ & et ille principans \textbf{ propter abundantiorem potentiam | melius posset politiam gubernare . } Tertia via sumitur \\\hline
3.2.3 & Et avn las abeias \textbf{ por que les es cosa natraal de } beuiren conpannian & Si apes etiam \textbf{ quia naturale est eis } in societate viuere , \\\hline
3.2.4 & que lees acomnedado . \textbf{ Lo primero deue auer razon abiuada e sotil ¶Lo segundo entencion derecha . } Lo terçero firmeza acabada . & ut bene regat populum sibi commissum . \textbf{ Primo enim debet habere perspicacem rationem . | Secundo rectam intentionem . } Tertio perfectam stabilitatem . \\\hline
3.2.4 & assi commo vn omne de muchos oios e de muchͣs manos . \textbf{ Por la qual cola meior sera este tal prinçipado } ca el omne assi conpuesto de muchs oios e la muchedunbre & unum hominem multorum oculorum et multarum manuum . \textbf{ Quare melior erit huius principatus , } quia homo sic constitutus \\\hline
3.2.4 & por que conuiene \textbf{ que el prinçipe sea regla derecha e firme et estable } assi que por ira & decet enim Principem \textbf{ esse regulam rectam et stabilem , ut per iram et concupiscentias } et per alias passiones non corrumpatur nec peruertatur . \\\hline
3.2.4 & Mas entre los malos prinçipados el prinçipado de vno \textbf{ que por nonbre comun es dich tirannia es muy mal prinçipado } mas desto diremos ayuso mas conplidamente & inter peruersos vero principatus , principatus , unius , \textbf{ qui communi nomine tyrannis nuncupatur , | est pessimus . } Sed de hoc infra dicetur : \\\hline
3.2.4 & ca mostraremos que assi commo la monarchia \textbf{ e el prinçipado real es muy bueno } assi por que al mayor bien es contrario el mayor mal & Sed de hoc infra dicetur : \textbf{ ostendetur enim quod sicut monarchia regia est optima ; } ita quia maiori bono maius malum opponitur , \\\hline
3.2.4 & assi por que al mayor bien es contrario el mayor mal \textbf{ por ende el prinçipado thiranico es muy malo } por que es contrario al prinçipado del regno & ita quia maiori bono maius malum opponitur , \textbf{ monarchia tyrannica est pessima . } Dominari autem plures dominio recto , \\\hline
3.2.4 & e segut derech \textbf{ sennorio meior es } que sea vn sennor que muchos . & regnum esse dignissimum principatum , \textbf{ et secundum rectum dominium melius est dominari unum , } quam plures . \\\hline
3.2.4 & Por ende si esto fuere \textbf{ assi en toda manera es cosa digna de aquel vno ser } prinçipe sigue conseio de buenos e de sabios & ideo si sic se habeat , \textbf{ omnino dignum est ipsum principari . } Si autem aliter se haberet , \\\hline
3.2.5 & al que pertenesçe de tegnar \textbf{ e de auer la dignidat real . } Et por ende paresça e a alguons que fablando sueltamente meiores & ad quem spectabit \textbf{ habere regiam dignitatem . } Absolute ergo loquendo , \\\hline
3.2.5 & quanto plaz ha el omne \textbf{ e quanta auna daia quando cuyda que tiene alguna cosa propria . } por que aquello que es natraal non puede ser oçioso & quantam dilectionem habeat , \textbf{ et quantum differat | patare aliquid proprium : } nam quod est naturale , \\\hline
3.2.5 & nin se ouieren bien \textbf{ çerca las cołas diuinales pocas uezes } contesçe que los fiios regnen en pos de los padres . & bene se habeant erga diuina , \textbf{ raro contingit filios regnare post patres : } et si regnant filii , \\\hline
3.2.5 & e regua con crueldat . \textbf{ Mas si el gouernamiento real ui merepot heredat los fijos de los Reyes } non se enssoberuesçen & et inflati corde et inerudite regnant . \textbf{ Sed si regale regimen | per haereditatem vadat , } filii ex hoc non inflantur \\\hline
3.2.5 & por que non sea mal gouernado \textbf{ e por que el gouerniamiento real non sea tornado en tirama . } que el poderio real o la dignidat re & in haereditatem paternam , \textbf{ expedit regno ne inerudite regatur , } et ne regale regimen conuertatur in tyrannidem , \\\hline
3.2.5 & e por que el gouerniamiento real non sea tornado en tirama . \textbf{ que el poderio real o la dignidat re } aluaya alos fijos & expedit regno ne inerudite regatur , \textbf{ et ne regale regimen conuertatur in tyrannidem , } ut regia dignitas per haereditatem transferatur ad posteros . \\\hline
3.2.5 & e desto puede paresçer \textbf{ que non solamente parte nesçe al regno } de determinar el & Ex hoc autem patere potest , \textbf{ quod non solum expedit regno determinare prosapiam , } ex qua praeficiendus est dominus , \\\hline
3.2.5 & do el sennorio viene por hedamiento \textbf{ ca si la dignidat Real passa alos fijos } por hedamiento conuiene alos pueblos & difficultatem non habet : \textbf{ nam si dignitas regia } per haereditatem transferatur ad posteros , \\\hline
3.2.5 & e que conuiene \textbf{ que esta dignidat real masspasse alos } mas los que alas fenbras & quia secundum lineam consanguinitatis filii parentibus maxime sunt coniuncti : \textbf{ oportet autem talem dignitatem } magis transferre \\\hline
3.2.5 & por hedat \textbf{ que la dignidat real se espone a ocasion e auentura } por que non saben & videlicet quod ire per haereditatem , dignitatem regiam , \textbf{ est exponere fortunae , } eo quod ignoretur \\\hline
3.2.5 & a qual de los fijos pertenesçra el regno . \textbf{ Por ende por que el bien comun non sea puesto a peligro de todos los fios deuen auer los padres grant cuydado } pho en el quanto libro delas poluenta tres cosas & et nescitur cui filiorum succedat regnum . \textbf{ Ideo ne periclitetur bonum commune , | de omnibus filiis Regis cura diligens est habenda . } Philosophus 5 Politic’ \\\hline
3.2.6 & ¶Lo segundo puede alguno sobrepuiar a otro \textbf{ por a unataia de obras uirtuosas . } ca por que de los buenos & Secundo potest aliquis praefici in Regem \textbf{ ab excessu virtuosarum actionum : } nam quia bonorum virtuosorum est diligere bonum commune potius quam priuatum , \\\hline
3.2.6 & que tases aun ataias \textbf{ e tales condiçions buenas sean falladas en el rey mas conplidamente despues que fuere puesto en el prinçipado } que ante ca conuiene & decens est tales excessus \textbf{ in ipsa monarchia perfectius reperiri . } Decet enim ipsum regem volentem recte regere \\\hline
3.2.6 & que el rey aya aquellas tres aun ataias \textbf{ e aquellas tres condiçiones buenas sobredichͣs . } ca si abondare en bien fazer seria muy amado del pueblo & Quare expedit regem \textbf{ habere praedictos tres excessus . } Nam si abundet in beneficiis tribuendis , \\\hline
3.2.6 & que otro bien singular \textbf{ por ende el omne bueno e uirtuoso mas procurara el bien comun } que el su bien pro ̉o priuado . & quam aliquod bonum singulare , \textbf{ magis procurabit vir bonus et virtuosus , | quam bonum aliquod proprium et priuatum . } Tertio expedit eum abundare \\\hline
3.2.6 & por que el regno es prinçipado derech . \textbf{ mas la tirania es sennorio tuerto e malo . Et pues que assi es commo el bien comun delas gentes sea mas diuinal que el bien de vno . } malamente e desigualmente & Nam regnum est principatus rectus , \textbf{ tyrannis vero est dominium peruersum . | Cum ergo bonum gentis sit } diuinius bono unius , \\\hline
3.2.6 & Et por ende la su entençion toda se pone en el auer \textbf{ o en los dinos creyendo que por ellos puede auer las otras cosas delectables . } Mas la entençion del Rey esta & nisi de delectationibus propriis , \textbf{ maxime versatur sua intentio circa pecuniam , credens se per eam posse huiusmodi delectabilia obtinere . } Sed regis intentio versatur circa virtutem , \\\hline
3.2.6 & por que menospreçia el bien comun e non ha cuydado \textbf{ si non delas sus delecta connes proprias . } veyendo se en carga e en aborresçimiento & Nam tyrannus eo quod spreto communi bono non curat \textbf{ nisi de delectationibus propriis , } videns se esse onerosum et tediosum \\\hline
3.2.6 & Por ende toda la guarda del su cuerpo \textbf{ acomienda a omnes estrannos . } Mas e Rey faze todo el contrario & qui sunt in regno , \textbf{ totam suam custodiam corporis committit extraneis : } sed Rex econuerso eo \\\hline
3.2.7 & por que quanto el sennorio de alguon ses mas contra uoluntad de los omes \textbf{ tanto mas deue ser dich des natural . } Et por ende la tirania es muy mala & magis est inuoluntarium , \textbf{ magis debet dici in naturale : } tyrannis igitur est pessima , \\\hline
3.2.7 & do dize que la tirania es muy mal prinçipado \textbf{ por que ninguno de los omes francos e libres non sufre } por uoluntad tal prinçipado & tyrannidem esse pessimum principatum , \textbf{ quia nullus liberorum voluntarie sustinet principatum talem . } Tertia via sumitur \\\hline
3.2.7 & que los çibdadanos sean sabios e entendidos . \textbf{ Et la razon por que los tiranos enbargan estos bienes sobredichos en las çibdades ayuso se dira . } Et cunpla agora de saber & nec etiam volunt ipsos esse sapientes et disciplinatos . \textbf{ Quare autem tyranni praedicta bona | impediunt in ciuibus ; } infra dicetur . \\\hline
3.2.7 & que la tirania es muy mal prinçipado \textbf{ por las razones sobredichͣs . } Mas en commo los Reyes en toda manera de una esquiuar de non enssennorear con sennorio de tirania & tyrannidem esse pessimum principatum \textbf{ propter rationes tactas . } Quod autem reges summo opere cauere debeant , \\\hline
3.2.8 & e por que sean y muchos sabios \textbf{ e muchs maestros entendidos } cado es la sabiduria & ut in suo regno uigeat studium litterarum , \textbf{ et ut ibi sint multi sapientes et industres . } Nam ubi viget sapientia \\\hline
3.2.8 & e pueda alcançar aquella fin . \textbf{ Et por ende parte nesçe al gouernador del regno de otdenar sus } subditosa uirtudes e a buenas costunbres ¶ & ut velit consequi finem illum : \textbf{ spectat igitur ad rectorem regni ordinare } suos subditos ad virtutes . \\\hline
3.2.8 & e para alcançar la fin \textbf{ que entienden en la uida çiuil . } Mas esto conmose puede fazer & prout deseruiunt ad bene viuere , \textbf{ et ad consequendum finem intentum in vita politica . } Hoc autem quomodo fieri possit , \\\hline
3.2.8 & Commo quier que en alguna manera sea declarado \textbf{ por las cosas sobredichos . } Enpero mas claramente se dira a de sante . & licet per praecedentia \textbf{ sit aliqualiter manifestum , } clarius tamen infra dicetur . \\\hline
3.2.8 & Et pues que assi es tirar vna cosa \textbf{ que enbarga much la buena uida çiuil es ordenar bien } en qual manera las hedades & remouere igitur unum maxime prohibentium bonam vitam politicam , \textbf{ est bene ordinare } quomodo haereditates decedentium perueniant ad posteros . \\\hline
3.2.8 & de los que mueren vengan alos hederos \textbf{ que fincan ¶lo segundo el estado bueno dela çibdat } e del regno se enbarga algunas vezes & quomodo haereditates decedentium perueniant ad posteros . \textbf{ Secundo , status tranquillus ciuitatis et regni aliquando impeditur } ex peruersitate ciuium : \\\hline
3.2.8 & Por ende pertenesçe al ofiçio del Rey de ser \textbf{ assy acuçioso çerca el poderio çiuil e çerca la sabiduria delas almas } por que pueda defender lo suyo & Spectat igitur ad regis officium sic solicitari \textbf{ circa ciuilem potentiam ; } et circa industriam armatorum , ut possint hostium rabiem prohibere . \\\hline
3.2.8 & Et pues que assi es auer acuçia çerca las cosas \textbf{ que son dichas parte nesçe al ofiçio del Rey } asnos podemos contar dies cosas & Solicitari igitur circa praedicta nouem , \textbf{ ad Regis officium pertinere videtur . } Narrare autem possumus decem \\\hline
3.2.9 & Et conmo quier que aquellas diez cosas \textbf{ en alguna manera general mente sean contenidas en las cosas sobredichos . } Enpero por que muchͣs uegadas en la sçiençia moral . & et quae Tyrannus se facere simulat . \textbf{ Illa enim decem licet aliquo modo in uniuersali contineantur in dictis , } tamen quia ( ut pluries dictum est ) \\\hline
3.2.9 & en alguna manera general mente sean contenidas en las cosas sobredichos . \textbf{ Enpero por que muchͣs uegadas en la sçiençia moral . } los sermones generales poco proprouechan . & Illa enim decem licet aliquo modo in uniuersali contineantur in dictis , \textbf{ tamen quia ( ut pluries dictum est ) } circa morale negocium uniuersales sermones proficiunt minus , \\\hline
3.2.9 & Enpero por que muchͣs uegadas en la sçiençia moral . \textbf{ los sermones generales poco proprouechan . } por ende sera bien de contar estas diez cosas cada vna & tamen quia ( ut pluries dictum est ) \textbf{ circa morale negocium uniuersales sermones proficiunt minus , } ideo bene se habet \\\hline
3.2.9 & e alas otras perssonas sin prouech . \textbf{ Lo segundo parte nesçe a derecho gouernador de regño } non solamente de ordenar las rentas del regno & et aliis personis inutilibus . \textbf{ Secundo spectat ad rectum rectorem regni } non solum redditus \\\hline
3.2.9 & mas quiere pare sçertal . \textbf{ lo quarto parte nesçe a } uerdadero Rey non despreciar a ninguon de los subditos & sed esse se simulat . \textbf{ Quarto spectat ad Regem , } nullum subditorum contemnere , \\\hline
3.2.9 & por que sean familiares e bien querençiosas alas mugers \textbf{ de los sobredichos omes buenos del regno } por que las muger smuch inclinan a sus maridos a sus uoluntades propreas & sed etiam ut ait Philosophus in Polit’ inducere debent uxores proprias \textbf{ ut sint familiares et beniuolae uxoribus praedictorum : } nam mulieres valde inclinant viros \\\hline
3.2.9 & Et pues que assi es . \textbf{ assi se deue auer buen Rey e buen gouernador deregas e de çibdat } Mas el tirano non se ha & sic ergo gerere se debet \textbf{ bonus rector regni aut ciuitatis . } Tyrannus autem non sic se habet , \\\hline
3.2.9 & assi commo auer dineros o auer delecta connes . \textbf{ Et por que es muy grant delectaçion sensible enlas uiandas e enlas luyias . } los tiranos sin fre no vsan delas delecta connes & ut bonum pecuniosum et delectabile , \textbf{ quia maxima delectatio sensibilis est | in cibis et venereis , } tyranni absque fraeno fruuntur voluptatibus illis . \\\hline
3.2.9 & mas matan los e destierran los \textbf{ ¶ Loye conuiene al Rey uerdadero de non enssanchar su regno } por tomar lo ageno & non honorant , sed perimunt . \textbf{ Nono decet verum Regem per usurpationem et iniustitiam } non dilatare suum dominium . \\\hline
3.2.9 & la qual cosa es muy uerdadera mayor mente \textbf{ si tal muchedunbre de poderio çiuil fuere ganada } tomando sennorio ageno & Quod maxime verum est \textbf{ si huiusmodi multitudo ciuilis potentiae acquisita sit } per usurpationem et iniustitiam . \\\hline
3.2.9 & que tal Rey faze todas las cosas con iustiçia \textbf{ e non faze ninguna cosa contuerto . } Enpero nos podemos traher otra & existimat enim talem semper iuste agere , \textbf{ et nihil iniquum exercere . } Possumus tamen ad hoc aliam meliorem rationem adducere dicentes \\\hline
3.2.9 & todas las cosas son manifiestas \textbf{ e el su poderio aqui non puede ser ninguna cosa contraria legnia } assi conmo cunple a su salut . & cui omnia sunt nota , \textbf{ et eius potentia cui nihil potest resistere , | continget eum } ut expedit suae saluti \\\hline
3.2.10 & Ca el tirano segunt \textbf{ qsu mala condiçion non solamente destruye los sabios } mas avn defiende & ( secundum quod huiusmodi est ) \textbf{ non solum sapientes destruit , } sed etiam truncat viam , \\\hline
3.2.10 & o por algun guerra derechurera . \textbf{ La nouena caute la del tirano es poner grant guarda en el su cuerpo } non por aquellos que son del regno & vel pro aliquo alio iusto bello . \textbf{ Nona , est custodiam corporis exercere } non per eos \\\hline
3.2.11 & e creyere que sera ayudado dellos \textbf{ por el poderio çiuil a cometera al tirano . } La quarta puede acaesçer & et credat se iuuari ab eis , \textbf{ propter ciuilem potentiam tyrannum inuadit . } Quarto hoc poterit accidere ex nimio ocio , \\\hline
3.2.11 & por las quales podran defender su çibdat \textbf{ e leunatarse contra el mal gouernador de los çibdadanos } Et pues que assi es contra estas quatto cosas procuran los tyranos de matar los nobles e los grandes & excogitant seditiones , \textbf{ quomodo possint turbare ciuitatem , et insurgere contra rectorem ciuium . } Contra haec ergo quatuor procurant tyranni perimere excellentes , \\\hline
3.2.11 & e declarar las cautelas de cada vno dellos . \textbf{ por que las cosas contrarias puestas çerca de ssi mismas } maclaramente paresçen & et utrasque cautelas describere : \textbf{ quia opposita iuxta se posita magis elucescunt : } magis enim apparet \\\hline
3.2.11 & por que es . muy mala . \textbf{ por que la cosa muy mala de ssi misma es much de foyr . } Et la muy buena dessi misma es much de segnir . & quia pessimum de se , \textbf{ est maxime fugiendum , } et optimum maxime prosequendum . \\\hline
3.2.12 & mas assi commo paresçe \textbf{ por las cosas sobredichas tres prinçipados son buenos e tres malos } ca dicho fue & Sunt enim ( ut patet ex habitis ) \textbf{ tres principatus boni , et tres peruersi . } Dicebatur autem \\\hline
3.2.12 & mas por que son ricos es llamado obligarçia \textbf{ que quiere dezer señorio tuerto . } Mas quando enssennore a todo el pueblo & sed quia diuites , \textbf{ est peruersus | et vocatur oligarchia . } Sed si dominatur totus populus \\\hline
3.2.12 & que tanto quiere dezer commo corrupçion e maldat del pueblo . Et pues que assi es la tirania \textbf{ e el sennorio corrupto de los ricos . } Et el sennorio malo del pueblo son tres señorios muy malos . & quod quasi peruersio et corruptio populi . \textbf{ Tyrannis vero corruptus principatus diuitum , } et iniquum dominium populi , \\\hline
3.2.12 & Conuiene a saber riquezas de dineros . \textbf{ Vv electa connes corporales . } Et guarda de su cuerro . & videlicet pecuniam , \textbf{ corporales delicias , } et custodiam corporis . \\\hline
3.2.12 & sienpte a delecta connes corporals \textbf{ e auer plazenterias senssibles . } ca si alguon quiere alguna cosa much & Rursus diuites inique principantes \textbf{ intendunt delicias corporales , et habere voluptates sensibiles . } Nam si quis vult aliquid , \\\hline
3.2.12 & e mayormente los que han la uoluntad desordenada orden en sus riquezas \textbf{ e sus uiçiosa delecta con nes corporales e a plazentenias dela carne siguese } que los ricos mas enssennorean & ordinent pecuniam et diuitias \textbf{ ad delectamenta corporea , | et ad voluptates sensibiles , } sequitur diuites inique dominantes , \\\hline
3.2.12 & que los otros uerdados Reyes . \textbf{ Et avn delas delecta connes conuenibles del cuerpo } non han tantas los tiranos & quam alios veros reges . \textbf{ Delectamenta etiam corporea } non tot habent tyranni , \\\hline
3.2.12 & ca asi commo paresçe \textbf{ por las cosas sobredh̃as el tiranᷤ } con todo su poder se esfuerça en abaxar los grandes & Hoc etiam tyrannus facit , \textbf{ quia ut patet ex habitis ipse pro viribus nititur } opprimere excellentes , \\\hline
3.2.13 & e con grand acnçia \textbf{ que non tira njz en dexado el gouna mj̊ derecho } Ca cuenta el philosofo en el & reges cura peruigili studere debent , \textbf{ ne delinquentes rectum gubernaculum , tyrannizent . } Narrat autem Philosophus \\\hline
3.2.14 & e si es mal entero non se puede sofrir . \textbf{ Pues que asi es la tiranja yor las maldades } e por las falsedades & et si integrum sit importabile sic . \textbf{ Tyrannis ergo propter peruersitates et nequitias } quae congregantur in ipsa , \\\hline
3.2.14 & e lançar ala mano derechͣ \textbf{ e ala mano desquierda son cosas contrarias . } Et en essa misma manera en este proponimiento & ut si multum opponitur pauco , \textbf{ proiectio ultra signum contrariatur proiectioni citra : et proiectio in dextrum proiectioni in sinistrum . } Sic etiam in proposito , \\\hline
3.2.14 & escontrana ala tirama del mal prinçipado \textbf{ Et vn prinçipado tiranico es contrario a otro prinçipado tiranico e malo } ca quando algun & ut tyrannis populi contrariatur tyrannidi monarchiae : \textbf{ et una monarchia tyrannica contrariatur alii . | Cum enim aliquis monarcha } vel aliquis unus Princeps tyrannizet in populum , \\\hline
3.2.14 & assi avn vna tirama de sennorio corronpe a otra \textbf{ por que muchͣs vezes vn prinçipe tirano se leunata contra otro prinçipe tirano } por que gane el su prinçipado . & Sic etiam una tyrannis monarchia corrumpit aliam : \textbf{ quia multotiens unus monarcha tyrannus insurgit in alium , } ut obtineat principatum eius . \\\hline
3.2.14 & que non tiraniz en commo en tantas maneras \textbf{ segunt dicho es se aya de destroyr el prinçipado tiranico . } Lo terçero se desfaze la tirani a non solamente & Debent ergo cauere Reges et Principes ne tyranizent , \textbf{ cum tot modis dissoluatur tyrannicus principatus . } Tertio dissoluitur tyrannis \\\hline
3.2.14 & Lo terçero se desfaze la tirani a non solamente \textbf{ por si misma o por otra tirama contraria . } Mas avn por gouernamiento bueno & non solum propter seipsam \textbf{ vel propter tyrannidem aliam , } sed etiam propter regnum . \\\hline
3.2.14 & non es bue no de tiranizar mas el regno \textbf{ e el sennorio bueno non es puesto atantos peligros } nin en tantas maneras se puede desatar commo la tirana . & dissolui eius principatus . Regium autem dominium \textbf{ non tot periculis exponitur , } nec tot modis habet dissolui . \\\hline
3.2.15 & bien vsar de los çibdadanos non solamente guarda la poliçia \textbf{ e el gouernamiento derecho . } Mas avn por esta razon el prinçipado & bene uti ciuibus \textbf{ non solum praeseruat politiam rectam , } sed etiam principatus ex hoc durabilior redditur , \\\hline
3.2.15 & comneçaron a auer guerra entre ssi . mismos . \textbf{ Mas esta cautela es propro prouechable en dos prinçipados } assi commo en el prinçipado & intra seipsos bellare coeperunt . \textbf{ Est autem haec cautela utilis solum duobus principatibus : } ut principatui constituto \\\hline
3.2.15 & Otrossi si alguno comneçare de nueuo a regnar \textbf{ por que contra tal prinçipe nueuo de ligero } seleuna tan los cibdadanos & Rursus , si quis super aliquos de nouo principari coepit , \textbf{ quia contra talem principatum facilius insurgitur ; } ne ciues insurgant in Principem , \\\hline
3.2.15 & e del regno conuiene \textbf{ que sea omne uirtuoso e uisto } e mas que iusto & et custos regni totius oportet \textbf{ quod sit virtuosus } et epiikis idest super iustus : \\\hline
3.2.16 & de que el omne non sabio \textbf{ e el omne loco toma consseio . } Mas de aquellas cosas de que el sabio & utique aliquis \textbf{ non pro quibus consiliatur insipiens , et insanus , } sed pro quibus sapiens , \\\hline
3.2.16 & mas si en tales cosas cae algun conse io esto non es \textbf{ por si mas en quanto siruen algunas obras nr̃as . } Assi commo si alguas obras nras mas coueniblemente se feziessen en el tpo caliente & hoc non est secundum se , \textbf{ sed prout deseruiunt actionibus nostris : } ut quia aliqua humana opera \\\hline
3.2.16 & assi comda e tallar thesoros . \textbf{ Lo v̊a vnño son todas las obras de los omes conseiables } nin taen so consseio . & quae sunt a fortuna , \textbf{ puta de thesauri inuentione . } Quinto non sunt consiliabilia \\\hline
3.2.16 & mas esto toma \textbf{ assi commo cosa çierta e conosçida } que el enfermo es de sanar & sed hoc accipit \textbf{ tanquam certum et notum , } egrum sanandum esse : \\\hline
3.2.16 & e toma assi \textbf{ conmocosa çierta e conosçida e ha consseio en qual manera estas cosas se pueden meior fazer . } Et pues que assi es los consseios son de aquellas cosas & sed haec accipit tanquam certa et nota , \textbf{ et consiliatur quomodo melius fieri possint . } Sunt ergo consiliabilia \\\hline
3.2.17 & e en las sçiençias delas naturas delas cosas \textbf{ e enlas sçiençias delas cosas perdurables . } Mas tales quastions & et circa naturas rerum , \textbf{ et circa aeterna fieri quaestiones multae , } sed huiusmodi quaestiones consilia \\\hline
3.2.17 & Por ende dize el philosofo en el tercero libro delas ethicas \textbf{ que çerca delas obras ciertas delas sçiençias } non ha consseio ninguon & Ideo dicitur tertio Ethicorum \textbf{ quod circa certas operationes disciplinarum } non est consilium puta de litteris , \\\hline
3.2.17 & que non tome mas conseio \textbf{ de quales se quier cosas pequanas mas de grandes . } Ca dicho fue desuso & Secundo est in consiliis attendendum , \textbf{ ut non consiliemur de quibuscunque minimis , sed de magnis . } Dicebatur enim supra , \\\hline
3.2.17 & Et por ende las cosas que son muy pequan ans assi que pueden acarrear muy \textbf{ pequano mal o enbargar } pequano bien non son de poner en consseio . & ut quae sunt apta nata \textbf{ efficere paruum bonum , | vel prohibere modicum malum , } non sunt consiliabilia . \\\hline
3.2.17 & en que deuen mucho petissar . \textbf{ mas por auentura meior po demos } dezir que cosseio sea dicha conssilendo & quia ibi plures simul consedere debent . \textbf{ Sed forte melius dicere possumus , | quod consilium dictum sit } a Con et Sileo \\\hline
3.2.17 & e mayormente enlos negoçios \textbf{ do se tractan los fechos e las negoçios comunes del regno . } por que cada vno de los consseieros tirando de ssi el & et maxime in consiliis \textbf{ ubi tractantur negocia communia et facta regni : } ut unusquisque consiliarius \\\hline
3.2.17 & quando ellos entra una tirando dessi el amor propreo \textbf{ assi se reuistien de amor comun e pubłico } que non dire & cuius limen intrantes abiecta priuata dilectione \textbf{ ita dilectionem publicam inducebant , } ut non dicam unum , \\\hline
3.2.17 & ¶Lo quintones de guardar en los consseios \textbf{ que non fablen y cosas plazenteras mas uerdaderas } ca los lisongeros estudiando de fazer plaza los prinçipes callan la uerdat & quod tam multorum auribus fuerat commissum . Quinto est in consiliis attendendum , \textbf{ ut non loquantur ibi placentia , sed vera . } Adulatores enim \\\hline
3.2.17 & ¶ La otra que non sean plazenteros \textbf{ assi que parezcan lisongeros auiendo mayor cuydado de fablar cosas plazenteras que uerdaderas . } En essa misma manera abn segunt dize el pho & ut quod essent adulatores , \textbf{ plus curantes loqui placentia , | quam vera . } Sic etiam ut recitat Philosophus \\\hline
3.2.17 & que luego lo pongamosen obra . \textbf{ ca quando viene el tp̃on coueinble para obrar } si derechͣmente queremos obrar & cito in opere exequamur . \textbf{ Nam cum adest opportunitas operandi , } et si recte volumus \\\hline
3.2.17 & pues que assi es muy bien es de \textbf{ escodrinnar con grant acuçia todo negoçio alto e noble si es prouechoso delo fazer . } mas despues que fuere conosçido derechamente & Bene ergo se habet diligenter \textbf{ quodlibet negocium discutere arduum , | an utile sit illud facere : } sed post quam per diuturnum consilium est recte cognitum \\\hline
3.2.17 & que es lo que deuemos fazer \textbf{ si ouieremos tp̃o conueinble e poder para obrar } mano a mano lo deuemos obrar & quid fiendum , \textbf{ si adsit operandi facultas , } prompte operari debemus . \\\hline
3.2.18 & e los bueons non quieren mentir \textbf{ de ligero creen los omes asodichos . } lo segundo se puede fazer creençia alos oydores & de facili creditur eorum dictis . \textbf{ Secundo potest fieri credulitas auditoribus } ex parte ipsorum auditorum : \\\hline
3.2.18 & assi por que el amigo es assi contado \textbf{ commo a ssi mismo . comunalmente se engannan los omes en sus amigos . } Ca assi commo dize el philosofo en el primero libro dela rectonça los que amamos & sic quia amicus reputatur alter ipse , \textbf{ communiter decipiuntur homines circa amicos . } Nam ut dicitur primo Rhetoricorum \\\hline
3.2.18 & e los que aborresçemos non los iudgamos egual mente . \textbf{ ca cosa comunales que los fechs de los amigos } por la mayor parte que los recontemos en vien . & non pariter iudicamus . \textbf{ Commune est enim ut amicorum facta } ut plurimum deferamus in bonum , \\\hline
3.2.18 & mas esta creençia et este amonestamiento es por si . \textbf{ Ca fazerse el omne digno de creer } e buen amonestador e razonador por si . & et haec persuasio est per se : \textbf{ nam reddere se credibilem } et bene persuadere per se , \\\hline
3.2.19 & que son menester \textbf{ para mantenençia dela uida corporal . } Ca deue penssar el prinçipe & cui principatur aliquis rector sufficiat sibi in alimento et in deseruientibus \textbf{ ad sufficientiam vitae : } considerandum est enim quantum alimentum est in regno \\\hline
3.2.19 & e puestas ordena \textbf{ çonns quals deuen o quales se pueden fazer } ca non es pequana cosa de auer consseio er la mantenençia & ut circa haec debita consilia \textbf{ et debitae ordinationes fieri possint : } non enim modicum consiliandum est circa alimentum , \\\hline
3.2.19 & ca los Reyes e los prinçipes non deuen sofrir \textbf{ que los malfechores bi una . } Avn son de penssar los logares & vel etiam totaliter extirpentur , \textbf{ quia Reges et Principes non debent pati maleficos viuere . } Sunt etiam consideranda loca \\\hline
3.2.20 & e son por dezir m \textbf{ or que en los capitulos sobredichos dixiemos del prinçipe demo } e det̃minamos del conseio & ut simul ex dictis et dicendis melius veritas patere possit . \textbf{ Quia in praecedentibus capitulis determinauimus de principe , | ostendendo qualis debeat esse princeps , } et determinauimus de consilio , \\\hline
3.2.20 & e con grant diligençia fueren escodrinnados \textbf{ que si luego man ama no fuesse dadas m̃a difinitiua . } por ende commo los fazedores delas leyes en muchtp̃o & quam si oporteat \textbf{ statim iudicatiuam sententiam proferre . } Itaque cum legum conditores multo tempore \\\hline
3.2.20 & nin son delas cosas generales \textbf{ mas son de las cosas espeçiales . } Et por que las perssonas espeçiales & non uniuersaliter , \textbf{ sed in particulari . } Incusantur enim determinatae personae , \\\hline
3.2.20 & Et muchas uezes tales perssonas tienen mientes a su bien propra o . \textbf{ por ende el iuez de ligero se t uerçeca los que aman } e desaman & et saepe talia annexum habent proprium commodum . \textbf{ Ideo iudex de facili obliquatur : } nam amantes , et odientes , \\\hline
3.2.20 & por que los fecho \textbf{ e las obras particulates non cien } conplidamente so recontamiento milo tuas & aliqua committere arbitrio iudicum , \textbf{ quia gesta particularia complete } sub narratione non cadunt , \\\hline
3.2.20 & pues que assi es \textbf{ quanto men or enemistaça fueren } los que han de dar los iuyzios & quia ( ut dicitur 6 Politicorum ) \textbf{ quanto utique minor inimicitia fuerit exequentibus iudicia , } tanto magis accipient \\\hline
3.2.21 & que assi passen el iuyzio \textbf{ por las palabras malas e sannudas } que pueden mouer los omes a mal & ut in iudicio procedant , \textbf{ ut sermones passionales prouocantes ad passiones , } ut ad iram et odium ; \\\hline
3.2.21 & ante los alcalłs rodemos lo prouar por tres razones ¶ \textbf{ La primera seqma par aquello que tales palabras han de to terçeres desegualar eliez } el qual conuiene de ser & Prima sumitur \textbf{ ex eo quod huiusmodi sermones | obligare habent iudicem , } quem esse oportet \\\hline
3.2.21 & el qual conuiene de ser \textbf{ assi commo regla derecha en } iudgando la segunda razon se toma & quem esse oportet \textbf{ quasi regulam in iudicando . } Secunda vero , \\\hline
3.2.21 & Et mientra que se non corronpen \textbf{ por alguna delas partes contrarias da iuyzio derecho de las cosas } que siente Rbigran & et quamdiu non inficitur \textbf{ secundum alterum contrariorum , } dat rectum iudicium de sensibus ; \\\hline
3.2.21 & que contienden non se enclinando a ninguna delas partes es \textbf{ assi commo regla derecha diziendo } e mostrando lo que es derecho & inter litigantes non declinans ad alteram partem , \textbf{ quasi regula recta decet } iustum esse iustum \\\hline
3.2.21 & que la regla se tuerca \textbf{ non es cosa conuenible de sofrir } que enco el uuzio se digan palabras malas e desiguales & permittere obliquari regulam , \textbf{ inconueniens est sustinere } in iudicio passionales sermones . \\\hline
3.2.21 & non es cosa conuenible de sofrir \textbf{ que enco el uuzio se digan palabras malas e desiguales } ca assi commo parezçca en lo que es de dezer los mezes & inconueniens est sustinere \textbf{ in iudicio passionales sermones . } Dato tamen quod contingat \\\hline
3.2.21 & enclinan la uoluntad de los omes \textbf{ e fagan paresçer alguna cosa derecha . } por que los que assi son munnidos & inclinent voluntatem , \textbf{ et faciant apparere aliquid iustum , } vel non iustum , \\\hline
3.2.21 & Mas esto non se deue fazer \textbf{ por las partes que contienden o por las palabras desiguales dichas delas partes . } La terçera razon se toma d esto & non autem hoc debet fieri \textbf{ per partes litigantes , | vel per sermones passionales } a partibus promulgatos . \\\hline
3.2.21 & por palabras contando le las miurias . \textbf{ las quales la parte contraria fizo al iuez } o contando le los bienes & aut narrare iniurias \textbf{ quas pars aduersa iudici intulit , } vel narrare bona \\\hline
3.2.21 & e a mal querençia dela parte contraria \textbf{ e a bien querençia de ssi mismo . } Esto non pertenesçe en ninguna guasa al proposito . & ad maliuolentiam partis aduersae , \textbf{ et ad beniuolentiam sui , } est omnino impertinens ad propositum : \\\hline
3.2.22 & que contienden \textbf{ fazen ser el iuyzio malo e desigual . } Ca assi commo paresçe & ad partes litigantes , \textbf{ facient iudicium iniquum : } nam ut patet ex habitis iudex debet esse quasi regula recta media inter utrasque partes : \\\hline
3.2.22 & por lo que dich̉es el iues deue ser \textbf{ assi commo regla derecha medianera entre amas las partes } por la qual cosa & facient iudicium iniquum : \textbf{ nam ut patet ex habitis iudex debet esse quasi regula recta media inter utrasque partes : } quare si huiusmodi regula a medio deuiat , \\\hline
3.2.22 & Ca algunas uezes \textbf{ por sospecha liuiana condepnaran a algunos . } pues que assi es el uuzio & facient iudicium suspitiosum , \textbf{ quia aliquando ex leui suspitione alios condemnabunt . } Non erit ergo rectum iudicium , \\\hline
3.2.22 & que el que ha el arte . \textbf{ Por que a conosçimiento delas condiçiones particulares . } assi en el iuyzio de los fechos & plus proficit expertus quam artifex , \textbf{ eo quod habeat notitiam conditionum particularium : } sic in iudicio agibilium \\\hline
3.2.22 & que los que saben las leyes . \textbf{ Et todas estas cosas sobredichͣs son menester } para iudgar derechamente e conuenible mente . & experti quam scientes iura . \textbf{ Omnia ergo praedicta requiruntur } ad recte et debite iudicandum . \\\hline
3.2.22 & por amor o por mal querençia delas partes . \textbf{ mas que por amor de iustiçia dens m̃as uerdaderas . } Otrossi que ayan prueua de los fechs & vel ut non ex amore vel odio partium , \textbf{ sed ex dilectione iustitiae sententias proferant : } habeant experientiam agibilium , \\\hline
3.2.23 & mayoraspeza e mayor dureza de quanta deue . \textbf{ Conuiene que por el entendimiento piadoso sea atenprada la guaueza dela pena } e esto es lo que dize el pho & sunt amplioris seueritatis contentiua , \textbf{ decet ut per pium intellectum moderetur supplicii magnitudo , } hoc est ergo quod dicitur 1 Rhetor’ \\\hline
3.2.23 & commo muestre la obra \textbf{ e por que las cosas dubdosas son de iudgar ala meior parte } por ende si el iues en alguna manera puede entender & ut opus ostendit : \textbf{ et quia dubia iudicanda sunt | in meliorem partem , } si aliquo modo potest \\\hline
3.2.23 & por la mayor parte much sseruiçios fizo . \textbf{ Et el que muchs seruiçios fizo por la mayor parte mucho t p̃o siruo } Et enpero contesçe que alas uezes & ut plurimum se committentur , quia qui multo tempore seruiuit \textbf{ ut plurimum multa seruitia fecit , } et econuerso ; \\\hline
3.2.23 & que en mui cħtp̃o ouo pocas oportuni dades deur . \textbf{ e en poco t p̃o o no muchͣ̃s para leruir } Et por ende estas esta razon que inclina al iues a piedat & quod in multo tempore occurrant opportunitates paucae , \textbf{ et in pauco multae . } Istud itaque sextum inclinatiuum ad pietatem \\\hline
3.2.24 & que es algun derechn atal . Et alguno es derecho delas gentes \textbf{ Et alguno es de techo çiuil dela çibdat . } Et pues que assi es en aquella man & et quoddam ius gentium , \textbf{ et quoddam ciuile . } Illo ergo modo quo iuristae separant \\\hline
3.2.24 & Et pues que assi es en aquella man \textbf{ era que los iuristas apartan el derecho natural del derecho delas } gentes podemos nos apartar el derecho natural del derecho delas animalias & et quoddam ciuile . \textbf{ Illo ergo modo quo iuristae separant | ius naturale a iure gentium , } possemus separare nos ius naturale \\\hline
3.2.24 & era que los iuristas apartan el derecho natural del derecho delas \textbf{ gentes podemos nos apartar el derecho natural del derecho delas animalias } e darla quanta distinçion & ius naturale a iure gentium , \textbf{ possemus separare nos ius naturale | a iure animalium : } et dare quintam distinctionem iuris , \\\hline
3.2.24 & O son dichos natales \textbf{ por que la razon natural los muestra de ser tales } o por que auemos natural apetito o natural inclinaçion & vel dicuntur iusta naturaliter \textbf{ quae dictat esse talia ratio naturalis , } vel ad quae habemus naturalem impetum et inclinationem . \\\hline
3.2.24 & a ellos aas los derechos positiuos \textbf{ e las leyes positiuas son dichos aquellos } que non por su natura son tales . & vel ad quae habemus naturalem impetum et inclinationem . \textbf{ Iusta vero positiua dicuntur , } quae non ex natura sua , \\\hline
3.2.24 & assi commo en italia . \textbf{ por que las cosas natraales son vnas a todos los omes } commo quier que non sean nonbradas & est ignis in alio ut in Italia : \textbf{ res enim eaedem sunt apud omnes ; } licet non eodem vocabulo nominentur . \\\hline
3.2.24 & Et por ende se sigue \textbf{ que el derecho natural es departido del derechpo sitiuo . } ca el derech natural & et edicta principum non sunt eadem apud omnes . \textbf{ Inde est quod ius naturale dicitur | differre a positiuo : } quia naturale ut traditur 5 Ethicor’ \\\hline
3.2.24 & de obligar alos omes . \textbf{ Mas la razon por que al derech natural conuinio anneder derecho positiuo es esta } por que muchas cosas son derechas naturalmente & habere ligandi efficaciam . \textbf{ Ratio autem , | quare iuri naturali oportuit } superaddere positiuum , \\\hline
3.2.24 & Et pues que assi es \textbf{ assi commo fablar es cosa natural alos omes } assi fablar tal & sermonem nobis esse datum a natura . \textbf{ Sicut ergo loqui est naturale , } sic autem loqui vel sic , est positiuum et ad placitum . \\\hline
3.2.24 & derechnatural \textbf{ por que la razon natural muestra } que se deuen fazer . & et cetera huiusmodi sunt , \textbf{ de iure naturali , } quia haec esse fienda \\\hline
3.2.24 & que son dadas por la natura \textbf{ Ca el derecho positiuo que es fallado } por arte & in his quae tradita sunt a natura , \textbf{ ius enim positiuum } per artem \\\hline
3.2.24 & que son dela natura \textbf{ Por la qual cosa si el derecho natural manda } que los ladrones & quae sunt naturae . \textbf{ Quare si ius naturale dictat fures et maleficos esse puniendos , } hoc praesupponens ius positiuum procedit ulterius , \\\hline
3.2.24 & e dos diferençias \textbf{ entre el derech natural e el positiuo . } La primera es que el derech natural & assignare possumus \textbf{ inter ius naturale , et positiuum . } Prima est , \\\hline
3.2.24 & luego en la primera faz se ofresçe al entendimiento \textbf{ mas el derecho positiuo non se ofresçe luego al entendimiento } mas es fallado & quia ius naturale prima facie se offert intellectui : \textbf{ Ius positiuum non statim se ostendit , } sed est per industriam hominum adinuentum . \\\hline
3.2.24 & fazen aquellas cosas \textbf{ que son dela ley e muestran la obra dela ley escpta en sus coraçons . } Mas el & naturaliter ea quae legis sunt faciunt , \textbf{ et ostendunt opus legis scriptum | in cordibus eorum . } Ius vero positiuum \\\hline
3.2.24 & La segunda diferençia es \textbf{ que el derecho narurales vno a todo los omes . } Et por ende es dicho ser derecho comunal . & Secunda differentia est , \textbf{ quia ius naturale est | idem apud omnes , } ideo dicitur esse ius commune : \\\hline
3.2.24 & que el derecho narurales vno a todo los omes . \textbf{ Et por ende es dicho ser derecho comunal . } Mas el derecho positiuo es departido en departidas çibdades & idem apud omnes , \textbf{ ideo dicitur esse ius commune : } sed ius positiuum diuersificatur apud diuersas ciuitates , \\\hline
3.2.24 & Et por ende es dicho ser derecho comunal . \textbf{ Mas el derecho positiuo es departido en departidas çibdades } e por ende es dicho derecho propreo . & ideo dicitur esse ius commune : \textbf{ sed ius positiuum diuersificatur apud diuersas ciuitates , } ideo vocatur ius proprium . \\\hline
3.2.24 & en el primero libro de la \textbf{ rectorica al derecho natural ayre o fuego } que continuadamente se estiende en claridat & ut recitat Philosoph’ 1 Rhet’ appellat \textbf{ ius naturale aetherem siue ignem , } qui continuat protendere \\\hline
3.2.24 & e es mas conosçido \textbf{ e mas claro que el derech positiuo . } Pues que assy es desto paresçe & quam ius positiuum , \textbf{ et est notius et clarius illo . } Ex hoc igitur patere potest , \\\hline
3.2.24 & que todas las disticonnes que pone el pho del derecho o del iusto \textbf{ son aduzidas a derecho positiuo e natraal . } Ca assi commo paresçe & de iure siue de iusto reducuntur \textbf{ ad ius positiuum , | et naturale : } nam \\\hline
3.2.24 & por lo que diches el \textbf{ derechon atraal es dicho non ser escpto } e ser comunal e ser segunt natura . & ( ut patet per habita ) \textbf{ ius naturale dicitur esse non scriptum , } et esse commune , \\\hline
3.2.25 & Et pues que assi es \textbf{ si aquellas cosas son del derech natural . } a que auemos natal inclinaçion & Si igitur ea sunt \textbf{ de iure naturali , ad quae habemus naturalem impetum et inclinationem : } huiusmodi naturalis impetus \\\hline
3.2.25 & a que auemos natal inclinaçion \textbf{ esta inclinaçion natural del apeti } too ligue lanr̃a natura & de iure naturali , ad quae habemus naturalem impetum et inclinationem : \textbf{ huiusmodi naturalis impetus } vel sequitur naturam nostram , \\\hline
3.2.25 & esta inclinaçion natural del apeti \textbf{ too ligue lanr̃a natura } en quanto somos omes & huiusmodi naturalis impetus \textbf{ vel sequitur naturam nostram , } ut sumus homines , \\\hline
3.2.25 & Et pues que assi es el derecho delas gentes es \textbf{ assi commo vn derecho natural mas espeçial . } Et por ende aquel derecho & Ius ergo gentium est \textbf{ quoddam ius naturale contractum . } Ius itaque illud quod natura omnia animalia docuit , \\\hline
3.2.25 & en quanto participamos con las otras aianlas \textbf{ en conparacion del derecho delas gentes es dicho derecho natural . } Ca si penssaremos los dichos del capitulo & ut communicamus cum animalibus aliis , \textbf{ respectu iuris gentium dicitur esse naturale . } Nam si considerentur dicta in praecedenti capitulo , \\\hline
3.2.25 & Ca si penssaremos los dichos del capitulo \textbf{ sobredicho el derecho natural es algun an cosa comun } e es algunan cosa conosçida & Nam si considerentur dicta in praecedenti capitulo , \textbf{ ius naturale est | quid commune , } quid notum , \\\hline
3.2.25 & e resçiben mayor mudamiento . \textbf{ Et pues que assi es con grant razon el derecho delas aianlias es dicho ser derecho natural en conparaçion del derecho delas gentes } ¶ & et maiorem mutationem suscipiunt : \textbf{ merito igitur huiusmodi ius , | naturale dicitur , } respectu iuris gentium . \\\hline
3.2.25 & Visto en qual manera el derecho delas gentes se \textbf{ departe del derecho natal de ligo puede paresçer } en qual manera el derecho delas aialias se departe del derech̉natiral . & differt \textbf{ a iure naturali , | de leui patere potest } quomodo ius animalium differt \\\hline
3.2.25 & assi podria ser de derecho natural \textbf{ en quanto el derecho naturales dicho ser } tal qual la natura mostro a todas las ainalias . & sic esse poterunt de iure naturali , \textbf{ prout ius naturale dicitur esse , } quod natura omnia animalia docuit . \\\hline
3.2.25 & assi seran de derecho natural \textbf{ en quanto el derecho natural es traydo al derecho delas gentes } el qual derecho esppreo solamente al linage humanal . & sic erit de iure naturali , \textbf{ prout ius naturale contractum est | ad ius gentium , } quod est proprium soli humano generi . \\\hline
3.2.25 & que assi commo el derecho delas gentes \textbf{ non es dicho assi derecho natural commo el derecho } que la natura enssenno a todas las asanlias . & quod ius gentium non dicitur \textbf{ ita ius naturale , } sicut ius quod nam omnia animalia docuit : \\\hline
3.2.25 & que es sinplementeposituo . \textbf{ Et pues que assi es tres cosas son en alguna manera del derecho natural Lo primero es que el sea ygualado proporçionado ala natura humanal } o que lo diga la razon nata & quod est simpliciter positiuum . \textbf{ Tria ergo sunt aliquo modo de iure naturali , } secundum quod inclinatio sequitur naturam nostram : \\\hline
3.2.25 & o que lo diga la razon nata \textbf{ lo que ayamos a ello inclinaçion natural . } Mas segunt que la intlinaçion sigue lanr̃a natura & Tria ergo sunt aliquo modo de iure naturali , \textbf{ secundum quod inclinatio sequitur naturam nostram : } Nam si inclinatio illa sequitur naturam nostram \\\hline
3.2.25 & e el derech delas aianlias \textbf{ e avn el derechçiuil se departe del derecho natural . } Saresçe que derecho çiuil o el derecho humanal & et ius animalium , \textbf{ et etiam ius ciuile differt | a iure naturali . } Videtur autem ius ciuile , \\\hline
3.2.26 & e avn el derechçiuil se departe del derecho natural . \textbf{ Saresçe que derecho çiuil o el derecho humanal } e positiuo es conparado a tres cosas . & a iure naturali . \textbf{ Videtur autem ius ciuile , | siue ius humanum et positiuum ad tria comparari , } videlicet ad ius naturale \\\hline
3.2.26 & Con uiene a saber . \textbf{ al derecho natural o ala ley dela natura dela qual tomo Rays e fuidamiento . } Et puede se conparar al bien comun & videlicet ad ius naturale \textbf{ siue ad legem naturalem , | a qua suscipit fundamentum : } ad bonum commune quod in ea intenditur : \\\hline
3.2.26 & Lo primero conuiene que la ley humanal o positiua sea derecha \textbf{ en quanto es conparada ala razon natural o ala ley de natura . } Ca si derechͣ non fuere non es ley mas es & siue positiuam esse iustam \textbf{ ut comparatur ad rationem naturalem | siue ad legem naturalem : } quoniam si iusta non sit , \\\hline
3.2.26 & tomare comienco dela ley de natura . \textbf{ Et si en alguna manera la razon natural non iudgare } que aquello deue ser establesçido . & ex lege naturali , \textbf{ et nisi aliquo modo ratio naturalis dictet illud statuendum esse . } Secundo lex humana et ciuilis debet esse utilis \\\hline
3.2.26 & Ca si enla ley non fuere entendido el bien comun \textbf{ non sera la ley derechͣ nin de Rey . } Mas sera mala e de tirano . & nam si in lege non intenditur bonum commune , \textbf{ tunc non est recta et regularis , } sed peruersa et tyrannica \\\hline
3.2.26 & e solamente en su prouecho . \textbf{ Assi la ley derechͣ e real } enla qual es entendido el bien comun & et intendit proprium et priuatum commodum : \textbf{ sic lex recta et regia } in qua intenditur bonum commune differt \\\hline
3.2.26 & enla qual es entendido el bien comun \textbf{ se departe dela ley mala e dela ley del tyrano } en la qual es entendido el bien propreo . & in qua intenditur bonum commune differt \textbf{ a peruersa et tyrannica } in qua intenditur priuatum bonum ; \\\hline
3.2.26 & e los prinçipes deuen poner Ca deuen poner buenas e aprouechables \textbf{ e parte nesçientes al pueblo } al qual son puestas de ligero puede paresçer & quae sunt iustae , \textbf{ utiles et competens populo , } cui imponuntur : \\\hline
3.2.26 & Assi que las palabras solas \textbf{ e los castigos solos los inclinan para faz̃ bien . } Mas otros son tan malos & ita quod soli sermones \textbf{ et solae increpationes | inclinant eos ad bonum . } Sed aliqui sunt adeo peruersi , \\\hline
3.2.26 & nin se emiendan \textbf{ por las palabras solas . } Et por ende conuenia que por estas tales fuessen establesçidas las leyes . & qui nec de se inclinantur ad bonum , \textbf{ nec per solos sermones corriguntur : } oportuit igitur saltem \\\hline
3.2.27 & establesçer leyes e reglas delas nuestras obras . \textbf{ por las quales leyes ymosa aquel bien . } Por la qual cosa commo el bien comun sea entendido & et regulas agibilium \textbf{ secundum quas intendimus in bonum illud : } quare cum bonum commune principaliter intendatur a tota communitate \\\hline
3.2.27 & de enderesçar los omes al bien comun . \textbf{ Ca si es ley diuinal e natural establesçida es de dios } a quienꝑtenesçe enderesçar todas las cosas asimesmo . & ab eo cuius est dirigere in bonum commune : \textbf{ nam si est lex diuina et naturalis , | condita est a Deo } cuius est omnia dirigere in seipsum , \\\hline
3.2.27 & ningnon si non fuere obligada e prigo nada . \textbf{ Ca commo la ley sea vn mandamiento del sennor mayor . } por el qual somos e reglados e ligados en las nr̃as obras & nisi sit promulgata . \textbf{ Nam cum lex sit | quoddam mandatum superioris , } per quod ligamur et regulamur \\\hline
3.2.27 & conuiene que sea publicada e pregonada . \textbf{ Mas commo otra sea la ley natural e otra la positiua en vna manera se deue publicar la vna } e en otra manera la otra . & oportet eam promulgatam esse . \textbf{ Sed cum alia sit lex naturalis , | alia positiua : } aliter propalatur haec , \\\hline
3.2.27 & e en otra manera la otra . \textbf{ Ca la ley naturales en tanto fincada enlos nuestros coraçons } que en cada vn omne es publicada e manifestada & aliter illa . \textbf{ Nam lex naturalis est a deo indita in cordibus nostris : } ideo in quolibet homine haec promulgatur et propalatur , \\\hline
3.2.27 & e qual cosa ha de foyr e de escusar \textbf{ segunt que esto pertenesçe al derecho natural . } Mas la ley humanal e positiua estonçe es publicada & quid sequendum et quid fugiendum , \textbf{ secundum quod haec pertinent ad ius naturale . } Sed lex humana tunc promulgatur , \\\hline
3.2.28 & assi establesçidas obedes tan bien los omes \textbf{ a demostramos enlos capitulos sobredichos quales deuen ser las leyes } que son de poner & ut legibus sic institutis bene obediatur . \textbf{ Ostendimus in praecedentibus capitulis , } quales debent esse leges condendae \\\hline
3.2.28 & por que todos los omes non pueden ser en todo acabados . \textbf{ Por ende parte nesçe al ponedor dela ley } non solamente conssentir aquellas cosas & esse omnino perfecti : \textbf{ spectat ad legislatorem } non solum permittere \\\hline
3.2.28 & nin dando pena por ellas . \textbf{ Mas avn parte nesçe al ponedor dela ley conssentir aquellas cosas } qua non se arriedran mucho del medio & nec puniendo , \textbf{ sed etiam spectat ad ipsum permittere } non solum quae non notabiliter recedunt a medio : \\\hline
3.2.28 & por todos los males \textbf{ o por quales quier culpas pequanas apenas o nunca poder e gouernar ningun pueblo . } Por ende non solamente son de conssentir aquellas cosas & quantumcunque modica prohibere et punire , \textbf{ vix aut nunquam posset | aliquem populum regere . } Ideo non solum permittenda sunt indifferentia , \\\hline
3.2.28 & mas despues que son ya fechas son de castigar . \textbf{ Et las obras buen asante } que se fagan son demandar & sunt punienda . \textbf{ Opera vero notabiliter bona , } antequam fiant , \\\hline
3.2.28 & Et despues que son fechas dar pena por ellas . \textbf{ Mas bna obra sola apodamos } e a & et punire facta : \textbf{ sed unum attribuimus legibus } respectu operum indifferentium \\\hline
3.2.29 & que sea fecho derechamente el Rey \textbf{ por su pode rio çiuil faga lo guardar . } Por la qual & ut quod iuste lex fieri praecipit , \textbf{ Rex per ciuilem potentiam obseruari facit . } Quare si quod est principalius , \\\hline
3.2.29 & e que los Reyes e los prinçipes \textbf{ que son establesçidos paser guardadores dela ley o seruidores delas leyes . } La segunda razon para mostrar esto mismo se toma & quia Reges aut Principes ita sunt instituendi , \textbf{ ut seruatores legis et ministri . } Secunda via ad inuestigandum hoc idem , \\\hline
3.2.29 & a razon paresçe \textbf{ que diga entendimiento solo . } Et por ende dize el pho en el terçero delas politicas & quia est aliquid pertinens ad rationem , \textbf{ videtur dicere intellectum solum : } ideo dicitur 3 Polit’ \\\hline
3.2.29 & conuiene de saber que el rey \textbf{ e qual se quier sennor deue ser medianero entre la ley natural e la ley positiua . } Ca ninguno non iudga derechamente nin & et quemlibet principantem \textbf{ esse medium | inter legem naturalem et positiuam ; } nam nullus recte principatur , \\\hline
3.2.29 & que sigua la ley natural \textbf{ la qual se leunata de razon derecha e de entendumento derecho . } Et por ende el rey en gouernando es a & oportet Regem in regendo alios \textbf{ sequi rectam rationem , } et per consequens sequi naturalem legem , \\\hline
3.2.29 & por que en tanto gouierna derechamente \textbf{ en quanto non se parte nin se arriedra dela ley natural . } Enpero es sobre la ley positiua & quia in tantum recte regit , \textbf{ in quantum a lege naturali non deuiat : } est tamen supra legem positiuam , \\\hline
3.2.29 & et si non feziere \textbf{ assi commo la razon derecha o el entendimiento manda . } En essa misma manera la ley positiua nunca liga & nisi innitatur lege naturali , \textbf{ et agat ut recta ratio dictat : } sic lex positiua nunquam recte ligat , \\\hline
3.2.29 & assi commo la razon derecha o el entendimiento manda . \textbf{ En essa misma manera la ley positiua nunca liga } nin obliga derechamente & et agat ut recta ratio dictat : \textbf{ sic lex positiua nunquam recte ligat , } nisi innitatur auctoritati legis \\\hline
3.2.29 & o por auctoridat de otro señor tlxv alguno . \textbf{ Ca el derecho positiuo e dela ley } assi commo dicho es de suso en el comienço non ha ningun departimiento & aut alterius principantis . \textbf{ Nam iustum positiuum et legale } ( ut supra dicebatur ) \\\hline
3.2.29 & por auctoridat de aquel quela pone ha departimiento e diferençia \textbf{ por la qual cosa la ley positiua es a quande del señor } que la manda fazer & propter auctoritatem ponentis differt : \textbf{ quare positiua , | lex est infra principantem , } sicut lex naturalis est supra . Et si dicatur legem \\\hline
3.2.29 & que la manda fazer \textbf{ assi commo la ley naturales sobre el señor } por que la non puede mudar . & lex est infra principantem , \textbf{ sicut lex naturalis est supra . Et si dicatur legem } aliquam positiuam esse supra principantem , \\\hline
3.2.29 & Et si dixiere alguno \textbf{ que alguna ley positiua deue ser sobre el sennor . } esto non es en quanto es ley positua & sicut lex naturalis est supra . Et si dicatur legem \textbf{ aliquam positiuam esse supra principantem , } hoc non est ut positiua est , \\\hline
3.2.29 & esto non es en quanto es ley positua \textbf{ mas en quanto enella es guardada la uirtud del derecho e dela ley natural . } Et pues que assi es quando es demandado enla quastiuo & hoc non est ut positiua est , \textbf{ sed ut in ea reseruatur | virtus iuris naturalis . } Cum ergo quaeritur \\\hline
3.2.29 & por muy buen Rey o por muy buena ley . \textbf{ si fablaremos dela leyna turͣal paresçe } que es mas prinçipal en & regnum aut ciuitatem Regi optimo Rege , aut optima lege . \textbf{ Si loquamur de lege naturali , } patet hanc principaliorem esse in regendo , \\\hline
3.2.29 & por que ninguon non es derecho rey \textbf{ si non en quanto se esfuerça enla ley natural . } Por la qual cosa bien dicho es & eo quod nullus sit rectus Rex \textbf{ nisi in quantum innititur illi legi . } Propterea bene dictum est \\\hline
3.2.29 & la qual dios puso en el entendimiento de cada vno . \textbf{ ¶ Mas si fablaremos dela ley positiua meior es } que el regno & quam Deus indidit intellectui cuiuscumque . \textbf{ Sed si loquamur de lege positiua } melius est Regi optimo Rege \\\hline
3.2.29 & por buen Rey \textbf{ que por buena ley que por la ley non puede determinar todas los casos particulates . } Por ende conuiene que el Rey o otro prinçipe & quod melius est Regi Rege , \textbf{ quam lege | eo quod lex particularia determinare non potest . } Ideo expedit Regem \\\hline
3.2.29 & Por ende conuiene que el Rey o otro prinçipe \textbf{ por razon derechͣo por ley natural . } la qual dios puso en voluntad de cada vn omne & aut alium principantem per rationem rectam , \textbf{ aut per legem naturalem , } quam Deus impressit \\\hline
3.2.29 & e se allegue algunas vezes ala vna parte e que obre mas manssamente con el que peca \textbf{ quela ley demanda o que la ley nidga . } Et algunas vezes conuiene que la regla se encorue & et agere mitius cum delinquente , \textbf{ quam lex dictat : } aliquando etiam oportet eam plicare ad partem oppositam , \\\hline
3.2.29 & maguera parezca \textbf{ conterra la iustiçia legal e positiua . } Enpero non son prinçipalmente nin sinplemente sin iustiçia . & videantur esse \textbf{ praeter iustitiam legalem et positiuam ; } non tamen sunt \\\hline
3.2.29 & si con razon se fezieren \textbf{ segunt que mandan las cercunstançias particulares . } Et pues que assi es dende viene & si rationabiliter fiant \textbf{ exigentibus particularibus circumstantiis . } Inde est ergo quod in iudicando , \\\hline
3.2.29 & ala parte contraria \textbf{ sera el iuizio de fortaleza o de iustiçia estrecha . } Et por que todas estas cosas se pueden fazer derechamente & ad partem oppositam , \textbf{ fiet iudicium ex rigore siue seueritate . } Et quia haec omnia iuste \\\hline
3.2.30 & e se mandan fazer todas las uirtudes \textbf{ Mas que sin la ley natural e humanal fue menester de dar ley e un aglical e diuinal . } podemos lo prouar por tres razones & et omnes virtutes praecipere . \textbf{ Sed quod praeter legem naturalem | et humanam fuerit } expediens \\\hline
3.2.30 & tan bien los de dento del coraçon commo los de fuera . \textbf{ Et los tris passadores desta ley diuirial } fuessen condenpnados en este siglo o en el otro & tam delicta interiora quam exteriora , \textbf{ cuius transgressores } vel in hoc seculo , \\\hline
3.2.30 & por que se escusen los mayores . \textbf{ assi commo las fornicaçonnes sinples son conssentidas } e non condep̃nadas por las leyes humanales & ut vitentur maiora : \textbf{ ut permittuntur fornicationes simplices , } et non puniuntur legibus humanis , \\\hline
3.2.30 & Ca commo este tal bien sea sobre el poderio dela nuestra natura . \textbf{ la ley natural e la } humanal que nos ayudan a alcançar este bien . & supra facultates nostrae naturae , \textbf{ lex naturalis et humana iuuantes nos } ad consecutionem illius boni \\\hline
3.2.30 & e de ser forma de beuir e regla de todas las obras . \textbf{ assi se auer ala leyna traal e diuinal e humanal } por que assi commo sobrepuian los otros en poderio e en diuinidat & et regulam agibilium , \textbf{ sic se habere ad legem diuinam , | naturalem , et humanam : } ut sicut excedunt alios potentia et dignitate , \\\hline
3.2.31 & quando disputa contra ypodomio \textbf{ si es cosa conuenible alas çibdades } de renouar las leyes dela tierra & cum disputat contra Hippodamum , \textbf{ utrum sit expediens ciuitatibus } innouare patrias leges , \\\hline
3.2.31 & diziendo que aquellas eran prouechosas ala çibdat . \textbf{ Et en esto desfazien e destruyen las leyes antiguas dela tr̃ra . } Et por ende non sin razon dubdauna si la opinion de ypodomio era buena & dicentes eas esse utiles et proficuas ciuitati , \textbf{ soluerent leges patrias et antiquas . } Merito ergo dubitatur , \\\hline
3.2.31 & de los que establesçen las leyes \textbf{ ¶la quarta dela non determinaçion delas cercunstançias particulares . } La primera razon paresçe & Tertia ex simplicitate condentium leges . \textbf{ Et quarta ex indeterminatione particularium circunstantiarum . } Prima via sic patet . \\\hline
3.2.31 & que las que fallaron los primeros padres \textbf{ que non son de guardar las leyes antiguas dela tierra dende adelante . } ¶ La segunda razon para mostrar esto mismo se toma de parte dela maldat de alguas leyes . & quam sint traditae a prioribus patribus , \textbf{ non sunt ulterius leges patriae obseruandae . } Secunda via ad ostendendum hoc idem , \\\hline
3.2.31 & e algun pariente uiniesse acometiendo contra algun çibdadano . \textbf{ Et aquel presentes alguons çibdadanos fuyesse del } maguer el non fuesse el matador & et aliquis consanguineus mortui inuaderet aliquem ciuem , \textbf{ et ille praesentibus aliquibus fugeret } ab eo reputabatur \\\hline
3.2.31 & Et pues que assi es por que algunas leyes dela tierra son malas . \textbf{ lon de renouar las leyes antiguas . } La tercera razon se toma dela sinpliçidat & ergo quia aliquae leges paternae sunt prauae , \textbf{ innouandae sunt . } Tertia via sumitur \\\hline
3.2.31 & ¶La quarta razon se toma dela non determinaçion delas particulares . \textbf{ Ca las obras particulares non son determinadas } nin se pueden saber acabada mente . & ex indeterminatione particularium actuum . \textbf{ Nam agibilia particularia indeterminata sunt , } et perfecte comprehendi non possunt : \\\hline
3.2.31 & Et por ende si los postrimeros sabios \textbf{ por la esperiençia delas obras particulares alguna cosa fallar en meior } non es cosa sin razon de tirar las leyes dela tierra antiguas & circa agibilia hominum . \textbf{ Si igitur posterioribus propter experientiam agibilium particularium } occurrit aliquid melius , \\\hline
3.2.31 & por las meiores leyes falladas nueuamente por ellos . \textbf{ Et por ende paresçe que estas razones sobredichas prueuna que cada que } acahesçiere algua cosa & et antiquas propter meliores leges nouiter inuentas . \textbf{ Videntur itaque hae rationes probare } quod quotiescunque occurrit aliquid melius , \\\hline
3.2.31 & es acostunbrar sea non obedesçer alas leyes \textbf{ Ca las leyes grant fuerça toman dela costunbre . } Et esto por que con grant & est assuescere non obedire legibus . \textbf{ Nam leges magnam efficaciam habent ex consuetudine : } de difficili enim \\\hline
3.2.31 & e alos reyes muestra lo el philosofo \textbf{ en el primero libro delaL rectorica } do dize que mas enpeesçe acostunbrar se los omes & non obedire Regibus et legibus , \textbf{ ostendit Philosophus 1 Rhetor’ } qui ait , magis nocere , \\\hline
3.2.31 & si fuere derecha conuiene que se raygͤ \textbf{ e se funde enla ley natural . } Et conuiene que determine las obras & sciendum quod lex positiua si recta sit , \textbf{ oportet quod innitatur legi naturali , } et quod determinet gesta particularia hominum . \\\hline
3.2.31 & Et conuiene que determine las obras \textbf{ e los fechos particulares de los omes . } Et por ende en dos maneras puede la ley posiua fallesçer o auer mengua¶ & oportet quod innitatur legi naturali , \textbf{ et quod determinet gesta particularia hominum . } Dupliciter ergo potest \\\hline
3.2.31 & e los fechos particulares de los omes . \textbf{ Et por ende en dos maneras puede la ley posiua fallesçer o auer mengua¶ } Lo primero si & et quod determinet gesta particularia hominum . \textbf{ Dupliciter ergo potest | huiusmodi lex habere defectum . } Primum si sit \\\hline
3.2.31 & Lo primero si \textbf{ fuerecontraria ala ley natural . } Lo segundo si non determinar e conplidamente los fechos e las obras particulares & Primum si sit \textbf{ contraria legi naturali . } Secundo si non sufficienter determinaret particularia gesta . \\\hline
3.2.31 & Lo segundo si non determinar e conplidamente los fechos e las obras particulares \textbf{ si en la primera manera fallesçieron las leyes positiuas dela tierra } siendo contra la ley natural non son leyes & Secundo si non sufficienter determinaret particularia gesta . \textbf{ Si primo modo deficiant leges paternae et positiuae , } non sunt leges sed corruptiones legum , \\\hline
3.2.31 & si en la primera manera fallesçieron las leyes positiuas dela tierra \textbf{ siendo contra la ley natural non son leyes } mas son corronpimiento de leyes . & Si primo modo deficiant leges paternae et positiuae , \textbf{ non sunt leges sed corruptiones legum , } propter hoc obseruari non debent . \\\hline
3.2.31 & por la qual cosa non se deuen guardar . \textbf{ Ca las leyes positiuas commo } quierque sean ennadidas & propter hoc obseruari non debent . \textbf{ Nam leges positiuae licet } sint additae naturalibus legibus , \\\hline
3.2.31 & assi commo paresçe en la \textbf{ institutado dize que las leyes humanales contrarias son al derecho natraal . } Ca de comienço todos los omes & ut patet ex Institutis de iure naturali , \textbf{ ubi dicitur quod leges humanae contrariae sunt iuri naturali ; | quia iure naturali } ab initio homines liberi nascebantur . \\\hline
3.2.31 & cuyo contrario dize \textbf{ e manda la razon natural e el entendimiento . } Et por que la seruidunbre es puesta & Nam illud proprie est contra naturam , \textbf{ cuius contrarium dictat ratio naturalis : } et quia propter utilitatem \\\hline
3.2.31 & Erpero non es contraria al derech natural \textbf{ nin es contra el derecho natural . } Ca la razon natural non contradize atal seruidunbre . & quod est natura productum : \textbf{ non tamen est contra ius naturale , } quia huic naturalis ratio non contradicit . \\\hline
3.2.31 & nin es contra el derecho natural . \textbf{ Ca la razon natural non contradize atal seruidunbre . } ¶ Et pues que assi es las leyes que fallesçen & non tamen est contra ius naturale , \textbf{ quia huic naturalis ratio non contradicit . } Leges ergo deficientes \\\hline
3.2.31 & mas de tirar e de derraygar . \textbf{ Mas si las leyes positiuas fallesçieren } por que non determinan conplidamente los fechos particula respuesto & sed extirpandae . \textbf{ Si vero leges sint defectiuae , } quia non complete determinant particularia agibilia , \\\hline
3.2.31 & por que non determinan conplidamente los fechos particula respuesto \textbf{ que sean falladas leyes meiores e mas conplidas . } Enpero non nos auemos a acostunbrar a renouar las leyes . & quia non complete determinant particularia agibilia , \textbf{ dato quod occurrant leges meliores et magis sufficientes , } non est assuescendum innouare leges . \\\hline
3.2.31 & Enpero non nos auemos a acostunbrar a renouar las leyes . \textbf{ Lo primero por que algunans vegadas contesçe que se engannan los omes } çerca tales cosas & non est assuescendum innouare leges . \textbf{ Primo , quia aliquando contingit } circa talia decipi , \\\hline
3.2.31 & nueuamente en alguna cosa fuessen mas sufiçientes . \textbf{ Enpero non deuen dexar de ser guardadas las leyes antiguas e las leyes dela tierra . } Ca en quanto dela vna parte a alguno prouecha & leges nouiter adinuentae , \textbf{ sunt tamen leges antiquae , | et paternae obseruandae , } quia quanto ex una parte quis perficit , \\\hline
3.2.32 & e abondasse muchen viandas \textbf{ e en los otros bienes tenporales . } Enperosi non visquiesse en conpannia & Si quis enim magna multitudine argenti et auri polleret , \textbf{ et omnibus victualibus abundaret , } et si non viueret in societate , \\\hline
3.2.32 & e en consolaçion mas avn \textbf{ por el beuir sola mente . } Ca los omes que estan en vna çibdat siruen assi mismos & non solum propter iocunde et delectabiliter conuersari , \textbf{ sed propter ipsum viuere . } Nam homines in eadem ciuitate existentes deseruiunt sibi ad vitam , \\\hline
3.2.32 & sean muy grandes bienes \textbf{ maguera que por todas las cosas sobredichas sea fecho la çibdat en alguna manera . } Enpero prinçipalmente fue ella establesçida & sunt maxima bonorum , \textbf{ licet propter omnia praedicta bona sit | aliquo modo ciuitas constituta , } potissime tamen constituta est \\\hline
3.2.32 & Ca si la çibdat e el regno son ordenados \textbf{ prinçipalmente a buena uida uertuosa . } los moradores del regno e el pueblo & Nam si ciuitas et regnum principaliter ordinatur \textbf{ ad vitam bonam et virtuosam , } habitatores regni et populum existentem \\\hline
3.2.32 & o en nobleza de linage \textbf{ o en otros bienes tenporales . } Et pues que assi es muchs cuydan & vel in nobilitate generis , \textbf{ vel aliis exterioribus bonis : } Credunt ergo multi esse ciues , \\\hline
3.2.33 & Et alguons muy pobres . \textbf{ Et otros son terçeros medianeros entre estos . } Et en esta manera que es departida la çibdat en tres partes & alii autem egeni valde , \textbf{ alii vero sunt horum medii . } Hoc ergo modo quo diuisa est ciuitas in tres partes , \\\hline
3.2.33 & e encobiertamente tomar e robar de sus biens . \textbf{ Mas si en el pueblo fueren muchas perssonas medianeras quedaran todo estos enpeesçimientos } e de ligero los omes obedescran ala razon . & et latenter eorum depraedari bona . \textbf{ Sed si in populo sint multae personae mediae , | cessabunt huiusmodi nocumenta , } et de facili rationi obediunt . \\\hline
3.2.33 & Mas si en el pueblo fueren muchas perssonas medianeras quedaran todo estos enpeesçimientos \textbf{ e de ligero los omes obedescran ala razon . } Et esto es lo que dize el philosofo en el quarto libro delas politicas & cessabunt huiusmodi nocumenta , \textbf{ et de facili rationi obediunt . } Hoc est quod dicitur 4 Politicorum \\\hline
3.2.33 & Et esto es lo que dize el philosofo en el quarto libro delas politicas \textbf{ que por que lo medianero es muy bueno la possesion medianera es muy buena . } por que muy de ligero obedesçe ala razon . & Hoc est quod dicitur 4 Politicorum \textbf{ quod quoniam mediocre est optimum , | optima est possessio media , } quia facilius rationi obediet . \\\hline
3.2.33 & e muy poderosa et̃ muy rica . \textbf{ la otra parte contraria desta } fuere muy menguada & superpotens , et superdiues : \textbf{ alia vero huic contraria sit superegena , } superdebilis et valde vilis , \\\hline
3.2.33 & alos otros es bueno \textbf{ que el pueblo abonde en perssonas medianeras . } la terçera razon se toma dela egualdat e dela iustiçia & Ut ergo sit mutuus amor inter ciues , \textbf{ bonum est propter hoc abundare in personis mediis . } Tertia via sumitur \\\hline
3.2.33 & que las çibdades deuien ser \textbf{ establesçidas de perssonas medianeras . } Et pues que assi es conuiene & Dixerunt enim eas constituendas esse \textbf{ ex personis mediis . } Decet ergo Reges et Principes adhibere cautelas , \\\hline
3.2.33 & nin ouiere cada vno liçençia de conprar qualsquier possessiones . \textbf{ Ca auiendo diligençia e acuçia conuenible en las conpras } e enlas uendidas de los canpos e delas tierras & quascunque possessiones emere , \textbf{ adhibita enim debita diligentia circa emptionem , } et venditionem agrorum et terrarum , \\\hline
3.2.34 & e guardaren las leyes . \textbf{ Mas quanta salut se leu nata en el regno dela obediençia del Rey } conplidamente se muestra & et obseruent leges . \textbf{ Quanta autem salus surgat in regno | ex obedientia Regis , } sufficienter ostenditur , \\\hline
3.2.34 & e assessiego de los çibdadanos \textbf{ e abondamiento de los bienes tenporales . } Ca assi conmodicho es de ssuso las leyes & et ex obseruatione legum oritur pax et tranquillitas ciuium , \textbf{ et abundantia exteriorum rerum . } Nam ( ut supra dicebatur ) leges et etiam legislatores , \\\hline
3.2.34 & si al que non por temor de pena se quite de aquellas malas obras . \textbf{ Et por ende cosa conuenible fue al regno } e ala çibdat de auer algun Rey o algun prinçipe & saltem timore poenae retrahatur ab illis . \textbf{ Expediens enim fuit regno et ciuitati } habere aliquem Regem \\\hline
3.2.34 & que los coraçones de los çibdadanos sean assessegados \textbf{ e que los çibdadanos bi una enparaz } e sean de vn coraçon e de vna uoluntad . & ut corda ciuium sint tranquilla , \textbf{ et ut ciues viuant pacifice et unanimes . } Unde et Philosophus 1 Rhet’ videtur velle , \\\hline
3.2.34 & e avn \textbf{ destose leunata en el regno abondamiento delos bienes tenporales . } Ca quando ay guerra en el regno & Consurgit etiam \textbf{ ex hoc abundantia exteriorum rerum . } Nam existente guerra in regno , \\\hline
3.2.35 & que viene del appetito de dar pena manifiesta \textbf{ por menospreçio aparesçiente de aquellas cosas } que son fechas contra el o contra aquellas cosas & ex appetitu apparentis punitionis , \textbf{ propter apparentem paruipensionem } eorum quae in ipsum , \\\hline
3.2.35 & por que non mueun a los Reyes \textbf{ assana non deuen fazer ninguna cosa mala contra el Rey } nin contra aquellas cosas que son del & ut non prouocent Reges ad iram , \textbf{ non debent fore facere | nec in Regem , } nec in ea quae sunt ipsius , \\\hline
3.2.35 & si ellos non fizieren aquello que deuen fazer \textbf{ o si fizieren cosas contrarias de aquello que deuen fazer . } la qual cosa contesçe mayormente & ut debent , \textbf{ vel si faciant contraria eorum quae debent , } quod maxime contingit , \\\hline
3.2.35 & e non le faziendo \textbf{ obediençia qual deuen e honrra conuenble . } finca de uer & non exhibere ei debitum honorem \textbf{ et obedientiam condignam . } Restat videre , \\\hline
3.2.35 & Et la muger e los fijos e los sus subditos . \textbf{ Et por ende parte nesçe alos moradores del regno sinon } quasieren mouer al Rey a saña & uxor filii , et subditi . \textbf{ Spectat itaque ad habitatores regni } si nolunt Regem \\\hline
3.2.35 & nin contra sus fijos \textbf{ nin contra ningunas perssonas subiectas a el } nin contra ningunos derechs del regno . & nec in filios , \textbf{ nec in aliquas personas ei subiectas , } nec in aliqua iura regni . \\\hline
3.2.36 & ca el pueblo non siente \textbf{ si non los bienes senssibles . } Et por ende ama e ha reuerençia alos bien fechores & Nam vulgus non percipit \textbf{ nisi sensibilia bona , } ideo beneficos , \\\hline
3.2.36 & a quales se quier uenturas \textbf{ por los bienes comunes . } por que cuyda sienpre el pueblo & Nam populus valde diligit fortes et magnanimos , \textbf{ exponentes se pro bonis communibus : } credit enim per tales salutem consequi . \\\hline
3.2.36 & Mas tres cosas son de penssar en la pena que dan los prinçipes . \textbf{ Conuiene a saber la pena misma quedan . } Et la perssona a quien la dan . & In punitione autem tria sunt consideranda \textbf{ videlicet punitionem ipsam , } personam punitam , \\\hline
3.2.36 & que por el amorsolo del bien honesto e del bien comun \textbf{ e por amor del prinçipe ponedor dela ley } cuya entençiones de tener mientes al bien comun & quod solo amore honesti et boni communis , \textbf{ et ex dilectione legislatoris , } cuius est intendere commune bonum , \\\hline
3.2.36 & assi commo es prouado \textbf{ por las cosas sobredichas . } Acabadas dos partes deste terçero libro . & in qua tractatur quomodo regenda ciuitas sit \textbf{ aut regnum tempore belli . } Peractis partibus duabus huius tertii libri , \\\hline
3.3.1 & Ca dicho es \textbf{ que alguno ha sabiduria singular o particular } quando sabe gouernar assi mesmo . & vel particularem prudentiam , \textbf{ quando seipsum scit regere et gubernare : } et haec est minor prudentia , \\\hline
3.3.1 & Conuiene que sea otra e departida de la sabiduria \textbf{ por la qual cada vno sabe gouernar a ssi mismo . } La terçera manera de la sabiduria es dicha regnatiua o positiua de leyes . & oportet esse aliam a prudentia , \textbf{ qua quis nouit seipsum regere . } Tertia species prudentiae dicitur esse regnatiua vel legum positiua . \\\hline
3.3.1 & La terçera manera de la sabiduria es dicha regnatiua o positiua de leyes . \textbf{ Ca assi commo alguna perssona singular es parte de la casa } assi la casa es parte de la çibdat e del regno . & Tertia species prudentiae dicitur esse regnatiua vel legum positiua . \textbf{ Nam sicut persona aliqua singularis est pars domus , } ita domus est pars ciuitatis , et regni : \\\hline
3.3.1 & assi la casa es parte de la çibdat e del regno . \textbf{ Et assi commo el bien de la casa es departido del bien de alguna perssona singular . } assi el bien de la çibdat toda & ita domus est pars ciuitatis , et regni : \textbf{ et sicut bonum commune est aliud | a bono alicuius singularis personae , } sic bonum ciuitatis et regni est aliud a bono domestico . \\\hline
3.3.1 & sobrepuia el bien de la casa \textbf{ e el bien de vna perssona singular . } en tanto la sabiduria & et regni excedit bonum domesticum , \textbf{ et bonum alicuius particularis personae , } tanto prudentia quae requiritur \\\hline
3.3.1 & del gouernamiento de vna casa \textbf{ o la sabiduria de algun omne particular . } por la qual cosa es bien dicho & excedere prudentiam patrisfamilias , \textbf{ vel prudentiam alicuius particularis hominis . } Propter quod bene dictum est \\\hline
3.3.1 & e el prinçipe es vna perssona en ssi \textbf{ e en quanto ha de gouernar a ssi mismo . } Mas en el segundo libro le mostramos ser sabio & prout Rex aut Princeps est quaedam persona in se , \textbf{ et prout habet seipsum regere : } In secundo vero docuimus ipsum \\\hline
3.3.1 & La quinta manera de sabiduria \textbf{ es dicha sabiduria caualleril . } Ca el gouernamiento del regno & nec oeconomica , nec regnatiua , quia non esset ciuitas nec pars ciuitatis . \textbf{ Quinta species prudentiae dicitur esse militaris . } Nam regimen regni , et ciuitatis , \\\hline
3.3.1 & assi que por la ponedora de las leyes toda la çibdat \textbf{ e todo el regno sigua las cosas prouechosas } e fuya de las enpesçibles . & ut per legum positiuam tota ciuitas \textbf{ et totum regnum prosequatur proficua , } et fugiat nociua : \\\hline
3.3.2 & o en quales tierras son meiores lidiadores . \textbf{ Conuiene de tener mientes en estas dos cosas sobredichas . } Et pues que asy es en las partes & in quibus regionibus meliores sunt bellatores , \textbf{ oportet attendere circa praedicta duo . } In partibus igitur nimis propinquis soli , \\\hline
3.3.2 & que de ningunas destas tierras non son de escoger los lidiadores \textbf{ mas de las tierras medianeras son de escoger los lidiadores } que non son muy arredrados & ex neutris partibus eligendi sunt bellatores ; \textbf{ sed ex regione media , } nec omnino a sole remota , \\\hline
3.3.2 & que commo quier que en las batallas tan bien la fortaleza del coraçon \textbf{ commo la sabiduria sean cosas neçessarias . } Empero mas aprouechable es la fortaleza del coraçon & quod licet in bellis tam animositas , \textbf{ quam etiam prudentia sit necessaria , } magis tamen animositas est utilis . \\\hline
3.3.2 & Enpero mas aprouechable es la gente de la tierra medianera \textbf{ Et entre las gentes medianeras mas de escoger son } para las obras de las batallas & non sunt penitus utiles actibus bellicis : \textbf{ magis tamen inter medias regiones eligendi sunt } ad opera bellica remotiores a sole , \\\hline
3.3.2 & Por ende los que non temen los periglos de los puercos monteses \textbf{ e de las otras bestias fuertes . } señal es que non & quam pugnare cum hoste . \textbf{ Nam non timentes aprorum pericula , } signum est eos non timere hostium bella . \\\hline
3.3.2 & nin esgrimira la espada \textbf{ aquel que deue auer la mano liuiana . } Et non es acostubrado de tener en la mano & Nam nunquam bene vibrant clauam , \textbf{ aut ensem qui debet habere manum leuem , } et non est assuetus retinere \\\hline
3.3.2 & para las obras de la batalla \textbf{ por que non han la arte semeiable a las obras de la batalla . } Enpero puede contesçer & non sunt eligendi ad huiusmodi opera : \textbf{ eo quod non habeant artem conformem operibus bellicosis . } Potest ergo contingere \\\hline
3.3.3 & e nos delectamos en ellas . \textbf{ Et si quisiere el ponedor de la ley fazer los çibdadanos buenos lidiadores } e fazer los apareiados para la batalla & nimis diligimus et delectamur in illis : \textbf{ si vult legislator ciues bellatores facere , } et reddere ipsos aptos ad pugnandum , \\\hline
3.3.3 & que los temerosos e de flacos coraçones . \textbf{ Otrossi los omnes fuertes de cuerpo e duros de mienbros } por que son mas poderosos en las uirtudes corporales & quam timidos . \textbf{ Rursus , homines fortes et duros corpore , } quia potentiores sunt viribus , \\\hline
3.3.3 & Lo terçero por aquellas señales \textbf{ segunt las qualos semeiamos a las animalias batalladoras . } Ca son algunas señales & Tertio autem per signa , \textbf{ secundum quae conformamur animalibus bellicosis . } Sunt autem signa , \\\hline
3.3.3 & e la sabiduria del alma \textbf{ en todo han maneras contrarias . } Ca assi conmo el dize & et industria mentis , \textbf{ omnino requirunt modum contrarium . } Nam ut scribitur in 2 de Anima , \\\hline
3.3.3 & Et por el contrario los que han las carnes duras \textbf{ e los neruios espessos e firmes } e los laçertos fuertes e rezios & duri carne habentes compactos neruos , \textbf{ et lacertos , } sunt virosi et fortiores corpore , \\\hline
3.3.3 & e los neruios espessos e firmes \textbf{ e los laçertos fuertes e rezios } estos tales son fuertes de cuerpo & duri carne habentes compactos neruos , \textbf{ et lacertos , } sunt virosi et fortiores corpore , \\\hline
3.3.3 & e ha las carnes duras \textbf{ e ha los neruios espessos e firmes } e los muslos duros & durus in carne , \textbf{ compactus in neruis et musculis , } habens longa brachia , \\\hline
3.3.3 & e los muslos duros \textbf{ e ha los braços luengos e los pechos anchos . } estos tales deuemos iudgar & compactus in neruis et musculis , \textbf{ habens longa brachia , | et latum pectus , } debemus arguere \\\hline
3.3.4 & Lo quinto \textbf{ que por razon de la iustiçia e del bien comun despreçien la uida corporal . } Lo sexto que non teman & Quarto non curare de incommoditate iacendi et standi . \textbf{ Quinto quasi non appretiare corporalem vitam . } Sexto non horrere sanguinis effusionem . \\\hline
3.3.4 & Mas entre todas las cosas \textbf{ que fazen al omne buen lidiador } es dessear & Inter caetera autem \textbf{ quae reddunt hominem bellicosum , } est diligere honorari expugna , \\\hline
3.3.4 & esto se entiende \textbf{ si ouieren batalla derecha . } Ca por defendimiento de la iustiçia & intelligendum est , \textbf{ si habeat iustum bellum . } Nam pro defensione iustitiae \\\hline
3.3.4 & e por defendimiento del bien comun \textbf{ es de poner la vida corporal a periglo de muerte } e non deue foyr & Nam pro defensione iustitiae \textbf{ et pro communi bono exponenda est periculo corporalis vita , } non est cauenda effusio sanguinis , \\\hline
3.3.5 & Creo que ninguno nunca pudo dubdar \textbf{ que los omnes rusticos e aldeanos non fuessen meiores para las armas } que los que son delicadamente criados . & Numquam credo potuisse dubitari \textbf{ aptiorem armis esse rusticam plebem . } Ad hoc etiam videntur facere \\\hline
3.3.5 & e las otras cosas \textbf{ que contamos en el capitulo sobredicho . } Et çierto es que el pueblo aldeano ha mas prinçipalmente estas cosas sobredichas & et cetera alia \textbf{ quae tetigimus in capitulo praecedenti . } Constat autem \\\hline
3.3.5 & que contamos en el capitulo sobredicho . \textbf{ Et çierto es que el pueblo aldeano ha mas prinçipalmente estas cosas sobredichas } que los nobles & quae tetigimus in capitulo praecedenti . \textbf{ Constat autem | ruralem populum habere maxime praedicta . } Sunt enim rustici assueti \\\hline
3.3.5 & a los quales abasta el agua en la sed \textbf{ e el pan grueso en comer . } Otrossi los aldeanos non se agranian en mal yazer nin en mal estar & quibus potus aquae satisfaciebat in siti , \textbf{ et grossus panis sufficiebat ad esum . } Amplius , rurales non affliguntur \\\hline
3.3.6 & Et non es cosa desconuenible \textbf{ que vn omme sabidor e entendido en vna cosa . } por non auer prueua de las cosas particulares & et magis habuissent bellandi industriam . \textbf{ Non est enim inconueniens virum prudentem et sagacem in uno , } propter particularium inexperientiam \\\hline
3.3.6 & Onde muchas vezes \textbf{ contesçe que los omnes sabios en muchas otras cosas } por non auer vso de las armas & Unde multotiens contingit , \textbf{ quod prudentes in rebus aliis } propter inexercitium armorum \\\hline
3.3.6 & assi commo si ouiessen de acometer la batalla . \textbf{ Et quando vieren los caudiellos maestros de las batallas } que alguno non guarda orden en la az & ac si deberent pugnam committere . \textbf{ Et cum viderit magister bellorum } aliquem non tenere ordinem debitum in acie , \\\hline
3.3.6 & que sin salto non lo pueden saltar nin passar \textbf{ por la qual cosa prouechosa cosa es el saltar } para tirar estos enbargos . & quae sine saltu in via transire non possunt : \textbf{ quare utile est ad remouenda impedimenta , } ut equites sic sint docti , \\\hline
3.3.7 & que las madres nunca les querien dar de comer \textbf{ fasta que ferien con la fonda en logar çierto . } Et este uso es muy prouechoso & ut matres nullum cibum eis exhiberent , \textbf{ quem non primo cum funda percuterent . } Est enim hoc exercitium utile , \\\hline
3.3.7 & Ca algunas uezes contesçe \textbf{ que en logares pedregosos es la batalla } e por ende por que el logar se defienda & Interdum tamen euenit , \textbf{ ut in lapidosis locis habeatur conflictus , } et ut mons si taliquis defendendus . \\\hline
3.3.7 & Enpero conuiene de tener mientes \textbf{ que algunos destos usos sobredichos pertenesçen mas propriamente a los caualleros } e algunos a los peones & quod praedictorum exercitiorum \textbf{ quaedam sunt magis propria equitibus , quaedam peditibus , } quaedam utrisque . \\\hline
3.3.7 & e algunos a todos . \textbf{ Et esto en qual manera sea non ha menester grant estudio . } Ca non se puede asconder a ome sabio . & quaedam utrisque . \textbf{ Quod quomodo sit , | non magna consideratione eget , } et solertem mentem latere non potest . \\\hline
3.3.7 & Et esto en qual manera sea non ha menester grant estudio . \textbf{ Ca non se puede asconder a ome sabio . } Ca sobir en los cauallos pertenesçe a los caualleros . & non magna consideratione eget , \textbf{ et solertem mentem latere non potest . } Nam ascendere equos , \\\hline
3.3.8 & que lieuan consigo assi commo vna çibdat guarnida . \textbf{ Visto commo es cosa prouechable a la hueste fazer carcauas e costruir guarniçiones e castiellos . } finca de demostrar en qual manera las tales guarniciones & Viso utile esse \textbf{ circa exercitum facere fossas | et construere castra : } restat ostendere , \\\hline
3.3.8 & Mostrado que prouechosa cosa es de fazer los castiellos . \textbf{ avn en qual manera los enemigos presentes son de fazer los castiellos } Lo otro que nos finca de declarar & Ostenso utile esse castra construere , \textbf{ et qualiter etiam praesentibus hostibus construenda sint castra : } reliquum est declarare \\\hline
3.3.8 & deuemos escoger çerca de aqual \textbf{ assentamiento que sea el ayre sano . } Ca en la hueste non tan solamente deuemos escusar las feridas de los enemigos . & et adsit possibilitas est eligenda \textbf{ circa situm salubritas aeris . } Nam in exercitu non solum cauenda sunt vulnera hostium , \\\hline
3.3.8 & deuen ser quadradas e luengas . \textbf{ Enpero por que la forma redonda conprehende } mas que las otras & habere formam quadrilateram oblongam . \textbf{ Attamen quia figura circularis est capacissima , } est elegibilius facere munitiones \\\hline
3.3.8 & e el logar do estan assentados \textbf{ Mas la puerta principal deue ser fechan de aquella parte } que cata a los enemigos & vel aliquam formam aliam quam requirit dispositio et aptitudo situs . \textbf{ Porta autem principalis | ex illa parte fienda est , } quae respicit hostes , \\\hline
3.3.9 & que se exponer \textbf{ sin prouision conuenible a auentura e a acaesçimiento . } Ca nos veemos & melius est enim pugnam non committere , \textbf{ quam absque debita praeuisione fortunae | et casui se exponere . } Videmus autem in bello duo existere , \\\hline
3.3.9 & en que es de acometer la batalla \textbf{ es el sol contrario a las caras de los enemigos } e si se leuanta algun viento & utrum tempore quo committenda est pugna , \textbf{ sol sit oppositus faciebus eorum , vel hostium ; } et utrum sit aliquis ventus flans \\\hline
3.3.9 & Et pues que assi es todas estas cosas vistas con grant sabiduria \textbf{ el cabdello sabio de la hueste } conplidamente puede entender sil & His itaque igitur omnibus diligenter inspectis , \textbf{ prudens dux exercitus sufficienter aduertere potest , } utrum debeat publicam pugnam committere . \\\hline
3.3.10 & por \textbf{ letraso por algunas señales manifiestas se mostrasse manifiestamente } de qual az o de qual conpaña era aquel pendon o aquella seña & ita ut in quolibet vexillo per literas , \textbf{ vel aliqua euidentia signa aperte | ostenderetur cuius aciei , } vel cuius turmae esset vexillum illud : \\\hline
3.3.10 & eran escerptas letras \textbf{ algunas o alguna señal manifiesta . } a la qual catando los deanes conosçian al su senora propreo & In galea enim centurionis scriptae erant literae aliquae , \textbf{ vel signum aliquod euidens ; } quod respicientes decani \\\hline
3.3.10 & deue ser escogido el alferes . \textbf{ de las cosas sobredichas puede paresçer } qual deua ser el cabdiello & cum magna diligentia vexillifer est quaerendus . \textbf{ Ex dictis etiam patere potest , } qualis esse debeat \\\hline
3.3.10 & mas deue ser prouado \textbf{ en las armas ligero de cuerpo } e fuerte en los mienbros & multo magis debet \textbf{ esse armorum expertus , | et procer corpore , } et fortis viribus \\\hline
3.3.11 & menos son enbargadas \textbf{ mas ayna son aduchas a su fin conuenible . } Et pues que assi es despues que es librado por el conseio & tanto quae sunt in consiliis deliberata minus impediuntur , \textbf{ et citius fini debito mancipantur . } Postquam igitur deliberatum est \\\hline
3.3.11 & e aquellas carreras touiere el cabdiello escriptas e pintadas \textbf{ e ouiere algunos omes guiadores fieles } quanto esto menos fuere publico & conscriptas et depictas , \textbf{ et habentur conductores aliqui fideles , } quanto hoc minus est publicum \\\hline
3.3.11 & de quales ha mayor conplimiento la hueste de peones o de caualleros . \textbf{ ca los caualleros meior se defienden en los canpos . } mas los peones meior en los riscos e en los montes . & utrum magis abundet peditibus , vel equitibus . \textbf{ Nam equites melius se defendunt in campis . } Pedites vero in locis syluestribus et montuosis . \\\hline
3.3.11 & ca los caualleros meior se defienden en los canpos . \textbf{ mas los peones meior en los riscos e en los montes . } Et por ende assi commo viere el senor de la hueste & Nam equites melius se defendunt in campis . \textbf{ Pedites vero in locis syluestribus et montuosis . } Itaque prout viderit dux belli se abundare in equitibus , \\\hline
3.3.11 & podra escoger los caminos de los canpos \textbf{ e carreras anchas o las de los montes } o de las siluas & vel in peditibus , \textbf{ eligere poterit vias campestres et amplas , } vel montanas , syluestres , et nemorosas , \\\hline
3.3.12 & para que mas ligeramente los venzcan . \textbf{ mas guardar orden conuenible en la az } e que los caualleros e los peones guarden su az & ut facilius deuincantur . \textbf{ Seruare autem debitum ordinem in acie } ut equites et pedites suam aciem seruent , \\\hline
3.3.12 & por luengos tienpos acostunbrar los lidiadores \textbf{ a guardar orden conuenible en la az } e a fazer aquellas cosas & per diuturna tempora debet exercitare pugnatores \textbf{ ad seruandum debitum ordinem , } et ad faciendum ea quae requiruntur in bello . Modus autem , \\\hline
3.3.12 & Et despues que se ordenen \textbf{ segunt forma redonda e almogotes . } Et assi de las otras maneras deuen costunbrar los lidiadores & et postea secundum triangularem , \textbf{ et deinde secundum rotundam : } et sic deinceps debet \\\hline
3.3.12 & Estonçe es de establesçer el az \textbf{ segunt forma redonda . } Et los lidiadores deuen se costreñir & tunc est construenda acies \textbf{ secundum rotundam formam ; } et pugnantes debent se magis constringere et constipare , \\\hline
3.3.12 & Et pues que ssi es el az \textbf{ establesçida en forma redonda es prouechosa para sofrir colpes . } Mas en forma de tigeras es prouechosa para çercar & Nam diuisis hostibus facilius debellantur . \textbf{ Acies ergo constructa in forma rotunda , } utilis est ad sustinendum . \\\hline
3.3.12 & quando son pocos . \textbf{ Mas la forma aguda en manera de pera ens prouechosa para fender } e departir los enemigos & est utilis ad circum dandum et concludendum , \textbf{ cum hostes sunt pauci . Sed in forma acuta et pyramidali , | utilis est ad scindendum et diuidendum , } cum hostes sunt plures . \\\hline
3.3.13 & m mostrado en qual manera son de establesçer \textbf{ e de ordenar las azes fincanos de mostrar en qual manera los lidiadores deuen ferir } e si es meior de ferir cortando o ferir de punta o estocando . & Ostenso qualiter sunt acies ordinandae et construendae , \textbf{ reliquum est ostendere , | qualiter pugnantes percutere debeant , } utrum eligibilius est percutere caesim vel punctim . \\\hline
3.3.13 & que los que son prouados en las batallas . \textbf{ dizen que los lidiadores sienpre deuen auer las lorigas anchas . } assi que les aniellos de las lorigas se ayunten & Inde est quod bellorum experti dicunt pugnantes \textbf{ semper debere habere loricas amplas ita , } ut annuli loricarum se constringant : \\\hline
3.3.13 & e de cortar muchos huessos . \textbf{ Mas feriendo de punta pequeno colpe mata al omen . } ca dos onças de sangre abastan & et multa ossa incidere : \textbf{ sed percutiendo punctim } duae unciae sufficiunt ad hoc \\\hline
3.3.13 & por que feriendo \textbf{ assi mas ayna se faze llaga mortal . } la terçera razon se toma de la & percutiendum est punctim , \textbf{ quia sic feriendo citius infligitur plaga mortifera . } Tertia via sumitur \\\hline
3.3.13 & Mas la ferida de punta fecha con muy pequeña \textbf{ fuerca faze llaga mortal . } La quinta razon se toma del descrubimiento del que fiere . & Sed puncta modico impetu inflicta , \textbf{ facit lethale vulnus . } Quinta via sumitur \\\hline
3.3.15 & mouiminento en todas las animalias \textbf{ prinçipalmente enbia su uirtud a la parte derecha . } Assi que la parte derecha en las animalias & Nam cor , quod est in animali principium motus , \textbf{ principalius influit in partem dextram ita , } quod pars dextra \\\hline
3.3.15 & prinçipalmente enbia su uirtud a la parte derecha . \textbf{ Assi que la parte derecha en las animalias } es mas fuerte en mouer & principalius influit in partem dextram ita , \textbf{ quod pars dextra } in animalibus fortior est in mouendo , \\\hline
3.3.15 & Et por ende quando el pie esquierdo se pone delante \textbf{ e el costado derecho se aluenga estonçe esta el omne bien apareiado } para lançar dardos o piedras o otras cosas & Cum igitur pes sinister anteponitur , \textbf{ et latus dextrum elongatur , | optime disponitur homo } ad iaciendum iacula et missilia : \\\hline
3.3.15 & Enpero maguer que podamos folgar tan bien sobre la parte derecha \textbf{ commmo sobre la esquierda . Empero la parte derecha sienpre es mas apareiada para mouimiento } e la esquierda para folgura . & tam secundum partem dextram quam \textbf{ secundum sinistram possumus quiescere et moueri , | dextra tamen est aptior ad mouendum , } et sinistra ad quiescendum . \\\hline
3.3.15 & de lo que dicho es \textbf{ assi que el pie derecho tengamos delante } e el esquirdo detras . & debemus e contrario nos habere , \textbf{ ita quod pedem dextrum teneamus ante , } et sinistrum post . \\\hline
3.3.15 & e el esquirdo detras . \textbf{ Ca por que el costado derecho es mas conuenible al mouimiento } si el estudiere mas cerca del enemigo por razon del mouimiento podra foyr meior los colpes . & et sinistrum post . \textbf{ Nam quia latus dextrum aptius est ad motum , } si illud sit hosti propinquius , \\\hline
3.3.15 & aquello que esta que lo que se mueue . \textbf{ Otrossi el costado derecho fuere mas çerca del enemigo meior le podra ferir . } Ca deuen los lidiadores & quam quod mouetur . \textbf{ Rursus , si dextrum latus sit hosti propinquius , | melius poterit ipsum percutere . } Debent enim bellatores , \\\hline
3.3.15 & e quando dieren los colpes \textbf{ deuense arredrar con el pie derecho . } Et pues que assi es temiendo & et cum volunt ictus fugere , \textbf{ cum eodem pede debent se retrahere ; } sic itaque tenendo pedem sinistrum immobilem , \\\hline
3.3.15 & Et pues que assi es temiendo \textbf{ assi el pie esquierto firme } e mouiendo se con el esquierdo pie & cum eodem pede debent se retrahere ; \textbf{ sic itaque tenendo pedem sinistrum immobilem , } et cum dextro se mouendo , \\\hline
3.3.15 & deue auer el cabdiello de la batalla . \textbf{ La primera es de parte de la su hueste propria . } Ca si el cabdiello ouiere conseio de non lidiar . & debet dux belli duplicem cautelam habere . \textbf{ Prima est , | quantum ad exercitum proprium . } Nam et si dux consilium habeat \\\hline
3.3.16 & Conuiene de saber . \textbf{ Batalla canpal . } Batalla de çerca . & Videntur omnia bella \textbf{ ad quatuor genera reduci , } videlicet ad campestre , obsessiuum , defensiuum , et nauale . \\\hline
3.3.16 & Et batalla de naues \textbf{ Mas batalla canpal es dicha toda lid fecha en la tierra } segunt la qual los lidiadores lidian vnos & videlicet ad campestre , obsessiuum , defensiuum , et nauale . \textbf{ Bellum autem campestre dicitur omnis pugna facta in terra , } secundum quam bellantes ad inuicem pugnant \\\hline
3.3.16 & segunt la qual los lidiadores lidian vnos \textbf{ contra otros sin muros o sin guarniçion medianera . } Enpero quanto esta lid es mas arredrada de guarniçiones o de muros & secundum quam bellantes ad inuicem pugnant \textbf{ absque munitione media . | Quanto tamen huiusmodi pugna } magis est \\\hline
3.3.16 & Et tal manera de batalla en la qual defiende cada vno su villa o su castiello \textbf{ es dicha batalla defenssiua . } Ca si en toda batalla es algun acometemiento en alguna manera . & quo quis defendit munitiones et castra , \textbf{ dicitur defensiuum . } Nam etsi in omni pugna est aliquo modo inuasio et defensio : \\\hline
3.3.16 & mas es ay defendemiento que acometemiento . \textbf{ Et por ende tal manera de batalla de con coraçon es dicha defenssiua . } La quarta manera de lidiar & quam inuasio . \textbf{ ideo tale genus pugnae merito dicitur defensiuum . } Quartus autem pugnandi modus dicitur naualis : \\\hline
3.3.16 & que sean aquellas aguas \textbf{ son dichas batallas nauales e de naues . } Por la qual cosa commo sean quatro maneras da batallas & cuiuscunque conditionis aquae illae existant , \textbf{ nauales dicuntur . } Quare cum sint quatuor genera pugnarum , \\\hline
3.3.16 & Por la qual cosa commo sean quatro maneras da batallas \textbf{ despues que dixiemos de la batalla de la tierra fincanos de dezir de las otras tres . } Conuieue de saber de la de çerca & Quare cum sint quatuor genera pugnarum , \textbf{ postquam diximus de campestri , | restat dicere de obsessiua , defensiua , et nauali . } Contingit enim aliquando Reges , \\\hline
3.3.16 & Ca contesçe algunas uezes que los Reyes e los prinçipes lidian en todas estas maneras de lidiar . \textbf{ Ca algunas uezes acometen batalla canpal e en el canpo . } Et algunas vezes çerca villas o castiellos o fortalezas . & omnibus his modis pugnandi . \textbf{ Nam aliquando committunt campestre bellum . } Aliquando vero obsident munitiones et castra . \\\hline
3.3.16 & Et algunas vezes çerca villas o castiellos o fortalezas . \textbf{ Et avn algunas vezes contesçe que algunos otros çercan sus villas o sus castiellos . } Por la qual cosa les conuiene de vsar de batalla defenssiua para se defender . & Aliquando vero obsident munitiones et castra . \textbf{ Contingit etiam aliquando aliquos | inuadere aliquas munitiones eorum ; } propter quod eos oportet \\\hline
3.3.16 & Et avn algunas vezes contesçe que algunos otros çercan sus villas o sus castiellos . \textbf{ Por la qual cosa les conuiene de vsar de batalla defenssiua para se defender . } Otrossi contesçe que en el prinçipado & inuadere aliquas munitiones eorum ; \textbf{ propter quod eos oportet | uti pugna defensiua . } Amplius in principatu et regno contingit \\\hline
3.3.16 & algunas vezes de ordenar \textbf{ e de fazer batallas nauales e de naues . } Et pues que assi es dicho de la lid canpal & et ne terrae marinae impugnentur , \textbf{ expedit regibus et principibus aliquando ordinare bella naualia . } Dicto itaque de bello campestri , \\\hline
3.3.16 & Et por ende uisto quantas son las maneras de las batallas . \textbf{ Et dicho que despues de la lid canpal del canpo primero } es de dezir de la lid & Viso ergo quot sunt bellorum genera , \textbf{ et dicto quod post castrum campestre primo dicendum est } de pugna obsessiua : \\\hline
3.3.16 & Ca muchas uegadas contesçe \textbf{ que el agua viene de lueñe fasta las fortalezas cercadas . } Et por ende si en el comienço de la fuente se destruyere el canno o la carrera & ab obsessis accipere aquam . Nam multotiens euenit , \textbf{ aquam a remoto principio deriuari | usque ad munitiones obsessas : } quare si in illo fontali principio destruatur \\\hline
3.3.17 & los que estan cercados \textbf{ et ay vna manera comun e publica de acometer e de lidiar } contra los que estan cercados & quot modis impugnare debent obsessos . \textbf{ Est autem unus modus impugnandi communis et publicus , } videlicet , per ballistas , arcus , \\\hline
3.3.17 & e para entrar los . \textbf{ Enpero sin estas maneras manifiestas de batalla } ay otras tres maneras & ut si possint ascendere ad partes illas . \textbf{ Praeter tamen hos modos impugnationis apertos , } est dare triplicem impugnationis modum non omnibus notum . \\\hline
3.3.17 & de las quales la vna es \textbf{ por cueuas conegeras . } Et la otra es por algarradas e por engeñios & est dare triplicem impugnationis modum non omnibus notum . \textbf{ Quorum unus est per cuniculos . } Alius est per machinas proiicientes lapides magnos et graues . \\\hline
3.3.17 & que es \textbf{ por las cueuas conegeras . } Et pues que assi es lo primero & sed primo de impugnatione per cuniculos . \textbf{ Primo igitur per cuniculos , } id est per vias subterraneas \\\hline
3.3.17 & assi fechas en algun tienpo de noche \textbf{ o en otro tienpo conuenible a esta manera de conbatir } deue ser puesto el fuego & in aliquo nocturno tempore , \textbf{ vel in aliquo alio congruo | ad pugnandum } per appositionem ignis fieri debet , \\\hline
3.3.18 & e assi podran ganar aquellas fortalezas . \textbf{ m muchas uegadas contesçe que algunas fortalezas çercadas son fundadas sobre pennas muy fuertes } o son cercadas de agua & Contingit autem pluries , \textbf{ munitiones aliquas obsessas | super lapides fortissimos esse constructas , } vel esse aquis circumdatas , \\\hline
3.3.18 & assi que por cueuas conegeras \textbf{ o por cueuas soterranas nunca o con muy grand trabaio se pueden tomar . } Et avn acaesçe muchas vezes & ut per viculos \textbf{ et per subterraneas vias nunquam , | vel valde de difficili obtineri possint . } Euenit etiam pluries \\\hline
3.3.18 & que si la fortaleza cercada se puede tomar \textbf{ por cueuas soterrañas . } enpero los çercados veyendo que los pueden entrar & ut si munitio obsessa \textbf{ per vias subterraneas capi possit , } obsessi tamen prouidentes fossionem impediunt eam , \\\hline
3.3.18 & e bueluesse e tornasse çerca del pertegal . \textbf{ Et esta manera de engeñio llaman los lidiadores romanos bifan } Mas este engeñio ha departimiento del trabuquete . & vertens se circa huiusmodi virgam . \textbf{ Et hoc genus machinae Romani pugnatores | appellauerunt Biffam . } Differt autem haec a Trabutio . \\\hline
3.3.18 & e este engeñio tal non lança las piedras tan grandes \textbf{ commo los tres engeñios sobredichos . } Enpero non es menester tanto tienpo & Huiusmodi enim machina non proiicit lapides ita magnos , \textbf{ sicut praedicta tria genera machinarum : } tamen non oportet tantum tempus apponere \\\hline
3.3.18 & si puede meior conbatir aquella fortaleza lançando derechamente \textbf{ o lançando mas alueñe o en manera medianera entre estas dos } o deue avn penssar & munitionem illam impugnare proiiciendo rectius vel longius , \textbf{ vel medio modo inter utrunque vel etiam magis posset obsessos offendere } proiiciendo spissius et frequentius . \\\hline
3.3.18 & cercandos lançando amenudo e muchas vezes . \textbf{ Ca assi commo viere que mas conuiene en todas aquellas maneras sobredichas de engeñios . } o en todas aquellas maneras de lançar & proiiciendo spissius et frequentius . \textbf{ Nam prout viderit | expedire omnibus praedictis machinis , } vel omnibus praefatis modis proiiciendi , \\\hline
3.3.19 & la otra era \textbf{ por engeñios lançadores de piedras . } la terçera era por artifiçios de madera enpuxados & Tertius vero , \textbf{ per aedificia lignea impulsa } ad muros munitionis obsessae . \\\hline
3.3.19 & por las cueuas conegeras \textbf{ o por los engeñios lançadores de piedras } fincanos de dezir del acometimiento & Dicto ergo de impugnatione facta per cuniculos , \textbf{ et per lapidarias machinas ; } restat dicere de impugnatione \\\hline
3.3.19 & e fazese este artificio \textbf{ quando las tablas gruessas e fuertes son bien iuntadas e dobladas } o se fazen dos tablados & Fit autem hoc , \textbf{ cum tabulae grossae | et fortes optime conligantur , } et duplicantur , \\\hline
3.3.19 & Ca si las fortalezas cercadas non se pueden tomar par los carneros \textbf{ nin por las viñas sobredichas deuen tomarla mesura } e el alteza de los muros de aquella fortaleza & Nam si nec per arietes , \textbf{ nec per vineas capi possunt munitiones obsessae , } accipienda est mensura murorum munitionis illius , \\\hline
3.3.19 & Lo primero por sonbra \textbf{ ca vn filo liuiano de çierta medida de palmos o de pies es de atar a la saeta } e deuen lançar la saeta & Primo per umbram . \textbf{ Nam leue filum , | cuius nota sit quantitas , } ligandum est ad sagittam , \\\hline
3.3.19 & e los cadahalsos et las torrezilas de la fortaleza que quieren tomar . \textbf{ Et la parte medianera a la qual quieren echar la puente sobre los muros . } Et la parte mas postrimera & et curriculas munitionis capiendae partem quasi mediam , \textbf{ ad quam applicantur pontes cadendi super illos muros : } et partem infimam , \\\hline
3.3.20 & Et por ende son de fazer los muros de las fortalezas a esquinas \textbf{ por que se pueda la fortaleza meior defender . } Lo tercero que faze la fortaleza mas fuerte & Fiendi itaque sunt muri angulares , \textbf{ ut munitio faciliter defendi possit . } Tertium , \\\hline
3.3.20 & para se non poder entrar \textbf{ son terrados o torres albarranas o muros çiegos fechos de tierra . } Ca en la fortaleza que es de fazer & quod reddit munitionem difficiliorem ad capiendum , \textbf{ dicuntur esse terrata , | vel muri ex terra facti . } Nam in munitione fienda \\\hline
3.3.20 & assi commo vn muro . \textbf{ Ca pueden se fazer torres albarranas de la tierra } que sean bien feridas e bien tapiadas & et efficiatur quasi murus . \textbf{ Contingit etiam turres ex terra facere , } si bene condensetur : \\\hline
3.3.20 & Por la qual cosa mucho cunple fazer tales muros \textbf{ e tales torres albarranas de tierra muy tapiada . } Et esto uale para defender la fortaleza & construere huiusmodi muros \textbf{ ex terra depressata ; | valet quidem constitutio talium murorum } ad defendendam munitionem , \\\hline
3.3.20 & si sse podiere fazer Et \textbf{ pues que assi es en estas maneras sobredichas son las fortalezas mas fuertes e peores de tomar . } Et por ende deuen catar en el comienço & quae ( si adsit facultas ) \textbf{ replendae sunt aquis . | His ergo modis sunt munitiones difficiliores ad capiendum . } Ideo videndum est a principio ab his \\\hline
3.3.21 & non se puedan aprouechar ellas \textbf{ nin puedan de los bienes proprios de la fortaleza cercada aprouecharse para la conbatir . } Et si temen de ser çercados & ne obsidentes superuenientes inde capiant emolumentum , \textbf{ et ex bonis propriis munitionis obsessae inpugnent ipsam . } Si autem timeatur de diuturnitate temporis , \\\hline
3.3.21 & por las conpañas \textbf{ e por sabios despensseros . } Onde si pudiere ser & sed etiam ut victualia delata \textbf{ per temperatos erogatores per familias dispensentur . } Unde ( si fieri posset ) \\\hline
3.3.21 & en cada uarrio de la çibdat \textbf{ deuen ser puestas las viandas en orrios e en alholis publicos . } Et deuen ser dispenssadas e partidas tenpradamente e escassamente & in qualibet contrata ciuitatis victualia \textbf{ reduci debent | ad horrea publica , } et parte , et temperate per viros prouidos dispensare . \\\hline
3.3.21 & Et deuen ser dispenssadas e partidas tenpradamente e escassamente \textbf{ por muy sabios despensseros . } Et si la fortaleza çercada es pequena & ad horrea publica , \textbf{ et parte , et temperate per viros prouidos dispensare . } quod si munitio obsessa \\\hline
3.3.21 & si non agua salada \textbf{ por que el agua dulçe ba muy lueñe } e non la pueden tomar & habere aquam nisi santam , \textbf{ eo quod dulcem aquam habeat distantem , } ad quam capiendam prohibent obsidentes : \\\hline
3.3.21 & en el libro de los 
                     Metaurores 
                   toda agua salada \textbf{ que passa por los foradillos menudos de la çera toda se torna dulçe . } Et avn conuiene de traer & Deferendum est etiam ad munitionem obsidendem \textbf{ in magna copia acerum , } et vinum , \\\hline
3.3.21 & que teme ser cercada \textbf{ por que beuiendo agua sola los lidiadores enflaquesçerse yan en tanto que non podrian defenderse de los enemigos . } Mostrado quales remedios se deuen tomar & ne ex potu solius equae bellatores adeo debilitentur , \textbf{ quod non possint viriliter resistere obsidentibus . } Ostenso quomodo sunt remedia adhibenda contra famem , \\\hline
3.3.22 & de acometer las fortalezas çercadas de las quales . \textbf{ La vna era por cueuas conegeras o por carreras soterrañas . } Et la otra era por engeñios lançaderos de piedras . & Enumerabantur supra tres speciales modi impugnandi munitiones obsessas . \textbf{ Quorum unus erat per cuniculos et vias subterraneas . } Alius per machinas lapidarias . \\\hline
3.3.22 & La vna era por cueuas conegeras o por carreras soterrañas . \textbf{ Et la otra era por engeñios lançaderos de piedras . } E la terçera manera por artifiçios . & Quorum unus erat per cuniculos et vias subterraneas . \textbf{ Alius per machinas lapidarias . } Et tertius per aedificia impulsa \\\hline
3.3.22 & que se pueden poner \textbf{ contra aquella manera de acometer que es por cueuas conegeras o por carreras soterrañas . } Et podemos contra esto poner dos remedios . & Primo ergo dicemus de remediis \textbf{ contra impugnationem per cuniculos . } Possumus autem circa haec , \\\hline
3.3.22 & quiestan cercados \textbf{ por carteras soterrañas . } Et puesto avn que las carcauas non se pueda finchir de agua & ne obsessos impugnare possint per cuniculos , \textbf{ et vias subterraneas . } Dato tamen quod fossae aquis repleri non possint , \\\hline
3.3.22 & non se pueden conbatir las fortalezas cercadas \textbf{ si las dichas carreras soterrañas non fuessen muy mas fondas que las carcauas . } Pues que assi es la fortaleza que se ha de & non possunt munitiones obsessae , \textbf{ nisi dictae viae subterraneae profundiores sint fossis . } Munitio ergo defendenda \\\hline
3.3.22 & Pues que assi es la fortaleza que se ha de \textbf{ defendero esta assentada sobre peña firme . } Et estonçe non se puede acometer & Munitio ergo defendenda \textbf{ vel est supra petram firmam : } et tunc propter duritiem lapidum \\\hline
3.3.22 & por que non puedan passar \textbf{ por las cueuas conegeras a acometer la fortaleza . } El segundo remedio contra las cueuas conegeras & ne per cuniculos deuincatur . \textbf{ Secundum remedium contra cuniculos } et vias subterraneas est , \\\hline
3.3.22 & El segundo remedio contra las cueuas conegeras \textbf{ e contra las carreras soterrañas es de fazer vna fortaleza cercada otra carrera } que responda a la carrera soterraña & Secundum remedium contra cuniculos \textbf{ et vias subterraneas est , | facere in munitione obsessa viam aliam } correspondentem viae subterraneae factae ab obsidentibus . \\\hline
3.3.22 & por la qual cosa temen del cometemiento \textbf{ por las cueuas conegeras deuen penssar con grand acuçia los cercados } si pudieren veer que llegan la tierra de alguna parte & propter quod timetur de impugnatione per cuniculos ; \textbf{ diligenter considerare debent obsessi , } utrum ab aliqua parte videant terram deferri , \\\hline
3.3.22 & o si por algunas señales pudieren conosçer \textbf{ que los que cercan comiençan a fazer cueuas coneieras . } Et quando esto entendieren & et utrum per aliqua signa cognoscere possint \textbf{ obsidentes inchoare cuniculos : } quod cum perceperint , \\\hline
3.3.22 & assy ayuntada deuen la echar \textbf{ sobre los que cercan que estan las cueuas coneieras . } Ca muchos de aquellos & effundere debent \textbf{ supra obsidentes existentes in cuniculis . } Temporibus enim nostris multi obsidentium sic periclitati sunt : \\\hline
3.3.22 & estopa \textbf{ llamaronle los lidiadores antigos ençendemiento . } Et esta saeta enuiada & ex oleo , sulphure , et pice , et resina : \textbf{ quem ignem cum stupa conuolutum bellatores antiqui Incendiarium vocauerunt . } Huiusmodi autem sagitta \\\hline
3.3.22 & mas por que estas cosas tales son muchas \textbf{ e non las puede omne conplidamente contar dexamoslas a iuyzio de omnes sabios . } Mostrado en qual manera nos podemos defender & sed quia talia complete \textbf{ sub narratione non cadunt , | prudentis iudicio relinquantur . } Ostenso quomodo resistendum sit cuniculis , et lapidariis machinis : \\\hline
3.3.22 & con el qual prenden a la cabeça del carnero o a la cabesça de la viga ferrada . \textbf{ la qual cabesça presa o en todo en todo } leuaran el & vel caput illius trabis ferratae : \textbf{ quo capto , } vel omnino Aries ad superiora trahitur , \\\hline
3.3.22 & Enpero contra esto daremos espeçial remedio . \textbf{ ca pueden se fazer cueuas coneieras de dentro e carreras soterrañas } e ascondidamente se puede cauar la tierra & adhibetur tamen speciale remedium contra ipsa , \textbf{ quia fiunt cuniculi , | et viae subterraneae , } et clam suffoditur terra \\\hline
3.3.23 & ca si la naue se faze de madera verde \textbf{ quando el humor natural dellos se va e se seca . } los maderos se encogen & Nam si ex lignis viridibus construatur nauis , \textbf{ quando naturalis eorum humor expirauerit , } contrahuntur ligna , \\\hline
3.3.23 & Et lançandolos assi en las naues quebrantan se los cantaros \textbf{ e aquel fuego fuerte ençiende } e quema la naue . & Ex qua proiectione vas frangitur , \textbf{ et illud incendiarium comburitur } et succendit nauem . \\\hline
3.3.23 & Lo quarto conuiene de colgar al maste de la \textbf{ naue vn madero luengo e ferrado de a mas partes . } para ferir tan bien en la & Quarto ad arborem nauis suspendendum est lignum quoddam longum \textbf{ ex utraque parte ferratum , } quod ad percutiendum tam nauem , \\\hline
3.3.23 & en la batalla de la mar \textbf{ conuiene de auer grand conplimiento de saetas anchas . } con las quales se pueden & Quinto in bello nauali habenda est \textbf{ copia ampliarum sagittarum , } cum quibus scindenda sunt vela hostium . \\\hline
3.3.23 & e la ordena algunan ganançia \textbf{ o la ordena a alguna otra vengança de saña o de cobdiçia mundanal . } Et enpero las batallas sy & ordinari ad lucrum , \textbf{ vel ad aliquam aliam satifactionem irae , vel concupiscentiae . } Bella tamen si iuste gerantur , \\\hline

\end{tabular}
