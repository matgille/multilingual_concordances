\begin{tabular}{|p{1cm}|p{6.5cm}|p{6.5cm}|}

\hline
1.1.1 & que demanda esta obra \textbf{ que nos fazemos , } E primero veamos qual es la manera & ut non fiant ulteriores sermones \textbf{ quam praesens opus requirat , } primo videndum est , \\\hline
1.1.1 & dizeque conplidamente se dize dela \textbf{ moralph̃ia si fuer fecha manifestaçion } e declaraçion della sengund la materia de que ella es & quod dicetur sufficienter de morali negocio , \textbf{ si manifestatio fiat } secundum subiectam materiam . \\\hline
1.1.1 & conujene al subdito \textbf{ De saber fazer ¶ } E si por qual manera Deuen mandar a los sus Subditos & si per hunc librum instruuntur Principes , \textbf{ quomodo debeant se habere , } et qualiter debeant \\\hline
1.1.2 & que son cerca delas obras \textbf{ commo las deue omne fazer } Mas las nr̃as obras & scrutari ea quae sunt circa operationes , \textbf{ quomodo faciendum sit eas . } Operationes autem nostrae ex quatuor \\\hline
1.1.2 & por que cada vnos deseos \textbf{ e cada vnas pasiones fazen en nos alguna inclinaçion } por que fagamos departidas obras & singulae enim affectiones et passiones \textbf{ aliquam inclinationem in nobis efficiunt , } ut alia , et alia opera faciamus . \\\hline
1.1.2 & e cada vnas pasiones fazen en nos alguna inclinaçion \textbf{ por que fagamos departidas obras } ¶Otrosy lo quarto paresçe & aliquam inclinationem in nobis efficiunt , \textbf{ ut alia , et alia opera faciamus . } Quarto etiam ipsi mores opera diuersificare videntur . \\\hline
1.1.2 & ¶Otrosy lo quarto paresçe \textbf{ que las costunbres fazen departimjento en las obras } por que los que h̃a costunbres de vieios & ut alia , et alia opera faciamus . \textbf{ Quarto etiam ipsi mores opera diuersificare videntur . } Nam habentes mores senum , \\\hline
1.1.2 & por que la fin en conparaçion delas obras \textbf{ que se ha de fazer es } mas prinçipal comjenço & quia finis respectu agendorum , \textbf{ est principalius principium , } quam aliquod aliorum . \\\hline
1.1.3 & or que asy commo dicho es esta obra tomamos \textbf{ e comneçamos a fazer porgera de ensseñar los pnçipes } et commo nunca el prinçipen ̃j otro & Quoniam ( ut dictum est ) \textbf{ opus istud suscepimus gratia eruditionis Principum : } cum nunquam quis plene erudiatur , \\\hline
1.1.3 & para preguntar actento e acuçioso para rretener e tomar Et \textbf{ pues que ya en el primero capitulo fizimos la magestad Real begniuola } e uolunterosa mostradol aquellas cosas & nisi sit beniuolus , docilis , et attentus : \textbf{ postquam in primo capitulo reddidimus | regiam maiestatem beniuolam , } ostendendo , quae dicenda sunt , \\\hline
1.1.3 & que nos prometiemos de tractar ligniamente ¶ \textbf{ et en el segundo capitulo fiziemos } essa mjsma magestad doçible & nos esse faciliter tractaturos : \textbf{ et in secundo reddidimus eam docilem , } narrando ordinem dicendorum : \\\hline
1.1.3 & contando la orden de las cosas que aqui auemos de dezir ¶finca que en este terçero capitulo \textbf{ fagamos la Real magestad atenta e acuçiosa } declarandol quanto es el prouecho delas cosas & restat ut in hoc capitulo tertio \textbf{ reddamus eam attentam , } declarando quanta sit utilitas in dicendis . \\\hline
1.1.3 & e atal arte ¶ \textbf{ pues que asy es en esta mjsma arte faremos el oydor begniuolo } e uolunteroso & sed magis est ordo rectus et debitus . \textbf{ In huiusmodi ergo arte } ex facilitate tradendi \\\hline
1.1.3 & por la orden de lans cosas \textbf{ que son de dezer le faremos doçible e engeñoso } por que cada vno mayormente es fechͣo ensennable e engennoso & sed ex ordine dicendorum \textbf{ redditur docilis ; } nam quis maxime efficitur docilis \\\hline
1.1.3 & Mas por el prouecho delas cosas \textbf{ que son de dezer es fecho el oydor atento e acuçioso } para aprender & ex utilitate autem dicendorum \textbf{ redditur auditor attentus , } nam quilibet attente audit , \\\hline
1.1.3 & que bien gouierna asy mesmo es digno \textbf{ que sea fecho gouernador e senonr de los otros } Ca el que ha sabiduria & ut efficiatur rector , \textbf{ et dominus aliorum . } Nam vigens prudentia , \\\hline
1.1.4 & Ca en rrelaçion desta obra \textbf{ que se sigue feziemos la Real magestad begniuola et uolunterosa para oyr } e a prinder & Praemissis quibusdam praeambulis necessariis ad propositum , \textbf{ quia respectu sequentis operis } ex facilitate modi tradendi \\\hline
1.1.4 & Et ahun diemos la \textbf{ e fiziemos la ensennable e engennosa } para preguntar & ex facilitate modi tradendi \textbf{ reddidimus regiam maiestatem beniuolam , } ex ordine dicendorum \\\hline
1.1.4 & que auemos de dezir \textbf{ Et aun fiziemos la actenta e acuçiosa } para tomar e rretener & ex ordine dicendorum \textbf{ reddidimus eam docilem , } ex utilitate reperta \\\hline
1.1.4 & Mas por que la fin es comjenço mas prinçipal de todas las obras \textbf{ que fazemos que njnguno delons otros prinçipios } asy commo dicho es de suso & principium agibilium principalius , \textbf{ quam aliquod aliorum , } ut dicebatur supra , \\\hline
1.1.4 & que es Razon derecha de todas las cosas \textbf{ que ha de obrar e de fazer ¶ } Mas es dicho bien auenturado en vida contenplatina & habendo in se prudentiam , \textbf{ quae est recta ratio agibilium : } dicatur felix contemplatiue , \\\hline
1.1.4 & pues que asy es llaman al omne acabado en las obras \textbf{ que ha de fazer bien auenturado } çiudadanamente & et aliquid melius homine . \textbf{ Perfectum igitur in agibilibus , } vocabant felicem politice : \\\hline
1.1.4 & mas mejor que omne \textbf{ Ca fazer e partiçipar enlas obras } con los otros omes & sed homine meliorem . \textbf{ Nam agere et communicare } in actionibus cum aliis , \\\hline
1.1.4 & entre el sabio en las obras \textbf{ que ha de fazer } Et el acabado ente las sçiençias especulatians quanta es & Tanta est ergo differentia \textbf{ inter prudentem in agibilibus , } et perfectum in speculabilibus , \\\hline
1.1.4 & entienden las obras \textbf{ faziendo cosas granadas } e honrra das et gouernando derechamente los sus . & non enim vitam contemplatiuam \textbf{ ponimus in pura speculatione , } ut Philosophi sentiebant . \\\hline
1.1.5 & tres cosas le son le mester ¶ \textbf{ Lo primero que faga bien } ¶Lo segundo que faga bien & tria requiruntur . \textbf{ Primo , quod agat bene . } Secundo , quod ex electione . \\\hline
1.1.5 & Lo primero que faga bien \textbf{ ¶Lo segundo que faga bien } por eleçion e por uoluntad ¶ & Primo , quod agat bene . \textbf{ Secundo , quod ex electione . } Tertio , quod agat delectabiliter . \\\hline
1.1.5 & por eleçion e por uoluntad ¶ \textbf{ Lo terçero que el bien que feziere } que lo fiza deletablemente & Secundo , quod ex electione . \textbf{ Tertio , quod agat delectabiliter . } Si enim non ageret bene sed male , \\\hline
1.1.5 & e de buena uoluntad . \textbf{ Ca sy non feziese bien } mas mal non podria alcançar buena fin & Tertio , quod agat delectabiliter . \textbf{ Si enim non ageret bene sed male , } non consequeretur finem , \\\hline
1.1.5 & e enbarga todo bien \textbf{ El qual bien tirado los que mal fazen dignos son } que ayan mala postremeria . & tollit omne bonum . \textbf{ Quare si male agentes digni sunt } ut mala consequantur , \\\hline
1.1.5 & Ca el mal es contrario del bien \textbf{ Et por ende el mal es contrario ala fin del que mal faze . } Por la qual rrazon los malos dignos son & quia malum contrariatur bono , \textbf{ per consequens contrariatur fini , | male agentes } ( secundum quod huiusmodi ) \\\hline
1.1.5 & mas el contrario e mala fin . \textbf{ Mas non solamente los que mal fazen non alcançan buena fin } mas avn los que pueden bien fazer & sed contrarium finis . \textbf{ Immo non solum male agentes | non consequuntur finem , } sed potentes bene agere , \\\hline
1.1.5 & Mas non solamente los que mal fazen non alcançan buena fin \textbf{ mas avn los que pueden bien fazer } sy non fazen bien non les es deuido corona & non consequuntur finem , \textbf{ sed potentes bene agere , } nisi bene agant , \\\hline
1.1.5 & mas avn los que pueden bien fazer \textbf{ sy non fazen bien non les es deuido corona } njn les es deuido buena fin njn buean andança ¶ & sed potentes bene agere , \textbf{ nisi bene agant , | non debetur eis corona , } nec debetur eis finis , \\\hline
1.1.5 & Enpero si non lidiaren de fech̃o non les es deuida corona ¶ \textbf{ pues que asi es conviene bien fazer de fecho } por que por las nr̃as obras merescamos de auer buena fino buena ventura & non debetur eis corona . \textbf{ Oportet igitur actu bene agere , } ut per opera nostra mereamur \\\hline
1.1.5 & Ca aqual \textbf{ que faze bien acaso e auentura por esto non es de alabar } njn por esto non le es deujda buean fin njn buena ventura . & quod ex electione agamus . \textbf{ Nam qui casu vel fortuitu bene agit , | ex hoc non est laudandus , } nec debetur ei \\\hline
1.1.5 & njn por esto non le es deujda buean fin njn buena ventura . \textbf{ Ca las cosas que se non fazen } por eleçio non son de uoluntad & ex hoc finis vel felicitas : \textbf{ nam quae ex electione non fiunt , } non sunt voluntaria . \\\hline
1.1.5 & por eleçio non son de uoluntad \textbf{ Et por las cosas que fazemos non de uoluntad } non deuemos ser loados njn denostados . & non sunt voluntaria . \textbf{ Ex inuoluntariis autem | ( ut patet per Philosophum 3 Ethic’ ) } non laudamur nec vituperamur : \\\hline
1.1.5 & si non obra de voluntad \textbf{ Mas aquellas cosas que nos non fazemos por } electiuo non las fazemos de uoluntad ¶ & quod nullus est beatus nisi volens . \textbf{ Sed quae non agimus ex electione , } non agimus volentes : \\\hline
1.1.5 & Mas aquellas cosas que nos non fazemos por \textbf{ electiuo non las fazemos de uoluntad ¶ } pues que asy es fablando propiamente de tales obras qua non son de uoluntad fechas segunt que tales son non se nos sigue buena fin & Sed quae non agimus ex electione , \textbf{ non agimus volentes : | ergo per se loquendo } ex talibus operibus \\\hline
1.1.5 & por que qua quanto cada vno mas se delecta en la obra \textbf{ que faze tanto } mas faze aquella obra & nam quanto quis in aliquo opere magis delectatur , \textbf{ tanto magis voluntarie } et ex habitu efficit opus illud . \\\hline
1.1.5 & que faze tanto \textbf{ mas faze aquella obra } por voluntad e por elecçio¶ & tanto magis voluntarie \textbf{ et ex habitu efficit opus illud . } Unde Philosophus 2 Ethic’ vult , \\\hline
1.1.5 & en el segundo libro delas ethicas \textbf{ que non cunple solamente fazer buenas obras } mas fazerlas bien njn cunple de obrar obras iustas & Unde Philosophus 2 Ethic’ vult , \textbf{ quod non sufficit agere bona , } sed bene : nec sufficit operari iusta , \\\hline
1.1.5 & que non cunple solamente fazer buenas obras \textbf{ mas fazerlas bien njn cunple de obrar obras iustas } mas fazer las iustamente e con iustiçia . & quod non sufficit agere bona , \textbf{ sed bene : nec sufficit operari iusta , } sed iuste . \\\hline
1.1.5 & mas fazerlas bien njn cunple de obrar obras iustas \textbf{ mas fazer las iustamente e con iustiçia . } por que contesçe que algunos malos fazen algunas buenas obras & sed bene : nec sufficit operari iusta , \textbf{ sed iuste . } Contigit enim aliquos prauos \\\hline
1.1.5 & mas fazer las iustamente e con iustiçia . \textbf{ por que contesçe que algunos malos fazen algunas buenas obras } Enpero por que las non . & sed iuste . \textbf{ Contigit enim aliquos prauos | facere aliqua de genere bonorum , } tamen quia non faciunt ea bene et delectabiliter , \\\hline
1.1.5 & Enpero por que las non . \textbf{ fazen bien njn } delectosamente non les conuiene & facere aliqua de genere bonorum , \textbf{ tamen quia non faciunt ea bene et delectabiliter , } non oportet per huiusmodi opera \\\hline
1.1.5 & Et quando estas tres cosas todas uienen en vno \textbf{ que nos fagamos buean sobras } e las fagamos de voluntad & Cum ergo ista tria contingunt , \textbf{ quod agamus bona , } ex electione , et delectabiliter , \\\hline
1.1.5 & que nos fagamos buean sobras \textbf{ e las fagamos de voluntad } e delectosamente estonçe contesçe que nos alcançemos conplidamente buena fin . & quod agamus bona , \textbf{ ex electione , et delectabiliter , } maxime contingit nos sic finem \\\hline
1.1.5 & esto es por ventura asy los que ante non conosçen la su fin \textbf{ si bien fazen } e alcançan la fin & sic non praecognoscentes finem , \textbf{ si bene agant , } et finem consequantur , \\\hline
1.1.5 & que para lanr̃auida grant acresçentamiento \textbf{ faze connosçer ante la fin } Ca por esta Razon alcançaremos ante la fin & ad vitam nostram magnum habet incrementum : \textbf{ consequemur enim } ex hoc magis ipsum finem ; \\\hline
1.1.5 & Lo terçero conesçer ante la fin \textbf{ non solamente nos faze bien obrar e obrar de uoluntad } mas ahun faze nos obrar delectosamente & Tertio praecognitio finis \textbf{ non solum facit | nos agere bene , } et ex electione , \\\hline
1.1.5 & non solamente nos faze bien obrar e obrar de uoluntad \textbf{ mas ahun faze nos obrar delectosamente } Ca pensada la bien andança & nos agere bene , \textbf{ et ex electione , } sed etiam delectabiliter . \\\hline
1.1.5 & e la buena uentraa \textbf{ que omne ha por las buean s obras las obras guaues e fuertes de fazer se } fazen muy delectables e plazenteras ¶ & sed etiam delectabiliter . \textbf{ Nam grauia efficiuntur delectabilia et dulcia , } considerata beatitudine , et felicitate , \\\hline
1.1.5 & que omne ha por las buean s obras las obras guaues e fuertes de fazer se \textbf{ fazen muy delectables e plazenteras ¶ } pues que asi e sacanda & sed etiam delectabiliter . \textbf{ Nam grauia efficiuntur delectabilia et dulcia , } considerata beatitudine , et felicitate , \\\hline
1.1.5 & que a otro ninguno . \textbf{ por que pueda fazer buenas obras e comunes } que son en alguna manera obras diuina les . & cognoscere suam felicitatem , \textbf{ ut opera communia , } quae sunt quodammodo diuina , \\\hline
1.1.5 & que son en alguna manera obras diuina les . \textbf{ Et estas obras que los faga bien e delectosamente e de buena uoluntad } ¶ lo segundo esto mesmo paresçe asy & quae sunt quodammodo diuina , \textbf{ exerceat bene , delectabiliter , | et ex electione . } Secundo patet hoc idem \\\hline
1.1.6 & La primera es que la bien andança es bien acabado \textbf{ e tal bien que por si mesmo faze el omne acabado . } Ca estonçe dezimos & Felicitas enim dicit perfectum , \textbf{ et per se sufficiens bonum . Nam tunc dicimus aliquem esse felicem , } quando assecutus est id , \\\hline
1.1.6 & e es bien acabado e bien sufiçiente \textbf{ por si para fazer al omne acabado ¶ } Et desto primero se sigue lo segundo & et perfectum , \textbf{ et per se sufficiens bonum . } Ex hoc autem primo sequitur secundum , \\\hline
1.1.6 & Por que quantomayores son los bienes \textbf{ tantom fazen la razon libre e desenbargada } Ca lo que es segunt razon non conuiene & quanto maiora sunt , \textbf{ tanto magis reddunt rationem liberam , et expeditam . } Nam quod secundum rationem existit , \\\hline
1.1.6 & que maguera que alguas delecta connes sean conuenibles e honestas \textbf{ por que las obras uirtuosas fazen al omnen bueno e uirtuoso e de grant delectaçion . } Enpero njnguna delectaçion non es feliçadat & quod licet sint delectationes aliquae licitae , et honestae , \textbf{ eo quod ipsa opera virtutum Homini bono , | et virtuoso magnam delectationem faciant : } nulla tamen delectatio est essentialiter ipsa felicitas , \\\hline
1.1.6 & la segunda que estas plazenterias carnales \textbf{ le fazen ser menospreçiado de los omes } ¶la terçera es que las plazenterias dela carne le fazen & quia ipsum maxime deprimunt . \textbf{ Secundo , quia eum contemptibilem reddunt . } Tertio , quia esse ipsum indignum principari faciunt . \\\hline
1.1.6 & le fazen ser menospreçiado de los omes \textbf{ ¶la terçera es que las plazenterias dela carne le fazen } que non sea digno de ser prinçipe¶ & Secundo , quia eum contemptibilem reddunt . \textbf{ Tertio , quia esse ipsum indignum principari faciunt . } Dicebatur enim supra , \\\hline
1.1.6 & es que por esso seria denostado e menospreçiado \textbf{ e farie que los otros le touiese en pos . } ¶ Onde el philosofo en el quinto libro delas politicas & ø \\\hline
1.1.7 & dende ¶ El pmero mal es que pierde muy grandes bienes ¶ \textbf{ El segundo es que se faze por ende tirano quiere dezer le unadoro . } apremiador del pueblo & Primo , quia amittit maxima bona . \textbf{ Secundo , quia efficitur Tyrannus . } Tertio , quia efficitur populi depiaedator . \\\hline
1.1.7 & nin granado \textbf{ el qual magnifico ha de fazer grandes espensas para ser granado } nin ahun puede ser mager fico & nunquam potest esse magnificus , \textbf{ cuius est facere magnos sumptus : } nec etiam potest esse Magnanimus , \\\hline
1.1.7 & que pone su bien andança en las riquezas corporales . \textbf{ ¶ Ca por esto se fare tirano } ca ay grant diferençia entre el Rey e tirano & vel Principi suam felicitatem ponere in diuitiis , \textbf{ quia hoc facto Tyrannus efficitur . } Est enim differentia \\\hline
1.1.7 & esto es por que del bien comun se le siguͤ bien ala persona \textbf{ Mas al tirano faze todo el contrario . } Ca prinçipalmente para mientes al su bien propio . & hoc est ex consequenti . \textbf{ Tyrannus vero econuerso , } principaliter intendit bonum priuatum : \\\hline
1.1.7 & e en los aueres \textbf{ prinçipalmente entiende de thesaurizar e fazer thesoro e llegar muchos dineros } Et por ende se sigue & ponens suam felicitatem in numismate , \textbf{ principaliter intendit reseruare sibi , | et congregare pecuniam . } Non ergo est Rex , \\\hline
1.1.7 & por que non solamente non les procura bien \textbf{ mas faze les mal . } Por la qual cosa si es muy contra razon que el Rey dexe muy grandes bienes . Et si es contra razon otrosi & quando non solum non procurat bonum , \textbf{ sed etiam infert malum . } Quare si detestabile est , \\\hline
1.1.8 & e non de dentro del alma ¶ \textbf{ Lo terçero por que la honrraes tal bien que es mas en aquel que la faze } que non en aquel que la resçibe . & Secundo , est bonum extrinsecum . \textbf{ Tertio , est magis in honorante , } quam in honorato . \\\hline
1.1.8 & do dize que honrra es reuerençia \textbf{ que fazen unos omes a otros en testimoino de uirtud } por que son omes uirtuosos ¶ & esse exhibitionem reuerentiae \textbf{ in testimonium virtutis . } Causa ergo , \\\hline
1.1.8 & Ca ya dixiemos \textbf{ que la honrraes reuerençia fecha a alguno en sennal de uertud . } Mas la señal o el testimoino & dictum est enim \textbf{ quod honor est reuerentia exhibita | in signum virtutis : } signum autem , \\\hline
1.1.8 & lo que cada vno pienssa en su coraçon \textbf{ mas lo que faze de fuera¶ } Pues que asi es si la reuerençia es honrra & quae quis in corde cogitat , \textbf{ sed quae exterius repraesentat . } Reuerentia ergo , \\\hline
1.1.8 & e ha de magnifestar la uirtud de aquel \textbf{ a quien la fazen non cunple } que la uertud sea penssada en el coraçon & si debet manifestare virtutem eius \textbf{ cui exhibetur , | non sufficit , } quod si cogitata in corde , \\\hline
1.1.8 & que esta de fuera \textbf{ por que es reuerençia fecha } por señales mostradas de fuera¶ & Honor ergo habet rationem boni extrinseci , \textbf{ cum sit reuerentia exhibita } per quaedam exteriora signa . \\\hline
1.1.8 & por que la honrra es mas enel \textbf{ que la faze que non en el que la resçibe } assi como dize el philosofo en el primero libro delas ethicas . & ex eo quod honor magis est in honorante , \textbf{ quam in honorato , } ut plane vult Philosophus 1 Ethicorum \\\hline
1.1.8 & Ca asi commo dize vn prouerbio la cortesia e la enseñaça \textbf{ es en aquel que la faze . } Assi la honrra es mas en aquel que la faze & curialitas est eius \textbf{ qui facit ipsam , } sic honor magis in eo est , \\\hline
1.1.8 & es en aquel que la faze . \textbf{ Assi la honrra es mas en aquel que la faze } que non en aquel que la resçibe¶ & qui facit ipsam , \textbf{ sic honor magis in eo est , | qui facit ipsum , } quam in eo qui per huiusmodi reuerentiam honoratur . \\\hline
1.1.8 & Et esso mesmo paresçe manifiestamente al seso . \textbf{ Por que sy alguno inclinando se faze reuerençia a otro o le honrra . } çierto es que aquella inclinaçion & Apparet autem hoc esse sensibiliter verum : \textbf{ nam si aliquis inclinat se reuerenter ad alium , | vel honorat ipsum , } constat inclinationem illam proprie esse in inclinante , \\\hline
1.1.8 & çierto es que aquella inclinaçion \textbf{ quel faze propnamente es } en el que se indina & vel honorat ipsum , \textbf{ constat inclinationem illam proprie esse in inclinante , } non in eo cui inclinatio exhibetur . \\\hline
1.1.8 & que la reuerençia o la honrra mas es en el honrrador \textbf{ e en el que la faze } que non en el honrrado & Inclinatio ergo illa et reuerentia siue honor \textbf{ magis est in honorante , } quam in honorato . \\\hline
1.1.8 & do dize \textbf{ que los que fazen fuerça tan solamente de ser honrrados } que estos son infintos e superfiçiales . & unde et Philosophus in Politicis , \textbf{ curantes de honore tantum , } dicit esse fictos , et superficiales . \\\hline
1.1.8 & si el prinçipe pusiere la su bien andança enlas honrras \textbf{ por que pueda delo que feziere honrra alcançar } presumira de poner los pueblos a todo peligro & si Princeps suam felicitatem in honoribus ponat , \textbf{ ut possit honorem consequi , } praesumet suam gentem exponere omni periculo . \\\hline
1.1.9 & e enlos epitafios Reuerençia de antiguedat \textbf{ que fazen algunos } por que son antigunoo e bueons . & recordationes in metris , decantatio , \textbf{ aut solutae orationis } ad laudem recitatio , \\\hline
1.1.9 & Ca la fama es vn claro conosçimiento con loor . \textbf{ Enpero si quisieremos fazer depart ineto entre la fama e la eglesia } diremos & cum laude notitia . \textbf{ Si tamen vellemus aliquo modo distinguere | inter gloriam , et famam : } diceremus quod fama oritur ex gloria : \\\hline
1.1.9 & por que dura por mucho stp̃os \textbf{ e non se puede desfazer . } Et ahun paresçe & cum per multa tempora contingat \textbf{ ipsam indelebilem esse . } Videtur ergo quod maxime Princeps \\\hline
1.1.9 & se departe el conosçimiento de dios del nuestro conosçimiento . \textbf{ Ca el conosçimiento de dios fizo } e faze todas las cosas & est quaedam clara cum laude notitia . \textbf{ Dei autem notitia } ( ut ad praesens spectat ) \\\hline
1.1.9 & Ca el conosçimiento de dios fizo \textbf{ e faze todas las cosas } mas el nuestro conostimiento es fecho delas cosas que dios fizo . & Dei autem notitia \textbf{ ( ut ad praesens spectat ) } in tribus differt a notitia nostra : \\\hline
1.1.9 & e faze todas las cosas \textbf{ mas el nuestro conostimiento es fecho delas cosas que dios fizo . } asi commo dize el philosofo & ( ut ad praesens spectat ) \textbf{ in tribus differt a notitia nostra : } nam notitia Dei causat res , \\\hline
1.1.9 & que cosa es la honrra en si¶ \textbf{ La otra manera es teniendo mientes al talante de aquellos que la fazen . } Ca la honrra en si non es & vel ut procedit \textbf{ ex affectione dantium . } Honor autem in se , \\\hline
1.1.9 & en quanto salle de buean uoluntad \textbf{ de aquellos que la fazen es en ella vn } gualardon conuenible & ex affectione dantium , \textbf{ habet ibi esse } quaedam congrua remuneratio : \\\hline
1.1.9 & gualardon conuenible \textbf{ Por que los que la fazen non han cosa mayor } nin mas conuenible & quaedam congrua remuneratio : \textbf{ nam eo ipso , | quod dantes non habent aliquid } maius quod retribuant , \\\hline
1.1.9 & Et en esta manera deuen resçebir la honrra \textbf{ que les fazen } Por la qual cosa & congruit Principi \textbf{ hoc modo honorem exhibitum acceptare . } Propter quod \\\hline
1.1.9 & de aquellos \textbf{ que fazen la honrra } e non propiamente la honrra en sy . & in huiusmodi retributione \textbf{ acceptatur affectio dantium , } et non proprie honor datus . \\\hline
1.1.9 & Ca si el prinçipe non resçibiese esta buena uoluntad delas \textbf{ que les fazen honrramas quesiese } que la su gente & et non proprie honor datus . \textbf{ Quod si tamen Principes non acceptarent hanc affectionem dantium , } sed requirerent a gente sibi commissa alia exteriora bona , \\\hline
1.1.10 & ize vegeçio enł primero libro \textbf{ que fizo dela caualleria } que sobre todas las cosas es de alabar la maestria & Quod non decet regiam maiestatem suam \textbf{ Vegetius in libro De re militari , } super omnia commendare videtur bellorum industriam . \\\hline
1.1.10 & ¶La quinta daquello por que tal sennorio \textbf{ en la mayor parte faze gerat danno¶ } La primera razon se puede assi declarar & Quinta vero , ex eo quod tale dominium \textbf{ ut plurimum infert nocumentum . } Prima via sic patet . \\\hline
1.1.10 & poner la su bien andança en el poderio çiuiles \textbf{ por que este sennorio faze grant danno en las mas cosas . } Ca commo la feliçidat & quia huiusmodi principatus infert \textbf{ ut plurimum nocumentum . } Nam cum felicitas sit finis omnium operatorum , \\\hline
1.1.10 & Et en aquello que en las mas cosas . \textbf{ faze danno siguese } que non es cosa conuenible & ad minora bona , \textbf{ et ut plurimum infert nocumentum : } inconueniens est etiam Principem \\\hline
1.1.11 & Ca perdida la sanidat \textbf{ fazen se los mienbros magros e flacos } e non finca en ellos color natraal & quia amissa sanitate , \textbf{ membra fiunt macilenta , } non remanet in eis color debitus : \\\hline
1.1.11 & muchos se le una tan para seer señores \textbf{ e fazen grand discordia en el pueblo ¶ } Otrossi deuen los prinçipes auer riquezas sufiçientes & ut dominentur , \textbf{ et faciunt dissensionem in Populo . } Debet enim Princeps \\\hline
1.1.11 & por que puedan defender los regnos \textbf{ e fazer obras de uertudes . } E conuiene al Rey de seer magnifico e largo & ut possit regnum defendere , \textbf{ et exercere operationes virtutum : } decet enim Regem esse magnificum , \\\hline
1.1.11 & por que pueda bien fazera las personas dignas \textbf{ la qual cosa non se puede fazer sin riquezas } ¶ & beneficiare personas dignas : \textbf{ quod sine diuitiis fieri non potest . } Sic etiam , \\\hline
1.1.11 & Ca por el \textbf{ menospreçiamientodel prinçipe muchͣs vezes contesçe que alguons fazen e obran malas cosas } e fazen muchos tuertos a otros & et expedit ei habere ciuilem potentiam : \textbf{ nam propter paruipensionem Principis , | ut plurimum aliqui operantur mala , } et offendunt alios , \\\hline
1.1.11 & menospreçiamientodel prinçipe muchͣs vezes contesçe que alguons fazen e obran malas cosas \textbf{ e fazen muchos tuertos a otros } la qual cosa non conuiene al Regno . & ut plurimum aliqui operantur mala , \textbf{ et offendunt alios , } quod Regno non expedit . \\\hline
1.1.11 & Pues que assi es quando el prinçipe es alabado \textbf{ por bueno los subditos toman manera para fazer bien . } Ahun en essa misma manera la sanidat e la fermosura & bonus praedicatur , \textbf{ subditi suscipiunt materiam benefaciendi . } Sic etiam , sanitas , pulchritudo , \\\hline
1.1.11 & para ganar la feliçidat e la bien andança . \textbf{ Et en quanto fazen a una claridat } e apostura dela feliçidat e dela bien andança ¶ & ut sunt organa ad felicitatem , \textbf{ et ut faciunt } ad quandam claritatem felicitatis . \\\hline
1.1.12 & que es derecha sabiduria de todas las obras \textbf{ que han de fazer } es bien auentado en la uidapolitica . & quicunque scit alios bene regulare \textbf{ secundum Prudentiam , } est felix politice : \\\hline
1.1.12 & ¶ Et destas dos feliçidades e bien andanças \textbf{ fizo mencion el philosofo } Et dela vna determino en el primero libro delas ethicas & est felix contemplatiue . \textbf{ De his ergo duabus felicitatibus mentionem fecit : } de quarum una determinatur 1 Ethicorum , \\\hline
1.1.12 & e que sea su ofiçial \textbf{ para fazer las sus obras . } Por la qual cosa si los ofiçiales & sit diuinum organum , \textbf{ siue sit minister Dei . } Quare si minister , \\\hline
1.1.12 & si es amigo de \textbf{ otrosi faze } aquello que el su & et ex hoc aliquis probetur \textbf{ esse dilectiuus alterius , } si agat quae ipse vult : \\\hline
1.1.13 & Ca muchos esta non tal estado \textbf{ que podrian mal fazer } e guardan se delo fazer & multi enim non existentes \textbf{ in statu quo possint mala facere , } praeseruant se a malo : \\\hline
1.1.13 & que podrian mal fazer \textbf{ e guardan se delo fazer } Enpero si a mayor estado & multi enim non existentes \textbf{ in statu quo possint mala facere , } praeseruant se a malo : \\\hline
1.1.13 & Enpero si a mayor estado \textbf{ fuese leunata dos aurian razon para fazer muchͣs males . } Et por esso dize aristotiles & quod si tamen ad statum dignitatis assumerentur , \textbf{ multas transgressiones efficerent . } Propter quod Ethic’ 5 scribitur , \\\hline
1.1.13 & quando es puesto en señorio \textbf{ en que pueda fazer bien e mal . } Et aquella hora entiendan los omes & qualis homo sit , \textbf{ cum in principatu existens , in quo potest bene et male facere , } cogitat qualiter se habeat . \\\hline
1.1.13 & por el bien comun menguarian \textbf{ enlo que han de fazer } e non acresçentarian en su meresçimiento . & dato quod non transgrediatur , \textbf{ quia indiscrete agit , } suum meritum non augmentatur . \\\hline
1.2.1 & Ca las uirtudes son vnos hornamentos e conponimientos e hunas perfectiones \textbf{ que fazen acabada el alma . } Pues que assi es conuiene & Virtutes autem quaedam sunt quidam ornatus , \textbf{ et quaedam perfectiones animae . Oportet ergo prius ostendere , } quot sunt potentiae animae , \\\hline
1.2.1 & nin por que cresçe bien \textbf{ e se faze grande en el cuerpo ¶ } La segunda razon por que en tales poderios naturales & digestiuam bonam , \textbf{ vel bonam augmentatiuam . } Secundo , in talibus non habent esse virtutes : \\\hline
1.2.1 & Mas segunt su manera \textbf{ e su poder fazen sienpre sus obras ¶ } La terçera razon por que enlos poderios naturales & sed semper secundum modum sibi possibilem \textbf{ suas actiones efficiunt . } Tertio , in talibus potentiis \\\hline
1.2.1 & que por las malas disposiconnes se determinan los poderios del alma \textbf{ para mal fazer . } Et por las buenas disposiconnes & ut bene vel male agant , \textbf{ ut per habitus vitiosos determinatur potentia ad agendum male , } per virtuosos ad agendum bene : \\\hline
1.2.2 & e esta scina \textbf{ non por que nos faga sabidores descina } mas porque nos faga buenos . & ( ut pluries diximus ) praesens opus , \textbf{ non ut sciamus , } sed ut boni fiamus : \\\hline
1.2.2 & non por que nos faga sabidores descina \textbf{ mas porque nos faga buenos . } La qual bondat non pue de ser en nos & non ut sciamus , \textbf{ sed ut boni fiamus : } quae bonitas , sine prudentia et virtutibus moralibus , \\\hline
1.2.2 & pues que assi es menguadamiente \textbf{ lo fiziera la natura } si diera alas aian lias appetito cobdiçiador & et vitare malum . \textbf{ Imperfecto ergo egisset natura , } si dedisset animalibus concupiscibilem , \\\hline
1.2.3 & que es uirtud \textbf{ para fazer grandes cosas ¶ } La septima es manssedunbre ¶ & quam bene vertibilitatem , \textbf{ vel societatem appellare possumus . } Igitur computata Iustitia , \\\hline
1.2.3 & que es uirtud \textbf{ que faze omne desçender } a seer buen conpannon de todos ¶ & et Prudentia duodecim sunt virtutes morales ; \textbf{ de quibus omnibus quid sunt , } et quomodo decet eas Reges habere , \\\hline
1.2.3 & e son pasiones del alma \textbf{ e son obras que fazemos de fuera Et en estas nos conuiene } e poner meatad e egualdat . & scilicet , rationes , passiones , et operationes exteriores : \textbf{ per virtutes enim debemus } habere rationes rectas , \\\hline
1.2.3 & e entender las otras diez uirtudes morales \textbf{ delas quales fezimos mençion en comienço deste capitulo ¶ } Pues que assi es la iustiçia & illae decem virtutes morales , \textbf{ de quibus in principio huius capituli fecimus mentionem . } Iustitia ergo , \\\hline
1.2.3 & quanto tenprando las passiones mesuradamente \textbf{ e tenpradamente fazemos las obras de fuera . Mas la iustiçia se ha } por el contrario . & inquantum moderatis passionibus , \textbf{ moderate ferimur in operationes exteriores . | Iustitia autem econuerso , } proprius aequat , \\\hline
1.2.3 & o las obras \textbf{ que fazemos de fuera . } Ca faze que a cada vno sea dado lo qual e conuiene & et moderat ipsas res , \textbf{ vel ipsas operationes exteriores : } facit enim , \\\hline
1.2.3 & que fazemos de fuera . \textbf{ Ca faze que a cada vno sea dado lo qual e conuiene } o lo quel deuen dar & vel ipsas operationes exteriores : \textbf{ facit enim , | quod cuilibet tribuatur } quod decet , \\\hline
1.2.3 & e las obras \textbf{ que fazemos de fuera } mesuranse e ygualan se las passiones & et aequatis rebus , \textbf{ et operibus exterioribus , } passiones in nobis rectificantur , \\\hline
1.2.3 & Mas yra bien de razon puede ser en tres maneras . \textbf{ O por que acaba este bien o por que lo faze o por que lo guarda e defiende . } Mas la pradençia va a bien de razon & potest esse tripliciter , \textbf{ vel quia huiusmodi bonum perficit , | vel quia ipsum efficit , } vel quia ipsum custodit , \\\hline
1.2.3 & Mas la iustiçia va a bien de razon \textbf{ por que faze esse mismo bien . } Ca la iustiçia en las obras de fuera & in bonum rationis , \textbf{ quia huiusmodi bonum efficit : } nam Iustitia in operibus exterioribus , \\\hline
1.2.3 & e en las cosas \textbf{ que fazemos faze aquel bien } e aquella yguasdat & nam Iustitia in operibus exterioribus , \textbf{ et in ipsis rebus efficit illud bonum , } et illam aequalitatem , \\\hline
1.2.3 & Ca aquestas uirtudes \textbf{ que mesuran e ygualan las passiones fazen } que por ellas el omne non ande deserrado & Nam huiusmodi virtutes , \textbf{ moderantes passiones , agunt , } ne per eas homo abducatur , \\\hline
1.2.3 & O sele una tan para dar pena a alguno . \textbf{ Et entonçe es sanna que por el menospreçio o por el mal qua nos fazen } dessea pena a otre en vengança de aquel mal . & et tunc est ira , \textbf{ quae est propter paruipensionem : | vel propter malum illatum } appetit paenam in vindictam . \\\hline
1.2.3 & Ca magnifiçençia es uirtud \textbf{ que faze grandes cosas . } Pues que assi es la liberalidat e franqueza & idem est enim magnificentia \textbf{ quod magnificens : } erit igitur liberalitas in concupiscibili , \\\hline
1.2.3 & que es uirtud \textbf{ que faze al coraçon alto } e ha de seer çerca grandes honrras & sic dicitur Magnanimitas , \textbf{ quae est circa honores . } Erit autem honoris amatiua in concupiscibili , \\\hline
1.2.3 & Et la magnificençia \textbf{ para fazer grandezas e grandes cosas . Las quales uirtudes se pueden tomar assi . } Ca la fortaleza e la manssedunbre son çerca delas passiones del coraçon & et Mansuetudo sunt \textbf{ circa passiones ortas ex malis , } ut Fortitudo est \\\hline
1.2.3 & Ca la magnifiçençia es çerca de los bienes grandes e prouechosos \textbf{ assi commo en fazer grandes espenssas . } En la qual cosa se muestra omne por magnifico e granado . & circa magna bona utilia , \textbf{ ut circa magnos sumptus : } Magnanimitas vero \\\hline
1.2.4 & assi commo la fortaleza e la manssedunbre et la magnifiçençia \textbf{ que es uirtud para fazer grandescosas . } Et la magnanimidat & quatuor in irascibili , \textbf{ ut Fortitudo , Mansuetudo , Magnificentia , et Magnanimitas , } et sex in concupiscibili , \\\hline
1.2.4 & e por las passiones fuertes \textbf{ conmo quier que bien faga . } Enpero non lo faze delectable mente & non ergo continens est plene virtuosus , \textbf{ quia propter passiones fortes licet bene agat , } non tamen est ei delectabile bene agere . \\\hline
1.2.4 & conmo quier que bien faga . \textbf{ Enpero non lo faze delectable mente } njn & quia propter passiones fortes licet bene agat , \textbf{ non tamen est ei delectabile bene agere . } Continentia ergo , \\\hline
1.2.4 & plazeterosa mente \textbf{ con mo faze el uirtuoso ¶ Et pues que assi es la continençia } e la perseuerançia & non tamen est ei delectabile bene agere . \textbf{ Continentia ergo , } et Perseuerantia sic accepta , \\\hline
1.2.4 & asi commo son la pradençia e la iustiçia \textbf{ Et las otras de que fiziemos mençion en el capitu lo ant̃ dich̃o ¶ } pues que assi es de todas estas quatro disposiconnes diremos & cuiusmodi sunt Prudentia , iustitia , et alia , \textbf{ de quibus in praecedenti capitulo fecimus mentionem . } De omnibus ergo his quatuor suo loco dicemus . \\\hline
1.2.5 & Ca es costunbrado entre los santos \textbf{ e ahun enre los philosofos de fazer departimiento entre las uirtudes . } Por que alguas son prinçipales e cardinales . & Consueuit enim apud Sanctos , \textbf{ et etiam apud Philosophos distingui inter virtutes : } quia quaedam sunt Cardinales et principales , \\\hline
1.2.5 & Et la quarta tenprança ¶ \textbf{ Mas las anexas e non prinçipales son las otras ocho delas quales fezimos mençion de suso ¶ } Mas que estas quatro uirtudes sean prinçipales e cardinales & videlicet , Prudentia , Iustitia , Fortitudo , et Temperantia . \textbf{ Annexae autem et non principales sunt aliae octo , | de quibus supra fecimus mentionem . } Has autem quatuor virtutes esse Cardinales et principales , \\\hline
1.2.5 & Por la qual de todas las obras \textbf{ que fazemos fagamos razones derechas ¶ } Otrosi commo contesca de obrar derechamente & quae sit recta ratio , \textbf{ per quam de ipsis agibilibus rectas rationes faciamus . } Rursus cum contingat operari recte et non recte , \\\hline
1.2.5 & por las quales somos muy prestos e inclinados \textbf{ para fazer aquells males } que cobdiçiamos & ut passiones concupiscibiles , \textbf{ quia proni sumus ad agendum illa : } quaedam vero retrahunt nos a bono , \\\hline
1.2.5 & Ca toda obra si deue ser uirtuosa . \textbf{ conuiene que se faga sabiamente } e iusta mente . & Nam omnis actus , \textbf{ si virtuosus esse debet , } oportet quod fiat prudenter , iuste , fortiter , et temperate : \\\hline
1.2.5 & que es uirtud \textbf{ para fazer grandes cosas ha alguna prinçipalidat } por la grandeza delas cosas espenssas & Sic etiam et Magnificentia \textbf{ quandam principalitatem habet } propter magnitudinem sumptuum , \\\hline
1.2.6 & si quisieremos bien obrar en aquellas cosas \textbf{ que fazemos tres cosas deuemos auer ¶ } Lo primero deuemos buscar muchas carreras e departidas ¶ & si circa agibilia bene negociari volumus , \textbf{ tria habere debemus . } Primo debemus \\\hline
1.2.6 & lo terçero deuemos mandar \textbf{ que se fagan las obras } segunt las carreras falladas e iudgadas . & Tertio debemus praecipere \textbf{ ut fiant opera } secundum inuenta et iudicata : \\\hline
1.2.6 & ¶ lo terçero deuemos mandar \textbf{ que se fagan las obras } segunt aquellas maneras falladas e iudgadas . & Tertio praecipiendum esset \textbf{ ut fierent opera } secundum inuenta et iudicata . \\\hline
1.2.6 & ¶ la terçera es uirtud \textbf{ por la qual mandamos que se fagan las obras todas segunt las cosas falladas e iudgadas } e esta dize el philosofo & Tertia , per quam praecipiamus \textbf{ ut fiant opera | secundum inuenta et iudicata , } et hanc dicimus esse prudentiam . \\\hline
1.2.6 & Et pues que assi es la pradençia mas derechamente regla las obras \textbf{ que se han de fazer } por que las manda luego fazer & in opera fienda , \textbf{ eo quod praecipiat illa fieri , } quam faciat virtus inuentiua et iudicatiua . \\\hline
1.2.6 & que se han de fazer \textbf{ por que las manda luego fazer } que la uirtud buscadora e falladora . & in opera fienda , \textbf{ eo quod praecipiat illa fieri , } quam faciat virtus inuentiua et iudicatiua . \\\hline
1.2.6 & por las quales nos podemos reglar e ordenar en todas las cosas \textbf{ que auemos de fazer . } lo quarto puede se conparar la pradençia ala sçiençia dela qual d se departe en esta manera . & et alia per quae regulari possumus in agendis . \textbf{ Quarto comparari habet prudentia } ad ipsam scientiam , \\\hline
1.2.6 & que pue den contesçer \textbf{ que son en nuestro senñorio de las fazer } o non las fazer . & et rerum contingentium , \textbf{ quae sunt in potestate nostra . } Quinto comparari potest prudentia ad artem , \\\hline
1.2.6 & que son en nuestro senñorio de las fazer \textbf{ o non las fazer . } Et aquella sabiduria es dicha pradençia & et rerum contingentium , \textbf{ quae sunt in potestate nostra . } Quinto comparari potest prudentia ad artem , \\\hline
1.2.6 & porque el arte es en conpara con delas cosas \textbf{ que se pueden fazer } e non requiere reglamiento derecho dela uoluntad . & Nam ars est respectu factibilium , \textbf{ et non praesupponit rectitudinem voluntatis : } Prudentia vero est respectu agibilium , \\\hline
1.2.6 & mas la pradençia es en conparaçion delas cosas \textbf{ que se han de fazer e de obranr . } Et requiere e demanda reglamiento derecho dela uoluntad . & et non praesupponit rectitudinem voluntatis : \textbf{ Prudentia vero est respectu agibilium , } et praesupponit rectitudinem appetitus : \\\hline
1.2.6 & que sin uoluntad . \textbf{ Ca uoluntad faze el } pecado¶ & peius est peccare voluntarie , \textbf{ quam inuoluntarie . } Prudentia ergo , \\\hline
1.2.6 & que la pradençia es razon derecha de todas las obras \textbf{ que auemos de fazer } que requiere e demanda reglamiento de uoluntad ¶ & sic diffiniri potest , \textbf{ quam est recta ratio agibilium , } praesupponens rectitudinem voluntatis . \\\hline
1.2.7 & Et el Rey sin sabiduria \textbf{ por que non conosçe nin faz cuenta } si non destos bienes senssibles & prudentia carens , \textbf{ quia non cognoscet , } non reputabit nisi sensibilia bona , \\\hline
1.2.7 & e en estos bienes senssibles e menguados . \textbf{ Et por ende faze se tomador e robador del pueblo } e enssennorea con tirania & et sensibilibus bonis . \textbf{ Efficietur ergo depraedator populi , } dominabitur tyrannice , \\\hline
1.2.8 & non por que las pueda mudar . \textbf{ Ca esto ninguno non lo pie de fazer . } Mas conuiene al Rey de auer memoria delans cosas passadas & non quod possit praeterita immutare , \textbf{ quia nulli agenti hoc est possibile , } sed decet Regem habere praeteritorum memoriam , \\\hline
1.2.8 & por que delas cosas passadas \textbf{ sepa lo que ha de fazer en lo que ha de venir . } Mas por razon dela manera & ut ex actis praeteritis sciat \textbf{ quid agere debeat in futurum . } Ratione vero modi \\\hline
1.2.8 & commo en las praticas . \textbf{ Ca assi commo se fazen razones demostratiuas } para demostrar & quam in agibilibus . \textbf{ Nam sicut fiunt rationes , } ut demonstretur , \\\hline
1.2.8 & e conosçer que cosa es la uerdat bien \textbf{ assi se fazen razones praticas para enduzir alos omes } qual es aquel bien & quid sit verum cognoscendum : \textbf{ sic fiunt rationes , | ut persuadeatur } quid sit bonum prosequendum . \\\hline
1.2.8 & que pueden ser prinçipios \textbf{ e reglas para lo que ha de fazer ¶ } Et otrosi conuiene al Rey & esse Principia , \textbf{ et regulae agendorum . } Oportet autem quod sit rationalis , \\\hline
1.2.8 & e por aquellos prinçipios \textbf{ que es aquello que le conuiene de fazer . } pues que assi es & speculando ex illis regulis \textbf{ quid agere congruit . } Sicut ergo ratione bonorum \\\hline
1.2.9 & e en estudiando \textbf{ e folgando se faze sabio ¶ } pues que assi es los Reyes & Anima in sedendo , \textbf{ et quiescendo sit prudens , } ut vult Philosophus 7 Physicorum . \\\hline
1.2.9 & Pues que assi es los Reyes \textbf{ e los prinçipes se pueden fazer sabios } por quatro maneras ¶ & ut non impediantur in regimine regni sui . \textbf{ Seipsos ergo poterunt prudentes facere , } ut naturaliter regnum regant : \\\hline
1.2.9 & Et por las cosas que passaron pueden saber \textbf{ con mon han de fazer } en lo que ha de venir . & ex quibus scire poterunt , \textbf{ quid agendum sit in futurum . } Nam semper debet suum regimen conformare regimini retroacto , \\\hline
1.2.9 & predeessores son mas sabios en todas aquellas cosas \textbf{ que han de fazer ¶ } La segunda manera es esta & sic Reges et Principes cogitando acta suorum praedecessorum , \textbf{ fiunt magis prudentes in agibilibus . } Secundo debent diligenter intueri futura bona , \\\hline
1.2.9 & e delas leyes \textbf{ e delas costunbres les faze auer manera } para bien gouernar . & ø \\\hline
1.2.9 & Et por ende tanto deue el Rey ser mas acuçioso cerca \textbf{ lo que deue fazer } quanto mas buean s leyes & Tanto ergo Rex magis intelligens est \textbf{ circa agibilia , } quanto plures bonas leges , \\\hline
1.2.9 & por las quales puede saber \textbf{ que ha de fazer en cada negoçio¶ } La quarta manera es que muchas e muchas uezes deue cuydar en qual manera ahun & ex quibus scire potest , \textbf{ quid in quolibet negotio sit agendum . } Quarto saepe saepius excogitare debet , \\\hline
1.2.9 & para todas las cosas \textbf{ que ha de fazer . } Ca non abasta seer entendido & eliciendo ex eis debitas conclusiones agibilium . \textbf{ Non enim sufficit esse intelligentem , } habendo cognitionem legum , \\\hline
1.2.9 & para obrar en las cosas \textbf{ que ha de fazer } si el non fuere razonable & quae sunt principia agibilium : \textbf{ nisi quis sit rationalis , } ex illis legibus , \\\hline
1.2.9 & e de aquellas costunbres razones e conclusiones conuenibles \textbf{ para lo que ha de fazer } la qual cosa & conclusiones agibilium eliciendo . \textbf{ Quod quomodo fieri debeat , } in tertio Libro , \\\hline
1.2.9 & la qual cosa \textbf{ commo se ha de fazer mostrar lo hemos mas conplidamente en el terçero libro } do diremos del gouernamiento del regno¶ & Quod quomodo fieri debeat , \textbf{ in tertio Libro , } ubi agetur de regimine regni , \\\hline
1.2.9 & e ala sabiduria \textbf{ delas quales fablamos ya en el capitulo soƀ dicho podran fazer assi mismos sabios } Mas por que la malicia es & quae ad prudentiam requiruntur , \textbf{ de quibus in praecedenti capitulo fecimus mentionem , | poterunt seipsos prudentes facere . } Verum quia malitia est corruptiua principii . \\\hline
1.2.9 & por maliçia es ciego en el entendimiento \textbf{ e en la razon por que iudge mal en lo que ha de fazer } Ca alas vezes & et deprauatam voluntatem , excoecatur in intellectu , \textbf{ ut male iudicet de agibilibus : } iudicat enim esse agendum \\\hline
1.2.9 & Ca alas vezes \textbf{ iudgaque ha de fazer aquello que deuia escusar } e alas vezes el contrario¶ pues que assi es bien dicho es & ut male iudicet de agibilibus : \textbf{ iudicat enim esse agendum } quod est fugiendum , \\\hline
1.2.9 & e que non ayan uoluntad mala nin desordenada \textbf{ por que por la maliçia dela uoluntad fagan las cosas sin razon } Et que non yerren en el iuyzio & et non habere voluntatem deprauatam : \textbf{ ne propter malitiam appetitus , imprudenter agant , } et iudicent esse agenda , \\\hline
1.2.9 & Judgando \textbf{ que han de fazer aquello } que deuien escusar & ne propter malitiam appetitus , imprudenter agant , \textbf{ et iudicent esse agenda , } quae sunt fugienda . \\\hline
1.2.10 & en el primero libro dela grand ph̃ia moral . \textbf{ La ley manda fazer las obras de todas las uirtudes . } Ca manda la ley obrar obras fuertes e obras tenpradas . & Sed ( ut dicitur primo Magnorum Moralium ) \textbf{ lex praecipit actus omnium virtutum . } Praecipit enim lex operari fortia et temperata , \\\hline
1.2.10 & Otrosi manda la ley \textbf{ que non faga luxuria } la qual cosa pertenesçe ala tenperança . Et otrosi manda non ferir nin contender & quod spectat ad fortitudinem . \textbf{ Et praecipit non moechari , } quod pertinet ad temperantiam . \\\hline
1.2.10 & la qual cosa pertenesçe ala tenperança . Et otrosi manda non ferir nin contender \textbf{ nin fazer tuerto a otro } que son obras de manssedunbre . & Et non percutere , \textbf{ neque contendere , } quae sunt opera mansuetudinis . \\\hline
1.2.10 & que son obras de manssedunbre . \textbf{ Et por ende la ley generalmente manda fazer e conplir todas las uirtudes } e esq̉uarton dos los males & quae sunt opera mansuetudinis . \textbf{ Lex igitur uniuersaliter iubet | omnem virtutem implere , } et malitiam fugere . \\\hline
1.2.10 & en alguno manera toda uirtud \textbf{ por que manda fazer las obras de todas las uirtudes . } Enpero conuiene de saber & propter quod legalis Iustitia dicta est \textbf{ quodammodo omnis virtus , quia exercet opera omnium virtutum . } Non est autem simpliciter legalis Iustitia omnis virtus , \\\hline
1.2.10 & Mas esta iustiçia legal departese de cada vna dela sotras uirtudes en dos cosas . \textbf{ Ca commo quier que el iusto legal faga essas mismas obras } que faze el fuerte e el tenprado . & a qualibet virtute in duobus : \textbf{ nam licet eadem opera agat Iustus legalis , } quae agit fortis , et temperatus : \\\hline
1.2.10 & Ca commo quier que el iusto legal faga essas mismas obras \textbf{ que faze el fuerte e el tenprado . } Enpero non las faze segunt aquella enteçion & nam licet eadem opera agat Iustus legalis , \textbf{ quae agit fortis , et temperatus : } non tamen aget ea \\\hline
1.2.10 & que faze el fuerte e el tenprado . \textbf{ Enpero non las faze segunt aquella enteçion } nin segunt aquella manera en que las fazen los otros . & quae agit fortis , et temperatus : \textbf{ non tamen aget ea | secundum eandem intentionem , } vel secundum eandem rationem formalem . \\\hline
1.2.10 & Enpero non las faze segunt aquella enteçion \textbf{ nin segunt aquella manera en que las fazen los otros . } Ca aquel que faze las obras fuertes & secundum eandem intentionem , \textbf{ vel secundum eandem rationem formalem . } Nam qui agit opera fortia , \\\hline
1.2.10 & nin segunt aquella manera en que las fazen los otros . \textbf{ Ca aquel que faze las obras fuertes } en quanto se delecta en ellas es dicho fuerte . & vel secundum eandem rationem formalem . \textbf{ Nam qui agit opera fortia , } quia delectatur in talibus , \\\hline
1.2.10 & en quanto se delecta en ellas es dicho fuerte . \textbf{ Et el que faze las obras tenpradas } en quanto se delecta en ellas es dicho tenprado . & fortis est , \textbf{ et agens temperata , } quia delectatur in ipsis , temperatus est . \\\hline
1.2.10 & en quanto se delecta en ellas es dicho tenprado . \textbf{ Mas aquel que faze estas obras } non en quanto se deleyte en ellas & quia delectatur in ipsis , temperatus est . \textbf{ Sed agens talia , } non quia delectatur in eis , \\\hline
1.2.10 & non en quanto se deleyte en ellas \textbf{ mas en quanto las manda fazer la ley } e el quiere conplir la ley es dicho iusto legal . & non quia delectatur in eis , \textbf{ sed quia ea lex praecipit , } et vult implere legem , \\\hline
1.2.10 & por otra entençion en \textbf{ quanto faziendo sus obras cunplen sa ley . } Et desta diferenços se sigue la segunda . & hoc est ex consequenti , \textbf{ prout agendo talia opera , | legem implet . } Ex ista autem differentia sequitur secunda : \\\hline
1.2.10 & si en las obras de sus uirtudes estas uirtudes \textbf{ fazen acabados } aquellos que las han en quanto son tales uirtudes . & in operibus talium virtutum , \textbf{ secundum se virtutes illae perficiunt habentem eas , } et ut est aliquis secundum se . \\\hline
1.2.10 & cunpliendo la ley la iustiçia legal \textbf{ non faze al omne acabado segunt si . } Pas faze lo acabado en quanto ha orden alos leyes . & prout implet legem , \textbf{ Iustitia legalis non perficit hominem secundum se , | sed perficit ipsum , } ut habet ordinem ad leges . \\\hline
1.2.10 & non faze al omne acabado segunt si . \textbf{ Pas faze lo acabado en quanto ha orden alos leyes . } Mas las leyes son fechas e dadas por enl prinçipe & sed perficit ipsum , \textbf{ ut habet ordinem ad leges . } Leges autem traduntur ab ipso Principe , \\\hline
1.2.10 & Et por ende la iustiçia legal \textbf{ commo quier que faga aquellas mismas obras } que faze la tenperança e la fortaleza & Iustitia ergo legalis licet \textbf{ faciat illa eadem opera , } quae facit Temperantia , et Fortitudo : \\\hline
1.2.10 & commo quier que faga aquellas mismas obras \textbf{ que faze la tenperança e la fortaleza } Enpero acaba aquel que la ha en orden a otro & faciat illa eadem opera , \textbf{ quae facit Temperantia , et Fortitudo : } perficit tamen ipsum habentem in ordine ad alium , \\\hline
1.2.10 & Et non se determina a ninguna manera espeçial \textbf{ mas faze las obras de cada vna delas uirtudes sp̃ales ¶ } Mas la iustiçia igual non es toda uirtud & quodammodo omnis virtus , \textbf{ et quod non determinat sibi specialem Iustitiam , sed agit opera specialium virtutum . } Iustitia vero aequalis \\\hline
1.2.10 & Mas la iustiçia igual non es toda uirtud \textbf{ nin faze las obras de cada vna delas uirtudes singulares } mas determina se amateria sp̃al en quanto en ella entiende algun bien span l . & non est omnis virtus , \textbf{ nec agit opera singularum virtutum , } sed determinat sibi specialem Iustitiam , \\\hline
1.2.10 & nin qualquier maldat de vn çibdadano \textbf{ non faze malo a otro cibdadano } Mas si en estos bien es de fuera alguno fuere malo & nec quaecunque prauitas unius ciuis \textbf{ per se loquendo infert malum alteri ciui . } Sed si in bonis exterioribus aliquis malus sit , \\\hline
1.2.10 & e cerca todas las obras delas uirtudes non tomando las segunt si . \textbf{ Mas en quanto por ellas es fecho conplimiento de ley¶ } Lo terçero ya es mostrado & non secundum se accepta , \textbf{ sed prout per ea est impletio legis . } Tertio ostensum est \\\hline
1.2.11 & nin toma ninguna parte dela iustiçia legal . \textbf{ esto es lo que los faze ser enteramente e coplidamente malos . } Mas assi como dize el philosofo enl de . x . qunto libro delas ethicas enl capitulo dela manssedunbre . & in aliquo legalem Iustitiam , \textbf{ est eos esse integre et perfecte malos . } Sed ( ut dicitur Ethic’ 4 cap’ de mansuetudine ) \\\hline
1.2.11 & maguer que cada vna desigualdat le enflaquezca \textbf{ e le faga enfermar . } Bien assi cada vna mengua de iustiçia & tamen quaelibet inaequalitas aegrotat , \textbf{ et infirmat ipsum : } sic non quaelibet Iniustitia \\\hline
1.2.11 & Ca partida el alma del cuerpo el \textbf{ cuerposedes faze } e se corronpe Bien & quia recedente ea , \textbf{ corpus dissoluitur , } et marcescit : \\\hline
1.2.11 & e conprehende las çibdades e los regnos . \textbf{ Ca sin ella la çibdat se desfaze } e los regnos non pueden estar . & et regna , \textbf{ quia sine ea dissoluitur ciuitas , } et non possunt regna subsistere . \\\hline
1.2.11 & Pues que assi es el que non guarda la iustiçia \textbf{ grant tuerto faze al regno e al rey . } Et si la iustiçia es tanto bien del rey & qui Iustitiam non obseruet . \textbf{ Si igitur Iustitia est tantum bonum Regis , et regni , } summo opere Rex studere debet , \\\hline
1.2.11 & Ca a qual quier que fuesse negada la iustiçia \textbf{ magnifiestamente seria fecho tuerto al Rey e al regno } su gregnos & Cuicumque enim negetur Iustitia , \textbf{ manifeste Regi et Regno infertur iniuria . } Satis per praecedens capitulum \\\hline
1.2.12 & Ca si la ley es regla de todas las obras \textbf{ que auemos de fazer } assi commo dize el philosofo & Nam si lex est regula agendorum , \textbf{ ut haberi potest } ex 5 Ethicor’ ipse iudex , \\\hline
1.2.12 & en aquellas cosas \textbf{ que se deuen fazer . } Ca el rey o el prinçipe es vna ley & cuius est leges ferre , \textbf{ debet esse quaedam regula in agendis . } Est enim Rex siue Princeps quaedam lex , \\\hline
1.2.12 & e egual e el Rey sea vna ley animada e vna regla . \textbf{ animada de todo lo que le ha de fazer . Paresçe de parte dela persona del Rey } que conuiene mucho al Rey & et quaedam animata regula agendorum , \textbf{ ex parte ipsius personae regiae } maxime decet \\\hline
1.2.12 & Ca la iustiçia es vn muy grant bien e muy resplandeçiente . \textbf{ por que assi commo dicho es faze al omne acabado en orden a otro . } Mas estonçe lanr̃a bondat resplandeçe mucho & Perficit enim \textbf{ ( ut dictum est ) | hominem in ordine ad alium . } Tunc autem maxime clarescit bonitas nostra , \\\hline
1.2.12 & que cada vna cosa es acabada \textbf{ enssi quando puede fazer otra tal commo si . } Et quando la su obra se estiende alos otros & quod unumquodque perfectum est , \textbf{ cum potest sibi simile producere , } et cum actio sua ad alios se extendit : \\\hline
1.2.12 & por que deue ser regla de todas las cosas \textbf{ que se deuen fazer en el regno . } lo otro por quela iustiçia es muy clara uirtud . & Decet ergo Reges et Principes esse iustos , \textbf{ tum quia debent esse regula agendorum , } tum quia Iustitia est praeclara virtus , \\\hline
1.2.13 & por la qual seamos reglados en las obras \textbf{ que auemos de fazer ¶ } pues que assi es commo los omes alguas vezes puedan & per quam regulentur in agendo . \textbf{ Cum igitur circa timores , } et audacias contingat \\\hline
1.2.13 & por que por ellos non sea el omne rerenido \textbf{ nin enbargado de fazer } e acometer aquellas cosas & Reprimit enim Fortitudo timores , \textbf{ ne per eos quis retrahatur ab eo , } quod ratio dictat . \\\hline
1.2.13 & Et lo otro por que ymaginamos \textbf{ que tales periglos fazen grant dolor } ¶ & sicut ex tactu gladii . \textbf{ Pericula ergo bellica | tum quia manifesta sunt , tum etiam , } quia maxime sentiuntur , \\\hline
1.2.13 & por que son mas manifiestos \textbf{ nin por que fazen mayor dolor mas ahun } por que ymaginamos & quia sunt magis manifesta , \textbf{ et doloris illatiua , } sed quia imaginamur \\\hline
1.2.13 & ymaginamos la muerte forcada \textbf{ Ca la muete y dada en la batalla fazese } por taiamiento de mienbros & quia per ea maxime apprehendimus mortem violentam , \textbf{ cum mors ibi illata sit } per mutilationem membrorum , \\\hline
1.2.13 & que han de venir \textbf{ ¶ lo terçero esto es mas guaue cosa por que acometer puede se fazer } adesora mas sofrir requiere mas luengotron . & quam contra mala futura . \textbf{ Tertio hoc est difficilius , | quia aggredi potest fieri subito : } sed sustinere requirit diuturnitatem , et tempus . \\\hline
1.2.13 & Pues que assi es fincanos de declarar \textbf{ en qual manera podemos fazer anos mismos fuertes } Pues que assi es deuen dos notar e entender que commo quier que la uirtud sea contraria . & restat ergo declarandum , \textbf{ quomodo possumus facere nos ipsos fortes . } Notandum ergo , \\\hline
1.2.13 & por que menos contradize ala fortaleza que el temor \textbf{ Et por ende si quisieremos fazer fuertes a nos mismos } conuiene de inclinar nos ante ala osadia & quam timor , \textbf{ si volumus nos ipsos facere fortes . } Declarata ergo sunt illa tria , \\\hline
1.2.13 & ¶ Lo terçero ya declaramos \textbf{ en qual manera podemos fazer a nos mismos fuertes . } Ca mayormente nos podemos fazer fuertes & Tertio declaratum fuit , \textbf{ quomodo possumus facere nos ipsos fortes : } quia maxime hoc faciemus , declinando magis ad audaciam , \\\hline
1.2.13 & en qual manera podemos fazer a nos mismos fuertes . \textbf{ Ca mayormente nos podemos fazer fuertes } si mas nos inclinaremos ala osadia que al temor . & quomodo possumus facere nos ipsos fortes : \textbf{ quia maxime hoc faciemus , declinando magis ad audaciam , } quae non tantum repugnat fortitudini , \\\hline
1.2.14 & e ordenando pena alos que fuyen \textbf{ e faziendo cauas e cercas . } por que non puedan foyr las conpannas & statuentes poenam fugientibus , \textbf{ faciendo foueas , } ne possit exercitus fugere , \\\hline
1.2.14 & quebranto todas las naues . \textbf{ Et assi fizo lidiar } e vençer a los suyos . & ne aliquis de suo exercito haberet materiam fugiemdi , \textbf{ omnes naues confregit . } Tertia fortitudo dicitur militaris , \\\hline
1.2.14 & en el libro del fecho dela caualleria \textbf{ ninguno non duda de fazer } e de acometer aquello que ha usado & Nam ( ut dicit Vegetius in libro De re militari ) , \textbf{ Nullus attentare dubitat , } quod se bene didicisse confidit . \\\hline
1.2.15 & mas ahun fuyendo dellas . \textbf{ Ca aquel que del todo faze abstinençia del comer e del beuer } e delas otras delecta connes corporales & sed etiam eas fugiendo . \textbf{ Nam qui adeo abstineret | a cibo et potu , } et a licitis delectationibus , \\\hline
1.2.15 & que can al omne \textbf{ e se fazen } por el seso del gostar & Nam delectationes nutrimentales \textbf{ quae fiunt per gustum , } ordinantur ad conseruationem propriae personae : \\\hline
1.2.15 & Mas las delecta conns del matermonio \textbf{ que se fazen } por el tannimiento & ordinantur ad conseruationem propriae personae : \textbf{ sed delectationes matrimoniales } quae fiunt per tactum , ordinantur ad procreationem , filiorum , \\\hline
1.2.15 & que en las obras delas viandas de quanoscamos . \textbf{ O por auentura esto se fizo la natura } por que cada vno de nos es mas cuydados & quam in operibus nutrimenti . \textbf{ Vel forte hoc ideo natura fecit : } quia unusquisque magis solicitus est \\\hline
1.2.15 & que por la tenpranca refrenemos las delectaçiones carnales \textbf{ que se fazen por el tannimiento } que las delecta connes delas viandas & refraenemus delectationes venereas \textbf{ quae fiunt per tactum , } quam nutrimentales quae fiunt per gustum . \\\hline
1.2.15 & que las delecta connes delas viandas \textbf{ que se fazen por el gostar . } Ante digo que en aquellas & quae fiunt per tactum , \textbf{ quam nutrimentales quae fiunt per gustum . } Immo in ipsis delectationibus nutrimentalibus \\\hline
1.2.15 & e tomaua muy grant delectaçion en ellas g̃rago a dios \textbf{ quel feziese la garganta } mas luenga que garganta de grulla & nomine Phyloxenus , \textbf{ qui , cum esset pultiuorax , orauit , } ut guttur eius longius quam gruis fieret . \\\hline
1.2.15 & mas non rogo \textbf{ quel feziese la lengua mas luenga que lengua de bue } porque mas se delectase por el gusto . & Non enim orauit , \textbf{ ut lingua eius esset latior lingua bouis , } ut magis delectaretur per gustum , \\\hline
1.2.15 & Et estas cosas vistas \textbf{ que dichas son de ligero paresçe commo nos mismos nos podemos fazer tenprados . } Ca la tenpranca e la fortaleza se ha & His visis de leui patet , \textbf{ quomodo nosipsos facere possumus temperatos . } Nam Temperantia , \\\hline
1.2.15 & Et pues que assi es assi conmo la fortaleza mas conuiene con la osadi \textbf{ Et si nos quisieremos fazer nos fuertes } mas auemos aser osados e temerosos . & magis conuenit cum audacia ; \textbf{ et si volumus esse fortes , } debemus magis esse audaces , \\\hline
1.2.15 & que con la senssiblidat de los sesos . \textbf{ Et por ende si nos quisieremos fazer a nos mismos tenprados deuemos } declinara aquella parte & sic Temperantia plus conuenit cum insensibilitate . \textbf{ Si ergo volumus nosipsos facere temperatos , } ad illam partem declinandum est , \\\hline
1.2.15 & ¶Lo quarto declaramos \textbf{ en qual manera podemos fazer a nos mismos tenprados . } Ca esto podemos fazer mayormente & Quarto vero declaratum fuit , \textbf{ quomodo possumus nosipsos facere temperatos : } quia hoc maxime faciemus \\\hline
1.2.15 & en qual manera podemos fazer a nos mismos tenprados . \textbf{ Ca esto podemos fazer mayormente } si nos guardaremos e arredraremos de las cosas delectables de los sesos . & quomodo possumus nosipsos facere temperatos : \textbf{ quia hoc maxime faciemus } a delectactionibus abstinendo . \\\hline
1.2.15 & Ca segunt el philosofo \textbf{ en el segundo delas ethicas deuemos fazer lo que fezieron los bieios de troya } contra & secundum Philosophum Ethicorum 2 hoc pati , \textbf{ quod senes Troiae patiebantur , } ad Helenam dicentes : \\\hline
1.2.16 & tanto es mas de denostar \textbf{ Otrosi quando alguno mas ligeramente puede fazer bien } e non lo faze mas es de denostar e de reprehender . & tanto magis est increpandus . \textbf{ Rursus quanto aliquis facilius potest benefacere , } si non benefaciat , \\\hline
1.2.16 & Otrosi quando alguno mas ligeramente puede fazer bien \textbf{ e non lo faze mas es de denostar e de reprehender . } Et por ende el que no es tenprado es mas de denostar e de reprehender & Rursus quanto aliquis facilius potest benefacere , \textbf{ si non benefaciat , | magis est detestandus , } et reprehensibilis . \\\hline
1.2.16 & Lo vno por que peta mas de uoluntad \textbf{ ¶Lo otro por que mas ligeramente puede bien fazer e ganar tenprança que fortaleza . } Mas que el & tum quia magis voluntarie peccat , \textbf{ tum etiam quia facilius est | ei facere bonum , } et acquirere temperantiam , \\\hline
1.2.16 & Mas fuyr e temer es cosatste . \textbf{ Et mas de uoluntad faze cada vno } lo que faze con & fugere autem et timere , est tristabile . \textbf{ Magis quis voluntarie agit } quod facit cum delectatione , \\\hline
1.2.16 & Et mas de uoluntad faze cada vno \textbf{ lo que faze con } delectaçion que lo que faze con tͥsteza . & Magis quis voluntarie agit \textbf{ quod facit cum delectatione , } quam quod facit cum tristitia . \\\hline
1.2.16 & lo que faze con \textbf{ delectaçion que lo que faze con tͥsteza . } Et por ende el que peca por & quod facit cum delectatione , \textbf{ quam quod facit cum tristitia . } Peccans igitur per intemperantiam , \\\hline
1.2.16 & por otra razon . \textbf{ Ca asi commo dize el philosofo el temor faze al omne acometido e faze al omne } que se non puede mouer & Secundo hoc idem patet : \textbf{ quia ( ut ait Philosophus ) } timor obstupefacit , \\\hline
1.2.16 & e que \textbf{ finque espantado la qual cola non fazela delectaçion } por es descenpramiento & et reddit naturam immobilem , et attonitam : \textbf{ quod non facit delectatio per intemperantiam . } Sed quando aliquis est in stupore , \\\hline
1.2.16 & Mas quando alguno esta acometido \textbf{ e esta fuera de ssi non faz aquello que faze por uoluntad ñcon delibramiento . } Et por ende mas de foyr & Sed quando aliquis est in stupore , \textbf{ est quasi extra se , | nec voluntarie et deliberate agit quod agit . } Tolerabilius est igitur peccare per timorem , \\\hline
1.2.16 & que el temeroso \textbf{ porque mas de uoluntad faze mal que el temeroso Otrosi por que mas ligeramente puede bien fazer } e ganar & quam timidum : \textbf{ quia magis voluntarie male agit . } Sic etiam dupliciter potest \\\hline
1.2.16 & destenprado es mas de denostar \textbf{ por que mas ligeramente puede fazer bien . } Ca mas ligeramente puede ganar la tenprança que el temeroso la fortaleza & ostendi ipsum esse magis increpandum : \textbf{ quia facilius potest benefacere . | Facilius enim potest } acquiri temperantia , \\\hline
1.2.16 & delecta connes \textbf{ senssibles puede se fazer sin todo periglo } mas acometer las cosas espantables & retrahi autem a delectationibus , \textbf{ potest fieri sine omni periculo : } sed aggredi terribilia , \\\hline
1.2.16 & mas acometer las cosas espantables \textbf{ e puar las batallas non se puede fazer sin periglo . } Et pues que assi es mucho es de denostar el & sed aggredi terribilia , \textbf{ et experiri bellum , sine periculo non potest . } Valde est ergo increpandus carens tempesantia , \\\hline
1.2.16 & mas non es assi dela fortaleza . \textbf{ por que cada vn acometemiento de batallas non nos faze fuertes } saluo si fuessen en aquellas batallas derechureras . & Non autem sic est de fortitudine : \textbf{ nam non quaelibet aggressio | bellorum facit nos fortes , } nisi bella ista sint iusta . \\\hline
1.2.16 & non le contesçra \textbf{ que faga vna batalla derechurera . } Et pues que asi es non nos podemos & Sed forte toto tempore vitae hominis , \textbf{ non occurrit ei unum iustum bellum . } Non ergo sic possumus assuefieri \\\hline
1.2.16 & e por escͥptos enbiaua todas sus razones alos ricos omes e alos prinçipe ᷤ en que les mandaua \textbf{ lo que auian de fazer } Et acaesçio que vn prinçipe mucho su priuado que grant t p̃o le auia seruido e fiel mente . & et per literas mittebat Baronibus et Ducibus , \textbf{ quid vellet eos facere . | Accidit autem , } quod , cum quidam Dux exercitus diu ei seruiuisset , \\\hline
1.2.16 & Et acaesçio que vn prinçipe mucho su priuado que grant t p̃o le auia seruido e fiel mente . \textbf{ Et el Rey que tiendo fazer plazer a aquel } prinçipe mando qual pusiessen dentro ante si . & et fideliter , \textbf{ Rex ille volens complacere illi Duci , } praecepit quod duceretur ad ipsum . \\\hline
1.2.16 & que los que se dan alas delecta connes dela carne son menospreçiados \textbf{ Mas por que los destenprados fazen tuerto alos otros en las perssonas } que les son muy ayuntadas & contemnuntur . \textbf{ Immo quia intemperati iniuriantur aliis , } in personis maxime coniunctis , \\\hline
1.2.17 & Et esto en qual manera se deue entender adelante lo mostrͣemos¶ \textbf{ pues que assi es en faziendo espenssas } contesçe alas vezes de fallesçer & in prosequendo patebit . \textbf{ Si igitur in faciendo sumptus conuenit deficere , } quod facit auaritia : \\\hline
1.2.17 & contesçe alas vezes de fallesçer \textbf{ e esto faze la auariçia . } Et contesçe alas vezes de sobrepuiar e dar . & Si igitur in faciendo sumptus conuenit deficere , \textbf{ quod facit auaritia : } et superabundare , \\\hline
1.2.17 & Et contesçe alas vezes de sobrepuiar e dar . \textbf{ mas que conuiene e esto faze el gastamiento . } Et por que cada vna destas cosas & et superabundare , \textbf{ quod facit prodigalitas , } quia utrunque est \\\hline
1.2.17 & dessegunt si mas \textbf{ por quelos orden e para fazer } espenssas quales deue fazer . & non diligat pecuniam secundum se , \textbf{ sed ut eam ordinet } ad debitos sumptus : \\\hline
1.2.17 & por quelos orden e para fazer \textbf{ espenssas quales deue fazer . } Enpero para que pueda fazer & sed ut eam ordinet \textbf{ ad debitos sumptus : } tamen ut possit debitos sumptus facere , \\\hline
1.2.17 & espenssas quales deue fazer . \textbf{ Enpero para que pueda fazer } espenssas quales deue fazer & ad debitos sumptus : \textbf{ tamen ut possit debitos sumptus facere , } non debet proprios redditus inaniter dispergere . \\\hline
1.2.17 & Enpero para que pueda fazer \textbf{ espenssas quales deue fazer } non deue las suᷤ propias rentas esparzer & ad debitos sumptus : \textbf{ tamen ut possit debitos sumptus facere , } non debet proprios redditus inaniter dispergere . \\\hline
1.2.17 & mas deue auer cuydado de su fazienda \textbf{ e delas sus rentas propias e fazer dellas sus espenssas quales conuiene ¶ } Estas tres cosas son aquellas en que ha de ser la franqueza & habere debitam curam de propriis , \textbf{ et ex eis debitos sumptus facere : } sunt illa tria circa quae videtur esse liberalitas . \\\hline
1.2.17 & mas non ha de ser çerca estas tres cosas egualmente e prinçipal mente . \textbf{ Ca primeramente en espender e en fazer } espenssas quales deuees la franqueza prinçipalmente e primero . & Non autem est circa haec tria aeque principaliter et primo . \textbf{ Nam circa expendere | et circa debitos sumptus facere , } est liberalitas principaliter , et primo . \\\hline
1.2.17 & nin libal por que gana de los \textbf{ amigosa los quales le conuenia bien fazer . } Et por ende la franqueza es en non tomar & lucratur enim ab amicis , \textbf{ quibus oportet bene facere . } Est igitur liberalitas \\\hline
1.2.17 & e tomando onde deue esto \textbf{ por tanto lo faze por que pueda fazer espessas quales deuede sus rentas prop̃as } e por que pueda espender & et accipiens unde debet , \textbf{ hoc ideo facit , | ut possit ex propriis redditibus debitos sumptus facere , } et expendere sicut debet : \\\hline
1.2.17 & Et por ende de razon paresçe \textbf{ que mas prinçipalmente es la franqueza en espender e en fazer bien alos otros . } Et despues desto es en guardar las sus rentas propreas & merito liberalitas principalius est \textbf{ in expendendo | et in benefaciendo aliis ; } ex consequenti autem est \\\hline
1.2.17 & que la fraquanza es mas en espender \textbf{ e en bien fazer alos otros } que en guardar lo suyo mismo & liberalitatem magis esse circa expendere \textbf{ et circa beneficiare alios , } quam circa proprios redditus custodire . \\\hline
1.2.17 & Lo segundo esto mismo se praeua assi por que ala uirtud \textbf{ mas prinçipal parte nesçe de fazer mayor bien . } Et mayor bien es en bien fazer & Secundo hoc idem patet , \textbf{ quia ad virtutem principalius spectat | facere maius bonum . } Maius autem bonum est benefacere , \\\hline
1.2.17 & mas prinçipal parte nesçe de fazer mayor bien . \textbf{ Et mayor bien es en bien fazer } que enbien sofrir . & facere maius bonum . \textbf{ Maius autem bonum est benefacere , } quam bene pati , \\\hline
1.2.17 & e parte alos otros los bienes \textbf{ que tiene este faze bien } Mas el que guarda las rentas propias e lo suyo propio & Qui autem debite expendit \textbf{ et aliis dona largitur , benefacit . } Custodiens vero proprios redditus , \\\hline
1.2.17 & mas guardasse de mal obrar . \textbf{ Et pues que assi es si meior cosa es bien fazer } que non mal fazer & non operatur turpia . \textbf{ Si ergo melius est benefacere , } quam non malefacere , \\\hline
1.2.17 & Et pues que assi es si meior cosa es bien fazer \textbf{ que non mal fazer } o que bien sofrir & Si ergo melius est benefacere , \textbf{ quam non malefacere , } vel quam bene pati : \\\hline
1.2.17 & Et mayor loor e mayor honrra se le unata en bien espendiendo \textbf{ e en bien faziendo alos otros } que en guardando los bienes propios & in bene expendendo , \textbf{ et aliis benefaciendo , quam in custodiendo propria , } vel in non usurpando aliena . \\\hline
1.2.17 & mas prinçipalmente esta en despender commo deue \textbf{ e en bien fazer alos otros } que en guardar lo suyo & in debite expendendo , \textbf{ et benefaciendo aliis . } Quarto hoc idem patet : \\\hline
1.2.17 & ca la uirtud mas prinçipalmente es cerca lo mas guaue . \textbf{ Et mas guaue es dar los sus bienes e fazer bien alos otros } que guardar las sus rentas propias & circa difficilius . \textbf{ Difficilius autem est aliis dona tribuere , } quam proprios redditus custodire , \\\hline
1.2.17 & Mas es muy amado \textbf{ mas si faze conuenibles espensas de lo suyo propio } e parte e da de lo suyo grandes dones alos buenos & Sed maxime diligitur \textbf{ si ex propriis redditibus debitos sumptus faciat , } et bonis et dignis magna dona tribuat . \\\hline
1.2.17 & Et assi de ligero paresçe \textbf{ en qual manera podemos fazer a nos mismos liberales e francos . } Ca bien conmo la fortaleza mas & de leui patet , \textbf{ quomodo nos possumus facere liberales . } Nam sicut quia fortitudo plus opponitur \\\hline
1.2.17 & comtradize al miedo que ala osadia . \textbf{ Et nos fazemos a nos mismos fuertes declinando ala osadia } assi que seamos mas osados que temerosos . & timori quam audaciae , \textbf{ facimus nosipsos fortes , | declinando ad audaciam ; } ita quod potius plus audeamus , \\\hline
1.2.18 & que menos dan de quanto les conuiene ᷤ dar \textbf{ Et menos fazen de quanto les conuiene de fazer . } Et desto puede bien paresçer & Semper ergo cogitare debent , \textbf{ quod minora faciunt , | quam deceat . } Ex hoc autem apparere potest \\\hline
1.2.18 & Ca quanto la auariçia es mas raygada en cada vno \textbf{ e quanto mas enuegesçe el omne tanto mas se faze auariento . } ¶ Et por ende si la cabesca del regno & quia quanto quis procedit in auaritia , \textbf{ et quanto plus senescit , | tanto magis auarus efficitur . } Si ergo caput regni , \\\hline
1.2.18 & por la qual razon commo el gastador non sea amador de los \textbf{ desbien commo el libal non lo es de ligero se puede fazer } qual quier gastador liberal e franco ¶ & Quare cum prodigus non sit amator pecuniae , \textbf{ sicut nec liberalis , | de leui } quis cum sit prodigus , \\\hline
1.2.18 & e alos prinçipes de ser liƀͣales . \textbf{ Et para ser libales conuiene les de bien fazer alos buenos } e por razon devien non & et propter inanem gloriam \textbf{ vel propter aliquam aliam causam . | Decet igitur Reges esse liberales : } et ut liberales sint , \\\hline
1.2.19 & Mas commo en cada cosa \textbf{ mas e menos non fagan departimiento en la naturaleza } e en la semeiança delas cosas & Sed cum magis , \textbf{ et minus non videantur | diuersificare speciem , } et naturam rerum , \\\hline
1.2.19 & que de razon de uirtudes \textbf{ que se a çerca bien grande e guaue de fazer . } por la qual cosa commo en las mayores espenssas sea fallada . & Sciendum ergo quod de ratione virtutis est , \textbf{ quod sit circa bonum , | et difficile . } quare cum in maioribus sumptibus reperiatur \\\hline
1.2.19 & de aquel \textbf{ que faze las espenssas . } Et pueden se conparar alas obras & Videlicet , ad facultates eius \textbf{ qui facit sumptus , } et ad opera \\\hline
1.2.19 & en que se espienden \textbf{ e se fazen las espenssas ¶ } Pues que assi es cerca los algos & et ad opera \textbf{ in quae sumptus illi expenduntur . } Erit ergo circa pecuniam duplex virtus . \\\hline
1.2.19 & non \textbf{ por si mas por conparaçion de aquel da e faze las espessas . } Et por ende el dador de muchas cosas & non secundum se , \textbf{ sed per comparationem ad dantem , | et ad facientem sumptus . } Ideo largitor multorum posset esse auarus , \\\hline
1.2.19 & de aquello \textbf{ que faze es nonbrada la magnificençia } ca es dichon magnifico aquel que faze grandes cosas & Ab ipsa enim factione , \textbf{ et ab ipso opere denominatur magnificentia . } Dicitur enim magnificus , \\\hline
1.2.19 & que faze es nonbrada la magnificençia \textbf{ ca es dichon magnifico aquel que faze grandes cosas } Et por ende por que en qual quier estado & et ab ipso opere denominatur magnificentia . \textbf{ Dicitur enim magnificus , | quasi magna faciens . } Inde est ergo , \\\hline
1.2.19 & si assi la libalidat es certa grandes espenssas \textbf{ si el que las faze ouiere muchas riquezas } e ahun sera cerca espenssas mesuradas si el que faze las espenssas & sic liberalitas est circa magnos sumptus , \textbf{ si faciens eos , | multas habeat facultates : } et etiam circa mediocres , \\\hline
1.2.19 & si el que las faze ouiere muchas riquezas \textbf{ e ahun sera cerca espenssas mesuradas si el que faze las espenssas } mesuradosmente abondare en las riquezas . & multas habeat facultates : \textbf{ et etiam circa mediocres , | si faciens sumptus illos , } mediocriter facultatibus abundet . \\\hline
1.2.19 & nin tiene oio quales sean las obras . \textbf{ Ca non es guaue cosa de fazer conuenibles espenssas } en quales se quier obras . & non respicit quaecunque opera : \textbf{ quia non est difficile facere decentes sumptus } in quibuscunque operibus , \\\hline
1.2.19 & Ca non entienden \textbf{ commo han de fazer las grandes obras } mas tienen mientes a poco espender . & quia non intendunt \textbf{ quomodo magna opera faciant , } sed quomodo parum expendant . \\\hline
1.2.19 & Mas ay otros que en las grandes obras \textbf{ fazen conuenibles espenssas } e tales son uirtuosos & Quidam vero in magnis operibus \textbf{ faciunt decentes sumptus : } et tales sunt virtuosi , \\\hline
1.2.19 & e tienpra los consumimientos e destruymientos . \textbf{ Et assi commo la libalidat faze espenssas } e da dones conuenibles a las riquezas & et moderans consumptiones . \textbf{ Et sicut liberalitas est faciens sumptus , } et dationes proportionatas facultatibus : \\\hline
1.2.19 & e da dones conuenibles a las riquezas \textbf{ assi la magnificençia faze espenssas conuenibles alas grandes obras ¶ } visto que cosa es la magnificençia & et dationes proportionatas facultatibus : \textbf{ sic magnificentia est faciens sumptus decentes magnis operibus . } Viso quid est magnificentia : \\\hline
1.2.19 & Et si cunplieren las sus riquezas \textbf{ deue fazer grandes eglesias e sac̀fiçios honrrados } et apareiamientos dignos & homo debet esse magnificus circa diuina , \textbf{ constituendo ( si facultates tribuant ) templa magnifica , sacrificia honorabilia , praeparationes dignas . } Ideo dicitur 4 Ethicorum , \\\hline
1.2.19 & en el quarto libro delas ethicas \textbf{ que el magnifico deue fazer honrradas espenssas en aquellas cosas } que parte nesçen a dios¶ & Ideo dicitur 4 Ethicorum , \textbf{ quod honorabiles sumptus , | quos debet facere magnificus , } sunt circa Deum . \\\hline
1.2.19 & si \textbf{ ouiereconplidamente las riquezas fazer espenssas conuenibles en toda la comunidat } por que los bienes comunes lon en alguna manera diuinales & ( si adsit facultas ) \textbf{ facere decentes sumptus | circa totam communitatem . } Nam ipsa bona communia \\\hline
1.2.19 & Ca en esto paresçe mayormente la magnificençia \textbf{ quando alguno faze grandes bienes } a aquellos que son mas dignos & Nam in hoc potissime apparet magnificentia , \textbf{ quando quis magna facit } iis qui sunt magis digni . \\\hline
1.2.19 & assi commo son grandes casas e grandes fortalezas \textbf{ o a qual las que se fazen pocas vezes en toda la uida del omne } assi commo son los casamientos e las caualłias . & cuiusmodi sunt domus , et aedificia . \textbf{ Vel quae fiunt raro in tota vita , } cuiusmodi sunt nuptiae , \\\hline
1.2.19 & e mas deue entender \textbf{ en qual manera deue fazer muuy marauillosas e muy durables las sus casas e las sus moradas } que en qual manera las fara sufisticas e aparesçientes e muy pintadas & et magis debet intendere quomodo facere \textbf{ debeat admirabiles , | et diuturnas domus , } quam quomodo faciat eas sophysticas et apparentes . \\\hline
1.2.19 & nin tan firmes en si . \textbf{ En essa misma manera conuiene al magnifico de fazer muy } honrradamente las sus bodas e las sus cauallerias & quam quomodo faciat eas sophysticas et apparentes . \textbf{ Sic etiam decet magnificum , nuptias , et militias , } et talia quae raro occurrunt , \\\hline
1.2.19 & e cerca quales cosas ha de ser ligeramente \textbf{ paresçe commo podemos fazer a nos mismos magnificos . } Ca assi commo la liberalidat & de leui patet , \textbf{ quomodo possumus | nosipsos magnificos facere . } Nam sicut liberalitas plus contrariatur auaritiae , \\\hline
1.2.19 & Et declinado al gastamiento \textbf{ podemos a nos mismos fazer libales e francos . } Assi la magnificençia es mas contraria ala pariuuficençia & Nam sicut liberalitas plus contrariatur auaritiae , \textbf{ quam prodigalitati : } sic magnificentia plus contrariatur paruificentiae , \\\hline
1.2.19 & Assi la magnificençia es mas contraria ala pariuuficençia \textbf{ que faze pequanas cosas } que al destruyemiento & sic magnificentia plus contrariatur paruificentiae , \textbf{ quam consumptioni . } Nos ergo ipsos \\\hline
1.2.19 & que al destruyemiento \textbf{ que las faze mayores que deue . } ¶ & quam consumptioni . \textbf{ Nos ergo ipsos } ( si facultas adsit ) \\\hline
1.2.19 & ¶ \textbf{ Pues que assi es nos faremos a nos mismos magnificos } si ouieremos de que declinando mas al gastar e al destroyr & Nos ergo ipsos \textbf{ ( si facultas adsit ) | faciemus magnificos , } declinando ad consumptionem , \\\hline
1.2.20 & que el paruifico fallesce en todas las cosas \textbf{ que ha de fazer } Et esto se praeua & Prima proprietas est , \textbf{ quia circa omnia deficit , } quod ex ipso nomine patet . \\\hline
1.2.20 & por que el parufico es ome \textbf{ que faze pequanas cosas e menguadas ¶ } La segunda propiendat es & Esse enim paruificum , \textbf{ est facere parua et defectiua . } Secunda proprietas est , \\\hline
1.2.20 & por vna dinarada de pimienta . \textbf{ Ca quando non fazen espenssas conuenibles en el grant conbite } esto non es si non por que quiere bazer pequanas el penssas & pro denariato piperis . \textbf{ Dum enim circa magnum conuiuium | non faciunt decentes sumptus , } volentes parcere modicae expensae , \\\hline
1.2.20 & que quales quier cosas \textbf{ que faze el paruifico } sienpre las faze tardando . & totum conuiuium indecens redditur . Tertia proprietas est , \textbf{ quod quaecunque facit paruificus , } semper facit tardans . \\\hline
1.2.20 & que faze el paruifico \textbf{ sienpre las faze tardando . } Ca paresçe leal paruifico & quod quaecunque facit paruificus , \textbf{ semper facit tardans . } Videtur enim ei , \\\hline
1.2.20 & Bien alłi puesto que el paruifico \textbf{ e al escasso sea dado de fazer grandes espenssas sienpre tarda } e fuye quanto puede que se no fagan ¶ & sic dato quod paruificum oporteat \textbf{ expensas facere , | tamen illos sumptus tardat , } et subterfugit quantum potest . \\\hline
1.2.20 & e al escasso sea dado de fazer grandes espenssas sienpre tarda \textbf{ e fuye quanto puede que se no fagan ¶ } La quarta propiedates que el paruifico non entiende & tamen illos sumptus tardat , \textbf{ et subterfugit quantum potest . } Quarta est , quia paruificus non intendit \\\hline
1.2.20 & La quarta propiedates que el paruifico non entiende \textbf{ en qual manera faga granada obra } e en qual manera faga sus dones granados e conuenibles & Quarta est , quia paruificus non intendit \textbf{ qualiter faciat magnum opus , } ut qualiter faciat debitas largitiones , \\\hline
1.2.20 & en qual manera faga granada obra \textbf{ e en qual manera faga sus dones granados e conuenibles } o en qual manera faga sus bodas conuenibles & qualiter faciat magnum opus , \textbf{ ut qualiter faciat debitas largitiones , } vel quomodo faciat decentes nuptias : \\\hline
1.2.20 & e en qual manera faga sus dones granados e conuenibles \textbf{ o en qual manera faga sus bodas conuenibles } mas toda su entençion es & ut qualiter faciat debitas largitiones , \textbf{ vel quomodo faciat decentes nuptias : } sed tota sua intentio est , \\\hline
1.2.20 & mas toda su entençion es \textbf{ en qual manera faga peannas espenssas . } por que la propiedat del paruifico es & sed tota sua intentio est , \textbf{ quomodo faciat paruos sumptus . } Est enim proprietas paruifici , \\\hline
1.2.20 & que la obra . \textbf{ por ende mayor acuçia pone en commo esperienda poco que en commo faga grand obra¶ } La quinto propiedat del pariufico es & Est enim proprietas paruifici , \textbf{ ut appretietur plus pecunia quam opus . } Quinta proprietas eius est , \\\hline
1.2.20 & assi en el apartamento del auer \textbf{ e de los desque se faze } por las espenssas que faze por que es ael & sic in separatione pecuniae , \textbf{ quae fit per expensas , } quia est ibi quasi quaedam diuisio continui , \\\hline
1.2.20 & e de los desque se faze \textbf{ por las espenssas que faze por que es ael } assi conmo vn taiamiento del cuerpo continuo & quae fit per expensas , \textbf{ quia est ibi quasi quaedam diuisio continui , } eo quod paruificus reputat suam pecuniam \\\hline
1.2.20 & La sexta propiendat es \textbf{ que quando el parufico faze alguna cosa } aparesçe ael & Sexta est , \textbf{ quia cum paruificus nihil faciat , } videtur tamen ei quod semper agat maiora , \\\hline
1.2.20 & aparesçe ael \textbf{ que sienpre faze mayores cosas } que deue por que & quia cum paruificus nihil faciat , \textbf{ videtur tamen ei quod semper agat maiora , } quod debeat . \\\hline
1.2.20 & e las espensas \textbf{ que faze } que son obras de uirtudes . & quam dationes , et sumptus , \textbf{ quae sunt opera virtutum . } Pecuniam enim reputat \\\hline
1.2.20 & que non preçia las obras \textbf{ que faze nin obras de uirtudes } e todo cuyda que es cosa vil & quid earum , \textbf{ opera virtutum aestimat } quid vile , \\\hline
1.2.20 & assi en essa misma manera non puede el \textbf{ parufico fazer despenssas tan pequanas } en qual si quier obra que faga que non le paresca sienpre a el & non potest paruificus \textbf{ ita modicum sumptum facere erga quodcunque opus , } quin semper videatur ei \\\hline
1.2.20 & parufico fazer despenssas tan pequanas \textbf{ en qual si quier obra que faga que non le paresca sienpre a el } que faze mayores espenssas que deua . & ita modicum sumptum facere erga quodcunque opus , \textbf{ quin semper videatur ei } quod agat maiora , \\\hline
1.2.20 & en qual si quier obra que faga que non le paresca sienpre a el \textbf{ que faze mayores espenssas que deua . } Por la qual cosa si cosa muy denostada es en la real magestad fallesçer e menguar en todas las cosas & quin semper videatur ei \textbf{ quod agat maiora , | quam debeat . } Quare si detestabile est \\\hline
1.2.20 & Et tardar sienpre en las cosas \textbf{ que ha de fazer } e non fazer ninguna cosa aꝑçebidamente e de uoluntad & magna bona pro modico perdere , \textbf{ semper tardare , } et nihil prompte facere , \\\hline
1.2.20 & que ha de fazer \textbf{ e non fazer ninguna cosa aꝑçebidamente e de uoluntad } e nunca entender en commo faga grandes obras de uirtud & semper tardare , \textbf{ et nihil prompte facere , | et nunquam intendere } quomodo faciat \\\hline
1.2.20 & e non fazer ninguna cosa aꝑçebidamente e de uoluntad \textbf{ e nunca entender en commo faga grandes obras de uirtud } mas penssar sienpre en commo espienda poco e fazer sienpre espenssa con tristeza e con dolor . & et nunquam intendere \textbf{ quomodo faciat | magna opera virtutum , } sed quomodo modicum expendat , \\\hline
1.2.20 & e nunca entender en commo faga grandes obras de uirtud \textbf{ mas penssar sienpre en commo espienda poco e fazer sienpre espenssa con tristeza e con dolor . } Et quando non faze ningunan cosa cree el & magna opera virtutum , \textbf{ sed quomodo modicum expendat , | semper facere sumptum } cum tristitia et dolore ; et cum nihil facit , \\\hline
1.2.20 & mas penssar sienpre en commo espienda poco e fazer sienpre espenssa con tristeza e con dolor . \textbf{ Et quando non faze ningunan cosa cree el } que faze grandescosas e grandes obras . & semper facere sumptum \textbf{ cum tristitia et dolore ; et cum nihil facit , } credere se magna operari , \\\hline
1.2.20 & Et quando non faze ningunan cosa cree el \textbf{ que faze grandescosas e grandes obras . } Et por que todas estas cosas ponen grand mengua en la Real magestad & cum tristitia et dolore ; et cum nihil facit , \textbf{ credere se magna operari , } quia omnia haec valde derogant regiae maiestati , \\\hline
1.2.20 & que el Rey sea periufico mas que conuengaal Rey de ser magnifico \textbf{ e de fazer grandes espenssas } conplidamente es prouado & Quod autem deceat \textbf{ ipsum esse magnificum , } sufficienter probant superiora dicta : \\\hline
1.2.20 & e los fijos auiendo moradas honrradas \textbf{ e faziendo bodas conuenibles e honrradas } e vsando de caualłias marauillosał & habendo habitationes honorabiles , \textbf{ faciendo nuptias decentes , } exercendo militias admirabiles . \\\hline
1.2.21 & Ca dixiemos de suso \textbf{ que conuenia al magnifico de fazer conuenientes espenssas en las grandes obras . } Mas conosçer en qual es grandes obras & Dicebatur enim spectare ad magnificum \textbf{ in magnis operibus facere decentes sumptus . } Cognoscere autem quibus magnis operibus \\\hline
1.2.21 & entendimiento¶ \textbf{ La segunda propiedat del magnifico es fazer grandes espenssas } non por que se muestre & nisi quis polleat scientia , et intellectu . \textbf{ Secunda proprietas magnifici , | est facere magnos sumptus , } non ut ostendat seipsum , \\\hline
1.2.21 & nin vana eglesia de los omes \textbf{ ¶ la terçera propiedat del magnifico es fazer espenssas e dardones } delectablemente et sin detenemiento . & et non fauorem , et gloriam hominum . \textbf{ Tertia proprietas magnifici , | est facere sumptus } et distributiones delectabiliter \\\hline
1.2.21 & delectablemente et sin detenemiento . \textbf{ Ca aquel que primeramente ante que faga las despenssas } quiere tomar cuenta dellas & et prompte . \textbf{ Nam qui prius quam sumptus faciat , } diu ratiocinatur , \\\hline
1.2.21 & en el quarto libro delas etris \textbf{ que la grant auariçia de tomar cuenta faze al omne de poca fazienda ¶ } La quarta propiedat es & Ideo dicitur 4 Ethic’ \textbf{ quod diligentia ratiocinii est paruifica . } Quarta , est magis intendere qualiter faciat \\\hline
1.2.21 & que el magnifico deue mas entender \textbf{ en qual manera faga obra muy buena } e muy conuenible que entender en qual manera & Quarta , est magis intendere qualiter faciat \textbf{ opus optimum , } et decentissimum , \\\hline
1.2.21 & e quanta despenssa fara en aquella obra . \textbf{ assi commo si el magnifico ouiese de fazer algun tenplo es alguna eglesia en honrra de dios } o ouiesse de dar e de partir alguons dones a personas dignis & ad opus illud : \textbf{ ut si debet | magnificus aliquod templum construere } in honorem diuinum , \\\hline
1.2.21 & Ca commo aquel sea magnifico \textbf{ que faze conuenibles espenssas enlas grandes obras } si faz conuenibles espenssas faz omne ser liberal fazer muy grandes & Cum enim ille sit magnificus , \textbf{ qui in magnis operibus facit decentes sumptus : } si facere decentes sumptus est \\\hline
1.2.21 & que faze conuenibles espenssas enlas grandes obras \textbf{ si faz conuenibles espenssas faz omne ser liberal fazer muy grandes } e muy conuenibles espenssas & qui in magnis operibus facit decentes sumptus : \textbf{ si facere decentes sumptus est | esse liberalem , } facere maximos decentes sumptus , \\\hline
1.2.21 & e muy conuenibles espenssas \textbf{ lo que faze el magnifico es ser mucho mas liberal ¶ } La sexta propiedat es del magnifico & facere maximos decentes sumptus , \textbf{ quos facit magnificus , | est esse maxime liberalem . } Sexta proprietas magnifici , \\\hline
1.2.21 & que de egual es \textbf{ penssa faga obra mas granada que otro ninguno . } Nas los paruficos e de pequana fazienda & est aequali sumptu facere \textbf{ opus magis magnificum . } Paruifici enim , \\\hline
1.2.21 & pequana cosa pierden lo mucho . \textbf{ Et pues que assi es el magnifico aqui parte nesçe de non auer cuydado de contar lo que despiende de despenssa egual faze mas granada obra e mas mognifica por que non perdona alas despenssas conuenibles . } Mas contesçe alas vegadas que el auariento despiende mas en alguna obra que el libal & ab aequali sumptu , \textbf{ facit opus magis magnificum , | quia non parcit decentibus sumptibus . } Contingit enim aliquando auarum \\\hline
1.2.21 & avn que sean pequeñas e de pequans valor . \textbf{ Por ende la obra en que despiende mucho non la faze commo conuiene . } Por la qual cosa dela espenssa egual o alguas vegadas dela & et parui valoris , \textbf{ ideo opus , | ubi multum expendit , } indecenter facit . Quare ab aequale sumptu , \\\hline
1.2.21 & e conosçedores quales despenssas a quales obras conuienen . \textbf{ Et aellos otrosi mucho mas pertenesçe de fazer grandes donaconnes } e lobre puiantes de espenssas & quibus operibus deceant . \textbf{ Ad eos autem maxime spectat | facere magnas largitiones , } et excellentes sumptus boni gratia \\\hline
1.2.21 & que otro ninguno \textbf{ e avn essa misma manera pertenesçe a ellos fazer despenssas muy delectablemente e sin detenimiento . } Ca assi commo dicho es de suso & in bonum dirigere . \textbf{ Sic etiam ad eos spectat delectabiliter , | et prompte sumptus facere . } Nam ( ut supra dicebatur ) \\\hline
1.2.21 & e en riquezas \textbf{ tantomas les conuiene aellos de fazer mayores particonnes e mayores dones } e mas espender delectable ment en sin detenimiento & et diuitiis , \textbf{ tanto magis decet | eos ampliores retributiones facere , } et magis delectabiliter , \\\hline
1.2.21 & e alos prinçipes de entender e cuydar \textbf{ mas en qual manera deuen fazer obras muy grandes de } uirtudesque cuydar en qual manera guarden los des & et Principes magis intendere , \textbf{ quomodo faciant excellentia opera virtutum , } quam quomodo parcant nummis et expensis . \\\hline
1.2.21 & e perdonen alas espenssas \textbf{ para las non fazer } Otrosi avn conuiene alos Reyes & ø \\\hline
1.2.21 & e alos prinçipes de ser liberales muy altamente \textbf{ e de fazer sienpre obras muy grandes e magnificas . } Et pues que assi es todas las propiedades del magnifico & Oportet etiam eos esse excellenter liberales , \textbf{ et semper facere magnifica opera . } Omnes igitur proprietates magnifici per amplius , \\\hline
1.2.21 & que non pue de cada vno ser magnifico \textbf{ por que non puede cada vno fazer grandes espenssas } Mas assi commo alli dize el philosofo tales son los nobles e los głiosos . & quod non quilibet possit esse magnificus : \textbf{ quia non quilibet potest | facere magnos sumptus . } Sed , ut ibidem dicitur , \\\hline
1.2.22 & que dessi son aptos e apareiados \textbf{ para fazer grandes cosas . } Et poderosos para vsar de cosas grandes e altas . & Videmus enim aliquos de se aptos ad magna , \textbf{ potentes magna et ardua exercere : } quadam tamen pusillanimitate ducti , \\\hline
1.2.22 & retrahen se destas cosas guanadas \textbf{ que poderan muy bien fazer . } Et por ende estos fallesçen en tales cosas & quadam tamen pusillanimitate ducti , \textbf{ retrahunt se ab huiusmodi magnis . } Tales ergo in talibus deficiunt . \\\hline
1.2.22 & Et por ende estos fallesçen en tales cosas \textbf{ mas otros ay que fazen el contrario } que sobrepuian en poner se adelante & Tales ergo in talibus deficiunt . \textbf{ Sed quidam econtrario superabundant , } ingerentes se ad aliqua , \\\hline
1.2.22 & que sobrepuian en poner se adelante \textbf{ para fazer algunas cosas } que digna mientre non las pueden conplir . & Sed quidam econtrario superabundant , \textbf{ ingerentes se ad aliqua , } quae digne complere non possunt , \\\hline
1.2.22 & nin fuye delas obras altas \textbf{ que puede bien fazer } assi commo el pusillanimo & ab arduis operibus , \textbf{ quae potest digne agere , } ut pusillanimus , \\\hline
1.2.22 & nin se mete mas adelante en aquello que non puede conplir \textbf{ assi commo faze el presuptuoso . } Et desto paresçe & quae digne complere non potest , \textbf{ ut praesumptuosus . } Quare manifeste patet , \\\hline
1.2.22 & e cerca quales cosas ha de ser sinca de demostrar \textbf{ en qual manera podemos a nos mismos fazer magnanimos . } Mas entre todas las otras cosas & Restat ostendere , \textbf{ quomodo possumus | nosipsos magnanimos facere . } Inter caetera autem , \\\hline
1.2.23 & e dador de los galardones \textbf{ es fazer obras de uirtudes . } Ca assi commo dize el philosofo en el quarto libro delas ethicas pertenesce mucho & Propter quod , quia esse plurimum retributiuum , \textbf{ est agere opera virtutum , } conuenit magnanimo esse plurimum retributiuum , \\\hline
1.2.23 & entroͤ los bienes de fuera \textbf{ non faze grant fuerça dellas . } Ca assi commo dicho es muy poco preçia los bienes de fuera¶ & Nam cum talia inter exteriora bona computentur , \textbf{ ipse non multum curat de eis , } quia ( ut dicebatur ) \\\hline
1.2.24 & Ca non auer cuydado dela honrra \textbf{ por que non quiere fazer obras dignas de honrra esto es de denostar } mas auer cuydado de honrra & Nam non curare de honore , \textbf{ quia non vult agere | opera honore digna , } vituperabile est . \\\hline
1.2.24 & mas auer cuydado de honrra \textbf{ en quanto quiere fazer obras dignas de honrra esto es de loar . } Et por ende nos deuemos auer cuydado de honrra & opera honore digna , \textbf{ vituperabile est . } Nobis igitur debet \\\hline
1.2.24 & Et pues que assi es essas mismas obras pueden ser delas otras uirtudes e dela magnanimidat . \textbf{ Ca aquel que faze las obras dela fortaleza } e acomete la batalla & esse aliarum virtutum , et magnanimitatis . \textbf{ Nam agens opera fortitudinis , } et aggrediens pugnam , \\\hline
1.2.24 & e acomete la batalla \textbf{ si esto faze por que se delecta en tales obras } este es dicho fuerte . & et aggrediens pugnam , \textbf{ si hoc facit , | quia delectatur in talibus actibus , } fortis est . \\\hline
1.2.24 & este es dicho fuerte . \textbf{ Mas si esto faze por que tal sobras son dignas de grant honrra } assi es dichomagranimo . & fortis est . \textbf{ Si vero hoc agit , | quia talia opera sunt magno honore digna , } magnanimus est . \\\hline
1.2.24 & Avn en essa misma manera \textbf{ si feziere obras de castidat } por que se delecte enllas es dicho casto e tenprado . & magnanimus est . \textbf{ Sic etiam si agat opera castitatis , } quia delectatur in eis , \\\hline
1.2.24 & por que se delecte enllas es dicho casto e tenprado . \textbf{ Mas si estas cosas feziere en quanto son dignas de grant honrra es dicho magranimo . } Et por ende la prop̃a materia dela magranimidat non son los periglos delas batallas & castus et temperatus est . \textbf{ Sed si hoc agat , | quia sunt magno honore digna , } magnanimus est . \\\hline
1.2.24 & por que quales se quier cosas \textbf{ que faga el magranimo todas las faze } en quanto son dignas de grant honrra . & dicitur esse honor , \textbf{ quia quaecunque agit magnanimus , } agit ea prout sunt magno honore digna . \\\hline
1.2.24 & Ca assi commo la magnificençia es vn honrramiento dela libalidat \textbf{ por que el magnifico faze obras de li beralidat } en mas alta manera que la libalidat . & quidam ornatus liberalitatis , \textbf{ qua magnificus opera liberalitatis } facit excellentiori modo : \\\hline
1.2.24 & Ca esto es obra de pradençia \textbf{ nin parte nesçe de fazer cosas non iustas } o que non conuienen ala iustiçia & quod est actus prudentiae , \textbf{ nec facere iniusta , } quod pertinet ad iniustitiam . \\\hline
1.2.24 & obrara las obras delas otras uirtudes \textbf{ e fazer las ha mas altamente . } Ca commo la honrra & Operatur ergo magnanimus opera aliarum virtutum , \textbf{ et ea faciet excellenter . } Nam cum honor inter exteriora bona sit bonum excellens , \\\hline
1.2.24 & sea mas alto e meior bien \textbf{ el que faze obras de uirtudes } en quanto son diguas de honrra & Nam cum honor inter exteriora bona sit bonum excellens , \textbf{ faciens opera virtutum , } inquantum sunt honore digna , \\\hline
1.2.24 & en quanto son diguas de honrra \textbf{ mas altamente las faze que si feziese tales obras } por que se delectase ensłas & inquantum sunt honore digna , \textbf{ excellentiori modo facit ea , | quam si ageret talia opera , } quia delectaretur in illis . \\\hline
1.2.24 & Et pues que assi es bien dich̃ones \textbf{ que la maguanimidat faze todas las uirtudes mayors } Et bien dicho es & Recte ergo magnanimitas \textbf{ dicitur omnes virtutes maiores facere , } et dicitur omnes eas perficere et ornare : \\\hline
1.2.24 & que las acaba todas e las honrra . \textbf{ Por que por la magranimidat fazemos las obras de todas las uirtudes } mas altamente e mas honrradamente e mas acabadamente ¶ & et dicitur omnes eas perficere et ornare : \textbf{ quia actus omnium earum per magnanimitatem } facimus excellentius , ornatius , et magis perfecte . \\\hline
1.2.24 & en la manera que dich̃ones de suso . \textbf{ Conuiene saber que amen e cobdicien fazer lobras } que sean dignas de honrra . & et honoris amatiuos . Reges enim et Principes decet honores diligere modo quo dictum est ; \textbf{ videlicet , ut diligant et cupiant facere opera , } quae sint honore digna . \\\hline
1.2.24 & Et en essa misma manera \textbf{ los que fazen obras dignas de honrra medianera } son dichos amadores de henrra & nisi in magno corpore . \textbf{ Sic facientes opera mediocri honore digna , } dicuntur honoris amatiui : \\\hline
1.2.24 & son dichos amadores de henrra \textbf{ Mas propreamente quando fazen obras dignas de grant honrra } estonçe son dichos magranimos & dicuntur honoris amatiui : \textbf{ tamen tunc proprie sunt magnanimi , } quando agunt opera magno honore digna . \\\hline
1.2.24 & altosñen los medianeros \textbf{ non fagan alguna cosa rorpe } mas por que sienpre fagan obras dignas de honrra . & Ut igitur Reges et Principes nec in negociis arduis , \textbf{ nec in mediocribus faciant aliquid turpe ; } sed ut semper agant opera honore digna : \\\hline
1.2.24 & non fagan alguna cosa rorpe \textbf{ mas por que sienpre fagan obras dignas de honrra . } Bien dicho es & nec in mediocribus faciant aliquid turpe ; \textbf{ sed ut semper agant opera honore digna : } bene dictum est , \\\hline
1.2.25 & La qual cosa paresçe ser contraria de ssi milma \textbf{ por que el humildoso faze reuerençia alos otros . } Mas el magnanimo assi commo dize el philosofo & omnis magnanimus esset humilis : \textbf{ quod videtur esse oppositum in obiecto , quia humilis alios reueretur , } magnanimus vero \\\hline
1.2.25 & en manera que non podamos desuiarnos del bien de razon . \textbf{ Mas que las passiones nos fagan desuiar del bien de razon } en dos maneras puede contesçer & ne deuiemus a bono rationis . \textbf{ Sed quod passiones deuiare nos faciant a bono | secundum rationem } ( ut supra dicebatur ) \\\hline
1.2.25 & en quanto ha razon de grande nos tira \textbf{ que non lo fagamos nin lo alcançemos } por razon dela guaueza & nos retrahit , \textbf{ ne prosequamur illud , } ratione difficultatis . \\\hline
1.2.25 & que el magranimo despreçia los otros non \textbf{ por que faga tuerto } en mala manera alos otros mas . & Ideo magnanimus dicitur alios despicere , \textbf{ non quod aliis vitiose iniurietur , } sed quia est tanti cordis , \\\hline
1.2.25 & por ellas delas obras uirtuosas \textbf{ Mas la humildat faze el contrario . } Ca prinçipalmente nos tira delas honrras & ab operibus virtuosis . \textbf{ Humilitas autem e contrario principaliter retrahit : } ex consequenti impellit . \\\hline
1.2.25 & por que cuydando en los sus desfallesçimientos propios en las cosas conuenibles e honestas . \textbf{ faze reuerençia alos otros ¶ } Lo segundo ha departimiento entre la magnanimidat e la humildat & in rebus licitis et honestis alios reueretur . \textbf{ Secundo differt haec ab illa , } quia magnanimitas principaliter moderat desperationem , \\\hline
1.2.26 & qua nos allegua a aquellos bienes \textbf{ mas la humil dat faze todo lo contrario } por que es uirtud & quae est excellentia expellens nos in illa . \textbf{ Sed humilitas e contrario , } quia est virtus retrahens , \\\hline
1.2.26 & La segunda de parte delas obras \textbf{ que se deuen fazer } Ca assi commo dicho es de suso ninguno non puede serudaderamente magnanimo & Prima sumitur ex parte magnanimitatis . \textbf{ Secunda vero ex parte operum fiendorum . } Dicebatur enim supra , \\\hline
1.2.26 & mas assi comma conuiene al su estado dellos . \textbf{ e esto es lo que fazen los humildosos . } Et otrosi que non pongan la su bien andança & sed ut decet eorum statum , \textbf{ quod faciunt humiles : } quod tamen suam felicitatem \\\hline
1.2.26 & Et otrosi que non pongan la su bien andança \textbf{ en sobrepuiança de honrra lo que fazen los sobuios } por que deuen fazer los Reyes bueans obras e dignas de honrra & non ponant \textbf{ in excellentia et honore , | quod faciunt superbi . } Debent enim agere bona opera \\\hline
1.2.26 & en sobrepuiança de honrra lo que fazen los sobuios \textbf{ por que deuen fazer los Reyes bueans obras e dignas de honrra } non por alabança & quod faciunt superbi . \textbf{ Debent enim agere bona opera } et honore digna boni gratia , \\\hline
1.2.26 & por razon delas obras \textbf{ que han de fazer . } Ca el sobra uio demandado & ø \\\hline
1.2.27 & por que aquel que dessea los males \textbf{ que le son fechos de ser vengado } o que el otro que resçiba penna & quod appetimus vindictam . \textbf{ Qui enim propter mala aliqua illata vindictam } aut punitionem appetit , \\\hline
1.2.27 & por que enssannar se de qual se quier cosa \textbf{ e sobre qual cosa dessear uengança la qual cosa fazen los sannudos } esto es de denostar . & et semper , \textbf{ et de quolibet vindictam exposcere , | quod faciunt iracundi , } vituperabile est . \\\hline
1.2.27 & que por los males \textbf{ e por las jniurias que nos fazen } desseemos de ser vengados & ex consequenti autem intendit moderare passiones oppositas irae . \textbf{ Nam naturale est nobis } ut ex malis nobis illatis appetamus punitionem , et vindictam . \\\hline
1.2.27 & desseemos de ser vengados \textbf{ e los otros que las fazen sean atormentados } e ayan pena & Nam naturale est nobis \textbf{ ut ex malis nobis illatis appetamus punitionem , et vindictam . } Rursus , quia malum proprium vix potest \\\hline
1.2.27 & e dar pena \textbf{ a aquellos que nos fazen alguons males . } Mas avn en alguna manera natural cosa esa nos de dessear de ser vengados dellos mas que deuemos . & ut velimus puniri \textbf{ inferentes nobis aliqua mala , } sed etiam quodammodo naturale est nobis \\\hline
1.2.27 & Mas avn en alguna manera natural cosa esa nos de dessear de ser vengados dellos mas que deuemos . \textbf{ Ca por que el mal que ellos nos fazen paresçe a nos } que es mayor de quanto es & appetere punitionem ultra condignum . \textbf{ Nam quia malum nobis illatum videtur nobis maius esse , } quam sit ; \\\hline
1.2.27 & por ende queremos \textbf{ que aquellos quanos fazen mal } e iniuria ayan mayor pena & quam sit ; \textbf{ iniuriatores nostros } plus puniri volumus , \\\hline
1.2.27 & que deuen \textbf{ por el mal que nos fazen . } Et por que muy & plus puniri volumus , \textbf{ quam puniendi sint . } Difficile est ergo valde reprimere iras , \\\hline
1.2.27 & por que si las penas non se diessen \textbf{ e las uenganças non se fiziesen en el regno } los omes serian fechos jniuriadores e forçadores de los otros . & indecens esset : \textbf{ quia si punitiones non fierent in regno , } homines fierent iniuriatores aliorum , \\\hline
1.2.27 & e alos prinçipes de se mouer a dar penas . \textbf{ e fazer uengancas } quanto mas pertenesçe a ellos de seer guardadores dela iustiçia & Tanto ergo magis decet Reges et Principes moueri \textbf{ ad punitionem faciendam , } quanto magis spectat ad ipsos \\\hline
1.2.27 & segunt orden de razon e de entendimiento \textbf{ para fazer uenganças e dar penas . } Et commo esto faga la manssedunbre & secundum ordinem rationis , \textbf{ ut fiant punitiones et vindictae , } cum hoc faciat mansuetudo , \\\hline
1.2.27 & para fazer uenganças e dar penas . \textbf{ Et commo esto faga la manssedunbre } conuiene a ellos de ser manssos & ut fiant punitiones et vindictae , \textbf{ cum hoc faciat mansuetudo , } decet eos mansuetos esse . \\\hline
1.2.28 & Ca estos en tanto se muestran conpanneros \textbf{ que non quieren fazer pesar nin tristeza a ninguno } Mas todos los dichos & Hi enim adeo se ostendunt communicabiles et sociales , \textbf{ ut nullum contristari velint ; } sed omnia dicta \\\hline
1.2.28 & e todos los fechos de los otros alaban . \textbf{ Mas otras algunos ay que fazen todo lo contrario } por que fallesçen mucho desta conuerssa conn tal . & et facta aliorum laudant . \textbf{ Aliqui vero econtrario , } ab hac conuersatione nimis deficiunt . \\\hline
1.2.28 & por mengua de palaura \textbf{ la qual cosa fazen los desacordabłs } e mouedores de pelea ¶ visto que cosa es la amistanca & nec deficiamus , \textbf{ quod faciunt litigiosi , et discoli . } Viso quid est amicabilitas , ut hic de ea loquimur , \\\hline
1.2.28 & Empero non deuen todos en vna manera seramigables e bien fablantes . \textbf{ por que la grant familiaridat pare e faze despreçiamiento . } Et pues que assi es los Reyes e los p̃nçipes & non tamen omnes eodem modo amicabiles debent esse . \textbf{ Nam quia nimia familiaritas contemptum parit , } Reges et Principes , \\\hline
1.2.28 & e alos prinçipes de paresçer \textbf{ perssonas reuerendas a quien deuen fazer reuerençia } por qua non sean auidos en despreçiamiento . & ait , quod decet Reges et Principes \textbf{ apparere personas reuerendas , } ne contemptibiles habeantur . \\\hline
1.2.29 & Et pues que assi es desta uirtud \textbf{ por la qual alguno se faze uerdadero e manifiesto . } algunos se desuian por sobrepuiança mostrando de ssi mismos & Ab hac ergo veritate , \textbf{ per quam quis reddit se veracem et manifestum , } aliqui deniant per superabundantiam , \\\hline
1.2.29 & en ninguna manera \textbf{ la qual cosa fazen muchos . } Ca otorgan alguons de ssi mismos grandes bondades & quam sint , \textbf{ quod multi faciunt . } Concedunt enim de se aliqui magnas bonitates , \\\hline
1.2.29 & mas deuemos nos inclinar alo menos \textbf{ si se pudiere fazer sin mentira . } Et non deue el omne partir se notablemente del medio e de la egualdat & sed magis declinandum in minus : \textbf{ dum tamen sine mendacio fiat , } et non notabiliter recedat a medio : \\\hline
1.2.29 & nin alabando dessi mayores cosas que son \textbf{ nin prometiendo alos otros mayores cosas que faran . } Mas por tanto conuiene alos Reyes & vel iactando de se maiora quam sint , \textbf{ vel promittendo aliis maiora quam faciant . } Immo tanto magis decet Reges et Principes cauere iactantiam , \\\hline
1.2.30 & e trabaiando en los negoçios del regno \textbf{ e faziendo otras cosas } muchas trabaiamos continuada mente . & vel negociis regni insistendo , \textbf{ vel alia faciendo , } continue laboramus , \\\hline
1.2.30 & contesçe de pecar \textbf{ e de bien fazer } conuiene erca tales iuegos & Quare cum in talibus contingat peccare , \textbf{ et bene facere , } oportet \\\hline
1.2.30 & de aquella prea alguna cosa en essa misma manera \textbf{ los que quieren fazer de todo en todo riso } e enduzir alos otros a escarnio & qualitercunque possent aliquid de illa praeda capere : \textbf{ sic volentes omnino facere risum , } et prouocare alios ad cachinnum , \\\hline
1.2.30 & tales sobrepuian en los trebeios . \textbf{ Mas ay otros que fazen todo el contrario } que fallesçen en todos los trebeios . & Hi ergo superabundant ludis . \textbf{ Aliqui vero econtrario , deficiunt : } et hi vocantur duri , et agrestes . \\\hline
1.2.30 & nin de iuego antes son tristes e malenconosos de aquellos \textbf{ que fazen los solazes e los iuegos } Et pues que assi es paresçe & nec dicentes aliquid ridiculum : \textbf{ immo dicentibus sunt molesti . } Patet ergo quid est iocunditas , \\\hline
1.2.30 & Ca uirtudes prinçipalmente çerca aquellas cosas \textbf{ que son mas guaues de fazer mas repremir las } superfluydadesde los iegos es muy mas & quia virtus semper est principalius \textbf{ circa difficilius . | Reprimere autem superfluitates } ludorum est difficilius , \\\hline
1.2.30 & mas esto conuiene alos Reyes e alos prinçipes en tanto vsar tenpradamente delas delecta connes delos iuegos \textbf{ que si esto feziesen algunas ottas personas comunes paresçeria } que serian montesinos e siluestres . & uti iocosis delectationibus , \textbf{ quod si hoc facerent personae communes , } viderentur esse durae et agrestes . \\\hline
1.2.30 & si fueren honestos o tenprados . \textbf{ Et esto faremos mas o menos segunt } que requeere el departimiento delas perssonas . & si sint honesta et moderata : \textbf{ et hoc magis et minus , } prout requirit diuersitas personarum . \\\hline
1.2.31 & Otrossi ay otros algunos \textbf{ que fazen todo lo contrario desto ca son castos . } Et enpero estudian en auaricia e en escasseza . & Aliquos vero , \textbf{ qui econtrario sunt casti : } et tamen auaritiae student . \\\hline
1.2.31 & empero que non pueden seer magnificos \textbf{ por que non pue den fazer grandes cosas } por que non han nin pueden fazer grandes espenssas . & qui tamen non possunt esse magnifici : \textbf{ quia nequeunt magna facere , } eo quod non habeant magnos sumptus . \\\hline
1.2.31 & por que non pue den fazer grandes cosas \textbf{ por que non han nin pueden fazer grandes espenssas . } Et pues que assi es deuedes saber & quia nequeunt magna facere , \textbf{ eo quod non habeant magnos sumptus . } Sciendum igitur , \\\hline
1.2.31 & los quales non son castos . \textbf{ Mas veemos a otros fazer el contrario desto } por que han algpradençia natural & qui non sunt casti : \textbf{ aliqui vero econtrario habent } quandam naturalem pudicitiam , \\\hline
1.2.31 & assi commo en las riquesas \textbf{ luego farian grandes cosas . } Et pues que assi es deuemos declarar & quia si bonis exterioribus abundarent , \textbf{ statim magnifica facerent . } Declarandum est ergo \\\hline
1.2.31 & e ordenaremos a nos mala fin \textbf{ assi commo fazen los que pecan } assi commo los auarientos & Primo , si proponamus nobis malum finem , \textbf{ ut vitiosi faciunt . } Auari enim proponunt sibi , \\\hline
1.2.31 & donde o de qual parte tomen \textbf{ para fazer estas obras } tanto que puedan dar algunos dones alos otros . & undecunque accipiant , \textbf{ dum possint aliis dona aliqua elargiri : } et aliquando per furtum , \\\hline
1.2.31 & e algunas uegadas por robo \textbf{ por que puedan fazer obras de largueza } e de franqueza . & aliquando per manifestam oppressionem alios depraedantur , \textbf{ ut exerceant opera largitatis . } Hi ergo licet \\\hline
1.2.31 & e enderesçan al omne ala fin . \textbf{ Mas la pradençia e la sabiduria faze obrar al omne } derechͣmente aquellas cosas & quod virtutes morales rectificant finem , \textbf{ Prudentia vero facit operari recte } ea quae sunt ad finem . \\\hline
1.2.31 & si aella non fuere ayuntada la pradençia \textbf{ e la sabiduria Et por ende entiende el philosofo fazer tal razon çerca la fin del sexto libro delas ethicas } aquel que acabadamente ha alguna uirtud moral & Talem ergo rationem intendit \textbf{ facere Philosophus | circa finem 6 Ethicor’ . } Habens perfecte \\\hline
1.2.31 & e acaba a aquel que la ha \textbf{ e faga la su obra buena . } Por ende commo havien escoger & et perficiat habentem , \textbf{ et opus suum bonum reddat : } cum ad bene eligere , \\\hline
1.2.31 & Por ende commo havien escoger \textbf{ e a buena obra fazer } non abasta de entender buena fin & cum ad bene eligere , \textbf{ et ad bonum opus , } sufficiat proponere bonum finem , \\\hline
1.2.31 & assi commo su fin \textbf{ non aurie cuydado de fazer lux̉ia } en tal que pudiesse ganar algo . & quam intenderet ut finem , \textbf{ non curaret moechari . } Incomplete ergo , \\\hline
1.2.32 & e es uençido por las tentaçiones \textbf{ mas non le es cosa delectable de mal fazer quando cae . } Et pues que assi es los non continentes & et per tentationes deiicitur , \textbf{ sed delectabile est ei malefacere . } Incontinentes ergo \\\hline
1.2.32 & Et pues que assi es los non continentes \textbf{ e los muelles non se delectan en mal fazer } mas escogen conueniblemente & Incontinentes ergo \textbf{ et molles | non delectantur in malefacere : } immo aliud eligunt , \\\hline
1.2.32 & por las quales son tentados tan bien los non continentes \textbf{ commo los muelles escogen bien fazer } e ponen assi muy buean sleyes & enim extra passiones tam incontinentes , \textbf{ quam molles , | eligunt benefacere , } et proponunt optimas leges . \\\hline
1.2.32 & que non puede bien gouernar el cuerpo van ala simestro En essa misma manera los muelles \textbf{ e los non continentes proponen de bien fazer } e escogen de yr ala diestra . & vadunt in sinistram . \textbf{ Sic molles et incontinentes proponunt benefacere , } et eligunt ire in dextram : \\\hline
1.2.32 & por que aquellos son dichos bestiales \textbf{ que fazen mal sobre manera de los omes . } Et son estos omes bestiales & Illi enim bestiales esse dicuntur , \textbf{ qui ultra modum hominum malefaciunt . } Sunt autem bestiales operantes ea , \\\hline
1.2.32 & otrossi el su fiio non era en casa tomaua prestado el fijo de otro su vezino \textbf{ e aprestaual para fazer el conbit } et prometial que quando quisiesse fazer conbit & a vicino suo mutuabat filium , \textbf{ et ipsum parabat in conuiuium , } spondens quod quando vellet conuiuium facere , \\\hline
1.2.32 & e aprestaual para fazer el conbit \textbf{ et prometial que quando quisiesse fazer conbit } que el qual daria su fijo & et ipsum parabat in conuiuium , \textbf{ spondens quod quando vellet conuiuium facere , } ei suum filium tribueret . \\\hline
1.2.32 & que el qual daria su fijo \textbf{ que fiziesse conbite del¶ } En essa misma manera avn çerca aquella tierra que llama una & ei suum filium tribueret . \textbf{ Sic etiam multae de Phalaride bestialitates narrantur . } Quaedam etiam aliae gentes bestiales \\\hline
1.2.32 & Et por ende aquellos son dichos bestiales \textbf{ que fazen mal fuera de razon } e dela manera comunal de los omes & Illi ergo sunt bestiales , \textbf{ qui ultra modum hominum male agunt . } Dicto de quatuor generibus malorum , \\\hline
1.2.32 & por que contener se el omne \textbf{ e tener se de fazer males mas queꝑ seuerar . } Ca quando alguno tentado poco o muy tentado non cae este es dicho ꝑse uerar . & Continere enim est plus , \textbf{ quam perseuerare . } Nam si quis etiam non tentatus , \\\hline
1.2.32 & contra las fuertes passiones \textbf{ la qual cosa faz en los continentes . Mas son assi castigados en el appetito e en el desseo } e han las passiones del alma & vel contra fortes passiones se tenent , \textbf{ quod faciunt continentes : | sed sunt ita castigati in appetitu , } et habent tam moderatas passiones , \\\hline
1.2.32 & que non sienten aquella batalla \textbf{ nin aquella tentaçion mas es los a ellos cosa delectable de bien fazer } ¶ Et pues que assi es & quod quasi pugnam non sentiunt , \textbf{ et delectabile est eis benefacere . } Sicut ergo perseuerantes opponuntur mollibus , \\\hline
1.2.32 & Ca assi commo es cosa delectable \textbf{ alos destenprados de mal fazer } assi es cosa delectable & sic temperati opponuntur intemperatis . \textbf{ Nam sicut delectabile est intemperatis mala facere , } sic delectabile est temperatis bona operari . \\\hline
1.2.33 & Las politicas amollesçen \textbf{ e desponen el coraçon a bien fazer e reduzen lo a medio¶ } Et las uirtudes segundas & scilicet politicae , \textbf{ molliunt idest ad medium reducunt . } Secundae , scilicet purgatoriae , \\\hline
1.2.33 & conuiene saber las pgatorias tiran \textbf{ e fazen oluidar las passiones } de que el alma padesçe ¶ & Secundae , scilicet purgatoriae , \textbf{ auferunt . } Tertiae , quae sunt purgati animi , \\\hline
1.2.33 & que son del coraçon pragado \textbf{ fazen oluidar del todo las passiones . } ¶ Mas en las quartas uirtades & Tertiae , quae sunt purgati animi , \textbf{ obliuiscuntur . } Sed in quartis , \\\hline
1.2.33 & pues que assi es las uirtudes politicas \textbf{ que amollesçen e ordenan el coraçon abien fazer } e lo reduzen a medio parte nesçen alos perseuerantes & Virtutes ergo politicae \textbf{ quae molliunt , | idest disponunt animum ad benefaciendum , } et reducunt ipsum ad medium , \\\hline
1.2.33 & que non tiran las passiones \textbf{ mas fazen las oluidar . } Ca el tenprado & nam istae non auferunt , \textbf{ sed obliuiscuntur . } Temperatus enim habet sic appetitum castigatum , \\\hline
1.2.33 & tal auer las uirtudes del coraçon pragado \textbf{ las quales le fazen escaeçer e oluidar las passiones } e las delectaçiones desordenadas & Merito ergo talis dicitur \textbf{ habere virtutes purgati animi facientes } ipsum obliuisci passiones illas crebras , \\\hline
1.2.34 & por que al uirtuoso es cosa delectable \textbf{ de bien obrar e de bien fazer . } Et el continente & Continentia enim non proprie est virtus , \textbf{ quia virtuoso delectabile est benefacere : } continens enim licet \\\hline
1.2.34 & por razon dela lid \textbf{ que siente non es a el cosa delectable de bien fazer Et } pues que assi es en quanto alguno es continente & ratione pugnae quam sentit , \textbf{ non est ei delectabile benefacere . } Quandiu ergo aliquis est continens , \\\hline
1.3.1 & Mas si alguno quisi esse trabaiar de enponer o fallar \textbf{ non bͤapio a cada vna cosa podia lo fazer } mas quando nos somos çiertos dela cosa non deuemos auer cuydado delas palauras . & Si quis autem laborare vellet , \textbf{ cuilibet posset | inuenire nomen proprium . } Sed cum constat de re , \\\hline
1.3.1 & Ca en quanto por el nos \textbf{ leunatamos a fazer uengança assi es saña . } O en quanto fallesçemos & quia vel ex hoc consurgimus \textbf{ ad faciendam vindictam , | et sic est ira : } vel a tali vindicta deficimus , \\\hline
1.3.3 & Empero desto mas conꝑlidamente tractaremos adelante \textbf{ or que las passiones fazen departimiento } en el nuestro gouernamiento & De hoc tamen infra diffusius tractabitur . \textbf{ Passiones autem } quia diuersificant regnum et vitam nostram , \\\hline
1.3.3 & Ante si el nuestro bien fuese destroydo \textbf{ e perdido dios donde el quisiese lo podria refazer } por la qual cosa ningun omne sin ayuda de dios non pue da & immo et si annihilatum esset bonum nostrum , \textbf{ Deus unde vellet , | posset illud reficere . } Quare cum nullus homo \\\hline
1.3.3 & por la qual cosa ningun omne sin ayuda de dios non pue da \textbf{ assi mismo fazer bueno o guardar } assymismo en bondat . La razon natural muestra & sine diuino auxilio possit \textbf{ seipsum bonum facere , | vel se in bonitate conseruare , } dictat naturalis ratio \\\hline
1.3.3 & e demandan grandia singular de su persona \textbf{ por la qual cosa fazen iniurias } e tuertos alos otros . & et quaerentes excellentiam singularem : \textbf{ propter quod fiunt iniuriatores aliorum , } depraedant sacra , \\\hline
1.3.3 & casasscans e despoian el pueblo \textbf{ e fazen e vsan de toda iniustiçia . } Onde ualerio maximo cuenta de dionsio seziliano & expoliant populum , \textbf{ omnem iniustitiam exercent . } Unde et Valerius Maximus de Dionysio Ciciliano recitat , \\\hline
1.3.3 & e matando los omes malos \textbf{ que fazen mal . } Por ende tales son de destroyr e de matar & nisi exterminando maleficos homines , \textbf{ extirpandi sunt tales , } ne pereat commune bonum . \\\hline
1.3.4 & e la delectaçion \textbf{ nos faze folgar en el . } Et esto que dicho es del bien & desiderium nos mouet : \textbf{ et delectatio nos quietat . } Et quod dictum est \\\hline
1.3.4 & se ayan bien alas cosas diuinales e de dios \textbf{ e que fagan obras uirtuosas } e que sea entre ellos paz e assesiego & bene se habeant ad diuina , \textbf{ quod agant opera virtuosa , } quod sint in eis pax tranquillitas , \\\hline
1.3.4 & por las cosas desiguales e malas . \textbf{ Et fazer o tris cosas tales delas quales nasçe e cuelga la salud del regno } ¶ & punire iniusta , \textbf{ et facere talia , | a quibus regni salus dependere videtur . } Viso , quae , \\\hline
1.3.5 & Mas avn les conuiene de entender \textbf{ en bien alto e grande e guaue de fazer De mas desto } quanto mayor es la comunidat & et Principes tendere in bonum , \textbf{ sed etiam decet eos tendere in bonum arduum . } Amplius quanto maior est communitas , \\\hline
1.3.6 & Empero conuiene de tractar primero estas cosas uniuersales et generales \textbf{ por que el conosçimiento dellas faze mucho al conosçimiento delas cosas que se siguen . } Et por ende en el terçero & Expedit tamen haec uniuersalia pertransire , \textbf{ quia horum cognitio faciet | ad cognitionem sequentium . } In tertio ergo libro \\\hline
1.3.6 & La segunda se to ma de parte dela obra \textbf{ que es de fazer } la primera manera se puede assi mostrar . & ex parte consilii habendi . \textbf{ Secunda ex parte operis fiendi . } Prima via sic patet : \\\hline
1.3.6 & en el capitulo del temor \textbf{ que el temor nos faze auer consseio . } Ca por que alguno teme alguna cosa & nam , ut dicitur 2 Rhetoric’ cap’ de timore , \textbf{ Timor consiliatiuos facit , } ex eo , quod aliquis ei timet , \\\hline
1.3.6 & Lo segundo podemos esso mismo mostrar \textbf{ de parte dela obra que es de fazer . } Ca non abasta de ser acuciosos & Secundo hoc idem inuestigare possumus \textbf{ ex parte operis fiendi . } Nam non sufficit solicitari \\\hline
1.3.6 & Mas el \textbf{ temortenprado non solamente faze alos Reyes tomadores de conseio mas faze avn que fagan las obras mas acuçiosa mente . } Ca si nos viniere algun temor tenprado mas & non solum consiliatiuos facit , \textbf{ sed etiam agit | ut opera diligentius operemur . } Nam si moderatus adsit timor , \\\hline
1.3.6 & Ca si nos viniere algun temor tenprado mas \textbf{ acuciosamente fazemos las obras } por las quales queremos foyr de aquel temor & Nam si moderatus adsit timor , \textbf{ diligentius agimus opera , } per quae fugere credimus timorem illum . \\\hline
1.3.6 & Ca el temor destenprado \textbf{ e sin razon primero faze } al ome ser encogido & quae omnino derogant regno . \textbf{ Nam timor immoderatus primo } reddit hominem immobilem , et contractum . \\\hline
1.3.6 & e non semauer a las cosas ¶ \textbf{ Lo segundo faze e omne que non puede auer consseio } ¶lotraçero faz al omne tremuliento qual triemen los mienbros ¶ & reddit hominem immobilem , et contractum . \textbf{ Secundo facit ipsum inconsiliatiuum . } Tertio facit eum tremulentum . \\\hline
1.3.6 & Lo segundo faze e omne que non puede auer consseio \textbf{ ¶lotraçero faz al omne tremuliento qual triemen los mienbros ¶ } Lo quarto faz al omne & Secundo facit ipsum inconsiliatiuum . \textbf{ Tertio facit eum tremulentum . } Quarto facit eum inoperatiuum . \\\hline
1.3.6 & temordestenprado \textbf{ e sin razon fazen al omne sin conseio . } Ca quando alguno teme destenpradamente & Secundo hoc est indecens , \textbf{ quia immoderatus timor facit hominem inconciliatiuum . } Cum enim quis immoderate timet , \\\hline
1.3.6 & e es puesto fuera de entendimiento \textbf{ e non sabe que faze . por la qual cosa non se acuerda de auer conseio . } Et sil fuere dado consseio & et est in agona ductus , \textbf{ et ignorat quid faciat , | propter quod non recordatur consiliari : } et si consilium ei detur , \\\hline
1.3.6 & Por la qual cosa si es cosa desconuenible \textbf{ que los fechos del regno se fagan sin consseio } e que el Rey sea sin conseio . & Quare si indecens est , \textbf{ ut facta regni sine consilio gerantur , } et ut Rex sit inconsiliatiuus ; \\\hline
1.3.6 & Lo terçero el temor deste prado \textbf{ e sin razon faze al ome tremuliento e tremedor } Ca por el temor la calentura natural & indecens est ipsum timere immoderato timore . \textbf{ Tertio immoderatus timor reddit hominem tremulentum . } Nam propter timorem calore progrediente ad interiora , \\\hline
1.3.6 & Lo quarto el temor destenprado \textbf{ e sin razon faze al omne } que non obre . & Quarto immoderatus timor \textbf{ reddit hominem inoperatiuum . } Nam homo propter timorem immoderatum tremens \\\hline
1.3.6 & e sin razon \textbf{ Esto tal faze periuyzio a todo el regno } por la qual cosa & et imperare non valeat propter immoderatum timorem , \textbf{ toti regno praeiudicium gignitur : } quare si hoc est indecens , \\\hline
1.3.7 & otrosi non creyere \textbf{ quel fizo algun mal } o en si o en sus fios o en amigos o en algunos otros & nisi credat \textbf{ ipsum fore fecisse vel in se , vel in filios , } vel in amicos , \\\hline
1.3.7 & si non de aquellos \textbf{ que fezieron mal o en si mismo o en aquellas cosas } que parte nesçen assi mismo . & nisi de forefacientibus , \textbf{ vel in ipsum , | vel in pertinentibus ad ipsum . } Sed odire , \\\hline
1.3.7 & si non a alguno en espeçial . \textbf{ Ca commo el omne en comun non faga iniuria nin tuerto a nos } mas sienpre el tuerto o la imiuria es acometida & nisi alicui speciali . \textbf{ Nam cum homo in communi non iniurietur nobis , } sed semper committatur iniuria \\\hline
1.3.7 & que paresçe al sannudo \textbf{ que ay fecha uengança couenible } estonçe le farta la lanna & quod videatur irato \textbf{ ultionem decentem factam esse , } satiatur ira , \\\hline
1.3.7 & La quarta diferençia es que el sanudo dessea contristar \textbf{ e de fazer triste a a qual contra quien a saña . } Mas el que quiere mal a alguno tenssea dele enpeçer . & Quarta differentia est , \textbf{ quia iratus appetit contristare : } sed odiens appeti nocere . \\\hline
1.3.7 & Ca el sannudo quiere dar dolor e tsteza \textbf{ mas el mal quariente quiere fazer danno e enpeçemiento¶ } La quinta diferençia es & inferre dolorem , et tristitiam : \textbf{ sed odiens vult | inferre damnum , et nocumentum . } Quinta differentia est , \\\hline
1.3.7 & contra quien ha la saña . \textbf{ Mas el que quiere mal non faze fuerça desto } por que non abasta al ayrado & quia sentiri quidem vult iratus , \textbf{ odienti autem nihil differt . } Non enim sufficit irato , \\\hline
1.3.7 & si non que lo sienta . \textbf{ Et si non paresçiere manifiestamente que el sañudo faze al otro } contra quien ha saña aquel mal & nisi sentiatur , \textbf{ et ut manifeste appareat } quod ipse ei inferat illud malum : \\\hline
1.3.7 & por que tanta es la quexura del coraçon del sañudo \textbf{ que sea fecha uengança } que fasta que sea fecha aquella uegaça esta sienpre commo en tristeza continuadamente . & tanta enim est anxietas irati \textbf{ ut vindictae fiat , } quod donec sit facta ultio , \\\hline
1.3.7 & que sea fecha uengança \textbf{ que fasta que sea fecha aquella uegaça esta sienpre commo en tristeza continuadamente . } Mas la mal querençia puede ser sintsteza & ut vindictae fiat , \textbf{ quod donec sit facta ultio , | quasi continue est in tristia . } Sed odium sine tristitia esse potest : \\\hline
1.3.7 & Porque commo la sanna se pueda fartar \textbf{ si muchos males fueren fechos al otro } contra quien ha sana el sañudo apiadasse del & Nam cum ira satietur , \textbf{ si multa mala inferantur alteri , } iratus miseretur \\\hline
1.3.7 & que el otro padesca mal \textbf{ fasta que sea fecha uengança conuenible . } Mas la mal querençia mata & quod alter contra patiatur , \textbf{ donec fiat condigna ultio . } Sed odium exterminat , \\\hline
1.3.7 & sat̃ agostin la saña passar se en mal querençia \textbf{ esto es de vna paia fazer ugalagar ¶ } Et pues que assi es la mal querençia es de esquiuar en toda manera alos Reyes e alos principes & Immo iram transire in odium secundum Augustinum , \textbf{ hoc est , trabem facere de festuca . } Est ergo huiusmodi odium cauendum a quolibet . \\\hline
1.3.7 & Et pues que assi es la mal querençia es de esquiuar en toda manera alos Reyes e alos principes \textbf{ por que podrien fazer mucho danno } e mucho mal a muchos . & Magis tamen cauendum est Regibus , et Principibus : \textbf{ quia inferre possunt pluribus nocumentum . } Sic igitur sentiendum est de odio et ira : \\\hline
1.3.7 & ante que entiendan conplidamente el mandamiento del corren \textbf{ para fazer e cunplir el su mandado . } Por la qual cosa les & currunt , \textbf{ ut exequantur mandatum ipsius ; } quare contingit eos deficere , \\\hline
1.3.7 & non departiendo nin conosçiendo si aquel que viene es amigo o enemigo . \textbf{ Bien assi faze la saña . } Ca luego que la razon & an veniens sit amicus , vel inimicus . \textbf{ Sic et ira facit : } statim enim cum ratio dicit \\\hline
1.3.7 & luego quiere correr \textbf{ por que sea fecha uengança } non espando sobresto iuyzio delanrazon e del entendimiento & statim vult currere , \textbf{ ut vindictam exequatur , } non expectans super hoc iudicium rationis , \\\hline
1.3.7 & non espando sobresto iuyzio delanrazon e del entendimiento \textbf{ en qual manera esta uenganca deue ser fecha . } Et pues que assi es la sanna desordenada & non expectans super hoc iudicium rationis , \textbf{ qualiter vindicta illa fieri debeat . } Est igitur cauenda ira inordinata , \\\hline
1.3.7 & e el vso dela razon . \textbf{ Et pues que assi es ante que iudguemos conplidamente por la razon que es aquello que deuemos fazer deuemos ser manssos . } Mas despues que fueriuisto conplidamente aquello que deuemos fazer podemos romar la sana & et rationis actum . \textbf{ Ante ergo quam per rationem iudicemus plene | quid agendum , } debemus esse mansueti . \\\hline
1.3.7 & Et pues que assi es ante que iudguemos conplidamente por la razon que es aquello que deuemos fazer deuemos ser manssos . \textbf{ Mas despues que fueriuisto conplidamente aquello que deuemos fazer podemos romar la sana } assi commo sierua dela razon & debemus esse mansueti . \textbf{ Sed postquam plene visum est , | quid facturi sumus , } possumus assumere \\\hline
1.3.8 & que todas las delectaçiones eran buenas \textbf{ e fazia esta razon que aquella cola que es desseada de todos } paresçe mucho mas ser buean et escogible que ninguna otra . & Eudoxus autem posuit omnem delectationem esse bonam : \textbf{ quia quod ab omnibus appetitur } maxime videtur esse bonum \\\hline
1.3.8 & e segunt alguna parte . \textbf{ Otrossi por que la delectacion se faze por ayuntamiento dela cosa conuenible con cosa conuenible } Por ende commo algunas cosas conuengan alas bestias & et secundum quid . \textbf{ Rursus quia delectatio contingit | ex coniunctione conuenientis cum conuenienti : } cum ergo alia conueniant bestiis , alia hominibus : \\\hline
1.3.8 & aquel que la resçibe \textbf{ e faze la obra mas desenbargada e mas conuenible . Et por ende si los Reyes e los prinçipes se delectaren en las obras dela pradençia } e en las obras uirtuosas mas desenbargadamente & et expeditiorem reddit operationem conuenientem . \textbf{ Si igitur Reges , | et Principes delectabuntur } in actibus prudentiae , \\\hline
1.3.8 & e en las obras uirtuosas mas desenbargadamente \textbf{ e mas acabadamente podrian fazer estas tales obras . } Ca quando alguno mas fuertemente se delecta en las obras uirtuosas & et in operibus virtuosis , \textbf{ expeditiori modo et magis perfecte efficere poterunt huiusmodi opera . Nam quanto quis vehementiori modo delectatur } in actibus virtuosis , \\\hline
1.3.8 & tanto \textbf{ mas altamente faze aquellas obras } Mas las delecta connssenssibles los Reyes & in actibus virtuosis , \textbf{ tanto excellentius efficit illos actus . } In delectationibus autem sensibilibus delectari \\\hline
1.3.8 & Ca si alguno vee \textbf{ que el fizo en algua manera } cosas torpes deue se doler & nisi supposito aliquo turpi . \textbf{ Si quis enim videt se in aliquo turpia egisse , } debet dolere et tristari . \\\hline
1.3.8 & porque lo vno iudgan por razon e por entendimiento \textbf{ e lo al fazen despues por la obra } Por la qual cosa commo ellos non ayan paz en si mismos non gozan de ssi mismos . & Nam unum ratione iudicant , \textbf{ et aliud passione agunt . } Quare cum in seipsis pacem non habeant , \\\hline
1.3.10 & por el qual es inclinado alguno a dar benefiçios a \textbf{ otroo es inclinado a fazer bien a otro . } ¶ Et pues que assi es el zelo & quam quidam motus animi , \textbf{ per quem inclinatur aliquis ad beneficia conferendum . } Zelus ergo et gratia reducuntur ad amorem . \\\hline
1.4.1 & e las riquezas \textbf{ fazen las costunbres muy departidas . } Ca por la mayor parte otras costunbres han los nobles & ut nobilitas , potentia , et diuitiae , non modicum mores diuersificant . \textbf{ Nam ut plurimum alios mores habent nobiles , } quam ignobiles : \\\hline
1.4.1 & por que los mançebos non se delectan mucho en aquellas cosas \textbf{ que fizieron } por que se acuerdan & iuuenes \textbf{ in iis quae fecerunt , } quia memorantur se modica fecisse : \\\hline
1.4.1 & por que se acuerdan \textbf{ que fizieron pocas cosas } Mas mucho se delectan en cuydando aquellas cosas & in iis quae fecerunt , \textbf{ quia memorantur se modica fecisse : } sed multum delectantur in cogitando , \\\hline
1.4.1 & Mas mucho se delectan en cuydando aquellas cosas \textbf{ que han de fazer } por que esperan que han de fazer grandes cosas . & sed multum delectantur in cogitando , \textbf{ quae facturi sunt . } Sperant enim se magna facere , \\\hline
1.4.1 & que han de fazer \textbf{ por que esperan que han de fazer grandes cosas . } Por la qual cosa & quae facturi sunt . \textbf{ Sperant enim se magna facere , } quare contingit eos animosos esse , \\\hline
1.4.1 & por que se tiene por digno para g̃ndescolas \textbf{ e entremetesse de fazer grandes cosas . } Et pues que assi es commo los mancebos & quia dignificat se magnis , \textbf{ et ingerit se ad faciendum magna . } Iuuenes ergo , \\\hline
1.4.1 & assi commo medios dioses \textbf{ non solamente non les conuiene de fazer cosas torpes } mas avn deuen aborresçer delas oyr nonbrar & quos decet esse quasi semideos , \textbf{ non solum quod turpia committant , } sed abominabile eis esse debet \\\hline
1.4.1 & mas que los otros \textbf{ por que peor cae a ellos de fazer mal que a otros . } Et pues que assi es las primeras çinco cosas & uerecundari deberent etiam plus quam alii , \textbf{ eo quod magis indecenter se gererent . } Prima ergo quinque , \\\hline
1.4.1 & Ca conuiene alos Reyes de ser liberales \textbf{ por que farien contra natura } si segunt la muchedunbre de sus riquezas & non decet simpliciter competere Regibus et Principibus . Decet enim Reges et Principes esse liberales : \textbf{ quia contra naturam agerent , } si multitudinem diuitiarum qua pollent , \\\hline
1.4.1 & en el segundo libro de la rectorica \textbf{ losomes las mas cosas fazen e obran mal } Et por ende la flaqueza & Nam ( ut ait Philosophus 2 Rhetoricorum ) homines \textbf{ ut plurimum mala faciunt , } ipsa ergo humana fragilitas \\\hline
1.4.1 & por si mas por la mala obra \textbf{ si la fezieren } assi commo es dicho de ssuso . & dicet miseratiuos esse . \textbf{ Esse autem verecundos | non decet eos simpliciter , } ut superius dicebatur . \\\hline
1.4.2 & ¶lo sexto non han manera en las obras \textbf{ que fazen mas todas las cosas fazen con sobre puiança ¶ Lo primero digo que los mancebos son tales } que sigeien las passiones & Sexto in actionibus non habent modum , \textbf{ sed omnia faciunt valde . | Sunt enim primo iuuenes insecutores passionum , } et maxime insequuntur concupiscentias circa corpus . \\\hline
1.4.2 & La primera razon es que por que los mancebos son muy calient s̃ \textbf{ e el cuerpo escalentado faze appetito } e desseo de luxia . & Nam cum iuuenes sint percalidi , \textbf{ et corpore calefacto fiat venereorum appetitus , } naturalis dispositio corporis \\\hline
1.4.2 & Et por que les paresce que son enxalçados \textbf{ quando fazen miurias e tuertos a algunos . } por ende son peleadores de ligero ¶ & quia ergo videtur eis quod excellant , \textbf{ contumelias inferunt ; } et sic de facili sunt contumeliosi . \\\hline
1.4.2 & Lo sexto non han manera en las sus obras \textbf{ Mas todas las cosas fazen forçadamente } Onde dize el philosofo & non habent modum , \textbf{ sed omnia faciunt valde . } Unde dicitur 2 Rhet’ \\\hline
1.4.2 & e aborresçen mucho \textbf{ e todas las cosas fazen sobrepuiada mente . } Et esto les contesçe & odiunt valde , \textbf{ et omnia faciunt valde . } Hoc autem ideo contingit , \\\hline
1.4.2 & mas que por razon non han las cobdiçias tenpdas \textbf{ mas todas las cosas fazen con sobrepuiamiento ¶ } visto quales son las costunbres delos mançebos & non habent concupiscentias moderatas , \textbf{ sed omnia faciunt valde . } Viso qui sunt mores iuuenum vituperabiles ; \\\hline
1.4.2 & que ha sennorio \textbf{ por que non fagan assi mismos seer despreçiados . } Ca si la mentira faze seer los omes ser menospreçiados & est mendacium fugiendum , \textbf{ ne seipsos contemptibiles reddant : } si enim mendacium reddit homines contemptibiles , \\\hline
1.4.2 & por que non fagan assi mismos seer despreçiados . \textbf{ Ca si la mentira faze seer los omes ser menospreçiados } quanto mas es cosa desconuenible & ne seipsos contemptibiles reddant : \textbf{ si enim mendacium reddit homines contemptibiles , } quanto magis indecens est \\\hline
1.4.3 & e sin manziella \textbf{ e non han fecho muchos males } por la su sinpleza et inoçençia iudgan todos los otros . & Pueri enim , \textbf{ quia non multa mala fecerunt } et innocentes sunt , \\\hline
1.4.3 & segunt aquellas cosas \textbf{ que ellos mismos fezieron . } por la qual cosa & mensurant facta aliorum \textbf{ secundum ea quae gesserunt in seipsis : } propter quod ut plurimum credunt \\\hline
1.4.3 & Et por ende poniendo en lo que han su esperança \textbf{ e su fiuza non osan fazer espenssas . } Etrossi son escassos e non francos & Ponentes ergo in eis suam spem et confidentiam , \textbf{ non audent expensas facere . } Rursus illiberales sunt \\\hline
1.4.3 & e cuydan que son pocas cosas \textbf{ aquellas que pueden fazer } ca non biuen & senes in sperando deficiunt , \textbf{ et modica se cogitant facturos ; } non enim viuunt , \\\hline
1.4.3 & sienpre cuentan las cosas passadas \textbf{ que ellos fizieron } mas non se delectan en & semper recitant res gestas , \textbf{ quas fecerunt ; } non autem delectantur in recitando res fiendas , \\\hline
1.4.3 & contando las cosas \textbf{ que son de fazer . } Las quales cosas ellos han de fazer & non autem delectantur in recitando res fiendas , \textbf{ quas sunt facturi , } eo quod videant se multa fecisse , \\\hline
1.4.3 & que son de fazer . \textbf{ Las quales cosas ellos han de fazer } por que parezca alos omes & non autem delectantur in recitando res fiendas , \textbf{ quas sunt facturi , } eo quod videant se multa fecisse , \\\hline
1.4.3 & por que parezca alos omes \textbf{ que ellos ayan fecho muchas cosas } e cuyden que han pocas por fazer . & quas sunt facturi , \textbf{ eo quod videant se multa fecisse , } et cogitent se pauca facturos . \\\hline
1.4.3 & que ellos ayan fecho muchas cosas \textbf{ e cuyden que han pocas por fazer . } Et por ende son de poca esperança & eo quod videant se multa fecisse , \textbf{ et cogitent se pauca facturos . } Sunt ergo difficilis spei , \\\hline
1.4.3 & por que en esperando fallesçen \textbf{ e esperan de fazer pocas cosas } ¶ Lo sexto los uieios son desuergonçados & quia in sperando deficiunt , \textbf{ et pauca se facere sperant . } Sexto senex sunt inuerecundi , \\\hline
1.4.3 & e apretandolas torna las colas mas pesadas \textbf{ e faz las dessear el logar mas bayo . } Ca nos veemos & reddit ipsa grauiora , \textbf{ et facit ea appetere inferiorem locum . } Videmus enim quod elementa frigida \\\hline
1.4.3 & por la friura son costnidos e apretados en si mismos \textbf{ e son fecho pesados } que se non pueden mouer en tal manera & et retrahantur in seipsis ; \textbf{ et redduntur immobiles , } ut nihil audeant vel credant ; \\\hline
1.4.3 & que se non pueden mouer en tal manera \textbf{ que non osan fazer ninguna cosa } nin creen ninguna cosa & et redduntur immobiles , \textbf{ ut nihil audeant vel credant ; } nihil sperent , \\\hline
1.4.3 & que non solamente conuiene alos Reyes \textbf{ e alos prinçipes de ser francos faziendo espenssas medianas } mas ahun les conuiene de sern magnificos & quod non solum decet \textbf{ Reges et Principes esse liberales , | faciendo mediocres sumptus : } sed etiam congruit eos esse magnificos , \\\hline
1.4.3 & mas ahun les conuiene de sern magnificos \textbf{ e granados fazie do grandes cosas ¶ } Lo quinto con uiene a ellos de ser de buena elperança . & sed etiam congruit eos esse magnificos , \textbf{ magnifica faciendo . } Quinto oportet eos esse bonae spei : \\\hline
1.4.4 & La quarta es \textbf{ que non fazen ninguna cosa } co sobre piuna & Tertio dubia non pertinaciter asserunt . \textbf{ Quarto nihil agunt valde . } Concupiscentiae enim senum \\\hline
1.4.4 & quando el cuerpo esta escalentado \textbf{ fazese en el mouiento de lux̉ia } e dela cobdicia dela carne . & corpore existente calefacto , \textbf{ fit incitatio venereorum , } et concupiscentiae ; \\\hline
1.4.4 & quando el cuerpo esta esfriado \textbf{ fazese en el abaxamiento e atenpramiento delas cobdiçias dela luxͣia . } Ca cierça cosa es que aquellos que dessean la lux̉ia & corpore existente infrigidato , \textbf{ fit remissio venereorum , } et concupiscientiarum . Constat enim quod concupiscens venerea per appetitum , \\\hline
1.4.4 & que mundado el instramento artifiçial \textbf{ faze se mudamiento } en la obra artifiçial Vien & sicut in opere artificiali , \textbf{ variato organo fit variatio operis ; } sic et in actionibus animae , \\\hline
1.4.4 & que demudado el cuerpo el alma sigue la conplision del cuerpo \textbf{ e faze se mudamiento en las obras del alma . } Ca esfriando el cuerpo el alma es inclinada & anima sequitur complexiones corporis , \textbf{ et fit variatio actionum eius ; } corpore igitur infrigidato , \\\hline
1.4.4 & non sentençian ninguna cosa firmemente ¶ \textbf{ Lo quarto non fazen ninguna cosa con sobrepunaça } mas en todas las sus obras quieren paresçer tenprados . & nihil firme sententiant . \textbf{ Quarto nihil agunt valde , } sed in omnibus operibus suis videntur esse temperati . \\\hline
1.4.4 & e las passiones en todas las cosas tienen sobrepuiamiento \textbf{ Et todas las cosas fazen forcadamente } e con sobeiama non teniendo manera & in omnibus tenent extremum , \textbf{ et omnia agunt valde : } sic senes , \\\hline
1.4.4 & e las cobdiçias abaxadas \textbf{ por la mayor parte fazen todas las cosas tenpradamente ¶ } Visto quales son las costunbres de los mançebos & et concupiscentias remissas , \textbf{ ut plurimum agunt omnia moderate . } Viso , qui sunt mores iuuenum , \\\hline
1.4.4 & e indinacion natural ha costunbres malas e de deno star . \textbf{ Empero pueden fazer contra aquella disposiconn } e inclina conn natural & et inclinationem ad mores vituperabiles : \textbf{ possunt tamen contra illam pronitatem facere } consequi laudabiles mores . \\\hline
1.4.4 & e indinaçion a costunbres bueans e de loar \textbf{ enpero pueden uenir e fazer contra esta disposiçion natural } assi que por la corrupçion del appetito & ad mores laudabiles , \textbf{ possunt tamen contra istam pronitatem facere , } ut per corruptionem appetitus sequantur \\\hline
1.4.5 & sienpre incline el coraçon de los nobles \textbf{ para fazer grandes cosas siguese } que los nobles han de ser magnificos & quare cum nobilitas semper inclinet animum nobilium \textbf{ ut faciant magna , } sequitur nobiles esse magnificos , \\\hline
1.4.5 & e escodrinnadores sotilmente de todo aquello \textbf{ que les conuiene de fazer } por que las sus obras & subtiliter inuestigantes \textbf{ quid decet eos facere , } ne opera eorum , \\\hline
1.4.5 & en todas sus obras \textbf{ que deuen fazer . } Et desto puede paresçer & ex diligenti consideratione suorum agibilium \textbf{ esse dociles , et industres . } Ex hoc autem apparere potest , \\\hline
1.4.5 & dellos disponen los \textbf{ e fazen los que non conoscan assi mismos } e que siguna voluntad & disponuntur , \textbf{ ut seipsos ignorent , } et sequantur voluptatem , non rationem . \\\hline
1.4.5 & si temieren de ser reprehendidos \textbf{ e si temieren de fazer cosas reprehenssibles e si cuydaren sotilmente todo lo que han de fazer } ¶La quatta condicion de los nobles es & si timentes reprehensibilia facere , \textbf{ diligenter considerent quid agendum . } Quarto nobiles contingit esse politicos , et affabiles . \\\hline
1.4.5 & por que buien señeros \textbf{ fazen serudos e montanneses . } Assi los nobles por el contrario & quia quasi solitarii viuunt , \textbf{ fiunt rudes et syluestres : } sic nobiles econtrario vita sociali viuentes , \\\hline
1.4.5 & por que biuen en conpania \textbf{ fazen se conpanonnes e bien fablantes . } Mas estas costunbres las quales pueden ser bueans e de loar & sic nobiles econtrario vita sociali viuentes , \textbf{ fiunt sociales et affibiles . } Hos autem mores , \\\hline
1.4.5 & Et por ende aquellos que comiençan a \textbf{ enrriqueçerquieten se fazer mas ricos . } Et aquellos que son honrrados quieren se fazer mas honrrados . & ideo qui incipiunt ditari , \textbf{ volunt fieri ditiores : } et qui sunt honorabiles , \\\hline
1.4.5 & enrriqueçerquieten se fazer mas ricos . \textbf{ Et aquellos que son honrrados quieren se fazer mas honrrados . } Et por ende los nobles & volunt fieri ditiores : \textbf{ et qui sunt honorabiles , | volunt esse honorabiliores . } Nobiles ergo , \\\hline
1.4.5 & por ende quieren acresçentar aquella honrra \textbf{ que han e quieren se fazer mas honrrados et por ende son much desseadores de grant } honrra¶Lo segundo son sobuios e despreçiadores de aquellos que los engendraron . & ad id quod habent , \textbf{ et volunt fieri honorabiliores ; | ideo sunt nimis honoris appetitiui . } Secundo sunt elati et despectatores progenitorum : \\\hline
1.4.5 & por que non deuemos dessear las honrras en lli . \textbf{ Ca esto fazen los orgullolos et los sob̃uios . } Mas deuemos dessear las obras & appetere ipsos honores in se , \textbf{ quia hoc faciunt elati et superbi : } sed debemus appetere opera honore digna , \\\hline
1.4.5 & que son dignas de honrra \textbf{ la qual cosa fazen los uirtuosos } e los de grand coraçon & sed debemus appetere opera honore digna , \textbf{ quod faciunt virtuosi et magnanimi . } Reges ergo et Principes , \\\hline
1.4.6 & Et commo por tal razon commo esta alguno parezca de ser mas alto \textbf{ si puede fazer tuertos alos otros } e denostar los . & cum ex hoc quis excellere videatur , \textbf{ si potest aliis contumelias inferre : } diuites , \\\hline
1.4.6 & e por que parezcan que son mayores que los otros \textbf{ por ende se mueuen a fazer } e a dezer tuertos & eos esse excellentiores illis , \textbf{ mouentur , } ut aliis contumelias inferant . \\\hline
1.4.6 & e son instrumentos e organos para ganarla . \textbf{ Et ahun fazen las riquezas alguna claridat dela bien andança . } Mas si las riquezas fueren ordenadas a alabança o a pelea o a destenprança o a otrasma las obras & et sunt organa ad ipsam , \textbf{ et faciunt ad quandam eius claritatem . } Sed si ordinentur ad iactantiam , \\\hline
1.4.6 & Mas si las riquezas fueren ordenadas a alabança o a pelea o a destenprança o a otrasma las obras \textbf{ estonçe mas fazen al omne mal andante que bien andante . } Et por ende ca vno o cuyde quesa digno de ser prinçipe & vel ad intemperantiam , \textbf{ vel ad alia opera vitiosa : | tunc magis reddunt hominem infelicem , } quam felicem . \\\hline
1.4.6 & que abonde la sabiduria del omne \textbf{ para tanto que el se pueda fazer rico } porque veenmos algunas vezes que los omes mas engennolos e mas labidores lon menosncos . & non videtur sufficere industria humana \textbf{ ad hoc quod aliquis fiat diues : } videmus enim aliquando homines magis industres , minus ditari . \\\hline
1.4.6 & mientesa esto \textbf{ farien grandes cosas çerca las cosas diuinales . } Et por ende non creyrien que ellos dan grandes dones a dios & quam in propriam industriam : \textbf{ quod si hoc bene attenderent diuites faciendo magnifica circa diuina , } non crederent se Deo dona largiri , \\\hline
1.4.7 & por que non desçenden de linage honrrado \textbf{ mas fezieron sericos del otro dia aca . } Pues que assi es diferençia ay & quia non processerunt ex honorabili genere , \textbf{ sed sunt nuper ditati . } Differunt ergo esse nobilem , \\\hline
1.4.7 & en que esta la uirtud \textbf{ e de non fazer obras uirtuosas } e pues que assi es el prinçipado & verecundatur omnino declinare a medio , \textbf{ et non agere opera virtuosa . } Ipse ergo principatus \\\hline
1.4.7 & e dansse a occiosidat \textbf{ e muy de ligero se inclinan a fazer obras de luxia } et fazen se destenprados . & vacant ocio , \textbf{ et de leui inclinantur , | ut dent operam rebus venereis , } et fiant intemperati . \\\hline
1.4.7 & e muy de ligero se inclinan a fazer obras de luxia \textbf{ et fazen se destenprados . } Mas los poderosos et los prinçipes & ut dent operam rebus venereis , \textbf{ et fiant intemperati . } Potentes vero et principantes , \\\hline
1.4.7 & en el segundo libro dela rectorica \textbf{ que si los poderosos fazen tuerto a } alguons non ge lo fazen en pequanas cosas mas en grandes . Ca los poderosos estando en gerad sennorio & Ideo scribitur 2 Rhetor’ \textbf{ si potentes iniuriantur , } non iniuriantur in paruis , \\\hline
1.4.7 & que si los poderosos fazen tuerto a \textbf{ alguons non ge lo fazen en pequanas cosas mas en grandes . Ca los poderosos estando en gerad sennorio } por que son en logar digno de grand honrra & si potentes iniuriantur , \textbf{ non iniuriantur in paruis , | sed in magnis . } Potentes enim existentes in Principatu , \\\hline
1.4.7 & non entienden si non en grandes cosas e altas . \textbf{ Et por ende si contezca que ellos fagan tuertos a otros } non les fazen tuerto en pequennas mas en grandes & non tendunt nisi in magna et in ardua . \textbf{ Ideo si contingat eos aliis iniuriari , } non iniuriabuntur in paruis , \\\hline
1.4.7 & Et por ende si contezca que ellos fagan tuertos a otros \textbf{ non les fazen tuerto en pequennas mas en grandes } por que non curan de fazer & Ideo si contingat eos aliis iniuriari , \textbf{ non iniuriabuntur in paruis , | sed in magnis . } Non enim curabunt \\\hline
1.4.7 & non les fazen tuerto en pequennas mas en grandes \textbf{ por que non curan de fazer } pequano tuerto nin pequeno danno . & sed in magnis . \textbf{ Non enim curabunt } facere paruam offensam , \\\hline
1.4.7 & pequano tuerto nin pequeno danno . \textbf{ Mas o en ninguna cosa non faran danno alos otros o les faran grand danno . } Et pues que assi es menos son peleadores & facere paruam offensam , \textbf{ sed vel in nullo damnificabunt alios , | vel inferent magnum damnum . } Minus igitur sunt contumeliosi potentes , \\\hline
1.4.7 & los podero łos que los nicos \textbf{ por que non curan de fazer pelea } nin denuesto a ninguno en las pequennas cosas & Minus igitur sunt contumeliosi potentes , \textbf{ quam diuites : } quia quamlibet contumeliam inferre \\\hline
1.4.7 & e la nobleza \textbf{ en la mayor parte fazen al omne mas } desauentraado que auenturado . & potentatus et nobilitas , \textbf{ ut plurimum magis reddunt hominem infelicem quam felicem . } Qui enim sic diues est , \\\hline
2.1.1 & Lo primero se declara assi . Ca assi deuemos ymaginar \textbf{ que la natura non faze ninguna cosa en vano . } Et por ende de aquella cosa & Sic enim imaginari debemus , \textbf{ quod natura nihil facit frustra ; } ei ergo , quod naturaliter fit , \\\hline
2.1.1 & Et por ende de aquella cosa \textbf{ que naturalmente es fecha todas aquellas cosas le son naturales } sin las quales non se puede bien guardar en su ser . & ei ergo , quod naturaliter fit , \textbf{ naturalia sunt ea , } sine quibus non potest \\\hline
2.1.1 & sin las quales non se puede bien guardar en su ser . \textbf{ Ca la nata en vano faria las cosas } si las cosas naturales en ninguna manera non se pudiessen guardar en si mesmas e en su ser . & bene conseruari in esse . \textbf{ Frustra enim natura ageret , } si res naturales nullo modo conseruarentur in esse , \\\hline
2.1.1 & todas aquellas cosas \textbf{ que fazen a bien beuir } e sin las quales non puede el omne abondar & omnia illa , \textbf{ quae faciunt ad bene viuere , } et sine quibus non potest \\\hline
2.1.1 & Mas entre las o triscosas \textbf{ que fazen ala bondamiento dela uida del ome es la conpannia . } Et por ende naturalmente el omne & inter alia autem , \textbf{ quae faciunt | ad sufficientiam vitae humanae , } est societas , \\\hline
2.1.1 & Et por ende naturalmente el omne \textbf{ esaianlia aconpanable e conpanera Mas que la conpannia mucho faga a } conplimien to deuida de omne & est societas , \textbf{ naturaliter ergo homo est animal sociabile . | Quod autem societas maxime faciat } ad sufficientiam vitae humanae , \\\hline
2.1.1 & Et por ende lo muelen e lo çiernen \textbf{ e fazen ende pan e lo cuezen } por que se auianda conuenible al omne ¶ Et & ideo molitur , \textbf{ et depuratum fit inde panis , | et coquitur , } ut sit hominum congruus cibus . \\\hline
2.1.1 & es organo e instrumento sobre todos los instrumentos . \textbf{ Ca por la mano podemos fazer todos los instrumentos } e todo aquello que puede ser & est organum organorum . \textbf{ Nam per manum omnia organa , } et quicquid ad defensionem facit , \\\hline
2.1.1 & para auer uianda conueinble nin uestidura \textbf{ nin para fazer para si armas e jnstrumentos } por los quales se pueda defender de los enemigos . & ad habendum congruum victum et vestitum , \textbf{ et ad fabricandum sibi arma et organa , } per quae a contrariis defendatur : \\\hline
2.1.1 & por inclinaçion natra al \textbf{ faze su tela conuenible } avn que nunca aya visto otras arannas texer en essa misma manera & ut aranea ex instinctu naturae \textbf{ debitam telam faceret , } si nunquam vidisset \\\hline
2.1.1 & avn que nunca aya visto otras arannas texer en essa misma manera \textbf{ avn las golondrinas fazen su nido conueniblemente } avn que nunca ayan visto & araneas alias texuisse . \textbf{ Sic etiam et hirundines debite facerent nidum , } si nunquam vidissent alias nidificasse . \\\hline
2.1.1 & avn que nunca ayan visto \textbf{ a otras golondinas fazer su nido . } Et la perra por inclinaçion de natura es enssennada & Sic etiam et hirundines debite facerent nidum , \textbf{ si nunquam vidissent alias nidificasse . } Et canis ex instinctu naturae instruitur , \\\hline
2.1.1 & e el vno resçiba doctrina e enssenança del otro . \textbf{ Et por que esto non se puede fazer } si non biuieremos en vno con los otros omes . & et unus ab alio suscipiat disciplinam . \textbf{ Et quia hoc fieri non potest , } nisi simul cum aliis conuiuamus : \\\hline
2.1.1 & Ca por çierto aquellos que escogen de non beuir con los otros \textbf{ o esto fazen por que son muy pecadores o muy menguados } e non pueden sofrir conpanna de los otros & eligentes non conuiuere aliis , \textbf{ vel hoc est , | quia nimis sunt scelerati , } et non possunt societatem aliorum supportare , \\\hline
2.1.2 & Et comunidat de regno . \textbf{ Ca assi commo de muchas perssonas se faz la comunidat dela casa } assi de muchas casas se faz la comunidat de vn uarrio & videlicet , domus , vici , ciuitatis , et regni . \textbf{ Nam sicut ex pluribus personis fit domus , } sic ex multis domibus fit vicus , \\\hline
2.1.2 & Ca assi commo de muchas perssonas se faz la comunidat dela casa \textbf{ assi de muchas casas se faz la comunidat de vn uarrio } e de muchos uarrios se faz comunidat de çibdat & Nam sicut ex pluribus personis fit domus , \textbf{ sic ex multis domibus fit vicus , } et ex multis vicis ciuitas , \\\hline
2.1.2 & assi de muchas casas se faz la comunidat de vn uarrio \textbf{ e de muchos uarrios se faz comunidat de çibdat } e de muchas çibdades se faz comunidat de vn regno & sic ex multis domibus fit vicus , \textbf{ et ex multis vicis ciuitas , } et ex multis ciuitatibus regnum ; \\\hline
2.1.2 & e de muchos uarrios se faz comunidat de çibdat \textbf{ e de muchas çibdades se faz comunidat de vn regno } por la qual cosa & et ex multis vicis ciuitas , \textbf{ et ex multis ciuitatibus regnum ; } quare sicut singulares personae sunt partes domus , \\\hline
2.1.2 & Ca las casas son parte sin medio ninguno del varrio \textbf{ por que el uarrio es fecho delas casas } sin otro medio ninguno & sed vici sunt pars immediata , \textbf{ quia immediate ex domibus constituitur vicus ; } ciuitatis vero domus partes esse dicuntur , \\\hline
2.1.2 & mas las casas son dichas partes de la çibdat e del regno \textbf{ por que faziendo uarrio siguese } que pueden fazer çibdat e regno . & ciuitatis vero domus partes esse dicuntur , \textbf{ quia constituendo vicum , } ex consequenti constituere possunt ciuitatem , et regnum . \\\hline
2.1.2 & por que faziendo uarrio siguese \textbf{ que pueden fazer çibdat e regno . } Et por ende la comunidat dela casa & quia constituendo vicum , \textbf{ ex consequenti constituere possunt ciuitatem , et regnum . } Hoc ergo modo communitas domus se habet \\\hline
2.1.2 & e comienço delas çibdades en esta manera . \textbf{ Ca primeramente fue fecha vna casa } e despues cresçiendo los fijos e las fijas & hoc modo existit , \textbf{ quia primo facta fuit una aliqua domus : } sed crescentibus filiis et filiabus , \\\hline
2.1.2 & e por que non podieron por muchedunbre dellos morar todos en vna cala \textbf{ e por ende fizieron ellos otras casas iuntas ala primera casa } e assi en esta manera fue secho de muchas casas vn & et non valentibus praemultitudine habitare in domo illa , \textbf{ construxerunt sibi domos annexas : } et sic ex multis domibus factus fuit vicus . \\\hline
2.1.2 & es por vezindat delas casas \textbf{ las quales se fizieron } de muchedunbre de mietos e de fijos . & est ex conuicinia domorum , \textbf{ quas construxit multitudo collectaneorum , } et puerorum , siue filiorum . \\\hline
2.1.2 & por que non podien todos morar en vna casa \textbf{ por fuerça ouieron de fazer muchas casas } e fizieron vn uarrio . & et non valentibus habitare in una domo , \textbf{ compulsi sunt facere domos plures , } et constituere vicum . \\\hline
2.1.2 & por fuerça ouieron de fazer muchas casas \textbf{ e fizieron vn uarrio . } Et assi yendo & compulsi sunt facere domos plures , \textbf{ et constituere vicum . } Sic procedente generatione ipsorum , \\\hline
2.1.2 & Et dende se sigue \textbf{ que se fizo vna uilla o vna çibdat . } Et mas adelante acresçentadas las uillas & facta est pluralitas vicorum , \textbf{ et per consequens factum est castrum vel ciuitas ; } ulterius vero multiplicatis castris , \\\hline
2.1.2 & e las çibdades \textbf{ fue fecho vn prinçipado e vn regno . } Mas si ay otra manera en que se pueda fazer uarrio o çibdat o regno sin esta & et ciuitatibus , \textbf{ factus est principatus , et regnum . } Utrum autem alio modo sit \\\hline
2.1.2 & fue fecho vn prinçipado e vn regno . \textbf{ Mas si ay otra manera en que se pueda fazer uarrio o çibdat o regno sin esta } que dicha es & factus est principatus , et regnum . \textbf{ Utrum autem alio modo sit | possibilis generatio vici , ciuitatis , vel regni , } quam ex crescentia collectaneorum vel filiorum , \\\hline
2.1.2 & en qualquier manera \textbf{ que sea fecho el varrio o la çibdat } o el regno & quicquid tamen sit de hoc , \textbf{ et qualitercunque constituatur vicus , ciuitas , siue regnum , } semper sic se domus habet \\\hline
2.1.3 & ca dezimos algunas vezes \textbf{ que alguna çibdat fizo esto } o aquello non & Dicimus enim aliquando ciuitatem \textbf{ aliquam hoc fecisse , } non quod muri vel aedificia hoc egerint , \\\hline
2.1.3 & o aquello non \textbf{ por que lo fiziessen los muros o las casas } mas por que lo fizieron los que moran en la çibdat . & aliquam hoc fecisse , \textbf{ non quod muri vel aedificia hoc egerint , } sed quia in colae ciuitatis fecerunt illud . \\\hline
2.1.3 & por que lo fiziessen los muros o las casas \textbf{ mas por que lo fizieron los que moran en la çibdat . } Bien assi alguons suelen dezir & non quod muri vel aedificia hoc egerint , \textbf{ sed quia in colae ciuitatis fecerunt illud . } Sic aliqui dicere \\\hline
2.1.3 & Bien assi alguons suelen dezir \textbf{ que las sus casas fizieron esto } o aquello non por que las piedras fizieron esto & Sic aliqui dicere \textbf{ consueuerunt domos suas hoc operatas esse , } non quia lapides illud egerint , \\\hline
2.1.3 & que las sus casas fizieron esto \textbf{ o aquello non por que las piedras fizieron esto } mas por que los sus padres o los sus fujos lo fizieron . & consueuerunt domos suas hoc operatas esse , \textbf{ non quia lapides illud egerint , } sed quia sui progenitores fecerunt illud , \\\hline
2.1.3 & o aquello non por que las piedras fizieron esto \textbf{ mas por que los sus padres o los sus fujos lo fizieron . } Por la qual razon & non quia lapides illud egerint , \textbf{ sed quia sui progenitores fecerunt illud , } quare sicut communicatio ciuium ciuitas nominatur , \\\hline
2.1.3 & Enpero non pertenesçea el de tractar delan casa prinçipalmente \textbf{ segunt que la casa nonbra hedifiçio fecho de paredes . } mas deue tractar dela casa & ad ipsum principaliter determinare de domo , \textbf{ ut nominat aedificium constructum : } sed determinare debet de domo , \\\hline
2.1.3 & mas deue tractar dela casa \textbf{ que es fecha de paredes } en quanto ha orden ala & sed determinare debet de domo , \textbf{ quae est aedificium , } prout habet ordinem ad domum , \\\hline
2.1.3 & assi commo el carpento el arca \textbf{ que faz de los maderos ¶as } por que non puede alcançar la fin & Agens enim primo et principaliter intendit finem . \textbf{ Verum quia non potest habere finem , } nisi per ea , \\\hline
2.1.3 & por generaçion e por tienpo . \textbf{ Ca aquel que es engendrado primeramente por tp̃o es fecho moço } primero que sea fecho varon . & ut puer est prior viro generatione et tempore , \textbf{ qua prius tempore aliquis generatur } et efficitur puer quam efficitur vir : \\\hline
2.1.3 & Ca aquel que es engendrado primeramente por tp̃o es fecho moço \textbf{ primero que sea fecho varon . } Enpero el varon es primero que el moço & qua prius tempore aliquis generatur \textbf{ et efficitur puer quam efficitur vir : } verumtamen vir est prius puero perfectione et complemento , \\\hline
2.1.3 & Ca la casa es \textbf{ para fazer eluarrio } e el varrio & ad uicum , ciuitatem , et regnum . \textbf{ Est enim domus propter uicum , } uicus propter ciuitatem , \\\hline
2.1.3 & e el varrio \textbf{ para fazer la çibdat } e la çibdat para fazer el regno . & Est enim domus propter uicum , \textbf{ uicus propter ciuitatem , } ciuitas propter regnum . \\\hline
2.1.3 & para fazer la çibdat \textbf{ e la çibdat para fazer el regno . } Et pues que assi es la comunidat del uarrio & uicus propter ciuitatem , \textbf{ ciuitas propter regnum . } Communitas ergo vici est finis communitatis domus , \\\hline
2.1.3 & diga \textbf{ que la çibdat se faze de muchos uarrios } assi comm̃el uarrio se faze de muchas casas . & cum ipsemet dicat ciuitatem procedere ex multiplicatione vici , \textbf{ sicut et vicus procedit ex multiplicatione domorum , } intelligendum est ergo hoc \\\hline
2.1.3 & que la çibdat se faze de muchos uarrios \textbf{ assi comm̃el uarrio se faze de muchas casas . } Et por ende deue se entender & cum ipsemet dicat ciuitatem procedere ex multiplicatione vici , \textbf{ sicut et vicus procedit ex multiplicatione domorum , } intelligendum est ergo hoc \\\hline
2.1.4 & que la casa es comunidat \textbf{ segunt natura construyda e fecha para cada dia } que quiere dezir & Philosophum 1 Politicorum sic describere communitatem domus : \textbf{ videlicet , quod domus est communitas secundum naturam , } constituta quidem in omnem diem . \\\hline
2.1.4 & para las obras \textbf{ que se fazen de cada dia . } Mas en esta declaraçion alguna cosa es ya declarada en el capitulo & videlicet , quod domus est communitas secundum naturam , \textbf{ constituta quidem in omnem diem . } In hac autem descriptione aliquid declaratum est \\\hline
2.1.4 & han mester cada dia conprar e vender . \textbf{ pues que assi es la comunidat dela casa fue fecha para aquellas cosas } que auemos mester de cada dia . & vel venditione continue egeant . \textbf{ Communitas ergo domus facta fuit propter ea , } quibus quotidie indigemus . \\\hline
2.1.5 & Et pues que assi es en esta manera estas dos comuindades \textbf{ fazen ser la casa cosa natural . } Ca la comunidat del uaron e dela mugnies ordenada ala generacion . & in esse conseruari . \textbf{ Hoc ergo modo hae duae communitates faciunt domum esse quid naturale : } quia communitas viri et uxoris ordinatur ad generationem , \\\hline
2.1.5 & e la otra ala con leruaçion \textbf{ fazen la primera cala } por que sin ellas la primera casa non puede estar conueniblemente . & et alia conseruationi , \textbf{ dicuntur facere primam domum : } quia sine eis domus congrue \\\hline
2.1.5 & Ca dellas segunt \textbf{ que dize el philosofo se faze la primera casa de ligero puede paresçer } commo para el establesçemiento desta casa & dicitur domus prima : \textbf{ de leui videri potest , } quomodo ad constitutionem huius domus saltem requiruntur \\\hline
2.1.5 & assi commo son los tan pobres que ni pueden auer bueye nin cauallo nin asno \textbf{ para fazer sus obras } nin para arar & neque equos habere possunt \textbf{ tanquam ministros arantes } et scindentes terram , \\\hline
2.1.6 & sobredicho dos comuni dades \textbf{ fazen la primera casa } o nuene de saber et uaron & videlicet , viri et uxoris , domini et serui , \textbf{ facere domum primam . } Sed tamen , \\\hline
2.1.6 & que es para la saluaçion \textbf{ fazen la primera casa . } ¶ Pues que assi es & quae est propter saluationem , \textbf{ faciunt domum primam . } Sic ergo saluatio comparatur ad rem generatam : \\\hline
2.1.6 & engendrada la natura es acuçiosa çerca de su salud . \textbf{ Empero faze engendrar cosa semeiante de ssi non es } assi conpado alas cosas natraales & solicitatur natura circa salutem eius ; \textbf{ producere tamen sibi simile , } non sic comparatur ad res naturales : \\\hline
2.1.6 & por que non puede la cosa natural \textbf{ luego que es fecha fazer otra semeiante } assi mas conuiene que ella primeramente sea acabada & cum est res naturalis , \textbf{ potest sibi simile producere , } sed oportet prius ipsam esse perfectam . \\\hline
2.1.6 & Enpero non puede luegero engendrar \textbf{ nin puede fazer } luego otro su semeinante & non tamen statim potest generare , \textbf{ nec statim potest sibi simile producere , } sed oportet prius ipsum esse perfectum : \\\hline
2.1.6 & saluo son de natraa del primer omne . \textbf{ Mas fazer e engendrar semeiante desi es de natura de omne ya acabado . } Ca quando el ome es primero & de ratione hominis primi : \textbf{ sed producere sibi simile , | est de ratione hominis iam perfecti . } Cum enim primo homo est , \\\hline
2.1.6 & Et la natura luego es acuçiosa de su salud . \textbf{ Enpero non puede fazer su semeiante } si non quando el omne es ya acabado . & et natura statim est solicita de salute eius ; \textbf{ non potest tamen sibi simile producere , } nisi sit iam perfectus . \\\hline
2.1.6 & commo ouiesse dicho primeramente \textbf{ que la comiundat del omne e dela muger e del sennor e del sieruo fazen la primera casa . } Despues en el segundo capitulo desse dicho libro & cum prius dixisset \textbf{ communitatem viri et vxoris , domini et serui facere communitatem primam : } postea in sequenti capitulo praedicti libri ait , \\\hline
2.1.6 & estonçe toda cosa es acabada \textbf{ quan do puede fazer } e engendrar su semeiante & Tunc unumquodque perfectum est , \textbf{ cum potest sibi simile producere . } Ad hoc enim quod aliquid sit perfectum , \\\hline
2.1.6 & Ca para ser la cosa acabada \textbf{ non conuiene que faga sienpre } e engendre su semeiante & Ad hoc enim quod aliquid sit perfectum , \textbf{ non oportet | quod sibi simile producat , } sed quod possit sibi simile producere : \\\hline
2.1.6 & e engendre su semeiante \textbf{ mas que aya poder delo fazer . } Ca la perfection e el cunplimiento & quod sibi simile producat , \textbf{ sed quod possit sibi simile producere : } perfectio enim consideranda est \\\hline
2.1.6 & quando tiene su materia proprea presente en que puede obrar \textbf{ e non puede fazer su semeiante . } Por la qual razon commo el propre o fazedor en la generaçion sea el mas lo . & cum praesente proprio passiuo , \textbf{ non producat sibi simile ; } quare cum proprium actiuum generationis sit masculus , \\\hline
2.1.6 & cosastales commo quier que non sean essençiales ala bien andança . \textbf{ Enpero fazen algunan mostrança } e algunan nobleza dela bien andança çiuil . & licet non sint essentialia felicitati : \textbf{ faciunt tamen ad quandam claritatem felicitatis politicae ; } unde Philosophus 1 Ethic’ ait , \\\hline
2.1.6 & y alguno que sea senñor \textbf{ e alguno que faga su mandado . } Por la qual cosa commo en la comunindat del uaron & ibi principans , \textbf{ et aliquid obsequens . } Quare cum in communitate maris et foeminae , \\\hline
2.1.7 & La terçera de parte de obras \textbf{ que han de fazer . } Ca prouado es en el primero deste segundo libro & Secunda ex parte procreationis prolis . \textbf{ Tertia ex parte operum . } Probabatur enim in primo capitulo huius secundi libri , \\\hline
2.1.7 & para querer engendrar otro semeiable \textbf{ assi por que entre los omes esto se faze conueniblemente por el casamiento . } por ende el omne es naturalmente & quare cum homo et omnia animalia naturaliter inclinentur , \textbf{ ut velint producere sibi simile , quia in hominibus hoc debite sit per coniugium , } homo naturaliter est animal coniugale . \\\hline
2.1.7 & luego man amano son departidas las obras del uaron e dela \textbf{ mugnỉca las obras del uaron son en fazer aquellas cosas } que son de fazer fuera de casa . & diuisa sunt opera viri , et uxoris . \textbf{ Opera enim uiri uidentur esse in agendo , } quae sunt fienda extra domum : \\\hline
2.1.7 & mugnỉca las obras del uaron son en fazer aquellas cosas \textbf{ que son de fazer fuera de casa . } Mas las obras dela muger son en guardando las alfaias dela casa & Opera enim uiri uidentur esse in agendo , \textbf{ quae sunt fienda extra domum : } opera uero uxoris in conseruando suppellectilia , \\\hline
2.1.7 & assi commo dize el philosofo en el viij ̊ libro delas ethicas . \textbf{ Ca ordenar assi los bienes propreos al bien comun fazen avn } abastamientode uida . & ut dicitur 8 Ethic’ . \textbf{ Nam sic propria ordinare | ad bonum commune , } facit ad quandam sufficientiam vitae . \\\hline
2.1.7 & que aquel que non quiere beuir conuigalmente e con su muger o esto es \textbf{ por que quiere mas libremente fazer forniçio . } Et por ende escoge assi uida mas baxa que de omne . & uel hoc est , \textbf{ quia uult liberius fornicari ; } quare eligit sibi uitam infra hominem , \\\hline
2.1.7 & commo quier que non biuen assi commo omes . \textbf{ Enpero non fazen por esso mal } ca son assi commo dioses o commo angeles e son meiors que omes . & ut homines , \textbf{ non tamen propter hoc male agunt : } quia sunt quasi dii , \\\hline
2.1.8 & que ningun repoyo entre el marido e la muger non ouo del tienpo \textbf{ que la çibdat de Roma fue fecha } fasta çiento et cinquanta a nons . & a tempore , \textbf{ quo urbs Romana fuit condita , } usque ad vigesimum et quingentesimum annum , nullum intercessit . \\\hline
2.1.8 & ca non auer cuydado de los fijos del Rey \textbf{ mas puede fazer danno a todo el regno } que non auer cuydado de los fijos & Incuria enim regiae prolis \textbf{ plus potest } inferre nocumenti ipsi regno , \\\hline
2.1.9 & Ca commo el matermonio sea cosa natural \textbf{ en qual manera se deua fazer } conueniblemente puede se muy bien demostrar & Nam cum coniugium sit quid naturale : \textbf{ quomodo debito modo fieri debeat } maxime inuestigari potest \\\hline
2.1.9 & non ay diferençia \textbf{ nin faz fuerça } si vn mas lo fuere ayuntado a muchͣs fenbras & ad nutritionem filiorum , \textbf{ non refert } utrum unus masculus pluribus coniungatur foemellis ; \\\hline
2.1.9 & e avn non ay departimiento \textbf{ nin faz fuerça } si mientra dura el parto si el mas so biua en vno con la fenbra & utrum unus masculus pluribus coniungatur foemellis ; \textbf{ nec etiam refert , } utrum durante partu masculus conuiuat foeminae . \\\hline
2.1.11 & mas que esto sea contra razon \textbf{ e que non deua ser fecho casamiento } nin ayuntado con padre e madre & contra rationis dictamen : \textbf{ et quod cum parentibus et consanguineis } nimia consanguineitate coniunctis \\\hline
2.1.11 & e muy conuenible \textbf{ que auemos de fazer al padre e ala madre } e alos parientes muy çercanos ¶ & Prima sumitur ex debita reuerentia , \textbf{ quae est parentibus et consanguineis exhibenda . } Secunda , ex bono quod ex coniugio consurgit . \\\hline
2.1.11 & que les deuen \textbf{ e les son tenudos de fazer . } Avn essa misma manera non les conuiene de casar con los parientes & propter mutuam reuerentiam , \textbf{ quam sibi inuicem debent . } Sic etiam non licet \\\hline
2.1.11 & entre el marido e la muger en sus obrassacado con dispenssaçion \textbf{ e en algun caso non se deuen fazer matermonios entre tales perssonas } que son muy cercanas por parentesto . & quae in agendis inter coniuges congrue reseruari non possunt \textbf{ ( nisi ex dispensatione et in casu ) inter personas nimia consanguineitate coniunctas } non sunt connubia contrahenda . \\\hline
2.1.11 & que son muy cercanas por parentesto . \textbf{ Et pues que assi es conuiene a todos los çibdadanos de non fazer matermonios } con quales quier perssonas . & non sunt connubia contrahenda . \textbf{ Decet ergo omnes ciues | non contrahere coniugia } cum quibuscunque personis ; \\\hline
2.1.11 & Por ende la razon natural dize \textbf{ que los matermonios non son de fazer } entre estos tales & dictat naturalis \textbf{ ratio coniugia contrahenda esse inter illos } qui non sunt nimia consanguineitate coniuncti : \\\hline
2.1.11 & losçibdadanos \textbf{ de non fazer mater monio } entre perssonas muy ayuntadas por parentesço . & Decet ergo omnes ciues \textbf{ non contrahere cum personis } nimia consanguineitate coniunctis : \\\hline
2.1.11 & Et pues que assi es conuiene a todos los çibdadanos \textbf{ de non fazer casamientos entre perssonas } que son muy ayuntadas en parentesço & quod oporteret eos nimium vacare venereis . \textbf{ Decet ergo omnes ciues non inire connubia } cum personis nimia consanguinitate coniunctis ; \\\hline
2.1.11 & e enlas obras çiuiles . \textbf{ Et por ende non se deue fazer mater moion } en grado muy cercano de parentesco . & circa salutem regni \textbf{ et circa ciuilia opera non diligenter intendant . } In nimis ergo propinquo gradu consanguineitatis \\\hline
2.1.11 & que se puede escusar \textbf{ puede se fazer casamiento } e mater moion ¶ & vel magnum malum vitandum , \textbf{ contrahi poterit copula coniugalis . } Bonorum autem \\\hline
2.1.12 & por las iniunas e desigualdades \textbf{ que se fazen } entre si sele una tan discordias e peleas . & et inaequalitates \textbf{ quas inter se exercent , } consurgunt dissensiones et bella : \\\hline
2.1.12 & e muchedunbre de amigos . \textbf{ Ca segunt el philosofo en la rectorica los omes de buenamente fazen iniustiçias e tuertos } quando pue den . & et pluralitas amicorum . \textbf{ Nam secundum Philosophum in Rhetoricis , | homines libenter iniustificant , } cum possunt . \\\hline
2.1.13 & e que non ame ser uagarosa . \textbf{ mas que ame fazer obras non seruiles } nin de sieruo . & et quod non amet esse ociosa , \textbf{ sed diligat facere opera non seruilia . } Quae autem sunt opera non seruilia \\\hline
2.1.13 & ¶ Et pues que assi es todas aquellas cosas \textbf{ que paresçen de fazer } para escusar la fornicaçion & omnia ergo illa , \textbf{ quae videntur facere } ad fornicationem vitandam , \\\hline
2.1.13 & e para engendrar conueniblemente los fijos deuen ser demandadas enla muger . \textbf{ Ca beemos que la grandeza del cuerpo faze al bien dela generaçion de los fijos . } Por que los fijos enla quantidat del cuerpo & et ad prolem debite producendam , \textbf{ in coniuge quaeri debent . Videmus autem quod magnitudo corporis facit ad bonum prolis . } Nam filii in quantitate corporis \\\hline
2.1.13 & Ca paresçe que la fermosura dela muger \textbf{ non solamente faze ala bondat de los fijos } mas avn faze & videtur enim pulchritudo coniugis \textbf{ non solum facere | ad bonitatem prolis , } sed etiam ad fornicationem vitandam , \\\hline
2.1.13 & non solamente faze ala bondat de los fijos \textbf{ mas avn faze } para esquiuar la fortcaçion & ad bonitatem prolis , \textbf{ sed etiam ad fornicationem vitandam , | ad quam vitandam est ipsum coniugium ordinatum . } Viso , quomodo quantum \\\hline
2.1.13 & quanto la destenprança delas mugers \textbf{ dellos puede fazer mayor danno e enpeçemiento que la destenprança delas mugers de los otros . } Et pues que al sy es conuiene & tanto tamen hoc decet Reges et Principes , \textbf{ quanto intemperantia coniugum ipsorum plus nocumenti inferre potest , | quam intemperantia coniugum aliorum . } Decet ergo coniuges temperatas esse . \\\hline
2.1.13 & que las muger ssean tenpradas . \textbf{ Et avn les conuiene aellas de amar fazer buenas obras . } Ca quando alguna persona esta de uagar mas ligeramente es inclinada a aquellas cosas & Decet ergo coniuges temperatas esse . \textbf{ Decet eas etiam amare operositatem : } quia cum aliqua persona ociosa existat , \\\hline
2.1.14 & La segunda de parte delas obras \textbf{ que son de fazer ¶ } La primera razon paresçe assi . & ex parte modi regendi . \textbf{ Secunda vero ex parte operum fiendorum . } Prima via sic patet . \\\hline
2.1.14 & en qual manera se departe este gouernamiento \textbf{ de aquel de parte delas obras que se han de fazer . } Ca el padre assi deue & quomodo differt hoc regimen \textbf{ ab illo | ex parte operum fiendorum . } Nam pater sic debet \\\hline
2.1.15 & assi commo de aquel que mueue toda la natura . \textbf{ Pues que assi es commo en el artifiçio fecho de maestro sabio } non es fallada ninguna cosa sobeia & tanquam a mouente naturam totam . \textbf{ Sicut ergo in artificio facto | a sapienti artifice } non inueniretur \\\hline
2.1.15 & Pues que assi es todo lo \textbf{ que faze la natura } e lo que es apareiado & nec deficit in necessariis . \textbf{ quicquid ergo natura agit , } et quicquid natura praeparatur , \\\hline
2.1.15 & ally la natura \textbf{ non faze ninguna cosa } atal commo fazien los ferreros que fazien vn & Nam ( ut Philosophus ait ibidem ) \textbf{ nihil tale natura facit , } sicut faciebant fabri formantes Delphicum gladium . \\\hline
2.1.15 & non faze ninguna cosa \textbf{ atal commo fazien los ferreros que fazien vn } cuchielbauenturado para muchos obras & nihil tale natura facit , \textbf{ sicut faciebant fabri formantes Delphicum gladium . } Apud Delphos enim sic fiebant gladii propter pauperes , \\\hline
2.1.15 & que llama una cuchiello de dios Et \textbf{ por que entre los del fes fazien se cuchiellos } para los pobres & sicut faciebant fabri formantes Delphicum gladium . \textbf{ Apud Delphos enim sic fiebant gladii propter pauperes , } ita quod unus gladius deseruiebat pluribus officiis : \\\hline
2.1.15 & Conuiene saber que por que los pobres non podian auer \textbf{ muchos instrumentos fazian fazer vn instrͤde } que podiesen vsar en muchos ofiçies & utputa pauperes non valentes plura habere instrumenta , \textbf{ faciebant aliquod instrumentum fabricari , } quo possent ad plura uti officia . \\\hline
2.1.15 & que podiesen vsar en muchos ofiçies \textbf{ mas la natura non faze } assi mas por que se fagan meior las cosas natraales & quo possent ad plura uti officia . \textbf{ Natura autem non sic agit , } sed ut melius fiant res naturales , \\\hline
2.1.15 & mas la natura non faze \textbf{ assi mas por que se fagan meior las cosas natraales } sienpre la natura ordena vna cosa prinçipalmente avn ofiçio . & Natura autem non sic agit , \textbf{ sed ut melius fiant res naturales , } semper ad unum officium principaliter \\\hline
2.1.16 & Ca el non saber delas cosas \textbf{ genera les nos faria muchͣs uezes } que non sopiessemos las particulares . & circa mores despiciendi non sunt : \textbf{ quia ignorantia uniuersalium saepe facit particularia ignorare : } ipsis tamen uniuersalibus sermonibus sunt particularia addenda , \\\hline
2.1.16 & por que non pueda bien entender \textbf{ e que non pueda faze sus obras libremente . } ¶ Et pues que assi es los que nasçen de tal casamiento & ne possit bene speculari , \textbf{ et ne possit libere exequi actiones suas . } Nascentes ergo ex tali coniugio \\\hline
2.1.16 & non solamente los fijos resçiben ende danno \textbf{ mas avn ellas mismas se fazen destenpradas e orguollosas } por que quando alguna perssona mucho & non solum filii inde laeduntur , \textbf{ sed etiam ipsae uxores efficiuntur | intemperatae et lasciuae : } quia quaelibet persona \\\hline
2.1.16 & dize fue costunbre entre los gentiles \textbf{ de fazer logar speçial de oracion } por el parto delas moças & fuit consuetudo apud gentiles \textbf{ speciale oraculum facere } pro partu iuuencularum , \\\hline
2.1.17 & madremas pueden guardar las ceraturas \textbf{ e fazer las mas fuertes } ¶La segunda razon para demostrar esto & magis possunt conseruare suos foetus , \textbf{ et eos perfectiores faciunt . } Secunda via ad inuestigandum hoc idem , \\\hline
2.1.17 & enmagresçe \textbf{ assi por que se faze } mayorconuerssion dela uianda en el cuerpo del omne . & non sic laedit corpora virorum , nec sic attenuat ea , \textbf{ eo quod maior sit } ibi conuersio alimenti . \\\hline
2.1.17 & esso mismo se toma de parte dela disposiçion del ayre . \textbf{ Ca el çierço faze el ayre puro } mas el ayre faze lo turbio . & sumitur ex aeris dispositione . \textbf{ Nam boreas reddit aerem purum : } auster vero reddit ipsum turbulentum . \\\hline
2.1.17 & Ca el çierço faze el ayre puro \textbf{ mas el ayre faze lo turbio . } Ca segunt el philosofo en el libro de los mechaurores & Nam boreas reddit aerem purum : \textbf{ auster vero reddit ipsum turbulentum . } Nam secundum Philosophum in Meteoris , \\\hline
2.1.17 & meiora se la conplision \textbf{ de aquellos que estan en el e fazen se meiores las generaçiones . } Por la qual cosa en taltp̃o & melioratur complexio existentium in eo , \textbf{ et fiunt meliores generationes . } Quare tali tempore magis est \\\hline
2.1.18 & en general \textbf{ non ouiemos cuydado de fazer capitulo espeçial delas costunbres delas mugers } mas diemos lo a entender & ubi uniuersaliter tractabamus de moribus , \textbf{ non curauimus speciale capitulum facere de moribus mulierum : } sed supposuimus coniecturandum esse de huiusmodi moribus \\\hline
2.1.18 & mucho es de alabar en ellas ser uergon cosas \textbf{ ca por la uerguença dexan de fazer muchs cosas torpes } que non dexarien de fazer & laudabile est in ipsis esse uerecundas : \textbf{ quia propter uerecundiam multa turpia dimittunt } quae non dimitterent , \\\hline
2.1.18 & ca por la uerguença dexan de fazer muchs cosas torpes \textbf{ que non dexarien de fazer } si non las constriniessen la cadena dela uerguença . & quia propter uerecundiam multa turpia dimittunt \textbf{ quae non dimitterent , } nisi eas uerecundiae cathena constringeret . \\\hline
2.1.18 & e piadat dellos \textbf{ por que cada vno de ligero se inclina a fazer alos otros lo que el quarne } que los otros fiziessen a el . & uolunt aliis misereri et compati ipsis : \textbf{ quare cum de facili quis inclinetur | ad faciendum aliis , } quod ab eis uult fieri sibi ; \\\hline
2.1.18 & por que cada vno de ligero se inclina a fazer alos otros lo que el quarne \textbf{ que los otros fiziessen a el . } Et por ende los uicios de ligero se enpiadan sobre los otros . & ad faciendum aliis , \textbf{ quod ab eis uult fieri sibi ; } senes de facili super aliis miserentur . \\\hline
2.1.18 & que apenas auria omes en el mundo tan desuergonçados \textbf{ que pudiessen fazer cosas tan torpes } Mas esta terçera cosa maguera pueda ser alabada en los buenos . & quod vix inuenirentur viri adeo inuerecundi \textbf{ ut possint tanta turpia operari . } Hoc autem tertium \\\hline
2.1.18 & que non son parleras nin baraiadoras \textbf{ esto fazen mas por uerguença que por razon . } Por la qual cosa quando se mueuen non & et non litigent , \textbf{ magis hoc faciunt | ex verecundia quam ex ratione . } Quare cum motae sunt , \\\hline
2.1.19 & que peor pueden pronunçiar . \textbf{ Ende leemos que algunos philosofos lo fizieron } assi los quales commo ouiessen las lenguas enbargadas & quae deterius proferre possent . \textbf{ Unde et aliquos Philosophos legimus sic fecisse , } qui cum essent impeditae linguae , \\\hline
2.1.19 & que peor pronunçiauan \textbf{ e assi fueron fechos bien fablantes . } Et pues que assi es en essa misma manera deuemos fazer çerca las obras & quas deterius proferebant , \textbf{ facti sunt eloquentes . } Hoc ergo modo et circa opera se habet . \\\hline
2.1.19 & e assi fueron fechos bien fablantes . \textbf{ Et pues que assi es en essa misma manera deuemos fazer çerca las obras } por que quando alguno vee & facti sunt eloquentes . \textbf{ Hoc ergo modo et circa opera se habet . } Cum enim quis se vel alium videt \\\hline
2.1.19 & e malas \textbf{ fazen algunan sospecha dela desonestad delas mugers . } Para que el padre sea çierto de sus fijos & et impudica \textbf{ quandam suspitionem adgenerent de incontinentia coniugis ; } ut pater sit certus de sua prole , \\\hline
2.1.19 & que quanto la mug̃res mas firme e mas estable \textbf{ tanto mayor firmeza faze en su marido } para quel guarde fialdat . & et firma , \textbf{ tanto maior credulitas adgeneratur viro , } ut ei debitam fidem seruet . \\\hline
2.1.20 & Ca el vso tenpdo por \textbf{ fuerça faze tres males ¶ } Lo & ø \\\hline
2.1.20 & amanziella el apetito \textbf{ e faz las muger sser destenpradas . } ¶ Digo lo primero que se destruye & tertio immoderat appetitum , \textbf{ et reddit coniuges intemperatas . } Destruitur enim \\\hline
2.1.20 & conplidamente \textbf{ lo que ha de fazer } Onde es prouado de suso por la autoridat del philosofo & ut non possit sufficienter considerare ; \textbf{ unde et supra per auctoritatem Philosophi probabatur , } quod talia , \\\hline
2.1.20 & manziella mucho el apetito \textbf{ e faz las mugers e los omes destenprados } por que quanto mas dan obra a cosas luxiosas & nimis maculat appetitum , \textbf{ et reddit coniuges intemperatas . } Nam quanto plus datur opera venereis actibus , tanto plus incitatur appetitus \\\hline
2.1.20 & mugertanto mas se abiua el apetito para vsar della \textbf{ e sienpra se faz mas destenprado . } Pues que assi es conuiene a todos los çibdadanos de vsar & tanto magis incitatur ad utendum , \textbf{ et semper intemperantior redditur . } Decet ergo omnes ciues \\\hline
2.1.20 & Ca en todas las obras la sabiduria es menester \textbf{ por que las obras se fagan en tienpo conuenible } e en logar conuenible & Nam in omnibus operibus discretio est adhibenda , \textbf{ ut fiant tempore debito , } loco conuenienti , et modo congruo . \\\hline
2.1.21 & e las condiconnes delas perssonas son conueinbles e honestas \textbf{ si se fizieren commo deuen e commo cunple . } Ca conuiene alos maridos de proueer conueniblemente a sus & et conditionibus personarum \textbf{ debite et ordinate fiant , | sunt licita et honesta . } Decet enim viros \\\hline
2.1.21 & las quales tanne andronico peri patetico \textbf{ en el libro que fizo delas uirtudes . } Et estas tres uirtudes son humisdat tenprança sinpleza . & quam tangit Andromicus Peripateticus in libello \textbf{ quem fecit de virtute . } Huiusmodi triplex virtus , \\\hline
2.1.21 & nin se afeytan por vana eglesia \textbf{ mas esto fazen por fazer plazer a sus maridos e por los tirar de forncacion e de luxuria . } Mas estonçe son tenpradas & sed agunt \textbf{ ut suis viris placentes , | eos a fornicatione retrahant . } Tunc vero sunt moderatae , \\\hline
2.1.21 & en el fallesçimiento del su conponimiento \textbf{ ¶ Lo primero si esto se fiziere } por ꝑeza o por negligençia & contingit delinquere circa defectum . \textbf{ Primo si hoc fiat } ex pigritia et negligentia . \\\hline
2.1.21 & aquel que es mas vil en su uestidura \textbf{ faze se mas soberuio si cre } e que por ende sera mas alabado de lonsomes que otro . & qui vilior est in habitu , \textbf{ magis superbus efficitur , } si credat ex hoc ab hominibus commendari . \\\hline
2.1.21 & que non sean sofisticas \textbf{ e engannolas en querer a feytes e aposturas } que non son suyas ¶ Lo segundo que sean homillosas & ne sint sophisticae , \textbf{ quaerentes fucum et figmentum . } Secundo , ut sint humiles , \\\hline
2.1.22 & que biuen bien \textbf{ e fazen lo que deuen son denostadas a tuerto de sus maridos } si ellos fueren muy çelosos & et debite se gerentes \textbf{ increpantur a viris , } si contingat eos nimis esse zelotypos : \\\hline
2.1.22 & e que sin su culpa sospecha los maridos mal dellas \textbf{ la qual cosa fazen los maridos muy çelosos . } Por ende non lo pueden sofrir las mugersen paciençia . & quod sine causa calumnientur , et quod earum uiri sine culpa suspicentur de ipsis mala , \textbf{ quod faciunt uiri zelotypi ; } non possunt patienter sufferre : \\\hline
2.1.23 & en que es conplimiento de sabiduria . Et por ende conuiene que aquellas cosas \textbf{ que faze la natura } que las faga ordenadamente & Natura enim cum moueatur ab intelligentiis , et a Deo , \textbf{ in quo est suprema prudentia ; } oportet quod agat ordinate et prudenter . \\\hline
2.1.23 & que faze la natura \textbf{ que las faga ordenadamente } e con sabiduria & in quo est suprema prudentia ; \textbf{ oportet quod agat ordinate et prudenter . } Prudentis est enim cito se expedire , \\\hline
2.1.24 & e por ende son mas reueladoras delas poridades que los uatones . \textbf{ Ca si el defendimiento a crrsçienta el desseo tanto son las mugers mas prestas para fazer mal . } Las cosas defendidas & quam sint viri . \textbf{ Nam si prohibitio auget concupiscentiam : | tanto sunt proniores mulieres } ad faciendum prohibita quam masculi , \\\hline
2.1.24 & luego que algunas ꝑssonas les comiençan a lisongar \textbf{ e a Reyr en su faz dellas } luego ellas a aquella perssona tienen por amiga & statim cum aliqua persona eis applaudet , \textbf{ et ridet in facie earum , } credunt ipsam esse amicam , \\\hline
2.2.4 & La primera razon se puede prouar assi . \textbf{ Ca quanto el amor mas dura tanto se faze mayor . } Mas el amor de los padres alos fijos es de & Prima via sic patet . \textbf{ Nam quanto amor magis durat , | tanto vehementior efficitur : } amor autem parentum ad filios diuturnior est , \\\hline
2.2.4 & que los padres alos fijos . \textbf{ por la qual razon commo el amor faga algun ayuntamiento los fijos } assi conmomas ayuntados & quam econuerso . \textbf{ Quare cum amor quandam unionem importet , } filii tanquam magis uniti et magis propinqui parentibus , \\\hline
2.2.4 & e los padres alos fijos nasçe dela cosa \textbf{ que es faze dora ala cosa que es fecha . } Et desçende delo mas alto alo mas baxo & Amor ergo parentum circa filios procedit \textbf{ a causa ad effectum , } et a superiori ad inferius : \\\hline
2.2.4 & que avn los fiios non pue den oyr \textbf{ nin sofrir los deuuestos que fazen los otros a sus padres . } Mas los padres non toman tan grand indignaçion & Naturale est enim quod etiam usque ad auditum \textbf{ contumelias parentum sustinere non possint . } Nam sic indignantur parentes de contumelia filiorum , \\\hline
2.2.4 & para allegar les los bienes \textbf{ que les faz menestra . } Commo allegar les los bienes & ut congregant eis bona : \textbf{ et congregare aliis bona , } et solicitari circa eorum vitam , \\\hline
2.2.5 & es tal que cree de ligero esto tanto \textbf{ mas lo deuen fazer los que tienen la ley } e la fex̉ana & eo quod illa aetas sit simpliciter creditiua : \textbf{ tanto magis hoc debent efficere } profitentes christianam fidem , \\\hline
2.2.5 & Tu puedes ver \textbf{ quanto faze la costunbre cuydando en las leyes de los moros o de los x̉anos o de otras gentes } quales quiera que sean & quod consuetum est , \textbf{ leges ostendunt , } in quibus fabulas et apologos \\\hline
2.2.5 & en las quales fallaras fabliellas e trufas mas allegadas al coraçon \textbf{ que non fazen las uerdades . } Et pues que assi es & plus possunt propter consuetudinem , \textbf{ quam si essent veritates . } Si ergo leges Gentilium continentes \\\hline
2.2.5 & dando razon de todos nuestros fechos . \textbf{ assi que aquellos que bien fezieron yran ala uida perdurable . } Mas aquellos que fizieron mal yran al fuegon del infierno ¶ Et pues que assi es todos los çibdadanos deuen ser acuçiosos de sus fiios & reddituri de factis propriis rationem . \textbf{ Ita quod qui bona egerunt , | ibunt in vitam aeternam : } qui vero mala , in ignem aeternum . \\\hline
2.2.5 & assi que aquellos que bien fezieron yran ala uida perdurable . \textbf{ Mas aquellos que fizieron mal yran al fuegon del infierno ¶ Et pues que assi es todos los çibdadanos deuen ser acuçiosos de sus fiios } por que enla moçedat sean enssennados en esta fe & ibunt in vitam aeternam : \textbf{ qui vero mala , in ignem aeternum . | Debent ergo omnes ciues solicitari circa proprios filios , } ut ab infantia instruantur in hac fide ; \\\hline
2.2.6 & e en los aueres del mundo \textbf{ por que puedan fazer asus fijos Ricos } e acorrer los & et circa numismata , \textbf{ ut possint subuenire filiis } quantum ad indigentiam corporalem : \\\hline
2.2.6 & estonçe son de enssennar en bueans costunbres \textbf{ e deuen les ser fechos amonestamientos conuenibles . } ¶ La segunda razon para prouar esto mismo se toma del fallesçimiento dela razon & sunt instruendi ad bonos mores , \textbf{ et debent eis fieri monitiones debitae . } Secunda via ad inuestigandum hoc idem , \\\hline
2.2.6 & mas que ala meatad \textbf{ por que la podamos fazer uenir al medio . } En essa misma manera nos en fuyendo delas cosas & ultra medium inclinatur \textbf{ ut possit ad medium redire : } sic et nos in fugiendo delectabilia , \\\hline
2.2.7 & lenguaie que es dicho latin o lenguage de letras \textbf{ el qual fizieron tan ancho } e tan conplido & vel idioma literale : \textbf{ quod constituerunt adeo latum et copiosum , } ut per ipsum possent omnes \\\hline
2.2.8 & assi commo vna gruessa logica . \textbf{ Ca assi commo son de fazer razones sotiles } en las sçiençias especulatinas & quasi quaedam grossa dialectica . \textbf{ Nam sicut fiendae sunt rationes subtiles } in scientiis naturalibus \\\hline
2.2.8 & en las sçiençias especulatinas \textbf{ assi son de fazer razones gruessas en las sciençias morales } que tractan delas obras & in scientiis naturalibus \textbf{ et in aliis scientiis speculabilibus , sic fiendae sunt rationes grossae | in scientiis moralibus , } quae tractant de agibilibus . \\\hline
2.2.8 & que tractan delas obras \textbf{ que auemos de fazer . } por la qual cosa & in scientiis moralibus , \textbf{ quae tractant de agibilibus . } Quare sicut necessaria fuit dialectica , \\\hline
2.2.9 & e deue ser entendudo en las obras \textbf{ que ha de fazer } e deue ser bueno en uida & Debet esse sciens in speculabilibus . \textbf{ Prudens in agibilibus . } Et bonus in vita . \\\hline
2.2.9 & en qual manera deue ser prudente e sabio en las obras \textbf{ que ha de fazer } por que enssenne los moços en bueans costunbres . & esse prudens in agibilibus \textbf{ ut instruat in bonis moribus . } Ad huiusmodi autem prudentiam describendam , \\\hline
2.2.9 & assi sabio en las cosas \textbf{ que ha de fazer } por que sea menbrado delas cosas & esse sic prudens in agibilibus , \textbf{ ut sit memor , } cautus , prouidus , et circumspectus . \\\hline
2.2.9 & que han de uenir Sabio en departir el mal del bien e acordado e prouado \textbf{ enlo que ha de fazer ¶ } Ca primero deue ser menbrado e acordado delas colas passadas . & cautus , prouidus , et circumspectus . \textbf{ Debet enim esse memor , } recolendo praeterita . \\\hline
2.2.9 & por que las cosas falssas non sean mezcladas alas uerdaderas . \textbf{ Assi en las obras que son de fazer } conuiene al omne de ser sabio & ne falsa admisceantur veris : \textbf{ sic in agendis conuenit hominem esse cautum , } ne mala admisceantur bonis : \\\hline
2.2.9 & por buenas palauras \textbf{ lo que han de fazer . } Enpero si fiziere por obra el contrario & quantumcunque puerorum doctor eis verba bona proponeret , \textbf{ si tamen opere contraria faceret , } iuuenes illi exemplo inducti \\\hline
2.2.9 & lo que han de fazer . \textbf{ Enpero si fiziere por obra el contrario } delo que dize los moços & quantumcunque puerorum doctor eis verba bona proponeret , \textbf{ si tamen opere contraria faceret , } iuuenes illi exemplo inducti \\\hline
2.2.9 & Enpero quanto ala sabiduria moral delas obras \textbf{ que son de fazer } conuiene le que el doctor sea menbrado e prouado e sabio e acatado . & tam inuentorum quam intellectorum . \textbf{ Quantum vero ad prudentiam agibilium , } decet ipsum esse memorem , \\\hline
2.2.10 & Ca la fabla de las cosas torpes \textbf{ faze en nos memoria delas cosas delectables e non conuenibles . } Ca los fechos acresçientan la cobdiçia & ipsa enim locutio turpem facit \textbf{ in nobis memoriam delectabilium illicitorum : } qua facta , augetur concupiscentia circa illa : \\\hline
2.2.10 & que non respondan luego a lo que les den e andan . \textbf{ Ca commo quier que ninguno non es fecho a desora grande } e las moços non pueden ser luego acabados nin sabios . & ne statim ad interrogata respondeant . \textbf{ Licet enim nullus repente fiat summus , } et iuuenes non statim possint \\\hline
2.2.10 & mas que conuiene Et desto nos viene \textbf{ de fazer cosas estrannas } e de fazer cosas torpes & propter quod oportet \textbf{ ab ipsis iuuenibus extranea | facere } quaecunque sunt turpia , \\\hline
2.2.10 & de fazer cosas estrannas \textbf{ e de fazer cosas torpes } e de faz cosas & facere \textbf{ quaecunque sunt turpia , } et quaecunque infectionem habent : \\\hline
2.2.10 & e de fazer cosas torpes \textbf{ e de faz cosas } que han manziella . & quaecunque sunt turpia , \textbf{ et quaecunque infectionem habent : } quia quae primo aspiciuntur , \\\hline
2.2.11 & por que esto non solamente enpeesçe al alma \textbf{ por que se fazen golosos } aquellos que toman el comer muy cobdiçiosamente & Nam hoc non solum nocet animae , \textbf{ quia nimis ardenter } et auide sumentes cibum fiunt gulosi et intemperati ; \\\hline
2.2.11 & Et assi se sigue \textbf{ que non faga nudrimiento conueinble¶ Lo . iij . } pecan si toma la uianda torpemente e suzia mente . & ei dominari non possit , non bene digeritur , \textbf{ et per consequens non causat debitum nutrimentum . } Tertio delinquitur , \\\hline
2.2.11 & commo se deuan auer çerca las viandas . \textbf{ Ca ninguno adesora non se faze grande . } Et por ende abasta alos mocos & ut se habeant circa cibos . \textbf{ Nam nullus repente fit summus . } Sufficit autem eos paulatim et pedetentim instruere , \\\hline
2.2.12 & Et por ende en la hedat de los moços deuemos guardar \textbf{ que non se fagan destenprados . } Ca la tenprança ha de ser puesta çerca de tres cosas . & in puerili aetate cauendum est \textbf{ ne iuuenes efficiantur intemperati . } Temperantia autem circa tria est adhibenda : \\\hline
2.2.12 & Ca non solamente la uianda tomada \textbf{ commo non conuiene faze destenprança mas avn el beuer faze esso mismo } Pues que assi es conuiene alos moços de ser guardados & non solum esse abstinentes , \textbf{ ut non efficiantur gulosi } ex sumptione cibi : \\\hline
2.2.12 & Pues que assi es conuiene alos moços de ser guardados \textbf{ non solamente que se non fagan gollosos por el comͣ } mas avn les conuiene de ser mesurados & ut non efficiantur gulosi \textbf{ ex sumptione cibi : } sed etiam decet eos esse sobrios , \\\hline
2.2.12 & mas avn les conuiene de ser mesurados \textbf{ que non se fagan beodos } por el beuer o por el much vino . & sed etiam decet eos esse sobrios , \textbf{ ut non efficiantur ebrii } ex sumptione potus . \\\hline
2.2.12 & Ca quanto pertenesçe alo presente el vino tomado \textbf{ sin mesura faze tres males . } ¶ El primer mal es que abiua el omne asa lux̉ia . & ( quantum ad praesens spectat ) \textbf{ tria mala causat . } Primo , quia venerea prouocat . \\\hline
2.2.12 & ¶ El primer mal es que abiua el omne asa lux̉ia . \textbf{ Ca escalentado el cuerpo faze se enł omne mayor inclinaçion alas obras de luxia . } Onde el vino tomado & Primo , quia venerea prouocat . \textbf{ Cum enim corpore calefacto maior fiat | incitatio ad actus venereos , } vinum quod maxime calorem efficit immoderate sumptum , \\\hline
2.2.12 & Onde el vino tomado \textbf{ destenpradamente faze en el omne grand calentraa } e abiualo a destenprança de lux̉ia . & incitatio ad actus venereos , \textbf{ vinum quod maxime calorem efficit immoderate sumptum , } incitat ad incontinentiam nimiam . \\\hline
2.2.12 & e ençiende la sangre \textbf{ por ende faze el omne de mayor coraçon } e mas sanudo la qual cosa & vinum , quod propter sui caliditatem inflammat sanguinem , \textbf{ reddit hominem animosum et irascibilem : } quo facto facilius prouocatur \\\hline
2.2.13 & que vsa de razon e de entendimiento . \textbf{ Et pues que assi es para fazer el omne obras conueinbles non es inclinado conplidamente } por la natura & ut homo , qui utitur ratione et intellectu ; \textbf{ et ad agendum sibi opera debita , } non sufficienter inclinatur ex natura . \\\hline
2.2.13 & mucho a menudo o leuna tan los onbros \textbf{ o fazen aquellas cosas } que non siruen en ninguna cosa ala fabla . & vel erigunt humeros , \textbf{ vel faciunt alia , } quae ad locutionem nihil deseruiunt . \\\hline
2.2.13 & por que sir una alas obras \textbf{ que entienden fazer . } Ca fazer alguons mouimientos de los mienbros & ut habeant tales gestus , et ut sic utantur motibus membrorum , \textbf{ ut deseruiant ad opera quae intendunt . } Nam agere aliquos motus membrorum \\\hline
2.2.13 & que entienden fazer . \textbf{ Ca fazer alguons mouimientos de los mienbros } que non siruen ala obra & ut deseruiant ad opera quae intendunt . \textbf{ Nam agere aliquos motus membrorum } non deseruientes operi intento , \\\hline
2.2.13 & que non siruen ala obra \textbf{ que entienden fazer . } Et estos salle & Nam agere aliquos motus membrorum \textbf{ non deseruientes operi intento , } vel procedit ex insipientia mentis , \\\hline
2.2.13 & e çerca toda delectaçion que es en ellas . \textbf{ Ca por esto se faze algᷤ } destenprado e temeroso por que de ligero cada vno va a loçania & et circa delectationem in ipsis : \textbf{ nam ex hoc efficitur } quis intemperatus et timidus . \\\hline
2.2.13 & e de ligero saltan en locania . \textbf{ ¶ Lo segundo la grand blandura delas vestiduras faze al omne temeroso } por que las armas del fierro han en ssi alguna dureza & sed molles et de facili in lasciuiam prorumpunt . \textbf{ Secundo , nimia mollicies vestium reddit hominem timidum . } Nam cum arma ferrea in se \\\hline
2.2.13 & dubdan de tomar las armas \textbf{ e fazen semedrosos . } Et por que mucho conuiene alos mançebos & dubitant arma arripere , \textbf{ et efficiuntur timidi . } Iuuenes , maxime cum ad aliam aetatem venerint , \\\hline
2.2.14 & Et non han vso acabado de razon de ligero son aduchos por los conpannones \textbf{ e de ligo les fazen los conpannones razones } por que ciean & de facili inducuntur a sociis , \textbf{ et de facili persuadetur eis , } ut credant bona sensibilia esse sequenda . \\\hline
2.2.15 & por que de ligero enferman los moços \textbf{ e se fazen de mala disposicion enel cuerpo } si en el tienpo en que manian se acostunbraren a beuer vino & De facili enim aegrotantur pueri \textbf{ et efficiuntur male dispositi in corpore , | si tempore quo } ut plurimum pascuntur lacte \\\hline
2.2.15 & que el vso que toman en el \textbf{ frio faze buena } disposiconn en los moços & unde idem Philosophus ait , \textbf{ quod exercitium ad frigora facit } bonum habitudinem in pueris \\\hline
2.2.15 & e los de nuruega de vannar los sus fijos en los trios muy frios \textbf{ por que los fagan muy fuertes } Enpero deuemos entender & in fluminibus frigidis balneare filios , \textbf{ ut eos fortiores reddant . } Attendendum est tamen , \\\hline
2.2.15 & por que segunt el philosofo el mouimiento tenprado \textbf{ en los mocos faze seys bienes ¶ } Lo primero faze el cuerpo mas sano & motus temperatus \textbf{ in pueris quatuor bona facit . } Primo , quia reddit corpora magis sana : \\\hline
2.2.15 & en los mocos faze seys bienes ¶ \textbf{ Lo primero faze el cuerpo mas sano } e mas tenprado & in pueris quatuor bona facit . \textbf{ Primo , quia reddit corpora magis sana : } moderatum enim exercitium \\\hline
2.2.15 & e el mouimiento tenprado \textbf{ en qual si quier hedat aprouecha ala sanidat ¶L segundo faze los cuerpos mas ligeros . } Ca si encomienço los moços se acostunbran a algunos mouimientos & in quacunque aetate videtur \textbf{ ad sanitatem proficere . | Secundo , quia reddit corpora agibilia . } Si enim a principio assuescant pueri \\\hline
2.2.15 & Ca si non fueren acostunbrados a algunos mouimientos \textbf{ e fazen se pesados e ꝑezosos e mal dispuyestos } ¶ & nam nisi ad aliquales motus assuescant , \textbf{ fiunt graues , pigri , et inertes . } Tertio facit ad augmentum . \\\hline
2.2.15 & ¶ \textbf{ Lo terçero faze el mouimiento } e tenprado a acresçentamiento del cuerpo & fiunt graues , pigri , et inertes . \textbf{ Tertio facit ad augmentum . } Nam eo ipso quod temperatum exercitium iuuat \\\hline
2.2.15 & tenprado ayuda al cozimiento dela uianda \textbf{ e faze abuean disposiçion del cuerpo } Et por ende se sigue & ad digestionem ipsam , \textbf{ et facit ad bonam dispositionem corporis , } sequitur quod sit \\\hline
2.2.15 & que sea aprouechoso alacresçentamiento del cuerpo . \textbf{ Ca commo el acresçentamiento se faga del nudrimiento } aquellas cosas & quoddam proficuum ad augmentum . \textbf{ Nam cum augmentum fiat } ex ipso alimento faciente corpus bene dispositum , \\\hline
2.2.15 & aquellas cosas \textbf{ que fazen el cuerpo bien dispuesto } e quel fazen que se cere bien son & Nam cum augmentum fiat \textbf{ ex ipso alimento faciente corpus bene dispositum , } et quod bene nutriatur , \\\hline
2.2.15 & que fazen el cuerpo bien dispuesto \textbf{ e quel fazen que se cere bien son } aprouechabłsal acresçentamiento del ¶ & ex ipso alimento faciente corpus bene dispositum , \textbf{ et quod bene nutriatur , } et alatur , sunt proficua ad augmentum . \\\hline
2.2.15 & dize \textbf{ que conuiene alos moços de faz quales quier mouimientos pequanos } para soldar los mienbros & unde Philosophus 7 Poli’ ait , \textbf{ quod expedit in pueris | facere motus quoscunque } et tantillos ad solidandum membra , \\\hline
2.2.15 & en tanto lo alaba el philosofo que diz que luego enł comienço de su nasçençia \textbf{ deuen fazer alguons instrumentos } en que se mueun a los moços & ut ab ipso primordio natiuitatis dicat , \textbf{ fienda esse aliqua instrumenta , } in quibus pueri vertantur , \\\hline
2.2.15 & escusase que non sean ꝑezosos \textbf{ e fazen se los cuerpos mas ligeros } Otrossi avn deuen rezar alos moços alguas estorias despues que comiençan at entender las significaçiones delas palabras . & et per moderatum ludum vitatur inertia , \textbf{ et redduntur corpora agiliora . } Sunt etiam pueris recitandae aliquae fabulae , \\\hline
2.2.15 & que non lloren \textbf{ por esse mismo defendimiento se faze } que retengan en ssi el spun e el eneldo . & Nam cum pueri a ploratu cohibentur , \textbf{ ex ipsa prohibitione fit , } ut retineant spiritum et anhelitum . \\\hline
2.2.15 & en el septimo libro delas politicas \textbf{ faze a fortaleza del cuerpo . } Et pues que assi es por que los moços sean mas fuertes & secundum Philosophum septimo Politicorum , \textbf{ facit ad robur corporis . } Ut ergo pueri robustiores fiant , \\\hline
2.2.16 & Ca por la desordenacion del appetito de los sesos \textbf{ se faze la desordenaçion en la uoluntad . } Et pues que assi es si las cobdiçias se tienen de parte del cuerpo . & nam ex inordinatione appetitus sensitiui \textbf{ redundat inordinatio in voluntate . } Si ergo concupiscentiae se tenent ex parte corporis , \\\hline
2.2.16 & e todas las cosas \textbf{ que fazen fazen las mucho } mas que deuen asi que quando aman am̃a much̃ . Etrͣndo & et de facili mentiuntur , \textbf{ et omnia faciunt valde , } ita quod cum amant nimis amant , \\\hline
2.2.16 & comiençan de trebeiar trebeian much̃ . \textbf{ Et assi que todas las cosas que fazen } fazen la ssienpre con sobrepuiança . & cum incipiunt ludere nimis ludunt , \textbf{ et in caeteris aliis semper excessum faciunt , } adhibenda est cautela \\\hline
2.2.16 & Et assi que todas las cosas que fazen \textbf{ fazen la ssienpre con sobrepuiança . } Por ende deuemos poner cautela & cum incipiunt ludere nimis ludunt , \textbf{ et in caeteris aliis semper excessum faciunt , } adhibenda est cautela \\\hline
2.2.16 & por que non sean mintrosos mas uerdaderos \textbf{ nin fagan todas las cosas } que fazen mas que deuen mas en todas sus obras & ne sint mendaces sed veridici , \textbf{ nec omnia agant valde } sed in suis actibus \\\hline
2.2.16 & nin fagan todas las cosas \textbf{ que fazen mas que deuen mas en todas sus obras } e en todas sus palabras & nec omnia agant valde \textbf{ sed in suis actibus } et sermonibus moderationem accipiant . \\\hline
2.2.16 & vso de razon non seanda todo mal apareiados ala sçiençia deuen ser acostunbrados alas otras artes delas \textbf{ quales ya fiziemos mençion . } ni emos dessuso que trs cosas deuemos entender c̃ca los fijos . & assuescendi sunt ad alias artes , \textbf{ de quibus fecimus mentionem . } Dicebatur supra circa filios tria intendenda esse , \\\hline
2.2.17 & por que puedan tomar trabaios conuenibles la qual cosa \textbf{ mayormente se pue de fazer } si vsaren los fijos a ex̉çiçios & ut possint debitos subire labores , \textbf{ quod maxime fieri contingit , } si ad debita exercitia assuescant . \\\hline
2.2.18 & por al si fuere tenprado a todos es aprouechable \textbf{ por que fazen en algua manera } para auer salud & omnibus videtur esse proficua , \textbf{ eo quod faciat ad quandam sanitatem , } et ad quandam bonam dispositionem corporis : \\\hline
2.2.18 & enłbso delas armas \textbf{ por que el mouimiento conuenible del cuerpo faze el cuerpo mas fuerte } e mas rezio para que pueda sofrir & circa armorum usum . \textbf{ Exercitatio enim corporalis debita | reddit corpus robustius , } ut facilius duriciem armorum sustinere possit . \\\hline
2.2.18 & por que los trabaios del cuerpo \textbf{ por los quales se faz la carne dura } enbargan la sotileza del alma e del entendimiento . & corporales igitur labores , \textbf{ ex quibus redditur caro dura , } impediunt subtilitatem mentis . \\\hline
2.2.18 & que el alma en seyendo \textbf{ e enfolgado se faze sabia } ca por se assesegar & quod anima in sedendo \textbf{ et quiescendo fit prudens . } Nam per sessionem et quietem redditur caro mollis , \\\hline
2.2.18 & ca por se assesegar \textbf{ e por folgar se faze la carne muelle } por la qual somos despuestos & et quiescendo fit prudens . \textbf{ Nam per sessionem et quietem redditur caro mollis , } per quam sumus apti ad speculandum ; \\\hline
2.2.18 & Mas por el trabaio \textbf{ e por el mouimiento se faze la carne dura } por la qual cosa se enbarga la sotileza del entendimiento . & per quam sumus apti ad speculandum ; \textbf{ per laborem vero et motum efficitur caro dura , } per quam impeditur mentis sublimitas . \\\hline
2.2.18 & nin de una assi escusar los trabaios del cuerpo \textbf{ por que le fagan mugeriles } en tal manera & fugere corporales labores ; \textbf{ ut effecti quasi muliebres , } nec pro defensione regni \\\hline
2.2.18 & que los otros \textbf{ por que por tales trabaios la carne dellos non se faga dura } e enbarguela sotileza del entendimiento . & minus sunt assuescendi ad corporales labores quam alii , \textbf{ ne propter huiusmodi labores } caro eorum indurata \\\hline
2.2.19 & ¶ La primera se toma por que sea tirado \textbf{ alas fijas manera de mal fazer . } ¶ La segunda por que non se fagan desuergonçadas ¶ la terçera & Prima sumitur , \textbf{ ut tollatur filiabus commoditas malefaciendi . } Secunda , ne fiant inuerecundae . \\\hline
2.2.19 & alas fijas manera de mal fazer . \textbf{ ¶ La segunda por que non se fagan desuergonçadas ¶ la terçera } por que non se fagan locanas nin desonestas & ut tollatur filiabus commoditas malefaciendi . \textbf{ Secunda , ne fiant inuerecundae . } Tertiae , ne fiant lasciuae et impudicae . \\\hline
2.2.19 & ¶ La segunda por que non se fagan desuergonçadas ¶ la terçera \textbf{ por que non se fagan locanas nin desonestas } por que comualmente todos los mas de los uarones son incliados para mal . & Secunda , ne fiant inuerecundae . \textbf{ Tertiae , ne fiant lasciuae et impudicae . } Communiter enim quasi omnes viri proni sunt ad malum , \\\hline
2.2.19 & Onde el philosofo en la rectorica dize \textbf{ que los omes en la mayor parte fazen mal } quando pueden . & unde et Philosophus in Rheto’ vult \textbf{ quod homines | ut plurimum male faciant , } cum possunt . \\\hline
2.2.19 & por la qual cosa se dize vn prouerbio \textbf{ que el azma de furtar faze el ladron } e por ende si en los omes & propter quod et prouerbialiter dicitur , \textbf{ Furandi commoditas facit furem . } Si ergo in viris , \\\hline
2.2.19 & e por ende por que non sea dada a ella \textbf{ sazina de mal fazer son de guardar conueniblemente } e deuenles defender & ne ergo eis detur commoditas malefaciendi , \textbf{ sunt debite custodiendae , } et prohibendae sunt \\\hline
2.2.19 & esto mismo se toma \textbf{ por que non se fagan desuergonçadas } ca toda aquella cosa & hoc idem , sumitur , \textbf{ ne fiant inuerecundae . } Nam omne insolitum est \\\hline
2.2.19 & Et pues que assi es entre todas las cosas \textbf{ que fazen las moças uergonçosas en } acatamientode los uarones es de non las acostunbrar entre los uarones & Inter cetera ergo , \textbf{ quae reddunt puellas verecundas | in aspectu virorum , } est non assuescere eas inter gentes . \\\hline
2.2.19 & acostunbran se auer los omes \textbf{ e fazen se familiares dellos } e tirasse dellas la uerguença & assuescant virorum aspectibus , \textbf{ fiunt familiares eis , } et tollitur ab ipsis verecundia \\\hline
2.2.19 & por el qual freno se retrahen \textbf{ que non fagan mal } ca en las fenbras & quo trahuntur , \textbf{ ne male agant . } Nam in foemina , \\\hline
2.2.19 & e mayormente delas moças es la uerguença \textbf{ por que non puedan sallir a fazer cosas torpes . } Et pues que assi es cosa conuenible es de defender es alas mocas & et potissime puellarum , \textbf{ ne prorumpant in turpia , } videtur esse verecundia . \\\hline
2.2.19 & nin anden uagando allende \textbf{ nin a quande por que non se fagan desuergoncadas } o por que non se tire dellas la uerguenca & a discursu et euagatione , \textbf{ ne fiant inuerecundae , } vel ne tollatur ab eis verecundia , \\\hline
2.2.19 & ¶ La terçera razon se toma \textbf{ por que non se fagan orgullosas } e loçanas e desonestas & Tertia sumitur , \textbf{ ne fiant lasciuae et impudicae : } puellae enim si debito modo sub custodia teneantur , \\\hline
2.2.19 & mas si se acostunbran a vsar con los omes \textbf{ fazen se mansas } en tal manera & etiam valde syluestria \textbf{ si assuescant conuersationibus hominum , domesticantur , } et permittunt se tangi et palpari : \\\hline
2.2.19 & o de beuir conellos \textbf{ fazen se assi commo montes } mas delar la conpannanan de los omes & et a conuersatione virorum sint inconsuetae , \textbf{ quasi syluestres ab ipsorum societate } difficilius ad lasciuiam \\\hline
2.2.20 & e en uagar \textbf{ mas que las fagan vsar } en alguas obras conuenibles e honestas ¶ & ut nolint ociosae viuere , \textbf{ sed ament se exercitare } circa opera aliqua licita et honesta . \\\hline
2.2.20 & por que dende salga fructo e prouecho \textbf{ por que se fagan ellas bueans e uirtuosas . } Mas si alguno demandare & et utilitas , \textbf{ et efficiantur bonae et virtuosae . } Si autem quaeratur \\\hline
2.2.21 & que es parlera en algua manera \textbf{ se faze muy familiar } e muy conpanera en algua & eo ipso quod est loquax , \textbf{ quodammodo se nimis familiarem exhibet , } et quodammodo se contemptibilem reddit ; \\\hline
2.2.21 & por que si las mugers fueren callantias en manera conueinble \textbf{ por que assi non se fazen familiares } e conp̃aneras alos omes & Si enim mulieres sint modo debito taciturnae , \textbf{ quia se sic non familiares exhibent , } eorum consortium videtur magis abesse , \\\hline
2.3.1 & que cunplen la mengua corporal \textbf{ e las que fazen agua edamiento } e a cunplimiento dela uida & in qua agitur de iis quae supplent indigentiam corporalem , \textbf{ et quae faciunt ad conseruationem } et ad sufficientiam uitae : \\\hline
2.3.1 & çerca aquellas cosas \textbf{ que fazen para beuir bien } e que son meneste para conplimiento dela uida & ut patrifamilias cuius est domum gubernare debite solicitetur \textbf{ circa ea quae faciunt ad bene viuere , } et quae requiruntur ad sufficientiam vitae : \\\hline
2.3.1 & por prop̃os estrumentos pueda alcançar estas cosas \textbf{ que fazen al abastamiento dela uida . } Ca el arte del gouernamiento dela & quia per haec tanquam propria organa consequi poterit , \textbf{ quae faciunt ad sufficientiam vitae : } habet enim ars gubernationis domus propria organa , \\\hline
2.3.1 & por que por ellos non queda \textbf{ non finca ninguna cosa fecha de fuera } Mas finca perfeccion enłalma & ø \\\hline
2.3.1 & mas los estrumentos delas artes mecanicas son factiuos e obradores \textbf{ por que por ellos finca algua cosa fecha de fuera } assi commo el arca . & ø \\\hline
2.3.1 & mas el gouernamiento dela casa la regla es sabiduria que es derecha razon de todas las cosas \textbf{ que ha de fazer . } Et pues que alsi es & sed in gubernatione domus \textbf{ regula est prudentia . } Sicut ergo ars differt a prudentia , \\\hline
2.3.1 & por que el arte mecanica es derecha razon delas cosas \textbf{ que ha de fazer de fuera . } Et por tal arte sale algcosa fechͣ & ab organis gubernationis . \textbf{ Differt autem prudentia ab arte , } quia ars est recta ratio factibilium et per artem resultat aliquid factum in materia extra : \\\hline
2.3.1 & Mas la sabiduria es derecha razon delas cosas \textbf{ que son de fazer } e por ella non sale propreamente ninguna cosa fechͣ de fuera & quia ars est recta ratio factibilium et per artem resultat aliquid factum in materia extra : \textbf{ sed prudentia est recta ratio agibilium , } et per eam non proprie \\\hline
2.3.1 & accioono alguna perfection \textbf{ en aquella que la faze } Mas los estrumentos delas artes me cauicas son fazedores & sed magis resultat aliqua actio , \textbf{ et aliqua perfectio in agente . } Organa ergo moechanicorum erunt factiua , \\\hline
2.3.1 & que es delas cosas \textbf{ que se fazen de fuera } mas los estrumentos del arte del gouernamiento dela casa son actiuos & quia sunt organa artis , \textbf{ quae est de factibilibus : } sed organa gubernationis erunt actiua , \\\hline
2.3.1 & que ha de ser çerca las cosas \textbf{ que se han de fazer } e fincan en el alma . & quia sunt organa prudentiae , \textbf{ quae circa agibilia habet esse . } Ars ergo gubernationionis domus licet \\\hline
2.3.1 & e delas possesiones \textbf{ que fazen a gouernamiento } e a conplimiento dela uida dela uida dela casa . & de domibus , numismatibus , et possessionibus : \textbf{ quae faciunt ad conseruationem , } et ad sufficientiam vitae . \\\hline
2.3.2 & por que los instrumentos sin alma \textbf{ por si non pueden fazer las obras } para que son fechos . & quia organa inanimata per se ipsa exercere \textbf{ non possunt illud , } ad quod sunt facta . \\\hline
2.3.2 & por que puedan conplir su obra \textbf{ propraa e fazer sus ofiçios . } Ca non es cosa conueinble & vel organa carentia ratione , \textbf{ ut opus proprium possint implere . } Indignum est enim \\\hline
2.3.2 & mas cosaco nueible es \textbf{ que fagan estas cosas } por los siruientes medianeros . & aut aliqua talia exercere : \textbf{ sed congruentius est haec } per medios ministros efficere . \\\hline
2.3.3 & agnitul traa es la sotileza dela obra \textbf{ por que la morada sea sotilmente et conueiblemente fecho } Lo otro es el tenpramiento del ayre & est industria operis , \textbf{ ut aedificium sit subtiliter | et debite factum : } et temperamentum aeris , \\\hline
2.3.3 & mas non es conuenible morada al magnifico \textbf{ si non es fecha } por muy marauillosa maestria . & praepare habitationem decentem : \textbf{ non est autem decens habitatio , } nisi magnifico opere , \\\hline
2.3.3 & e alos prinçipes parte nesçe \textbf{ de fazer tan grandes cosas } e de estroyr & quod Principes decet \textbf{ sic magnifica facere , } et talia aedificia construere , \\\hline
2.3.3 & e de estroyr \textbf{ e de fazer tales moradas } que el pueblo que lo viere & sic magnifica facere , \textbf{ et talia aedificia construere , } quod populus ea videns , \\\hline
2.3.3 & cada vno del pueblo se guar da \textbf{ que non faga bolliçio } contra el prinçipe & quilibet ex populo retrahitur , \textbf{ ne dissensionem faciat contra Principem , } si aspiciat ipsum tantum , \\\hline
2.3.3 & ca la grandeza delas moradas \textbf{ maguer non se de una fazer aparesçençia } nin a grandezaauana eglesia & Magnitudo enim aedificiorum licet \textbf{ non sit fienda } ad ostentationem et inanem gloriam : \\\hline
2.3.3 & enpero conuiene alos Reyes \textbf{ e alos prinçipes de fazer moradas costosas e nobles } assi commo el su estado demanda & decet tamen Reges et Principes , \textbf{ ne in contemptum habeantur a populo , | facere aedificia magnifica , } prout requirit decentia status , \\\hline
2.3.3 & ueinblemente en las casas \textbf{ que ellos fazen a } conuiene & sed etiam multitudo ministrorum debite commorari possint \textbf{ in aedificiis constructis , } oportet ipsa esse magnifica . \\\hline
2.3.3 & por las quales podemos conosçer \textbf{ en que ayre es de fazer la morada } o dize & ex quibus cognoscere possumus , \textbf{ in quo aere sit aedificium construendum . } Dicit enim salubritatem aeris primo \\\hline
2.3.3 & por que las aguas estantias \textbf{ en la mayor parte se fazen gruessas } e se podresçen en essa misma manera el ayre & eo quod aquae stantes \textbf{ ut plurimum ingrossantur et putrescunt : } sic aer reclusus in vallibus , \\\hline
2.3.3 & por que non ha mouimiento libre \textbf{ fazese commo gruesso e non sano . } Et pues que assi es deuemos cuydar & eo quod non habeat liberum motum , \textbf{ quasi ingrossatur , | et efficitur non salubris ; } est ergo propter salubritatem aeris considerandum \\\hline
2.3.3 & Et pues que assi es deuemos cuydar \textbf{ por la sanidat del ayre en las moradas que auemos de fazer } que non se fagan & est ergo propter salubritatem aeris considerandum \textbf{ in aedificiis construendis , } ut non fiat talis constructio \\\hline
2.3.3 & por la sanidat del ayre en las moradas que auemos de fazer \textbf{ que non se fagan } nin se costruyan en los valłs muy baxos & in aedificiis construendis , \textbf{ ut non fiat talis constructio } in vallibus infimis . \\\hline
2.3.3 & cuydarl \textbf{ que aquel lugar en que deuemos fazer la morada sea guardado delas timebras dela } meblaca en alguna parte dela tierra & ut locus ille , \textbf{ in quo est aedificium construendum , | sit a nebularum tenebris absolutus . } Nam in aliqua parte terrarum , \\\hline
2.3.3 & si lo podemos mudar \textbf{ non son de fazer las moradas en aquel loguar . } ¶ Lo terçero que muestra la sanidat del ayte es cuydar enlos moradores & Ideo si vitari potest , \textbf{ non sunt ibi aedificia construenda . } Tercium , quod declarat salubritatem aeris , \\\hline
2.3.3 & que estan enł \textbf{ por que si en algun loguar queremos fazer algunas moradas } si contezca que alguos moren cerca aquella t rrason de uer & in ipso \textbf{ si enim alicubi aedificare volumus , } si contingat circa regionem illam aliquos habitare \\\hline
2.3.4 & Enpero si tanta fuere la neçesidat \textbf{ que nos costranga de fazer alli moradera } e non podamos auer & de quibus fecimus mentionem . \textbf{ Quod si tamen aedificandi necessitas urgeat , } nec tamen ibi aquae salubris sit copia , \\\hline
2.3.4 & segunt manda aquel philosofo \textbf{ pala dio faze alli çisternas } e algibes en que se puedan coger las aguas dela luuia & nec tamen ibi aquae salubris sit copia , \textbf{ est ibi ( secundum Palladium ) construenda cisterna , } in qua pluuiales aquae colligendae sunt . \\\hline
2.3.4 & por que por el nadamiento de los peces \textbf{ el agua estante semeie en lignieza al agua que corre ¶ Visto en qual manera son de fazer las moradas } quanto ala salud delas agunas finca de ver & ut horum natatu aqua \textbf{ stans agilitatem currentis imitetur . | Viso , qualiter est aedificium construendum quantum } ad salubritatem aquae : \\\hline
2.3.4 & quanto ala salud delas agunas finca de ver \textbf{ en qual manera son de fazer } quanto ala orden del mundo & restat videre , \textbf{ qualiter construendum sit | quantum } ad ordinem Uniuersi . \\\hline
2.3.4 & segunt que demanda la morada \textbf{ que es de fazer son de penssar tres cosas } conuiene a saber la condicion del çielo & prout requiri aedificium construendum , \textbf{ sunt tria consideranda , } videlicet conditio caelestis : \\\hline
2.3.4 & trauiesso menos cal entra a engendra que el derecho ¶ \textbf{ Lo segundo en fazer la morada } auemos de penssar el departimiento de los uientos & quam directus . \textbf{ Secundo in aedificando aedificio } attendenda est diuersitas ventorum : \\\hline
2.3.4 & e del çierco \textbf{ por que faze el aire mas puro } e es mas sano & Nam ventus septentrionalis , \textbf{ eo quod puriorem aerem facit , } salubrior esse videtur ; \\\hline
2.3.4 & en que los omes muy ligeramente enferman \textbf{ son de fazer alguas camaras contrarias al uiento set enteronal } e del çierço & in quo homines facilius infirmantur , \textbf{ aedificandae sunt aliquae camerae oppositae vento septentrionali , } ut in eis salubrior custodiatur vita . \\\hline
2.3.4 & auemos de penssar la disposiconn de las tierras \textbf{ por que en tal loguar sean fechos las moradas } por que puedan auer uergeles e fructales e huertas & Tertio quantum ad ordinem uniuersi consideranda est dispositio terrarum , \textbf{ ut in tali loco aedificium construatur , } cui viridaria et pomeria esse possunt connexa : \\\hline
2.3.4 & mas podrien se dezer alguas otras cosas \textbf{ mas particulares en el fazer delas moradas } assi commo diremos quales deuen ser los çilleros e las bodeguås & per ea ad hylaritatem et sanitatem confert . \textbf{ Essent autem in aedificiis construendis } quaedam alia particularia dicenda ; \\\hline
2.3.5 & e cosas neçessarias ala uida alas aianlias non acabadas much \textbf{ mas esto faze e deue fazer alas aianlias acabadas . } Et por ende el philosofo dize en el primero libro delas politicas & praeparat nutrimentum et necessaria vitae , \textbf{ multo magis hoc facit animalibus perfectis . } Ideo ait Philosophus primo Polit’ \\\hline
2.3.5 & por que la natan engendro \textbf{ e fizo estas cosas senssibles para el omne . } ca assi commo es dicho & quia natura produxit \textbf{ huiusmodi sensibilia propter hominem . } Sumus enim quodammodo nos finis omnium , \\\hline
2.3.5 & assi conmodio al omne a beuir \textbf{ assi fizo las aianlias e las plantas e las yerbas } por que el omne se enssenorear se dellas & Natura ergo sicut dedit boni viuere , \textbf{ sic fecit animalia , plantas , et herbas : } ut homo eis dominaretur , \\\hline
2.3.6 & delas possessiones delas cosas de fuera en quento \textbf{ esto faze al gouernamiento dela } casaca si los omes en la mayor parte non ouiessen los desseos corronpidos & de possessione rerum exteriorum , \textbf{ prout facit ad regimen et gubernationem domus . } Si enim homines \\\hline
2.3.6 & commo contesçe \textbf{ quando alguna cosa es mandada a muchs siruientes que la fagan . } Ca quando esto se manda & sicut accidit , \textbf{ cum aliquid committitur pluribus ministris : } cum enim hoc fit , \\\hline
2.3.6 & e se tira \textbf{ que non faga } aquello qual es mandado & quilibet ministrorum retrahitur , \textbf{ ne faciat quod mandatur , } sperans alium implere \\\hline
2.3.6 & e las tierras et las otras cosas \textbf{ que fazen fructo sean labrados } el pho enł primero libro delas politicas & ø \\\hline
2.3.7 & esto es por algun açidente en quanto aquellos en algua manera \textbf{ fazen o fizieron algua cosa desagnisada contra ellos . } ¶ Et pues que assi es el ome peca en faziendo mal al omne & inquantum illi aliquo modo forefaciunt \textbf{ vel forefecerunt in ipsos . } Delinquit ergo homo offendendo hominem : \\\hline
2.3.7 & fazen o fizieron algua cosa desagnisada contra ellos . \textbf{ ¶ Et pues que assi es el ome peca en faziendo mal al omne } enpero fablando sinplemente & vel forefecerunt in ipsos . \textbf{ Delinquit ergo homo offendendo hominem : } per se tamen loquendo , \\\hline
2.3.7 & enpero fablando sinplemente \textbf{ e por ssi non pecaen faziendo mal alas bestias } Et si en faziendo mal alas bestias es pecado & per se tamen loquendo , \textbf{ non delinquit offendendo bestias . } Si autem in offensione bestiarum est delictum , \\\hline
2.3.7 & e por ssi non pecaen faziendo mal alas bestias \textbf{ Et si en faziendo mal alas bestias es pecado } esto es & non delinquit offendendo bestias . \textbf{ Si autem in offensione bestiarum est delictum , } hoc est quasi per accidens , \\\hline
2.3.7 & mas avn deuen tomar a ellos en su perssona \textbf{ por que recusan de fazer aquello a que son teriuidos . } mas por que non es de fazer tuerto & sed eos etiam accipere in praeda , \textbf{ ex quo recusant facere | quod tenentur . } Verum quia nulli est iniuria facienda , \\\hline
2.3.7 & por que recusan de fazer aquello a que son teriuidos . \textbf{ mas por que non es de fazer tuerto } nin eniuria a ninguno fablado & quod tenentur . \textbf{ Verum quia nulli est iniuria facienda , } per se loquendo , \\\hline
2.3.8 & que biuna segunt los delectes del cuerpo \textbf{ e por que las riquezas prinçipalmente fazen esto } assi que por ellas cada vno cuyda & ut viuant secundum corporis voluptatem ; \textbf{ cum diuitiae maxime videantur hoc efficere , } ut per eas quilibet consequi possit \\\hline
2.3.8 & e segunt mesura dela sabud \textbf{ que ha de fazer enł entermo . } Et pues que assi es los omes comunalmente & sed potionem appetit dare \textbf{ secundum modum et mensuram sanitatis . } Communiter ergo homines \\\hline
2.3.8 & de aquella auemientra esta enł casco . \textbf{ Bien assi si la natura faze el aialia } e la ceratura dela sangre mestrual dela fenbra & quantum sufficiat ad nutrimentum illius auis . \textbf{ Sic etiam si natura } ex menstruo facit animal , et ex lacte nutrit ipsum , \\\hline
2.3.8 & para nudermiento de la catura \textbf{ si el gouernador dela casa quiere fazer contra natura } mas si quiere gouernar su casa & et lac in uberibus ad nutrimentum foetus : \textbf{ si gubernator domus non vult contra naturam agere , } sed vult suam domum regere \\\hline
2.3.9 & e estan enla casa \textbf{ por la qual cosa toda la mia dela casa se faze del padre familas } que es curador dela casa & ø \\\hline
2.3.9 & e mas honrrados \textbf{ que dellos se pueden fazer basos } que son aprouechables & et sunt utilia , et honorabilia : \textbf{ ex eis enim possunt fieri vasa , } quae sunt hominibus utilia , \\\hline
2.3.10 & assi alguas uezes \textbf{ por algun menester que acaesçe assi commo por fazer uasos o escudiellas e alguas otras cosas . } Alguons diueros son fondidos en massa & sic aliquando aliqua necessitate interueniente , \textbf{ ut propter vasa fienda , | vel propter aliquid aliud , } denarii resoluuntur in massam . \\\hline
2.3.10 & e por esta razon acaesçe por auentura \textbf{ que de tantos dineros en cuento se faze massa de mayor peso . } Et desta & ex totidem denariis numero , \textbf{ confici massam maioris ponderis : } ex quo casu ars sumpsit originem , \\\hline
2.3.10 & e tienpo quissiesse auer doze \textbf{ la qual cosa faze el arte pecumatiua dela usura } assi con & post aliquod tempus vult habere duodecim \textbf{ quod facit pecuniatiua usuraria , ut plane patet , } vult quod denarii illi pariant et generent : \\\hline
2.3.11 & e lo que parte nesçe a el \textbf{ conueiblemente lo pue de fazer } non faziendo tuerto a ninguno . & et quod pertinet ad ipsum , \textbf{ licite potest , } et nulli iniuriatur . \\\hline
2.3.11 & conueiblemente lo pue de fazer \textbf{ non faziendo tuerto a ninguno . } Et por ende assi faziendo non robanada & licite potest , \textbf{ et nulli iniuriatur . } Nihil ergo rapit , \\\hline
2.3.11 & non faziendo tuerto a ninguno . \textbf{ Et por ende assi faziendo non robanada } nin toma uso ageno si retiniendo en ssi el señorio dela casa vede la morada & et nulli iniuriatur . \textbf{ Nihil ergo rapit , | et nihil usurpat , } si retinens sibi dominium domus , \\\hline
2.3.11 & Et uso noppreo es uender la o canbiar la . \textbf{ Ca muchos fazen casas non para morar en ellas } mas para vender las . & uel commutare : \textbf{ multi enim domos fabricant | non ad inhabitandum , } sed ad uendendum . \\\hline
2.3.11 & para paresçer con ellas . \textbf{ La qual cosa fazen los rảcadores muchͣs uezes } por que parescan ricos & sed ad apparendum : \textbf{ quod forte multotiens mercatores faciunt , } qui ut appareant diuites , \\\hline
2.3.11 & de defender las usuras \textbf{ que non se fagan } por que son contrael derecho natural & prohibere usuras , \textbf{ ne fiant eo } quod iuri naturali contradicant . \\\hline
2.3.12 & Et esto contesçe \textbf{ que se parue de fazer } assi commo en çinco maneras ¶ & quibus numismata acquiruntur . \textbf{ Contingit enim hoc fieri quasi quinque viis . } Quarum una dicitur possessoria . \\\hline
2.3.12 & e de quales puede meior acorrer ala mengua dela cala . \textbf{ Et esto le puede fazer } si lopieren quales aianlias & et ex quibus potest melius subueniri indigentiae corporali domesticae siue gubernationi domus . \textbf{ Hoc autem fieri contingit , } si sciatur quae in quibus partibus abundant , \\\hline
2.3.12 & Ca segunt el philosofo la mercaduria se parte en tres partes . \textbf{ En nauigaria que se faze en leunado merçadurias } por la mar en las naueᷤ¶ & vel assistit deferentibus mercationes ipsas . \textbf{ Diuiditur autem ( secundum Philosophum ) mercatoria in tres partes , } in nauclariam quae sit per mare : \\\hline
2.3.12 & La segunda es leuadora de pesos \textbf{ que se faze } por la trra que llamamos aoqueria¶ & et ponderis portatiuam \textbf{ quae fit per terram : } et assistricem . \\\hline
2.3.12 & de los quales se ganaron grandes aueres . \textbf{ ¶ El primero es que fizo talez mi les io vno de los siete sabios } que primeramente comneçara a philosofo far . & duo particularia gesta , quibus fuit pecunia acquisita Primum est , \textbf{ quod fecit Thales Milesius unus de septem sapientibus , } qui primo philosophari coeperunt . \\\hline
2.3.12 & entre todas las cosas \textbf{ que acresçientan las riquezas es fazer monopolia } que quiere dezer vendiconn de vno solo . & ( secundum Philos’ ) \textbf{ est facere monopoliam , } idest facere vendationem unius : \\\hline
2.3.12 & quando alguno \textbf{ por su arte faze alguas obras } por que gana dineros . Ca commo quier la fin dela arte dela caualłia sea uictoria & Quinta via dicitur esse artifica , \textbf{ quando quis per artem suam aliqua exerceret , } propter quae pecuniam lucratur . \\\hline
2.3.12 & Ca los fisicos e los ferreros e los carpenteros \textbf{ que fazen las casas . } Et avn los caualleros & cum ex opere vel ex arte facto pecuniam intendunt . \textbf{ Medici enim , fabri , domificatores , } et etiam ipsi milites , \\\hline
2.3.12 & sienpre tx de sus greyes \textbf{ e esto fazia tan bien en las bestias } commo en las aues . & ubi victualia modici precii existebant ; nihilominus quasi tamen semper ex propriis alimenta carnium volebat assumere , \textbf{ et hoc tam in bestiis , } quam etiam in volucribus . \\\hline
2.3.13 & La primera razon paresçe assi . \textbf{ Ca nunca algunas cosas muchas fazen vna cosa } segunt orden conuenible & Quarta ex diuersitate sexuum in specie humana . Prima via sic patet . \textbf{ Nam numquam aliqua multa } secundum debitum ordinem efficiunt aliquid unum , \\\hline
2.3.13 & que nunca algunas cosas \textbf{ muchas fazen naturalmente alguna cosa } que sea vna si non & quod nunquam aliqua plura constituunt \textbf{ naturaliter aliquid unum , } nisi ibi naturaliter aliud sit praedominans : \\\hline
2.3.13 & assi commo es prouado de suso \textbf{ mas conplidamente nunca de mucho omes se faria naturalmente } vna conpannia o vna poliçia & ut superius diffusius probabatur , \textbf{ numquam ex pluribus hominibus fieret naturaliter una societas vel una politia , } nisi naturale esset \\\hline
2.3.14 & connino de dar \textbf{ e de fazer alg̃s leyes pointiuas } legunt las quales se gouernassen los regnos e las çibdades & propter commune bonum oportuit \textbf{ dare leges aliquas positiuas , } secundum quas regentur regna et ciuitates : \\\hline
2.3.14 & que son bienes del alma \textbf{ fazen señorio notural } e sinplemente e sin condiconn . & quae sunt bona animae , \textbf{ reddunt dominium naturale et simpliciter : } est enim dignum naturaliter \\\hline
2.3.14 & e bienes de fuera \textbf{ non fazen sennorio natural sinplemente . } Mas mas fazen sennorio legal & quae sunt bona corporalia et exteriora , \textbf{ non faciunt dominium simpliciter naturale , } sed magis faciunt ipsum legale et positiuum . \\\hline
2.3.14 & non fazen sennorio natural sinplemente . \textbf{ Mas mas fazen sennorio legal } e positiuo por establesçimiento de los omes . & non faciunt dominium simpliciter naturale , \textbf{ sed magis faciunt ipsum legale et positiuum . } Quod enim superans in bonis corporis , \\\hline
2.3.14 & que segunt los biens del cuerpo \textbf{ Mas assi commo el dize non es semeiante cosa fazer paresçer la fermosura del alma } e la fermosura del cuerpo ¶ & quam secundum bona corporis ; \textbf{ sed ( ut ait ) non similiter esse facile , } videre pulchritudinem animae , et corporis . \\\hline
2.3.14 & serique mas enclinados a matar \textbf{ e a fazer honuçidio si } soperiessen que nigunt pro non aurian de tal uençimiento . & homines enim alios debellantes proniores essent ad homicidium , \textbf{ si scirent se ex eis nullam utilitatem consecuturos ; } sed cum cogitant eos acquirere in seruos , \\\hline
2.3.15 & que en todo tienpo de su uida \textbf{ non fazen ningua batalla iusta } por que por ella puedan ganar algunos seruientes e sieruos . & Rursus , quia contingit aliquando plures etiam ex nobili genere ortos toto tempore vitae suae \textbf{ non agere aliquod iustum bellum , } ut ex eo possent \\\hline
2.3.15 & que ellos reçiban \textbf{ mas del bien fecho del prinçipe } e que sean ermas honrrados & Quare dignum est ipsos \textbf{ plus de influentia recipere , } et amplius honorari , et praemiari a principante . \\\hline
2.3.16 & por que aquel offiçio non sea menospreçiado \textbf{ e non se faga confusamente } e desordenadamente conuiene de poner y vn & ne illud negligatur , \textbf{ et ne fiat confuse et inordinate , } praeficiendus est unus architector ministris illis , \\\hline
2.3.16 & por que vno non cunpliria \textbf{ para fazer aquel oficio e aquella obra . } ¶ Pues que assi es en a comne dar estos ofiçios & eo quod unus non sufficeret \textbf{ exequi opus illud . } Est igitur in commissione officiorum \\\hline
2.3.16 & por que podria la casa ser tan pequanan que muy ligeramente vno \textbf{ podria fazer ser seruidor dela mesa e guardador dela puerta . } Mas en las casas de los Reyes & idem posset \textbf{ esse totus minister mensae , | et custos portae . } Sed in domibus Regum et Principum , \\\hline
2.3.16 & por que comunalmente los ofiçiales acostunbraron de fallesçer en dos cosas \textbf{ Ca algunos fazen mal los ofiçios } que les son acomnedados & Communiter enim ministri in duobus consueuerunt deficere : \textbf{ nam aliqui male exequuntur opus iniunctum , } quia sunt decipientes , \\\hline
2.3.16 & por que son engannadores e menguadores de los derechos reales . \textbf{ as otros ay que fazen mal el ofiçio } que les & et defraudantes iura legalia : \textbf{ aliqui vero male consequuntur ipsum , } sed non ex malitia voluntatis , \\\hline
2.3.16 & e buen proueedor e bien acatado e bien aguardado delas cosas \textbf{ que ha de fazer } e si ouiere las o triscosas & ø \\\hline
2.3.17 & nin por aparesçençia uanas \textbf{ e de una fazer tales cosas . } Enpero por que los Reyes e los prinçipes sean guardados en su estado granado & Nam licet non ad inanem gloriam , \textbf{ nec ad ostentationem talia sint fienda : } tamen ut Reges et Principes conseruent se \\\hline
2.3.17 & Enpero por que los Reyes e los prinçipes sean guardados en su estado granado \textbf{ e por qua non sean despreçiados de los pueblos conuieneles de fazer grandes } assi commo prueua el philosofo & in statu suo magnifico , \textbf{ et ne a populis condemnantur , | decet eos magnifica facere , } ut probat Philosophus 7 Poli’ . \\\hline
2.3.17 & la muy marauillosa sabiduria \textbf{ de aquel que la fizo que es dios . } Por la qual cosa enlas casas de los Reyes & in quo declaratur conditoris mirabilis sapientia . \textbf{ Quare in domibus Regum et Principum , } quarum quaelibet propter varietatem ministrantium \\\hline
2.3.18 & ay falssedat \textbf{ ca assi commo dize el philosofo en las politicas la natura quiere sienpte fazer alguna cosa . } Enpero muchͣs uezes non la puede fazer & Huic autem probabilitati aliquando subest falsitas , \textbf{ quia ( ut dicitur in Politicis ) | natura vult } qui de hoc facere multotiens , \\\hline
2.3.18 & ca assi commo dize el philosofo en las politicas la natura quiere sienpte fazer alguna cosa . \textbf{ Enpero muchͣs uezes non la puede fazer } mas fallesçe por algun enbargo & natura vult \textbf{ qui de hoc facere multotiens , | tamen non potest , } sed deficit . \\\hline
2.3.18 & en essa misma manera son dichos algunos curiales \textbf{ si conueiblemente se ouieren en fazer guaades despessas } la qual cosa es obra de magnifiçençia . & Sic et liberales dicuntur , \textbf{ si decenter se habeant in magnis sumptibus , } quod est opus magnificentiae . \\\hline
2.3.18 & que \textbf{ seancurialmente contra sus çibdadanos sinon les fezieren tuerto en las mugers } e en las fijas & Curiales etiam dicuntur homines se habere erga suos ciues , \textbf{ si non eis iniuriam inferant in uxoribus , } vel in filiabus , \\\hline
2.3.18 & assi commo si alguno dieres o bienes alos sus çibdadanos liberalmente \textbf{ si esto faze } por quel plaze tal obra es dichͣ liberal & ut si quis suis conciuibus bona sua prompte largitur , \textbf{ si haec agit , } quia ei placent , \\\hline
2.3.18 & por quel plaze tal obra es dichͣ liberal \textbf{ mas si lo faze por que cunpla la ley } e por que lo manda la ley es dichn isto legual & huiusmodi actus liberalis est ; \textbf{ sed si ut impleat legem } quia hoc lex praecipit , \\\hline
2.3.18 & e por que lo manda la ley es dichn isto legual \textbf{ mas si esto faze } por que esto conuiene alas costunbres dela corte & iustus legalis est : \textbf{ si vero id agat } quia hoc decet mores curiae et mores nobilium , \\\hline
2.3.18 & algbiuiere et morare con los otros alegremente \textbf{ e amigablemente si esto faze } por quel plazen tales obras es dicho bien amigable & quis cum aliis conuersetur , \textbf{ si hoc agit } quia ei placent huiusmodi opera , \\\hline
2.3.18 & por quel plazen tales obras es dicho bien amigable \textbf{ mas si lo faze } por que cunpla la ley es iusto legual & affabilis est : \textbf{ sed si ut legem impleat , } iustus legalis erit : \\\hline
2.3.18 & por que cunpla la ley es iusto legual \textbf{ mas aquel que lo faze por guardar las costunbron dela corte } e de los nobles es dicho curial & iustus legalis erit : \textbf{ qui vero ut seruet mores curiae et nobilium , } curialis esse dicetur . \\\hline
2.3.18 & por que muchs son los \textbf{ que fazen obras de uirtudes } partiendo los sus bienes alos otros & curialis esse dicetur . \textbf{ Sunt enim multi facientes opera virtutum } ut bona sua aliis largientes , \\\hline
2.3.18 & partiendo los sus bienes alos otros \textbf{ e esto non lo fazen } por que les plaze de despender & ut bona sua aliis largientes , \textbf{ non agentes hoc quia eis placeat expendere ; } nec quod delectentur in dando , \\\hline
2.3.18 & nin por que se delecten en dar \textbf{ la gual cosa faze el omne liberal } nin otrossi non lo faze & nec quod delectentur in dando , \textbf{ quod facit liberalis ; } nec quod ex hoc velint \\\hline
2.3.18 & la gual cosa faze el omne liberal \textbf{ nin otrossi non lo faze } por que quiera cunplir la ley & quod facit liberalis ; \textbf{ nec quod ex hoc velint } implere legem hoc precipientem , \\\hline
2.3.18 & por que quiera cunplir la ley \textbf{ que lo manda la qual cosa faze el iusto legal . } Mas por que el quiere retener las costunbres dela corte & implere legem hoc precipientem , \textbf{ quod facit iustus legalis : } sed quia volunt retinere mores curiae et nobilium , \\\hline
2.3.19 & e lo postrimero conuiene de saber \textbf{ en qual manera los señores les han de fazer bien } e en qual manera les deuen fazer grans ¶ & Quinto et ultimo oportet cognoscere , \textbf{ qualiter sunt beneficiandi , } et quomodo sunt eis gratiae impendendae . \\\hline
2.3.19 & en qual manera los señores les han de fazer bien \textbf{ e en qual manera les deuen fazer grans ¶ } Mas lo primero destas cosas & qualiter sunt beneficiandi , \textbf{ et quomodo sunt eis gratiae impendendae . } Primum autem horum aliquo modo est \\\hline
2.3.19 & conplidamente çierto \textbf{ si non le uieremos fazer las cosas sabiamente } por luengo tienpo . & Sic etiam et de prudentia alicuius plene non constat , \textbf{ nisi per diuturnum tempus viderimus ipsum prudenter egisse . } Hoc ergo modo sunt ministris officia committenda , \\\hline
2.3.19 & que por que las costunbres \textbf{ de los que en el otro dia se fezieron ricos } e los que de nueno suben en alto estado & quod quia mores nuper ditatorum , \textbf{ et de nouo ascendentium ad altum statum , } ut dicitur 2 Rhetor’ \\\hline
2.3.19 & Et pues que assi es non conuiene al prinçipe \textbf{ de se fazer tan familiar alos sus siruientes } por que sea despreçiado de los & Non ergo decet Principem \textbf{ tam familiarem se exhibere ministris , } ut habeatur in contemptu , \\\hline
2.3.19 & e alos prinçipes \textbf{ de se faz menos familiares } que los otros & omnino enim decet Reges et Principes \textbf{ minus se exhibere quam caeteros , } et ostendere se esse personas magis graues \\\hline
2.3.20 & quando vn estrumento es ordenado a vna obra \textbf{ por el qual estrumento se deue fazer } e acabar aquella obra & tunc secundum naturam unumquodque perficitur , \textbf{ quando unum organum ordinatur ad unum opus . } Immo quia in operibus naturae non debet \\\hline
2.3.20 & e de los angeles . \textbf{ puesto que la natura faga vn estrumento para dos obras } enpero por que non sea confusion en las obras & ab ipso deo et intelligentiis ordinata : \textbf{ dato quod natura faciat | idem organum ad duo opera , } ne sit in operibus confusio , \\\hline
2.3.20 & contra natural ordenes \textbf{ quando por aquel estrumento entendemos fazen vna de aquellas cosas } si en aquel tienpo non quedaremos dela otra obra . & contra naturalem ordinem est \textbf{ cum per illud organum intendimus | circa unum illorum operum , } si per illud tempus non cessemus ab alio . \\\hline
2.3.20 & que es la lengua a aquella obra que es gostar . \textbf{ La qual cosa se faze enla mesa en el tienpo del comer } si estonçe non quedaremos & quod est gustare , \textbf{ quod sit in mensa tempore comestionis , } si tunc temporis non cessemus \\\hline
2.3.20 & assi commo escalentados \textbf{ fazense osados } e de buenamente se estienden en palabras . & et homines abundantes vino quasi calefacti et audaces , \textbf{ libenter in verba prorumpunt , } videbuntur magis esse ebrii quam sobrii . \\\hline
3.1.1 & que todos los omes obran \textbf{ e fazen todas las cosas } que fazen & et primo Politicorum scribitur , \textbf{ quod gratia eius quod videtur bonum , } omnia operantur omnes . \\\hline
3.1.1 & e fazen todas las cosas \textbf{ que fazen } por grande alguna cosa & quod gratia eius quod videtur bonum , \textbf{ omnia operantur omnes . } Si ergo omnes homines ordinant sua opera in id quod videtur bonum , \\\hline
3.1.2 & en el primero libro delas politicas \textbf{ que fue fecha la çibdat } non solamente por grande beuir & Ideo dicitur primo Politicorum \textbf{ quod facta } quidem igitur est ciuitas viuendi gratia : \\\hline
3.1.3 & si tiene materia sienpre escalentada \textbf{ e do quier que es sienpre faze obra de escalentamiento } por la qual cosa si la çibdat fuesse alguna cosa natural & semper calefacit , \textbf{ et ubicumque est , | habet exercere calefactionis actum . } Quare si ciuitas \\\hline
3.1.3 & assi commo son los religiosos \textbf{ ca aquel que non casa e el que non biue çiuilmente meior faze } por que mas liberalmente buia & bene faciat ; \textbf{ tamen qui non nubit , | et qui non ciuiliter viuit , } ut liberius contemplationi vacet , \\\hline
3.1.3 & mas o son bestiales o diuinales \textbf{ que fazen uida contenplatiua } el capitulo sobredicho soluiemos las obiecconnes contrarias & sed vel est bestia \textbf{ vel est deus . } Remouebantur in praecedenti capitulo obiectiones contrariae , \\\hline
3.1.4 & sobiecconnes contrarias \textbf{ por las quales se desfaze la uerdat } si non aduxieremos razones propreas & remouere omnes obiectiones \textbf{ contrarias veritatem aliquam impugnantes , } nisi adducantur rationes propriae \\\hline
3.1.4 & que acaban la casa son cosa natural . \textbf{ ca la casa se faze de comuidat de omne e de su muger e de sennor e de sieruo e de padre e de fijos . Et cada vna destas comuidades es cosa segunt natura bien } assi avn la comunidat del uarrio es cosa natural & sunt quid naturale : \textbf{ constat enim domus | ex communitate viri et uxoris , domini , et serui , patris et filii , quarum quaelibet est secundum naturam . } Sic etiam communitas vici est \\\hline
3.1.4 & mas avn por que la generaçion del uarro ha de ser cosa natal \textbf{ ca fazesse el uarion a tal monte de acresçentamiento de fijos e de metos e de parientes e de vezinos } assi commo dize el philosofo & fit enim vicus naturaliter \textbf{ ex crescentia filiorum collectaneorum , | et nepotum , } ut vult Philosophus 1 Polit’ \\\hline
3.1.4 & diol natural inclinaçion \textbf{ para fazer aquellas cosas } por las quales se pueden manteñ en la uida e esto contesçe mayormente segunt que dize el pho & dedit ei naturalem impetum ad faciendum ea \textbf{ per quae possit | sibi in vita sufficere . } Hoc autem maxime contingit \\\hline
3.1.4 & para beuir politicas miente en çibdat \textbf{ e para fazer çibdat } mas commo aquello a que auemos inclinaçion natural sea cosa natural & ad viuendum politice , \textbf{ et ad constituendum ciuitatem . } Sed cum id , \\\hline
3.1.5 & que la comunidat acabada \textbf{ que es çibdat se faga de muchsuarios } e aya en ssi conplimiento delas cosas & quod communitas perfecta , \textbf{ quae est ciuitas constans | ex pluribus vicis , } est habens terminum omnis \\\hline
3.1.5 & ca la entencion del \textbf{ que faze la ley } assi commo dicho es de suso & sumitur ex parte adoptionis virtutis . \textbf{ Intentio enim legislatoris } ( ut supra tangebatur ) \\\hline
3.1.5 & por que biuna los omes segunt ley \textbf{ e uirtuosamente ante el que faze la ley } cantomas prinçipalmente deue tener mientes a esto & sed etiam ut viuant \textbf{ secundum legem et virtuose . } Immo tanto principalius debet \\\hline
3.1.5 & que los gouernadores de la çibdat ayan poderio çiuil \textbf{ por que puedan costrennir e fazer iustiçia } en los que non quieren beuiruirtuosamente & habere ciuilem potentiam , \textbf{ ut possint cogere et punire } nolentes virtuose viuere , \\\hline
3.1.5 & enemigos \textbf{ faze amistança con otra çibdat } por que pueda meior cotra dezir & cum aliqua ciuitas impugnatur \textbf{ confoederat se ciuitati alii , } ut melius possit \\\hline
3.1.6 & enpero la vna es mas natural que la otra \textbf{ ¶La primera manera es aquella dela qual en el segundo libro feziemos mençion desuso do dixiemos } que por las cresçençias de los fijos e de los nietos e de los bisnietos e de los parientes & aliter tamen est naturalior altero . \textbf{ Primus est ille de quo supra | in secundo libro fecimus mentionem , } ubi diximus quod propter excrescentiam filiorum collectaneorum \\\hline
3.1.6 & mas conplidamente aquellas cosas \textbf{ que fazen meester } para la mengua dela uida & in qua simul morantes habere possent sufficientius \textbf{ quae requiruntur ad indigentiam vitae . } Hoc ergo modo posset \\\hline
3.1.6 & para la mengua dela uida \textbf{ ¶ Et pues que assi es en esta manera se podia fazer el establesçemiento del regno } assi commo sy muchͣs çibdades & quae requiruntur ad indigentiam vitae . \textbf{ Hoc ergo modo posset | fieri constitutio regni , } ut si multae ciuitates et castra simul confoederarentur et concordarent , \\\hline
3.1.6 & e los mietos en vna casa \textbf{ e non podiendo morar en vno fagan para ssi muchos casas } e establescan vn uarrio . & ex crescentibus filiis et nepotibus in eadem domo , \textbf{ et non valentibus simul habitare , | faciant sibi plures domos , } et constituant sibi vicum ; \\\hline
3.1.6 & Et despues mas adelante cresçiendo \textbf{ e non podiendo morar en vn uarrio fagan mas adelante } para si much suarrios & et ulterius excrescentibus \textbf{ et non valentibus habitare in uno vico , } faciant sibi vicos plures , \\\hline
3.1.6 & e noo podiendo morar en vna çibdat \textbf{ fagan para si muchͣs çibdades } e establezçen vn regno & et non valentibus habitare in una ciuitate , \textbf{ fabricent sibi ciuitates plures , } et constituant regnum : \\\hline
3.1.6 & e dela çibdat es natural \textbf{ assi commo quando se faze tal establesçimiento } por concordia de los omes & ut cum ex concordia hominum \textbf{ talis constitutio habet esse , } licet non sit adeo naturalis \\\hline
3.1.6 & e pueden mas defender se de los enemigos \textbf{ que les quieren mal fazer } et esta tal inclinaçion es natural & magis pacifice viuere , \textbf{ et magis resistere hostibus volentibus impugnare ipsos . } Est enim huius impetus naturalis : \\\hline
3.1.6 & e porque cunplan assi en la uida \textbf{ la qual cosa se faze } por la conpannia çiuil . & et ut sufficiant sibi in vita , \textbf{ quod fit per ciuilem societatem : } sic naturalem habent impetum \\\hline
3.1.6 & e por que puedan mas ligeramente defender se de sus enemigos \textbf{ la qual cosa se faze } por conpania de prinçipado e de regno & et ut naturaliter resistant hostibus , \textbf{ quod fit per societatem principatus et regni ; } videmus enim ciuitates non existentes \\\hline
3.1.6 & por tirania e por poderio ciuil apremiasse las otras çibdades \textbf{ e se feziesse rey sobre ellos . } ¶ Et pues que assi es visto & opprimat ciuitates alias , \textbf{ et faciat se Regem constitui super illas . } Viso diuersos esse modos generationis ciuitatis et regni , \\\hline
3.1.7 & ¶ Et pues que assi es si la nata \textbf{ assi faze } que la vena del oro & vena argenti similiter se habere . \textbf{ Si ergo natura sic agit } quod venam auri non conuertit \\\hline
3.1.7 & delas \textbf{ quales si acaesçiere logar nos podremos dellas fazer mençion . } on conuiene de demandar en todas las cosas & circa regimen ciuitatis , \textbf{ de quibus si locus occurrat mentio fieri poterit . } Maximam unitatem et aequalitatem \\\hline
3.1.8 & que de vn regno \textbf{ ca la casa se faze de muchͣs personas } e eluarrio de muchͣ̃s casas & et ciuitatis quam regni . \textbf{ Constat enim domus ex pluribus personis , } vicus ex pluribus domibus , \\\hline
3.1.8 & e el regno de muchͣs çibdades \textbf{ ¶ Et pues que assi es si la casa en tanto se feziesse vna } assi que todas las perssonas & et regnum ex pluribus ciuitatibus . \textbf{ Si ergo tantum uniretur domus , } quod omnes habitantes in ipsa fieret una persona , \\\hline
3.1.8 & assi que todas las perssonas \textbf{ que morassen en ella se feziessen vna perssona } ya non seria casa mas serie vn omne singular . & Si ergo tantum uniretur domus , \textbf{ quod omnes habitantes in ipsa fieret una persona , } iam non remaneret domus , \\\hline
3.1.8 & ya non seria casa mas serie vn omne singular . \textbf{ En essa misma manera si el uarrio en tanto se feziesse vno que se feziesse vna casa } ya non fincaria uarrio & sed fieret homo unus aliquis singularis : \textbf{ sic si tantum uniretur vicus , | quod fieret una domus , } iam non remaneret vicus : \\\hline
3.1.8 & ya non fincaria uarrio \textbf{ e en essa misma manera si la çibdat fuesse en tanto vna que se feziesse } vnuamno o vna & iam non remaneret vicus : \textbf{ eodem etiam modo | si tantum uniretur ciuitas , } quod fieret unius vicus vel una domus , \\\hline
3.1.8 & y departidos mienbros \textbf{ que fagan estas obras departidas . } En essa misma manera por que para conplir la mengua dela uida & et auditus ideo oportet \textbf{ ibi dare diuersa membra exercentia diuersos actus : } sic quia ad indigentiam vitae \\\hline
3.1.9 & ca assi commo non puede ser \textbf{ que de todos los çibdadanos se fagan vn } cuerporealmente & nam sicut fieri non potest , \textbf{ quod ex omnibus ciuibus fiat unum corpus realiter ; } ita esse non potest , \\\hline
3.1.9 & si fuere conosçido \textbf{ çierto mayor amor faze } que la filiaçion delos fijos & si sit nota et certa , \textbf{ maiorem dilectionem causat , } quam filiatio , \\\hline
3.1.10 & e cada vnos sean çiertos de sus parientes \textbf{ por que por esta non sabiduria los fiios non ayan de fazer miurias } nin tuertosa sus padres e a sus parientes & et quoslibet certificari de eorum consanguineis , \textbf{ ne propter ignorantiam filii } in proprios parentes et consanguineos \\\hline
3.1.10 & por esta non sabiduria \textbf{ nin çertidunbre farien tuerto alos parientes e asus padres . } El segundo mal se muestra & de facili \textbf{ propter ignorantiam iniurari consanguineis suis . } Secundum malum sic ostenditur . \\\hline
3.1.10 & assi commo poca miel puesta en vn grant rio \textbf{ non puede fazer todo el rio dulçe } assi amor de dos o de tro fiios & Sicut ergo parum mellis totum unum fluuium \textbf{ non posset facere dulcem , } sic amor duorum \\\hline
3.1.10 & de que son en vna çibdat \textbf{ nin puede fazer } que aquella muchedunbre sea plazible & innumerabilem multitudinem puerorum existentium in ciuitate una , \textbf{ non posset reddere placibilem et dilectam . } Sed non existente dilectione ciuium ad pueros , \\\hline
3.1.10 & quando esta muy de pierta \textbf{ e muy golola por grant muchedunbre de uiandas es cosa guaue de fazer al omne astinençia . } assi el appetito de luxuria & per multitudinem ciborum \textbf{ difficile est esse abstinentes , } sic prouocaris venereis \\\hline
3.1.10 & puesto que los prinçipes defendiessen alos fijos \textbf{ que non fiziessen lururia con sus madres } e los padres con sus fiias & et non iudicabantur eis proprii parentes , \textbf{ dato quod prohiberetur filio actus venereus circa matrem , } et patri circa filiam , \\\hline
3.1.11 & e los fructos uernon a comun \textbf{ la qual cosa se fazia en el tpon antiguo } entre algunas naçiones & et fructus redigentur in commune , \textbf{ quod antiquitus } ( ut recitat Philosophus ) \\\hline
3.1.11 & commo entre los hr̃manos de vn vientre \textbf{ por que si se fiziesse } assi commo dizia socrates & sicut inter uterinos fratres , \textbf{ nam et si fieret } ut Socrates dicebat , \\\hline
3.1.12 & e los esforçados han se de estendera muchos cosas . \textbf{ Et por ende la calentura faze al omne sera ionso e esforçado } Mas la frialdat faze al que la ha temeroso & ad alia se extendere , \textbf{ calor enim reddit | habentem animosum et virilem , } frigiditas vero timidum et pusillanimum . \\\hline
3.1.12 & Et por ende la calentura faze al omne sera ionso e esforçado \textbf{ Mas la frialdat faze al que la ha temeroso } e de flaco coraçon . & habentem animosum et virilem , \textbf{ frigiditas vero timidum et pusillanimum . } Quare mulieres , \\\hline
3.1.12 & e conuiene les de auer fuertes braços \textbf{ para fazer fuertes colpes . } Et por que las mugers esto non pueden auer & et habere fortia brachia \textbf{ ad faciendum percussiones fortes : } mulieres igitur eo quod habent carnes molles \\\hline
3.1.12 & segunt ordenamiento conueinble en aquellas cosas \textbf{ que las bestias fazen } sin razon los omes non las deuen segnir & Quare in iis , \textbf{ in quibus bestiae praeter rationem agunt , eas sequi non debent . } Dicebat Socrates \\\hline
3.1.13 & e partiendo los a departidas \textbf{ ꝑsonas faze a buen estado e paçifico dela çibdat } e de los çibdadanos & et distribuere eos diuersis personis , \textbf{ ut innuit Philosophus | 2 Polit’ videtur facere ad quietum } et pacificum statum ciuium . \\\hline
3.1.13 & dizeque socrates \textbf{ sienpre fazie vnos mismos prinçipes } que era razon de discordia e de vanderia en la çibdat & cum ait . \textbf{ Socrates semper facit eosdem Principes , } quod est seditionis causa \\\hline
3.1.14 & en el segundo libro delas politicas \textbf{ el que quiere poner leyes o fazer ordenaçion alguna en la çibdat a tres } co sas deue deuer mietes . & Nam secundum Philosophum secundo Politicorum , \textbf{ volens ponere leges | vel facere ordinationem aliquam in ciuitate , } ad tria debet respicere , \\\hline
3.1.16 & que el otro \textbf{ ca assi commo el dizia esto se podia faz ligeramente } quando la çibdat en el comienço se establesçia & quam alter : \textbf{ quod ut dicebat a principio } quando ciuitas constituitur , \\\hline
3.1.16 & ca por ganar las possessions \textbf{ fazen los çibdadanos desagnisados } e tuercos vnos a otros & ut tollantur de ciuitate iniuriae et contumeliae : \textbf{ nam ex eo quod ciues libenter sibi possessiones appropriant , } dicente , \\\hline
3.1.16 & ca para ganar possessiones \textbf{ fazen los çibdadanos tuerto vno a otro en sus ppreas personas . } An fazense en la çibdat furtos & Nam pro acquirendis possessionibus \textbf{ inferunt sibi ciues iniurias | et contumelias in personis propriis ; } fiunt autem in ciuitate furta , \\\hline
3.1.16 & fazen los çibdadanos tuerto vno a otro en sus ppreas personas . \textbf{ An fazense en la çibdat furtos } e robos i home çie doi dios & et contumelias in personis propriis ; \textbf{ fiunt autem in ciuitate furta , } rapinae et homicidia \\\hline
3.1.17 & las quales miurias deue escusar muchel \textbf{ que faze la ley } ca pertenesçe al prinçipe de auer grant cuydado & quae possunt in ciuitate consurgere , \textbf{ quas legislator summo studio cauere debet . } Spectat enim ad principem , \\\hline
3.1.17 & que los çibdadanos non sean bulliçiosos ni turbadores dela paz \textbf{ e que non fagan iniurias } nin tuertos los vnos alos otros & ne ciues sint insolentes , \textbf{ et ne sibi inuicem iniurias inferant : } sed si in ciuitate statuitur \\\hline
3.1.17 & assy commo dize felleas contesçeria muchos uegadas \textbf{ que los ricos se farian pobres } e los pobres ricos & et non dando dotes illis , contingit multotiens diuites fieri pauperes , \textbf{ et econuerso . } Nam si diuites aliqui plures habent filios quam pauperes , \\\hline
3.1.17 & contesçerien iniurias e tuertos e uaraias en la çibdat . \textbf{ Lo primero quando los pobres se fazen ricos } non saben sofrir la su buena uentura & et iurgia in ciuitate . \textbf{ Primo quia pauperes cum ditantur nesciunt fortunas ferre , } ut plane ostendit \\\hline
3.1.17 & assi commo el philosofo muestra llanamente en el segundo libro de la rectoriça \textbf{ e por ende farien tuertos los fijos de los pobres } alos fiios de los ricos & Philosop’ 2 Rhet’ . \textbf{ Iniuriabuntur ergo aliis . } Filii enim pauperum inflati , \\\hline
3.1.17 & que los otros \textbf{ por ende mueuensse a fazer les tuerto } e por ende se farian iniurias e tuertos en la çibdat . & plus aliis in diuitiis abundare , \textbf{ iniustificabunt in eos , } et fient iniuriae in ciuitate . \\\hline
3.1.17 & por ende mueuensse a fazer les tuerto \textbf{ e por ende se farian iniurias e tuertos en la çibdat . } lo segundo estos tuertos & iniustificabunt in eos , \textbf{ et fient iniuriae in ciuitate . } Rursus , huiusmodi iniuriae et contumeliae orirentur inter ciues \\\hline
3.1.17 & non lo pue den sofrir \textbf{ e por ende fazer se yan sobuios e turbarian los otros . } ¶ La terçera razon para mostrar & non valentes se continere , \textbf{ fient insolentes , | et turbabunt alios . } Tertia via ad ostendendum legem Phaleae \\\hline
3.1.17 & que non podrien ser lo ƀales \textbf{ nin fazer ligeramente obras de largueza } Otrossi auiendo las possessiones ygualadas & quod oporteret eos ita parce viuere \textbf{ quod opera liberalitatis de facili exercere non valerent . } Rursus habendo possessiones aequatas possent \\\hline
3.1.18 & nin la entençion prinçipal \textbf{ del que faze la ley deue ser çerca delas possessiones } mas mayormente deue entender en la reprehension delas cobdiçias & ut ciues possessiones aequatas habeant ; \textbf{ nec intentio legislatoris principaliter esse debet circa possessiones , } sed principalius debet \\\hline
3.1.18 & enpero mienbranos \textbf{ que fiziemos vn tractado del partimiento } que es entre la ethica e la rectorica e la politica & Meminimus tamen , \textbf{ nos edidisse quendam tractatum . } De differentia Ethicae Rhetoricae et Politicae , \\\hline
3.1.18 & que la prinçipal entençion del \textbf{ que faze la ley non deue ser en } mesurarlas possessiones de fuera & Ad praesens autem scire sufficiat , \textbf{ principalem intentionem legislatoris non debere esse } circa possessiones exteriores mensurandas . \\\hline
3.1.18 & e algunas son honrradas e graçiosas pertenesçe \textbf{ al que faze la ley non solamente de establesçer } que las possessiones sean ordenadas conueinblemente & quidam gratiosae et honorabiles : \textbf{ spectat ad legislatorem | non solum statuere , } ut possessiones debite ordinentur , \\\hline
3.1.18 & las quales los omes siguen en la mayor parte \textbf{ por que los omes non solamente fazen tuertos } e inuirias & quas homines ut plurimum insequuntur . \textbf{ Non enim homines solum iniustificant , } et iniurias inferunt in res exteriores propter auaritiam : \\\hline
3.1.18 & e por la auariçia \textbf{ mas avn fazen tuertos en las fijas } e en las mugers de los çibdadanos & et iniurias inferunt in res exteriores propter auaritiam : \textbf{ sed etiam iniuriantur in filias , } et in uxores ciuium \\\hline
3.1.18 & por que los çibdadanos non sean destenprados \textbf{ nin fagan tuertos a sus uezinos en las mugerse en las fijas } ante tanto deuen ser mas acuçiosos en esto & ne ciues sint intemperati , \textbf{ et iniurientur in uxores , | et in filias aliorum . } Immo tanto magis solicitari debent \\\hline
3.1.18 & ca lo s ons quieren gozar de delecta çonnes sin tristezas \textbf{ e por ende fazen tuertos alos otros non } por conplir grant mengua & Volunt enim homines gaudere delectationibus absque tristitiis , \textbf{ ideo iniuriantur aliis in se } non solum propter supplendam indigentiam , \\\hline
3.1.18 & Mas por que cuydan que los otros pueden enbargar sus delecta connes \textbf{ o porque cuydan que les pueden fazer tristeza . } Et pues que assi es non solamente son de esta & sed quia existimant alios posse eorum delectationibus impedire , \textbf{ vel quia existimant eis posse tristitiam inferre . } non ergo solum propter possessiones sunt instituendae leges , \\\hline
3.1.19 & Ca qual se quier \textbf{ que faze tuerto a otro } o le faze tuerto en las sus cosas & videlicet de nocumento , iniuria , et morte : \textbf{ quicunque enim iniustificat in alium , } vel iniustificat in res nocendo \\\hline
3.1.19 & que faze tuerto a otro \textbf{ o le faze tuerto en las sus cosas } enpeesçiendol o dannando gelas o le faze tuerto enla persona & quicunque enim iniustificat in alium , \textbf{ vel iniustificat in res nocendo } et damnificando ipsum : \\\hline
3.1.19 & o le faze tuerto en las sus cosas \textbf{ enpeesçiendol o dannando gelas o le faze tuerto enla persona } e esto en dos maneras & vel iniustificat in res nocendo \textbf{ et damnificando ipsum : | vel in personam , } et hoc dupliciter , \\\hline
3.1.19 & dandol feridas \textbf{ e faziendo le llagas en el cuerpo } mas el tuerto & vel dehonorando eam , \textbf{ faciendo ei opprobria et vituperia : } vel offendendo ipsam , \\\hline
3.1.19 & que fuesse buena e conueinble ala çibdat \textbf{ qual feziessen grant honrra } por ello¶ & quod esset expediens ciuitati , \textbf{ quod retribueretur ei debitus honor . } Quantum ad bellatores ordinauit , \\\hline
3.1.19 & ordeno que los fijos de aquellos \textbf{ que moriessen en la batalla fecha } por defendemiento delatrra & quod filii eorum , \textbf{ qui morerentur in bello facto } pro defensione patriae \\\hline
3.1.19 & por los sus dichos \textbf{ que el prinçipe non deue ser fecho } por heredamiento mas por el ectiuo la qual elecçion daua a todo el pueblo ¶ & ( ut apparet ex dictis suis ) \textbf{ principem debere esse per haereditatem , } sed per electionem , \\\hline
3.1.19 & que non pueden guardar su derecho \textbf{ por que tales perssonas los otros de ligero les fazen tuerto } por que non pueden defender su derecho & et personis impotentibus specialem curam gere : \textbf{ eo quod talibus alii de facili iniuriantur , } cum non possint defendere iura sua . \\\hline
3.1.20 & por que podiessen tirar las discordias e las uaraias \textbf{ que se fazen en la çibdat } e por que podiessen defender latr̃ra delos enemigos . & ut possent seditiones \textbf{ et iurgia facta in ciuitate remouere , } et patriam ab hostibus defendere : \\\hline
3.1.20 & sienpre quieren ser senors en las çibdades \textbf{ e los omes de grado fazen tuerto quando puede . } Puesto que el dizie siguese que los menestrales & et quia semper potentiores dominantur , \textbf{ et homines libenter iniustificant | cum possunt ; } hoc posito artifices , \\\hline
3.1.20 & non puede estar con el \textbf{ establesçimiento que fizo de los lidiadores } que ellos fuessen mas poderosos que los otros & stare non potest \textbf{ cum statuto de bellatoribus , } ut quod ipsi sint potentiores aliis , \\\hline
3.2.1 & entp̃o dela paz \textbf{ por las leyes conuiene de fazer tractado destas quatro cosas sobredichͣs en este gouernamiento ¶ } La segunda razon para prouar & Quare si considerentur quae requiruntur ad hoc quod tempore pacis per leges bene gubernetur ciuitas , \textbf{ oportet in huiusmodi regimine | de praedictis quatuor considerationem facere . } Secunda via ad inuestigandum hoc idem sumitur ex fine \\\hline
3.2.1 & aquello que nos enpeesçe \textbf{ e fagamos derech } e escusemos tuerto . & et vitemus nociuum : \textbf{ faciamus iustum , } et vitemus iniustum : \\\hline
3.2.1 & por las leyes \textbf{ para fazer buenas obras } por que en las leyes son mandadas las cosas & populus enim ad bene agendum , \textbf{ per leges maxime inducendus est , } eo quod ipsis legibus praecipiuntur laudabilia , \\\hline
3.2.2 & e por el gouernamiento destos se gouernaua toda la çibdat . \textbf{ Onde los malefiçios que se fazian eran departidos } por el departimiento delos prinçipados & et horum gubernatione tota ciuitas regebatur : \textbf{ unde et maleficia facta | distinguebantur } ex diuersitate principatuum . \\\hline
3.2.2 & ca alli es demandado el consentimiento de todo el pueblo \textbf{ para los establesçimientos que han de fazer } ca maguera sienpre llamen ally alguna & ibi enim requiritur consensus totius populi \textbf{ in statutis condendis , | in potestatibus eligendis , } et etiam in potestatibus corrigendis . \\\hline
3.2.2 & e de castigar le \textbf{ si mal feziere . } Et a todo el pueblo pertenesçe de fazer los establesçimientos & eo quod totius populi est eum eligere et corrigere , \textbf{ si male agat : } etiam eius totius est statuta condere , \\\hline
3.2.2 & si mal feziere . \textbf{ Et a todo el pueblo pertenesçe de fazer los establesçimientos } los quales establesçimientos non puede passar el sennor & si male agat : \textbf{ etiam eius totius est statuta condere , } quae non licet dominium transgredi . \\\hline
3.2.3 & e esta concordia \textbf{ mas la puede fazer aquello que es vna cosa } por ssi & hanc autem unitatem et concordiam \textbf{ magis efficere potest } quod est per se unum ; \\\hline
3.2.3 & por cuyo mouimiento se gouiernan \textbf{ e se fazen todos los mouimientos } que se fazen aqui en la trra & est illud per cuius motum reguntur \textbf{ et causantur motus facti inferius : } et in toto uniuerso est \\\hline
3.2.3 & e se fazen todos los mouimientos \textbf{ que se fazen aqui en la trra } E en essa milma manera en todo el mundo es vn dios & est illud per cuius motum reguntur \textbf{ et causantur motus facti inferius : } et in toto uniuerso est \\\hline
3.2.4 & que muchos omes assi \textbf{ enssennoreantes fazen } assi commo vn omne de muchos oios e de muchͣs manos . & unde Philosophus 3 Politicorum ait , \textbf{ quod plures homines sic principantes quasi constituunt } unum hominem multorum oculorum et multarum manuum . \\\hline
3.2.4 & el sennorio de vno es meior \textbf{ e fazer tal monarchia de vno } si se faze en manera & inquantum tenent locum unius : \textbf{ dominari unum et facere monarchiam , } si debito modo fiat , \\\hline
3.2.4 & e fazer tal monarchia de vno \textbf{ si se faze en manera } conueinble sera en toda manera & dominari unum et facere monarchiam , \textbf{ si debito modo fiat , } erit omnino rectior et dignior . \\\hline
3.2.5 & de si nin auentura \textbf{ mas es fecho por arte e por sabiduria por el que meior e el mas sabio sera puesto en el sennorio } mas si esto fuere por heredat es pone se el regno & videtur tale regnum non esse expositum casui et fortunae , \textbf{ sed factum esse per artem , | eo quod praeficietur melior et industrior . } Sed si per haereditatem hoc fiat , \\\hline
3.2.5 & por la mayor parte thiranzan \textbf{ e fazense thiranos } e son soberiuos de coraçon & est quasi quaedam ineruditio regiae dignitatis \textbf{ tales quidem ut plurimum tyrannizant , } et inflati corde et inerudite regnant . \\\hline
3.2.5 & por ello \textbf{ nin se fazen chiranos nin inchados . } ca non tienen por desaguisado si heredar & filii ex hoc non inflantur \textbf{ nec efficiuntur elati , quia non reputant magnum , } si illud habeant \\\hline
3.2.5 & que son acostunbrados \textbf{ fazen se assi conmonaiturales } e por ende el pueblo & ø \\\hline
3.2.5 & e tuelle las thiranas \textbf{ e faze el sennorio } assi commo natural . & tollit tyrannidem , \textbf{ efficit quasi dominium naturale . } Litigia enim sedat , \\\hline
3.2.5 & los tales que son tomados \textbf{ por electon mas ligerament ethiranizan Avn faze el hedamiento } que el ssennorio sea natural . & tales facilius tyrannizant . \textbf{ Facit etiam hoc dominium naturale , } quia populus quasi naturaliter inclinatur \\\hline
3.2.6 & e por ende antiguamente los mas de los sennores fueron tomados en Reyes . \textbf{ por que si alguno era atal que feziera bien al pueblo } aquella gente inclinada a el & Inde est quod antiquitus plures sic praeficiebantur in Reges . \textbf{ Nam si aliquis fuerat primo beneficus , } gens illa tracta ad amorem eius \\\hline
3.2.6 & e aquellas tres condiçiones buenas sobredichͣs . \textbf{ ca si abondare en bien fazer seria muy amado del pueblo } por aquellos bienes que parterie al pueblo . & habere praedictos tres excessus . \textbf{ Nam si abundet in beneficiis tribuendis , | diligetur a populo . } Si excellat in actionibus virtuosis , \\\hline
3.2.6 & acomienda a omnes estrannos . \textbf{ Mas e Rey faze todo el contrario } por que vee que ha muy grant cuydado del regno & totam suam custodiam corporis committit extraneis : \textbf{ sed Rex econuerso eo } quod videat se maximam curam habere de bono regni et communi , \\\hline
3.2.6 & de aquellos que son en el su regno . \textbf{ Et por ende se faze guardar } de los sus çibdadanos propreos & qui sunt in regno . \textbf{ Ideo facit se custodiri a propriis , } non ab extraneis . \\\hline
3.2.7 & por que aquella es obra natural \textbf{ quando cada cosa se faze } assi commo se deue fazer & Nam illa est naturalis operatio erga aliquid , \textbf{ quando sic agitur } ut est aptum natum agi : \\\hline
3.2.7 & quando cada cosa se faze \textbf{ assi commo se deue fazer } por la qual cosa el regno estonçe es naturalmente gouernado & quando sic agitur \textbf{ ut est aptum natum agi : } quare tunc regnum naturaliter agitur , \\\hline
3.2.7 & ca por la uirtud ayuntada en el \textbf{ puede fazer muchͣs bueans cosas . } Et si por auentura el prinçipe ha la entençion tuerta & quia propter unitatem virtutes potest \textbf{ multa bona efficere : } si vero monarchia habet intentionem peruersam , \\\hline
3.2.7 & ca por el su poderio muy grande \textbf{ que es ayuntado en vno puede fazer muchs males } e esta razon tanne el philosofo & quia propter suam unitam potentiam potest \textbf{ multa mala efficere . } Hanc autem rationem tangit Philosophus quinto Politicorum ubi ait , \\\hline
3.2.8 & quando la enbiamos a alguna sennal \textbf{ que primero la fazemos derecha } por que pueda yr meior ala señal . & in signum aliquod : \textbf{ quia primo efficitur recta , } ut possit melius in finem tendere : \\\hline
3.2.8 & por que pueda yr meior ala señal . \textbf{ Lo segundo la fazemos enpennolada } por que pueda & ut possit melius in finem tendere : \textbf{ secundo efficitur pennata , } ut melius aerem scindat \\\hline
3.2.8 & que entienden en la uida çiuil . \textbf{ Mas esto conmose puede fazer } e en qual manera se deua ordenar la çibdat por que enlła sean falladas todas aquellas cosas & et ad consequendum finem intentum in vita politica . \textbf{ Hoc autem quomodo fieri possit , } et quomodo ordinanda fit ciuitas \\\hline
3.2.8 & e arredrar la rauia de los enemigos . \textbf{ Mas en qual manera esto se aya de fazer en la terçera parte deste terçero libro do } fablaremos delas batallas & et circa industriam armatorum , ut possint hostium rabiem prohibere . \textbf{ Quomodo autem hoc fieri habeat , | in tertia parte huius tertii libri , } ubi agetur de bellis , \\\hline
3.2.8 & aquello deuen cunplir \textbf{ e esto se puede fazer } por conseio de sabios & illud est supplendum : \textbf{ hoc autem fieri potest } per consilium sapientum , \\\hline
3.2.8 & ¶ Lo terçero los que bien obran \textbf{ e mayormente los que fazen aquellas cosas } por que viene grant bien al comun son de gualardonar & Tertio , bene operantes \textbf{ et maxime facientes ea } per quae resultat commune bonum , \\\hline
3.2.8 & por que viene grant bien al comun son de gualardonar \textbf{ e de les fazer bien . Ca commo quier que las cosas menguadas se cunplan } e los bienes ordenados se guarden . & per quae resultat commune bonum , \textbf{ sunt remunerandi et praemiandi : | nam licet commissa supplere , } et bene ordinata conseruare , \\\hline
3.2.8 & Enpero mucho uale galardonar las buean sobras \textbf{ e los que bien fazen . } Ca assi commo dize el philosofo & maxime tamen deseruire videtur ; \textbf{ bene apta remunerare . } Nam ( ut dicitur 3 Ethicorum capitulo de fortitudine ) \\\hline
3.2.9 & que deue obrar el uerdadero rey \textbf{ Et estas diez cosas el tirano se enfinze delas fazer } e non las faze & quae debet operari bonus Rex , \textbf{ et quae Tyrannus se facere simulat . } Illa enim decem licet aliquo modo in uniuersali contineantur in dictis , \\\hline
3.2.9 & Et estas diez cosas el tirano se enfinze delas fazer \textbf{ e non las faze } Et conmo quier que aquellas diez cosas & quae debet operari bonus Rex , \textbf{ et quae Tyrannus se facere simulat . } Illa enim decem licet aliquo modo in uniuersali contineantur in dictis , \\\hline
3.2.9 & por si¶ Et la primera \textbf{ que parte nesçe de fazer aludadero Reyes } que mucha ya cuydado de procurar & Est autem primum quod spectat \textbf{ ad verum Regem facere , } ut maxime procuret bona communia , \\\hline
3.2.9 & e en el bien del regno \textbf{ Mas los tirannos fingen se de fazer estas cosas . } Enpero non las fazen . & vel in bonum regni : \textbf{ tyranni vero hoc simulant facere , } non tamen faciunt : \\\hline
3.2.9 & Mas los tirannos fingen se de fazer estas cosas . \textbf{ Enpero non las fazen . } Mas commo dize el pho & tyranni vero hoc simulant facere , \textbf{ non tamen faciunt : } sed ( ut ait Philosophus ) eos tribuunt meretricibus , et adulatoribus , \\\hline
3.2.9 & et los derechs del regno . \textbf{ Et maguera que los tiranos se enfingan de fazer estas cosas } enpero non las fazen . & et iura regni debet \textbf{ maxime custodire et obseruare . | Quod tyranni licet se facere simulant , } non tamen faciunt , \\\hline
3.2.9 & Et maguera que los tiranos se enfingan de fazer estas cosas \textbf{ enpero non las fazen . } ante non guardan los bienes de los otros & Quod tyranni licet se facere simulant , \textbf{ non tamen faciunt , } immo bona aliorum rapiunt , \\\hline
3.2.9 & ¶ Lo terçeto conuiene al Rey et al prinçipe de non mostrarsse muy espantable nin muy cruel . \textbf{ nin le conuiene otrosi de se fazer muy familiar alos omnes } Mas deue paresçer perssona pesada & et Principem non ostendere se nimis terribilem et seuerum , \textbf{ nec decet se nimis familiarem exhibere , } sed apparere debet persona grauius et reuerenda , \\\hline
3.2.9 & e de muy grant reuerençia . \textbf{ la qual cosa non se pue de fazer conueniblemente sin uirtud . } Et por ende el uerdadero Rey deue ser uirtuoso & sed apparere debet persona grauius et reuerenda , \textbf{ quod congrue sine virtute fieri non potest : } ideo verus Rex vere virtuosus existit : \\\hline
3.2.9 & uerdadero Rey non despreciar a ninguon de los subditos \textbf{ nin fazer tuerto a ninguno } nin en las fij̉as & nullum subditorum contemnere , \textbf{ nulli iniuriari , } nec in filiabus , \\\hline
3.2.9 & nin en ningundelas otras cosas . \textbf{ Et si contesçiesse que alguno en el regno feziesse algun mal deuel castigar } e dar pena non en razon de uaraia & nec in aliquibus aliis : \textbf{ et si contingeret aliquem | ex regno fore facere , } non propter contumeliam \\\hline
3.2.9 & e por conplir iustiçia . \textbf{ mas esto non fazen los tiranos . } Mas por que ellos non entienden & sed propter bonum commune et iustitiae ipsum punire debet . \textbf{ Hoc autem tyranni non faciunt , } sed quia ipsi intendunt \\\hline
3.2.9 & e por auer algo e por conplir sus delecta connes . \textbf{ por ende fazen muchs tuertosa } los çibdadanos tan bien en las mugers & bonum pecuniosum et delectabile , \textbf{ iniuriantur ciuibus in uxoribus , } et in filiabus , \\\hline
3.2.9 & e contra los prinçipes \textbf{ e farian e mouerian discordias en el regno } e enl prinçipado & inducerent viros \textbf{ ut seditiones mouerent in regno aut principatu ; } sic ergo gerere se debet \\\hline
3.2.9 & mas el desmesurado e el enbnago . \textbf{ Mas los tiranos esto non fazen } ca commo ellos non entienden & sed qui ebrius . \textbf{ Tyranni autem hoc non faciunt : } nam cum non intendant \\\hline
3.2.9 & si entre ellos estudiessen . \textbf{ la qual cosa non fazen los tiranos } ca non honrran los sabios & si inter ipsos existerent , \textbf{ quod tyranni non faciunt : } nam sapientes et bonos , \\\hline
3.2.9 & que si les dexaua menor regno en quantidat \textbf{ enpero dexauales mayor regno e mas duradero por tienpo la qual cosa non fazen los tiranos . } Et maguer sienpre enfingan & quod si dimittebat eis minus in quantitate regnum , \textbf{ dimittebat tamen maius et diuturnius tempore . Tyranni autem hoc non faciunt ; } nam licet semper se simulent iuste agere , \\\hline
3.2.9 & Et maguer sienpre enfingan \textbf{ que fazen las cosas derechamente } enpero en muchas cosas & dimittebat tamen maius et diuturnius tempore . Tyranni autem hoc non faciunt ; \textbf{ nam licet semper se simulent iuste agere , } tamen in multis suum dominium iniuste ampliant , \\\hline
3.2.9 & ca sienpre cuyda \textbf{ que tal Rey faze todas las cosas con iustiçia } e non faze ninguna cosa contuerto . & et habere amicum Deum : \textbf{ existimat enim talem semper iuste agere , } et nihil iniquum exercere . \\\hline
3.2.9 & que tal Rey faze todas las cosas con iustiçia \textbf{ e non faze ninguna cosa contuerto . } Enpero nos podemos traher otra & existimat enim talem semper iuste agere , \textbf{ et nihil iniquum exercere . } Possumus tamen ad hoc aliam meliorem rationem adducere dicentes \\\hline
3.2.9 & assi conmo cunple a su salut . \textbf{ Et fazel ser sienpre bien } auentraado en todos sus fechos & ut expedit suae saluti \textbf{ semper in suis actibus prosperari . } Immo propter sanctitatem regis , \\\hline
3.2.9 & auentraado en todos sus fechos \textbf{ Et por ende por la sanidat del Rey dios muchas uezes faze muchs bienes } a aquellos que son en el su regno . & semper in suis actibus prosperari . \textbf{ Immo propter sanctitatem regis , | multotiens Deus } multa bona confert existentibus in ipso regno . \\\hline
3.2.10 & si non commo podra matar los grandes e los nobles . \textbf{ Ante quando alguons cobdician de fazer tiranias } non solamente matan los grandes & nisi quomodo possit excellentes perimere . \textbf{ Immo cum aliqui tyrannizare cupiunt , } non solum excellentes alios , \\\hline
3.2.10 & la qual cosaes muy mala señal de tirama . \textbf{ Mas el uerdadero Rey faze todo el contrario } entendiendo en el bien comun & quod signum est tyrannidis pessimae . \textbf{ Verus autem Rex econuerso intendens commune bonum , } et cognoscens se diligi ab ipsis \\\hline
3.2.10 & es destroyr los sabios . \textbf{ Ca ueyendo que aquello que fazen es } contra razon derech̃tu eyendo & est sapientes destruere . \textbf{ Vident enim se contra dictamen rectae rationis agere , } et non intendere bonum commune sed proprium : \\\hline
3.2.10 & ca conosçiendo la mourien el pueblo contra ellos . \textbf{ Ca sienpre el que mal faze } abortesçe la luz e la sabiduria & semper enim \textbf{ qui male agit , odit lucem , } et non diligit sapientes , \\\hline
3.2.10 & quanto puede destruye los sabios . \textbf{ Mas el uerdadero Rey faze todo el confͣrio . } Ca sabeque lo que el faze faze lo con razon de techͣ & sapientes pro posse destruit . \textbf{ Verus autem Rex econtrario sciens se } secundum rectam rationem agere , \\\hline
3.2.10 & Mas el uerdadero Rey faze todo el confͣrio . \textbf{ Ca sabeque lo que el faze faze lo con razon de techͣ } por ende salua los sabios & Verus autem Rex econtrario sciens se \textbf{ secundum rectam rationem agere , } sapiente , saluat , promouet , \\\hline
3.2.10 & e uieda las sçiençias e el estudio \textbf{ por que non se fagan algunos sabios por ellas . } Ca sienpre teme el tirano de ser reprehendido & et inhibet studium et disciplinam , \textbf{ ne efficiantur aliqui sapientes : semper enim timet } per sapientiam reprehendi . \\\hline
3.2.10 & por la sabiduria \textbf{ e por la sçiençia a as el uerdadero Rey faze el contrario } que promueue el estudio & per sapientiam reprehendi . \textbf{ Verus autem Rex econtrario studium promouet , } et conseruat , \\\hline
3.2.10 & nin sean conosçidos vnos con otros . \textbf{ Ca assi commo dize el pho la conosçençia faze fe . } Ca por esso mismo & nec esse ad inuicem notos : \textbf{ nam ( ut ait Philosophus ) | notitia fidem facit : } eo enim ipso quod quis habet notitiam alterius , \\\hline
3.2.10 & por los males \textbf{ e los tuertos que les faze . } Mas eludadero Rey faze todo el contrario & propter iniurias \textbf{ quas eis infert , | contra ipsum insurgant . } Verus autem Rex econtrario permittit sodalitates ciuium , \\\hline
3.2.10 & e los tuertos que les faze . \textbf{ Mas eludadero Rey faze todo el contrario } ca consiente todas las conpannias & contra ipsum insurgant . \textbf{ Verus autem Rex econtrario permittit sodalitates ciuium , } et vult ciues sibi inuicem esse notos , \\\hline
3.2.10 & que non se le encubra ninguna cosa \textbf{ delo que fazen los çibdadanos . } Ca commo los tyranos sepan & et tenptare non latere ipsum \textbf{ quicquid a ciuibus agitur . } Cum enim tyranni sciant se non diligi a populo , \\\hline
3.2.10 & e ninguna dellas non se leunata contra el tyrano . \textbf{ Mas el uerdadero Rey faze el contrario . } ca non procura turbaçion & neutra insurgit contra tyrannum . \textbf{ Verus autem Rex econtrario non procurat turbationem existentium in regno , } sed pacem et concordiam : \\\hline
3.2.10 & ca non entendrie en el bien comun . \textbf{ La vij ͣ̊ cautela del tirano es fazer los subditos pobres } en tanto que el non aya menester guarda & quia non intenderet commune bonum . \textbf{ Septima , est pauperes facere subditos adeo } ut ipse tyrannus nulla custodia egeat . \\\hline
3.2.10 & en que han de de beuir de cada dia \textbf{ por que no les uague de fazer ayuntamiento contra ellos } nin los tiranos non ayan menester ninguna guarda & quibus indigent , \textbf{ ut non vacet eis aliquid machinari contra ipsos , nec oporteat ipsos habere aliquam custodiam propter illos . } Verus autem Rex \\\hline
3.2.10 & guerrasa partes estrannas \textbf{ e sienpre faze lidiar sus çibdadanos } e los que son en el regno & mittere bellatores ad partes extraneas , \textbf{ et semper facere bellare | eos } qui sunt in regno : \\\hline
3.2.10 & mas por los estrannos . \textbf{ Mas el uerdadero Rey faze el contrario } assi commo mas coplidamente dixiemos de ssuso ¶ & sed per extraneos . \textbf{ Verus autem Rex econuerso se habet , } ut supra plenius dicebatur . \\\hline
3.2.10 & que con vn unado atormente al otro assi que con vn clauo atenaçe el otro . \textbf{ Mas el buen Rey faze todo el contrario } non procura departimientos & ut clauum clauo retundat . \textbf{ Rex autem econtrario non procurat diuisiones } et partes in regno , \\\hline
3.2.10 & por los destroyr \textbf{ e por los desfazer . } o das estas cautelas de los tiranos & sed si quae ibi existunt , \textbf{ eas amouere desiderat . } Omnes cautelas tyrannicas , \\\hline
3.2.11 & para que alguno \textbf{ seleunate o faga algun ayuntamiento contra el tyrano de alguna destas quatro coosas puede sallir ¶ } La primera sale le de sabiduria e de grant entendimiento . & vel machinetur aliquid \textbf{ contra tyrannum | ex aliquo praedictorum quatuor videtur procedere . } Primo enim potest hoc accidere ex magnanimitate , \\\hline
3.2.12 & entiende a obras mas bueans e mas uirtuosas \textbf{ por que se faga meior } et mas uirtuolo & et ad virtutem , \textbf{ ut efficiatur melior et virtuosior . } Sed si principatur quia diues , \\\hline
3.2.12 & e por que vse del delecta connes \textbf{ e de plazenterias faze mal de su cuerpo } e faze much stuertos alos çibdadanos e enlas mugres e en las fijnas . & ut vero voluptatibus fruatur fore facit , \textbf{ et infert contumeliam ciuibus } quantum ad uxores et filias . \\\hline
3.2.12 & e de plazenterias faze mal de su cuerpo \textbf{ e faze much stuertos alos çibdadanos e enlas mugres e en las fijnas . } Et por ende veyendo se aborresçido del pueblo & et infert contumeliam ciuibus \textbf{ quantum ad uxores et filias . } Videt ergo se esse odiosum populo , \\\hline
3.2.12 & que nunca mostraua la cara alegte \textbf{ e aquel tirano quariendo dar razon desto fizo despoiar a su hͣmano } e fizola tar & et quare nunquam hylarem vultum ostenderet . \textbf{ Tyrannus ille volens reddere causam quaesiti , | eum expoliari fecit , } et ligari : \\\hline
3.2.12 & e aquel tirano quariendo dar razon desto fizo despoiar a su hͣmano \textbf{ e fizola tar } e fizol colgar una espada sobre su cabesça muy aguda de vn filo muy delgado & eum expoliari fecit , \textbf{ et ligari : } et supra caput eius acutissimum gladium \\\hline
3.2.12 & e fizola tar \textbf{ e fizol colgar una espada sobre su cabesça muy aguda de vn filo muy delgado } e fizo & et ligari : \textbf{ et supra caput eius acutissimum gladium | pendentem tenuissimo filo apponi fecit : } circa ipsum quosdam homines cum ballistis , sagittis appositis , \\\hline
3.2.12 & e fizol colgar una espada sobre su cabesça muy aguda de vn filo muy delgado \textbf{ e fizo } poñuallesteros con ballestas armadas contrael . & pendentem tenuissimo filo apponi fecit : \textbf{ circa ipsum quosdam homines cum ballistis , sagittis appositis , } stare faciebat . \\\hline
3.2.12 & e el otro respodio \textbf{ que lo non podia fazer } por muchs peligros qual esta una aprestados . & hylarem ostende vultum . \textbf{ Respondente illo quod non posset } propter imminentia pericula : \\\hline
3.2.12 & e quanto puede para abaxar los nobles e los altos \textbf{ E abn esto faze el tirano } ca asi commo paresçe & sed satagit opprimere nobiles , et insignes . \textbf{ Hoc etiam tyrannus facit , } quia ut patet ex habitis ipse pro viribus nititur \\\hline
3.2.13 & gouernemjento derecho sea tiranizar \textbf{ e fazer tuerto alos subditos } e non entender al bien comun & cum deuiare a recto regimine sit tyrannizare , \textbf{ et iniuriari subditis , } et non intendere commune bonum ; \\\hline
3.2.13 & asi commo desanparados acometen alos otros \textbf{ e fazense muny buenos } onde el prouerbio dize & quasi desperantes inuadunt alios , \textbf{ et efficiuntur probi . } Unde et prouerbialiter dicitur , \\\hline
3.2.13 & onde el prouerbio dize \textbf{ que quien muncho faze foyr al temeroso } por fuerça lo costrange desee oscido en essa misma manera & Unde et prouerbialiter dicitur , \textbf{ quod nimis fugans timidum , } vi compellit esse audacem . \\\hline
3.2.13 & e algunas vegadas Lo matan \textbf{ ca por la mayor parte los tiranos fazen algunas cosas } por que se fazen despreçiados de los pueblos & et aliquando perimunt ipsum : \textbf{ quia ut plurimum tyranni faciunt ea } per quae se contemptibiles reddunt . \\\hline
3.2.13 & ca por la mayor parte los tiranos fazen algunas cosas \textbf{ por que se fazen despreçiados de los pueblos } ca por qua non qͥeren el bien comun & quia ut plurimum tyranni faciunt ea \textbf{ per quae se contemptibiles reddunt . } Nam cum non quaerant bonum commune , \\\hline
3.2.13 & mas son luxiosos \textbf{ e por que esto fazen son menospreçiados delos omes } e de ligero los acometen & sed dant operam venereis : \textbf{ et quia hoc agentes se contemptibiles reddunt , } de facili inuaduntur . \\\hline
3.2.13 & por que ayan el su señorio mas por que paresca alos omes \textbf{ que fazen algunos omes apartadas } ca algunos quieren ser en alguna nonbrada o en alguna fama & sed ut videantur \textbf{ facere actiones aliquas singulares . } Volunt enim aliqui esse in aliquo nomine \\\hline
3.2.13 & e por ende por que sean nonbrados \textbf{ fazen alguna cosa maran llosa } e algund fhon estraño & ideo ut nominentur \textbf{ faciunt aliquid mirabile } et aliquod singulare factum : \\\hline
3.2.13 & por ende algunos le unatanse contra el mas tales \textbf{ que por esta rrazon fazen acometimjento al tirano } son muny pocos & Sed tales , \textbf{ qui propter hanc causam facientes impetum in tyrannum } ( ut ait Philos’ ) \\\hline
3.2.13 & com muy grand \textbf{ acuçiaqua non se faga tirano } ca si non fisjere tueᷤto alos sbditos en si fuere bueno rmesurador & summa diligentia cauere debet , \textbf{ ne conuertatur in tyrannum . } Nam si non iniuriatur subditis , \\\hline
3.2.14 & segunt dicho es se aya de destroyr el prinçipado tiranico . \textbf{ Lo terçero se desfaze la tirani a non solamente } por si misma o por otra tirama contraria . & cum tot modis dissoluatur tyrannicus principatus . \textbf{ Tertio dissoluitur tyrannis | non solum propter seipsam } vel propter tyrannidem aliam , \\\hline
3.2.14 & e en tantas \textbf{ manerasse ha de desfazer el su prinçipado } non es bue no de tiranizar mas el regno & et tot modis habet \textbf{ dissolui eius principatus . Regium autem dominium } non tot periculis exponitur , \\\hline
3.2.15 & e dela çibdat \textbf{ las quals conuiene al Rey de fazer } para que se pueda man tener en lu prinçipado e en lu lennorio ¶ & quae politiam saluant , \textbf{ et quae oportet facere Regem ad hoc } ut se in suo principatu praeseruet . \\\hline
3.2.15 & e el gouernamiento del regno es bien vsar \textbf{ e bien fazer aquellos que son en el regno } poniendo los en alguons prinçipados & et regnum regium , \textbf{ est bene uti iis qui sunt in regno , } introducendo eos ad aliquos principatus , \\\hline
3.2.15 & e honrrando los \textbf{ e non les faziendo tuerto . } Ca assi commo dize el philosofo & honorando eos , \textbf{ et non iniuriando eis . } Nam ut innuit Philosophus in Poli’ \\\hline
3.2.15 & Mas avn por esta razon el prinçipado \textbf{ se faze mas durable } puesto que en el sea alguna cosa meztlada de maldat ¶ & non solum praeseruat politiam rectam , \textbf{ sed etiam principatus ex hoc durabilior redditur , } dato quod in ipso sit aliquid obliquitatis ad mixtum . \\\hline
3.2.15 & e en el regno \textbf{ ca las corrupconnes alongadas de fecho } e allegadas por temor & qui sunt in politia : \textbf{ nam corruptiones longe secundum rem , } prope autem secundum timorem politiam saluant : \\\hline
3.2.15 & que son de dentro dela çibdat \textbf{ e faze quelos çibdadanos sean mas acordados e mas ayuntados en vno . } Et desto auemos exenplo en los romanos & Guerra enim exterius tollit seditiones intrinsecas , \textbf{ et reddit ciues magis unanimes et concordes . } Exemplum huius habemus in Romanis , \\\hline
3.2.15 & e departidos los nobles \textbf{ fazense discordias } e departimientos en el regno & Nam baronibus dissentientibus \textbf{ fiunt seditiones in regno , } et per consequens fit praeparamentum \\\hline
3.2.15 & por la mayor parte corronpen las uoluntades de los oens \textbf{ por que se fagan traspassadores dela iustiçia } Mas esta cautela es muy aprouechosa & ut plurimum corrumpunt mentes hominum , \textbf{ ut fiant transgressores iustitiae . Est autem haec cautela maxime utilis ad homines , } de quibus Rex certam \\\hline
3.2.15 & e mala çerca aquello que ama \textbf{ ca el temor faze alos omes tomar conseio } e ser sabios & ne aliquod inconueniens accidat circa amatum . \textbf{ Timor autem consiliatiuos facit , } ut dicitur 2 Rhet’ \\\hline
3.2.15 & que biue de furto o de rapina . \textbf{ ca assi fazie do podra guardar la iustiçia } e guardar el regno de malefiçios & vel ex male ablato viuat . \textbf{ Sic enim faciendo ista , | poterit seruare iustitiam , } et praeseruare regnum a maleficis , \\\hline
3.2.16 & Reyr \textbf{ quales cosas le conuiene de fazer } para que derechamente gouierne el pueblo qual es acomendado . & manifestauimus item quod sit Regis officium , \textbf{ et quae oporteat ipsum facere } ut recte regat populum sibi commissum : \\\hline
3.2.16 & por si mas en quanto siruen algunas obras nr̃as . \textbf{ Assi commo si alguas obras nras mas coueniblemente se feziessen en el tpo caliente } e otras en el tp̃ofrio mas . & sed prout deseruiunt actionibus nostris : \textbf{ ut quia aliqua humana opera | congruentius fiunt tempore calido , } aliqua vero tempore frigido : \\\hline
3.2.16 & Las mouimientos delas estrellas \textbf{ que han a fazer calentura o frio segunt } departidost pons pueden caer en nuestro consseio & cursus syderum quae inducere habet \textbf{ secundum diuersa tempora calorem et frigiditatem , } non per se , \\\hline
3.2.16 & por que sepamos en \textbf{ quetp̃o son de fazer algunas obras } e en que tp̃o orͣ̃s . & ut sciamus quo tempore \textbf{ quae opera sunt fienda . } Tertio non sunt consiliabilia \\\hline
3.2.16 & aquellas cosas \textbf{ que se fazen muchͣs uezos li le fazen naturalmente } e por ende delas luuias & Tertio non sunt consiliabilia \textbf{ etiam quae fiunt frequenter , | si fiunt a natura . } Ideo de imbribus \\\hline
3.2.16 & e por ende delas luuias \textbf{ que sienpre se fazen en el tpon del yuierno } e delas calenturas & Ideo de imbribus \textbf{ quae semper fiunt tempore hyemali , } et de caumatibus \\\hline
3.2.16 & e delas calenturas \textbf{ que muchas uezes se fazen en el tro del } estiuo non auemos a tomar consseio & et de caumatibus \textbf{ quae sepe contingunt tempore aestiuali , } non habet esse consilium : \\\hline
3.2.16 & en aquellas cosas \textbf{ que se fazen pocas vezes } si contesçe a auentura & Quarto non sunt consiliabilia \textbf{ quae etiam fiunt raro , } si ex fortuna contingant . \\\hline
3.2.16 & cada vne de los omes toma conseio de aquellas obras \textbf{ que se puden fazer por el } ¶Lo vi̊ non caen sosico consseio todas aqllas cosas & consiliantur de iis operabilibus , \textbf{ quae fieri possunt per ipsos . } Sexto non sunt consiliabilia \\\hline
3.2.16 & e toma assi \textbf{ conmocosa çierta e conosçida e ha consseio en qual manera estas cosas se pueden meior fazer . } Et pues que assi es los consseios son de aquellas cosas & sed haec accipit tanquam certa et nota , \textbf{ et consiliatur quomodo melius fieri possint . } Sunt ergo consiliabilia \\\hline
3.2.17 & que pueden obrar los omes \textbf{ ca pueden se fazer questiones } en las sciençias especulatiuas & sed solum de agibilibus humanis : \textbf{ possunt autem circa speculabilia , } et circa naturas rerum , \\\hline
3.2.17 & ca es question delas obras \textbf{ que pueden fazer los omes finca de ver } en qual manera es de tomar el conseio & quia est quaestio agibilium humanorum : \textbf{ restat videre qualiter est consiliandum , } et quem modum in consiliis habere debemus . \\\hline
3.2.17 & e non son çiertas del todo \textbf{ nin determinadas en qual manera se deue fazer . } Por ende dudamos & et non sunt omnino certa \textbf{ et determinata qualiter fieri debeant , } ideo dubitamus \\\hline
3.2.17 & pues que assi es vna manera es de toraar en los consseios e es esta . \textbf{ que quando algun fecho del regno se propone } quanto al fech & Est ergo unus modus in consiliis adhibendus , \textbf{ quod quando aliquod factum regni proponitur , } quanto pluribus modis fieri potest \\\hline
3.2.17 & quanto al fech \textbf{ por mas maneras se puede fazer . } Et quanto menos ha çiertas & quod quando aliquod factum regni proponitur , \textbf{ quanto pluribus modis fieri potest } et quanto minus habet certas et determinatas vias , \\\hline
3.2.17 & e determinadas carreras \textbf{ para se fazer tanto mayor tienpo ha menester omne } para tomar consseio dello . & et quanto minus habet certas et determinatas vias , \textbf{ tanto per plus tempus est consiliandum , } ut de illis viis facilior et melior eligatur . \\\hline
3.2.17 & Ca dicho fue desuso \textbf{ que el temor faze alos omes tomar conseio . } Et por ende paresçe & Dicebatur enim supra , \textbf{ quod timor consiliatiuos facit : } qui ergo consiliatur , \\\hline
3.2.17 & connusco otros con los quales ayamos acuerdo delas cosas \textbf{ que auemos de fazer } ca commo quier que el omne entre ssi mismo pueda fallar carreras e maneras para fazer alguna cosa & inter quos conferamus \textbf{ de negociis fiendis . } Nam licet homo inter seipsum possit \\\hline
3.2.17 & que auemos de fazer \textbf{ ca commo quier que el omne entre ssi mismo pueda fallar carreras e maneras para fazer alguna cosa } enpero non es sabio aquel & de negociis fiendis . \textbf{ Nam licet homo inter seipsum possit | inuenire vias et modos ad aliquid peragendum , } attamen imprudens est \\\hline
3.2.17 & que con los otros ayamos acuerdo \textbf{ delo que auemos de fazer la qual cosa paresçe por dos cosas . } ca los conseios assi conmo dicho es & ut cum aliis conferamus quid agendum , \textbf{ quod ex duobus patet . } Nam consilia \\\hline
3.2.17 & por que el conseio es delas obras particulares \textbf{ que fazemos en las quales vale much la prueua } ca en las tales cosas & circa agibilia particularia , \textbf{ in quibus multum valet experientia . } Nam in talibus , \\\hline
3.2.17 & que non fablen y cosas plazenteras mas uerdaderas \textbf{ ca los lisongeros estudiando de fazer plaza los prinçipes callan la uerdat } e dizen las cosas & ut non loquantur ibi placentia , sed vera . \textbf{ Adulatores enim | dum Principi placere student , } vera silentes , \\\hline
3.2.17 & si derechͣmente queremos obrar \textbf{ e non lo fazemos e esto es por que non sabemos } si auemos de fazer aquella cosa o non ¶ & et si recte volumus \textbf{ et non illud facimus , } hoc est quia ignoramus an expediat illud fieri . \\\hline
3.2.17 & e non lo fazemos e esto es por que non sabemos \textbf{ si auemos de fazer aquella cosa o non ¶ } pues que assi es muy bien es de & et non illud facimus , \textbf{ hoc est quia ignoramus an expediat illud fieri . } Bene ergo se habet diligenter \\\hline
3.2.17 & pues que assi es muy bien es de \textbf{ escodrinnar con grant acuçia todo negoçio alto e noble si es prouechoso delo fazer . } mas despues que fuere conosçido derechamente & Bene ergo se habet diligenter \textbf{ quodlibet negocium discutere arduum , | an utile sit illud facere : } sed post quam per diuturnum consilium est recte cognitum \\\hline
3.2.17 & por el conse io prolongado \textbf{ que es lo que deuemos fazer } si ouieremos tp̃o conueinble e poder para obrar & sed post quam per diuturnum consilium est recte cognitum \textbf{ quid fiendum , } si adsit operandi facultas , \\\hline
3.2.17 & e luego e que conuiene de touiar conseio prolongadamente \textbf{ mas conuiene de fazerl cosas conseiadas mucho ayna . } mar ala Real magestado das aquellas cosas & et quod oportet consiliari tarde , \textbf{ sed facere consiliata velociter . } Omnia autem illa quae habere debet bene persuadens \\\hline
3.2.18 & por razones \textbf{ e en quantas maneras les ha omne de fazer fe . } e en quantas maneras son los omes enclinados & quot modis persuadetur hominibus , \textbf{ vel quot modis fit eis fides } et inclinantur ad credendum sermones auditos . \\\hline
3.2.18 & de ligero creen los omes asodichos . \textbf{ lo segundo se puede fazer creençia alos oydores } de parte de los dizidores la qual cosa contesçe & de facili creditur eorum dictis . \textbf{ Secundo potest fieri credulitas auditoribus } ex parte ipsorum auditorum : \\\hline
3.2.18 & de quanto son enlos negoçios \textbf{ que han de fazer comunalmente creen los omes } que veen mas cosas de quanta sueen & et esse magis sapientes quam sint , \textbf{ et in negotiis fiendis communiter credunt homines } plus videre quam videant , \\\hline
3.2.18 & por que las sabe conosçer e iudgar uiene la creençia \textbf{ e faze fe el sabio alos oydores . } mas esta creençia et este amonestamiento es por si . & cognoscere et iudicare , \textbf{ facit fidem auditoribus . } Haec autem credulitas \\\hline
3.2.18 & mas esta creençia et este amonestamiento es por si . \textbf{ Ca fazerse el omne digno de creer } e buen amonestador e razonador por si . & et haec persuasio est per se : \textbf{ nam reddere se credibilem } et bene persuadere per se , \\\hline
3.2.18 & razonadorca sabe tomar razones e argumentos \textbf{ por los quales faga fe alos oydores } e porque los sabios saben esto fazer & de quibus loquitur scire assumere rationes et argumenta , \textbf{ per quae fides fiat audientibus . } Et quia prudentes sciunt facere , \\\hline
3.2.18 & por los quales faga fe alos oydores \textbf{ e porque los sabios saben esto fazer } e aquellos que son tenidos & per quae fides fiat audientibus . \textbf{ Et quia prudentes sciunt facere , } et qui existimantur prudentes , \\\hline
3.2.18 & por sabios son contados \textbf{ para fazer tales cosas } Morende para que alguno faga fe delas cosas de que fabla o conuiene que sea sabio & et qui existimantur prudentes , \textbf{ existimantur talia facere : } ideo ad hoc quod aliquis ex rebus \\\hline
3.2.18 & para fazer tales cosas \textbf{ Morende para que alguno faga fe delas cosas de que fabla o conuiene que sea sabio } o que sea tenido por sabio . & existimantur talia facere : \textbf{ ideo ad hoc quod aliquis ex rebus | de quibus loquitur fidem faciat , vel oportet quod sit prudens } vel quod credatur esse prudens . \\\hline
3.2.18 & ca conosçen los negoçios \textbf{ que han de fazer } e saben en qual manera los han de fazer . & de quibus loquuntur ; \textbf{ quia cognoscent negocia agibilia , } et scient qualiter sit agendum . \\\hline
3.2.18 & que han de fazer \textbf{ e saben en qual manera los han de fazer . } Et pues que assi es estas tres cosas & quia cognoscent negocia agibilia , \textbf{ et scient qualiter sit agendum . } Haec ergo tria quaerenda sunt in consiliariis : \\\hline
3.2.19 & e puestas ordena \textbf{ çonns quals deuen o quales se pueden fazer } ca non es pequana cosa de auer consseio er la mantenençia & ut circa haec debita consilia \textbf{ et debitae ordinationes fieri possint : } non enim modicum consiliandum est circa alimentum , \\\hline
3.2.19 & Et en estas cosas \textbf{ que parte nesçen para la uida deuense fazer mudaçiones } e canbios conuenibles & quia aliter iam non esset ciuitas : \textbf{ ut in huiusmodi sufficientibus ad vitam fieri debent debitae commutationes , } ut debitae emptiones , \\\hline
3.2.19 & e del regno \textbf{ la qual cosa se puede fazer en dos maueras . } ca lo primero es de poner guarda & circa custodiam ciuitatis et regni , \textbf{ quod dupliciter fieri habet : } nam primo est custodia adhibenda \\\hline
3.2.19 & ca lo primero es de poner guarda \textbf{ por que se non le una ten discordias nin se fagan malefiçios entre los çibdadanos . } por ende deuemos cuydar con grant acuçia & nam primo est custodia adhibenda \textbf{ ne insurgant seditiones et malitia inter ciues . } Ideo attendendum est diligenter \\\hline
3.2.19 & Avn son de penssar los logares \textbf{ en los quales se suelen fazer malos malefiçios } ca assi commo alguons delos omes fazen mayores tuertos & Sunt etiam consideranda loca \textbf{ in quibus consueuerunt magis maleficia perpetrari : } nam sicut quidam hominum magis iniustificant quam alii \\\hline
3.2.19 & en los quales se suelen fazer malos malefiçios \textbf{ ca assi commo alguons delos omes fazen mayores tuertos } que los otros & in quibus consueuerunt magis maleficia perpetrari : \textbf{ nam sicut quidam hominum magis iniustificant quam alii } sic sunt quaedam loca magis apta ad iniustificandum quam alia : \\\hline
3.2.19 & assi son algunos logares mas apareiados \textbf{ para fazer mal e mi usticia que los otros } assi commo contesçe & nam sicut quidam hominum magis iniustificant quam alii \textbf{ sic sunt quaedam loca magis apta ad iniustificandum quam alia : } ut in ciuitate contingit \\\hline
3.2.19 & e son brios mas apareiados \textbf{ para fazer mal que los otros . } e por ende deue ser tomado consseio & loca aliqua sunt nemorosa et umbrosa \textbf{ magis apta ad iniustificandum , | quam alia : } debet ergo adhiberi consilium , \\\hline
3.2.19 & por essos mismos çibdadanos o por aquellos que son en el regno \textbf{ que el vno non faga tuerto contra el otro } mas avn deuen tomar consseios & qui sunt in regno , \textbf{ ne unus iniustificet in alium : } sed etiam propter ipsos extraneos : \\\hline
3.2.19 & que non sea con razon e con derecho . \textbf{ por que fazer tuerto alos otros } e apremiar sos sin derecho es mala cosa por si & Primo , ut nunquam capiatur iniustum bellum , \textbf{ quia iniustificari in alios , } et eos indebite opprimere , \\\hline
3.2.19 & Ca puesto que los mas \textbf{ poderolos alos quales non podemos contradezer nos fagan alguna fuerca o algun tuerto grant sabiduria } es non nos leunatar contra ellos & Posito enim quod potentiores , \textbf{ ad quos resistere non valemus , | in nos forefaciant , prudentiae est , } non insurgere in ipsos , \\\hline
3.2.20 & que pueden acaesçer çerca desta materia \textbf{ mas commo el iuyzio se deua fazer } por las leyes o por aluedrio o por amas estas cosas . & et alia quae circa istam occurrunt materiam . \textbf{ Sed cum iudicium fiat per leges , } aut per arbitrium , \\\hline
3.2.20 & Et en aquella misma çibdat \textbf{ en que se fazen las leyes } pueden ser mas uiezes & Immo illa eadem ciuitate \textbf{ in qua leges conduntur } contingit \\\hline
3.2.20 & corronper nin mudar . \textbf{ ca en lłas non deue ser fecha mudaçion ninguna o muy pequana . } Enpero los iuezes & Nam leges si iustae sint , debent esse quasi immortales et immutabiles : \textbf{ quia circa eas nulla aut modica mutatio fieri debet . } Iudices tamen iudicantes \\\hline
3.2.20 & e delas cosas que auien de venir diziendo \textbf{ que qual quier que tal cosa fiziere tal pena aura } non sabiendo si serie amigo o enemigo & in uniuersali et de futuris , \textbf{ dicentes quicunque sic egerit , } sic puniatur , ignorantes an amicus , \\\hline
3.2.20 & non sabiendo si serie amigo o enemigo \textbf{ aquel que auie de fazer aquella cosa } e deuie passar por tal suina . & sic puniatur , ignorantes an amicus , \textbf{ vel inimicus sit illa facturus , } et debeat illam subire sententiam . \\\hline
3.2.20 & ca si por auentra asopiessen ellos \textbf{ que su amigo auie de fazer aquella cosa } por & et debeat illam subire sententiam . \textbf{ Nam si scirent quod amicus , } forte obliquerentur in iudicando , \\\hline
3.2.20 & non se tuerçen en iudgando \textbf{ nin en faziendo las leyes enclinados se } por amor o por mal querençia . & non peruertuntur \textbf{ in iudicando amore , } vel odio inclinati . \\\hline
3.2.20 & por amor o por mal querençia . \textbf{ Mas los iuezes non fazen } assi ca los iuyzios de los uiezes & vel odio inclinati . \textbf{ Iudices autem non sic . } Nam iudicum iudicia non sunt de futuris , \\\hline
3.2.20 & Lo primero por que mas ligeramente pueden auer los omes vn sabio o pocos que muchos . \textbf{ lo segundo por que los establesçimientos delas leyes se fazen } por penssamiento prolongado & unam aut paucos sapientes , \textbf{ quam multos . | Deinde quia legislatores fiunt } ex consideratis ex multo tempore , \\\hline
3.2.20 & los que han de dar los iuyzios \textbf{ tanto mas ayna uerna fazer essecuçion } e dar fin a los iuyzios & quanto utique minor inimicitia fuerit exequentibus iudicia , \textbf{ tanto magis accipient } finem executione iudiciorum . \\\hline
3.2.20 & ca non paresçe \textbf{ que lo faze el } por si mas & quia non videtur ipse \textbf{ secundum se hoc agere , } sed lege compulsus dicitur \\\hline
3.2.20 & por si mas \textbf{ que lo faze costrennido por la ley } e por ende es dicho dar e publicar tal snïa & secundum se hoc agere , \textbf{ sed lege compulsus dicitur } talem sententiam promulgare . \\\hline
3.2.21 & sabiendo que tienen mal pleito \textbf{ non cuentan lo que es fecho } o lo que non es fecho & multi enim litigantium cognoscentes se habere malam causam , \textbf{ non narrant quid factum et quid non factum , } sed conuertunt se \\\hline
3.2.21 & non cuentan lo que es fecho \textbf{ o lo que non es fecho } mas tornassea mouer el uuez a sana o a aborrençia & multi enim litigantium cognoscentes se habere malam causam , \textbf{ non narrant quid factum et quid non factum , } sed conuertunt se \\\hline
3.2.21 & e iudgar mal e desigual mente . \textbf{ Et por que esto fazen las palabras desiguales } et malas consentir tales palabras en el iuyzio non es otra cosa & quasi regula tortuosa peruerse iudicabit , \textbf{ et quia hoc faciunt sermones passionales , } permittere talia in iudicio nihil est aliud quam regulam obliquare : \\\hline
3.2.21 & enclinan la uoluntad de los omes \textbf{ e fagan paresçer alguna cosa derecha . } por que los que assi son munnidos & inclinent voluntatem , \textbf{ et faciant apparere aliquid iustum , } vel non iustum , \\\hline
3.2.21 & Ca las partes mouiendo el iuez \textbf{ assi fazen paresçer } alguͣ cosa derechͣo non de rethica . & quia partes passionando iudicem , \textbf{ ei faciunt apparere aliquid iustum vel iniustum , } quod non est officium partium , \\\hline
3.2.21 & e qual non es el derecho . \textbf{ Mas esto non se deue fazer } por las partes que contienden o por las palabras desiguales dichas delas partes . & quid iustum et quid iniustum : \textbf{ non autem hoc debet fieri } per partes litigantes , \\\hline
3.2.21 & por palabras contando le las miurias . \textbf{ las quales la parte contraria fizo al iuez } o contando le los bienes & aut narrare iniurias \textbf{ quas pars aduersa iudici intulit , } vel narrare bona \\\hline
3.2.21 & o contando le los bienes \textbf{ que ellos fizieron a liiez . } Et en esta manera inclinar al iuez a malenconia & vel narrare bona \textbf{ quae ipsi iudici contulerunt , } et hoc modo prouocare iudicem \\\hline
3.2.22 & al fazedor dela ley \textbf{ fazen que el iuyzio sea fortado . } Et entonçe los iuezes non se han derechamente al fazedor dela ley & si non recte se habent ad legislatorem , \textbf{ faciunt iudicium usurpatum . } Tunc quidem dicuntur iudices non recte se habere ad legislatorem , \\\hline
3.2.22 & que les es a comne dada . \textbf{ Lo segundo los iuezes fazen iuyzio loco } quando non se han derechamente & quando excedunt auctoritatem sibi commissam . \textbf{ Secundo dicuntur iudices | facere iudicium temerarium , } si non recte se habent ad leges , \\\hline
3.2.22 & que contienden \textbf{ fazen ser el iuyzio malo e desigual . } Ca assi commo paresçe & ad partes litigantes , \textbf{ facient iudicium iniquum : } nam ut patet ex habitis iudex debet esse quasi regula recta media inter utrasque partes : \\\hline
3.2.22 & en las obras de los omnes \textbf{ faran el iuyzio ser sospechoso e arrebatado . } Ca algunas uezes & ut si sint inexperti humanorum actuum , \textbf{ facient iudicium suspitiosum , } quia aliquando ex leui suspitione alios condemnabunt . \\\hline
3.2.23 & es muchedunbre de buean sobras . \textbf{ ca por auentra a aquel que agora peca fizo ante muchas bueans obras } Et por ende eliez non deue & Nam sorte ille \textbf{ qui nunc deliquit | multa bona opera prius fecit : } debet ergo iudex non ita respicere ad partem \\\hline
3.2.23 & nin deue tener mientes a vna obra \textbf{ que fizo mas a todas las buenas obras } que auia fecho ¶ & quod iudicans non debet respicere ad partem , \textbf{ sed ad totum . } Sextum est diuturnitas temporis retroacti . \\\hline
3.2.23 & que fizo mas a todas las buenas obras \textbf{ que auia fecho ¶ } Lo sexto que inclina ali es a piedat es alongamiento & quod iudicans non debet respicere ad partem , \textbf{ sed ad totum . } Sextum est diuturnitas temporis retroacti . \\\hline
3.2.23 & Lo sexto que inclina ali es a piedat es alongamiento \textbf{ detpo passado por que contesçe que alas uezes alguno en poco tp̃o faze muchas buenas obras . } Et por ende dos cosas deuen endozir al Rey o al prinçipe & Sextum est diuturnitas temporis retroacti . \textbf{ Nam contingit etiam in pauco tempore facere multa bona opera : } duo ergo debent inducere Regem \\\hline
3.2.23 & ca el que mucho t pon sirmo \textbf{ por la mayor parte much sseruiçios fizo . } Et el que muchs seruiçios fizo por la mayor parte mucho t p̃o siruo & ut plurimum se committentur , quia qui multo tempore seruiuit \textbf{ ut plurimum multa seruitia fecit , } et econuerso ; \\\hline
3.2.23 & por la mayor parte much sseruiçios fizo . \textbf{ Et el que muchs seruiçios fizo por la mayor parte mucho t p̃o siruo } Et enpero contesçe que alas uezes & ut plurimum se committentur , quia qui multo tempore seruiuit \textbf{ ut plurimum multa seruitia fecit , } et econuerso ; \\\hline
3.2.23 & que algun subdito peca agora en algua parte detp̃o \textbf{ el que fizo bien en todo } elt pon passado & in aliqua parte temporis delinquere : \textbf{ qui toto tempore se bene habuit praecedenti , } est cum ipso misericorditer agendum , \\\hline
3.2.23 & que resçibieron del que \textbf{ erroque del tuerto que les fizo . } Lo viij̊ que enduze el iuez amisicordia & quae passi sunt a delinquente , \textbf{ quam iniuriae quam fecit . } Octauum est patientia incusati . \\\hline
3.2.23 & que por solo deuuesto e por la sola \textbf{ palabrase meioran e dexan de fazer mal } e atalon deuemos mucho perdonar . & quod sola increpatione , \textbf{ idest solo sermone meliorantur | et desinunt praua agere : } talibus ergo est valde indulgendum , \\\hline
3.2.23 & e esto prouamos \textbf{ por que los canes non fazen mal a aquellos que se les homillan } e que se echa ante ellos . & cum bestiae hoc agant : \textbf{ canes quidem non offendunt humiliantes se , } et prosternentes se coram eis . \\\hline
3.2.24 & Mas nos podemos tan bien dela ley \textbf{ commo del derecho fazer çinco departimientos } de los quales los dos pone el pho enel primero libro de la rectorica . & Possumus autem tam de lege \textbf{ quam de iusto | quinque distinctiones facere , } quarum duae tanguntur 1 Rhet’ tertia \\\hline
3.2.24 & Et estos çinco departimientos \textbf{ que fiziemos del derecho dela cosa } derechͣ podemos fazer dela ley . & Has ergo quinque distinctiones , \textbf{ quas fecimus de iure siue de iusto , } facere possumus de ipsa lege . Ut ergo haec omnia melius patefiant , \\\hline
3.2.24 & que fiziemos del derecho dela cosa \textbf{ derechͣ podemos fazer dela ley . } Et por ende por que estas cosas todas sean meior mostradas . & quas fecimus de iure siue de iusto , \textbf{ facere possumus de ipsa lege . Ut ergo haec omnia melius patefiant , } et ut has diuersitates ad concordiam reducamus sciendum \\\hline
3.2.24 & por que la razon natural muestra \textbf{ que se deuen fazer . } Et auemos natural inclinaçion & de iure naturali , \textbf{ quia haec esse fienda | dictat ratio naturalis , } et habemus naturalem impetum \\\hline
3.2.24 & Et auemos natural inclinaçion \textbf{ que estas cosas se fagan . } ca estas cosas sele una tan dela natura dela cosa . & et habemus naturalem impetum \textbf{ ut haec fiant . } Surgunt enim ista \\\hline
3.2.24 & que non han ley natural \textbf{ fazen aquellas cosas } que son dela ley e muestran la obra dela ley escpta en sus coraçons . & Nam gentes quae legem non habent , \textbf{ naturaliter ea quae legis sunt faciunt , } et ostendunt opus legis scriptum \\\hline
3.2.25 & Mas si aquellas reglas se tomaren en quanto el omne naturalmente \textbf{ dessea fazer fiios e carlos } assi podria ser de derecho natural & ex eo quod homo naturaliter \textbf{ appetit filios producere et educare : } sic esse poterunt de iure naturali , \\\hline
3.2.26 & et con las costunbres de los omes e dela tierra e del tienpo . \textbf{ Ca en las obras que fazemos } algunan cosa auemos de dar ala costunbre & et compossibilis consuetudini patriae et tempori : \textbf{ nam in agibilibus aliquid } dandum est consuetudini , tempori , et patriae \\\hline
3.2.26 & e esto queremos alcançar \textbf{ conuiene que fagamos estas cosas . } Et pues que assi estales deuen ser las leyes & et hoc sequi volumus , \textbf{ oportet hoc agere . } Tales ergo debent esse leges , \\\hline
3.2.26 & para fazerobras uirtuosas \textbf{ e para dexar de fazer obras malas . } Mas alguons otros son tales & ut agant opera virtuosa , \textbf{ et desistant agere peruerse . Aliqui vero si ex seipsis , } id est , \\\hline
3.2.26 & non son conplidamente inclinados \textbf{ para dexar el mal e fazer el bien . } Enpero son de ligero disçiplinables e corrigibles por los otros buenos . & in seipsis \textbf{ non sufficienter inclinantur ad bonum , } attamen sunt facile disciplinabiles per alios , \\\hline
3.2.27 & son algunas reglas delas obras \textbf{ que auemos de fazer } que nos ordenan a bien comun & sunt quaedam regulae agibilium , \textbf{ ordinantes nos in commune bonum , } habentes coactiuam potentiam . \\\hline
3.2.27 & que non parte nesçe aquel \textbf{ si quier de fazer leyes ¶ } La primera se toma de aqual & Duplici ergo via venari possumus , \textbf{ quod non cuiuslibet est leges condere . } Prima via sumitur \\\hline
3.2.27 & Ca puede cada vno del pueblo amonestar e enduzir a otro o a otros \textbf{ que fagan bien . } Mas estas muniçonnes e estas condiçiones tales non son dich̃ͣs leyes & et persuadere alteri \textbf{ ut bene agat , } sed huiusmodi monitiones et persuasiones non dicuntur leges , \\\hline
3.2.27 & en el primero libro delas politicas \textbf{ dizia omero que cada vno podia fazer leyes a sus fijos e a so mugers . } Caquaria ommo que las leyes e los mandamientos & quod unusquisque statuit \textbf{ legis pueris et uxoribus : | volebat enim Homerus } quod monitiones et praecepta , \\\hline
3.2.27 & Caquaria ommo que las leyes e los mandamientos \textbf{ que faze el padre familias a su muger } e asus fijos & quod monitiones et praecepta , \textbf{ quae facit paterfamilias uxori , filiis , } et aliis existentibus in domo , \\\hline
3.2.27 & quando comiença de auer uso de razon e de entendimiento \textbf{ por el qual conosçe qual cosa ha de fazer e de escoger } e qual cosa ha de foyr e de escusar & quando incipit habere rationis usum , \textbf{ per quam cognoscit } quid sequendum et quid fugiendum , \\\hline
3.2.28 & ¶ \textbf{ El vno es delas obras que son de fazer ante que sean fechͣs . } El otro es delas obras & duplex cura esse potest : \textbf{ una cum opera sunt futura , | priusquam sint effectui mancipata : } alia vero opera \\\hline
3.2.28 & Et en estas dos maneras deuen auer cuydado \textbf{ los que fazen las leyes delas obras humanales ¶ } Et pues que assi es dende & utroque modo debet \textbf{ esse cura conditoribus legum de humanis actibus . } Inde est ergo quod aliqui effectus legum sumuntur \\\hline
3.2.28 & Et pues que assi es dende \textbf{ uiene que algunas obras delas leyes se toman en conparaçion delas obras que son de fazer . } Et alguas se toman en conparaçion delas obras & esse cura conditoribus legum de humanis actibus . \textbf{ Inde est ergo quod aliqui effectus legum sumuntur } respectu operum fiendorum , \\\hline
3.2.28 & Mas en conparaçion delas obras \textbf{ que son de fazer } podemos apodar alas leyes tres cosas conuiene de saber . & aliqui vero respectu operum iam factorum . \textbf{ Respectu fiendorum quidem tria possumos attribuere legibus , } videlicet praecipere , permittere , et prohibere . \\\hline
3.2.28 & para alinpiat la casa \textbf{ o para fazer alguna otra obra buean } por la buena entençion & si vero eleuando eam vellet purgare domum \textbf{ vel facere aliquod aliud opus pium , } propter bonam intentionem operantis , \\\hline
3.2.28 & por la buena entençion \textbf{ de aquel que la faze esta tal obra } que dessi non es buena nin mala & vel facere aliquod aliud opus pium , \textbf{ propter bonam intentionem operantis , } quod de se est quasi indifferens , \\\hline
3.2.28 & enpero puede ser uirtuosa e de loar . \textbf{ pues que assi es segunt estas tres maneras delas obras que son de fazer podemos a podar } e a proprear tres cosas & esse potest virtuosum et laudabile . \textbf{ Secundum igitur haec tria genera fiendorum , } tria legibus attribuimus , \\\hline
3.2.28 & en conparaçion delas obras \textbf{ que son de fazer . } de ligero puede paresçer & Viso quae attribuenda sunt legibus respectu operum fiendorum : \textbf{ de leui apparere potest } quae attribuenda sunt \\\hline
3.2.28 & e dar gualardon . \textbf{ assi que las malas obras ante que se fagan son mucho de defender . } mas despues que son ya fechas son de castigar . & punire et praemiare . \textbf{ Ut opera notabiliter mala , | antequam fiant , } sunt prohibenda : \\\hline
3.2.28 & Et las obras buen asante \textbf{ que se fagan son demandar } e de consseiar por las leyes . & Opera vero notabiliter bona , \textbf{ antequam fiant , } per leges sunt praecipienda et consulenda : \\\hline
3.2.28 & Mas las que non son buenas nin malas o nin son muy buean so muy malas . \textbf{ estas tales ante que se fagan son de conssentir } por las leyes humanales . & aut notabiliter mala , \textbf{ dicuntur legibus humanis esse permissa ; } facta vero nec puniuntur nec praemiantur . \\\hline
3.2.28 & Las dos en conparacion delas buean sobras . \textbf{ Assi commo es mandar que se fagan } e gualardonarlas despues que son fechas . & duo respectu operum bonorum , \textbf{ ut praecipere fienda , } et praemiare facta : \\\hline
3.2.28 & Et otras dos en conparacion delas malas obras \textbf{ assi conmoes vedar que se non fagan . } Et despues que son fechas dar pena por ellas . & et duo respectu malorum , \textbf{ ut prohibere fienda , } et punire facta : \\\hline
3.2.29 & assi que aquello quela ley manda \textbf{ que sea fecho derechamente el Rey } por su pode rio çiuil faga lo guardar . & ut quod iuste lex fieri praecipit , \textbf{ Rex per ciuilem potentiam obseruari facit . } Quare si quod est principalius , \\\hline
3.2.29 & que sea fecho derechamente el Rey \textbf{ por su pode rio çiuil faga lo guardar . } Por la qual & ut quod iuste lex fieri praecipit , \textbf{ Rex per ciuilem potentiam obseruari facit . } Quare si quod est principalius , \\\hline
3.2.29 & Ca ninguno non iudga derechamente nin \textbf{ enssennorea si non faze } assi commo manda la razon derecha & nam nullus recte principatur , \textbf{ nisi agat } ut recta ratio dictat : \\\hline
3.2.29 & sy non se esforçare en la ley natural \textbf{ et si non feziere } assi commo la razon derecha o el entendimiento manda . & nisi innitatur lege naturali , \textbf{ et agat ut recta ratio dictat : } sic lex positiua nunquam recte ligat , \\\hline
3.2.29 & assi commo dicho es de suso en el comienço non ha ningun departimiento \textbf{ que se faga assi en otra manera . } Mas quando se pone & ex principio nihil differt sic , \textbf{ vel aliter ; } quando autem ponitur , \\\hline
3.2.29 & por la qual cosa la ley positiua es a quande del señor \textbf{ que la manda fazer } assi commo la ley naturales sobre el señor & ø \\\hline
3.2.29 & que non es de guardar general mente . \textbf{ Et pues que assi es segunt esto concluya la razo que es fecha en el contrario que meior es de ser gouernado el regno } por buen Rey & quod non est uniuersaliter obseruandum . \textbf{ Secundum hoc ergo concludebat | ratio in oppositum facta , } quod melius est Regi Rege , \\\hline
3.2.29 & Enpero non son prinçipalmente nin sinplemente sin iustiçia . \textbf{ si con razon se fezieren } segunt que mandan las cercunstançias particulares . & praeter iustitiam simpliciter , \textbf{ si rationabiliter fiant } exigentibus particularibus circumstantiis . \\\hline
3.2.29 & del que peca la regla dela ley se encorua ala parte dela mibicordia . \textbf{ Estonçe el iuyzio fecho es dicho nasçer de gran o de piadat . } Mas si la dicha regla fuere encoruada & iudicium tunc factum dicetur \textbf{ procedere ex gratia vel ex clementia . } Sed si dicta regula plicetur \\\hline
3.2.29 & sera el iuizio de fortaleza o de iustiçia estrecha . \textbf{ Et por que todas estas cosas se pueden fazer derechamente } e con razon la reziedunbre dela iustiçia & fiet iudicium ex rigore siue seueritate . \textbf{ Et quia haec omnia iuste } et rationabiliter fieri possunt , clementia et seueritas simul cum iustitia possunt existere . \\\hline
3.2.30 & en las quales paresçe que se defienden todos los pecados \textbf{ e se mandan fazer todas las uirtudes } Mas que sin la ley natural e humanal fue menester de dar ley e un aglical e diuinal . & quae videntur omnia vitia prohibere , \textbf{ et omnes virtutes praecipere . } Sed quod praeter legem naturalem \\\hline
3.2.30 & e ningun bien non fincasse sia gualardon . \textbf{ Mas para esto fazer non cunple la ley natural } assi commo paresçra adelante . & et nullum bonum irremuneratum . \textbf{ Ad hoc autem faciendum non sufficit lex naturalis , } ut in prosequendo patebit : \\\hline
3.2.30 & que si fuere penssada la entençion del \textbf{ que faze la ley todos los pecados son defendidos } o deuen ser defendidos por ley humanal . & si consideretur intentio legislatoris , \textbf{ lege humana omnia peccata prohibentur , } vel prohiberi debent \\\hline
3.2.30 & o qual quier otro fazendor de ley \textbf{ si non entendiere fazer a todos los sus çib } dadanos los mas uirtuosos que pudiere ¶as commo ninguno non pueda venir a acatadas uirtudes sim̃o entendiere escusar todos los pecados & alius legislator , \textbf{ nisi intendat suos conciues | facere } quam virtuosiores potest . \\\hline
3.2.31 & La tercera razon se toma dela sinpliçidat \textbf{ e del poco saber de aquellos que fazen las leyes . } Ca algunas uegadas contesçe & Tertia via sumitur \textbf{ ex simplicitate condentium leges . } Nam si aliquando condentes leges contingit \\\hline
3.2.32 & por que auemos natural inclinaçion \textbf{ e desseo para establesçer e fazer la çibdat . } Enpero non se faze la çibdat & quid naturale , eo quod naturalem habemus \textbf{ impetum ad ciuitatem constituendam : } non tamen efficitur , \\\hline
3.2.32 & e desseo para establesçer e fazer la çibdat . \textbf{ Enpero non se faze la çibdat } nin se acaba & impetum ad ciuitatem constituendam : \textbf{ non tamen efficitur , } nec perficitur ciuitas , \\\hline
3.2.32 & Mas todas aquellas cosas \textbf{ que se fazen } por los omes & nisi ex opere , et industria hominum . \textbf{ Quae autem ex arte humana efficiuntur , diffiniuntur , } et cognoscuntur potissime \\\hline
3.2.32 & si sopieremos \textbf{ que ella es fecha por este bien } por que nos defienda delas suuias & si sciuerimus ipsam esse facta \textbf{ propter hoc bonum , } ut nos defendat a pluiis et caumatibus ; \\\hline
3.2.32 & e por que les pudiesse dar de los sus bienes non termie todos aquellos bienes en much . \textbf{ Et pues que assi es la çibdat fue fecha } por que muchs omes biuiessen en vno en vn logar . & non multum reputaret illa . \textbf{ Facta est ergo ciuitas , } ut homines simul in uno loco viuentes , \\\hline
3.2.32 & e non morassen vnos con otros . \textbf{ Lo terçero fue fecha la çibdat } por defendimiento dessi mismos & si homines solitarii morarentur . \textbf{ Tertio facta fuit ciuitas compugnationis gratia , } et propter non iniustum pati . \\\hline
3.2.32 & Lo quarto fue la çibdat ordenada \textbf{ e fecha por razon delos camios } e delas mudaçiones e de los contractos . & Quarto fuit ciuitas ordinata \textbf{ propter commutationes et contractus . } Dicebatur enim supra \\\hline
3.2.32 & quando fablauamos delas leyes \textbf{ que fazer mudaçiones } e contracto sera & cum de legibus tractabamus , \textbf{ quod facere commutationes , } et contractus erant \\\hline
3.2.32 & dende le una \textbf{ tar fazen sus casamientos } e fazen se cunnados & quod vident inde consurgere , \textbf{ iungunt connubia } et fiunt \\\hline
3.2.32 & tar fazen sus casamientos \textbf{ e fazen se cunnados } e parientes los vnos con los otros . & iungunt connubia \textbf{ et fiunt } ad inuicem affines . \\\hline
3.2.32 & ca mas se pueden castigar \textbf{ los que yerran e fazen mal si los omes biuieren en vno en la çibdat } que si morassen apartados & Nam magis possunt puniri delinquentes et malefici , \textbf{ si homines simul conuiuant in ciuitate , } quam si morarentur dispersim , \\\hline
3.2.32 & que si morassen apartados \textbf{ e fiziessen uida apartada . } Et dende viene que & quam si morarentur dispersim , \textbf{ et vitam solitariam ducerent . } Inde est igitur \\\hline
3.2.32 & muchs dexan de fazermal \textbf{ e acostunbran se a fazer buenas obras . } la qual cosa faziendo los omes & quod timore poenae multi desinunt malefacere , \textbf{ et assuescunt ad operationes bonas : } quod faciendo , disponuntur , \\\hline
3.2.32 & e acostunbran se a fazer buenas obras . \textbf{ la qual cosa faziendo los omes } ordenansse para ser bueons e uirtuosos ¶ & et assuescunt ad operationes bonas : \textbf{ quod faciendo , disponuntur , } et fiunt boni , et virtuosi . \\\hline
3.2.32 & sean muy grandes bienes \textbf{ maguera que por todas las cosas sobredichas sea fecho la çibdat en alguna manera . } Enpero prinçipalmente fue ella establesçida & sunt maxima bonorum , \textbf{ licet propter omnia praedicta bona sit | aliquo modo ciuitas constituta , } potissime tamen constituta est \\\hline
3.2.33 & Ca aquellos que fueren muy ricos \textbf{ e muy poderosos faran tuerto alos otros } e los otros que fueren pobres seran muy malos & hi autem videlicet superpauperes , \textbf{ fient nequi , } valde astuti et latenter insidiantes diuitibus . \\\hline
3.2.33 & e poner so pie alos otros \textbf{ e los pobres contradiziendo alos ricos faran discordia en la çibdat } e si contezca que los pobres venzcan & et suppeditare alios . \textbf{ Alii vero contra nitentes dissensionem faciunt , } et si contingat pauperes obtinere , \\\hline
3.2.33 & que entre ellos non aya egualdat . \textbf{ la qual cosa se puede fazer } si por qual quier razon los çibdadanos & ut ne aliis ad nimiam paupertatem deuenientibus efficiantur reliqui nimis diuites : \textbf{ quod fieri poterit , } si non pro quacunque causa liceat \\\hline
3.2.34 & por que la su entençion es enduzir los otros a uirtud . \textbf{ Et la uirtud faze al que la ha buenon } Esta buean obra conuiene que sea en el gouernamiento derech & inducere alios ad virtutem , \textbf{ cum virtus faciat habentem bonum ; } et opus bonum , \\\hline
3.2.34 & e dela çibdat \textbf{ e los guardadores del regno fazen se libres e francos } si obedesçieren alos Reyes & Saluatur itaque salus regni et ciuitatis , \textbf{ si habitatores regni efficiuntur liberi , } si obediant regibus , \\\hline
3.2.34 & mas fincan por labrar . \textbf{ Et fazen se robos e furtos } e non vienen bien los tenporales . & terrae manent incultae , \textbf{ fiunt depraedationes , } oriuntur sterilitates \\\hline
3.2.35 & ¶ Et pues que assi es quando alguno en alguna destas dichas maneras \textbf{ faze tuerto a otro manifiesta mente . } aquel a quien es fech tu & vel ea quae aliquo modo ordinantur ad ipsum . \textbf{ Cum ergo quis aliquo dictorum modorum manifeste forefacit in alium , } ille tristatus ex appetitu punitionis , \\\hline
3.2.35 & por que non mueun a los Reyes \textbf{ assana non deuen fazer ninguna cosa mala contra el Rey } nin contra aquellas cosas que son del & ut non prouocent Reges ad iram , \textbf{ non debent fore facere | nec in Regem , } nec in ea quae sunt ipsius , \\\hline
3.2.35 & por que non cayan en sanna del reyes \textbf{ non fazer ninguna cosa } mala contra el Rey & ut non incurrant regiam iram , \textbf{ non forefacere in ipsum Regem . } Regi autem duo debentur , \\\hline
3.2.35 & Mas por razon que a el parte nesçe degniar los otros \textbf{ ael deue ser fecha subiectiuo e obediençia } Por la qual cosa en dos maneras pueden los que son en el regno errar contra el Rey¶ & quia ipsius est dirigere alios , \textbf{ debetur ei subiectio et obedientia . } Quare dupliciter potest forefieri ad Regem \\\hline
3.2.35 & Por la qual cosa en dos maneras pueden los que son en el regno errar contra el Rey¶ \textbf{ La primera si non le fizieren honrra } qual deuen e reuerençia conueinble . & ab iis qui sunt in regno . \textbf{ Primo , si non ei exhibeant honorem debitum , } et reuerentiam dignam . \\\hline
3.2.35 & assi commo deuene . \textbf{ Et razon desto es que aquellos que esto fazen paresçe que nos despreçian . } Ca si non nos despreçiassen & Et est ratio , \textbf{ quia sic se habentes | videntur nos despicere . } Nam si nos non despicerent , \\\hline
3.2.35 & Ca si non nos despreçiassen \textbf{ fazernos } yan aquella honrra que deuien . & Nam si nos non despicerent , \textbf{ impenderent nobis honorem dignum . } Sic etiam ibidem dicitur , \\\hline
3.2.35 & que nos nos enssannassemos contra los menores \textbf{ si ellos non fizieren aquello que deuen fazer } o si fizieren cosas contrarias de aquello que deuen fazer . & quod et ad minores irascimur , \textbf{ si non sic se habeant | ut debent , } vel si faciant contraria eorum quae debent , \\\hline
3.2.35 & si ellos non fizieren aquello que deuen fazer \textbf{ o si fizieren cosas contrarias de aquello que deuen fazer . } la qual cosa contesçe mayormente & ut debent , \textbf{ vel si faciant contraria eorum quae debent , } quod maxime contingit , \\\hline
3.2.35 & non deuen mouer el Rey a saña errando contra el \textbf{ e non le faziendo } obediençia qual deuen e honrra conuenble . & forefaciendo in ipsum , \textbf{ non exhibere ei debitum honorem } et obedientiam condignam . \\\hline
3.2.35 & quasieren mouer al Rey a saña \textbf{ non solamente de non fazer ningun tuerto contra el rey en su perssona . } Mas avn de non fazer contra sus parientes & ad iram prouocare , \textbf{ non solum non forefacere in ipsum Regem , } sed etiam non forefacere in cognatos , \\\hline
3.2.35 & non solamente de non fazer ningun tuerto contra el rey en su perssona . \textbf{ Mas avn de non fazer contra sus parientes } nin contra su muger & non solum non forefacere in ipsum Regem , \textbf{ sed etiam non forefacere in cognatos , } uxorem , filios , \\\hline
3.2.35 & son subiectos al Rey \textbf{ quando alguno fiziere tuerto } contra alguno de los que son en el regno faze tuerto al rey . & Et quia omnes qui sunt in regno ipsi regi sunt subditi , \textbf{ forefacit in regem } quotiescunque alicui existenti \\\hline
3.2.35 & quando alguno fiziere tuerto \textbf{ contra alguno de los que son en el regno faze tuerto al rey . } Et esto es lo que dize el philosofo & Et quia omnes qui sunt in regno ipsi regi sunt subditi , \textbf{ forefacit in regem } quotiescunque alicui existenti \\\hline
3.2.35 & contra aquellos \textbf{ que fazen tuerto anros parientes } e a nuestros fijos & quotiescunque alicui existenti \textbf{ in regno efficitur iniuria aliqua . Hoc est ergo quod dicitur 2 Rhet’ cap’ de ira , } quod irascimur forefacientibus \\\hline
3.2.35 & e obedescerle \textbf{ e non fazer tuerto contra los sus parientes } nin contra sus fijos & obedire ei : \textbf{ non forefacere in cognatos eius , } nec in filios , \\\hline
3.2.36 & que amamos a aquellos \textbf{ que nos pueden fazer bien } saluandonos e librado nos . & quod quia diligimus beneficos in salutem , \textbf{ id est eos qui possunt nobis benefacere nos saluando et liberando , } ideo diligimus fortes et cordatos . Tertio , \\\hline
3.2.36 & es por las penas dela iustiçia \textbf{ que fazen en los subditos . } Mas tres cosas son de penssar en la pena que dan los prinçipes . & propter punitiones , \textbf{ quas exercent in subditos . } In punitione autem tria sunt consideranda \\\hline
3.2.36 & Ca nin por fijo nin por amigo \textbf{ nin por otro ninguon non deue dexar de fazer iustiçia } e de obrar derechureramente e bien & nec pro amico , \textbf{ nec pro aliquo alio } dimittendum est operari iuste et bene . \\\hline
3.2.36 & nin a ninguons otros \textbf{ quando veen que mal fazen . } Et pues que assi es cada vno del pueblo teme de mal fazer cuydando & nec aliis parcunt , \textbf{ si viderint eos forefacere . } Timet igitur tunc quilibet ex populo forefacere , \\\hline
3.2.36 & quando veen que mal fazen . \textbf{ Et pues que assi es cada vno del pueblo teme de mal fazer cuydando } que non podra escapar dela pena . & si viderint eos forefacere . \textbf{ Timet igitur tunc quilibet ex populo forefacere , } cogitans se non posse punitionem effugere . \\\hline
3.2.36 & e mas cruelmente se ayan contra los amigos \textbf{ quando mal fizieren } que contra los otros & et seuerius se gerere contra amicos , \textbf{ si contingat eos valde forefacere , } quam contra alios . \\\hline
3.2.36 & e los sus mjnos ascondidamente e cautelosamente en dar las penas \textbf{ e en fazer la iustiçia } que ninguno malo non pueda foyr de sus manos & in punitionibus exequendis , \textbf{ et in iustitia facienda : } quod mali effugere non possunt , \\\hline
3.2.36 & e non duzir alos otros a uirtud . \textbf{ Et pues que assi es todo bien fazer } por el qual los çibdadanos son mas bueons & inducere alios ad virtutem . \textbf{ Omne ergo bonum } per quod ciues sunt magis boni et virtuosi , \\\hline
3.2.36 & e los que son en el regno \textbf{ si fazen bien } e guardan las leyes & Cum ergo ciues et existentes in regno \textbf{ si bene agant , } et obseruent leges , \\\hline
3.2.36 & e al Rey son mas uirtuosos \textbf{ e mas bueon ssi esto assi fazen que si lo fiziessen } por temor de pena & sint magis boni et virtuosi , \textbf{ quam si hoc facerent timore poenae , } et ne punirentur ; \\\hline
3.2.36 & e los prinçipes de ser amados de los pueblos \textbf{ e que los pueblos fagan bien } por amor de los prinçipes & et quod amore boni , \textbf{ populi bene agant , } quam timeri ab eis , \\\hline
3.2.36 & que por temor dellos \textbf{ nin que por temor de pena se escusen de fazer malas obras . } Et pues que assi es estas dos cosas ser temidos & quam timeri ab eis , \textbf{ et quod timore poenae cauere sibi ab actibus malis . } Utrumque enim est necessarium , \\\hline
3.2.36 & cuya entençiones de tener mientes al bien comun \textbf{ que por ende queden los omes de mal fazer . } Por la qual cosa conuiene & cuius est intendere commune bonum , \textbf{ quiescant male agere : } oportuit ergo aliquos inducere ad bonum , \\\hline
3.3.1 & que partenesçe al Rey o al prinçipe \textbf{ a quien pertenesçe de fazer leyes } e de gouernar el regno e la çibdat . & idest prudentia quae requiritur in Rege et principante , \textbf{ cuius est leges ferre , } et regere regnum et ciuitatem , \\\hline
3.3.1 & el assessiego de los çibdadanos e el bien comun . \textbf{ Et esto deue fazer los caualleros } segunt el mandamieto del Reyo & ø \\\hline
3.3.2 & Mas quanto es de la manera de la arte \textbf{ e en quanto el arte faze al omne apareiado o desapareiado } a la obra de la batalla & Sed quantum est ex genere artis , \textbf{ et prout ars ipsa reddit hominem aptum ad opus bellicum , } aliqua genera artium diximus utilia ad actiones bellicas , \\\hline
3.3.3 & e nos delectamos en ellas . \textbf{ Et si quisiere el ponedor de la ley fazer los çibdadanos buenos lidiadores } e fazer los apareiados para la batalla & nimis diligimus et delectamur in illis : \textbf{ si vult legislator ciues bellatores facere , } et reddere ipsos aptos ad pugnandum , \\\hline
3.3.3 & Et si quisiere el ponedor de la ley fazer los çibdadanos buenos lidiadores \textbf{ e fazer los apareiados para la batalla } deue ante tomar el tienpo de la mançebia & si vult legislator ciues bellatores facere , \textbf{ et reddere ipsos aptos ad pugnandum , | potius debet } praeuenire tempus quam praetermittere . \\\hline
3.3.3 & Mas las señales \textbf{ que nos fazen semeiantes a las animalias lidiadoras } son grandeza de los mienbros e anchura de los pechos . & Signa vero conformantia \textbf{ nos animalibus bellicosis , } sunt magnitudo extremitatum , \\\hline
3.3.4 & en la batalla de coner poco \textbf{ e de fazer abstineçia } e non se finchir de mucha uianda & immo et si adesset pugnantibus ciborum ubertas , \textbf{ adhuc esset eis necessaria abstinentia , } et non grauari ex nimio cibo , \\\hline
3.3.4 & aborresçra esparzer la sangre \textbf{ e non osara fazer llagas a los enemigos . } Et assi se sigue & horreat effundere sanguinem ; \textbf{ non audebit hostibus plagas infligere , } et per consequens bene bellare non potest . \\\hline
3.3.4 & Mas entre todas las cosas \textbf{ que fazen al omne buen lidiador } es dessear & Inter caetera autem \textbf{ quae reddunt hominem bellicosum , } est diligere honorari expugna , \\\hline
3.3.4 & e todas las otras cosas dichas \textbf{ que son de fazer } por las quales la iustiçia & et caetera alia sunt fienda , \textbf{ per quae iustitia et commune bonum defendi potest . } Ex his autem plane patet , \\\hline
3.3.5 & que los que son delicadamente criados . \textbf{ Et a esto fazen avn aquellas cosas } que dichas son de suso . & aptiorem armis esse rusticam plebem . \textbf{ Ad hoc etiam videntur facere } quae superius diximus . \\\hline
3.3.5 & que dize el philosofo en el tercero libro de las . \textbf{ Lo que fizo a ector atreuido en las armas e buen lidiador . } Ca dizia ector & ( ut ait Philosophus 3 Ethic’ ) \textbf{ quod Hectorem fecit audacem . } Dicebat enim Hector , \\\hline
3.3.5 & que polimidias otro cauallero le denostaria muy mal . Avn en essa misma manera \textbf{ diomedes fue fecho atreuido cauallero . } Ca dizie que si boluiesse las espaldas en la batalla & Polydamas mihi redargutiones ponet . \textbf{ Sic etiam et Diomedes | hoc modo effectus fuit strenuus , } quia dicebat , \\\hline
3.3.5 & Enpero conuiene de saber \textbf{ que para que los nobles de toda parte sean fechos estremados } lidiadores deuen se acostunbrar a sofrir el peso de las armas & Sciendum tamen quod \textbf{ ut nobiles ex omni parte efficiantur strenui bellatores , } assuefaciendi sunt \\\hline
3.3.6 & e en cada fecho de grant osadia \textbf{ por que non ayan miedo los omnes de fazer aquello que han usado } por que segunt que dize Uegeçio & Exercitium enim in quolibet negocio praebet audaciam , \textbf{ ut non metuat illud facere . } Nam quia secundum Vegetium : \\\hline
3.3.6 & por que segunt que dize Uegeçio \textbf{ ninguno non tenie de fazer } aquello que ha bien aprendido e en que tieue fiuza & Nam quia secundum Vegetium : \textbf{ Nemo facere metuit , } quod se bene didicisse confidit . \\\hline
3.3.6 & e que non sea enbargada para ferir . \textbf{ La qual cosa non se puede fazer } non guardando grado conuenible & et non impediatur ad percutiendum . \textbf{ Quod , nisi seruato debito gradu , } et debito ordine in incessu , \\\hline
3.3.6 & Lo segundo para espantar los enemigos . \textbf{ Lo tercero para fazer mayores llagas . } Ca contesçe algunas vezes de fallar algunas carcauas e arroyos e açequias & Secundo ad terrendum aduersarios . \textbf{ Tertio ad infligendum maiores plagas . } Contingit enim aliquando inuenire fossas \\\hline
3.3.6 & por razon del mouimiento vale \textbf{ para fazer mayores llagas . } e et podemos sin aquellas tres cosas & Rursus , ipse saltus ratione motus facit \textbf{ ut plaga amplior infligatur . } Possumus autem praeter tria praedicta , \\\hline
3.3.7 & por el arte del esgrima \textbf{ mas desto faremos espeçial capitulo . } lo primero dezimos que son de vsar los lidiadores & ad percutiendum cum gladiis et ensibus . \textbf{ Sed de hoc speciale capitulum faciemus . } Primo enim sunt bellatores exercitandi \\\hline
3.3.7 & Et los moços que querian \textbf{ acostunbrara fazer los buenos lidiadores . } vsauan los a ferir en aquellos palos & et iuuenes quos volebant \textbf{ facere optimos bellatores exercitabant ad palos illos ita , } ut quilibet haberet scutum dupli ponderis \\\hline
3.3.7 & e assi se yua cobriendo del escudo \textbf{ e faziendo todas las otras cosas } que son menester & et contra palum illum sic impetuose \textbf{ se gerebat percutiendo ipsum , et alia faciendo } quae requiruntur ad bellum , \\\hline
3.3.7 & e a ferir con lanças \textbf{ la qual cosa avn fazien les mançebos al palo fincado . } ca assi fazien antiguamente & et ad percutiendum cum lancea : \textbf{ quod etiam ad defixum palum fieri habet . } Fiebat enim antiquitus \\\hline
3.3.7 & la qual cosa avn fazien les mançebos al palo fincado . \textbf{ ca assi fazien antiguamente } ca quando los mançebos eran usados a ferir en los palos fincados con las maças & quod etiam ad defixum palum fieri habet . \textbf{ Fiebat enim antiquitus } ut cum iuuenes exercitati erant \\\hline
3.3.7 & por el mayor mouimineto \textbf{ que el ome faze en el ayre } mas aluene fiere & propter maiorem motum \textbf{ quem efficit in aere , longius pergit } et amplius vulnus infligit . \\\hline
3.3.7 & mas aluene fiere \textbf{ e mayor colpe faze . } Lo quarto son de vsarlos lidiadores & quem efficit in aere , longius pergit \textbf{ et amplius vulnus infligit . } Quarto exercitandi sunt bellantes \\\hline
3.3.7 & los lidiadores son de usar a sobir ligeramente en los cauallos . \textbf{ Ca assi commo cuenta vegeçio antiguamente fazien cauallos de madera . } Et los mançebos vsauan en el yuierno a sobir & ad ascensiones equorum . \textbf{ Nam , ut Vegetius recitat , | fiebant antiquitus equi lignei , } ad quos ascendendos iuuenes , \\\hline
3.3.8 & nin sabiamente luego mueren o fuyen . \textbf{ En tal manera que fuyendo fazense temerosos en manera que non pueden ser vençedores de sus enemigos } e apenas o nunca osan acometer batalla contra ellos . & vel in fugam versi \textbf{ adeo efficiuntur timidi , } quod contra suos victores vix aut nunquam audent bella committere . \\\hline
3.3.8 & o quiere \textbf{ y fazer mayor tardança } si aquel logar en algun caso & alicubi vult pernoctare , \textbf{ vel ulteriorem moram contrahere , } si ad locum illum in aliquo casu , \\\hline
3.3.8 & o a auentura puedan a desora los enemigos venir \textbf{ luego que llegan al logar deuen fazer enderredor carcauas } e deuen leuantar algunas guarniçiones & hostes superuenire possunt , \textbf{ statim circa exercitum fiendae sunt fossae , } erigendae munitiones aliquae \\\hline
3.3.8 & que lieuan consigo assi commo vna çibdat guarnida . \textbf{ Visto commo es cosa prouechable a la hueste fazer carcauas e costruir guarniçiones e castiellos . } finca de demostrar en qual manera las tales guarniciones & Viso utile esse \textbf{ circa exercitum facere fossas | et construere castra : } restat ostendere , \\\hline
3.3.8 & finca de demostrar en qual manera las tales guarniciones \textbf{ et los tales castiellos se deuen fazer . } Ca si los enemigos non estudieren cerca de ligero pueden fazer carcauas çerca de la hueste & quomodo huiusmodi monitiones \textbf{ et castra sunt construenda . } Nam si hostes sunt absentes \\\hline
3.3.8 & et los tales castiellos se deuen fazer . \textbf{ Ca si los enemigos non estudieren cerca de ligero pueden fazer carcauas çerca de la hueste } e leuantar & et castra sunt construenda . \textbf{ Nam si hostes sunt absentes | facile est fossas circa exercitum fodere , } munitiones erigere et castra construere . \\\hline
3.3.8 & e leuantar \textbf{ guarnicoñes e fazer castiellos . } Mas si los enemigos fueren cerca & facile est fossas circa exercitum fodere , \textbf{ munitiones erigere et castra construere . } Sed si aduersarii praesentes adsint , \\\hline
3.3.8 & graue cosa es de guarnescer la hueste \textbf{ e de fazer castiellos . } Ca en tal caso commo este dos cosas son menester . & Sed si aduersarii praesentes adsint , \textbf{ difficilius est castra munire . } Sunt enim in tali casu duo necessaria , \\\hline
3.3.8 & Lo primero estar e lidiar contra los enemigos \textbf{ Lo segundo fazer los castiellos . } Pues que assi es en tal auenemiento conmo este & videlicet hostibus resistere , \textbf{ et castra construere . } In tali ergo euentu \\\hline
3.3.8 & e alguna partida de peones \textbf{ deuen ser ordenados en vna az } para refrenar e tornar a çaga el arrebato de los enemigos . & et una pars peditum debet \textbf{ ordinari in acie } ad pellendum impetum hostium : \\\hline
3.3.8 & que puede abastar \textbf{ para fazer ligeramente los castiellos } deue los costruyr & quae possit sufficere \textbf{ ad celerem constructionem castrorum , } debet celeriter castra construere . \\\hline
3.3.8 & deue los costruyr \textbf{ e fazer muy apriessa . } Mas conuiene de poner algunos maestros & ad celerem constructionem castrorum , \textbf{ debet celeriter castra construere . } Oportet autem semper construendis castris , \\\hline
3.3.8 & para costruyr los castiellos \textbf{ e fazer las carcauas } que acuçien los negligentes & Oportet autem semper construendis castris , \textbf{ et faciendis fossis aliquos magistros praestitui , } qui negligentes solicitent , \\\hline
3.3.8 & que acuçien los negligentes \textbf{ e manden a cada vno qual cosa deua fazer . } Mostrado que prouechosa cosa es de fazer los castiellos . & qui negligentes solicitent , \textbf{ et unicuique iniungant | quod ipsum oporteat facere . } Ostenso utile esse castra construere , \\\hline
3.3.8 & e manden a cada vno qual cosa deua fazer . \textbf{ Mostrado que prouechosa cosa es de fazer los castiellos . } avn en qual manera los enemigos presentes son de fazer los castiellos & quod ipsum oporteat facere . \textbf{ Ostenso utile esse castra construere , } et qualiter etiam praesentibus hostibus construenda sint castra : \\\hline
3.3.8 & Mostrado que prouechosa cosa es de fazer los castiellos . \textbf{ avn en qual manera los enemigos presentes son de fazer los castiellos } Lo otro que nos finca de declarar & Ostenso utile esse castra construere , \textbf{ et qualiter etiam praesentibus hostibus construenda sint castra : } reliquum est declarare \\\hline
3.3.8 & nin avn sea tomado tan pequeno espaçio por que la hueste este a mayor estrechura que deue . \textbf{ Lo quarto si conueniere que aquella hueste aya de fazer } en aquel logar alguna tardança & oporteat exercitum constringi et constipari . \textbf{ Quarto si oporteat in loco illo exercitum moram contrahere , } et adsit possibilitas est eligenda \\\hline
3.3.8 & en aquel logar alguna tardança \textbf{ e fuere cosa que se puede fazer } deuemos escoger çerca de aqual & Quarto si oporteat in loco illo exercitum moram contrahere , \textbf{ et adsit possibilitas est eligenda } circa situm salubritas aeris . \\\hline
3.3.8 & que las guarniçiones e las carcauas \textbf{ que son de fazer çerca de la hueste } deuen ser quadradas e luengas . & Videtur autem velle Vegetius , \textbf{ munitiones et fossas fiendas circa exercitum debere } habere formam quadrilateram oblongam . \\\hline
3.3.8 & por ende es \textbf{ mas de escoger de fazer las guarniçiones } segunt la figura redonda & Attamen quia figura circularis est capacissima , \textbf{ est elegibilius facere munitiones } secundum circularem formam , \\\hline
3.3.8 & por que si temen mucho del cometemiento de los enemigos \textbf{ coñuiene de fazer carcauas de muchos rencones } por que aquella figura es mas conuenible & quia si multum timeretur de impetu hostium , \textbf{ oporteret foueas facere multorum angulorum , } eo quod illa est magis defensioni apta , \\\hline
3.3.8 & contesçe que el assentamiento non sufre tal figura . \textbf{ Et por ende en tal caso deuen se fazer los castiellos } en figura de medio cerco o quedrados o de tres rencones & illum non pati talem formam . \textbf{ In tali ergo casu construenda sunt castra semicircularia , } triangularia , quadrata , \\\hline
3.3.8 & en qual manera de guarnimiento es de catar \textbf{ en el fazer de los castiellos . } Ca si la hueste mucho ouiere de morar & quis munitionis modus attendendus sit \textbf{ in constructione castrorum . } Nam si exercitus diu \\\hline
3.3.8 & e de escoger mas fuertes guarniçiones \textbf{ e son de fazer mas anchas carcauas . } mas solamente quieren y estar vna noche o por poco tienpo non conuiene de fazer tantas guarniçiones . & eligendae sunt fortiores munitiones , \textbf{ et fiendae ampliores fossae . | Sed si solum ibi pernoctare cupit , } aut ibi debet \\\hline
3.3.8 & e son de fazer mas anchas carcauas . \textbf{ mas solamente quieren y estar vna noche o por poco tienpo non conuiene de fazer tantas guarniçiones . } Mas la manera e la quantidat de las carcauas pone la vegeçio & aut ibi debet \textbf{ per modicum tempus existere , | non oportet tantas munitiones expetere . } Modum autem , \\\hline
3.3.8 & la carcaua deue ser muy ancha de nueue pies e alta de siete . \textbf{ Mas si la fuerça de los enemigos paresciere mas fuerte conuiene de fazer las carcauas mas anchas et mas fondas } si han uagar para las fazer & fossa debet esse lata pedes nouem , alta septem . \textbf{ Sed si aduersariorum vis acrior imminet , | contingit fossam ampliorem } et altiorem facere ita , \\\hline
3.3.8 & Mas si la fuerça de los enemigos paresciere mas fuerte conuiene de fazer las carcauas mas anchas et mas fondas \textbf{ si han uagar para las fazer } assi que sea la carcaua ancha de doze pies e alta de nueue . & contingit fossam ampliorem \textbf{ et altiorem facere ita , } ut sit lata pedes duodecim , \\\hline
3.3.8 & nueue pies echando la tierra a la parte de la hueste \textbf{ fazese la carcaua mas alta de quatro pies } assi que toda la carcaua sera alta de treze pies . & propter terram eiectam \textbf{ supra fossam crescit | quasi pedes quatuor : } ita quod tota fossa alta erit quasi pedes tresdecim : \\\hline
3.3.9 & porque son seys cosas de parte de los enemigos lidiadores \textbf{ que fazen ganar victoria . } Lo primero es el cuento de los lidiadores . & Sunt autem sex ex parte hominum bellatorum , \textbf{ quae faciunt ad obtinendam victoriam . } Primum est , numerus bellantium . \\\hline
3.3.9 & e sin mayor trabaio \textbf{ e pena faze las obras que ha acostunbradas . } Lo terçero deue ser & et cum minori labore \textbf{ et poena faciat opera consueta . } Tertio , attendenda est tolerantia \\\hline
3.3.9 & e si se leuanta algun viento \textbf{ que faga poluo } contra si o contra los enemigos . & et utrum sit aliquis ventus flans \textbf{ et eleuans puluerem contra ipsos , } vel contra aduersarios : \\\hline
3.3.10 & que lieuen los pendones \textbf{ por que cada vno sepa lo que ha de fazer . } por que tan grande es el espanto en la batalla & et ferentes vexilla : \textbf{ ut quilibet sciat | quid debeat agere . } Est enim tantus terror in bello \\\hline
3.3.10 & se sepa tener ordenadamente en su az \textbf{ e sepa que ha de fazer } e desto puede paresçer & se tenere ordinate in acie , \textbf{ et cognoscat quid sit acturus . } Ex hoc autem patere potest , \\\hline
3.3.10 & por que el alferez \textbf{ que leuaua la seña fizo falsedat } encubriendo la seña & cuiusdam deuictum esse a bellatoribus paucis , \textbf{ eo quod vexillifer fraudem committens } velauit vexillum \\\hline
3.3.10 & Por que en la batalla de los caualleros \textbf{ se faze mayor pelea } que en la batalla de los peones . & qui est equitibus praeponendus : \textbf{ quia in bello equestri maior conflictus efficitur , } quam in pedestri pugna . \\\hline
3.3.10 & por que lidien fuertemente e quel alinpien las armas \textbf{ e sepan fazer todas las otras cosas } que son menester para la batalla . & ut fortiter pugnent , arma tergant , \textbf{ et alia faciant } quae requiruntur ad bellum . \\\hline
3.3.11 & Mas seguramente podria guiar su hueste \textbf{ por que assi lo fazen los marineros . } Ca veyendo los periglos de la mar . & tutius posset suum exercitum ducere . \textbf{ Sic etiam marinarii faciunt , } qui videntes maris pericula , \\\hline
3.3.11 & por el qual es sabida en pintura o en su semeiança . \textbf{ Enpero por que los guiadores non puedan fazer algunos engaños } deue el señor de la hueste poner en ellos buenas guardas & vel in alio simili . \textbf{ Ne tamen conductores moliantur fraudes aliquas , } debet circa eos dux belli bonas \\\hline
3.3.11 & e vsados en las batallas \textbf{ de cuyo conseio faga todas aquellas cosas } que ouiere de fazer . & habere secum plures sapientes fideles principi , exercitatos in bellis , \textbf{ de quorum consilio agat } quicquid viderit ipse dux belli esse fiendum . \\\hline
3.3.11 & de cuyo conseio faga todas aquellas cosas \textbf{ que ouiere de fazer . } Ca do puede contesçer tan grant periglo & de quorum consilio agat \textbf{ quicquid viderit ipse dux belli esse fiendum . } Nam ubi tantum currit periculum , \\\hline
3.3.11 & que deue el señor de la hueste en cada conpaña \textbf{ e en cada vna az auer vnos caualleros muy fieles } e muy estremados & et in qualibet acie \textbf{ habere aliquos equites fidelissimos et strenuissimos , } habentes equos veloces et fortes ; \\\hline
3.3.11 & en alguna parte \textbf{ no pueden fazer daño en la hueste . } Ca maguera quel conseio del cabdiello non sea sabido a ninguno . & ne hostes aliqui latitantes \textbf{ ex aliqua parte molestent exercitum . } Nam etsi nullis esset notum ducis consilium , \\\hline
3.3.11 & que si los enemigos fuessen pressentes \textbf{ non les pudiessen fazer daño ninguno . } Et por ende dize e prouerbio & ut si et tunc hostes praesentes adessent , \textbf{ ei non possent efficere nocumentum . } Unde et prouerbialiter dicitur , \\\hline
3.3.12 & e que los caualleros e los peones guarden su az \textbf{ non se puede fazer } sin grant vso de las armas . & ut equites et pedites suam aciem seruent , \textbf{ non sine magno exercitio fieri potest . } Qui igitur in tempore aliquo vult bellare , \\\hline
3.3.12 & a guardar orden conuenible en la az \textbf{ e a fazer aquellas cosas } que son menester en la batalla . & ad seruandum debitum ordinem , \textbf{ et ad faciendum ea quae requiruntur in bello . Modus autem , } per quem pugnatores huiusmodi ordinem seruare discunt , \\\hline
3.3.12 & segunt aquella distançia \textbf{ que demanda el az de los caualleros o de los peones . } Et despues desto deue mandar & secundum distantiam \textbf{ quam requirit acies equestris vel pedestris . } Postea praecipere debet \\\hline
3.3.12 & Et despues desto deue mandar \textbf{ que se doble el az } assi que la meatad de la az & Postea praecipere debet \textbf{ ut duplicent aciem } ita quod medietas aciei statim separet se a medietate alia , \\\hline
3.3.12 & o enpos della . \textbf{ Et esto fecho luego deue el cabdiello de la batalla mandar } que fagan az quadrada . & vel post ipsam . \textbf{ Quo facto statim debet praecipere dux belli , } ut aciem quadratam faciant , \\\hline
3.3.12 & Et esto fecho luego deue el cabdiello de la batalla mandar \textbf{ que fagan az quadrada . } e desende que establezcan vn triangulo & Quo facto statim debet praecipere dux belli , \textbf{ ut aciem quadratam faciant , } et deinde , ut constituant trigonum : \\\hline
3.3.12 & que quiere dezir forma de tres linnas \textbf{ e esto se faz ligeramente . } Ca despues que el az esta quadrada & et deinde , ut constituant trigonum : \textbf{ quod faciliter fit . } Nam acie quadruplicata \\\hline
3.3.12 & e va por medio fasta el otro canto . \textbf{ Et las partes quadradras ayuntadas en vno fazen vn triangulo } que es figura de tres liñas . & et secata diametro , \textbf{ et partibus quadratis coniunctis | simul faciunt trigonum . } Vel , ut sit ad unum dicere , \\\hline
3.3.12 & que entre todas las otras formas de la az la quadrada es mas sin prouecho . \textbf{ Et por ende nunca es de formar el az } sinplemente & inter caeteras formas esse magis inutilem : \textbf{ ideo secundum hanc formam nunquam formanda est acies simpliciter , } sed in casu : \\\hline
3.3.12 & assi que el az non pueda ser ronpida de los enemigos . \textbf{ Et çerca del az ençima } e departe de fuera contra los enemigos & ab hostibus transcindi . \textbf{ Circa aciem autem in summitate , } et in exteriori parte constituendi sunt homines grauioris armaturae \\\hline
3.3.12 & sin el cuento de los lidiadores \textbf{ que fazen el az son de guardar algunos buenos } et fuertes lidiadores & praeter numerum pugnatorum constituentium aciem , \textbf{ reseruandi sunt aliqui strenui bellatores extra ipsam aciem } qui possint ad illam partem succurrere \\\hline
3.3.12 & en el ordenamiento de las azes . \textbf{ Lo primero que el az sea bien ordenada } segunt forma redondao aguda o segunt forma de tigeras & in constitutione acierum . \textbf{ Primo , ut acies bene ordinetur } secundum formam acutam , rotundam , et forficularem : \\\hline
3.3.12 & que vieren \textbf{ que mas faz meester } e mas ayna puede fallesçer . & qui possint succurrere ad partem illam , \textbf{ erga quam viderint } aciem titubare , et deficere . \\\hline
3.3.13 & ca si alguno avn que estudiesse desarmado en la ferida \textbf{ que se faze cortando fuesse ferido } ante que el colpe veniesse al coraçon o a los mienbros de vida & Nam et si quis quasi inermis existeret , \textbf{ in percussione caesim priusquam perueniretur } ad cor vel ad membra vitalia , \\\hline
3.3.13 & ante que el colpe veniesse al coraçon o a los mienbros de vida \textbf{ conuernie de fazer muy grant llaga } e de cortar muchos huessos . & ad cor vel ad membra vitalia , \textbf{ oporteret magnam plagam facere } et multa ossa incidere : \\\hline
3.3.13 & ca dos onças de sangre abastan \textbf{ para que se fagan llaga mortal } mas deuemos penssar & duae unciae sufficiunt ad hoc \textbf{ ut fiat plaga mortalis , | et sit lethale vulnus . } Considerare quidem debemus , \\\hline
3.3.13 & por que feriendo \textbf{ assi mas ayna se faze llaga mortal . } la terçera razon se toma de la & percutiendum est punctim , \textbf{ quia sic feriendo citius infligitur plaga mortifera . } Tertia via sumitur \\\hline
3.3.13 & por que los dardos que son ante vistos menos daño \textbf{ fazen que los que non son vistos . } Mas en feriendo cortando . & et citius potest illa vitare : \textbf{ quia iacula praeuisa minus laedunt . } In percutiendo autem caesim , \\\hline
3.3.13 & Mas en feriendo cortando . \textbf{ por que conuiene de fazer grand mouimiento de los braços } ante que se de el colpe el enemigo & In percutiendo autem caesim , \textbf{ quia oportet fieri magnum brachiorum motum prius quam infligatur plaga , } aduersarius ex longinquo potest prouidere vulnus , \\\hline
3.3.13 & lueñe se puede guardar \textbf{ que nol faga llaga . } Et por ende puede se mas guardar & quia oportet fieri magnum brachiorum motum prius quam infligatur plaga , \textbf{ aduersarius ex longinquo potest prouidere vulnus , } ideo magis sibi cauere potest \\\hline
3.3.13 & que venga tarde mata . \textbf{ Mas la ferida de punta fecha con muy pequeña } fuerca faze llaga mortal . & raro occidit . \textbf{ Sed puncta modico impetu inflicta , } facit lethale vulnus . \\\hline
3.3.13 & Mas la ferida de punta fecha con muy pequeña \textbf{ fuerca faze llaga mortal . } La quinta razon se toma del descrubimiento del que fiere . & Sed puncta modico impetu inflicta , \textbf{ facit lethale vulnus . } Quinta via sumitur \\\hline
3.3.13 & por quel pueda mas ligeramente ferir . \textbf{ Ca mas ligeramente faze daño } e enpeesçimiento en el cuerpo desnuyo que en el cubierto . & ut possit nos laedere . \textbf{ Nam leuius infertur laesio et nocumentum corpore nudato , quam tecto . } Ut dicebatur superius , \\\hline
3.3.14 & Et pues que assi es todas aquellas cosas \textbf{ que fazen los enemigos ser fuertes } para lidiar con sus enemigos & et econuerso . \textbf{ Quaecunque igitur reddunt hostes fortiores } ad resistendum bellantibus , \\\hline
3.3.14 & las contrarias les son desprouechosas \textbf{ e los fazen ser mas flacos } por que non puedan lidiar contra sus enemigos . & eorum opposita sunt eis nociua , \textbf{ et reddunt eos debiliores } ne possint impugnantibus resistere . \\\hline
3.3.14 & mas fuertes seran de vençerLo . \textbf{ segundo que faze los enemigos mas fuertes } para lidiar es el logar do se assientan . & difficilius euincuntur . \textbf{ Secundum quod reddit | hostes fortiores } ad resistendum , \\\hline
3.3.14 & e tomados los enemigos \textbf{ faze los ser mas flacos para lidiar . } Assi el logar conuenible & reddit eos debiliores ad bellandum : \textbf{ sic locus aptus facit eos potentiores ad resistendum . } Tertium , est ipsum tempus . \\\hline
3.3.14 & Assi el logar conuenible \textbf{ e bueno fazelos mas fuertes para se defender . } Lo terçero es el tienpo can en el tienpo en que el viento es mas contra los enemigos & Tertium , est ipsum tempus . \textbf{ Nam tempore in quo ventus est contra hostes , } et in quo puluis facies eorum percutit , \\\hline
3.3.14 & e el sol non les es contrario son mas apareiados para lidiar . \textbf{ Lo quarto que fazen los enemigos mas esforçados e mas aperaiados } e para lidiar es prouision . & hostes habiliores sunt ad pugnandum . \textbf{ Quartum quod reddit hostes | magis animosos } et magis promptos ad renitendum , \\\hline
3.3.14 & Et pues que assi es la folgura conueible \textbf{ faze los omnes mas poderosos . } Lo vj° es amistança e concordia entre ralmente mas de ligero se pueden vençer & Sextum quidem , est amor et concordia ipsorum . \textbf{ Nam si hostes diuisi corporaliter } facilius deuincuntur , \\\hline
3.3.14 & Por la qual cosa si la vnidat del logar \textbf{ e el ayuntamiento de los lidiadores los faze mas poderosos . } mucho mas el amor & Quare si unitas loci \textbf{ et congregatio bellantium | eos potentiores facit , } amor et unitas cordium eos viriliores reddit . \\\hline
3.3.14 & e la vnidat de los coraçones \textbf{ los faze meiores e mas prouechosos para se ayudar Lo vij . } que faze los enemigos mas poderosos & ø \\\hline
3.3.14 & los faze meiores e mas prouechosos para se ayudar Lo vij . \textbf{ que faze los enemigos mas poderosos } para se defender & amor et unitas cordium eos viriliores reddit . \textbf{ Septimum , quod facit hostes potentiores ad renitendum , } est latentia propriarum conditionum existentium \\\hline
3.3.14 & Et por ende contadas aquellas cosas \textbf{ que fazen los enemigos mas poderosos } para se defender de ligero puede paresçer commo & Enumeratis itaque \textbf{ quae reddunt hostes potentiores ad renitendum , } de facili patere potest , \\\hline
3.3.14 & Lo quarto el señor de la hueste se deue tenprar \textbf{ assi que en tal ora faga tomar la vianda a los caualleros } e folgar e dar çeuada a los cauallos & Quarto dux exercitus sic se temperare debet : \textbf{ ut tali hora faciat | suos commilitones cibum capere , } et requiescere : \\\hline
3.3.14 & Lo quinto deue el cabdiello escodriñar con grant acuçia \textbf{ quando los enemigos fizieren grant iornada } e touieren los cauallos canssados . & Quinto debet diligenter explorare , \textbf{ quando hostes magnam fecerunt dietam , } sunt fatigati habent laxatos equos : \\\hline
3.3.14 & Ca estonçe si los quisieren acometer \textbf{ de ligero les faran } boluer las espaldas a foyr . & si eos inuadere poterint , \textbf{ de facili terga vertent . } Sexto ( secundum Vegetium ) debet dux belli \\\hline
3.3.14 & assi que non fien dessi mismos . \textbf{ Et esto fecho si los acometieren non auiendo fiuza enssi mismos de ligero se a foyr . } Mas esta cautela commo quier que la ponga vegeçio & si eos inuadat , \textbf{ non habentes fiduciam de se inuicem , | de facili conuertentur in fugam . } Sed haec cautela licet \\\hline
3.3.15 & el qual esgrimido mueue el ayre mas reziamente \textbf{ e faz mas fuerte colpe . } Enpero maguer que podamos folgar tan bien sobre la parte derecha & quo vibrato vehementius mouet aerem , \textbf{ et fortius ferit . } Licet enim \\\hline
3.3.15 & e quando quieren ferir \textbf{ deuen se fazer adelante con el pie derecho } e quando dieren los colpes & et cum volunt percutere , \textbf{ cum pede dextro debent se antefacere , } et cum volunt ictus fugere , \\\hline
3.3.15 & Ca estonçe con desesperamiento \textbf{ assi commo costreñidos por fuerça fazen se mas osados } veyendo que non les finca si non la muerte . & quod non pateat aliquis aditus fugiendi : \textbf{ quia desperantes quasi necessitate compulsi efficiuntur audaces , } videntes enim se necessario moriendos , \\\hline
3.3.15 & e defender sse \textbf{ non les pudiessen fazer ningun enpeesçimiento . } Pues que assi es mostrado & quod quantumcunque debellare vellent \textbf{ nullum possent nocumentum efficere . } Ostenso itaque qualiter debeant stare pugnantes , \\\hline
3.3.15 & que esto non lo sepan los enemigos . \textbf{ Et por ende muchos de noche fazen esto mas que de dia . } e muchos ouieron esta cautela & ut hoc hostes lateat . \textbf{ Ideo multi tempore nocturno | potius quam diuino hoc agunt : } et plures hanc habuere cautelam , \\\hline
3.3.15 & que quando se assi escusa la batalla \textbf{ nunca la az se deue departir . } ca podrie contesçer & quod quando sic declinatur pugna , \textbf{ nunquam acies se debent diuidere : } quia si contingeret \\\hline
3.3.15 & si fuere menester \textbf{ que los sus contrarios los ayan a fazer foyr . } p paresçe que todas e las batallas se puenden adozir a quatro maneras . & ad quem posset confugere exercitus , \textbf{ si fugaretur ab hostibus . } Videntur omnia bella \\\hline
3.3.16 & Et batalla de naues \textbf{ Mas batalla canpal es dicha toda lid fecha en la tierra } segunt la qual los lidiadores lidian vnos & videlicet ad campestre , obsessiuum , defensiuum , et nauale . \textbf{ Bellum autem campestre dicitur omnis pugna facta in terra , } secundum quam bellantes ad inuicem pugnant \\\hline
3.3.16 & Et estas batallas \textbf{ que se fazen en las aguas } de qual se quier condiçion & quam terrestres . \textbf{ Huiusmodi autem pugna in aquis facta } cuiuscunque conditionis aquae illae existant , \\\hline
3.3.16 & algunas vezes de ordenar \textbf{ e de fazer batallas nauales e de naues . } Et pues que assi es dicho de la lid canpal & et ne terrae marinae impugnentur , \textbf{ expedit regibus et principibus aliquando ordinare bella naualia . } Dicto itaque de bello campestri , \\\hline
3.3.16 & es de dezir de la lid \textbf{ que se faz por çercar } por que por tal lid & et dicto quod post castrum campestre primo dicendum est \textbf{ de pugna obsessiua : } cum per huiusmodi pugnam \\\hline
3.3.16 & non los matan \textbf{ Mas por deçepamiento de los mienbros fazenlos sin prouecho } e despues enbianlos a las fortalezas cercadas & non occidunt illos \textbf{ sed per mutilationem membrorum | eos reddunt inutiles , } et postea illos remittunt \\\hline
3.3.16 & por que y con los otros comedores cercados \textbf{ fagan mayor fanbre } e mayor desfallesçimiento en viandas . & apud ipsos obsessos \textbf{ maiorem famem et inopiam inducant . } Tertius modus obtinendi munitiones est per pugnam : \\\hline
3.3.16 & Ca si por sed son de ganar las fortalezas \textbf{ meior es de fazer la çerca } en el tienpo del estiuo & Nam si per sitim sunt munitiones obtinendae , \textbf{ melius est facere obsessionem tempore aestiuo , } eo quod tunc magis desiccantur aquae , \\\hline
3.3.16 & e estan en las casas \textbf{ Pues que assi es o las cercas son de fazer } en el tienpo del estiuo o łi por muchos tienpos han de durar las cercas & quam obsessos manentes in domibus . \textbf{ Vel igitur obsessiones fiendae sunt tempore aestiuo , } vel si per multa tempora obsessiones durare debent , \\\hline
3.3.17 & quanto podrie lançar la vallesta o el dardo \textbf{ e fazer carcauas enderredor de ssi } e finçar y grandes palos & vel iaculi debent castrametari , \textbf{ et circa se facere fossas , } et figere ibi ligna , \\\hline
3.3.17 & e finçar y grandes palos \textbf{ e fazer algunas fortalezas } assi que si los que estan çercados & et figere ibi ligna , \textbf{ et construere propugnacula : } ut si oppidani eos repente vellent inuadere , resistentiam inuenirent . \\\hline
3.3.17 & Ca cauando ally \textbf{ e faziendo cauas soterrañas } assi commo cauan los que buscan la plata & ubi incipiant fodere : \textbf{ ibi enim faciendo vias subterraneas } sicut faciunt fodientes argentum \\\hline
3.3.17 & por aquellas carreras soterrañas \textbf{ faziendolas toda via mayores } e mas anchas e mas fondas que las carcauas de la çibdat o de la fortaleza & debent per vias illas , \textbf{ faciendo eas profundiores , } quam sint fossae munitionis deuincendae , \\\hline
3.3.17 & fasta los muros de aquel logar . \textbf{ Et si esto se puede fazer } ligera cosa es de tomar aquella fortaleza o aquel logar & pergere usque ad muros munitionis praedictae : \textbf{ quod si hoc fieri potest , } leue est munitionem capere . \\\hline
3.3.17 & por que non puedan luego caer \textbf{ nin fazer daño a los que cauan } e quando todos los muros o grant parte dellos & et supponere ibi ligna \textbf{ ne statim cadant . } Et cum omnes muros , \\\hline
3.3.17 & que sotienen los muros \textbf{ e fazer que todos los muros o grand parte dellos cayan en vno a desora . } Otrossi deuen fenchir las carcauas & statim debent apponere ignem in lignis sustinentibus muros \textbf{ et facere omnes muros | vel facere magnam eorum partem cadere , } et replere fossas : \\\hline
3.3.17 & por que non los maten los muros \textbf{ quando cayeren pues que assi es assi auemos de fazer en este conbatemiento } que es por cueuas conegeras & ne laedantur per murorum casum . \textbf{ Sic ergo agendum est } in impugnatione per cuniculos , \\\hline
3.3.17 & assi que por ellas puedan entrar a la çibdat o al castiello . \textbf{ Et estas cosas todas deuense fazer muy encubiertamente } por que non lo sepan & haberi ingressus ad ciuitatem et castrum : \textbf{ quae omnia latenter fieri possunt } absque eo quod sentiantur ab obsessis : \\\hline
3.3.17 & nin lo sientan los cercados . \textbf{ Et maguera que todas estas cosas non se puedan fazer } sin muy grant graueza & absque eo quod sentiantur ab obsessis : \textbf{ licet tamen sine difficultate } et diuturnitate temporis , \\\hline
3.3.17 & ayan entrada al castiello o a la çibdat \textbf{ e por la entrada que se faze } por do caen los muros & et munitiones suffossae : \textbf{ et per vias subterraneas fiat ingressus ad castrum , vel ad ciuitatem : } et per aditum factum ex muris cadentibus reliqui obsidentes ingrediantur castrum , \\\hline
3.3.18 & nin por engaño \textbf{ nin por encubierta nin assecha . Et esto commo se ha de fazer mostrar lo hemos en los capitulos } que se siguen & Quod quomodo fieri habeat , \textbf{ ostendemus in sequentibus capitulis , } ubi agetur de defensiua pugna . \\\hline
3.3.18 & Mas este tal leuantamento del pertegal \textbf{ algunas vegadas se faze por contrapeso } e algunas vegadas non abasta el contrapeso . & Huiusmodi autem eleuatio virgae \textbf{ aliquando fit per contra pondus , } aliquando autem non sufficit contrapondus , \\\hline
3.3.18 & arroian las piedras . \textbf{ Et pues que assi es si por el contrapeso solamente se faze tal arroiamiento de piedras } este contra peso o es fincado o es mouible o es conpuesto de amos ados & qua eleuata iaciuntur lapides . \textbf{ Si ergo per solum contrapondus fit huiusmodi proiectio : | contrapondus illud est } vel ex fixum , \\\hline
3.3.18 & mas lanca la piedra \textbf{ que non faze el trabuquete . } La quarta manera del engeñio es & se vertentis longius emittit lapidem quam Trabutium . \textbf{ Quartum vero genus machinae est , } quod loco contraponderis habet funes , \\\hline
3.3.18 & Ca si conplidamente sopiere todas estas maneras de engennios \textbf{ de las quales fazemos mençion } conplidamente sabra en qual manera por los engenmos que lançan piedras se puede conbatir & Si enim plena notitia habeatur de machinis , \textbf{ de quibus mentionem fecimus , } sufficienter scietur , \\\hline
3.3.19 & Et pues que assi es despues que dixiemos del conbatimiento \textbf{ que se faze } por las cueuas conegeras & ad muros munitionis obsessae . \textbf{ Dicto ergo de impugnatione facta per cuniculos , } et per lapidarias machinas ; \\\hline
3.3.19 & fincanos de dezir del acometimiento \textbf{ que se puede fazer } por los artifiçios de madera enpuxados & restat dicere de impugnatione \textbf{ quam fieri contingit } per aedificia impulsa ad muros , \\\hline
3.3.19 & dura fruente para ferir \textbf{ e fazer grant colpe . } Et esta viga a tal atanla con cuerdas & quia ratione ferri ibi appositi durissimam habet \textbf{ frontem ad percutiendum . } Huiusmodi autem trabs funibus , \\\hline
3.3.19 & Et esta viga a tal atanla con cuerdas \textbf{ e con cadenas de fierro a la bouada fecha de madera } e a manera de & Huiusmodi autem trabs funibus , \textbf{ vel cathenis ferreis alligatur | ad testudinem factam ex lignis , } et ad modum arietis se subtrahit : \\\hline
3.3.19 & e otros llaman a este artifiçio gata \textbf{ e fazese este artificio } quando las tablas gruessas e fuertes son bien iuntadas e dobladas & quod vocant Vineam . \textbf{ Fit autem hoc , } cum tabulae grossae \\\hline
3.3.19 & quando las tablas gruessas e fuertes son bien iuntadas e dobladas \textbf{ o se fazen dos tablados } por que las piedras que echaren & et fortes optime conligantur , \textbf{ et duplicantur , | siue fit duplex tabulatum , } ne lapides emissi possint \\\hline
3.3.19 & por que non le puedan quemar . \textbf{ Et solien fazer este artifiçio } de ocho pies en ancho & ne ab igne possint offendi . \textbf{ Consueuit autem tale aedificium fieri } in latitudine octo pedum , \\\hline
3.3.19 & e el alteza de los muros de aquella fortaleza \textbf{ e segunt aquella mesma o avn segunt mas asta medida son de fazer las torres } o los castiellos de madera & et secundum huiusmodi mensuram , \textbf{ vel etiam secundum altiorem construendae sunt ligneae turres vel castra , } quae tegenda sunt crudis coriis , \\\hline
3.3.19 & Mas por que el sol non paresce \textbf{ nin faze sonbra . } Ca algunas vezes esta cubierto de nuues . & erit altitudo murorum . Verum quia non semper sol splendet \textbf{ et facit umbram , } sed aliquando tegitur nubibus , \\\hline
3.3.19 & e en el quadrante \textbf{ mas desto non fazemos fuerça . } ca cunple quanto a lo presente tantas cosas dezir desto & per regulas traditas in Astrolabio , et Quadrante . \textbf{ Sed de hoc nobis non sit curae : } sufficiat autem de talibus ad praesens tanta dicere , \\\hline
3.3.19 & fasta la fortaleza que tienen cercada . \textbf{ Et esto quando assi fuere fecho } en tres maneras puede acometer la fortaleza . & usque ad munitionem obsessam : \textbf{ quod cum factum est , } tripliciter impugnanda est munitio . \\\hline
3.3.19 & en tres maneras puede acometer la fortaleza . \textbf{ ca que el castiello assi fecho } para conbatir la fortaleza & tripliciter impugnanda est munitio . \textbf{ Nam in castro sic aedificato } ad munitionem impugnandam , \\\hline
3.3.19 & aquellos que estan en la parte mas alta deuen lançar piedras \textbf{ e fazer foyr } los que estan en los muros & illi qui sunt in parte superiori debent proiicere lapides , \textbf{ et fugare eos , } qui sunt in muris . \\\hline
3.3.20 & e determinamos de la lid \textbf{ que se fazia } por los que cercauan mostrando & ø \\\hline
3.3.20 & lomanera los cercados se deuen defender de los que çercan . \textbf{ Mas lo primero que faze mucho } para que non se tome la çibdat cercada de aquellos qua la çercan & defendere ab obsidentibus . \textbf{ Primum autem quod maxime facit } ne obsessa ciuitas deuincatur ab obsidentibus , \\\hline
3.3.20 & para que non se tome la çibdat cercada de aquellos qua la çercan \textbf{ e lo que mas faze } para que los çercados ligeramente puedan defender las fortalezas & ne obsessa ciuitas deuincatur ab obsidentibus , \textbf{ et maxime facit } ut obsessi faciliter possint \\\hline
3.3.20 & e saber en qual manera son de construyr \textbf{ e de fazer los castiellos et las cibdades e las otras fortalezas } por que non se puedan conbatir ligeramente . & qualiter aedificanda sunt castra , \textbf{ et ciuitates , | et munitiones ceterae , } ne faciliter impugnentur . \\\hline
3.3.20 & quando son de fundar \textbf{ e de fazer las fortalezas } por que se puedan defender & A principio igitur \textbf{ quando aedificandae sunt munitiones , } defendendae ab exteriori pugna , \\\hline
3.3.20 & e la sanera del pueblo quieren se defender en algunan fortaleza \textbf{ o si ouieren poderio es de fazer tal fortaleza } por que la natura del & volunt se tueri in munitione aliqua : \textbf{ si adsit facultas quaerenda est munitio talis , } quae ex ipsa natura loci fortior existat , \\\hline
3.3.20 & mas ligeramente se defienden dellos \textbf{ e mas ligeramente les pueden fazer mal e ferir a los que los çercan . } Ca por que la çerca es fecha a esquinas & obsessi facilius se tuentur ab illis , \textbf{ et leuius offendunt obsidentes . } Nam propter angularitatem murorum \\\hline
3.3.20 & e mas ligeramente les pueden fazer mal e ferir a los que los çercan . \textbf{ Ca por que la çerca es fecha a esquinas } non solamente las pueden ferir de delante & et leuius offendunt obsidentes . \textbf{ Nam propter angularitatem murorum } non solum ex parte anteriori , \\\hline
3.3.20 & a los que lleguan a la fortaleza o a los muros . \textbf{ Et por ende son de fazer los muros de las fortalezas a esquinas } por que se pueda la fortaleza meior defender . & impugnantes munitionem illam . \textbf{ Fiendi itaque sunt muri angulares , } ut munitio faciliter defendi possit . \\\hline
3.3.20 & por que se pueda la fortaleza meior defender . \textbf{ Lo tercero que faze la fortaleza mas fuerte } para se non poder entrar & ut munitio faciliter defendi possit . \textbf{ Tertium , | quod reddit munitionem difficiliorem ad capiendum , } dicuntur esse terrata , \\\hline
3.3.20 & son terrados o torres albarranas o muros çiegos fechos de tierra . \textbf{ Ca en la fortaleza que es de fazer } non solamente es de catar la bondat do esta assentada & vel muri ex terra facti . \textbf{ Nam in munitione fienda } non solum est quaerenda bonitas situs , \\\hline
3.3.20 & e que sean los muros fechos de esquinas . \textbf{ Mas avn cerca aquella fortaleza son de fazer dos muros arredrados algun poco . } assi commo la cerca e la barbacana & et angularitas murorum , \textbf{ sed circa munitionem illam adificandi sunt } duo muri aliqualiter distantes : \\\hline
3.3.20 & que sacan de las carcauas . \textbf{ las quales carcauas son de fazer enderredor de la fortaleza e de los muros } o ssi alli non ouiesse tierra deue se traer de otra parte & quae fodienda est de fossis , \textbf{ quae fiendae sunt | circa munitionem illam , } vel est aliunde terra apportanda , \\\hline
3.3.20 & Et esta tierra es anssi de tapiar \textbf{ que amos los muros se fagan } assi commo vn muro . & Est etiam huiusmodi terra \textbf{ inter tale spatium posita ita densanda , } quod ad inuicem conglutinetur , \\\hline
3.3.20 & assi commo vn muro . \textbf{ Ca pueden se fazer torres albarranas de la tierra } que sean bien feridas e bien tapiadas & et efficiatur quasi murus . \textbf{ Contingit etiam turres ex terra facere , } si bene condensetur : \\\hline
3.3.20 & que sean bien feridas e bien tapiadas \textbf{ e fazen la fortaleza mas fuerte . } Por la qual cosa mucho cunple fazer tales muros & si bene condensetur : \textbf{ propter quod non est inconueniens } construere huiusmodi muros \\\hline
3.3.20 & e fazen la fortaleza mas fuerte . \textbf{ Por la qual cosa mucho cunple fazer tales muros } e tales torres albarranas de tierra muy tapiada . & propter quod non est inconueniens \textbf{ construere huiusmodi muros } ex terra depressata ; \\\hline
3.3.20 & que a los otros muros que son de piedra . \textbf{ Et este muro fecho de tierra deue ser muy gruesso . } ca estonçe resçibra en ssi todas las piedras del engeñio sin grand daño . & quam muri alii ; \textbf{ debet quidem talis murus ex terra factus esse grossus , } quia tunc quasi absque laesione suscipiet lapides emissos a machinis . \\\hline
3.3.20 & ca estonçe resçibra en ssi todas las piedras del engeñio sin grand daño . \textbf{ Lo quarto que faze las fortalezas mas fuertes son torres e menas e cadahalsos . } Ca sienpre son de fazer en los muros torres & quia tunc quasi absque laesione suscipiet lapides emissos a machinis . \textbf{ Quartum autem quod facit munitiones fortiores sunt turres , | et propugnacula . } Nam in ipsis muris construendae sunt turres , \\\hline
3.3.20 & Lo quarto que faze las fortalezas mas fuertes son torres e menas e cadahalsos . \textbf{ Ca sienpre son de fazer en los muros torres } e cadahalsos por que se pueda meior defender la fortaleza . & et propugnacula . \textbf{ Nam in ipsis muris construendae sunt turres , } et propugnacula , \\\hline
3.3.20 & e cadahalsos por que se pueda meior defender la fortaleza . \textbf{ Et mayormente son de fazer las torres } e los & ut munitio leuius defendi possit . \textbf{ Maxime autem ante portam quamlibet ipsius munitionis , } de qua timetur , \\\hline
3.3.20 & e ella toda cobierta de fierro \textbf{ por que non puedan entrar los enemigos nin puedan fazer daño con el fuego . } Ca si los que cercan quisieren llegar a quemar las puertas de la fortaleza . & undique etiam ferrata , \textbf{ prohibens ingressum hostium , | et incendium ignis . } Nam si obsidentes vellent \\\hline
3.3.20 & si contesçiesse que los enemigos pusiessen fuego a las puertas . \textbf{ Lo quinto que faze las fortalezas mas fuertes e peores para las entrar } es quando las carcauas son muy anchas e muy fondas & si contingeret ipsum ab obsidentibus esse appositum . \textbf{ Quintum quod facit munitiones magis inacessibiles , et fortiores : } est latitudo , et profunditas fossarum : \\\hline
3.3.20 & las quales carcauas deuen ser lleñas de agua \textbf{ si sse podiere fazer Et } pues que assi es en estas maneras sobredichas son las fortalezas mas fuertes e peores de tomar . & est latitudo , et profunditas fossarum : \textbf{ quae ( si adsit facultas ) } replendae sunt aquis . \\\hline
3.3.20 & Et por ende deuen catar en el comienço \textbf{ aquellos que quieren fazer las fortalezas } por que las puedan defender de los enemigos & Ideo videndum est a principio ab his \textbf{ qui volunt munitiones defendere } ab obsidentibus eas , \\\hline
3.3.21 & N Non abasta dezir \textbf{ en qual manera son de fazer las fortalezas } e quales muros deuen auer & Non sufficit scire , \textbf{ quomodo aedificandae sunt munitiones , } et quales muros debent habere , \\\hline
3.3.21 & Et si la fortaleza çercada es pequena \textbf{ non es graue cosa de fazer esto . } Ca non aprouecha nada traer muchas viandas & modici esset ambitus , \textbf{ hoc efficere non est difficile } quasi enim nihil prodest \\\hline
3.3.21 & sean enbiadas fuera a otra parte \textbf{ si se puede fazer . } Ca tales perssonas gastan & ad defensionem munitionis obsessae , \textbf{ si commode fieri potest , } sunt ad partes alias transmittendae : \\\hline
3.3.21 & e si y non ouieren fuentes \textbf{ deuen fazer pozos . } Et sy por auentura el logar es tan seco & quod si vero ibi non sint fontes , \textbf{ fodendi sunt putet : } quod si etiam locus sit siccus , \\\hline
3.3.21 & que non pueda \textbf{ y auer poço pueden se ay fazer cisternas } e algibes en que coisgan el agua & quod si etiam locus sit siccus , \textbf{ ut ibi nec putei fieri possint : } fiendae sunt cisternae , \\\hline
3.3.21 & por los enemigos que la tienen cercada . \textbf{ Estonçe pueden el agua salado fazer dulçe colando la } por la çera & ad quam capiendam prohibent obsidentes : \textbf{ tunc mediante caera poterit dulcificari . } Nam secundum philosophum in Meteoris : \\\hline
3.3.21 & por la çera \textbf{ e faziendo pellas hueças } e echando las en el agua salada . & tunc mediante caera poterit dulcificari . \textbf{ Nam secundum philosophum in Meteoris : } Quicquid ex aqua salita \\\hline
3.3.21 & por que les non fallezca \textbf{ assi que de la madera puedan fazer astas } para las saetas & in debita abundantia deportanda , \textbf{ ut per ligna hastae sagittarum , } et telorum , \\\hline
3.3.21 & e para las lanças \textbf{ e avn que puedan fazer } cadahalsos & et telorum , \textbf{ et etiam aedificia necessaria munitioni fieri possint . } Per ferra vero etiam reparari possint arma , \\\hline
3.3.21 & cadahalsos \textbf{ los que fezieren menester en la fortaleza . } Et del fierro puedan las armas & et etiam aedificia necessaria munitioni fieri possint . \textbf{ Per ferra vero etiam reparari possint arma , } et fieri tela ; \\\hline
3.3.21 & Et del fierro puedan las armas \textbf{ e fazer fierros de dardos et de saetas } e las otras cosas & Per ferra vero etiam reparari possint arma , \textbf{ et fieri tela ; } et sagittae , et alia per quae impugnari valeant obsidentes . \\\hline
3.3.21 & Et avn es menester \textbf{ mucha cal fecha poluo } e conuiene de la traer en grand abondança & et turres munitionis obsessae . \textbf{ Calcem etiam puluerizatam deferendum est } ad ipsam munitionem in magna abundantia , \\\hline
3.3.21 & e conuiene de la traer en grand abondança \textbf{ donde quier que la podieren fazer a la fortaleza . } Et conuiene de finchir della & Calcem etiam puluerizatam deferendum est \textbf{ ad ipsam munitionem in magna abundantia , } et ex ea replenda sunt multa vasa ; \\\hline
3.3.21 & que tienen cercada la fortaleza . \textbf{ Et tan grand daño les faze } que assi commo çiegos non veyen quien les fiere & quibus fractis puluis illius subintrat obsidentium oculos , \textbf{ et adeo offendit eos , } ut quasi caeci , \\\hline
3.3.21 & e para las otras cosas \textbf{ que fazen menester } E si por auentura fallesçieren los neruios en logar & et alia praeparanda , \textbf{ quod si nerui deficiant , } loco eorum adhiberi poterunt crines equi , \\\hline
3.3.21 & e dieron los a sus maridos \textbf{ de los quales cabellos fezieron sogas } e refezieron los engenios & eos suis maritis tradiderunt : \textbf{ per quos machinis reparatis } aduersariorum impetum repulerunt . \\\hline
3.3.21 & de los quales cabellos fezieron sogas \textbf{ e refezieron los engenios } e enpuxaron arredraron de ssi los enemigos . & eos suis maritis tradiderunt : \textbf{ per quos machinis reparatis } aduersariorum impetum repulerunt . \\\hline
3.3.22 & El segundo remedio contra las cueuas conegeras \textbf{ e contra las carreras soterrañas es de fazer vna fortaleza cercada otra carrera } que responda a la carrera soterraña & Secundum remedium contra cuniculos \textbf{ et vias subterraneas est , | facere in munitione obsessa viam aliam } correspondentem viae subterraneae factae ab obsidentibus . \\\hline
3.3.22 & que responda a la carrera soterraña \textbf{ que es fecha de los que çercan . } ca si la fortaleza çercada non ha cueuas muy fondas & facere in munitione obsessa viam aliam \textbf{ correspondentem viae subterraneae factae ab obsidentibus . } Si enim obsessa munitio \\\hline
3.3.22 & o si por algunas señales pudieren conosçer \textbf{ que los que cercan comiençan a fazer cueuas coneieras . } Et quando esto entendieren & et utrum per aliqua signa cognoscere possint \textbf{ obsidentes inchoare cuniculos : } quod cum perceperint , \\\hline
3.3.22 & luego sin detenimiento \textbf{ ninguno deuen fazer otras cueuas soterrañas } que respondan a aquellas cueuas coneieras & quod cum perceperint , \textbf{ statim debent viam aliam subterraneam } facere correspondentem illis cuniculis , \\\hline
3.3.22 & que vengan derechamente contra ellas . \textbf{ Enpero assi lo deuen fazer } que aquellas carreras desçendan & ø \\\hline
3.3.22 & contra aquellas \textbf{ que fazen los que çercan . } Et estonçe por aquellas assi foradadas & ita tamen quod via illa pendeat \textbf{ contra obsidentes : et tunc per viam illam sic perforatam } ( cuius partem fecerunt obsidentes , \\\hline
3.3.22 & Et estonçe por aquellas assi foradadas \textbf{ de las quales fizieron vna parte los que çercan } e otra fezieron los cercados se & contra obsidentes : et tunc per viam illam sic perforatam \textbf{ ( cuius partem fecerunt obsidentes , } et partem obsessi ) \\\hline
3.3.22 & de las quales fizieron vna parte los que çercan \textbf{ e otra fezieron los cercados se } deue acometer la batalla continuadamente & ( cuius partem fecerunt obsidentes , \textbf{ et partem obsessi ) } debet esse bellum continuum , \\\hline
3.3.22 & e deuen salir de aquella cueua \textbf{ la qual cosa fecho toda aquella agua o aquella orina } assy ayuntada deuen la echar & et exire foueam illam \textbf{ quo facto totam aquam aut urinam congregatam } effundere debent \\\hline
3.3.22 & tienpo resçibieron grand peligro desto . \textbf{ Por la qual cosa si esto alguna vegada fue fecho } non deuemos cuydar & Temporibus enim nostris multi obsidentium sic periclitati sunt : \textbf{ quare si hoc aliquando factum fuit , } non debemus reputare impossibile \\\hline
3.3.22 & non deuemos cuydar \textbf{ que se non pueda fazer otra vegada . } Visto en qual manera auemos de contrallar a la batalla fecha & non debemus reputare impossibile \textbf{ ne iterum fieri possit . Viso quomodo resistendum sit debellationi factae per cuniculos : } restat videre quomodo obsessi debeant \\\hline
3.3.22 & que se non pueda fazer otra vegada . \textbf{ Visto en qual manera auemos de contrallar a la batalla fecha } por los engenios que lançan las piedras . & ne iterum fieri possit . Viso quomodo resistendum sit debellationi factae per cuniculos : \textbf{ restat videre quomodo obsessi debeant } obuiare impugnationi factae per lapidarias machinas . \\\hline
3.3.22 & e pongan fuego al engeñio . \textbf{ Et esto fecho los de suso resçiban los } por cuerdas a la fortaleza . & machinam incendunt : \textbf{ quo peracto trahuntur superius per funes } ad munitionem illam . Est etiam et tertius modus destruendi machinas \\\hline
3.3.22 & La iij° manera de destroyr los engeñios \textbf{ e las algarradas es fazer saetas } que llaman ruecas & ad munitionem illam . Est etiam et tertius modus destruendi machinas \textbf{ faciendo sagittas } quas appellant telos . \\\hline
3.3.22 & en la qual deue ser puesto fuego de alquitran \textbf{ que es fecho de olio } e de piedra sufre & Est autem sagitta illa in medio quasi quaedam cauea , \textbf{ in qua ponitur ignis fortis factus } ex oleo , sulphure , et pice , et resina : \\\hline
3.3.22 & que lançan las piedras \textbf{ es fazer otros engeñios de dentro } que lançen a ellos e los destruyan & Quarto etiam modo resistitur machinis lapidariis , \textbf{ faciendo alias machinas interius , } percutiendo eas , \\\hline
3.3.22 & Mas entre todos los otros remedios este es el meior \textbf{ despues que es fecho el engeñio de los de dentro } fazerle vna fonda con cadenas de fierro o & Inter caetera autem summum remedium est , \textbf{ postquam constituta est machina , } interius facere ei fundam ex cathenulis ferreis , \\\hline
3.3.22 & despues que es fecho el engeñio de los de dentro \textbf{ fazerle vna fonda con cadenas de fierro o } texidade fierro . & postquam constituta est machina , \textbf{ interius facere ei fundam ex cathenulis ferreis , } vel testam ex ferro ; \\\hline
3.3.22 & texidade fierro . \textbf{ Et cerca de aquel engeñio deuen fazer vna fragua } en que pongan vn grand pedaço de fierro & vel testam ex ferro ; \textbf{ et iuxta machinam illam construere fabricam } in qua aliquod magnum ferrum bene ignatur , \\\hline
3.3.22 & nin la madera non se le puede defender \textbf{ ca toda cosa fecha de madera se puede quemar en esta manera . } Mas avn ay otras muchas cautelas particulares & ligna non habent resistentiam : \textbf{ omne enim aedificium ligneum | hoc modo comburi potest . } Sunt autem et multae aliae particulares cautelae , \\\hline
3.3.22 & et contra este \textbf{ carnero se puede fazer vn fierro } coruo dentado de dientes muy fuertes e muy agudos & Contra hoc autem constituitur \textbf{ quoddam ferrum curuum dentatum dentibus fortissimis , } et acutis , et ligatum funibus , \\\hline
3.3.22 & Enpero contra esto daremos espeçial remedio . \textbf{ ca pueden se fazer cueuas coneieras de dentro e carreras soterrañas } e ascondidamente se puede cauar la tierra & adhibetur tamen speciale remedium contra ipsa , \textbf{ quia fiunt cuniculi , | et viae subterraneae , } et clam suffoditur terra \\\hline
3.3.22 & çerca de aquellos muros deuen algunos castiellos de madera \textbf{ o si pueden deuen fazer muros de piedra } assi que si los que çercan entraren de dentro de la fortaleza sean retenidos e ençerrados entre aquellos muros & iuxta illos muros erigantur aedificia lignea , \textbf{ vel ( si sit possibile ) | aedificentur muri lapidei : } ut si continget obsidentes intrare munitionem , \\\hline
3.3.22 & Enpero deuen tener mientes acuciosamente en su fazienda . \textbf{ ca algunas vezes los que çercan fazen se que fuyen . } Et assi por çeladas & Est tamen diligenter aduertendum , \textbf{ quod aliquando obsidentes | fingunt se fugere , } et sic per insidias , \\\hline
3.3.23 & Mas çerca esta manera de lidiar primeramente es de veer \textbf{ en qual manera es de fazer la naue . } ca la naue mal fecha & primo videndum est , \textbf{ qualiter fabricanda fit nauis : } nam nauis male fabricata , \\\hline
3.3.23 & en qual manera es de fazer la naue . \textbf{ ca la naue mal fecha } por pequena batalla de los enemigos de ligero peresçe . & qualiter fabricanda fit nauis : \textbf{ nam nauis male fabricata , } ex modica impugnatione hostium de facili perit . \\\hline
3.3.23 & Et pues que assi es conuiene de saber \textbf{ que segunt que dize vegeçio que los maderos que se deue fazer la } naue non son de taiar & Sciendum ergo , \textbf{ quod secundum Vegetium , | ligna ex quibus construenda est nauis , } non sunt de quolibet tempore incidenda . \\\hline
3.3.23 & non es bueno de taiar los arboles \textbf{ de los quales deue ser fecha la naue . } Mas en el tienpo del iullio e del & non est bonum incidere arbores , \textbf{ ex quibus fabricanda est nauis . } Sed tempore Iulii et Augusti vel aliquo alio tempore , \\\hline
3.3.23 & Otrossi taiados los maderos non es \textbf{ luego de fazer la naue dellos . } Mas primero los arboles deuen ser serrados & Rursus , non statim incisis lignis est \textbf{ ex eis fabricanda nauis : } sed primo arbores sunt diuidendae per tabulas ; \\\hline
3.3.23 & assi por que se puedan secar . \textbf{ ca si la naue se faze de madera verde } quando el humor natural dellos se va e se seca . & ut desiccari possint . \textbf{ Nam si ex lignis viridibus construatur nauis , } quando naturalis eorum humor expirauerit , \\\hline
3.3.23 & los maderos se encogen \textbf{ e fazese a venturas en las naues } e ninguna cosa non puede ser mas periglosa en las naues & contrahuntur ligna , \textbf{ et faciunt in nauibus rimas , } quibus in nauibus nihil periculosius esse potest . \\\hline
3.3.23 & e de poner a secar \textbf{ por que della se pueda fazer la } naue conueniblemente . & et quomodo reseruanda , \textbf{ ut ex eis nauis debite valeat fabricari : } restat videre ; \\\hline
3.3.23 & finca de veer commo son de acometer las batallas en la \textbf{ naue bien fecha e bien formada . } ca la batalla de las naues & quomodo in naui bene fabricata \textbf{ committenda sunt bella . } Habet autem nauale bellum quantum \\\hline
3.3.23 & Bien assi todas estas cosas \textbf{ fazen menester } en la batalla de la naue . & et hostibus vulnera infligere : \textbf{ sic et haec requiruntur in bello nauali . } Immo in huiusmodi pugna oportet \\\hline
3.3.23 & Ca quebrantados aquellos cantaros en las naues \textbf{ en aquellos logares fazen se escorredizos por el xabon } en tal manera que los enemigos non pueden y tener los pies & Nam vasis illis confractis in huiusmodi locis , \textbf{ loca illa per saponem liquidam redduntur adeo lubrica , } quod hostes ibi ponentes pedes statim labuntur in aquis . \\\hline
3.3.23 & e foraden la pordiuso . \textbf{ Et faziendo muchos forados } los quales non podran los enemigos çercar & et eam in profundo perforare , \textbf{ faciendo ibi plura foramina , } quae foramina ab hostibus reperiri non poterunt , \\\hline
3.3.23 & Mostrado en qual manera es de taiar la madera \textbf{ para fazer las naues } e en qual manera auemos de lidiar en la mar . & Ostenso qualiter incidenda sunt ligna \textbf{ ex quibus construenda est nauis , } et quomodo bellandum est in nauali bello . \\\hline
3.3.23 & Et enpero las batallas sy \textbf{ derechamente las fezieren } e las tomaren conueniblemente son de ordenar . & vel ad aliquam aliam satifactionem irae , vel concupiscentiae . \textbf{ Bella tamen si iuste gerantur , } et debite fiant , \\\hline
3.3.23 & e non sobrepuia el vno sobre el otro \textbf{ e non faze menester xarope nin sangria } assi mientra los omnes estan ordenados & et non est ibi humorum excessus , \textbf{ non indigemus potione nec phlebotomia ; } sic quamdiu homines debite se habent , \\\hline
3.3.23 & commo deuen \textbf{ e el vno non faze tuerto al otro non ay } porque auer batalla ninguna . & sic quamdiu homines debite se habent , \textbf{ et unus non iniuriatur alteri ; } non sunt committenda bella . \\\hline

\end{tabular}
