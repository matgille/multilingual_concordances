\begin{tabular}{|p{1cm}|p{6.5cm}|p{6.5cm}|}

\hline
1.2.17 & e en partiendo e en dando los sus bienes alos ctros . \textbf{ visto que cosa es la franqueza . } Ca es uirtud & principaliter liberalitas consistit . \textbf{ Viso quid est liberalitas , } quia est virtus reprimens auaritias , \\\hline
1.2.19 & assi la magnificençia faze espenssas conuenibles alas grandes obras ¶ \textbf{ visto que cosa es la magnificençia } fincanos de veer & sic magnificentia est faciens sumptus decentes magnis operibus . \textbf{ Viso quid est magnificentia : } restat videre circa quae habet esse . \\\hline
1.2.22 & que non pueden acabar . \textbf{ ¶ Visto que cosa es la uirtud } que dizen magranimidat & et moderans praesumptiones . \textbf{ Viso quid est magnanimitas , } de leui patet , \\\hline
1.2.26 & entre la presup̃çion e la pusillanimidat . \textbf{ ¶ Visto que cosa es la humisdat } ligeramente puede paresçer & sicut magnanimitas est media inter praesumptionem , et pusillanimitatem . \textbf{ Viso quid est humilitas , } de leui videri potest \\\hline
1.2.26 & e non son uirtudes ¶ \textbf{ Visto cerca quales cosas es lahcudildat } ca prinçipalmente es çerca la presup̃çion & est quid iactatiuum . \textbf{ Viso circa quae est humilitas , } quia principaliter est \\\hline
1.2.27 & que es nunca se enssannar ¶ \textbf{ Visto que cosa es la mansedunbre ligeramente paresçe } çerca quales cosas ha de ser . & et moderat irascibilitates . \textbf{ Viso quid est mansuetudo , | de leui patet } circa quae habet esse . \\\hline
1.2.28 & por que ninguno non le deue en tanto mostrar conpanero alos otros \textbf{ por que sea visto plazentero e falaguero . } Otrossi non se deue tirar en tanto dela conpana & tantum aliis ostendere socialem , \textbf{ ut videatur placidus , } et blanditor : nec se debet tantum a societate subtrahere , \\\hline
1.2.28 & Otrossi non se deue tirar en tanto dela conpana \textbf{ por que sea visto desacordable e vara ador . } Et pues que assi es commo la uirtud sea medianera & et blanditor : nec se debet tantum a societate subtrahere , \textbf{ ut uideatur discolus , et litigiosus . } Cum igitur uirtus sit \\\hline
1.2.28 & la qual cosa fazen los desacordabłs \textbf{ e mouedores de pelea ¶ visto que cosa es la amistanca } segunt que della fablamos aqui . & quod faciunt litigiosi , et discoli . \textbf{ Viso quid est amicabilitas , ut hic de ea loquimur , } quia est uirtus reprimens litigia , \\\hline
1.2.29 & por la sobrepuiança de carga . \textbf{ ¶ Visto que cosa es la uerdat } dela qual aqui fablamos & propter onerosas esse superabundantias . \textbf{ Ostenso quid est veritas } de qua loquimur , \\\hline
1.2.30 & que non para que fuyamos dellas . \textbf{ ¶ Visto que cosa es la entropelia o alegria } e çerca quales cosas ha de seer . & quam ut fugiamus illas . \textbf{ Viso quid est eutrapelia siue iocunditas , } et circa quae habet esse , \\\hline
1.3.2 & que es tomada en razon de bien es primero que la tristeza \textbf{ que es tomada en razon de mal Et pues que assi es por que visto es } ya de suso quantas son las passiones & prior est tristitia , \textbf{ quae est respectu mali . | Quia ergo visum est , } quot sunt passiones , \\\hline
1.3.3 & e dannaua las eglesias e las casas santas \textbf{ ¶ visto en qual manera los Reyes } e los prinçipes se de una auer al amor & et depraedabat sacra . \textbf{ Viso quomodo Reges et Principes se habere debeant ad amorem , } quia principaliter debent amare bonum diuinum et commune : \\\hline
1.3.4 & ¶ \textbf{ Visto quales cosas deuen los Reyes dessear } e en qual manera & Viso , quae , \textbf{ et quomodo Reges et Principes desiderare debent , } quia sicut debent amare primo \\\hline
1.3.5 & tenesçen ala espança conuenible \textbf{ ¶ visto conmolos Reyes } e los prinçipes se deuen bien auer & quae ad spem debitam requiruntur . \textbf{ Viso , quomodo decet Reges } et Principes \\\hline
1.3.6 & e sin razon . \textbf{ ¶ visto en qual manera los Reyes se deuen auer al temor } por que cosa mas guaue es de repremir el temor que tenprar la osadia & Regem immoderato timore timere . \textbf{ Viso quomodo Reges se habere debeant ad timorem , } quia difficilius est reprimere timorem , \\\hline
1.3.8 & e enbargan las obras uirtuosas \textbf{ ¶ Visto en qual manera los Reyes } e los prinçipes se deuen auer alas delectaçiones & et operationes impediunt virtuosas . \textbf{ Viso , quomodo Reges , } et Principes se habere debeant ad delectationes : \\\hline
1.4.1 & nin honrra los mançebos de ligero toman uerguença \textbf{ ¶ visto quales costunbres son de loar en los mançebos de ligero } puede paresçer en qual manera los Reyes & iuuenes de facili erubescunt . \textbf{ Viso qui mores sunt laudabiles de iuuenibus , } de leui patere potest , \\\hline
1.4.2 & mas todas las cosas fazen con sobrepuiamiento ¶ \textbf{ visto quales son las costunbres delos mançebos } que son de denostar de ligero pue de omne & sed omnia faciunt valde . \textbf{ Viso qui sunt mores iuuenum vituperabiles ; } de facili videri potest , \\\hline
1.4.3 & por que la cosa fria non ha de querer logar alto mas baxo . \textbf{ visto quales son las costunbres de los uieios } que son de denostar de ligero & appetere locum superiorem , sed inferiorem . \textbf{ Viso qui sunt mores senum vituperabiles ; } de leui patere potest \\\hline
1.4.4 & por la mayor parte fazen todas las cosas tenpradamente ¶ \textbf{ Visto quales son las costunbres de los mançebos } e de los uieios de ligero puede paresçer & ut plurimum agunt omnia moderate . \textbf{ Viso , qui sunt mores iuuenum , } et senum : \\\hline
1.4.5 & Et en qual manera bien razonados e \textbf{ corteses¶ Visto quales costunbres delos nobłs son de loar } finca de uer quales costunbres son de denostar . & et quomodo affabiles et sociales . \textbf{ Viso qui mores nobilium sunt laudabiles : | videre restat } qui mores sunt vituperabiles . \\\hline
1.4.6 & ¶ \textbf{ Visto quales son las malas costunbres de los ricos } e que conuiene alos Reyes e alos & et ad opera virtuosa . \textbf{ Viso qui sunt mali mores diuitum , } et quod decet Reges , \\\hline
1.4.7 & Por la qual cosa non es vna cosa ser el omne rico e ser toderoso¶ \textbf{ Visto quales son las costunbres delons nobłs } e de los ricos finça de & et esse non potentem . \textbf{ Viso ergo | qui sunt mores nobilium , } et qui diuitum restat videre , \\\hline
2.1.1 & faze su tela conuenible \textbf{ avn que nunca aya visto otras arannas texer en essa misma manera } avn las golondrinas fazen su nido conueniblemente & debitam telam faceret , \textbf{ si nunquam vidisset | araneas alias texuisse . } Sic etiam et hirundines debite facerent nidum , \\\hline
2.1.3 & mas conplidamente en el terçero libro \textbf{ Visto en qual manera la comunidat dela casa es primero en alguna manera que las otras comuni dades de ligero puede paresçer } en alguna manera esta comunidat dela cała es natural & ut in tertio libro plenius ostendetur . \textbf{ Viso , quomodo communitas domus | aliquo modo est prior , } quam communitates aliae : \\\hline
2.1.5 & ¶ \textbf{ Visto en qual manera } alo menos estas dos comunidades & infra clarius ostendetur . \textbf{ Viso , quomodo saltem hae duae communitates requiruntur ad domum , } quia secundum Philosophum ex eis constare \\\hline
2.1.6 & segunt que el sennor deue ser antepuesto alos sieruos \textbf{ Visto commo en la casa acabada } ian de ser tres comuindades & secundum quod dominus praeest seruis . \textbf{ Viso , in domo perfecta esse communitates tres , } et tria regimina : \\\hline
2.1.12 & ¶ \textbf{ Visto en qual manera los Reyes } e los prinçipes & quam ex tali coniugio acquiratur multitudo numismatis . \textbf{ Viso quomodo Reges , } et Principes in suis coniugibus \\\hline
2.1.13 & quales bienes de fuera son de demandar enla muger . \textbf{ Esto uisto finca nos de dezir } en qual manera las mugers de los çibdadanos deuen ser honrradas & qualia bona exteriora sunt quaerenda in coniuge . \textbf{ Reliquum est ut dicamus , } quomodo coniuges ipsorum virorum bonis \\\hline
2.1.13 & a E ordenado el casamiento . \textbf{ ¶ Visto en qual manera } quanto alos bienes del cuerpo son & in coniuge magnitudo , et pulchritudo . \textbf{ Restat videre quomodo quantum } ad bona animae \\\hline
2.1.14 & Ca en ninguna manera los fijos non escogen assi su padre ¶ \textbf{ Visto en qual manera se departe el } gouernamientoma termoianl del paternal & quia filii nullo modo eligunt sibi patrem . \textbf{ Viso , quomodo differt regimen coniugale } a paternali ex modo regendi , \\\hline
2.1.18 & Enpero enlos otros es muchos de denostar ¶ \textbf{ Visto quales cosas son de alabar en las muger } s finca de dezir que cosas son de denostar en ellas . & in malis vero est vituperabile . \textbf{ Viso , quae sunt laudabilia in foeminis : } restat narrare quae sunt vituperabilia in eis . \\\hline
2.1.19 & por amonestamientos conueninbles a las uirtudes e bondades sobredich̃ͣs¶ \textbf{ Visto en qual manera se deua gouernar la muger } por que ella sea tenprada & ad bonitates praehabitas . \textbf{ Viso , coniugem sic regendam esse , } ut sit debite temperata : \\\hline
2.1.20 & mas avn amistança honesta . \textbf{ ¶ Visto en qual manera conuiene alos uarons de vlar labiamente } e tenpradamente de sus muger sfinca de ver & non solum amicitia delectabilis , sed honesta . Viso , \textbf{ quomodo decet viros suis uxoribus moderate uti et discrete : } restat videre , \\\hline
2.2.3 & ¶ \textbf{ Visto que el gouernamiento del padre nasçe de amor paresçe } que el padre deue & nisi paternum regimen ex amore nasceretur . \textbf{ Viso , quod paternum regimen ex amore nascitur , } patet quod filiis debet \\\hline
2.2.4 & commo los fijos de los padres . \textbf{ ¶ Visto qual es el amor de los radres alos fijos . } Ca los padres son inclia a dos a amar los fijos & quam econuerso . \textbf{ Viso , qualis est amor | inter patrem et filium , } quia parentes afficiuntur ad filios \\\hline
2.2.10 & para fablar palabras sin reprehenssion . \textbf{ ¶ Visto en qual manera los ayos } e los maestros de los moços los deuen enssennar & ut proferant sermones irreprehensibiles . \textbf{ Viso qualiter paedagogi et doctores } iuuenum debent eos instruere \\\hline
2.2.12 & por que sean guardados e mesurados ¶ \textbf{ Visto en qual manera los moços se deuen auer çerca el comer } e el beuer finca de ver & quod sint abstinentes et sobrii . \textbf{ Viso , qualiter iuuenes debeant se habere | circa cibum et potum . } Restat videre , \\\hline
2.2.13 & delas politicas ¶ \textbf{ Visto en qual manera los moços se deuen auer çerca los trebeios finca de ver } en qual manera se deuen auer çerca los gestos ¶ & ut vult Philosophus 7 Politicorum . \textbf{ Viso qualiter iuuenes se habere debeant circa ludos . | Restat videre , } qualiter se habere debeant circa gestus . \\\hline
2.2.16 & por que non enbarguen la cresçençia delos mienbros ¶ \textbf{ Visto que para que los moços sean bien despuestos e bien ordenados } quanto al cuerpo del vi̊ año fasta xuiij . & ne impediatur incrementum . \textbf{ Viso , quod , ut iuuenes sint | bene dispositi quantum ad corpus , } a septimo usque ad quartumdecimum annum \\\hline
2.2.17 & e amouimientos conuenibles ¶ \textbf{ Visto en qual manera del . xiiij . año adelante deuen los padres auer cuydado de los fijos } por que ayan el cuerpo bien ordenado & si ad debita exercitia assuescant . \textbf{ Viso , quomodo a quartodecimo anno ultra solicitari debent patres erga filios , } ut habeant dispositum corpus . \\\hline
2.3.2 & ¶ \textbf{ Visto en qual manera se departe los estrumentos del gouernamiento dela casa finca de ver } en qual manera son conparados los vnos alos otros . & quia haec sunt animata , illa inanimata . \textbf{ Viso , quomodo distinguuntur | organa gubernationis domus : } restat ostendere , \\\hline
2.3.3 & que ellos sean muy grandes e muy costosas \textbf{ ¶ Visto quales deuen ser las moradas } quanto ala grandeza & oportet ipsa esse magnifica . \textbf{ Viso , qualia debent esse aedificia , } quantum ad magnificentiam et industriam operis : \\\hline
2.3.4 & por que por el nadamiento de los peces \textbf{ el agua estante semeie en lignieza al agua que corre ¶ Visto en qual manera son de fazer las moradas } quanto ala salud delas agunas finca de ver & ut horum natatu aqua \textbf{ stans agilitatem currentis imitetur . | Viso , qualiter est aedificium construendum quantum } ad salubritatem aquae : \\\hline
2.3.14 & e de cosa çierta . \textbf{ Visto fue alos establesçedores delas leyes } que los bençedores en la fazienda & ut lex daret iudicium de aliquo certo , \textbf{ visum fuit legum latoribus , } ut superantes in bello congrue dominarentur aliis superatis , \\\hline
2.3.18 & que han con los omes \textbf{ Visto que cosa es la curialidat et la cortesia de ligero puede paresçer } que los seruientes de los Reyes & ad conuersationem requiruntur in ipsis . \textbf{ Viso quid est curialitas , | de leui patet } quod decet ministros Regum et Principum curiales esse . \\\hline
2.3.19 & ante que ellos suban atan al cogrado . \textbf{ ¶ Visto commo son de a comne dar los ofiçios } alos ofiçiales finca deuer & prius quam ad aliud altum ascendant . \textbf{ Viso quomodo ministris sunt officia committenda , } restat videre quomodo sunt \\\hline
3.2.5 & Et quando han gouernador muchͣs uezes t hiranzan . \textbf{ Onde muchs males auemos visto en tales gouernamientos } que non podemos contar & et cum gubernatorem habent \textbf{ ut plurimum tyrannizant : | plura enim huiusmodi mala } in talibus regiminibus vidimus , \\\hline
3.2.6 & contra la paz del regno \textbf{ Visto en quales cosas el Rey deue sobrepuiar } e auer a una taia de los otros . & et turbare pacem regni . \textbf{ Viso in quibus Rex alios debet excedere : } restat ostendere , \\\hline
3.2.8 & Enpero mas claramente se dira a de sante . \textbf{ Visto que parte nesçe al ofiçio del Rey } de ser acuçioso çerca aquellas cosas & clarius tamen infra dicetur . \textbf{ Viso quod spectat ad Regis officium solicitari } circa ea per quae possit populus \\\hline
3.2.17 & commo estas non deuen ser dicho consseios ¶ \textbf{ Visto que cola es consseio } ca es question delas obras & dici non debent . \textbf{ Viso quid est consilium , } quia est quaestio agibilium humanorum : \\\hline
3.2.24 & tales cosas castigadas o condep̃nadas . \textbf{ ¶ Esto uisto quanto pertenesçe alo presente podemos mostrar dos departimientos } e dos diferençias & determinans qua poena sint talia punienda . \textbf{ Hoc viso quantum | ad praesens spectat } duplicem differentiam \\\hline
3.2.25 & ¶ \textbf{ Visto en qual manera el derecho delas gentes se } departe del derecho natal de ligo puede paresçer & respectu iuris gentium . \textbf{ Viso quomodo ius gentium | differt } a iure naturali , \\\hline
3.2.27 & enssennoreare ¶ \textbf{ Visto que non parte nesçe a cada vno } establesçer las leyes de ligo puede paresçer & si tota huiusmodi multitudo principetur . \textbf{ Viso quod non est } cuiuslibet leges condere , \\\hline
3.2.28 & ¶ \textbf{ Visto que obras son de apodar alas leyes } en conparaçion delas obras & annexam permitti possint a legislatore . \textbf{ Viso quae attribuenda sunt legibus respectu operum fiendorum : } de leui apparere potest \\\hline
3.2.32 & por si uida acabada e conplida . \textbf{ Visto que cosa es la çibdat de ligero se puede ver } que cosa es el regno . & et per se sufficientem vitam . \textbf{ Viso quid est ciuitas , | de leui videri potest } quid est regnum . \\\hline
3.2.35 & ¶ \textbf{ Visto en qual manera los moradores del regno } non deuen mouer el Rey a saña errando contra el & et non obediunt regi quasi praecellenti . \textbf{ Viso quomodo habitatores regni non debent } prouocare Regem ad iram , \\\hline
3.2.36 & que much amamos los derechureros . \textbf{ Visto en qual manera los Reyes } e los prinçipes se de una auer & quod iustos maxime diligimus . \textbf{ Viso quomodo Reges et Principes } debeant se habere \\\hline
3.2.36 & mas son temidos que los manifiestos . \textbf{ Visto commo se deuen auer los Reyes } e los prinçipes & quam manifesti . \textbf{ Viso quomodo Reges et Principes se habere debeant } ut amentur , \\\hline
3.3.2 & que las otras gentes . \textbf{ Visto de quales partes son los meiores lidiadores } finca de ver & ut magis animositate participent . \textbf{ Viso ex quibus partibus meliores sunt bellatores : } videre restat , \\\hline
3.3.3 & e lidiadores escogidos e fuertes . \textbf{ Visto en qual hedat son de acostunbrar a obras de batalla } aquellos que se deuen fazir caualleros e ser lidiadores . & viros exercitatos et bellatores strenuos debet assumere . \textbf{ Viso in qua aetate assuescendi sunt } qui debent effici bellatores \\\hline
3.3.6 & para foyr e para morir . \textbf{ Visto en qual manera el uso de las armas es muy prouechoso } para las obras de la batalla & ad fugam et ad eaedem . \textbf{ Viso armorum exercitium } esse perutile ad opera bellica , \\\hline
3.3.8 & que lieuan consigo assi commo vna çibdat guarnida . \textbf{ Visto commo es cosa prouechable a la hueste fazer carcauas e costruir guarniçiones e castiellos . } finca de demostrar en qual manera las tales guarniciones & Viso utile esse \textbf{ circa exercitum facere fossas | et construere castra : } restat ostendere , \\\hline
3.3.15 & podrian mas fuertemente ferir los enemigos e mas ligeramente foyr los colpes dellos . \textbf{ Visto commo deuen estar los lidiadores } quando deuen ferir los enemigos . & et eorum ictus facilius fugere . \textbf{ Viso quomodo debeant stare bellantes , } si debeant hostes percutere : \\\hline
3.3.16 & e que quiere dezir batalla de cercamiento . \textbf{ Et por ende uisto quantas son las maneras de las batallas . } Et dicho que despues de la lid canpal del canpo primero & Primo tamen dicemus de bello obsessiuo . \textbf{ Viso ergo quot sunt bellorum genera , } et dicto quod post castrum campestre primo dicendum est \\\hline
3.3.17 & por que los non puedan enpesçer . \textbf{ visto en qual manera se deuen guarnesçer los çercadores } por que non pueda resçebir daño de los cercados & ut si oppidani eos repente vellent inuadere , resistentiam inuenirent . \textbf{ Viso quomodo se munire debent obsidentes , } ne ab oppidanis molestentur : \\\hline
3.3.21 & nin es vso de las comer . \textbf{ Visto en qual manera la fortaleza cercada se escusa } e se guarda & quae ad esum vetat communis usus . \textbf{ Viso quomodo munitio obsessa vitat , } ne capiatur fame : \\\hline
3.3.22 & que se non pueda fazer otra vegada . \textbf{ Visto en qual manera auemos de contrallar a la batalla fecha } por los engenios que lançan las piedras . & ne iterum fieri possit . Viso quomodo resistendum sit debellationi factae per cuniculos : \textbf{ restat videre quomodo obsessi debeant } obuiare impugnationi factae per lapidarias machinas . \\\hline
3.3.23 & por las quales la naua puede peresçer . \textbf{ visto en qual manera es de taiar la madera . } e en qual manera es de guardar & ne puppis per rimas naufragium patiatur . \textbf{ Viso qualiter incidenda sunt ligna , } et quomodo reseruanda , \\\hline

\end{tabular}
