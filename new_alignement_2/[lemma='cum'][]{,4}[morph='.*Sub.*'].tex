\begin{tabular}{|p{1cm}|p{6.5cm}|p{6.5cm}|}

\hline
1.1.13 & sed etiam totum regnum . \textbf{ Cum ergo magnae virtuti debeatur magna merces , } magnum erit meritum & mas ahun a todo el regno . \textbf{ Et pues que assi es commo grant uirtud deua auer grant merçed } e gm̃t gualardon gerad sera el meresçimiento e el gualardon de los Reyes . \\\hline
1.2.2 & et malum inquantum habent rationem difficilis , et ardui . \textbf{ Nam cum bonum secundum se dicat prosequendum , } malum vero quid fugiendum : & en quanto han razon de cosa guaue e fuerte . \textbf{ Ca commo el bien | por si diga tal cosa } que deue el omne seguir \\\hline
1.2.6 & circa quam versatur . \textbf{ Cum enim Prudentia sit circa agibilia , } et agibilia sint singularia , & ala materia en que obra . \textbf{ Ca commo la pradençia aya de ser en las obras . } Et las obras ayan de ser \\\hline
1.2.12 & ut sint Reges . \textbf{ Cum enim deceat regulam esse rectam et aequalem , } Rex quia est quaedam animata lex , & enpero non son dignos de seer Reyes . \textbf{ Ca commo conuenga ala regla de ser derecha } e egual e el Rey sea vna ley animada e vna regla . \\\hline
1.2.19 & et proportionati diuitibus . \textbf{ Cum ergo liberalitas non respiciat sumptus secundum se , } sed ut proportionantur facultatibus , & e conuenibles alos ricos \textbf{ ¶Pues que assi es commo la libalidat | non cate alas espenssas } segunt \\\hline
1.2.26 & Non tamen aequae principaliter operatur utrunque : \textbf{ nam cum magnanimi sit tendere in magnum , } magnanimitas magis est & e el entendimiento muestran . Enpero estas dos cosas non las obra egualmente nin prinçipalmente \textbf{ Ca commo almagranimo pertenesca de yr | e entender en cosas grandes la magranimidat } mas es uirtud \\\hline
1.2.27 & ut fiant punitiones et vindictae , \textbf{ cum hoc faciat mansuetudo , } decet eos mansuetos esse . & para fazer uenganças e dar penas . \textbf{ Et commo esto faga la manssedunbre } conuiene a ellos de ser manssos \\\hline
1.2.29 & Concedunt enim de se aliqui magnas bonitates , \textbf{ cum tamen illis careant : } et promittunt amicis & Ca otorgan alguons de ssi mismos grandes bondades \textbf{ commo quier que en ellos nen sean } e prometen alos amigos conosçidos grandes bienes e grandes ayudas . \\\hline
1.2.31 & quia perfecte una virtus sine aliis haberi non potest . Immo expedit Regibus et Principibus , \textbf{ cum non possint se excusare } per defectum exteriorum bonorum , & mas ante conuiene alos Reyes \textbf{ e alos prinçipes | commo ellos non se puedan escusar } por mengua de los bienes \\\hline
1.3.7 & non autem odio . \textbf{ Nam cum ira satietur , } si multa mala inferantur alteri , & Mas la mal querençia non \textbf{ Porque commo la sanna se pueda fartar } si muchos males fueren fechos al otro \\\hline
1.3.7 & impedimur ab usu rationis , \textbf{ quare cum per iram accendatur sanguis circa cor , } corpus redditur intemperatum , & Ca el cuerpo non estando en tenpramiento conuenible somos enbargados en el vso dela razon . \textbf{ Por la qual cosa commo por la saña se ençienda la sangre cerca el coraçon tornasse el cuerpo destenprado } e non podemos conueniblemente vsar de la razon . \\\hline
1.3.8 & ex coniunctione conuenientis cum conuenienti : \textbf{ cum ergo alia conueniant bestiis , alia hominibus : } aliquae delectationes sunt conuenientes bestiis , & Otrossi por que la delectacion se faze por ayuntamiento dela cosa conuenible con cosa conuenible \textbf{ Por ende commo algunas cosas conuengan alas bestias | e algunos alos omes } algunans delectaçiones seran conuenientes alas bestias \\\hline
1.3.9 & spes et timor sunt principales passiones respectu irascibilis . \textbf{ Sed cum ex passionibus diuersificari habeant opera nostra , } decet nos diligenter intendere , & en conparacion del appetito enssannador . \textbf{ Mas commo las nr̃as obras ayan de ser departidas | por estas passiones . } Cconuiene a nos de acuçiosamente entender \\\hline
1.4.2 & eos esse nimis creditiuos . \textbf{ Nam cum multos habeant adulatores , } et plurimi sint in eorum auribus susurrantes , & e alos prinçipes de çreer de ligero . \textbf{ Ca commo ellos ayan muchos lisongeros } e muchos les estenruyendo alas oreias \\\hline
1.4.5 & in filiis quam in parentibus : \textbf{ quare cum nobilitas semper inclinet animum nobilium } ut faciant magna , & antiguadas las riquezas en los fijos que en los padres . \textbf{ Por la qual razon commo la nobleza | sienpre incline el coraçon de los nobles } para fazer grandes cosas siguese \\\hline
1.4.7 & et habet multos sub suo dominio ; \textbf{ quare cum multos nobiles videamus esse impotentes , } et non posse principari , & e ha muchos so su sennorio . \textbf{ por la qual cosa commo nos veamos muchos ser nobles | qua non son poderosos } e non pueden ser prinçipes \\\hline
2.1.9 & est amicitia excellens et naturalis . \textbf{ Sed cum excellens amor non possit esse ad plures , } ut vult Philosophus 9 Ethicor’ , & entre ellos es amistança muy grande e muy natural . \textbf{ Mas commo el grand amor non pueda ser departido amuchͣs partes } assi conmo dize el philosofo en elix̊ . \\\hline
2.1.10 & ad quam coniugium ordinatur . \textbf{ Nam cum quilibet moleste ferat , } si in usu suae rei delectabilis impeditur ; & ala qual es ordenado el casamiento . \textbf{ Ca commo qual si quier sufra } guauemente si le enbargaren del vso de aquella cosa \\\hline
2.1.11 & Prima via sic patet . \textbf{ Nam cum ex naturali ordine debeamus parentibus debitam subiectionem , } et consanguineis debitam reuerentiam , & La primera razon se declara assi . \textbf{ Ca commo por la orden natural deuamos auer | subiectiuo al padre e ala madre } e reuerençia conueible alos parientes \\\hline
2.1.19 & Unde et aliquos Philosophos legimus sic fecisse , \textbf{ qui cum essent impeditae linguae , } accipientes specialem conatum & Ende leemos que algunos philosofos lo fizieron \textbf{ assi los quales commo ouiessen las lenguas enbargadas } tomaron especial esfuerço çerca aquellas letras \\\hline
2.2.3 & quando potest sibi simile generare . \textbf{ Quare cum quilibet suam perfectionem diligat , } naturaliter pater diligit filium , & quando ꝑuede engendrar su semeiante . \textbf{ Et commo quier que cada vno ame su perfecçion . } Emperona traalmente el padre ama el fijo \\\hline
2.2.4 & quam econuerso . \textbf{ Quare cum amor quandam unionem importet , } filii tanquam magis uniti et magis propinqui parentibus , & que los padres alos fijos . \textbf{ por la qual razon commo el amor faga algun ayuntamiento los fijos } assi conmomas ayuntados \\\hline
2.3.17 & et congruentia temporum . \textbf{ Cum enim deceat Regem esse magnificum , } ut supra in primo libro diffusius probabatur , & La conueniençia de los tiepos . \textbf{ Ca commo conuenga alos Reyes | e alos prinçipes ser magnificos } assi commo es prouado mas conplidamente en el primero libro \\\hline
3.1.8 & et aliqui subiecti . \textbf{ Quare cum hoc diuersitatem requirat , } oportet in ciuitate & e alguons que fuessen subditos \textbf{ por ende commo estas cosas demanden departimiento } conuiene de dar en la çibdat algun departimiento . \\\hline
3.1.11 & tanto magis ad inuicem conuersantur : \textbf{ sed cum esse non possit , } aliquos valde & mas han de beuir en vno \textbf{ mas commo non pueda ser } que alguons \\\hline
3.1.12 & quam eos in societate habere \textbf{ nam cum humanum sit timere mortem , } viriles etiam et animosi trepidant & que auerlos en su conpannia \textbf{ ca commo todos los omes | teman la muerte los esforçados } e de grandes coraçones temen \\\hline
3.1.13 & vel ad aliquem magistratum assumitur . \textbf{ Quare cum deceat regia maiestatem } et uniuersaliter omnem ciuem , & commo se conosçe despues que esle un atada en alguna dignidat o en algun maestradgo o en algun poderio \textbf{ por la qual razon commo venga ala real magestad } e generalmente a qual quier que ha de dar \\\hline
3.2.10 & et de se confidere ; \textbf{ nam cum intendat bonum ipsorum ciuium et subditorum , } naturale est & e que fien vnos de otros . \textbf{ Ca commo el entienda enl bien de los çibdadanos natural } cosaes que sea amado dellos . \\\hline
3.2.10 & quicquid a ciuibus agitur . \textbf{ Cum enim tyranni sciant se non diligi a populo , } eo quod in multis offendant ipsum , & delo que fazen los çibdadanos . \textbf{ Ca commo los tyranos sepan | que non lon amados del pueblo . } por que en muchͣs cosas le aguauian quieren auer muchs assechadores \\\hline
3.2.16 & quae tractanda sunt circa ipsum . \textbf{ Sed , cum dicat Philosophus } 3 Ethic’ & quales cosas son de trattrar çerca el . \textbf{ Mas commo el pho diga en el segundo libro delas ethicas } que por çierto alguno tomara consseio non de aquellas cosas \\\hline
3.2.20 & et alia quae circa istam occurrunt materiam . \textbf{ Sed cum iudicium fiat per leges , } aut per arbitrium , & que pueden acaesçer çerca desta materia \textbf{ mas commo el iuyzio se deua fazer } por las leyes o por aluedrio o por amas estas cosas . \\\hline
3.3.4 & victoria esse finis . \textbf{ Quare cum maxime contingat bellantes vincere , } si bene sciant se protegere & la uictoria es dicha fin de todas las obras de la batalla . \textbf{ Por la qual cosa commo mayormente contezca a los lidiadores vençer } si bien se sopieren cobrir \\\hline

\end{tabular}
