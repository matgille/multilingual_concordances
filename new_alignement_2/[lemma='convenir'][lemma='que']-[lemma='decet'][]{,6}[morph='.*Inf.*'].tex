\begin{tabular}{|p{1cm}|p{6.5cm}|p{6.5cm}|}

\hline
2.2.7 & Lo primero paresce assi . \textbf{ Ca conuiene que los que quieren aprinder sçiençia de letris } que aprendan pronunçiar departidamente las palabras delas letras ¶ & quae est ex scientia acquirenda . \textbf{ Decet enim volentes literas discere , } literales sermones scire distincte proferre . \\\hline
2.3.15 & e el amor de bien los inclina asuir . \textbf{ Conuiene que los prinçipes se ayan çerca ellos } assi commo cerca de fijos . & quos virtus et amor boni inclinat ad seruiendum , \textbf{ decet principantes se habere quasi ad filios , } et decet eos regere non regimine seruili , \\\hline
3.3.1 & Et todas estas tres sabidurias \textbf{ conuiene que aya el Rey . } Conuiene a saber . & et gubernare ciues . \textbf{ Omnes autem tres prudentias decet habere Regem , } videlicet particularem , oeconomicam et regnatiuam . \\\hline

\end{tabular}
