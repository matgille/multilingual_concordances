\begin{tabular}{|p{1cm}|p{6.5cm}|p{6.5cm}|}

\hline
1.2.2 & et quomodo esse habent in intellectu et \textbf{ Ostenso , } quod nec in potentiis naturalibus , & que en estos tales sean puestas las uirtudes morales \textbf{ ues ya es mostrado } que las uirtudes ¶ morales non son en los poderios naturales \\\hline
1.2.19 & circa personas dignas , \textbf{ et circa seipsum . Ostenso quid est magnificentia , } et circa quae habet esse : & fechas cerca las personas dignas \textbf{ e çerca la su persona miłma ¶ | Mostrado que cosa es la magnificençia } e cerca quales cosas ha de ser ligeramente \\\hline
1.2.22 & et parum bene fortunatum extolli . \textbf{ Ostenso quid est magnanimitas , } et circa quae habet esse . & e en poto bien auenturado se leunata en vana eglesia \textbf{ ¶Mostrado que cosa es la magnan midat } e cerca quales cosas ha de ser sinca de demostrar \\\hline
1.2.29 & propter onerosas esse superabundantias . \textbf{ Ostenso quid est veritas } de qua loquimur , & por la sobrepuiança de carga . \textbf{ ¶ Visto que cosa es la uerdat } dela qual aqui fablamos \\\hline
1.3.3 & esse debet quid amandum . \textbf{ Ostenso ergo quomodo Reges et Principes } quodam speciali modo prae aliis debent & que cosa ha de amar \textbf{ Et por ende mostrado | que los Reyes et los prinçipes } por alguna manera especial sobre todos los otros deuen amar el bien diuinal \\\hline
2.1.20 & uxorem propriam honorifice pertractare . \textbf{ Ostenso , quomodo decet viros suis uxoribus moderate et discrete } uti , & segunt su estado de tractar muy honrradamente a su muger . \textbf{ ¶ Mostrado en qual manera conuiene alos maridos usen de sus mugers } sabiamente e tenprada mente . \\\hline
2.2.10 & quod ab infantia assueuit . \textbf{ Ostenso , quomodo instruendi sunt } iuuenes quantum ad loquelam , et visionem : & en que es acostunbrado de su moçedat . \textbf{ ¶ Mostrado en qual manera son de enssennar los moços } e los mançebos quento ala fabla \\\hline
2.2.13 & supra , cum egimus de regimine coniugali , diffusius diximus . \textbf{ Ostenso , } quomodo iuuenes debent esse abstinentes in cibo , & quando dixiemos del gouernamiento del casamiento \textbf{ ostrado en qual manera los as . moços deuen ser guardados enla vianda } e mesurados enel beuer \\\hline
2.2.17 & vel usu propriae coniugis sint contenti . \textbf{ Ostenso , quomodo in iuuenibus a quartodecimo anno ultra est bene disponendum corpus , } et rectificandus appetitus : & e sean continentes e pagados de sus mugers propraas ¶ \textbf{ Mostrado en qual manera es de ordenar bien el cuerpo } e enderescar el appetito en los mançebos de los xiiij̊ \\\hline
2.2.21 & circa quae opera deceat foeminas esse intentas . \textbf{ Ostenso , } quod non decet puellas esse vagabundas , & que las mugers fuesen acuçiosas . \textbf{ ostrado que non conuiene alas moças de andar uagarosas a quande e allende } nin les conuiene de beuir ociosas \\\hline
2.3.19 & sed ne regia dignitas contemnatur . \textbf{ Ostenso qualiter ministris sunt officia committenda , } et quomodo sunt erga eos solicitandi , & por que non sea despreçiada la dignidat real \textbf{ mostrado en qual manera se deuen dar los ofiçios a los seruientes } e en qual manera deuen ser acuçiosos çerca ellos \\\hline
3.2.8 & plenius ostendetur . \textbf{ Ostenso quomodo Reges et Principes solicitari debent , } ut populus sibi commissus habeat & sera mostrado mas conplida mente . \textbf{ Mostrado en qual manera los Reyes | e los prinçipes deuen ser acuçiosos } por que el pueblo que le es a comnedado aya aquellas cosas \\\hline
3.2.32 & ut sub Rege . \textbf{ Ostenso quid est ciuitas , } et quid regnum : & assi commo son vn Rey . \textbf{ Mostrado que cosa es la çibdat } e que cosaes regno de ligero puede paresçer \\\hline
3.3.8 & quod ipsum oporteat facere . \textbf{ Ostenso utile esse castra construere , } et qualiter etiam praesentibus hostibus construenda sint castra : & e manden a cada vno qual cosa deua fazer . \textbf{ Mostrado que prouechosa cosa es de fazer los castiellos . } avn en qual manera los enemigos presentes son de fazer los castiellos \\\hline
3.3.13 & aciem titubare , et deficere . \textbf{ Ostenso qualiter sunt acies ordinandae et construendae , } reliquum est ostendere , & e mas ayna puede fallesçer . \textbf{ m mostrado en qual manera son de establesçer } e de ordenar las azes fincanos de mostrar en qual manera los lidiadores deuen ferir \\\hline
3.3.15 & nullum possent nocumentum efficere . \textbf{ Ostenso itaque qualiter debeant stare pugnantes , } si velint hostes percutere , & non les pudiessen fazer ningun enpeesçimiento . \textbf{ Pues que assi es mostrado | en qual manera deuen estar los peones lidiadores } si quisieren ferir los enemigos \\\hline
3.3.16 & in sequenti capitulo ostendetur . \textbf{ Ostenso quot sunt genera bellorum , } et quot modis deuincendae sunt munitiones obsessae : & mostrar lo hemos en el capitulo que se sigue . \textbf{ Mostrado quantas son las maneras de las batallas } e en quantas maneras son de vençer \\\hline
3.3.21 & quod non possint viriliter resistere obsidentibus . \textbf{ Ostenso quomodo sunt remedia adhibenda contra famem , } et sitim per quae obsessa munitio deuinci consueuit : & por que beuiendo agua sola los lidiadores enflaquesçerse yan en tanto que non podrian defenderse de los enemigos . \textbf{ Mostrado quales remedios se deuen tomar | contra la fanbre e contra la sed . } Por las quales cosas las fortalezas cercadas se suelen tomar \\\hline
3.3.22 & prudentis iudicio relinquantur . \textbf{ Ostenso quomodo resistendum sit cuniculis , et lapidariis machinis : } reliquum est declarare , & e non las puede omne conplidamente contar dexamoslas a iuyzio de omnes sabios . \textbf{ Mostrado en qual manera nos podemos defender | de las cueuas coneieras } e de los engeñios \\\hline
3.3.23 & quas tradidimus erga nauale bellum . \textbf{ Ostenso qualiter incidenda sunt ligna } ex quibus construenda est nauis , & cerca las batallas de las naues . \textbf{ Mostrado en qual manera es de taiar la madera } para fazer las naues \\\hline

\end{tabular}
