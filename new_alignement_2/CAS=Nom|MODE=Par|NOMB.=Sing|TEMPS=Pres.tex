\begin{tabular}{|p{1cm}|p{6.5cm}|p{6.5cm}|}

\hline
1.1.2 & et bonis operibus regulatis ordine rationis : \textbf{ volens tractare de regimine sui , } oportet ipsum notitiam tradere de omnibus his & por orden de Razon \textbf{ el que quiere tractar del gouernaiento } e fablarde sy mesmo conujene de tractar \\\hline
1.1.3 & et dominus aliorum . \textbf{ Nam vigens prudentia , } et aliis virtutibus moralibus , & que sea fecho gouernador e senonr de los otros \textbf{ Ca el que ha sabiduria } e las otras uirtudes morales \\\hline
1.1.5 & unde in eodem 3 dicitur , \textbf{ quod nullus est beatus nisi volens . } Sed quae non agimus ex electione , & en el terçero libro de las ethicas \textbf{ que njguon non es bien auer turado | si non obra de voluntad } Mas aquellas cosas que nos non fazemos por \\\hline
1.1.5 & habitus electiuus \textbf{ in medietate consistens , } ut dicitur 2 Ethic . ) & que muestra a omne escoger . \textbf{ Et esta sabiduria esta en medio delas buenas obras } Ca asy lo praeua el philosofo \\\hline
1.1.5 & propter quod sicut \textbf{ si non esset agens primum , } nullum agens ageret : & por la qual cosa \textbf{ asi commo si el primero mouedor non fuese ninguno otro non seria mouedor } asi si la fin postrimera non fuese \\\hline
1.1.5 & si non esset agens primum , \textbf{ nullum agens ageret : } sic si non esset finis ultimus , & por la qual cosa \textbf{ asi commo si el primero mouedor non fuese ninguno otro non seria mouedor } asi si la fin postrimera non fuese \\\hline
1.1.5 & Unde Philosophus 1 Ethicor’ \textbf{ volens ostendere } necessariam esse praecognitionem finis , ait , & mas es por auentra \textbf{ a¶ Onde el philosofo quariendo mostrar en el primero libro delas ethicas } que es neçesario de connosçer ante la fin \\\hline
1.1.5 & intendere bonum gentis et commune , \textbf{ quod est magis expediens et diuinius , } quam bonum aliquod singulare . & e al bien comun \textbf{ que es mas conuenible | e mas diuinal } que ningun bien singłar njn personal ¶ \\\hline
1.1.6 & Unde Philosophus 1 Ethicorum \textbf{ describens felicitatem , } ait , & en el primero libro delas ethicas \textbf{ quariendo mostrar | que cosa es la fe liçidat } e la bien andança . \\\hline
1.1.6 & ut in seipso quilibet experiri potest . \textbf{ unde Philosophus 3 Ethi’ loquens de talibus delectationibus ait , } quod insatiabilis est delectabilis appetitus . & asi commo cada vno prueba en si mesmo . \textbf{ ¶ Onde el philosofo en el terçero libro delas ethicas fablando de tales delectaçonnes dize } que el apetito delectable de los sesos \\\hline
1.1.6 & et magnitudine bonitatis . \textbf{ In tanto ergo gradu existens , } indignum est , & e en grandeza de bondat ¶ \textbf{ pues que asi es el que esta en tan alto grado } non deue escoger uida de bestia . \\\hline
1.1.6 & nec qui uigil , \textbf{ sed qui dormiens , } decet Regiam maiestatem & e ueladoͬ e espierto . \textbf{ Mas quando es dormidor } e enbriago es de menospreçiar \\\hline
1.1.7 & nec etiam potest esse Magnanimus , \textbf{ quia metuens pecuniam perdere , } nihil magnum attentabit . & nin de grant coraçon . \textbf{ Ca temiendo deꝑder los des e las riquezas nunca acometra grandes cosas } Et la razon es esta \\\hline
1.1.7 & Cum ergo finis maxime diligatur , \textbf{ ponens suam felicitatem in numismate , } principaliter intendit reseruare sibi , & a aquel que pone las un feliçidat \textbf{ e la su bien andança en las riquezas | e en los aueres } prinçipalmente entiende de thesaurizar e fazer thesoro e llegar muchos dineros \\\hline
1.1.8 & suam felicitatem in honoribus ponere . \textbf{ Tertio hoc est indicens ei , } ne sit iniustus et inaequalis : & nin se ensoƀuezca mucho nol conuiene de poner su bien andança en las honrras ¶ \textbf{ Lo terçero se demuestra | assi porque non conuiene al prinçipe en ninguna manera } que sea iniusto nin desegual ¶ \\\hline
1.1.10 & in quo suam felicitatem ponit : \textbf{ ponens ergo suam felicitatem } in ciuili potentia , & e toda su bien andança . \textbf{ Et por ende aquel que pone su feliçidat | e su bien andança } en poderio çiuil \\\hline
1.1.10 & et incurret nocumentum \textbf{ secundum animam . Propter quod Philosophus 7 Politicorum vituperans Lacedaemones , } ponentes felicitatem & Et contesçer le ha muy grant daño segunt su alma . \textbf{ por la qual cosa dize el philosofo | en el septimo libro delas politicas } deno stando alos griegos \\\hline
1.1.12 & Nec etiam voluit esse ponendam eam in habitibus , \textbf{ quia habens habitum , } et non operans , & que son en el alma \textbf{ que el que ha las scians } e non obra segunt el lassemeia a aquel que duerme . \\\hline
1.1.12 & quia habens habitum , \textbf{ et non operans , } quasi assimilatur dormienti : & que el que ha las scians \textbf{ e non obra segunt el lassemeia a aquel que duerme . } Ca mientre los omes duermen \\\hline
1.1.12 & quod est multitudinis rector : \textbf{ nam regens multitudinem debet } intendere commune bonum . & es por que es gouernador de mucho \textbf{ Ca el que gouienna a muchos deue tener } mientesal bien comun de todos . \\\hline
1.1.13 & qualis homo sit , \textbf{ cum in principatu existens , in quo potest bene et male facere , } cogitat qualiter se habeat . & por que estonçe paresçe qual es el omne \textbf{ quando es puesto en señorio | en que pueda fazer bien e mal . } Et aquella hora entiendan los omes \\\hline
1.2.1 & vel subtiliter audit , \textbf{ nisi forte hoc esset per accidens , } ut si quis ex superflua comestione , & o oye sotil mente . saluo ende \textbf{ si por auentura esto fuese | por algun açidente } assi commo alguon \\\hline
1.2.2 & et quia quanto aliquid sub uniuersaliori modo accipitur , \textbf{ tanto unum et idem existens } ad plura se extendit , & que el appetito del seso . \textbf{ Et por que cada cosa quanto es tomada so manera mas general tanto essa misma cosa se } estieñde amas cosas \\\hline
1.2.2 & appetitus tamen intellectiuus unus , \textbf{ et idem existens fertur } in omne bonum intelligibile . & Enpero el appetito del entendimiento seyendo vno \textbf{ e esse mismo ua a todo bien que se puede entender ¶ pues que assi es non es otro } nin departido el appetito del entendimiento \\\hline
1.2.3 & sed sit Eutrapelus \textbf{ et bene se vertens , } ut se habeat circa ludos & Mas sea buen conpanon \textbf{ e sepa bien beuir con los omes } assi commo conuiene en los trebeios . \\\hline
1.2.6 & quod prudentia est virtus \textbf{ secundum uniuersales maximas particularia facta concernens . } Huiusmodi autem uniuersales regulae & diziendo que la pradençia es uirtud \textbf{ que iudga alos negoçios particulares seg̃t las reglas vniuerssales . } las quales reglas generales son buenas leyes e buenas costunbres . \\\hline
1.2.6 & quam est recta ratio agibilium , \textbf{ praesupponens rectitudinem voluntatis . } Ex omnibus ergo his , & que auemos de fazer \textbf{ que requiere e demanda reglamiento de uoluntad ¶ } Et pues que assi es de todas estas cosas sobredichas podemos tomar vna \\\hline
1.2.6 & et iudicata secundum uniuersales maximas , \textbf{ particularia contingentia agibilia concernens , } praesupponens rectitudinem voluntatis . & e iudgandas \textbf{ que cata sienpre las obras particulares | que pue den contesçer segunt las reglas uniuerssales } e que demanda endereçamiento e reglamiento de uoluntad \\\hline
1.2.6 & particularia contingentia agibilia concernens , \textbf{ praesupponens rectitudinem voluntatis . } Viso quid est prudentia , & que pue den contesçer segunt las reglas uniuerssales \textbf{ e que demanda endereçamiento e reglamiento de uoluntad } isto que cosa es la prudençia \\\hline
1.2.7 & cuiusmodi sunt diuitiae et bona exteriora ; \textbf{ prudentia carens , } quia non cognoscet , & e estos otros bienes de fuera \textbf{ que son del cuerpo . | Et el Rey sin sabiduria } por que non conosçe nin faz cuenta \\\hline
1.2.7 & et illud natureliter dominatur , \textbf{ semper principans pollet prudentia , a qua deficit } qui naturaliter seruus existit . & Et do quier que ay sabiduria ha naturalmente sennorio . \textbf{ Por ende sienpre el prinçipe deue auer sabiduria . | por la qual sera sennor } por cuya mengua el sieruo esta naturalmente en su seruidunbre . \\\hline
1.2.8 & ratione propriae personae \textbf{ quae alios est dirigens , oportet quod sit solers , et docilis : } ratione vero gentis quam dirigit , & por razon de la su propia persona \textbf{ que ha de guiar los otros . | Conuiene le de ser sotil e doctrinable ¶ } Mas por razon dela gente \\\hline
1.2.8 & siue oportet \textbf{ quod sit intelligens et rationale . } Modus enim , & Conuiene le que aya entendimiento \textbf{ e razon o conuiene le que sea entendido e razonable . } Ca la manera por que el Rey guia el su pueblo \\\hline
1.2.8 & qui est inditus hominibus , \textbf{ volens alios dirigere , } oportet quod sit intelligens , & çer que es enxerida naturalmente alos omes . \textbf{ El que quiere alos otros guiar } conuiene le que sea entendido \\\hline
1.2.8 & volens alios dirigere , \textbf{ oportet quod sit intelligens , } cognoscendo principia , & El que quiere alos otros guiar \textbf{ conuiene le que sea entendido } conosciendo los prinçipios e las razones . \\\hline
1.2.8 & ex illis praemissis cunclusiones intentas . \textbf{ Vel oportet quod sit intelligens , } sciendo leges , & e las razones que quiere ençerrar ¶ \textbf{ Et otrosi conuiene al Rey | que sea entendido } e sabio sabiendo las leys \\\hline
1.2.8 & Sed ratione propriae personae \textbf{ quae est alios dirigens , } oportet quod sit solers , et docilis . & Mas por razon dela su persona propia \textbf{ que es tal que ha de gouernar los otros . } Conuiene le de sor sotil e doctrinable . \\\hline
1.2.8 & Patet ergo quod ratione propriae personae \textbf{ quae est alios dirigens , } oportet Regem esse solertem , et docilem . & que por razon dela su propia persona \textbf{ que es gouernandor de los otros . } Conuiene al rey de ser engennioso e doctrinable . \\\hline
1.2.9 & sicut intellectus principiorum est . \textbf{ Tanto ergo Rex magis intelligens est } circa agibilia , & e delas costunbres les faze auer manera \textbf{ para bien gouernar . | Et por ende tanto deue el Rey ser mas acuçioso cerca } lo que deue fazer \\\hline
1.2.9 & Verum quia malitia est corruptiua principii . \textbf{ Sicut enim quis habens corruptum gustum , } male iudicat de saporibus , & corronpadera dela razon e del comienco para obrar . \textbf{ Ca assi commo aquel que ha el gosto } corronpido mal iudga delos sabores . \\\hline
1.2.9 & et e conuerso : \textbf{ sic habens infectam , } et deprauatam voluntatem , excoecatur in intellectu , & Et lo que es amargo que es dulçe Bien \textbf{ assi aquel que ha corrupta e desordenada la uoluntad } por maliçia es ciego en el entendimiento \\\hline
1.2.10 & fortis est , \textbf{ et agens temperata , } quia delectatur in ipsis , temperatus est . & en quanto se delecta en ellas es dicho fuerte . \textbf{ Et el que faze las obras tenpradas } en quanto se delecta en ellas es dicho tenprado . \\\hline
1.2.10 & quia delectatur in ipsis , temperatus est . \textbf{ Sed agens talia , } non quia delectatur in eis , & en quanto se delecta en ellas es dicho tenprado . \textbf{ Mas aquel que faze estas obras } non en quanto se deleyte en ellas \\\hline
1.2.12 & manifeste Regi et Regno infertur iniuria . \textbf{ Satis per praecedens capitulum } persuadetur Principibus atque Regibus , & su gregnos \textbf{ poro quier que assaz es ya prouado | por el capitulo sobredicho } que conuiene alos prinçipes e alos Reyes \\\hline
1.2.13 & Erit igitur Fortitudo virtus \textbf{ quaedam reprimens timores , } et moderans audacias . & Et por ende la fortaleza es vna uirtud \textbf{ que repreme los temores } e tienpra las osadias . \\\hline
1.2.13 & quaedam reprimens timores , \textbf{ et moderans audacias . } Reprimit enim Fortitudo timores , & que repreme los temores \textbf{ e tienpra las osadias . } Ca la fortaleza repreme los temores \\\hline
1.2.13 & Non enim sic per fugam vitare possumus aegritudines : \textbf{ quia cum aegritudo sit aliquid in nobis existens , } per fugam eam vitare non possumus . & assi por foyr escapar las enfermedades \textbf{ por que la enfermedat es alguna cosa | que esta en nos } e por foyr non la podemos escusar . \\\hline
1.2.13 & debilioris est . \textbf{ Aggrediens autem comparatur ad alios , sicut ad debiliores sed sustinens , } sicut ad fortiores : & por que el que acomete es conparado a aquellos a quien acomete \textbf{ assi commo a omes mas flacos . | Mas el que sufre es conparado alos otros } assi commo amas fuertes . \\\hline
1.2.13 & Secundo est difficilius , \textbf{ quia aggrediens imaginatur malum ut futurum : } sed sustinens habet malum prae oculis , & Lo segundo esto es mas guaue por que aquel que acomete \textbf{ ymagina el mal | assi commo cosa que ha de venir . } Mas el que sufre ha el mal ante los oios \\\hline
1.2.13 & quia aggrediens imaginatur malum ut futurum : \textbf{ sed sustinens habet malum prae oculis , } et ut praesens . & assi commo cosa que ha de venir . \textbf{ Mas el que sufre ha el mal ante los oios } e assi conmo presente . \\\hline
1.2.13 & quid est Fortitudo : \textbf{ quia est virtus reprimens timores , } et moderans audacias . & que cosa es la fortaleza \textbf{ ca es uirtud | que te pree me los temores } e refrena las osadias . \\\hline
1.2.13 & quia est virtus reprimens timores , \textbf{ et moderans audacias . } Rursus manifestum est , & que te pree me los temores \textbf{ e refrena las osadias . } Otrosi ya es mostrado en quales cosas es la fortaleza . \\\hline
1.2.14 & Fortitudo enim ciuilis est , \textbf{ quando aliquis timens verecundiam , } et volens honorem adipisci , & ¶La fortaleza çeuiles \textbf{ quando alguno temiendo uerguença } e quariendo ganar honrra \\\hline
1.2.14 & quando aliquis timens verecundiam , \textbf{ et volens honorem adipisci , } aggreditur aliquod terribile , & quando alguno temiendo uerguença \textbf{ e quariendo ganar honrra } acomete alguna cosa fuerte e espantable . \\\hline
1.2.14 & Hector fortis erat , \textbf{ qui timens increpationes Polydamantis , } aggrediebatur terribilia . & enxienplo que ector era fuerte en esta manera \textbf{ que temie ser denostado de polimas . | Et por ende } acometie cosas espatables \\\hline
1.2.14 & si non strenue bellaret , \textbf{ Hector laudans se in Troianis diceret } Diomedem ab eo deuictum esse . & Ca dize que si non lidiase reziamente su contrario ector \textbf{ alabandose entre los troyanos } dirie que diomedes era flaco \\\hline
1.2.14 & quis pugnam aggrediatur , \textbf{ inueniens resistentiam , } pugnam non sustinet ; & Mas si por la sana alguno acomete la batal la \textbf{ quando falla resistençia } e fortaleza non sufre la batalla . \\\hline
1.2.14 & Sexta fortitudo dicitur esse bestialis , \textbf{ ut cum aliquis ignorans fortitudinem aduersarii , bellatur . } Ut puta si habitantes in septentrione sunt fortes , et audaces , & La sexta fortaleza es testial e de bestia \textbf{ assi commo quando alguno comiença de lidiar non sabiendo | nin conosçiendo la fortaleza del su contrario . } Enxient lo desto . \\\hline
1.2.15 & Nam sicut fortitudo est \textbf{ reprimens timores , } et moderans audacias , & Ca assi commo la fortaleza \textbf{ reprime los temotes } e refrena las osadias \\\hline
1.2.15 & reprimens timores , \textbf{ et moderans audacias , } inter quas ipsa fortitudo habet esse : & reprime los temotes \textbf{ e refrena las osadias } entre las quales cosas ha de ser la fortaleza . \\\hline
1.2.15 & inter quas ipsa fortitudo habet esse : \textbf{ sic temperantia est reprimens } delectationes sensibiles , & entre las quales cosas ha de ser la fortaleza . \textbf{ assi la tenpranca } repreme las delecta connes corporales \\\hline
1.2.15 & delectationes sensibiles , \textbf{ et moderans insensibilitates , } inter quas habet esse . & repreme las delecta connes corporales \textbf{ e resten a los non sentimientos . } Et pues que assi es si la tenpranca reprime \\\hline
1.2.15 & Si autem est circa delectationes aliorum sensuum , \textbf{ hoc est per accidens : } quia aliis sensus per accidens percipiunt & Mas si ella ha de seer çerca las \textbf{ delectaconnes de los otros sesos es | por algun accidente } por razon que los otros sesos \\\hline
1.2.15 & hoc est per accidens : \textbf{ quia aliis sensus per accidens percipiunt } delectabilia gustus , et tactus . & por algun accidente \textbf{ por razon que los otros sesos | por algun accidente resçiben } delecta connes del gostar e del tanner . \\\hline
1.2.15 & in aliis vero sensibus \textbf{ delectantur per accidens . } Propter quod in eodem libro dicitur , & Et en los otros sesos se delectan \textbf{ por algun accidente . } por la qual razon dize el philosofo en esse milmo terçero \\\hline
1.2.15 & et ex consequenti circa gustum : \textbf{ per accidens autem est } circa delectabilia aliorum sensuum . & e desi cerca del gusto \textbf{ Mas por algun accidente es cerca las cosas delectables } de los otros sesos . \\\hline
1.2.15 & quid est Temperantia : \textbf{ quia est virtus reprimens } delectationes sensibiles , & que cosa es la tenpranca . \textbf{ Ca es uirtud que repreme las delectaçonnes sensibles de los cinco sesos } e tienpra los non sentimientos \\\hline
1.2.15 & delectationes sensibiles , \textbf{ et moderans insensibilitates . } Secundo vero declarabatur circa quae habet esse : & Ca es uirtud que repreme las delectaçonnes sensibles de los cinco sesos \textbf{ e tienpra los non sentimientos } ¶L segundo auemos declarado cerca quales cosas ha descer la tenpranca . \\\hline
1.2.15 & ex consequenti circa gustum : \textbf{ per accidens vero } circa alios sensus . & e desi çerca el gusto \textbf{ mas por algun acçidente ha de ser cerca los otros sesos ¶ } Lo terçero auemos mostrado \\\hline
1.2.16 & et experiri bellum , sine periculo non potest . \textbf{ Valde est ergo increpandus carens tempesantia , } cum eam sine periculo possit acquirere : & e puar las batallas non se puede fazer sin periglo . \textbf{ Et pues que assi es mucho es de denostar el | que non ha } tenpranca \\\hline
1.2.16 & non autem adeo increpandus est \textbf{ carens fortitudine , } quia virtus illa est difficilis , & Mas non es tanto de denostas el \textbf{ que non ha fortaleza } por que aquella uirtud dela fortaleza es mas guaue \\\hline
1.2.16 & et fideliter , \textbf{ Rex ille volens complacere illi Duci , } praecepit quod duceretur ad ipsum . & Et acaesçio que vn prinçipe mucho su priuado que grant t p̃o le auia seruido e fiel mente . \textbf{ Et el Rey que tiendo fazer plazer a aquel } prinçipe mando qual pusiessen dentro ante si . \\\hline
1.2.16 & Dux autem ille assuetus rebus bellicis , \textbf{ videns Regem suum esse totum muliebrem et bestialem , } statim ipsum habuit in contemptum : & veyendo \textbf{ que el su Rey era todo mugeril | e toda su } conuerssaçion era entre mugers e era bestial . \\\hline
1.2.16 & voluit eum inuadere . \textbf{ Rex autem timens , fugit : } et quia credebat se non posse fugere manus illius Ducis , & e quaso yr contra el para lo matar . \textbf{ Et el Rey temiendo lo fuyo . } Et por que creya que non podia foyr delas manos \\\hline
1.2.17 & quia est media \textbf{ inter timores et audacias , ideo est virtus reprimens timores , } et moderans audacias : & e por E ende es uirtud \textbf{ que repreme los temo eres } e tienpra las osadias . \\\hline
1.2.17 & inter timores et audacias , ideo est virtus reprimens timores , \textbf{ et moderans audacias : } ita liberalitas & que repreme los temo eres \textbf{ e tienpra las osadias . } Assi la fre anqueza es medianera \\\hline
1.2.17 & quia est media inter auaritias et prodigalitates , \textbf{ ideo est virtus reprimens auaritias , } et moderans prodigalitates . & entre las auariçias e los gastamientos \textbf{ Et por ende es uirtud | que repreme las auariçias } e tienpra los gastamientos . Et esta uirtud esta en bien vsar del auer e de los dineros . \\\hline
1.2.17 & ideo est virtus reprimens auaritias , \textbf{ et moderans prodigalitates . } Consistit autem haec virtus & que repreme las auariçias \textbf{ e tienpra los gastamientos . Et esta uirtud esta en bien vsar del auer e de los dineros . } Mas para bien usar del auer \\\hline
1.2.17 & est ex consequenti . \textbf{ Usurpans enim bona utilia , } et non accipiens ea sicut debet , & Et despues es en non tomar nin forcar los bienes agenos . \textbf{ Ca aquel que vsurpa e toma los bienes prouechosos agenos malamente commo non deue . | este paresçe que es muy cobdicioso de auer ¶ } Et por esso el philosofo \\\hline
1.2.17 & Usurpans enim bona utilia , \textbf{ et non accipiens ea sicut debet , } nimis videtur auidus pecuniae . Propter quod Philosophus 4 Ethic’ usurarios , lenones , & este paresçe que es muy cobdicioso de auer ¶ \textbf{ Et por esso el philosofo | en el quarto libro delas ethicas llama a estos tales non liberales } que quiere dezir non francos \\\hline
1.2.17 & et illud magis . \textbf{ Si enim liberalis conseruans proprios redditus , } et accipiens unde debet , & por que lo es . es dicha mastal . \textbf{ Et por ende si alguno es liberal e franco en guardando las sus rentas propreas } e tomando onde deue esto \\\hline
1.2.17 & Si enim liberalis conseruans proprios redditus , \textbf{ et accipiens unde debet , } hoc ideo facit , & Et por ende si alguno es liberal e franco en guardando las sus rentas propreas \textbf{ e tomando onde deue esto } por tanto lo faze por que pueda fazer espessas quales deuede sus rentas prop̃as \\\hline
1.2.17 & Custodiens vero proprios redditus , \textbf{ et accipiens pecuniam a propriis possessionibus , } bene patitur siue bene recipit & Mas el que guarda las rentas propias e lo suyo propio \textbf{ e toma el auer delas sus possesiones propias } este sufre bien o resçibe bien \\\hline
1.2.17 & eo modo quo debet . \textbf{ Non usurpans autem aliena , } non operatur turpia . & en aquella manera que deue . \textbf{ Mas aquel que non vsurpa | nin toma las cosas agenas } este non obra mal \\\hline
1.2.17 & secundum se difficultatem habet : \textbf{ quia propria bona sunt aliquid ad nos pertinens , } et naturaliter afficimur ad illa . Immo auari adeo afficiuntur & Mas dar los sus biens propios ha alguna guaueza por si . \textbf{ Ca los bienes propios son | cosaque parte nesçen a nos mismos } e naturalmente amamos lo que pertenesçe anos . \\\hline
1.2.17 & Viso quid est liberalitas , \textbf{ quia est virtus reprimens auaritias , } et moderans prodigalitates . & visto que cosa es la franqueza . \textbf{ Ca es uirtud | que te preme las auariçias } e que tienpra los gastamientos \\\hline
1.2.17 & quia est virtus reprimens auaritias , \textbf{ et moderans prodigalitates . } Et ostenso circa quae habet esse , & que te preme las auariçias \textbf{ e que tienpra los gastamientos | e las superfluydades delas espenssas . } Et mostrado en quales cosas ha de ser la franqueza . \\\hline
1.2.19 & Dicitur enim magnificus , \textbf{ quasi magna faciens . } Inde est ergo , & que faze es nonbrada la magnificençia \textbf{ ca es dichon magnifico aquel que faze grandes cosas } Et por ende por que en qual quier estado \\\hline
1.2.19 & sic liberalitas est circa magnos sumptus , \textbf{ si faciens eos , } multas habeat facultates : & si assi la libalidat es certa grandes espenssas \textbf{ si el que las faze ouiere muchas riquezas } e ahun sera cerca espenssas mesuradas si el que faze las espenssas \\\hline
1.2.19 & et etiam circa mediocres , \textbf{ si faciens sumptus illos , } mediocriter facultatibus abundet . & si el que las faze ouiere muchas riquezas \textbf{ e ahun sera cerca espenssas mesuradas si el que faze las espenssas } mesuradosmente abondare en las riquezas . \\\hline
1.2.19 & Magnificentia vero , \textbf{ respiciens sumptus } ut sunt decentes operibus , & guaue cosa es dese auer bien los omes en ellas . \textbf{ Mas la magnificençia que cata alas espenssas en quanto son conuenbles alas obras non cata } nin tiene oio quales sean las obras . \\\hline
1.2.19 & inter auaritiam , et prodigalitatem . \textbf{ Patet ergo quid est magnificentia . Nam sicut liberalitas est reprimens auaritias , } et moderans pro digalitates : & Pues que assi es desto paresçe \textbf{ que cosa es la magnificençia . | ca assi commo la liberalidat } repreme las auariçias \\\hline
1.2.19 & Patet ergo quid est magnificentia . Nam sicut liberalitas est reprimens auaritias , \textbf{ et moderans pro digalitates : } sic magnificentia est reprimens paruificentias , & ca assi commo la liberalidat \textbf{ repreme las auariçias | et tienpra los gastamientos } assi la magnificençia \\\hline
1.2.19 & et moderans pro digalitates : \textbf{ sic magnificentia est reprimens paruificentias , } et moderans consumptiones . & et tienpra los gastamientos \textbf{ assi la magnificençia | repreme las paruifiçençias e las pequenas espenssas } e tienpra los consumimientos e destruymientos . \\\hline
1.2.19 & sic magnificentia est reprimens paruificentias , \textbf{ et moderans consumptiones . } Et sicut liberalitas est faciens sumptus , & repreme las paruifiçençias e las pequenas espenssas \textbf{ e tienpra los consumimientos e destruymientos . } Et assi commo la libalidat faze espenssas \\\hline
1.2.19 & et moderans consumptiones . \textbf{ Et sicut liberalitas est faciens sumptus , } et dationes proportionatas facultatibus : & e tienpra los consumimientos e destruymientos . \textbf{ Et assi commo la libalidat faze espenssas } e da dones conuenibles a las riquezas \\\hline
1.2.19 & et dationes proportionatas facultatibus : \textbf{ sic magnificentia est faciens sumptus decentes magnis operibus . } Viso quid est magnificentia : & e da dones conuenibles a las riquezas \textbf{ assi la magnificençia faze espenssas conuenibles alas grandes obras ¶ } visto que cosa es la magnificençia \\\hline
1.2.20 & quod quaecunque facit paruificus , \textbf{ semper facit tardans . } Videtur enim ei , & que faze el paruifico \textbf{ sienpre las faze tardando . } Ca paresçe leal paruifico \\\hline
1.2.22 & Sicut igitur circa ipsa bona utilia est duplex virtus \textbf{ una respiciens magnos sumptus , } ut magnificentia , & Por enl de assi commo çerca los bienes aprouechosos son dos uirtudes . \textbf{ La vna que cata alas grandes espenssas } assi commo es la magnificençia . \\\hline
1.2.22 & ideo est virtus \textbf{ quaedam reprimens auaritias , } et moderans prodigalitates : & Por ende es dicha uirtud \textbf{ que repreme las auariçias } e tienpra los gastamientos en espender . \\\hline
1.2.22 & quaedam reprimens auaritias , \textbf{ et moderans prodigalitates : } sic magnanimitas , & que repreme las auariçias \textbf{ e tienpra los gastamientos en espender . } En essa misma manera la magnanimidat \\\hline
1.2.22 & et praesumptionem , \textbf{ est virtus quaedam reprimens pusillanimitates , } et moderans praesumptiones . & es dicha uirtud \textbf{ que repreme las pus illanimidades | que son flaquezas de coraçon } e tienpra las presunpçiones \\\hline
1.2.22 & est virtus quaedam reprimens pusillanimitates , \textbf{ et moderans praesumptiones . } Viso quid est magnanimitas , & que son flaquezas de coraçon \textbf{ e tienpra las presunpçiones | que son sobrepuiamientos } en cometer las grandes cosas \\\hline
1.2.24 & esse aliarum virtutum , et magnanimitatis . \textbf{ Nam agens opera fortitudinis , } et aggrediens pugnam , & Et pues que assi es essas mismas obras pueden ser delas otras uirtudes e dela magnanimidat . \textbf{ Ca aquel que faze las obras dela fortaleza } e acomete la batalla \\\hline
1.2.24 & Nam agens opera fortitudinis , \textbf{ et aggrediens pugnam , } si hoc facit , & Ca aquel que faze las obras dela fortaleza \textbf{ e acomete la batalla } si esto faze por que se delecta en tales obras \\\hline
1.2.24 & Nam cum honor inter exteriora bona sit bonum excellens , \textbf{ faciens opera virtutum , } inquantum sunt honore digna , & sea mas alto e meior bien \textbf{ el que faze obras de uirtudes } en quanto son diguas de honrra \\\hline
1.2.25 & aeque principaliter \textbf{ et per se non est eadem virtus retrahens et impellens : } nec est eadem virtus principaliter moderans passiones , & que vna uirtud egualmente \textbf{ e por li nos tire | e nos allegue a aquello } que dize la razon \\\hline
1.2.25 & et per se non est eadem virtus retrahens et impellens : \textbf{ nec est eadem virtus principaliter moderans passiones , } et impellens nos & que dize la razon \textbf{ nin es vna uirtud prinçipal | que tienpra las passiones } que nos allegan \\\hline
1.2.25 & nec est eadem virtus principaliter moderans passiones , \textbf{ et impellens nos } ab eo quod ratio vetat . & que tienpra las passiones \textbf{ que nos allegan } a aquello que la razon vieda e que nos tira de aquello que la razon manda . \\\hline
1.2.25 & Ideo humilis dicitur alios reuereri , \textbf{ quia considerans proprios defectus , } in rebus licitis et honestis alios reueretur . & que ha en reuerençia alos otros \textbf{ por que cuydando en los sus desfallesçimientos propios en las cosas conuenibles e honestas . } faze reuerençia alos otros ¶ \\\hline
1.2.25 & Humilitas vero principaliter moderat ipsam spem , \textbf{ ne aliquis nimis sperans de ipso bono , } ultra rationem prosequatur magnos honores . & Mas la humildat prinçipalmente tienpra la esꝑanca \textbf{ por que alguno auiendo grand esperança del bien } non vaya en pos grandes honrras \\\hline
1.2.25 & quia ille dicitur humilis , \textbf{ qui moderans spem ipsam } ad adipiscendum honores magnos , & por que aquel es dicho humildoso \textbf{ que tienpra la esperança de ganar grandes honrras } e va a ellas medianeramente . \\\hline
1.2.25 & cum virtute \textbf{ illa quam Philosophus distinguens a magnanimitate appellat eam honoris amatiuam ? } Non est praesentis speculationis . & sinplemente con aquella uirtud \textbf{ la qual el philosofo aparta dela magnanimidat | e llama la amadora de honrra . } Esto non es de esta presente arte \\\hline
1.2.26 & magnanimitas magis est \textbf{ virtus impellens in magna , } quam retrahens nos ab illis . & mas es uirtud \textbf{ quanos allega | e nos esfuerca a cosas grandes } que non retrayendo nos dellas . \\\hline
1.2.26 & virtus impellens in magna , \textbf{ quam retrahens nos ab illis . } Principalius ergo magnanimitas reprimit desperationem , & e nos esfuerca a cosas grandes \textbf{ que non retrayendo nos dellas . } Et pues que assi es mas prinçipalmente la magnanimidat repremela deses paracion \\\hline
1.2.26 & Nam ( ut patet ex dictis ) \textbf{ magnanimitas est virtus non impellens , } ne ratione difficultatis retrahamur , & Ca assi commo paresce delas cosas \textbf{ ya dichas lama granimidat es uirtud | que nos allegua } e nos esfuerca \\\hline
1.2.26 & Sic etiam ex dictis patet , \textbf{ quod humilitas est virtus nos retrahens , } ne ratione bonitatis et delectabilitatis , & assi commo ya dicho es paresçe \textbf{ que la humildat es uirtud | que nos tira } por que non siguamos las grandes honrras \\\hline
1.2.26 & Est enim hoc notabiliter attendendum , \textbf{ quod cum virtus magis sit retrahens quam impellens , } principaliter opponitur superabundantiae , & que quando la uirtud . \textbf{ mas nos trahe e tira | que nos allega e esfuerca . } Estonçe prinçipalmente \\\hline
1.2.26 & per quem retrahimur . \textbf{ Sed quum virtus magis est impellens quam retrahens , } e contrario se habet : & por el qual somos retraydos e tirados dellas \textbf{ mas quando la uirtud mas es allegante } e esforcante que retrayente nin tirante hase en manera contraria \\\hline
1.2.26 & Magnanimitas ergo , \textbf{ quia est virtus impellens in ardua bona , } magis opponitur pusillanimitati , & Et pues que assi es la magnanimidat \textbf{ por que es uirtud allegante | e esforcante nos alos bienes muy altos } mas es contraria ala pusillanmidat \\\hline
1.2.26 & magis opponitur pusillanimitati , \textbf{ quae est defectus nos retrahens } e talibus bonis , & mas es contraria ala pusillanmidat \textbf{ que es fallescemiento | que nos retraye } e tira de tales bienes \\\hline
1.2.26 & quam opponatur praesumptioni , \textbf{ quae est excellentia expellens nos in illa . } Sed humilitas e contrario , & que alla presup̃çion \textbf{ que es dela sobrepuiança | qua nos allegua a aquellos bienes } mas la humil dat faze todo lo contrario \\\hline
1.2.26 & Sed humilitas e contrario , \textbf{ quia est virtus retrahens , } ne ultra rationem & mas la humil dat faze todo lo contrario \textbf{ por que es uirtud | que nos retraye } por que non podamos seguat las sobrepuianças e las honrras . \\\hline
1.2.26 & Secundo decet eos esse humiles ratione operum fiendorum . \textbf{ Nam superbus quaerens suam excellentiam ultra quam debeat , } ut plurimum tendit & que han de fazer . \textbf{ Ca el sobra uio demandado | e quariendo su excellençia e sobrepuiamiento } mas que deue \\\hline
1.2.28 & Viso quid est amicabilitas , ut hic de ea loquimur , \textbf{ quia est uirtus reprimens litigia , } et moderans blanditias , & segunt que della fablamos aqui . \textbf{ ca es uirtud | que repreme las peleas } et tienpra los falages . \\\hline
1.2.28 & quia est uirtus reprimens litigia , \textbf{ et moderans blanditias , } ut potest haberi & que repreme las peleas \textbf{ et tienpra los falages . } assi commo el philosofo dize en el segundo \\\hline
1.2.29 & haec autem communi nomine Veritas nuncupatur . \textbf{ Patet ergo quid est veritas : quia est virtus moderans despectiones , } et reprimens iactantias . & Et pues que assi es paresçe \textbf{ que cosa es la uerdat | ca es uirtud } que tienpra los despreçiamientos \\\hline
1.2.29 & Patet ergo quid est veritas : quia est virtus moderans despectiones , \textbf{ et reprimens iactantias . } Viso quid est veritas , & ca es uirtud \textbf{ que tienpra los despreçiamientos } e repreme los alabamientos ¶visto que cosa es la uerdat finca de veer \\\hline
1.2.29 & ne homines sint aliis onerosi . \textbf{ Hanc autem rationem tangit Philosophus in eodem 4 Ethicorum dicens , } declinandum esse in minus & por que los omes non sean alos otros pesados e guaues . \textbf{ Et esta razon misma tanne el philosofo en esse mismo quarto libro delas ethicas } o dize que deuemos \\\hline
1.2.30 & et ut hic de ea loquimur , \textbf{ quia est reprimens } superfluitates ludi , & Et segunt que della aqui fablamos \textbf{ por que es uirtud } que repreme las superfluydades del iuego \\\hline
1.2.30 & superfluitates ludi , \textbf{ et moderans duritias . } Huiusmodi autem virtus habet esse & que repreme las superfluydades del iuego \textbf{ et tienpra la menguas e los fallesçimientos del . } Et pues que assi es esta tal uirtud ha de seer tan bien \\\hline
1.2.31 & habet perfectam prudentiam : \textbf{ sed habens perfectam prudentiam , } habet omnem virtutem : & ha acabadamente la pradençia e la sabiduria \textbf{ Mas aquel que ha acabadamente la pradençia } e la sabiduria ha todas las uirtudes morales \\\hline
1.2.31 & habet omnem virtutem : \textbf{ ergo habens aliquam virtutem , } habet omnem virtutem . & e la sabiduria ha todas las uirtudes morales \textbf{ Por ende aquel que ha alguna uirtud } ha todas las uirtudes . \\\hline
1.2.31 & non tamen oportet \textbf{ quod habens perfecte unam virtutem moralem , } habeat omnes virtutes morales . & Enpero non conuiene \textbf{ que aquel que ha vna uirtud moral } aya todas las uirtudes morales \\\hline
1.2.32 & et ipsum parabat in conuiuium , \textbf{ spondens quod quando vellet conuiuium facere , } ei suum filium tribueret . & e aprestaual para fazer el conbit \textbf{ et prometial que quando quisiesse fazer conbit } que el qual daria su fijo \\\hline
1.2.32 & appellatur a Philosopho heroica \textbf{ idest principans , et dominatiuat . } Ex hoc ergo manifeste patet , & es llamada del philosofo eroyca \textbf{ que quiere dezir prinçipante e sennor ante | por que es señora delas otras uirtudes } ¶ Et pues que assi es por esto paresçe manifiestamente que los Reyes \\\hline
1.2.32 & quod Reges , et Principes \textbf{ si debens recte dominari , } non sufficit eos fugere omnes gradus malorum , & ¶ Et pues que assi es por esto paresçe manifiestamente que los Reyes \textbf{ e los prinçipes si derechamente deuen ensseñorear } non les abasta a ellos de foyr \\\hline
1.2.32 & oportet quod habeant virtutem illam , \textbf{ quae est dominans } et principans respectu aliarum , & Conuienele que aya aquella uirtud \textbf{ que es sennora e prinçipante a todas las otras uirtudes . } la qual quando la ouieren seran buenos \\\hline
1.2.32 & quae est dominans \textbf{ et principans respectu aliarum , } et sint boni ultra modum aliorum , & Conuienele que aya aquella uirtud \textbf{ que es sennora e prinçipante a todas las otras uirtudes . } la qual quando la ouieren seran buenos \\\hline
1.2.33 & et cognoscat defectum suum , \textbf{ videns vitam et perfectionem principantis . } Quare apud Reges et Principes & que cada vno de los sus subditos tome del forma e manera de beuir \textbf{ e conosca can vno su mengua veyendo la uida } e la grant perfeçion del prinçipe e del señor . \\\hline
1.2.34 & quam sit virtus . \textbf{ Unde Philosophus 7 Ethicorum loquens de hac virtute , } ait , quod non est virtus , & que uirtud \textbf{ ¶ Onde el philosofo en el septimo libro delas etl sfablando desta uirtud dize } que non es uirtud \\\hline
1.3.1 & quia omnis passio \textbf{ et omnis motus animae pertinens ad concupiscibilem , } vel sumitur respectu boni , & assi se puede tomar \textbf{ por que toda passion | e todo mouimiento del alma } que pertenesçe al appetito desseador o se torna en conparaçion de algun bien \\\hline
1.3.3 & ex consequenti autem \textbf{ et quasi per accidens intendit bonum commune , } inquantum ex bono communi & ca prinçipalmente entiende en el su bien propio mas despues desto \textbf{ e assi conmo por açidente entiende en el bien comun } en quanto del bien comun se leunata a el algun bien ppreo . \\\hline
1.3.3 & Immo quia speciali modo Rex \textbf{ et quilibet principans est minister Dei et persona publica et communis , } speciali modo spectat ad Reges & Mas avn por que en espeçial manera el rey \textbf{ e cada vn prinçipe es ofiçial de dios | e ꝑson a publica e comun . } espeçialmente ꝑtenesçe alos Reyes \\\hline
1.3.4 & ut est oditum , \textbf{ est aliquid displicens } et difforme voluntati nostrae : & que non nos plaze \textbf{ e es cosa non conformada } nin conuenible ala nuestra uoluntad . \\\hline
1.3.6 & Timor autem si moderatus sit , \textbf{ expediens est Regibus et Principibus . } Moderato enim timore omnes principantes timere debent , & e de ser osados \textbf{ por que el temor si fuere tenprado es conuenible alos Reyes e alos prinçipes . } Ca por temor tenprado todos los prinçipes deuen temer \\\hline
1.3.6 & sic cum quis timet , \textbf{ calor existens in exterioribus membris , } statim confugit ad interiora ; & En essa misma manera \textbf{ quando alguno teme la calentura natural | que esta en los mienbros de fuera } luego fuye alos mienbros de dentro . \\\hline
1.3.6 & reddit hominem inoperatiuum . \textbf{ Nam homo propter timorem immoderatum tremens } et obstupefactus immobilitatur , & que non obre . \textbf{ Ca el omne por el temor destenprado | e sin razon } trieme e esta atomeçido \\\hline
1.3.7 & quia iratus appetit contristare : \textbf{ sed odiens appeti nocere . } Vult enim iratus & e de fazer triste a a qual contra quien a saña . \textbf{ Mas el que quiere mal a alguno tenssea dele enpeçer . } Ca el sannudo quiere dar dolor e tsteza \\\hline
1.3.7 & inferre dolorem , et tristitiam : \textbf{ sed odiens vult } inferre damnum , et nocumentum . & Ca el sannudo quiere dar dolor e tsteza \textbf{ mas el mal quariente quiere fazer danno e enpeçemiento¶ } La quinta diferençia es \\\hline
1.3.7 & Secundo , quia eam obnubilat . \textbf{ Ira enim rationem praecedens } secundum Philosoph’ & Lo segundo por quela sanna escuresçe \textbf{ e a nunbla el entendimiento . | Ca la saña que es ante de la razon e del entendimiento } segunt el philosofo \\\hline
1.3.7 & latrant , non distinguentes , \textbf{ an veniens sit amicus , vel inimicus . } Sic et ira facit : & luego ladran \textbf{ non departiendo nin conosçiendo si aquel que viene es amigo o enemigo . } Bien assi faze la saña . \\\hline
1.3.7 & ut vindictam exequatur , \textbf{ non expectans super hoc iudicium rationis , } qualiter vindicta illa fieri debeat . & por que sea fecha uengança \textbf{ non espando sobresto iuyzio delanrazon e del entendimiento } en qual manera esta uenganca deue ser fecha . \\\hline
1.3.7 & Cauenda est ergo ira inordinata , \textbf{ et rationem praecedens . } Sed si sequatur imperium rationis , & que es de esquiuar la sanna desordenada \textbf{ e aquella que viene ante iuyzio de la razon . } Mas si la sanna \\\hline
1.3.8 & absque delectatione aliqua nullus viuere potest . \textbf{ Si ergo negans loquelam , } concedit loquelam & sin alguna delectaçon \textbf{ Et pues que assi es | assi commo aquel } que niega la fabla otorga \\\hline
1.3.8 & ( ut patet per Philos 4 Metaphy’ ) \textbf{ sic ponens omnem delectationem esse fugiendam , } ponit aliquam delectationem esse prosequendam . & en el quarto libro delas ethicas \textbf{ En essa misma manera el que pone que toda delectaçiones de esquiuar e de foyr pone que alguna delectaciones de segnir . } Ca assy commo la fabla non puede ser negada sinon por la fabla . \\\hline
1.3.8 & nisi per loquelam , \textbf{ negans loquelam , } loquitur : & ø \\\hline
1.3.8 & ei omnem delectationem fugere ; \textbf{ sequitur quod fugiens omnem delectationem , } sequatur aliquam delectationem . & si non fuere a el delectable de foyr toda delectaçion siguese \textbf{ que aquel que fuye toda delectaçion } sigue algunan delectaçion . \\\hline
1.3.8 & esse \textbf{ quoddam pondus aggrauans animam . } Sicut ergo in pondere corporali & nono de las ethicas dize el philosofo por qui paresçe que latsteza es vn peso \textbf{ que agua uia el alma } Et por ende assi commo en el peso corporal \\\hline
1.3.8 & ideo eis amissis non dolebimus , \textbf{ nisi forte per accidens , } inquantum per amissionem & Et por ende aquellos perdudos \textbf{ non nos dolemos dellos sinon por auentura por accidente alguno en } quanto por perdimiento de aquellos bienes somos enbargados delas obras uirtuosas . \\\hline
1.4.3 & prout disponitur , \textbf{ existens in aliqua passione , } naturaliter passionatur passione illa . & qual quier que es inclinado naturalmente a alguna passion \textbf{ assi commo aquel que esta puesto en aquella passion } natraalmente es passionado de aquella passion \\\hline
1.4.4 & fit remissio venereorum , \textbf{ et concupiscientiarum . Constat enim quod concupiscens venerea per appetitum , } in alia se extendit . & fazese en el abaxamiento e atenpramiento delas cobdiçias dela luxͣia . \textbf{ Ca cierça cosa es que aquellos que dessean la lux̉ia | por su appetito se estienden } e van a ella . \\\hline
2.1.1 & intelligendum est de cibariis aliis . \textbf{ numquam cum homo existens } solus sufficit sibi & de todas las otras uiandas \textbf{ por que nunca el omne estando señero puede conplir assy mismo } para auer viandas conuenibles \\\hline
2.1.1 & Ideo dicitur primo Politicorum \textbf{ quod eligens solitariam vitam , } non est pars ciuitatis : & Et por ende dize el philosofo \textbf{ en el primero libro delas politicas el que toma e escoge de beuir uida sola } e apartada non es parte dela çibdat \\\hline
2.1.2 & Dubitare forte aliquis , \textbf{ dicens nos scientiarum limites ignorare : } quia societas , siue communitas illa , & ora uentura dubdaria alguno et diria \textbf{ que nos non sabiamos los terminos destas sçiençias } por que aquellas conpania o aquella comunidat \\\hline
2.1.3 & Unde et Philosophus 1 Politicorum \textbf{ comparans ciuitatem } ad vicum et domum , & Onde el philosofo en el primero libro delas politicas \textbf{ conparando la çibdat aluarrio } e ala casa dize \\\hline
2.1.4 & In hac autem descriptione aliquid declaratum est \textbf{ per praecedens capitulum , } et aliquid restat ulterius declarandum . & que se fazen de cada dia . \textbf{ Mas en esta declaraçion alguna cosa es ya declarada en el capitulo } sobredich̃o Et alguna cola finca adelante de declarar . \\\hline
2.1.4 & non solum est \textbf{ expediens communitas domus , } sed et vici , ciuitatis , et regni . & por las cosas ya dichͣs en la uida humanal \textbf{ non solamente es menester la comunidat dela casa } mas ahun la comunidat del uarrio e dela çibdat e del regno \\\hline
2.1.4 & quam propter iam dictas , \textbf{ sit expediens communitas ciuitatis , et regni , } in tertio libro plenius ostendetur . & que por las que ya dichos son \textbf{ es menester la comunidat dela çibdat } e del regno esto se mostrara mas conplidamente en el terçero libro . \\\hline
2.1.5 & ( ut patet per Philosophus 1 Politic’ ) \textbf{ qui deficiens intellectu , } pollet fortitudine corporali Seruus ergo , & en el primero libro delas politicas \textbf{ que fallesçe en el entondemiento } e ha mayor fortalleza en el cuerpo . \\\hline
2.1.5 & ut consequatur salutem per eum , \textbf{ et ut dirigatur per ipsum . Nam deficiens intellectu , } est quasi coecus , & por que alcançe salud por el \textbf{ e por que sea enderescado por el en sus obras . | Ca el que fallesçe en el entendemiento es } assi commo çiego \\\hline
2.1.6 & aliquam bene ordinatam , \textbf{ nisi aliquid sit ibi dirigens , } et aliquid directum : & Ca nunca podemos dar comunindat ninguna bien ordenada \textbf{ si non fuere . } y algun gouernador o si non fuere \\\hline
2.1.6 & vel nisi aliquid sit \textbf{ ibi principans , } et aliquid obsequens . & y algun gouernador o si non fuere \textbf{ y alguno que sea senñor } e alguno que faga su mandado . \\\hline
2.1.6 & Quare cum in communitate maris et foeminae , \textbf{ mas debet esse principans , } et foemina obsequens : & Por la qual cosa commo en la comunindat del uaron \textbf{ et dela fenbra el uaron deua ser prinçipal et ordenador } e la fenbra obediente . \\\hline
2.1.6 & in communitate vero patris et filii , \textbf{ pater debet esse imperans , } et filius obtemperans ; & Mas en la comunidat del padre \textbf{ e del fijo el padre deua sienpre mandar } e el fij̉o ser obediente . \\\hline
2.1.6 & pater debet esse imperans , \textbf{ et filius obtemperans ; } in communitate quidem domini et serui , & e del fijo el padre deua sienpre mandar \textbf{ e el fij̉o ser obediente . } Et en la comunidat del señor \\\hline
2.1.6 & in communitate quidem domini et serui , \textbf{ dominus debet esse praecipiens , } et seruus ministrans et seruiens , & Et en la comunidat del señor \textbf{ e del sieruo el señor deue mandar } e el sieruo seruir e ministrar . \\\hline
2.1.6 & dominus debet esse praecipiens , \textbf{ et seruus ministrans et seruiens , } in domo perfecta & e del sieruo el señor deue mandar \textbf{ e el sieruo seruir e ministrar . } Siguese que en la casa acabada \\\hline
2.1.7 & et quod homo naturaliter est animal coniugale . \textbf{ Sciendum ergo quod Philosophus 8 Ethic’ volens } ostendere & esaianl coniugable e ayuntable en mater moino . \textbf{ Et pues que assi es que el philosofo en el octauo delas ethicas } quariendo mostrar \\\hline
2.1.7 & probat amicitiam illam esse secundum naturam : \textbf{ adducens triplicem rationem } quod homo sit naturaliter animal coniugale . & que aquella amistança es segunt natura . \textbf{ Et aduze para esto tres razones } que el omne sea naturalmente \\\hline
2.1.7 & Hanc autem rationem tangit Philosophus \textbf{ 8 Ethic’ dicens : } Homo enim natura & Et esta razon tanne el philosofo \textbf{ en el octauo libro delas ethas | do dize } e que el omne \\\hline
2.1.7 & de societate politica , \textbf{ videlicet quod eligens solitudinem , } et nolens ciuiliter viuere , & conmodiziemos dela uida politica e de çibdat . \textbf{ Conuiene a saber que el que escoge beuir solo } e non quiere beuir \\\hline
2.1.7 & videlicet quod eligens solitudinem , \textbf{ et nolens ciuiliter viuere , } vel est bestia , & Conuiene a saber que el que escoge beuir solo \textbf{ e non quiere beuir | çiuilmente } este tal o es bestia o es assi commo dios \\\hline
2.1.7 & sic et de coniugio dicere possumus . \textbf{ Nam nolens coniugaliter uiuere , } uel hoc est , & En essa misma manera podemos dezir del mater moion \textbf{ que aquel que non quiere beuir conuigalmente e con su muger o esto es } por que quiere mas libremente fazer forniçio . \\\hline
2.1.8 & est quod diuidat et distinguat . \textbf{ Hanc autem rationem tangit Philosophus 8 Ethic’ dicens , } quod quia commune continet & assi commo dela razon del bien propreo es que ayunte e desayunte el vno del otro . \textbf{ Et esta razon pone el philosofo en el viii̊ libro delas ethicas } do dize que por que el bien comunal contiene \\\hline
2.1.14 & sed secundum eas quas ciues instituerunt . \textbf{ Cum enim principans in ciuitate ipse } secundum seipsum principatur , & que establesçieron los çibdadanos . \textbf{ Ca quando el que | enssennorea en la çibdat el mesmo ensseñorea por si e el } establesçe leyes aquel gouernamiento toma nenbre \\\hline
2.1.15 & quod unumquodque organorum optime perficiet suum opus , \textbf{ si non multis operibus sit seruiens , sed uni . } Quare cum natura ordinauerit & Ende en esse logar dize el philosofo \textbf{ que qual si quier de los instrumentos fara conplidamente su obra si non siruiere en muchos obras mas en vna . } Por la qual cosa commo la natura aya ordenada la mugr \\\hline
2.1.15 & idem sunt foemina et seruus , \textbf{ ut recitat Philosophus 1 Politicorum dicens , } quod inter Barbaros foemina et seruus eundem habent ordinem . & assi commo entre los bartaros vna ꝑlona es muger e lieruo . \textbf{ Assi conmo dize el philosofo } en el primero libro delas politicas do dize que entre los barbaros la fenbra \\\hline
2.1.15 & ( ut recitatur ibidem ) \textbf{ quia inter Barbaros nullus est naturaliter principans , } sed idem est & assi commo el philosofo dize en esso mismo logar \textbf{ por que entre los barbaros non era ninguno natrealmente sennor } mas vn omne era naturalmente barbaro e sieruo . \\\hline
2.1.15 & nisi careat usu rationis et intellectus . \textbf{ Sed cum carens rationis usu sit naturaliter seruus , } quia nescit seipsum dirigere , & si non fuese priuado de vso de razon e de entendimiento . \textbf{ Mas commo aquel que es priuado de vso de razon e de entendemiento sea naturalmente sieruo } por que non sabe gniar assi mismo \\\hline
2.1.17 & quia calor eius interius \textbf{ propter frigus circunstans } non exalat , & por la calentura natural se ençierran de dentro \textbf{ por el frio qual çerca de fuera } e non salle la calentura mas esforçada \\\hline
2.1.19 & in uniuersali tractare de regimine coniugali , \textbf{ dicendo ipsum esse differens } a paternali et seruili , & assi en general del gouernamiento matermonial del marido ala mugni \textbf{ que es departido del gouernamiento paternal } e del gouernamiento \\\hline
2.1.21 & ostendit Philosophus 1 Rhet’ \textbf{ qui loquens de Lacedaemoniis , } ait , eos esse infelices secundum dimidium , & en el primero libro de las rectorica \textbf{ do dize fablando de los laçedemonios | que son gente de greçia } e desauentraados en la meytad de su fazienda \\\hline
2.1.21 & humilis \textbf{ non ornans se propter vanam gloriam , } posset delinquere in ornatum , & que la muger eł uaron fuesse humillosa \textbf{ e non se posie con sse } por uana eglesia podria pecar \\\hline
2.1.23 & propter quod magis obsequitur ei : \textbf{ et anima existens in tali corpore , } liberius utitur operibus propriis , & mas es proporçionado e egualado al alma . \textbf{ Et el alma que esta en tal cuerpo meior vsa de sus obras propiçias } e mas \\\hline
2.1.23 & quod mala herba cito crescit \textbf{ quia natura de ea modicum curans , } cito perducit & que yerba mala mas ayna cresçe \textbf{ por que la natura ha poco cuydado della } e ayna la aduze a su conplimiento . \\\hline
2.2.3 & Sciendum ergo triplex esse regimen . \textbf{ Nam omnis regens alios , } vel regit eos & que tres son los gouernamientos . \textbf{ Ca todo omne que gouierna alos otros } o los gouierna segunt ciertas leyes \\\hline
2.2.3 & et hoc potest esse dupliciter : \textbf{ quia vel sic regens } intendit bonum proprium , & Et esto puede ser en dos maneras . \textbf{ Ca el que gouierna o entiende el su bien prop̃o } e assi es dicho gouernamiento despotico o suilo \\\hline
2.2.3 & quia filius naturaliter est \textbf{ quaedam similitudo procedens a patre : } cum secundum naturam ad huiusmodi similia fit dilectio , & Por la qual cosa si el gonernamiento del padre desto tora a comienco \textbf{ por que el fijo naturalmente es vria semerança que desçende del cadre . } Canmo segunr nacsta \\\hline
2.2.4 & quia de illis certiores existunt . \textbf{ Tertia via probans hoc idem , } sumitur ex unione parentum ad filios . & ¶ \textbf{ La tercera razon que peua esto mismo se toma del ayuntamiento de los padres alos fijos . } Ca los fijos son mas ayuntados \\\hline
2.2.4 & qui est quaedam gleba \textbf{ a parentibus descendens , } et quia est quaedam pars ipsorum ; & que es vn çesped \textbf{ que desçende del padre es vna parte del } e non ay ninguna cosa en el fijo \\\hline
2.2.5 & Unde et Philosophus 2 Meta’ \textbf{ volens probare } consuetudinem esse magnae efficaciae , & Onde el philosofo en el segundo libro dela \textbf{ methafisica quariendo prouar } que la costunbre es de grand fuerça dize assi . \\\hline
2.2.8 & quae est quaedam grossa dialectica \textbf{ docens modum arguendi } grossum et figuralem . & assi fue neçessaria la rectorica \textbf{ que es assi commo vna logica gruessa } que nos mostrasse manera gruessa e figural \\\hline
2.2.8 & dicitur esse arithmetica , \textbf{ docens proportiones numerorum , } ad quam forte filii liberorum ideo tradebantur , & La quinta sçiençia libal es dicha arismetica \textbf{ que muesterra las proporçiones | e las concordanças de los cuentos } ala qual sçiençia \\\hline
2.2.8 & longe nobiliores istis . \textbf{ Nam Naturalis Philosophia docens } cognoscere naturas rerum , & qua non estas . \textbf{ Ca la natural ph̃ia } que muestra conosçer las naturas delas cosas \\\hline
2.2.9 & sed quod sit perspicax \textbf{ et intelligens aliorum dicta . } Tertio oportet ipsum esse iudicatiuum : & que alguno sea inuentiuo e fallador dessi mas conuiene que sea agudo \textbf{ e que entienda los dichͣs de los otros . } ¶ Lo terçero conuiene que sea iudgador \\\hline
2.2.9 & qui sit inuentiuus ex se , \textbf{ intelligens alios , } et bene iudicatiuus & que sea fallador dessi \textbf{ e entendedor de los otros } e buen iudgador tan bien delas cosas entendidas \\\hline
2.2.9 & recolendo praeterita . \textbf{ Nam sicut volens rectificare virgam , } nunquam eam rectificare posset & Ca primero deue ser menbrado e acordado delas colas passadas . \textbf{ Ca assi commo aquel que quiere enderesçar la pierte } ga nunca la puede enderesçar \\\hline
2.2.9 & ex qua parte esset obliquata : \textbf{ sic volens alios rectificare } nunquam eos congrue rectificare posset & ga nunca la puede enderesçar \textbf{ si non conosçiere de qual parte esta tuerta . En essa misma manera aquel que quiere enderesçar los otros } nunca los podria enderesçar \\\hline
2.2.9 & Sic qui vult iuuenes dirigere debet esse , \textbf{ cautus proponens eis bona } sine admixtione malorum . & deue ser sabio \textbf{ aponiendo les los bienes } sin mesclamiento de ninguons males \\\hline
2.2.12 & In qua autem aetate debeant uti coniugio , \textbf{ ostendit Philosophus 7 Poli’ dicens , } quod in muliere requiritur aetas decem et octo annorum , & Mas en qual hedat deuen los omes vsar del casamiento \textbf{ muestra lo el philosofo | en el septimo libro delas politicas } o dize \\\hline
2.2.20 & non vacaret ociose , \textbf{ sed saepe saepius librum arripiens , } se lectionibus occuparet . & non estudiesse baldia nin oçiosa mas muchͣs \textbf{ e muchos uegadas tomando el libro s } e trabaiasse en rezar sus leçiones o sus salmos o sus oronnes . \\\hline
2.3.1 & et spectat ad fabrum talia instrumenta cognoscere . \textbf{ Sic volens tradere notitiam de arte textoria , } debet determinare de pectinibus , & cognosçertales estrumentos . \textbf{ Et dessa misma manera | el que quiere dar conosçimiento del arte del texer } deue determinar de los peinnes \\\hline
2.3.1 & talia instrumenta cognoscere . \textbf{ Quare volens tradere notitiam } de arte gubernationis domus , & e pertenesçe al texedor de conosçer tales estrumentos . \textbf{ Por la qual cosa el que quisiere dar conosçimiento del arte del gouernamiento dela casa } deue determinar de los hedifiçios \\\hline
2.3.3 & et talia aedificia construere , \textbf{ quod populus ea videns , } quasi sit mente suspensus propter vehementem admirationem . & e de fazer tales moradas \textbf{ que el pueblo que lo viere } finque assi commo espantado \\\hline
2.3.3 & Nam populus minus insurgit contra principem , \textbf{ videns ipsum sic magnificum : } quilibet enim de populo & contra el prinçipe \textbf{ quando vee | que el es muy magnifico } por que cada vno del pueblo \\\hline
2.3.3 & Sic enim imaginari debemus , \textbf{ quod sicut aqua currens sanior est quam stans , } eo quod aquae stantes & porque deuemos assi ymaginar \textbf{ que assi commo el agua corre es mas ssana | que non la estançia } que non corre \\\hline
2.3.4 & eo quod eis sit \textbf{ aqua quodammodo stans , } ut plurimum habent aquam non salubrem . & por que en ellas es el agua \textbf{ estantiaqua non corre } e por ende por la mayor parte non han el agua sana¶ \\\hline
2.3.4 & ut horum natatu aqua \textbf{ stans agilitatem currentis imitetur . } Viso , qualiter est aedificium construendum quantum & por que por el nadamiento de los peces \textbf{ el agua estante semeie en lignieza al agua que corre ¶ Visto en qual manera son de fazer las moradas } quanto ala salud delas agunas finca de ver \\\hline
2.3.4 & si aedificium \textbf{ secundum suam ampliorem partem respiciat oriens hyemale : } tunc enim eo quod in hyeme oppositum sit soli , & Lo primero puede contesçer si la morada \textbf{ segunt la su parte mayor | catare a lorsete del yuierno } por que estonçe la morada \\\hline
2.3.5 & sic natura ordinauit , \textbf{ ponens in ipsis ouis album et rubeum , } ita quod ex albo generatur auis , & assi commo son las aues la natura \textbf{ assi ordeno poniendo en los hueuos blanco e bermeio } assi que del blanco se engendra el aue \\\hline
2.3.6 & quod esset utile \textbf{ et expediens ciuitati } quod ciues propriis possessionibus non gauderent , & que cola aprouechosa \textbf{ e conueniente serie ala çibdat } que los çibdadanos non gozassen de sus possessiones proprias \\\hline
2.3.6 & ne faciat quod mandatur , \textbf{ sperans alium implere } quod iubetur ; & aquello qual es mandado \textbf{ elperando que el otro cunplira aquello que a el es mandado . } Por la qual cosa conuiene \\\hline
2.3.7 & quia natura talia produxit , \textbf{ ordinans ea ad usum hominis : } semper enim imperfecta & por quela natura engendro tales cosas \textbf{ ordenando la sal vso del omne | ca sienpre las cosas } que non son acabadas son ordenadas alas mas acabadas \\\hline
2.3.7 & Si enim homines aliqui contra alios iuste bellant , \textbf{ hoc quasi per accidens , } inquantum illi aliquo modo forefaciunt & conuenibleca si algunos omes batallan derchamente contra otros \textbf{ esto es por algun açidente en quanto aquellos en algua manera } fazen o fizieron algua cosa desagnisada contra ellos . \\\hline
2.3.7 & Si autem in offensione bestiarum est delictum , \textbf{ hoc est quasi per accidens , } inquantum talis offensa redundat & Et si en faziendo mal alas bestias es pecado \textbf{ esto es } por açidente en quanto tal danno se torna en danno del omne \\\hline
2.3.9 & quibus factis videtur \textbf{ utens illis esse in honore et gloria . } Primitus ergo inuentae fuerunt commutationes ad metalla solum secundum pondera : & los quales assi fechs paresçen alos omes \textbf{ que los que vsan dellos son en eglesia e en honrra . } Et pues que assi es primeramente fueron falladas las mudaçiones delas cosas alos metalles \\\hline
2.3.11 & et nihil usurpat , \textbf{ si retinens sibi dominium domus , } uendit inhabitationem , & Et por ende assi faziendo non robanada \textbf{ nin toma uso ageno si retiniendo en ssi el señorio dela casa vede la morada } e el uso della mas en los dineros non es \\\hline
2.3.11 & ipsam substantiam alienare . \textbf{ Quia ergo accidens a substantia dependet , } et usus ex ipsa re sumit originem , & dellos commo atal uso pertenezca de enagenar la sustançia . \textbf{ Et pues que assi es por que el açidente desçende dela sustançia del subieto } en que esta e el uso toma nasçençia dela sustançia \\\hline
2.3.13 & secundum debitum ordinem efficiunt aliquid unum , \textbf{ nisi sit ibi aliquid praedominans respectu aliorum : } ut si plures voces efficiunt aliquam harmoniam , & segunt orden conuenible \textbf{ siy non fuere entre ellas algunan cosa | que en ssennore } e en conpara conn delas otras . \\\hline
2.3.13 & ad constitutionem eiusdem corporis mixti , \textbf{ oportet aliquod elementum praedominans , } secundum quod illi mixto competat & conuiene de dar \textbf{ ay algun helemento | que ssennoree sobre los otros } segunt el qual conuiene aquel cuerpo mesclado mouimiento con ueinble o asentamiento con ueible \\\hline
2.3.13 & naturaliter aliquid unum , \textbf{ nisi ibi naturaliter aliud sit praedominans : } cum societas hominum sit naturalis , & muchas fazen naturalmente alguna cosa \textbf{ que sea vna si non | fuereynatraalmente alguna cosa } que en ssennore e a todas aquellas muchͣs . \\\hline
2.3.13 & in partes naturales diuiditur in corpus et animam : \textbf{ ubi anima est quasi dominans , } et corpus est quasi obsequens et obediens . & assi commo en partes essençiales . \textbf{ En la qual particion el alma es | assi commo señora . } Et el cuerpo es \\\hline
2.3.14 & Prima congruitas sic patet : \textbf{ oportet enim dominans } ( ut dicitur in Politic’ ) & la primera razon paresçe assi . \textbf{ Ca conuiene que el sennor | segunt } que dize el philosofo \\\hline
2.3.14 & sed magis faciunt ipsum legale et positiuum . \textbf{ Quod enim superans in bonis corporis , } ut in fortitudine , & e positiuo por establesçimiento de los omes . \textbf{ Ca aquel que ha auentaia en los bienes del cuerpo } assi commo en fortaleza o en poderio ciuil deue \\\hline
2.3.15 & ille enim mercenarius dicitur , \textbf{ qui principaliter mercedem intendens , } obsequitur ei , & por que aquel es dich merçenario \textbf{ que prinçipalmente entiende resçebir merçed } e por ende obedesçe \\\hline
2.3.15 & ampliora beneficia tribuenda : \textbf{ cum ergo virtuosus seruiens } ex amore honesti , & Ca sienpre alos mas digunos son de dar los mayores benefiçios . \textbf{ Et por ende commo el seruiente uirtuoso } que sirue por amor de bien e de honestad sea mas digno que el que sirue \\\hline
2.3.16 & nam saepe quilibet ministrantium huiusmodi ministerium negligit , \textbf{ credens quod alius exequatur illud : } ubicunque enim est multitudo , & Ca muchͣs uezes cada vno de aquellos seruientes \textbf{ menospreçia aquel seruiçio cuydando que el otro lo fara . } Por que do quier que ay muchedunbre alli es confusion \\\hline
2.3.19 & et ab usu rationis deficiat . \textbf{ De his autem loquens Philosophus } circa finem primi Politicorum ait , & e aquel que fallesçe de uso de razon e de entendimiento \textbf{ e atal non deuen descobrir sus poridades . | Mas fablando elpho destas cosas } cerca la fin del prim̃o libro \\\hline
3.1.1 & et eam quae est omnium maxime principalis , \textbf{ et omnes alias circumplectens , potissime gratia boni constitutam esse contingit : } haec autem est communitas politica , & Et toda cosa que encieira en ssi las o triscosas \textbf{ couiene que sea establesçida prinçipalmente | por grande algun bien } e esta es comunidat politica \\\hline
3.1.2 & et secundum aliquas laudabiles ordinationes , \textbf{ recusans sic viuere , } non vult viuere ut homo , & e segunt alguas ordenaconnes de loar \textbf{ el que refusare de beuir } assi non quiere beuir commo omne \\\hline
3.1.2 & vel in uno vico non reperiuntur \textbf{ omnia quae ad vitam sufficiunt , ideo fuit ciuitas constituta habens plures vicos , } in quorum uno exercetur ars fabrilis , & que cunplen ala uida del omne . \textbf{ Et por ende sue establesçida la çibdat | que ha muchos uartios en el vno } de los quales se obra la ferreria \\\hline
3.1.2 & huiusmodi communitas est \textbf{ habens terminum omnis } per se sufficientiae vitae . & ca esto se sigͤel dezir \textbf{ que tal comunidat es la que ha termino } por si e todo conplimiento e abastamiento de uida \\\hline
3.1.2 & quidem igitur est ciuitas viuendi gratia : \textbf{ existens autem gratia bene viuendi } forte & non solamente por grande beuir \textbf{ mas por grande beuir bien } e segunt ley \\\hline
3.1.2 & constituerunt ciuitatem , \textbf{ ut unus alterius defectum supplens } haberet sufficientia in vita . & por ende establesçieron çibdat \textbf{ por que vn uario cunpliesse la mengua del otro } e ouiessen abondança en la uida \\\hline
3.1.3 & quos ( ut recitat Philosophus 1 Poli’ ) \textbf{ maledicebat Homerus , dicens , } Maledictus insocialis , inlegalis , sceleratus , affectator belli , & en el primero libro delas politicas \textbf{ do dize que maldicho es el | qua non tiene conpan nina } e el que non es legal nin guarda ley \\\hline
3.1.3 & veluti sicut \textbf{ sine iugo existens sicuti volatilia . } Tertio aliqui non ciuiliter viuunt & assi commo si fuesses un yugo \textbf{ e sin ley commo las aues todos estos linages de oms son maldichos } ¶la terçera razon \\\hline
3.1.3 & vel est bestia et sceleratus et sine iugo , \textbf{ non valens legem et societatem supportare , } vel est quasi deus idest diuinus , & o pecadar e sin yugo de ley \textbf{ que non quiere conplir les | nin sofrir conpannia } o por que es diuinal \\\hline
3.1.3 & vel est quasi deus idest diuinus , \textbf{ eligens altiorem vitam : } propter quod scribitur primo Poli’ & assi commo dios \textbf{ que escoge uida mas alta que los otros } por la qual cosa dize el philosofo \\\hline
3.1.4 & per quem distincte significatur \textbf{ quid conferens , } quid nociuum , & departidamente \textbf{ qual cosa le es delectable } e qual enpeçible \\\hline
3.1.4 & et quid iniustum . \textbf{ Si ergo communitas domestica ordinatur ad prosequendum conferens , } et ad fugiendum nociuum : & e qual non iusta \textbf{ por ende | si la comunidat dela casa es ordenada a alcançar lo que es delectable } e para foyr \\\hline
3.1.4 & quia in ea non solum quaeritur \textbf{ quid conferens et quid nociuum , } sed etiam quid iustum et quid iniustum : & por que non solamente en la çibdat es demandado \textbf{ que cosa es delectable | e que cosa es enpesçible } mas avn y es demandada \\\hline
3.1.4 & datum a natura repraesentatur \textbf{ conferens et nociuum , } et iustum et iniustum . & por quela palabra anos en dada naturalmente demostrar \textbf{ qual cosa nos es delectable | e qual enpesçible } e qual nos es iusta \\\hline
3.1.5 & ex pluribus vicis , \textbf{ est habens terminum omnis } per se sufficientiae vitae . & que es çibdat se faga de muchsuarios \textbf{ e aya en ssi conplimiento delas cosas } que ꝑtenesçen ala uida esto non es assi de entender que sienpre conuenga \\\hline
3.1.5 & quanto minorem haberet potentiam , \textbf{ tanto magis esset expediens ciuitati . } Tertia via ad inuestigandum hoc idem , & en todo quanto menor poderio ellos ouieren tanto \textbf{ mas es pro dela çibdat } ¶ La terçera razon para prouar esto mismo se toma \\\hline
3.1.6 & et per tyrannidem constitui posset , \textbf{ ut si quis dominans alicui ciuitati per tyrannidem } et per ciuilem potentiam & e por tirania se podia establesçer \textbf{ assi como si alguno quisiesse | senorear alguna çibdat } por tirania e por poderio ciuil apremiasse las otras çibdades \\\hline
3.1.7 & circa naturas rerum , \textbf{ videns circa naturalem scientiam } magnam difficultatem esse , & as socrates commo ouiesse phophado luengo tienpo çerca las naturas delas cosas \textbf{ ueyendo muy grant guaueza cerca la sciençia natural } assi commo cuenta el pho \\\hline
3.1.8 & oportet in ciuitate diuersitatem esse . \textbf{ Tertia via declarans } et manifestans vias praedictas , & conuiene que enla çibdat sea algun departimiento . \textbf{ La tercera razon que declara e manifiesta las razones } ya dichos se toma en conparaçion del todo alas partes \\\hline
3.1.8 & nam ciuitas est communitas \textbf{ habens terminum omnis } per se sufficientiae vitae : & ca la çibdat escomunidat \textbf{ que ha } por si termino de abastamiento deuidaso \\\hline
3.1.8 & et diuersos habere vicos , \textbf{ ut expediens ad vitam } quod non neperitur in uno vico , & e de auer departidos uarrios \textbf{ assi que abonden a la uida } por que aquello que non fuer fallado en vn uarrio \\\hline
3.1.9 & quam ponebat Socrates , \textbf{ non est expediens ciuitati , } Decet autem hoc Reges , & pues que assi es aquella comuidat \textbf{ que ponian platon e socrates non era conueinble ala çibdat . } Mas conuiene alos Reyes \\\hline
3.1.10 & et patres filias . \textbf{ Socrates volens hoc inconueniens vitare , } dixit , & e los padres con sus fijas . \textbf{ Empero socrates quariendo escusar este mal dix̉o } que al prinçipe dela çibdat pertenesçia de auer cuydado e acuçia \\\hline
3.1.11 & ( ut Philosophus ait ) \textbf{ superficietenus considerata valde expediens ciuitati : } cum enim quis audit ciuitatem & assi commo dize el philosofo \textbf{ si fuere penssada a de fuera | e liuianamente paresçe muy conueinble a la cibdat } ca quando algud vee \\\hline
3.1.11 & et possessiones proprias quantum ad dominum : \textbf{ nam quilibet dominans bonis propriis } adhibebit & e las possessiones prop̃as quanto al sennorio . \textbf{ ca cada vn sennor de sus bienes propreos aura mayor acuçia de aquellos bienes } que si fuessen comunes \\\hline
3.1.12 & sicut et viri : \textbf{ assumens exemplum ex bestiis , } ex quibus tam mares quam foeminae bellant . & assi commo los omes \textbf{ e tomaua enxienplo delas bestias } entre las quales las fenbras lidianta bien commo los mas los . \\\hline
3.1.12 & a similitudine bestiarum , \textbf{ insufficiens est : } quia secundum Philosophum 2 Poli’ & la qual razon tomo delas semesaças delas bestias \textbf{ non es buena nin conplida } por que segunt el philosofo \\\hline
3.1.13 & Possumus autem triplici via ostendere , \textbf{ quod non sit expediens ciuitati } semper praeponere & por tons razones \textbf{ que non es conuenible ala çibdat } que sienpre sean puestos vnos maestros \\\hline
3.1.13 & non sic potest iniustificare , \textbf{ ut publica et habens potestatem ; } et quia non tot oculi respiciunt & assi mostrat la su maldat \textbf{ commo quando es persona publica | e ha poderio } por que non paran mientes tantos oios \\\hline
3.1.13 & non scitur qualis sit vir , \textbf{ donec est persona priuata non habens potestatem , } sicut cognoscitur postquam & assi saber quales el uaron mientra es persona preuada \textbf{ que non ha poderio } commo se conosçe despues que esle un atada en alguna dignidat o en algun maestradgo o en algun poderio \\\hline
3.1.14 & Numerum autem bellantium statuebat , \textbf{ dicens , } in qualibet ciuitate & Ca dizia que los lidiadores deuen ser parte dela çibdat apartada de los otros çibdadanos \textbf{ e establesçie cuento de los lidiadores } diziedo que en cada vna çibdat deuen ser \\\hline
3.1.14 & tantam multitudinem pascere . \textbf{ Propter quod Philosophus 2 Politicorum reprehendens Socratem de huiusmodi ordine ciuitatis , } ait , quod oportet ciuitatem illam sic institutam esse in regione Babilonica , & tal gouernamiento de çibdat \textbf{ do dize | que aquella çibdat assi establesçida deuie ser } tamannera \\\hline
3.1.14 & Nam secundum Philosophum secundo Politicorum , \textbf{ volens ponere leges } vel facere ordinationem aliquam in ciuitate , & en el segundo libro delas politicas \textbf{ el que quiere poner leyes o fazer ordenaçion alguna en la çibdat a tres } co sas deue deuer mietes . \\\hline
3.1.14 & esse circa particularia signata , \textbf{ volens tradere artem de regimine ciuitatum , } non potest statuere determinatum numerum bellatorum : & cerca las cosas particulares e senñaladas . \textbf{ El que quiere dar arte e sçiençia de gouernamiento dela çibdat } non puede \\\hline
3.1.15 & in agricolas , artifices , et bellatores , \textbf{ volens ciuitatem } ad minus mille continere bellatores . & e en batalladores quariendo \textbf{ que alo menos la çibdat ouiesse mil ł batalladores } por auentura \\\hline
3.1.16 & 2 Politicorum intromisit se de ordine ciuitatis , \textbf{ statuens quomodo posset } optime politia ordinari . & que se entremi tio del ordenamiento dela çibdat \textbf{ establesçiendo en qual manera se podria ordenas muy bien la poliçia e la çibdat } ca dizia \\\hline
3.1.16 & in possessionibus aliis ciuibus . \textbf{ Sic etiam perdens placitum , } non multum amitteret : & ygualado en las possessiones con los otros çibdadanos . \textbf{ En essa misma manera } el que perdiesse el pleito non perderia mucho \\\hline
3.1.16 & statuit potissime curandum esse de possessionibus ciuium , \textbf{ volens eos aequatas possessiones habere . } Si considerentur dicta Philosophi 2 Politicorum , & que much deuia ser tomado grant cuydado delas possessiones de los çibdadanos \textbf{ e quaria que los çibdadanos ouiessen las possessiones eguales } i fueren penssados los dichos del philosofo \\\hline
3.1.19 & Praeter Socratem et Phaleam fuit quidam alius Philosophus nomine Hippodamus , \textbf{ intromittens se de regimine ciuitatis , } statuens multa pertinentia ad regimen ciuium . & y podo mio \textbf{ que se entremetio del gouernamiento dela çibdat } establesçiendo muchas cosas \\\hline
3.1.19 & intromittens se de regimine ciuitatis , \textbf{ statuens multa pertinentia ad regimen ciuium . } Videntur autem quasi ad sex reduci & que se entremetio del gouernamiento dela çibdat \textbf{ establesçiendo muchas cosas | que parte n esçiençia al gouernamiento de los çibdadanos } e paresçe que aquellas cosas \\\hline
3.1.19 & diuersa genera personarum . \textbf{ Hippodamus autem statuens suam politiam , } primo intromisit se de multitudine & que tannian alguons linages de personas dezimos \textbf{ que y podo mio | establesciendo su poliçia } primero se entremetio dela muchedunbre \\\hline
3.1.19 & et distinctione ciuium , \textbf{ dicens quod optima quantitas ciuium est circa decem millia virorum . } Hanc autem quantitatem distinxit in tres partes , & e del deꝑtimiento de los çibdadanos \textbf{ e dizia que sia muy buena quantidat de çibdadanos | si fuessen fasta diez minl uarones } e esta quantidat departia en tres partes \\\hline
3.1.19 & et determinauit de distinctione possessionum , \textbf{ diuidens totam regionem idest } totum territorium ciuitatis & entremetiose e determino del departimiento delas possessiones \textbf{ partiendo todo el regno } o todo el terretorio de la çibdat en tres partes . \\\hline
3.1.19 & quod quicunque inueniret aliquid , \textbf{ quod esset expediens ciuitati , } quod retribueretur ei debitus honor . & establesçio que qual quier que fallasse algua cosa \textbf{ que fuesse buena e conueinble ala çibdat } qual feziessen grant honrra \\\hline
3.1.20 & et specialiter quantum ad legem quam statuit erga sapientes . \textbf{ Nam si quicunque sapiens inueniens aliquid expediens ciuitati , } deberet ex hoc debitum honorem accipere ; & que establesçio alos sabios diziendo \textbf{ que si qual quier sabio | que fallasse alguna cosa conuenible } e buena para la çibdat deue \\\hline
3.1.20 & qui inuenit \textbf{ aliquid expediens ciuitati } ( ut ait Philosophus ) & que el sabio \textbf{ que alguna cosa fallaua a prouechosa ala çibdat deuia ser honrrado | por ello esto } assi commo dize el philosofo visto \\\hline
3.2.1 & debet enim intendere legislator \textbf{ ut per leges consequamur conferens , } et vitemus nociuum : & ca deue entender el fazedor dela ley \textbf{ que por las leyes alcançemos | lo que nos cunple } e escusemos \\\hline
3.2.1 & quorum est cognoscere \textbf{ quid conferens et quid nociuum , } et quales debeant esse iudices & alos quales parte nesçe de conosçer \textbf{ que cosa non es buena | e que cosa non es enpeesçible } e quales deuen ser los iuezes alos \\\hline
3.2.2 & regis autem est intendere commune bonum . \textbf{ Si vero ille unus dominans } non intendit commune bonum , & ca al Rey parte nesçe de enteder el bien comun . \textbf{ Et li aquel vno assi } enssennoreante non entiende el bien comun \\\hline
3.2.2 & non intendit commune bonum , \textbf{ sed per ciuilem potentiam opprimens alios , } omnia ordinat in bonum proprium et priuatum , & enssennoreante non entiende el bien comun \textbf{ Mas entiende por poderio çiuil apremiar los otros } e todas las cosas ordena al su bien propio \\\hline
3.2.2 & si rectus sit . \textbf{ Sed si populus sic dominans non intendit bonum omnium secundum suum statum , } sed vult tyrannizare et opprimere diuites , & si derecho es . \textbf{ Mas si el pueblo assi | enssennoreante non entiende a bien de todos } segunt su estado \\\hline
3.2.3 & efficacior esset ; \textbf{ et ille principans } propter abundantiorem potentiam & fuesse ayuntado en vn prinçipante e vn sennor mas fuerte seria . \textbf{ Et aquel prinçipe por ma . } yor cunplimiento de poderio meior podria gouernar la çibdat \\\hline
3.2.3 & semper totum illud regnum reducitur \textbf{ in aliquod unum principans . } Ut si in eodem corpore & ca do quier que es gouernamiento natra al sienpre todo aquel gouernamiento es reduzido en algun prinçipe gouernador \textbf{ assi commo si en vn cuerpo son departidos mienbros ordenados } a departidos ofiçios \\\hline
3.2.3 & et in toto uniuerso est \textbf{ unus deus singula regens et disponens . } Si apes etiam & E en essa milma manera en todo el mundo es vn dios \textbf{ que gouierna todas las cosas | e las ordena cada vna a ssu fin . } Et avn las abeias \\\hline
3.2.3 & semper videmus multitudinem quamlibet reduci \textbf{ in unum aliquod principans et gubernans . } Nam sicut naturale est , & que cada vna muchedunbre es reduzida a vn prinçipe \textbf{ e avn gouernador } ca assi commo es cosa natraal \\\hline
3.2.4 & quia homo sic constitutus \textbf{ et multitudo sic principans , } efficacio erit in principando . & Por la qual cola meior sera este tal prinçipado \textbf{ ca el omne assi conpuesto de muchs oios e la muchedunbre } que assi ensseñorea sera mas abiuada \\\hline
3.2.4 & quae requiritur in principante . \textbf{ Tunc enim principans rectam habet intentionem , } si non intendat bonum proprium sed commune : quanto igitur minus intenditur commune bonum , & que deue ser en el prinçipe \textbf{ por que estonçe ha derecha entençion } si non entiende abien propo mas a bien comun . \\\hline
3.2.4 & quasi bonum omnino priuatum , \textbf{ sic intendens recedit } quasi omnino a communi bono . & assi commo bien apropiado el \textbf{ que assi entiende el bien propre } o apartasse much del bien comun . \\\hline
3.2.4 & dubitando , \textbf{ assignans rationes multas , } quod melius sit dominari multitudinem : & que el pho pone en el terçero libro delas politicas dudado \textbf{ e poniendo muchͣs razones para esto } que meior es que much \\\hline
3.2.5 & quam per haereditatem . \textbf{ Nam superficialiter considerata veritate quaesiti , dubitans . } An melius sit regiam dignitatem & que non por heredamiento \textbf{ por que penssada la uerdat desta | questiuo liuianamente el que duda } si es meior de ser la dignidat real \\\hline
3.2.6 & quantus honor sequitur , \textbf{ et quanto honore est dignius intendens commune bonum . } Quare si tyrannus intendit & assi non se puede dezir nin contar quanta honrra se sigue \textbf{ e quanto es digno de honrra | el que entiende en el bien comun . } por la qual cosa si el thirano entiende el su bien propreo siguese \\\hline
3.2.6 & nisi de delectationibus propriis , \textbf{ maxime versatur sua intentio circa pecuniam , credens se per eam posse huiusmodi delectabilia obtinere . } Sed regis intentio versatur circa virtutem , & si non delas sus delectaçonnes propreas . \textbf{ Et por ende la su entençion toda se pone en el auer | o en los dinos creyendo que por ellos puede auer las otras cosas delectables . } Mas la entençion del Rey esta \\\hline
3.2.6 & nisi de delectationibus propriis , \textbf{ videns se esse onerosum et tediosum } ab iis & si non delas sus delecta connes proprias . \textbf{ veyendo se en carga e en aborresçimiento } de aquellos que son en el regno . \\\hline
3.2.7 & Sed si tyrannus dominetur , \textbf{ cum unus sit dominans , } et non intendat nisi bonum proprium ; & Mas si el tirano \textbf{ enssennoreare commo vno sea el señor } e non entienda \\\hline
3.2.9 & ut appareat magis esse procurator communis boni , \textbf{ quam tyrannus quaerens utilitatem propriam . } Octauo decet verum Regem & assi que parezca mas ser procurador del bien comun \textbf{ que tirano que quiere sienpre supro . } Lo octauo conuiene al uerdadero . \\\hline
3.2.10 & quod signum est tyrannidis pessimae . \textbf{ Verus autem Rex econuerso intendens commune bonum , } et cognoscens se diligi ab ipsis & la qual cosaes muy mala señal de tirama . \textbf{ Mas el uerdadero Rey faze todo el contrario | entendiendo en el bien comun } e conosçiendo que el es amado de todos \\\hline
3.2.10 & Verus autem Rex econuerso intendens commune bonum , \textbf{ et cognoscens se diligi ab ipsis } qui sunt in regno , excellentes , et nobiles , & entendiendo en el bien comun \textbf{ e conosçiendo que el es amado de todos } los que son en su regno \\\hline
3.2.10 & et conseruat , \textbf{ videns quod per ipsum , bonum commune , } et bonus status regni , & e mantiene le ueyendo \textbf{ que por el bien comun } e el buen estado del regno \\\hline
3.2.10 & aeque de facili eius potentiae resisti : \textbf{ nam tunc quaelibet partium timens alteram , } neutra insurgit contra tyrannum . & Entre tanto non puede de ligero contradezer al su mal poderio \textbf{ por que estonçe cada vna delas partes ha miedo dela otra } e ninguna dellas non se leunata contra el tyrano . \\\hline
3.2.11 & Primo enim potest hoc accidere ex magnanimitate , \textbf{ ut quia insurgens est tanti cordis } ut nihil reputet magnum . & La primera sale le de sabiduria e de grant entendimiento . \textbf{ por que cuyda que podra fallar tantas carreras e tantas maneras } por que pueda matar el tyrano \\\hline
3.2.11 & Tertio potest contingere ex potentia , \textbf{ ut si insurgens confidat de aliis ciuibus , } et credat se iuuari ab eis , & ¶ La . iij es que esto puede ser por fiança \textbf{ que si el que se leunata contra el tyrano | fiare delos otros çibdadanos } e creyere que sera ayudado dellos \\\hline
3.2.12 & Intendunt enim pecuniam : \textbf{ quia quilibet principans } intendit augmentum illius , & Aopmero entienden en los dineros . \textbf{ por que cada vn prinçipan te entiende acresçentamiento de aquella cosa . } por la qual cosa se vee ser prinçipe . \\\hline
3.2.12 & ideo non credit se multitudini , \textbf{ sed semper dubitans de furia populi , } maxime solicitatur circa custodiam corporis . & non fia dela muchedunbre de los çibdadanos \textbf{ massienpre dubda dela sanna e dela ira del pueblo . } e por ende ha muy grant acuçia dela guarda del su cuerpo . \\\hline
3.2.12 & et quare nunquam hylarem vultum ostenderet . \textbf{ Tyrannus ille volens reddere causam quaesiti , } eum expoliari fecit , & que nunca mostraua la cara alegte \textbf{ e aquel tirano quariendo dar razon desto fizo despoiar a su hͣmano } e fizola tar \\\hline
3.2.12 & Et cum frater eius timore horribili inuaderetur , \textbf{ timens ab acuto gladio vulnerari , } et a ballistis perfodi , & Et estonçe commo aquel su hͣrmano tomasse grant espanto \textbf{ e grant temor | e ouiesse miedo de ser ferido del cuchiello } e llagado delas ballestas dixo el tyrano h̃mano gozate \\\hline
3.2.13 & Unde et prouerbialiter dicitur , \textbf{ quod nimis fugans timidum , } vi compellit esse audacem . & onde el prouerbio dize \textbf{ que quien muncho faze foyr al temeroso } por fuerça lo costrange desee oscido en essa misma manera \\\hline
3.2.13 & qui spreto communi bono totum se dedit venereis . \textbf{ Quidam vero dux contemnens eum , } eo quod vitam pecudum elegisset , & tomun todo se dapla lururia \textbf{ e vn prinçipe despciandol } por que suiera vida bescia lacomenol \\\hline
3.2.13 & et peremit ipsum . \textbf{ Sic etiam Dionysius posterior tyrannizans , } et curans magis de gula & por que suiera vida bescia lacomenol \textbf{ e matol aver en essa is mermana diomsio el tipano } que cupana mas dela garganta \\\hline
3.2.13 & Sic etiam Dionysius posterior tyrannizans , \textbf{ et curans magis de gula } quam de bono communi , & e matol aver en essa is mermana diomsio el tipano \textbf{ que cupana mas dela garganta } que del bien comun era despreçiado de los sbraditos \\\hline
3.2.13 & eo quod quasi semper esset ebrius : \textbf{ quidam enim nomine Dion videns ipsum } quasi semper esse ebrium , & e por ende vn omne \textbf{ que auja nonbre dion viendol } que sienpre estaua enbriago \\\hline
3.2.14 & tres modos corruptionis tyrannidis , \textbf{ dicens , } Tyrannidem corrumpi a se , & Ca cuenta el phon enel quinto libro delas politicas tres maneras dela corrupçion dela tiranja \textbf{ e dize que la tiranja corrope de si mismar coronpese desta çirana } e corronpese por El regno ¶ \\\hline
3.2.14 & vel aliquis unus Princeps tyrannizet in populum , \textbf{ gens illa oppressa non valens } sustinere tyrannidem Principis , & enparadoro algun \textbf{ prinçipe vno titaniza en el pueblo aquella gente apremiada non podie do sofrir su tira } maleunatasse e tiraniza contrael prinçipe matandol o echandol del prinçipado . \\\hline
3.2.14 & insurgit et tyrannizat in ipsum , \textbf{ eum perimens vel expellens . } Totus ergo populus efficitur & prinçipe vno titaniza en el pueblo aquella gente apremiada non podie do sofrir su tira \textbf{ maleunatasse e tiraniza contrael prinçipe matandol o echandol del prinçipado . } Et por ende todo el pueblo es fech \\\hline
3.2.15 & etiam modicae . \textbf{ Secundum praeseruans politiam } et regnum regium , & e deuen avn ser defendidos los males pequanos . \textbf{ La segunda cosa que guarda la poliçia } e el gouernamiento del regno es bien vsar \\\hline
3.2.15 & ne repente constituatur aliquis in maximo principatu . \textbf{ Septimum saluans regnum et politiam , } est Regem siue principantem habere dilectionem & que adesora non sea ninguno puesto en muy grant senorio . \textbf{ La vi jncosa que salua el regno | e la poliçia es } que el Rey \\\hline
3.2.15 & Quare si Rex bonum regni diligat , saluabitur regnum ; \textbf{ quia timens ne in regno aduersa contingant , } adhibebit multa consilia qualiter possit & saluat se ha el regno \textbf{ ca temiendo que | cotezccan alguas cosas contrarias en el regno } aura tomar muchs consseios \\\hline
3.2.15 & et periculis imminentibus obuiare . \textbf{ Octauum saluans regnum et politiam , } est habere ciuilem potentiam . & e pueda contradezir alos peligros \textbf{ que pueden acaesçer¶ La . viijn . | cosa que salua el regno } e la poliçia es auer poderio \\\hline
3.2.15 & et transgressoribus iusti . \textbf{ Nonum maxime saluans regnum , } est esse regem bonum et virtuosum . & que traspassan la iustiçia . \textbf{ La ixͣ cosa que much salua el regno es } que el rey sea bueno e uirtuoso . \\\hline
3.2.16 & sed pro quibus sapiens , \textbf{ et intellectum habens . } Ideo primo videndum est & Mas de aquellas cosas de que el sabio \textbf{ e el que ha buen entendimiento toma consseio } por ende primero deuemos ver \\\hline
3.2.16 & non per se , \textbf{ sed per accidens potest } sub consilia cadere , & non por si mas \textbf{ por algun acçidente } por que sepamos en \\\hline
3.2.17 & quia ( ut dicitur 6 Ethicorum ) \textbf{ consilians siue bene siue male consiliatur , } quaerit aliquid , et ratiocinatur . & en el sesto libro delas ethicas \textbf{ aquel que demanda conseio | si quier demande conseio bien } si quier mal demanda alguna cosa \\\hline
3.2.17 & non enim consiliatur scriptor \textbf{ ( nisi sit omnino ignorans ) qualiter debeat scribere litteras , } quia hoc sufficienter determinatum est & Ca el esceruano non toma consseio \textbf{ commo esceruir a las letris | si non fuere del todo nesçio } que non sepa en \\\hline
3.2.17 & capitulo de Institutis antiquis , \textbf{ commendans Romanos consiliatores , } ait , quod fidum et altum erat & establesçimientos antigos \textbf{ alabando alas consseieros de roma } dize que de grant fe \\\hline
3.2.17 & 2 Rhetor’ \textbf{ quidam poeta nomine Alexander videns } Priamum in consiliis esse secretarium et veracem , commendans eum dicebat , & en el terçero libro de la rectorica \textbf{ que vn poeta | que auie nonbre alixandre veyendo } que primero era muy guardado enlos conseios \\\hline
3.2.17 & quidam poeta nomine Alexander videns \textbf{ Priamum in consiliis esse secretarium et veracem , commendans eum dicebat , } Iste est qui consuluit . & que auie nonbre alixandre veyendo \textbf{ que primero era muy guardado enlos conseios | e muy uerdadero } alabandolo dize del este es aquel que conseia \\\hline
3.2.18 & sed facere consiliata velociter . \textbf{ Omnia autem illa quae habere debet bene persuadens } et bene creditiuus apparenter , & mar ala Real magestado das aquellas cosas \textbf{ que deue auer aquel | que bien amonesta } e bien razona \\\hline
3.2.18 & Bene igitur dictum est \textbf{ quod quicumque bene persuadens , } habet apparenter , & Et pues que assi es bien dicho es \textbf{ que todas aquellas colas } que ha el \\\hline
3.2.18 & quod contingit , \textbf{ si dicens sit bonus , } vel credatur bonus . & que oyen puede ser de parte de aquel que las dize e esto \textbf{ contesçe quando el que lo dize es en ssi bueno } o creen los omes que es bueno . \\\hline
3.2.18 & ex parte dicentis ; \textbf{ nam quia dicens creditur esse bonus , } cum tales mentiri nolint , & mas esta fe uiene de parte de aquel que dize e fabla . \textbf{ ca por que el dezidor es creydo | que es bueno } e los bueons non quieren mentir \\\hline
3.2.18 & quod contingit \textbf{ si proferens sermones } sit beniuolus et amicus . & de parte de los dizidores la qual cosa contesçe \textbf{ si el que dize las palabras } fuereentre los oydores bien quarido \\\hline
3.2.18 & quod contingit , \textbf{ si dicens sit prudens } vel credatur esse prudens . & la qual cosa contesçe \textbf{ quando el dezidor es sabio } o es tenido por sabio . \\\hline
3.2.18 & vel quod credatur esse prudens . \textbf{ Omnis ergo bene persuadens , } vel omnis ille cuius dictis creditur & o que sea tenido por sabio . \textbf{ Et pues que assi es todo buen amonestador o razonador } o todo aquel \\\hline
3.2.18 & Itaque cum dictum sit \textbf{ quod qui bene persuadens , } et ille cui fides adhibetur , & Et por ende commo sea dicho \textbf{ que el que es buen amonestador e razonador } e aquel a que los omes creen \\\hline
3.2.19 & secundum quem dominatur habet saluari et corrumpi : \textbf{ ut eligens optimum modum principandi , } ferat leges iustissimas , & enssennorea se ha de saluar o corronper . \textbf{ por que escogiendo la meior manera de prinçipar o de } enssennorear \\\hline
3.2.20 & et quam paucissima arbitrio iudicum committere . \textbf{ Has autem tres rationes tangit Philosophus 1 Rhetoricorum dicens } quod maxime quidem contingit & e en aluedrio de los iuezes . \textbf{ Et estas tres cosas tanne elpho | en el primero libro de la rectorica } diziendo \\\hline
3.2.21 & sic debet se habere inter partes litigantes , \textbf{ sicut lingua volens discernere de saporibus , } vel sicut quilibet alius sensus volens discernere & que contienden \textbf{ commo la lengua | quando quiere iudgar de los sabores } o si commo cada vno de los otros sesos \\\hline
3.2.21 & sicut lingua volens discernere de saporibus , \textbf{ vel sicut quilibet alius sensus volens discernere } de proprii sensibilibus , & quando quiere iudgar de los sabores \textbf{ o si commo cada vno de los otros sesos | quando quieren iudgar delas otras cosas } que sienten propriamente \\\hline
3.2.21 & infecto aliquo humore , recte iudicat , \textbf{ dicens amarum esse amarum , } et dulce dulce . & por algun humor iudga derechͣmente diziendo \textbf{ que lo amargo es amargo } e lo dulçe es dulçe . \\\hline
3.2.21 & colera vel phlegmate vel aliquo alio humore , \textbf{ tunc non quasi existens in medio , } sed contracta ad alterum contrariorum , & o por alguno de los otros humores estonçe \textbf{ assi commo aquella que non esta en medio } mas esta trayda a \\\hline
3.2.21 & peruersae iudicat , \textbf{ dicens dulce esse amarum , } et econuerso , & algundelas partes contrarias iudga mal diziendo \textbf{ que lo dulçe es amargo } e lo amargo es dulçe . \\\hline
3.2.21 & quando est medius \textbf{ inter litigantes non declinans ad alteram partem , } quasi regula recta decet & assi el uiez mientra el esta medianero \textbf{ entre las partes | que contienden non se enclinando a ninguna delas partes es } assi commo regla derecha diziendo \\\hline
3.2.21 & Si autem discit ab eis quid iustum , \textbf{ hoc est per accidens , } inquantum allegant leges conditas a legumlatore . & Mas delas partes aprende qual es el fecho \textbf{ o qual non es el fecho . Mas si aprende delas partes quales el derecho esto es auentura en quanto } allegansas leyes establesçidas \\\hline
3.2.21 & nihil oportet dici in iudicio \textbf{ nisi pertinens ad rem vel ad negocium , } de quo est litigium : & non se deue dezir ninguna cosa en iuizio \textbf{ si non lo que pertenesçe a aquel fecho | o a aquel negoçio } de que contienden \\\hline
3.2.23 & Ideo dicitur 1 Rheto’ \textbf{ quod iudicans potius debet } respicere ad legislatorem , & Et por ende dize el pho enel primero libro de la rectorica \textbf{ que el iuez } mas deue tener mientes al ponedor dela ley \\\hline
3.2.23 & quam ad leges . \textbf{ Tertium inclinans ad pietatem } est pius intellectus legum . & que alas leyes . \textbf{ Lo terçero que inclina al iuez a piedat } e es piadoso entendimiento delas leyes . \\\hline
3.2.23 & hoc est ergo quod dicitur 1 Rhetor’ \textbf{ quod iudicans non debet } respicere ad verba legum , & enl primero libro de la rectorica \textbf{ que el iuez non deue parar mientes } alas palabras delas leyes mas al \\\hline
3.2.23 & dicitur 1 Rhet’ \textbf{ quod iudicans debet } aspicere non ad actionem , & en el primero libro de la rectorica \textbf{ que el que iudga non deue tener } mientesa la obra mas ala entençion . \\\hline
3.2.23 & sed ad electionem . \textbf{ Quintum inducens ad misericordiam , } est multitudo bonorum operum . & mientesa la obra mas ala entençion . \textbf{ Lo quinto que enduze el iuez ami bicordia } es muchedunbre de buean sobras . \\\hline
3.2.23 & ideo dicitur 1 Rhetor’ \textbf{ quod iudicans non debet respicere ad partem , } sed ad totum . & en el primero libro de la rectonca \textbf{ que el uiez non deue catar ala parte mas al todo | nin deue tener mientes a vna obra } que fizo mas a todas las buenas obras \\\hline
3.2.23 & Istud itaque sextum inclinatiuum ad pietatem \textbf{ respiciens diuturnitatem temporis , } non est idem cum quinto , & Et por ende estas esta razon que inclina al iues a piedat \textbf{ catando alongamiento de tp̃o non es vna } nin es essa misma \\\hline
3.2.23 & est cum ipso misericorditer agendum , \textbf{ et magis respiciendum est ad multum et ad totum tempus praecedens , } quam ad modicum & deue el iuez con el passar mibicordiosamente . \textbf{ Et en tal cosa commo esta mas deue omne tener mientes alo much | assi commo a todo el tp̃o passado } que alo poco \\\hline
3.2.23 & semper in multo tempore praecedenti . \textbf{ Septimum inclinans ad pietatem , } est excessus bonitatis supra malitiam . & mas qual fue en el mucho t pon que passo¶ La . vij̊ \textbf{ que enduze al iuez a piedat } es sobrepuiança de bondat sobre la maldat . \\\hline
3.2.23 & Ideo Philos’ 1 Rhet’ \textbf{ volens iudicantes } ad misericordiam adducere & por ende el pho en el primero libro de la \textbf{ rectorica quariendo enduzir los iuezes a misericordia } contra los que yerran contra ellos dize \\\hline
3.2.23 & si credat magis peccantem iudicari velle sermone quam opere . \textbf{ Decimum inclinans ad clementiam , } est subiectio et humiltas delinquentis : & mas iudgado e castigado \textbf{ por palabta que por obra . | Lox̊ que inclina el iues a mihicordia } e a piedat es la \\\hline
3.2.23 & est subiectio et humiltas delinquentis : \textbf{ nam si delinquens omnino se subiicit , } omnino se humiliat , & subiectonn e la humildat del que peca . \textbf{ ca si el que peca en toda manera se somete } e se homilla e se pone todo en el aluedrio del iuez \\\hline
3.2.24 & Quare si ius naturale dictat fures et maleficos esse puniendos , \textbf{ hoc praesupponens ius positiuum procedit ulterius , } determinans qua poena sint talia punienda . & e los mas fechores sean castigados \textbf{ e ayan pena el derech | positiuo prisu pone esto va adelante } determinando de qual pena de una ser \\\hline
3.2.24 & hoc praesupponens ius positiuum procedit ulterius , \textbf{ determinans qua poena sint talia punienda . } Hoc viso quantum & positiuo prisu pone esto va adelante \textbf{ determinando de qual pena de una ser | tales cosas castigadas o condep̃nadas . } ¶ Esto uisto quanto pertenesçe alo presente podemos mostrar dos departimientos \\\hline
3.2.29 & sed lex \textbf{ quia est aliquid pertinens ad rationem , } videtur dicere intellectum solum : & mas la ley por que es alguna cosa \textbf{ que parte nesçe } a razon paresçe \\\hline
3.2.30 & et humanam fuerit \textbf{ expediens } dare legem euangelicam et diuinam , & ø \\\hline
3.2.30 & ut vitentur adulteria . \textbf{ Secunda via ostendens necessariam esse legem euangelicam et diuinam , } sumitur ex parte cognitionis nostrae , & por que sean escusados los adulterios . \textbf{ ¶ La segunda razon que muestra la ley en angelical } e diuinal ser neçessaria es tomada deꝑte del nuestro conosçimiento \\\hline
3.2.30 & Quare cum in humanis iudiciis cadere possit dubieras et error , \textbf{ expediens fuit lex euangelica } et diuina & cosa muy aprouechable \textbf{ e muy neçessaria fue la ley e una gelical e diuinal } en la qual non puede caer yerro . \\\hline
3.2.31 & cum disputat contra Hippodamum , \textbf{ utrum sit expediens ciuitatibus } innouare patrias leges , & quando disputa contra ypodomio \textbf{ si es cosa conuenible alas çibdades } de renouar las leyes dela tierra \\\hline
3.2.31 & ( ut supra tetigimus ) \textbf{ quod inueniens aliquam legem ciuitati proficuam , } honorem acciperet in ciuitate illa ; & assi commo dixiemos de suso \textbf{ que aquel que fallasse algunan ley a prouechosa ala çibdat } resçibiesse honrra en aquella çibdat . \\\hline
3.2.31 & ab eo reputabatur \textbf{ fugiens reus homicidii : } dicebat enim legislator & si suye tenien le \textbf{ por culpado del omiçidio . } Ca dizia el fazedor dela ley \\\hline
3.2.32 & quod bonorum illorum sit potius . \textbf{ Narrat quidem Philosophus 3 Politic’ volens diffinire } quid sit ciuitas , & qual de aquellos bienes es el meior . \textbf{ Et cuenta el philosofo enel terçero libro delas politicas | quariendo de el arar } que cosa es la çibdat seys bienes \\\hline
3.2.32 & et propter non iniustum pati . \textbf{ Nam quia homo unus solitariam vitam ducens , } non est sufficiens resistere impugnantibus , & e por que non pudiessen resçebir tuerto delos enemigos . \textbf{ Por que vn omne biuienda solo } apartadamente non se podria \\\hline
3.2.32 & ut homo qui solitarius se non potest tueri \textbf{ ab hostibus existens pars multitudinis , } tute et absque formidine viueret . & non se podria defender de los enemigos . \textbf{ Et seyendo parte de muchedunbre de çibdat } puede benir seguro \\\hline
3.2.32 & de leui patere potest , \textbf{ qualis debeat esse populus existens } in ciuitate et regno . & e que cosaes regno de ligero puede paresçer \textbf{ qual deue ser el pueblo que es en el regno e enla çibdat . } Ca si la çibdat e el regno son ordenados \\\hline
3.2.33 & nec unde ei inuideat \textbf{ videns se ei quasi aequalem existere , } et non esse magnum excessum inter ipsos . & nin donde aya enuidia al otro \textbf{ ueyendo que es su egual e veyendo } qua non ay grant auna taia entre el vno e el otro . \\\hline
3.2.34 & quantum sit utile \textbf{ et expediens populo } obedire Regibus et Principibus , & por tres razones \textbf{ quanto es prouechoso e conuenible al pueblo de obedesçer alos Reyes } e guardar las leyes . \\\hline
3.2.34 & et ei cuius est leges ferre . \textbf{ Quare si principans recte regat populum sibi commissum , } quia intentio eius est & e a aquel a quien pertenesçe de poner las leyes . \textbf{ Por la qual cosa si el prinçipe gouernar e derechamente el pueblo qual es acomne dado } por que la su entençion es enduzir los otros a uirtud . \\\hline
3.2.34 & Imo tanto magis est \textbf{ hoc expediens nobilibus quam ignobilibus , } quanto decentius est & ante digo que tanto mas es meester tan bien alos nonbles \textbf{ commo alos non nobles } quanto mas conuenible es \\\hline
3.2.35 & Ita autem secundum Philosophum \textbf{ in principio 2 Rhet’ est tristitia proueniens } ex appetitu apparentis punitionis , & Ca la sanna segunt el philosofo \textbf{ enl comienço del segundo libro de la | rectorica est steza } que viene del appetito de dar pena manifiesta \\\hline
3.2.36 & Ideo dicitur 7 Politicorum \textbf{ quod bene operans nulli parcit : } quia nec pro patre , & Por ende dize el philosofo en el vij̊ . libro delas politicas . \textbf{ que el que bien obra } e el iusto non perdona a ninguno por iustiçia . \\\hline
3.2.36 & Timet igitur tunc quilibet ex populo forefacere , \textbf{ cogitans se non posse punitionem effugere . } Imo , ut vult Philos’ 7 Polit’ & Et pues que assi es cada vno del pueblo teme de mal fazer cuydando \textbf{ que non podra escapar dela pena . } Ante assi conmo dize el philosofo \\\hline
3.3.3 & Quando igitur in homine videmus , \textbf{ quod sit vigilans oculis , } erectus ceruice , & Et pues que assi es quando nos veemos en el omne \textbf{ que es bien despierto | e tiene los oios bien abiertos } e ha la çeruiz derecha \\\hline
3.3.3 & compactus in neruis et musculis , \textbf{ habens longa brachia , } et latum pectus , & e los muslos duros \textbf{ e ha los braços luengos e los pechos anchos . } estos tales deuemos iudgar \\\hline
3.3.4 & non sic de facili percuteretur ab arcu , \textbf{ sic homo se circumuoluens , } non sic de facili vulneratur ab hoste . & non se podrie ferir tan de ligero del uallestero \textbf{ assi el ome andando | e volviendosse de vna parte a otra } non puede el enemigo de ligero ferirle . \\\hline
3.3.4 & bene mori in bello , \textbf{ quando iuste bellans , } ut pro defensione patriae , & cada vno bien morir en la batalla \textbf{ quando lidia derechamente } e assi commo por defendimiento de la tierra \\\hline
3.3.4 & horrere sanguinis effusionem . \textbf{ Nam si quis cor molle habens , } muliebris existens , & aborresçer el derramamiento de la sangre \textbf{ por que si alguno ouiere el coraçon muele } e fuere assi commo mugeril \\\hline
3.3.4 & Nam si quis cor molle habens , \textbf{ muliebris existens , } horreat effundere sanguinem ; & por que si alguno ouiere el coraçon muele \textbf{ e fuere assi commo mugeril } aborresçra esparzer la sangre \\\hline
3.3.4 & industria protegendi , \textbf{ et feriendi valde est expediens bellatoribus . } Qualiter , autem talis industria habeatur , et qualiter sit feriendum hostem , & e de se defender \textbf{ e de ferir a los otros | mucho es conuenible a los lidiadores } Mas en qual manera tal sabiduria se puede auer \\\hline
3.3.5 & Huiusmodi autem opinionis visus est \textbf{ esse Vegetius , dicens : } Numquam credo potuisse dubitari & Et desta opinion fue \textbf{ vegeçio diziendo assi . } Creo que ninguno nunca pudo dubdar \\\hline
3.3.8 & Modum autem , \textbf{ et quantitatem fossarum tradit Vegetius dicens , } quod si non immineat magna vis hostium , & mas solamente quieren y estar vna noche o por poco tienpo non conuiene de fazer tantas guarniçiones . \textbf{ Mas la manera e la quantidat de las carcauas pone la vegeçio } diziendo que si paresciere grant fuerça de los enemigos . \\\hline
3.3.9 & aut Princeps , aut Dux exercitus \textbf{ qui debet esse vigilans , } sobrius , prudens , et industris , & o el cabdiello de la hueste \textbf{ que deue ser acucioso e mesurado e sabio e entendido } deue penssar \\\hline
3.3.9 & Et tunc dux sobrius , \textbf{ et vigilans prout viderit suum exercitum } in his conditionibus abundare , & Estonçe el cabdiello de la hueste mesurado \textbf{ e en viso segunt que viere la su hueste } ha conplimiento en estas seys condiçiones \\\hline
3.3.9 & sol sit oppositus faciebus eorum , vel hostium ; \textbf{ et utrum sit aliquis ventus flans } et eleuans puluerem contra ipsos , & es el sol contrario a las caras de los enemigos \textbf{ e si se leuanta algun viento } que faga poluo \\\hline
3.3.9 & et utrum sit aliquis ventus flans \textbf{ et eleuans puluerem contra ipsos , } vel contra aduersarios : & e si se leuanta algun viento \textbf{ que faga poluo } contra si o contra los enemigos . \\\hline
3.3.10 & In galea enim centurionis scriptae erant literae aliquae , \textbf{ vel signum aliquod euidens ; } quod respicientes decani & eran escerptas letras \textbf{ algunas o alguna señal manifiesta . } a la qual catando los deanes conosçian al su senora propreo \\\hline
3.3.10 & cuiusdam deuictum esse a bellatoribus paucis , \textbf{ eo quod vexillifer fraudem committens } velauit vexillum & fue vençido de pocos lidiadores . \textbf{ por que el alferez | que leuaua la seña fizo falsedat } encubriendo la seña \\\hline
3.3.10 & portare scutum ad se protegendum , \textbf{ vigilans , agilis , sobrius , } habens omnium armorum experientiam ; & Rodearse e cobrirse del escudo \textbf{ para se guardar e despierto e vigilante e ligero e mesurado } e que aya prueua de todas las armas \\\hline
3.3.10 & vigilans , agilis , sobrius , \textbf{ habens omnium armorum experientiam ; } ut sciat erudire & para se guardar e despierto e vigilante e ligero e mesurado \textbf{ e que aya prueua de todas las armas } e que sepa ensseñar los lidiadores \\\hline
3.3.10 & Nam ipse armorum nitor terrorem incutit hostibus , \textbf{ ut portans huiusmodi arma credatur } bonus esse bellator . Ipsa enim rubigo armorum in eo & por que el resplandesçimiento de las armas pone grant espanto a los enemigos \textbf{ assi que el que traye tales armas es tenido } por buen lidiador \\\hline
3.3.11 & cui commissa est tantorum vita , \textbf{ debet esse attentus et vigilans , } ne hostes eum inuadere possent & a qui es acomendada la uida de tantos omnes \textbf{ deue ser muy acuçioso e despierto } por que los enemigos non puedan acometer los \\\hline
3.3.13 & Omnis ergo ille modus percutiendi est magis eligendus , \textbf{ secundum quem seriens minus discooperitur et detegitur ; } quia sic feriendo , & Et pues que assi es toda aquella manera de ferir es mas de escoger \textbf{ segunt la qual el que fiere se descubre menos . } por que assi feriendo menor daño le puede contesçer . \\\hline
3.3.15 & quia tunc quiescit in sinistra , \textbf{ et mouetur in dextra vibrans ipsum iaculum ; } quo vibrato vehementius mouet aerem , & Ca estonçe fuelga el omne en la parte esquierda \textbf{ e mueuese en la derecha | que ha de esgrimir el dardo . } el qual esgrimido mueue el ayre mas reziamente \\\hline
3.3.17 & ad ipsa ligna sustinentia murum , \textbf{ apponens huiusmodi ignem et existentes } cum eo debent & que sotiene los muros \textbf{ el que pone el fuego } e los que estan con el \\\hline
3.3.18 & Nam in omni tali machina est \textbf{ dare aliquid trahens } et eleuans virgam machinae , & puedense adozer a quatro maneras . \textbf{ Ca en todo tal engeñio es de dar alguna cosa que traya } e leuante el pertegal del engennio \\\hline
3.3.18 & dare aliquid trahens \textbf{ et eleuans virgam machinae , } ad quam coniuncta est funda , & Ca en todo tal engeñio es de dar alguna cosa que traya \textbf{ e leuante el pertegal del engennio } al qual pertegal esta \\\hline
3.3.18 & quaedam cassa immobiliter \textbf{ adhaerens virgae plena lapidibus et arena , } vel plena plumbo , & que non se mueue \textbf{ e ayuntada al pertegal llena de piedras o de arena o llena de plomo } o de algun otro cuerpo mas pesado . \\\hline
3.3.18 & contrapondus mobiliter \textbf{ adhaerens circa flagellum , } vel circa virgam ipsius machinae , & que ha el contrapeso mouible \textbf{ que se llega çerca el pertegal del engennio } e bueluesse e tornasse çerca del pertegal . \\\hline
3.3.18 & vel circa virgam ipsius machinae , \textbf{ vertens se circa huiusmodi virgam . } Et hoc genus machinae Romani pugnatores & que se llega çerca el pertegal del engennio \textbf{ e bueluesse e tornasse çerca del pertegal . } Et esta manera de engeñio llaman los lidiadores romanos bifan \\\hline
3.3.18 & quod Tripantum nuncupant , \textbf{ habens utrumque contrapondus , } unum infixum virgae , & que llaman algunos . \textbf{ que ha el contrapeso doblado tan bien de vno commo de otro } e es conpuesto de amos los contrapesos tan bien del fincado commo del mouible . \\\hline
3.3.18 & unum infixum virgae , \textbf{ et aliud mobiliter se vertens circa ipsam : } hoc enim ratione ponderis infixi rectius proiicit , & Et el vno es fincado en el pertegal . \textbf{ Et el otro es mouible | e bueluesse çerca del pertegal . } Et este engeñio \\\hline
3.3.19 & dato quod quis non possit pertingere usque ad muros eius . \textbf{ Nam quia huiusmodi trabs habens caput sic ferratum retrahitur et impingitur , } poterit percuti murus ipsius munitionis obsessae , & puesto que non puedan llegar a los muros della . \textbf{ Ca por que esta viga ha la cabeça | assi ferrada tiranla afuera } e despues enpuxan la \\\hline
3.3.19 & erigendum est aliquod lignum in altum , \textbf{ faciens tantam umbram ; } et secundum altitudinem illius ligni & es de alçar vn madero en alto \textbf{ segunt la quantidat de aquella sonbra . | Et este madero fara tan grand sonbra en aquel alto commo es el muro } e segund la alteza del madero \\\hline
3.3.19 & si vero visus protendatur magis basse , cum tabula sit existente ad pedes , \textbf{ et sic iacens in terra , } elonget se ab aedificio praedicto , & mas con la tabla en tal manera \textbf{ que catando } por ençima de la tabla vea la mas alta parte del muro \\\hline
3.3.20 & et ante huiusmodi portam ponenda est cataracta \textbf{ pendens annulis ferreis } undique etiam ferrata , & Et ante destas puertas se deue poner la puerta de la trayçion \textbf{ que esta colgada con cadenas de fierro } e ella toda cobierta de fierro \\\hline
3.3.20 & undique etiam ferrata , \textbf{ prohibens ingressum hostium , } et incendium ignis . & e ella toda cobierta de fierro \textbf{ por que non puedan entrar los enemigos nin puedan fazer daño con el fuego . } Ca si los que cercan quisieren llegar a quemar las puertas de la fortaleza . \\\hline
3.3.20 & Rursus supra cataractam debet \textbf{ esse murus perforatus recipiens ipsam , } per quem locum poterunt proiici lapides , & Otrossi sobre las puertas de la trayçion \textbf{ deue ser el muro foradado de guisa | que la puedan leuantar arriba et baxarla cada que quisieren . } Et por aquel logar pueden lançar piedras . \\\hline
3.3.21 & Salis etiam multitudo multum est \textbf{ expediens munitioni obsessae , } eo quod ad multa sit utilis . & e mas dura que tedos los otros granos . \textbf{ Avn basteçer de grand conplimiento de carnes saladas e de mucha sal . } Ca la muchedunbre de la sal mucho es prouechosa \\\hline
3.3.23 & Debet autem sic ordinari lignum illud , \textbf{ ut ligamentum retinens } ipsum possit deprimi , et eleuari : & assi ser ordenado \textbf{ que con el atadura } que el tiene se pueda alçar \\\hline
3.3.23 & habere ferrum quoddam curuatum \textbf{ ad modum falcis bene incidens , } quod applicatum ad funes retinentes vela ; & Et lo sexto suelen los marineros auer vn fierro coruo bien agudo \textbf{ e bien taiante a manera de foz . } el qual echan a las cuerdas del maste \\\hline

\end{tabular}
