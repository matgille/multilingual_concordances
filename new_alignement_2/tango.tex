\begin{tabular}{|p{1cm}|p{6.5cm}|p{6.5cm}|}

\hline
1.1.1 & Hanc autem rationem videtur \textbf{ tangere Philosophus 1 Ethicorum , } cum ait , & E esta rrazon tañe el philosopho en \textbf{ eL segundo libro delas ethicas quando } dizeque conplidamente se dize dela \\\hline
1.1.4 & Quod falsum est . \textbf{ Nam ( ut supra tangebatur ) } omnibus volentibus viuere recte , & la qual cosa magnifiestamente es falsa . \textbf{ Ca asy commo ya dixiemos a todos } los que quirien bien beuir derechamente \\\hline
1.1.7 & impetrauit a Deo , \textbf{ ut quicquid tangeret , } fieret aurum . & E asi commo dla fabliella gano de dios \textbf{ que quanto el tanxiesse que todo se le tornasse oro } asi que quanto el tannia todo se le tornaua oro \\\hline
1.1.7 & in singulis partibus corporis , \textbf{ etiam ore nihil tangere poterat , } quin conuertetur in aurum . & Et por que el seso del tanner es en todas las partes del cuerpo \textbf{ e ahun en la boca non podia tanner ninguna cosa | que tonda non se le } tornauaoro \\\hline
1.2.3 & quas enumerauimus , \textbf{ quas tangit Philosophos } circa finem 2 Ethicor’ . & que ya contamos de suso . \textbf{ las quales pone el philosofo } en el segundo libro de las ethicas . \\\hline
1.2.12 & quadruplici via venari , \textbf{ secundum quatuor quae tanguntur de Iustitia in 5 Ethicorum . } Prima via sumitur & en quatro maneras \textbf{ segunt quatro cosas | que tanne el philosofo dela iustiçia en el quimo libro delas ethicas ¶ } La primera manera se toma de parte dela persona del Rey ¶ \\\hline
1.2.13 & sicut pericula bellica ; \textbf{ nec ex tactu aquae sic sentimus } et imaginamur laesionem , & que los otros lo vno \textbf{ por que son mas manifiestos . | Et lo otro por que lo sentimos mucho mas . } Et lo otro por que ymaginamos \\\hline
1.2.13 & et imaginamur laesionem , \textbf{ sicut ex tactu gladii . } Pericula ergo bellica & Et lo otro por que lo sentimos mucho mas . \textbf{ Et lo otro por que ymaginamos } que tales periglos fazen grant dolor \\\hline
1.2.15 & sed non possumus gustare , \textbf{ et tangere , } nisi nobis coniuncta . & mas non podemos gostar nin tanner \textbf{ si non las cosas que son ayuntadas anos . } Por ende en estos dos sesos \\\hline
1.2.15 & Reliqua autem animalia in gustu , \textbf{ et tactu delectantur per se : } in aliis vero sensibus & Mas laso trisa inalias se delectan engostar \textbf{ e en el taner por si . } Et en los otros sesos se delectan \\\hline
1.2.15 & Immo in ipsis delectationibus nutrimentalibus \textbf{ magis delectamur in tactu , } quam in gustu . & delectaconnes delas viandas mas nos \textbf{ delectamostanniendo que gostando . } Ca comiendo e beuiendo \\\hline
1.2.16 & et Principes temperatos esse . Est enim intemperantia \textbf{ ( ut ibidem tangitur ) } vitium maxime bestiale , puerile , & et alos prinçipes de ser tenprados . \textbf{ por que la destenpranca assi commo } y dize el philosofo es pecado muy bestial e de bestia \\\hline
1.2.21 & Philosophus 4 Ethicorum capitulo De magnificentia , \textbf{ tangit sex proprietates magnifici , } quas habere decet Reges et Principes . & euedes saber que el philosofo enl quarto libro delas ethicas capitulo dela magnificençia pone seys \textbf{ propiedadesdel magnifico las quales conuiene alos Reyes e alos prinçipes auer ¶ } La primera es que el magnifico es semeiante al sabio \\\hline
1.2.29 & quam sint . \textbf{ Hanc autem rationem tangit Philosophus } in cap’ praetacto , & por que somos inclinados alos nuestros bienes propreos paresçen nos mayores de quanto son . \textbf{ Et esta razon tanne el philosofo en la auctoridat } que dicha es \\\hline
1.2.29 & ne homines sint aliis onerosi . \textbf{ Hanc autem rationem tangit Philosophus in eodem 4 Ethicorum dicens , } declinandum esse in minus & por que los omes non sean alos otros pesados e guaues . \textbf{ Et esta razon misma tanne el philosofo en esse mismo quarto libro delas ethicas } o dize que deuemos \\\hline
1.3.4 & de bono respectu amoris , desiderii , et delectationis , \textbf{ ut supra tangebatur , } intelligendum est de malo respectu odii , abominationis , et tristitiae . & en conparaçion del amor e del desseo e dela delectaçion \textbf{ assi commo dessuso dixiemos . } assi de uemos entender del mal \\\hline
1.3.8 & per quae huiusmodi tristitia vitari possit . \textbf{ Videtur autem Philosophus tria remedia tangere , } per quae tristitia vitatur ; & por los quales podamos escusar esta tal tristeza e esquiuar la . \textbf{ Et paresçe que el philosofo tanne tres remedios } por los quales la tristeza se puede esquiuar . \\\hline
1.3.8 & Haec enim ratio licet \textbf{ tangatur 9 Ethicor’ } dicitur tamen fuisse Platonis , & Mas esta razon commo quier que fuesse tannida \textbf{ enel nono delas ethicas . } Empero dize \\\hline
1.4.1 & quidam vituperabiles . \textbf{ Inter alia quidem quae tangit Philosophus } de iuuenibus 2 Rhetoricorum , & e algunas de denostar . \textbf{ Mas entre las otras cosas | que tanne el philosofo de los mançebos } en el segundo de la rectorica tanne seys costunbres de loar \\\hline
1.4.1 & de iuuenibus 2 Rhetoricorum , \textbf{ tangit sex mores laudabiles , } et sex vituperabiles . & que tanne el philosofo de los mançebos \textbf{ en el segundo de la rectorica tanne seys costunbres de loar } e seys de denostar ¶ \\\hline
1.4.1 & quare de facili erubescunt , \textbf{ Philosophus tamen secundo Rhetoricorum aliam causam videtur tangere , } quare iuuenes sunt naturaliter verecundi . & uerguenna \textbf{ empero el philosofo tanne otra razon en el segundo libro dela rectorica } por que los mançebos son natraalmente uergoñosos . \\\hline
1.4.1 & non multiplicarent in debitos et pios usus , \textbf{ ut supra in tractatu de liberalitate sufficienter tetigimus . } Decet etiam eos & e por las quales los preçian en las non espender en vsos o en obras conuenibles e piadosas \textbf{ assi commo dessuso dixiemos | e tractamos suficientemente en el terctado dela liberalidat } ¶avn conuiene alos Reyes en \\\hline
1.4.2 & sic enumerare possumus sex vituperabiles : \textbf{ quas etiam tangit Philosophus 2 Rhetoricorum . } Primo enim iuuenes sunt passionum insecutiui . & que son de denostar \textbf{ las quales pone el philosofo | en el segundo libro de la rectorica¶ } Lo primero son los mançebos \\\hline
1.4.3 & Philosophus autem 2 Rhetoricorum , \textbf{ inter alios mores quos tangit de senibus , } enumerat sex uituperabiles mores . & Ca el philosofo en el segundo libro de la rectorica \textbf{ entre las otras costunbres | que tanne de los uieios cuenta seys costunbres } que son de denostar \\\hline
1.4.3 & Posset autem una causa assignari , \textbf{ quae quasi communis est ad omnia tacta . } Dictum est enim senes esse frigidos . & Et por esta razon aceesçe alos uieios de ser desuergoncados \textbf{ Mas puede aqui ser fallada vna razon que es a comun a todas estas cosas de suso dichͣs . } Ca dicho es de suso \\\hline
1.4.4 & Videtur autem Philosophus 2 Rhetoricorum , \textbf{ circa senes tangere quatuor mores , } qui possunt esse laudabiles . & fincanos de poner las costunbres dellos qson de loar \textbf{ Mas paresçe que el philosofo en el segundo libro dela rectorica pone quatro costunbres de los uieios } que pueden ser de loar ¶ \\\hline
2.1.1 & inter alias rationes , \textbf{ quas tangit , } probantes hominem naturaliter & en el primero libro delas politicas entre las otras razones \textbf{ que tanne } por las quales praeua el omne es naturalmente aial aconpannable \\\hline
2.1.1 & et nolentes ciuiliter viuere , \textbf{ ut supra in primo libro tetigimus , } et ut infra tangentur , & assy commo çibdadanos non biuen commo omes \textbf{ assi conmodixiemos de sulo en el primero libro } e avn diremos adelante . \\\hline
2.1.1 & ut supra in primo libro tetigimus , \textbf{ et ut infra tangentur , } quasi non viuunt ut homines ; & assi conmodixiemos de sulo en el primero libro \textbf{ e avn diremos adelante . } Et pues que assi es esto contesçe alos omes \\\hline
2.1.2 & et puerorum , siue filiorum . \textbf{ Nam , ut tangebatur , } crescentibus collectaneis & las quales se fizieron \textbf{ de muchedunbre de mietos e de fijos . } Ca assi commo dicho es de suso cresçiendo los nietos e los fijos de los fijos \\\hline
2.1.4 & Ad praesens autem sufficiat \textbf{ in tantum tangere de regno et ciuitate , } in quantum eorum notitia & Mas quantoalo presente abasta de dezir en \textbf{ tantodel regno e dela çibdat } en quanto el conosçimiento dellos \\\hline
2.1.4 & et ad sciendum qualiter sit regenda . \textbf{ Nam ( ut superius tangebatur ) } in hoc secundo libro intenditur principaliter regimen domus , & Et para saber en qual manera se ha de gouernar la \textbf{ casaca assi commo es dicho de suso } en este segundo libro \\\hline
2.1.7 & Communitas autem in vita humana \textbf{ ( ut supra tangebatur ) } ad quadruplex genus reducitur : & Mas la comunidat en la uida humanal \textbf{ assi commo dicho es dessuso } aduze se aquetro linages o a quatro maneras ¶ \\\hline
2.1.7 & est prior vico , regno , et ciuitate . \textbf{ Hanc autem rationem tangit Philosophus } 8 Ethic’ dicens : & nin dela çibdat nin del regno . \textbf{ Et esta razon tanne el philosofo } en el octauo libro delas ethas \\\hline
2.1.7 & homo naturaliter est animal coniugale . \textbf{ Hanc autem rationem tangit Philosophus 1 Politicorum , et 8 Ethicorum , } ubi probat coniugium competere homini secundum naturam , & aianl conuuigable e ayuntable por casamiento . \textbf{ Et esta razon tanne el philosofo en el primero libro delas politicas | e en el octauo delas ethicas do prueua } que el casamiento conuiene alos omes segunt natura \\\hline
2.1.8 & Ad quod ostendendum adducere possumus duas vias , \textbf{ quas philosophi tetigerunt . } Prima via sumitur ex parte fidei , & Et para esto mostrar podemos dezir dos razones \textbf{ las quales posieron los philosofos } ¶ La primera razon se toma de parte dela fe \\\hline
2.1.8 & Hanc autem rationem videtur \textbf{ tangere Valerius Maximus in libro de factis memorabilibus , } capitulo de institutis antiquis , & Et esta razon tanne vałio el grande \textbf{ en el segundo libro de los fechos rememorables } en el capitulo delas constituçiones antiguas \\\hline
2.1.8 & est quod diuidat et distinguat . \textbf{ Hanc autem rationem tangit Philosophus 8 Ethic’ dicens , } quod quia commune continet & assi commo dela razon del bien propreo es que ayunte e desayunte el vno del otro . \textbf{ Et esta razon pone el philosofo en el viii̊ libro delas ethicas } do dize que por que el bien comunal contiene \\\hline
2.1.21 & ibi triplicem virtutem concurrere , \textbf{ quam tangit Andromicus Peripateticus in libello } quem fecit de virtute . & conuiene de ser tres uirtudes \textbf{ las quales tanne andronico peri patetico } en el libro que fizo delas uirtudes . \\\hline
2.1.21 & Hoc ergo modo regendae sunt coniuges quantum ad ornatum corporis , \textbf{ ut circa illa sex quae tetigimus , } diligenter instruantur et moueantur . & quanto al conponimiento del cuerpo \textbf{ assi que en aquellas seys cosas | que dixiemos sean las mugers enssennadas } ᷤ \\\hline
2.2.3 & quia uxori \textbf{ ( ut in praehabitis tangebatur ) } non quis debet & por que ala muger \textbf{ assi conmo dicho es de } suso non deue ninguno ensennorear sinplemente \\\hline
2.2.8 & addiscere musicam . \textbf{ Sed de his forte infra tangetur . } Quinta scientia liberalis & de aprender la musica \textbf{ mas destas razones | por auentura fablaremos adelante ¶ } La quinta sçiençia libal es dicha arismetica \\\hline
2.2.9 & licet enumerare possemus omnia illa octo \textbf{ quae in primo libro de prudentia tetigimus , } sufficiat tamen ad praesens quatuor de illis enumerare . & entre aquellas ocho cosas \textbf{ que dixiemos enel primero libro dela sabiduria . } Enpero cunple nos de contar las quatro cosas de aquellas ocho . \\\hline
2.2.9 & ipsum esse cautum , \textbf{ quia ut in primo libro tetigimus , } sicut in cognoscendo et speculando & que sea sabio \textbf{ en departir el manl del bien . | Ca assi commo dixiemos en el primer libro } assi commo en conosçiendo \\\hline
2.2.19 & si assuescant conuersationibus hominum , domesticantur , \textbf{ et permittunt se tangi et palpari : } si vero a conuersationibus hominum sint remotae , & fazen se mansas \textbf{ en tal manera | que se dexan tanner e palpar } mas si fueren alongadas de vsar con los omes \\\hline
2.3.3 & sumitur ex parte ipsius populi : \textbf{ et hanc tangit Philosophus 6 Politicorum , } ubi ait , & ¶La segunda razon para prouar esto mismo se toma de parte del pueblo \textbf{ e esta razon tanne el philosofo | enł vi̊ libro delas politicas } do dize que alos Reyes \\\hline
2.3.12 & et Principes \textbf{ inter vias tactas solae } duae viae videntur esse utiles : & Mas alos Reyes e alos prinçipes \textbf{ entre las otras maneras | dichͣs } paresçen dos maneras tan solamente conueinbles para auer dineros . \\\hline
3.1.5 & Intentio enim legislatoris \textbf{ ( ut supra tangebatur ) } non solum esse debet , & que faze la ley \textbf{ assi commo dicho es de suso } non solamente deue ser por que los çibdadanos ayan aquellas cosas \\\hline
3.1.7 & Determinando autem Socrates et Plato de moralibus , \textbf{ quinque tetigisse videntur } circa ciuitatem & Mas determinado socrates e platon delas cosas morales paresçe \textbf{ que pusieron e dixieron cerca çinco cosas ser la çibdat } e el gouernamiento çiuil . \\\hline
3.1.8 & indiget operibus , \textbf{ ut ambulatione , tactu , visione , } et auditus ideo oportet & ca assi commo vn cuerpo ha mester departidas obras \textbf{ assi commo de andar e de tanner e de oyr e deuer . } por ende conuiene de dar . \\\hline
3.1.9 & potius esset causa litigii quam pacis . \textbf{ Nam ( ut supra in secundo libro tetigimus ) } haec est uita perfectorum : & que de paz \textbf{ ca assi commo dixiemos de suso } en el segundo libro esta es uida de o ensacabados non auer prỏo \\\hline
3.1.10 & enumerare possumus quinque mala , \textbf{ quae Philosophus tangit ibidem sequentia ex tali communitate . } Primum est , iniuria consanguineorum . & e quanto parte nesçe a lo presente podemos contar c̃co males \textbf{ que se sigune de tal comuidat | los quales el pho pone en esse mismo logar . } El primero esiuria \\\hline
3.1.11 & Volebat enim \textbf{ ut superius tangebatur , } quod si nihil esset proprium in ciuitate nec possessiones nec fructus ne filii nec uxores , & assi commo ordenos ocͣtesca quarie socrates \textbf{ assi commo diches de suso } que ninguna cosa non fuesse prop̃a en la çibdat \\\hline
3.1.13 & Hanc autem tertiam rationem improbantem ordinationem Socraticam \textbf{ tangit Philosophus 2 Poli’ } cum ait . & Et esta razon que repraeua e denuesta el ordenamiento de socrates \textbf{ pone el pho | en el segundo libro delas politicas } quando \\\hline
3.1.14 & Dicebat enim Socrates , \textbf{ ut supra tangebam , } ciuitatem quinque in se debere habere , & Ca dizia socrates \textbf{ assi commo es dicho desuso } que toda çibdat deue auer \\\hline
3.1.19 & Sexto et ultimo statuit \textbf{ quasdam leges tangentes } diuersa genera personarum . & Lo sesto \textbf{ e lo postrimo establesçio algunas leyes } que tannian alguons linages de personas dezimos \\\hline
3.1.19 & Sexto statuit quasdam leges , \textbf{ tangentes diuersa personarum genera . } Statuit autem quatuor , & Lo sesto establesçio el dicho philosofo algunas leyes \textbf{ que tannian a algunos linages de personas } e destas tales leyes establesçio quatro . \\\hline
3.1.19 & Statuit autem quatuor , \textbf{ quarum prima tangebat sapientes , } secunda bellatores , & e destas tales leyes establesçio quatro . \textbf{ Lapmera tannia alos sabios¶ } La segunda alos lidiaderes . \\\hline
3.1.20 & quem statuit , \textbf{ tangentes diuersa genera personatum . } Primo enim dictus Phil’ deferre fecit statuendo impossibilia . & que el establesçio \textbf{ tanniendo departidos linages de perssonas . } Ca lo primero el dicho philosofo fallesçio \\\hline
3.1.20 & quantum ad leges quas statuit , \textbf{ tangentes diuersa personarum genera , } et specialiter quantum ad legem quam statuit erga sapientes . & que establesçio \textbf{ que tannian departidos estados de omes | espeçialmente fallesçio } quanto ala ley \\\hline
3.1.20 & infra diffusius tractabuntur : \textbf{ ad praesens tamen sufficiat hoc tetigisse de opinionibus Philosophorum . } Et in hoc terminetur prima pars huius tertii libri , & e las otras cosas \textbf{ que en esta materia son de dezer adelante lo tractaremos mas conplidamente . } Enpero quanto alo presente c̃ple de auer tranniendo esto \\\hline
3.2.1 & in tali regimine . \textbf{ Videtur autem Philos’ 3 Polit’ tangere , } quatuor quae consideranda sunt & en el gouernamiento del regno e dela çibdat \textbf{ Mas el philosofo en el terçero libro delas politicas } tanne quatro cosas \\\hline
3.2.4 & Philosophus 3 Politicorum videtur \textbf{ tangere tres rationes , } per quas probari videtur , & e han cunplimiento delas cosas \textbf{ lpho en el terçero libro delas politicas tanne tres razons } por las quales paresçe que se puede prouar \\\hline
3.2.4 & quod melius sit dominari multitudinem : \textbf{ postea in eodem 3 tangit quaedam , } per quae obiectiones huiusmodi solui possunt . & senssennore en que vno . \textbf{ Despues en esse mismo terçero tanne algunas cosas } por las quales se pueden solcar aquellas razones \\\hline
3.2.5 & ut in pluribus continent veritatem . \textbf{ Quod vero superius tangebatur , } videlicet quod ire per haereditatem , dignitatem regiam , & e sean uerdaderas . \textbf{ Mas aqual lo que dessuso fue dich } conuiene saber \\\hline
3.2.7 & ab intentione communis boni . \textbf{ Hanc autem rationem tangit Philosophus } circa principium 4 Politicorum ubi ait , & quanto por ella mas se arriedra el tirano dela entençion del bien comun . \textbf{ Et esta razon tanne el philosofo çerca el comienço del quarto libro delas politicas . } do dize que assi conmo el regno es muy buena et muy digna poliçia . \\\hline
3.2.7 & et maxime contra naturam , eo quod sit maxime subditorum afflictiua . \textbf{ Hanc autem rationem tangit Philosophus } in eodem 4 Politicorum ubi ait , & por que es mucho atormentadora de los subdictos . \textbf{ Et esta razon tanne avn el philosofo } en el quarto libro delas politicas \\\hline
3.2.7 & multa mala efficere . \textbf{ Hanc autem rationem tangit Philosophus quinto Politicorum ubi ait , } tyrannidem esse oligarchiam & que es ayuntado en vno puede fazer muchs males \textbf{ e esta razon tanne el philosofo | en el quinto libro delas politicas } do dize que la tirnia es la postrimera obligarçia \\\hline
3.2.7 & tyrannidem esse pessimum principatum \textbf{ propter rationes tactas . } Quod autem reges summo opere cauere debeant , & que la tirania es muy mal prinçipado \textbf{ por las razones sobredichͣs . } Mas en commo los Reyes en toda manera de una esquiuar de non enssennorear con sennorio de tirania \\\hline
3.2.10 & sed simulat se talem esse . \textbf{ Multas cautelas tangit Philosophus 5 Polit’ } ex quibus quantum ad praesens spectat , & e por auentura de paresçer tal . \textbf{ nchas cautelas tanne el philosofo en el quinto libro delas politicas delas quales quanto par tenesçe alo presente podemos tomar diez . } por las qualose esfuerça el tiranno de se mantener en su sennorio . \\\hline
3.2.16 & de quibus non sunt consilia adhibenda . \textbf{ Possumus autem tangere sex , } quantum ad praesens spectat , & de que non deuemos tomar conseios . \textbf{ Mas quanto alo presente nos podemos poner seys cosas } que non caen so consseio . \\\hline
3.2.20 & Quod quadruplici via inuestigare possumus , \textbf{ quarum tres tanguntur 1 Rhet’ } quarta vero tangitur 1 Polit’ . & por quatro razones delas quales las tres tanne el philosofo \textbf{ en el primero libro de la rectorica } e la quat catanne en el sexto libro delas politicas \\\hline
3.2.20 & quarum tres tanguntur 1 Rhet’ \textbf{ quarta vero tangitur 1 Polit’ . } Prima via sic patet . & en el primero libro de la rectorica \textbf{ e la quat catanne en el sexto libro delas politicas | cos son los fazedores delas leyes } que los La primera razon paresçe \\\hline
3.2.20 & et quam paucissima arbitrio iudicum committere . \textbf{ Has autem tres rationes tangit Philosophus 1 Rhetoricorum dicens } quod maxime quidem contingit & e en aluedrio de los iuezes . \textbf{ Et estas tres cosas tanne elpho | en el primero libro de la rectorica } diziendo \\\hline
3.2.23 & Quantum ad praesens spectat decem numerare possumus , \textbf{ quae videtur tangere Philos’ 1 Rhet’ } ad quae decet & uanto pertenesçe alo presente podemos contar diez cosas de la rectorica . \textbf{ Las quales dies cosas con que tanne el philosofo | en el primero libro } uiene que tenga el iuez sienpre mientes \\\hline
3.2.24 & quinque distinctiones facere , \textbf{ quarum duae tanguntur 1 Rhet’ tertia } ponitur 5 Ethic’ & commo del derecho fazer çinco departimientos \textbf{ de los quales los dos pone el pho enel primero libro de la rectorica . } Et el terçero pone en el quinto libro delas ethicas . \\\hline
3.2.25 & Omnes distinctiones de iure factas a Philosopho , \textbf{ quas in praecedenti capitulo tetigimus , } bimembres erant ; & por el philosofo del derecho \textbf{ las quales pusiemos en el capitulo lante dicho } eran de dos mienbros \\\hline
3.2.31 & Ordinauerat enim Hippodamus \textbf{ ( ut supra tetigimus ) } quod inueniens aliquam legem ciuitati proficuam , & por que ypodomio ordenara \textbf{ assi commo dixiemos de suso } que aquel que fallasse algunan ley a prouechosa ala çibdat \\\hline
3.3.5 & et cetera alia \textbf{ quae tetigimus in capitulo praecedenti . } Constat autem & e las otras cosas \textbf{ que contamos en el capitulo sobredicho . } Et çierto es que el pueblo aldeano ha mas prinçipalmente estas cosas sobredichas \\\hline
3.3.9 & Nam habenter carnes molles \textbf{ ( ut supra tangebatur ) } sunt aptiores ad intelligendum , & Ca assi commo dixiemos de suso \textbf{ los que han las carnes muelles } son mas apareiados para entender \\\hline
3.3.14 & Nam cum ipsa virtus unita , \textbf{ ( ut etiam supra tangebatur ) } fortior sit se ipsa dispersa : & Ca commo la uertud ayuntada \textbf{ assi commo dicho es } dessuso sea mas fuerte \\\hline
3.3.19 & in funda machinae imponendus . \textbf{ Tangebatur autem supra tres modi impugnandi munitiones obsessas . } Quorum unus erat per cuniculos . & e qual o quant pesada piedra se deue poner en la fonda del engeñio . \textbf{ t tres maneras de conbatir las fortalezas cercadas } fueron puestas dessuso de las quales la vna era \\\hline

\end{tabular}
