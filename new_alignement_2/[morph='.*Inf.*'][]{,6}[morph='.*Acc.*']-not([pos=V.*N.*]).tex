\begin{tabular}{|p{1cm}|p{6.5cm}|p{6.5cm}|}

\hline
1.1.2 & secundo scire suam familiam gubernare , \textbf{ tertio scire regere regnum , et ciuitatem . In primo autem libro in quo agetur de regimine sui , } sunt quatuor declaranda . & e sus çibdades ¶ \textbf{ pues que asy es en el primo libro | en el qual tractaremos del gouerna mjeto del omne . } En sy mesmo son quatro cosas de declarar e de demostrͣ \\\hline
1.1.2 & Videntur autem haec quatuor \textbf{ habere aliquam analogiam adinuecem . } Nam ex aliis , et aliis motibus , & pues que asy es paresçe \textbf{ que estas quatro cosas dichas han alguna conparaçion entre sy } por que departidas costunbres han de nasçer departidos pasiones \\\hline
1.1.3 & et maxime regiam maiestatem \textbf{ implorare diuinam gratiam . } Nam quanto maiestas regia in loco altiori consistit , & e mayormente prinçipe o Rey \textbf{ que demande mucho afincadomente la gera de dios } Ca quanto la magestad rreal esta en logar \\\hline
1.1.3 & ut possit virtutum opera exercere , \textbf{ et sibi subditos valere inducere ad virtutem . } Quot sunt modi viuendi , & por que pueda usar de obras de uirtudes \textbf{ e por que pueda enduzir } e traher los sus subienctos a uirtudes . \\\hline
1.1.4 & videlicet , voluptuosam , politicam , et contemplatiuam . \textbf{ Videbant enim hominem esse medium inter superiora , et inferiora : } est autem homo naturaliter medius & Ca veyen los philosofos \textbf{ que el omne es medianero entre las cosas | que son desuso } que son çelestiałs e las cosas \\\hline
1.1.4 & nam in vita voluptuosa \textbf{ negauerunt esse felicitatem , } quod et Theologi negant : & Ca en la vida delectos a \textbf{ negaron algunos philosofos la feliçidat | e la bien andança } diziendo que non es en elła . \\\hline
1.1.4 & et per omnem modum potuerunt \textbf{ attingere veritatem . } Nam licet vere dixerunt & e segunt manera acabada . \textbf{ Ca mager que dissiese nudat } que en la vida seliçonsa non es de poner bien andança \\\hline
1.1.4 & Quod autem vitam contemplatiuam dixerint \textbf{ esse potiorem , } quam vitam politicam et actiuam , & Mas en lo que ellos dixieron \textbf{ que la vida contenplatian es mejor que la vida politica e actiua } que esta en las obras en esto non descordaron de los cheologos njn dela uerdat . \\\hline
1.1.4 & tanto magis decet \textbf{ habere reges et principes , } quanto apud tribunal summi Iudicis & tantomas conviene dela auer los rreys \textbf{ e los prinçipes } por quanto han de dar mayor cuenta \\\hline
1.1.5 & vel suam felicitatem , \textbf{ habere praecognitionem ipsius finis . } Possumus autem dicere & e la su bien andança de auer \textbf{ ante algun conosçimjento dela su fin e dela su bien andança . } Mas podemosdezer quanto pertenesçe alo presente \\\hline
1.1.5 & eos \textbf{ consequi finem vel felicitatem . } Cum ergo ista tria contingunt , & delectosamente non les conuiene \textbf{ que por aquellas obras alcançen buena fin | nin bue an uentura ¶ } Et quando estas tres cosas todas uienen en vno \\\hline
1.1.5 & quod maxime decet regiam maiestatem \textbf{ cognoscere suam felicitatem , } ut opera communia , & que muy mas conuiene al Reio al prinçipe conosçer la su fin \textbf{ e la su bien andança | que a otro ninguno . } por que pueda fazer buenas obras e comunes \\\hline
1.1.5 & eo quod sit sagittae director : sic magis expedit regiam maiestatem felicitatem , \textbf{ et finem cognoscere quam populum , } eo quod fit populi directiua . & que mas conuiene al Rei o al prinçipe de conosçer la su bien andança e la su \textbf{ finque non al pueblo . } Por que el es ginador del pueblo \\\hline
1.1.6 & Felicitas enim dicit perfectum , \textbf{ et per se sufficiens bonum . Nam tunc dicimus aliquem esse felicem , } quando assecutus est id , & La primera es que la bien andança es bien acabado \textbf{ e tal bien que por si mesmo faze el omne acabado . } Ca estonçe dezimos \\\hline
1.1.6 & si felicitas ponitur \textbf{ esse perfectum bonum , oportet quod sit bonum } secundum intellectum , et rationem : & Pues si la bien auentraança es bien acabado e conplido . \textbf{ Conuiene que sea bien segunt el en tedimiento } e segunt Razon \\\hline
1.1.6 & Est ergo detestabile cuilibet Homini \textbf{ ponere suam felicitatem in voluptatibus . } Sed maxime hoc est detestabile Regiae maiestati : & ̉ qual quier omne \textbf{ que toda su bien andança pone en delecta connes dela carne } mas mucho mas es de denostar el Rey \\\hline
1.1.6 & Dictum est enim \textbf{ quod decet Principem esse supra Hominem , } et totaliter diuinum . & Ca dicho n auemos ya desuso \textbf{ que conuiene al Rei | e al prinçipe sier } mas que omne e ser del todo diuinal . \\\hline
1.1.6 & quod sicut non differt \textbf{ esse Puerum aetate , et moribus : } sic non refert & pues que asi es muy acuçiosamente deuemos notar \textbf{ que asy commo non ay departimento entre moço en hedat e moço en constunbres . } asy non ay deꝑtimiento \\\hline
1.1.6 & sic non refert \textbf{ esse Senem moribus et aetate , } propter quod sicut si sit Senex tempore , & asy non ay deꝑtimiento \textbf{ entre vieio en costunbres e uieio en hedat . } por la qual cosa asy commo sy \\\hline
1.1.6 & quare si constat \textbf{ eos habere mores seniles , } et vigere Prudentia , & por la qual cosa si fuer çierto \textbf{ qualos mançebos han costunbres de me nos } e han sabiduria e entendimiento \\\hline
1.1.7 & Cuilibet ergo Homini detestabile est \textbf{ ponere suam felicitatem in diuitiis , } sed maxime detestabile est regiae maiestati . & ¶pres que assi es mucho es de denostar todo en que pone su feliçidat \textbf{ e su bien andança en las riquezas corporales . } Mas mayor mente es de denostar la Real magestad \\\hline
1.1.7 & quid importatur nomine finis , \textbf{ non potest eum latere quemlibet , } omni via qua potest , & que quiere dezir e quanto lieua este nonbre fin \textbf{ e bien andança non se le puede esconder } por ninguna manera \\\hline
1.1.7 & Quare si detestabile est , \textbf{ Regem admittere maxima bona , } esse Tyrannum , & mas faze les mal . \textbf{ Por la qual cosa si es muy contra razon que el Rey dexe muy grandes bienes . Et si es contra razon otrosi } que sea robador del pueblo \\\hline
1.1.7 & Regem admittere maxima bona , \textbf{ esse Tyrannum , } et depraedatorem detestabile & Por la qual cosa si es muy contra razon que el Rey dexe muy grandes bienes . Et si es contra razon otrosi \textbf{ que sea robador del pueblo } e sea tyrano ¶ Bien asi es cosa contra razon \\\hline
1.1.8 & qui vult honorem \textbf{ esse exhibitionem reuerentiae } in testimonium virtutis . & por el philosofo en el primero libro delas ethicas \textbf{ do dize que honrra es reuerençia } que fazen unos omes a otros en testimoino de uirtud \\\hline
1.1.8 & quam in eo qui per huiusmodi reuerentiam honoratur . \textbf{ Apparet autem hoc esse sensibiliter verum : } nam si aliquis inclinat se reuerenter ad alium , & que non en aquel que la resçibe¶ \textbf{ Et esso mesmo paresçe manifiestamente al seso . } Por que sy alguno inclinando se faze reuerençia a otro o le honrra . \\\hline
1.1.8 & Indecens est ergo cuilibet homini \textbf{ ponere suam felicitatem in honoribus , } ut credat se esse felicem , & es que ningun omne . \textbf{ ponga su bien andança en las honrras } assi que crea que es bien andante \\\hline
1.1.8 & ponere suam felicitatem in honoribus , \textbf{ ut credat se esse felicem , } si ab hominibus honoratur . & ponga su bien andança en las honrras \textbf{ assi que crea que es bien andante } si los omes le honrran \\\hline
1.1.8 & curantes de honore tantum , \textbf{ dicit esse fictos , et superficiales . } Si ergo maxime decet & que los que fazen fuerça tan solamente de ser honrrados \textbf{ que estos son infintos e superfiçiales . } Et pues que assi es si mucho conuiene al Rey de ser bueno uerdaderamente \\\hline
1.1.8 & Quod non decet regiam maiestatem , \textbf{ suam ponere felicitatem in gloria , } vel in & ø \\\hline
1.1.9 & et pro eodem , \textbf{ posset forte alicui videri felicitatem ponendam } esse in fama et gloria , & alguon que la feliçidat \textbf{ e la bien andança es de poner } e en fama e en eglesia \\\hline
1.1.9 & inconueniens enim est , \textbf{ quod Rex se credat esse felicem , } si sit famosus apud Homines , & Et non es cosa conuenible \textbf{ nin cosa con razon | que el Rey crea } que es bien auentra ado si es famoso entre los omes \\\hline
1.1.9 & vel si sit in populis gloriosus . \textbf{ Non igitur debet Rex se credere esse beatum , } si sit in gloria apud homines : & o si es głioso en los pueblos ¶ \textbf{ Et pues que assi es el rey | non deue creer } que es bien auenturado \\\hline
1.1.9 & Non est intelligendus textus Philosophi , \textbf{ quod Reges principaliter pro suo merito quaerere debeant gloriam , } et famam Hominum , & assi entender \textbf{ que los Reys prinçipalmente por su meresçimiento deuen demandar } e quere reglesia e fama de los omes . \\\hline
1.1.10 & Non ergo Rex debet \textbf{ se credere esse felicem , } si per violentiam , & nin el prinçipe \textbf{ que es bien auentraado | si enseñorea sobre el pueblo } por fuerca e por poderio çiuil . \\\hline
1.1.10 & Impossibile est autem \textbf{ in aliquo esse maximum bonum , } nisi ille bene viuat , & Mas esto non puede ser \textbf{ que en alguno sea muy grand bien } si el non biuiere bien \\\hline
1.1.10 & sicut impossibile est in aliquo \textbf{ esse intensam albedinem , } nisi ille fit intense albus . & assi commo non puede seer \textbf{ que en alguna cosa sea grand blancura } si aquella cosa non fuere muy blanca . \\\hline
1.1.10 & ut non videretur in eis \textbf{ esse aliquid molle , } nec clementia aliqua . & Et fueron de tan grant crueldat \textbf{ que non paresçia en ellos ningunan cosablanda } nin de piedat \\\hline
1.1.10 & quia si Princeps se crederet \textbf{ esse felicem , } si abundet in ciuili potentia , & Por que si el prinçipe o el Rey crea \textbf{ que es bien auen traado } por que abonda en poderio çiuil non ordenara los çibdadanos \\\hline
1.1.10 & inconueniens est etiam Principem \textbf{ ponere suam felicitatem } in ciuili potentia , & que non es cosa conuenible \textbf{ que el prinçipe ponga su bien andança en poderio çiuil . } Et por que cuyde que es bien \\\hline
1.1.10 & in ciuili potentia , \textbf{ et quod credat se esse felicem , } si possit sibi subiicere nationes multas . & que el prinçipe ponga su bien andança en poderio çiuil . \textbf{ Et por que cuyde que es bien } auentraado \\\hline
1.1.10 & et quod credat se esse felicem , \textbf{ si possit sibi subiicere nationes multas . } Quod non deceat Regiam maiestatem & Et por que cuyde que es bien \textbf{ auentraado | quando pudiere subiugar } assi las naçiones . \\\hline
1.1.11 & quam talia bona praegustentur , \textbf{ creduntur esse maiora , } quam sint : & Ca primeramente que omne goste e siente \textbf{ que le son estos bienes corporales paresçen le mayores de quanto son } Mas despues que los ha auido paresçen meno res de quanto el cuydaua . \\\hline
1.1.11 & nec in pulchritudine corporis , sed animae . \textbf{ Non igitur quis credat se esse felicem , } si habeat aequatos humores , & nin en fortaleza del cuerpo mas del alma . \textbf{ Et pues que assi es non crea ninguno | que es bien auentra ado } si ouiere los humores egualados \\\hline
1.1.11 & Non igitur quis credat se esse felicem , \textbf{ si habeat aequatos humores , } et sit sanus corpore : & que es bien auentra ado \textbf{ si ouiere los humores egualados } e fuere sano del cuerpo \\\hline
1.1.11 & tunc ( ut exigit suus status ) \textbf{ credat se esse felicem . } Dicimus autem & Estonçe crea el \textbf{ que es bien auentraado segunt su estado } Et dezimos segunt \\\hline
1.1.12 & sed etiam mali participant . \textbf{ Nec etiam voluit esse ponendam eam in habitibus , } quia habens habitum , & en las disposiconnes \textbf{ nin en las scinas | que son en el alma } que el que ha las scians \\\hline
1.1.12 & nam regens multitudinem debet \textbf{ intendere commune bonum . } In eo ergo debet & Ca el que gouienna a muchos deue tener \textbf{ mientesal bien comun de todos . } Et por ende deue poner la su feliçidat \\\hline
1.1.12 & Si ergo Rex debet in Deo \textbf{ ponere suam felicitatem , } oportet ipsum huiusmodi felicitatem ponere & ¶ Et pues que el Rey deue poner la su feliçidat \textbf{ e la su bien andança en dios . } Conuiene le dela poner en la obra de aquella uirtud \\\hline
1.1.12 & unitiuam \textbf{ quandam dicimus esse virtutem . } In amore ergo diuino est ponenda felicitas . & commo aquel commo humanal o angelical o diuinal ha muy grant uirtud de ayuntar al que ama con lo que ama . \textbf{ Pues que assi es en el amor de dios } es de poner la feliçidat en la bien andança \\\hline
1.1.13 & Magnum autem esse praemium Regis , \textbf{ et magnam eius esse felicitatem , } si per prudentiam , & quant grant es el gualardon de los reyes \textbf{ e quant grande es la su feliçidat | e las un bien andança } que han de auer \\\hline
1.1.13 & cum semper amor sit ad similes , et conformes , \textbf{ oportet esse similem , } et conformem Deo , & Et commo el amor sienpre sean los semeiables e acordables con el . \textbf{ Conuiene que aquel que es para de ser } gualardonado de dios \\\hline
1.2.2 & Dictum est enim , \textbf{ virtutem esse aliquid secundum rationem : } oportet ergo esse rationalem potentiam , & Ca dicho es \textbf{ que la uirtud es alguna cosa | segunt razon } e por ende conuiene \\\hline
1.2.2 & virtutem esse aliquid secundum rationem : \textbf{ oportet ergo esse rationalem potentiam , } in qua potest esse virtus . & segunt razon \textbf{ e por ende conuiene | que sea poderio razonable } aquel en que esta la uirtud ¶ \\\hline
1.2.2 & vel ipsa ratio essentialiter , \textbf{ dicitur tamen participare rationem , } quia est aptus natus rationi obedire . & nin sea la razon por eennçia \textbf{ Enpero partiçipa con la razon } por que es apareiado e inclinado de obedesçer al entendimiento \\\hline
1.2.2 & videlicet , sensitiuus , \textbf{ qui sequitur formam apprehensam per sensum : } et intellectiuus , & assi en nos ay dos apetitos vno \textbf{ senssitiuo que sigue la forma tomada e conosçida por el seso . } Et el otro intellectiuo \\\hline
1.2.3 & nisi circa ea quae sunt in potestate nostra , \textbf{ in quibus decet nos ponere medietatem , } vel aequalitatem , siue rectitudinem : & que son en nuestro poder . \textbf{ En las quales cosas nos conuiene } de poner meatado ygualdat o derechura . \\\hline
1.2.3 & non sic possunt \textbf{ habere rationem ardui sicut utilia , et honesta , } licet tam ex bonis utilibus & Et daqui paresçe que por que los bienes delectables non pueden auer \textbf{ assi manera de guaueza | commo han los bienes prouechosos } e los honestos maguera \\\hline
1.2.6 & Per virtutes ergo morales \textbf{ praestituimus nobis debitos fines : } sed per prudentiam & por la pradençia ¶ Et pues que assi es por las uirtudes morales somos ordenados \textbf{ a nuestros fines buenos e conuenibles . } Mas por la pradençia somos reglados \\\hline
1.2.6 & secundum inuenta et iudicata , \textbf{ et hanc dicimus esse prudentiam . } Prudentia ergo respectu virtutis inuentiuae et iudicatiuae & por la qual mandamos que se fagan las obras todas segunt las cosas falladas e iudgadas \textbf{ e esta dize el philosofo | que es pradençia . } Et pues que assi es la pradençia \\\hline
1.2.6 & Cum ergo in moralibus actus \textbf{ et opera dicantur esse potiora , } Prudentia , quae immediatius se habet ad ea , & Et pues que assi es commo en las uirtudes morales las obras \textbf{ e los fechos sean dichos mayores e meiors . } la pradençia que mas derechanmente cata alas obras es mayor e meior \\\hline
1.2.6 & sic prudentiae est praecipere . \textbf{ Tertio , Prudentia comparari potest ad materiam , } circa quam versatur . & assi la pradençia es uirtud para mandar . \textbf{ Lo terçero la pradençia se pue de conparar } ala materia en que obra . \\\hline
1.2.6 & quae sunt in potestate nostra . \textbf{ Quinto comparari potest prudentia ad artem , } a qua etiam distingui habet . & o non las fazer . \textbf{ Et aquella sabiduria es dicha pradençia | por la qual cosa seg̃t que la pradençia ha departimiento dela sçina } pue dese \\\hline
1.2.7 & Hoc etiam modo iuuenes naturaliter decet \textbf{ antiquioribus esse subiectos , } quia inexperti agibilium & en esta misma gusa las moços \textbf{ e los mançebos | conuiene que naturalmente sean subiectos de los mas antigos } por que non son espiertos \\\hline
1.2.7 & quod polleat prudentia , et intellectu . \textbf{ Quot , et quae oporteat habere Regem , } si & ø \\\hline
1.2.8 & Haec autem octo , \textbf{ quae dicuntur esse partes prudentiae , } sic accipi possunt . & Mas estas ocho cosas \textbf{ que son dichas part | s̃ dela sabiduria } assi se pueden tomar . \\\hline
1.2.8 & quae possunt \textbf{ esse utilia toti regno , } cum hoc quod Regem expedit & Mas porque ningun omne non puede conplidamente penssar aquellas cosas \textbf{ que son aprouechables a todo el regno . } Enpero con esto que conuiene al Rey de ser sotil e agudo de si penssando los bienes \\\hline
1.2.8 & quod non decet \textbf{ ipsum fugere commouentem . } Non enim decet Regem & do dize que non conuiene al magnanimo \textbf{ menospreçiara | aquel que bien le conseia } por la qual cosa non le conuiene al Rey de seguir en todas cosas su cabeça \\\hline
1.2.8 & et eligendo bona simpliciter , \textbf{ ad quae debet dirigere gentem sibi commissam . } Quomodo Reges , & ¶Et otrosi para escoger las buenas \textbf{ que son dessi buenas } alas quales deue el Rey guiar \\\hline
1.2.9 & et consuetudines debite regnum regat , \textbf{ eliciendo ex eis debitas conclusiones agibilium . } Non enim sufficit esse intelligentem , & puede bien gouernar su regno \textbf{ tomando delas razones conuenibles conclusiones | para todas las cosas } que ha de fazer . \\\hline
1.2.9 & Verum quia malitia est corruptiua principii . \textbf{ Sicut enim quis habens corruptum gustum , } male iudicat de saporibus , & corronpadera dela razon e del comienco para obrar . \textbf{ Ca assi commo aquel que ha el gosto } corronpido mal iudga delos sabores . \\\hline
1.2.9 & sic habens infectam , \textbf{ et deprauatam voluntatem , excoecatur in intellectu , } ut male iudicet de agibilibus : & assi aquel que ha corrupta e desordenada la uoluntad \textbf{ por maliçia es ciego en el entendimiento } e en la razon por que iudge mal en lo que ha de fazer \\\hline
1.2.9 & quae superius diximus , \textbf{ oportet ipsos esse bonos , } et non habere voluntatem deprauatam : & que dixiemos de suso \textbf{ conuieneles | que sean buenos } e que non ayan uoluntad mala nin desordenada \\\hline
1.2.9 & oportet ipsos esse bonos , \textbf{ et non habere voluntatem deprauatam : } ne propter malitiam appetitus , imprudenter agant , & que sean buenos \textbf{ e que non ayan uoluntad mala nin desordenada } por que por la maliçia dela uoluntad fagan las cosas sin razon \\\hline
1.2.10 & Perfici ergo in ordine ad leges , \textbf{ est perfici in ordine ad Principem , } cuius est legem ferre , & alas leyes es seer acabado en orden al prinçipe \textbf{ al qual parte nesçe } commo dicho es confirmar la ley \\\hline
1.2.11 & Dicebatur in praecedenti capitulo \textbf{ duas esse Iustitias , } unam generalem , et aliam specialem . & a assi commo dicho es en este capitulo \textbf{ sobredicho dos son las iustiçias vna general e otra espeçial . } Mas para que los regnos esten en su estado \\\hline
1.2.11 & et perfecta malitia . \textbf{ In nullo ergo obseruare leges , } et ciues non participare & es maliçia entera e acabada¶ pues que assi es quando los çibdadanos \textbf{ ennigua cosa non guardan las leyes } nin toma ninguna parte dela iustiçia legal . \\\hline
1.2.12 & ostendit \textbf{ eos esse perfecte bonos . } Sic enim videmus in aliis rebus & muestra \textbf{ que ellos son acabados e buenos . } Ca assi lo veemos en todas las otras cosas \\\hline
1.2.12 & quae perficiunt hominem in se , \textbf{ se videntur habere ad Iustitiam , } quae perficit hominem in ordine ad alterum , & Et por ende todas las otras uirtudes morałs̃ que acaban el omne en ssi deuen auer la iustiçia \textbf{ assi commo reina e sennora } porque acaba el omne en orden alos otros \\\hline
1.2.13 & et non recte , \textbf{ oportet dare virtutem aliquam } circa timores , et audacias . & e en las osadias \textbf{ por la qual sea el omne reglado en ellos . | por que contesce que algunos remen algunas cosas } que han de temer e alas uegadas temen alguas cosas \\\hline
1.2.13 & sicut pericula belli . \textbf{ Cum ergo difficilius sit durare , et sustinere pericula illa } quae per fugam vitare possumus , & assi escusar commo los periglos delas batallas . \textbf{ Et pues que assi es commo sea mas | guaue cosa de endurar } e de sufrir aquellos periglos que podemos escusar \\\hline
1.2.14 & quum sunt in ignotis partibus , \textbf{ committere aliqua turpia , } quae inter ciues et notos nullatenus attentarent . & do non son conosçidos \textbf{ acometen alguas torpedades } las quales non quarrian acometer nin tentar entre los sus çibdadanos en ningunan manera \\\hline
1.2.14 & Hoc autem modo quidam Dux dicitur exercitum suum \textbf{ coegisse ad Fortitudinem . } Nam , cum nauigiis , & Et en esta manera vn caudiello dizen \textbf{ que costrino sus conpannas a fortaleza . | por que fuesen fuertes . } Ca commo el estudiese en sus naues \\\hline
1.2.15 & quam ratio dictet , \textbf{ fugere delectationes corporales sensibiles . } Qui igitur omnes delectationes insequitur , & Et pues que assi es el non sentirse es foyr delas delecta con \textbf{ nessenssibles e corporales | mas que la razon manda } Et por ende aquel que se da a todas las delecta \\\hline
1.2.15 & et tactus magis directe \textbf{ et immediate videntur ordinari ad conseruationem nostram : } ut delectabilia gustus & son mas derechamente \textbf{ e mas ayuntadamente ordenades alanr̃auida | e alanr̃a conseruaçion } assi commo las cosas \\\hline
1.2.15 & Oportet enim vere temperatum \textbf{ non exercere opera venerea , neque gestus . } Prout ergo abstinet & e la linpieza refreña \textbf{ e abaxan las delecta connes | e los uicios dela carne } Et por ende conuiene aquel que uerdaderamente es \\\hline
1.2.16 & ei facere bonum , \textbf{ et acquirere temperantiam , } quam sit acquirere fortitudinem . & ¶Lo otro por que mas ligeramente puede bien fazer e ganar tenprança que fortaleza . \textbf{ Mas que el } destenprado pequemas de voluntad que el temeroso puede se demostrar \\\hline
1.2.16 & et acquirere temperantiam , \textbf{ quam sit acquirere fortitudinem . } Quod autem magis voluntarie peccet intemperatus & Mas que el \textbf{ destenprado pequemas de voluntad que el temeroso puede se demostrar } por dos razons¶ \\\hline
1.2.16 & quam timidius dupliciter ostendi potest . \textbf{ Primo , quia insequi voluntates intemperatas , } est delectabile : & por dos razons¶ \textbf{ La primera es por que segnir } plazenterias desfenpradas es cosa delectable \\\hline
1.2.16 & quam non esse fortes . \textbf{ Si ergo Regem non esse virilem , } et non esse constantem & que por non ser fuertes . \textbf{ Et por ende si el Rey non fuere fuerte } e non fuer firme en el coraçon es de deno star por ello . \\\hline
1.2.16 & esse bestialem et seruilem : \textbf{ indecens est ipsum esse intemperatum . } Secundo intemperantia est vitium maxime puerile . & enssennorear alos otros de ser bestial e sieruo \textbf{ e non es cosa conuenible | que el sea destenp̃do ¶ } Lo segundo la \\\hline
1.2.16 & Ideo maxime videmus eos sequi delectabilia , \textbf{ et esse insecutores passionum . } Unde Philosophus 3 Ethicorum vim concupiscibilem , & mas sigue las cosas delecta bles \textbf{ e son seguidores delas pasiones | e de los apetitos que los otros } ¶ Onde el philosofo en el terçero libro delas ethicas \\\hline
1.2.16 & et non sequi rationem , sed passionem : \textbf{ indecens est ipsum esse intemperatum . } Tertio est hoc indecens Regi : & e en entendimiento mas passiones \textbf{ e delectaçiones | non es cosa conuenible } que el Rey sea es tenprado ¶ \\\hline
1.2.16 & se reuerendam et honore dignam , \textbf{ maxime indecens est eam esse intemperatam . } Exemplum autem huius habemus & e muy digna de honrra . \textbf{ Much̃o desconuenible cosa es | que el Rey sea destenprado } e desto auemos \\\hline
1.2.16 & Dux autem ille assuetus rebus bellicis , \textbf{ videns Regem suum esse totum muliebrem et bestialem , } statim ipsum habuit in contemptum : & veyendo \textbf{ que el su Rey era todo mugeril | e toda su } conuerssaçion era entre mugers e era bestial . \\\hline
1.2.17 & et ostendimus quomodo Reges et Principes illis virtutibus decet \textbf{ esse ornatos . } Reliquum est pertransire & e los prinçipes deuen ser conpuestos e honrrados \textbf{ dellas fincanos } de dezir delas otras och̃o uirtudes . \\\hline
1.2.17 & magnificentia vero dicitur \textbf{ respicere magnos sumptus ; } quod quomodo sit intelligendum , & e non son grandes nin pequenas . \textbf{ Mas la magnificençia es tal uirtud que cata alas grandes despenssas . } Et esto en qual manera se deue entender adelante lo mostrͣemos¶ \\\hline
1.2.17 & Spectat autem ad liberalem \textbf{ non usurpare alios redditus , } et custodire proprios . & Ca pertenesçe al franco \textbf{ que non tome | nin fuerce las rentas de los otros } e que guarde \\\hline
1.2.17 & non usurpare alios redditus , \textbf{ et custodire proprios . } Nam licet liberales & nin fuerce las rentas de los otros \textbf{ e que guarde | e tome las suyas . } Ca maguera que el franco non ha menos \\\hline
1.2.17 & expoliatores mortuorum , et aleatores , \textbf{ dicit esse turpia lucra : } et omnes tales appellat illiberales . & e los iugadores delas tablas \textbf{ e de los otros iuegos | Dize } que talon iuegos e ganançias commo estas son torpes e de sone stos . \\\hline
1.2.17 & sed magis est acquirere \textbf{ et generare ipsam . } Propter quod patet liberalitatem & lo suyo non es husar del auerante es mas ganar lo e allegar lo . \textbf{ por la qual cosa paresçe } que la franqueza es mas en espender \\\hline
1.2.17 & circa debitos sumptus , \textbf{ et circa debitas rationes ; } ex consequenti autem est & Ca prinçipalmente es en las espenssas conuenibles \textbf{ e en las dona connes conuenibles } e despues desto ha de ser \\\hline
1.2.18 & ad possessiones dantis . \textbf{ Ideo Philosophus ait Tyrannos non esse prodigos : } quia non videntur posse & en conparaçion delas possesiones e de las rentas del queda \textbf{ Et por ende dize el philosofo | que los Reyes non son gastadores } por razon que non pueden \\\hline
1.2.18 & propter quod omnino detestabile est \textbf{ Reges et Principes esse auaros : } tam enim fugienda est auaritia a principibus & por la qual razon muy de denostar son los Reyes \textbf{ e los prinçipes | si fueren auarientos . } Ca en tanto deue ser arredrada la auariçia de los prinçipes \\\hline
1.2.18 & Si ergo omnino decens est \textbf{ Regem esse virtuosum , } tanto detestabilius est & qual quier gastador liberal e franco ¶ \textbf{ pues que assi es si es conueinble al Rey } en toda manera de ser uirtuoso tanto \\\hline
1.2.18 & et quod omnino detestabile est \textbf{ eos esse auaros : } restat ostendere , & e que muchon son de denostar \textbf{ si fueren auarientos fincanos de demostrar } que conuiene alos Reyes \\\hline
1.2.19 & et minus non videantur \textbf{ diuersificare speciem , } et naturam rerum , & Mas commo en cada cosa \textbf{ mas e menos non fagan departimiento en la naturaleza } e en la semeiança delas cosas \\\hline
1.2.19 & quae est circa magnos sumptus , \textbf{ esse virtutem aliam a liberalitate , } quae est circa mediocres . & que es en las grandes \textbf{ espenssassea otra uirtud e apartada dela liberalidat } que es çerca delas medianas mesuradas espenssas . \\\hline
1.2.20 & Videtur enim ei , \textbf{ quod remouere a se pecuniam , } sit abscindere membra a proprio corpore . & sienpre las faze tardando . \textbf{ Ca paresçe leal paruifico } que tirar el auer de ssi \\\hline
1.2.20 & qualiter faciat magnum opus , \textbf{ ut qualiter faciat debitas largitiones , } vel quomodo faciat decentes nuptias : & en qual manera faga granada obra \textbf{ e en qual manera faga sus dones granados e conuenibles } o en qual manera faga sus bodas conuenibles \\\hline
1.2.20 & cum tristitia et dolore ; et cum nihil facit , \textbf{ credere se magna operari , } quia omnia haec valde derogant regiae maiestati , & Et quando non faze ningunan cosa cree el \textbf{ que faze grandescosas e grandes obras . } Et por que todas estas cosas ponen grand mengua en la Real magestad \\\hline
1.2.20 & distribuere bona regni , \textbf{ maxime decet ipsum esse magnificum . Nam quia est caput regni , } et gerit in hoc Dei vestigium , & e a el pertenesca de partir los bienes del regno mucho le conuiene a el de ser magnifico . \textbf{ Ca porque es cabeça del regno } e ha en esto semeiança de dios \\\hline
1.2.20 & maxime spectat ad eum magnifice \textbf{ se habere circa bona communia , } et circa ea quae respiciunt regnum totum . & mucho parte nesçe a el de se auer granadamente \textbf{ e honrradamente çerca los bien es comuns } e cerca todas aquellas cosas \\\hline
1.2.21 & esse liberalem , \textbf{ facere maximos decentes sumptus , } quos facit magnificus , & si faz conuenibles espenssas faz omne ser liberal fazer muy grandes \textbf{ e muy conuenibles espenssas } lo que faze el magnifico es ser mucho mas liberal ¶ \\\hline
1.2.21 & Sed , ut ibidem dicitur , \textbf{ tales oportet esse nobiles et gloriosos . } Quare quanto est nobilior aliis , & por que non puede cada vno fazer grandes espenssas \textbf{ Mas assi commo alli dize el philosofo tales son los nobles e los głiosos . } por la quel cosa en quanto el Rey es mas noble \\\hline
1.2.22 & 4 Ethicor’ velle , \textbf{ magnanimitatem esse circa honores , } circa diuitias , et principatus , & en el quarto libro delas ethicas dize \textbf{ que la magranimidat | es cerca las honrras } et cerca las riquezas \\\hline
1.2.22 & cum non reputamus \textbf{ talia esse simpliciter maxima bona . } Et si contingat nos infortunari circa ea , & por que non cuydaremos \textbf{ que tales bienes son los mayores bienes . } Et si contesçiere quenos \\\hline
1.2.23 & non parcat vitae , \textbf{ ut Philosophus ait 4 Ethic’ . Secundo competit magnanimo se habere bene circa retributiones . } Magnanimus enim parum appreciatur exteriora bona , & assi commo dize el philosofo \textbf{ en el quarto libro delas ethicas¶ | La segunda propiedat que parte nesçe al magnanimo es } auer se bien çerca las particones delos dones dando a cada vno \\\hline
1.2.23 & est agere opera virtutum , \textbf{ conuenit magnanimo esse plurimum retributiuum , } ut dicitur 4 Ethicor’ . & es fazer obras de uirtudes . \textbf{ Ca assi commo dize el philosofo en el quarto libro delas ethicas pertenesce mucho } almagnanimo ser mucho partidor e dador de gualardones ¶ \\\hline
1.2.23 & omnia negocia \textbf{ quantumcumque modica expedire per seipsos , } nec decet eos omnium esse operatiuos ; & por si mismos todos los negoçios \textbf{ mayormente los que son pequa nons } nin conuiene aellos de seer obradores de todas las cosas \\\hline
1.2.23 & quae sunt multa . \textbf{ Quarto decet esse apertos , } ut esse veridicos ; & que son muchos alos otros ¶ \textbf{ Lo quarto conuiene alos Reyes } de seer manifiestos e claros e seer uerdaderos \\\hline
1.2.24 & quod non congruit \textbf{ magnanimo fugere commouentem , } quod est actus prudentiae , & que non pertenesce almagranimo foyr \textbf{ de aquel qual bien conseia . } Ca esto es obra de pradençia \\\hline
1.2.24 & quae sint honore digna . \textbf{ Videtur enim honoris amatiua se habere ad magnanimitatem , } sicut formositas corporis & que sean dignas de honrra . \textbf{ por que paresçe que la uirtud es dicha amadora de honrra se ha ala magnanimidat | assi commo la fermosura cortoral se ha } ala apostura grande de todo el cuerpo . \\\hline
1.2.25 & Nam cum impossibile sit \textbf{ esse magnanimum non existentem bonum , } ut probat Philosophus 4 Ethic’ & Ca commo non pueda seer \textbf{ que alguno sea magnanimo | e que non sea bueno } assi \\\hline
1.2.25 & de qua loquitur Philosophus , \textbf{ non esse per omnem modum idem cum humilitate : } quia illa de qua Philosophus loquitur , & mostrariamos que la uirtud de que fabla el philosofo \textbf{ non es en toda manera vna cosa misma con la humildat } por que aquella de que fabla el philosofo \\\hline
1.2.27 & et deficere , \textbf{ oportet ibi dare virtutem aliquam , } per quam dirigamur ad bene agendum , & e tal sesçer conuiene de dar y . \textbf{ alguna uirtud por la qual seamos enderesçados } abien obrar \\\hline
1.2.27 & praeter ordinem rationis . \textbf{ Ratio enim dictat punitiones aliquas esse faciendas , } et quod est irascendum , & fuera de orden de razon e de entendimiento \textbf{ por que la razon demanda | que algunas penas sean dadas } e algunas venganças sean fechas \\\hline
1.2.28 & et verba nostra \textbf{ ad debitam conuersationem . } Secundo , verba , et opera nostra & e las nuestras obras a buena conuerssaçion e conuenible . \textbf{ lo segundo las nr̃as palauras } e las nr̃as obras siruennos ala uerdat \\\hline
1.2.28 & et uerba , \textbf{ ut ordinantur ad debitam conuersationem in uita . } Si enim homo est naturaliter animal sociale , & Ca ha de seer \textbf{ çerca las palauras en quanto son ordenadas a buena conuerssaçion en la uida del omne . } Ca si el omne es naturalmente animalia aconpanable \\\hline
1.2.28 & sic in conuersatione hominum . \textbf{ Aliqua enim familiaritas reputatur regi ad virtutem , } et dicitur ex hoc amicabilis esse : & el qual seria pequano para el sanno assi en essa misma manera en la conuerssacion de los omes \textbf{ alguna familiaridat es contada al Rey a uirtud } e es dicho por ende amigable \\\hline
1.2.29 & quod affecti ad propria bona , \textbf{ videntur nobis illa esse maiora , } quam sint . & Ca deuemos cuydar que nos \textbf{ por que somos inclinados alos nuestros bienes propreos paresçen nos mayores de quanto son . } Et esta razon tanne el philosofo en la auctoridat \\\hline
1.2.29 & declinandum esse in minus \textbf{ propter onerosas esse superabundantias . } Ostenso quid est veritas & declinaralo menos \textbf{ por la sobrepuiança de carga . } ¶ Visto que cosa es la uerdat \\\hline
1.2.29 & quam sint , \textbf{ videntur esse derisores , et contemptibiles . } Excedentes vero in plus , & e mas viles de quanto son . \textbf{ paresçe que son escarnidores e despreçiadores de ssi mismos . } mas aquellos que sobrepuian en lo mas \\\hline
1.2.29 & vel promittendo aliis maiora quam faciant . \textbf{ Immo tanto magis decet Reges et Principes cauere iactantiam , } quanto plures habent incitantes ipsos ad iactantiam , & nin prometiendo alos otros mayores cosas que faran . \textbf{ Mas por tanto conuiene alos Reyes | e alos prinçipes de escusar } e de foyr el alabança \\\hline
1.2.30 & quod videtur requies \textbf{ et ludus esse aliquid necessarium in vita . } Sicut ergo sensus corporales , & que paresce que la folgura \textbf{ e el trebeio es vna cosa necessaria en la uida de los omes . } Et pues que assi es \\\hline
1.2.30 & aliquam puerilitatem videtur \textbf{ habere annexam . } Tanto igitur decet Reges et Principes moderate & e honesto pareste \textbf{ que aya en el alguna moçedat ayuntada Et pues que assi es en tanto conuiene alos Reyes } e alos prinçipes de vsar \\\hline
1.2.31 & quae sine aliis virtutibus \textbf{ omnibus perfecte possit haberi . Sic etiam tractatores veritatis senserunt } dicentes virtutes connexas esse . & sin las otras uirtudes todas \textbf{ Et en esta misma manera avn todos los que tractaron delas uirtudes sentieron esto } e dixieron \\\hline
1.2.31 & Philosophum circa finem 6 Ethicor’ \textbf{ manifeste probare virtutes connexas esse . } Sed ut soluat huiusmodi obiectiones , & Et pues que assi es deuedes saber \textbf{ quel philosofo çerca la fin del sexto libro delas ethicas prueua manifiestamente que todas las uirtudes son conexas } e ayuntadas vna con otra . \\\hline
1.2.31 & Sic etiam ex ipsa pueritia \textbf{ videmus aliquos mox inclinari ad opera largitatis , } qui non sunt casti : & que en el tp̃o dela su moçedat \textbf{ luego son inclinados a obras de largueza e de franqueza } los quales non son castos . \\\hline
1.2.31 & non tamen perfecte liberales \textbf{ dici debent : quia ad perfectam virtutem spectat } non solum proponere bonum finem , & acabadamente liberales nin franços . \textbf{ por que pertenesçe ala uirtud acabada } non solamente establesçer fin conuenible \\\hline
1.2.31 & si vero sit multum infecta phlegmate dulci , \textbf{ videtur participare quandam dulcedinem . } Sic quales sumus & si mucho es llena de flema \textbf{ dulçe paresçen le todas las cosas dulçes . } En essa misma manera quales somos segunt nr̃a uoluntad \\\hline
1.2.31 & Quare sic decet Reges , \textbf{ et Principes esse quasi semideos , } et habere virtutes perfectas : & sin todas las otras . \textbf{ Por la qual cosa si conuiene alos Reyes e alos prinçipes de ser } assi commo medios dioses \\\hline
1.2.32 & per quam quis debet \textbf{ esse bonus ultra modum humanum , } appellatur a Philosopho heroica & Mas aquella uirtud por la qual alguno es dich̃o bueno \textbf{ sobre la manera comunal de los omes } es llamada del philosofo eroyca \\\hline
1.2.33 & Virtutes autem politicas , \textbf{ esse virtutus acquisitas , } per quas homines bene se habent & Et las uirtudes politicas son uirtudes g̃nadas \textbf{ que ganan los . | omes por buean sobras } por las quales los omes bien se han \\\hline
1.2.33 & et purgati animi dicunt \textbf{ esse virtutes infusas , } per quis quis bene se habet ad diuina . & e las de pragado coraçon \textbf{ son dichas uirtudes enuiadas } que enuia dios en el alma del omne \\\hline
1.2.33 & et tales dicuntur \textbf{ habere virtutes purgatorias . Aliqui vero sunt } quodammodo iam assecuti similitudinem illam : & e tales son dichos auer uirtudes pgatorias . \textbf{ Mas otros algunos son que en algua manera han ya conssigo esta semeiança diuinal } e tales son dichos auer uirtudes de pgado coraçon . \\\hline
1.2.33 & secundum se vera dicant , \textbf{ non tamen videntur accedere ad intentionem eorum , } qui hoc modo de virtutibus sunt locuti . & Mas commo quier que estos digan cosas uerdaderas dessi \textbf{ enpero non paresçe que se allegan ala entençion } de aquellos \\\hline
1.2.33 & quam ponebant , \textbf{ dicebant esse acquisitam . } Sectando ergo Philosophorum viam , & que los philosofos ponian dizian \textbf{ que era ganada de los omes | por vso de buenas obras } ¶Et pues que assi es siguiendo el camino \\\hline
1.2.33 & habere virtutes purgati animi facientes \textbf{ ipsum obliuisci passiones illas crebras , } quia iam habet animum purgatum et castigatum , & las quales le fazen escaeçer e oluidar las passiones \textbf{ e las delectaçiones desordenadas } por que ya ha el \\\hline
1.2.33 & sed etiam nominare , \textbf{ et audire turpia nefas esse debet . } Bene ergo eis competunt exemplares virtutes , & mas avn delas oyr no obra ¶ \textbf{ Et pues que assi es mucho ꝑ } tenesçen aellos las uirtudes exenplares \\\hline
1.2.34 & magis tamen videtur \textbf{ esse dispositio ad virtutem , } quam virtus . & largamente tomado la uirtud . \textbf{ Empero mas paresçe que sea disposiçion } ala uirtud que uirtud . \\\hline
1.3.1 & sed prout tendimus in ipsum , \textbf{ habet esse in nobis desiderium : } prout vero quietamur in eo , & Mas en quanto aquel bien ha de ser \textbf{ en nos es en nos desseo . } Et en quanto folgamos en aquel bien esta en nos gozo e delectaçion . \\\hline
1.3.3 & Sic etiam antiquitus \textbf{ si perspeximus ciuitatem aliquam dominari et tenere monarchiam : } hoc erat , quia ciues pro Republica non dubitabant & Et en essa misma manera avn si cataremos al tp̃o \textbf{ quando alguna çibdat auie señorio e tenie sennorio sobre las otras esto era } por que los çibdadanos non duda una de se poner ala muerte \\\hline
1.3.3 & Inter caetera autem , \textbf{ quae inducere possent alios ad virtutes , } est , ut bonum diuinum & Mas entre todas las uirtudes \textbf{ que pueden los Reyes } e los prinçipes aduzir a uirtudes es que amen prinçipalmente el bien diuianl e el bien comun \\\hline
1.3.3 & et depraedabat sacra . \textbf{ Viso quomodo Reges et Principes se habere debeant ad amorem , } quia principaliter debent amare bonum diuinum et commune : & e dannaua las eglesias e las casas santas \textbf{ ¶ visto en qual manera los Reyes | e los prinçipes se de una auer al amor } Ca nal e comun de ligero puede paresçer \\\hline
1.3.3 & Viso quomodo Reges et Principes se habere debeant ad amorem , \textbf{ quia principaliter debent amare bonum diuinum et commune : } de facili patere potest , & e los prinçipes se de una auer al amor \textbf{ Ca nal e comun de ligero puede paresçer } en qual manera \\\hline
1.3.3 & potissimum ergo in intentione cuiuslibet \textbf{ esse debet quid amandum . } Ostenso ergo quomodo Reges et Principes & Et pues que assi es la prinçipal entençion de cada vno deue ser \textbf{ que cosa ha de amar } Et por ende mostrado \\\hline
1.3.4 & intendere \textbf{ et amare bonum regni et commune . } Quare si desiderium debet & Conuiene alos Reyes e alos prinçipes entender e amar \textbf{ prinçipalmente el bien del regno e el bien comun . } Por la qual cosa si el desseo deue tomar mesura del amor \\\hline
1.3.4 & ex amore , \textbf{ principaliter Reges et Principes debent desiderare bonum statum regni : } ut quod qui in regno sunt , & Por la qual cosa si el desseo deue tomar mesura del amor \textbf{ Conuiene que los Reyes e los prinçipes desse en prinçipalmente el buen estado del regno } assi que todos quantos son en el regno \\\hline
1.3.5 & Possumus autem quadrupliciter ostendere , \textbf{ quod deces Reges et Principes decenter se habere circa spem , } et sperare speranda , & Mas nos podemos mostrar en quatro maneras \textbf{ que conuiene alos Reyes | e alos prinçipes de se auer } conueniblemente cerca la esperança \\\hline
1.3.5 & quodcunque bonum possit \textbf{ esse amor vel desiderium : } spes tamen esse non habet , & Ca commo quier que el amor e el desseo puedan ser cerca \textbf{ qual si quier bien . } Enpero la esperança non ha de ser \\\hline
1.3.5 & nisi sibi videatur \textbf{ esse bonum arduum , } et difficile . & si non cerca de bien alto e guaue de alcançar . \textbf{ Ca ninguno non es para si non bien alto } e guaue de alcançar \\\hline
1.3.5 & ad Reges et Principes leges ponere , \textbf{ spectat ad eos sperare bonum . } Rursus quia principale intentum & e alos prinçipes de poner las leyes . \textbf{ Et parte nesçe a ellos de es par algun bien . } Otrosi por que la prinçipal entençion del fazedor delas leyes \\\hline
1.3.5 & ab aliquibus bonis arduis , \textbf{ videntur mereri indulgentiam , } quia ciuilis potentia , diuitiae , & e si se tiran de algunos bienes altos \textbf{ e grandes meresçen perdon } por que el poderio ciuil e las riquezas \\\hline
1.3.5 & Quare cum Reges et Principes \textbf{ tendere debeant in bona ardua , } et debeant prouidere bona futura possibilia ipsi regno : & e los prinçipes \textbf{ de una entender alos bienes altos e grandes } e de una proueer los biens \\\hline
1.3.5 & et debeant prouidere bona futura possibilia ipsi regno : \textbf{ decet eos esse bene sperantes per magnanimitatem , } quia habent omnia & que han de venir e los bienes que pueden acahesçer a su regno . \textbf{ Por ende conuiene a ellos de serbine esparautes | por la magnanimidat } que han en ssi \\\hline
1.3.5 & inexperti enim non possunt \textbf{ cognoscere arduitatem operis . } Contingit etiam hoc ex immoderatione passionis : & enlos fechͣs non pueden saber las cosas altas e grandes abiertamente \textbf{ por que non sopieron la guaueza | nin la alteza delas obras } Et esto mismo contesçe avn \\\hline
1.3.6 & ut consiliatiui reddantur , \textbf{ habere aliquem moderatum timorem . } Secundo hoc idem inuestigare possumus & conuiene alos Reyes e alos prinçipes de auer algun temor tenprado \textbf{ sienpre tomado conseio sobrello ¶ } Lo segundo podemos esso mismo mostrar \\\hline
1.3.6 & in seipso contrahitur , \textbf{ et redditur immobilis . Quare si indecens est caput regni siue Regem esse immobilem et contractum , } indecens est ipsum timere timore immoderato . & et pierde el mouimiento . \textbf{ Et por ende si es cosa desconuenible | que la cabeca del regno o el Rey } sea tal que se non mueua \\\hline
1.3.6 & quia tunc nihil aggreditur . \textbf{ Moderate ergo se habere ad timorem , } et ad audaciam Regibus et Principibus omnino congruit . & por que estonçe non acometria ninguna cosa¶ Et pues \textbf{ que assi es auersse tenpradamente al temor } e ala osadia es cosa en todo en todo conuenible alos Reyes e alos prinçipes . \\\hline
1.3.7 & nisi credat \textbf{ ipsum fore fecisse vel in se , vel in filios , } vel in amicos , & otrosi non creyere \textbf{ quel fizo algun mal } o en si o en sus fios o en amigos o en algunos otros \\\hline
1.3.7 & cum scimus aliquem esse malum , \textbf{ ut cum scimus aliquem esse furem , } possumus ipsum odire , & Por que luego quando sabemos \textbf{ que alguno es ladron podemos le mal querer } si quiera aya fecho mal a nos o a \\\hline
1.3.7 & statim enim cum ratio dicit \textbf{ vindictam esse fiendam , } statim vult currere , & ento dize \textbf{ que sea techa vengança } luego quiere correr \\\hline
1.3.7 & cauendum est \textbf{ habere rationem obnubilatam , } et non plene rationi obedire : & Et ponen de si en cada vn omne es de esquiuar \textbf{ que aya la razon | e el entendimiento oscuresçido } e non obedezca conplidamente ala razon a cada vno es de foyr \\\hline
1.3.8 & Dicebatur enim supra , delectationes , \textbf{ et tristitias tenere ultimum gradum } in ordine passionum : & que las delecta connes \textbf{ e las tristezas tienen el } postrimero grado en la orden delas \\\hline
1.3.8 & ut patet per Philos’ 10 Ethicor’ . \textbf{ Eudoxus autem posuit omnem delectationem esse bonam : } quia quod ab omnibus appetitur & assi commo paresçe por el philosofo en el decimo libro delas ethicas \textbf{ Es heudoxio puso | que todas las delectaçiones eran buenas } e fazia esta razon que aquella cola que es desseada de todos \\\hline
1.3.8 & quod sint amici : \textbf{ et quia delectabile est habere amicos , } delectamur : & que son nros amigos . \textbf{ Et por que es cosa delectable } auer amigos delectamos nos \\\hline
1.3.8 & Nam per huiusmodi considerationem cognoscimus \textbf{ talia esse modica bona : } ideo eis amissis non dolebimus , & que tales bienes \textbf{ commo estos son muy pequa nons bienes . | Et por ende aquellos perdudos } non nos dolemos dellos sinon por auentura por accidente alguno en \\\hline
1.3.8 & eorum impedimur \textbf{ ab operibus virtuosis . Patet ergo non esse dolendum , } nisi de turpibus , & quanto por perdimiento de aquellos bienes somos enbargados delas obras uirtuosas . \textbf{ Et pues que assi es paresçe | que non nos deuemos doler } sinon delas cosas torpes \\\hline
1.3.9 & Nam omnes aliae passiones videntur \textbf{ ordinari ad istas ; } ut passiones sumptae respectu boni , & assi commo las passiones \textbf{ que son tomadas } en conparacion de algun bien \\\hline
1.3.9 & ut passiones sumptae respectu boni , \textbf{ ordinari videntur ad spem , } et gaudium , & en conparacion de algun bien \textbf{ son ordenadas ala esperança e al gozo . } Mas las que son tomadas \\\hline
1.3.9 & sumptae autem respectu mali , \textbf{ ordinari videntur ad timorem , et tristitiam . } Nam passio sumpta respectu boni , & en conparaçion de algun mal \textbf{ son ordenadas al temor e ala tristeza . } Ca la passion tomada en conparaçion de algun bien . \\\hline
1.3.9 & oportet delectationem et tristitiam \textbf{ esse principales passiones respectu concupiscibilis . } Spes autem et timor sunt principales passiones respectu irascibilis . & Conuiene que la delectaçion e la tristeza \textbf{ sean prinçipales passiones | en conparaçion del appetito cobdiçiador . } Mas la esperança e el temor son passiones prinçipales \\\hline
1.3.10 & sicut timentes pallescunt . \textbf{ Nam ex eo , quod aliquis credit se amittere vitam , } quod est bonum interius , & assi commo los temerosos se tornan amariellos . \textbf{ Ca por tanto que alguno cree | que perdera la uida } que es bien de dentro teme mas \\\hline
1.3.10 & quia verecundia consurgit \textbf{ ex eo quod quis se credit amittere exteriora bona . } Duplex ergo est timor , & por que la uerguença se leunata de aquello que alguno cree \textbf{ que pierde los bienes de fuera . } Et pues que assi es dos son los temores . \\\hline
1.3.10 & et maxime si credit \textbf{ ipsum indigne pati illud malum , } sic est misericordia . & e mayormente si cree \textbf{ que alguno sufre aquel mala tuerto } assi es miscderia . \\\hline
1.3.11 & licet videantur \textbf{ esse laudabiles passiones , } non tamen simpliciter & Mas la uerguença e la nemessis \textbf{ commo quier que parescan passiones de loar } Enpero non son sinplemente de loar \\\hline
1.4.1 & quia non credunt \textbf{ alios esse malos , } sed ut plurimum credunt & por que non creen \textbf{ que los otros sean malos . } mas por la mayor parte creen \\\hline
1.4.1 & sed ut plurimum credunt \textbf{ omnes homines esse bonos . } Cuius ratio est , & mas por la mayor parte creen \textbf{ que todos los omes son buenos . } Et la razon desto es \\\hline
1.4.1 & Reges tamen et Principes , \textbf{ quos decet esse quasi semideos , } non solum quod turpia committant , & por que los Reyes e los prinçipes alos quales conuiene de ser \textbf{ assi commo medios dioses } non solamente non les conuiene de fazer cosas torpes \\\hline
1.4.1 & nisi ex suppositione : \textbf{ quia si contingeret eos operari turpia , } uerecundari deberent etiam plus quam alii , & por alguna razon \textbf{ assi commo si contesçiesse | que ellos obrassen algunas cosas } torꝑes deuen auer uerguença ahun \\\hline
1.4.1 & sunt digniora quam alia . \textbf{ Rursus decet eos esse magnanimos : } quia ( ut dicebatur & e alos prinçipes de ser magranimos \textbf{ e de grand coraçon } Ca assi commo es dicho dessuso \\\hline
1.4.1 & in malam partem , \textbf{ contingeret eos esse tyrannos , } et esse vastatores gentium . & Ca si los fechos de los subditos lienpre le interpetrassen en mala parte contesçrie \textbf{ que los Reyes serien tiranos e destruydores delas gentes } Et otrossi conuiene alos Reyes \\\hline
1.4.2 & Primo , quia non sunt maligni moris . \textbf{ Non enim putant alios esse malos , } sed sua innocentia alios mensurant . & por que non son maliçiosos de uoluntad \textbf{ nin cuydan de los otros | que son malos . } Mas por la su moçençia \\\hline
1.4.2 & quod quis de facili credat ei , \textbf{ quem credit esse bonum , } et quem non cogitat & Et pues que assi es commo natural cosa sea \textbf{ que qual quier omne de ligero cree a aquel que cuyda que es bueno . } et aquel que cuyda que non fabla en enganno \\\hline
1.4.2 & sunt pertinaces in mendatio cogitant enim \textbf{ se esse ingloriosos , } si appareat non sic esse & por que cuydan \textbf{ que ellos ya son en eglesia } si paresçiere alos otros \\\hline
1.4.2 & in Regibus et Principibus , \textbf{ qui debent esse caput et regula aliorum . } Indecens enim est Reges et Principes & e en los prinçipes \textbf{ que deuen lercabesca e regla de todos los otros . } Et por ende cosa desconuenible es alos Reyes de ser segnidores delas passiones \\\hline
1.4.2 & Nam cum inconueniens sit \textbf{ regulam esse obliquam , } Reges et Principes , & Porque cosa desconuenible es que la regla sea tuerta \textbf{ e ellos son commo regla . } Et por ende los Reyes e los prinçipes \\\hline
1.4.3 & non de facili fit eis fides , \textbf{ sed credunt omnes alios esse deceptores . } Ideo dicitur 2 Rhetoricorum , & nin dan fe alos sy dichos . \textbf{ Mas cuydan que todos los otros | sanmint tosos e engannadores . } Por ende dize el philosofo \\\hline
1.4.3 & sequitur quod sit naturaliter formidolosus . \textbf{ Sequitur ergo senes esse naturaliter timidos , } quia deficit in eis naturalis calor , & Et por ende siguese \textbf{ que los uieios son naturalmente temerosos . Ca fallesçe enellos la calentura natural } e han los mienbs naturalmente frios ¶ \\\hline
1.4.3 & quia enim multis annis vixerunt , \textbf{ credibile est eos fuisse passos indigentias multas . } Timentes ergo indigentiam pati , & Ca por que biuieron muchos a nons de creer es \textbf{ que ellos sufrieron muchͣs menguas . } Et por ende temiendo que auran adelante menguas son escassos . \\\hline
1.4.3 & quae quasi communis est ad omnia tacta . \textbf{ Dictum est enim senes esse frigidos . } Frigidus enim omnia constipat , & Mas puede aqui ser fallada vna razon que es a comun a todas estas cosas de suso dichͣs . \textbf{ Ca dicho es de suso | que los uieios son frios } e el frio tondas las cosas estrinne e aprieta e costͥmedolas \\\hline
1.4.3 & faciendo mediocres sumptus : \textbf{ sed etiam congruit eos esse magnificos , } magnifica faciendo . & e alos prinçipes de ser francos faziendo espenssas medianas \textbf{ mas ahun les conuiene de sern magnificos } e granados fazie do grandes cosas ¶ \\\hline
1.4.4 & Videtur autem Philosophus 2 Rhetoricorum , \textbf{ circa senes tangere quatuor mores , } qui possunt esse laudabiles . & fincanos de poner las costunbres dellos qson de loar \textbf{ Mas paresçe que el philosofo en el segundo libro dela rectorica pone quatro costunbres de los uieios } que pueden ser de loar ¶ \\\hline
1.4.4 & possunt tamen contra illam pronitatem facere \textbf{ consequi laudabiles mores . } Sic et illi & Empero pueden fazer contra aquella disposiconn \textbf{ e inclina conn natural } e segnir bueans costunbrs e de loar . \\\hline
1.4.5 & quod magnanimos et magnificos decet \textbf{ esse nobiles et gloriosos . } Vult enim ibidem , & e de grandes coraçones e magnificos \textbf{ e de grandes fechos e głiosos e much̃ honrrados } por que dize alli el philosofo \\\hline
1.4.5 & rationabile est , \textbf{ eos habere corpus bene dispositum , } et bene complexionatum . & con razon es \textbf{ que ellos ayan los cuerpos bien ordenados e bien conplissionados . } Et pues que assi es conmolos bien conplissionados \\\hline
1.4.5 & diligenter considerent quid agendum . \textbf{ Quarto nobiles contingit esse politicos , et affabiles . } Nam quia ut plurimum in curiis nobilium consueuit & e si temieren de fazer cosas reprehenssibles e si cuydaren sotilmente todo lo que han de fazer \textbf{ ¶La quatta condicion de los nobles es | que son corteses e amigables . } Ca porque en la mayor parte en las cortes delos nobles \\\hline
1.4.6 & quia habendo diuitias aliquas , \textbf{ credunt se acquisiuisse omnia bona . } Unde et 2 Rhetoricorum dicitur , & por que auyendo las riquezas \textbf{ cuydan | que han todos los bienes . } Onde en el segundo libro de la rectorica dize \\\hline
1.4.6 & Habentes ergo numismata , \textbf{ aestimant se habere omnia bona , } eo quod reputent & Et por ende los que han las monedas cuydan \textbf{ que han todos los bienes } por que cuydan que los dineros son dignidat e preçio de todas las otras cosas . \\\hline
1.4.6 & eo quod reputent \textbf{ pecuniam esse dignitatem , } et pretium omnium aliorum ; & que han todos los bienes \textbf{ por que cuydan que los dineros son dignidat e preçio de todas las otras cosas . } Por la qual cosa en los susco raçons se ensoƀueçen \\\hline
1.4.6 & dispicientes alios , \textbf{ et credentes se esse super eos , } eo quod videant illos indigere bonis eorum . & despreçiando alos otros \textbf{ e cuydando que son mayores que ellos } por que veen \\\hline
1.4.6 & Diuitiae enim , \textbf{ quia videntur esse bona fortunae , } non videtur sufficere industria humana & e que parte nesçen a dios . \textbf{ Ca las riquezas | por que paresçen biens de auentura } non paresçe \\\hline
1.4.6 & per ordinationem diuinam \textbf{ habere huiusmodi bona . } Hoc autem dictum Philosophicum & e por los ordenamientos de dios \textbf{ han estos bienes tenporales e estas riquezas . } Et por ende este dicho tan sotil del philosofo \\\hline
1.4.7 & sed sunt nuper ditati . \textbf{ Differunt ergo esse nobilem , } et esse diuitem . & mas fezieron sericos del otro dia aca . \textbf{ Pues que assi es diferençia ay } entre ser noble e ser rico \\\hline
1.4.7 & Quare contingit potentes \textbf{ magis esse temperatos , } quam diuites . & por la qual cosa contesçe \textbf{ que los poderosos son mas tenprados que los ricos ¶ } Lo terçero los poderosos son menos peleadores que los ricos . \\\hline
1.4.7 & Non enim curabunt \textbf{ facere paruam offensam , } sed vel in nullo damnificabunt alios , & por que non curan de fazer \textbf{ pequano tuerto nin pequeno danno . } Mas o en ninguna cosa non faran danno alos otros o les faran grand danno . \\\hline
1.4.7 & et dulcedine scientiarum , \textbf{ statim percipit ea esse maiora bona , } quam crederent . & por que aquel que comiença agostar dela bondat delas uirtudes e dela dulçedunbre delas sçiençias \textbf{ luego entiende | que aquellas cosas son mayores } e meiores bienes que el cuydaua . \\\hline
2.1.1 & ex quibus quadruplici via venari possumus , \textbf{ ipsum esse communicatiuum et sociale . } Prima via sumitur ex victu , & que el omne es natra \textbf{ alnen te comunal a todos | e conpannero ¶ } La primera manera se toma \\\hline
2.1.1 & Nam quia non habent complexionem ita puram , \textbf{ et ita redactam ad medium , } ut homo , non indigent cibo ita deputato , & Ca por que non han la conplission tan pura las oinanlias \textbf{ nin assi trayda atenpmiento } medianere commo el omne non han \\\hline
2.1.1 & quia magis habet complexionem puram \textbf{ et redactam ad medium , } indiget alimento praeparato et depurato . & Mas el omne por que ha conplission mas apurada \textbf{ e mas aducha atenpramiento medianero } por esso ha mester vianda mas apareiada e mas apurada \\\hline
2.1.1 & quasi naturale indumentum , \textbf{ habere videntur lanam , vel pennas . } Homini autem non sufficienter prouidet natura in vestitu : & que han natural uestido \textbf{ assi commo las bestias han la lana e las aues las pennolas . } Mas la natura non prouee al omne tan conplidamente en uestidura \\\hline
2.1.1 & ut aranea ex instinctu naturae \textbf{ debitam telam faceret , } si nunquam vidisset & por inclinaçion natra al \textbf{ faze su tela conuenible } avn que nunca aya visto otras arannas texer en essa misma manera \\\hline
2.1.2 & In praecedenti ergo capitulo determinauimus de societate humana , \textbf{ ostendentes eam esse necessariam ad vitam nostram : } quia per hoc manifeste ostenditur & Et pues que assi es en el capitulo sobredicho auemos determinado dela conpannia humanal \textbf{ mostrando que es neçessario a lanr̃auida } por que por esta razon se muestra manifiestamente \\\hline
2.1.2 & quia per hoc manifeste ostenditur \textbf{ necessariam esse communitatem domesticam : } cum omnis alia communitas communitatem illam praesupponat . & por que por esta razon se muestra manifiestamente \textbf{ que la comunidat dela casa es neçessaria } por que todas las comuni dades ençierran en ssi \\\hline
2.1.3 & et typo ostendere , \textbf{ quod decet homines habere habitationes decentes } secundum suam possibilem facultatem ; & por que ael parte nesçe generalmente demostrar \textbf{ por figera e por exienplo que conuiene alos omes de auer conueibles moradas } segunt el su poder e la su riqueza . \\\hline
2.1.3 & communitatem domus \textbf{ esse priorem aliis tempore et generatione ; } esse tamen posteriorem & Bien dicho es que la comuidat dela casa es primero \textbf{ por generaçion | e por tienpo que las otras . } Enpero es postrima \\\hline
2.1.3 & ad Reges et Principes , \textbf{ quia sicut regnum vel ciuitas praesupponunt esse domum , } sic regimen regni et ciuitatis & Mas espeçialmente esto parte nesçe a los Reyes e alos prinçipes . \textbf{ Ca assy commo el regno | e la çibdat } ante ponen la comunidat dela casa \\\hline
2.1.4 & Sciendum ergo , \textbf{ Philosophum 1 Politicorum sic describere communitatem domus : } videlicet , quod domus est communitas secundum naturam , & Pues que assi es deuedes saber \textbf{ que el philosofo en el primero libro delas | politicasasse declara } e difine la comunidat dela casa \\\hline
2.1.4 & oportuit \textbf{ dare communitatem ciuitatis . } Communitas ergo ciuitatis esse videtur & conuiene de dar comunidat ala çibdat \textbf{ sobre la comunidat deluarrio . } Et por ende \\\hline
2.1.4 & ad expugnandam ciuitatem aliam \textbf{ confoederare se alteri ciuitati ; } quare cum confoederatio ciuitatum utilis sit & para que pueda lidiar con otra çibdat \textbf{ que aya conpanna e amistança con otra çibdat } que la pueda ayudar . \\\hline
2.1.4 & quanto ex incuria propriae domus magis potest \textbf{ insurgere praeiudicium ciuitati et regno , } quam ex incuria aliorum . & quanto por mal gouernamiento de su casa proprea \textbf{ mas se puede leuna tar piuyzio ala çibdat | e al regno } por mal gouernamiento de los otros . \\\hline
2.1.5 & quod Damascenus ait , \textbf{ generationem esse quid naturale , } et esse opus naturae . & a qual lo que dize damasçeno \textbf{ que la generaçion es cosa natural } e es obra de natura . \\\hline
2.1.5 & generationem esse quid naturale , \textbf{ et esse opus naturae . } Rursus rerum conseruatio , & que la generaçion es cosa natural \textbf{ e es obra de natura . } Otrossy la conseruaçion \\\hline
2.1.5 & Hoc ergo modo hae duae communitates faciunt domum esse quid naturale : \textbf{ quia communitas viri et uxoris ordinatur ad generationem , } communitas vero domini & fazen ser la casa cosa natural . \textbf{ Ca la comunidat del uaron e dela mugnies ordenada ala generacion . } Mas la comunidat del sennor e del sieruo es ordenada ala \\\hline
2.1.5 & quia sine eis domus congrue esse non valet . \textbf{ Quod autem communicatio viri et uxoris sit propter generationem , } videre non habet dubium : & por que sin ellas non puede ser la cosa conueniblemente . \textbf{ Mas que la comunidat del uaron | e dela muger sea } para la generaçion non adubda ninguna \\\hline
2.1.5 & requiritur \textbf{ communitas viri et uxoris propter generationem , } sic requiritur ibi communitas domini & es men ester la comunidat del uaron \textbf{ e dela mugni | para la generaçion } en essa misma manera es \\\hline
2.1.5 & et alia conseruationi , \textbf{ dicuntur facere primam domum : } quia sine eis domus congrue & e la otra ala con leruaçion \textbf{ fazen la primera cala } por que sin ellas la primera casa non puede estar conueniblemente . \\\hline
2.1.5 & vel habent aliquid aliud loco bonis . \textbf{ Decet autem omnes ciues cognoscere partes , } ex quibus componitur domus : & o alguna otra cosa en logar de bueye . \textbf{ Et conuiene a todos los çibdadanos de conosçer } e saber las partes de que se conpone la casa \\\hline
2.1.5 & Quia ergo cognitio partium domus , \textbf{ et scire quot genera personarum , } et quot communitates requiruntur ad domum : & Ca saber las partes dela casa \textbf{ e saƀ quantos son los linages delas perssonas | e quintas son las comuindades } que son menester ala casa \\\hline
2.1.6 & videlicet , viri et uxoris , domini et serui , \textbf{ facere domum primam . } Sed tamen , & sobredicho dos comuni dades \textbf{ fazen la primera casa } o nuene de saber et uaron \\\hline
2.1.6 & si domus debet esse perfecta , \textbf{ oportet ibi dare communitatem tertiam , } scilicet patris et filii . & Emposi la casa fuere acabada conuiene de dar \textbf{ y la terçera comunidat } que es de padre e de fijo . \\\hline
2.1.6 & potest sibi simile producere , \textbf{ sed oportet prius ipsam esse perfectam . } statim enim , & luego que es fecha fazer otra semeiante \textbf{ assi mas conuiene que ella primeramente sea acabada } enssi \\\hline
2.1.6 & sed oportet prius ipsum esse perfectum : \textbf{ producere ergo sibi similem , } non est de ratione rei naturalis & luego otro su semeinante \textbf{ mas conuiene que primeramente el sea acabado . } Et pues que assi es engendrar su semeiante non pertenesçe a cosa natural tomada en qual quier manera mas pertenesçe a cosa natural en quanto ella es acabada . \\\hline
2.1.6 & et ea quae videmus in domo , \textbf{ reducere volumus in naturales causas , } dicemus duas communitates , & e las cosas que veemos en la casa queremos traer \textbf{ a razones naturales diremos que las dos comuidades } que son de varon e de muger e de señor e de sieruo \\\hline
2.1.6 & cum prius dixisset \textbf{ communitatem viri et vxoris , domini et serui facere communitatem primam : } postea in sequenti capitulo praedicti libri ait , & commo ouiesse dicho primeramente \textbf{ que la comiundat del omne e dela muger e del sennor e del sieruo fazen la primera casa . } Despues en el segundo capitulo desse dicho libro \\\hline
2.1.6 & Patet ergo quod ad hoc quod domus habeat esse perfectum , \textbf{ oportet ibi esse tres communitates : } unam viri et uxoris , aliam domini et serui , & Et por ende paresçe que para que la casa sea acabada \textbf{ que conuiene que sean enlla tres comuundades . } ¶ La vna del uaron e dela muger ¶ \\\hline
2.1.6 & oportet ibi esse tres communitates : \textbf{ unam viri et uxoris , aliam domini et serui , } tertiam patris et filii . & que conuiene que sean enlla tres comuundades . \textbf{ ¶ La vna del uaron e dela muger ¶ | La otra del senor e del sieruo ¶ } La terçera del padre e del fij̉o . \\\hline
2.1.6 & quod ibi oportet \textbf{ esse quatuor genera personarum . } Videretur tamen forte alicui & que conuiene que sean y . \textbf{ quatro linages de perssonas } Empero podrie paresçer a alguno por auentura que deuen y ser seys linages de perssonas \\\hline
2.1.6 & Nam cum in domo perfecta sint tria regimina , \textbf{ oportet hunc librum tres habere partes ; } in quarum prima tractetur primo de regimine coniugali : & Ca commo en la casa acabada sean tres gouernamientos . \textbf{ Ca conuiene que este libro sea partido en tres partes . } ¶ En la primera delas quales tractaremos del gouernamiento mater moianl . \\\hline
2.1.7 & ostendere \textbf{ qualis amicitia sit viri ad uxorem , } probat amicitiam illam esse secundum naturam : & quariendo mostrar \textbf{ qual es el amistança del uaron } a la muger prueua \\\hline
2.1.7 & qualis amicitia sit viri ad uxorem , \textbf{ probat amicitiam illam esse secundum naturam : } adducens triplicem rationem & qual es el amistança del uaron \textbf{ a la muger prueua | que aquella amistança es segunt natura . } Et aduze para esto tres razones \\\hline
2.1.7 & Hanc autem rationem tangit Philosophus 1 Politicorum , et 8 Ethicorum , \textbf{ ubi probat coniugium competere homini secundum naturam , } quia naturale est homini , & e en el octauo delas ethicas do prueua \textbf{ que el casamiento conuiene alos omes segunt natura } por que natural cola es al omne \\\hline
2.1.7 & Quare si naturale est homini , \textbf{ habere impetum ad sufficientiam vitae : } naturale est ei , & ø \\\hline
2.1.8 & cum non fit naturalis amicitia \textbf{ inter aliquos nisi obseruent sibi debitam fidem ; } ad hoc quod coniugium sit secundum naturam , & ca entre algunos \textbf{ si non guardaren | assi mesmos fe conuenible } para el casamiento \\\hline
2.1.9 & ut vult Philosophus 9 Ethicor’ , \textbf{ indecens est quoscunque ciues plures habere uxores : } quia eas non tanta amicitia diligerent , & assi conmo dize el philosofo en elix̊ . \textbf{ delas ethicas cosa desconuenible es | a quales si quier çibdadanos } e a quales se quier uatones de auer muchͣs mugieres \\\hline
2.1.9 & ut unus masculus uni adhaereat foeminae , \textbf{ sequitur in hominibus esse quid naturale , } ut quam diu filii indigent parentibus , & Por en de liguele \textbf{ que en los omes le acola natural } que mientra que los fijos han menester ayuda del padre \\\hline
2.1.9 & huius secundi libri plenius dicebatur . \textbf{ Postquam ergo pulli auium apposuerunt debitas pennas , } et peruenerunt ad debitum incrementum : & assi commo es dichon mas conplidamente ençima en el primero capitulo deste segundo libro ¶ \textbf{ Et pues que assi es despues que alos pollos delas aues cresçieren pennolas conuenibles } e vinieren acres çentamiento conueni \\\hline
2.1.9 & decet omnes ciues \textbf{ una sola uxore esse contentos . } Et tanto magis hoc decet Reges et Principes , & Conuiene que todos los çibdadanos sean pagados \textbf{ cada vno de vna sola mugier . } Et tanto esto mas pertenesçe a los Reyes \\\hline
2.1.9 & quanto decet eos meliores esse aliis , \textbf{ et magis sequi ordinem naturalem . } Patet ergo quod non solum ex parte viri et uxoris , & que todos los otros . \textbf{ Et pues que assi es paresçe } que non lolamente es conuenible de parte del uaron \\\hline
2.1.10 & in congruum unum virum \textbf{ etiam simul habere plures uxores ; } apud nullas tamen gentes & sts̃ naçiones baruaras non se tenido \textbf{ por desconuenible que vn uaron aya muchͣs mugieres en vno . } Empero entre ningunas gentes que biuen \\\hline
2.1.10 & vel propter aliquam figuram \textbf{ et signationem legimus , unius viri plures fuisse uxores : } tamen quod reperitur in paucis , & o por alguna significaçion o figera . \textbf{ Leemos que vn omne ouo muchͣs mugers . } Enpero aquella cosa que es fallada en pocos \\\hline
2.1.10 & Secundum enim commune dictamen rationis detestabile est \textbf{ unum virum simul plures habere uxores : } detestabilius tamen esset , & e de entendemiento cosa de denostares \textbf{ que vn uaron aya en vno muchͣ̃s mugers . } Empero mas de denostares \\\hline
2.1.10 & nam naturale est foeminam \textbf{ esse subiectam viro , } eo quod vir prudentia et intellectu sit praestantior ipsa . & por que el uaron es meior \textbf{ que la muger } por sabiduria e por entendemiento¶ \\\hline
2.1.10 & ad conseruationem ordinis naturalis , \textbf{ et ad debitam pacem , } sed etiam ordinatur ad procreationem filiorum . & e aguarda dela orden natural \textbf{ e apaz conueinble mas avn es ordenado a generacion de los fijos . } ¶ Lo quarto \\\hline
2.1.10 & ad filiorum procreationem : \textbf{ sic ordinatur ad eorum debitam nutritionem . } Inconueniens est ergo unam foeminam , & assi commo el casamiento es ordenado a generaçion de los fijos \textbf{ assi es ordenado anudermiento conuenible dellos . } Et por ende non es cosa conueniente \\\hline
2.1.10 & simul viris pluribus detestabilius esse debet . \textbf{ Decet ergo coniuges omnium ciuium uno viro esse contentas : } multo magis tamen hoc decet & que vna muger case en vno con mugons varones . \textbf{ ¶ Et pues que assi es conuiene | quelas mugers de todos los çibdadanos sean pagadas de vn uaron . } Enpero mucho mas conuiene esto alas mugers de los Reyes \\\hline
2.1.10 & Igitur ex parte procreationis filiorum omnino indecens est \textbf{ unam foeminam plures habere uiros . } Nam etsi unus masculus potest & Et por ende departe dela generaçion de lons fijos es \textbf{ cosadesconuenible en qua vna fenbra aya muchos maridos } ca commo quier que vn mas lo puede enprenniar muchͣs fenbras . \\\hline
2.1.10 & ne impediatur earum foecunditas , \textbf{ uno viro esse contentas . Tanto tamen hoc magis decet Regum , } et Principum coniuges , & por que non sea enbargado el \textbf{ sunconçebemiento sean paragadas de vn marido solo . | Enpero tanto mas conuiene esto } alas mugers de los Reyes e de los prinçipes \\\hline
2.1.10 & Nam ex hoc parentes solicitantur circa pueros , \textbf{ quia firmiter credunt eos esse eorum filios : } quicquid ergo impedit certitudinem filiorum , & çerca los moços \textbf{ por que creen firmemente | que ellos son sus fiios . } Et pues que assi es \\\hline
2.1.10 & Detestabile est ergo unum virum plures habere uxores : \textbf{ sed detestabilius est unam uxorem plures habere viros , } quia per hoc magis impeditur certitudo filiorum . & que vn ome aya muchas mugers . \textbf{ Et mucho mas de denostares | que vna muger aya muchos maridos } ca por esto se enbargaria mas la çertidunbre de los fijos . \\\hline
2.1.11 & ( dum tamen una foemina per coniugium uni copuletur viro ) \textbf{ licitum esse illud coniugium , } cuiuscunque generis , & que vna fenbra sea ayuntada avn uaron \textbf{ por matermonio a aqlmat monio es connenible } de qual se quier linage \\\hline
2.1.11 & Nam cum ex naturali ordine debeamus parentibus debitam subiectionem , \textbf{ et consanguineis debitam reuerentiam , } cum huiusmodi reuerentia debita non reseruetur & subiectiuo al padre e ala madre \textbf{ e reuerençia conueible alos parientes } e commo esta reuerençia conueinble non sea guardada \\\hline
2.1.12 & ( ut patet ex dictis ) \textbf{ ad debitam societatem , } ad pacificum esse , & assi commo paresçe en las cosas \textbf{ sobredichͣs | aconpannia conuenible } e abien de paz e abastamiento dela uida . \\\hline
2.1.12 & Prout ergo coniugium ordinatur \textbf{ ad debitam societatem , } apud Reges , & que el mater moion es ordenado \textbf{ ala conpannia conuenible } conuiene alos Reyes \\\hline
2.1.12 & coniugibus quaerenda est nobilitas generis : \textbf{ sed prout ordinatur ad esse pacificum , } quaerenda est multitudo amicorum : & en sus mugi eres nobleza de liuage \textbf{ mas en quanto el matermoino es ordenado abien de paz } deuen querer \\\hline
2.1.12 & nisi coniugium ordinaretur \textbf{ in quandam societatem debitam et naturalem . } Cum ergo debite et congrue nobili societur : & Mas esto non si asi el casamiento non fuesen ordenado a algua conpanna conuenible e natural . \textbf{ ¶ Et pues que assi es commo deuidamente | e conueniblemente } el noble deua ser aconpannado \\\hline
2.1.12 & ut possit nociua expellere , \textbf{ sic ad esse pacificum requiritur } abundantia ciuilis potentiae & quel enpesçen \textbf{ assi para el bien de paz | conuiene } que aya abondança de poderio çiuil \\\hline
2.1.12 & in esse pacifico . \textbf{ Coniugium igitur prout ordinatur ad esse pacificum , } quaerenda est ex eo amicorum pluralitas . & e non le dexan beuir en paz . \textbf{ Et pues que assy es el casamiento en quanto es ordenado a bien de paz } por ende es de querer en ellos bienes \\\hline
2.1.13 & ut in praecedenti capitulo dicebatur , \textbf{ ordinetur ad societatem debitam , et ad esse pacificum , } et ad sufficientiam vitae : & sobredicho sea ordenado \textbf{ aconpania conuenible e abien de paz e aconplimiento deuida . } Enpero avn es ordenado a generaçion conuenible de los fijos \\\hline
2.1.13 & ordinatur etiam nihilominus \textbf{ ad debitam prolis productionem , } et ad fornicationem vitandam . & aconpania conuenible e abien de paz e aconplimiento deuida . \textbf{ Enpero avn es ordenado a generaçion conuenible de los fijos } e a esquiua la fornicaçion . \\\hline
2.1.13 & quod agere secundum rationem , \textbf{ et insequi passiones , } modo opposito se habent , & Et dicho es de ssuso que obrar segunt razon e seguir las passiones \textbf{ lonco las contrarias } alli que quanto alguno mas sigue las passiones \\\hline
2.1.13 & ( quantum est de se ) \textbf{ magis videntur esse insecutores passionum , } quam viri , & quanto es dessi \textbf{ mas son seguidoras de passiones que los varones . } Ca el uaron es mas acabado en razon \\\hline
2.1.13 & Patet ergo ex iam dictis , \textbf{ quale debet esse coniugium , } et qualiter omnes ciues , & ya dichas qual deueler el casamiento . \textbf{ Et en qual manera todos los çibdadanos } e mayormente los Reyes \\\hline
2.1.15 & Dicebatur superius \textbf{ in domo esse tria distincta regimina : } nuptiale , & e la razon natural muestra . \textbf{ a dixiemos de suso que en la casa ay tres gouernamientos departidos } de los quales el vno es mater moian \\\hline
2.1.15 & sed idem est \textbf{ esse naturaliter barbarum et seruum ; } esse enim barbarum ab aliquo , & por que entre los barbaros non era ninguno natrealmente sennor \textbf{ mas vn omne era naturalmente barbaro e sieruo . } Ca ser barbaro de alguno este es ser estranno del \\\hline
2.1.15 & et expedit ei quod ab aliquo alio dirigatur , \textbf{ idem est esse natura barbarum et seruum . } Quare si apud Barbaros eundem habent ordinem uxor et seruus , & e conuiene que sea gado de otro este \textbf{ tal es naturalmente barbaro e sieruo } Por la qual cosa sient los barbaros han vna orden la muger \\\hline
2.1.15 & Ex parte igitur ordinis naturalis \textbf{ patet aliud esse regimen coniugale quam seruile : } et non esse utendum uxoribus tanquam seruis . & e de ser menguados de razon e de encendemiento . \textbf{ Et pues que assi es de parte de la orden natural paresçe que otra cosa es el gouernamiento del marido ala mug̃r } que del señor al sieruo . \\\hline
2.1.15 & sumitur ex parte perfectionis domus . \textbf{ Videtur enim domus esse imperfecta , et habere penuriam rerum , } et non sufficere sibi in vita , & e del abastamiento dela casa . \textbf{ Ca paresçe que la casa non es acabada | e que a ninguna delas cosas } e non abasta assi en la uida \\\hline
2.1.16 & quia regimen coniugale est aliud a paternali et seruili : \textbf{ et ostendere quod aliter debet se habere vir } tam erga uxorem , & matermoian les otro que el paternal \textbf{ e que el suil e mostrar } que en otra manera se deua auer el uaron cerca la mugni \\\hline
2.1.16 & ne possit bene speculari , \textbf{ et ne possit libere exequi actiones suas . } Nascentes ergo ex tali coniugio & por que non pueda bien entender \textbf{ e que non pueda faze sus obras libremente . } ¶ Et pues que assi es los que nasçen de tal casamiento \\\hline
2.1.19 & Nam quicunque vult aliquid bene regere , \textbf{ oportet ipsum speciales habere cautelas ad ea , } circa quae videt ipsum magis deficere . & conuiene \textbf{ que el aya algunas cautelas espeçiales | para aquellas cosas } en las quales vee \\\hline
2.1.19 & et honestas . \textbf{ Nam non sufficit coniuges esse castas , } et cauere sibi ab operibus illicitis : & las de ser linpias e honestas \textbf{ por que non abasta | que las muger ssean castas } e se guarden de malas obras . \\\hline
2.1.19 & tanto maior credulitas adgeneratur viro , \textbf{ ut ei debitam fidem seruet . } Tali ergo regimine regendae sunt coniuges , & tanto mayor firmeza faze en su marido \textbf{ para quel guarde fialdat . } ¶ Et pues que assi es por tal gouernamiento \\\hline
2.1.20 & et tanto magis hoc decet Reges , et Principes , \textbf{ quanto indecentius est eos propter huiusmodi actus habere corpus debilitatum , } mentem depressam , & e alos prinçipes \textbf{ quanto mas desconuenible es aellos | por tales obras carnales } auer el cuerpo enflaqueçido \\\hline
2.1.20 & qualiter cum eis debeant conuersari . \textbf{ Tunc autem viri ad uxorem est conuersatio congrua , } si ei ostendat debita signa amicitiae , & en qual manera deuen beuir conellas \textbf{ mas estonçe es dicha la conuersaçion | e la uida conuenible e buena entre el marido e la muger } si se mostraren sennales conuenibles de amistança e de amor . \\\hline
2.1.20 & si ei ostendat debita signa amicitiae , \textbf{ et si eas per debitas monitiones instruat . } Declarare autem quae sunt signa amicitiae debita , & si se mostraren sennales conuenibles de amistança e de amor . \textbf{ Et si el marido enssencare ala muger | por conuenibles castigos . } Mas declarar quales son las señales conueinbles dela mistança \\\hline
2.1.20 & et inspectis conditionibus personarum , \textbf{ suis uxoribus ostendere debita amicitiae signa , } et eas ( ut expedit ) & e catadas las condiconnes delas perssonas mostrar a sus \textbf{ mugersseñales conueibles de amor } e enssennarlas \\\hline
2.1.20 & et eas ( ut expedit ) \textbf{ per debitas monitiones instruere . } Decet Reges , et Principes , & e enssennarlas \textbf{ assi commo conuiene | por conuenibłs castigos . } onuiene alos Reyes e alos prinçipes \\\hline
2.1.21 & Tertio decet foeminas \textbf{ circa ornatum corporis esse simplices , } ut non nimia solicitudine ornamenta requirant . & ¶ Lo terçero conuiene alas mugers de ser \textbf{ sinples enel conponimiento de su cuerpo } por que non demanden \\\hline
2.1.22 & increpantur a viris , \textbf{ si contingat eos nimis esse zelotypos : } eo quod ipsi zelotypi & e fazen lo que deuen son denostadas a tuerto de sus maridos \textbf{ si ellos fueren muy çelosos } por que los çelosos son acostunbrados \\\hline
2.1.22 & si contingat suos viros \textbf{ esse nimis zelotypos . } Commune est enim & que por ende las muger sson mas abiuadas a mal \textbf{ quando sus maridos son muy çelosos dellas . } Ca comunal cosa es sienpre \\\hline
2.1.22 & de suis coniugibus \textbf{ esse nimis zelotypos . } Nec etiam decet eos & ø \\\hline
2.1.23 & oportet foeminas deficere a ratione , \textbf{ et habere consilium inualidum . } Nam quantum corpus est melius complexionatum , & que las mugers \textbf{ que fallezcan de vso de razon e que ayan el conseio flaço . } Ca quando el cuerpo es meior conplissionado tanto \\\hline
2.1.23 & In casu tamen potest \textbf{ esse muliebre consilium melius quam virile : } ut quia illud est citius in suo complemento , & que del omne \textbf{ por que el consseio de la muger es mas ayna el su conplimiento que deluats . } por que si acaesçiesse de obrar alguna cosa adesora \\\hline
2.1.24 & quanto ab usu rationis deficientes , \textbf{ minus possunt refraenare incitamenta concupiscentiarum quam viri . } Secunda via ad inuestigandum hoc idem , & e tan comenos pueden refrenar los abiuamientos de las cobdiçias \textbf{ que los uarones | quanto mas fallesçe enlłas razon } que en los omes \\\hline
2.1.24 & et ridet in facie earum , \textbf{ credunt ipsam esse amicam , } et reuelant ei omnia secreta cordis . & e a Reyr en su faz dellas \textbf{ luego ellas a aquella perssona tienen por amiga } e descubienle todas las poridades de su coraçon \\\hline
2.1.24 & reuelare secreta . \textbf{ Nam cum dicimus hos esse mores iuuenum , } hos mulierum , hos senum . & en qual manera los maridos de una descobrir a sus mugieres los sus secretos . \textbf{ Ca quando nos dezimos | que estas son las costunbres de los mançebos } e estas las de los uieios \\\hline
2.1.24 & nisi per diuturna tempora sint experti , \textbf{ eas esse discretas , prudentes , et stabiles , } et non esse secretorum propalatiuas . & saluo a aquellas de que han prouado de luengot \textbf{ pon que son sabias e entendidas e estables en vn proponimiento } e que non son descobrideras delos secretos \\\hline
2.1.24 & eas esse discretas , prudentes , et stabiles , \textbf{ et non esse secretorum propalatiuas . } His visis , & pon que son sabias e entendidas e estables en vn proponimiento \textbf{ e que non son descobrideras delos secretos } ¶ vistas estas cosas \\\hline
2.1.24 & quomodo Reges et Principes , \textbf{ et uniuersaliter omnes ciues se habere debeant ad suas coniuges , } et quomodo cum eis debeant conuersari , & en qual manera los Reyes e los prinçipes \textbf{ e generalmente todos los çibdad a uos se de una auer alus mugers } e como de una beuir con ellas . \\\hline
2.1.24 & non sufficienter \textbf{ esse traditam notitiam regiminis nuptialis , } eo quod non ostensum fit , & que non auiemos dado \textbf{ conplidamente sabiduria del gouernamiento de los casados e del casamiento } por que non es mostrado a vna \\\hline
2.1.24 & Sed quia de eis infra dicetur , \textbf{ volumus ea hic silentio praeterire . Primae partis secundi libri de regimine Principum finis , } in qua traditum fuit , & suso deuemos mostrar \textbf{ quales obras conuiene que vsen las mugers } Mas por que dellas diremos adelante \\\hline
2.2.1 & et filii pertineat \textbf{ ad domum iam inesse perfectam : } quia domus prima praecedit & Mas la comunindat del padre e del fijo parte nesçan ala casa ya acabada en su ser \textbf{ por que la casa primera es ante que la } casaque es ya acabada . \\\hline
2.2.1 & quia domus prima praecedit \textbf{ domum iam in esse perfectam , } ideo forte videretur alicui statim & por que la casa primera es ante que la \textbf{ casaque es ya acabada . } por ende por auentura parescria a alguno que luego despues que dixiemos del gouernamiento del casamiento \\\hline
2.2.1 & natura est solicita \textbf{ dare animalibus ora et alia organa , } per quae possunt sumere cibum et nutrimentum . & luego es cuydados a de dar a todas las aian lias bocas \textbf{ e todos los organos e instrumentos } por los quales puedan tomar la uianda qual les conuiene . \\\hline
2.2.1 & et ea regulant et conseruant : \textbf{ videmus enim super caelestia corpora influere in haec inferiora , } et ea regere , et conseruare . & e guardan las en su ser . \textbf{ Ca ueemos que los cuerpos | çelestiales enbian de suso su uirtud enlos cuerpos de yuso } e gouiernan los \\\hline
2.2.1 & quem habent ad filios , \textbf{ solicitari circa eos . } Licet omnes patres deceat solicitari & Conuiene que los padres por amor natural \textbf{ que han alos fijos sean cuy dadosos della } aguer que todos los padres de una \\\hline
2.2.2 & magis habet solicitudinem circa filios : \textbf{ naturale est enim quemlibet diligere sua opera , } ut Philosophus in Ethicorum & mas ha cuydado de sus fijos \textbf{ Ca natural cosa es que cada vno ame sus obras } assi commo dize el philosofo en las ethicas \\\hline
2.2.3 & patet , paternale regimen \textbf{ non esse idem quod dominatiuum , } sed sumit originem ex amore tamen , & paresçe que el gouernamiento del padre \textbf{ non es tal commo el gouernamiento a uereruo } ca toma comienço de amor . \\\hline
2.2.3 & possumus duplici via venare , \textbf{ paternale regimen trahere originem ex amore . } Prima via sumitur ex ordine naturali . & todemos prouar pardas rasones \textbf{ quel gouernamiento | qł padre toma comie y de amor¶ } La primera razon se torna de la orden natural \\\hline
2.2.4 & Dicebatur in praecedenti capitulo , \textbf{ paternale regimen sumere originem ex amore . } Videndum est igitur quantus sit amor patrum ad filios , & ssi commo es dicho en el capitulo sobredich̃ . \textbf{ El gouernamiento patrinal toma comienço del amor Et } pues que assi es deuemos uer \\\hline
2.2.4 & Sciendum ergo per Philosophum 8 Ethic’ triplici ratione probare , \textbf{ parentes plus diligere filios quam econtra . } Prima via sumitur & que por tres razones podemos prouar \textbf{ que los padres aman mas alos fijos | que los fijos alos padres } ¶La primera razon se toma del alongamiento del tp̃o¶ \\\hline
2.2.4 & nisi ex quibusdam signis , \textbf{ ut quia puer videt personas aliquas magis affici ad eum quam alias , } arguit illas parentes eius , & si non por algunas sennales o por oydo \textbf{ o por que el moço vio algunas perssonas } que se inclina una \\\hline
2.2.4 & quam econuerso ; \textbf{ cum diligere aliquem , } idem sit quod velle ei bonum , & que los fijos alos padres \textbf{ commo amara alguno sea essa misma cosa } que querer bien \\\hline
2.2.4 & quam econuerso . \textbf{ Simpliciter tamen parentes plus dicuntur diligere filios , } quam filii ipsos : & Enpero sienpre dezimos \textbf{ que los padres mas aman alos fijnos } que los fijos alos padres \\\hline
2.2.6 & a lasciuiis retrahantur . \textbf{ Decet ergo omnes ciues solicitari erga filios , } ut ab ipsa infantia instruentur ad bonos mores . & e por bueons castigos sean tirados delas loçanias . \textbf{ Et pues que assi es | conuieneque todos los çibdadanos ayan grand cuydado de sus fijos } assi que luego en su moçedat \\\hline
2.2.7 & quis debite et distincte \textbf{ proferre aliquod idioma , } nisi sit in eo in ipsa infantia assuefactus ; & e departidamente algun \textbf{ lenageiaie si non fuere acostunbrado ael de su moçedat . } Ca aquellos que se mudan en hedat acabada a tierras luengas do los legunaies son departidos del \\\hline
2.2.7 & et ab incolis illius terrae semper cognoscitur \textbf{ ipsum fuisse aduenam , } et non fuisse in illis partibus oriundus . & Mas luego son conosçidos de los moradores de aqual la tierra \textbf{ que son auenedizos } e que non nasçieron en aquella tierra . \\\hline
2.2.7 & ad perfectionem scientiae , \textbf{ nisi quasi ab ipsis cunabulis vacare incipiat ad ipsam . } Nam licet intelligentiae & a perfectiuo de sçiençia \textbf{ si non lo comne care de pequanon . } Ca commo quier que los angeles \\\hline
2.2.7 & si volunt suos filios distincte \textbf{ et recte loqui literales sermones , } et si volunt eos esse feruentes , & que los sus fijos departidamente \textbf{ e derechamente fablen las palabras delas letras } Et si quieren \\\hline
2.2.8 & ut vigere possint prudentia et intellectu . \textbf{ Septem scientias esse famosas apud antiquos , } antiqua auctoritas protestatur . & ø \\\hline
2.2.8 & est ut per debita argumenta , \textbf{ et per debitas rationes manifestemus propositum . } Oportuit ergo inuenire aliquam scientiam docentem modum , & que por argumentos conuenibles \textbf{ e por razones derechas | i anifestamos nr̃a uoluntad e nr̃a entençion . } Et por ende conuiene de fallar algua sçiençia \\\hline
2.2.8 & Nam cum oporteat \textbf{ eos esse quasi semideos , } et debite et absque negligentia & e mayormente delos Reyes e de los prinçipes . \textbf{ Ca commo les conuenga a ellos de ser } assi commo medios dioses e de entender conueinblemente \\\hline
2.2.9 & quam doctor : \textbf{ decet igitur ipsum esse inuentiuum . } Secundo decet ipsum esse intelligentem et perspicacem . & este tal mas es rezador que doctor ¶ \textbf{ Et pues que assi es conuiene al maestro | que non tan solamente sea fallador delas cosas } mas que sea entendido e sotil . Ca assi commo ninguon non puede abastar \\\hline
2.2.9 & decet igitur ipsum esse inuentiuum . \textbf{ Secundo decet ipsum esse intelligentem et perspicacem . } Nam sicut nullus bene et perfecte & que non tan solamente sea fallador delas cosas \textbf{ mas que sea entendido e sotil . Ca assi commo ninguon non puede abastar } asi en la uida bien \\\hline
2.2.9 & esse doctor iuuenum , \textbf{ ut eos per debitos sermones , } et per debitas monitiones & Et pues que assi es tal deue ser el doctor \textbf{ e el maestro de los mocos } que los pueda endozir \\\hline
2.2.10 & per se prauum et fugiendum , \textbf{ per debitas monitiones et correptiones inducendi sunt } ut relinquentes mendacium adhaereant veritati , & Por ende son de endozir \textbf{ por castigos | e por conseios conuenibles } que dexen la mentira \\\hline
2.2.10 & in illa quae vident . \textbf{ Quare si contingat eos videre turpia , } magis recordantur de illis , & e mayor acuçiavan a aquellas cosas que veen . \textbf{ Por la qual cosa si contesca que ellos vean cosas torpes } mas se acuerdan dellas \\\hline
2.2.10 & quia sicut decens est \textbf{ audire eos honesta , } et pulchra , & quanto a aquellos que oyen . \textbf{ Ca assi commo es cosa conuenible a ellos de oyr } cosas honestas e fermosas \\\hline
2.2.11 & oportet \textbf{ ipsum esse proportionatum calori naturali . } Quare si in tanta quantitate sumatur , & Ca si la vianda se ouiere bien a cozer \textbf{ conuiene que sea bien proporçionada ala calentura natural } Por la qual cosa si en tan grand quantia se \\\hline
2.2.11 & ideo cum quis assuescit , \textbf{ sumere cibum in aliqua hora , } ut plurimum appetit sumptionem eius in eadem hora . & por ende quando alguno se acostunbra a tomar la uianda \textbf{ en algua ora desordenada } por la mayor parte dessea dela tomar en aquella misma ora . \\\hline
2.2.11 & Nam etiam in vilibus cibariis potest \textbf{ quis ostendere se nimis gulosum , } si nimio studio velit ea esse parata . & por que avn en las viles viandas cada vno se puede mostrar \textbf{ por muy goloso } si las quisiere auer apareiadas con quant estudio . \\\hline
2.2.11 & Sufficit autem eos paulatim et pedetentim instruere , \textbf{ ut cum ad debitam aetatem peruenerint , } sint sufficienter instructi , & que poco a poco sean enformados e enssennados \textbf{ por que quando venieren a hedat } conueinble e acabada puedan ser enssennados \\\hline
2.2.13 & expedit aliquando habere aliquos ludos , \textbf{ et habere aliquas deductiones } licitas et honestas . & Conuiene a nos algunas uegadas de auer algunos trebeios \textbf{ e algunos solazes conuenibles e honestos . } Mas quales son estos trebeios \\\hline
2.2.13 & Videmus enim prudentes et bonos habere \textbf{ gestus ordinatos et honestos : } cohibent enim sua membra , & Ca veemos que los sabios \textbf{ e los buenos han gestos ordenados e honestos } por que estos tales costramnen e apetan sus mienbros \\\hline
2.2.13 & tempora , et aetates . \textbf{ Nam habentes complexiones magis depressas et minus porosas , } non sic laeduntur a calore et frigore , & departiendo entre las conplissiones e los tienpos e las hedades . \textbf{ Ca los que han las conplissiones espessas | e menos ralas non rsçiben } assi danno dela calentura \\\hline
2.2.14 & maxime competere iuuenibus , \textbf{ fugere societatem prauam , } sumitur ex eo quod iuuenes sunt & que conuiene alos mançebos de foyr \textbf{ la mala conpannia se toma desto que los mançebos son muy muelles } e muy tristor nabłs \\\hline
2.2.14 & est virtus organica siue corporalis . \textbf{ Quare oportet talem appetitum sumere modum , } et mensuram ex ipso corpore . & mas el appetito de los sesos es uirtud organica o corporal . \textbf{ Por la qual cosa conuiene } que tal desseo tome manera e mesura del cuerpo . \\\hline
2.2.15 & maxime videtur \textbf{ esse proportionatum proprio filio . } Secundo pueri sunt prohibendi a vino , & por que la leche dela madre paresçe mas mucho proporçio nada \textbf{ e mas conueinble al fijo | que otra } ning¶lo segundo alos moços es de defendeᷤ el vino mayormente en aquel tienpo \\\hline
2.2.15 & unde Philosophus septimo Politi’ ait , \textbf{ quod mox expedit pueris paruis consuescere ad frigora . } Assuescere enim pueros ad frigora utile est ad duo . & Onde el philosofo en el septimo libro delas politicas \textbf{ dizeque luego conuiene alos mocos pequanos } de acostunbrar los alos frios \\\hline
2.2.15 & quod expedit in pueris \textbf{ facere motus quoscunque } et tantillos ad solidandum membra , & dize \textbf{ que conuiene alos moços de faz quales quier mouimientos pequanos } para soldar los mienbros \\\hline
2.2.15 & Nam ipsi nihil tristes sustinere possunt : \textbf{ ideo bonum est , eos assuescere ad aliquos moderatos ludos , } et ad honestas aliquas & por que los moços non pueden sostir ninguna cosa triste . \textbf{ Por ende es bien de los acostunbrara algs trebeios tenprados } e a alguas delectaçiones honestas \\\hline
2.2.16 & ut cum dicimus , \textbf{ usque ad septem annos sic esse regendos : } a septimo usque ad decimumquartum sic esse instruendos , & assi deuian ser gouernados los moços . \textbf{ Et de los siete años | fasta los que torze } assi deuian ser enssennados . \\\hline
2.2.16 & exercitandi sunt per debita exercitia , \textbf{ et per debitos motus . } Ut habeant voluntatem bene ordinatam , & e bien ordenado son de vsar \textbf{ por bsos e por mouimientos conuenibles . } Mas por que ayan la uoluntad bien dispuesta \\\hline
2.2.16 & eos ordinari ad virtutes , \textbf{ ut habeant dispositam voluntatem . } Sciendum ergo , & finca de demostrar en qual manera conuiene alos moços de ser dispuestos e ordenados alas uirtudes \textbf{ porque ayan bien dispuesta e bien ordenada la uoluntad e el entendimiento . } Et pues que assi łes deuedes saber \\\hline
2.2.16 & perfecte scire non possunt . \textbf{ Ne tamen cum incipiunt habere rationis usum , } omnino sint indispositi ad scientiam , & fallesçe de vso de razon non pueden saber las sçiençias acabada mente . \textbf{ Enpero por que quando comiençan a auer } vso de razon non seanda todo mal apareiados ala sçiençia deuen ser acostunbrados alas otras artes delas \\\hline
2.2.17 & ( licet imperfecte ) \textbf{ incipiunt participare rationis usum , } ideo in illo tempore non solum curandum est & commo quier \textbf{ que non acabada mente } por ende en aqł tienpo \\\hline
2.2.17 & Nam quia a decimoquarto anno ultra incipiunt \textbf{ habere perfectum rationis usum , } ut dicebatur , & por que del xiiij ̊ año adelante comiença los mançebos de auer \textbf{ mas acabadamente vso de razon . } Ca assi commo es dicho desde \\\hline
2.2.17 & ex tunc potest \textbf{ instrui non solum in grammatica } quae est scientia verborum , & estonçe pueden ser enssennados \textbf{ non tan solamente en la guamatica } que es assi commo sçiençia de palabras o en logica \\\hline
2.2.19 & qualis cura gerenda sit circa filias . \textbf{ Nam sicut decet coniuges esse continentes , } pudicas , abstinentes , et sobrias : & çerca delas fijas \textbf{ ca assi commo conuiene alas madres } de ser continentes e castas e guardadas e mesuradas en essa misma manera conuiene alas fijas de ser tales \\\hline
2.2.20 & infra declarandum esse , \textbf{ circa quae opera deceat foeminas esse intentas . } Ostenso , & casamien toca y dixiemos que adelante serie de declarar cerca quales obras conuenia \textbf{ que las mugers fuesen acuçiosas . } ostrado que non conuiene alas moças de andar uagarosas a quande e allende \\\hline
2.2.21 & ut foeminae etiam a puellari aetate discant \textbf{ cautos proferre sermones , } decet eas non esse loquaces : & tomadesto \textbf{ que las mugrͣ̃s non sean prestas avaraias e apeleas } ca commo las muger se mayormente las mocas \\\hline
2.3.1 & vel ad sufficientiam vitae , \textbf{ quae supplere videntur indigentiam corporalem . } Determinabimus igitur & e a conplimiento dela uida \textbf{ e que paresçen que cunplen la mengua corporal . } Et por ende determinaremos en esta terçera parte deste segundo libro \\\hline
2.3.2 & Suprema autem in quolibet negotio \textbf{ esse videntur architectores et domini : } infima vero sunt organa inanimata : & Mas las cosas mas altas en cada vn negoçio \textbf{ son los maestros | e los gouernadores e los sennores . } Et las cosas mas baxas son instrumentos sin alma . \\\hline
2.3.2 & per se ipsos esse praeparatores mensarum , \textbf{ vel esse ostiarios , } aut aliqua talia exercere : & por si mismos sean apareiadores delas mesas \textbf{ o que sean porteroso } que vsen de o tristales cosas . \\\hline
2.3.3 & Quod autem Reges et Principes debeant \textbf{ habere habitationes mirabiles , } et subtili industria constructas , & Mas que los Reyes e los prinçipes de una auer \textbf{ moradas marauillosas e labradas } por engennio muy sotil \\\hline
2.3.3 & nam secundum Philosophum 4 Ethicorum capitulo de Magnificentia , \textbf{ maxime gloriosos et nobiles decet esse magnificos : } Reges ergo et Principes , & en el quarto libro delas ethicas \textbf{ enł capitulo dela magnifiçençia | que much mas conuiene alos Reyes } e alos prinçipes \\\hline
2.3.3 & In domibus ergo Regum et Principum \textbf{ oportet multos abundare ministros , } ut ergo non solum personas Regis et Principis , & e de los prinçipes conuiene \textbf{ que ayan muchos ofiçiales | e much ssiruient s̃ Et pues que assi es } por que non solamente la persona del Rey o del prinçipe mas avn \\\hline
2.3.3 & propter circumstantiam montium contingit \textbf{ ipsum non esse salubrem . } Sic enim imaginari debemus , & et por ende contesçe \textbf{ que el ayre | y non sea sano } porque deuemos assi ymaginar \\\hline
2.3.4 & illas aquas generari , \textbf{ vel transire per aliqua loca infecta , } a quibus talem odorem , & que aquellas aguaas son engendradas en logares corruptoso \textbf{ que passan por algunos logares non sanos } delos quales trahen tal color o tal sabor \\\hline
2.3.4 & In ordine autem Uniuersi , \textbf{ prout requiri aedificium construendum , } sunt tria consideranda , & mas en la arden del mundo \textbf{ segunt que demanda la morada } que es de fazer son de penssar tres cosas \\\hline
2.3.6 & et suprema dilectio in ciuitate . \textbf{ Tunc enim omnes viri diligerent omnes foeminas tanquam proprias , } sic etiam omnes homines diligerent omnes pueros tanquam filios proprios , & e grand amorio en la çibdat \textbf{ por que estonçe todos los omes | a marien a todas las mugers } assi commo sus prop̃as mugers . \\\hline
2.3.6 & Possumus autem ex diuersis locis \textbf{ in libro Polit’ accipere tria , } per quae triplici via venari possumus , & de departidos logares \textbf{ enł libro delas politicas tres cosas } por las quales podemos prouar \\\hline
2.3.6 & per quae triplici via venari possumus , \textbf{ quod expedit ciuitati ciues habere proprias possessiones . } Prima via sumitur , & por tres razons \textbf{ que conuiene ala çibdat | que los çibdadanos ayan possessiones propias } ¶ \\\hline
2.3.6 & ut plurimum contingeret ciuitatem \textbf{ illam sic ordinatam venire ad inopiam , } ut ciues non possent sibi in vita sufficere ; & por ende contesçeria en la mayor parte \textbf{ que aquella çibdat | assi orde nada uerme a grant pobreza } por que los çibdadanos non podrien abondar assi enla uida \\\hline
2.3.6 & utile est ciuitati ciues \textbf{ habere possessiones proprias , } ne propter ignauiam circa communia , & assi commo dicho es pro prouechosa cosa es ala çibdat \textbf{ que los çibdadanos ayan | possessionspropreas } por que non auiendo cuy dado çerca las cosas comunes dela casa \\\hline
2.3.7 & quia sapientes naturaliter debent dominari insipientibus , \textbf{ iustum habere bellum contra ipsos , } si eis nolint esse subiecti . & sennorsnaturalmente de los non sabios \textbf{ e por ende han batalla decha contra ellos } si non quisieren sorsus subiectos . \\\hline
2.3.8 & infinita est diuitiarum concupiscentia , \textbf{ cuius causa est studere homines circa viuere , } non circa bene viuere ; & que la cobdiçia delas riquezas es sin fin e sin mesura \textbf{ e la razon desto es | por que los omes estudian cerca beuir } e non çerca bien beuir . \\\hline
2.3.8 & et putant ipsum finem \textbf{ in diuitiis esse ponendum , } appetunt eas in infinitum . & que han de poner su fin \textbf{ e su bien andança en las riquezas } dessean las sin mesura e sin fin . \\\hline
2.3.9 & quod in prima communitate quae est domus , \textbf{ manifestum est nullum esse opus ipsius commutationis igitur } propter communitates & que es comuidat dela casa es cosa prouada \textbf{ que non es menest obra de muda conn ninguna } Et por ende por las otras comuidades \\\hline
2.3.9 & cum ex una parte regni oportebat \textbf{ eos accedere ad aliam , } portando secum victualia onerosa , & e que estan en vn logar del regno \textbf{ commo contesçe alas vezes | e los que estan en vna parte del regno an de yr } ala otra parte del regno leunado \\\hline
2.3.9 & et rerum ad numismata , \textbf{ oportuit inuenire commutationem numismatum ad numismata . } Patet ergo quot sunt commutationes , & e delas cosas alos \textbf{ diueros otra mudaçiones | que es de monedas alas monedas . } Et pues que assi es paresçe \\\hline
2.3.10 & ex totidem denariis numero , \textbf{ confici massam maioris ponderis : } ex quo casu ars sumpsit originem , & e por esta razon acaesçe por auentura \textbf{ que de tantos dineros en cuento se faze massa de mayor peso . | Et desta } auentraa tomo comienço esta arte \\\hline
2.3.10 & Quarta species pecuniatiuae \textbf{ dicitur esse tachos , } quod in latino idem sonat & La terçera manera del arte pecuniatiua \textbf{ es dicha en gniego | tal zez que es usura . } Et en latin tato suena commo parto \\\hline
2.3.10 & Videtur enim haec ars parere \textbf{ et generare denarios , } quam nos communi nomine appellamus usuram : & Et en latin tato suena commo parto \textbf{ por que pare por que paresçe que esta pare e engendradinos la qual arte nos } por nonbre comunal llamamos usura \\\hline
2.3.11 & Volens ergo accipere pensionem de usu denariorum , \textbf{ dicitur committere usuram , } uel dicitur usurpare , & de los dineros dezimos \textbf{ que comete usura } e tal es dichusurar e robar uso \\\hline
2.3.12 & oeconomicum et dispensatorem domus \textbf{ esse expertum circa possessiones , } sciendo quae sunt magis fructiferae , & Ca conuiene segunt el philosofo al mayordomo e al despenssero dela casa de ser prouado \textbf{ e sabio | derca las possessiones } sabien \\\hline
2.3.12 & quibus pecuniam sunt lucrati , \textbf{ dicitur scire lucratiuam experimentalem . } Recitat enim Philosophus & por los quales fechos ganaron alguas riquezas \textbf{ Esta prueua tal es dicho ganançiosa por prueua . } Ca el philosofo cuenta dos fechos particulares \\\hline
2.3.12 & vel per se , \textbf{ vel per alios esse expertos , } sciendo particulares conditiones regni , & por que conuiene a ellos \textbf{ que por si o por otros ayan prouada } de saber las condiconnes particulares del regno \\\hline
2.3.13 & Ostendemus enim primo seruitutem aliquam naturalem esse , \textbf{ et quod naturaliter expedit aliquibus aliis esse subiectos : } quod probat Philosophus primo Polit’ quadruplici via , & que alguna suidunbre es dichͣ natural \textbf{ e que conuiene que alg ssean subietos naturalmente a algunos otros } la qual cosa praeua el philosofo \\\hline
2.3.13 & Corpus enim non posset \textbf{ seipsum dirigere ad operationes debitas , } sed dirigitur ad huiusmodi opera & Ca el cuerpo non puede enderesçat \textbf{ assi mismo a obras conueinbles } si non fuere enderescado atales obras \\\hline
2.3.13 & quasi corpus ad animam , \textbf{ sequitur eos esse naturaliter seruos . } Sunt enim aliqui carentes prudentia et intellectu , & assi commo el cuerpo al alma siguesse \textbf{ que aquellos sean naturalmente sieruos } Et por que algunos son menguados de entendimiento e de sabideria \\\hline
2.3.13 & ut canes , et equos in multis \textbf{ consequi salutem propter prudentiam hominum , } quam ex propria industria & assi commo los canes e los cauallos \textbf{ que en muchͣs cosas han salud | por la sabiduria de los omes } la qual non podrian auer \\\hline
2.3.13 & a rationis usu quam foeminae a viris , \textbf{ sequitur eos naturaliter esse subiectos . } Quare seruitus est & que las fenbras de los uarones \textbf{ por ende se sigue | que algunos sean naturalmente subietos e sieruos } Pot la qual cosa la piudunbre es en alguna manera cosa natural \\\hline
2.3.14 & ( ut dicitur in Politic’ ) \textbf{ habere aliquem excessum respectu serui . } Huiusmodi autem excessus dupliciter esse potest , & en las politicas \textbf{ aya algua auentaia sobre el su sieruo . } Et esta auentaia puede ser en dos maneras \\\hline
2.3.14 & Videtur tamen huiusmodi iustum \textbf{ aliquo modo esse congruum , } si considerentur legum conditores . & mas es derech segunt prigon de ley . \textbf{ Enpero paresçe que este derecho en algua manera sea conuenible } si pararemos mientes alos establesçedores delas leyes . \\\hline
2.3.14 & sed ( ut ait ) non similiter esse facile , \textbf{ videre pulchritudinem animae , et corporis . } Secunda congruitas sumitur & Mas assi commo el dize non es semeiante cosa fazer paresçer la fermosura del alma \textbf{ e la fermosura del cuerpo ¶ } La segunda razon se toma dela defenssion delatrra . \\\hline
2.3.14 & si scirent se ex eis nullam utilitatem consecuturos ; \textbf{ sed cum cogitant eos acquirere in seruos , } reseruant ipsos propter utilitatem & soperiessen que nigunt pro non aurian de tal uençimiento . \textbf{ Mas quando pienssan que aquellos a quien vençe } que los gana \\\hline
2.3.15 & propter quod tales contingit \textbf{ esse naturaliter seruos , } ut est per habita manifestum . & Por la qual cosa conuiene \textbf{ que tales natraalmente sean sieruos } assi commo es manifiesto e prouado por las cosas ya dichͣs \\\hline
2.3.15 & Impotentes vero contingit \textbf{ esse ministros ex lege : } ut si qui in potentia deficientes ; & Mas aquellos que non son poderosos \textbf{ conuiene que sean ministros e siruientes por ley } assi commo si algunos fallesciessen enl poderio \\\hline
2.3.15 & Mercenarios vero contingit \textbf{ esse ministros ex conducto : } ille enim mercenarius dicitur , & Otrossi los merçenarios conuiene \textbf{ que sean ministros | por alquiler } por que aquel es dich merçenario \\\hline
2.3.15 & Rursus , quia contingit aliquando plures etiam ex nobili genere ortos toto tempore vitae suae \textbf{ non agere aliquod iustum bellum , } ut ex eo possent & que en todo tienpo de su uida \textbf{ non fazen ningua batalla iusta } por que por ella puedan ganar algunos seruientes e sieruos . \\\hline
2.3.15 & ad supplendum indigentiam domesticam oportuit \textbf{ esse aliquos ministros conductos seruientes } intuitu mercedis , & conuiene para cunplimiento dela mengua dela casa \textbf{ que ouiessen algunos seruientes alquilados | que los seruiessen por el gualardon } e por la merçed \\\hline
2.3.15 & quos virtus et amor boni inclinat ad seruiendum , \textbf{ decet principantes se habere quasi ad filios , } et decet eos regere non regimine seruili , & e el amor de bien los inclina asuir . \textbf{ Conuiene que los prinçipes se ayan çerca ellos | assi commo cerca de fijos . } Et conuiene les alos prinçipes delos gouernar non \\\hline
2.3.16 & si debet esse ordinata , \textbf{ oportet reduci in unum aliquem , } a quo ordinetur . & En essa misma manera cada vna muchedunbre si bien ordenada es \textbf{ conuiene que sea aduchͣa vn ordenador } de quien ella sea ordenada . \\\hline
2.3.16 & non multi possunt \textbf{ praesidere in officiis et ubi officia commissa } non magnam curam habent annexam , & do non pueden muchs auer los ofiçios \textbf{ por la poquedat de los moradores . | Et do los ofiçios acomnedados } non han grand cura anexa \\\hline
2.3.17 & et quia debita prouisio maxime videtur \textbf{ facere ad honoris statum , } ut instruantur Reges , et Principes , & e por que prouision conueible delas uestiduras \textbf{ mayormente parte nesçe a estado de onrra } por ende por que los Reyes e los prinçipes sean ensennados \\\hline
2.3.17 & cognoscatur \textbf{ eos esse unius Principis ministros . } Tertio circa prouisionem indumentorum & por que por la semeiança delas uestidas sea conosçidos \textbf{ que son seruientes de vn prinçipe¶ } Lo terçero çerca la prouision delas uestidas es de penssar la condiçion delas personas \\\hline
2.3.17 & Nam non omnes decet \textbf{ habere aequalia indumenta . } In tantis enim domibus & por que non conuiene que todos sean uestidos \textbf{ de eguales uestiduras caenta } grandescasas non solamente son legos mas avn aycłigos \\\hline
2.3.17 & eos aliter \textbf{ et aliter esse ordinatos . } Sic enim videmus & ø \\\hline
2.3.17 & Videmus enim communiter homines \textbf{ adeo affici ad patrias consuetudines , } et ad conuersationes regionis propriae , & mas de buenamente la veemos \textbf{ e por ende los ons en tanto lon mas inclinados | alas costunbres propreas dela su tierra } e alas conuersaconnes de su regno \\\hline
2.3.18 & et secundum quem decet \textbf{ eos esse meliores aliis ; } si probabile est ex bonis bonos , & Et segunt el su estado conuienel es de ser meiores \textbf{ que los | otrossi cosa prouable es } que de los bueons nasçen buenos \\\hline
2.3.18 & qui sunt ex nobilibus natalibus orti dicuntur \textbf{ esse nobiles secundum opinionem , } quia opinio probabilitati innititur , & aquellos tales son dichos ser nobles \textbf{ segunt opinion de los omes } por que aquella opinion se funda en alguna nobleza \\\hline
2.3.18 & et tales ( ut patet per praehabita ) \textbf{ probabile est esse prudentes , et bonos . } Huic autem probabilitati aliquando subest falsitas , & assi commo paresçe por lo que dicho es \textbf{ Et por ende cosa prouable es | que ellos son sabios e buenons . } Enpero en esta opinion prouable algunas uezes \\\hline
2.3.18 & esse tales secundum veritatem , \textbf{ decens est nobiles genere esse nobiles secundum mores . } Ex hoc ergo curialitas venisse videtur . & Et por ende cosa conueinble es \textbf{ que los nobles | por linage sean nobles } por costunbres e desto paresçe \\\hline
2.3.18 & et quia decet nobiles \textbf{ et magnos esse nobiles secundum mores , } inde sumptum est , & e por que conuiene \textbf{ que los nobles e los guaades sean nobles en costunbres } dende fue tomado \\\hline
2.3.18 & inde sumptum est , \textbf{ ut dicantur esse curiales habentes mores nobiles : } propter quod curialitas morum & dende fue tomado \textbf{ que sean dichs curiales e corteses | los que han nobles costunbres } por la qual cosa la curialidat e la cortesia \\\hline
2.3.18 & quod est opus temperantiae : \textbf{ non dimittere aciem , } quod est opus fortitudinis : & que es obra de tenprança . \textbf{ Otrossi manda que los caualleros non descçian par en el az } que es obra de fortaleza \\\hline
2.3.18 & quod est opus temperantiae . \textbf{ Curiales etiam dicuntur homines se habere erga suos ciues , } si non eis iniuriam inferant in uxoribus , & nin torpemente la qual cosa es obra de tenprança . \textbf{ avn los omes son dicho | que } seancurialmente contra sus çibdadanos sinon les fezieren tuerto en las mugers \\\hline
2.3.19 & in tertio Libro patebit . \textbf{ Solicitari vero circa quosdam ministros } et velle se de quibuscumque inimicis intrommittere , & mas non conuiene alos reyes e alos prinçipes de ser acuçiosos \textbf{ çerca quales si quier ofiçiales | nin cerca de sus ofiçios } nin se deuen entremeter \\\hline
2.3.19 & qui respectu eorum sunt inferiores et humiles , \textbf{ debent se ostendere moderatos : } quia erga eos velle se habere & tonprados a los sus seruientes propreos \textbf{ los quales en conparacion dellos son humillosos e baxos } por que contra ellos non se deuen mostrar en grand alteza \\\hline
2.3.19 & omnino enim decet Reges et Principes \textbf{ minus se exhibere quam caeteros , } et ostendere se esse personas magis graues & e alos prinçipes \textbf{ de se faz menos familiares | que los otros } e de se mostrar \\\hline
2.3.19 & si per diuturnitatem temporis constet \textbf{ ipsos esse beniuolos , } fideles , et prudentes , & e por dilectonn del prinçipe \textbf{ si fueren çiertos | por tpoluengo } que ellos son beniuolos e fieles e sabios atales podran descobrir sus poridades \\\hline
2.3.20 & Dicto quales debent \textbf{ esse ministri Regum et Principum , } et qualiter Reges et principes debeant se habere de ipsis : & ich quales deuen ser los seruientes de los Reyes \textbf{ e de los prinçipes | e en qual manera los Reyes } e los prinçipes se de una auer \\\hline
2.3.20 & restat ut dicamus qualiter in mensis . \textbf{ Principum circa eloquia habere se debeant } tam ipsi Reges et Principes & a ellos finca \textbf{ que digamos en qual manera en las mesas de los prinçipes se de una auer } enl fablar tan bien los Reyes \\\hline
2.3.20 & et ne intemperati appareant , \textbf{ decet in mensis vitare sermonum multitudinem , } decet etiam hoc ipsos ministrantes , & e por que non parezcan destenprados \textbf{ assi commo dicho es . } Avn esto mismo conuiene alos seruientes por que la orden e la manera del seruir \\\hline
3.1.1 & Quoniam omnem ciuitatem contingit \textbf{ esse communitatem quandam , } cum omnis communitas fit & e todo cunplimiento ha de ser \textbf{ or que toda çibdat conuiene que sea alguna comunindat } commo toda comunidat sea por graçia de algun bien . \\\hline
3.1.1 & gratia alicuius boni , \textbf{ oportet ciuitatem ipsam constitutam esse propter aliquod bonum . } Probat autem Philosophus primo Polit’ duplici via , & commo toda comunidat sea por graçia de algun bien . \textbf{ Conuiene que la çibdat sea establesçida por algun bien | Ca pruena el pho } enl primero libro delas politicas \\\hline
3.1.1 & utilem esse in vita humana , \textbf{ et esse principaliorem communitate ciuitatis . } Videtur enim suo modo communitas regni & que la comunidat del regno es prouechosa en la uida humanal \textbf{ e es mas prinçipal | que la comunidat dela çibdat ca paresçe } que assi se ha la comunidat del regno \\\hline
3.1.2 & et aliud virtuose viuere . \textbf{ Nam esse latissimum est , } ut dicitur in libro de Causis : & e otra cosa es beuir uirtuosamente \textbf{ por que el seres cosa muy general | e muy ancha } assi commo es dicho en el libro de causis \\\hline
3.1.3 & Videmus autem multos societatem politicam retinentes , \textbf{ eligere solitariam vitam , et campestrem . } Sed hae et aliae dubitationes & que muchos fuyen la conpanna politicas \textbf{ e çiuil | e escogen uida solitaria e montanensa } assi commo los hermitaons . . \\\hline
3.1.3 & quia ignoratur , \textbf{ quomodo naturale est homini esse animal ciuile . } Non enim hoc est sic homini naturale , & por que non saben los \textbf{ que en esto dubdan | en qual maneran atra al cosa es al omne de ser } aianlçiuil \\\hline
3.1.3 & aliqui enim sic habent \textbf{ appetitum corruptum et voluntatem peruersam , } quod nequeunt viuere in societate & ca algunos omes assi han el appetito corrupto \textbf{ e la uoluntad desordenada } que non pue den beuir en conpannia e segunt ley . \\\hline
3.1.4 & per quas probari uidebatur , \textbf{ ciuitatem non esse aliquid secundum naturam , } et hominem non esse naturaliter animal ciuile . & por las quales se prouaua \textbf{ que la çibdat non era cosa natural } e que el oen non era naturalmente aianlçiuil . \\\hline
3.1.4 & ciuitatem non esse aliquid secundum naturam , \textbf{ et hominem non esse naturaliter animal ciuile . } Cum ergo non satis sit & que la çibdat non era cosa natural \textbf{ e que el oen non era naturalmente aianlçiuil . } Et pues que assi es commo non cunpla soluer la \\\hline
3.1.4 & intendimus in hoc capitulo adducere rationes ostendentes ciuitatem esse quid naturale , \textbf{ et hominem esse naturaliter animal ciuile . } Possumus autem duplici uia ostendere & que muestren \textbf{ que la çibdat es cosa natural | e que el omne es naturalmente aianlçiuil } e podemos por dos razones mostrat \\\hline
3.1.4 & communitatem politicam \textbf{ siue ciuitatem esse aliquid secundum naturam . } Prima uia sumitur & que la comiundat politicas \textbf{ o la çibdat es algunan cosa segunt natura . } ¶ La primera razon se tomadesto \\\hline
3.1.4 & Probabatur enim supra , \textbf{ quod uiuere erat homini secundum naturam ; } ut natura non deficiat in necessariis , & ¶Lo primero se prueua \textbf{ assi ca fue prouado dessuso | que beuir es cosa natural al omne } e por que la nafa non pueda fallesçer en las cosas neçessarias \\\hline
3.1.4 & quam communitates illae , \textbf{ oportet eam esse secundum naturam . } Secunda uia ad inuestigandum hoc idem , & que estas dos comuidades \textbf{ por ende conuiene | que la çibdat lea comuidat natraal ¶ } La segunda razon para prouar \\\hline
3.1.4 & tanquam ad finem et complementum , \textbf{ ordinatur ad ciuitatem . Viso , ciuitatem esse aliquid secundum naturam : } reliquum est ostendere , & assi commo assu fin e asu conplimiento . \textbf{ Et pues que assi es iusto | que la çibdat es cosa natural . } finca de demostrar \\\hline
3.1.4 & alteri cani per suum latratum \textbf{ significare tristitiam , } vel delectationem quam habet . & quando se trista puede a otro can de mostrar \textbf{ por su ladrado sutsteza o su delecta conn que ha mas al omne } sobre todo esto le es dada la palabra \\\hline
3.1.4 & oportet communitatem domesticam \textbf{ et ciuilem esse quid naturale . } Nam si natura dedit homini sermonem , & que la comunidat dela casa \textbf{ e la comunidat dela çibdat sean cosas naturales } ca si la natura dio al omne palabra natural aquella comunidat \\\hline
3.1.4 & quae ordinatur ad illa , \textbf{ quae sunt apta nata exprimi per sermonem : } iustum enim et iniustum non proprie habet & que se han de demostrar \textbf{ conueinblemente | por la palabra conuiene } que sea natural \\\hline
3.1.4 & oportet ciuitatem \textbf{ esse quid naturale , } vel esse aliquid secundum naturam . & conuiene quela çibdat sea cosa natural \textbf{ e sea alguna cosa segunt natura } e podemos mostrar por tres razones \\\hline
3.1.4 & esse quid naturale , \textbf{ vel esse aliquid secundum naturam . } Possumus autem triplici via ostendere , & conuiene quela çibdat sea cosa natural \textbf{ e sea alguna cosa segunt natura } e podemos mostrar por tres razones \\\hline
3.1.5 & quod semper oporteat ciuitatem \textbf{ ex propriis possessionibus habere omnia quae requiruntur ad vitam : } sed sufficit ciuitatem sic esse sitam , quod per mercationes , & para aquellas cosas \textbf{ que son menester ala uida | mas cunple } que assi sea la çibdat establesçida \\\hline
3.1.5 & ex propriis possessionibus habere omnia quae requiruntur ad vitam : \textbf{ sed sufficit ciuitatem sic esse sitam , quod per mercationes , } et ponderis portatiuam , & mas cunple \textbf{ que assi sea la çibdat establesçida | que por mercadores } e por acarreo trayendo cargas \\\hline
3.1.5 & in eadem ciuitate \textbf{ congregari diuersos vicos , } ut facilius habeantur & assi commo cosa apuechosa es ala uida humanal \textbf{ que en vna çibdat sean ayuntados muchos uarrios } por que mas ligeramente e meior se puedan acoirer los vnos alos otros \\\hline
3.1.5 & Oportet ergo rectores ciuitatis \textbf{ habere ciuilem potentiam , } ut possint cogere et punire & Pues que assi es conuiene \textbf{ que los gouernadores de la çibdat ayan poderio çiuil } por que puedan costrennir e fazer iustiçia \\\hline
3.1.5 & Quare cum peruersi in ciuitate aliqua \textbf{ non audeant insurgere contra principem , } si sciant ipsum magnam habere ciuilem potentiam , & por la qual cosa commo los malos en alguna çibdat \textbf{ non se osenle unatat | contra el prinçipesi sopieren } que el ha grant poderio en la çibdat \\\hline
3.1.5 & non audeant insurgere contra principem , \textbf{ si sciant ipsum magnam habere ciuilem potentiam , } et dominare in ciuitatibus multis , & contra el prinçipesi sopieren \textbf{ que el ha grant poderio en la çibdat } e que han grant señorio en muchͣs cibdades \\\hline
3.1.6 & et faciat se Regem constitui super illas . \textbf{ Viso diuersos esse modos generationis ciuitatis et regni , } restat uidere & ¶ Et pues que assi es visto \textbf{ que son maneras departidas | de establesçimiento de çibdat e de Regno . } finca de ver en quantas partes conuiene \\\hline
3.1.7 & arguitur esse summe bonus . \textbf{ Videtur ergo ciuitas esse potissime bona , } si sit potissime una ; & que dios es muy bueno \textbf{ e por ende paresçe que la çibdat es muy | buenasi fuere muy vna . } Et pues que assi es quanto mas se allega a vnidat \\\hline
3.1.7 & reputarent \textbf{ quemlibet puerorum esse filium proprium . } Videbatur enim Socrati et Platoni totam dissensionem ciuium consurgere & mas cuydarian de cada vno moço \textbf{ que era su fijo propreo } por que paresçia a socrates e a platon \\\hline
3.1.7 & antiquiores reputarent se \textbf{ habere maximam unitatem } cum iunioribus , & e grant ayuntamiento de los padres alos fijos los mas antiguos \textbf{ cuydarian | que auian muy grant vnidat con los moços } e esso mismo los mocos cuydarian \\\hline
3.1.7 & et quod crederent \textbf{ eos esse suos filios , } illi vero opinarentur & por que los antiguos creerian \textbf{ que los moços eran sus fiios } e los mocos cuydarian \\\hline
3.1.7 & illi vero opinarentur \textbf{ eos esse suos patres . } Tertium vero quod senserunt & e los mocos cuydarian \textbf{ que ellos eran sus padres . } ¶ Lo terçero que sintieron los dichs philosofos cerca el gouernamiento dela çibdat . \\\hline
3.1.7 & est , quia dixerunt ciuitatem \textbf{ quamlibet diuidendam esse in quinque partes , } videlicet in agricolas , artifices , bellatores , consiliarios , et principem . & que dixieron \textbf{ que cada vna çibdat | demaser partida en çinco partes . } Conuiene a saber en labradores \\\hline
3.1.8 & non esse ciuitatem , \textbf{ et regnum non esse regnum . } Secunda via ad inuestigandum hoc idem , & commo dizian socrates e platones dezer que la çibdat non sea çibdat \textbf{ e el regno non sea regno . } La segunda razon para prouar esto mismo se toma \\\hline
3.1.8 & et auditus ideo oportet \textbf{ ibi dare diuersa membra exercentia diuersos actus : } sic quia ad indigentiam vitae & y departidos mienbros \textbf{ que fagan estas obras departidas . } En essa misma manera por que para conplir la mengua dela uida \\\hline
3.1.9 & cessarent litigia , \textbf{ quia crederent ciues omnes pueros esse filios suos , } et sic esset in ciuitate maximus amor . & e las contiendas enla çibdat \textbf{ por que cuydarian los çibdadanos | que todos los moços eran sus fijos propreos } e por ende en la çibdat seria muy grant amor Et pues que assi es nos podemos mostrar \\\hline
3.1.9 & Immo quia impossibile est omnes ciues \textbf{ aequaliter esse prudentes et bonos , } et esse aequaliter utiles ciuitati , & qua non puede ser \textbf{ que todos los çibdadanos sean egualmente sabios } e todos sean bien egualmente prouechosos ala çibdat \\\hline
3.1.9 & uellet secundum dignitatem suam \textbf{ ei fieri retributionem . } Hanc autem aequalitatem non de facili esset possibile reseruari inter ciues & por meior que el otro quarria \textbf{ que segunt la su dignỉdat le diessen mayor gualardo delas cosas comunes } mas esta egualdat non se podria guardar de ligero \\\hline
3.1.9 & non oporteret ciues \textbf{ omnes pueros reputare filios proprios . } Immo quia puerorum aliqui essent & assi comunes non conuernia \textbf{ que los çibdadanos cuydassen | que todos los moços fuessen sus fijos propreos } por que alguons de los moços son semeiantes \\\hline
3.1.9 & quilibet ciuis appropriaret sibi in filium , \textbf{ quem videret sibi esse similem . } Unde et Philosophus narrat 2 Politicor’ & e cada vno de los çibdadanos apropriaria \textbf{ assi por fijo a aquel que viesse | que lo semeiaua . } Onde el pho cuenta en el segundo libro delas politicas \\\hline
3.1.9 & Videlicet , ut ciues omnes pueros crederent \textbf{ esse proprios filios . } Tertia via sic patet . & que todos los çibdadanos creyessen \textbf{ que todos los moços serian sus fijos propreos } La terçera razon paresçe \\\hline
3.1.9 & si firmiter crederet \textbf{ ipsum esse talem ; } quam pater filium , & si creyesse uerdaderamente \textbf{ que era su nieto } que el padre amaria al fijo \\\hline
3.1.10 & Nam oportet in ciuitate \textbf{ consurgere lites , vulnerationes , et contumelias , } quae tanto detestabiliores sunt , & El quinto es abusion de los parientes ¶ Lo primero paresçe assi ca conuiene \textbf{ que en la çibdat se leunatenlides e feridas e deniestos } las quales cosas tanto son \\\hline
3.1.10 & propter honestatem \textbf{ et bonitatem morum parentes esse certos de suis filiis , } et quoslibet certificari de eorum consanguineis , & por bondat e honestad de costunbres \textbf{ que los padres sean çiertos de sus fiios } e cada vnos sean çiertos de sus parientes \\\hline
3.1.10 & et exaltare ignobiles , \textbf{ et non saluare amicitiam inter eos . } Tertium malum sic declaratur . & e enxalcar los viles \textbf{ e assi se salua la amistança entre ellos | ¶ } El terçero mal se declara \\\hline
3.1.10 & Quare si quilibet ciuis crederet \textbf{ quemlibet puerorum esse proprium filium , quia partiretur eius amor in tantam multitudinem , } modicum diligeret unumquemque , & por la qual cosa si cada vno de los çibdadanos cuydasse que cada vno de los moços era su fijo propreo \textbf{ por que el amor del se partia en tanta muchedunbre de fijos } muy \\\hline
3.1.10 & nullo modo suspicari posset \textbf{ omnes pueros esse suos filios . } Si ergo omnes diligerent tanquam filios , & si non fuesse loco en ninguna manera non podria sospechͣr \textbf{ que todos los moços fuessen sus fijos . } Et pues que assi es si a todos amassen \\\hline
3.1.10 & hoc esset ratione duorum vel trium puerorum , \textbf{ quos crederent esse proprios filios : } et quia illi non essent eis certitudinaliter noti , & por razon de dos o tres moços \textbf{ los quales cuydaria que eran suᷤ fijos propreos } e por que aquellos nonl serian conosçidos çiertamente \\\hline
3.1.10 & propter duos vel tres \textbf{ vel propter paucos pueros velle magnam multitudinem diligere puerorum tanquam proprios filios , } hoc est ponere parum de melle in multa aqua . & Mas esto reprahende el philosofo enel segundo libro delas politicas \textbf{ ca por dos o por tres o por pocos mocos querera mar grant muchedunbre de moços | assi conmo a fijos propreos } esto es poner poco de miel en muchͣ agua . \\\hline
3.1.11 & et firmiter credunt \textbf{ se esse tanta consanguinitate coniunctos , } multa habent litigia , & pocoscreen firmemente \textbf{ que son ayuntados en tan grant parentesco } e han muchas contiendas \\\hline
3.1.11 & propter communitatem mulierum et uxorum crederent \textbf{ se esse consanguinitate coniunctos ; } attamen inter eos & e delas mugers creyessen \textbf{ que eran ayuntados | por mayor parentesco . } Enpero segunt uerdat entre ellos \\\hline
3.1.11 & ne inter ciues oriantur dissensiones et iurgia , \textbf{ non sic esse possessiones communes , } ut Socrates statuebat . & por que non nazcan entre los çibdadanos uarias e contiendas \textbf{ qua non sean las possessio nes } assi comunes commo establesçio socrates mas lamzon delpho \\\hline
3.1.11 & adhibebit \textbf{ debitam diligentiam circa illa . } Expedit autem talia esse communia secundum liberalitatem : & ca cada vn sennor de sus bienes propreos aura mayor acuçia de aquellos bienes \textbf{ que si fuessen comunes } mas conuiene que las cosas sean comunes \\\hline
3.1.11 & debitam diligentiam circa illa . \textbf{ Expedit autem talia esse communia secundum liberalitatem : } quia cives inter se debent liberales esse , & que si fuessen comunes \textbf{ mas conuiene que las cosas sean comunes | segunt uirtud de franqueza } por que los çibdadanos entre ssi deuen ser francos ꝑtiendo sus bienes entre ssi . \\\hline
3.1.12 & Homines enim bellatores decet \textbf{ esse mente cautos et prouidos : } corde viriles et animosos : & ca los omes lidiadores conuiene que sean cuerdos \textbf{ por entendimiento e sabios } e sean rezios e esforcados de coraçon e fuertes e ualientes de cuerpo \\\hline
3.1.12 & quam eos in societate habere \textbf{ nam cum humanum sit timere mortem , } viriles etiam et animosi trepidant & que auerlos en su conpannia \textbf{ ca commo todos los omes | teman la muerte los esforçados } e de grandes coraçones temen \\\hline
3.1.13 & sed eius est \textbf{ intendere sanitatem tanquam finem : } sic cuius est ciuitatem ordinare , & mas ha de tener mienteᷤ en la sanidat \textbf{ assi comm̃en su fin bien } assi aquel que ha de ordenar la çibdat \\\hline
3.1.13 & videntes enim nullam dignitatem possidere , \textbf{ si contingat eos esse viriles et animosos , } seditiones mouent . & nin gͤdignidat en la çibdat \textbf{ si contezca | que ellos sean tales } que sean poderosos \\\hline
3.1.14 & Quare cum patefactum sit in praecedentibus , \textbf{ non expedire ciuitati possessiones , } uxores , et filios esse communes , & por las cosas dichͣs de suso \textbf{ que non conuiene ala çibdat } que las possessiones nin los fijos nin las mugieres sean comunes \\\hline
3.1.14 & nec esse decens , \textbf{ mulieres ordinari ad opera bellica ; nec esse utile , } eosdem semper in eisdem magistratibus praefici , & que las mugieres sean ordenadas \textbf{ alas obras de batalla | nin es prouechoso } que sienpre vnos ofiçialon sean puestos en essos mismos ofiçios \\\hline
3.1.14 & famulos et liberos , \textbf{ pascere tantam multitudinem bellatorum . } Tertio delinquebat Socrates & assi commo dize el philosofo gouernar atanta muchedunbre de lidiadores \textbf{ sin çibdadanos e sin mugers e sin siruientes e sin fijos } Lo terçero erraua socrates \\\hline
3.1.15 & volens ciuitatem \textbf{ ad minus mille continere bellatores . } Forte per bellatores intendebat nobiles , & e en batalladores quariendo \textbf{ que alo menos la çibdat ouiesse mil ł batalladores } por auentura \\\hline
3.1.16 & quomodo ciues habeant possessiones aequatas . \textbf{ Volebat enim tunc esse ciuitatem optime ordinatam , } si nullus ciuium haberet plures redditus , & en qual manera los çibdadanos de una auer las possesipnes igualadas \textbf{ ca creya | que estonçe seria la çibdat muy bien ordenada } si ninguno de los çibdadanos non ouiesse mayores rentas o mayores possessiones \\\hline
3.1.16 & de facili rector ciuitatis posset \textbf{ diuidere aequaliter possessiones illas inter ciues . } Sed ciuitate iam constituta , & podria partir el rectoor dela çibdat \textbf{ egualmente aquellas possessions entre los çibdadanos } mas la çibdat ya establesçida los çibdadanos \\\hline
3.1.16 & Statuit enim Phaleas ciuitatis rectorem \textbf{ hoc modo reducere hanc inaequalitatem } ad aequalitatem mediantibus dotibus statuendo & que el rector dela çibdat \textbf{ en esta manera aduxiesse esta desegualdat a egualdat | Conuiene a saber } por las arras \\\hline
3.1.16 & et scirent se non posse \textbf{ excedere suos conciues in possessionibus , } frustra propter hoc insurgerent lites et placita . & que vno non podia sobrepuiar \textbf{ los otros çibdadanos sus uesnos en possessiones } por esto debalde se leunatarian entre ellos las contiendas e las uaraias \\\hline
3.1.17 & non est possibile statuere in ciuitate \textbf{ omnes ciues habere aequalem numerum filiorum . } Propter quod ex parte procreationis prolis manifeste ostenditur & que otros por ende non se puede poner ley en la çibdat \textbf{ que todos los çibdadanos ayan ygual cuento de fijos } por la qual cosa se muestra manifiesta miente de parte dela generaçion de los fijos \\\hline
3.1.17 & praedictam legem \textbf{ non esse congruentem ; } eo quod congrue obseruari non possit . & por la qual cosa se muestra manifiesta miente de parte dela generaçion de los fijos \textbf{ quela ley puesta por felleas non es conuenible } por que se non puede guardar conueinblemente ¶ \\\hline
3.1.17 & decet enim ipsos \textbf{ esse liberales et temperatos : } non ergo bene dictum est quod ad bonum regimen ciuitatis sufficit ciues habere possessiones aequatas , & por que conuiene \textbf{ que los çibdadanos sean liberales e francos } e por ende non es bien dicho \\\hline
3.1.17 & esse liberales et temperatos : \textbf{ non ergo bene dictum est quod ad bonum regimen ciuitatis sufficit ciues habere possessiones aequatas , } nisi aliquid determinetur & que los çibdadanos sean liberales e francos \textbf{ e por ende non es bien dicho | que a buen gouernamiento dela çibdat } cunple de ser las possessiones egualadas \\\hline
3.1.18 & et in causa quam alibi , \textbf{ magis debent intendere rectores ciuium } circa reprimendas concupiscentias quam circa alia , & e en el comienço donde nasçe \textbf{ que en otra cosa los rectores delas çibdades } mas deuen entender en repreheder las cobdiçias \\\hline
3.1.18 & Meminimus tamen , \textbf{ nos edidisse quendam tractatum . } De differentia Ethicae Rhetoricae et Politicae , & enpero mienbranos \textbf{ que fiziemos vn tractado del partimiento } que es entre la ethica e la rectorica e la politica \\\hline
3.1.18 & si non possint \textbf{ consequi honorem debitum et condignum . } Quare si ciuium quidam personae sunt pauperes , & e con ueinble a ellos . \textbf{ Et por ende mas contienden los honrrados sobre la honra | que sobre la sustançia } por la qual cosa \\\hline
3.1.19 & Volebat autem bellatores \textbf{ debere habere arma , } et non terram . & e en labradores caquaria \textbf{ que los lidiadores troxiessen armas } e que non labrassen la tr̃ra \\\hline
3.1.20 & oportebat bellatores \textbf{ habere maiorem potentiam , } quam agricolae , & Et segunt esto conuiene \textbf{ que los lidiadores ouiessen mayor poderio } que los menestrales nin los labradores todos en vno . \\\hline
3.2.1 & quae vim legum obtinent . \textbf{ Videntur ergo sic se habere arma ad tempus belli , } sicut leges ad tempus pacis . & que han fuerça de leyes \textbf{ et por ende assi sean las armas | alt pon de guerra } commo las leyes al tro de paz . \\\hline
3.2.1 & sed de laudabili et vituperabili est exclamatio siue concionatio , \textbf{ quae potest respicere totum populum : } populus enim ad bene agendum , & e de denostar es llamamiento e conuiramiento \textbf{ que parte nesçe a todo el pueblo } ca el pueblo es de abiuar \\\hline
3.2.2 & vocat eum Philosophus nomine communi , \textbf{ et dicit ipsum esse Politiam . } Politia enim quasi idem est , & llamalle el philosofo nonbre comun \textbf{ e diz el poliçia } por que poliçia es \\\hline
3.2.4 & Philosophus 3 Politicorum videtur \textbf{ tangere tres rationes , } per quas probari videtur , & e han cunplimiento delas cosas \textbf{ lpho en el terçero libro delas politicas tanne tres razons } por las quales paresçe que se puede prouar \\\hline
3.2.4 & quasi omnino a communi bono . \textbf{ Peius est igitur principari unum , } quam plures . & o apartasse much del bien comun . \textbf{ Et pues que assi es meior cosa es de prinçipar muchs que vno¶ } La terçera razon se toma dela firmeza \\\hline
3.2.4 & decet enim Principem \textbf{ esse regulam rectam et stabilem , ut per iram et concupiscentias } et per alias passiones non corrumpatur nec peruertatur . & por que conuiene \textbf{ que el prinçipe sea regla derecha e firme et estable | assi que por ira } nin por cobdiçia \\\hline
3.2.4 & assignans rationes multas , \textbf{ quod melius sit dominari multitudinem : } postea in eodem 3 tangit quaedam , & e poniendo muchͣs razones para esto \textbf{ que meior es que much | senssennore en que vno . } Despues en esse mismo terçero tanne algunas cosas \\\hline
3.2.4 & cum ipse pluries dicat in eisdem politicis , \textbf{ regnum esse dignissimum principatum : } inter principatus enim rectos , & en esse mismo libro delas politicas \textbf{ que el regno es prinçipado muy digno } por que entre los prinçipados derechs el prinçipado de vno \\\hline
3.2.4 & non est dignius , \textbf{ quam dominari unum ; } cum nunquam plures recte dominari possint , & derech non es mas digna cosa nin meior \textbf{ que si enssennoreas se vno . } ca nunca pueden much \\\hline
3.2.4 & Censendum est igitur , \textbf{ regnum esse dignissimum principatum , } et secundum rectum dominium melius est dominari unum , & Et pues̃ que assi es deuemos otorgar \textbf{ que el regno es prinçipado muy digno } e segut derech \\\hline
3.2.4 & regnum esse dignissimum principatum , \textbf{ et secundum rectum dominium melius est dominari unum , } quam plures . & que el regno es prinçipado muy digno \textbf{ e segut derech | sennorio meior es } que sea vn sennor que muchos . \\\hline
3.2.4 & Non ergo dici poterit \textbf{ talem unum monarchiam non cognoscere multa ; } quia quantum spectat & Et pues que assi es non se puede dezer \textbf{ que vn tal monarchia | o tal prinçipe assi fech̃ de muchos que non conogca } e non sepa muchͣs cosas . \\\hline
3.2.4 & et bonos quos sibi associauit , \textbf{ contingeret esse peruersos : } talis enim maxime intendit commune bonum ; & e todos los buenos omes \textbf{ que assi ouo aconpannado fuessen tristornados e corronpidos } por que tales prinçipalmente entienden el bien comun . \\\hline
3.2.5 & An melius sit regiam dignitatem \textbf{ ire per electionem , } an per haereditatem : & si es meior de ser la dignidat real \textbf{ por elecçion } que ꝑ non por heredamiento paresçe \\\hline
3.2.5 & videtur tale regnum non esse expositum casui et fortunae , \textbf{ sed factum esse per artem , } eo quod praeficietur melior et industrior . & de si nin auentura \textbf{ mas es fecho por arte e por sabiduria por el que meior e el mas sabio sera puesto en el sennorio } mas si esto fuere por heredat es pone se el regno \\\hline
3.2.5 & Absolute ergo loquendo , \textbf{ melius est Principem praestituendum esse per electionem , } quam per haereditatem . & Et por ende paresça e a alguons que fablando sueltamente meiores \textbf{ que el prinçipe sea establesçido | por elecçion } que por heredat . \\\hline
3.2.5 & quanto credit ipsum regnum \textbf{ magis esse bonum suum et bonum proprium : } quare si Rex videat & que el regno es mas su bien \textbf{ e mas su bien propo | que de otro ninguno } Por la qual cosa si el Rey viere \\\hline
3.2.5 & magis reputabit bonum regni \textbf{ esse bonum suum , } et ardentius solicitabitur & mas avn por heredat en sus fijos . \textbf{ mas terna que el bien del regno es su bien propreo } e con mayor \\\hline
3.2.5 & Innuit enim hoc esse \textbf{ quasi virtutis diuinae , et excedere humanum modum . } Sed forte ideo hoc Philosophus dixit , & ca dize que esto es assi commo por uirtud de dios \textbf{ e que sobrepiua la manera | humanabeas por auentura esto } por ende lo dize el philosofo \\\hline
3.2.5 & sicut haereditarie principantes : \textbf{ et quia hoc est esse tyrannum , } non intendere bonum regni , & conmo aquellos que enssennorean por h̃edamiento \textbf{ e vn por que esta es condiçion de thirano } non tener mientes en el bien del regno \\\hline
3.2.5 & per haereditatem transferatur ad posteros , \textbf{ oportet eam transferre in filios , } quia secundum lineam consanguinitatis filii parentibus maxime sunt coniuncti : & por hedamiento conuiene alos pueblos \textbf{ que tomne alos fijos } ca segunt el linage del patente \\\hline
3.2.5 & ut pater ampliori solicitudine curet de bono regni , \textbf{ sciens ipsum peruenire ad filium plus dilectum . } Et si dicatur quod contingit aliquando magis diligere minores . & por que el padre con mayor acuçia aya cuydado del bien del regno \textbf{ sabiendo | que el regno parte nesçe al su fiio mas amado } Et si dixiere alguno \\\hline
3.2.5 & sciens ipsum peruenire ad filium plus dilectum . \textbf{ Et si dicatur quod contingit aliquando magis diligere minores . } Talibus obiectionibus de facili respondetur : & que el regno parte nesçe al su fiio mas amado \textbf{ Et si dixiere alguno | que contesçe algunas uezes } que los padres mas aman alos menores \\\hline
3.2.6 & nimis ardenter mouetur in eorum amorem , \textbf{ et optat eos habere in dominos . } Inde est quod antiquitus plures sic praeficiebantur in Reges . & e bien fechores mueuense con grant ardor alos amar \textbf{ e dessean de los auer | por sennores } e por ende antiguamente los mas de los sennores fueron tomados en Reyes . \\\hline
3.2.6 & Quare expedit regem \textbf{ habere praedictos tres excessus . } Nam si abundet in beneficiis tribuendis , & que el rey aya aquellas tres aun ataias \textbf{ e aquellas tres condiçiones buenas sobredichͣs . } ca si abondare en bien fazer seria muy amado del pueblo \\\hline
3.2.6 & tendere in bonum , \textbf{ eius erit magis tendere in maius bonum , } bonum ergo gentis et commune & ca si la uirtud parte nesçede se estender a mayor bien \textbf{ e en mayor bien dela gente es el bien comun } que es mas diuinal \\\hline
3.2.6 & in ciuili potentia , ut possit corrigere volentes insurgere , \textbf{ et turbare pacem regni . } Viso in quibus Rex alios debet excedere : & por que pueda castigar los que se quisieren le una tar \textbf{ contra la paz del regno } Visto en quales cosas el Rey deue sobrepuiar \\\hline
3.2.7 & Quadruplici via venari possumus , \textbf{ tyrannidem esse pessimum principatum . } Prima sumitur ex eo quod tale dominium maxime recedit & or quatro razones podemos prouar \textbf{ que la thirama es muy mal prinçipado | ¶La primera se toma } por razon que tal sennorio mucho se arriedra dela entençion del bien comun . \\\hline
3.2.7 & in eodem 4 Politicorum ubi ait , \textbf{ tyrannidem esse pessimum principatum , } quia nullus liberorum voluntarie sustinet principatum talem . & en el quarto libro delas politicas \textbf{ do dize que la tirania es muy mal prinçipado } por que ninguno de los omes francos e libres non sufre \\\hline
3.2.7 & et concordiam adinuicem : \textbf{ rursus nolunt eos esse magnanimos et virtuosos : } nec etiam volunt ipsos esse sapientes et disciplinatos . & que los çibdadanos ayan paz nin concordia entre ssi . \textbf{ Otrossi non quiere | que ellos se que de grandes coraçones e uirtuosos } e avn non quiere \\\hline
3.2.7 & rursus nolunt eos esse magnanimos et virtuosos : \textbf{ nec etiam volunt ipsos esse sapientes et disciplinatos . } Quare autem tyranni praedicta bona & que ellos se que de grandes coraçones e uirtuosos \textbf{ e avn non quiere | que los çibdadanos sean sabios e entendidos . } Et la razon por que los tiranos enbargan estos bienes sobredichos en las çibdades ayuso se dira . \\\hline
3.2.7 & Sufficiat autem ad praesens scire , \textbf{ tyrannidem esse pessimum principatum } propter rationes tactas . & Et cunpla agora de saber \textbf{ que la tirania es muy mal prinçipado } por las razones sobredichͣs . \\\hline
3.2.8 & ut possit \textbf{ consequi finem intentum . } Secundo , ut remoueantur prohibentia et deuiantia & Lo primero que en tal manera sea el pueblo apareiado e ordenado por que pue da alcançar su fin que entiende . \textbf{ Lo segundo conuiene que sean arredradas todas aquellas cosas } que enbargan de alcançar aquella fin \\\hline
3.2.8 & organice deseruiunt res exteriores . \textbf{ Decet ergo Reges et Principes sic regere ciuitates et regna , } ut sibi subiecti abundent rebus exterioribus & assi commo son las riquezas e los algos . \textbf{ Et por ende conuiene alos Reyes | e alos prinçipes de gouernar } assi las çibdades e los regnos \\\hline
3.2.8 & quasi enim nihil esset \textbf{ vitare interiora discrimina , } nisi prohibentur exteriora pericula . & por que non seria nada escusar los males \textbf{ de dentro del alma | e los pecados } si non fuessen tirados \\\hline
3.2.9 & et regni redditus studeat \textbf{ expendere in bonum commune , } vel in bonum regni : & que las rentas del regno se pongan \textbf{ enł bien comun } e en el bien del regno \\\hline
3.2.9 & Tertio decet Regem , \textbf{ et Principem non ostendere se nimis terribilem et seuerum , } nec decet se nimis familiarem exhibere , & nin los derechos del regno . \textbf{ ¶ Lo terçeto conuiene al Rey et al prinçipe de non mostrarsse muy espantable nin muy cruel . } nin le conuiene otrosi de se fazer muy familiar alos omnes \\\hline
3.2.9 & tyrannus autem non est , \textbf{ sed esse se simulat . } Quarto spectat ad Regem , & Mas el tirano non es tal \textbf{ mas quiere pare sçertal . } lo quarto parte nesçe a \\\hline
3.2.9 & per quos bonus status regni conseruari potest , \textbf{ sed etiam ut ait Philosophus in Polit’ inducere debent uxores proprias } ut sint familiares et beniuolae uxoribus praedictorum : & por los quales se puede guardar el buen estado del regno . \textbf{ Mas avn assi commo dize el philosofo | en el terçero libro delas politicas deue enduziras Ꝯmugres propraas } por que sean familiares e bien querençiosas alas mugers \\\hline
3.2.9 & ut seditiones mouerent in regno aut principatu ; \textbf{ sic ergo gerere se debet } bonus rector regni aut ciuitatis . & e enl prinçipado \textbf{ Et pues que assi es . } assi se deue auer buen Rey e buen gouernador deregas e de çibdat \\\hline
3.2.9 & omnino est subiectus Regi \textbf{ quem credit esse deicolam , } et habere amicum Deum : & Ca el pueblo segunt que dize el philosofo es del todo subiecto al Rey \textbf{ quando vor que es honrrador de dios } e que ha a dios por amigo \\\hline
3.2.9 & quem credit esse deicolam , \textbf{ et habere amicum Deum : } existimat enim talem semper iuste agere , & quando vor que es honrrador de dios \textbf{ e que ha a dios por amigo } ca sienpre cuyda \\\hline
3.2.10 & Vident enim se contra dictamen rectae rationis agere , \textbf{ et non intendere bonum commune sed proprium : } ideo vellent omnes suos subditos & contra razon derech̃tu eyendo \textbf{ que ellos non entienden enl bien comun | mas en el su bien propreo . } Por ende querrien que todos los sus subditos fuessen sin sabiduria e nesçios \\\hline
3.2.10 & ideo vellent omnes suos subditos \textbf{ esse ignorantes et inscios , } ne cognoscentes eorum nequitiam , & mas en el su bien propreo . \textbf{ Por ende querrien que todos los sus subditos fuessen sin sabiduria e nesçios } por qua non conosçiessen \\\hline
3.2.10 & et eos qui sunt in regno \textbf{ non esse sodales , } nec esse ad inuicem notos : & que los çibdadanos que son en el regno \textbf{ non sean conpannones nin amigos } nin sean conosçidos vnos con otros . \\\hline
3.2.10 & non esse sodales , \textbf{ nec esse ad inuicem notos : } nam ( ut ait Philosophus ) & non sean conpannones nin amigos \textbf{ nin sean conosçidos vnos con otros . } Ca assi commo dize el pho la conosçençia faze fe . \\\hline
3.2.10 & Verus autem Rex econtrario permittit sodalitates ciuium , \textbf{ et vult ciues sibi inuicem esse notos , } et de se confidere ; & ca consiente todas las conpannias \textbf{ e quiere que los çibdada nos sean conosçidos vnos con otros } e que fien vnos de otros . \\\hline
3.2.10 & ut diligatur ab eis : \textbf{ quare vult eos esse confoederatos et coniunctos , } quia tunc magis unanimiter diligunt bonum Regis . & cosaes que sea amado dellos . \textbf{ Por la qual cosa quiere | que los çibdadanos sean amigos et ayuntados } por conpannias por que estonçe los çibdadanos aman \\\hline
3.2.10 & Omnino enim esset peruersus populus , \textbf{ si cognosceret se habere verum Regem , } et diligere commune bonum , & Ca en todo en todo sia malo el pueblo \textbf{ si conosçiesse | que auia buen Rey e uerdadero } e que amaua el bien comun \\\hline
3.2.10 & si cognosceret se habere verum Regem , \textbf{ et diligere commune bonum , } si viceuersa non diligeret ipsum Regem . & que auia buen Rey e uerdadero \textbf{ e que amaua el bien comun } si el esso mismo non amasse mucho al Rey \\\hline
3.2.10 & volunt habere exploratores multos , \textbf{ ut si viderent aliquos ex populo machinari aliquid contra eos , } possint obuiare illis . & por que en muchͣs cosas le aguauian quieren auer muchs assechadores \textbf{ por que si vieren | que alguon ssele una tan contra ellos } que los puedan contradezer ante \\\hline
3.2.10 & Immo eo ipso quod ciues credunt \textbf{ tyrannum habere exploratores multos , } non audent congregari & que los çibdadanos creen \textbf{ que ay muchos assechadores } non se osan ayuntar \\\hline
3.2.10 & Utrum autem deceat Reges \textbf{ habere exploratores in regno propter aliam causam , } quam ne populus insurgat in ipsum , & Mas si conuiene al Rey auer assechadores en el regno \textbf{ por otra razon que por la que dichͣes . } que el pueblo non se leunate contra el adelante se dira . \\\hline
3.2.10 & sed magis procurat eorum bona . \textbf{ Octaua , est procurare bella , } mittere bellatores ad partes extraneas , & Mas ha cuydado de acrescentar sus bienes \textbf{ ¶ La . viij n . cautela del tirano } es procurar guerras e enbiar \\\hline
3.2.11 & Secundo ex industria et sagacitate , \textbf{ ut quia credit se tot adinuenire vias et versutias posse , } ut valeat tyrannum perimere . & La segunda paresçe que salle de omne de grant coraçon . \textbf{ por que el que se leunata contra el tyra | no es de tan grant coraçon } que non tiene por grant cosa de \\\hline
3.2.12 & sed pecuniam , \textbf{ intendere corporales delectationes . } Tertio diuites sic principantes & e non entienden en el bien comun \textbf{ mas en las riquezas entienden en las delectaçonnes corporales } ¶ \\\hline
3.2.12 & quantum ad uxores et filias . \textbf{ Videt ergo se esse odiosum populo , } ideo non credit se multitudini , & e faze much stuertos alos çibdadanos e enlas mugres e en las fijnas . \textbf{ Et por ende veyendo se aborresçido del pueblo } non fia dela muchedunbre de los çibdadanos \\\hline
3.2.12 & Priuatur ergo tyrannus a maxima delectatione , \textbf{ cum videat se esse populis odiosum . } Viso tyrannidem cauendam esse , & e por ende el tirano es pri uado de grant delectaçion \textbf{ quando bee | que es aborresçido delos pueblos } Disto que la tirauja es de esquiuar e de aborresçer \\\hline
3.2.13 & quod nimis fugans timidum , \textbf{ vi compellit esse audacem . } Sic etiam et alia animalia & que quien muncho faze foyr al temeroso \textbf{ por fuerça lo costrange desee oscido en essa misma manera } avn en las otras ainalias \\\hline
3.2.13 & quidam enim nomine Dion videns ipsum \textbf{ quasi semper esse ebrium , } propter despectionem insurrexit in ipsum . & que auja nonbre dion viendol \textbf{ que sienpre estaua enbriago | por despechon quel auje } e despreçiandol \\\hline
3.2.13 & et non honorare subditos , \textbf{ et non quaerere commune bonum , } volentes adipisci honorem & e non quiere honrrar los subditos \textbf{ njn quiere el bien comun } quariendo algunos alcançar la gloriar la honrra \\\hline
3.2.13 & et perimunt ipsum . \textbf{ Sic etiam quia multi reputant pecuniam esse maximum bonum , } videntes tyrarannum non intendere & Ca essa misma manera avri \textbf{ por que munchos cuda | que el auer es muy grand bien veyendo } que el tirano non entiende \\\hline
3.2.13 & sed ut videantur \textbf{ facere actiones aliquas singulares . } Volunt enim aliqui esse in aliquo nomine & por que ayan el su señorio mas por que paresca alos omes \textbf{ que fazen algunos omes apartadas } ca algunos quieren ser en alguna nonbrada o en alguna fama \\\hline
3.2.13 & Sexto contingit aliquos insidiari tyrannis \textbf{ et perimere ipsos , } ut liberent patriam & que al gunos a echa alos tiranos \textbf{ e los matan } por que libren la trrͣa dela grad \\\hline
3.2.14 & eo quod esset habet multipliciter : \textbf{ contingit enim uno modo percutere signum , } propter quod in hoc non est diuersitas nec contrarietas : & por que ha de ser en muchͣs maneras . \textbf{ ca contesçe que en vna manera tiran ala sennal derechomente } por esso en tal cosa commo esta non ha contrariedat nin departimiento . \\\hline
3.2.14 & Decet ergo regiam maiestatem \textbf{ summo studio cauere tyrannidem , } ne praedictis periculis exponatur . & Et pues que assi es conuiene ala Real magestad de escusar con grant estudio \textbf{ e con grant acuçia la tirama } por que non se pongan alos peligros sobredichos \\\hline
3.2.15 & quae politiam saluant , \textbf{ et quae oportet facere Regem ad hoc } ut se in suo principatu praeseruet . & e el gouernamiento del regno \textbf{ e dela çibdat | las quals conuiene al Rey de fazer } para que se pueda man tener en lu prinçipado e en lu lennorio ¶ \\\hline
3.2.15 & ut se in suo principatu praeseruet . \textbf{ Primo est , non permittere in suo regno transgressiones modicas . } Nam multae modicae transgressiones & para que se pueda man tener en lu prinçipado e en lu lennorio ¶ \textbf{ La primera es que non consienta en su regno muchos pequanos males } ca muchs pequannos males \\\hline
3.2.15 & quod expendunt , et quomodo possunt \textbf{ reddere rationem sui victus : } nam qui huiusmodi rationem non potest reddere , & et commo pueden dar razon de su uida \textbf{ e de comm̃ se mantienen } ca aquel que non puede dar razon desto señal \\\hline
3.2.15 & Nonum maxime saluans regnum , \textbf{ est esse regem bonum et virtuosum . } Nam ut dicitur 5 Politicorum , & La ixͣ cosa que much salua el regno es \textbf{ que el rey sea bueno e uirtuoso . } ca assi commo dize el philosofo \\\hline
3.2.15 & et epiikis idest super iustus : \textbf{ decet enim talem esse quasi semideum , } ut sicut alios dignitate et potentia excellit , & ca conuiene \textbf{ que el tal que sea | assi commo dios } assi que commo lieua auna taia de los otros en dignidat e en poderio \\\hline
3.2.16 & et qui illorum peruersi , \textbf{ et declarauimus regnum esse optimum principatum , } et tyrannidem pessimum ; & e quales tuertos . \textbf{ Et declaramos en commo el regno era muy buen prinçipado } e la tirania muy malo . \\\hline
3.2.16 & nec propter nostra opera \textbf{ immutari possunt eorum cursus , } ideo circa talia non est consilium adhibendum . & e non se pueden mudar los sus mouimientos \textbf{ por las nuestras obras } por ende nos deuemos tomar conseio \\\hline
3.2.16 & Ideo dicitur in Ethic’ \textbf{ non esse consilium de his , } quae sunt a fortuna , & e por ende dize el philosofo enl terçero libro delas ethins \textbf{ que non ay consero de aquellas cosas } que son auentura \\\hline
3.2.16 & oportet enim in consilio \textbf{ praesupponere finaliter intentum , } et non consiliari de ipso , & ca el nuestro consseio non es dela fu . \textbf{ por que conuiene que en el conseio sorongamos la fin } e que non tomemos consseio della \\\hline
3.2.17 & et circa naturas rerum , \textbf{ et circa aeterna fieri quaestiones multae , } sed huiusmodi quaestiones consilia & e en las sçiençias delas naturas delas cosas \textbf{ e enlas sçiençias delas cosas perdurables . } Mas tales quastions \\\hline
3.2.17 & quanto pluribus modis fieri potest \textbf{ et quanto minus habet certas et determinatas vias , } tanto per plus tempus est consiliandum , & por mas maneras se puede fazer . \textbf{ Et quanto menos ha çiertas | e determinadas carreras } para se fazer tanto mayor tienpo ha menester omne \\\hline
3.2.17 & ut quae sunt apta nata \textbf{ efficere paruum bonum , } vel prohibere modicum malum , & Et por ende las cosas que son muy pequan ans assi que pueden acarrear muy \textbf{ pequano mal o enbargar } pequano bien non son de poner en consseio . \\\hline
3.2.17 & quidam poeta nomine Alexander videns \textbf{ Priamum in consiliis esse secretarium et veracem , commendans eum dicebat , } Iste est qui consuluit . & que auie nonbre alixandre veyendo \textbf{ que primero era muy guardado enlos conseios | e muy uerdadero } alabandolo dize del este es aquel que conseia \\\hline
3.2.18 & Sed ad hoc quod aliquis sit bene creditiuus , \textbf{ non oportet ipsum esse existenter talem , } sed sufficit quod videatur & mas para que alguno sea bien de creer \textbf{ non conuiene | que el sea tal fechmas cunple } que parezca tal cael o en iudga las cosas que paresçen de fuera por las cosas que vee \\\hline
3.2.18 & ut credamus eos plus valere quam valeant , \textbf{ et esse meliores quam sint . } Quare si auditores credunt & que valen mas de quanto ualen \textbf{ e que son meiores de quanto son . . } por la qual cosa si los oydores creen alos bien querençiosos \\\hline
3.2.18 & nam reddere se credibilem \textbf{ et bene persuadere per se , } est ex ipsis rebus , & Ca fazerse el omne digno de creer \textbf{ e buen amonestador e razonador por si . } nasçe de aqual las cosas \\\hline
3.2.19 & circa haec ergo quinque oportet \textbf{ consiliatores esse instructos . } Primo enim contingit esse Regis consilium circa prouentus , & Et en estas çinco cosas pueden ser enformados \textbf{ e enssenados los conseieros e los sabidores dellas . } Lo primero conuiene que el conseio del Rey \\\hline
3.2.19 & consiliatores esse instructos . \textbf{ Primo enim contingit esse Regis consilium circa prouentus , } in quo duo sunt attendenda . & e enssenados los conseieros e los sabidores dellas . \textbf{ Lo primero conuiene que el conseio del Rey | sea cerca las sus rentas } en la qual cosa dos cosas conuiene de penssar \\\hline
3.2.19 & probabatur enim supra , \textbf{ Regem debere esse talem , } quod esset bonus virtuosus & ca prouado es de suso \textbf{ que el Rey deue ser } tal que sea bueno e uirtuoso \\\hline
3.2.19 & existentes in regno promoueret et honoraret : \textbf{ quod esse non posset , si bona eorum quae sunt in regno usurparet iniuste . } Rursus est attendendum , & et ꝓmueua los q̃ son eñl su regno e los hõ rre . \textbf{ la qual cosa non podria ser } si tomasse los bienes \\\hline
3.2.19 & apponatur et augeatur . \textbf{ Secundo debet esse consilium de alimento , } ut sciatur utrum ciuitas vel regnum & e sean acresçentadas las sus rentas ¶ \textbf{ Lo segundo deueser tomado consseio en fech delas uiandas | e enla mantenençia de los omes . } por que sea sabido \\\hline
3.2.19 & ut in ciuitate contingit \textbf{ esse vicos aliquos } magis esse suspectos quam alios : & assi commo contesçe \textbf{ que en la çibdat son alguon suarrios mas sospethosos que los otros . } por que los mal fechores se acostunbraron de esconder se \\\hline
3.2.19 & esse vicos aliquos \textbf{ magis esse suspectos quam alios : } quia iniustificantes ibidem possunt magis latere , & assi commo contesçe \textbf{ que en la çibdat son alguon suarrios mas sospethosos que los otros . } por que los mal fechores se acostunbraron de esconder se \\\hline
3.2.19 & per se est malum , et fugiendum . \textbf{ Deinde , si visum sit bellum esse iustum , } consideranda est potentia regni , vel ciuitatis , & e es muchͣ de escusar \textbf{ despues si fuere iusto | que la guerra es derechͣ } deue ser penssado el poderio del regno o dela çibdat \\\hline
3.2.20 & Nam in qualibet ciuitate oporteret \textbf{ esse aliquod praetorium ordinarium } ad quod causae reducantur & ca en cada vna çibdat conuiene \textbf{ que aya vna alcalłia otdinaria } ala qual deuen venir todos los pleitos \\\hline
3.2.20 & quod maxime quidem contingit \textbf{ recte positas leges , } quaecunque possibile est determinare : & que mucho conuiene \textbf{ que las leyes | que son derechamente puestas determinen } quanto pueden ser todas las cosas \\\hline
3.2.20 & sufficienter enim iudex excusatur , \textbf{ cum secundum positas leges aliquid iudicat ; } quia non videtur ipse & e muy pocas cosas son de dexar en aluedrio de los mueze ᷤ \textbf{ quando iudga alguna cosa | segunt las leyes puestas } ca non paresçe \\\hline
3.2.21 & in iudicio prohibeantur : \textbf{ multi enim litigantium cognoscentes se habere malam causam , } non narrant quid factum et quid non factum , & assi comm̃ayra a abortençia sean defendidas en łmyzio \textbf{ ca muchos de los que contienden en iuyzio | sabiendo que tienen mal pleito } non cuentan lo que es fecho \\\hline
3.2.21 & infecto aliquo humore , recte iudicat , \textbf{ dicens amarum esse amarum , } et dulce dulce . & por algun humor iudga derechͣmente diziendo \textbf{ que lo amargo es amargo } e lo dulçe es dulçe . \\\hline
3.2.21 & peruersae iudicat , \textbf{ dicens dulce esse amarum , } et econuerso , & algundelas partes contrarias iudga mal diziendo \textbf{ que lo dulçe es amargo } e lo amargo es dulçe . \\\hline
3.2.21 & quasi si inconueniens est \textbf{ permittere obliquari regulam , } inconueniens est sustinere & cosasi non es cosa conueible \textbf{ que la regla se tuerca } non es cosa conuenible de sofrir \\\hline
3.2.22 & faciunt iudicium usurpatum . \textbf{ Tunc quidem dicuntur iudices non recte se habere ad legislatorem , } quando excedunt auctoritatem sibi commissam . & fazen que el iuyzio sea fortado . \textbf{ Et entonçe los iuezes non se han derechamente al fazedor dela ley } quando sobressallen dela auctoridat \\\hline
3.2.22 & Secundo dicuntur iudices \textbf{ facere iudicium temerarium , } si non recte se habent ad leges , & que les es a comne dada . \textbf{ Lo segundo los iuezes fazen iuyzio loco } quando non se han derechamente \\\hline
3.2.23 & ad quae decet \textbf{ respicere iudicem , } ut humanis indulgeat , & en el primero libro \textbf{ uiene que tenga el iuez sienpre mientes } para que perdone alas obras de los omes \\\hline
3.2.23 & quod iudicans debet \textbf{ aspicere non ad actionem , } sed ad electionem . & que el que iudga non deue tener \textbf{ mientesa la obra mas ala entençion . } Lo quinto que enduze el iuez ami bicordia \\\hline
3.2.23 & debet ergo iudex non ita respicere ad partem \textbf{ ut ad hoc particulare negocium in quo delinquunt , } sicut ad totum & Et por ende eliez non deue \textbf{ assi catara vna obra particular } en que peco commo a todos los bienes \\\hline
3.2.23 & Sextum est diuturnitas temporis retroacti . \textbf{ Nam contingit etiam in pauco tempore facere multa bona opera : } duo ergo debent inducere Regem & Lo sexto que inclina ali es a piedat es alongamiento \textbf{ detpo passado por que contesçe que alas uezes alguno en poco tp̃o faze muchas buenas obras . } Et por ende dos cosas deuen endozir al Rey o al prinçipe \\\hline
3.2.23 & Patet ergo quomodo decet \textbf{ iudices esse magis clementes quam seueros : } et si hoc decet iudices , & en qual manera conuiene \textbf{ que los miezes sean mas piadosos que crueles } Et si esto conuiene alos iuezes mucho \\\hline
3.2.24 & quae sunt naturae . \textbf{ Quare si ius naturale dictat fures et maleficos esse puniendos , } hoc praesupponens ius positiuum procedit ulterius , & que son dela natura \textbf{ Por la qual cosa si el derecho natural manda | que los ladrones } e los mas fechores sean castigados \\\hline
3.2.25 & ut conuenimus cum animalibus aliis : \textbf{ sic dicitur esse ius naturale . } Ideo in Instituta , & siguiere la nuestra natura en quanto auemos conueniençia con las otras aian lias \textbf{ assi es dich derech natural . } Et por ende en la instituta del derecho natural \\\hline
3.2.25 & ut communicamus cum animalibus aliis , \textbf{ respectu iuris gentium dicitur esse naturale . } Nam si considerentur dicta in praecedenti capitulo , & en quanto participamos con las otras aianlas \textbf{ en conparacion del derecho delas gentes es dicho derecho natural . } Ca si penssaremos los dichos del capitulo \\\hline
3.2.25 & in societate viuere \textbf{ secundum debitas conuentiones et pacta ; } sic erit de iure naturali , & dessea beuir en conpannia \textbf{ segunt | establesçimientos e posturas cs̃uenbles } assi seran de derecho natural \\\hline
3.2.25 & quod ius consequens naturam nostram \textbf{ prout appetimus esse et bonum , } est naturale respectu iuris animalium , & que el derecho que ligue lanr̃a natura en quanto desseamos ser \textbf{ e desseamos bien es natural en conparaçion del derech delas aianlias } o en conparaçion del derech \\\hline
3.2.25 & Si ut conuenit cum animalibus aliis , \textbf{ sic habet esse ius illud , } quod natura omnia animalia docuit . & Mas en quanto conuiene el omne con todas las otras \textbf{ ai alias | assi se toma aquel derecho } que la natura demostro \\\hline
3.2.25 & cum omnibus entibus , \textbf{ sic habet esse ius illud , } quod per antonomasiam dicitur esse naturale . & con todas las sustançias \textbf{ assi se toma aquel derech } que es dicho natural pora un ataia de los otros derechos . \\\hline
3.2.25 & sic habet esse ius illud , \textbf{ quod per antonomasiam dicitur esse naturale . } Appetere enim esse et bonum , & assi se toma aquel derech \textbf{ que es dicho natural pora un ataia de los otros derechos . } Por que dessear el bien e el ser \\\hline
3.2.26 & ad quem debet applicari \textbf{ et debet regulari per huiusmodi legem , } oportet quod sit competens & a que es dada \textbf{ el qual pueblo deueser reglado por aquella ley . } conuiene que sea conuenible \\\hline
3.2.26 & quod non oportet \textbf{ adaptare politias legibus , } sed leges politiae , & que non conuiene de apropar las comunidades \textbf{ delas çibdades alas leyes . } Mas las leyes alas comunidades \\\hline
3.2.28 & Ostendimus in praecedentibus capitulis , \textbf{ quales debent esse leges condendae } a Regibus et Principibus & a demostramos enlos capitulos sobredichos quales deuen ser las leyes \textbf{ que son de poner } por los Reyes \\\hline
3.2.28 & continere huiusmodi leges . \textbf{ Dicuntur autem quinque esse effectus legum , } vel quinque esse opera legalia , & e quales e quantas obras deuen contener estas leyes \textbf{ e conuiene de sabra | que çinco son los fechos } o las obras delas leyes \\\hline
3.2.28 & Dicuntur autem quinque esse effectus legum , \textbf{ vel quinque esse opera legalia , } videlicet praecipere , permittere , prohibere , praemiare , et punire . & que çinco son los fechos \textbf{ o las obras delas leyes | que son estas . } Mandar ¶ Conssentir . \\\hline
3.2.28 & quae in ea traduntur , \textbf{ vult regulare et aequare humanos humores : } sic scientia politica & ø \\\hline
3.2.28 & vult aequare \textbf{ et regulare actiones humanas , } ut ciues iuste viuant , & que le contienen en aquella sçiençia \textbf{ por que los çibdadanos biuna derechamente } e saayan conmose deuen auer . \\\hline
3.2.29 & quia est aliquid pertinens ad rationem , \textbf{ videtur dicere intellectum solum : } ideo dicitur 3 Polit’ & que parte nesçe \textbf{ a razon paresçe | que diga entendimiento solo . } Et por ende dize el pho en el terçero delas politicas \\\hline
3.2.29 & His ergo rationibus videtur ostendi , \textbf{ melius esse regnum et ciuitatem Regi lege , } quam Rege . & Et por esta razon e paresçe ser mostrado \textbf{ que meior es que el regno o la çibdat se gouernada } por ley que por Rey . \\\hline
3.2.29 & oportet Regem in regendo alios \textbf{ sequi rectam rationem , } et per consequens sequi naturalem legem , & Et assi se sigue \textbf{ que sigua la ley natural | la qual se leunata de razon derecha e de entendumento derecho . } Et por ende el rey en gouernando es a \\\hline
3.2.29 & sequi rectam rationem , \textbf{ et per consequens sequi naturalem legem , } quia in tantum recte regit , & la qual se leunata de razon derecha e de entendumento derecho . \textbf{ Et por ende el rey en gouernando es a | quande dela ley natural } por que en tanto gouierna derechamente \\\hline
3.2.29 & in mente cuiuslibet hominis , \textbf{ dirigere legem positiuam , } et esse supra iustitiam legalem , & la qual dios puso en voluntad de cada vn omne \textbf{ que enderesçe la ley positiua } e que sea sobre la iustiçia legal \\\hline
3.2.29 & dirigere legem positiuam , \textbf{ et esse supra iustitiam legalem , } et non obseruare legem , & que enderesçe la ley positiua \textbf{ e que sea sobre la iustiçia legal } e qua non guarde la ley positiua \\\hline
3.2.29 & et esse supra iustitiam legalem , \textbf{ et non obseruare legem , } ubi non est obseruanda . & e que sea sobre la iustiçia legal \textbf{ e qua non guarde la ley positiua } do non la deue guardar . \\\hline
3.2.29 & quae sit applicabilis humanis actibus . \textbf{ Oportet igitur aliquando legem plicare ad partem unam , } et agere mitius cum delinquente , & e allegar alas obras delos omes . \textbf{ Et por ende conuiene quela ley que se ençorue } e se allegue algunas vezes ala vna parte e que obre mas manssamente con el que peca \\\hline
3.2.29 & quam lex dictat : \textbf{ aliquando etiam oportet eam plicare ad partem oppositam , } et rigidius punire peccantem , & quela ley demanda o que la ley nidga . \textbf{ Et algunas vezes conuiene que la regla se encorue | ala parte contraria } e que mas reziamente de pena \\\hline
3.2.29 & aliquando etiam oportet eam plicare ad partem oppositam , \textbf{ et rigidius punire peccantem , } quam lex determinet . & ala parte contraria \textbf{ e que mas reziamente de pena } al que peca que la ley demandan que determina . \\\hline
3.2.30 & expediens \textbf{ dare legem euangelicam et diuinam , } triplici via possumus venari & ø \\\hline
3.2.30 & Oportuit igitur praeter legem humanam \textbf{ dari aliquam legem , } ut nullum malum remaneret impunitum , & Et por ende conuiene que sin la ley humanal fuesse \textbf{ dada otra ley diuinal | e e un agłica l . } por que ningun mal non fincasse sin pena \\\hline
3.2.30 & sed quantum ad punitionem , \textbf{ sic dicitur non prohibere mentem et animum , } eo quod talia delicta non puniat . & humanal non quanto ala entençion del ponedor della \textbf{ mas quanto ala pena que pone assi digo que non defiende la uoluntad e el coraçon . } por que non condep̃na tales pecados \\\hline
3.2.30 & ut vitentur adulteria . \textbf{ Secunda via ostendens necessariam esse legem euangelicam et diuinam , } sumitur ex parte cognitionis nostrae , & por que sean escusados los adulterios . \textbf{ ¶ La segunda razon que muestra la ley en angelical } e diuinal ser neçessaria es tomada deꝑte del nuestro conosçimiento \\\hline
3.2.31 & ut inuenientes consuetudines nouas , \textbf{ dicentes eas esse utiles et proficuas ciuitati , } soluerent leges patrias et antiquas . & para fallar costunbres nueuas \textbf{ diziendo que aquellas eran prouechosas ala çibdat . } Et en esto desfazien e destruyen las leyes antiguas dela tr̃ra . \\\hline
3.2.31 & contingit \textbf{ esse malas et barbaricas : } sicut erat lex olim apud Graecos , & Ca contesçe que algunas leyes dela tr̃ra \textbf{ assi commo dize el philosofo son deseguales e malas e barbaricas } assi commo fue aquella ley que era establesçida entre los gniegos . \\\hline
3.2.31 & ut ciues possent uxores suas vendere . \textbf{ Sic etiam contingit leges aliquas esse stultas , } utputa legem illam quam & por las quales los çibdadanos pudiessen vender so mugers . \textbf{ assi avn contesçe que algunas leyes son locas } assi commo aquella ley \\\hline
3.2.31 & Nam si aliquando condentes leges contingit \textbf{ esse simplices , } irrationale esset , & que los fazedores delas leyes son sinples \textbf{ e de poco saber . } Et por ende cosa sin razon seria \\\hline
3.2.31 & et per consequens est \textbf{ tollere principatum et regnum . } Quantum autem malum sequitur & e alos prinçipes dela qual cosa se signirie \textbf{ que se tiraria el prinçipado e el regno . } Mas quanto mal se se sigue \\\hline
3.2.32 & Nam regnum supra ciuitatem videtur \textbf{ addere multitudinem nobilium et ingenuorum . } Est enim ciuitas pars regni ; & Ca el regno eñade sobre la çibdat muchedunbre \textbf{ de nobles omes e de alto linage . } por que la çibdat es parte del regno . \\\hline
3.2.32 & decet nobiles et ingenuos \textbf{ esse magis bonos et virtuosos } quam ciues alios : & que los nobles e los altos \textbf{ e los mas fijos dalgo sean mas buenos | e mas uirtuosos } que los otros çibdadanos . \\\hline
3.2.32 & propter quod regem ipsum tanquam omnibus excellentiorem \textbf{ decet esse optimum , } et quasi semideum . & assi commo aquel que sobrepula todos los otros en dignidat \textbf{ e en pero de rio sea muy bueno } e sea assi commo medio dios . \\\hline
3.2.32 & in ciuitate et regno , \textbf{ oportet esse talem , } quod viuat bene et virtuose . & que es en el regno e enla çibdat . \textbf{ conuiene que sea atal que biuna bien e uirtuosamente . } Et por ende assi conmo dize el philosofo en el terçero libro delas politicas \\\hline
3.2.33 & quod tres oportet \textbf{ esse partes ciuitatis . } Nam alii quidem sunt opulenti valde , & uenta el philosofo en el quarto libro delas politicas \textbf{ que conuiene que sean tres partes dela çibdat . } Ca alguons son muy ricos . \\\hline
3.2.33 & ostendere \textbf{ optimam esse ciuitatem et regnum , } si ibi sit populus & Mas la entençion deste capitulo es mostrar \textbf{ que es muy buena la çibdat e el regno } si y fuere pueblo establesçido de muchͣs perssonas medianeras \\\hline
3.2.33 & ex quibus sumi possunt quatuor viae , \textbf{ ostendentes meliorem esse politiam , } vel melius esse regnum et ciuitatem , & delas quales se pueden tomar quatro razonnes \textbf{ que muestran que meior es la poliçia } o meior es el regno o la çibdat \\\hline
3.2.33 & ostendentes meliorem esse politiam , \textbf{ vel melius esse regnum et ciuitatem , } si ibi sit populus abundans & que muestran que meior es la poliçia \textbf{ o meior es el regno o la çibdat } si y fuere pueblo \\\hline
3.2.33 & videns se ei quasi aequalem existere , \textbf{ et non esse magnum excessum inter ipsos . } Huic auctoritati attestatur , & ueyendo que es su egual e veyendo \textbf{ qua non ay grant auna taia entre el vno e el otro . } Et a esta uerdat da testimo \\\hline
3.2.33 & ex personis mediis . \textbf{ Decet ergo Reges et Principes adhibere cautelas , } ut in regno suo abundent multae personae mediae ; & establesçidas de perssonas medianeras . \textbf{ Et pues que assi es conuiene | que los reyes e los prinçipes ayan cautelas e sabidurias . } por que en el su regno sean muchͣs perssonas medianeras \\\hline
3.2.34 & quanto decentius est \textbf{ eos esse bonos , et virtuosos . } Secunda via ad inuestigandum hoc idem , & quanto mas conuenible es \textbf{ que ellos sean bueons e uirtuosos . } la segunda razon para prouar esto mesmo se tomadesto \\\hline
3.2.34 & Credunt enim aliqui , \textbf{ quod obseruare leges , } et obedire regi , & Ca assi commo dize el philosofo \textbf{ en el primero libro de la rectorica enlas leyes es salud dela çibdat . | Et maguer algunos cuyden que guardan las leyes } e obedesçer el Rey lea algunasiudunbre . \\\hline
3.2.34 & sic pessimum est regno \textbf{ deserere leges regias } et praecepta legalia , & assi es muy mala cosa \textbf{ que el regno desanpare las leyes } e los mandamientos reales \\\hline
3.2.34 & et praecepta legalia , \textbf{ et non regi per Regem . } Tertia via ad ostendendum hoc idem , & e los mandamientos reales \textbf{ e que non se gouierne por el Rey ¶ } La terçera razon para mostrar esto mismo se \\\hline
3.2.36 & Nam maxime prouocatur populus ad odium Regis , \textbf{ si viderit ipsum non obseruare iustitiam : } Ideo dicitur 2 Rhet’ & Ca el pueblo mayormente se le una taria a mal querençia del Rey \textbf{ si viesse | que el non guardaua nistiçia . } Et por ende el philosofo \\\hline
3.2.36 & quod homines timent eos , \textbf{ de quibus sunt conscii fecisse aliquid dirum . } Secundo timentur Reges & que los omes temen a aquellos de que son sabidores \textbf{ que fizien es alguna cosa muy cruel ¶ } Lo segundo son temidos los Reyes \\\hline
3.2.36 & magis punire , \textbf{ et seuerius se gerere contra amicos , } si contingat eos valde forefacere , & que mayor penaden \textbf{ e mas cruelmente se ayan contra los amigos } quando mal fizieren \\\hline
3.2.36 & per quod ciues sunt magis boni et virtuosi , \textbf{ debet esse magis intentum a legislatore . } Cum ergo ciues et existentes in regno & e mas uirtuo los deue ser \textbf{ prinçipalmente quarido e entendido del ponedor dela ley . } Et pues que assi es quando los çibdadanos \\\hline
3.2.36 & quiescant male agere : \textbf{ oportuit ergo aliquos inducere ad bonum , } et retrahere a malo timore poenae . & Por la qual cosa conuiene \textbf{ que alguon s | enduxiessemosa bien } e arredrassemos del mal \\\hline
3.3.1 & et ad quid sit instituta . \textbf{ Sciendum igitur militiam esse quandam prudentiam , } siue quandam speciem prudentiae . & Et pues que assi es deuedes saber \textbf{ que la caualleria es vna prudençia o vna manera de sabiduria . . | Mas podemos quanto pertenesçe a lo presente } departir çinco maneras de prudençia e de sabiduria . \\\hline
3.3.1 & per quam quis scit regere domum et familiam , \textbf{ oportet esse aliam a prudentia , } qua quis nouit seipsum regere . & por la qual cada vno sabe gouernar la casa e la conpaña . \textbf{ Conuiene que sea otra e departida de la sabiduria } por la qual cada vno sabe gouernar a ssi mismo . \\\hline
3.3.1 & et gubernare ciues . \textbf{ Omnes autem tres prudentias decet habere Regem , } videlicet particularem , oeconomicam et regnatiuam . & e en quanto ha de poner leyes e gouernar los çibdadanos . \textbf{ Et todas estas tres sabidurias | conuiene que aya el Rey . } Conuiene a saber . \\\hline
3.3.1 & Quare cum commune bonum directe videatur \textbf{ impediri per impugnationem hostium , } ex consequenti vero & derechamente parezca de ser enbargado \textbf{ por la guerra de los enemigos } e de si por la turbacion \\\hline
3.3.1 & Hanc autem prudentiam videlicet militarem , \textbf{ maxime decet habere Regem . } Nam licet executio bellorum , et remouere impedimenta ipsius communis boni , & Et esta sabiduria de caualleria \textbf{ mas pertenesçe al rey que a otro ninguno . | Ca commo quier que pertenezca a los caualleros } la essecuçion de las batallas \\\hline
3.3.1 & ad dignitatem militarem , \textbf{ nisi constet ipsum diligere bonum regni et commune , } et nisi spes habeatur & para dignidat de caualleria \textbf{ si non fueren çiertos | que el ama el bien del regno e el bien comun } e si non ouieren esperança \\\hline
3.3.1 & remouere quaecunque \textbf{ impedire possunt commune bonum . } Ex hoc etiam patere potest & quales se quier cosas \textbf{ que enbarguen el bien comun . } Et desto puede parescer \\\hline
3.3.2 & quia non sine magna audacia contingit \textbf{ aliquos inuadere apros . } Sunt ergo tales animosi et strenui ad bellandum . & sin grant osadia acometer los puercos monteses \textbf{ e les otros fuertes venados . } Et por ende tales son de grant coraçon \\\hline
3.3.2 & Nam non timentes aprorum pericula , \textbf{ signum est eos non timere hostium bella . } Rursus venatores ceruorum non sunt repudiandi & e de las otras bestias fuertes . \textbf{ señal es que non | temerien lasbatallas de los enemigos . } Otrossi los caçadores de los çieruos non son de refusar \\\hline
3.3.3 & quasi fortuito videtur \textbf{ peruenire ad palmam , } si caret industria bellandi . & Ca si quier sea cauallero si quier peon el que ha de lidiar paresçe \textbf{ que por uentura alcaça uictoria } si non ouiere sabiduria de lidiar \\\hline
3.3.3 & Sciendum igitur viros audaces et cordatos \textbf{ utiliores esse ad bellum , } quam timidos . & que los omnes osados eatreuidos \textbf{ e de grandes coraçones son mas prouechosos para la batalla } que los temerosos e de flacos coraçones . \\\hline
3.3.3 & Sed e contrario dicimus , \textbf{ duri carne habentes compactos neruos , } et lacertos , & son sotiles de coraçon . \textbf{ Et por el contrario los que han las carnes duras } e los neruios espessos e firmes \\\hline
3.3.3 & et latitudo pectoris . \textbf{ Videmus enim leones animalium fortissimos habere magna brachia , } et latum pectus . & son grandeza de los mienbros e anchura de los pechos . \textbf{ Ca veemos que los leones | que son mas fuertes que todas las otras animalas } por que han grandes braços e anchos pechos . \\\hline
3.3.3 & debemus arguere \textbf{ ipsum esse bellicosum , } et aptum ad pugnam . & estos tales deuemos iudgar \textbf{ por lidiadores e apareiados para la batalla . } Et pues que assi es tales son de escoger los lidiadores \\\hline
3.3.3 & quia ut plurimum contingit \textbf{ eos esse aptos ad actiones bellicas . } Quantum ad praesens spectat , & que por la mayor parte sean apareiados \textbf{ para las obras de la batalla . |  } q quanto pertenesçe a lo presente ocho cosas podemos contar \\\hline
3.3.4 & Quarto non curare de incommoditate iacendi et standi . \textbf{ Quinto quasi non appretiare corporalem vitam . } Sexto non horrere sanguinis effusionem . & nin de mal estar \textbf{ Lo quinto | que por razon de la iustiçia e del bien comun despreçien la uida corporal . } Lo sexto que non teman \\\hline
3.3.4 & Sexto non horrere sanguinis effusionem . \textbf{ Septimo habere aptitudinem , } et industriam ad protegendum se et feriendum alios . & nin aborrezcan de derramar su \textbf{ sangreLo septimo conuiene | que ayan buena disposiçion e buena sabidura } para defender assi \\\hline
3.3.4 & ad defensionem exercitus , \textbf{ deferre in abundantia victualium copiam : } immo et si adesset pugnantibus ciborum ubertas , & para defendimieto de la hueste \textbf{ que lo pongan | e que lo lieuen en muchedunbre de uiandas } Ca ante deuen dexar el peso de las talegas \\\hline
3.3.5 & Numquam credo potuisse dubitari \textbf{ aptiorem armis esse rusticam plebem . } Ad hoc etiam videntur facere & Creo que ninguno nunca pudo dubdar \textbf{ que los omnes rusticos e aldeanos non fuessen meiores para las armas | que los que son delicadamente criados . } Et a esto fazen avn aquellas cosas \\\hline
3.3.5 & Hos etiam probabile est \textbf{ non multum timere mortem . } Nam tanto quis magis mortem timere videtur , & Et avn asas es cosa prouada \textbf{ que estos poco temen la muerte } ca tanto mas teme cada vno la muerte \\\hline
3.3.5 & Hi etiam non videntur \textbf{ horrere effusionem sanguinis . } Nam inter ceteras gentes & Et avn paresçe \textbf{ que estos non aboresçen derramamiento de sangre . } Ca entre todas las gentes \\\hline
3.3.5 & Ad haec igitur intendentibus videtur \textbf{ censendum esse meliores bellatores esse rurales . } Sunt autem alia , & e los villanos son meiores \textbf{ para las batallas | que los nobles nin los fijosdalgo . } Mas ay otras cosas \\\hline
3.3.5 & velle honorari ex pugna , \textbf{ et erubescere turpem fugam . } Hoc est enim & e tomar uerguença de foyr \textbf{ torpemente | assi commo dicho es dessuso . } Et esto es segunt \\\hline
3.3.5 & quam rusticis , \textbf{ ii meliores esse videntur ad pugnam , } eo quod verecundentur fugere . & Porende meiores son los nobles \textbf{ e los | fijosdalgo para las batallas } que los villanos e los aldeanos \\\hline
3.3.6 & Viso armorum exercitium \textbf{ esse perutile ad opera bellica , } restat ostendere & Visto en qual manera el uso de las armas es muy prouechoso \textbf{ para las obras de la batalla } finca de demostrar \\\hline
3.3.6 & Et cum viderit magister bellorum \textbf{ aliquem non tenere ordinem debitum in acie , } ipsum increpet et corrigat : & Et quando vieren los caudiellos maestros de las batallas \textbf{ que alguno non guarda orden en la az } deuenle denostar e castigar \\\hline
3.3.6 & ut sint habiles in praecurrendo . \textbf{ Videtur enim hoc valere ad tria . } Primo ad explorandum inimicorum facta . & quando venieren a la fazienda . \textbf{ Ca paresçe que esto les vale atres cosas } Lo primero para assechar e ascuchar el estado de los enemigos . \\\hline
3.3.7 & postquam per magnam partem \textbf{ dici exercitati essent ad arma , } si tempus erat natationi congruum , & que auian de ser lidiadores \textbf{ despues que por vna grant parte del dia eran usados en las armas } si tienpo era conuenible para nadar \\\hline
3.3.8 & quam circa ipsa in aliquo neglexisse . \textbf{ Nam recitat Vegetius dixisse Catonem , } quod in aliis rebus & Ca en ninguna guisa non deuen ser negligentes en ellas . \textbf{ Onde dize Uegeçio | que gaton el sabio dixo } que en las otras cosas \\\hline
3.3.8 & quod , exercitu absque fossis et castris existente , \textbf{ et non credentes hostes esse propinquos , } superuenientibus hostibus fugit exercitus debellatus . & e sin castiellos o otros defendimientos non cuydando \textbf{ que sus enemigos estan çerca vienen a desora los } e es vençida la hueste . \\\hline
3.3.8 & Nam contingit aliquando situm \textbf{ illum non pati talem formam . } In tali ergo casu construenda sunt castra semicircularia , & Ca algunas uegadas \textbf{ contesçe que el assentamiento non sufre tal figura . } Et por ende en tal caso deuen se fazer los castiellos \\\hline
3.3.10 & si careant centurione et duce , \textbf{ qui debet esse eorum caput et eorum directiuum . } Inde est quod antiquitus & e \textbf{ mayorales que sean cabeças dellos e guiadores en la hueste . } Et por ende antiguamente \\\hline
3.3.10 & et ducem militaris belli \textbf{ esse habilem corpore , } ut possit etiam armatus agiliter equum conscendere : & que el que es antepuesto es cabdiello de la caualleria en la batalla \textbf{ que sea ligero en el cuerpo } por que pueda avn que sea armado sobir ligeramente en el cauallo \\\hline
3.3.11 & per viam aliquam \textbf{ in qua pati possit insidias , } nisi qualitates viarum , montes , flumina , & nin de las çeladas \textbf{ si el cabdiello de la batalla non ouiere escriptas . } o pintadas las qualidades de los caminos . \\\hline
3.3.11 & ut simul cum hoc quod habet vias \textbf{ et qualitates viarum conscriptas et depictas , } ducat dux belli conductores aliquos & que deue auer las carreras \textbf{ e las qualidades de los caminos escriptas e pintadas . } avn ayan otros guiadores \\\hline
3.3.11 & Tertia est , \textbf{ habere secum plures sapientes fideles principi , exercitatos in bellis , } de quorum consilio agat & La terçera cautela es \textbf{ que el cabdiello aya consigo muchos sabios | e fieles al prinçipe } e vsados en las batallas \\\hline
3.3.11 & et vias illas Dux habet \textbf{ conscriptas et depictas , } et habentur conductores aliqui fideles , & por quales caminos deue yr la hueste \textbf{ e aquellas carreras touiere el cabdiello escriptas e pintadas } e ouiere algunos omes guiadores fieles \\\hline
3.3.12 & Nam pugnantes vel solum volunt se defendere \textbf{ et sustinere ictus , } vel volunt alios inuadere . & ø \\\hline
3.3.14 & Secundo debet diligenter \textbf{ explorare eorum itinera , } ut ad transitus fluuiorum , & Lo segundo deue escudriñar \textbf{ con grand acuçia los caminos dellos } assi commo el passo de los rios \\\hline
3.3.14 & quando hostes magnam fecerunt dietam , \textbf{ sunt fatigati habent laxatos equos : } tunc enim , & quando los enemigos fizieren grant iornada \textbf{ e touieren los cauallos canssados . } Ca estonçe si los quisieren acometer \\\hline
3.3.15 & hostes \textbf{ insequi fugientes a bello , } plures occiderent ; & ca podrie contesçer \textbf{ que los enemigos persiguirien a los que fuyessen de la batalla } e matarian muchos dellos \\\hline
3.3.16 & in tanta multitudine esse , \textbf{ et tantam habere potentiam , } ut non expectant hostes exire ad campum , & que algunos lidiadores son en tan grand muchedunbre \textbf{ e de tan grant poder } que no esperan \\\hline
3.3.16 & et tantam habere potentiam , \textbf{ ut non expectant hostes exire ad campum , } sed ipsas munitiones obsideant et inuadant : & e de tan grant poder \textbf{ que no esperan | que salgan los enemigos al canpo . } Mas ellos acometen las villas \\\hline
3.3.16 & sed ipsas munitiones obsideant et inuadant : \textbf{ sic contingit aliquos esse adeo paucos et tam debiles , } ut non putent in campo & e los castiellos o las fortalezas e los çercan . \textbf{ assi contesçe que alguno lidiadores son tan pocos et tan flacos } que non cuydan \\\hline
3.3.16 & Contingit etiam aliquando aliquos \textbf{ inuadere aliquas munitiones eorum ; } propter quod eos oportet & Et algunas vezes çerca villas o castiellos o fortalezas . \textbf{ Et avn algunas vezes contesçe que algunos otros çercan sus villas o sus castiellos . } Por la qual cosa les conuiene de vsar de batalla defenssiua para se defender . \\\hline
3.3.16 & vel per aliquam industriam possint \textbf{ ab obsessis accipere aquam . Nam multotiens euenit , } aquam a remoto principio deriuari & o por alguna sotileza puedan tomar el agua de los cercados . \textbf{ Ca muchas uegadas contesçe } que el agua viene de lueñe fasta las fortalezas cercadas . \\\hline
3.3.16 & per quam pergit aqua ad obsessos , \textbf{ oportebit ipsos pati aquarum penuriam . } Rursus , aliquando munitiones sunt altae , & por do viene el agua a los cercados han de auer los cercados \textbf{ por fuerça mengua de agua . } Otrossi algunas uegadas las fortalezas son altas \\\hline
3.3.16 & debent obsidentes adhibere omnem diligentiam , \textbf{ quomodo possint obsessis prohibere aquam . } Secundus modus impugnandi munitiones , & los que cercan deuen auer grant acuçia \textbf{ en commo defiendan el agua a los cercado o gela tiren . } La segunda manera para ganar las fortalezas es por fanbre . \\\hline
3.3.16 & volentes citius opprimere munitiones , \textbf{ si contingat eos capere aliquos de obsessis , } non occidunt illos & mas ayna ganar las fortalezas \textbf{ si contezca | que prendan algunos de los cercados } non los matan \\\hline
3.3.17 & non est possibile obsidentes \textbf{ semper esse paratos aeque . } Ideo nisi sint muniti , & que los que cercan \textbf{ sienpre esten apareiados de vna guisa | nin de vna manera . } Et por ende si non estudieren guarnesçidos puede les contesçer \\\hline
3.3.17 & Praeter tamen hos modos impugnationis apertos , \textbf{ est dare triplicem impugnationis modum non omnibus notum . } Quorum unus est per cuniculos . & Enpero sin estas maneras manifiestas de batalla \textbf{ ay otras tres maneras | que non son manifiestas a todos } de las quales la vna es \\\hline
3.3.18 & munitiones aliquas obsessas \textbf{ super lapides fortissimos esse constructas , } vel esse aquis circumdatas , & e assi podran ganar aquellas fortalezas . \textbf{ m muchas uegadas contesçe que algunas fortalezas çercadas son fundadas sobre pennas muy fuertes } o son cercadas de agua \\\hline
3.3.18 & super lapides fortissimos esse constructas , \textbf{ vel esse aquis circumdatas , } vel habere profundissimas foueas , & m muchas uegadas contesçe que algunas fortalezas çercadas son fundadas sobre pennas muy fuertes \textbf{ o son cercadas de agua } o han carcauas muy fondas \\\hline
3.3.18 & vel esse aquis circumdatas , \textbf{ vel habere profundissimas foueas , } vel aliquo alio modo esse munitas : & o son cercadas de agua \textbf{ o han carcauas muy fondas } o son \\\hline
3.3.18 & vel habere profundissimas foueas , \textbf{ vel aliquo alio modo esse munitas : } ut per viculos & o han carcauas muy fondas \textbf{ o son | enfortalezidas en alguna otra manera } assi que por cueuas conegeras \\\hline
3.3.19 & ideo appellatur aries , \textbf{ quia ratione ferri ibi appositi durissimam habet } frontem ad percutiendum . & Ca por razon del fierro \textbf{ que ponen y . | ha muy fuerte et muy . } dura fruente para ferir \\\hline
3.3.20 & volumus de bello defensiuo : \textbf{ ut postquam docuimus obsidentes qualiter debeant inuadere obsessos , } volumus docere ipsos obsessos qualiter & que es para se defender los çercados . \textbf{ assi que despues que ensseñamos | en qual manera } los que çercan deuen acometer los \\\hline
3.3.21 & et non possit \textbf{ habere aquam nisi santam , } eo quod dulcem aquam habeat distantem , & e non podieren auer \textbf{ si non agua salada } por que el agua dulçe ba muy lueñe \\\hline
3.3.21 & et etiam aedificia necessaria munitioni fieri possint . \textbf{ Per ferra vero etiam reparari possint arma , } et fieri tela ; & cadahalsos \textbf{ los que fezieren menester en la fortaleza . | Et del fierro puedan las armas } e fazer fierros de dardos et de saetas \\\hline
3.3.22 & et tunc propter duritiem lapidum \textbf{ non est facile per cuniculos debellare eam , } vel est supra petram & por la dureza de la peña \textbf{ que se non puede } cauaro esta la fortaleza assentada sobre peña blanda \\\hline
3.3.22 & statim debent viam aliam subterraneam \textbf{ facere correspondentem illis cuniculis , } ita tamen quod via illa pendeat & ninguno deuen fazer otras cueuas soterrañas \textbf{ que respondan a aquellas cueuas coneieras | que vengan derechamente contra ellas . } Enpero assi lo deuen fazer \\\hline
3.3.22 & restat videre quomodo obsessi debeant \textbf{ obuiare impugnationi factae per lapidarias machinas . } Contra has autem quadrupliciter subuenitur . & Visto en qual manera auemos de contrallar a la batalla fecha \textbf{ por los engenios que lançan las piedras . } Et podemos dar contra los engeñios quatro maneras de acorro . \\\hline
3.3.22 & trabem ferratam percutientem muros munitionis obsessae \textbf{ propter duritiem capitis vocari Arietem . } Contra hoc autem constituitur & para ferir en los muros de la fortaleza \textbf{ por la dureza de la cabeça es llamado carnero } et contra este \\\hline
3.3.22 & et clam suffoditur terra \textbf{ unde debet transire castrum ; } qua suffossa , et castro demerso in ipsam propter magnitudinem ponderis , & e ascondidamente se puede cauar la tierra \textbf{ por que puedan passar } allende del castiello o de la villa çercada . \\\hline
3.3.22 & Quod si tamen contingeret \textbf{ per huiusmodi aedificia perforari muros munitionis obsessae : } cum de hoc dubitatur , & Enpero si por auentura contesçiere \textbf{ que por estos artifiçios fueren foradados los muros de la fortaleza çercada . } quando desto dubdaren ante que los muros sean foradados \\\hline
3.3.22 & aedificentur muri lapidei : \textbf{ ut si continget obsidentes intrare munitionem , } retineantur clausi inter muros illos ; & o si pueden deuen fazer muros de piedra \textbf{ assi que si los que çercan entraren de dentro de la fortaleza sean retenidos e ençerrados entre aquellos muros } assi que se non puedan defender \\\hline
3.3.23 & simul bellis naualibus , \textbf{ et exponere se periculis , } ne puppis per rimas naufragium patiatur . & a la batalla de las naues \textbf{ e al periglo de las aberturas . } por las quales la naua puede peresçer . \\\hline
3.3.23 & Immo in huiusmodi pugna oportet \textbf{ homines melius esse armatos , quam in terrestri : } quia cum pugnatores marini quasi fixi stent in naui , & Ante conuiene que en esta batalla de la \textbf{ naue sean los omnes . | meior armados que en la de la tierra } por que los lidiadores de la mar esten firmes \\\hline
3.3.23 & in tali bello vident \textbf{ sibi imminere mortem : } quare si oculi bellantium & los que estan en las naues \textbf{ non veen donde les viene la muerte . } Por la qual cosa si los oios de los que lidian en la \\\hline
3.3.23 & diu valentem durare sub aquis ; \textbf{ qui accepto penetrali sub aquis debet accedere ad hostilem nauem , } et eam in profundo perforare , & que puedan y mucho estar e lieuen taladros para foradar \textbf{ e llegunen se a la naue de los enemigos so el agua } e foraden la pordiuso . \\\hline
3.3.23 & et ex cupiditate eorum \textbf{ ordinari ad lucrum , } vel ad aliquam aliam satifactionem irae , vel concupiscentiae . & o nasçen de cobdiçia \textbf{ e la ordena algunan ganançia } o la ordena a alguna otra vengança de saña o de cobdiçia mundanal . \\\hline
3.3.23 & sic in conuersatione , et societate hominum est \textbf{ dare plures personas , et plures homines . } Et sicut quamdiu humores sunt aequati in corpore , & e en la conpañia de los omnes \textbf{ ay muchas perssonas e muchos omens } e assi commo mientra los humores son ygualados en el cuerpo del omne \\\hline
3.3.23 & et eorum qui sunt in regno . \textbf{ Supposito ergo Reges et Principes habere iustum bellum , } et hostes eorum iniuste perturbare pacem & e de todos los que son en el regno . \textbf{ Et pues que assi es puesto } que los Reyes e los prinçipes ayan batalla derecha \\\hline
3.3.23 & et commune bonum : \textbf{ non est inconueniens docere eos omnia genera bellandi , } et omnem modum & Et pues que assi es puesto \textbf{ que los Reyes e los prinçipes ayan batalla derecha } et los sus enemigos turben la paz \\\hline

\end{tabular}
