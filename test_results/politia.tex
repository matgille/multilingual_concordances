\begin{tabular}{|p{1cm}|p{6.5cm}|p{6.5cm}|}

\hline
3.2.1 & praemittendo quaedam praeambula ad propositum , et recitando opinionem diuersorum Philosophorum instituentium politiam , et tradentium artem de regimine ciuitatis et regni : & ante pomiendo alguons preanbulos al nuestro proponimiento e rezando opiniones de departidos philosofos que establesçieron poliçias | e rezando opiniones de departidos philosofos que establesçieron poliçias e dieron arte del gouernamiento dela çibdar \\\hline
3.2.2 & Nam regnum aristocratia , et politia sunt principatus boni : tyrannides , & ca el regno e la aristo carçia que quiere dezer sennorio de buenos e la poliçia que quiere dezer pueblo bien enssenoreante son bueons prinçipados . | que quiere dezer sennorio de buenos e la poliçia que quiere dezer pueblo bien enssenoreante son bueons prinçipados . La thirama que quiere dezer sennorio malo \\\hline
3.2.2 & et quia talis principatus non habet nomen proprium , vocat eum Philosophus nomine communi , et dicit ipsum esse Politiam . Politia enim quasi idem est , quod ordinatio ciuitatis & llamalle el philosofo nonbre comun e diz el poliçia por que poliçia es | e diz el poliçia por que poliçia es assi commo ordenamiento bueno de çibdat \\\hline
3.2.2 & quantum ad maximum principatum qui dominantur omnibus aliis . Politia enim consistit maxime in ordine summi principatus , qui est in ciuitate . & e prinçipalmente quanto al grant prinçipado que enssennorea a todos los otros ca la poliçia esta mayormente en el ordenamiento del grant prinçipado que es en la çibdat \\\hline
3.2.2 & Omnis ergo ordinatio , ciuitatis Politia dici potest . Principatus tamen populi & que es en la çibdat Pues que assi es todo ordenamiento de çibdat puede ser dicħ poliçia . Enpero el prinçipado del pueblo \\\hline
3.2.2 & si rectus sit , eo quod non habeat commune nomen , Politia dicitur . Nos autem talem principatum appellare possumus gubernationem populi , si rectus sit . & si derecho es por que non ha nonbre comun es dich poliçia e nos podemos llamar atal prinçipado gouernamiento del pueblo | por que non ha nonbre comun es dich poliçia e nos podemos llamar atal prinçipado gouernamiento del pueblo si derecho es . \\\hline
3.2.3 & et qui peruersi . Quia intendimus ostendere quae sit optima politia , et quis sit optimus principatus . & e quales malos ir que entendemos mostrar qual es la muy buena poliçia | ir que entendemos mostrar qual es la muy buena poliçia e que cosa es el muy buen prinçipado despues \\\hline
3.2.3 & quae est in pluribus principantibus , congregaretur in uno Principe , efficacior esset ; et ille principans propter abundantiorem potentiam melius posset politiam gubernare . Tertia via sumitur ex his quae videmus in natura . & Et aquel prinçipe por ma . yor cunplimiento de poderio meior podria gouernar la çibdat que muchos ¶ | por ma . yor cunplimiento de poderio meior podria gouernar la çibdat que muchos ¶ La terçera razon se toma de aquellas cosas \\\hline
3.2.5 & ( ut ait Philosophus in Politiis ) decet iuniores senioribus obedire . Immo & que alos otros segunt que dize el pho en las politicas conuiene | segunt que dize el pho en las politicas conuiene que los mas mançebos obedescan alos mas uieios e avn \\\hline
3.2.7 & ubi ait , quod sicut Regnum est optima et dignissima politia , sic tyrannis est pessima : & Et esta razon tanne el philosofo çerca el comienço del quarto libro delas politicas . do dize que assi conmo el regno es muy buena et muy digna poliçia . | do dize que assi conmo el regno es muy buena et muy digna poliçia . assi la tirania es muy mala \\\hline
3.2.7 & ut ibi dicitur ) quia tyrannis plurimum distat a politia , idest a communi bono . Secunda via ad inuestigandum hoc idem , sumitur ex eo quod tale dominium maxime est naturale . & y dize el pho por quela tirama much se arriedra dela poliçia e del bien comun . la segunda manera \\\hline
3.2.14 & praeparatur via ut corrumpatur principatus ille . Politia ergo quanto de se magis a iustitia recedit , tanto ex se habet & pues que assi es quanto el gouernamiento mas se arriedra dela iustiçia | quanto el gouernamiento mas se arriedra dela iustiçia tanto ha dessi que sea mas ayna corronpida \\\hline
3.2.15 & Tangit autem Philosophus 5 Polit’ decem quae politiam saluant , et quae oportet facere Regem & anne el pho en el quànto libro delas politicas diez cosas que saluna la poliçia e el gouernamiento del regno | que saluna la poliçia e el gouernamiento del regno e dela çibdat \\\hline
3.2.15 & etiam modicae . Secundum praeseruans politiam et regnum regium , est bene uti iis & La segunda cosa que guarda la poliçia e el gouernamiento del regno \\\hline
3.2.15 & et non iniuriando eis . Nam ut innuit Philosophus in Poli’ bene uti ciuibus non solum praeseruat politiam rectam , sed etiam principatus ex hoc durabilior redditur , & Ca assi commo dize el philosofo en el primero libro delas pol . bien vsar de los çibdadanos non solamente guarda la poliçia e el gouernamiento derecho . | en el primero libro delas pol . bien vsar de los çibdadanos non solamente guarda la poliçia e el gouernamiento derecho . Mas avn por esta razon el prinçipado se faze mas durable \\\hline
3.2.15 & dato quod in ipso sit aliquid obliquitatis ad mixtum . Tertium est , incutere timorem iis qui sunt in politia : nam corruptiones longe & La terçera cosa que guarda al gouernamiento del regno es meter mie do aquellos que son enla çibdat e en el regno ca las corrupconnes alongadas de fecho e allegadas \\\hline
3.2.15 & prope autem secundum timorem politiam saluant : ciues enim magis sunt subiecti Principi et plus ei obediunt , & e saluna lo poliçia e el gouernamiento dela çibdat ca los çibdadanos son mas subiectos al prinçipe | e el gouernamiento dela çibdat ca los çibdadanos son mas subiectos al prinçipe e meior le obedesçen \\\hline
3.2.15 & et antecessores sui obtinuerunt huiusmodi principatum , tanta cautela non magnam utilitatem habere videtur . Quartum autem quod politiam saluare videtur , est cauere seditiones & que dichͣes non paresçe que puede ser muy prouechosa la quarta cosa que salua la poliçia | que puede ser muy prouechosa la quarta cosa que salua la poliçia es escusar las discordias e las contiendas delos nobles \\\hline
3.2.15 & aut magistratum . Nihil enim adeo regnum conseruat et politiam saluat , sicut praeficere homines bonos & tanto el regno e la poliçia commo poner los bueons e los uirtuosos en las dignidades \\\hline
3.2.15 & et conferre eis dominia et principatus . Quare maxime saluatiuum politiae est , regiam maiestatem considerare diligenter quos praeficit in aliquibus magistratibus : & e dar les los señorios e los prinçipados . por la qual cosalo que mucho salua la poliçia | por la qual cosalo que mucho salua la poliçia es que el Rey piensse con grant acuçia \\\hline
3.2.15 & ne repente constituatur aliquis in maximo principatu . Septimum saluans regnum et politiam , est Regem siue principantem habere dilectionem & La vi jncosa que salua el regno e la poliçia es | que salua el regno e la poliçia es que el Rey e el prinçipe aya grant amor al bien del regno \\\hline
3.2.15 & et amorem ad bonum regni , et ad politiam , in qua principatur . & que el Rey e el prinçipe aya grant amor al bien del regno e al bien dela çibdat en quien enssennorea \\\hline
3.2.15 & et periculis imminentibus obuiare . Octauum saluans regnum et politiam , est habere ciuilem potentiam . & e pueda contradezir alos peligros que pueden acaesçer¶ La . viijn . cosa que salua el regno | que pueden acaesçer¶ La . viijn . cosa que salua el regno e la poliçia es auer poderio çiuilca \\\hline
3.2.15 & sic eos bonitate superet : hoc enim maxime saluabit regnum et politiam , si Rex sit bonus et virtuosus , & assi lieue aun ataia en bondat e esto es lo que much salua el regno e la poliçia si el Rey fuere bueno e uirtuoso \\\hline
3.2.15 & quia intendet bono regni et communi . Decimum , est Regem non ignorare qualis sit illa politia secundum quam principatur , & La xͣ cosa que salua la poliçia es que el Rey sepa aquella poliçia e gouernamiento | que salua la poliçia es que el Rey sepa aquella poliçia e gouernamiento segunt el qual el enssennorea \\\hline
3.2.19 & quod esset bonus virtuosus et politiam diligeret , existentes in regno promoueret et honoraret : quod esse non posset , si bona eorum quae sunt in regno usurparet iniuste . Rursus est attendendum , & la qual cosa non podria ser si tomasse los bienes de aquellos que son en el su regno | si tomasse los bienes de aquellos que son en el su regno sin derech \\\hline
3.2.26 & et moribus illius gentis . Ideo dicitur 4 Politicorum quod non oportet adaptare politias legibus , sed leges politiae , quas leges oportet diuersas esse secundum diuersitatem politiarum . & que non conuiene de apropar las comunidades delas çibdades alas leyes . Mas las leyes alas comunidades de las çibdades las quales leyes conuiene de ser departidas | Mas las leyes alas comunidades de las çibdades las quales leyes conuiene de ser departidas segunt el departimiento delas comunidades . \\\hline
3.2.26 & quod non oportet adaptare politias legibus , sed leges politiae , quas leges oportet diuersas esse secundum diuersitatem politiarum . Volens ergo leges ferre , & las quales leyes conuiene de ser departidas segunt el departimiento delas comunidades . Et pues que assi es el que quisiere poner leyes con grand acuçia \\\hline
3.2.29 & Adducit autem rationes duas , quod melius sit politiam regni Regi optima lege , quam optimo Rege . & Mas para esto prouar aduze dos razones que meior es de ser gouernado el regno por muy buena ley | que meior es de ser gouernado el regno por muy buena ley que por muy buen Rey ¶ \\\hline
3.2.33 & si ibi sit populus ex multis personis mediis constitutus . Tangit autem Philosophus 4 Politicorum , quatuor , ex quibus sumi possunt quatuor viae , ostendentes meliorem esse politiam , vel melius esse regnum et ciuitatem , & Et pone el pho en el quarto libro delas politicas quatro cosas delas quales se pueden tomar quatro razonnes que muestran que meior es la poliçia | delas quales se pueden tomar quatro razonnes que muestran que meior es la poliçia o meior es el regno o la çibdat \\\hline
3.2.33 & et Charondas , et Licurgus tradiderunt de Politiis . Dixerunt enim eas constituendas esse ex personis mediis . Decet ergo Reges et Principes adhibere cautelas , & non lo que dixieron sałon e carendas e liguago los quales phos dixieron que las çibdades deuien ser establesçidas de perssonas medianeras . | e liguago los quales phos dixieron que las çibdades deuien ser establesçidas de perssonas medianeras . Et pues que assi es conuiene \\\hline
3.2.34 & intentio legislatoris est inducere ciues ad virtutem . In recta enim Politia ( ut vult Philosophus ) & es enduzer los çibdadanos o uirtud . Ca en la derecha poliçia assi commo dize el philosofo çerca el comienço del quarto libro delas polticas \\\hline
3.2.35 & et obseruantiam legum utilium : et ad obseruandum ea quae requirit politia , vel regimen regni , & e aguardar aquellas cosas que demanda la poliçia o el gouernamiento del regno en que estan \\\hline
3.2.36 & qui ultra modum regnum et politiam per turbant , inexquisitas crudelitates exerceant . & Et la manera en que la dan . Et pueᷤ que assi es por la pena que dan son temidos los Reyes e los prinçipes si mostraren grandescrueldades \\\hline

\end{tabular}
