\begin{tabular}{|p{1cm}|p{6.5cm}|p{6.5cm}|}

\hline
1.1.2 & en que deue la Real magestado el rrey & Nam Primo ostendetur in quo regia maiestas debeat suum finem , \\\hline
1.1.9 & assi commo demanda el su estado ¶ & ut exigit status suus . \\\hline
1.1.11 & segunt su estado & tunc ( ut exigit suus status ) \\\hline
1.1.11 & Et dezimos segunt que requiere su estado & tunc ( ut exigit suus status ) \\\hline
1.1.13 & Mas el estado del Rey demanda que sea mas acordable et mas acordable & Principis autem status requirit , \\\hline
1.1.13 & Ca muchos esta non tal estado & multi enim non existentes in statu quo possint mala facere , \\\hline
1.1.13 & Enpero si a mayor estado fuese leunata dos aurian razon & quod si tamen ad statum dignitatis assumerentur , \\\hline
1.1.13 & que el prinçipado e el estado muestraqles el omne & quod principatus virum ostendit . Tunc enim apparet qualis homo sit , cum in principatu existens , in quo potest bene et male facere , \\\hline
1.2.1 & e qual parte nesçe a su estado & nam tunc Reges habent felicitatem suo statui debitam , \\\hline
1.2.1 & e el su estado & et ex cognitione et dilectione eius studium suum , \\\hline
1.2.3 & assi commo estado de honrra¶ & est enim honestum quasi honoris status . \\\hline
1.2.11 & Mas para que los regnos esten en su estado & Ad hoc autem ut regna subsistant \\\hline
1.2.14 & e guardase de ser deno stado & et notos quis maxime fugit \\\hline
1.2.19 & por que en qual quier estado & quod quia in quocunque statu homo sit , \\\hline
1.2.22 & que conueniblemente se deue auer en qual si quier estado . & In quolibet enim statu nouit magnanimus se decenter habere . \\\hline
1.2.22 & Ca assi commo dicho es de suso el magnanimo sabe sosrir buenas venturas e sabe conueniblemente se auer en todo estado & et in quolibet statu scit se decenter habere . Causa autem , \\\hline
1.2.26 & mas que su estado demanda sin razon & si quis ultra quam suus status requirat , \\\hline
1.2.26 & e que non conosçiesse si estado & et notabiliter se deiiceret : \\\hline
1.2.26 & mas que el su estado demandaua creyendo & vilius induebantur : \\\hline
1.2.26 & mas assi comma conuiene al su estado dellos . & sed ut decet eorum statum , \\\hline
1.3.3 & por que el ma estro es en estado & quia magister est in statu in quo ipso debet scientiam aliis tradere : \\\hline
1.3.3 & por que el estado del Rey demanda & quia status regius requirit \\\hline
1.3.4 & Conuiene que los Reyes e los prinçipes desse en prinçipalmente el buen estado del regno & principaliter Reges et Principes debent desiderare bonum statum regni : \\\hline
1.3.4 & e por si et essençialmente nasçe el buen estado del regno . & a quibus per se et essentialiter dependet bonus status regni . \\\hline
1.3.4 & Mas quales cosas son aquellas que guardan el regno en buen estado & Quae sunt autem illa quae regnum conseruant in bono statu , \\\hline
1.3.6 & que pueda dannar e menguar el estado del regno . & quod eius bonum statum deprauare possit . | Possumus autem \\\hline
1.4.1 & Ca o tris costunbres han aquellos que estan en estado de buena uentura . & alios senes , \\\hline
1.4.1 & al si non estado de honira . & iuuenes multum affectant ea quae importare uidentur honoris statum , \\\hline
1.4.1 & que traen a estado de honrra . Et por el contrario muchon temen aquellas cosas & et per locum ab oppositis , | multum timent quae important ignominiam et inhonorationem : \\\hline
1.4.1 & Et pues que assi es non son ellos en estado & Non ergo sunt in statu quod debeant uerecundari . \\\hline
1.4.3 & que a auer honrra o estado honrado . & quam ad ea quae requirit honoris status , \\\hline
1.4.3 & assi que non ayan cuydado de estado de honrra & ut quod non curent de honoris statu , \\\hline
1.4.4 & que son en el estado me dianero . & qui sunt in statu . \\\hline
1.4.4 & Ca aquellos que ya son en lu estado son mediannos entre los uieios et los mançebos & Nam illi \\\hline
1.4.4 & Ca por que estos son en estado medianero & ut vult Philosophus secundo Rhetoricorum , habent quicquid laudabilitatis est in senibus , \\\hline
1.4.4 & que los que son en el estado medianero & sed habent se medio modo . Ideo dicitur 2 Rhetoricorum , quod qui sunt in statu , nec sunt omnibus credentes , \\\hline
1.4.4 & en los que son en el estado medianero . & totum reperitur in iis qui sunt in statu . \\\hline
1.4.4 & en los que son en el estado medianero & in iis qui sunt in statu : \\\hline
1.4.4 & e en los uieios es alongado de aquellos que son en estado medianero . & totum reperiri debet in iis qui sunt in statu . \\\hline
1.4.4 & e tirar de aquellos que son en estado medianero . & qui sunt in statu : quicquid vituperabilitatis est in illis , | remouetur \\\hline
1.4.4 & e aquellos que son en estado medianero han alguna inclinacion natural alas costunbres & et illi qui sunt in statu , | quandam pronitatem , et inclinationem habent \\\hline
1.4.4 & que son en estado medianero & qui sunt in statu , \\\hline
1.4.4 & Vien assi ahun aquellos que son en estado medianero maguera & qui sunt in statu , \\\hline
1.4.7 & e los que estan en estado medianero & et illi qui sunt in statu , potentes , \\\hline
1.4.7 & segunt departidos estados e departidas edades . & secundum diuersas aetates , | et secundum diuersos status : \\\hline
2.1.11 & Ca quanto son en mayor estado & quia quanto sunt in maiori statu et in altiori gradu , \\\hline
2.1.11 & quanto estan en mas alto estado & quanto in altiori gradu existunt , \\\hline
2.1.12 & quanto el estado dellos es ma salto & quanto eorum status , \\\hline
2.1.20 & segunt su estado de tractar muy honrradamente a su muger . & secundum suum statum uxorem propriam honorifice pertractare . Ostenso , \\\hline
2.1.20 & si non fuere catado con grand acuçia el departimiento delos estados . & nisi inspecta diuersitate statuum , \\\hline
2.1.20 & penssando el su estado propreo & Decet ergo quoslibet viros , considerato proprio statu , \\\hline
2.1.21 & pensando el estado propio e las condiconnes delas perssonas son conueinbles e honestas & quae si considerato proprio statu et conditionibus personarum debite et ordinate fiant , sunt licita \\\hline
2.1.21 & Ca conuiene alos maridos de proueer conueniblemente a sus mugerssegunt sus estados & et honesta . Decet enim viros \\\hline
2.1.21 & quando catado el su estado non quieren & quando considerato suo statu non superflua vestimenta quaerunt . \\\hline
2.1.21 & mas que demanda el su estado . & et ultra quam suus status requireret , appeteret ornamenta . \\\hline
2.1.21 & mas de quanto demanda su estado . & nec ultra suum statum ornamenta appeteret : \\\hline
2.1.21 & que el su estado demandaua . & qui infra suum statum vestimenta quaerentes \\\hline
2.1.21 & que el su estado demanda¶ & ne ultra eorum statum vestimenta requirant . Quarto , \\\hline
2.1.21 & por ꝑeza notablemente en su estado dellas & ne totaliter infra eorum statum propter pigritiam \\\hline
2.2.2 & segunt que algunos son en mayorestado e en mas alta dignidat & secundum quod aliqui sunt in maiori statu et in altiori dignitate , \\\hline
2.2.11 & maque demanda el su estado . & si quaerantur cibaria nimis lauta et delicata ultra quam eius status requirat . Delicatio enim ciborum accipienda est \\\hline
2.2.11 & e seg̃t el estado dela nobleza del omne . Et pues que assi es aquel que quiere & et secundum statum nobilitatis eius . | Qui ergo , \\\hline
2.2.11 & e mas que conuiene al su estado demanda viandas delicadas peca enllo . & et ultra quam eius status requirat , | delicata cibaria quaerat , delinquit : \\\hline
2.2.13 & que estado de honrra . & quod honoris status . \\\hline
2.3.3 & segunt que pertenesçe a cada vno en su estado & ut competit propriae facultati . \\\hline
2.3.3 & assi commo el su estado demanda & prout requirit decentia status , \\\hline
2.3.8 & qual abastan segunt el mester de su estado deue ser pagado dellas & secundum exigentiam sui status bene sufficiant ad gubernationem domus , \\\hline
2.3.8 & quantas demanda el menester de su estado & quantas requirit exigentia sui status . \\\hline
2.3.12 & e demanda el su estado de cada vno . & habere curam de acquisitione pecuniae , | secundum quod exigit suus status : \\\hline
2.3.17 & mayormente parte nesçe a estado de onrra & et quia debita prouisio maxime videtur facere ad honoris statum , \\\hline
2.3.17 & Enpero por que los Reyes e los prinçipes sean guardados en su estado granado & tamen ut Reges et Principes conseruent se in statu suo magnifico , \\\hline
2.3.17 & segunt el su estado & secundum suum statum cuilibet sunt talia tribuenda , \\\hline
2.3.18 & segunt linageson en tal estado & secundum genus sunt in statu , \\\hline
2.3.18 & Et segunt el su estado conuienel es de ser meiores & et secundum quem decet eos \\\hline
2.3.19 & por que guarden el estado e la honrra dela corte conueiblemente & ut seruent decentiam curiae | et honoris statum curiales esse \\\hline
2.3.19 & e los que de nueno suben en alto estado & et de nouo ascendentium ad altum statum , \\\hline
2.3.19 & Etrossi non se deue mostrar tantlto el su estado & nec debet se sic excellentem ostendere , \\\hline
3.1.5 & e en los que turban la paz e el bue estado de los otros çibdadanos & et turbantes pacem et bonum statum aliorum ciuium . \\\hline
3.1.8 & segunt su estado sea muy acabado conuietie de dar & secundum suum statum sit maxime perfectum , \\\hline
3.1.10 & quando cada vno se ha segunt su proporcion e segunt el su estado & quando quilibet se habet \\\hline
3.1.11 & e el estado de los omes & et status hominum , \\\hline
3.1.13 & ꝑsonas faze a buen estado e paçifico dela çibdat e de los çibdadanos & ut innuit Philosophus 2 Polit’ videtur facere ad quietum \\\hline
3.1.20 & que tannian departidos estados de omes espeçialmente fallesçio & et specialiter \\\hline
3.2.1 & que por el estado paçifico del pueblo & ut propter pacificum statum populi quae traduntur \\\hline
3.2.2 & segunt su estado & secundum suum statum , sic est aequale et rectum . \\\hline
3.2.2 & segunt su estado estonçe el prinçipado es derech e ygual . & secundum suum statum : | et tunc est rectus et aequalis : \\\hline
3.2.2 & segunt su estado & Sed si populus sic dominans non intendit bonum omnium secundum suum statum , \\\hline
3.2.5 & que pudiere mouerse a procurar todo buen estado del regno & ideo omni cura qua poterit mouebitur ad procurandum bonum statum regni , \\\hline
3.2.8 & pues que as susi es much se puede turbar el buen estado e la paz dela çibdat & Valde ergo turbari potest tranquillus status | et pax ciuitatis \\\hline
3.2.8 & ¶lo segundo el estado bueno dela çibdat & status tranquillus ciuitatis \\\hline
3.2.9 & por los quales se puede guardar el buen estado del regno . & per quos bonus status regni conseruari potest , \\\hline
3.2.9 & por los quales se deue guardar el buen estado vieren & et aliorum per quos bonus status regni conseruari habet , \\\hline
3.2.10 & por los quales el buen estado del regno se puede guardar mas saluales & per quos bonus status regni conseruari potest , \\\hline
3.2.10 & e el buen estado del regno & et bonus status regni , \\\hline
3.2.11 & para penssar del estado del regno . &  \\\hline
3.2.12 & a njnguno en su estado mas esfuercas & non intendit quodlibet seruare in suo statu , \\\hline
3.2.13 & que demada el estado dellos & et alios existentes in regno ut requirit eorum status ; \\\hline
3.2.15 & mas si el regnado fuere en estado de antiguedat & sed si regnum diu in statu perstiterit , \\\hline
3.2.15 & qual cosa corronpe el buen estado del regno &  \\\hline
3.2.15 & fue meior guardado el buen estado del regno & bonus status regni , \\\hline
3.2.16 & que el regno se en buen estado . & et utrum regnum oporteat esse in bono statu : \\\hline
3.2.18 & mar ala Real magestado das aquellas cosas &  \\\hline
3.2.18 & e buen amonestador e razonador por si . & et bene persuadere per se , \\\hline
3.2.18 & de que fabla el amonestador & et ex ipsis negotiis de quibus loquitur \\\hline
3.2.18 & Et pues que assi es todo buen amonestador o razonador & Omnis ergo bene persuadens , \\\hline
3.2.18 & que el que es buen amonestador e razonador & Itaque cum dictum sit quod qui bene persuadens , \\\hline
3.2.19 & e a buen estado del Rey e del pueblo & et ad bonum statum eius : \\\hline
3.2.23 & e el buen estado de los çibdadanos non podria estar & et bonus status ciuium \\\hline
3.3.6 & Lo primero para assechar e ascuchar el estado de los enemigos . & Primo ad explorandum inimicorum facta . \\\hline
3.3.6 & que vayan e escuchar e a saber las condiciones e el estado delos enemigos & explorantes conditiones | et facta hostium . \\\hline
3.3.10 & e el estado de muchos omnes es puesta &  \\\hline
3.3.10 & e altos en el estado del cuerpo & proceri statura , scientes proiicere hastas \\\hline
3.3.10 & grande en su estado & procer statura , \\\hline

\end{tabular}
