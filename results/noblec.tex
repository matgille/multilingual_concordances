\begin{tabular}{|p{1cm}|p{6.5cm}|p{6.5cm}|}

\hline
1.1.7 & e ennobleçida de tales uirtudes & esse ornatam talibus virtutibus , \\\hline
2.1.15 & por que es mas enoblesçido e mayor en razon & eo quod ratione praestantior : \\\hline
2.1.18 & los mas los mas son ennoblesçidos & Nam secundum eundem Philosophum in Polit’ mares plus vigent ratione quam foeminae : \\\hline
2.1.18 & Et en essa misma manera avn los moços non son tan noblesçidos & Sic etiam et pueri non sic ratione vigent \\\hline
2.1.18 & por ende commo las mugers comunalmente non sean ennoblesçidas & Quare cum mulieres communiter non tanta bonitate polleant sicut uiri ; \\\hline
2.2.7 & si non fuere enoblesçido & nisi vigeat prudentia et intellectu . \\\hline
2.2.7 & et mas enoblesçido por sabidia e por entendimiento & et magis viget prudentia et intellectu : \\\hline
2.2.7 & Ca si los prinçipes non fueren ennoblesçidos & Nam nisi princeps vigeat prudentia \\\hline
2.2.7 & por que se pueda enoblesçer & ut vigere possint prudentia et intellectu . \\\hline
2.3.7 & que son ennoblesçidos & naturaliter debent esse subiecti hominibus pollentibus subtilitate et prudentia . \\\hline
2.3.7 & Et segunt esta manera tal de fablar los çibdadanos que son mas noblesçidos & Secundum quem modum loquendi , | ciues , \\\hline
2.3.13 & o por conparaçion delas fenbras a los uatones . Ca ueemos que por que el uaton es mas ennoblesçido en razon & eo quod sit ratione prestantior , \\\hline
2.3.18 & e menos ennoblesçidos en costunbres que los otros . & et sunt peruersiores aliis | secundum mores . \\\hline

\end{tabular}
