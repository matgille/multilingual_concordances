\begin{tabular}{|p{1cm}|p{6.5cm}|p{6.5cm}|}

\hline
1.1.6 & Ca asi commo dize el philosofo en el quinto libro delas politicas el prinçipado & ( ut dicitur 5 Politicorum principatus debet respondere magnitudini , \\\hline
1.1.10 & que tal prinçipado e tal sennorio non dura mucho ¶ & ex eo quod talis principatus non multum durat . Secunda , \\\hline
1.1.10 & que tal prinçipado e tal sennorio puede seer sin bondat deuida ¶ & ex eo quod esse potest sine bonitate vitae . \\\hline
1.1.10 & que tal prinçipado pue de seer non digno¶ & ex quod est indignus . \\\hline
1.1.10 & que por este prinçipado los çibdadanos son ordenados alos menores bienes & ex eo quod per huiusmodi principatum ciues ordinantur ad minora bona . \\\hline
1.1.10 & tal prinçipado non puede mucho durar & talis principatus diu durare non potest . \\\hline
1.1.10 & pues que assi es commo los prinçipados & Cum ergo Principatus se extendant adinuicem , \\\hline
1.1.10 & que el prinçipado & quod principatus liberorum , \\\hline
1.1.13 & que el prinçipado e el estado muestraqles el omne & quod principatus virum ostendit . Tunc enim apparet qualis homo sit , cum in principatu existens , in quo potest bene et male facere , \\\hline
1.2.11 & e vn prinçipado . & et quidam principatus . \\\hline
1.2.11 & e el principado sea de los subditos por conparaçion alas leyes e al prinçipe & et | principatus sit ipsorum subditorum per respectum ad leges , \\\hline
1.2.12 & fasta que son puestos en alguna dignidat o en algun prinçipado de sennorio . Ca quando algun omne non ha de gouernar & nisi constituantur in aliquo principatu . Nam quandiu aliquis non habet regere nisi seipsum , \\\hline
1.2.12 & Mas quando es puesto en algun prinçipado o en algun sennorio & Sed quando constituitur in principatu aliquorum , \\\hline
1.2.12 & quanto cada vno es puesto en mayor prinçipado e en mayor dignidat & quanto aliquis in maiori principatu constituitur ; \\\hline
1.2.12 & que el prinçipado e la dignidat muestran & Propter quod et prouerbialiter dicitur : \\\hline
1.2.12 & quel ponga en alguna dignidat o en algun prinçipado ¶ & ponat ipsum in aliquo principatu . \\\hline
1.2.16 & por ende si fueren puestos en prinçipado o en regno son mucho menospreçiados & si in principatu sint , | maxime contemnuntur : \\\hline
1.2.22 & e cerca los prinçipados . Et generalmente çerca todas las cosas auenturadas e non auenturadas . & et uniuersaliter circa omnia fortunia , | et infortunia : \\\hline
1.2.22 & e çerca los prinçipados & Ex consequenti autem est circa diuitias , et principatus , \\\hline
1.2.22 & Ca las riquezas e los prinçipados & sicut honor : \\\hline
1.2.22 & Mas despues desto es çerca las riquezas e los prinçipados e çerca los otros bienes de fuera en tal manera & tanquam circa maxima bona exteriora : \\\hline
1.2.22 & que el magnanimo conueniblemente se ha enauiendo las riquezas e los prinçipados . & ex consequenti autem est circa diuitias , | et principatus , \\\hline
1.2.22 & si quier prinçipados si quier riquezas & siue principatus , | siue diuitiae , \\\hline
1.3.4 & e finalmente se entiende la salud del Regno e del prinçipado & et finaliter intendatur salus Regni et Principatus , \\\hline
1.4.6 & Ca creen que las riquezas son dignidat de sennorio e de prinçipado & credunt enim , | quod dignitas principatus sint diuitiae . \\\hline
1.4.6 & Ca la dignidat del prinçipado & Nam dignitas principatus principaliter innititur virtutibus \\\hline
1.4.7 & por que es en algun prinçipado & quia est in aliquo principatu , \\\hline
1.4.7 & que son cerca su prinçipado . &  \\\hline
1.4.7 & Ca aquellos que son en algun poderio o en algun prinçipado & Nam qui est in aliquo potentatu , | vel in aliquo principatu , \\\hline
1.4.7 & e pues que assi es el prinçipado & Studiosiores igitur sunt potentes , \\\hline
1.4.7 & por el su prinçipado non pueden & quia diuersis curis intendunt propter principatum , \\\hline
1.4.7 & por su prinçipado & Rursus quia pollent principatu et potentia , \\\hline
2.1.2 & fue fecho vn prinçipado & factus est principatus , \\\hline
2.1.3 & e del prinçipado empero pertenesçe entender este bien & ad Reges et Principes spectat intendere bonum regni et principatus : \\\hline
2.1.4 & comuidat de regno e de prinçipado & et ciuitatis , inuenta fuit communitas regni et principatus , \\\hline
2.2.2 & Enpero por que cada vno dellos es establesçido en algun prinçipado & quia singuli constituuntur in aliquo principatu \\\hline
2.2.2 & que deuen auer el prinçipado e el senorio en el regno & qui debent habere principatum et dominium in regno ; \\\hline
2.3.15 & Et por ende en la mayor parte los prinçipados & ut plurimum principatus sunt peruersi : \\\hline
2.3.16 & nin muchos prinçipados avn apssona & In magna enim ciuitate non sunt congreganda officia et principatus , \\\hline
3.1.6 & para establesçer prinçipado e regno & ut constituant principatum | et regnum ; \\\hline
3.1.6 & por conpania de prinçipado & quod fit per societatem principatus et regni ; \\\hline
3.1.7 & assi comm̃los prinçipantes de mayor prinçipado & ut principantes maiori principatu , \\\hline
3.1.13 & en el quinto libro delas ethicas el prinçipado & Nam ut dicitur 5 Ethic’ principatus virum ostendit prius enim quam assumantur homines ad aliquam dignitatem , \\\hline
3.1.13 & quales pone en los prinçipados e enlos maestradgos & cognoscere quales praeficiant praepositos | et magistros ; \\\hline
3.1.13 & Si los prinçipados e los maestradgos & si principatus | et magistratus \\\hline
3.1.13 & e mudado los maestradgos e los prinçipados & Mutare autem aliquando magistratus et principatus , et distribuere eos diuersis personis , \\\hline
3.2.1 & que dichas son conuiene desaƀ del prinçipado del conseio del alcalłia del pueble . & De omnibus ergo his quatuor , videlicet de principatu , consilio , \\\hline
3.2.1 & Enpero primero diremos del prinçipado . & primo tamen dicemus de ipso principatu . \\\hline
3.2.2 & departe el philosofo seys linaies de prinçipados & Tertio Politicorum distinguit Philosophus sex modos principantium , \\\hline
3.2.2 & e la poliçia que quiere dezer pueblo bien enssenoreante son bueons prinçipados . & et politia sunt principatus boni : \\\hline
3.2.2 & enssennoreante son malos prinçipados . & et democratia sunt mali . \\\hline
3.2.2 & Et alli muestra el pho departir el buen prinçipado del malo . & Docet enim idem ibidem discernere bonum principatum a malo . \\\hline
3.2.2 & ca si en algun señorio o prinçipado es entendido el bien comun & Nam si in aliquo dominio | aut principatu intenditur bonum commune \\\hline
3.2.2 & podemos tomar la suficiençia de los prinçipados & secundum viam Philosophi accipere possumus sufficientiam principantium tam peruersorum quam rectorum . Nam in ciuitate , \\\hline
3.2.2 & que dos principados se le una tan del sennono de vno prinçipado de derechas & sed Tyrannus . Duo ergo principatus consurgunt ex dominio unius unus rectus , \\\hline
3.2.2 & Et otro prinçipado malo & ut cum propter bonum commune dominatur Rex : et alius peruersus , \\\hline
3.2.2 & que se fazian eran departidos por el departimiento delos prinçipados & unde et maleficia facta distinguebantur ex diuersitate principatuum . Dicebatur autem de \\\hline
3.2.2 & e tal prinçipado es dicħa ristrocaçia & et tunc talis principatus dicitur Aristocratia , \\\hline
3.2.2 & que quiere dezer prinçipado de buenos omes e uir̉tuosos & quod idem est quod principatus bonorum et virtuosorum . Inde autem venit \\\hline
3.2.2 & Et pues que assi es dos prinçipados se leuna & Consurgit igitur duplex principatus ex dominio paucorum : \\\hline
3.2.2 & Lo tercero se pueden departir los prinçipados & Tertio possunt distingui principatus , \\\hline
3.2.2 & e por ende en tal prinçipado & In tali ergo principatu , \\\hline
3.2.2 & segunt su estado estonçe el prinçipado es derech e ygual . & secundum suum statum : | et tunc est rectus et aequalis : \\\hline
3.2.2 & Et por que tal prinçipado non ha nonbre propio & et quia talis principatus non habet nomen proprium , vocat eum Philosophus nomine communi , \\\hline
3.2.2 & quanto a todos los prinçipados & quantum ad omnes principatus \\\hline
3.2.2 & e prinçipalmente quanto al grant prinçipado que enssennorea a todos los otros & et principaliter | quantum ad maximum principatum \\\hline
3.2.2 & ca la poliçia esta mayormente en el ordenamiento del grant prinçipado & qui dominantur omnibus aliis . Politia enim consistit maxime in ordine summi principatus , \\\hline
3.2.2 & Enpero el prinçipado del pueblo & Principatus tamen populi \\\hline
3.2.2 & e nos podemos llamar atal prinçipado gouernamiento del pueblo & eo quod non habeat commune nomen , Politia dicitur . Nos autem talem principatum appellare possumus gubernationem populi , \\\hline
3.2.2 & es dich prinçipado de malos & est principatus peruersus , \\\hline
3.2.2 & quantos son los prinçipados & Patet ergo quot sunt principatus , \\\hline
3.2.3 & e que cosa es el muy buen prinçipado despues & et quis sit optimus principatus . \\\hline
3.2.3 & que deptimos las maneras de los prinçipados e de los señorios & Postquam distinximus modus principandi , \\\hline
3.2.3 & e quales son malos finca de demostrar entre los prinçipados derechs e buenos & et qui peruersi : | restat ostendere \\\hline
3.2.3 & que el regno es muy buen prinçipado & quod regnum est optimus principatus , \\\hline
3.2.3 & por ssi en el prinçipado & in uno principatu , \\\hline
3.2.3 & quando enssennorean muchs nunca puede ser paz en tal prinçipado & nunquam potest esse pax in huiusmodi principatu , \\\hline
3.2.4 & Por la qual cola meior sera este tal prinçipado & Quare melior erit huius principatus , \\\hline
3.2.4 & tanto peoras el prinçipe e el prinçipado & quia bonum multorum est \\\hline
3.2.4 & que el regno es prinçipado muy digno & regnum esse dignissimum principatum : \\\hline
3.2.4 & por que entre los prinçipados derechs el prinçipado de vno & inter principatus enim rectos , | principatus unius , \\\hline
3.2.4 & Mas entre los malos prinçipados el prinçipado de vno & inter peruersos vero principatus , | principatus , \\\hline
3.2.4 & que por nonbre comun es dich tirannia es muy mal prinçipado & unius , | qui communi nomine tyrannis nuncupatur , est pessimus . \\\hline
3.2.4 & e el prinçipado real es muy bueno & ostendetur enim quod sicut monarchia regia est optima ; \\\hline
3.2.4 & por ende el prinçipado thiranico es muy malo & ita quia maiori bono maius malum opponitur , monarchia tyrannica est pessima . Dominari autem plures dominio recto , \\\hline
3.2.4 & por que es contrario al prinçipado del regno & non est dignius , \\\hline
3.2.4 & que el regno es prinçipado muy digno & secundum rectum dominium melius est dominari unum , \\\hline
3.2.4 & e vn Rey en todo vn prinçipado & vel unus Rex in toto principatu \\\hline
3.2.4 & e el tal sennorio es prinçipado muy malo . & et ut infra planius ostendetur ) tyrannis est pessimus principatus . \\\hline
3.2.5 & e el prinçipado venga por elecçio & ire per electionem quam per haereditatem . \\\hline
3.2.6 & mas conplidamente despues que fuere puesto en el prinçipado & decens est tales excessus in ipsa monarchia perfectius reperiri . Decet enim ipsum regem volentem \\\hline
3.2.6 & por que el regno es prinçipado derech . & Nam regnum est principatus rectus , \\\hline
3.2.7 & que la thirama es muy mal prinçipado & tyrannidem esse pessimum principatum . \\\hline
3.2.7 & por razon que tal prinçipado es muy afincado por enpesçer . & ex eo quod est efficacissimum ad nocendum . Quarta , \\\hline
3.2.7 & por razon que tal prinçipado ha de enbargar muy grandeᷤ bienes delos çibdadanos ¶ & ex eo quod impedire habet maxime bona ipsorum ciuium . Prima via sic patet . \\\hline
3.2.7 & assi ca quando el Rey tiene el prinçipado & Quia si dominatur Rex , \\\hline
3.2.7 & que tal sennor auer prinçipado es & est quasi principari multitudinem , \\\hline
3.2.7 & assi commo partir vn prinçipado en muchs . & est quasi principari multitudinem , \\\hline
3.2.7 & o que es esso mismo que auer muchs el prinçipado . & eo quod in tali principatu intendatur bonum multorum . \\\hline
3.2.7 & por que en tal prinçipado es entendido el bien de muchs . & eo quod in tali principatu intendatur bonum multorum . \\\hline
3.2.7 & Pues que assi es el prinçipado & nisi sit quodammodo \\\hline
3.2.7 & por ende tanto la tirania es peor prinçipado & tanto tyrannis est principatus peior , \\\hline
3.2.7 & do dize que la tirania es muy mal prinçipado & ubi ait , | tyrannidem esse pessimum principatum , \\\hline
3.2.7 & por uoluntad tal prinçipado & quia nullus liberorum voluntarie sustinet principatum talem . \\\hline
3.2.7 & por que tal prinçipado es muy afincado para enpeesçer . casi commo el prinçipado del Rey & Nam sicut principatus Regis eo quod sit maxime unitus , \\\hline
3.2.7 & e el su prinçipado es muy bueno . & tunc est Rex et est optimus principatus : \\\hline
3.2.7 & et el su prinçipado es muy malo & et est pessimus , \\\hline
3.2.7 & que la tirania es muy mal prinçipado & tyrannidem esse pessimum principatum propter rationes tactas . \\\hline
3.2.7 & que es muy mal prinçipado & qui est pessimus principatus . \\\hline
3.2.9 & e farian e mouerian discordias en el regno e enl prinçipado Et pues & ut seditiones mouerent in regno aut principatu ; \\\hline
3.2.12 & que es en los otros malos prinçipados & et iniquitatis est in aliis peruersis principatibus , \\\hline
3.2.12 & por las cosas sobredichas tres prinçipados son buenos e tres malos & ( ut patet ex habitis ) tres principatus boni , | et tres peruersi . \\\hline
3.2.12 & por el bien comun este prinçipado es derech & est principatus rectus , \\\hline
3.2.12 & e por que los malos prinçipados & quia iniqui principatus diuitum congregantur in ea : restat \\\hline
3.2.14 & nin el prinçipado rreal & quam rectus \\\hline
3.2.14 & e prinçipado derech non es contra no al prinçipado derech . & principatus rectus non contrariatur recto principatui : \\\hline
3.2.14 & Mas el mal prinçipado & sed peruersus peruerso . \\\hline
3.2.14 & escontrana ala tirama del mal prinçipado & ut tyrannis populi contrariatur tyrannidi monarchiae : \\\hline
3.2.14 & Et vn prinçipado tiranico es contrario a otro prinçipado tiranico e malo & et una monarchia tyrannica contrariatur alii . \\\hline
3.2.14 & matandol o echandol del prinçipado . & eum perimens vel expellens . Totus ergo populus \\\hline
3.2.14 & por que gane el su prinçipado . & ut obtineat principatum eius . \\\hline
3.2.14 & segunt dicho es se aya de destroyr el prinçipado tiranico . & cum tot modis dissoluatur tyrannicus principatus . \\\hline
3.2.14 & e en tantas manerasse ha de desfazer el su prinçipado &  \\\hline
3.2.15 & para que se pueda man tener en lu prinçipado & ad hoc ut se in suo principatu praeseruet . \\\hline
3.2.15 & poniendo los en alguons prinçipados & introducendo eos ad aliquos principatus , \\\hline
3.2.15 & Mas avn por esta razon el prinçipado se faze mas durable & sed etiam principatus ex hoc durabilior redditur , \\\hline
3.2.15 & Mas esta cautela es propro prouechable en dos prinçipados & Est autem haec cautela utilis solum duobus principatibus : \\\hline
3.2.15 & assi commo en el prinçipado & ut principatui constituto ex hominibus assuetis ad bella , \\\hline
3.2.15 & Et en el prinçipado & et principatui in quo quis nouiter principari coepit . Homines enim bellatores \\\hline
3.2.15 & que sea apareiado el principado Real & ut dissoluatur regius principatus . Quintum , est diligenter aspicere , \\\hline
3.2.15 & o algun prinçipado o algun maestadgo . & vel ad | aliquem principatum , \\\hline
3.2.15 & e dar les los señorios e los prinçipados . & et conferre eis dominia et principatus . \\\hline
3.2.16 & e las maneras de los prinçipados & et distinximus quot sunt genera principatus , \\\hline
3.2.16 & Et declaramos en commo el regno era muy buen prinçipado & et declarauimus regnum esse optimum principatum , \\\hline
3.2.17 & Et por ende ponen todo su prinçipado & exponunt periculo totum principatum \\\hline
3.2.19 & o quantas son las maneras de los prinçipados & et \\\hline
3.2.19 & e qual prinçipado es mas durable & qui principatus est durabilior , \\\hline
3.2.19 & e en qual manera el prinçipado es mas durable . & qui principatus est durabilior , \\\hline
3.2.19 & Et en qual manera el prinçipado & et qualiter principatus \\\hline
3.2.19 & segunt las quales aquel prinçipado se ha de saluar . & secundum quas saluari habet principatus ille . \\\hline
3.2.19 & por que se salue el prinçipado & Quae autem leges sunt ferendae , ut saluetur principatus eius , \\\hline
3.2.30 & por que el prinçipado pueda durar en el pueblo . & ad hoc ut possit durare principatus in populo : \\\hline
3.2.31 & que se tiraria el prinçipado e el regno . & et regnum . \\\hline
3.2.31 & e alos prinçipes de guardar las bueans costunbres del prinçipado e del regno & Decet ergo reges et principes obseruare bonas consuetudines principatus et regni , \\\hline
3.3.1 & en quanto es cabeça del regno o del prinçipado & In tertio vero eruditur Rex aut Princeps ut est caput regni aut principatus , \\\hline
3.3.16 & que en el prinçipado e en el regno ay puertos e tierras marinas assentadas çerca de la mar . & propter quod eos oportet uti pugna defensiua . Amplius in principatu et regno contingit esse portus et terras maritimas iuxta mare sitas : \\\hline

\end{tabular}
