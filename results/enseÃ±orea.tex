\begin{tabular}{|p{1cm}|p{6.5cm}|p{6.5cm}|}

\hline
1.1.10 & Mas ensennorear & ponenda est in Principatu optimo , et digno . Principari autem per ciuilem potentiam , est principari seruis , \\\hline
1.1.10 & que ensennorear despotice & quam principari despotice , \\\hline
1.1.10 & que quiere dezir enssennorear . seruilmente e sobre los sieruos ¶ & idest dominaliter . \\\hline
1.2.7 & nin enssennorear natural mente . & quia sine ea non possunt naturaliter dominari . \\\hline
1.2.16 & que ha de enssennorear alos otros & cuius est aliis dominari , \\\hline
1.3.4 & e de enssennorear prinçipalmente & Cum ergo in arte regnandi et principandi principaliter \\\hline
2.1.3 & por que cada vno deue estudiar de ser digno para gouernar e para enssennorear . & tum quia quilibet studere debet | ut sit dignus regere \\\hline
2.1.12 & Mas vno se esforçara de enssennorear al otro & sed unus alteri , \\\hline
2.1.14 & Mas el padre deue enssennorear alos fiion & Sed pater debet praeesse filiis \\\hline
2.1.14 & Ca enssennorear realmente es senoreat en toda manera & Nam praeesse regaliter est praeesse totaliter et secundum arbitrium : \\\hline
2.1.14 & mas enllennore ar çiuilmente es enssennorear & sed secundum quasdam conuentiones | et pacta . Rursus , \\\hline
2.1.14 & Ca el padre assi deue enssennorear alos fijes & Nam pater sic debet praeesse filiis , \\\hline
2.2.1 & e deuen enssennorear a ellos & et conseruant : \\\hline
2.2.3 & non deue ninguno ensennorear sinplemente & quia uxori ( ut in praehabitis tangebatur ) non quis debet praeesse simpliciter ex arbitrio , \\\hline
2.2.3 & mas deue enssennorear la segunt & sed ei praeesse debet \\\hline
2.2.3 & que el uaron deue ensennorear ala mug̃ & virum praeesse mulieri , \\\hline
2.2.3 & Mas ala muger deue enssennorear çibdadana niete & sed mulieri quidem politice , natis autem regaliter . Tertium autem regimen quod est in domo , \\\hline
2.2.3 & Ca si el padre deue enssennorear alos fiios realmente & Nam si pater debet praeesse filiis regaliter \\\hline
2.2.3 & e non çibdadanamente non entender enssennorear el bien de los sieruos & praeesse aliquibus dominatiue , | non intendere bonum ipsorum , \\\hline
2.2.3 & Empero el padre deue enssennorear alos fuos & pater tamen debet praeesse filiis propter bonum ipsorum filiorum . \\\hline
2.2.3 & que el padre deue enssennorear alos fiios & patet quod filiis debet \\\hline
2.2.7 & por que puedan enssennorear mas sabiamente & quanto decet eos intelligentiores \\\hline
2.2.8 & e de enssennorear al pueblo el qual pueblo non puede entender & inter gentes et dominari populo , | qui non potest percipere \\\hline
2.2.8 & assi es si conuiene de saber la sçiençia moral a aquellos que dessean enssennorear & Sic ergo morale negocium scire expedit ab iis | qui cupiunt principari : \\\hline
2.2.8 & en qual manera de una enssennorear & qualiter debeat principari , \\\hline
2.2.10 & e de enssennorear & qui debent principari \\\hline
2.2.17 & les que son dignos de enssennorear & quod digni sint dominari , \\\hline
2.3.13 & e enla çibdat bien ordenadlos sabios deuen enssennorear & et corpus obedit : sic in politia bene ordinata sapientes debent dominari , \\\hline
2.3.14 & assi commo en fortaleza o en poderio ciuil deue enssennorear & ut in fortitudine , | vel in ciuili potentia , \\\hline
3.2.3 & que si fuere vno lo lo el que enssennoreaua aura mayor paz en la çibdat & ergo si solus unus principaretur | inter eos , \\\hline
3.2.4 & ca si el Rey dessea enssennorear & nam si Rex recte dominari desiderat , \\\hline
3.2.7 & de non enssennorear con sennorio de tirania & ne principentur principatu tyrannico , \\\hline
3.2.7 & de non enssennorear con señorio de tirania & ne dominetur | per tyrannidem \\\hline
3.2.12 & Mas si enssennorear en pocos & et vocatur aristocratia siue principatus bonorum . Si vero dominentur non quia boni , \\\hline
3.2.12 & por que pueda mas enssennorear & ut magis principari possit , \\\hline
3.2.15 & en qual manera deua enssennorear & ut sciat cognoscere qualiter principari debeat , \\\hline
3.2.19 & por que escogiendo la meior manera de prinçipar o de enssennorear ponga leyes muy derechos . & ut eligens optimum modum principandi , | ferat leges iustissimas , \\\hline
3.2.19 & Mas qual es la manera muy buena de prinçipar o de enssennorear . & Quis sit autem optimus modus principandi , \\\hline
3.2.29 & que aquel que manda enssenerorear al entendimiento manda enssennorear a dios e ala ley . & quod qui iubet principari intellectum , | iubet principari deum et legem ; \\\hline
3.2.29 & as quien manda enssennorear al omne & sed qui iubet principari hominem , \\\hline
3.2.29 & que en el derecho gouernamiento non deue enssennorear la bestia & quod in recto regimine principari non debet bestia , \\\hline
3.2.30 & Ca non puede ser que ninguno pueda enssennorear prolongadamente & non enim esset possibile aliquem diuturne principari , \\\hline
3.2.33 & e en grand pobreza non saben enssennorear & secundum excessum sunt indigentes et pauperes , \\\hline

\end{tabular}
