\begin{tabular}{|p{1cm}|p{6.5cm}|p{6.5cm}|}

\hline
3.1.8 & a algun prinçipe o algun sennor e commo en la çibdat conuenga de dar alguons ofiçioso alguons maestradgos o algunas alcaldias la qual cosa non seria & ut cum in ciuitate oporteat dare aliquos magistratus , et aliquas praeposituras , | dare aliquos magistratus , et aliquas praeposituras , quod non esset , \\\hline
3.1.19 & Lo quarto se entremetio el dicho philosofo del departimiento de aquellos que iudgan ca dize que dos deuian ser los linages de los iudgadores e delos alcalłs e dias audiençias . La vna ordinaria & Dicebat enim debere esse duo genera iudicantium , et duplex praetorium : unum ordinarium , \\\hline
3.2.1 & e estas son ¶ El prinçipe Et el conseio . Et el alcalłia ¶ Et el pueblo . Enpero podemos de aquellas cosas & quatuor quae consideranda sunt in regimine ciuitatis . Haec autem sunt princeps , consilium , praetorium , \\\hline
3.2.1 & e guardadas por el prinçipe esto parte nesçe al alcaldia o alos uezes por que ellos son aqual los que segunt tales leyes deuen iudgar los fechs de los çibdadanos . Mas bien guardar las leyes puestas & et custoditas per principem , spectat ad praetorium siue ad iudices : ipsi enim sunt \\\hline
3.2.1 & Pues que assi es de todas estas quatro cosas que dichas son conuiene desaƀ del prinçipado del conseio del alcalłia del pueble . diremos breuemente & et laudabile et fugiat vituperabile et turpe . De omnibus ergo his quatuor , videlicet de principatu , consilio , praetorio , et populo , \\\hline
3.2.16 & Et el conseio . Et el alcalłia Et el pueblo . Et pues que assi es despues & videlicet Principem , consilium , praetorium , et populum . Postquam ergo auxiliante deo determinauimus de Principe , \\\hline
3.2.20 & segunt la orden sobredichͣ que digamos del alcalłia o del iuyzio mostrando & secundum ordinem praetaxatum ut exequamur de praetorio , siue de iudicio , | ut exequamur de praetorio , siue de iudicio , inuestigando qualiter iudicandum sit , \\\hline
3.2.20 & ca en cada vna çibdat conuiene que aya vna alcalłia otdinaria ala qual deuen venir todos los pleitos & secundum leges iam conditas . Nam in qualibet ciuitate oporteret esse aliquod praetorium ordinarium ad quod causae reducantur et litigia exorta in ciuitate illa : \\\hline
3.2.20 & et de las cosas generales Mas el alcalłe o el iues iudga delas cosas presentes e delas perssonas señaladas . & et in uniuersali : sed praefectus aut iudex iudicat de praesentibus et determinatis , ad quos est amare , \\\hline
3.2.20 & alas quales puede auer amor o abortençia . Alos quales alcalłs muchos vezes se ayunta algun pro para si mesmos & ad quos est amare , et odire , et quibus proprium commodum annexum est saepe . Quarta via ad ostendendum hoc idem , \\\hline
3.2.21 & e de deue dar ante los alcalłs rodemos lo prouar por tres razones ¶ La primera seqma &  \\\hline

\end{tabular}
