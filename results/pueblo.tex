\begin{tabular}{|p{1cm}|p{6.5cm}|p{6.5cm}|}

\hline
1.1.1 & enpero todo el pueblo se ha de enseñar por este libro & totus tamen populus erudiendus est per ipsum . \\\hline
1.1.1 & ¶ E pues que asi es todo el pueblo deue seer Oydor & totus ergo populus auditor quodammodo est huius artis , \\\hline
1.1.1 & que quanto mayor es el pueblo tendo menores & propter quod dicitur 3 Rhetoricorum , \\\hline
1.1.1 & E pues que asi es commo todo el pueblo pueda entender las cosas sotiles deuemosyr en este libro & Cum igitur totus populus subtilia comprehendere non possit , \\\hline
1.1.1 & e esta sçiençia estender la fasta el pueblo & oportet doctrinam hanc extendere usque ad populum , \\\hline
1.1.5 & e la su finque non al pueblo . & et finem cognoscere quam populum , \\\hline
1.1.5 & Por que el es ginador del pueblo & eo quod fit populi directiua . \\\hline
1.1.5 & e el pueblo non del . & eo quod fit populi directiua . \\\hline
1.1.6 & a el pueblo comunalmente non siente & Vulgus communiter non percipit , \\\hline
1.1.6 & por que non sea menospreçiado de su pueblo¶ & sed qui dormiens , decet Regiam maiestatem tales delectationes immoderatas fugere , \\\hline
1.1.7 & apremiador del pueblo & quia amittit maxima bona . Secundo , \\\hline
1.1.7 & que es Robador e despoblador del pueblo . & quia efficitur populi depiaedator . Diligens enim pecuniam \\\hline
1.1.7 & que ha de ser robador e despoblador del pueblo & quod Rex sit Populi depraedator . \\\hline
1.1.7 & e si robare el pueblo & si depraedetur populum et Rem publicam , \\\hline
1.1.7 & que non procure el bien del pueblo & quod tamen bonum Reipublicae non procuret : \\\hline
1.1.7 & mas es robador del pueblo & sed tunc est depraedator , \\\hline
1.1.7 & que sea robador del pueblo & Regem admittere maxima bona , \\\hline
1.1.8 & e pornia los pueblos en peligro¶ & ponere suam felicitatem in honoribus , quia ex hoc efficietur periclitator Populi , et praesumptuosus : \\\hline
1.1.8 & por que pueda delo que feziere honrra alcançar presumira de poner los pueblos a todo peligro & ut possit honorem consequi , | praesumet suam gentem exponere omni periculo . Exemplum huiusmodi habemus \\\hline
1.1.8 & Et sera malo al pueblo & erit malus in suis rebus , \\\hline
1.1.8 & Ca non fara fuerça de poner el pueblo a grandes peligros presuptuosamente e arrebatadamente . & secundum personarum dignitatem . \\\hline
1.1.9 & o si es głioso en los pueblos ¶ & vel si sit in populis gloriosus . \\\hline
1.1.9 & Ca segunt que dize boeçio la fama del pueblo romano & Dici potest , \\\hline
1.1.9 & Et por ende serie dicho Robador del pueblo . & nam ex hoc efficerentur populi praedatores . | Quod non decet regiam maiestatem suam \\\hline
1.1.10 & si el pueblo liberalmente & si populus libere , \\\hline
1.1.10 & si enseñorea sobre el pueblo & si per violentiam , et per ciuilem potentiam dominetur : \\\hline
1.1.10 & assi las naçiones e los pueblos . & et ad ea , per quae sibi possit subiicere nationes . \\\hline
1.1.10 & e en subiugar las naçions e las gentes e los pueblos . & et in subiiciendo sibi nationes , \\\hline
1.1.11 & e fazen grand discordia en el pueblo ¶ & ut dominentur , | et faciunt dissensionem in Populo . \\\hline
1.1.11 & tal es el pueblo . & talis est Populus . \\\hline
1.1.12 & e es gouernador dela muchedunbre del pueblo . & et est rector multitudinis . \\\hline
1.1.12 & que los Reyes e los prinçipes gouiernen sus pueblos iustamente e santamente & ut per prudentiam , | et legem populum sibi commissum iuste , \\\hline
1.1.13 & si gouernare el su pueblo & et legem recte regat populum sibi commissum , \\\hline
1.1.13 & la qual materia es la gente e la muchedunbre delons pueblos esta muestra & quae est gens , et multitudo , | indicat eius praemium esse magnum . Primae partis primi libri de regimine Principum finis . \\\hline
1.2.1 & segund la sabiduria del gouernamiento gouernan las sus gentes e los sus pueblos & secundum prudentiam regitiuam , gentem sibi commissam \\\hline
1.2.1 & e guyen el pueblo & secundum ordinem rationis dirigant Populum sibi commissum . \\\hline
1.2.7 & ala qual es de guiar el pueblo . & in quem est populus dirigendus . \\\hline
1.2.7 & assi el Rey non puede gouernar el pueblo & sic nec Rex potest populum dirigere siue regere , \\\hline
1.2.7 & que ha de guiar el pueblo . &  \\\hline
1.2.7 & Et por ende faze se tomador e robador del pueblo & et sensibilibus bonis . Efficietur ergo depraedator populi , dominabitur tyrannice , \\\hline
1.2.7 & si non commo podra sacardes e algo del su pueblo . & non curabit qualitercunque possit pecuniam extorquere . Tertio decet Reges , \\\hline
1.2.8 & por que el Rey guia el su pueblo & quo Rex suum populum dirigit , \\\hline
1.2.8 & su pueblo le conuiene de ser acordable e prouisor . & et prouidum : \\\hline
1.2.8 & por la qual ha de guiar el pueblo le conuiene de ser entendido e razonable . & oportet ipsum esse intelligentem , | et rationalem . \\\hline
1.2.8 & que es puesto para gouernar tanta gente e tanto pueblo . & ut tantam gentem regere habeat , oportet quod sit industris , et solers , \\\hline
1.2.8 & que conuiene a su pueblo e asu gente ¶ & bona gentis sibi commissae . \\\hline
1.2.8 & e del pueblo & Sed ratione gentis quam dirigit , \\\hline
1.2.8 & e de su pueblo & et aliam gentem , sunt alia , \\\hline
1.2.8 & de ser my prouado conosçiendo las condiconnes particulares de su gente e de su pueblo & et alia exquirenda . Oportet igitur Principem respectu gentis cui praeest , esse expertum , cognoscendo particulares conditiones gentis sibi commissae , \\\hline
1.2.8 & por que pue da meior guiar e gouernar su pueblo e su gente & et alia exquirenda . Oportet igitur Principem respectu gentis cui praeest , esse expertum , cognoscendo particulares conditiones gentis sibi commissae , \\\hline
1.2.8 & e enderesçar su pueblo e su gente ¶ & ad quae debet dirigere gentem sibi commissam . Quomodo Reges , et Principes possunt \\\hline
1.2.14 & Et a esta manera de fortaleza enduzen los pueblos los caudiellos dela hueste & Ad hanc Fortitudinem inducunt populum Duces exercitus , statuentes poenam fugientibus , \\\hline
1.2.14 & e el su pueblo a periglos de batallas & ut non exponant suam gentem periculis bellicis , \\\hline
1.2.16 & Por ende mucho mueuen alos pueblos en sanna contra si . & maxime prouocant alios contra se . \\\hline
1.2.16 & que el pueblo non se leuate en sanna contra ellos & ne furor populi \\\hline
1.2.23 & Ca por esta manera enduziran todo el pueblo & quia hoc modo maxime inducent totum populum sibi commissum , \\\hline
1.2.26 & en las mas cosas pone el su pueblo a periglo . ¶ & ut plurimum periclitator efficitur populorum . \\\hline
1.3.3 & por las quales pue da meior gouernar su pueblo . & per quam possit melius suum populum regere . \\\hline
1.3.3 & e despoian el pueblo &  \\\hline
1.3.5 & Reyeᷤ¶ la segunda de parte dela gente del pueblo quales acomnedado . & ex parte gentis sibi commissae . \\\hline
1.3.5 & e alos prinçipes de parte del pueblo & Secundo hoc decet eos ex parte populi sibi commissi , \\\hline
1.3.9 & por que caran al bien dela gente e del pueblo & eo quod respiciant bonum gentis . \\\hline
1.3.11 & por que sean temidos de lons pueblos e amados . & ut rei cognoscentia postulabit . Ibi enim ostendemus , quomodo Reges et Principes se habere debeant , ut a populis timeantur , et amentur : \\\hline
1.4.6 & por que paresçe segunt la opinion comun de los pueblos & et quaedam dignitas aliorum . Videtur enim esse in communi opinione vulgarium omnia mensurari numismate , \\\hline
1.4.6 & que los omes baxos e de los pueblos & a vulgaribus , \\\hline
2.1.21 & Onde el philosofo en el quarto libro delas ethicas denuesta a aquellos pueblos & unde Philos’ 4 Ethicor’ vituperat Laconios , \\\hline
2.2.7 & que nigun lenguage del pueblo non es conplido nin acabado & nullum idioma vulgare esse completum \\\hline
2.2.7 & Et por esta razd̃ serian tir annos e robadores del pueblo . & Erit ergo Tyrannus , | et populi depraedator . \\\hline
2.2.8 & e de enssennorear al pueblo el qual pueblo non puede entender & inter gentes et dominari populo , | qui non potest percipere \\\hline
2.2.8 & Ca assi commo los legos e los omes del pueblo & Nam sicut laici et vulgares , \\\hline
2.2.18 & e generalmente todo sennor del pueblo commo quier que en lidiando e en tomando armas non vala & licet in bellando et in assumendo arma quasi non plus valeat quam unus homo , \\\hline
2.2.18 & Enpero por la sabidia puede mas valer a todo el pueblo & tamen per prudentiam praeualere potest toti populo sibi commisso . \\\hline
2.2.18 & que por las armas ca todo el pueblo & Nam totus populus , \\\hline
2.3.3 & delas quales la primera se toma de parte dela grandeza real . ¶ La segunda de parte del pueblo & Quarum prima sumitur ex parte magnificentiae regiae . | Secunda ex parte populi . Possumus autem \\\hline
2.3.3 & para prouar esto mismo se toma de parte del pueblo & sumitur ex parte ipsius populi : \\\hline
2.3.3 & que el pueblo que lo viere finque & quod populus ea videns , \\\hline
2.3.3 & por que cada vno del pueblo & quilibet enim de populo hoc viso opinatur principem esse tantum , quod quasi impossibile sit ipsum inuadere : \\\hline
2.3.3 & cada vno del pueblo se guar da & quilibet ex populo retrahitur , \\\hline
2.3.3 & por que non sean auidos en menospreçio del pueblo . & in quo existunt . \\\hline
2.3.8 & que enl pueblo & detestabilius est in Rege non habere ueram aestimationem de fine quam in populo , \\\hline
2.3.8 & por que el pueblo es enderescado e gado por el Rey & eo quod populus a Rege dirigitur , \\\hline
2.3.9 & assi commo plazia de establesçer en aquel tienpo alos pueblos e alos Reyes . & ut placebat tunc temporis populis | et regibus instituere . \\\hline
2.3.17 & e por qua non sean despreçiados de los pueblos conuieneles de fazer grandes & et ne a populis condemnantur , | decet eos magnifica facere , \\\hline
2.3.18 & por que los omes e el pueblo comunalmente non sienten & et communiter populus non percipiunt \\\hline
2.3.18 & assi que non sea memoria en el pueblo que los sus padres nin los sus auuelos fueron pobres & ita quod non sit memoria in populo progenitores suos fuisse pauperes , \\\hline
2.3.18 & que son nobles seg̃t reputaçion del pueblo & secundum reputationem populi \\\hline
3.1.9 & e a todo el pueblo & et toti populo debet esse talis , \\\hline
3.1.9 & que pertenezca a todo el pueblo & quae competat omni populo \\\hline
3.1.9 & e ala uida comunal del pueblo & et communi uitae ; \\\hline
3.1.11 & que el pueblo comunalmente se desuia de carrera derecha & quod communiter populus a via perfecta deuiare , \\\hline
3.1.11 & en quanto el pueblo deue guardar la ley & prout communiter populus observatiuus est legum \\\hline
3.1.19 & la terçera a todo el pueblo . & tertia tantum populum , \\\hline
3.1.19 & quanto a todo el pueblo establesçio & Tertio quantum ad totum populum statuit , \\\hline
3.1.19 & que todo el pueblo tan bien los lidiadores commo los menestrales & ut totus populus uidelicet tam bellatores \\\hline
3.1.19 & mas por el ectiuo la qual elecçion daua a todo el pueblo ¶ & sed per electionem , | quam electionem toti populo attribuebat . \\\hline
3.1.20 & que todo el pueblo escoia el prinçipe non puede estar con el establesçimiento & quod totus populus eligat principem , | stare non potest cum statuto de bellatoribus , \\\hline
3.2.1 & Et el alcalłia ¶ Et el pueblo . & Haec autem sunt princeps , \\\hline
3.2.1 & que por el estado paçifico del pueblo & ut propter pacificum statum populi quae traduntur \\\hline
3.2.1 & que el pueblo ha de guardar & ut debite inuenire possint quae populus obseruare debet . Bene vero iudicare \\\hline
3.2.1 & eston pertenesce a todos los çibdadanos o a todo el pueblo & Sed bene obseruare leges spectat ad omnes ciues , siue ad totum populum . \\\hline
3.2.1 & que parte nesçe a todo el pueblo & quae potest respicere totum populum : populus enim ad bene agendum , \\\hline
3.2.1 & ca el pueblo es de abiuar & quae potest respicere totum populum : populus enim ad bene agendum , \\\hline
3.2.1 & e qual deue ser el pueblo & et qualis debeat esse populus \\\hline
3.2.2 & e la poliçia que quiere dezer pueblo bien enssenoreante son bueons prinçipados . & et politia sunt principatus boni : \\\hline
3.2.2 & Et la democraçia que quiere dez maldat del pueblo & oligarchia , \\\hline
3.2.2 & todo el pueblo romano era gouernado & regebatur totus Romanus populus quibusdam paucis viris : \\\hline
3.2.2 & que los mayores en el pueblo & ut maiores in populo , \\\hline
3.2.2 & e los que deuen gouernar el pueblo son dich sobtimates & et qui debent populum regere vocati sunt optimates , \\\hline
3.2.2 & assi commo todo el pueblo & ut totus populus : ibi enim requiritur consensus totius populi in statutis condendis , in potestatibus eligendis , \\\hline
3.2.2 & ca alli es demandado el consentimiento de todo el pueblo & ut totus populus : ibi enim requiritur consensus totius populi in statutis condendis , in potestatibus eligendis , \\\hline
3.2.2 & Enpero mas enssennore a todo el pueblo & magis tamen dominatur totus populus , \\\hline
3.2.2 & por que a todo el pueblo ꝑtenesçe de escoger le & eo quod totius populi est eum eligere et corrigere , \\\hline
3.2.2 & Et a todo el pueblo pertenesçe de fazer los establesçimientos & etiam eius totius est statuta condere , \\\hline
3.2.2 & assi commo todo el pueblo & ut totus populus , \\\hline
3.2.2 & Enpero el prinçipado del pueblo & Principatus tamen populi \\\hline
3.2.2 & e nos podemos llamar atal prinçipado gouernamiento del pueblo & eo quod non habeat commune nomen , Politia dicitur . Nos autem talem principatum appellare possumus gubernationem populi , \\\hline
3.2.2 & Mas si el pueblo assi enssennoreante non entiende a bien de todos & Sed si populus sic dominans non intendit bonum omnium secundum suum statum , \\\hline
3.2.2 & e en nonbre gniego es dicho democraçia mas nos podemos le llamar destruymiento e desordenamiento del pueblo . & et in graeco nomine dicitur Democratia . | Nos autem ipsum appellare possumus peruersionem populi . \\\hline
3.2.3 & Pues que assi es bueon es el gouernamiento del pueblo o de muchos & Bonum est igitur regimen populi siue multitudinis , \\\hline
3.2.4 & para que bien gouierne su pueblo &  \\\hline
3.2.5 & que gouienna el pueblo ¶ & qui debet in haereditatem succedere . \\\hline
3.2.5 & la terçera de parte del pueblo que deue ser gouernado & qui debet per tale regimen gubernari . \\\hline
3.2.5 & ¶ La terçera razon se toma de parte del pueblo & Tertia via sumitur | ex parte populi \\\hline
3.2.5 & e por ende el pueblo & quasi naturalia . \\\hline
3.2.5 & e mas ligeramente obedezca el pueblo & et facilius obediat populus mandatis regis , \\\hline
3.2.5 & ca el pueblo inclinase naturalmente a obedesçer los mandamientos de tal Rey & quia populus quasi naturaliter inclinatur | ut obediat iussionibus talis Regis . \\\hline
3.2.5 & por hedamiento conuiene alos pueblos & nam si dignitas regia per haereditatem transferatur ad posteros , oportet eam transferre in filios , quia secundum lineam consanguinitatis filii parentibus maxime sunt coniuncti : \\\hline
3.2.6 & ca el pueblo & Nam quia populus non percipit \\\hline
3.2.6 & por que si alguno era atal que feziera bien al pueblo aquella gente inclinada a el & Nam si aliquis fuerat primo beneficus , \\\hline
3.2.6 & por ende aquel que el pueblo tierne & qui a populo creditur virtuosus . \\\hline
3.2.6 & ¶ lo primero que sea amado del pueblo . & ut ametur a populo . Secundo , ut cura peruigili procuret commune bonum . \\\hline
3.2.6 & e alos que turban el pueblo . & et turbantes populum . \\\hline
3.2.6 & seria muy amado del pueblo & Nam si abundet \\\hline
3.2.6 & que parterie al pueblo . & diligetur a populo . \\\hline
3.2.7 & o si enssennoreare todo el pueblo & vel si dominetur totus populus , \\\hline
3.2.8 & que en tal manera sea el pueblo apareiado e ordenado & ut populus taliter disponatur et ordinetur , \\\hline
3.2.8 & que el pueblo sea enderesçado & et deuiantia ab huiusmodi fine . Tertio , ut dirigatur et promoueatur in finem praedictum . \\\hline
3.2.8 & e cada vn gouernador del pueblo es & et quilibet director populi , \\\hline
3.2.8 & Et el pueblo es & populus vero , \\\hline
3.2.8 & e su pueblo a su fin &  \\\hline
3.2.8 & por que el pueblo pueda alcançar su fin & ut populus possit consequi finem intentum \\\hline
3.2.8 & conuiene que de ally tome todo el pueblo algun enssennamiento & oportet \\\hline
3.2.8 & por las quales el pueblo puede alcançar su fin & clarius tamen infra dicetur . Viso quod spectat ad Regis officium solicitari circa ea per quae possit populus consequi finem intentum : \\\hline
3.2.8 & que enbargan al pueblo & huiusmodi prohibentia remouere . Quae \\\hline
3.2.8 & por que el pueblo &  \\\hline
3.2.8 & en qual manera ellos deuengar e enderesçar su pueblo ala fin que entienden . & restat ostendere quomodo eos debeant in finem dirigere . \\\hline
3.2.8 & muchualeni para que el pueblo seagniado derechamente en su fin . & ut populus recte dirigatur in finem , \\\hline
3.2.9 & por qua non sean menospreçiados de los pueblos & ne contemnantur a populis , non deberent suam intemperantiam ostendere : \\\hline
3.2.9 & Ca el pueblo segunt que dize el philosofo es del todo subiecto al Rey & Populus enim ( ut recitat Philosophus ) | omnino est subiectus Regi \\\hline
3.2.10 & ca conosçiendo la mourien el pueblo contra ellos . & ne cognoscentes eorum nequitiam , incitent populum contra ipsos : \\\hline
3.2.10 & e enduzen el pueblo al amor del Rey . & populum commouent ad amorem eius . Tertia , \\\hline
3.2.10 & Ca en todo en todo sia malo el pueblo & quia tunc magis unanimiter diligunt bonum Regis . Omnino enim esset peruersus populus , \\\hline
3.2.10 & que non lon amados del pueblo . & Cum enim tyranni sciant se non diligi a populo , \\\hline
3.2.10 & que el pueblo non se leunate & quam ne populus insurgat in ipsum , \\\hline
3.2.10 & e el pueblo con los nobles & et peruertere . Volunt enim tyranni turbare amicos cum amicis , populum cum insignibus , \\\hline
3.2.12 & Mas quando enssennore a todo el pueblo & Sed si dominatur totus populus et intendat bonum omnium tam insignium quam aliorum , est principatus rectus , et vocatur regimen populi . \\\hline
3.2.12 & e es llamado gouernamiento de pueblo . &  \\\hline
3.2.12 & Mas si el pueblo tiranizare & Si vero populus tyrannizet \\\hline
3.2.12 & que tanto quiere dezer commo corrupçion e maldat del pueblo . Et pues que assi es la tirania & quod idem est quod quasi peruersio et corruptio populi . Tyrannis vero corruptus principatus diuitum , \\\hline
3.2.12 & Et el sennorio malo del pueblo son tres señorios muy malos . & et iniquum dominium populi , sunt regimina peruersa . Tyrannis \\\hline
3.2.12 & e en el mal señorio del pueblo todo es ayuntado enla tirania & et in peruerso dominio populi , | totum congregatur in tyrannidem . \\\hline
3.2.12 & si non aguauiando el pueblo en muchͣs cosas . & ad pecuniam et voluptates corporis , nisi in multis offendat populum : \\\hline
3.2.12 & Et por ende veyendo se aborresçido del pueblo & Videt ergo se esse odiosum populo , \\\hline
3.2.12 & e dela ira del pueblo . & sed semper dubitans de furia populi , \\\hline
3.2.12 & que los tiranos de mala e de robo del pueblo & ex praedatione populi . \\\hline
3.2.12 & e ser muy amado del pueblo es muy delectable & et diligi a populo , | est maxime delectabile : \\\hline
3.2.12 & e aborresçido dela muchedunbre del pueblo & et credere se odiosum esse multitudini , \\\hline
3.2.12 & que es aborresçido delos pueblos Disto & cum videat se esse populis odiosum . Viso tyrannidem cauendam esse , \\\hline
3.2.12 & por que en ella son ayuntados los males del mal priͥnçipado del pueblo & eo quod etiam in ipsa congregantur mala peruersi principatus populi . \\\hline
3.2.12 & por que quando el pueblo enseñorea malamente non entiende guaedar & Cum enim populus principatur peruerse , \\\hline
3.2.13 & por que derechamente e egualmente gouiernen el pueblo & ut recte et debite gubernent populum sibi commissum : \\\hline
3.2.13 & e non gouernar derechamente el pueblo & et non recte gubernare populum : \\\hline
3.2.13 & por que se fazen despreçiados de los pueblos & quia ut plurimum tyranni faciunt ea per quae se contemptibiles reddunt . \\\hline
3.2.14 & mas que gouionen derechamente el pueblo & sed ut recte regant populum sibi commissum : \\\hline
3.2.14 & assi cotio la tirania del pueblo & ut tyrannis populi contrariatur tyrannidi monarchiae : \\\hline
3.2.14 & ca quando algun enparadoro algun prinçipe vno titaniza en el pueblo & Cum enim aliquis monarcha vel aliquis unus Princeps tyrannizet in populum , \\\hline
3.2.14 & Et por ende todo el pueblo es fech & eum perimens vel expellens . Totus ergo populus \\\hline
3.2.14 & la tirama del pueblo corronpe & et tyrannis populi corrumpit tyrannidem monarchiam . Sic \\\hline
3.2.16 & Et el alcalłia Et el pueblo . & consilium , praetorium , et populum . \\\hline
3.2.16 & para que derechamente gouierne el pueblo qual es acomendado . & ut recte regat populum sibi commissum : | probauimus \\\hline
3.2.19 & corporal¶ Lo terçero dela guarda dela çibdat e del pueblo . & de custodia ciuitatis , \\\hline
3.2.19 & e a buen estado del Rey e del pueblo & et ad bonum statum eius : \\\hline
3.2.24 & Mas los establesçimientos de los pueblos e los pleitos de los çibdadanos & Sed statuta populi , | et pacta ciuium , \\\hline
3.2.26 & Mas en quanto es conparada al pueblo & sed ut refertur ad populum ad quem debet applicari \\\hline
3.2.26 & a que es dada el qual pueblo deueser reglado & sed ut refertur ad populum ad quem debet applicari \\\hline
3.2.26 & Ca assi commo dize el pueblo comunalmente & Nam et vulgo dicitur , \\\hline
3.2.26 & en quanto es conparada al pueblo & ad populum cui est imponenda , \\\hline
3.2.26 & qual es el pueblo & qualis sit populus , \\\hline
3.2.26 & que cunple al pueblo tales leyes les deue poner . & tales debet eis leges imponere . \\\hline
3.2.26 & e parte nesçientes al pueblo & utiles | et competens populo , \\\hline
3.2.27 & assi commo de todo el pueblo & ut a toto populo , vel a principante , \\\hline
3.2.27 & o deuen ser establesçidas de todo el pueblo & vel condendae sunt a toto populo , \\\hline
3.2.27 & si todo el pueblo en ssennorea & si totus populus principetur , \\\hline
3.2.27 & es establesçida del prinçipe o de todo el pueblo & vel a toto populo , \\\hline
3.2.27 & si todo el pueblo enssennoreare . & si totus populus principetur ; \\\hline
3.2.27 & Ca el prinçipe o avn todo el pueblo & Princeps enim aut totus populus cum principatur , \\\hline
3.2.27 & Ca puede cada vno del pueblo amonestar e enduzir a otro o a otros & Potest enim quilibet ex populo mouere \\\hline
3.2.27 & que enssennorea amuchedunbre del pueblo & vel a multitudine , \\\hline
3.2.27 & o dela muchedunbre del pueblo & si tota huiusmodi multitudo principetur . \\\hline
3.2.27 & que han de poner al pueblo & non modicum solicitari quas leges imponant populo \\\hline
3.2.28 & e conueninbles al pueblo & et conuenientes populo cui imponuntur leges . \\\hline
3.2.28 & e gouernar ningun pueblo . & vix \\\hline
3.2.30 & la qual cosa contesçe por dos razones La primera es por que el pueblo comunalmente non puede alcançar forma de beuir en punto . & quod duplici de causa contingit . Prima est , | quia communiter populus non potest attingere punctalem formam viuendi , \\\hline
3.2.30 & por que el prinçipado pueda durar en el pueblo . & ad hoc ut possit durare principatus in populo : \\\hline
3.2.32 & Lo quarto qual deue ser el pueblo . & et qualis populus . \\\hline
3.2.32 & Conuiene de saber del pueblo . & scilicet de populo . \\\hline
3.2.32 & para saber que cosa es pueblo . & scire | quid sit ciuitas , \\\hline
3.2.32 & qual deue ser el pueblo & qualis debeat esse populus existens in ciuitate et regno . \\\hline
3.2.32 & los moradores del regno e el pueblo & Nam si ciuitas et regnum principaliter ordinatur \\\hline
3.2.33 & se puede departir cada pueblo & Hoc ergo modo quo diuisa est ciuitas in tres partes , diuidi potest quilibet populus et quodlibet regnum . \\\hline
3.2.33 & si y fuere pueblo establesçido de muchͣs perssonas medianeras & si ibi sit populus ex multis personis mediis constitutus . Tangit autem Philosophus 4 Politicorum , quatuor , \\\hline
3.2.33 & si y fuere pueblo & vel melius esse regnum et ciuitatem , \\\hline
3.2.33 & de aquello que tal pueblo commo este biue & ex eo quod talis populus magis rationabiliter viuit . \\\hline
3.2.33 & Ca si en el pueblo fueren muchs muy ricos & Prima via sic patet . | Nam si in populo sint multi valde diuites , \\\hline
3.2.33 & Mas si en el pueblo fueren muchas perssonas medianeras quedaran todo estos enpeesçimientos & Sed si in populo sint multae personae mediae , | cessabunt huiusmodi nocumenta , \\\hline
3.2.33 & que el pueblo abonde en perssonas medianeras . & inter ciues , bonum est propter hoc abundare in personis mediis . \\\hline
3.2.33 & Por la qual cosasi el pueblo fuere establesçido sola . & Quare si populus sit solum ex duabus partibus constitutus , \\\hline
3.2.33 & Por la qual cosa el pueblo es muy bueno & Quare optimus est populus ex multis personis mediis constitutus , \\\hline
3.2.34 & uanto alo presente parte nesçe el pueblo alcança tres bienes & Consequitur autem populus | ( quantum ad praesens spectat ) tria , \\\hline
3.2.34 & quanto es prouechoso e conuenible al pueblo de obedesçer alos Reyes & quantum sit utile et expediens populo obedire Regibus et Principibus , et obseruare leges . \\\hline
3.2.34 & Ca lo primero desto alçança el pueblo uirtudes e grandes bienes & Primo enim | ex hoc consequitur populus virtutes , \\\hline
3.2.34 & Por la qual cosa si el prinçipe gouernar e derechamente el pueblo & Quare si principans | recte regat populum sibi commissum , \\\hline
3.2.34 & Et por ende con muy grant acuçia deue estudiar el pueblo & vel quam sanitas corporum . Summo ergo opere studere \\\hline
3.2.34 & Ca avn puesto que en alguna cosa tira nizen deue estu diar avn el pueblo de obedesçer los . & sed etiam dato quod in aliquo tyrannizarent , studeret populus obedire illis . \\\hline
3.2.35 & es pues que enssennamos al pueblo & Postquam docuimus populum , \\\hline
3.2.36 & por que sean amados del pueblo . & ut amentur a populo , \\\hline
3.2.36 & que para que los Reyes e los prinçipes comunalmente sean amados del pueblo deuen auer en ssi tres cosas prinçipalmente . & Sciendum itaque quod ut Reges et Principes communiter amentur a populo , \\\hline
3.2.36 & ca el pueblo non siente & nisi sensibilia bona , \\\hline
3.2.36 & que el pueblo ama & quod populus amat , \\\hline
3.2.36 & para que los Reyes sean ama dos del pueblo es que deuon ser fuertes e de grandes coraçones & Secundo ut Reges amentur in populo , | debent esse fortes et magnanimi , \\\hline
3.2.36 & Ca el pueblo muchama los fuertes & Nam populus valde diligit fortes \\\hline
3.2.36 & por que cuyda sienpre el pueblo &  \\\hline
3.2.36 & para que los Reyes sean amados del pueblo & et cordatos . Tertio , ut Reges diligantur a populo , \\\hline
3.2.36 & Ca el pueblo mayormente se le una taria a mal querençia del Rey & Nam maxime prouocatur populus ad odium Regis , \\\hline
3.2.36 & por que sean amados del pueblo . fiça deuer en qual manera se de una auer & videre restat , \\\hline
3.2.36 & porque sean temidos del pueblo . & videre restat , \\\hline
3.2.36 & Et pues que assi es cada vno del pueblo teme de mal fazer cuydando & Timet igitur tunc quilibet ex populo forefacere , \\\hline
3.2.36 & de ser amados de los pueblos & et quod amore boni , \\\hline
3.2.36 & e que los pueblos fagan bien & populi bene agant , \\\hline
3.3.5 & que el pueblo aldeano & Constat autem ruralem populum habere maxime praedicta . \\\hline
3.3.5 & que la fortaleza del pueblo . & quam corporis fortitudo . \\\hline
3.3.6 & que subiugaron todo el mundo al pueblo de los romanos . & et industriam bellandi fuisse ea , quae terrarum orbem Romano Populo subiecerunt . \\\hline
3.3.7 & por el pueblo de roma non cuydaua vençer en otra manera a los enemigos & non aliter contra hostes se obtinere credebat , \\\hline
3.3.10 & que todo el pueblo de la çibdat fue vençido de pocos lidiadores . & et strenuum bellatorem . Contigit enim nostris temporibus totum populum ciuitatis cuiusdam deuictum esse a bellatoribus paucis , \\\hline
3.3.11 & Et por ende alli do paresçen la muerte del pueblo & Ubi ergo quaeritur mors populi , \\\hline
3.3.19 & tanto mas es espantado el pueblo & tanto plus terrentur obsessi , \\\hline
3.3.20 & e la sanera del pueblo & aut Domini metuentes furorum populi , volunt se tueri in munitione aliqua : \\\hline

\end{tabular}
