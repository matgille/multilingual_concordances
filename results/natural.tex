\begin{tabular}{|p{1cm}|p{6.5cm}|p{6.5cm}|}

\hline
1.1.1 & en quanto la naturaleza dessa mismͣ cosa lo demanda & inquantum natura rei recipit . Videtur enim natura \\\hline
1.1.1 & Ca semeja la naturaleza Dela sçiençia moral del todo ser contraria ala sçiençia matematica & rei moralis omnino esse opposita negocio mathematico . \\\hline
1.1.2 & e de naturaleza primeramente es de rrazon & quae sunt ad alterum , \\\hline
1.1.2 & por orden natural & In ipsis etiam speculabilibus ordine naturali semper ex imperfecto ad perfectum procedimus , \\\hline
1.1.2 & Segund orden natural ala rreal magestad primeramente & quanta in gubernatione ciuitatis et regni : ordine naturali decet regiam maiestatem \\\hline
1.1.2 & Mas los mançebos son naturalmente liƀales & et auari : iuuenes vero sunt naturaliter liberales , \\\hline
1.1.3 & Et estos son sotileza del entendemiento e engennjo natural & huiusmodi sunt industria mentis , ingenium naturale , potentiae animae : \\\hline
1.1.3 & Et naturalmente mas es sieruo & magis tamen est dignus subiici quam principari , \\\hline
1.1.4 & Ca el omes es naturalmente medianero entre las bestias & est autem homo naturaliter medius \\\hline
1.1.4 & por que el omne es natural . & quia homo \\\hline
1.1.4 & naturalmente aina la conpanable çibdadano & ( ut ibi probatur ) est naturaliter animal sociale , ciuile , \\\hline
1.1.4 & ca creyeron que qual quier ome naturalmente syn otra ayuda . & crediderunt enim , | quod ex puris naturalibus absque alio auxilio gratiae posset \\\hline
1.1.10 & que es naturalmente caliente & qui est naturae calidae , \\\hline
1.1.11 & Et sienpre la forma es mas natural dela cosa & et semper forma magis dicit naturam rei , \\\hline
1.1.12 & e non lo ha conplidamente es instrumento de aquel que la ha naturalmente e conplidamente . & est instrumentum et organum eius , | quod habet illud essentialiter et perfecte , \\\hline
1.1.12 & que el amor tan bien natural & quod amorem , | siue diuinum , \\\hline
1.2.1 & ca alguons destos poderios del alma son naturales & quia potentiae animae quaedam sunt naturales , quaedam cognitiuae sensitiuae , quaedam appetitiuae , \\\hline
1.2.1 & e algunos intellectiuos e entendedores ¶ Naturales poderios son aquellos & et quaedam intellectiuae . Naturales potentiae sunt illae , in quibus communicamus cum vegetabilibus , \\\hline
1.2.1 & Mas que en los poderios naturales & vel in aliquibus horum . In potentiis autem naturalibus esse non possunt , \\\hline
1.2.1 & La primera es que por las cosas naturales non somos alabados nin denostados . & quia ex naturalibus nec laudamur , \\\hline
1.2.1 & por que en tales poderios naturales non han de seer las uirtudeses . & in talibus non habent esse virtutes : \\\hline
1.2.1 & segund que son naturales & ( secundum quod huiusmodi ) \\\hline
1.2.1 & por que enlos poderios naturales non pueden seer las disposiçonnes & in talibus potentiis non sunt habitus , \\\hline
1.2.1 & e los poderios naturales sean determinandos conplidamente & et potentiae naturales sufficienter determinentur ad agendum , \\\hline
1.2.1 & por la qual cosa no pueden ser las uirtudes en los poderios naturales & quare in eis huiusmodi virtutes esse non possunt . \\\hline
1.2.1 & que las uirtudes non son en los poderios naturales & per quas probatum est virtutes non esse in potentiis naturalibus , \\\hline
1.2.1 & por los poderios naturales & nec vituperamur ex potentiis naturalibus : sic non laudamur , nec vituperamur ex sensibus . \\\hline
1.2.1 & porque assi commo los poderios naturales non obedesçen a la razon . & quia sicut potentiae naturales non obediunt rationi , \\\hline
1.2.1 & Ca assi commo los poderios naturales son conplidamente determinados a sus obras por naturaleza . & quia sicut potentiae naturales sufficienter determinantur ad actiones suas per suam naturam : \\\hline
1.2.1 & por su naturaleza . & sic et sensus . \\\hline
1.2.1 & Et estos poderios naturales e senssibles son determinados & haec autem sufficienter determinantur ad actiones proprias per naturam : \\\hline
1.2.1 & por su naturaleza a sus obras & haec autem sufficienter determinantur ad actiones proprias per naturam : \\\hline
1.2.1 & si las uirtudes morales non son en los poderios naturales & nec in sensibus est virtus moralis , \\\hline
1.2.1 & nin en los sesos naturales . & cum praeter potentias naturales , \\\hline
1.2.1 & Commo sin estos poderios naturales e sesos naturales & cum praeter potentias naturales , \\\hline
1.2.2 & ues ya es mostrado que las uirtudes ¶ morales non son en los poderios naturales nin en los sesos . & Ostenso , | quod nec in potentiis naturalibus , \\\hline
1.2.2 & assi commo la ph̃ia natural e la geometria e la methaphisica & ut naturalis Philosophia , Geometria , Metaphysica , et caetera talia . Virtutes vero simpliciter morales sunt illae , quae sunt in appetitu , siue appetitus ille sit sensitiuus , siue intellectiuus , \\\hline
1.2.2 & mas los poderios naturales & nec sunt rationales per essentiam , \\\hline
1.2.2 & Ca assi commo el appetito natural sigue a su forma naturalmente auida . & Nam sicut appetitus naturalis sequitur formam naturaliter adeptam , \\\hline
1.2.2 & e vna inclinaçion natural & sequitur ea quaedam naturalis inclinatio , \\\hline
1.2.2 & por que dessean natural . & et quidam appetitus naturalis , \\\hline
1.2.2 & que el fuego naturalmente es caliente e liuiano & Videmus enim quod ignis naturaliter est calidus , \\\hline
1.2.7 & que en sennor ee natural . mente . & ne suus principatus in tyrannidem conuertatur . Tertio studere debet , \\\hline
1.2.7 & la terçera que sin la pradençia non puede seer senor naturalmente¶ & quia sine ipsa non potest naturaliter dominari . \\\hline
1.2.7 & nin enssennorear natural mente . & quia sine ea non possunt naturaliter dominari . \\\hline
1.2.7 & que por esso es dicho alguno naturalmente sieruo & ex hoc est aliquis naturaliter seruus , \\\hline
1.2.7 & Et por ende es dicho alguno naturalmente señor & Ex hoc autem naturaliter est Dominus , \\\hline
1.2.7 & e confirman la todos los gouernamientos naturales . & sed etiam confirmant singula regimina naturalia . Videmus enim naturaliter homines dominari bestiis , \\\hline
1.2.7 & Ca uehemos que los omes son naturalmente sennores delas bestias & sed etiam confirmant singula regimina naturalia . Videmus enim naturaliter homines dominari bestiis , \\\hline
1.2.7 & Ca los omes naturalmente sen sennores delas bestias & Senes pueris . Homines naturaliter dominantur bestiis , \\\hline
1.2.7 & Et por ende por la mayor parte la muger deue ser naturalmente subietta al ome & Ergo foemina viro naturaliter debet esse subiecta , \\\hline
1.2.7 & por que naturalmente fallesçe dela sabiduria del omne & eo quod naturaliter deficiat a viri prudentia . \\\hline
1.2.7 & que naturalmente sean subiectos de los mas antigos & Hoc etiam modo iuuenes naturaliter decet antiquioribus esse subiectos , \\\hline
1.2.7 & pues que assi es do quier que ay mengua de sabiduria ay naturalmente seruidunbre . & et illud natureliter dominatur , semper principans pollet prudentia , a qua deficit qui naturaliter seruus existit . \\\hline
1.2.7 & Et do quier que ay sabiduria ha naturalmente sennorio . & et illud natureliter dominatur , semper principans pollet prudentia , a qua deficit qui naturaliter seruus existit . \\\hline
1.2.7 & por cuya mengua el sieruo esta naturalmente en su seruidunbre . & quod polleat prudentia , et intellectu . Quot , \\\hline
1.2.7 & por que naturalmente sea sennor conuiene & si \\\hline
1.2.8 & que es enxerida naturalmente alos omes . & qui est inditus hominibus , \\\hline
1.2.9 & naturalmente deuen cuydar primero en los tpons passados & excogitando primo tempora retroacta , \\\hline
1.2.11 & e canda vna comunidat semeia avn cuerpo natural . & et quaelibet congregatio assimilatur cuidam corpori naturali . Sicut enim videmus corpus animalis constare ex diuersis membris connexis , \\\hline
1.2.11 & el cuerpo natural non podria estat & corpus naturale non posset subsistere . Sic \\\hline
1.2.11 & por sabiduria natural & sine qua corpus naturale durare non posset : \\\hline
1.2.11 & assi conmo el cuerpo natural no podria estar & et bona tribuere . Sicut ergo corpus naturale | non subsisteret , \\\hline
1.2.11 & que assi commo cada vna desigualdat non tuelle la uida del cuerpo natural & quod sicut non quaelibet inaequalitas tollit vitam corporis naturalis , \\\hline
1.2.13 & Et naturalmente cada vno fuye dela tristoza & ut plurimum tristia sunt . Tristia autem naturaliter quilibet fugit , sicut naturaliter sequitur delectabilia . \\\hline
1.2.13 & assi commo naturalmente sigue las cosas delectables . & ut plurimum tristia sunt . Tristia autem naturaliter quilibet fugit , sicut naturaliter sequitur delectabilia . \\\hline
1.2.13 & Et pues que assi es commo nos natural mente fuyamos dela tristeza & Cum ergo naturaliter tristia fugiamus , \\\hline
1.2.17 & Ca cada hun omne es naturalmente inclinado a amar asi mismo & quia unusquisque naturaliter inclinatur | ut se diligat , \\\hline
1.2.17 & e naturalmente amamos & et naturaliter afficimur ad illa . Immo auari adeo afficiuntur \\\hline
1.2.18 & Ca si el regno de cada vno dios Reyes deue ser natural & Nam si regnum alicuius debet esse naturale , \\\hline
1.2.18 & que nos veemos en la nataleza e en las cosas naturales & debet his quae videmus in natura . In naturalibus autem nihil est ociosum , \\\hline
1.2.18 & por la qual razon commo la naturaleza de los omes se tenga por pagada de pocas cosas & cum natura humana modicis contenta sit , \\\hline
1.2.19 & Mas commo en cada cosa mas e menos non fagan departimiento en la naturaleza & Sed cum magis , | et minus non videantur diuersificare speciem , \\\hline
1.2.27 & por lo que ha razon de se ensannar . Ca natural cosa es anos & Nam naturale est nobis \\\hline
1.2.27 & Et non solamente naturalmente somos inclinados & non solum naturaliter inclinamur , \\\hline
1.2.27 & Mas avn en alguna manera natural cosa esa nos de dessear & sed etiam quodammodo naturale est nobis appetere punitionem ultra condignum . \\\hline
1.2.28 & Ca si el omne es naturalmente animalia aconpanable & Si enim homo est naturaliter animal sociale , \\\hline
1.2.29 & La primera por que cada vno naturalmente es Inclinado a querer su bien propio & Quilibet enim ita naturaliter afficitur | ad proprium bonum , \\\hline
1.2.31 & O en quanto son naturales & vel ut sunt naturales , \\\hline
1.2.31 & Mas lasuirtudes naturales e non conplidas & Virtutes autem naturales et imperfectae , \\\hline
1.2.31 & por que veemos alguons naturalmente auer alguna n industria & Videmus enim aliquos naturaliter habere quandam industriam , \\\hline
1.2.31 & por que han algpradençia natural empero non son liberales Et pues que assi es en esta manera las uirtudes & non tamen liberales existunt . Sic ergo virtutes non connectuntur . \\\hline
1.2.31 & que es aꝑcebimiento natural . & et industria , \\\hline
1.2.33 & que por prinçipios puros naturales podriemos escusar todos los males & violentium quod ex puris naturalibus possemus | omnia mala vitare , \\\hline
1.3.2 & mas conosçida anos la naturaleza delas passiones & quia ex omnibus his magis innotescit nobis natura ipsarum passionum , \\\hline
1.3.2 & e en figua a la qual naturaleza conosçida poremos conosçer & qua cognita , \\\hline
1.3.3 & assymismo en bondat . La razon natural muestra & naturalis ratio \\\hline
1.3.3 & por inclinacion natural & ex naturali enim instinctu cum quis vult percuti , \\\hline
1.3.3 & Et esto por inclinaçion natural pone el braço a periglo &  \\\hline
1.3.4 & Ca los fechͣs e las obras morales son semeiables en alguna manera alas cosas naturales . & Nam gesta moralia quodammodo rebus naturalibus sunt similia . \\\hline
1.3.4 & Ca assi commo los cuerpos naturales & Nam sicut corpora naturalia per suas formas , \\\hline
1.3.4 & Ca assi commo en las cosas naturales beemos & Sic enim in naturalibus videmus , \\\hline
1.3.4 & Empero naturalmente el mouimiento ayuso toma medida e manera dela pesadura & naturaliter | tamen motus deorsum sumit mensuram \\\hline
1.3.4 & es entendida la salud del cuerpo natural . & sicut in arte medicandi principaliter intenditur sanitas corporis : naturaliter decet Reges \\\hline
1.3.5 & mucho la calentura natural & nam quia nimis abundat in eis calor , \\\hline
1.3.6 & por que quando alguno teme la calentura natural & Cum enim quis timet , calor ad interiora progreditur ; modum enim , \\\hline
1.3.6 & podemos lo ueer en la calentura del cuerpo natural . & aspicere possumus in calore corporis naturalis . \\\hline
1.3.6 & quando alguno teme la calentura natural & vel ad arcem : \\\hline
1.3.6 & Ca por el temor la calentura natural tornase alos mienbros de dentro e los mienbros de fuera fincan frios . & Nam propter timorem calore progrediente ad interiora , \\\hline
1.4.1 & por que en ellos abonda mucho calentura natural . & quia in eis multum abundat calor : \\\hline
1.4.1 & por la calentura natural & corde ergo et aliis membris inflammatis ex calore existente in ipsis iuuenibus , fiunt iuuenes bonae spei , \\\hline
1.4.1 & e segunt cursso natural deuen much beuir en el tienpo & et | secundum cursum naturalem debent multum viuere in futuro . \\\hline
1.4.1 & por la calentura natural desse an sobrepuiar & qui percalidi nimis affectent excellere , \\\hline
1.4.2 & Por ende la natural disposiçion del cueꝑpo mueue & naturalis dispositio corporis incitat iuuenes ad concupiscentias venereorum . \\\hline
1.4.2 & Et pues que assi es commo natural cosa sea & Cum ergo naturale sit , \\\hline
1.4.3 & qual quier que es inclinado naturalmente a alguna passion & Quicunque naturaliter sic disponitur , \\\hline
1.4.3 & Et qual si quier que naturalmente es frio naturalmente es temeroso & quicunque est naturaliter frigidus , sequitur \\\hline
1.4.3 & Et por ende siguese que los uieios son naturalmente temerosos . Ca fallesçe enellos la calentura natural & quod sit naturaliter formidolosus . Sequitur ergo senes esse naturaliter timidos , | quia deficit in eis naturalis calor , \\\hline
1.4.3 & e han los mienbs naturalmente frios ¶ & et habent membra naturaliter frigida . \\\hline
1.4.3 & e en la calentura natural & et in calore naturali : sic deficiunt in animo , \\\hline
1.4.4 & Et esto es contra la razon natural del frio . & Hoc autem est contra rationem frigidi . \\\hline
1.4.4 & por que segunt cuisu natural & quia \\\hline
1.4.4 & e aquellos que son en estado medianero han alguna inclinacion natural alas costunbres & et illi qui sunt in statu , | quandam pronitatem , et inclinationem habent \\\hline
1.4.4 & por que assi commo los vieios e los mançebos maguera ayan disposiconn e indinacion natural ha costunbres malas & quia sicut senes , | et iuuenes habent quandam pronitatem naturalem , \\\hline
1.4.4 & e inclina conn natural & possunt tamen contra illam pronitatem facere consequi laudabiles mores . \\\hline
1.4.4 & e fazer contra esta disposiçion natural & possunt tamen contra istam pronitatem facere , \\\hline
1.4.5 & Ca natural cosa es & Naturale est enim , \\\hline
1.4.5 & por que los fijos son fechuras de los padron natural cosa es & cum filii sint quidam effectus parentum , \\\hline
1.4.5 & Ca natural cosa es & Naturale est enim , \\\hline
1.4.5 & e los prinçipes non puedan naturalmente ensseñorear & cum non possint naturaliter dominari , \\\hline
2.1.1 & que naturalmente es fecha todas aquellas cosas le son naturales & quod naturaliter fit , | naturalia sunt ea , \\\hline
2.1.1 & si las cosas naturales en ninguna manera non se pudiessen guardar & si res naturales nullo modo conseruarentur in esse , \\\hline
2.1.1 & assi mesmo en la uida son colas naturales al omne & quae faciunt ad bene viuere , et sine quibus non potest sibi in vita sufficere , sunt homini naturalia : \\\hline
2.1.1 & Et por ende naturalmente el omne esaianlia aconpanable e conpanera & naturaliter ergo homo est animal sociabile . \\\hline
2.1.1 & e las o tristales dales gouernamiento delas otrasaian lias que naturalmen te son engendradas . & et caeteris talibus , administratur nutrimentum ex caeteris animalibus , | quae naturaliter producuntur . \\\hline
2.1.1 & que auemos mester El omne es naturalmente con panno & quo indigemus , | homo est naturaliter animal sociale . \\\hline
2.1.1 & que han natural uestido & quasi naturale indumentum , habere videntur lanam , vel pennas . Homini autem non sufficienter prouidet natura in vestitu : \\\hline
2.1.1 & siguese que el o omne ha natural inclinamiento & quod homo naturalem impetum habeat \\\hline
2.1.1 & para guardar la uida natural delos omes & Sed si haec sunt necessaria ad conseruandam hominis naturalem vitam , \\\hline
2.1.1 & e en comunidat es en alguna manera natural alos omes ¶ & et in societate est quodammodo homini naturale . \\\hline
2.1.1 & Par la qual cosa si natural cosa es al omne de dessear conseruaçion & Quare si naturale est homini desiderare conseruationem vitae , \\\hline
2.1.1 & por ende natural cosa es ael & naturale est ei , ut desideret viuere in communitate , \\\hline
2.1.1 & que deuen por inclinaçion natural sin ninguno otro ensseñamiento primero & Nam alia animalia sufficienter inclinantur ad opera sibi debita ex instinctu naturae absque introductione aliqua praecedente : \\\hline
2.1.1 & por inclinaçion natural alas obras qual conuienen & ad opera sibi debita , \\\hline
2.1.1 & Por ende natural cosa es al omne de beuir con los otros omes & naturale est homini simul conuiuere cum aliis , \\\hline
2.1.1 & que tanne por las quales praeua el omne es naturalmente aial aconpannable & quas tangit , probantes hominem naturaliter esse sociale animal , potissime innititur huic rationi , videlicet , \\\hline
2.1.1 & mas naturalmente es aial aconpanable & sequitur hominem magis naturaliter esse animal sociale , \\\hline
2.1.1 & assi es cosa natural al omne & Quare si sic naturale est , hominem esse animal sociale : recusantes societatem , \\\hline
2.1.2 & Et pues que assi es natural fazimiento e comienço del uarrio & Naturalis ergo origo vici , \\\hline
2.1.2 & que esta generaciones muy natural & ad praesens vero in tantum dictum sit , quod huiusmodi generatio est maxime generalis : \\\hline
2.1.3 & en alguna manera esta comunidat dela cała es natural & de leui videri potest , | quomodo sit huiusmodi communitas naturalis . \\\hline
2.1.3 & que dela cosa naturales presunpuesta e antepuesta non puede ser propreamente cosa artifiçial & non proprie quid artificiale erit , | sed oportet illud \\\hline
2.1.3 & mas conuiene que aquello en quanto es tal sea cosa natural & ( secundum quod huiusmodi est ) | naturale esse . \\\hline
2.1.3 & Por la qual cosa si el omne es naturalmente ai al comiuncable & Quare si homo est naturaliter animal communicatiuum \\\hline
2.1.3 & conuiene quala comunidat dela casa o la casa sea cosa natural . & oportet communitatem domesticam siue domum | quid naturale esse . \\\hline
2.1.3 & ca es comunidat en alguna manera natural & et ut cognoscant quae et qualis est communitas domus \\\hline
2.1.4 & que el omne es naturalmente ainalia domestica e de casa & cum ostensum sit quod homo est naturaliter animal domesticum , \\\hline
2.1.4 & e quela comunidat dela casa es en alguna manera natural . & et quod communitas domus est quodammodo naturalis . \\\hline
2.1.4 & segunt natura e natural de suso es prouado gruesamente e figuaalmente & secundum naturam , superius grosse et figuraliter probabatur , \\\hline
2.1.4 & e prouaremos que cada vna ꝑtetal dela casa es cosa natural . & quod quaelibet talis pars est aliquid naturale . \\\hline
2.1.4 & Ca es comunidat natural e establesçida & quia est communitas naturalis constituita propter opera diurnalia et quotidiana . \\\hline
2.1.5 & mas adelante es cosa natural . & est quid naturale . \\\hline
2.1.5 & Ca prinçipalmente cosa natural es la generaçion delas cosas & Maxime autem quid naturale esse videtur , \\\hline
2.1.5 & et commo las cosas naturales resçiban su naturaleza propra a & et cum res naturales per generationem propriam naturam accipiant , \\\hline
2.1.5 & que la generaçion es cosa natural & generationem esse quid naturale , \\\hline
2.1.5 & Ca en vano seria algua cosa engendrada naturalmente & Nam frustra esset aliquid naturaliter generatum , \\\hline
2.1.5 & fazen ser la casa cosa natural . & quid naturale : \\\hline
2.1.5 & e la conseruaçion es cosa natural & et serui ad conseruationem . Quare si generatio et conseruatio est quid naturale , \\\hline
2.1.5 & conuiene que la casa sea cosa natural ¶ & oportet domum quid naturale esse . Amplius , \\\hline
2.1.5 & Por que naturalmente los sennores han mayor sabiduria & nam naturaliter domini vigent prudentia , et intellectu . \\\hline
2.1.6 & Ca ueemos en las cosas naturales & Videmus enim in naturalibus rebus quod statim quum generatae sunt , \\\hline
2.1.6 & por que non puede la cosa natural & quia non statim cum est res naturalis , \\\hline
2.1.6 & non pertenesçe a cosa natural tomada & non est \\\hline
2.1.6 & mas pertenesçe a cosa natural & de ratione rei naturalis quocumque modo , sumptae ; sed est de ratione eius , \\\hline
2.1.6 & Et pues que assi es si la casa es cosa natural & Si ergo domus est quid naturale , \\\hline
2.1.6 & a razones naturales & reducere volumus in naturales causas , \\\hline
2.1.6 & segunt orden natural & propter quod ordine naturali quae videmus in multitudine domestica , \\\hline
2.1.6 & e dela fructificaçion natural & et fructificationis naturalis . \\\hline
2.1.6 & para prouar esto mesmo se toma de acabamiento de comunindat natural duradera & sumitur ex parte naturalis perpetuitatis . \\\hline
2.1.6 & Por que la morada continuada dela casa en vna manera es natural & ut si per creationem filiorum domus ista continue habitetur . \\\hline
2.1.6 & assi comm̃ natural casual e auentura & Alio modo est quasi casualis , \\\hline
2.1.6 & Et pues que assi es la morada natural dela casa & Naturalis ergo habitatio domestica naturaliter perpetuari non potest , \\\hline
2.1.6 & non puede durar naturalmente & Naturalis ergo habitatio domestica naturaliter perpetuari non potest , \\\hline
2.1.6 & naturalmente en alguna manera puede durar & ubi desunt filii , \\\hline
2.1.7 & Et que el omne naturalmente esaianl coniugable e ayuntable en mater moino . & et quod homo naturaliter est animal coniugale . Sciendum ergo quod Philosophus 8 Ethic’ volens ostendere qualis amicitia sit viri \\\hline
2.1.7 & que el omne sea naturalmente aianl conuigable¶ & secundum naturam : adducens triplicem rationem quod homo sit naturaliter animal coniugale . \\\hline
2.1.7 & que el omne es naturalmente aina l aconpannable e comun incatiuo & hominem esse naturaliter animal sociale et communicatiuum . Communitas autem in vita humana \\\hline
2.1.7 & Et el omne naturalmente es & homo naturaliter magis est animal de mesticum , \\\hline
2.1.7 & que es natural al omne & quam ciuile : et communitas domus magis videtur esse naturalis ipsi homini , \\\hline
2.1.7 & mas es natural al omne & quin communitas domus magis sit naturalis homini , \\\hline
2.1.7 & Ca aquella cosaparesçe ser mayormente natural & Nam illud maxime videtur naturale , \\\hline
2.1.7 & ala qual el omne ha natural inclinaçion & ad quod homo habet naturalem impetum : \\\hline
2.1.7 & por la qual cosa commo todas las ainalias naturalmente sean inclinadas & et omnia animalia naturaliter inclinentur , \\\hline
2.1.7 & por ende el omne es naturalmente aianl conuuigable e ayuntable por casamiento . & homo naturaliter est animal coniugale . \\\hline
2.1.7 & por que natural cola es al omne & quia naturale est homini , \\\hline
2.1.7 & e atondas las aianlias auer natural inclinaçion & et omnibus animalibus , \\\hline
2.1.7 & Por la qual cosa si natural cosa es al omne de auer inclinaçion & Quare si naturale est homini , \\\hline
2.1.7 & e appetito al abastamiento dela uida natural cosa es a el de querer ser a i al conuigable & habere impetum ad sufficientiam vitae : | naturale est ei , \\\hline
2.1.7 & mas si el ma termonio es cosa natural siguese & Sed si coniugium est \\\hline
2.1.7 & assi conmo aquella que es contraria ala cosa natural . & tanquam aliquid contrarium rei naturali : \\\hline
2.1.7 & Ca si el casamiento es al omne natural & Nam si coniugium est homini naturale , \\\hline
2.1.7 & Ca si natural cosa es al omne & Nam si naturale est homini esse animal coniugale , \\\hline
2.1.8 & o de parte dela amistança natural & vel ex parte amicitiae naturalis , \\\hline
2.1.8 & Ca commo entre el uaron e la muger sea amistança natural & et econuerso . Cum enim inter virum et uxorem sit amicitia naturalis , \\\hline
2.1.8 & commo non sea natural amistan & cum non fit naturalis amicitia \\\hline
2.1.8 & e para que entre el uaron e la muger sea amistança natural conuiene que guarden vno a otro fe e lealtad & et ad hoc quod inter uxorem et virum sit amicitia naturalis , | oportet quod sibi inuicem seruent fidem , \\\hline
2.1.8 & por que naturalmente aman a sus fijos & qui naturaliter diligunt suam prolem , ex dilectione naturali quam habent ad ipsam , \\\hline
2.1.8 & por el amor natural & qui naturaliter diligunt suam prolem , ex dilectione naturali quam habent ad ipsam , \\\hline
2.1.8 & que han con ellos acresçientase entre ellos amorio natural & augmentatur eorum amicitia naturalis . \\\hline
2.1.9 & delas ethicas entre ellos es amistança muy grande e muy natural . & est amicitia excellens | et naturalis . \\\hline
2.1.9 & Ca commo el matermonio sea cosa natural & Nam cum coniugium sit | quid naturale : \\\hline
2.1.9 & non abaste la fenbra sola natural cosa es alos omes & naturale est hominibus \\\hline
2.1.9 & Ca nos deuemos iudgar las cosas naturales & Ea enim naturalia iudicare debemus \\\hline
2.1.9 & que natural cosa es al omne & ut naturale est homini quod sit dexter , \\\hline
2.1.9 & Empero por que non es iudgar la cosa natural & quia tamen naturale non est iudicandum illud quod est in paucioribus , \\\hline
2.1.9 & que es cosa natural & reliquum est ut in hominibus iudicetur quid naturale , \\\hline
2.1.9 & e dela madre sea cosa natural & naturale sit \\\hline
2.1.9 & que en los omes le acola natural & ut unus masculus uni adhaereat foeminae , sequitur in hominibus esse quid naturale , \\\hline
2.1.9 & Ca alas otras ain alias conplidamente la naturales apareia su uianda & quia eis natura sufficienter parat victum , \\\hline
2.1.9 & e alos prinçipes de segnir mas orden natural & Et tanto magis hoc decet Reges et Principes , \\\hline
2.1.10 & Ca en el mater moino primeramente es guardada la orden natural . & omnino detestabile esse unam foeminam nuptam esse pluribus viris . In coniugio enim primo reseruatur ordo naturalis : \\\hline
2.1.10 & e aguarda dela orden natural & ordinatur enim coniugium non solum ad conseruationem ordinis naturalis , \\\hline
2.1.10 & e por esto se tolleria la orden natural & Tolletur enim ex hoc naturalis ordo ; \\\hline
2.1.10 & que por esto se tire la orden natural & Quod autem ex hoc tollatur naturalis ordo , \\\hline
2.1.10 & Ca segunt la orden natural & Nam | secundum ordinem naturalem \\\hline
2.1.10 & Otrossi segunt la orden natural en essas mismas obras & secundum ordinem naturalem in eisdem operibus nullus aeque per se duobus \\\hline
2.1.10 & Et por ende contradize ala orden natural & et esse sub ipso repugnat ergo ordini naturali eundem duobus esse subiectum . \\\hline
2.1.10 & segunt que son muchos non puede ser segunt orden natural . & ut plures sunt , \\\hline
2.1.10 & por que en el casamiento dellos conuiene de guardar la orden natural & quia in eorum coniugio magis quam in alio decet naturalem ordinem conseruare . \\\hline
2.1.11 & Ca commo por la orden natural deuamos auer subiectiuo al padre e ala madre & Nam cum ex naturali ordine debeamus parentibus debitam subiectionem , \\\hline
2.1.11 & Mas en tanto esto paresçe conuenible a razon natural & hoc naturali rationi consentaneum , \\\hline
2.1.11 & se con razon natural saca algunas perssonas & sola ratione naturali ductus exceptuat personas aliquas a contractione connubii : \\\hline
2.1.11 & quanto mas conuiene a ellos de guardar la orden natural & quanto magis eos obseruare decet ordinem naturalem . \\\hline
2.1.11 & Por ende la razon natural dize que los matermonios non son de fazer entre estos & inter illos \\\hline
2.1.11 & assi es commo ayan natural amor las perssonas & Cum ergo ad personas nimia affinitate coniunctas habeatur naturalis amor , \\\hline
2.1.12 & por que el omne naturalmente es aian la conpannable ama termoino & eo quod homo naturaliter esse animal sociale : prima autem naturalis societas ( ut patet per Philosophum in Polit’ ) \\\hline
2.1.12 & ca la primera con pama natural & eo quod homo naturaliter esse animal sociale : prima autem naturalis societas ( ut patet per Philosophum in Polit’ ) \\\hline
2.1.12 & si asi el casamiento non fuesen ordenado a algua conpanna conuenible e natural . & nisi coniugium ordinaretur in quandam societatem debitam | et naturalem . \\\hline
2.1.13 & Ca es cosa natural & quia est quid naturale , \\\hline
2.1.14 & e mas natural . & magis totale et naturale : regimen vero politicum est magis paternale et ex electione . \\\hline
2.1.14 & Otrossi commo quier que todo gouernamiento si es derechurero sea natural & quantumcunque sit rectum , | non est adeo naturale , \\\hline
2.1.14 & nin escogen el padre para si mas naturalmente son egendrades del padre . & nec eligunt sibi patrem , | sed naturaliter producuntur ab ipso . \\\hline
2.1.14 & assi commo el omne naturalmente es despuesto a fablar & sicut homo naturaliter est aptus ad loquendum , \\\hline
2.1.14 & que la orden e la razon natural muestra . & et ratio naturalis . \\\hline
2.1.15 & que los mouimientos naturales & Impetus naturales \\\hline
2.1.15 & ¶ Et pues que assi es non es orden natural & Non est ergo naturalis ordo , virum praeesse uxori eo regimine \\\hline
2.1.15 & mas vn omne era naturalmente barbaro e sieruo . & sed idem est esse naturaliter barbarum | et seruum ; \\\hline
2.1.15 & e de entendemiento sea naturalmente sieruo & Sed cum carens rationis usu sit naturaliter seruus , \\\hline
2.1.15 & este tal es naturalmente barbaro e sieruo & idem est esse natura barbarum et seruum . \\\hline
2.1.15 & e conosçer la manerar la orden natural & et cognoscere modum | et ordinem naturalem ; \\\hline
2.1.15 & Et pues que assi es de parte de la orden natural paresçe que otra cosa es el gouernamiento del marido ala mug̃r & Ex parte igitur ordinis naturalis patet aliud esse regimen coniugale quam seruile : \\\hline
2.1.16 & por que assi commo veemos en las otras cosas naturales & Sic enim videmus in aliis rebus naturalibus , \\\hline
2.1.17 & Et dizen que esto otorgan tan bien los naturales & et ait , | quod tam a naturalibus , \\\hline
2.1.17 & Otrossi los poros abiertos salle la calentura natural . & apertis poris exalat naturalis calor , \\\hline
2.1.17 & por la calentura natural se ençierran de dentro por el frio qual çerca de fuera & quia calor eius interius propter frigus circunstans non exalat , \\\hline
2.1.17 & por que non salle dellos la cal entra a natural & quia non exalat inde calor ; \\\hline
2.1.18 & Ca commo las muger ssean naturalmente temerosas & Nam cum mulieres sint naturaliter adeo timidae , \\\hline
2.1.22 & por que sea entre ellos amistança natural delectable e honesta & inter eos amicitia naturalis delectabilis , et honesta . \\\hline
2.1.24 & e dela mug̃ sea natural & quomodo communitas viri et uxoris sit naturalis , \\\hline
2.2.1 & de ser amistança natural ¶ & ex eo quod inter eos debet esse amicitia naturalis . Prima via sic patet . \\\hline
2.2.1 & Ca assi commo veemos en las cosas naturales & Nam sicut in naturalibus rebus aspicimus \\\hline
2.2.1 & e los fijos naturalmente han el ser de los padres . Conuiene alos padres de auer cuydado de los fijos & et filii naturaliter a patribus esse habent , | decet patres habere curam filiorum , \\\hline
2.2.1 & por que naturalmente las cosas de suso & videmus enim super caelestia corpora influere \\\hline
2.2.1 & Por la qual cosasi natural cosa es & Quare si naturale est , \\\hline
2.2.1 & por que los padres enssenore a naturalmente alos fijos & Patres ergo | eo ipso quod naturaliter praesunt filiis , \\\hline
2.2.1 & Conuiene que los padres por amor natural & decet patres ex ipso amore naturali , \\\hline
2.2.2 & Ca natural cosa es que cada vno ame sus obras & naturale est enim quemlibet diligere sua opera , \\\hline
2.2.2 & Onde los padres naturalmente aman los fijos & ut Philosophus in Ethicorum unde et patres naturaliter diligunt filios , et poetae sua poemata tanquam proprium opus . \\\hline
2.2.2 & e generalmente todos los señores sy de una naturalmente ensseñorear conuiene les & et uniuersaliter omnes dominantes , | si debeant naturaliter dominari , \\\hline
2.2.3 & La primera razon se torna de la orden natural & paternale regimen trahere originem ex amore . Prima via sumitur ex ordine naturali . \\\hline
2.2.3 & por que el fijo naturalmente es vria semerança que desçende del cadre . & quia filius naturaliter est quaedam similitudo procedens a patre : \\\hline
2.2.3 & assi en ellos es natural appetito & naturalis impetus ad eos diligendum , \\\hline
2.2.4 & Ca si entre los padres e los fijos es amor natural & Nam si inter parentes et filios est amor naturalis , \\\hline
2.2.4 & Et non es natural cosa que las de diyuso &  \\\hline
2.2.6 & La primera se toma dela naturaleza dela delectacion & Prima via sumitur | ex naturalitate delectationis . Secunda , \\\hline
2.2.6 & en tanto es natural anos de nos delectar enla ninnes & adeo connaturale est nobis delectari , \\\hline
2.2.6 & Et paues que assi es dela naturaleza e dela delectaçion paresçe & ex ipsa ergo connaturalitate delectationis , statim cum pueri sunt sermonum capaces , \\\hline
2.2.7 & Ca assi conmodicho es de suso ninguno non es dich̃ sennor naturalmente & Nam ( ut superius dicebatur ) nullus est naturaliter dominus , \\\hline
2.2.7 & e mas natural mente . & et prudentiores esse , \\\hline
2.2.8 & por ende segunt que dize este mismo philosofo la musica es conueinble ala naturaleza de los mançebos por que les muestra delectaçicen sin danno ¶ & quare ( secundum eundem Philosophum ) musica est consentanea naturae iuuenum , | quia habent innocuas delectationes . \\\hline
2.2.8 & Ca la natural ph̃ia & Nam Naturalis Philosophia docens cognoscere naturas rerum , \\\hline
2.2.8 & Et la sçiençia dela fisica es sola ph̃ia natural . & sub naturali Philosophia . \\\hline
2.2.8 & delas otras sçiençias falladas por los omes . Et en pos estos deuen ser mas honrrados los philosofos naturales & inter scientias humanitus inuentas metaphysica primatum tenet . Post hos quidem honorari debent naturales Philosophi : \\\hline
2.2.8 & por que lph̃ia natural commo quier que sea a quande dela methasisica . & quia naturalis Philosophia licet sit infra metaphysicam , tenet \\\hline
2.2.11 & por que mas liga mente passasse la calentura natural ala bianda & facilius pateretur a calore naturali , \\\hline
2.2.11 & Mas esta orden natural en la mayor parte & Hunc autem ordinem naturalem , \\\hline
2.2.11 & conuiene que sea bien proporçionada ala calentura natural & Si enim cibus digeri debeat , oportet ipsum esse proportionatum calori naturali . \\\hline
2.2.11 & si en tan grand quantia se tomaque la calentura natural non pueda enssennorar & Quare si in tanta quantitate sumatur , quod calor naturalis ei dominari non possit , non bene digeritur , \\\hline
2.2.11 & Ca todas las obras naturales & Nam omnes actiones naturales \\\hline
2.2.13 & assi los mouimientos naturales ni obra assi & nec sic agit ex naturali instinctu , \\\hline
2.2.13 & por que ha menos de calentura natural de dentro & eo quod magis caret calore naturali intrinseco , \\\hline
2.3.2 & Ca si la conpania dela casa es cosa natural & Si enim societas domus est quid naturale , \\\hline
2.3.5 & por tres razons la possession delas cosas es natural en algua manera al omne & quod rerum possessio est quodammodo naturalis . \\\hline
2.3.5 & assi casi los omes biuen naturalmente & Si enim homines naturaliter viuunt , \\\hline
2.3.5 & e la conpanna dela çibdat es en alguna manera natural al omne & et societas politica est quodammodo homini naturalis , \\\hline
2.3.5 & conuiene en algunan manera que las cosas naturales sean neçessarias enla uida politica . & oportet aliquomodo naturalia esse | quae sunt necessaria in vita politica ; \\\hline
2.3.5 & en alguna manera es natural al omne ¶ & est quodammodo homini naturalis . \\\hline
2.3.5 & por ende han señorio natural sobrellas . & habet naturale dominium super ipsa : \\\hline
2.3.5 & por la qual cosa natural cosa es al ome & quare naturale est homini \\\hline
2.3.5 & do prueua que la possession de tales cosas es natural . & ubi probat possessionem talium naturalem esse , \\\hline
2.3.5 & dize que naturalmente es batalla derecha de los omes contra las bestias & ait , | quod hominum ad bestias naturaliter est iustum bellum : \\\hline
2.3.5 & assi commo el omne naturalmente enssennorea alas bestias & Sicut ergo homo naturaliter dominatur bestiis , \\\hline
2.3.5 & avn en essa misma manera naturalmente enssennorea a todas las otras cosas de fuera . & sic et naturaliter dominatur aliis exterioribus rebus ; \\\hline
2.3.5 & si la possession delas cosas en algua manera non fuesse natural al omne ¶ & nisi rerum possessio quodammodo naturalis esset . \\\hline
2.3.5 & mas naturalmente apareia el nudmiento conuenible a ellas & sed naturaliter praeparat eis debitum nutrimentum ; congruum est \\\hline
2.3.5 & e por ende el sennorio delas cosas de fuera es en algua manera natural al omne . & Habere ergo dominium rerum exteriorum est | quodammodo homini naturale : \\\hline
2.3.5 & assi es cosa natural & est homini naturale : \\\hline
2.3.6 & viij̊ libro delas ethicas es amistança natural & inter quos secundum Philosophum 8 Ethicorum est amicitia naturalis , \\\hline
2.3.7 & naturalmente deuen ser subietos alos omes & naturaliter debent esse subiecti hominibus pollentibus subtilitate et prudentia . \\\hline
2.3.7 & que los sabios de una ser sennorsnaturalmente de los non sabios & quia sapientes naturaliter debent dominari insipientibus , \\\hline
2.3.8 & segunt manera natural non deue dessear possessiones nin riquezas sin mesura e sin fin . & et ordinem naturalem , | non debet infinitas diuitias \\\hline
2.3.10 & e entgo es abondar en riquezas e possessiones naturales & est abundare in diuitiis naturalibus : \\\hline
2.3.10 & delas quales nasçen las riquezas naturales & ex quibus oriuntur diuitiae naturales , annectitur tractatur de numismatibus , quae sunt diuitiae artificiales . \\\hline
2.3.10 & pone quatro maneras de dineros conuiene saber . Natural . & Distinguit autem Philos’ in Poli’ quatuor species pecuniatiuae : \\\hline
2.3.10 & por aquello que las cosas naturales se mudan en dineros . & quasi naturalis : | quae fit ex eo quod res naturales commutantur in pecuniam . \\\hline
2.3.10 & e entrago el qual resçibiesse naturalmente destas cosas dineros . tales dineros e tal riqueza serie dichͣ & quae naturaliter producuntur , | ex eis pecuniam susciperet , \\\hline
2.3.10 & assi commo natural & talis pecuniatiua | quasi naturalis diceretur , \\\hline
2.3.10 & por razon que ha comienco delas cosas naturales . & quia a rebus naturalibus inciperet . Secunda species pecuniatiuae dicitur esse campsoria : haec enim \\\hline
2.3.10 & e Mas esta arte pecumatiua de dineros non deue ser dicha natural & Haec enim pecuniatiua , | naturalis dici non debet ; \\\hline
2.3.10 & por que non com . iença de cosas naturales & quia nec a rebus naturalibus incipit , \\\hline
2.3.10 & nin se termina en cosas naturales & nec ad naturalia terminatur . \\\hline
2.3.10 & assi commo yconomica es natural & quae est quasi oeconomica et naturaliter , \\\hline
2.3.10 & e assi commo natural & quae est oeconomica et quasi naturalis , decet . Decet enim ipsos abundare in possessionibus \\\hline
2.3.11 & es cosa proprea en las cosas naturales & est proprium naturalibus , \\\hline
2.3.11 & por que son cosas naturales & quia sunt res naturales , \\\hline
2.3.11 & por que son contrael derecho natural & ne fiant eo quod iuri naturali contradicant . \\\hline
2.3.13 & que alguna suidunbre es dichͣ natural & ut de seruis . Ostendemus enim primo seruitutem aliquam naturalem esse , \\\hline
2.3.13 & que alg ssean subietos naturalmente a algunos otros & et quod naturaliter expedit aliquibus aliis esse subiectos : \\\hline
2.3.13 & que nunca algunas cosas muchas fazen naturalmente alguna cosa & quod nunquam aliqua plura constituunt naturaliter aliquid unum , \\\hline
2.3.13 & Et commo la conpannia de los omes sean a falpor el ome es aianlia aconpannable naturalmente & quia homo est naturaliter animal sociale , \\\hline
2.3.13 & mas conplidamente nunca de mucho omes se faria naturalmente vna conpannia o vna poliçia & numquam ex pluribus hominibus fieret naturaliter una societas vel una politia , \\\hline
2.3.13 & si naturalmente non fuesse a ellos conuenible & nisi naturale esset aliquos principari \\\hline
2.3.13 & Et pues que assi es algunos son natraalmente sieruos e alguas naturalmente senores ¶ & Sunt ergo aliqui naturaliter domini , \\\hline
2.3.13 & assi commo el cuerpo al alma siguesse que aquellos sean naturalmente sieruos Et por que algunos son menguados de entendimiento e de sabideria & ad alias quasi corpus ad animam , sequitur eos esse naturaliter seruos . Sunt enim aliqui carentes prudentia et intellectu , \\\hline
2.3.13 & assi como es diche de suso naturalmente enssennorea & ut supra dicebatur ) naturaliter dominatur bestiis . Videmus enim multas bestias domesticas , \\\hline
2.3.13 & e es cosa natural & et naturale est ipsis subiici homini ; \\\hline
2.3.13 & Por ende assi commo natural cosa es alas bestias de seruir alos omes & sic naturale est ignorantes subiici prudentibus expedit enim eis sic esse subiectos , \\\hline
2.3.13 & assi cosa naturales & ut per eorum industriam dirigantur \\\hline
2.3.13 & que algunos sean naturalmente subietos e sieruos & sequitur eos naturaliter esse subiectos . \\\hline
2.3.13 & Pot la qual cosa la piudunbre es en alguna manera cosa natural & Quare seruitus est aliquo modo quid naturale , \\\hline
2.3.13 & Et natural mente conuiene ala conpannia human & et naturaliter expedit societati humanae aliquos seruire , \\\hline
2.3.14 & que lin la hudunbre natural & quod praeter seruitutem naturalem , \\\hline
2.3.14 & Ca cada vno de los omes es naturalmente establesçido & Nam homo quilibet naturaliter ex his duabus partibus est compositus . \\\hline
2.3.14 & non fazen sennorio natural sinplemente . & non faciunt dominium simpliciter naturale , \\\hline
2.3.14 & e esto non es derech natural & non est iustum naturale , \\\hline
2.3.15 & ca algunos son tałs naturalmente . & Nam aliqui sunt tales naturaliter , \\\hline
2.3.15 & de ser senno res dellos natural mente . & secundum quorum excessum contingit aliquos naturaliter dominari . \\\hline
2.3.15 & Mas conuiene de dar a ministraçion de alquiler e de amor sin la ministt̃ion natural et segunt ley . & praeter ministrationem naturalem | et secundum legem : \\\hline
2.3.15 & non guardamos sienpre la orden natural . & et non semper reseruamus ordinem naturalem , \\\hline
2.3.19 & ca algunos son sieruos naturalmente & quia quidam sunt serui naturaliter , quidam ex lege , \\\hline
2.3.19 & Et pues que assi es commo alos que son sieruos naturalmente & Cum seruis ergo naturaliter non sunt communicanda secreta neque consilia : \\\hline
2.3.19 & nunca alguno es dicho naturalmente & nunquam est quis naturaliter seruus , \\\hline
2.3.20 & desto que tal cosa commo esta contradize ala orden natural . & etiam omnium ciuium abundare eloquiis . Prima via sumitur ex eo quod hoc repugnat ordini naturali . \\\hline
2.3.20 & e por que el vno non enbargue al otro contra natural ordenes & et ne unum impediat aliud , | contra naturalem ordinem est \\\hline
2.3.20 & Et por ende contra orden natural es & et locutionem contra naturalem ordinem est , \\\hline
2.3.20 & que es el fablar contradize ala orden natural . & repugnat ergo hoc ordini naturali . \\\hline
2.3.20 & e guardar la orden natural en toda meranera & quos decet maxime temperatos esse , et obseruare ordinem naturalem omnino in suis mensis , \\\hline
2.3.20 & por que non sea tirada la ordenn natural & ne tollatur naturalis ordo , \\\hline
3.1.1 & e amouemiento naturala aquel bien . & aliquando ad bonum illud habemus impetum a natura , \\\hline
3.1.1 & por que han natural inclinaçion han establesçimiento della & eo quod habent naturalem impetum | ad constitutionem eius , \\\hline
3.1.1 & Ca commo quier que toda comun dar natural sea ordenada a bien & Nam licet omnis communitas naturalis ordinetur \\\hline
3.1.3 & si la çibdat es cosa natural & An ciuitas sit aliquid | secundum naturalem ? \\\hline
3.1.3 & por la qual cosa si la çibdat fuesse alguna cosa natural & Quare si ciuitas quid naturale esset , \\\hline
3.1.3 & e el omne naturalmente fuesse aianl & et homo naturaliter esset animal ciuile , \\\hline
3.1.3 & assi natural al omne conmoes cosa natural al fuego de escalentar & sicut est naturale igni calefacere , \\\hline
3.1.3 & assi parte nesçen naturalmente al fuego & quia talia sic eis naturaliter competunt quod ad contrarium assuefieri non possunt , \\\hline
3.1.3 & assi naturalmente aianl çiuil & homo ergo non sic naturaliter est animal ciuile , \\\hline
3.1.3 & mas es dicho qual conuiene naturalmente de seraianl ciuil & sed dicitur ei hoc naturaliter conuenire , \\\hline
3.1.3 & e algun appetito natural & et quandam aptitudinem naturalem , \\\hline
3.1.3 & que assi son naturales & Quae autem sic sunt naturalia , \\\hline
3.1.3 & que maguera natural cosa sea al omne & ut licet naturale sit homini esse dextrum , \\\hline
3.1.3 & assi maguera natural cosa sea al omne de beuir çiuil mente . & sic licet naturale sit homini viuere ciuiliter , \\\hline
3.1.4 & que la çibdat non era cosa natural & ciuitatem non esse aliquid | secundum naturam , \\\hline
3.1.4 & e que el oen non era naturalmente aianlçiuil . & et hominem non esse naturaliter animal ciuile . \\\hline
3.1.4 & que muestren que la çibdat es cosa natural & adducere rationes ostendentes ciuitatem esse \\\hline
3.1.4 & e que el omne es naturalmente aianlçiuil & quid naturale , \\\hline
3.1.4 & que beuir es cosa natural al omne & quod uiuere erat homini | secundum naturam ; \\\hline
3.1.4 & conuiene que sea cosa natural todo aquello & oportet quid naturale esse quicquid \\\hline
3.1.4 & e avn del uartio son cosas naturales & et etiam uici naturalia sunt , \\\hline
3.1.4 & lo que es fin dela generaçion delas cosas naturales & ut arguit Philosophus primo Politicorum quod est finis generationis naturalium , \\\hline
3.1.4 & es cosa natural & est quid naturale , \\\hline
3.1.4 & assi commo lo que es fin dela generaçio del omne es cosa natural al omne & ut quod est finis generationis hominis est | quid naturale , \\\hline
3.1.4 & mas ueemos que la comuidat dela casa es cosa natural & et est ipsa natura . Videmus autem quod communitas domus est \\\hline
3.1.4 & que acaban la casa son cosa natural . & quae perficiunt domum , \\\hline
3.1.4 & bien assi avn la comunidat del uarrio es cosa natural & Sic etiam communitas vici est quid naturale , \\\hline
3.1.4 & por la qual cosa si tal acresçentamiento es cosa natural & propter quod si tale crementum est naturale , \\\hline
3.1.4 & que el uarriosa cosa natural & vicus ipse quid naturale erit . \\\hline
3.1.4 & que es fin de la generaçion dela casa e del vairio sera cosa natural & quae est finis generationis domus | et vici erit \\\hline
3.1.4 & por que es fin dela generaçion de cosas naturales & secundum naturam , | eo quod sit finis generationis naturalium : \\\hline
3.1.4 & que la çibdat es cosa natural . & reliquum est ostendere , \\\hline
3.1.4 & finca de demostrar que el omne es naturalmente aianl politicas & hominem esse naturaliter animal politicum et ciuile , \\\hline
3.1.4 & La segunda de parte dela inclinaçion natural & Secunda ex parte impetus naturalis . Probatur enim in principio secundi libri , \\\hline
3.1.4 & que el omne es naturalmente aian la conpannable & et disciplinam . \\\hline
3.1.4 & e la comunidat dela çibdat sean cosas naturales & et ciuilem esse quid naturale . \\\hline
3.1.4 & ca si la natura dio al omne palabra natural aquella comunidat & Nam si natura dedit homini sermonem , \\\hline
3.1.4 & que sea natural & iustum enim et iniustum non proprie habet esse in communitate domestica , \\\hline
3.1.4 & por quela palabra anos en dada naturalmente & eo quod per sermonem nobis datum a natura repraesentatur conferens \\\hline
3.1.4 & para prouaͬes pose toma de parte dela inclinacion natural & sumitur ex parte impetus naturalis . \\\hline
3.1.4 & ca todas las ainalias han natural inclinaçon & Nam omnia animalia habent naturalem impetum ad conseruandum ea quae sunt eis a natura tributa : \\\hline
3.1.4 & por la qual cosa si la natura dio al omne el beuir diol natural inclinaçion & quare si natura dedit homini viuere , | dedit ei naturalem impetum \\\hline
3.1.4 & e por ende en todos los omes es inclinaçion natural & quae ad vitam sufficiunt . Inerit ergo hominibus impetus naturalis \\\hline
3.1.4 & mas commo aquello a que auemos inclinaçion natural sea cosa natural & Sed cum id , | ad quod habemus impetum naturalem , \\\hline
3.1.4 & conuiene quela çibdat sea cosa natural & secundum naturam , | oportet ciuitatem esse quid naturale , \\\hline
3.1.6 & e cada vna dlłas es en alguna manera natural & quorum quilibet est aliquo modo naturalis , \\\hline
3.1.6 & enpero la vna es mas natural & aliter tamen est naturalior altero . \\\hline
3.1.6 & mas la primera es mas natural que la segunda & sed primus est naturalior secundo . Prima enim constitutio ciuitatis \\\hline
3.1.6 & e del regno es natural en dos maneras & et regni est dupliciter naturalis . Primo enim naturalis est , \\\hline
3.1.6 & ¶La primera es natural & et regni est dupliciter naturalis . Primo enim naturalis est , \\\hline
3.1.6 & ¶ la segunda es natural & quae est opus naturae . Secundo naturalis existit , \\\hline
3.1.6 & por que los omes naturalmente han inclinaçion a establesçer çibdat e regno & quia homines naturalem habent impetum ad constituendam ciuitatem | et regnum . \\\hline
3.1.6 & non solamente por que los omes han natural inclinacion atal establesçimiento & quia homines habent naturalem impetum ad talem constitutionem , \\\hline
3.1.6 & e dela çibdat es natural & ut cum ex concordia hominum \\\hline
3.1.6 & commo quier que non sea tanna tal commo la primera enpero es natural & licet non sit adeo naturalis | ut prima attamen naturalis est , \\\hline
3.1.6 & Avn en essa misma manera han natural inclinaçion & in qua reperiuntur sufficientia ad vitam . Sic etiam habent naturalem impetum , \\\hline
3.1.6 & que les quieren mal fazer et esta tal inclinaçion es natural & et magis resistere hostibus volentibus impugnare ipsos . Est enim huius impetus naturalis : \\\hline
3.1.6 & e del tegno delas quales cada vna puede ser dichͣ natural & quorum quilibet dici potest naturalis : \\\hline
3.1.7 & ueyendo muy grant guaueza cerca la sciençia natural & videns circa naturalem scientiam magnam difficultatem esse , \\\hline
3.1.7 & que segunt la orden natural & secundum ordinem naturalem in quo communicamus \\\hline
3.1.7 & Et mayormente la çibdat sera bien gouernada si semeiare ala orden natural & et maxime bene regitur ciuitas si imitatur ordinem naturalem : \\\hline
3.1.14 & o ca natural cosaes & Nam semper bonum commune praeponendum est bono priuato : \\\hline
3.1.14 & Pues que assi es sin orden natural biue todo çibdadano & praeter ergo ordinem naturalem agit quilibet ciuis , \\\hline
3.1.19 & por la calentura natural & et cibo propter calorem naturalem consumentem huiusmodi radicale . \\\hline
3.2.3 & si cada vna delas cosas naturales fueren penssadas & Si igitur singula naturalia considerentur , \\\hline
3.2.5 & ca naturalmente ha cada vno amor assi mismo & naturaliter autem quilibet habet amicitiam ad seipsum : \\\hline
3.2.5 & assi commo natural . & efficit quasi dominium naturale . Litigia enim sedat , \\\hline
3.2.5 & que el ssennorio sea natural . & etiam hoc dominium naturale , \\\hline
3.2.5 & ca el pueblo inclinase naturalmente a obedesçer los mandamientos de tal Rey & quia populus quasi naturaliter inclinatur | ut obediat iussionibus talis Regis . \\\hline
3.2.5 & que muchs males nasçen enlas çibdades e en los regnos do non ay algun sennor natural ca contesçe & ea aliquando diu carere gubernatore , \\\hline
3.2.7 & La segunda se toma por razon que tal sennorio es muchodes natural ¶ & Secunda , | ex eo quod est maxime innaturale . \\\hline
3.2.7 & por razon que tal sennorio es muy desnatural & Nam illa est naturalis operatio erga aliquid , \\\hline
3.2.7 & por que aquella es obra natural & Nam illa est naturalis operatio erga aliquid , \\\hline
3.2.7 & por la qual cosa el regno estonçe es naturalmente gouernado & quare tunc regnum naturaliter agitur , \\\hline
3.2.7 & e ha razon estonçe es naturalmente gouernado & tunc naturaliter regitur \\\hline
3.2.7 & mas deue ser dicho desnatural . &  \\\hline
3.2.7 & tanto mas deue ser dich des natural . & debet dici in naturale : tyrannis igitur est pessima , \\\hline
3.2.8 & qual es el su ofiçio deue penssar con grant acuçia en las cosas naturales & quod sit eius officium : | diligenter considerare debet in naturalibus rebus . \\\hline
3.2.8 & que veemos enlas cosas naturales deue descender el gouernamiento & quod videmus in naturalibus , deriuari debet regimen , \\\hline
3.2.8 & Ca el arte semeia mucha la natura . Et en las cosas naturales & est enim ars imitatrix naturae . \\\hline
3.2.8 & lo terçero las cosas naturales &  \\\hline
3.2.8 & naturalmente una a sus terminos o a ssus fines & Tertio per huiusmodi collata naturaliter intendunt in suos fines siue in suos terminos . \\\hline
3.2.8 & por aquellas cosas que resçibe dela natura naturalmente sube arriba . & per ea quae accepit a natura , | naturaliter tendit sursum . \\\hline
3.2.8 & e natural tres cosas son meciester ¶ & et naturale , \\\hline
3.2.8 & Ca lo omes son en ssi mismos naturalmente corruptibles & Nam homines in seipsis sunt naturaliter corruptibiles : \\\hline
3.2.8 & dessean naturalmente de durar en sus fijos . & naturaliter appetunt perpetuari in suis filiis siue sint naturales siue adoptiui . \\\hline
3.2.8 & si quier sean naturales siquier por fuados . & naturaliter appetunt perpetuari in suis filiis siue sint naturales siue adoptiui . \\\hline
3.2.10 & Ca commo el entienda enl bien de los çibdadanos natural cosaes & nam cum intendat bonum ipsorum ciuium et subditorum , | naturale est \\\hline
3.2.10 & por que non fia delos sus naturales & eo quod diffidat de illis , \\\hline
3.2.15 & Et aquell sennor fuere natural de antiguo tienpo & et dominus ille sit naturalis , \\\hline
3.2.16 & li le fazen naturalmente & si fiunt a natura . Ideo de imbribus quae semper fiunt tempore hyemali , \\\hline
3.2.16 & por que tales cosas commo estas son naturales & quia talia naturalia sunt , \\\hline
3.2.24 & Et alguon es natural . & quoddam est commune , quoddam proprium ; quoddam est naturale , quoddam legale siue positiuum . \\\hline
3.2.24 & que los iuristas apartan el derecho natural del derecho delas gentes & Illo ergo modo quo iuristae separant | ius naturale a iure gentium , \\\hline
3.2.24 & podemos nos apartar el derecho natural del derecho delas animalias & possemus separare nos ius naturale a iure animalium : \\\hline
3.2.24 & Conuiene a saber ende recħ natural & dicendo quod quadruplex est ius , \\\hline
3.2.24 & Conuiene a saber natural e positiua puesta por omne . & vel duplex est lex , naturalis , \\\hline
3.2.24 & Ca los derechsson dichos naturales & vel duplex est lex , naturalis , \\\hline
3.2.24 & por que la razon natural los muestra & quae sunt adaequata et proportionata ex natura sua , vel dicuntur iusta naturaliter quae dictat esse talia ratio naturalis , \\\hline
3.2.24 & o por que auemos natural apetito o natural inclinaçion & vel ad quae habemus naturalem impetum et inclinationem . \\\hline
3.2.24 & que el derecho natural es departido del derechpo sitiuo . & quod ius naturale dicitur differre a positiuo : \\\hline
3.2.24 & ca el derech natural & quia naturale \\\hline
3.2.24 & por que al derech natural conuinio anneder derecho positiuo es esta & quare iuri naturali oportuit superaddere positiuum , est : quia multa sunt sic iusta naturaliter , \\\hline
3.2.24 & por que muchas cosas son derechas naturalmente & quare iuri naturali oportuit superaddere positiuum , est : quia multa sunt sic iusta naturaliter , \\\hline
3.2.24 & assi commo natural cosa es al ome de fablar & sicut est naturale homini loqui : \\\hline
3.2.24 & ca auemos natural apetito e natural inclinacion para fablar & habemus enim naturalem impetum | et naturalem inclinationem ut loquamur , \\\hline
3.2.24 & assi commo fablar es cosa natural alos omes & non pati maleficos viuere , \\\hline
3.2.24 & e o tristales cosas son de derechnatural & quia haec esse fienda dictat ratio naturalis , \\\hline
3.2.24 & por que la razon natural muestra que se deuen fazer . & quia haec esse fienda dictat ratio naturalis , \\\hline
3.2.24 & Et auemos natural inclinaçion & et habemus naturalem impetum \\\hline
3.2.24 & Ca natural cosa es &  \\\hline
3.2.24 & Et pues que assi es natural cosa es & naturale est ergo talia punire . \\\hline
3.2.24 & Et pues que assi es en aquel logar ose termina el derecho natural & Ubi ergo terminatur ius naturale , \\\hline
3.2.24 & e por sabiduria delos omes presurone el derecho natural & et industriam hominum adinuentum praesupponit ius naturale , \\\hline
3.2.24 & Por la qual cosa si el derecho natural manda & Quare si ius naturale dictat fures \\\hline
3.2.24 & e dos diferençias entre el derech natural e el positiuo . & inter ius naturale , | et positiuum . \\\hline
3.2.24 & La primera es que el derech natural & Prima est , \\\hline
3.2.24 & Por ende las leyes naturales & quia sic se offerunt intellectui nostro naturales leges , \\\hline
3.2.24 & e aquello que es del derecho natural & quod est de iure naturali simpliciter dicitur esse scriptum in cordibus nostris . \\\hline
3.2.24 & que non han ley natural & Nam gentes quae legem non habent , \\\hline
3.2.24 & tan bien el natural commo el positiuo se puede escͥuir en algun libro & Potest itaque utrunque ius scribi in aliqua exteriori substantia tam naturale quam positiuum : | naturale \\\hline
3.2.24 & Enpero non es tanto meester de se escuir el natural commo el positiuo & tamen non sic indiget | ut scribatur \\\hline
3.2.24 & por que non pue de assi caer dela memoria el natural commo el positiuo ¶ & sicut positiuum , | nam non sic potest a memoria recedere \\\hline
3.2.24 & al derecho natural & appellat ius naturale aetherem siue ignem , \\\hline
3.2.24 & que los otros helementos semeia al derecho natural & assimilatur iuri naturali , \\\hline
3.2.25 & de los quales el vn mienbro se contenie so el derecho natural & quarum unum membrum continebatur sub iure naturali , \\\hline
3.2.25 & si aquellas cosas son del derech natural . & Si igitur ea sunt de iure naturali , ad quae habemus naturalem impetum et inclinationem : \\\hline
3.2.25 & a que auemos natal inclinaçion esta inclinaçion natural & Si igitur ea sunt de iure naturali , ad quae habemus naturalem impetum et inclinationem : \\\hline
3.2.25 & assi es dich derech natural . & sic dicitur esse ius naturale . Ideo in Instituta , \\\hline
3.2.25 & Et por ende en la instituta del derecho natural & sic dicitur esse ius naturale . Ideo in Instituta , \\\hline
3.2.25 & dize el derecho natural enssenna a todas las ainalias . & quod ius naturale , | est quod natura omnia animalia docuit . \\\hline
3.2.25 & Et pues que assi es segunt esto de derecho natural & Secundum hoc ergo est de iure naturali coniunctio maris \\\hline
3.2.25 & assi commo vn derecho natural mas espeçial . & Ius ergo gentium est quoddam ius naturale contractum . \\\hline
3.2.25 & e que sigue la nuestra inclinaçion natural & et quod sequitur inclinationem nostram naturalem ut communicamus \\\hline
3.2.25 & es dicho derecho natural . & respectu iuris gentium dicitur esse naturale . \\\hline
3.2.25 & sobredicho el derecho natural & ius naturale est quid commune , \\\hline
3.2.25 & tanto mas meresçe de auer nonbre derech natural . Et por ende aquel derecho & tanto magis meretur nomen iuris naturalis . | Ius ergo , \\\hline
3.2.25 & es dicho ser derecho natural & naturale dicitur , \\\hline
3.2.25 & en quanto el omne es aianl conuiene con las naturales delas otras aianlias . & conuenit cum naturis aliorum animalium , | sic in quantum viuit \\\hline
3.2.25 & Et por ende la inclinacion natural puede seguir la natura del ome & et cum entibus omnibus . Poterit ergo inclinatio naturalis sequi naturam hominis \\\hline
3.2.25 & Ca el omne naturalnse te dessea ser guardado en su ser & Nam homo naturaliter appetit conseruari in esse , \\\hline
3.2.25 & que el ome naturalmente dessea ser . & quod homo naturaliter appetit esse : \\\hline
3.2.25 & assi tales reglas podran ser de derecho natural & sic huiusmodi regulae poterunt esse de iure naturali , \\\hline
3.2.25 & Mas si aquellas reglas se tomaren en quanto el omne naturalmente dessea fazer fiios e carlos & Si vero regulae | illae sumantur ex eo quod homo naturaliter appetit filios producere et educare : \\\hline
3.2.25 & assi podria ser de derecho natural & sic esse poterunt de iure naturali , \\\hline
3.2.25 & en quanto el derecho naturales dicho ser & prout ius naturale dicitur esse , \\\hline
3.2.25 & en quanto el omne naturalmente dessea beuir en conpannia & prout homo naturaliter appetit in societate viuere \\\hline
3.2.25 & assi seran de derecho natural & sic erit de iure naturali , \\\hline
3.2.25 & en quanto el derecho natural es traydo al derecho delas gentes & prout ius naturale contractum est ad ius gentium , \\\hline
3.2.25 & assi derecho natural commo el derecho & sicut ius quod nam omnia animalia docuit : \\\hline
3.2.25 & assi natural commo es aquel derecho & sicut ius illud quod sequitur inclinationem naturae nostrae : \\\hline
3.2.25 & mas de derech natural & est plus de iure naturali , \\\hline
3.2.25 & e desseamos bien es natural en conparaçion del derech delas aianlias & prout appetimus esse et bonum , est naturale respectu iuris animalium , \\\hline
3.2.25 & Avn en essa misma este dereches dicho natural & siue respectu iuris quod natura omnia animalia docuit : \\\hline
3.2.25 & Et el detecho delas gentes es natural & sic etiam huiusmodi ius est naturale respectu iuris ciuilis , \\\hline
3.2.25 & Et pues que assi es tres cosas son en alguna manera del derecho natural Lo primero es que el sea ygualado proporçionado ala natura humanal & Tria ergo sunt aliquo modo de iure naturali , \\\hline
3.2.25 & lo que ayamos a ello inclinaçion natural . & secundum quod inclinatio sequitur naturam nostram : \\\hline
3.2.25 & que es dicho natural pora un ataia de los otros derechos . & quod per antonomasiam dicitur esse naturale . Appetere enim esse et bonum , \\\hline
3.2.25 & e el mal es aquello que desseamos naturalmente & quod naturaliter appetimus , \\\hline
3.2.25 & Et assi avn es el derech natural & sic est de iure naturali : \\\hline
3.2.25 & si quier sean naturales & siue sint naturales , \\\hline
3.2.25 & e avn el derechçiuil se departe del derecho natural . & etiam ius ciuile differt a iure naturali . \\\hline
3.2.26 & al derecho natural o ala ley dela natura & videlicet ad ius naturale siue ad legem naturalem , \\\hline
3.2.26 & en quanto es conparada ala razon natural & ut comparatur ad rationem naturalem siue ad legem naturalem : \\\hline
3.2.26 & Et si en alguna manera la razon natural non iudgare & et nisi aliquo modo ratio naturalis dictet illud statuendum esse . \\\hline
3.2.27 & Ca si es ley diuinal e natural establesçida es de dios a quienꝑtenesçe enderesçar todas las cosas asimesmo . & nam si est lex diuina et naturalis , | condita est a Deo cuius est omnia dirigere in seipsum , \\\hline
3.2.27 & Mas commo otra sea la ley natural e otra la positiua & Sed cum alia sit lex naturalis , alia positiua : \\\hline
3.2.27 & Ca la ley naturales & Nam lex naturalis est a deo indita in cordibus nostris : \\\hline
3.2.27 & segunt que esto pertenesçe al derecho natural . & secundum quod haec pertinent | ad ius naturale . \\\hline
3.2.27 & por las leyes naturales & et iustae et a legibus naturalibus determinatae , \\\hline
3.2.29 & e qual se quier sennor deue ser medianero entre la ley natural & inter legem naturalem \\\hline
3.2.29 & que sigua la ley natural & et per consequens sequi naturalem legem , \\\hline
3.2.29 & Et por ende el rey en gouernando es a quande dela ley natural &  \\\hline
3.2.29 & nin se arriedra dela ley natural . & in quantum a lege naturali non deuiat : \\\hline
3.2.29 & sy non se esforçare en la ley natural & nisi innitatur lege naturali , \\\hline
3.2.29 & assi commo la ley naturales sobre el señor & lex est infra principantem , \\\hline
3.2.29 & e dela ley natural . & sed ut in ea reseruatur virtus iuris naturalis . \\\hline
3.2.29 & que es mas prinçipal en gouernando la ley natural & patet hanc principaliorem esse in regendo , \\\hline
3.2.29 & si non en quanto se esfuerça enla ley natural . & eo quod nullus sit rectus Rex nisi in quantum innititur illi legi . \\\hline
3.2.29 & nin dela ley natural & et a lege naturali , \\\hline
3.2.29 & por ley natural . & aut per legem naturalem , \\\hline
3.2.30 & Mas que sin la ley natural e humanal fue menester de dar ley & Sed quod praeter legem naturalem | et humanam fuerit expediens dare legem euangelicam \\\hline
3.2.30 & non cunple la ley natural & Ad hoc autem faciendum non sufficit lex naturalis , \\\hline
3.2.30 & la ley natural e la humanal & lex naturalis \\\hline
3.2.30 & que nos ayudan a alcançar este bien . el qual non podemos natural mente & et humana iuuantes nos ad consecutionem illius boni \\\hline
3.2.30 & que es sobre natural . & quod possumus naturaliter adipisci , \\\hline
3.2.31 & e se funde enla ley natural . & oportet quod innitatur legi naturali , \\\hline
3.2.31 & Lo primero si fuerecontraria ala ley natural . & Primum \\\hline
3.2.31 & siendo contra la ley natural non son leyes & non sunt leges \\\hline
3.2.31 & commo quierque sean ennadidas alas leyes naturales & licet sint additae naturalibus legibus , \\\hline
3.2.31 & que contrario del derecho natural es aquello que non es enduzido dela natura . & quod non est a natura inductum , \\\hline
3.2.31 & segunt la qual manera de fablar el omne es desnudo natural mente . & secundum quem modum loquendi homo est nudus naturaliter , \\\hline
3.2.31 & Et segunt esta manera de fablar fablan los iuristas del derecho natural & Secundum hunc modum loquendi loquuntur Iuristae , ut patet ex Institutis de iure naturali , \\\hline
3.2.31 & e manda la razon natural e el entendimiento . & cuius contrarium dictat ratio naturalis : \\\hline
3.2.31 & maguer que sea ennadido e sobrepuesto al derecho natural & et adductum corpori nudo \\\hline
3.2.31 & Erpero non es contraria al derech natural & non tamen est contra ius naturale , \\\hline
3.2.31 & nin es contra el derecho natural . & non tamen est contra ius naturale , \\\hline
3.2.31 & Ca la razon natural non contradize atal seruidunbre . & quia huic naturalis ratio non contradicit . \\\hline
3.2.31 & e son contra derecho natural & et contra ius naturale , \\\hline
3.2.31 & saluo si fuessen contrarias ala razon natural e ala razon derecha & nisi fuerit rectae rationi contrariae . \\\hline
3.2.32 & que commo quier que la çibdat en alguna manera sea cosa natural & quod cum ciuitas sit aliquo modo \\\hline
3.2.32 & por que auemos natural inclinaçion & quid naturale , eo quod naturalem habemus impetum ad ciuitatem constituendam : \\\hline
3.2.34 & tanto mas es natural miente sieruo & tanto est magis naturaliter seruus . Esse quidem sceleratum et affectatorem belli , \\\hline
3.3.2 & ca estos tales naturalmente han miedo de las feridas . & quia naturaliter metuunt vulnera . \\\hline
3.3.2 & Ca por que naturalmente han poca sangre naturalmente temen de perder la sangre . & Nam cum naturaliter habeant modicum sanguinis , | naturaliter timent sanguinis amissionem : \\\hline
3.3.7 & assi conmo vna naturaleza & quasi natura quaedam . \\\hline
3.3.23 & quando el humor natural dellos se va & quando naturalis eorum humor expirauerit , contrahuntur ligna , \\\hline

\end{tabular}
