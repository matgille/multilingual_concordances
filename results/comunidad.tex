\begin{tabular}{|p{1cm}|p{6.5cm}|p{6.5cm}|}

\\hline
1.1.7 & e desfiziere la comunidat & si depraedetur populum et Rem publicam , \\\hline
1.2.10 & mas por la comunidat & quia aliquando aliqui plus laborantes pro Republica , \\\hline
1.2.11 & e toda comunidat es vna orden & et omnis politia est quidam ordo , \\\hline
1.2.11 & Et pues que assi es non se guardaria dende adelante en ellos comunidat & Non ergo ulterius reseruaretur | in eis politia , \\\hline
1.2.11 & e canda vna comunidat semeia avn cuerpo natural . & et quaelibet congregatio assimilatur cuidam corpori naturali . Sicut enim videmus corpus animalis constare ex diuersis membris connexis , \\\hline
1.2.11 & e cada vna delas comunidades es conpuesta de de pattidas personas ayuntadas e ordenadas a vna cosa . & et quaelibet congregatio constat ex diuersis personis connexis , | et ordinatis ad unum aliquid . \\\hline
1.2.11 & Bien assi cada vna mengua de iustiçia non corronpe del todo el regno e la comunidat . & et infirmat ipsum : sic non quaelibet Iniustitia corrumpit totaliter regnum , \\\hline
1.2.11 & e la comunidat apareiada acorruy conn¶ & et politia infirmatur , \\\hline
1.2.19 & si ouiereconplidamente las riquezas fazer espenssas conuenibles en toda la comunidat & ( si adsit facultas ) | facere decentes sumptus circa totam communitatem . \\\hline
1.2.19 & mente en toda la comunidat ¶ &  \\\hline
1.2.20 & la qual toda la comunidat dela gente & ad quam ordinatur tota communitas , \\\hline
1.2.27 & o por amor de la comunidat & vel propter amorem Reipublicae \\\hline
1.2.27 & por que sin ella la comunidat non podrie durar . & quia sine ea Respublica durare non posset . \\\hline
1.2.27 & et mantenedores dela comunidat . & et conseruatores Reipublicae . \\\hline
1.3.3 & e dela comunidat al bien propio e personal de cada vno . & et quia in communi bono includitur \\\hline
1.3.5 & quanto mayor es la comunidat & Amplius quanto maior est communitas , \\\hline
2.1.1 & mas commo la conpanna o la casa sea vna comunidat & Sed cum familia domus sit communitas quaedam , \\\hline
2.1.1 & e sea comunidat natraal & et sit communitas naturalis : \\\hline
2.1.1 & e es la comunidat dela casa & necessaria ergo fuit communitas domus , et communitates aliae , \\\hline
2.1.1 & assi commo son comunidades de varrio et de çibdat & cuiusmodi sunt communitas ciuitatis , et regni , \\\hline
2.1.1 & e en comunidat es en alguna manera natural alos omes ¶ & et in societate est quodammodo homini naturale . \\\hline
2.1.2 & por que aquellas conpania o aquella comunidat & quia societas , | siue communitas illa , \\\hline
2.1.2 & que sea comunidat çiuil de casa & non videtur esse communitas domestica , \\\hline
2.1.2 & mas paresce que sea comunidat çiuil e de çibdat . & sed ciuilis : \\\hline
2.1.2 & por la qual cosa sin determinarde la comunidat de los çibdadanos & Quare si determinare de communitate ciuium \\\hline
2.1.2 & que pertenesçen ala comunidat dela çibdat . & determinando in praecedenti capitulo aliqua pertinentia ad communitatem ciuitatis . \\\hline
2.1.2 & en qual manera la comunidat dela casa se ha a todas las otrascomuidades & quomodo communitas domestica se habet ad communitates alias : \\\hline
2.1.2 & en ssi la comunidat dela casa & cum quaelibet communitas includat communitatem domesticam , \\\hline
2.1.2 & siguese que la comunidat dela casa es neçessaria a esta uida . & sequitur communitatem domus | ad huiusmodi vitam necessariam esse . \\\hline
2.1.2 & que la comunidat dela casa es neçessaria & ad vitam nostram : \\\hline
2.1.2 & e ante ponen esta comunidat dela casa ¶ & quia per hoc manifeste ostenditur necessariam esse communitatem domesticam : \\\hline
2.1.2 & Comuidat dela casa Et comunidat de uarrio . & apparebit quadruplicem esse communitatem ; videlicet , domus , \\\hline
2.1.2 & Et comunidat de çibdat . & vici , ciuitatis , et regni . \\\hline
2.1.2 & Et comunidat de regno . & vici , ciuitatis , et regni . \\\hline
2.1.2 & Ca assi commo de muchas perssonas se faz la comunidat dela casa & Nam sicut ex pluribus personis fit domus , \\\hline
2.1.2 & assi de muchas casas se faz la comunidat de vn uarrio & sic ex multis domibus fit vicus , et ex multis vicis ciuitas , \\\hline
2.1.2 & e de muchos uarrios se faz comunidat de çibdat & sic ex multis domibus fit vicus , et ex multis vicis ciuitas , \\\hline
2.1.2 & e de muchas çibdades se faz comunidat de vn regno & et ex multis ciuitatibus regnum ; quare sicut singulares personae sunt partes domus , \\\hline
2.1.2 & Et por ende la comunidat dela casa sea alas trͣs comunidades & Hoc ergo modo communitas domus se habet ad communitates alias : \\\hline
2.1.2 & en tal manera que todas las otras comunidades ençierren & quia omnes aliae ipsam praesupponunt : \\\hline
2.1.2 & por que ella es en alguna manera parte de todas las otras comunidades . & et ipsa est quodammodo pars omnium aliarum . Naturalis enim origo ciuitatis \\\hline
2.1.2 & sienpre la comunidat dela casa se ha alas otras comuindades & semper sic se domus habet | ad communitates alias , \\\hline
2.1.2 & en qual manera la comunidat dela casa se ha alas otras comuidades de ligero puede parescer & quomodo communitas domus se habeat ad communitates alias : | de leui patet , \\\hline
2.1.2 & en qual manera esta comunidat es neçessaria ala uida humanal . & quomodo huiusmodi communitas sit necessaria in humana vita . \\\hline
2.1.2 & Ca si todas las otras comunidades antepo nen la comuidat dela calali alguna corunidat es neçessaria & Nam si omnes communitates aliae domum praesupponunt : \\\hline
2.1.2 & Conuiene que la comunidat dela casa sea mas neçessaria Et pues que assi es los Reyes e los prinçipes & si aliqua communitas est necessaria ad per se sufficientiam vitae , oportet communitatem domus necessariam esse . Reges ergo et Principes , quorum officium est dirigere alios ad bene viuere , ignorare non debent , \\\hline
2.1.3 & en quanto estas cosas son ordenadas ala comunidat & ut ordinantur ad communitatem , \\\hline
2.1.3 & que es comunidat delas perssonas dela casa & quae est communitas personarum domesticarum , \\\hline
2.1.3 & en qual manera ella es la comunidat primera . & quomodo sit communitas prima . \\\hline
2.1.3 & en qual manera la comunidat dela casa se ha ala comunindat dela çibdat & quomodo communitas domus se habet ad communitatem ciuitatis , et ad communitates alias . \\\hline
2.1.3 & e alas otras comunidades . & quomodo communitas domus se habet ad communitatem ciuitatis , et ad communitates alias . \\\hline
2.1.3 & e en la entençion laso trisco munindades son primero que la comunidat dela & et in intentione communitates illae praecedunt communitatem domesticam . Rursus uia generationis et temporis domestica communitas praecedit communitates alias : \\\hline
2.1.3 & casa¶Otrossi la comunidat dela casa es primero & et in intentione communitates illae praecedunt communitatem domesticam . Rursus uia generationis et temporis domestica communitas praecedit communitates alias : \\\hline
2.1.3 & que las otras comunidades & sed in uia perfectionis \\\hline
2.1.3 & mas las otras comunindades son primero que la comunidat delan casa & et complementi communitates aliae praecedunt ipsam . \\\hline
2.1.3 & por que paresçe que la comunidat dela casa se ha en dos maneras alas otras comunidades . & Videtur communitas domus ad communitates alias dupliciter se habere . Primo , \\\hline
2.1.3 & por que esta comunidat en conparaçion delas otras es mas menguada & quia huiusmodi communitas respectu aliarum est imperfecta : \\\hline
2.1.3 & por que todas las otras comunidades ençierran & cum enim omnis alia communitas includat communitatem domus , \\\hline
2.1.3 & en ssi la comunidat dela casa & cum enim omnis alia communitas includat communitatem domus , \\\hline
2.1.3 & Et por ende todas las otras comunidades son mas conplidas que ella . & omnes aliae communitates sunt perfectiores ea . Habet ergo se communitas domus \\\hline
2.1.3 & Otrossi la comunidat dela casa se ha alas otras comunidades & sicut pars ad totum . Rursus huiusmodi communitas se habet ad alias , \\\hline
2.1.3 & assi es la comunidat del uarrio & Communitas ergo vici est finis communitatis domus , \\\hline
2.1.3 & es fin dela comunidat dela casa & Communitas ergo vici est finis communitatis domus , \\\hline
2.1.3 & e la comunidat dela çibdat es fin dela comunidat del uarrio . & communitas ciuitatis communitatis vici : \\\hline
2.1.3 & Mas la comunidat del regno es fin de todas las otras comuindades sobredichas . & sed communitas regni est finis omnium praedictorum . \\\hline
2.1.3 & quela comunidat dela çibdat es la primera . & quod prima communitas est communitas ciuitatis ; \\\hline
2.1.3 & Otrossi commo la comunidat dela casa non solamente se aya alas otras comuindades & Amplius cum communitas domus ad communitates alias non solum se habeat \\\hline
2.1.3 & Visto en qual manera la comunidat dela casa es primero en alguna manera que las otras comuni dades de ligero puede paresçer & quomodo communitas domus aliquo modo est prior , | quam communitates aliae : \\\hline
2.1.3 & en alguna manera esta comunidat dela cała es natural & de leui videri potest , | quomodo sit huiusmodi communitas naturalis . \\\hline
2.1.3 & commo todas las comunidades presupongan & cum omnis communitas praesupponat communitatem domus , \\\hline
2.1.3 & e ante pongan la comunidat dela casa & cum omnis communitas praesupponat communitatem domus , \\\hline
2.1.3 & conuiene quala comunidat dela casa o la casa sea cosa natural . & oportet communitatem domesticam siue domum | quid naturale esse . \\\hline
2.1.3 & Ca assy commo el regno e la çibdat ante ponen la comunidat dela casa & quia sicut regnum vel ciuitas praesupponunt esse domum , \\\hline
2.1.3 & e qual es la comunidat dela casa & et ut cognoscant quae et qualis est communitas domus \\\hline
2.1.3 & ca es comunidat en alguna manera natural & et ut cognoscant quae et qualis est communitas domus \\\hline
2.1.3 & Et en algunan manera esta comunidat se ha al regno & ut se habet ad regnum et ciuitatem , \\\hline
2.1.4 & qual es la comunidat dela casa . & qualis sit communitas domus : \\\hline
2.1.4 & e quela comunidat dela casa es en alguna manera natural . & et quod communitas domus est quodammodo naturalis . \\\hline
2.1.4 & casapor ende entendemos de dezir algunas cosas dela comunidat dela casa . & ideo intendimus aliqua dicere | de communitate domestica . \\\hline
2.1.4 & e difine la comunidat dela casa & Philosophum 1 Politicorum sic describere communitatem domus : \\\hline
2.1.4 & diziendo que la casa es comunidat & videlicet , | quod domus est communitas \\\hline
2.1.4 & Ca que la casa sea comunidat & Nam quod domus sic communitas \\\hline
2.1.4 & en qual manera la casa sea comunidat establesçida para cada dia . & quomodo domus sit communitas constituta in omnem diem . \\\hline
2.1.4 & pues que assi es la comunidat dela casa fue fecha para aquellas cosas & et ut viatores , \\\hline
2.1.4 & para la uida non cunplie la comunidat de vna casa & non sufficiebat communitas domestica , \\\hline
2.1.4 & mas conuiene de dar comunidat de varrio . & sed oportuit dare communitatem vici , \\\hline
2.1.4 & que assi commo la comunidat dela casa es establesçida & quod sicut communitas domus constituta est in omnem diem , \\\hline
2.1.4 & conuiene de dar comunidat ala çibdat sobre la comunidat deluarrio . & praeter communitatem \\\hline
2.1.4 & que la casa es comunidat & secundum naturam constituta in omnem diem . Vicus autem est communitas constituta in opera non diurnalia . \\\hline
2.1.4 & Mas el uatrio es comunidat estableçida & Ciuitas vero est communitas constituta \\\hline
2.1.4 & Et la çibdat es comunidat establesçida &  \\\hline
2.1.4 & Mas el regno es comunidat establesçida & Sed regnum est communitas constituta non solum ad supplendum indigentias vitae , \\\hline
2.1.4 & Et por ende paresçe qual es la comunidat dela casa . & Patet ergo | qualis sit communitas domus , \\\hline
2.1.4 & Ca es comunidat natural e establesçida & quia est communitas naturalis constituita propter opera diurnalia et quotidiana . \\\hline
2.1.4 & por las cosas sobredichͣs la casa es vna comunidat & sit communitas quaedam et societas personarum : \\\hline
2.1.4 & Et commo non sea propreamente comunidat nin conpannia de vno & cum non sit proprie communitas nec societas ad seipsum , \\\hline
2.1.4 & comunidades la qual cosa non puede ser sin muchͣs perssonas . & sed in domo oportet dare plures communitates : \\\hline
2.1.4 & non solamente es menester la comunidat dela casa & in vita humana non solum est expediens communitas domus , \\\hline
2.1.4 & mas ahun la comunidat del uarrio e dela çibdat e del regno & sed et vici , | ciuitatis , et regni . \\\hline
2.1.4 & que por las que ya dichos son es menester la comunidat dela çibdat & quam propter iam dictas , | sit expediens communitas ciuitatis , \\\hline
2.1.5 & De comunidat de uaron et de muger . & ex communitate viri et uxoris , \\\hline
2.1.5 & Et de comunidat de sennor e de sieruo . & domini et serui , \\\hline
2.1.5 & Ca la comunidat del uaron & quid naturale : \\\hline
2.1.5 & Mas la comunidat del sennor e del sieruo es ordenada ala conseruaçique & quia communitas viri et uxoris ordinatur ad generationem , communitas vero domini \\\hline
2.1.5 & Mas que la comunidat del uaron & quia sine eis domus congrue esse non valet . Quod autem communicatio viri \\\hline
2.1.5 & Mas que la comunidat del sennor e del sieruo sea por salud & Sed quod communitas domini | et serui sit propter salutem \\\hline
2.1.5 & es men ester la comunidat del uaron e dela mugni & quia sicut ad constitutionem domus requiritur communitas viri \\\hline
2.1.5 & menester la comunidat del sennor e del sieruo & sic requiritur ibi communitas domini \\\hline
2.1.5 & Mas si la comunidat del sennor e del sieruo en otra manera es establesçida para salud e conseruaçion & Utrum autem communitas domini \\\hline
2.1.5 & Visto en qual manera alo menos estas dos comunidades son menester & et seruus propter dominum , infra clarius ostendetur . Viso , quomodo saltem hae duae communitates requiruntur ad domum , \\\hline
2.1.5 & que commo la casa primera sea establesçida de dos comunidades & quod cum domus prima constet | ex duabus communitatibus , \\\hline
2.1.5 & e cada vna delas comunidades aya menester dos perssonas o dos linages de perssonas & et quaelibet communitas requirat duas personas | uel duo genera personarum , \\\hline
2.1.6 & y la terçera comunidat & oportet ibi dare communitatem tertiam , \\\hline
2.1.6 & por ende con razon la comunidat del uaron e dela muger & merito ergo communitas viri et uxoris , \\\hline
2.1.6 & e la comunidat del sennor e del sieruo & et domini et serui , quae est propter saluationem , faciunt domum primam . Sic ergo saluatio comparatur ad rem generatam : \\\hline
2.1.6 & Mas la terçera comunidat & sed tertiam etiam communitatem , \\\hline
2.1.6 & por la qual cosa commo la comunidat del padre al fijo tome nasçençia & Quare cum communitas patris ad filium sumat originem \\\hline
2.1.6 & e comienço de aquello que el padre e la madre engendran su semeiança esta tal comunidat non es dicha de razon dela primera casa & ex eo quod parentes sibi simile produxerunt : | huiusmodi communitas non dicitur esse de ratione domus primae , \\\hline
2.1.6 & Mas que ala perfectiuo dela casafaga menester esta terçera comunidat & Quod autem ad perfectionem domus requiratur haec tertia communitas , \\\hline
2.1.6 & Pues que assi es que la terçera comunidat & tertia communitas , \\\hline
2.1.6 & sea la primera parte dela casar la primera comunidat & et prima communitas , \\\hline
2.1.6 & Mas en la comunidat del padre & in communitate vero patris et filii , \\\hline
2.1.6 & Et en la comunidat del señor & et filius obtemperans ; in communitate quidem domini \\\hline
2.1.6 & quantas son las comunidades & Patet ergo quot communitates sunt in domo perfecta , et quot regimina , \\\hline
2.1.7 & en la comunidat dela casa & primum oportet congregare marem , \\\hline
2.1.7 & Et pues que assi es la comunidat del uaton e dela muger & et potest sibi simile producere . Communitas ergo maris \\\hline
2.1.7 & Mas la comunidat en la uida humanal & ( \\\hline
2.1.7 & Mas todas estas comunidades ante ponen la comunidat dela casa . & Omnes autem hae communitates praesupponunt communitatem domesticam . \\\hline
2.1.7 & Et la comunidat dela casa mas paresçe & quam ciuile : et communitas domus magis videtur esse naturalis ipsi homini , \\\hline
2.1.7 & que la comunidat del uarrio & quam communitas vici , ciuitatis , et regni . \\\hline
2.1.7 & assi commo es ordenada la comunidat dela casa & sicut communitas domus : \\\hline
2.1.7 & Ca si las comunidades dichas del uarrio e dela çibdat son ordenadas al mantenemiento e ala generaçion humanal & nam si praedictae communitates ordinantur ad nutritionem , | et ad generationem ; \\\hline
2.1.7 & commo la primera comunidat dela casa sea ayuntamientode uaron & cum prima communitas ipsius domus sit coniunctio viri et uxoris , sequitur ex parte ipsius communitatis humanae , \\\hline
2.1.7 & por comunidat coniugable & communicatiuum communitate coniugali , \\\hline
2.1.7 & que por comunidat de barrio & quam communitate vici ciuitatis , et regni : \\\hline
2.2.1 & que commo la comunidat & quod cum communitas viri et uxoris , \\\hline
2.2.18 & Mas avn es neçessario para el bien e para el defendimiento dela comunidat & sed etiam necessarius pro bono Reipublicae , \\\hline
2.2.20 & nin c̃ca los gouernamientos dela comunidat & nec regiminibus reipublicae ; \\\hline
2.3.9 & que es la comunidat primera non seria ninguna muda conn neçessaria & nulla commutatio esset necessaria . Nam in domo dominatur paterfamilias , \\\hline
2.3.9 & que en la primera comunidat & quod in prima communitate quae est domus , \\\hline
2.3.9 & assi commo es comunidat del uarrio o dela çibdat & ut communitas vici , vel ciuitatis , vel totius regni , \\\hline
2.3.9 & ca si ala comuidat de vn barrio o de vna çibdatur llgua manera abastasse la mudaçion delas cosas alas cosas enpero ala comunidat & vel ciuitatis , | aliquo modo sufficeret commutatio rerum ad res : \\\hline
3.1.1 & commo toda comunidat sea & cum omnis communitas fit gratia alicuius boni , \\\hline
3.1.1 & por conparaçion alas otras comunidades ¶ & Secunda ex parte ciuitatis constitutae . \\\hline
3.1.1 & ca assi commo dize el philosofo enl primero libro delas politicas vna inclinaçion de natura es en todos los orans atal comunidat & natura quidem impetus in omnibus inest | ad talem communitatem , \\\hline
3.1.1 & enpero mayormente es ordenada aquel bien la comunidat & ad bonum , | maxime tamen ordinatur \\\hline
3.1.1 & e esta tal es la comunidat dela çibdat & huius autem est communitas ciuitatis , quae respectu communitatis domus , \\\hline
3.1.1 & la qual en conparacion dela comunidat dela casa & huius autem est communitas ciuitatis , quae respectu communitatis domus , \\\hline
3.1.1 & mas conplidamente en el segundo libro la comunidat del barno & ut supra in secundo libro diffusius probabatur : | communitas vici , \\\hline
3.1.1 & e avn la comunidat dela çibdat & et ad hoc communitas ciuitatis , \\\hline
3.1.1 & que la comunidat del barrio & quae est principalissima communitas respectu vici , et domus , \\\hline
3.1.1 & que toda comunidat es establesçida & quod si communitatem omnem gratia alicuius boni dicimus constitutam , \\\hline
3.1.1 & e esta es comunidat politica & haec autem est communitas politica , \\\hline
3.1.1 & mas a vn otra comunidat ay mas prinçipal & et vicio . | Est autem alia communitas principalior ea , \\\hline
3.1.1 & que ella la qual es comunidat del regno & cuiusmodi est communitas regni , \\\hline
3.1.1 & ca mostraremos que la comunidat del regno es prouechosa en la uida humanal & ostendemus enim communitatem regni utilem esse in vita humana , \\\hline
3.1.1 & que la comunidat dela çibdat &  \\\hline
3.1.1 & que assi se ha la comunidat del regno ala comunidat dela çibdat & et esse principaliorem communitate ciuitatis . Videtur enim suo modo communitas regni se habere ad communitatem ciuitatis , sicut haec communitas se habet ad domum , et vicum . Nam ciuitas sicut complectitur domum , et vicum ; \\\hline
3.1.1 & commo la comunidat dela çibdat se ha ala comiundat dela casa e del uarrio . & et esse principaliorem communitate ciuitatis . Videtur enim suo modo communitas regni se habere ad communitatem ciuitatis , sicut haec communitas se habet ad domum , et vicum . Nam ciuitas sicut complectitur domum , et vicum ; \\\hline
3.1.1 & bien assi la comunidat del regno ençierra & sic communitas regni circumplectitur communitatem ciuitatis , \\\hline
3.1.1 & en ssi la comunidat de la çibdat & sic communitas regni circumplectitur communitatem ciuitatis , \\\hline
3.1.1 & que la comunidat dela çibdat & quam communitas illa . \\\hline
3.1.2 & e beuir en comunidat politica e de çibdat & Nam ipsum viuere consequuntur homines ex communitate politica : \\\hline
3.1.2 & que la comunidat & quod communitas , \\\hline
3.1.2 & et tales comunidat acabada & quae est ciuitas constans ex pluribus vicis , est communitas perfecta : \\\hline
3.1.2 & que tal comunidat es la que ha termino por si & huiusmodi communitas est habens terminum omnis per se sufficientiae vitae . \\\hline
3.1.2 & nin durar otdenaron la comunidat politica & cum sine lege et iustitia constituta ciuitas stare non posset , ordinauerunt communitatem politicam , \\\hline
3.1.4 & que esta comunindat ençierra en si la comunidat dela casa & secundum naturam . \\\hline
3.1.4 & e la comunidat del uartio ¶ & secundum naturam . \\\hline
3.1.4 & e por ende la comunidat dela casa & Communitas ergo domestica \\\hline
3.1.4 & que la çibdat es fin et conplimiento de las dichͣs dos comunidades & hoc idem , sumitur ex eo quod ciuitas est illarum communitatum finis et complementum . \\\hline
3.1.4 & bien assi avn la comunidat del uarrio es cosa natural & Sic etiam communitas vici est quid naturale , \\\hline
3.1.4 & si la comunidat dela casa es ordenada a alcançar lo que es delectable & quid iniustum . Si ergo communitas domestica ordinatur ad prosequendum conferens , \\\hline
3.1.4 & Et la comunidat dela çibdat & communitas vero ciuitatis ultra hoc ordinatur \\\hline
3.1.4 & que la comunidat dela casa & oportet communitatem domesticam \\\hline
3.1.4 & e la comunidat dela çibdat sean cosas naturales & et ciuilem esse quid naturale . \\\hline
3.1.4 & ca si la natura dio al omne palabra natural aquella comunidat & Nam si natura dedit homini sermonem , \\\hline
3.1.4 & non han de ser propreamente en la comuidat dela casa mas enla comunidat dela çibdat & sed in communitate ciuili . In ciuitate enim , \\\hline
3.1.4 & por la comunidat dela çibdat & per communitatem ciuilem , \\\hline
3.1.5 & que sin la comunidat dela çibdat cosa aprouechosa fue ala uida humanal de establesçer comunidat de regno ¶ & quod praeter communitatem ciuitatis , | utile est humanae vitae statuere communitatem regni . Prima via sumitur ex parte sufficientiae vitae . \\\hline
3.1.5 & que la comunidat acabada & quod communitas perfecta , \\\hline
3.1.9 & Et pues que assi es la comunidat delas possessiones & appreciatur se plus ualere quam ualeat . Communitas ergo possessiones \\\hline
3.1.9 & Et pue tal que assi es puesta comunidat delas mugers & Non ergo supposita communitate uxorum esset \\\hline
3.1.9 & ca muchͣ comunidat & ut opinabatur Socrates : \\\hline
3.1.9 & assi commo conuiene ala comunidat de los çibdadanos & ut expedit communitati ciuium . \\\hline
3.1.10 & que ordeno socrates tire la cercidunbre de los fijos e el conosçimiento del parentesco non es de ella bartal comunidat & et contumelias inferant . | Quare cum communitas uxorum , \\\hline
3.1.10 & ¶ El terçero mal se declara assi ca conmo de la comunidat sobredichͣ delas mugiets & Nam supposita praedicta communitate , \\\hline
3.1.10 & que puesta tal comunidat commo ordeno soctateᷤ & sequitur quod supposita communitate , | quam ordinauerat Socrates , \\\hline
3.1.10 & por ende much era de reprehender la opimon de socrates dela comunidat & et amor libidinosus , reprehensibilis erat opinio Socratis de communitate uxorum et filiorum . Decet ergo Reges \\\hline
3.1.10 & por que defendia la comunidat delas fenbras & et Principes sic ordinare ciuitatem , ut prohibita communitate foeminarum \\\hline
3.1.11 & do es tanta comunidat de los çibdadanos tienen la çibdat & tanta communitas ciuium , \\\hline
3.1.11 & e que biuen sin contienda entre los quales se guarda tan grant comunidat . & et absque litigio viuere , | inter quos tanta communitas obseruatur . \\\hline
3.1.11 & por la comunidat delas fenbras & attamen inter eos \\\hline
3.1.15 & e en los fijos deue ser guardada comunidat & reseruari communitas quantum ad amorem : \\\hline
3.1.15 & quanto ala comunidat de los çibdadanos & quantum ad communitatem ciuium : \\\hline
3.1.15 & assi es poniendo la entençio de socrates dela comunidat delas cosas & et de unitate ciuium , \\\hline
3.2.17 & ca esto fue lo que enssalço la comunidat de Roma fieldat de buenos consseieros & Hoc enim fuit , | quod apud Romanam Rempublicam exaltauit fidelitas consiliantium : \\\hline
3.2.17 & e muy alto era conssisto no secreto dela comunidat de roma & quod fidum et altum erat secretum consistorium reipublicae , silentique salubritate munitum : \\\hline
3.2.19 & e ame la comunidat et ꝓmueua los q̃ son eñl su regno &  \\\hline
3.2.26 & que non conuiene de apropar las comunidades delas çibdades alas leyes . & et moribus illius gentis . Ideo dicitur 4 Politicorum \\\hline
3.2.26 & Mas las leyes alas comunidades de las çibdades & quod non oportet adaptare politias legibus , sed leges politiae , quas leges oportet diuersas esse \\\hline
3.2.26 & segunt el departimiento delas comunidades . & secundum diuersitatem politiarum . \\\hline
3.2.27 & prinçipalmente de toda la comunidat & quare | cum bonum commune principaliter intendatur a tota communitate \\\hline
3.2.32 & e dez que es comunidat de çibdadanos & quod est communicatio ciuium propter bene , \\\hline

\end{tabular}
