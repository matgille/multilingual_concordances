\begin{tabular}{|p{1cm}|p{6.5cm}|p{6.5cm}|}

\hline
1.1.4 & que son falladas en ellas . Enpero non pudieron alcançar la uerdat conplidamente e segunt manera acabada . & et de felicitatibus repertis in ipsis , Philosophi distinxerunt : non tamen ad plenum , et per omnem modum potuerunt attingere veritatem . Nam licet vere dixerunt quod in vita voluptuosa non est quaerenda felicitas , \\\hline
1.1.4 & tantomas conviene dela auer los rreys e los prinçipes por quanto han de dar mayor cuenta ant̃la siella del primer alcałł que es dios Et quanto de mayores conpannas han de dar Razon e cuenta & tanto magis decet habere reges et principes , quanto apud tribunal summi Iudicis reddituri sunt de pluribus rationem . Quod maxime expedit regiae maiestati \\\hline
1.1.5 & sy la su fin non sopiere . Conviene a todo omne que quiera alcançar e auer su fin & ignorato ipso fine , expedit volenti consequi suum finem , vel suam felicitatem , \\\hline
1.1.5 & Ca para que cada vno por las sus obras alcançe la su fin tres cosas le son le mester ¶ Lo primero que faga bien & ut sit finis consecutiuus . Secunda vero , inquantum est aliorum directiuus . Prima via sic patet . Nam ad hoc quod aliquis per suas operationes finem consequatur , tria requiruntur . Primo , \\\hline
1.1.5 & Ca sy non feziese bien mas mal non podria alcançar buena fin mas avria el contrario & Si enim non ageret bene sed male , non consequeretur finem , | sed male , non consequeretur finem , sed contrarium finis : \\\hline
1.1.5 & Por la qual rrazon los malos dignos son non que alcançen buena fin mas el contrario e mala fin . & digni sunt , non ut consequantur finem , sed contrarium finis . Immo non solum male agentes non consequuntur finem , \\\hline
1.1.5 & mas el contrario e mala fin . Mas non solamente los que mal fazen non alcançan buena fin mas avn los que pueden bien fazer & non ut consequantur finem , sed contrarium finis . Immo non solum male agentes non consequuntur finem , sed potentes bene agere , \\\hline
1.1.5 & njn buena ventura de los . ¶ Mas commo nos alcançamos la buena ventura por obras uirtuosas & et felicitatem . Immo cum ex operibus virtuosis felicitatem consequamur ( quia virtus est habitus electiuus in medietate consistens , \\\hline
1.1.5 & que conuiene que las obras por las quales nos alcançamos la fin que salgan denr̃a elecçio e denr̃a uoluntad ¶ & ut dicitur 2 Ethic . ) oportet operationes , per quas finem consequimur , ex electione procedere . \\\hline
1.1.5 & njn delectosamente non les conuiene que por aquellas obras alcançen buena fin nin bue an uentura ¶ & tamen quia non faciunt ea bene et delectabiliter , non oportet per huiusmodi opera eos consequi finem vel felicitatem . Cum ergo ista tria contingunt , \\\hline
1.1.5 & e delectosamente estonçe contesçe que nos alcançemos conplidamente buena fin . Mas estas cosas contesçen & et delectabiliter , maxime contingit nos sic finem consequi haec autem maxime contingunt , cognito fine . \\\hline
1.1.5 & que nos que conoscamos primero alanr̃a fin por que la podamos alcançar Ca asy lo deuemos ymaginar & Expedit ergo ( ut finem consequamur ) finem praecognoscere . Sic enim imaginari debemus , quod sicut est in causis efficientibus , \\\hline
1.1.5 & si bien fazen e alcançan la fin esto non es por uoluntad & si bene agant , et finem consequantur , hoc non est ex electione , sed à fortuna . \\\hline
1.1.5 & que para lanr̃auida grant acresçentamiento faze connosçer ante la fin Ca por esta Razon alcançaremos ante la fin si la connosçieremos & ad vitam nostram magnum habet incrementum : consequemur enim ex hoc magis ipsum finem ; quemadmodum sagittatores signum habentes , \\\hline
1.1.6 & que el omne es bien auenturado quando el alcança aquello en que esta el bien acabado ¶onde dize el philosofo en el primero libro delas ethicas & quando assecutus est id , in quo consistit suum perfectum bonum . Unde Philosophus 1 Ethicorum describens felicitatem , \\\hline
1.1.7 & e por qual quier carrera que pudiere por que pueda alcançar aquella fin e aquel bien & quod quilibet studet omni via , omni modo , quo potest , consequi finem suum . Ponens igitur suam felicitatem in diuitiis , \\\hline
1.1.7 & que non pueda querer seguir la su fin ante se trabaia dela alcançar quanto puede ¶ & non potest eum latere quemlibet , omni via qua potest , velle consequi suum finem . Est igitur Rex Tyrannus , \\\hline
1.1.8 & si el prinçipe pusiere la su bien andança enlas honrras por que pueda delo que feziere honrra alcançar presumira de poner los pueblos a todo peligro por que pueda alcançar aquella honrra¶ & si Princeps suam felicitatem in honoribus ponat , ut possit honorem consequi , praesumet suam gentem exponere omni periculo . Exemplum huiusmodi habemus | ut possit honorem consequi , praesumet suam gentem exponere omni periculo . Exemplum huiusmodi habemus de filio cuiusdam Romani Principis nomine Torquati , \\\hline
1.1.8 & por que pueda delo que feziere honrra alcançar presumira de poner los pueblos a todo peligro por que pueda alcançar aquella honrra¶ Et desto auemos enxienplo en vn fijo de vn prinçipe Romano & praesumet suam gentem exponere omni periculo . Exemplum huiusmodi habemus de filio cuiusdam Romani Principis nomine Torquati , qui nimii honoris auidus , \\\hline
1.1.9 & Ca commo quier los de altos coraçones entienden prinçipalmente de tomar honrra mas de alcançar algun bien enpero la honrra les parte nesçe a ellos . Et conuiene les alos Reys de resçebir la honrra & Magnanimi autem licet non intendant principaliter honorem , sed bonum : honor tamen eos consequitur , \\\hline
1.1.11 & assi commo de instrumentos para alcançar la feliçidat e la bien andança . Ca deue husar de viandas & omnibus tamen istis debet uti , ut sunt organa ad felicitatem . Debet enim uti cibis , \\\hline
1.1.13 & Et pues que assi es quanto cada vno mas alcança dela semeiança de dios e mas se conforma con el mayor gualardon resçibra del . & qui ab eo remunerari desiderat : quanto ergo quis magis gerit imaginem eius , et plus se conformat ei , \\\hline
1.1.13 & e por el su buen gouierno alcançara grant merçed e grant gualardon de dios . Ca por el bien dela gente & ex operibus eorum consequenter mercedem magnam : quia pro bono gentis . Et pro bono totali , totaliter se exponunt . \\\hline
1.2.2 & Vnos enssituio por el qual alcançan su propia folgura e su propia delectaçion assi commo es el appetito desseador . & Tribuit ergo eis duplicem appetitum sensitiuum , unum per quem prosequuntur propriam quietem , et propriam delectationem , | unum per quem prosequuntur propriam quietem , et propriam delectationem , ut concupiscibilem : \\\hline
1.2.2 & segund el qual segnimos los bienes delectables por el entendimiento . Et acometemos los bienes fuertes de alcançar assi commo es otro & Non ergo est alius appetitus intellectiuus , secundum quem prosequimur bona delectabilia per intellectum , et aggredimur bona ardua : sicut est alius appetitus sensitiuus , secundum quem prosequimur \\\hline
1.2.3 & que ha de seer çerca delas espessas comunales o aquel bien prouechoso es e alto e guaue de alcançar Et assi es manificençia & quae est circa mediocres sumptus . Vel est illud bonum arduum , et sic est Magnificentia , \\\hline
1.2.3 & que son grandeza e alteza de coraçon son çerca de los bienes guaues e fuertes de alcançar . Enpero de departidas maneras . & ut Fortitudo est circa passiones ortas ex malis futuris , Mansuetudo circa passiones ortas ex malis praesentibus . Magnificentia vero , et Magnanimitas sunt circa bona ardua , aliter et aliter : quia Magnificentia est circa magna bona utilia , \\\hline
1.2.3 & assi commo çerca de gran deshonrras . Ca el que quiere alcançar grandes honrras es magnanimo et de alto coraçon ¶ Mas en el appetito cobdiçiador son seys uirtudes & Magnanimitas vero circa magna bona honesta , ut circa magnos honores . In concupiscibili autem sunt sex virtutes , videlicet , Temperantia , Liberalitas , \\\hline
1.2.6 & o a fin conuenible delas otras uirtudes morales si non sopiere en qual manera el puede alcançar tal fin . e esto ha de saber & vel in finem aliarum virtutum moralium , nisi sciamus , quomodo possumus consequi talem finem , | nisi sciamus , quomodo possumus consequi talem finem , quod fit per prudentiam . Per virtutes ergo morales praestituimus nobis debitos fines : \\\hline
1.2.6 & Mas por la pradençia somos reglados en qual manera pondemos alcançar aquellas fines . Mas la pradençia toma aquellas fines delas uirtudes morałs & sed per prudentiam ( cum habemus ) dirigimur in fines illos . Accipiat autem prudentia fines illos a virtutibus moralibus : \\\hline
1.2.6 & Et por aquellas cosas que son ordenadas a aquellos fines la pradençia regla derechamente a omne para alcançar aquellos fines . Por la qual cosa la pradençia & et per ea quae sunt ad finem , recte dirigit in fines illos . Quare prudentia , \\\hline
1.2.6 & que fablamos si son buenas para alcançar aquello que queremos . ¶ lo terçero deuemos mandar & quos castrum istud capi posset . Secundo iudicandum esset de viis inuentis , utrum bonae essent ad propositum prosequendum . Tertio praecipiendum esset \\\hline
1.2.8 & Ca los omes que bien proueen de los bienes que han de venir pienssan carreras e maneras por las quales puedan ligeramente alcançar aquellos bienes ¶ pues que assi es & quia homines prouidentes futura bona , excogitant vias , per quas faciliter illa adipisci valeant . | excogitant vias , per quas faciliter illa adipisci valeant . Ergo ratione bonorum , \\\hline
1.2.8 & que aya prouision delas cosas que han de venir por que mas ligeramente pueda alcançar aquellos bienes que han de venir . Et conuiene le & expedit ut habeat prouidentiam futurorum , ut facilius illa futura bona adipisci valeat , et ut habeat memoriam praeteritorum , \\\hline
1.2.9 & por que los males puedan meior esquiuar . Et los bienes mas ligeramente alcançar¶ La terçera manera es que los Reyes et los prinçipes deuen penssar muchas uezes & Nam ex hoc habebunt prouidentiam futurorum , ut possint mala expeditius vitare , et bona facilius adipisci . Tertio debent saepe recogitare bonas consuetudines , \\\hline
1.2.13 & ¶Pues que assi es porque non podemos en punto alcançar el medio entre la osadia e el temor . por ende auemos de inclinar nos mas ala osadia & et plus repugnat , et contradicit timor fortitudini , quam faciat audacia . Quia igitur non possumus punctualiter attingere medium inter audaciam , \\\hline
1.2.14 & e guisa en commo escuse el denuesto e alcançe honrra . Ca veemos & et notos quis maxime fugit opprobria , et quaerit honores . Videmus enim aliquos , \\\hline
1.2.17 & assi commo son los logreros e los garçons e los que biuen de alcauteria e los que despoian los muertos & lenones , idest viuentes de meretricio , expoliatores mortuorum , \\\hline
1.2.18 & non solamente non pueden ser gastadores dando mas apenas pueden alcançar a que sean francos dando e espendiendo . Ca sienpre deuen penssar & non solum non possunt esse prodigi , sed vix possunt attingere ut sint liberales . | sed vix possunt attingere ut sint liberales . Semper ergo cogitare debent , \\\hline
1.2.22 & Por que los omes las mas vezes ordenan estas cosas para alcançar por ellas honrra . & quia ut plurimum homines haec ordinant , ut consequantur honores . Sicut ergo dicebatur de fortitudine , \\\hline
1.2.25 & en quanto ha razon de grande nos tira que non lo fagamos nin lo alcançemos por razon dela guaueza & unam impellentem , et aliam retrahentem . Videmus autem quod magnanimitas est circa magnos honores . Magnus autem honor est quodammodo magnum bonum . Magnum autem bonum , ratione qua magnum est , \\\hline
1.2.25 & que uayamos a ello e lo alcançemos . Et pues que assi es cerca las grandeshonrras & ne prosequamur illud , ratione difficultatis . Prout vero bonum est , nos allicit , ut tendamus in ipsum . Ergo circa magnos honores , \\\hline
1.2.25 & si mas que la razon e el entendimientomanda nos tiremos de aquellas cosas por que son muy altas e muy graues de alcançar ¶Lo segundo si desmesuradamente & et circa magna bona dupliciter contingit peccare . Primo , si ultra quam ratio dictet prosequamur ea inquantum bona sunt . Secundo , si infra quam ratio dictet retrahamus nos ab illis , \\\hline
1.2.25 & e mas de quanto manda la razon fueremos e quisieremos alcançar aquellas cosas por razon dela bondat & si ultra quam ratio dictet prosequamur ea inquantum bona sunt . Secundo , si infra quam ratio dictet retrahamus nos ab illis , eo quod ardua et difficilia sint . \\\hline
1.2.25 & por que non nos podamos tirar atras por la graueza que es en alcançar . Et esta uirtud es magnammidat ¶ & Quare circa talia duplici virtute indigemus . Una impellente nos , ne retrahamur propter difficultatem , et haec est magnanimitas : \\\hline
1.2.25 & por razon dela guaueza que non pueda alcançar obras dignas de grand sonrra . Mas la humildat prinçipalmente tienpra la esꝑanca & ne aliquis ratione difficultatis desperet , ne tendat in opera magno honore digna . Humilitas vero principaliter moderat ipsam spem , ne aliquis nimis sperans \\\hline
1.2.30 & assi ca cosa baldia et vana es aquella que non alcanca su fin conuenible . Et pues que assi es qual si quier cosa & Nam ociosum est illud , quod caret debito fine ; quicquid ergo de se est ordinabile ad debitum finem , \\\hline
1.2.31 & e por la sabiduria razon amos bien de aquellas cosas que son ordenadas a alcançara aquella fin e escogemos a nos derech̃o & Nam proponentes nobis bonum finem per virtutes morales , per prudentiam bene ratiocinamur de iis quae sunt ad finem , et eligimus nobis rectam viam , \\\hline
1.2.31 & qual li quier fin falla carreras e caminos por que mas ayna alcançe aquella fin . Por la qual cosa los destenprados & qui proposito quocunque fine inueniat vias , ut citius consequatur finem illum : propter quem temperati , \\\hline
1.2.31 & si sopieren cuydar las carreras e los caminos por los quales pueden alcançar las cosas delectables segunt la carne &  \\\hline
1.3.1 & Ca las passiones del apetito cobdiçiador catan al bien o al mal en quanto es guaue de alcançar Et pues que assi es estas tales passiones & sed passiones irascibiles respiciunt bonum vel malum in eo quod arduum . Huiusmodi ergo passiones vel sumuntur respectu boni , \\\hline
1.3.1 & si non yra algun bien alto e gue de alcançar Pot que cerca los bienes tomados sueltamente &  \\\hline
1.3.5 & si non cerca de bien alto e guaue de alcançar . Ca ninguno non es & nisi circa bonum arduum ; nullus enim sperare dicitur , nisi sibi videatur esse bonum arduum , \\\hline
1.3.5 & para si non bien alto e guaue de alcançar ¶lo terçero la esperança ha de ser cerca el bien futuro que ha de uenir . & nisi sibi videatur esse bonum arduum , et difficile . Tertio spes habet esse circa bonum futurum : \\\hline
1.3.5 & e las riquezas e la nobleza non les siruen aellos por que puedan alcançar tales bienes . Mas los Reyes e los prinçipes alos quales sirue la nobleza de linage & et subtrahunt se ab aliquibus bonis arduis , videntur mereri indulgentiam , quia ciuilis potentia , diuitiae , \\\hline
1.3.5 & si non creyeren que ellos pueden alcançar tan grandes bienes e tan dignos de grand honrra . & et nobilitas non adminiculantur eis , ut possint prosequi talia bona : Reges autem et Principes , \\\hline
1.3.5 & que non pueden acabar nin alcançar ¶ En elsa misma manera avn podemos dezir & prorumpunt ut attentent aliqua quae consumate non possunt . Sic etiam dicere possemus , quod ebriosi plus sperant quam debent : \\\hline
2.1.3 & que faz de los maderos ¶as por que non puede alcançar la fin sin aquellas cosas & Agens enim primo et principaliter intendit finem . Verum quia non potest habere finem , nisi per ea , \\\hline
2.1.3 & e en execuçion dela obra es la manera contraria Ca por la obra alcançamos la fin obrando aquellas cosas que son ordenadas ala fin & sed in operando , et exequendo est econtrario . Nam per opus consequitur finem , operando ea quae sunt ad finem , ita quod ea quae sunt ad finem licet sint posteriora et in voluntate et in intentione , \\\hline
2.1.5 & e que sirua a aquel en que ha sabiduria e entendemiento por que alcançe salud por el e por que sea enderescado por el en sus obras . & qui viget prudentia et intellectu , ut consequatur salutem per eum , et ut dirigatur per ipsum . Nam deficiens intellectu , \\\hline
2.2.11 & en el terçero libro delas ethicas pequana delectaçiones quando la lengua alcança la uianda mas mayor delectaçiones & Nam ( ut innuit Philosophus 3 Ethicorum ) modica delectatio est , cum cibus attingit linguam : sed maior est , \\\hline
2.2.13 & por dos razones . ¶ Lo pripreo por escusar cuydado desconneible ¶ Lo segundo por alcançar fin conuenible ¶ La primera razon paresçe assi . & ut probat Philosophus 8 Poli’ est necessarius in vita quod ( quantum ad praesens spectat ) duplici via declarari potest . Primo , ex vitatione illicitae solicitudinis . Secundo , ex adeptione finis intenti . Prima via sic patet . \\\hline
2.2.13 & La segunda razon para prouar esto mismo se toma del alcançamiento dela fin & et quae sunt illae deductiones , circa quas debent vacare pueri , infra dicetur . Secunda via ad ostendendum hoc idem , sumitur ex adeptione finis intenti . Nam non semper statim quis habere potest finem intentum : ne ergo propter continuos labores deficiat a consecutione finis , \\\hline
2.2.13 & por que non fallezca el omne por los grandes trabaios de alcançar su fin conuienel de entreponer alguons trebeios & expedit aliquos ludos et aliquas deductiones interponere suis curis , ut ex hoc aliquam requiem recipientes , \\\hline
2.2.13 & assi que en esto resçibiendo alguno folgua a puedan mas trabaiar para alcançar su fin . Onde el philosofo enłviij̊ delas politicas dize & magis possint laborare in consecutione finis . Unde et Philosophus 8 Politicorum ait , quod quia homo non potest requiescere in fine adepto , \\\hline
2.2.13 & en que trabaia luengamente ante que alcançe aquella fin por ende conuienele de entroponer alguons trebeios & in quo diu laborat , antequam consequatur illum , ideo oportet interponere aliquos ludos , \\\hline
2.2.13 & e algunas delectaconnes por que non fallesca de alcançar a qual la fin . & et aliquas delectationes , ne deficiat a consecutione finis . Sic ergo instruendi sunt pueri erga ludos , \\\hline
2.2.21 & quanto alguna cosa que se puede auer mas patesçegue e alta de alcançar tanto mas parełçe &  \\\hline
2.2.21 & que los omes non pue den auer su conpanma nin pueden alcançar lo que quieren con ellas & eorum consortium videtur magis abesse , et videntur ipsae magis inaccessibiles : propter quod non sic vilipenduntur \\\hline
2.3.1 & assi conmo por prop̃os estrumentos pueda alcançar estas cosas que fazen al abastamiento dela uida . & quia talia sunt organa huius artis . Spectat ergo ad gubernationem domus talia cognoscere : quia per haec tanquam propria organa consequi poterit , quae faciunt ad sufficientiam vitae : \\\hline
2.3.8 & assi que por ellas cada vno cuyda que podra alcancar aquello que dessea . por ende los omes non se fartan de riquezas nin de possessions & cum diuitiae maxime videantur hoc efficere , ut per eas quilibet consequi possit quod appetit , ut melius homines possint explere \\\hline
2.3.8 & e segunt mesura de aquella fin que ha de alcançar assi commo si dixiessemos & ea vero quae sunt ad finem , secundum modum et mensuram ipsius finis . Ut si finis medicinae est sanare , \\\hline
3.1.2 & e quantos bienes son aquellos que los omes alcançan por establesçimiento dela çibdat & quae et quot sunt illa bona , quae ex constitutione ciuitatis homines consequuntur . Huiusmodi autem bona ( quantum ad praesens spectat ) tria esse contingit . Ordinatur enim ciuitas ad viuere , \\\hline
3.1.2 & ¶ Et pues que assi es el establesçimiento dela çibdat es razon de muchos e muy grandes bienes porque por ella alcançan los omes ser acabados e beuir en comunidat politica e de çibdat & Multorum igitur et maximorum bonorum causa est constitutio ciuitatis , quia per eam homines consequuntur omnia tria praedicta bona . Nam ipsum viuere consequuntur homines ex communitate politica : \\\hline
3.1.2 & Et por ende fablando del beuir assi commo de beuir commo omne los omes alcançan tal beuir por partiçipamiento politico o por establestimiento dela çibdat ¶ & de viuere ut homo ) consequuntur homines excommunicatione politica siue ex constitutione ciuitatis . Secundo ex tali constitutione homines non solum consequuntur viuere , \\\hline
3.1.2 & si los moradores della non puede fallar todas las cosas en ella que cunplen para la uida del omne ¶ Et pues que assi es nos alcançamos por establesçimiento dela çibdat & nunquam enim est ciuitas perfecta nisi habitatores eius ibi inuenire possint omnia sufficientia ad vitam . Consequimur ergo ex ciuitate non solum viuere ut homo , \\\hline
3.1.2 & e todo conplimiento e abastamiento de uida ¶ Lo terçero por establesçimiento de la çibdat alcançamos beuir uirtuosamente por que la entençion del fazedor dela ley non solamente deue ser & huiusmodi communitas est habens terminum omnis per se sufficientiae vitae . Tertio ex constitutione ciuitatis consequimur virtuose viuere : nam inceptio legislatoris \\\hline
3.1.4 & e qual non iusta por ende si la comunidat dela casa es ordenada a alcançar lo que es delectable e para foyr lo que es enpeçible . & et quid iniustum . Si ergo communitas domestica ordinatur ad prosequendum conferens , et ad fugiendum nociuum : \\\hline
3.1.8 & a algun prinçipe o algun sennor e commo en la çibdat conuenga de dar alguons ofiçioso alguons maestradgos o algunas alcaldias la qual cosa non seria & ut cum in ciuitate oporteat dare aliquos magistratus , et aliquas praeposituras , | dare aliquos magistratus , et aliquas praeposituras , quod non esset , \\\hline
3.1.10 & esto es vilificar los nobles e enxalcar los viles e assi se salua la amistança entre ellos & et exaltare ignobiles , et non saluare amicitiam inter eos . Tertium malum sic declaratur . Nam supposita praedicta communitate , \\\hline
3.1.18 & e mas se traban si non pueden alcançar la honrra digna e con ueinble a ellos . &  \\\hline
3.1.19 & Lo quarto se entremetio el dicho philosofo del departimiento de aquellos que iudgan ca dize que dos deuian ser los linages de los iudgadores e delos alcalłs e dias audiençias . La vna ordinaria & Dicebat enim debere esse duo genera iudicantium , et duplex praetorium : unum ordinarium , \\\hline
3.2.1 & e estas son ¶ El prinçipe Et el conseio . Et el alcalłia ¶ Et el pueblo . Enpero podemos de aquellas cosas & quatuor quae consideranda sunt in regimine ciuitatis . Haec autem sunt princeps , consilium , praetorium , \\\hline
3.2.1 & e guardadas por el prinçipe esto parte nesçe al alcaldia o alos uezes por que ellos son aqual los que segunt tales leyes deuen iudgar los fechs de los çibdadanos . Mas bien guardar las leyes puestas & et custoditas per principem , spectat ad praetorium siue ad iudices : ipsi enim sunt \\\hline
3.2.1 & ca deue entender el fazedor dela ley que por las leyes alcançemos lo que nos cunple & debet enim intendere legislator ut per leges consequamur conferens , et vitemus nociuum : \\\hline
3.2.1 & Pues que assi es de todas estas quatro cosas que dichas son conuiene desaƀ del prinçipado del conseio del alcalłia del pueble . diremos breuemente & et laudabile et fugiat vituperabile et turpe . De omnibus ergo his quatuor , videlicet de principatu , consilio , praetorio , et populo , \\\hline
3.2.8 & que la natura primeramente da a todas las cosas aquello por que pueden alcançar su fin . ¶ Lo segundo les da aquellas cosas &  \\\hline
3.2.8 & que en tal manera sea el pueblo apareiado e ordenado por que pue da alcançar su fin que entiende . & ut populus taliter disponatur et ordinetur , ut possit consequi finem intentum . Secundo , ut remoueantur prohibentia \\\hline
3.2.8 & Lo segundo conuiene que sean arredradas todas aquellas cosas que enbargan de alcançar aquella fin ¶Lo terçero conuiene & ut possit consequi finem intentum . Secundo , ut remoueantur prohibentia et deuiantia ab huiusmodi fine . Tertio , ut dirigatur et promoueatur in finem praedictum . \\\hline
3.2.8 & por que la gente que les acomnedada aya aquellas cosas por las quales puede alcançar la fin que entiende . & Quia primo solicitari debet , ut gens sibi commissa habeat per quae possit consequi finem intentum . Secundo debet prohibentia remouere . \\\hline
3.2.8 & ¶ Lo segundo deue arredrar todas aquellas cosas que enbargan de alcançar la fin . Lo terçero deue el Rey enderesçar & Secundo debet prohibentia remouere . Tertio gentem ipsam debet in finem dirigere . Ea vero quae deseruiunt \\\hline
3.2.8 & Mas aquellas cosas que siruen aesto por que el pueblo pueda alcançar su fin e pueda bien beuir son estas . & Ea vero quae deseruiunt ut populus possit consequi finem intentum et bene viuere , \\\hline
3.2.8 & mas es tirano . Lo segundo para alcançar la fin que entiende siruen buenos apareiamientos del alma e uirtudes & sed tyrannus . Secundo ad consequendum finem intentum deseruiunt boni habitus et virtutes . Non enim sufficit finem cognoscere , \\\hline
3.2.8 & en tal manera que quiera e pueda alcançar aquella fin . Et por ende parte nesçe al gouernador del regno de otdenar sus subditosa uirtudes & et habeat ordinatum appetitum , ut velit consequi finem illum : | ut velit consequi finem illum : spectat igitur ad rectorem regni ordinare suos subditos ad virtutes . Tertio ad consequendum \\\hline
3.2.8 & e a buenas costunbres ¶ Lo terçero para alcançar la fin que entiende enla uida çiuil & spectat igitur ad rectorem regni ordinare suos subditos ad virtutes . Tertio ad consequendum finem qui intenditur in vita politica , organice deseruiunt \\\hline
3.2.8 & en quanto ellas siruen a bien beuir e para alcançar la fin que entienden en la uida çiuil . & ad bene viuere , et ad consequendum finem intentum in vita politica . Hoc autem quomodo fieri possit , \\\hline
3.2.8 & de ser acuçioso çerca aquellas cosas por las quales el pueblo puede alcançar su fin que entiende finca de demostrar & licet per praecedentia sit aliqualiter manifestum , clarius tamen infra dicetur . Viso quod spectat ad Regis officium solicitari circa ea per quae possit populus consequi finem intentum : restat ostendere , \\\hline
3.2.8 & que enbargan al pueblo de alcançar su fin los quales enbargos son tres de los quales . ¶ El vno toma nasçençia dela natura & quomodo spectat ad Reges et Principes huiusmodi prohibentia remouere . Quae etiam tria sunt , \\\hline
3.2.8 & que le es a comnedado aya aquellas cosas por que pueda alcançar su fin . Et en qual manera deuen arredr e tirar los enbargos & plenius ostendetur . Ostenso quomodo Reges et Principes solicitari debent , ut populus sibi commissus habeat per quae possit finem consequi , et quomodo debeant prohibentia remouere : \\\hline
3.2.13 & njn quiere el bien comun quariendo algunos alcançar la gloriar la honrra que veen enel tirano acometen ler matanle , & et non quaerere commune bonum , volentes adipisci honorem et gloriam quam conspiciunt in tyranno , inuadunt eum , \\\hline
3.2.16 & Et el conseio . Et el alcalłia Et el pueblo . Et pues que assi es despues & videlicet Principem , consilium , praetorium , et populum . Postquam ergo auxiliante deo determinauimus de Principe , \\\hline
3.2.16 & Et por que escusemos los males e alcançemos los bienes . Et pues que assi es aquellas cosas & et ut vitemus mala , et ut consequamur bona : quae ergo vitari non possunt , \\\hline
3.2.16 & mas de aquellas cosas por que podemos alcançar aquella fin . Ca el fisico & et non consiliari de ipso , sed de iis per quae consequi possumus illud . Medicus enim quia finaliter intendit sanitatem , \\\hline
3.2.17 & que contezca alguna dos auentura por que non pueda alcançar el bien que desseao & et dubitare ne aliquo infortunio contingente deficiat a consecutione optati boni , vel incurrat aliquod damnum , \\\hline
3.2.20 & segunt la orden sobredichͣ que digamos del alcalłia o del iuyzio mostrando & secundum ordinem praetaxatum ut exequamur de praetorio , siue de iudicio , | ut exequamur de praetorio , siue de iudicio , inuestigando qualiter iudicandum sit , \\\hline
3.2.20 & ca en cada vna çibdat conuiene que aya vna alcalłia otdinaria ala qual deuen venir todos los pleitos & secundum leges iam conditas . Nam in qualibet ciuitate oporteret esse aliquod praetorium ordinarium ad quod causae reducantur et litigia exorta in ciuitate illa : \\\hline
3.2.20 & et de las cosas generales Mas el alcalłe o el iues iudga delas cosas presentes e delas perssonas señaladas . & et in uniuersali : sed praefectus aut iudex iudicat de praesentibus et determinatis , ad quos est amare , \\\hline
3.2.20 & alas quales puede auer amor o abortençia . Alos quales alcalłs muchos vezes se ayunta algun pro para si mesmos & ad quos est amare , et odire , et quibus proprium commodum annexum est saepe . Quarta via ad ostendendum hoc idem , \\\hline
3.2.21 & e de deue dar ante los alcalłs rodemos lo prouar por tres razones ¶ La primera seqma &  \\\hline
3.2.25 & e en el se fuudaron . Ca en todos los derechos es entendido o alcançar el bien o foyr del mal . & et in eo fundantur : nam in omnibus attenditur vel consecutio boni , | nam in omnibus attenditur vel consecutio boni , vel fuga mali . \\\hline
3.2.26 & e esto queremos alcançar conuiene que fagamos estas cosas . Et pues que assi estales deuen ser las leyes & et hoc sequi volumus , oportet hoc agere . Tales ergo debent esse leges , \\\hline
3.2.28 & a todos los omes e todos los omes comunalmente non pueden alcançar al medio nin al conplimiento dela bondat nin dela uirtud & tamen quod lex imponitur quasi communiter omnibus , et omnes communiter punctaliter non attingunt medium bonitatis , quia non omnes possunt esse omnino perfecti : \\\hline
3.2.30 & nin da pena por todos los pecados la qual cosa contesçe por dos razones La primera es por que el pueblo comunalmente non puede alcançar forma de beuir en punto . Por ende conuiene & lex humana non omnia peccata punit , quod duplici de causa contingit . Prima est , quia communiter populus non potest attingere punctalem formam viuendi , | quod duplici de causa contingit . Prima est , quia communiter populus non potest attingere punctalem formam viuendi , ideo oportet aliqua peccata dissimulare \\\hline
3.2.30 & la ley natural e la humanal que nos ayudan a alcançar este bien . el qual non podemos natural mente alcançar non cunplen & lex naturalis et humana iuuantes nos ad consecutionem illius boni quod possumus naturaliter adipisci , \\\hline
3.2.30 & que nos ayudan a alcançar este bien . el qual non podemos natural mente alcançar non cunplen para alcançar este bien & lex naturalis et humana iuuantes nos ad consecutionem illius boni quod possumus naturaliter adipisci , \\\hline
3.2.30 & alcançar non cunplen para alcançar este bien que es sobre natural . & et humana iuuantes nos ad consecutionem illius boni quod possumus naturaliter adipisci , non sufficiunt ad consequendum illud bonum supernaturale ; \\\hline
3.2.34 & podria ser guardada algunan egualdat entre los çibdadan uanto alo presente parte nesçe el pueblo alcança tres bienes si obedesçiere alos Reyes e alos prinçipes con grand acuçia . & et terrarum , poterit aliqualis aequalitas reseruari inter ciues . Consequitur autem populus ( quantum ad praesens spectat ) tria , | Consequitur autem populus ( quantum ad praesens spectat ) tria , si cum magna diligentia obediat regibus , \\\hline
3.2.36 & por que cuyda sienpre el pueblo que por tales alcançara salut et bien . Et por ende dize el pho en el segundo libro de la rectorica & exponentes se pro bonis communibus : credit enim per tales salutem consequi . Ideo dicitur 2 Rhetor’ quod quia diligimus beneficos in salutem , id est eos \\\hline
3.3.1 & do tracta del Fecho de la . Mas uale para alcançar uictoria que la muchedubre o ahun la fortaleza de los lidiadores & ut patet per Vegetium in De re militari , plus confert ad obtinendam victoriam , quam faciat multitudo | plus confert ad obtinendam victoriam , quam faciat multitudo vel fortitudo bellantium . Opus autem bellicum \\\hline
3.3.3 & Ca si quier sea cauallero si quier peon el que ha de lidiar paresçe que por uentura alcaça uictoria si non ouiere sabiduria de lidiar & Nam siue equitem siue peditem oportet esse bellantem , quasi fortuito videtur peruenire ad palmam , si caret industria bellandi . \\\hline
3.3.4 & e se mueue de vna parte a otra apenas o nunca le puede alcançar ninguna ferida . Mas por la mayor parte sienpre fuye los colpes . & expertum est enim quod homine continue se ducente et mouente , vix aut nunquam ad plenum aliqua percussio potest ipsum attingere , sed semper vulnera subterfugit . \\\hline
3.3.6 & ca buena cosa es en la fazienda auer algunos mas ligeros que los otros que non puedan ser alcançados do ligero de los enemigos que vayan e escuchar e a saber las condiciones e el estado delos enemigos & Nam bonum est in exercitu aliquos exiliores praecurrere , qui de facili non possint ab ipsis hostibus comprehendit , explorantes conditiones \\\hline
3.3.6 & para seguir e alcançar los enemigos quando fuyen . Ca no puede ninguno de ligero foyr de las manos de aquellos & facilius obtinebunt aptiorem locum ad pugnandum . Est etiam hoc utile ad prosequendum hostes fugientes . Nam non de facili quis potest euadere manus agilium \\\hline
3.3.7 & nin en sofrir quales quier otros trabaios de la batalla . Lo terçero son de usar los lidiadores a alcançar dardos e azconetas e a ferir con lanças & vel in sustinendo quoscunque labores bellicos . Tertio exercitandi sunt bellatores admittendum tela et iacula , | Tertio exercitandi sunt bellatores admittendum tela et iacula , et ad percutiendum cum lancea : \\\hline
3.3.9 & Ca quanto mas sabios son los lidiadores tanto mas ayna alcançan victoria . Lo sexto es de penssar el esfuerço & Nam quanto cautiores sunt bellatores , tanto citius victoriam obtinent . Sexto , attendenda est virilitas et audacia mentis , quia audaciores \\\hline
3.3.9 & e de mayores coraçons por la mayor parte alcançan en la batalla la victoria . Et pues que assi es el rey o el prinçipeo el señor de la hueste & et magis cordati ut plurimum in pugna victoriam obtinent . Rex igitur \\\hline

\end{tabular}
