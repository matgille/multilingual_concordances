\begin{tabular}{|p{1cm}|p{6.5cm}|p{6.5cm}|}

\hline
1.1.11 & e por guarda del su linage & siue propter procreationem prolis : \\\hline
1.2.7 & por que el linage delos . omes ha sabiduria & quia hominum genus viget prudentia : \\\hline
1.2.15 & e a continuaçion del humanal linage . & ut matrimonia , \\\hline
1.2.15 & e por que fuesse saluado el humanal linage . & et tactus , \\\hline
1.2.32 & finca de dezir delos quatro linages de los buenos & restat dicere de quatuor generibus bonorum . \\\hline
1.2.33 & assi conuiene de dar den parti dos linages de uirtudes . & ita quod \\\hline
1.2.33 & Et pues que assi es assi conmo el philosofo pone quatro linages de buenos commo paresçe & Sicut ergo Philosophus innuit quatuor genera bonorum , \\\hline
1.2.33 & assi que a cada vn linage de los bueons demos su orden de uirtudes . & aliquos vero diuinos , \\\hline
1.2.33 & Et que estos linages destas uirtudesse & et purgatoriae politicas . \\\hline
1.2.33 & alos dichos linages de los buenos esto paresçe & Quod autem haec genera virtutum adaptari debeant praedictis generibus bonorum , \\\hline
1.2.33 & mas baroguado en el linage de los buenos . & ut de ea Philosophus loquitur ) tenent infimum gradum in genere bonorum . \\\hline
1.2.34 & conuiene aellos de conosçer estos linages & decet eos haec genera dispositionum cognoscere . \\\hline
1.3.2 & Por ende en el linage delas passiones & ideo in genere passionum ira mansuetudinem praecedit . \\\hline
1.3.5 & Mas los Reyes e los prinçipes alos quales sirue la nobleza de linage & videntur mereri indulgentiam , \\\hline
1.3.8 & si la delectaçion non fuesse sinplemente del linage de los bienes ¶ & nisi delectatio simpliciter esset de genere bonorum . \\\hline
1.3.11 & e por que son en linage de los bienes & quae minima sunt in genere bonorum , \\\hline
1.4.5 & que honrra de linage . & Ex hoc enim aliqui dicuntur esse nobiles , \\\hline
1.4.5 & por que desçenden de linage honrrado . & quia processerunt ex genere honorabili . Genus autem honorabile dicitur \\\hline
1.4.5 & Mas el linage es dicho honrrado & quia processerunt ex genere honorabili . Genus autem honorabile dicitur \\\hline
1.4.5 & si antigua miente desçendieron de aquel linage muchos granados omes et muchos nobles . & si ab antiquo ex illo genere processerunt multi praesides , \\\hline
1.4.5 & Et pues que assi es la uirtud del linage & et multi insignes . Virtus ergo generis , \\\hline
1.4.5 & si non ser de algunt linage alto &  \\\hline
1.4.5 & e en su linage fueres mucho nobles e ricos leunatase el coraçon de los nobles & quia ergo nobiles ex antiquo fuerunt praesides , et in suo genere fuerunt multi insignes et diuites , \\\hline
1.4.5 & que en el su linage fueron muchos nobles & quod in eorum genere fuerunt multi insignes , \\\hline
1.4.5 & o non es al si non uirtud e honrra del linage . & vel quod virtus \\\hline
1.4.5 & Mas el linage de algunos prinçipalmente es tenido por hanrrado & et honorabilitas generis . Genus autem alicuius maxime reputatur honorabile , \\\hline
1.4.5 & por el su linage & quia ex suo genere videntur esse honorabiles , ideo volunt accumulare \\\hline
1.4.7 & por que desçendende linage honrrado . & quia processerunt ex aliquo nobili genere , \\\hline
1.4.7 & por que non desçenden de linage honrrado & quia non processerunt ex honorabili genere , \\\hline
1.4.7 & e non desçendiere de linage antigo e honrrado & si non sit nobilis , et non processerit ex quodam genere antiquo \\\hline
2.1.2 & Et assi yendo actesçentandose el linage dellos & Sic procedente generatione ipsorum , \\\hline
2.1.5 & alo menos son menester y . tres linages de perssonas . & quomodo ad constitutionem huius domus saltem requiruntur ibi tria genera personarum . \\\hline
2.1.5 & e cada vna delas comunidades aya menester dos perssonas o dos linages de perssonas & et quaelibet communitas requirat duas personas | uel duo genera personarum , \\\hline
2.1.5 & por razon destas dos comini dades ha menester quatro linages de ꝑssonas . & quod domus prima ratione duarum communitatum requirat quatuor genera personarum , \\\hline
2.1.5 & Pues que assi es tres linages de perssonas & Tria ergo genera personarum , \\\hline
2.1.5 & quantos son los linages delas perssonas & et scire quot genera personarum , \\\hline
2.1.6 & que conuiene que sean y . quatro linages de perssonas & quod ibi oportet esse quatuor genera personarum . Videretur \\\hline
2.1.6 & Empero podrie paresçer a alguno por auentura que deuen y ser seys linages de perssonas alssi que la primera perssona deue ser ꝑ el uaron¶ La segunda dela muger ¶ & tamen forte alicui ibi debere esse sex genera personarum , | ita quod prima persona sit ibi vir , secunda uxor , \\\hline
2.1.6 & e quantos los gouernamientos et quantos los linages delas perssonas . & et quot genera personarum . \\\hline
2.1.7 & aduze se aquetro linages o a quatro maneras ¶ & ad quadruplex genus reducitur : \\\hline
2.1.7 & que conuiene mucho al mantene miento del humanal linage & quae maxime expedit conseruationi speciei , \\\hline
2.1.7 & e al mantenemiento del humanal linage & et regnum non sic immediate ordinantur ad nutritionem propter bonum personae propriae , et ad generationem propter conseruationem speciei , \\\hline
2.1.9 & e en otros muchos linages de aues & et in aliis pluribus generibus auium , quae alternatim oua fouent ) \\\hline
2.1.11 & de qual se quier linage & cuiuscunque generis , \\\hline
2.1.11 & por su linage a algun uaron non es de tomar & quod nimis propinqua ex suo genere non est per coniugium socianda , \\\hline
2.1.12 & e en tres linages . & quae ( quantum ad praesens spectat ) in triplici genere habent esse . \\\hline
2.1.12 & Ca la honrra del linage & Nam honorabilitas generis , \\\hline
2.1.12 & ala nobleza del linage & Sed ad nobilitatem generis , \\\hline
2.1.12 & e por si deuen entender a la nobleza del linage & inter exteriora bona debent intendere \\\hline
2.1.12 & que sea noble por linage & quae sit nobilis genere , \\\hline
2.1.12 & ala nobleza del linage & ad debitam societatem , | ad pacificum esse , \\\hline
2.1.12 & que son nobles por linage & quos constat esse nobiles genere , \\\hline
2.1.12 & e de los nobles es de demandar nobleza de linage . & et nobilium quaerenda est nobilitas generis , \\\hline
2.1.12 & que ellas sean nobles de linage & quod sint nobiles genere , \\\hline
2.1.12 & Conuiene a saberhonrra de linage . & videlicet , honorabilitas generis , multitudo amicorum , \\\hline
2.1.13 & assi commo enlas o trisaian lias en la mayor parte de grand linage sallen grandes animalias & ut plurimum ex magno genere magna procedunt : \\\hline
2.1.13 & Et en qual manera son nobles por linage . & ut quomodo sint nobiles genere , \\\hline
2.2.5 & e que el humanal linage fue ensuziado & et spiritus sanctus . Quod Adam primo parente nostro peccante , et humano genere per peccatum eius infecto , \\\hline
2.3.9 & todas las muda connes son aduchos a tres linages & omnes commutationes | quasi ad tria genera reducuntur . \\\hline
2.3.13 & La terçera de departidos linages de ainalias ¶ & Tertia ex diuersis speciebus animalium . Quarta ex diuersitate sexuum in specie humana . Prima via sic patet . \\\hline
2.3.13 & La quarta del departimiento dela mena la muger en el humanal linage & Tertia ex diuersis speciebus animalium . Quarta ex diuersitate sexuum in specie humana . Prima via sic patet . \\\hline
2.3.13 & por departimiento de los linages masculino e femenino & Videmus enim virum , \\\hline
2.3.15 & que mucho son nasçidos de noble linage & quia contingit \\\hline
2.3.18 & assi conmola nobleza del linage ¶ Et otra segunt uerdat & ut nobilitatem generis : | et aliam \\\hline
2.3.18 & assi commo son dos linages de grandes bienes . & nisi in aliquo excederent illum . Sicut ergo sunt duo genera bonorum magnorum , \\\hline
2.3.18 & assi commo son honrra de linage riquezas e poderio çiuil & huiusmodi sunt honorabilitas generis , diuitiae , | et ciuilis potentia . \\\hline
2.3.18 & e por ende el que viene de honrrado linage & Qui ergo processit ex genere horabili , ut ex diuitibus , vel ex potentibus , \\\hline
2.3.18 & e estos tales son dichos auer nobleza de linage & dicitur habere nobilitates generis , \\\hline
2.3.18 & por que vemos enla mayor parte que los nobles de linage son de meiors costunbres & huiusmodi vulgaris opinio alicui probabilitati innititur . Videmus enim ut plurimum quod nobiles genere sunt nobiliorum morum quam alii : \\\hline
2.3.18 & por que alguons nobles de linage de su naian se & Quidam enim nobiles genere degenerant a naturae nobilitate , \\\hline
2.3.18 & si la nobleza del linage es nobleza & Si ergo nobilitas generis est nobilitas magis \\\hline
2.3.18 & Et por ende cosa conueinble es que los nobles por linage sean nobles por costunbres & decens est nobiles genere esse nobiles | secundum mores . \\\hline
3.1.2 & que han ser conplido segunt su natura e su linage & quae secundum suam speciem habent esse completum . \\\hline
3.1.3 & e sin ley commo las aues todos estos linages de oms son maldichos & Aliqui autem sunt tantae perfectionis , \\\hline
3.1.9 & assi commo aquel que viene de su linage & qui sunt ex duobus fratribus nati ) alius vero diligit ipsum tanquam contribuelem idest tanquam natum ex eadem tribu , \\\hline
3.1.19 & que tannian alguons linages de personas dezimos & Quinto de modo iudicandi . Sexto et ultimo statuit quasdam leges tangentes diuersa genera personarum . Hippodamus autem statuens suam politiam , \\\hline
3.1.19 & Et para estas tres cosas siruen los tres linages sobredichos de los uarones & Ad haec enim tria deseruiunt praedicta tria genera virorum . \\\hline
3.1.19 & Lo quarto se entremetio el dicho philosofo del departimiento de aquellos que iudgan ca dize que dos deuian ser los linages de los iudgadores & Quarto intromisit se de distinctione iudicantium . | Dicebat enim debere esse duo genera iudicantium , \\\hline
3.1.19 & que tannian a algunos linages de personas & tangentes diuersa personarum genera . \\\hline
3.1.20 & que el establesçio tanniendo departidos linages de perssonas . & quem statuit , tangentes diuersa genera personatum . Primo enim dictus Phil’ deferre fecit statuendo impossibilia . \\\hline
3.2.2 & departe el philosofo seys linaies de prinçipados & Tertio Politicorum distinguit Philosophus sex modos principantium , \\\hline
3.2.5 & e el linage &  \\\hline
3.2.5 & de determinar el linage donde ha de ser tomado el sennor . & ex qua praeficiendus est dominus , \\\hline
3.2.5 & de qual linage ha de ser tomado el Rey & et lites , \\\hline
3.2.5 & en qual linage deua ser prinçipe & si ignoretur ex qua prosapia assumendus sit Rex : sic etiam litigia oriuntur , \\\hline
3.2.5 & ca segunt el linage del patente &  \\\hline
3.2.16 & quantos son los linages & et distinximus quot sunt genera principatus , \\\hline
3.2.25 & mas es comuna todo el humanal linage Et pues que assi es deste derecho ssallen & quod non est commune animalibus aliis : sed commune est omni humano generi . \\\hline
3.2.25 & el qual derecho esppreo solamente al linage humanal . & quod est proprium soli humano generi . \\\hline
3.2.32 & Et era propreo al humanal linage & et erat proprium humano generi : \\\hline
3.2.32 & e de alto linage . & Nam regnum supra ciuitatem videtur addere multitudinem nobilium et ingenuorum . Est enim ciuitas pars regni ; \\\hline
3.2.32 & que si abondasse en riquezas o en nobleza de linage & quam si abundaret in diuitiis , | vel in nobilitate generis , \\\hline
3.2.32 & que aquellos que han nobleza de linage & quam habentes nobilitatem generis , ciuilem potentiam , et multitudinem ipsorum bonorum exteriorum . \\\hline
3.3.5 & Ca entre todas las gentes el linage de los rusticos & Nam inter ceteras gentes ruralium gens videtur esse crudelius . \\\hline

\end{tabular}
