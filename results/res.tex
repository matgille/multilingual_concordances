\begin{tabular}{|p{1cm}|p{6.5cm}|p{6.5cm}|}

\hline
1.1.1 & Oportet ut latitudo sermonis in unaquaque re sit & onuiene que la largueza delos sermones | e delas palabras en cada cosasea \\\hline
1.1.1 & secundum exigentiam illius rei , & segund que demanda aquella cosa \\\hline
1.1.1 & inquantum natura rei recipit . Videtur enim natura & en quanto la naturaleza dessa mismͣ cosa lo demanda \\\hline
1.1.1 & rei moralis omnino esse opposita negocio mathematico . & Ca semeja la naturaleza Dela sçiençia moral del todo ser contraria ala sçiençia matematica \\\hline
1.1.1 & et grossae . Geometrae igitur est non persuadere , sed demonstrare : & donde se sigue quel geometrico non ha de amonestar | mas de demostrar \\\hline
1.1.2 & quod hunc totalem librum intendimus in tres partiales libros diuidere . & que todo este libro entendemos partir en tres libros particulares \\\hline
1.1.2 & quia finis respectu agendorum , est principalius principium , & por que la fin en conparaçion delas obras | que se ha de fazer \\\hline
1.1.3 & restat & e engannosa contando la orden de las cosas que aqui auemos de dezir ¶finca que en este terçero capitulo fagamos la Real magestad atenta \\\hline
1.1.3 & ubi modus rei requirit traditionem figuralem & que dicho nes sea corruꝑçion del apetito \\\hline
1.1.4 & quia respectu sequentis operis & Ca en rrelaçion desta obra | que se sigue \\\hline
1.1.4 & restat dicere seriatim quae in hoc opere sunt dicenda . & que se falla enestas cosas | que auemos de dezer fincanos de dezer ordenadamente \\\hline
1.1.4 & Secundum has tres considerationes sumptae sunt a Philosophis praedictae tres vitae voluerunt enim & Segunt estos tres pensamjentos tomaron los philosofos las tres uidas sobredichas | Ca quisieron alguons philosofos \\\hline
1.1.4 & ut angelus . Distinxerunt ergo has tres vitas , & pues que asy es asy departieron los philosofos estas tres vidas \\\hline
1.1.4 & siue hos tres modos viuendi : & o estas tres maneras de beuir \\\hline
1.1.5 & tria requiruntur . Primo , & Lo primero que faga bien \\\hline
1.1.5 & Cum ergo ista tria contingunt , & Et quando estas tres cosas todas uienen en vno \\\hline
1.1.6 & Quantum enim ad praesens spectat , felicitas tria importare videtur . Felicitas enim dicit perfectum , & non es de poner la feliçidat e la bien andança . Ca quanto parte nesçe alo prèsente la bien andança ençierra en si tres cosas ¶ | La primera es \\\hline
1.1.6 & bona corporis sunt imperfecta respectu bonorum animae : & asi los bienes del cuerpo deuen ser ordenados alos bienes de łalma | Ca son menguados en conparaçonn delos bienes del alma¶ \\\hline
1.1.6 & ( ut dicitur 5 Politicorum principatus debet respondere magnitudini , & Ca asi commo dize el philosofo en el quinto libro delas politicas el prinçipado \\\hline
1.1.7 & quae naturaliter ex rebus naturalibus producuntur , & que uienen n atraalmente de cosas natraales \\\hline
1.1.7 & In neutris autem diuitiis est ponenda felicitas . Tangit enim 1 Politicor’ tria , & Ca el philosofo tanne tres razones | enl primero libro delas politicas \\\hline
1.1.7 & Cum ergo tactus reseruetur in singulis partibus corporis , & asi que quanto el tannia todo se le tornaua oro | Et por que el seso del tanner es en todas las partes del cuerpo \\\hline
1.1.7 & tria maxima mala inde consequuntur . Primo , & tres males muy grandes se le siguen dende ¶ El pmero mal es que pierde muy grandes bienes ¶ | El segundo es que se faze \\\hline
1.1.7 & principaliter intendit reseruare sibi , & prinçipalmente entiende de thesaurizar | e fazer thesoro \\\hline
1.1.7 & si depraedetur populum et Rem publicam , & e si robare el pueblo | e desfiziere la comunidat \\\hline
1.1.7 & quod tamen bonum Reipublicae non procuret : & que non procure el bien del pueblo | mas el suyo propio . \\\hline
1.1.8 & Sunt autem in honoribus tria attendenda , & Mas en las honrras son tres cosas de cuydar \\\hline
1.1.8 & Si enim Rex suam felicitatem in honoribus ponat , sequentur ipsum tria mala : erit enim superficialiter bonus erit praesumptuosus , et erit iniustus , et inaequale . & Ca si el Rey pone su bienandança en las honrras | siguen se le tres males ¶ \\\hline
1.1.8 & erit malus in suis rebus , & Et sera malo al pueblo \\\hline
1.1.9 & cum scientia nostra non sit ipsa res , & Como el nuestro connosçimiento non sea aque|p{1cm}|p{6.5cm}|p{6.5cm}|la cosa \\\hline
1.1.9 & nec sit causa rerum , & de que es | nin sea razon dellas \\\hline
1.1.9 & ( ut ad praesens spectat ) in tribus differt a notitia nostra : & en tres cosas | se departe el conosçimiento de dios del nuestro conosçimiento . \\\hline
1.1.9 & nam notitia Dei causat res , & Ca el conosçimiento de dios fizo | e faze todas las cosas \\\hline
1.1.9 & notitia nostra causatur a rebus , ut vult Commen’ 12 Met’ . & mas el nuestro conostimiento es fecho delas cosas que dios fizo . | asi commo dize el philosofo \\\hline
1.1.9 & quasi punctus respectu caeli , & asi common punto en conparaçion del çielo . \\\hline
1.1.9 & cum totum tempus vitae praesentis sit quasi punctale respectu Dei aeternitatis : & Ca todo el tpon dela uida presente de los omeses | assi commo vn momneto \\\hline
1.1.10 & Vegetius in libro De re militari , & ize vegeçio enł primero libro | que fizo dela caualleria \\\hline
1.1.10 & quia maxime dederunt operam rebus bellicis , & por que sobre todo lo al se dieron alas obras delas batallas \\\hline
1.1.11 & tamen tria bona corporis , & Empero tres son los bienes corporales \\\hline
1.1.11 & et semper forma magis dicit naturam rei , & que el cuerpo . | Et sienpre la forma es mas natural dela cosa \\\hline
1.1.11 & in qua reseruatur decor , & en que es guardada la color e la fermosura ¶ \\\hline
1.1.13 & Rex maxime conformatur ei : ergo respectu Dei , & Et por ende en conparaçon de dios \\\hline
1.1.13 & multas transgressiones efficerent . Propter quod Ethic’ 5 scribitur , & para fazer muchͣs males . | Et por esso dize aristotiles en el quinto libro delas ethicas \\\hline
1.2.1 & restat ostendere quibus virtutibus Reges pollere debeant . Virtutes autem quaedam sunt quidam ornatus , & e assignada finca nos de demostrar en quales uirtudes deuen resplandesçer los Reyes e los prinçipes . | Ca las uirtudes son vnos hornamentos e conponimientos e hunas perfectiones \\\hline
1.2.1 & His ergo tribus rationibus , & Et pues que assi es | por estas tres razones mismas \\\hline
1.2.1 & vel bene crescit : & nin por que cresçe bien \\\hline
1.2.2 & restat ostendere , & Et mostrado en qual manera han de seer en el entendimiento e en el appetito . \\\hline
1.2.2 & si natura elementis et rebus inanimatis dedit duplicem potentiam , & si la natura de los helementos dio a todas las cosas | que non han alma dos appetitos e dos poderios . \\\hline
1.2.2 & et aliam per quam resistunt , & e en que fuelgan . | Et dioles otro poderiopo el qual contradizen alos sus contrarios \\\hline
1.2.2 & ut per eam resisteret , & con que pudiesse uençer \\\hline
1.2.2 & et rebus inanimatis natura dedit duplicem potentiam , unam per quam adipiscuntur propriam quietem , et aliam per quam agunt in prohibentia et contraria : & Et otro por que pudiessen uençer | e corronper a sus contrarios mucho mas la natura dio estos dos po de rios alas aian lias \\\hline
1.2.2 & et alium per quem resistunt , & por el qual contradizen | e uan contra todas aquellas cosas \\\hline
1.2.2 & per frigiditatem vero resistunt contrariis , & que les es conuenible por natura | e por la frialdat se arrefieren de sus contrarios \\\hline
1.2.2 & quod concupiscibilis respicit bonum , & qual es ael conueinble . | Por la qual cosa es bien dicho \\\hline
1.2.2 & secundum se : irascibilis vero respicit bonum , & Mas el appetito enssanador cata | por el bien e por el mal \\\hline
1.2.2 & secundum quam aggredimur , et resistimus prohibentibus , et nociuis , & e los males que sienten tomando los segunt que son en si ¶ \\\hline
1.2.2 & per quam resistant , & por el qual pueden acometer e arredrar tondas las cosas \\\hline
1.2.2 & cum intellectus uniuersaliori modo respiciat suum obiectum quam sensus , & Ca commo el entendimiento sea mas general que el seso . | Mas generalmente cata aquello en que ha de obrar \\\hline
1.2.3 & huiusmodi autem , tria sunt , scilicet , & Et estos son tres conuiene de saber quales . \\\hline
1.2.3 & proprius aequat , et moderat ipsas res , & Ca primeramente yguala e mesura las cosas o las obras \\\hline
1.2.3 & et aequatis rebus , & Ca quanto mesuradas e ygualadas las cosas e las obras \\\hline
1.2.3 & et in ipsis rebus efficit illud bonum , & e en las cosas | que fazemos faze aquel bien \\\hline
1.2.3 & ( ut dicitur 4 Ethicorum ) quia non habemus proprium nomen respectu huiusmodi virtutis , & Enpero assi commo dize el philosofo en el quarto libro delas ethicas | por que non auemos nonbre propio \\\hline
1.2.3 & ut communicamus cum aliis , sic ( ut dicitur secundo Ethicorum ) sumuntur tres virtutes . & assi commo dize el philosofo en el segundo libro delas ethicas . | pueden se tomar tres uirtudes . \\\hline
1.2.3 & nec sit agrestis , & nin sea monte \\\hline
1.2.3 & Omnes autem hae tres virtutes , & Mas estas tres uirtudes ya dichas \\\hline
1.2.3 & quia tres harum , & Ca las tres dellas \\\hline
1.2.3 & aliae vero tres , & Mas las otras tres se toman \\\hline
1.2.3 & secundum ergo utraque bona , sunt tres virtutes . & pues que assi es segund estos de ꝑti dos bienes son tres uirtudes . \\\hline
1.2.3 & Nam bona hominis in se tria sunt : & Ca los bienes del omne en si son tres . \\\hline
1.2.5 & et cardinales respectu aliarum , & e caddinales en conparaçonn delas otras . \\\hline
1.2.5 & cum iustitia sit circa commutationes rerum , & Por que la iusticia es cerca de aquellas cosas | que se canbian vna por otra . \\\hline
1.2.6 & respectu quarum est praeceptiua : & alas quales ha demandar¶ \\\hline
1.2.6 & respectu quorum est praeceptiua : & en conparaçon delas quales es señora e mandadora . \\\hline
1.2.6 & tria habere debemus . Primo debemus diuersas vias inuenire . & que fazemos tres cosas deuemos auer ¶ | Lo primero deuemos buscar muchas carreras e departidas ¶ \\\hline
1.2.6 & In intellectu ergo nostro debent esse tres virtutes . & Et pues que assi es en el nuestro entendimiento | deuen ser tres uirtudes ¶ Vna \\\hline
1.2.6 & et hanc dicimus esse prudentiam . Prudentia ergo respectu virtutis inuentiuae & que es pradençia . | Et pues que assi es la pradençia en conparaçion dela uirtud falladora \\\hline
1.2.6 & Prudentia ergo respectu virtutis inuentiuae & Et pues que assi es la pradençia en conparaçiondestas dos uirtudes falladora e iudgadora \\\hline
1.2.6 & est de rebus necessariis , & que han sustançia non ꝑ mudable | que seño puede mudar . \\\hline
1.2.6 & et rerum contingentium , & pue dese | assi difinir e declarar . \\\hline
1.2.6 & Nam ars est respectu factibilium , & porque el arte es en conpara con delas cosas | que se pueden fazer \\\hline
1.2.6 & Prudentia vero est respectu agibilium , & mas la pradençia es en conparaçion delas cosas | que se han de fazer \\\hline
1.2.7 & restat ostendere , & finca de demostrar \\\hline
1.2.7 & ( quantum ad praesens spectat ) tria quae maxime Rex attendere debet . & Mas quanto pertenesçe alo presente tres cosas son | a que muchon deue tener mientes el Rey \\\hline
1.2.7 & secundum rei veritatem , & Lo segundo deue estudiar el Rey | que el su prinçipadgo \\\hline
1.2.7 & secundum rei veritatem , & segunt uerdatmas \\\hline
1.2.7 & secundum rei veritatem , & que sin la prudençia niguno non puede ser rey segunt uirdat . \\\hline
1.2.7 & sit Rex non solum nomine sed re , & por que el sea Rey non solamente segunt el nonbre \\\hline
1.2.8 & Experientia enim est rerum particularium . & por que la praeua es delas cosas particulares \\\hline
1.2.8 & et alia exquirenda . Oportet igitur Principem respectu gentis cui praeest , esse expertum , cognoscendo particulares conditiones gentis sibi commissae , & de ser my prouado conosçiendo las condiconnes particulares de su gente e de su pueblo | por que pue da meior guiar e gouernar su pueblo e su gente \\\hline
1.2.8 & respuendo apparenter bona , & e despreçiar aquellas cosas | que paresçen bueans \\\hline
1.2.10 & et ad rempublicam , & e al bien comun \\\hline
1.2.10 & erit in eis Iustitia aequalis . Bonum enim commune resultat & assi es enllos iustiçia ygual . Ca el bien comun nasçe de todos los bienes de los çibdadanos . \\\hline
1.2.10 & res enim publica , et tota ciuitas melior est , & Ca el bien comun | e toda la çibdat es meior \\\hline
1.2.10 & ex hoc resultet commune bonum , & e donde se leunato el bien comun \\\hline
1.2.10 & secundum quod huiusmodi resultat bonitas alterius ciuis : & por si en quanto tales non se leunata bondat del otro çibdadano \\\hline
1.2.10 & quia aliquando aliqui plus laborantes pro Republica , & Ca alas vezes algunos trabaian | mas por la comunidat \\\hline
1.2.10 & quod declarata sunt illa tria , & Ca declaradas son aquellas tres cosas \\\hline
1.2.11 & principatus sit ipsorum subditorum per respectum ad leges , & e el principado sea de los subditos por conparaçion alas leyes e al prinçipe \\\hline
1.2.11 & non reseruaretur in eis ordo nec ad leges , & non seria en ellos guardada orden . \\\hline
1.2.11 & Non ergo ulterius reseruaretur & Et pues que assi es non se guardaria dende adelante en ellos comunidat \\\hline
1.2.11 & aliquo modo reseruatur in ipsis Iustitia distributiua . & en alguna manera es fallada en ellos iustiçia partidora \\\hline
1.2.11 & deficit tamen a potentia gressiua , & para si | e para los otros mienbros \\\hline
1.2.11 & quia non potest pergere . Pes autem abundat ingressiua potentia : & Enpero fallesçe en poderio de andar . | Ca non puede el oio andar \\\hline
1.2.11 & Pes autem per potentiam gressiuam qua pollet , & Mas el pie por el poderio que ha de andar acorre ala mengua del oio \\\hline
1.2.11 & nisi in eo reseruaretur quaedam distributiua Iustitia , & si en el non fuesse guardada vna iustiçia partidora engsa \\\hline
1.2.11 & nisi reseruetur in ea distributiua iustitia , & si en ellos non fuere guardada la iustiçia partidora \\\hline
1.2.11 & reseruatur in eis Iustitia distributiua . & es en ellos guardada la iustiçia partidora . \\\hline
1.2.12 & ( ut dictum est ) hominem in ordine ad alium . Tunc autem maxime clarescit bonitas nostra , quando usque ad alios se extendit . Unde 5 Ethic’ dicitur , & por que assi commo dicho es faze al omne acabado | en orden a otro . \\\hline
1.2.12 & sed si sit in Regibus et Principibus ostendit eos esse perfecte bonos . Sic enim videmus in aliis rebus quod unumquodque perfectum est , & que ellos son acabados e buenos . | Ca assi lo veemos en todas las otras cosas \\\hline
1.2.13 & restat videre circa quae esse habeat talis virtus . Sciendum igitur , quod timere , et audere , proprie respiciunt pericula . Nullus autem timet , & e auer osadia propiamente catan | e han de ser enlos peligros . \\\hline
1.2.13 & si inuenerint resistentiam , non sustinent , & si fallan fortaleza | non sufren \\\hline
1.2.13 & principalius tamen est circa repressionem timorum , & e refrenando las osadias . Enpero mas prinçipalmente es cerca aquellas cosas | que repremen los temores \\\hline
1.2.13 & Fortitudinem principalius esse circa repressionem timorum , & que la fortaleza mas prinçipalmente es cerca del repremiento de los temores \\\hline
1.2.13 & restat ergo declarandum , & Pues que assi es fincanos de declarar \\\hline
1.2.13 & si volumus nos ipsos facere fortes . Declarata ergo sunt illa tria , & Pues que assi es declaradas son aquellas tres cosas \\\hline
1.2.14 & Nam ( ut dicit Vegetius in libro De re militari ) , & Ca assi commo dize vegeçio | en el libro del fecho dela caualleria \\\hline
1.2.14 & Immo si ex furore quis pugnam aggrediatur , inueniens resistentiam , & que se sigue della | Mas si por la sana alguno acomete la batal la quando falla resistençia \\\hline
1.2.14 & quia si inueniant resistentiam , & Ca si fallaren alguna resistençia o fortaleza \\\hline
1.2.15 & quae est principalior Temperantia . Restat dicere de ipsa Temperantia , quae & Et dela fortaleza que es mas prinçipal que la tenperança . \\\hline
1.2.15 & Vocamus autem insensibilem et agrestem , & que se non sienten . Ca nos llamamos aquel omne insenssible | que non se siente . \\\hline
1.2.15 & Qui autem omnes fugit , insensibilis et agrestis est . & e es siluestre \\\hline
1.2.15 & non respiciamus in ipsam . & e dixieron echemos la de nos que quieredezir tanto commo non la catemos nin la veamos . \\\hline
1.2.16 & nam non quaelibet aggressio bellorum facit nos fortes , & por que cada vn acometemiento de batallas non nos faze fuertes \\\hline
1.2.16 & de ipsa intemperantia tria , & ca tres cosas | delas quales podemos tomar tres razones \\\hline
1.2.16 & ex quibus tres rationes sumi possunt , quod maxime decet Reges et Principes temperatos esse . Est enim intemperantia & para prouar | que mucho conuiene alos Reyes et alos prinçipes de ser tenprados . \\\hline
1.2.16 & praecepit quod duceretur ad ipsum . Dux autem ille assuetus rebus bellicis , & Mas aquel prinçipe | por que era acostunbrado delas batallas \\\hline
1.2.17 & Virtutes autem aliae vel respiciunt exteriora bona , & que esta s ochon uirtudes o catan alos bienes tenporales de fuera o alos males tenporales de fuera . \\\hline
1.2.17 & vel in ordine ad aliud . Primo ergo dicemus de virtutibus respicientibus exteriora bona & tu desque catan alos males tenporales de fuera . | Et otrossi diremos delas uirtudes \\\hline
1.2.17 & secundum se . Postea determinabimus de virtutibus respicientibus exteriora bona in ordine ad aliud . & que catan alos bienes tenporales de fuera | en quanto son ordenados a otra cola . \\\hline
1.2.17 & Primo ergo dicemus de virtutibus respicientibus bona utilia : & Et pues que assi es primeramente diremos delas uirtudes | que caran alos bienes aprouechosos \\\hline
1.2.17 & et postea de respicientibus bona honesta . & e despues diremos delas uirtudes | que catan alos bienes honestos . \\\hline
1.2.17 & de virtutibus respicientibus bona utilia . & uirtudesque catan dos bienes prouechosos . \\\hline
1.2.17 & et magnificentia . Hae autem duae virtutes respiciunt sumptus , et pecuniam : & La largueza e la magnificençia . | Ca estas dos uirtudes catan alas despenssas e alos dineros mas departida mente . \\\hline
1.2.17 & ( quae alio modo largitas nuncupatur ) dicitur respicere sumptus mediocres : & que en otro nonbre es dicha largueza dezimos | que cata alas espenssas mesuradas e medianas \\\hline
1.2.17 & magnificentia vero dicitur respicere magnos sumptus ; & Mas la magnificençia es tal uirtud | que cata alas grandes despenssas . \\\hline
1.2.17 & Ad rectum autem usum pecuniae tria requiruntur . & Mas para bien usar del auer | e de lons dineros tres cosasson menester \\\hline
1.2.17 & sunt illa tria circa quae videtur esse liberalitas . & Estas tres cosas son aquellas | en que ha de ser la franqueza \\\hline
1.2.17 & Non autem est circa haec tria aeque principaliter & mas non ha de ser çerca estas tres cosas egualmente e prinçipal mente . \\\hline
1.2.18 & restat ostendere , & si fueren auarientos fincanos de demostrar \\\hline
1.2.18 & Spectat autem ad liberalem primo respicere quantitatem dati , & Mas par tenesce al libal e alstan ço de catar tres cosas ¶ | La primera deue catar la quantidat delo que da \\\hline
1.2.18 & Secundo debet respicere quibus det , & Lo segundo deue catar aqui lo da \\\hline
1.2.19 & quae respicit moderatos sumptus , & ¶La vna que cata alas espenssas mesuradas \\\hline
1.2.19 & quae respicit sumptus magnos , & La otra cata alas grandes espenssas \\\hline
1.2.19 & et naturam rerum , & e en la semeiança delas cosas \\\hline
1.2.19 & et difficile . quare cum in maioribus sumptibus reperiatur specialis ratio bonitatis et difficultatis , & por la qual cosa commo en las mayores espenssas sea fallada . | mas espeçial razon de bondat e de guaueza \\\hline
1.2.19 & Cum ergo liberalitas non respiciat sumptus & ¶Pues que assi es commo la libalidat non cate alas espenssas \\\hline
1.2.19 & quae respicit sumptus proportionatos facultatibus , & que cata alas espenssas conparadas alas riquezas se estiende \\\hline
1.2.19 & respiciens sumptus & Mas la magnificençia | que cata alas espenssas \\\hline
1.2.19 & non respicit quaecunque opera : & non cata | nin tiene oio quales sean las obras . \\\hline
1.2.19 & restat videre circa quae habet esse . & en quales cosas ha de ser . | Onde conuiene saber \\\hline
1.2.19 & debet enim quis magnifice se habere circa magna opera respectu personae propriae . & por que deue cada vno granadamenᷤte se auer cerca las grandes obras en conparaçion dela su perssona \\\hline
1.2.20 & ideo sicut in incisione corporis propter diuisionem continui resultat ibi tristitia & por la diuision del cuerpo | que es continuo nasçe ende tristeza e dolor \\\hline
1.2.20 & et circa ea quae respiciunt regnum totum . Rursus quia ad ipsum maxime spectat distribuere bona regni , & e proprouechando las mucho ¶ | Otrosi por que a el parte nesce prinçipalmente partir los bienes del regno \\\hline
1.2.21 & et erga rempublicam , & e en el bien comun \\\hline
1.2.22 & Sicut igitur circa ipsa bona utilia est duplex virtus una respiciens magnos sumptus , & Por enl de assi commo çerca los bienes aprouechosos son dos uirtudes . | La vna que cata alas grandes espenssas \\\hline
1.2.22 & et alia quae respicit mediocres , & Et otra que cata alas despenssas medianas \\\hline
1.2.22 & una quae respicit magnos honores , & La vna que cata las grandes horras \\\hline
1.2.22 & et alia quae respicit mediocres , & Et otra es que cata las honrras medianas \\\hline
1.2.22 & quid est magnanimitas , et circa quae habet esse . Restat ostendere , quomodo possumus nosipsos magnanimos facere . & e cerca quales cosas ha de ser sinca de demostrar | en qual manera podemos a nos mismos fazer magnanimos . \\\hline
1.2.23 & et maxime non debet esse talis respectu exteriorum bonorum . & Et mayormente non deue ser tal en conpara raçonn de los bienes de fuera \\\hline
1.2.23 & negocia enim ardua pauca sunt respectu aliorum . & por que los negoçios muy altos son pocos | e en conpara connde los otros . \\\hline
1.2.24 & quae respicit sumptus , & que cata alas despenssas \\\hline
1.2.24 & quae respicit sumptus & que cata las despenssas \\\hline
1.2.24 & Una quae respicit honores mediocres , & Lauona | que cata las honrras medianeras \\\hline
1.2.24 & quae respicit magnos honores , & que cata alas grandes honrras | assi commo la magnanimidat \\\hline
1.2.24 & Recte ergo magnanimitas dicitur omnes virtutes maiores facere , & Et pues que assi es bien dich̃ones | que la maguanimidat faze todas las uirtudes mayors \\\hline
1.2.25 & in rebus licitis & por que cuydando en los sus desfallesçimientos propios en las cosas conuenibles e honestas . \\\hline
1.2.26 & quae est in rebus honorificis , & que es enellas \\\hline
1.2.26 & quia principaliter est circa repressionem superbiae , & ca prinçipalmente es çerca la presup̃çion | e sobrepiuamiento dela sobraia \\\hline
1.2.26 & restat videre quod decet Reges & finca de ver | que conuiene alos Reyes \\\hline
1.2.27 & de virtutibus respicientibus exteriora bona , & que catan alos bienes de fuera . \\\hline
1.2.27 & restat dicere de mansuetudine , & fincanos agora de dezir dela mansedunbre \\\hline
1.2.27 & quae respicit exteriora mala . & que es uirtud | que cata alos males de fuero . \\\hline
1.2.27 & vel propter amorem Reipublicae & o por amor de la comunidat \\\hline
1.2.27 & quia sine ea Respublica durare non posset . & por que sin ella la comunidat non podrie durar . | por la qual cosa si el Rey o el prinçipe o otro \\\hline
1.2.27 & et conseruatores Reipublicae . & de seer guardadores dela iustiçia | et mantenedores dela comunidat . \\\hline
1.2.28 & sufficienter diximus de virtutibus respicientibus bona & suficientemente dixiemos de las uirtudes \\\hline
1.2.28 & et de respicientibus exteriora mala . & e delas que catan alos małs̃ de fuera . \\\hline
1.2.28 & Restat dicere & fincanos de dezir de las uirtudes \\\hline
1.2.28 & de virtutibus respicientibus bona exteriora , & que catan alos bienes de fuera \\\hline
1.2.28 & ad tria nobis deseruiunt , & en las quales partiçipamos con los otros siruen a nos en tres cosas . \\\hline
1.2.28 & et agrestes , non valentes cum aliis conuersari . Uterque autem a recta ratione deficiunt , & que no saben beuir con los otros . | Mas cada vno destos fallesçen en cada vna destas razo nes \\\hline
1.2.29 & restat videre circa quae habet esse . & traca quales cosas ha de seer . \\\hline
1.2.29 & Est igitur veritas circa repressionem iactantiarum , & Et pues que assi es la uerdat es uirtud | para repremir los alabamientos \\\hline
1.2.30 & et hi vocantur duri , et agrestes . Huiusmodi autem sunt non sustinentes & Et estos son llamados duros e montesmos . | Et estos tales non sufren ningun trebeio \\\hline
1.2.30 & quam circa repressionem superfluitatum in ludo : & commo çerca el repremimiento delas sobrepuiancas en el mego \\\hline
1.2.30 & Restat ergo videre , & Et pues que assi es finca deuer \\\hline
1.2.30 & secundum rem , & commo esta en la cosa \\\hline
1.2.30 & viderentur esse durae et agrestes . & que serian montesinos e siluestres . \\\hline
1.2.31 & aliquando per manifestam oppressionem alios depraedantur , & por manifiesta o prasion apremiando los otros | e algunas uegadas por robo \\\hline
1.2.32 & ut omnes praegnantes resideret ; & e quaria que todas las prennadas abriessen \\\hline
1.2.32 & restat dicere de quatuor generibus bonorum . & ¶ dicho de los quatroguados delons malos | finca de dezir delos quatro linages de los buenos \\\hline
1.2.32 & quae est dominans et principans respectu aliarum , & que es sennora | e prinçipante a todas las otras uirtudes . \\\hline
1.2.33 & quaedam exemplares . & Et alguas son exenplares . \\\hline
1.2.33 & virtutes exemplares esse in ipso Deo . Virtutes autem politicas , & que son en dios . | Et las uirtudes politicas son uirtudes g̃nadas \\\hline
1.2.33 & per quas homines bene se habent in rebus humanis . & por las quales los omes bien se han en todas las cosas humanales . \\\hline
1.2.33 & sed diuini habent virtutes exemplares . Propter quod sicut diuini meliores sunt temperatis , temperati continentibus , continentes perseuerantibus : sic virtutes exemplares excellunt virtutes purgati animi : virtutes vero purgati animi excellunt purgatorias : & que las uirtudes del coraçon pragado . | Et las uirtudes del coraçon pragado son mas altas \\\hline
1.2.33 & videlicet , exemplares adaptari possunt hominibus diuinis : & las ezenplares pueden part | e nesçer alos omes diuinales \\\hline
1.2.33 & sed decet eos quodammodo esse diuinos . Virtutes ergo competentes eis , possunt exemplares dici : & mas conuienel es de ser en algunan manera diuinales | Et por ende las uirtudes que parten esten aellos pueden ser dichas exenplares \\\hline
1.2.33 & Bene ergo eis competunt exemplares virtutes , & segunt las quales uirtudes \\\hline
1.3.1 & Restat exequi de tertia parte huius primi libri , & Agora finca de dezir dela tercera parte deste libro \\\hline
1.3.1 & Sed cum constat de re , & mas quando nos somos çiertos dela cosa non deuemos auer cuydado delas palauras . \\\hline
1.3.1 & vel sumitur respectu boni , & que pertenesçe al appetito desseador o se torna en conparaçion de algun bien o en conparaçion de algun mal \\\hline
1.3.1 & vel respectu mali . & Si se toma en conparaçion de algun bien \\\hline
1.3.1 & Si respectu boni , & Si se toma en conparaçion de algun bien \\\hline
1.3.1 & et delectatio sumuntur respectu boni . Odium vero , & e la delectaçion son tomados en conparaçion de algun bien | Mas la mal querençia \\\hline
1.3.1 & sumuntur respectu mali . & e la aborrençia e la tristeza son tomadas en conparaçion de algun mal \\\hline
1.3.1 & Viso quomodo accipiuntur passiones concupiscibiles : restat videre , quomodo sumendae sunt passiones irascibiles . & visto en qual manera se toman las passiones del appetito desseador \\\hline
1.3.1 & quia passiones concupiscibiles respiciunt bonum & deuer en qual manera se han de tomar las passiones del appetito enssannador | Mas estas passiones han diferençia \\\hline
1.3.1 & sed passiones irascibiles respiciunt bonum & Ca las passiones del apetito cobdiçiador catan al bien o al mal \\\hline
1.3.1 & Huiusmodi ergo passiones vel sumuntur respectu boni , & Et pues que assi es estas tales passiones | o son tomadas en conparaçion de algun bien \\\hline
1.3.1 & vel respectu mali . & o en conparaçion de algun mal \\\hline
1.3.1 & Si respectu boni , & saien conparaçion de algun bien \\\hline
1.3.1 & et desperatio sumuntur respectu boni . & Et por ende la esperança | e la desesperança son tomadas en conparaçion de algun bien . \\\hline
1.3.1 & Aliae vero passiones irascibiles sumuntur respectu mali . & Mas las otspassiones del appetito enssannador son tomadas \\\hline
1.3.2 & et desperatio , cum sumantur respectu boni , praecedunt timorem , et audaciam , iram , & por razon de bien son puestas primero | que el temor e la oladia \\\hline
1.3.2 & quae sumuntur respectu mali . Timor autem , & que son tomadas por razon de mal . \\\hline
1.3.2 & quae sumuntur respectu mali futuri , & que son tomadas por razon de mal \\\hline
1.3.2 & quae sumuntur respectu mali praesentis . & que son tomadas | por razon de mal presente ¶ \\\hline
1.3.2 & quia semper passio sumpta respectu boni & por que sienpre la passion | que es tomada en razon de bien \\\hline
1.3.2 & ( secundum quod huiusmodi ) prior est passione sumpta respectu mali . & segunt que estal es primero que la passion | que es tomada en razon de mal \\\hline
1.3.2 & quae est respectu boni , & que es tomada en razon de bien es primero que la tristeza \\\hline
1.3.2 & quae est respectu mali . & que es tomada en razon de mal Et pues que assi es \\\hline
1.3.3 & reseruatur in Deo , & e mas noblemente es guardado el su bien en dios \\\hline
1.3.3 & quia ciues pro Republica non dubitabant se morti exponere . & por que los çibdadanos non duda una de se poner ala muerte | por el bien comun de todos . \\\hline
1.3.3 & ad Rempublicam fecit Romam esse principantem & e publicofizo a Roma ser sennora \\\hline
1.3.3 & ad tria comparari potest scilicet ad tyrannidem , & Conuiene saber ala tirania . | del tirano \\\hline
1.3.4 & dicere restat , & que son las primeras passiones finca de dezir \\\hline
1.3.4 & Nam gesta moralia quodammodo rebus naturalibus sunt similia . & Ca los fechͣs e las obras morales son semeiables en alguna manera alas cosas naturales . \\\hline
1.3.4 & et leuibus est tria considerare . Primo formam grauis vel leuis , per quam conformatur loco sursum vel deorsum . & Lo primero la forma del cuerpo pesado o liuiano | por la qual cosa es conformado al su logar de yuso o de suso ¶ \\\hline
1.3.4 & est tria considerare . & assi commo todos dizen comunalmente deuemos penssar estas tres cosas . \\\hline
1.3.4 & Et quod dictum est de bono respectu amoris , desiderii , et delectationis , & Et esto que dicho es del bien en conparaçion del amor | e del desseo \\\hline
1.3.4 & intelligendum est de malo respectu odii , & assi de uemos entender del mal en conparaçion de la mal querençia \\\hline
1.3.5 & Restat vero videre quomodo Reges & Et pues que assi os finca de veer \\\hline
1.3.5 & nisi de rebus futuris , & Et non delas passadas | nin delas presentes . \\\hline
1.3.5 & restat videre quomodo se habere debeant in non sperando non speranda . & Ca conuiene alos Reyes de cuydar \\\hline
1.3.6 & quomodo se habere debeant circa spem et desperationem quae respiciunt futurum bonum ; & en qual manera se deuen auer çerca la escanca | e cerca la desesperanca \\\hline
1.3.6 & quae respiciunt futurum malum . Videtur autem forte aliquibus Reges , & que los reyes e los prinçipes | en ninguna cosa non deuen ser temerosos \\\hline
1.3.8 & Dicto ergo de omnibus aliis passionibus restat dicere quomodo Reges , & ¶Et pues que assi es | que dicho es de todas la s otras passiones finca de dezir \\\hline
1.3.8 & videre restat , & finça deuer en qual maneras \\\hline
1.3.8 & per quae huiusmodi tristitia vitari possit . Videtur autem Philosophus tria remedia tangere , & Et paresçe que el philosofo tanne tres remedios \\\hline
1.3.9 & ut passiones sumptae respectu boni , & assi commo las passiones \\\hline
1.3.9 & et gaudium , sumptae autem respectu mali , & que son tomadas en conparacion de algun bien son ordenadas ala esperança e al gozo . \\\hline
1.3.9 & Nam passio sumpta respectu boni , & Ca la passion tomada en conparaçion de algun bien . \\\hline
1.3.9 & Respectu vero mali incipit ab odio , et procedit in fugam , vel abominationem , et terminatur in timorem , & Mas quando es en conparaçion de algun mal comiença en la mal querençia | e vayendo para la foyr \\\hline
1.3.9 & quia ad eas ordinantur passiones sumptae respectu mali : & por que son ordenadas aellas las o tris passiones . que son tomadas en conparaçion de mal | assi commo ala esperança \\\hline
1.3.9 & sicut ad spem et gaudium ordinatur passiones sumptae respectu boni . & e al gozoson ordenadas las passiones | que son tomadas en conparacion de bien . \\\hline
1.3.9 & vel sumitur respectu boni , & en conparaçion de algun bien \\\hline
1.3.9 & vel respectu mali . Rursus bonum , & o en conparaçion de algun mal . \\\hline
1.3.9 & Tertio modo hae passiones sumi possunt respectu potentiarum animae , & La terçera manera es que estas passiones se pueden tomar | en conparaçion delas potençias del alma . \\\hline
1.3.9 & ut respectu irascibilis , & assi commo en conparacion del appetito enssannador \\\hline
1.3.9 & et tristitiam esse principales passiones respectu concupiscibilis . & Conuiene que la delectaçion e la tristeza sean prinçipales passiones \\\hline
1.3.9 & Spes autem et timor sunt principales passiones respectu irascibilis . & en conparaçion del appetito cobdiçiador . | Mas la esperança e el temor son passiones prinçipales \\\hline
1.3.9 & spes et timor sunt principales passiones respectu irascibilis . & e el temor son passiones prinçipales | en conparacion del appetito enssannador . \\\hline
1.3.9 & eo quod respiciant bonum gentis . & por que caran al bien dela gente e del pueblo \\\hline
1.3.10 & ideo zelus respectu horum diffiniri consueuit , & Por ende se suele difinir | e declarar el zelo en conparaçion destas cosas \\\hline
1.3.10 & Sed respectu bonorum intellectualium , & Mas en conparaçion de los bienes intellectuales e en conpaçion delas uirtudes \\\hline
1.3.10 & et respectu virtutum , & Mas en conparaçion de los bienes intellectuales e en conpaçion delas uirtudes \\\hline
1.3.10 & Huiusmodi ergo zelus respectu bonorum honorabilium diffinitur a Philosopho 2 Rheto’ & e declarado | assi por el philosofo en el segundo libro dela rectorica \\\hline
1.3.10 & Restat videre , & al temor fica \\\hline
1.3.10 & Sciendum ergo haec tria esse species tristitiae . & Et pues que assi es deuedes saber | que aqui son tres maneras de tristeza . \\\hline
1.3.11 & ut rei cognoscentia postulabit . Ibi enim ostendemus , quomodo Reges et Principes se habere debeant , ut a populis timeantur , et amentur : & Cay mostraremos en qual manera los Reyes se deuen auer | por que sean temidos de lons pueblos e amados . \\\hline
1.4.1 & Expeditis tribus partibus huius primi libri : & Es enbargadas las tres partes deste primero libro | Ca mostrado es \\\hline
1.4.1 & et quas passiones debent sequi . Restat exequi de parte quarta , & e quels passiones deuen segnir ¶ | finça de dezir dela quarta parte . \\\hline
1.4.1 & Cum ergo memoria sit respectu praeteritorum , & que ha de uenir . Et pues que assi es commo la memoria sea en conparaçion del tienpo passado | por que es recordaçion delas cosas que passaron \\\hline
1.4.1 & et spes respectu futurorum : & e la esperança es en conparacion deltp̃o | que ha de uenir . \\\hline
1.4.2 & ad pauca respicientes , & por que non son docternados | por muchͣs razones \\\hline
1.4.2 & non valentes ad multa respicere , & non pueden catar a muchͣs cosas \\\hline
1.4.2 & et omnia scire putant . Ideo de omnibus respondent , & e creen que saben todas las cosas . | Et por ende de todas las cosas responden \\\hline
1.4.3 & Restat uidere , & finca de uer quales son las costunbres \\\hline
1.4.3 & semper recitant res gestas , & sienpre cuentan las cosas passadas \\\hline
1.4.3 & non autem delectantur in recitando res fiendas , & mas non se delectan en contando las cosas \\\hline
1.4.3 & quia cum negocia respicientia totum regnum , & Por que conmo los negoçios e los fech̃d | que pertenesçen a todo el regno çerca \\\hline
1.4.4 & restat enumerare mores ipsorum laudabiles . & que non son de loar fincanos de poner las costunbres dellos qson de loar \\\hline
1.4.4 & quia frigidum restringitur , & Et por que el frio se restune \\\hline
1.4.4 & ne ita res se habeat , & e que seran engannados . \\\hline
1.4.5 & et diuitiae videntur esse pretium rei cuiuslibet , & e las riquezas paresçen ser preçio | de qual si quier cosa . \\\hline
1.4.5 & quia multos habent respicientes ad eos , & por que han muchͣs | que tienen mientes a ellos \\\hline
1.4.5 & videre restat & finca de uer quales costunbres son de denostar . \\\hline
1.4.5 & maiores se reputant , & por mayores se tienen \\\hline
1.4.6 & non valentes passionibus resistere : & por que non pueden corͣdezir a sus passiones \\\hline
1.4.6 & fieri diuitem , quam sapientem . Respondit , & Et ella respondio \\\hline
1.4.6 & videre restat , & e alos prinçipesarredrar se de tales costunbrs finca de ueer \\\hline
1.4.7 & et qui diuitum restat videre , & e de los ricos | finça \\\hline
1.4.7 & quam diuites . Narrat autem Philosophus tria , & que los ricos . | Ca cuenta el philosofo \\\hline
1.4.7 & ad quam multi respiciunt , verecundatur omnino declinare a medio , & las quales catan muchos uergunença | han en toda manera de arredrar se del medio \\\hline
1.4.7 & ut dent operam rebus venereis , & e muy de ligero se inclinan a fazer obras de luxia \\\hline
1.4.7 & quia ut plurimum his tribus exterioribus affluunt , & por que por la mayor partida abonda en estos tres bienes de fuera \\\hline
2.1.1 & Restat ergo dicere de regimine familiae , & Et pues que assi es finca de dezir del gouernamiento delan conpanna o del gouernamiento dela casa . \\\hline
2.1.1 & si res naturales nullo modo conseruarentur in esse , & si las cosas naturales en ninguna manera non se pudiessen guardar \\\hline
2.1.2 & sed crescentibus filiis et filiabus , & e despues cresçiendo los fijos e las fijas \\\hline
2.1.2 & crescentibus collectaneis idest nepotibus , & Ca assi commo dicho es de suso cresçiendo los nietos e los fijos de los fijos \\\hline
2.1.2 & quam ex crescentia collectaneorum vel filiorum , & por acresçentamiento de fuos e de nietos adelante lo mostraremos mas conplidamente \\\hline
2.1.3 & per amplius et perfectius reseruantur in viro quam in puero . & e mas conplidamente son falladas en el uaron | que en el moço . \\\hline
2.1.3 & quia huiusmodi communitas respectu aliarum est imperfecta : & por que esta comunidat en conparaçion delas otras es mas menguada \\\hline
2.1.3 & ad ipsum finem : domus respectu aliarum communitatum non solum est prior tempore , & en conparaçion delas otras comuindades | non solamente es primera por tienpo e postrimera por perfetçion \\\hline
2.1.4 & In hac autem descriptione aliquid declaratum est per praecedens capitulum , et aliquid restat ulterius declarandum . & Mas en esta declaraçion alguna cosa es ya declarada en el capitulo sobredich̃o | Et alguna cola finca adelante de declarar . \\\hline
2.1.4 & Restat ergo declarare in descriptione praedicta , & Pues que assi es finça de declarar en la difiniçion sobredichͣ \\\hline
2.1.5 & rerum generatio , et earum conseruatio . Cum enim generatio ( ut dicitur 2 Physicorum ) & e la conseruaçiondellas | por que assi commo es dicho en el segundo libro de los fisicos \\\hline
2.1.5 & et cum res naturales per generationem propriam naturam accipiant , & e carrera en la natura | et commo las cosas naturales resçiban su naturaleza propra a \\\hline
2.1.5 & Rursus rerum conseruatio , & Otrossy la conseruaçion \\\hline
2.1.5 & videlicet , rerum generatio , & La generaçion delas cosas \\\hline
2.1.5 & conseruatio rerum generatarum esse non potest & La conseruaçion delas cosas engendradas non puede ser \\\hline
2.1.5 & quomodo ad constitutionem huius domus saltem requiruntur ibi tria genera personarum . & commo para el establesçemiento desta casa | alo menos son menester y . tres linages de perssonas . \\\hline
2.1.5 & Tria ergo genera personarum , & Pues que assi es tres linages de perssonas \\\hline
2.1.5 & ad huiusmodi officium faciendum obsequitur hos vel equus . Domus ergo prima dicitur constare ex tribus : & para tal ofiçio | ca les sirue el bueye o el cauallo o el asno . \\\hline
2.1.5 & dicentem domum constare ex tribus , & que dixo | que la casa es estableçida de tres perssonas . \\\hline
2.1.6 & Videmus enim in naturalibus rebus quod statim quum generatae sunt , & Ca ueemos en las cosas naturales | que luego que son engendradas las cosas dla natura \\\hline
2.1.6 & et domini et serui , quae est propter saluationem , faciunt domum primam . Sic ergo saluatio comparatur ad rem generatam : & e la comunidat del sennor e del sieruo | que es para la saluaçion fazen la primera casa . \\\hline
2.1.6 & quia statim cum res est genita , solicitatur natura circa salutem eius ; & assi la saluaçion es conpada ala cosa engendrada | que luego que la cosa es engendrada la natura es acuçiosa çerca de su salud . \\\hline
2.1.6 & non sic comparatur ad res naturales : & de ssi non es assi conpado alas cosas natraales \\\hline
2.1.6 & quia non statim cum est res naturalis , & por que non puede la cosa natural \\\hline
2.1.6 & de ratione rei naturalis quocumque modo , sumptae ; sed est de ratione eius , & en qual quier manera | mas pertenesçe a cosa natural \\\hline
2.1.6 & quodammodo in una persona singulari reseruantur . & en alguna manera son falladas en vna perssona singular \\\hline
2.1.6 & rei , per quam aliquid est in actu , & e dela forma de essa misma cosa . | por la qual toda cosa es en su ser conplido \\\hline
2.1.6 & et maxime principalium resultat imperfectio totius , & e mayormente delas prinçipales | seleunata mengua en el todo \\\hline
2.1.6 & et nisi sit ibi excrescentia filiorum . & e multiplicaçion de fiios . \\\hline
2.1.6 & cum omni sua claritate . Patet ergo quod ad hoc quod domus habeat esse perfectum , oportet ibi esse tres communitates : unam viri et uxoris , aliam domini & que conuiene que sean enlla tres comuundades . | ¶ La vna del uaron e dela muger \\\hline
2.1.6 & quod oportet in domo perfecta esse tria regimina . & que en la casa acabada deuen ser tres gouernamientos . \\\hline
2.1.6 & sunt tria regimina , & tres son los gouernamientos . \\\hline
2.1.6 & in domo perfecta esse communitates tres , & commo en la casa acabada | ian de ser tres comuindades \\\hline
2.1.6 & et tria regimina : & e tres gouernamientos de ligero puede paresçer \\\hline
2.1.6 & in quo tractatur de regimine domus . Nam cum in domo perfecta sint tria regimina , oportet hunc librum tres habere partes ; & Ca commo en la casa acabada sean tres gouernamientos . | Ca conuiene que este libro sea partido en tres partes . \\\hline
2.1.6 & Haec autem tria regimina bene cognoscere maxime decet Reges et Principes ; & ¶ Empero mucho conuiene alos Reyes | e alos prinçipes de conosçer bien estos tres gouernamientos \\\hline
2.1.7 & tria esse determinanda in hoc secundo libro , & que tres cosas son de determinar en este segundo libro \\\hline
2.1.7 & secundum quod in ipsa domo tria contingit esse regimina ; & segunt que enla casa han de ser tres gouernamientos . \\\hline
2.1.7 & quae non respicit domum primam , & que non cata ala casa primera . la casa tomada en qual si quier manera mas cata ala casa \\\hline
2.1.7 & quod respicit communitatem viri et uxoris , & que cata ala comuidat del uaron | e dela muger \\\hline
2.1.7 & quod respicit communitatem domini & que cata la comuidat del señor \\\hline
2.1.7 & quod respicit communitatem patris et filii . In determinando autem de regimine coniugali , & que cata la comunindat del padre et del fiio . | Mas avn en determinando del gouernamiento coniugal \\\hline
2.1.7 & tanquam aliquid contrarium rei naturali : & assi conmo aquella que es contraria ala cosa natural . \\\hline
2.1.9 & inter eos et suas coniuges maxime reseruari debet amor debitus coniugalis . & por que entre ellos e sus mugers sea mucho mas guardado el amor matermoinal . \\\hline
2.1.10 & omnino detestabile esse unam foeminam nuptam esse pluribus viris . In coniugio enim primo reseruatur ordo naturalis : & que vna muger sea casada con muchos uarones | Ca en el mater moino primeramente es guardada la orden natural . \\\hline
2.1.10 & non resultabit inde pax et concordia , & e por esto se tolleria la concordia e la paz e nasçria dende discordia e enemistança \\\hline
2.1.10 & rei delectabilis impeditur ; absque dissensione & en que se delecta \\\hline
2.1.10 & rei delectabilis & mientra que el vno de aquellos enbarguase al otro en el vso del matermo non delectable \\\hline
2.1.11 & cum huiusmodi reuerentia debita non reseruetur & e commo esta reuerençia conueinble non sea guardada entre la muger e el uaron \\\hline
2.1.11 & inter coniuges congrue reseruari non possunt & por razon de la reuerençia deuida alos parientes | que non se puede guardar conueniblemente entre el marido e la muger en sus obrassacado con dispenssaçion \\\hline
2.1.11 & tanto indigent habere maiores , & tanto han menester mayores \\\hline
2.1.12 & ad tria praedicta bona . & nin prinçipalmente entender a estos tres bienes sobredichos \\\hline
2.1.12 & et pluralitas amicorum , quam multitudo diuitiarum . Omnia tamen haec tria aliquo modo sunt attendenda . & Ca mas deuen entender enla \\\hline
2.1.12 & quasi sanitas respectu humorum . & assi commo la sanidat en conparaçion de los humores . \\\hline
2.1.12 & et quomodo ex tali coniugio quaerenda est pluralitas amicorum . Restat ostendere quomodo ex eo quaeri debeat diuitiarum multitudo . Quaeruntur enim ex coniuge dotes & Et en qual manera por tal casamiento jes de demandar muchedunbre de amigos finca | de demostrar \\\hline
2.1.12 & ut sibi coniuges fidem non seruent . Quaerenda sunt ergo in coniuge tria praedicta bona exteriora : & por que los casados non se guardassen fialdat . | ¶ Et pues que assi es las tres cosas sobredichͣs \\\hline
2.1.12 & reseruetur tam animae quam corporis . & assi que entre los casados sea guardada alguna egualdat \\\hline
2.1.13 & quam ad tria praedicta . & por que el bien dela fialdat es \\\hline
2.1.13 & Restat videre quomodo quantum ad bona animae quaerenda sunt in ea temperantia , & en qual manera quanto alos bienes del alma son de demandar en ella tenprança \\\hline
2.1.14 & reseruatur in uno homine : unde et ab eis homo appellatur minor mundus . & Et por ende los pp̃os llaman al omne menor mundo . \\\hline
2.1.14 & Immo adeo modus uniuersi reseruatur in quolibet homine , & en tanto es fallada en cada vno de los omes \\\hline
2.1.14 & cum ciuitas sit pars uniuersi , regimen totius ciuitatis multo magis reseruabitur in una domo . & se aparte de todo el mundo | el gouernamiento de teda la çibdat \\\hline
2.1.14 & si totum uniuersum per quandam similitudinem reseruatur in uno homine , & Et si todo el mundo por algunan semeiança es fallado en vn omne mucho | mas los gouernamientos \\\hline
2.1.14 & multo magis regimina quae sunt in ciuitate per quandam similitudinem reseruantur in domo . & que son en vna çibdat | por alguna semeiança son falladasen vna casa . \\\hline
2.1.15 & Dicebatur superius in domo esse tria distincta regimina : & a dixiemos de suso | que en la casa ay tres gouernamientos departidos \\\hline
2.1.15 & quia ad alia opera est hoc quam illud . Restat ostendere , & e para otras el paternal fincanos de demostrar \\\hline
2.1.15 & sed ut melius fiant res naturales , & mas por que se fagan meior las cosas natraales \\\hline
2.1.15 & Videtur enim domus esse imperfecta , et habere penuriam rerum , & Ca paresçe que la casa non es acabada | e que a ninguna delas cosas \\\hline
2.1.16 & Sic enim videmus in aliis rebus naturalibus , & por que assi commo veemos en las otras cosas naturales \\\hline
2.1.16 & si tempore augmenti et crescente corpore utantur venereis . & si en el t pon del cresçer | e cresçiendo el cuerpo vsaren de lux̉ia . \\\hline
2.1.16 & cum ad tempus augmenti communiter in hominibus requirantur tria septennia , & Como el tp̃o dela cresçençia demande communalmente en los omes tres setenas de años . \\\hline
2.1.17 & quia calor naturalis magis reseruatur interius , plus possumus conuertere & e tornada adentromas podemos conuter en nuestros mienbros dela uianda . Por la qual cosa el vso del ayuntamiento del casamiento | en tal tp̃o \\\hline
2.1.18 & Quia ergo , respectu perfectionis , & e por que en conparaçion dela perfeçion \\\hline
2.1.18 & quia non uident in seipsis scientiam unde gaudere possint ; quod non habent in rei ueritate , uolunt habere in hominum opinione . & donde se puedan gozar | lo que non han en uerdat quieren paresçerdelo auer en opinion de los omes ¶ \\\hline
2.1.18 & restat narrare quae sunt vituperabilia in eis . Possumus autem narrare tria in mulieribus vituperabilia . Primo , & Et podemos dez | que tres cosas son de denostar en ellas ¶ \\\hline
2.1.19 & restat ostendere , quomodo regenda & commo cunple finca de demostrar | en qual manera se deua gouernar \\\hline
2.1.20 & quomodo circa eas debeant se habere . Sunt autem tria & en qual manera se de una auer çerca ellas . | Mas son tres cosas \\\hline
2.1.20 & et ultra quam sufficiat ad restaurandum : & mas de aquello que abasta | para restaurar e cobrar aquello queꝑ \\\hline
2.1.20 & restat videre , & ¶ Visto \\\hline
2.1.20 & restat ostendere , & finca de demostrar \\\hline
2.1.21 & His ergo tribus modis contingit & Et pues que assi es en estas tres maneras contesçe de pecar \\\hline
2.1.22 & tria mala consurgunt ; ex quibus tres rationes sumi possunt , & tres males se le una tan dende | de los quales podemos tomar tres razones \\\hline
2.1.22 & Sed cum res prohibita , & Lo segundo la cosaue dada | por que es mas uedada paresçe \\\hline
2.1.23 & et modicum tempus apponere in rebus vilioribus & mas ayna desenbargar se | e mas pequano tienpo poner en las cosas uiles \\\hline
2.1.23 & quod mala herba cito crescit & que yerba mala mas ayna cresçe \\\hline
2.2.1 & restat exequi de secunda , & fincanos de dezir dela segunda parte \\\hline
2.2.1 & Nam sicut in naturalibus rebus aspicimus & Ca assi commo veemos en las cosas naturales \\\hline
2.2.1 & qui singulis rebus praeest , habet solicitudinem , et prouidentiam totius Uniuersi . & ha grand cuydado | e grand prouidençia de todo el mundo . ¶ Et pues que assi es \\\hline
2.2.1 & est ut solicitet amantem circa rem amatam , & que aquel que ama aya grand cuydado dela cosa que ama . \\\hline
2.2.3 & His autem tribus regiminibus , & Mas estos tres gouernamientos seguñ los quales veemos algunos regnar en las çibdades \\\hline
2.2.3 & assimilantur tria regimina reperta in una domo . & e en las villas son semeiantes trs gouernamientos | que son fallados en vna casa . \\\hline
2.2.3 & ex ipsa perfectione patris . Prima via sic patet . Nam secundum Philosophum ideo natura dedit vim generatiuam rebus , & assy que segunt el philosofo | por ende la natura die uirtud de engendrar alas cosas \\\hline
2.2.5 & Videntur autem tria ipsi fidei conuenire , & assegunt que paresçe tres cosas conuienen ala fe \\\hline
2.2.5 & Quod tertia die resurrexit a mortuis . & e que te suçito de muerte ala uida en el terçero dia . \\\hline
2.2.5 & et omnes resurgemus , & Et que todos resuçitaremos \\\hline
2.2.6 & Si ergo sic ab ipsa infantia nobiscum crescit concupiscentia delectabilium , & para las cosas delectables | luego en la moçedat deuemos poner freno \\\hline
2.2.6 & ab ipsa infantia est tali concupiscentiae resistendum : & e contradezir ala cobdiçia | enla nr̃a moçedat . \\\hline
2.2.7 & et perfectum , per quod perfecte exprimere possent naturas rerum , & por al qual pudiessen conplidamente pronunçiar las natraas delas cosas e las costunbres de los omes \\\hline
2.2.7 & et ad cognoscendum naturas rerum : & para entender e conosçer las uaturas delas cosas . \\\hline
2.2.8 & et quantitates rerum . & que muestra conosçer las mesuras e las quantidades delas cosas . \\\hline
2.2.8 & Nam Naturalis Philosophia docens cognoscere naturas rerum , & qua non estas . | Ca la natural ph̃ia \\\hline
2.2.9 & tria in se habere debet . & tres cosas deue auer enssi . \\\hline
2.2.9 & Ad hoc autem quod sit sciens in speculabilibus , requiruntur tria . & mas para que alguno sea sabio en las sciençias speculatiuas tres cosas son menester \\\hline
2.2.9 & nisi tam de intellectis quam de inuentis nouerit iudicare quae sunt tenenda , et quae respuenda . & quales cosas son de retener | e quales de refusar . \\\hline
2.2.9 & ut doceat in scientia : restat videre , & para enssennar los maços en la sçiençia finca de uer | en qual manera deue ser prudente e sabio en las obras \\\hline
2.2.9 & qualem proponant suis numismatibus , possessionibus , et rebus inanimatis : & e enlas o triscosas | que non han alma . \\\hline
2.2.10 & ad pauca respicientes enunciant facile , & e conosçen poco catando alas pocas cosas \\\hline
2.2.10 & ne statim ad interrogata respondeant . & que non respondan luego | a lo que les den \\\hline
2.2.10 & ut praemeditati respondeant , & Enpero si se acostunbraten a responder con penssamiento \\\hline
2.2.10 & et doctores iuuenum debent eos instruere quomodo se habeant ad loquelam . Restat videre , quomodo sunt instruendi , ut se habeant circa visum . & e los maestros de los moços | los deuen enssennar e enformar \\\hline
2.2.10 & ut non solum prohibeantur iuuenes ad videndum turpia in re , & que non solamente sea defendido alos mançebos | de ver cosas torpes \\\hline
2.2.10 & ut non habeant oculos vagabundos . Inclinatur enim aetas illa ( eo quod omnia respiciat tanquam noua ) & e que non echen los oios a cada parte con locura . | C aquella hedat es inclinada para catar todas las cosas \\\hline
2.2.10 & restat ostendere , & quento ala fabla | e quanto ala iusta finca de demostrar \\\hline
2.2.10 & quantum ad res auditas . Secundo , & ¶La primera quanto alas cosas que oyen ¶ La segunda \\\hline
2.2.10 & quantum ad eos quos audit . In rebus autem auditis obseruatur cautela quantum ad iuuenes , & quanto alas personas que oyen . | Mas quanto alas cosas oydas se guarda cautella en los mançebos \\\hline
2.2.11 & Restat ostendere , & deuer \\\hline
2.2.12 & Restat dicere , & finca nons de dezer \\\hline
2.2.12 & Temperantia autem circa tria est adhibenda : & Ca la tenprança ha de ser puesta çerca de tres cosas . \\\hline
2.2.12 & tria mala causat . Primo , & ¶ El primer mal es \\\hline
2.2.12 & quod inducit nimia sumptio vini , est depressio rationis . Nam ascendentibus fumositatibus vini ad caput , & que viene del tomar mucho el vino | es çegamiento de la razon e del entendimiento . \\\hline
2.2.12 & Restat videre , & e el beuer finca de ver \\\hline
2.2.12 & restat dicere , & por que non sean golosos finca de dezer \\\hline
2.2.13 & et modeste se habere cum uxore iam ducta . Restat ostendere , & e commo se deuen auer tenpradamente con sus mugers | que han ya tomadas finca de demostrar \\\hline
2.2.13 & Viso qualiter iuuenes se habere debeant circa ludos . Restat videre , & Visto en qual manera los moços se deuen auer çerca | los trebeios \\\hline
2.2.13 & restat exequi de tertio , & assi vistas finca de seguir | e de tractar delo terçero \\\hline
2.2.13 & Vestimenta quidem ad tria videntur ordinari : & Et paresçe que los vestidos son ordenados a tres cosas . Conuiene a sabera la delectaçion \\\hline
2.2.13 & Restat ostendere , & en quanto ellas siruen ala delectaçion finca de demostrar \\\hline
2.2.14 & Quare cum mollia et ductilia facilius recipiant impressionem ex iis ex quibus coniunguntur , quam dura , pueri et iuuenes & Por la qual cosa commo las cosas muelles e tristornables | mas ligeramente resçiben in pression de aquellas cosas \\\hline
2.2.16 & decimumquartum annum , tria sunt consideranda circa regimen filiorum . & fasta el año xiiij̊ . | son de penssar tres cosas enl gouernamiento de los fijos . \\\hline
2.2.16 & et appetitum . Tria ergo attendenda sunt in filiis . Primo , & que es la uoluntad . | Et por ende tres cosas deuemos cuydar en los fijos ¶ \\\hline
2.2.16 & Restat videre , & finca de demostrar \\\hline
2.2.16 & His visis restat videre , & ¶ Vistas estas cosas | finca de demostrar \\\hline
2.2.16 & ut de ipsis rebus considerare possint : & que puedan penssar en las cosas . \\\hline
2.2.17 & Dicebatur supra circa filios tria intendenda esse , & ni emos dessuso | que trs cosas deuemos entender c̃ca los fijos . \\\hline
2.2.17 & Haec autem tria triplici septennio possumus adaptare . & Et estas tres colas podemos nos coparara tres se tenarios de años . \\\hline
2.2.17 & ad quas sciendas oportet recurrere ad intellectum rerum , & alas quales nos deuemos acoger | para saber el entendimiento delas cosas \\\hline
2.2.17 & ut a quartodecimo anno et deinceps intendendum est circa tria , & que comiença del xiiii año dende adelante | deuen entender çerca tres cosas \\\hline
2.2.17 & aliquando sustinere fortes labores pro defensione reipublicae : & conuiene les de sofrir alguas uegadas fuertes trabaios | por defendemiento dela tierra . \\\hline
2.2.17 & et congruitas quod respublica defensione indigeat , habeant corpus sic dispositum , & en que la tierra aya meester defendimiento | ayan el cuerpo o bien ordenado \\\hline
2.2.17 & ut per eos respublica possit defendi . & e pueda defender la tierra . \\\hline
2.2.17 & ut habeant dispositum corpus . Restat videre , & por que ayan el cuerpo bien ordenado | finca de ver \\\hline
2.2.17 & sed sint subiecti et obedientes suis patribus et senioribus . Tangit autem Philosophus 8 Poli’ tres breues rationes , quare decet filios esse subiectos , et obedire senioribus , & libro delas politicas tres razono breues | por que conuiene alos fijos de ser lubiectos e obedientes a lus padres e alos vieios . \\\hline
2.2.18 & sed etiam necessarius pro bono Reipublicae , & Mas avn es neçessario para el bien e para el defendimiento dela comunidat \\\hline
2.2.18 & et a suis haeredibus magis est vitanda inertia , & e abueans costunbres pueden ellos | e sus herederos \\\hline
2.2.19 & restat dicere , & cerca los fiios fincanos \\\hline
2.2.20 & nec regiminibus reipublicae ; & nin c̃ca los gouernamientos dela comunidat | nin dela çibdat \\\hline
2.2.20 & ut ex hoc resultet fructus , & por que dende salga fructo e prouecho \\\hline
2.2.21 & restat ut nunc tertio ostendamus , & que agora lo terçero mostremos \\\hline
2.3.1 & Restat exequi de parte tertia , & finca de dezer dela terçera ꝑte \\\hline
2.3.1 & quia ars est recta ratio factibilium et per artem resultat aliquid factum in materia extra : & que ha de fazer de fuera . | Et por tal arte sale algcosa fechͣ en la materia de fuera . \\\hline
2.3.1 & et per eam non proprie resultat aliquid factum extra : & e por ella non sale propreamente ninguna cosa fechͣ de fuera Massale alguna accioono alguna perfection en aquella | que la faze \\\hline
2.3.1 & sed magis resultat aliqua actio , & Mas los estrumentos delas artes me cauicas son fazedores \\\hline
2.3.2 & quia quodlibet aliquo modo est quaedam res possessa : & en esto conuienen | que cada vno dellos es en algunan manera possession del amen \\\hline
2.3.2 & restat ostendere , & finca | de ver \\\hline
2.3.3 & quasi ad tria reducuntur : & e al abastamiento dela uida son aduchͣsa tres cosas \\\hline
2.3.3 & restat videre , & e quanto ala maestera dela obra finca de ver \\\hline
2.3.3 & quantum ad aeris temperamentum . Tangit autem Palladius in libro de Agricultura , tria , & quanto ala ceprança del ayre | mas tanne peladio enłlibro dela agnicultura tres cosas \\\hline
2.3.3 & et putrescunt : & e se podresçen en essa misma manera el ayre \\\hline
2.3.4 & restat videre , & en qual manera son de fazer \\\hline
2.3.4 & sunt tria consideranda , & que es de fazer son de penssar tres cosas \\\hline
2.3.4 & secundum suam ampliorem partem respiciat oriens & segunt la su parte mayor catare a lorsete del yuierno \\\hline
2.3.4 & eo quod oblique respiciatur a sole , & por quela cata entrauiesso el sol \\\hline
2.3.5 & restat dicere de possessionibus . Possumus autem triplici via venari , & que contamos de suso finca de dezer delas possessiones | maspodemos prouar \\\hline
2.3.5 & quod rerum possessio est quodammodo naturalis . & por tres razons la possession delas cosas es natural en algua manera al omne \\\hline
2.3.5 & secundum Philosophum primo Polit’ necessaria est rerum possessio in gubernatione domus , & mas segunt el philosofo en el primero libro delas politicas | la possession de las cosas es neçessaria en el gouernamientode la casa \\\hline
2.3.5 & eo ergo ipso quod rerum possessio deseruit necessitati vitae , & Et pues que assi es por esta razon misma | por que la possession delas cosas sirue ala neçessidat dela uida \\\hline
2.3.5 & eo enim ipso quod homo respectu corporalium & en conparaconn de las cosas corporales e sensibles \\\hline
2.3.5 & sic et naturaliter dominatur aliis exterioribus rebus ; & avn en essa misma manera naturalmente enssennorea a todas las otras cosas de fuera . \\\hline
2.3.5 & nisi rerum possessio quodammodo naturalis esset . & si la possession delas cosas en algua manera non fuesse natural al omne ¶ \\\hline
2.3.5 & idest res a quibus nutrimentum accipimus , & que estas cosas de que tomamosnudmiento son dadas a nos por nata . \\\hline
2.3.5 & Naturale est ergo nobis habere res exteriores . & Et pues que assi es natal cosa es a nos de auer las cosas de fuera \\\hline
2.3.5 & Habere ergo dominium rerum exteriorum est & e por ende el sennorio delas cosas de fuera es en algua manera natural al omne . \\\hline
2.3.5 & et proponit viuere absque dominio exteriorum rerum , non proponit viuere ut homo , & de las cosas de fuera non proponen beuir | assi commo omne \\\hline
2.3.5 & et dominium aliquarum rerum exteriorum propter sufficientiam vitae . Natura ergo sicut dedit boni viuere , & Et pues que assi es la natura | assi conmodio al omne a beuir \\\hline
2.3.6 & quid sentiendum sit de possessione rerum exteriorum , & que auemos de sentir en fecho delas possessiones delas cosas de fuera \\\hline
2.3.6 & Rebus & e reluze mas claramente enpero las cosas estando assi como agoraes tan cosa aprouechosa es ala çibdat \\\hline
2.3.6 & quod essent contenti viuere tali vita . Possumus autem ex diuersis locis in libro Polit’ accipere tria , & que quisiessen ser pagados de beuir en tal uida comun | mas nos podemos tomar de departidos logares \\\hline
2.3.6 & In rebus ergo sic se habentibus , & Et pues que assi es estando las cosas | assi commo dicho es \\\hline
2.3.6 & propter quod oportet rem illam vel non produci ad effectum , & Por la qual cosa conuiene | que aquella cosa non sea aduchͣobra \\\hline
2.3.7 & quod ex alio et alio usu exteriorum rerum , & que por departidos usos de usar delas cosas \\\hline
2.3.7 & quod non omnes utuntur eisdem rebus , & que non usan todas de essas mismas cosas \\\hline
2.3.7 & Homines etiam diuersimode utuntur exterioribus rebus , & Et avn los omes vsan en deꝑtidas maneras delas cosas de fuera \\\hline
2.3.7 & ex quibus conbinatis resultant alii modi viuendi , & delas quales ayuntadas se le una tan departidas maneras de beuir \\\hline
2.3.7 & uel rebus exterioribus utatur illicite . & e delas cosas conueniblemente \\\hline
2.3.9 & quasi ad tria genera reducuntur . & todas las muda connes son aduchos a tres linages \\\hline
2.3.9 & Quarum una est commutatio rerum ad res : & de los quales la vna es muda connde alguas cosas | a otras cosas \\\hline
2.3.9 & Alia est commutatio rerum ad numismata . & que cunplen la mengua corporal . Otra es muda conn delas cosas alos dinos o de los dineros alas cosas \\\hline
2.3.9 & propter has ergo communitates introductae sunt illae tres species commutationum . & por eltas comiundades sobredichͣs fueron puestas aquellas tres maneras de mudaçonnes \\\hline
2.3.9 & aliquo modo sufficeret commutatio rerum ad res : & ca si ala comuidat de vn barrio o de vna çibdatur llgua manera abastasse la mudaçion delas cosas alas cosas enpero ala comunidat \\\hline
2.3.9 & oportuit introduci commutationem rerum ad denarios , & que hades es en todo el regno conuiene de poner m̃udaçion delas cosas alos dineros \\\hline
2.3.9 & Rursus si in communicatione totius regni sufficeret commutatio rerum ad denarios , & ¶Ot ssi si en la comundaf de tedo el regno abasta sse la mudaçion delas cosas alos dineros \\\hline
2.3.9 & oportuit introduci non solum commutationem rerum ad res , vel rerum ad denarios ; & prouinçias conuiene de poner non sola mente mudaçion delas cosas alas cosas | o delas cosas alos dineros \\\hline
2.3.9 & quasi solum commutabant res ad res : & buuendo en su sinpliçidat muda una vnas cosas en otras cosas \\\hline
2.3.9 & adhuc reseruatur & avn es guarda entre muchͣs naçiones barbaras e estrannas \\\hline
2.3.9 & solum res ipsas commutant . Hic autem modus forte in uno vico , & Mas commo quier que esta manera atal se podiesse guardar en vn barrio o en vna uilla non se poda guardar conueinblemente \\\hline
2.3.9 & Quare commutatio rerum ad denarios , et econuerso , & por la qual cosa si la mudacion de las cosas alos dineros | e de los dmños alas cosas \\\hline
2.3.9 & praeter commutationem rerum ad res , & por ende conniene \\\hline
2.3.9 & et rerum ad numismata , & que los que mora una en apartadas prouinçias ouiessen de fallar | sin la mudaçion delas cosas alas cosas . \\\hline
2.3.10 & numismata et pecuniam , restat dicere , quot sunt species pecuniatiuae . & finca | de dezer quantas son las maneras de los dineros . \\\hline
2.3.10 & quae fit ex eo quod res naturales commutantur in pecuniam . & por aquello que las cosas naturales se mudan en dineros . \\\hline
2.3.10 & quia a rebus naturalibus inciperet . Secunda species pecuniatiuae dicitur esse campsoria : haec enim & por razon que ha comienco delas cosas naturales . | ¶ la segunda manera de los dineros es dichͣ camiadora . \\\hline
2.3.10 & quia nec a rebus naturalibus incipit , & por que non com . iença de cosas naturales \\\hline
2.3.10 & denarii resoluuntur in massam . & Alguons diueros son fondidos en massa \\\hline
2.3.10 & et qui essent maioris ponderis resoluerentur in massam , & e los que fuessen de mayo rpeso | que los tornassen en massa \\\hline
2.3.10 & nunquam enim aliqua crescunt in se ipsis , & por que niguas cosas nunca cresçen en ssi mismas saluo \\\hline
2.3.10 & Aliae vero tres & por que toda arte \\\hline
2.3.11 & quia sunt res naturales , & por que son cosas naturales \\\hline
2.3.11 & res aliud usus rei , & e otra asa es el vso della \\\hline
2.3.11 & In quibuscunque igitur potest concedi usus rei absque eo quod concedatur eius substantia , & Et por ende en quales se quier cosas | en que se puede otorgar el uso dela cosa \\\hline
2.3.11 & dato quod res illa in nullo deterioraretur . & puesto que aquella cosa se enpeor | e por aquel uso \\\hline
2.3.11 & et usus ex ipsa re sumit originem , & e el uso toma nasçençia dela sustançia \\\hline
2.3.11 & quasi cuiuslibet rei est duplex usus : & enl primero delas politicas de cada cosa | ay dos usos \\\hline
2.3.11 & et de aliis rebus . & e delas otras cosas . \\\hline
2.3.11 & In rebus autem aliis & Mas en las otras cosas \\\hline
2.3.12 & vel assistit deferentibus mercationes ipsas . Diuiditur autem ( secundum Philosophum ) mercatoria in tres partes , & Ca segunt el philosofo la mercaduria se parte en tres partes . | En nauigaria \\\hline
2.3.13 & Expeditis ergo tribus , & que assi es desenbargadas | e determinadas las tres cosas \\\hline
2.3.13 & restat exequi de quarto , & finca de dezer dela quarta | assi commo son los sieruos \\\hline
2.3.13 & sumpta ex quadruplici similitudine . Prima via sumitur ex similitudine reperta in rebus inanimatis . & ¶ La primera razon se toma dela semeiança | que es fallada en las cosas \\\hline
2.3.13 & nisi sit ibi aliquid praedominans respectu aliorum : & que en ssennore | e en conpara conn delas otras . \\\hline
2.3.13 & Quare si sic est in rebus inanimatis & por la qual cosa siassy es enlas cosas | que non han alma \\\hline
2.3.13 & ut ex homine respectu aliorum animalium : homo enim & assi commo del omne en conparaçion de otras asalias \\\hline
2.3.13 & eo quod sit ratione prestantior , & o por conparaçion delas fenbras a los uatones . Ca ueemos que por que el uaton es mas ennoblesçido en razon \\\hline
2.3.13 & cum ergo videamus aliquos homines respectu aliorum plus deficere a rationis usu quam foeminae a viris , & que algs omes en conparaçion de los otros pueden | mas fallesçer de uso de razon \\\hline
2.3.14 & oportet enim dominans ( ut dicitur in Politic’ ) habere aliquem excessum respectu serui . & segunt que dize el philosofo en las politicas | aya algua auentaia sobre el su sieruo . \\\hline
2.3.14 & secundum bona corporis respectu illius , & que es segunt los biens del cuerpo | en conparaçion dela otra auentaia es \\\hline
2.3.14 & et bonos non resistere ordinationi legali : & e los bueons non contradigan al ordenamiento dela ley . \\\hline
2.3.14 & reseruant ipsos propter utilitatem quam inde consequi sperant . & por sieruos quardan los de matar | por enl pro \\\hline
2.3.14 & quia tales a victoribus reseruantur in bello ; & por que tales son guardados enla batalla delos vençedores e non los matan \\\hline
2.3.15 & et non valentes debellantibus resistere , & e non se podiessen defender de los lidiadores \\\hline
2.3.15 & et non semper reseruamus ordinem naturalem , & non guardamos sienpre la orden natural . \\\hline
2.3.16 & aliqui vero aliis rebus prout requirit modus domesticus : & Et algunos alas otras cosas | assi commo demanda la mana dela casa . \\\hline
2.3.16 & tria sunt attendenda , & tres cosas deuemos pessar en esto . | Conuiene de saber . \\\hline
2.3.16 & ut reseruetur ibi debitus ordo ministrandi : & assi acomnedados alos seruientes | por que sea y guardada la orden conuenible del seruiçio \\\hline
2.3.17 & reseruata condictione personarum , maxime debet attendi & segunt la condiçion delas personas \\\hline
2.3.17 & sed considerata conditione rerum aliqua pollent maiori pulchritudine , & mas penssada la condiçonn de las cosas alguas cosas resplandesçen | por mayor fermosura \\\hline
2.3.18 & et quia multi oculi in ipsos respiciunt , & por que muchos oios catan aellos comunalmente mas uerguença toman que los otros \\\hline
2.3.18 & sicut legalis iustitia est tota virtus per respectum ad impletionem legis . & ca assi la commo la iustiçia legales toda uirtud | por conparaçion al cunplimiento dela ley \\\hline
2.3.19 & restat ostendere qualiter Reges et Principes & de demostrar | en qual manera los Reyes e los prinçipes \\\hline
2.3.19 & restat videre quomodo sunt in commissis officiis solicitandi . Per se enim ipsos habere curam & finca deuer en qual manera son de acuçiar | por que cunplan bien sus ofiçios \\\hline
2.3.19 & et velle se de quibuscumque inimicis intrommittere , nullatenus decet ipsos . Hoc viso restat ostendere tertium , & ca esto ꝑtenesçe alos menores . | ¶ Esto iusto finça de demostrar lo terçero \\\hline
2.3.19 & qui respectu eorum sunt inferiores et humiles , & alos quales conuiene de ser magn animos \\\hline
2.3.20 & restat ut dicamus qualiter in mensis . & finca | que digamos en qual manera en las mesas de los prinçipes \\\hline
3.1.1 & Igitur per respectum ad homines ciuitatem constituentes , & Et pues que assi es por conparaçion | que es alos omes \\\hline
3.1.1 & huius autem est communitas ciuitatis , quae respectu communitatis domus , & e esta tal es la comunidat dela çibdat | la qual en conparacion dela comunidat dela casa \\\hline
3.1.1 & quae est principalissima communitas respectu vici , et domus , & que es much mas prinçipal | que la comunidat del barrio \\\hline
3.1.1 & sed respectu communitatis domus , & e en toda manera mas en conparaçion dela comuundat dela casa e del barrio es dichͣ mas prinçipal \\\hline
3.1.2 & Huiusmodi autem bona ( quantum ad praesens spectat ) tria esse contingit . Ordinatur enim ciuitas ad viuere , & quanto parte nesçe alo presente son tres biemes | por que la çibdat es ordenada abenir \\\hline
3.1.2 & est enim prima rerum creaturarum , & por que los comienços de todas las cosas ceradas \\\hline
3.1.2 & Si enim alicui rei deficiat aliqua perfectio competens suae speciei , & por que si a algunan cosa fallesçiere algun acabamiento que pertenezca ala suspeno ala su semeiança | commo quier que puede auer aquella cosa aquel ser menguado en alguna manera . \\\hline
3.1.2 & licet possit habere illa res esse aliquod , & que aquel ser conplidamente ¶ | Et pues que assi es mas larga cosa es el ser \\\hline
3.1.2 & quia per eam homines consequuntur omnia tria praedicta bona . & porque por ella alcançan los omes ser acabados \\\hline
3.1.3 & reperiuntur tamen multi campestre viuentes . Tangit autem Philosophus 1 Polit’ tria , & que non biuen çiuilmente . | Mas el pho tanne enl primero libro delas politicas tres razones \\\hline
3.1.4 & Nam finis generationis rerum est forma , & ca la fin dela generaçion delas cosas | es la forma de cada cosa \\\hline
3.1.4 & fit enim vicus naturaliter ex crescentia filiorum collectaneorum , & ca fazesse el uarion a tal monte de acresçentamiento de fijos e de metos e de parientes e de vezinos \\\hline
3.1.5 & ut melius possit resistere impugnationem hostium : & por que pueda meior cotra dezir | e con tristar alos enemigos \\\hline
3.1.6 & ubi diximus quod propter excrescentiam filiorum collectaneorum & do dixiemos | que por las cresçençias de los fijos \\\hline
3.1.6 & et ciuitas in regnum excrescere . & e la çibdat en regno \\\hline
3.1.6 & Si enim per generationem ex crescentibus filiis & por que si por generaçion cresçiendo los fijos e los mietos en vna casa \\\hline
3.1.6 & et ulterius excrescentibus & Et despues mas adelante cresçiendo \\\hline
3.1.6 & amplius autem ipsis excrescentibus & Et despues mas adelante cresçiendo \\\hline
3.1.6 & et magis resistere hostibus volentibus impugnare ipsos . Est enim huius impetus naturalis : & e pueden mas defender se de los enemigos | que les quieren mal fazer et esta tal inclinaçion es natural \\\hline
3.1.6 & et ut naturaliter resistant hostibus , & e por que puedan mas ligeramente defender se de sus enemigos \\\hline
3.1.6 & restat uidere in quot partes oportet hunc tertium librum diuidere , & finca de ver en quantas partes conuiene de partir este terçero libro \\\hline
3.1.6 & et ad resistendum uolentibus turbare pacem , & e para yr | contra los que quisieren turbar la paz \\\hline
3.1.6 & Totum ergo hunc librum tertium diuidemus in tres partes . & Et pues que assy es todo este re terçero | libro partiremos en tres partes . \\\hline
3.1.7 & Socrates autem quandiu philosophatus esset circa naturas rerum , & as socrates commo ouiesse phophado luengo tienpo çerca las naturas delas cosas \\\hline
3.1.7 & maiores corpore et audaciores corde et praestantiores viribus sunt foeminae quam masculi : & que biuen de rapina mayores son de cuerpo | e mas osadas de coraçon \\\hline
3.1.7 & vel existentes in maioribus principatibus ; & e semeia | alos que estan en mayores senno rios \\\hline
3.1.8 & Maximam unitatem et aequalitatem non oportet quaerere in omnibus rebus . & on conuiene de demandar en todas las cosas \\\hline
3.1.8 & reseruari in una specie , & ca por que toda la bondat del mundo non puede ser fallada en vna espeçie \\\hline
3.1.8 & reseruetur maior perfectio , & ca en muchos espeçies e semeianças delas cosas se salua mayor perfectiuo que en vna tan sola mente . \\\hline
3.1.9 & Hanc autem aequalitatem non de facili esset possibile reseruari & mas esta egualdat non se podria guardar de ligero entre los çibdadanos \\\hline
3.1.10 & vel trium puerorum , & por razon de dos o tres moços \\\hline
3.1.10 & vel tres vel propter paucos pueros velle magnam multitudinem diligere puerorum tanquam proprios filios , hoc est ponere parum de melle in multa aqua . & ca por dos o por tres o por pocos mocos querera mar grant muchedunbre de moços | assi conmo a fijos propreos \\\hline
3.1.10 & vel trium filiorum innumerabilem multitudinem puerorum & que sea amada grant muchedunbre et sin cuenta de mocos \\\hline
3.1.11 & Esse res communes , & ue las cosas sean comunes \\\hline
3.1.11 & Nam in rebus deseruientibus ad victum est considerare duo : & ca en las cosas que siruen ala uida e ala uianda del omne auemos de penssar dos cosas | Conuiene a saber las cosas \\\hline
3.1.11 & videlicet res fructiferas , & que lie una fructo \\\hline
3.1.11 & qui oritur ex talibus rebus , & que nasçe de tales cosas \\\hline
3.1.11 & vel erunt communes res fructiferae , et diuidentur fructus : & o serun comunes las cosas | que lie una fructa e partiran los fructos \\\hline
3.1.11 & res fructifefae quam fructus , ut Socrates ordinauit . & tan bien las cosas | que lie una fructo commo los fructos dellas \\\hline
3.1.11 & quod si res ciuium communes essent , & que si las cosas de los çibdadanos fuessen comunes mayores contiendas se le unatarien en la çibdat \\\hline
3.1.11 & Secunda ex parte communicantium in rebus illis . & La segunda de parte de aquellos | que han las cosas comunes¶ \\\hline
3.1.11 & cum unus ab usu et fructu illius rei communis propter alium impeditur , & del uso | e del fructo de aquella cosa comun de alli se leuna talid e discordia entre ellos \\\hline
3.1.11 & secundum rei veritatem non esset & e delas mugers creyessen | que eran ayuntados \\\hline
3.1.11 & ait enim possessiones et res civium debere esse proprias , et communes . Proprias quidem quantum ad dominum , communes vero propter virtutem liberalitatis . & ca dize que las possessiones e las cosas de los çibdadanos deuen ser propreas | e comunes propraas \\\hline
3.1.11 & expedit cuilibet habere res et possessiones proprias & conuiene alos çibdadanos de auer las cosas | e las possessiones prop̃as \\\hline
3.1.12 & secundum tria quae requiruntur ad bellum . Homines enim bellatores decet esse mente cautos & que son meester para la batalla ca los omes lidiadores conuiene | que sean cuerdos por entendimiento e sabios \\\hline
3.1.12 & quia secundum Vegetium in De re militari , & segunt dizeuegeçio en el libro del negoçio dela caualleria \\\hline
3.1.13 & et quia non tot oculi respiciunt personam priuatam quam publicam , & por que non paran mientes tantos oios en la persona priuada commo en la publica por ende non se puede \\\hline
3.1.14 & Restat ergo exequi de iussione et ordine ciuitatis , & pues que assi es finca de tractar del sdepartimiento | e de la ordenaçion dela çibdat \\\hline
3.1.14 & secundum tria quae de ipsis bellantibus Socrates statuebat . & segunt aquellas tres cosas | que el establesçia en los lidiadores ¶ \\\hline
3.1.14 & ad tria debet respicere , & a tres co sas deue deuer mietes . \\\hline
3.1.14 & ad tria deberet respicere . & dela çibdat a tres cosas deuia tener mientes | conuiene a saber . a los çibdadanos \\\hline
3.1.15 & rei veritatem non possibile neque utile : & Enpero si dixieremos | que estas cosas son comunes \\\hline
3.1.15 & de rebus aliorum , & o quanto pudiere delas cosas de los otros \\\hline
3.1.15 & reseruari communitas quantum ad amorem : & e en los fijos deue ser guardada comunidat \\\hline
3.1.15 & sed in possessionibus non solum debet reseruari communitas quantum ad amorem & non solamente | quanto al amor \\\hline
3.1.15 & quando ciues se amando et diligendo maxime unirentur . Sic ergo exposita mente Socratis de communitate rerum & quando los çibdadanos amandose | e quariendose muy bien fuessen much ayuntados en amor . \\\hline
3.1.16 & vel maiores possessiones , & si ninguno de los çibdadanos non ouiesse mayores rentas o mayores possessiones \\\hline
3.1.18 & et dominium exteriorum rerum : & e por el sennorio delas cosas de fuera \\\hline
3.1.18 & et iniurias inferunt in res exteriores propter auaritiam : & por las cosas de fuera | e por la auariçia \\\hline
3.1.19 & Hanc autem quantitatem distinxit in tres partes , & e esta quantidat departia en tres partes | conuiene a saber en lidiadores e en menestrales \\\hline
3.1.19 & ( quantum ad praesens spectat ) tribus indigere , & quanto pertenesçe alo presente ha mͣester tres cosas \\\hline
3.1.19 & Ad haec enim tria deseruiunt praedicta tria genera virorum . & Et para estas tres cosas siruen los tres linages sobredichos de los uarones \\\hline
3.1.19 & diuidens totam regionem idest totum territorium ciuitatis in tres partes & partiendo todo el regno o todo el terretorio de la çibdat en tres partes . \\\hline
3.1.19 & Dicebat enim omnia iudicia debere esse de tribus , & et dizia que todos los iuyzios deuian ser de tres maneras \\\hline
3.1.19 & secundum quod de tribus litigant ciues : & segunt que por tres cosas contienden los çibdadanos . \\\hline
3.1.19 & vel iniustificat in res nocendo et damnificando ipsum : & o le faze tuerto en las sus cosas enpeesçiendol \\\hline
3.1.19 & res appellabat nocumentum : & quanto alas cosas llamaua enpeesçimiento . \\\hline
3.1.19 & et postea in pugillaribus scriptam adduceret suam sententiam : & por si deuia penssar | e despues poner sus nina en esc̀pto \\\hline
3.1.19 & ut princeps principalem curam haberet de tribus , & que ouiesse el prinçipe prinçipalmente cuydado de tres cosas . \\\hline
3.1.19 & videlicet de rebus communibus , & Conuiene a saber delas cosa comunes ¶ \\\hline
3.1.19 & de rebus communibus , & e delos pelegninos \\\hline
3.1.20 & ad praesens spectat , sequendo dicta Philos’ 2 Pol’ increpare Hippodamum quantum ad tria . Primo , & rephender a ipodomio | quanto a tres cosas \\\hline
3.1.20 & secundum ipsum distingui debeat in tres partes , & segunt el dixo | se deuia partir en tres partes . \\\hline
3.2.1 & restat nunc exequi de aliis duabus partibus & finca nos de dezir delas otras dos partes . \\\hline
3.2.1 & quae potest respicere totum populum : populus enim ad bene agendum , & que parte nesçe a todo el pueblo | ca el pueblo es de abiuar \\\hline
3.2.2 & quorum tres sunt boni , & de los quales tro son buenos \\\hline
3.2.2 & et tres sunt mali . & e los trsson malos . \\\hline
3.2.2 & et aliorum oppressio , sic est corruptum et peruersum . Hoc enim modo & assi es señorio corruyto e malo | e en esta manera segunt derecho e tuerto commo pone el pho . \\\hline
3.2.2 & ut maiores in populo , & que los mayores en el pueblo \\\hline
3.2.3 & restat ostendere & e quales son malos finca de demostrar entre los prinçipados derechs e buenos \\\hline
3.2.3 & eo quod ibi perfectior unitas reseruetur . & por que yes fallada mas acabada vnidat ¶ \\\hline
3.2.4 & Philosophus 3 Politicorum videtur tangere tres rationes , & lpho en el terçero libro delas politicas tanne tres razons \\\hline
3.2.4 & Videntur enim in Principe tria esse necessaria , & que por vno ca paresçe | que en el prinçipe son tr̃o cosas neçessarias \\\hline
3.2.4 & Ex his autem sumi possunt tres viae , & Et destas trs cosas se pueden tomar tres razons \\\hline
3.2.5 & aliquando magis diligere minores . Talibus obiectionibus de facili respondetur : & que alos mayores atales | argunentos de ligero podemos responder \\\hline
3.2.6 & Philosophus 5 Politic’ narrat tria , & pho en el quanto libro delas poluenta tres cosas \\\hline
3.2.6 & ( quantum ad praesens spectat ) ad tria solicitari . Primo , & quanto parte nesçe alo presente | que de tres cosas aya cuydado . \\\hline
3.2.6 & Quare expedit regem habere praedictos tres excessus . & por la qual cosa conuiene | que el rey aya aquellas tres aun ataias \\\hline
3.2.6 & restat ostendere , & e auer a una taia de los otros . \\\hline
3.2.6 & quia Rex respicit bonum commune : & ¶ El primero es que el rey cata al bien comun \\\hline
3.2.6 & quia ex hoc maxime resultat bonum honorificum , & por que desto nasçe prençipalmente el bien honrrado \\\hline
3.2.7 & etiam satagit impedire eorum maxima bona . Tangit autem Philosophus 5 Polit’ tria maxima bona , & para enbargar los bienes dellos | e tanne espho en el quinto libro delas politicas muy grandes tres bienes \\\hline
3.2.8 & diligenter considerare debet in naturalibus rebus . & qual es el su ofiçio deue penssar con grant acuçia en las cosas naturales \\\hline
3.2.8 & quod natura primo dat rebus & que la natura primeramente da a todas las cosas aquello \\\hline
3.2.8 & per quem agit et resistit contrariis . Tertio ignis & por la qual obrase de fien Lo terçero de de todos sus contrarios el fuego \\\hline
3.2.8 & tria requiruntur . Primo , & Lo primero \\\hline
3.2.8 & est quasi sagitta quaedam dirigenda in finem et in bonum . Tria igitur spectant ad regis officium . & e gara la fin e al bien comun . | Et pues que assi estes cosas pertenesçen al offiçio del Rey . \\\hline
3.2.8 & sunt tria , & e pueda bien beuir son estas . \\\hline
3.2.8 & res exteriores . Decet ergo Reges et Principes sic regere ciuitates & assi commo son las riquezas e los algos . | Et por ende conuiene alos Reyes \\\hline
3.2.8 & ut sibi subiecti abundent rebus exterioribus & por que los sus subditos abonden en las cosas de fuera \\\hline
3.2.8 & restat ostendere , & que entiende finca de demostrar \\\hline
3.2.8 & etiam tria sunt , & los quales enbargos son tres de los quales . ¶ El vno toma nasçençia dela natura \\\hline
3.2.8 & restat ostendere quomodo eos debeant in finem dirigere . & de demostrar | en qual manera ellos deuengar e enderesçar su pueblo ala fin que entienden . \\\hline
3.2.8 & Haec etiam tria sunt . & Et para esto ver son menester tres cosas . \\\hline
3.2.8 & ea per quae resultat commune bonum , sunt remunerandi et praemiandi : & por que viene grant bien al comun son de gualardonar \\\hline
3.2.9 & respondit Rex ille , & Et el Rey respondiol \\\hline
3.2.9 & cui omnia sunt nota , et eius potentia cui nihil potest resistere , & la prouidençia de dios aqui todas las cosas son manifiestas | e el su poderio aqui non puede ser ninguna cosa contraria legnia \\\hline
3.2.10 & tamdiu non potest aeque de facili eius potentiae resisti : & Entre tanto non puede de ligero contradezer al su mal poderio \\\hline
3.2.12 & ( ut patet ex habitis ) tres principatus boni , & por las cosas sobredichas tres prinçipados son buenos e tres malos \\\hline
3.2.12 & et tres peruersi . & por las cosas sobredichas tres prinçipados son buenos e tres malos \\\hline
3.2.12 & quantum , ad praesens spectat , tria intendunt , & quanto parte nesçe alo presente tres cosas entienden . \\\hline
3.2.12 & hylarem ostende vultum . Respondente illo quod non posset propter imminentia pericula : & que lo non podia fazer | por muchs peligros qual esta una aprestados . \\\hline
3.2.12 & quia iniqui principatus diuitum congregantur in ea : restat & e por que los malos prinçipados | e los malos señorios de los rricos son ayuntados en ella , \\\hline
3.2.13 & ut liberent patriam ab oppressione eorum . & por que libren la trrͣa dela grad serindunbre dela grad primia dellos . | pues que asi es \\\hline
3.2.14 & et regius principatus . Narrat autem Philosophus 5 Politicorum tres modos corruptionis tyrannidis , dicens , & Ca cuenta el phon enel quinto libro delas politicas | tres maneras dela corrupçion dela tiranja e dize que la tiranja corrope de si \\\hline
3.2.15 & non permittere in suo regno transgressiones modicas . & La primera es que non consienta en su regno muchos pequanos males \\\hline
3.2.15 & Nam multae modicae transgressiones & ca muchs pequannos males egualan se a vn grant mal \\\hline
3.2.15 & paruae enim transgressiones & por que los pequa non s males \\\hline
3.2.15 & disponunt ad transgressiones magnas . & si fueren muchs apareian a omne a grandes males . \\\hline
3.2.15 & secundum rem , & ca las corrupconnes alongadas de fecho e allegadas \\\hline
3.2.15 & ut fiant transgressores iustitiae . Est autem haec cautela maxime utilis ad homines , & por que se fagan traspassadores dela iustiçia | Mas esta cautela es muy aprouechosa \\\hline
3.2.15 & nisi per potentiam ciuilem puniantur transgressores iusti . & sinon dando pena poderio çiuil \\\hline
3.2.15 & et vult punire transgressores iusti , habere multos exploratores , & auer much sassechadores | e muchs pesquiridores \\\hline
3.2.15 & et transgressoribus iusti . Nonum maxime saluans regnum , & que traspassan la iustiçia . | La ixͣ cosa \\\hline
3.2.16 & Restat ergo de consilio pertransire quae tractanda sunt circa ipsum . & Et pues que assi es fincanos de fablar del consseio | quales cosas son de trattrar çerca el . \\\hline
3.2.17 & possunt autem circa speculabilia , et circa naturas rerum , et circa aeterna fieri quaestiones multae , & e en las sçiençias delas naturas delas cosas | e enlas sçiençias delas cosas perdurables . \\\hline
3.2.17 & restat videre qualiter est consiliandum , & finca de ver | en qual manera es de tomar el conseio \\\hline
3.2.17 & esse debent de rebus magnis . & assi conmo dicho es deuen ser de grandes cosas \\\hline
3.2.17 & quod apud Romanam Rempublicam exaltauit fidelitas consiliantium : & ca esto fue lo que enssalço la comunidat de Roma fieldat de buenos consseieros \\\hline
3.2.17 & quod fidum et altum erat secretum consistorium reipublicae , silentique salubritate munitum : & e muy alto era conssisto no secreto dela comunidat de roma | alos quel guardananca era guaruido de grant fialdat . \\\hline
3.2.17 & quod alii respectu eius consiliatores dici non debent , & que los otros en conparaçion del non deuien ser dichos consseieros mas lisongeros . \\\hline
3.2.18 & Haec autem sunt tria , & e estas son tres cosas conuiene saber . \\\hline
3.2.18 & secundum quod in omni locutione tria sunt consideranda , & ¶ el dezidor que fabla \\\hline
3.2.18 & et rem de qua loquatur . & Et la cosa de que fabla . \\\hline
3.2.18 & Tertio fit credulitas ex parte ipsarum rerum : quod contingit , & Lo terçero viene la creeçia de parte de aquellas cosas | que omne fabla . \\\hline
3.2.18 & scire et cognoscere ipsas res , & e de conosçer | daquellas cosas \\\hline
3.2.18 & et ex ipsis rebus & e delas cosas mismas \\\hline
3.2.18 & est ex ipsis rebus , & nasçe de aqual las cosas \\\hline
3.2.18 & et qui existimantur prudentes , existimantur talia facere : ideo ad hoc quod aliquis ex rebus de quibus loquitur fidem faciat , vel oportet quod sit prudens & para fazer tales cosas Morende | para que alguno faga fe delas cosas \\\hline
3.2.18 & quia non mentientur ex parte rerum de quibus loquuntur ; & assi que non mientan de parte de aquellas cosas \\\hline
3.2.18 & et scient qualiter sit agendum . Haec ergo tria quaerenda sunt in consiliariis : & e saben en qual manera los han de fazer . | Et pues que assi es estas tres cosas son menester en los consseieros \\\hline
3.2.19 & venditores res suas vendere vellent . Tertio , & que se venden | si los vendedores quisieren vender las cosas \\\hline
3.2.19 & ad quos resistere non valemus , & alos quales non podemos contradezer \\\hline
3.2.20 & Restat & segunt la orden sobredichͣ \\\hline
3.2.20 & quarum tres tanguntur 1 Rhet’ quarta vero tangitur 1 Polit’ . & por quatro razones delas quales las tres tanne el philosofo en el primero libro de la rectorica | e la quat catanne en el sexto libro delas politicas \\\hline
3.2.20 & Quare si legum conditores respectu iudicum sunt pauci , & por la qual cosa si los fazedores son pocos en conparaçion de los iuezes \\\hline
3.2.20 & et quam paucissima arbitrio iudicum committere . Has autem tres rationes tangit Philosophus 1 Rhetoricorum dicens & e en aluedrio de los iuezes . | Et estas tres cosas tanne elpho en el primero libro de la rectorica \\\hline
3.2.21 & ad rem vel ad negocium , & si non lo que pertenesçe a aquel fecho o a aquel negoçio \\\hline
3.2.23 & ad quae decet respicere iudicem , & para que perdone alas obras de los omes | et para que sea mas piadoso que cruel \\\hline
3.2.23 & quod iudicans potius debet respicere & mas deue tener mientes al ponedor dela ley \\\hline
3.2.23 & quod dicitur 1 Rhetor’ quod iudicans non debet respicere ad verba legum , & que el iuez non deue parar mientes alas palabras delas leyes | mas al entendimientodellas ¶ \\\hline
3.2.23 & debet ergo iudex non ita respicere ad partem ut ad hoc particulare negocium in quo delinquunt , & Et por ende eliez non deue | assi catara vna obra particular \\\hline
3.2.23 & ideo dicitur 1 Rhetor’ quod iudicans non debet respicere ad partem , & Et por ende dize el pho en el primero libro de la rectonca | que el uiez non deue catar ala parte \\\hline
3.2.23 & ad pietatem respiciens diuturnitatem temporis , & que inclina al iues a piedat | catando alongamiento de tp̃o \\\hline
3.2.23 & quod respicit multitudinem operum . & que cata ala muchedunbre delas obras \\\hline
3.2.23 & est cum ipso misericorditer agendum , et magis respiciendum est ad multum & Et en tal cosa commo esta | mas deue omne tener mientes \\\hline
3.2.23 & qui resident . & Et pues que assi es peiresçe \\\hline
3.2.24 & Has autem tres distinctiones iuris Phil’ tradidit : & que es puesto por omes . | Estos tres departimientos del derecho puso el philosofo . \\\hline
3.2.24 & Itaque cum naturae rerum sint eaedem ubique , & Et pues que assi es commo las naturas delas cosas sean vnas en todos logares \\\hline
3.2.24 & res enim eaedem sunt apud omnes ; & por que las cosas natraales son vnas a todos los omes \\\hline
3.2.24 & ut haec fiant . Surgunt enim ista ex ipsa natura rei , & que estas cosas se fagan . | ca estas cosas sele una tan dela natura dela cosa . \\\hline
3.2.25 & respectu iuris gentium dicitur esse naturale . & en conparacion del derecho delas gentes | es dicho derecho natural . \\\hline
3.2.25 & respectu iuris gentium . & en conparaçion del derecho delas gentes ¶ \\\hline
3.2.25 & prout appetimus esse et bonum , est naturale respectu iuris animalium , & e desseamos bien es natural en conparaçion del derech delas aianlias | o en conparaçion del derech \\\hline
3.2.25 & siue respectu iuris quod natura omnia animalia docuit : & que la natura demostro a todas las aian lias ¶ | Avn en essa misma este dereches dicho natural \\\hline
3.2.25 & sic etiam huiusmodi ius est naturale respectu iuris ciuilis , & Et el detecho delas gentes es natural | en conparaçion del derech ciuil \\\hline
3.2.25 & Tria ergo sunt aliquo modo de iure naturali , & Et pues que assi es tres cosas son en alguna manera del derecho natural Lo primero es que el sea ygualado proporçionado ala natura humanal \\\hline
3.2.26 & siue ius humanum et positiuum ad tria comparari , & Saresçe que derecho çiuil o el derecho humanal e positiuo es conparado a tres cosas . \\\hline
3.2.26 & quae per illam legem est regulanda . Tria igitur lex habere debet , & la qual gente ha de ser reglada | por aquella ley . \\\hline
3.2.26 & prout ad haec tria comparatur . & en quanto es conparada a estas tres cosas . \\\hline
3.2.28 & quod aliqui effectus legum sumuntur respectu operum fiendorum , aliqui vero respectu operum iam factorum . & Et pues que assi es dende uiene | que algunas obras delas leyes se toman en conparaçion delas obras \\\hline
3.2.28 & Respectu fiendorum quidem tria possumos attribuere legibus , & que son de fazer . | Et alguas se toman en conparaçion delas obras \\\hline
3.2.28 & Respectu factorum vero duo legibus attribuimus , & Mas en conparaçion delas obras \\\hline
3.2.28 & Secundum igitur haec tria genera fiendorum , & pues que assi es | segunt estas tres maneras delas obras que son de fazer podemos a podar \\\hline
3.2.28 & tria legibus attribuimus , & e a proprear tres cosas \\\hline
3.2.28 & sed etiam quae modicam malitiam habent annexam permitti possint a legislatore . Viso quae attribuenda sunt legibus respectu operum fiendorum : & estas tales deue las conssentir el fazedor dela ley ¶ | Visto que obras son de apodar alas leyes \\\hline
3.2.28 & de leui apparere potest quae attribuenda sunt eis respectu operum iam factorum . & de ligero puede paresçer | que obras son de apodar alas leyes \\\hline
3.2.28 & nec praemiantur . Quinque igitur attribuimus legibus : duo respectu operum bonorum , & Las dos en conparacion delas buean sobras . | Assi commo es mandar \\\hline
3.2.28 & et duo respectu malorum , & Et otras dos en conparacion delas malas obras assi conmoes vedar que se non fagan . \\\hline
3.2.28 & sed unum attribuimus legibus respectu operum indifferentium & Mas bna obra sola apodamos | e a propiamonsa las leyes en conparaçion delas obras \\\hline
3.2.29 & sed ut in ea reseruatur virtus iuris naturalis . & mas en quanto enella es guardada la uirtud del derecho | e dela ley natural . \\\hline
3.2.30 & et omnes transgressiones corrigere . Rursus & e castigar todos los trasgreedores . | ¶ \\\hline
3.2.30 & et si omnes transgressiones possent puniri a principante , & Otrossi si todos los trasgreemientos conosçidos ꝑudiessen ser codepnados del prinçipe . \\\hline
3.2.30 & cuius transgressores & Et los tris passadores desta ley diuirial fuessen condenpnados en este siglo o en el otro \\\hline
3.2.32 & Expeditis ergo tribus , & Et por ende desenbargadas las tres cosas \\\hline
3.2.32 & restat dicere de quarto , & finca de dezer de la quarta . \\\hline
3.2.32 & non est sufficiens resistere impugnantibus , & non se podria defenderde | los que mal le quisiessen \\\hline
3.2.32 & quod intenditur in re accipienda est eius notitia , benedictum est & es de tomar la declaraçion | e el conosçimiento de aquella cosa . \\\hline
3.2.32 & commutatio rerum , et cetera talia sunt ea , & Et la comutaçion delas cosas . | Et o tristales cosas son aquellas \\\hline
3.2.33 & quod tres oportet esse partes ciuitatis . & que conuiene que sean tres partes dela çibdat . \\\hline
3.2.33 & Hoc ergo modo quo diuisa est ciuitas in tres partes , diuidi potest quilibet populus et quodlibet regnum . & se puede departir cada pueblo | et cada regno en tres partes . \\\hline
3.2.33 & reseruabitur & e se amaran los vnos alos otros \\\hline
3.2.33 & et terrarum , poterit aliqualis aequalitas reseruari inter ciues . & e delas o tris possessiones | podria ser guardada algunan egualdat entre los çibdadan \\\hline
3.2.34 & ( quantum ad praesens spectat ) tria , & uanto alo presente parte nesçe el pueblo alcança tres bienes \\\hline
3.2.34 & et abundantia exteriorum rerum . Prima via sic patet . & e abondamiento de los bienes tenporales \\\hline
3.2.34 & et tranquillitas ciuium , et abundantia exteriorum rerum . & e assessiego de los çibdadanos | e abondamiento de los bienes tenporales . \\\hline
3.2.34 & Consurgit etiam ex hoc abundantia exteriorum rerum . & e avn destose leunata en el regno abondamiento delos bienes tenporales . \\\hline
3.2.35 & forefaciendo in ipsum , non exhibere ei debitum honorem et obedientiam condignam . Restat videre , quomodo non debent ipsum prouocare , & e non le faziendo obediençia | qual deuen e honrra conuenble . \\\hline
3.2.36 & tria potissime in se habere debent . Primo quidem esse debent benefici , & La primera que sean bien fechores e liberales e francos ¶ \\\hline
3.2.36 & videre restat , & por que sean amados del pueblo . fiça deuer en qual manera se de una auer | porque sean temidos del pueblo . \\\hline
3.2.36 & quas exercent in subditos . In punitione autem tria sunt consideranda videlicet punitionem ipsam , personam punitam , et modum puniendi . Quantum ergo ad punitionem , & Mas tres cosas son de penssar | en la pena que dan los prinçipes . \\\hline
3.3.1 & ut patet per Vegetium in De re militari , & assi commo prueua Uegeçio en el libro | do tracta del Fecho de la . \\\hline
3.3.1 & in primo libro de re militari , & lo que dize Uegeçio en el primer libro del Fecho de la caualleria \\\hline
3.3.1 & hunc totalem librum diuisimus in tres libros . & e los prinçipes partimos este libro todo en tres libros \\\hline
3.3.1 & Omnes autem tres prudentias decet habere Regem , & Et todas estas tres sabidurias conuiene | que aya el Rey . \\\hline
3.3.1 & et resistit prohibentibus . & Et otra por la qual acomete e contradize a las cosas \\\hline
3.3.1 & principaliter respiciunt commune bonum , & catan al bien comun \\\hline
3.3.1 & et ex oppressione debilium personarum , dicere possumus quod sicut ad fortem & Et otrossi por el agrauiamiento de las perssonas flacas podemos dezir \\\hline
3.3.1 & et omnes oppressiones eorum & e todos agrauiamientos aquellos que son el regno segunt \\\hline
3.3.2 & ut ait Philosophus 7 Polit’ . Ratio autem huius assignatur a Vegetio primo libro De re militari capitulo secundo , & assi commo dize el philosofo en el . vij . libro de las . | la razon desto muestra vegeçio en el primer libro del Fecho de la caualleria \\\hline
3.3.2 & videre restat , & finca de ver \\\hline
3.3.3 & Quod concordat cum Vegetio de re militari dicente , & el qual dicho concuerda con vegeçio del Fecho de la caualleria \\\hline
3.3.3 & videre restat , & finca de ver \\\hline
3.3.3 & Tribus igitur generibus signorum & Et pues que asy es \\\hline
3.3.5 & restat inquirere , & quales son los meiores lidiadores . \\\hline
3.3.5 & qui ad hos et ad maiores sunt continue assueti . Rursus , & Ca a estos e a mayores trabaios son vsados de cadal dia . \\\hline
3.3.6 & Recitat Vegetius in libro De re militari , & c Uegecio en el libro del Fecho de la caualleria \\\hline
3.3.6 & quod prudentes in rebus aliis propter inexercitium armorum non sunt industres in bellis . Exercitium enim in quolibet negocio praebet audaciam , & por non auer vso de las armas non son sabidores en las faziendas . | Ca el vso en cada vn negocio \\\hline
3.3.6 & restat ostendere quomodo exercitandi sunt bellantes ad incedendum gradatim , & finca de demostrar | en qual manera se deuen vsar los lidiadores \\\hline
3.3.6 & Videtur enim hoc valere ad tria . & Ca paresçe que esto les vale atres cosas \\\hline
3.3.6 & Quod etiam ad tria est utile . Primo ad remouendum impedimenta . & la qual cosa es prouechosa a tres cosas | Lo primero para tirar los enbargos \\\hline
3.3.6 & Secundo ad terrendum aduersarios . Tertio ad infligendum maiores plagas . & Lo segundo para espantar los enemigos . | Lo tercero para fazer mayores llagas . \\\hline
3.3.7 & Possumus autem praeter tria praedicta , & e et podemos sin aquellas tres cosas aque dixiemos \\\hline
3.3.8 & quod in aliis rebus & que en las otras cosas \\\hline
3.3.8 & restat ostendere , & e costruir guarniçiones e castiellos . \\\hline
3.3.8 & videlicet hostibus resistere , & Lo primero estar e lidiar contra los enemigos \\\hline
3.3.8 & tria sunt consideranda , & Et por ende son de penssar tres cosas . \\\hline
3.3.8 & quae respicit hostes , & que cata a los enemigos \\\hline
3.3.8 & et alta nouem . Est tamen aduertendum quod si fossa sit alta pedum nouem , propter terram eiectam supra fossam crescit & e alta de nueue . | Enpero conuiene de saber \\\hline
3.3.8 & ita quod tota fossa alta erit quasi pedes tresdecim : & assi que toda la carcaua sera alta de treze pies . \\\hline
3.3.10 & quod respicientes decani agnoscebant centurionem proprium , & a la qual catando los deanes conosçian al su senora propreo \\\hline
3.3.11 & rei per quam cognoscitur in se ipsa , & por el qual la sabemos en si mesma | que aquel conosçimiento \\\hline
3.3.11 & possent inuadentibus resistere . Sic enim dicendo , & puedan defender se de los acometedores . | Ca assi diziendo \\\hline
3.3.12 & ulterius dicere restat , & e segunt que paresçe non nos finca de dezir ninguna cosa en esta materia \\\hline
3.3.12 & quod in qualibet acie praeter numerum pugnatorum constituentium aciem , reseruandi sunt aliqui strenui bellatores extra ipsam aciem qui possint & sin el cuento de los lidiadores | que fazen el az son de guardar algunos buenos et fuertes lidiadores fuera del az \\\hline
3.3.12 & Haec igitur tria obseruanda sunt in constitutione acierum . Primo , & Et por ende estas tres cosas son de guardar en el ordenamiento de las azes . \\\hline
3.3.12 & ut extra quamlibet aciem reseruentur aliqui milites strenui & lo tercero que fuera de cada vna de las azes sean guardados algunos estremados caualleros e osados \\\hline
3.3.13 & Secunda ex resistentia ossium . & La segunda se toma de la resistençia e dureza de los huessos . \\\hline
3.3.13 & tanto propter armorum resistentiam difficilius itur ad carnem . & mas toma de las armas . | tanto mas tarde viene el colpe a la carne \\\hline
3.3.13 & sumitur ex resistentia ossium . & para puar esto se toma del defendimiento de los huessos \\\hline
3.3.14 & ad resistendum bellantibus , & para lidiar con sus enemigos \\\hline
3.3.14 & ne possint impugnantibus resistere . & por que non puedan lidiar contra sus enemigos . \\\hline
3.3.14 & Secundum quod reddit hostes fortiores ad resistendum , & segundo que faze los enemigos mas fuertes | para lidiar es el logar \\\hline
3.3.14 & sic locus aptus facit eos potentiores ad resistendum . Tertium , & Assi el logar conuenible | e bueno fazelos mas fuertes \\\hline
3.3.14 & difficilius possunt hostes resistere : & con mayor trabaio se pueden defender de sus enemigos . \\\hline
3.3.14 & quia non habebunt potentiam resistendi . Secundo debet diligenter explorare eorum itinera , & por que non han poder de se defender . | Lo segundo deue escudriñar con grand acuçia los caminos dellos \\\hline
3.3.15 & videre restat , & finca nos de ver qual manera son los enemigos de ençerrar e de çercar . \\\hline
3.3.15 & Restat nunc tertio declarare , & finca nos agora lo terçero de mostrar \\\hline
3.3.15 & et non possumus illis resistere . & e non pueden estar contra ellos | nin lidiar con ellos . \\\hline
3.3.16 & ut non putent in campo posse resistere impugnantibus . & que enl canpo podrian estar | nin se defender de los enemigos . \\\hline
3.3.16 & quam terrestres . Huiusmodi autem pugna in aquis facta cuiuscunque conditionis aquae illae existant , nauales dicuntur . & de qual se quier condiçion | que sean aquellas aguas son dichas batallas nauales e de naues . \\\hline
3.3.16 & restat dicere de obsessiua , defensiua , & fincanos de dezir de las otras tres . | Conuieue de saber de la de çerca \\\hline
3.3.16 & dicendum est de aliis generib’ bellorum . & e del canpo es de dezir | de las otras maneras de las batallas . \\\hline
3.3.16 & restat dicere quot modis talia deuinci possunt . & fincanos de dezir en quantas maneras tales fortalezas pueden ser vençidas . | Et conuiene de saber \\\hline
3.3.16 & restat ostendere , & finca de demostrar en que tienpo es meior de çercar las çibdades e las castiellos . \\\hline
3.3.17 & resistentiam inuenirent . & fallen enbargo | por que los non puedan enpesçer . \\\hline
3.3.17 & restat ostendere quot modis impugnare debent obsessos . & finca de demostrar | en quantas maneras se deuen acometer \\\hline
3.3.17 & ad maiores munitiones et ad maiora moenia castri , vel ciuitatis obsessae , & e a los mas fuertes adarues del castielloo de la çibdat çercada . | Et por cueuas deuen venir \\\hline
3.3.18 & Quare si modus artis debet imitari naturam quae semper faciliori via res ad effectum producit : & si la manera del arte deue semeiar a la natura . | la qual natura sienpre aduze \\\hline
3.3.18 & sicut praedicta tria genera machinarum : & commo los tres engeñios sobredichos . \\\hline
3.3.19 & Tangebatur autem supra tres modi impugnandi munitiones obsessas . & t tres maneras de conbatir las fortalezas cercadas fueron puestas dessuso de las quales la vna era por cueuas conegeras \\\hline
3.3.19 & restat dicere de impugnatione quam fieri contingit & fincanos de dezir del acometimiento | que se puede fazer \\\hline
3.3.19 & est tria considerare , & deuemos penssar tres cosas \\\hline
3.3.20 & propter quod non est inconueniens construere huiusmodi muros ex terra depressata ; & Por la qual cosa mucho cunple fazer tales muros | e tales torres albarranas de tierra muy tapiada . \\\hline
3.3.21 & tria sunt attendenda , & tres cosas son de penssar e de proueer . \\\hline
3.3.21 & quod non possint viriliter resistere obsidentibus . & por que beuiendo agua sola los lidiadores enflaquesçerse yan en tanto que non podrian defenderse de los enemigos . \\\hline
3.3.21 & restat videre , & e vençer finca de ver quales remedios son de poner \\\hline
3.3.21 & et non possent eorum machinas reparare ad resistendum bellatoribus mulieres Romanae abscissis crinibus eos suis maritis tradiderunt : & e non podien adobar los engeñios | para se defender de los enemigos . las mugeres de roma cortaron se los cabellos \\\hline
3.3.21 & et per ea quae dicta sunt resistere poterunt obsessi ; & e por aquellas cosas | que son dichas se podran defender \\\hline
3.3.22 & Enumerabantur supra tres speciales modi impugnandi munitiones obsessas . & C Contadas son de suso tres maneras espeçiales de acometer las fortalezas çercadas de las quales . \\\hline
3.3.22 & facere in munitione obsessa viam aliam correspondentem viae subterraneae factae ab obsidentibus . & otra carrera | que responda a la carrera soterraña \\\hline
3.3.22 & statim debent viam aliam subterraneam facere correspondentem illis cuniculis , & sin detenimiento ninguno deuen fazer otras cueuas soterrañas | que respondan a aquellas cueuas coneieras \\\hline
3.3.22 & ne iterum fieri possit . Viso quomodo resistendum sit debellationi factae per cuniculos : & que se non pueda fazer otra vegada . | Visto en qual manera auemos de contrallar a la batalla fecha \\\hline
3.3.22 & restat videre quomodo obsessi debeant obuiare impugnationi factae per lapidarias machinas . & por los engenios que lançan las piedras . \\\hline
3.3.22 & et pice , et resina : & e de pez e de rasina . \\\hline
3.3.22 & Quarto etiam modo resistitur machinis lapidariis , & La quarta manera para destroyr los engeñios \\\hline
3.3.22 & ligna non habent resistentiam : & nin la madera non se le puede defender \\\hline
3.3.22 & sed quia talia complete sub narratione non cadunt , prudentis iudicio relinquantur . Ostenso quomodo resistendum sit cuniculis , et lapidariis machinis : & e non las puede omne conplidamente contar dexamoslas a iuyzio de omnes sabios . | Mostrado en qual manera nos podemos defender de las cueuas coneieras \\\hline
3.3.22 & quaecumque diximus contra resistentiam machinarum . & que dixiemos dessuso | para destroyr los engeñios . \\\hline
3.3.23 & et quomodo reseruanda , & e en qual manera es de guardar \\\hline
3.3.23 & restat videre ; & finca de veer | commo son de acometer las batallas \\\hline
3.3.23 & quomodo in naui bene fabricata committenda sunt bella . Habet autem nauale bellum quantum ad aliqua similem modum bellandi cum ipsa pugna terrestri . & en la naue bien fecha e bien formada . | ca la batalla de las naues \\\hline
3.3.23 & Nam sicut terrestri pugna oportet pugnantes bene armatos esse , & Ca assi commo en la batalla de la tierra . | conuiene que los lidiadores sean bien armados \\\hline
3.3.23 & et haec requiruntur in bello nauali . Immo in huiusmodi pugna oportet homines melius esse armatos , quam in terrestri : & meior armados | que en la de la tierra \\\hline
3.3.23 & ut supra diximus in bello terrestri ) & assi commo dixiemos dessuso en la batalla de la tierra . \\\hline

\end{tabular}
