\begin{tabular}{|p{1cm}|p{6.5cm}|p{6.5cm}|}

\hline
1.1.4 & la qual los theologos llaman vida actiua que quiere dezir vida de bien obrar Et dela vida contenplatian e intellectual non sintieron la uerdat conplidamente & de vita tamen politica , quam Theologi vocant vitam actiuam , et de vita contemplatiua non usquequaque vera senserunt : \\\hline
1.1.7 & en el capitulo dela maganimidat que quiere dezir grandeza de coraçon Ca en la opinion del auariento & ut vult Philosophus 4 Ethicorum cap’ de Magnanimitate ) quia in opinione auari , \\\hline
1.1.7 & Ca aquel que bien entiende que quiere dezir e quanto lieua este nonbre fin & qui enim bene intelligit , quid importatur nomine finis , non potest eum latere quemlibet , omni via qua potest , \\\hline
1.1.10 & que ensennorear despotice que quiere dezir enssennorear . seruilmente e sobre los sieruos ¶ La quarta razon es & quam principari despotice , idest dominaliter . Quarto non est ponenda felicitas in ciuili potentia : \\\hline
1.2.3 & La otra es eutropolia que quiere dezir buena conuerssaçion o buena manera de beuir . & opera autem , et verba , ut communicamus cum aliis deseruiunt nobis ad veritatem , vitam , et ludum . \\\hline
1.2.3 & Mas entropolia que quiere dezir buena conpanma o buena manera de beuir en conpanna es & non est discolus , sed est affabilis , et curialis . Eutrapelia vero siue bona versio , est , quando aliquis sic se habet in ludis , \\\hline
1.2.6 & la qual llama el philosofo sinesis . que quiere dezir uirtud de bien iudgar ¶ la terçera es uirtud & quam Philosophus appellat synesin , idest bene iudicatiuam . Tertia , per quam praecipiamus \\\hline
1.2.8 & Et enssennable ¶ Espierto ¶ Et prouado ¶ Cauto que quiere dezir conosçedor e escogendor delo meior . ¶ &  \\\hline
1.2.12 & por nonbre comunal que quiere dezir cosa clara e cosa apuesta . Et esta estrella algunas vezes nasçe ante del sol &  \\\hline
1.2.17 & llama a estos tales non liberales que quiere dezir non francos assi commo son los logreros e los garçons & sicut debet , nimis videtur auidus pecuniae . Propter quod Philosophus 4 Ethic’ usurarios , lenones , idest viuentes de meretricio , \\\hline
1.2.19 & la qual laman magnifiçençia que quiere dezir grandeza en despender . Mas commo en cada cosa mas e menos non fagan departimiento en la naturaleza & quae respicit sumptus magnos , quam magnificentiam nominant . Sed cum magis , \\\hline
1.2.19 & a estos tales llama los han asos que quiere dezir fuegos e fornos por que estos tałs & Philosophus vero vocat eos chaunos idest ignes et fornaces , quia tales sicut fornax omnia consumunt . Quidam vero in magnis operibus faciunt decentes sumptus : \\\hline
1.2.21 & mas pariufico que quiere dezir de pequena fazienda . Et por ende dize el philosofo . & non est magnificus , sed paruificus . Ideo dicitur 4 Ethic’ \\\hline
1.2.22 & Et estos son dichos magnanimos que quiere dezir omes de grand coraçon ca nos ueemos algunos que dessi son aptos e apareiados & ut magnanimi . Videmus enim aliquos de se aptos ad magna , potentes magna et ardua exercere : \\\hline
1.2.22 & Et estos tałs ̃ llama el philosofo caymos que quiere dezir fumosos e ventosos mas nos podemos los llamar prasunptuosos &  \\\hline
1.2.26 & e alabadores de ssi que quiere dezir alabadores que se alaban & iactatores et superbos appellat : quia ultra quam eorum status requireret , vilius induebantur : \\\hline
1.2.28 & assi commo deuemos somos amigables e afabiles que quiere dezir amigos bien fablantes . Pues que assi es non es otra cosa amistanȩ & et recipiendo ipsos ut debemus , sumus amicabiles , et affabiles . Nihil est ergo aliud amicabilitas , \\\hline
1.2.28 & Et otrosi alegera conuenible la qual el philosofo llama heutropeliam que quiere dezir buena conpanina . Et pues que assi es & quae apertio nuncupatur : et debita iocunditas , quam eutrapeliam vocat . Communicando igitur cum aliis , si bene conuersari volumus , \\\hline
1.2.29 & Et estos llama el philosofo yrones que quiere dezir escarnidores e despreçiadores dessi mismos . & quos Philosophus vocat irones , idest irrisores , et despectores . Oportet ergo dare aliquam virtutem mediam , per quam moderentur diminuta , \\\hline
1.2.32 & es llamada del philosofo eroyca que quiere dezir prinçipante e sennor ante por que es señora delas otras uirtudes ¶ & a Philosopho heroica idest principans , et dominatiuat . Ex hoc ergo manifeste patet , \\\hline
1.3.1 & Mas el appetito senssitiuo del seso assi commo mas largamente dixiemos de suso partese en apetito iraçibile que quiere dezir enssannador e concupiçible que quiere dezir desseador . & in appetitu sensitiuo . Sensitiuus autem appetitus ( ut supra diffusius diximus ) diuiditur in irascibilem , et concupiscibilem . \\\hline
1.3.1 & que quiere dezir enssannador e concupiçible que quiere dezir desseador . Et por ende las sobredichas passiones & ( ut supra diffusius diximus ) diuiditur in irascibilem , et concupiscibilem . Praedictae ergo passiones sic distinguuntur , \\\hline
1.3.10 & gera Njemesim que quiere dezir tanto commo indignacion dela buena andança de los malos . Misericordia e jnuidia . & videlicet , zelum , gratiam , nemesin ( quod idem est quod indignatio de prosperitatibus malorum ) misericordiam , inuidiam , et erubescentiam siue verecundia . Sed omnes hae passiones reducuntur ad aliquas passiones praedictarum : \\\hline
1.3.10 & e es dicha uerguença o herubesçençia que quiere dezir en bermegecimiento . Et pues que assi es la uergunença es temor espeçial & et dicitur verecundia , vel erubescentia . Verecundia ergo est quidam timor , \\\hline
1.3.10 & assi es dicha enemessis o indignaçion que quiere dezir desden . Ca segunt el philosofo en el segundo libro dela rectorica . Nemessis o indignaçiones auer tristeza de aquel & vel indignatio . Nam ( secundum Philosophum 2 Rhetoricorum ) nemesis vel indignatio , est tristari de eo qui indigne videtur bene prosperari . \\\hline
2.1.3 & assi commo alinconicos nico que quiere dezir ordenador de casa de deter minar de los heditiçios delas calas por que ael parte nesçe generalmente demostrar & spectat enim ad moralem Philosophum , ut ad oeconomicum , determinare de aedificiis domorum : | ut ad oeconomicum , determinare de aedificiis domorum : quia spectat ad ipsum uniuersaliter et typo ostendere , \\\hline
2.1.3 & ala ca la que es comuidat delas perssonas assy commo parte nesçe al politico que quiere dezir ordenador de çibdat de tractar de la orden delas casas & prout habet ordinem ad domum , quae est communitas personarum ; sicut spectat ad Politicum determinare de ordine domorum , et de constructione vici , \\\hline
2.1.4 & segunt natura construyda e fecha para cada dia que quiere dezir para las obras & quod domus est communitas secundum naturam , constituta quidem in omnem diem . \\\hline
2.1.7 & que el omne es naturalmente aina l aconpannable e comun incatiuo que quiere dezir ꝑtiçipante con otro Mas la comunidat en la uida humanal & hominem esse naturaliter animal sociale et communicatiuum . Communitas autem in vita humana ( ut supra tangebatur ) \\\hline
3.2.2 & e los que deuen gouernar el pueblo son dich sobtimates que quiere dezir muy buenos ca muy buenos deuen ser aquellos que dessean ser mayores que los otros . & et qui debent populum regere vocati sunt optimates , quia optimi debent esse qui aliis praeesse desiderant . \\\hline
3.2.2 & tal es dich obligartia que quiere dezir prançipado de ricos . Et pues que assi es dos prinçipados se leuna & et opprimentes alios intendunt proprium lucrum huiusmodi principatus Oligarchia dicitur , quod idem est quod principatus diuitum . Consurgit igitur duplex principatus ex dominio paucorum : \\\hline
3.3.12 & e desende que establezcan vn triangulo que quiere dezir forma de tres linnas e esto se faz ligeramente . &  \\\hline
3.3.16 & Enpero diremos de la batalla osse ssiua e que quiere dezir batalla de cercamiento . Et por ende uisto quantas son las maneras de las batallas . & non oportet circa alia bellorum genera diutius immorari . Primo tamen dicemus de bello obsessiuo . Viso ergo quot sunt bellorum genera , \\\hline

\end{tabular}
