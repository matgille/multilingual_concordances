\begin{tabular}{|p{1cm}|p{6.5cm}|p{6.5cm}|}

\hline
1.1.5 & que en olimpiedes que quiere dezer en aquellas faziendas o es aquellas batallas non son coronados los muy fuertes & quod in Olimpidiadibus , idest in illis bellis et agonibus non coronantur fortissimi , sed agonizantes : \\\hline
1.2.6 & en el sexto libro delas ethicas . Eubullia que quiere dezer uirtud para bien coseiar . la otra es & quam Philosophus Ethic’ 6 appellat eubuliam , idest bene consiliatiua . Alia vero per quam bene iudicamus de inuentis , \\\hline
1.2.19 & Et estos son llamados pariuficos que quiere dezer omes de poca fazienda e que espienden poco . &  \\\hline
1.3.11 & e los prinçipes en tanto deuen ser nemessicos que quiere dezer desdennosos . Et en tanto deuen ser manssos &  \\\hline
2.2.16 & que fasta la hedat dela pubeçençia que quiere dezer fasta la hedat de los xiuf ͤ años deuen vsar de los vsos et mouimientos mas ligeros & quod usque ad pubescentiam , idest usque ad decimumquartum annum , leuiora quaedam exercitia sunt assumenda , \\\hline
2.3.10 & Obolostica que quiere dezer maunera de tornar los dineros en pasta . & obolostaticam , et tacos siue usuram : his enim quatuor modis possideri consueuit multitudo pecuniae . \\\hline
2.3.10 & ¶La terçera manera del arte pecuniatiua de dineros es obolostica que quiere dezer arte de peso sobrepuiante que por auentura fue fallada assi . Ca assi commo la massa del metal es partida en los dineros & obolostatica , vel ponderis excessiua : quae forte sic inuenta fuit . | vel ponderis excessiua : quae forte sic inuenta fuit . Nam sicut massa metalli in denarios diuiditur , et imprimitur ibi signum publicum ; \\\hline
2.3.12 & por que segunt el philosofo entre todas las cosas que acresçientan las riquezas es fazer monopolia que quiere dezer vendiconn de vno solo . Ca quando vno solo uende taxa el preçio & ( secundum Philos’ ) est facere monopoliam , idest facere vendationem unius : nam \\\hline
3.2.2 & ca el regno e la aristo carçia que quiere dezer sennorio de buenos e la poliçia que quiere dezer pueblo bien enssenoreante son bueons prinçipados . & Nam regnum aristocratia , et politia sunt principatus boni : tyrannides , \\\hline
3.2.2 & que quiere dezer sennorio de buenos e la poliçia que quiere dezer pueblo bien enssenoreante son bueons prinçipados . La thirama que quiere dezer sennorio malo & Nam regnum aristocratia , et politia sunt principatus boni : tyrannides , \\\hline
3.2.2 & e la poliçia que quiere dezer pueblo bien enssenoreante son bueons prinçipados . La thirama que quiere dezer sennorio malo e la obligaçia que quiere dezer sennorio duro . & et politia sunt principatus boni : tyrannides , oligarchia , \\\hline
3.2.2 & La thirama que quiere dezer sennorio malo e la obligaçia que quiere dezer sennorio duro . Et la democraçia que quiere dez maldat del pueblo & tyrannides , oligarchia , et democratia sunt mali . \\\hline
3.2.2 & e tal prinçipado es dicħa ristrocaçia que quiere dezer prinçipado de buenos omes e uir̉tuosos e dende vienen & et tunc talis principatus dicitur Aristocratia , quod idem est quod principatus bonorum et virtuosorum . Inde autem venit ut maiores in populo , \\\hline
3.2.7 & que la tirnia es la postrimera obligarçia que quiere dezer muy mala obligacion por que es muy enpesçedera alos subditos ¶ & ubi ait , tyrannidem esse oligarchiam extremam idest pessimam : quia est maxime nociua subditis . \\\hline
3.2.12 & e es llamado anstrocraçia que quiere dezer señorio de buenos . Mas si enssennorear en pocos & quia boni et virtuosi , est rectus principatus , et vocatur aristocratia siue principatus bonorum . Si vero dominentur non quia boni , \\\hline
3.2.12 & mas por que son ricos es llamado obligarçia que quiere dezer señorio tuerto . Mas quando enssennore a todo el pueblo & et vocatur aristocratia siue principatus bonorum . Si vero dominentur non quia boni , sed quia diuites , est peruersus et vocatur oligarchia . Sed si dominatur totus populus et intendat bonum omnium tam insignium quam aliorum , est principatus rectus , et vocatur regimen populi . \\\hline
3.2.23 & que el iese pieques que quiere dezer mas que iusto deue perdonar a las cosas humanales & Ideo dicitur 1 Rhet’ quod Iudex epiikis idest superiustus debet indulgere humanis , si viderit delinquentem magis velle ire ad arbitrium , \\\hline
3.2.27 & Lo segundo que sean bien guardadas que quiere dezer tanto commo que atales leyes assi establesçidas obedes tan bien los omes & ut bene custodiantur , vel ( quod idem est ) ut legibus sic institutis bene obediatur . \\\hline
3.2.36 & por la salur que quiere dezer que amamos a aquellos que nos pueden fazer bien saluandonos e librado nos . & quod quia diligimus beneficos in salutem , id est eos qui possunt nobis benefacere nos saluando et liberando , ideo diligimus fortes \\\hline

\end{tabular}
