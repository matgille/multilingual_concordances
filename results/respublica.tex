\begin{tabular}{|p{1cm}|p{6.5cm}|p{6.5cm}|}

\hline
1.1.7 & quod tamen bonum Reipublicae non procuret : & que non procure el bien del pueblo | mas el suyo propio . \\\hline
1.2.10 & et ad rempublicam , & e al bien comun \\\hline
1.2.10 & quia aliquando aliqui plus laborantes pro Republica , & Ca alas vezes algunos trabaian | mas por la comunidat \\\hline
1.2.21 & et erga rempublicam , & e en el bien comun \\\hline
1.2.27 & vel propter amorem Reipublicae & o por amor de la comunidat \\\hline
1.2.27 & quia sine ea Respublica durare non posset . & por que sin ella la comunidat non podrie durar . | por la qual cosa si el Rey o el prinçipe o otro \\\hline
1.2.27 & et conseruatores Reipublicae . & de seer guardadores dela iustiçia | et mantenedores dela comunidat . \\\hline
1.3.3 & quia ciues pro Republica non dubitabant se morti exponere . & por que los çibdadanos non duda una de se poner ala muerte | por el bien comun de todos . \\\hline
1.3.3 & ad Rempublicam fecit Romam esse principantem & e publicofizo a Roma ser sennora \\\hline
2.2.17 & aliquando sustinere fortes labores pro defensione reipublicae : & conuiene les de sofrir alguas uegadas fuertes trabaios | por defendemiento dela tierra . \\\hline
2.2.17 & et congruitas quod respublica defensione indigeat , habeant corpus sic dispositum , & en que la tierra aya meester defendimiento | ayan el cuerpo o bien ordenado \\\hline
2.2.17 & ut per eos respublica possit defendi . & e pueda defender la tierra . \\\hline
2.2.18 & sed etiam necessarius pro bono Reipublicae , & Mas avn es neçessario para el bien e para el defendimiento dela comunidat \\\hline
2.2.20 & nec regiminibus reipublicae ; & nin c̃ca los gouernamientos dela comunidat | nin dela çibdat \\\hline
3.2.17 & quod apud Romanam Rempublicam exaltauit fidelitas consiliantium : & ca esto fue lo que enssalço la comunidat de Roma fieldat de buenos consseieros \\\hline
3.2.17 & quod fidum et altum erat secretum consistorium reipublicae , silentique salubritate munitum : & e muy alto era conssisto no secreto dela comunidat de roma | alos quel guardananca era guaruido de grant fialdat . \\\hline

\end{tabular}
