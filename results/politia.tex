\begin{tabular}{|p{1cm}|p{6.5cm}|p{6.5cm}|}

\hline
1.2.11 & Secunda via ad inuestigandum hoc idem sumitur ex parte ipsius regni . Regnum enim et omnis politia est quidam ordo , et quidam principatus . & para prouar esto mismo se toma de parte del regno . Ca el regno e toda comunidat es vna orden | Ca el regno e toda comunidat es vna orden e vn prinçipado . \\\hline
1.2.11 & Non ergo ulterius reseruaretur in eis politia , nec esset ulterius regnum . & nin alas leyes nin al prinçipe . Et pues que assi es non se guardaria dende adelante en ellos comunidat ni seria dende adelante regno . \\\hline
1.2.11 & et infirmat ipsum : sic non quaelibet Iniustitia corrumpit totaliter regnum , et politiam , tamen per quamlibet Iniustitiam regnum , & Bien assi cada vna mengua de iustiçia non corronpe del todo el regno e la comunidat . Enpero que por qual quier menguade iustiçia es enfermo el regno e la comunidat apareiada acorruy conn¶ \\\hline
1.2.11 & tamen per quamlibet Iniustitiam regnum , et politia infirmatur , et disponitur ad corruptionem . Patet igitur & Enpero que por qual quier menguade iustiçia es enfermo el regno e la comunidat apareiada acorruy conn¶ pues que assi es paresçe \\\hline
1.2.27 & homines fierent iniuriatores aliorum , et politia durare non posset . Nullus igitur debet irasci per odium , & e forçadores de los otros . Et la paliçia que es ordenança dela çibdat non podria durar ¶ | Et la paliçia que es ordenança dela çibdat non podria durar ¶ Pues que assi es ninguno non deue dessear uengança e pena por rancor \\\hline
2.1.3 & ut ordinantur ad communitatem , et ad politiam ciuium . Intendimus ergo ostendere de domo , & en quanto estas cosas son ordenadas ala comunidat e ala polliçia e ordenamiento de los çibdadanos . Et pues que assi es nos entendemos de determinar dela casa \\\hline
2.1.14 & et principante , sed magis ab ipsa politia et ab ipsis ciuibus . Dicitur ergo tale regimen politicum & a qual gouernamiento non se deue nonbrar del regñate nin del prinçipante . Mas deue se nonbrar dela çibdat e de los lus çibdadanos e es dicho tal gouernamiento politico e çiuil . \\\hline
2.3.13 & ut superius diffusius probabatur , numquam ex pluribus hominibus fieret naturaliter una societas vel una politia , nisi naturale esset aliquos principari & assi commo es prouado de suso mas conplidamente nunca de mucho omes se faria naturalmente vna conpannia o vna poliçia si naturalmente non fuesse a ellos conuenible \\\hline
2.3.13 & anima dominatur , et corpus obedit : sic in politia bene ordinata sapientes debent dominari , et insipientes obedire : & e el cuerpo obedesçe assi en la poliçia e enla çibdat bien ordenadlos sabios deuen enssennorear | assi en la poliçia e enla çibdat bien ordenadlos sabios deuen enssennorear e los non si bios deuen obedesçer \\\hline
2.3.13 & quam sapientes . Sed hoc accidit ex peruersitate politiae : nam sicut in homine pestilente , & que los que son sabios . Mas esto contesçe por desordenança dela poliçia e dela çibdat . Ca assi commo en el omne pestilençial e malo \\\hline
2.3.13 & corpus et sensualitas dominatur magis quam anima vel ratio : sic in politiis pestilentibus , et corruptis magis dominantur ignorantes , & es mas senoraque el alma nin la razon | mas senoraque el alma nin la razon assi enlas poliçias delas çibdades pestilençiosas e desordenadas \\\hline
2.3.19 & Nam sicut decet ciues ut debitam politiam seruent esse iustos legales , & ca assi conmo conuiene alos çibdadanos de ser iustos e legales para guardar su poliçia conueniblemente assi conuiene alos sermient s̃ de los sennores de ser curiales \\\hline
3.1.6 & et regna . Secundo ostendetur , quae sit optima politia siue optimum regnum , et quibus cautelis uti debeant principantes , & Lo segundo mostraremos qual es la muy buean politica o çibdat o muy vuen regno e de quales cautelas deuen usar los prinçipes e los . Reyes \\\hline
3.1.9 & ne circa ipsa contingat error . Quare cum in regimine ciuitatis primo sit politia ordinanda , diu inuestigandum est , & por que çerca ellos non contezca yerro . por la qual cosa commo en el gouernamiento dela çibdat primeramente se ha de ordenar la poliçia | por la qual cosa commo en el gouernamiento dela çibdat primeramente se ha de ordenar la poliçia muy luengamente es de buscar e de escodrinnar \\\hline
3.1.14 & seditiones mouent . Quia credimus Philosophos determinantes de Politiis , et de ordine ciuium , & que sean poderosos e aionsos mueuen contiendas e peleas enla çibdat sobre los ofiçios Or que creemos que los philosofos fablaron dela çibdat e de la orden de los çibdadanos \\\hline
3.1.15 & hi videlicet nobiles potissime debent defendere patriam , et eorum maxime est vacare circa armorum industriam . Volebat ergo Socrates politiam aliquam non debere nominari ciuitatem , nisi saltem contineret mille nobiles , quos per quandam excellentiam vocabat bellatores . & de entender çerca la sabiduria delas armas . Et pues que assi esquariasocrates que algua poliçia non deuiesse ser llamada çibdat | Et pues que assi esquariasocrates que algua poliçia non deuiesse ser llamada çibdat si a lo de menos non ouiesse en ella minłłomes nobles los quales llamaua \\\hline
3.1.16 & ut narrat Philosophus 2 Politicorum intromisit se de ordine ciuitatis , statuens quomodo posset optime politia ordinari . Dicebat autem , & que se entremi tio del ordenamiento dela çibdat establesçiendo en qual manera se podria ordenas muy bien la poliçia e la çibdat ca dizia \\\hline
3.1.16 & quod inter caetera quae debet intendere legislator et rector politiae , est , quomodo ciues habeant possessiones aequatas . & aque deue parar mientes el fazedor della es en qual manera los çibdadanos de una auer las possesipnes igualadas ca creya que estonçe seria la çibdat muy bien ordenada \\\hline
3.1.16 & et contumeliae . Tertio , ex his quae videbat in politiis aliis . Prima via sic patet . & La terçera por aquellas cosas que ueya en las otras poliçias . ¶ La primera razon paresçe \\\hline
3.1.16 & ex his quae videbat in ciuitatibus aliis : nam in multis politiis bene ordinatis magna cura fuit legislatoribus & que el veya enlas otras çibdades ca veya en muchͣs çibdades bien ordenadas que auyan grant cura los fazedores dela ley delas possessiones delos çibdadanos . \\\hline
3.1.16 & ideo Phalas forte motus ex his quae videbat in politiis aliis , statuit potissime curandum esse de possessionibus ciuium , & Et por ende felleas por auentura fue mouido por estas cosas que veya en las otrå sçibdades . | por estas cosas que veya en las otrå sçibdades . Et por ende establesçio \\\hline
3.1.19 & Quarto de distinctione iudicantium . Quinto de modo iudicandi . Sexto et ultimo statuit quasdam leges tangentes diuersa genera personarum . Hippodamus autem statuens suam politiam , primo intromisit se de multitudine & ¶Lo quarto del departimiento delos que iudgan . Lo quinto dela manera de iudgar ¶ Lo sesto e lo postrimo establesçio algunas leyes que tannian alguons linages de personas dezimos | Lo quinto dela manera de iudgar ¶ Lo sesto e lo postrimo establesçio algunas leyes que tannian alguons linages de personas dezimos que y podo mio establesciendo su poliçia \\\hline
3.1.20 & ut recte iudicet . Hippodami ergo opinionem recitauimus , quia in sua politia multas bonas sententias promulgauit : aliqua tamen incongrue statuit . Possumus autem & por que iudgue derechamente e por ende contamos la opinion de ipodomio por que el en la su poliçia manifesto muchͣs bueanssmans . Empo algunas cosas establesçio non conuenible mente . | por que el en la su poliçia manifesto muchͣs bueanssmans . Empo algunas cosas establesçio non conuenible mente . Et por ende podemos \\\hline
3.1.20 & hoc posito artifices , et agricolae non haberent partem in politia ; et bellatores non permitterent eos participare in electione principis . & Puesto que el dizie siguese que los menestrales e los labradores non los dexanien auer parte en la elecçion del prinçipe . Et por ende establesçer \\\hline
3.1.20 & citius degenerabunt hoc modo quam aliter . Unde et Philosophus innuit , quod in quibusdam bonis politiis econtrario statuitur quam ordinauerit Hippodamus : ubi ordinantur iudices posse loqui sibi inuicem publice , & Onde el phoda a entender que en algunos buenos ordenamientos de çibdat el contra no es establesçido delo que ordeno ypodomio | que en algunos buenos ordenamientos de çibdat el contra no es establesçido delo que ordeno ypodomio enlos quales ordenamientos es ordenado \\\hline
3.2.1 & praemittendo quaedam praeambula ad propositum , et recitando opinionem diuersorum Philosophorum instituentium politiam , et tradentium artem de regimine ciuitatis et regni : & ante pomiendo alguons preanbulos al nuestro proponimiento e rezando opiniones de departidos philosofos que establesçieron poliçias | e rezando opiniones de departidos philosofos que establesçieron poliçias e dieron arte del gouernamiento dela çibdar \\\hline
3.2.2 & Nam regnum aristocratia , et politia sunt principatus boni : tyrannides , & ca el regno e la aristo carçia que quiere dezer sennorio de buenos e la poliçia que quiere dezer pueblo bien enssenoreante son bueons prinçipados . | que quiere dezer sennorio de buenos e la poliçia que quiere dezer pueblo bien enssenoreante son bueons prinçipados . La thirama que quiere dezer sennorio malo \\\hline
3.2.2 & et quia talis principatus non habet nomen proprium , vocat eum Philosophus nomine communi , et dicit ipsum esse Politiam . Politia enim quasi idem est , quod ordinatio ciuitatis & llamalle el philosofo nonbre comun e diz el poliçia por que poliçia es | e diz el poliçia por que poliçia es assi commo ordenamiento bueno de çibdat \\\hline
3.2.2 & quantum ad maximum principatum qui dominantur omnibus aliis . Politia enim consistit maxime in ordine summi principatus , qui est in ciuitate . & e prinçipalmente quanto al grant prinçipado que enssennorea a todos los otros ca la poliçia esta mayormente en el ordenamiento del grant prinçipado que es en la çibdat \\\hline
3.2.2 & Omnis ergo ordinatio , ciuitatis Politia dici potest . Principatus tamen populi & que es en la çibdat Pues que assi es todo ordenamiento de çibdat puede ser dicħ poliçia . Enpero el prinçipado del pueblo \\\hline
3.2.2 & si rectus sit , eo quod non habeat commune nomen , Politia dicitur . Nos autem talem principatum appellare possumus gubernationem populi , si rectus sit . & si derecho es por que non ha nonbre comun es dich poliçia e nos podemos llamar atal prinçipado gouernamiento del pueblo | por que non ha nonbre comun es dich poliçia e nos podemos llamar atal prinçipado gouernamiento del pueblo si derecho es . \\\hline
3.2.3 & et qui peruersi . Quia intendimus ostendere quae sit optima politia , et quis sit optimus principatus . & e quales malos ir que entendemos mostrar qual es la muy buena poliçia | ir que entendemos mostrar qual es la muy buena poliçia e que cosa es el muy buen prinçipado despues \\\hline
3.2.3 & quae est in pluribus principantibus , congregaretur in uno Principe , efficacior esset ; et ille principans propter abundantiorem potentiam melius posset politiam gubernare . Tertia via sumitur ex his quae videmus in natura . & Et aquel prinçipe por ma . yor cunplimiento de poderio meior podria gouernar la çibdat que muchos ¶ | por ma . yor cunplimiento de poderio meior podria gouernar la çibdat que muchos ¶ La terçera razon se toma de aquellas cosas \\\hline
3.2.5 & ( ut ait Philosophus in Politiis ) decet iuniores senioribus obedire . Immo & que alos otros segunt que dize el pho en las politicas conuiene | segunt que dize el pho en las politicas conuiene que los mas mançebos obedescan alos mas uieios e avn \\\hline
3.2.7 & ubi ait , quod sicut Regnum est optima et dignissima politia , sic tyrannis est pessima : & Et esta razon tanne el philosofo çerca el comienço del quarto libro delas politicas . do dize que assi conmo el regno es muy buena et muy digna poliçia . | do dize que assi conmo el regno es muy buena et muy digna poliçia . assi la tirania es muy mala \\\hline
3.2.7 & ut ibi dicitur ) quia tyrannis plurimum distat a politia , idest a communi bono . Secunda via ad inuestigandum hoc idem , sumitur ex eo quod tale dominium maxime est naturale . & y dize el pho por quela tirama much se arriedra dela poliçia e del bien comun . la segunda manera \\\hline
3.2.14 & praeparatur via ut corrumpatur principatus ille . Politia ergo quanto de se magis a iustitia recedit , tanto ex se habet & pues que assi es quanto el gouernamiento mas se arriedra dela iustiçia | quanto el gouernamiento mas se arriedra dela iustiçia tanto ha dessi que sea mas ayna corronpida \\\hline
3.2.15 & Tangit autem Philosophus 5 Polit’ decem quae politiam saluant , et quae oportet facere Regem & anne el pho en el quànto libro delas politicas diez cosas que saluna la poliçia e el gouernamiento del regno | que saluna la poliçia e el gouernamiento del regno e dela çibdat \\\hline
3.2.15 & etiam modicae . Secundum praeseruans politiam et regnum regium , est bene uti iis & La segunda cosa que guarda la poliçia e el gouernamiento del regno \\\hline
3.2.15 & et non iniuriando eis . Nam ut innuit Philosophus in Poli’ bene uti ciuibus non solum praeseruat politiam rectam , sed etiam principatus ex hoc durabilior redditur , & Ca assi commo dize el philosofo en el primero libro delas pol . bien vsar de los çibdadanos non solamente guarda la poliçia e el gouernamiento derecho . | en el primero libro delas pol . bien vsar de los çibdadanos non solamente guarda la poliçia e el gouernamiento derecho . Mas avn por esta razon el prinçipado se faze mas durable \\\hline
3.2.15 & dato quod in ipso sit aliquid obliquitatis ad mixtum . Tertium est , incutere timorem iis qui sunt in politia : nam corruptiones longe & La terçera cosa que guarda al gouernamiento del regno es meter mie do aquellos que son enla çibdat e en el regno ca las corrupconnes alongadas de fecho e allegadas \\\hline
3.2.15 & prope autem secundum timorem politiam saluant : ciues enim magis sunt subiecti Principi et plus ei obediunt , & e saluna lo poliçia e el gouernamiento dela çibdat ca los çibdadanos son mas subiectos al prinçipe | e el gouernamiento dela çibdat ca los çibdadanos son mas subiectos al prinçipe e meior le obedesçen \\\hline
3.2.15 & et antecessores sui obtinuerunt huiusmodi principatum , tanta cautela non magnam utilitatem habere videtur . Quartum autem quod politiam saluare videtur , est cauere seditiones & que dichͣes non paresçe que puede ser muy prouechosa la quarta cosa que salua la poliçia | que puede ser muy prouechosa la quarta cosa que salua la poliçia es escusar las discordias e las contiendas delos nobles \\\hline
3.2.15 & aut magistratum . Nihil enim adeo regnum conseruat et politiam saluat , sicut praeficere homines bonos & tanto el regno e la poliçia commo poner los bueons e los uirtuosos en las dignidades \\\hline
3.2.15 & et conferre eis dominia et principatus . Quare maxime saluatiuum politiae est , regiam maiestatem considerare diligenter quos praeficit in aliquibus magistratibus : & e dar les los señorios e los prinçipados . por la qual cosalo que mucho salua la poliçia | por la qual cosalo que mucho salua la poliçia es que el Rey piensse con grant acuçia \\\hline
3.2.15 & ne repente constituatur aliquis in maximo principatu . Septimum saluans regnum et politiam , est Regem siue principantem habere dilectionem & La vi jncosa que salua el regno e la poliçia es | que salua el regno e la poliçia es que el Rey e el prinçipe aya grant amor al bien del regno \\\hline
3.2.15 & et amorem ad bonum regni , et ad politiam , in qua principatur . & que el Rey e el prinçipe aya grant amor al bien del regno e al bien dela çibdat en quien enssennorea \\\hline
3.2.15 & et periculis imminentibus obuiare . Octauum saluans regnum et politiam , est habere ciuilem potentiam . & e pueda contradezir alos peligros que pueden acaesçer¶ La . viijn . cosa que salua el regno | que pueden acaesçer¶ La . viijn . cosa que salua el regno e la poliçia es auer poderio çiuilca \\\hline
3.2.15 & sic eos bonitate superet : hoc enim maxime saluabit regnum et politiam , si Rex sit bonus et virtuosus , & assi lieue aun ataia en bondat e esto es lo que much salua el regno e la poliçia si el Rey fuere bueno e uirtuoso \\\hline
3.2.15 & quia intendet bono regni et communi . Decimum , est Regem non ignorare qualis sit illa politia secundum quam principatur , & La xͣ cosa que salua la poliçia es que el Rey sepa aquella poliçia e gouernamiento | que salua la poliçia es que el Rey sepa aquella poliçia e gouernamiento segunt el qual el enssennorea \\\hline
3.2.19 & quod esset bonus virtuosus et politiam diligeret , existentes in regno promoueret et honoraret : quod esse non posset , si bona eorum quae sunt in regno usurparet iniuste . Rursus est attendendum , & la qual cosa non podria ser si tomasse los bienes de aquellos que son en el su regno | si tomasse los bienes de aquellos que son en el su regno sin derech \\\hline
3.2.26 & et moribus illius gentis . Ideo dicitur 4 Politicorum quod non oportet adaptare politias legibus , sed leges politiae , quas leges oportet diuersas esse secundum diuersitatem politiarum . & que non conuiene de apropar las comunidades delas çibdades alas leyes . Mas las leyes alas comunidades de las çibdades las quales leyes conuiene de ser departidas | Mas las leyes alas comunidades de las çibdades las quales leyes conuiene de ser departidas segunt el departimiento delas comunidades . \\\hline
3.2.26 & quod non oportet adaptare politias legibus , sed leges politiae , quas leges oportet diuersas esse secundum diuersitatem politiarum . Volens ergo leges ferre , & las quales leyes conuiene de ser departidas segunt el departimiento delas comunidades . Et pues que assi es el que quisiere poner leyes con grand acuçia \\\hline
3.2.29 & Adducit autem rationes duas , quod melius sit politiam regni Regi optima lege , quam optimo Rege . & Mas para esto prouar aduze dos razones que meior es de ser gouernado el regno por muy buena ley | que meior es de ser gouernado el regno por muy buena ley que por muy buen Rey ¶ \\\hline
3.2.33 & si ibi sit populus ex multis personis mediis constitutus . Tangit autem Philosophus 4 Politicorum , quatuor , ex quibus sumi possunt quatuor viae , ostendentes meliorem esse politiam , vel melius esse regnum et ciuitatem , & Et pone el pho en el quarto libro delas politicas quatro cosas delas quales se pueden tomar quatro razonnes que muestran que meior es la poliçia | delas quales se pueden tomar quatro razonnes que muestran que meior es la poliçia o meior es el regno o la çibdat \\\hline
3.2.33 & et Charondas , et Licurgus tradiderunt de Politiis . Dixerunt enim eas constituendas esse ex personis mediis . Decet ergo Reges et Principes adhibere cautelas , & non lo que dixieron sałon e carendas e liguago los quales phos dixieron que las çibdades deuien ser establesçidas de perssonas medianeras . | e liguago los quales phos dixieron que las çibdades deuien ser establesçidas de perssonas medianeras . Et pues que assi es conuiene \\\hline
3.2.34 & intentio legislatoris est inducere ciues ad virtutem . In recta enim Politia ( ut vult Philosophus ) & es enduzer los çibdadanos o uirtud . Ca en la derecha poliçia assi commo dize el philosofo çerca el comienço del quarto libro delas polticas \\\hline
3.2.35 & et obseruantiam legum utilium : et ad obseruandum ea quae requirit politia , vel regimen regni , & e aguardar aquellas cosas que demanda la poliçia o el gouernamiento del regno en que estan \\\hline
3.2.36 & qui ultra modum regnum et politiam per turbant , inexquisitas crudelitates exerceant . & Et la manera en que la dan . Et pueᷤ que assi es por la pena que dan son temidos los Reyes e los prinçipes si mostraren grandescrueldades \\\hline

\end{tabular}
