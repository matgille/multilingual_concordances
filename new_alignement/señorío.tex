\begin{tabular}{|p{1cm}|p{6.5cm}|p{6.5cm}|}

\hline
1.1.10 & tomadaquello que tal prinçipado \textbf{ e tal sennorio non dura mucho ¶ } La segunda daquello que tal prinçipado & ex eo quod talis principatus \textbf{ non multum durat . } Secunda , ex eo quod esse potest \\\hline
1.1.10 & La segunda daquello que tal prinçipado \textbf{ e tal sennorio puede seer } sin bondat deuida ¶ & Secunda , ex eo quod esse potest \textbf{ sine bonitate vitae . } Tertia vero , ex quod est indignus . \\\hline
1.1.10 & e non alos mayores \textbf{ ¶La quinta daquello por que tal sennorio } en la mayor parte faze gerat danno¶ & ad minora bona . \textbf{ Quinta vero , ex eo quod tale dominium } ut plurimum infert nocumentum . \\\hline
1.1.10 & et la bien andança non es de poner en este poderio çiuil . \textbf{ Por que este sennorio non es muy bueno nin muy digno . } Ca si la feliçidat & non est ponenda felicitas , \textbf{ quia huiusmodi Principatus non est optimus , | nec est dignus . } Si enim felicitas \\\hline
1.1.10 & Ca si la feliçidat \textbf{ e la bien andança deue ser puesta en algun sennorio deue seer puesta en sennorio . } muy bueno e muy digno . & Si enim felicitas \textbf{ in aliquo Principatu poni debet , } ponenda est in Principatu optimo , et digno . \\\hline
1.1.10 & e non alos libres que han libertad . \textbf{ Ca tal señorio es sennorio } por costrinimiento e por fuerça¶ & non liberis : \textbf{ nam talis Principatus est } per coactionem , et violentiam . \\\hline
1.1.10 & pues que assi es commo los prinçipados \textbf{ e los sennorios se estiendan } segunt aquellos a quien & per coactionem , et violentiam . \textbf{ Cum ergo Principatus se extendant adinuicem , } secundum eos quibus aliquis principatus , \\\hline
1.1.10 & poner la su bien andança en el poderio çiuiles \textbf{ por que este sennorio faze grant danno en las mas cosas . } Ca commo la feliçidat & quia huiusmodi principatus infert \textbf{ ut plurimum nocumentum . } Nam cum felicitas sit finis omnium operatorum , \\\hline
1.2.7 & que el su prinçipadgo \textbf{ e el su sennorio non se torne en tirania } que es señorio malo e desigual & Secundo studere debet , \textbf{ ne suus principatus in tyrannidem conuertatur . } Tertio studere debet , \\\hline
1.2.7 & pues que assi es do quier que ay mengua de sabiduria ay naturalmente seruidunbre . \textbf{ Et do quier que ay sabiduria ha naturalmente sennorio . } Por ende sienpre el prinçipe deue auer sabiduria . & Ubicunque igitur hoc naturaliter seruit , \textbf{ et illud natureliter dominatur , } semper principans pollet prudentia , a qua deficit \\\hline
1.2.12 & fasta que son puestos en alguna dignidat \textbf{ o en algun prinçipado de sennorio . Ca quando algun omne non ha de gouernar } si non assi mismo non paresçe bien & nisi constituantur in aliquo principatu . \textbf{ Nam quandiu aliquis non habet regere | nisi seipsum , } non plene apparet qualis sit , \\\hline
1.2.12 & quales nin es conostida la su bondat acabadamente ¶ \textbf{ Mas quando es puesto en algun prinçipado o en algun sennorio } por que la su bondat se ha de estender a otros & nec perfecte cognoscitur bonitas eius . \textbf{ Sed quando constituitur | in principatu aliquorum , } quia oportet , \\\hline
1.3.3 & Et en essa misma manera avn si cataremos al tp̃o \textbf{ quando alguna çibdat auie señorio e tenie sennorio sobre las otras esto era } por que los çibdadanos non duda una de se poner ala muerte & Sic etiam antiquitus \textbf{ si perspeximus ciuitatem aliquam dominari et tenere monarchiam : } hoc erat , quia ciues pro Republica non dubitabant \\\hline
1.3.3 & e publicofizo a Roma ser sennora \textbf{ e auer sennorio en todo el mundo . } Pues que assi es que esto conuiene a todos los omes de ser amadores & Dilectatio enim quam habebant Romani \textbf{ ad Rempublicam fecit Romam esse principantem et monarcham . } Hoc ergo modo quoslibet homines decet esse amatiuos , \\\hline
1.4.2 & e generalmente a todos los omes \textbf{ que ha sennorio } por que non fagan assi mismos seer despreçiados . & et uniuersaliter ab omnibus dominantibus , \textbf{ est mendacium fugiendum , } ne seipsos contemptibiles reddant : \\\hline
1.4.6 & por que el omne es digno de ser señor . \textbf{ Ca creen que las riquezas son dignidat de sennorio e de prinçipado e semeiales a ellos que las riquezas lon tan grand bien } que qualquier omne & id quo quis efficitur dignus principari : \textbf{ credunt enim , | quod dignitas principatus } sint diuitiae . \\\hline
1.4.6 & que son dignos de ser sennores . \textbf{ Ca la dignidat del prinçipado e del sennorio } prinçipalmente esta en uirtudes e en sabiduria & se esse dignos principari . \textbf{ Nam dignitas principatus principaliter } innititur virtutibus et prudentiae , \\\hline
1.4.7 & por que es en algun prinçipado \textbf{ e ha muchos so su sennorio . } por la qual cosa commo nos veamos muchos ser nobles & quia est in aliquo principatu , \textbf{ et habet multos sub suo dominio ; } quare cum multos nobiles videamus esse impotentes , \\\hline
1.4.7 & los quales fallesçen en poderio çiuil \textbf{ ca ninguons non se gouiernan so su sennorio . } Por la qual cosa non es vna cosa ser el omne rico e ser toderoso¶ & in ciuili potentia : \textbf{ quia nulli reguntur sub eius imperio ; } quare non est idem esse diuitem , \\\hline
1.4.7 & que si los poderosos fazen tuerto a \textbf{ alguons non ge lo fazen en pequanas cosas mas en grandes . Ca los poderosos estando en gerad sennorio } por que son en logar digno de grand honrra & si potentes iniuriantur , \textbf{ non iniuriantur in paruis , | sed in magnis . } Potentes enim existentes in Principatu , \\\hline
2.1.14 & Et por gouernamiento real . \textbf{ Mas alguon es dicho ser adelantado en sennorio real } quando es adelantado segunt aluedrio & politico scilicet et regali . \textbf{ Dicitur autem quis praeesse regali dominio , } cum praeest secundum arbitrium et secundum leges , \\\hline
2.1.14 & e sinplemente en toda manera que el marido ala muger . \textbf{ Otrossi el sennorio del padre mas es segunt natura } que el mater moinal . & quam uxori . \textbf{ Rursus , dominium paternale | magis est secundum naturam , } quam coniugale : \\\hline
2.1.14 & nin escogen el padre para si mas naturalmente son egendrades del padre . \textbf{ Et por ende el sennorio del padre es dicho mas segina natura } que el sennorio matrimoinal . & sed naturaliter producuntur ab ipso . \textbf{ Dicitur dominium paternale esse | plus secundum naturam , } quam coniugale : \\\hline
2.1.14 & Et por ende el sennorio del padre es dicho mas segina natura \textbf{ que el sennorio matrimoinal . } Ca commo quier que el omne sean atal mente & plus secundum naturam , \textbf{ quam coniugale : } quia licet sit \\\hline
2.2.1 & Conuiene a cada vn sennor \textbf{ que ha sennorio sobre otro de ser muy cuydadoso en qual manera } por ayudas conueinbles enbie su uirtud & decet quemlibet dominantem , \textbf{ et praeeminentem solicitari , } quomodo per debita auxilia influat , \\\hline
2.2.2 & Enpero por que cada vno dellos es establesçido en algun prinçipado \textbf{ e en algun sennorio } en quel conuiene gouernar los otros & in aliquo principatu \textbf{ et in aliquo dominio , } in quo oportet eos alios gubernare ; \\\hline
2.2.7 & e de los prinçipes \textbf{ quando son puestos en algun sennorio non tiraniçen } nin sean tirannos & et Principum \textbf{ cum ponuntur in aliquo dominio tyrannizent , } decet ipsos etiam ab ipsa infantia insudare literis , \\\hline
2.2.8 & por los omes \textbf{ entre las quales la methaphisica tiene sennorio . } ¶ Otrossi la theologia & de scientiis humanitus inuentis , \textbf{ inter quas Metaphysica primatum tenet . } Amplias Theologia , \\\hline
2.2.8 & los quales assi conmodicho es \textbf{ de suso tienen sennorio } entre todos los sabios & quia ( ut superius dicebatur ) \textbf{ inter scientias humanitus inuentas } metaphysica primatum tenet . \\\hline
2.3.5 & Et pues que assi es natal cosa es a nos de auer las cosas de fuera \textbf{ e por ende el sennorio delas cosas de fuera es en algua manera natural al omne . } por que la natan engendro & habere res exteriores . \textbf{ Habere ergo dominium rerum exteriorum est quodammodo homini naturale : } quia natura produxit \\\hline
2.3.5 & enl primero libro delas politicas de auer possession \textbf{ e sennorio de algers cosas de fuera } para conplimiento dela uida . & ut vult Philosophus primo Polit’ habere possessionem , \textbf{ et dominium aliquarum rerum exteriorum } propter sufficientiam vitae . \\\hline
2.3.7 & que la natura dio anos tales cosas \textbf{ e las ordeno a vso e añro sennorio . } ¶ Et pues que assi es cosa conuenible es & quod natura dedit nobis talia , \textbf{ ordinauit enim ea ad usum et dominium nostrum ; } licitum est ergo sumere nutrimentum ex agris , \\\hline
2.3.14 & e bienes de fuera \textbf{ non fazen sennorio natural sinplemente . } Mas mas fazen sennorio legal & quae sunt bona corporalia et exteriora , \textbf{ non faciunt dominium simpliciter naturale , } sed magis faciunt ipsum legale et positiuum . \\\hline
2.3.14 & non fazen sennorio natural sinplemente . \textbf{ Mas mas fazen sennorio legal } e positiuo por establesçimiento de los omes . & non faciunt dominium simpliciter naturale , \textbf{ sed magis faciunt ipsum legale et positiuum . } Quod enim superans in bonis corporis , \\\hline
2.3.15 & sienpre la orden natural . \textbf{ Et por ende en la mayor parte los prinçipados e los sennorios son malos } ca los nesçios e los malos prèuados de uso de razon e de entendimiento & et non semper reseruamus ordinem naturalem , \textbf{ ut plurimum principatus sunt peruersi : } nam ignoratis priuatim bonis animae , \\\hline
3.1.11 & ca dize que las possessiones e las cosas de los çibdadanos deuen ser propreas \textbf{ e comunes propraas quanto al sennorio . } e comunes por uirtud de liberalidat & et res civium debere esse proprias , et communes . \textbf{ Proprias quidem quantum ad dominum , communes vero } propter virtutem liberalitatis . \\\hline
3.1.11 & de auer las cosas \textbf{ e las possessiones prop̃as quanto al sennorio . } ca cada vn sennor de sus bienes propreos aura mayor acuçia de aquellos bienes & expedit cuilibet habere res \textbf{ et possessiones proprias quantum ad dominum : } nam quilibet dominans bonis propriis \\\hline
3.1.11 & tanta era la franqueza \textbf{ que maguera que omes se sus bienes propreos quanto al sennorio . } Enpero por la franqueza eran comunes aquellos çibdadanos & tanta erat liberalitas , \textbf{ quia licet quilibet haberet propria bona quantum ad dominium , } tamen propter liberalitatem quantum ad usum erant illis ciuibus communes serui , et equi , et canes : \\\hline
3.1.13 & en quanto le toma \textbf{ por cada vn sennorio } o por cada vn maestradgo & prout sumitur \textbf{ pro quolibet dominio , } vel pro quolibet magistratu , \\\hline
3.1.18 & por las possessiones \textbf{ e por el sennorio delas cosas de fuera } mas assi cuydado era engannado & propter possessionem \textbf{ et dominium exteriorum rerum : } et sic opinando fallebatur , \\\hline
3.2.2 & ca el regno e la aristo carçia \textbf{ que quiere dezer sennorio de buenos } e la poliçia & et tres sunt mali . \textbf{ Nam regnum aristocratia , } et politia sunt principatus boni : \\\hline
3.2.2 & enssenoreante son bueons prinçipados . \textbf{ La thirama que quiere dezer sennorio malo } e la obligaçia que quiere dezer sennorio duro . & et politia sunt principatus boni : \textbf{ tyrannides , oligarchia , et democratia sunt mali . } Docet enim idem ibidem \\\hline
3.2.2 & La thirama que quiere dezer sennorio malo \textbf{ e la obligaçia que quiere dezer sennorio duro . } Et la democraçia & tyrannides , oligarchia , et democratia sunt mali . \textbf{ Docet enim idem ibidem } discernere \\\hline
3.2.2 & segunt su estado \textbf{ assi es sennorio ygual e derech̃ . } mas si es entendido & secundum suum statum , \textbf{ sic est aequale et rectum . } Sed si intenditur ibi bonum aliquorum \\\hline
3.2.2 & e bien de los subdictos \textbf{ e estonçe es dicho tal sennorio monarch̃ia o e egno } ca al Rey parte nesçe de enteder el bien comun . & Si intendit bonum commune et subditorum , \textbf{ tunc dicitur Monarchia siue Regnum : } regis autem est intendere commune bonum . \\\hline
3.2.2 & Et pues que assi es dos prinçipados \textbf{ se leuna tan del sennorio de po cos vno derecho } assi commo quando enslennorean alguons & Consurgit igitur duplex principatus \textbf{ ex dominio paucorum : | unus rectus , } ut cum dominantur aliqui , \\\hline
3.2.4 & que fue la entençion del philosofo \textbf{ que el ssennorio de muchos es mas de alabar } que el ssennorio de vno & simpliciter fuisse de intentione Philosophi , \textbf{ dominium plurium esse comendabilius dominio unius , } dum tamen utrunque sit rectum , \\\hline
3.2.4 & que el ssennorio de muchos es mas de alabar \textbf{ que el ssennorio de vno } Puesto que amos los ssennorios sean derechs & simpliciter fuisse de intentione Philosophi , \textbf{ dominium plurium esse comendabilius dominio unius , } dum tamen utrunque sit rectum , \\\hline
3.2.4 & mas tal por la qual cosa \textbf{ si el sennorio de muchos } en tanto es derecho e digno & propter quod unumquodque et illud magis . \textbf{ Quare si dominari plures } in tantum est rectum et dignum , \\\hline
3.2.4 & en quanto ellos tienen logar de vno \textbf{ el sennorio de vno es meior } e fazer tal monarchia de vno & inquantum tenent locum unius : \textbf{ dominari unum et facere monarchiam , } si debito modo fiat , \\\hline
3.2.4 & e segut derech \textbf{ sennorio meior es } que sea vn sennor que muchos . & regnum esse dignissimum principatum , \textbf{ et secundum rectum dominium melius est dominari unum , } quam plures . \\\hline
3.2.4 & Por ende si fuere la monarchia \textbf{ e el sennorio do fuere sen nor vn prinçipe } e vn Rey en todo vn prinçipado & ø \\\hline
3.2.4 & e mas llanamente la tirania \textbf{ e el tal sennorio es prinçipado muy malo . } ora uentaçea alguon que seria meior en toda manera & et ut infra planius ostendetur ) \textbf{ tyrannis est pessimus principatus . } Videretur forte alicui \\\hline
3.2.5 & de si nin auentura \textbf{ mas es fecho por arte e por sabiduria por el que meior e el mas sabio sera puesto en el sennorio } mas si esto fuere por heredat es pone se el regno & videtur tale regnum non esse expositum casui et fortunae , \textbf{ sed factum esse per artem , | eo quod praeficietur melior et industrior . } Sed si per haereditatem hoc fiat , \\\hline
3.2.5 & e tuelle las thiranas \textbf{ e faze el sennorio } assi commo natural . & tollit tyrannidem , \textbf{ efficit quasi dominium naturale . } Litigia enim sedat , \\\hline
3.2.5 & por electon mas ligerament ethiranizan Avn faze el hedamiento \textbf{ que el ssennorio sea natural . } ca el pueblo inclinase naturalmente & tales facilius tyrannizant . \textbf{ Facit etiam hoc dominium naturale , } quia populus quasi naturaliter inclinatur \\\hline
3.2.5 & non hagniueza ninguna \textbf{ do el sennorio viene por hedamiento } ca si la dignidat Real passa alos fijos & difficultatem non habet : \textbf{ nam si dignitas regia } per haereditatem transferatur ad posteros , \\\hline
3.2.6 & ca dize que antiel Rey deue sobrepiuar \textbf{ e auerguamente los Reyes eran establesçidos enlos sennorios } por tres aun ataias & quod antiquitus Reges \textbf{ a triplici excessu constituebantur . } Primo ab excessu beneficii . \\\hline
3.2.6 & por que el regno es prinçipado derech . \textbf{ mas la tirania es sennorio tuerto e malo . Et pues que assi es commo el bien comun delas gentes sea mas diuinal que el bien de vno . } malamente e desigualmente & Nam regnum est principatus rectus , \textbf{ tyrannis vero est dominium peruersum . | Cum ergo bonum gentis sit } diuinius bono unius , \\\hline
3.2.7 & ¶La primera se toma \textbf{ por razon que tal sennorio mucho se arriedra dela entençion del bien comun . } La segunda se toma & tyrannidem esse pessimum principatum . \textbf{ Prima sumitur ex eo quod tale dominium maxime recedit | ab intentione communis boni . } Secunda , ex eo quod est maxime innaturale . \\\hline
3.2.7 & La segunda se toma \textbf{ por razon que tal sennorio es } muchodes natural ¶ & ab intentione communis boni . \textbf{ Secunda , ex eo quod est maxime innaturale . } Tertia , ex eo quod est efficacissimum ad nocendum . \\\hline
3.2.7 & la segunda manera para prouar esto mismo se toma . \textbf{ por razon que tal sennorio es muy desnatural } por que aquella es obra natural & idest a communi bono . \textbf{ Secunda via ad inuestigandum hoc idem , sumitur ex eo quod tale dominium maxime est naturale . } Nam illa est naturalis operatio erga aliquid , \\\hline
3.2.7 & mas deue ser dicho desnatural . \textbf{ por que quanto el sennorio de alguon ses mas contra uoluntad de los omes } tanto mas deue ser dich des natural . & et libere obedit : \textbf{ quare quanto dominium aliquorum | magis est inuoluntarium , } magis debet dici in naturale : \\\hline
3.2.7 & La quarta razon se toma \textbf{ por que por tal sennorio son enbargados grandes bienes de los çibdadanos . } ca el tirano non solamente procura los males & Quarta via sumitur \textbf{ ex eo quod per tale dominium impediuntur | maxima bona ipsorum ciuium . } Tyrannus enim non solum procurat mala \\\hline
3.2.7 & por las razones sobredichͣs . \textbf{ Mas en commo los Reyes en toda manera de una esquiuar de non enssennorear con sennorio de tirania } esto non es guaue de mostrar & propter rationes tactas . \textbf{ Quod autem reges summo opere cauere debeant , | ne principentur principatu tyrannico , } videre non est difficile . \\\hline
3.2.7 & ca qtanto peor es el prinçipe \textbf{ quanto peor es el su sennorio } por la qual cosa el Rey deue poner muy grant acuçia & Nam tanto peior est Princeps , \textbf{ quanto peiori dominio principatur : | quare } si omnem diligentiam adhibere debet Rex ne sit pessimus , \\\hline
3.2.9 & si tal muchedunbre de poderio çiuil fuere ganada \textbf{ tomando sennorio ageno } e por fuerça e sin iustiçia . & Quod maxime verum est \textbf{ si huiusmodi multitudo ciuilis potentiae acquisita sit } per usurpationem et iniustitiam . \\\hline
3.2.10 & nchas cautelas tanne el philosofo en el quinto libro delas politicas delas quales quanto par tenesçe alo presente podemos tomar diez . \textbf{ por las qualose esfuerça el tiranno de se mantener en su sennorio . } La primera cautela del tirano es matar los grandes omes e los poderosos . & possumus enumerare decem \textbf{ per quas nititur | tyrannus se in suo dominio praeseruare . } Prima cautela tyrannica , \\\hline
3.2.11 & e alguna cosa delas arterias de los tiranos \textbf{ e tanto meior es el sennorio } quanto mas se allega a buen gouernamiento de Rey & et aliquid de versutiis tyrannorum : \textbf{ et tanto est melius dominium , } quanto plus accedit ad regnum , \\\hline
3.2.12 & e es llamado regno . \textbf{ Mas si ensseñoreare par su bien propio es llamado sennorio } e este sennorio tal es llamado tyrama . & et vocatur regnum . \textbf{ Si vero propter bonum proprium , | est peruersus , } et vocatur tyrannis . \\\hline
3.2.12 & Mas si ensseñoreare par su bien propio es llamado sennorio \textbf{ e este sennorio tal es llamado tyrama . } Mas si enssennorearen pocos & est peruersus , \textbf{ et vocatur tyrannis . } Si vero principentur pauci , \\\hline
3.2.12 & Mas si enssennorearen pocos \textbf{ por que son buenos e uirtuosos este sennorio es derech } e es llamado anstrocraçia & Si vero principentur pauci , \textbf{ et dominentur quia boni et virtuosi , | est rectus principatus , } et vocatur aristocratia \\\hline
3.2.12 & e entienda de abaxar los ricos \textbf{ es sennorio corrupto } e tal sennorio es llamado de mocraçia & et intendat opprimere diuites , \textbf{ est principatus corruptus , } et vocatur Democratia , \\\hline
3.2.12 & es sennorio corrupto \textbf{ e tal sennorio es llamado de mocraçia } que tanto quiere dezer commo corrupçion e maldat del pueblo . Et pues que assi es la tirania & est principatus corruptus , \textbf{ et vocatur Democratia , } quod idem est \\\hline
3.2.12 & que tanto quiere dezer commo corrupçion e maldat del pueblo . Et pues que assi es la tirania \textbf{ e el sennorio corrupto de los ricos . } Et el sennorio malo del pueblo son tres señorios muy malos . & quod quasi peruersio et corruptio populi . \textbf{ Tyrannis vero corruptus principatus diuitum , } et iniquum dominium populi , \\\hline
3.2.12 & e el sennorio corrupto de los ricos . \textbf{ Et el sennorio malo del pueblo son tres señorios muy malos . } Enpero la tyrania es peor sennorio & Tyrannis vero corruptus principatus diuitum , \textbf{ et iniquum dominium populi , | sunt regimina peruersa . } Tyrannis tamen est peruersior principatus : \\\hline
3.2.12 & Et el sennorio malo del pueblo son tres señorios muy malos . \textbf{ Enpero la tyrania es peor sennorio } que ninguno de los otros . & sunt regimina peruersa . \textbf{ Tyrannis tamen est peruersior principatus : } quia ut probat Philosophus 5 Polit’ \\\hline
3.2.12 & qual quier cosa de maldat es \textbf{ en el mal sennorio de los ricos } e en el mal señorio del pueblo todo es ayuntado enla tirania & quicquid peruersitatis est \textbf{ aliquo principatu diuitum , } et in peruerso dominio populi , \\\hline
3.2.14 & por la qual cosa quando algͤse arriedra dela iustiçia apareia en ssi carrera \textbf{ por que sea corrun pido el su sennorio } pues que assi es quanto el gouernamiento & praeparatur via \textbf{ ut corrumpatur principatus ille . } Politia ergo quanto de se magis a iustitia recedit , \\\hline
3.2.14 & e los prinçipes \textbf{ si quieren durar en su sennorio } mucho deuen escusar & Reges ergo et principes \textbf{ si volunt suum durare dominium , } summe cauere debent \\\hline
3.2.14 & fuerca atenprar la tirania \textbf{ ca quanto mas poco tiranzar en tanto mas dura el su sennorio ¶ } Lo segundo la tirania se corronpe & suam tyrannidem pro viribus moderare debent , \textbf{ quia quanto remissius tyrannizabunt , | tanto durabilius principabuntur . } Secundo tyrannis corrumpitur \\\hline
3.2.14 & la tirama del pueblo corronpe e destruye la tirana del prinçipe bien \textbf{ assi avn vna tirama de sennorio corronpe a otra } por que muchͣs vezes vn prinçipe tirano se leunata contra otro prinçipe tirano & et tyrannis populi corrumpit tyrannidem monarchiam . \textbf{ Sic etiam una tyrannis monarchia corrumpit aliam : } quia multotiens unus monarcha tyrannus insurgit in alium , \\\hline
3.2.14 & non es bue no de tiranizar mas el regno \textbf{ e el sennorio bueno non es puesto atantos peligros } nin en tantas maneras se puede desatar commo la tirana . & dissolui eius principatus . Regium autem dominium \textbf{ non tot periculis exponitur , } nec tot modis habet dissolui . \\\hline
3.2.15 & a quales dio los maestradgos e las dignidades . \textbf{ Et si bien se ouieren acresçentar les los sennorios } e si mal tirargelos . & in aliquibus magistratibus : \textbf{ et si bene se habuerint , | augere eorum dominia : } si male , minuere : \\\hline
3.2.15 & Et en tanto se podrian auer mal \textbf{ que serien de tirar de los sennorios } o a ende condepnar & vel adeo male se habere possent , \textbf{ quod essent totaliter a dominio remouendi , } vel etiam capitali sententia condemnandi . \\\hline
3.2.15 & es non dara ninguer muy grant señorio \textbf{ ca los grandes sennorios } por la grant bondat dela uentura & Sextum est , nulli valde magnum dominium conferre . \textbf{ Nam magna dominia } ex nimia bonitate et fortitudine \\\hline
3.2.19 & e qual deua ser el ofiçio del rey \textbf{ e en qual manera el Rey se deua guardar en su sennorio mostrado fue conplidamente en los dichos de ssuso . } Ca ya por los dichos dessuso & et quale debeat esse regis officium , \textbf{ et quomodo Rex se debeat | in suo dominio praeseruare , } fuit in superioribus patefactum . \\\hline
3.2.19 & por que se salue el prinçipado \textbf{ e el sennorio del Rey . } Enpero entendemos en los capitulos & ø \\\hline
3.3.20 & e algunos teniendo la yra de los señores \textbf{ o temiendo el señorio } e la sanera del pueblo quieren se defender en algunan fortaleza & Vel si non vacat munitiones de nouo aedificare , \textbf{ et aliqui timentes iram dominorum , } aut Domini metuentes furorum populi , \\\hline

\end{tabular}
