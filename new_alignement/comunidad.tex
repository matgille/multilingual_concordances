\begin{tabular}{|p{1cm}|p{6.5cm}|p{6.5cm}|}

\hline
1.1.7 & e si robare el pueblo | e desfiziere la comunidat solamente | que el pueda allegar riquezas e dineros . & si opprimat viduas , et pupillos , | si depraedetur populum et Rem publicam , | dum tamen possit pecuniam congregare , \\\hline
1.2.10 & e en los partires . | Ca alas vezes algunos trabaian mas por la comunidat | e resçiben menos ante esta iustiçia & Rursus contingit inaequalitas in distributionibus , | quia aliquando aliqui plus laborantes pro Republica , | minus accipiunt : | quia aliquando aliqui plus laborantes pro Republica , | minus accipiunt : | immo haec Iustitia quodammodo deprauata est , \\\hline
1.2.11 & La segunda manera para prouar esto mismo se toma de parte del regno . | Ca el regno e toda comunidat es vna orden e vn prinçipado . | ¶ pues que assi es como la orden & ex parte ipsius regni . | Regnum enim et omnis politia est quidam ordo , | et quidam principatus . | Regnum enim et omnis politia est quidam ordo , | et quidam principatus . | Cum igitur ordo , \\\hline
1.2.11 & Et pues que assi es non se guardaria dende adelante | en ellos comunidat | ni seria dende adelante regno . & nec ad Principem . | Non ergo ulterius reseruaretur in eis politia , | nec esset ulterius regnum . \\\hline
1.2.11 & Ca cada vno de los regnos | e canda vna comunidat semeia avn cuerpo natural . | Ca assi commo veemos & Quodlibet enim regnum , | et quaelibet congregatio assimilatur cuidam corpori naturali . | Sicut enim videmus corpus animalis constare \\\hline
1.2.11 & assi cada vno dellos regnos | e cada vna delas comunidades es conpuesta de de pattidas personas ayuntadas e ordenadas a vna cosa . | Et por ende pues nos conuiene de fablar & sic quodlibet regnum , | et quaelibet congregatio constat | ex diuersis personis connexis , | et quaelibet congregatio constat | ex diuersis personis connexis , | et ordinatis ad unum aliquid . \\\hline
1.2.11 & Bien assi cada vna mengua de iustiçia | non corronpe del todo el regno e la comunidat . | Enpero que por qual quier menguade iustiçia & sic non quaelibet Iniustitia | corrumpit totaliter regnum , et politiam , | tamen per quamlibet Iniustitiam regnum , \\\hline
1.2.11 & es enfermo el regno | e la comunidat apareiada acorruy conn¶ | pues que assi es paresçe & et politia infirmatur , | et disponitur ad corruptionem . | Patet igitur quod prout ciues habent ordinem ad se inuicem , \\\hline
1.2.19 & si | ouiereconplidamente las riquezas fazer espenssas conuenibles en toda la comunidat | por que los bienes comunes lon en alguna manera diuinales & ( si adsit facultas ) | facere decentes sumptus | circa totam communitatem . | facere decentes sumptus | circa totam communitatem . | Nam ipsa bona communia \\\hline
1.2.19 & resplandesçe mas fermosa . | mente en toda la comunidat | ¶ & valde debiliter repraesentatur | in una persona singulari : sed in tota communitate pulchrius elucescit diuinum bonum . | Tertio magnificus se decenter habere debet \\\hline
1.2.20 & Mas por el que es persona publica e comuna | la qual toda la comunidat dela gente | e todo el regno es ordenado & Quia vero est persona publica , | ad quam ordinatur tota communitas , | et totum regnum , \\\hline
1.2.27 & et por zelo de iustiçia | o por amor de la comunidat | por que sin ella la comunidat non podrie durar . & et zelum iustitiae , | vel propter amorem Reipublicae | quia sine ea Respublica durare non posset . \\\hline
1.2.27 & o por amor de la comunidat | por que sin ella la comunidat non podrie durar . | por la qual cosa si el Rey o el prinçipe & vel propter amorem Reipublicae | quia sine ea Respublica durare non posset . | Quare si quis in tantum esset mitis , \\\hline
1.2.27 & quanto mas pertenesçe a ellos de seer guardadores dela iustiçia | et mantenedores dela comunidat . | Por la qual cosa si alos Reyes non conuiene de seer sannudos & esse custodes iustitiae , | et conseruatores Reipublicae . | Quare si Reges non debent esse iracundi , \\\hline
1.3.3 & sienpredeuemos ante poner el bien comun | e dela comunidat al bien propio e personal de cada vno . | Ca nos natraalmente veemos & includitur bonum priuatum , | semper bono priuato praeponendum est commune bonum . | Naturaliter enim videmus \\\hline
1.3.5 & en bien alto e grande e guaue de fazer De mas desto | quanto mayor es la comunidat | tantomas cosas le pueden auenir e contesçer . & sed etiam decet eos tendere in bonum arduum . | Amplius quanto maior est communitas , | tanto plura possunt ei contingere , \\\hline
2.1.1 & determinaremos del gouernamiento de la casa | mas commo la conpanna o la casa sea vna comunidat | e sea comunidat natraal & In hoc ergo secundo libro determinabitur de regimine domus . | Sed cum familia domus | sit communitas quaedam , | Sed cum familia domus | sit communitas quaedam , | et sit communitas naturalis : \\\hline
2.1.1 & mas commo la conpanna o la casa sea vna comunidat | e sea comunidat natraal | si queremos fablar dela casa & sit communitas quaedam , | et sit communitas naturalis : | si de domo determinare volumus , \\\hline
2.1.1 & Et pues que assi es neçessaria fue | e es la comunidat dela casa | e las otras comuindades . & ut sit animal sociale ; | necessaria ergo fuit communitas domus , | et communitates aliae , \\\hline
2.1.1 & e las otras comuindades . | assi commo son comunidades de varrio et de çibdat e de regno | para esto & et communitates aliae , | cuiusmodi sunt communitas ciuitatis , et regni , | ad hoc quod homines perfecte sibi in vita sufficiant . \\\hline
2.1.1 & siguese que beuir en conpannia | e en comunidat es en alguna manera natural alos omes | ¶ & Sed si haec sunt necessaria ad conseruandam hominis naturalem vitam , | viuere in communitate et in societate est quodammodo homini naturale . | Tertia via ad inuestigandum hoc idem , \\\hline
2.1.2 & que nos non sabiamos los terminos destas sçiençias | por que aquellas conpania o aquella comunidat | por la qual abastamos a nos en la uianda e en el uestido & dicens nos scientiarum limites ignorare : | quia societas , siue communitas illa , | per quam nobis sufficimus in victu et vestitu , \\\hline
2.1.2 & e en las otras cosas neçessarias ala uida non paresçe | que sea comunidat çiuil de casa mas paresce | que sea comunidat çiuil e de çibdat . & et in aliis necessariis ad vitam , | non videtur esse communitas domestica , | sed ciuilis : \\\hline
2.1.2 & que sea comunidat çiuil de casa mas paresce | que sea comunidat çiuil e de çibdat . | Ca segunt dize el philosofo & non videtur esse communitas domestica , | sed ciuilis : | quia secundum Philosophum 1 Politicorum , \\\hline
2.1.2 & por la qual cosa sin | determinarde la comunidat de los çibdadanos | non parte nesçe a este libro & ut in tertio libro plenius ostendetur . | Quare si determinare de communitate ciuium | non spectat ad hunc librum , \\\hline
2.1.2 & en el qual diremos del gouernamiento dela çibdat . paresçe que auemos trispassado los terminos desta arte determinando en el capitulo passado algunas cosas | que pertenesçen ala comunidat dela çibdat . | Mas si con estudio e con acuçia penssaremos & videmur transgressi fuisse limites huius artis , | determinando in praecedenti capitulo aliqua pertinentia ad communitatem ciuitatis . | Sed si diligenter aspicimus , \\\hline
2.1.2 & Mas si con estudio e con acuçia penssaremos | en qual manera la comunidat dela casa se ha a todas las | otrascomuidades & Sed si diligenter aspicimus , | quomodo communitas domestica se habet ad communitates alias : | cum quaelibet communitas \\\hline
2.1.2 & commo cada vna delas otras comundades | ençierre en ssi la comunidat dela casa | por que non puede ser çibdat nin varrio & cum quaelibet communitas | includat communitatem domesticam , | nec possit esse ciuitas , \\\hline
2.1.2 & humanal siguese | que la comunidat dela casa es neçessaria a esta uida . | Et pues que assi es en el capitulo sobredicho auemos determinado dela conpannia humanal & si communitas aliqua est necessaria in humana vita : | sequitur communitatem domus | ad huiusmodi vitam necessariam esse . | sequitur communitatem domus | ad huiusmodi vitam necessariam esse . | In praecedenti ergo capitulo determinauimus de societate humana , \\\hline
2.1.2 & por que por esta razon se muestra manifiestamente | que la comunidat dela casa es neçessaria | por que todas las comuni dades ençierran en ssi & quia per hoc manifeste ostenditur | necessariam esse communitatem domesticam : | cum omnis alia communitas communitatem illam praesupponat . \\\hline
2.1.2 & por que todas las comuni dades ençierran en ssi | e ante ponen esta comunidat dela casa ¶ Et | pues que assy es deuedes saber & necessariam esse communitatem domesticam : | cum omnis alia communitas communitatem illam praesupponat . | Aduertendum ergo quod \\\hline
2.1.2 & Comuidat dela casa | Et comunidat de uarrio . | Et comunidat de çibdat . &  \\\hline
2.1.2 & Et comunidat de uarrio . | Et comunidat de çibdat . | Et comunidat de regno . & apparebit quadruplicem esse communitatem ; | videlicet , domus , vici , ciuitatis , et regni . | Nam sicut ex pluribus personis fit domus , \\\hline
2.1.2 & Et comunidat de çibdat . | Et comunidat de regno . | Ca assi commo de muchas perssonas se faz la comunidat dela casa & apparebit quadruplicem esse communitatem ; | videlicet , domus , vici , ciuitatis , et regni . | Nam sicut ex pluribus personis fit domus , \\\hline
2.1.2 & Et comunidat de regno . | Ca assi commo de muchas perssonas se faz la comunidat dela casa | assi de muchas casas se faz la comunidat de vn uarrio & videlicet , domus , vici , ciuitatis , et regni . | Nam sicut ex pluribus personis fit domus , | sic ex multis domibus fit vicus , \\\hline
2.1.2 & Ca assi commo de muchas perssonas se faz la comunidat dela casa | assi de muchas casas se faz la comunidat de vn uarrio | e de muchos uarrios se faz comunidat de çibdat & Nam sicut ex pluribus personis fit domus , | sic ex multis domibus fit vicus , | et ex multis vicis ciuitas , \\\hline
2.1.2 & assi de muchas casas se faz la comunidat de vn uarrio | e de muchos uarrios se faz comunidat de çibdat | e de muchas çibdades se faz comunidat de vn regno & sic ex multis domibus fit vicus , | et ex multis vicis ciuitas , | et ex multis ciuitatibus regnum ; \\\hline
2.1.2 & e de muchos uarrios se faz comunidat de çibdat | e de muchas çibdades se faz comunidat de vn regno | por la qual cosa & et ex multis vicis ciuitas , | et ex multis ciuitatibus regnum ; | quare sicut singulares personae sunt partes domus , \\\hline
2.1.2 & que pueden fazer çibdat e regno . | Et por ende la comunidat dela casa | sea alas trͣs comunidades & ex consequenti constituere possunt ciuitatem , et regnum . | Hoc ergo modo communitas domus se habet | ad communitates alias : \\\hline
2.1.2 & Et por ende la comunidat dela casa | sea alas trͣs comunidades | en tal manera que todas las otras comunidades ençierren & Hoc ergo modo communitas domus se habet | ad communitates alias : | quia omnes aliae ipsam praesupponunt : \\\hline
2.1.2 & sea alas trͣs comunidades | en tal manera que todas las otras comunidades ençierren | e ante ponen en ssi la comuidat dela casa & ad communitates alias : | quia omnes aliae ipsam praesupponunt : | et ipsa est quodammodo pars omnium aliarum . \\\hline
2.1.2 & e ante ponen en ssi la comuidat dela casa | por que ella es en alguna manera parte de todas las otras comunidades . | Ca assi commo paresçe por el philosofo & quia omnes aliae ipsam praesupponunt : | et ipsa est quodammodo pars omnium aliarum . | Naturalis enim origo ciuitatis | et ipsa est quodammodo pars omnium aliarum . | Naturalis enim origo ciuitatis | ut patet per Philosophum 1 Politicorum , \\\hline
2.1.2 & o el regno | sienpre la comunidat dela casa se ha alas otras comuindades | en esta manera & et qualitercunque constituatur vicus , ciuitas , siue regnum , | semper sic se domus habet | ad communitates alias , | semper sic se domus habet | ad communitates alias , | quod est pars omnium aliarum , \\\hline
2.1.2 & e la ponen ante si¶ visto | en qual manera la comunidat dela casa se ha alas otras comuidades de ligero puede parescer | en qual manera esta comunidat & et aliquo modo eam omnes aliae praesupponunt . | Viso , quomodo communitas domus se habeat | ad communitates alias : | Viso , quomodo communitas domus se habeat | ad communitates alias : | de leui patet , \\\hline
2.1.2 & en qual manera la comunidat dela casa se ha alas otras comuidades de ligero puede parescer | en qual manera esta comunidat | es neçessaria ala uida humanal . & de leui patet , | quomodo huiusmodi communitas sit necessaria in humana vita . | Nam si omnes communitates aliae domum praesupponunt : \\\hline
2.1.2 & es neçessaria ala uida humanal . | Ca si todas las otras comunidades | antepo nen la comuidat dela & quomodo huiusmodi communitas sit necessaria in humana vita . | Nam si omnes communitates aliae domum praesupponunt : | si aliqua communitas est necessaria \\\hline
2.1.2 & e por si vale a conplimiento dela uida . | Conuiene que la comunidat dela casa sea mas neçessaria | Et pues que assi es los Reyes e los prinçipes & ad per se sufficientiam vitae , | oportet communitatem domus necessariam esse . | Reges ergo et Principes , \\\hline
2.1.3 & e dela fechura dela çibdat | en quanto estas cosas son ordenadas ala comunidat | e ala polliçia e ordenamiento de los çibdadanos . & et de fabrica ciuitatis , | ut ordinantur ad communitatem , | et ad politiam ciuium . \\\hline
2.1.3 & Et pues que assi es nos entendemos de determinar dela casa | que es comunidat delas perssonas dela casa | en qual manera ella es la comunidat primera . & Intendimus ergo ostendere de domo , | quae est communitas personarum domesticarum , | quomodo sit communitas prima . \\\hline
2.1.3 & que es comunidat delas perssonas dela casa | en qual manera ella es la comunidat primera . | Et por ende deuemos notar & quae est communitas personarum domesticarum , | quomodo sit communitas prima . | Notandum ergo , \\\hline
2.1.3 & Et pues que assi es departidas las cosas primeras en esta manera de ligero puede paresçer | en qual manera la comunidat dela casa se ha | ala comunindat dela çibdat & de leui patet , | quomodo communitas domus se habet | ad communitatem ciuitatis , \\\hline
2.1.3 & ala comunindat dela çibdat | e alas otras comunidades . | Ca en la obra dela execucion la casa es primero que el uarrio & ad communitatem ciuitatis , | et ad communitates alias . | Nam in executione et opere domus \\\hline
2.1.3 & e en la entençion | laso trisco munindades son primero que la comunidat dela | casa¶Otrossi la comunidat dela casa es primero & sed in uoluntate | et in intentione communitates illae | praecedunt communitatem domesticam . \\\hline
2.1.3 & laso trisco munindades son primero que la comunidat dela | casa¶Otrossi la comunidat dela casa es primero | que las otras comunidades & et in intentione communitates illae | praecedunt communitatem domesticam . | Rursus uia generationis \\\hline
2.1.3 & casa¶Otrossi la comunidat dela casa es primero | que las otras comunidades | en manera de generaçion & et in intentione communitates illae | praecedunt communitatem domesticam . | Rursus uia generationis \\\hline
2.1.3 & mas las otras comunindades son primero | que la comunidat delan casa | en manera de perfectiuo & Rursus uia generationis | et temporis domestica communitas praecedit communitates alias : | sed in uia perfectionis \\\hline
2.1.3 & ot de conplimiento | por que paresçe que la comunidat dela casa se ha en dos maneras | alas otras comunidades . & et complementi communitates aliae praecedunt ipsam . | Videtur communitas domus | ad communitates alias dupliciter se habere . \\\hline
2.1.3 & por que paresçe que la comunidat dela casa se ha en dos maneras | alas otras comunidades . | ¶ La primera es por que esta comunidat en conparaçion delas otras es mas menguada & Videtur communitas domus | ad communitates alias dupliciter se habere . | Primo , quia huiusmodi communitas \\\hline
2.1.3 & alas otras comunidades . | ¶ La primera es por que esta comunidat en conparaçion delas otras es mas menguada | e las otras son mas conplidas que ella & ad communitates alias dupliciter se habere . | Primo , quia huiusmodi communitas | respectu aliarum est imperfecta : | Primo , quia huiusmodi communitas | respectu aliarum est imperfecta : | omnes uero aliae sunt perfectiores ipsa ; \\\hline
2.1.3 & e las otras son mas conplidas que ella | por que todas las otras comunidades ençierran en ssi la comunidat dela casa | e ennaden alguna cosa sobre ella . & omnes uero aliae sunt perfectiores ipsa ; | cum enim omnis alia communitas | includat communitatem domus , | cum enim omnis alia communitas | includat communitatem domus , | et addat aliquid supra ipsam ; \\\hline
2.1.3 & e ennaden alguna cosa sobre ella . | Et por ende todas las otras comunidades son mas conplidas que ella . | Et pueᷤ que assi es la comuidat dela casa se ha & et addat aliquid supra ipsam ; | omnes aliae communitates sunt perfectiores ea . | Habet ergo se communitas domus \\\hline
2.1.3 & assi commo la parte al su todo ¶ | Otrossi la comunidat dela casa se ha alas otras comunidades | assi commo las çosas que son ordenadas & et sicut pars ad totum . | Rursus huiusmodi communitas | se habet ad alias , | Rursus huiusmodi communitas | se habet ad alias , | sicut quod est ad finem se habet ad ipsum finem . \\\hline
2.1.3 & e la çibdat para fazer el regno . | Et pues que assi es la comunidat del uarrio | es fin dela comunidat dela casa & ciuitas propter regnum . | Communitas ergo vici est finis communitatis domus , | communitas ciuitatis communitatis vici : \\\hline
2.1.3 & Et pues que assi es la comunidat del uarrio | es fin dela comunidat dela casa | e la comunidat dela çibdat & ciuitas propter regnum . | Communitas ergo vici est finis communitatis domus , | communitas ciuitatis communitatis vici : \\\hline
2.1.3 & es fin dela comunidat dela casa | e la comunidat dela çibdat | es fin dela comunidat del uarrio . & Communitas ergo vici est finis communitatis domus , | communitas ciuitatis communitatis vici : | sed communitas regni est finis omnium praedictorum . \\\hline
2.1.3 & e la comunidat dela çibdat | es fin dela comunidat del uarrio . | Mas la comunidat del regno es fin de todas las otras comuindades sobredichas . & Communitas ergo vici est finis communitatis domus , | communitas ciuitatis communitatis vici : | sed communitas regni est finis omnium praedictorum . \\\hline
2.1.3 & es fin dela comunidat del uarrio . | Mas la comunidat del regno es fin de todas las otras comuindades sobredichas . | Por la qual cosa conmola cosa non acabada sea primero & communitas ciuitatis communitatis vici : | sed communitas regni est finis omnium praedictorum . | Quare cum imperfectum \\\hline
2.1.3 & e ala casa dize | quela comunidat dela çibdat es la primera . | Et esto non se deue entender que es primera por generaçion & ad vicum et domum , | ait , quod prima communitas est communitas ciuitatis ; | quod non est intelligendum \\\hline
2.1.3 & por manera de perfecçion e de cunplimiento . | Otrossi commo la comunidat dela casa | non solamente se aya alas otras comuindades & et complementi . | Amplius cum communitas domus ad communitates alias | non solum se habeat \\\hline
2.1.3 & mas conplidamente en el terçero libro | Visto en qual manera la comunidat dela casa es primero en alguna manera que las otras comuni dades de ligero puede paresçer | en alguna manera esta comunidat dela cała es natural & ut in tertio libro plenius ostendetur . | Viso , quomodo communitas domus | aliquo modo est prior , | Viso , quomodo communitas domus | aliquo modo est prior , | quam communitates aliae : \\\hline
2.1.3 & Visto en qual manera la comunidat dela casa es primero en alguna manera que las otras comuni dades de ligero puede paresçer | en alguna manera esta comunidat dela cała es natural | Ca commo la natura non presupone & quam communitates aliae : | de leui videri potest , | quomodo sit huiusmodi communitas naturalis . | de leui videri potest , | quomodo sit huiusmodi communitas naturalis . | Nam cum natura non praesupponat artem , \\\hline
2.1.3 & ai al comiuncable e aconpanable | commo todas las comunidades presupongan | e ante pongan la comunidat dela casa & naturaliter animal communicatiuum et sociale , | cum omnis communitas | praesupponat communitatem domus , \\\hline
2.1.3 & commo todas las comunidades presupongan | e ante pongan la comunidat dela casa | conuiene quala comunidat dela casa o la casa sea cosa natural . & cum omnis communitas | praesupponat communitatem domus , | oportet communitatem domesticam siue domum \\\hline
2.1.3 & e ante pongan la comunidat dela casa | conuiene quala comunidat dela casa o la casa sea cosa natural . | Et por ende conuiene alos Reyes e alos prinçipes & praesupponat communitatem domus , | oportet communitatem domesticam siue domum | quid naturale esse . | oportet communitatem domesticam siue domum | quid naturale esse . | Reges ergo et Principes decet \\\hline
2.1.3 & e la çibdat | ante ponen la comunidat dela casa | assi el gouernamiento del regno & quia sicut regnum vel ciuitas praesupponunt esse domum , | sic regimen regni et ciuitatis | praesupponit \\\hline
2.1.3 & e que conoscan que cosa | e qual es la comunidat dela casa | ca es comunidat en alguna manera natural & ut sciant domum propriam gubernare , | et ut cognoscant quae et qualis est communitas domus | ut se habet ad regnum et ciuitatem , \\\hline
2.1.3 & e qual es la comunidat dela casa | ca es comunidat en alguna manera natural | Et en algunan manera esta comunidat se ha al regno & ut sciant domum propriam gubernare , | et ut cognoscant quae et qualis est communitas domus | ut se habet ad regnum et ciuitatem , \\\hline
2.1.3 & ca es comunidat en alguna manera natural | Et en algunan manera esta comunidat se ha al regno | e ala çibdat & et ut cognoscant quae et qualis est communitas domus | ut se habet ad regnum et ciuitatem , | ut est in praesenti capitulo declaratum : \\\hline
2.1.4 & e ya es declarado en alguna manera en el capitulo | sobredicho qual es la comunidat dela casa . | Ca ya mostrado es & Est autem ex praecedenti capitulo aliqualiter declaratum , | qualis sit communitas domus : | cum ostensum sit \\\hline
2.1.4 & que el omne es naturalmente ainalia domestica e de casa | e quela comunidat dela casa es en alguna manera natural . | Empero por que esto non auemos & quod homo est naturaliter animal domesticum , | et quod communitas domus est quodammodo naturalis . | Attamen quia per hoc non sufficienter habetur \\\hline
2.1.4 & conplidamente qual es esta comuidat dela | casapor ende entendemos de dezir algunas cosas dela comunidat dela casa . | Pues que assi es deuedes saber & qualis sit huiusmodi communitas , | ideo intendimus aliqua dicere de communitate domestica . | Sciendum ergo , \\\hline
2.1.4 & politicasasse declara | e difine la comunidat dela casa | diziendo &  \\\hline
2.1.4 & diziendo | que la casa es comunidat | segunt natura construyda e fecha para cada dia & Philosophum 1 Politicorum sic describere communitatem domus : | videlicet , quod domus est communitas secundum naturam , | constituta quidem in omnem diem . \\\hline
2.1.4 & sobredich̃o Et alguna cola finca adelante de declarar . | Ca que la casa sea comunidat | segunt natura e natural de suso es prouado gruesamente e figuaalmente & et aliquid restat ulterius declarandum . | Nam quod domus sic communitas secundum naturam , | superius grosse et figuraliter probabatur , \\\hline
2.1.4 & Pues que assi es finça de declarar en la difiniçion sobredichͣ | en qual manera la casa sea comunidat establesçida para cada dia . | Et para auer esto deuedes saber & in descriptione praedicta , | quomodo domus sit communitas constituta in omnem diem . | Ad cuius euidentiam aduertendum , \\\hline
2.1.4 & han mester cada dia conprar e vender . | pues que assi es la comunidat dela casa fue fecha para aquellas cosas | que auemos mester de cada dia . & vel venditione continue egeant . | Communitas ergo domus facta fuit propter ea , | quibus quotidie indigemus . \\\hline
2.1.4 & todas las cosas neçessarias para la uida | non cunplie la comunidat de vna casa | mas conuiene de dar comunidat de varrio . & Verum quia in una domo non reperiuntur omnia necessaria ad vitam , | non sufficiebat communitas domestica , | sed oportuit dare communitatem vici , \\\hline
2.1.4 & non cunplie la comunidat de vna casa | mas conuiene de dar comunidat de varrio . | Por que commo el uarrio sea fech̃ de muchas casas & non sufficiebat communitas domestica , | sed oportuit dare communitatem vici , | ita quod cum vicus constet \\\hline
2.1.4 & en el primero libro delas politicas | que assi commo la comunidat dela casa es establesçida | para las obras de cada dia & Propter quod Philosophus 1 Politicorum ait , | quod sicut communitas domus | constituta est | quod sicut communitas domus | constituta est | in omnem diem , \\\hline
2.1.4 & todas las cosas neçessarias ala uida | conuiene de dar comunidat ala çibdat | sobre la comunidat deluarrio . & omnia necessaria ad vitam , | praeter communitatem vici | oportuit | praeter communitatem vici | oportuit | dare communitatem ciuitatis . \\\hline
2.1.4 & conuiene de dar comunidat ala çibdat | sobre la comunidat deluarrio . | Et por ende & oportuit | dare communitatem ciuitatis . | Communitas ergo ciuitatis esse videtur \\\hline
2.1.4 & ordende todas estas cosas | que la casa es comunidat | segunt natura establesçida para cada dia . & Erit ergo hic ordo , | quod domus est communitas | secundum naturam constituta in omnem diem . \\\hline
2.1.4 & segunt natura establesçida para cada dia . | Mas el uatrio es comunidat estableçida | para las obras & secundum naturam constituta in omnem diem . | Vicus autem est communitas constituta | in opera non diurnalia . \\\hline
2.1.4 & que non son mester de cada dia . | Et la çibdat es comunidat establesçida | para conplimiento delas cosas & in opera non diurnalia . | Ciuitas vero est communitas constituta | ad sufficientiam in vita tota . \\\hline
2.1.4 & que son menester para toda la uida | Mas el regno es comunidat establesçida | non solamente para conplir las menguas dela uida & ad sufficientiam in vita tota . | Sed regnum est communitas constituta | non solum ad supplendum indigentias vitae , \\\hline
2.1.4 & e otros muchos castiellos | Et por ende paresçe qual es la comunidat dela casa . | Ca es comunidat natural & adiunctae aliae plurimae ciuitates et castra . | Patet ergo qualis sit communitas domus , | quia est communitas naturalis constituita \\\hline
2.1.4 & Et por ende paresçe qual es la comunidat dela casa . | Ca es comunidat natural | e establesçida para las obras cotidianas & Patet ergo qualis sit communitas domus , | quia est communitas naturalis constituita | propter opera diurnalia et quotidiana . \\\hline
2.1.4 & Ca assi commo paresçe por las cosas | sobredichͣs la casa es vna comunidat | e vna conpannia de muchͣs ꝑssonas . & ( ut patet ex habitis ) | sit communitas quaedam et societas personarum : | cum non sit proprie communitas nec societas ad seipsum , \\\hline
2.1.4 & e vna conpannia de muchͣs ꝑssonas . | Et commo non sea propreamente comunidat | nin conpannia de vno & sit communitas quaedam et societas personarum : | cum non sit proprie communitas nec societas ad seipsum , | si in domo communitatem saluare volumus , \\\hline
2.1.4 & non solamente la casa es vna comiundat | mas en la casa conuiene de dar muchͣs comunidades | la qual cosa non puede ser sin muchͣs perssonas . & non solum domus est communitas quaedam , | sed in domo oportet | dare plures communitates : | sed in domo oportet | dare plures communitates : | quod sine pluralitate personarum \\\hline
2.1.4 & por las cosas ya dichͣs en la uida humanal | non solamente es menester la comunidat dela casa | mas ahun la comunidat del uarrio e dela çibdat e del regno & in vita humana | non solum est | expediens communitas domus , | non solum est | expediens communitas domus , | sed et vici , ciuitatis , et regni . \\\hline
2.1.4 & non solamente es menester la comunidat dela casa | mas ahun la comunidat del uarrio e dela çibdat e del regno | mas si por otras razones & expediens communitas domus , | sed et vici , ciuitatis , et regni . | Utrum autem propter alias causas , \\\hline
2.1.4 & que por las que ya dichos son | es menester la comunidat dela çibdat | e del regno esto se mostrara mas conplidamente en el terçero libro . & quam propter iam dictas , | sit expediens communitas ciuitatis , et regni , | in tertio libro plenius ostendetur . \\\hline
2.1.5 & Conuiene a saber | De comunidat de uaron et de muger . | Et de comunidat de sennor e de sieruo . & quod ex duabus communitatibus , | videlicet , ex communitate viri et uxoris , domini et serui , | constat domus prima . \\\hline
2.1.5 & De comunidat de uaron et de muger . | Et de comunidat de sennor e de sieruo . | Ca dize que la primera casa deue ser establesçida destas dos comuindades & quod ex duabus communitatibus , | videlicet , ex communitate viri et uxoris , domini et serui , | constat domus prima . \\\hline
2.1.5 & fazen ser la casa cosa natural . | Ca la comunidat del uaron e dela mugnies ordenada ala generacion . | Mas la comunidat del sennor e del sieruo es ordenada ala & Hoc ergo modo hae duae communitates faciunt domum esse quid naturale : | quia communitas viri et uxoris ordinatur ad generationem , | communitas vero domini \\\hline
2.1.5 & Ca la comunidat del uaron e dela mugnies ordenada ala generacion . | Mas la comunidat del sennor e del sieruo es ordenada ala | conseruaçique por la qual cosa si la generaçion e la conseruaçion es cosa natural & quia communitas viri et uxoris ordinatur ad generationem , | communitas vero domini | et serui ad conseruationem . Quare si generatio et conseruatio est quid naturale , \\\hline
2.1.5 & por que sin ellas non puede ser la cosa conueniblemente . | Mas que la comunidat del uaron | e dela muger sea & quia sine eis domus congrue esse non valet . | Quod autem communicatio viri et uxoris sit propter generationem , | videre non habet dubium : \\\hline
2.1.5 & e dela fenbra . | Mas que la comunidat del sennor e del sieruo | sea por salud e por conseruaçion dela casa & instituta est societas maris et foeminae . | Sed quod communitas domini et serui sit | propter salutem et conseruationem , \\\hline
2.1.5 & que assi conmo para el establesçemiento dela casa | es men ester la comunidat del uaron | e dela mugni & quia sicut ad constitutionem domus | requiritur | communitas viri et uxoris propter generationem , \\\hline
2.1.5 & en essa misma manera es | y . menester la comunidat del sennor | e del sieruo & communitas viri et uxoris propter generationem , | sic requiritur ibi communitas domini | et serui propter salutem \\\hline
2.1.5 & por que sin ellas la primera casa non puede estar conueniblemente . | Mas si la comunidat del sennor e del sieruo | en otra manera es & non potest existere . | Utrum autem communitas domini | et serui naturaliter instituta sit \\\hline
2.1.5 & Visto en qual manera | alo menos estas dos comunidades | son menester para la casa . & Viso , quomodo saltem hae duae communitates requiruntur ad domum , | quia secundum Philosophum ex eis constare | dicitur domus prima : \\\hline
2.1.5 & Ca por auentra a paresçrie alguno que commo la casa primera | sea establesçida de dos comunidades | e cada vna delas comunidades & quod cum domus prima constet | ex duabus communitatibus , | et quaelibet communitas requirat \\\hline
2.1.5 & sea establesçida de dos comunidades | e cada vna delas comunidades | aya menester dos perssonas o dos linages de perssonas & ex duabus communitatibus , | et quaelibet communitas requirat | duas personas uel duo genera personarum , \\\hline
2.1.6 & Emposi la casa fuere acabada conuiene de dar | y la terçera comunidat | que es de padre e de fijo . & si domus debet esse perfecta , | oportet ibi dare communitatem tertiam , | scilicet patris et filii . \\\hline
2.1.6 & son las primeras obras dela natura . | por ende con razon la comunidat del uaron | e dela muger que es para la generaçion & sunt prima opera naturae ; | merito ergo communitas viri et uxoris , | quae est propter generationem ; \\\hline
2.1.6 & e dela muger que es para la generaçion | e la comunidat del sennor e del sieruo | que es para la saluaçion & quae est propter generationem ; | et domini et serui , | quae est propter saluationem , \\\hline
2.1.6 & ca sin ellas non puede ser la primera casa conuenible mente . | Mas la terçera comunidat | que es de padre e de fijo & quia sine eis domus congrue esse non potest : | sed tertiam etiam communitatem , | quae est patris et filii , \\\hline
2.1.6 & por la qual cosa | commo la comunidat del padre al fijo tome nasçençia e comienço de aquello que el padre e la madre | engendran su semeiança esta tal comunidat non es dicha de razon dela primera casa & nisi sit iam perfectus . | Quare cum communitas patris ad filium sumat originem | ex eo quod parentes sibi simile produxerunt : | Quare cum communitas patris ad filium sumat originem | ex eo quod parentes sibi simile produxerunt : | huiusmodi communitas non dicitur \\\hline
2.1.6 & commo la comunidat del padre al fijo tome nasçençia e comienço de aquello que el padre e la madre | engendran su semeiança esta tal comunidat non es dicha de razon dela primera casa | mas es de razon dela casa ya acabada & ex eo quod parentes sibi simile produxerunt : | huiusmodi communitas non dicitur | esse de ratione domus primae , | huiusmodi communitas non dicitur | esse de ratione domus primae , | sed est de ratione domus perfectae ; \\\hline
2.1.6 & Mas que ala perfectiuo dela | casafaga menester esta terçera comunidat | pademos lo prouar & videlicet patris et filii . | Quod autem ad perfectionem domus requiratur haec tertia communitas , | triplici via venari possumus . \\\hline
2.1.6 & que non son bien auentraadas . | Pues que assi es que la terçera comunidat | que es del padͤ & Sic et feliciora sunt perfectiora infelicibus . | Ergo quod ad perfectionem domus requiratur tertia communitas , | quae est patris et filii , \\\hline
2.1.6 & o por mengua de amos | Mas commo el maslo e la fenbra e el marido e la muger sea la primera parte dela casar la primera comunidat | que es menester para la uida dela casa & vir et uxor sit | prima pars domus | et prima communitas , | prima pars domus | et prima communitas , | quae requiritur in vita domestica : \\\hline
2.1.6 & e la fenbra obediente . | Mas en la comunidat del padre | e del fijo el padre deua sienpre mandar & et foemina obsequens : | in communitate vero patris et filii , | pater debet esse imperans , \\\hline
2.1.6 & e el fij̉o ser obediente . | Et en la comunidat del señor | e del sieruo el señor deue mandar & et filius obtemperans ; | in communitate quidem domini et serui , | dominus debet esse praecipiens , \\\hline
2.1.6 & ¶ Pues que assi es paresçe | quantas son las comunidades | en la casa ̀ acabada & et dominus seruorum . | Patet ergo quot communitates sunt in domo perfecta , | et quot regimina , \\\hline
2.1.7 & en el primero libro delas politicas | en la comunidat dela casa | primeramente conuiene de ayuntar el uaron con la mugni & quia secundum Philosophum 1 Politic’ | in communitate domestica , | primum oportet \\\hline
2.1.7 & e puede engendrar semeiable dessi ¶ | Et pues que assi es la comunidat del uaton e dela muger | que es para la generaçion es la primera parte dela casa & et potest sibi simile producere . | Communitas ergo maris et foeminae , | quae est propter generationem , \\\hline
2.1.7 & que quiere dezir ꝑtiçipante con otro | Mas la comunidat en la uida humanal | assi commo dicho es dessuso & hominem esse naturaliter animal sociale et communicatiuum . | Communitas autem in vita humana | ( ut supra tangebatur ) \\\hline
2.1.7 & ¶ Et la otra de regno . | Mas todas estas comunidades | ante ponen la comunidat dela casa . & quaedam ciuitatis , quaedam regni . | Omnes autem hae communitates | praesupponunt communitatem domesticam . \\\hline
2.1.7 & Mas todas estas comunidades | ante ponen la comunidat dela casa . | Et pues que assi es commo la casa sea primero & Omnes autem hae communitates | praesupponunt communitatem domesticam . | Cum ergo domus sit prior vico , ciuitate , et regno : \\\hline
2.1.7 & que çiuil e de çibdat . | Et la comunidat dela casa mas paresçe | que es natural al omne & magis est animal de mesticum , | quam ciuile : et communitas domus | magis videtur esse naturalis ipsi homini , \\\hline
2.1.7 & que es natural al omne | que la comunidat del uarrio | nin dela çibdat nin del regno . & magis videtur esse naturalis ipsi homini , | quam communitas vici , ciuitatis , et regni . | Nam si considerentur ea , \\\hline
2.1.7 & e al mantenemiento del humanal linage | assi commo es ordenada la comunidat dela casa | Ca si las comunidades dichas del uarrio & propter conseruationem speciei , | sicut communitas domus : | nam si praedictae communitates \\\hline
2.1.7 & assi commo es ordenada la comunidat dela casa | Ca si las comunidades dichas del uarrio | e dela çibdat son ordenadas al mantenemiento & sicut communitas domus : | nam si praedictae communitates | ordinantur \\\hline
2.1.7 & que de çibdat | commo la primera comunidat dela casa sea | ayuntamientode uaron & quam politicum : | cum prima communitas ipsius domus sit coniunctio viri et uxoris , | sequitur ex parte ipsius communitatis humanae , \\\hline
2.1.7 & e que mas es ayuntable | por comunidat coniugable e de matermoino | que por comunidat de barrio & quod homo magis sit animal coniugale | quod politicum ; et quod magis sit communicatiuum communitate coniugali , | quam communitate vici ciuitatis , et regni : \\\hline
2.1.7 & por comunidat coniugable e de matermoino | que por comunidat de barrio | nin de çibdat nin de regno . & quod politicum ; et quod magis sit communicatiuum communitate coniugali , | quam communitate vici ciuitatis , et regni : | quia domus , \\\hline
2.2.1 & Et pues que assi es | deuedessaber que commo la comunidat | e la conpannia del uaron e dela muger e del sennor e del sieruo part enescan ala casa primera & Sciendum igitur , | quod cum communitas viri et uxoris , | et domini et serui pertineant \\\hline
2.2.18 & Mas avn es neçessario para el bien | e para el defendimiento dela comunidat | por ende alos que quieren beuir & non solum aliquando est licitus , | sed etiam necessarius pro bono Reipublicae , | volentibus politice viuere , \\\hline
2.2.20 & quales si quier obras | nin c̃ca los gouernamientos dela comunidat nin dela çibdat | Si la uoluntad del omne & et non vacant ciuilibus operibus , | nec regiminibus reipublicae ; | si mens humana \\\hline
2.3.9 & que si non fuesse sinon la comiundat dela casa | que es la comunidat primera | non seria ninguna muda conn neçessaria & nisi communitas domus | quae est communitas prima , | nulla commutatio esset necessaria . Nam in domo dominatur paterfamilias , \\\hline
2.3.9 & enł primero libro delas politicas | que en la primera comunidat | que es comuidat dela casa es cosa prouada & quod dicitur 1 Politicorum | quod in prima communitate quae est domus , | manifestum est nullum esse opus ipsius commutationis igitur \\\hline
2.3.9 & sin la comiundat dela casa | assi commo es comunidat del uarrio o dela çibdat | o de todo el regno o dela prouiçia . & a communitate domus : | ut communitas vici , vel ciuitatis , | vel totius regni , et prouinciae , \\\hline
2.3.9 & llgua manera abastasse la mudaçion delas cosas alas cosas | enpero ala comunidat | que hades es en todo el regno & commutatio rerum ad res : | tamen ad communicationem | quod habetur in toto regno , \\\hline
3.1.1 & or que toda çibdat conuiene que sea alguna comunindat | commo toda comunidat sea por graçia de algun bien . | Conuiene que la çibdat sea establesçida por algun bien & esse communitatem quandam , | cum omnis communitas fit | gratia alicuius boni , | cum omnis communitas fit | gratia alicuius boni , | oportet ciuitatem ipsam constitutam esse propter aliquod bonum . \\\hline
3.1.1 & La segunda se toma de parte dela çibdat establesçida | por conparaçion alas otras comunidades ¶ | La primera razon se declara & ex parte hominum constituentium ciuitatem . | Secunda ex parte ciuitatis constitutae . | Prima via sic patet . \\\hline
3.1.1 & vna inclinaçion de natura es en todos los orans | atal comunidat | qual es la comuidat dela çibdat . & natura quidem impetus in omnibus inest | ad talem communitatem , | qualis est communitas ciuitatis . \\\hline
3.1.1 & Ca commo quier que toda comun dar natural sea ordenada a bien | enpero mayormente es ordenada aquel bien la comunidat | que es mas prinçipal & Nam licet omnis communitas naturalis ordinetur ad bonum , | maxime tamen ordinatur | ad ipsum communitas principalissima : \\\hline
3.1.1 & que es mas prinçipal | e esta tal es la comunidat dela çibdat | la qual en conparacion dela comunidat dela casa e del barrio es mas prinçipal & ad ipsum communitas principalissima : | huius autem est communitas ciuitatis , | quae respectu communitatis domus , \\\hline
3.1.1 & e esta tal es la comunidat dela çibdat | la qual en conparacion dela comunidat dela casa e del barrio es mas prinçipal | por la qual razon & huius autem est communitas ciuitatis , | quae respectu communitatis domus , | et vici principalissima existit . | quae respectu communitatis domus , | et vici principalissima existit . | Quare si communitas domestica ordinatur ad bonum \\\hline
3.1.1 & assi commo es prouado de suso | mas conplidamente en el segundo libro la comunidat del barno | que es mas prinçipal & et etiam ad multa bona , | ut supra in secundo libro diffusius probabatur : | communitas vici , | ut supra in secundo libro diffusius probabatur : | communitas vici , | quae est principalior communitate domestica , \\\hline
3.1.1 & mas sera ordenada a bien | e avn la comunidat dela çibdat | que es much mas prinçipal & multo magis ordinatur ad bonum : | et ad hoc communitas ciuitatis , | quae est principalissima communitas respectu vici , \\\hline
3.1.1 & que es much mas prinçipal | que la comunidat del barrio | nin dela casa much mas es ordenada abien que todas las otras & et ad hoc communitas ciuitatis , | quae est principalissima communitas respectu vici , | et domus , \\\hline
3.1.1 & que si nos dizimos | que toda comunidat es establesçida | por grande algun bien & Hoc est ergo quod dicitur primo Polit’ | quod si communitatem omnem gratia alicuius boni dicimus constitutam , | maxime autem principalissimam omnium , \\\hline
3.1.1 & por grande algun bien | e esta es comunidat politica | que es llamada & et omnes alias circumplectens , potissime gratia boni constitutam esse contingit : | haec autem est communitas politica , | quae communi nomine vocatur ciuitas . \\\hline
3.1.1 & mas prinçipal | mas a vn otra comunidat ay mas prinçipal | que ella la qual es comunidat del regno & sed respectu communitatis domus , et vicio . | Est autem alia communitas principalior ea , | cuiusmodi est communitas regni , \\\hline
3.1.1 & mas a vn otra comunidat ay mas prinçipal | que ella la qual es comunidat del regno | dela qual diremos en su logar & Est autem alia communitas principalior ea , | cuiusmodi est communitas regni , | de qua suo loco dicetur : \\\hline
3.1.1 & ca mostraremos | que la comunidat del regno es prouechosa en la uida humanal | e es mas prinçipal & de qua suo loco dicetur : | ostendemus enim communitatem regni | utilem esse in vita humana , | ostendemus enim communitatem regni | utilem esse in vita humana , | et esse principaliorem communitate ciuitatis . \\\hline
3.1.1 & e es mas prinçipal | que la comunidat dela çibdat ca paresçe | que assi se ha la comunidat del regno & utilem esse in vita humana , | et esse principaliorem communitate ciuitatis . | Videtur enim suo modo communitas regni \\\hline
3.1.1 & que la comunidat dela çibdat ca paresçe | que assi se ha la comunidat del regno | ala comunidat dela çibdat & et esse principaliorem communitate ciuitatis . | Videtur enim suo modo communitas regni | se habere | Videtur enim suo modo communitas regni | se habere | ad communitatem ciuitatis , \\\hline
3.1.1 & que assi se ha la comunidat del regno | ala comunidat dela çibdat | commo la comunidat dela çibdat se ha & se habere | ad communitatem ciuitatis , | sicut haec communitas se habet ad domum , et vicum . \\\hline
3.1.1 & ala comunidat dela çibdat | commo la comunidat dela çibdat se ha | ala comiundat dela casa e del uarrio . & ad communitatem ciuitatis , | sicut haec communitas se habet ad domum , et vicum . | Nam ciuitas sicut complectitur domum , et vicum ; \\\hline
3.1.1 & que estas dichos dos comuindades . | bien assi la comunidat del regno ençierra en ssi la comunidat de la çibdat | e es mucho mas acabada et conplida en la uida humanal & quam communitates praedictae : | sic communitas regni | circumplectitur communitatem ciuitatis , | sic communitas regni | circumplectitur communitatem ciuitatis , | et est multo perfectior \\\hline
3.1.1 & e es mucho mas acabada et conplida en la uida humanal | que la comunidat dela çibdat | on a basta de dezer & et magis sufficiens in vita , | quam communitas illa . | Non sufficit dicere ciuitatem constitutam \\\hline
3.1.2 & porque por ella alcançan los omes ser acabados | e beuir en comunidat politica e de çibdat | ca sin ella la uida del omne non puede ser & quia per eam homines consequuntur | omnia tria praedicta bona . | Nam ipsum viuere consequuntur homines ex communitate politica : | omnia tria praedicta bona . | Nam ipsum viuere consequuntur homines ex communitate politica : | quia sine ea vita hominis \\\hline
3.1.2 & en el primero libro delas politicas | que la comunidat | que es cibdat es cosa & quod scribitur primo Politicorum | quod communitas , | quae est ciuitas constans \\\hline
3.1.2 & que es fechͣ de much suarrios | et tales comunidat acabada | ca esto se sigͤel dezir & ex pluribus vicis , | est communitas perfecta : | quia iam \\\hline
3.1.2 & ca esto se sigͤel dezir | que tal comunidat es la que ha termino | por si e todo conplimiento e abastamiento de uida & ( ut consequens est dicere ) | huiusmodi communitas est | habens terminum omnis | huiusmodi communitas est | habens terminum omnis | per se sufficientiae vitae . \\\hline
3.1.2 & nin durar | otdenaron la comunidat politica | que era fechͣ & constituta ciuitas stare non posset , | ordinauerunt communitatem politicam , | quae facta erat ad viuere , \\\hline
3.1.4 & ¶ La primera razon se tomadesto | que esta comunindat ençierra en si la comunidat dela casa | e la comunidat del uartio ¶ & Prima uia sumitur | ex eo quod communitas complectitur domum et uicum . | Secunda ex eo quod est illarum finis et complementum . \\\hline
3.1.4 & que esta comunindat ençierra en si la comunidat dela casa | e la comunidat del uartio ¶ | La segunda razon desto & Prima uia sumitur | ex eo quod communitas complectitur domum et uicum . | Secunda ex eo quod est illarum finis et complementum . \\\hline
3.1.4 & que sirue a conplimiento de uida | e por ende la comunidat dela casa | e avn del uartio son cosas naturales & ad sufficientiam uitae . | Communitas ergo domestica | et etiam uici naturalia sunt , \\\hline
3.1.4 & que la çibdat es fin | et conplimiento de las dichͣs dos comunidades | ca assi commo prueua el pho & sumitur ex eo quod ciuitas est | illarum communitatum finis et complementum . | Nam ut arguit Philosophus primo Politicorum \\\hline
3.1.4 & ca la casa se faze de comuidat de omne e de su muger e de sennor e de sieruo e de padre e de fijos . Et cada vna destas comuidades es cosa segunt natura bien | assi avn la comunidat del uarrio es cosa natural | non solamente por que sirue a conplimiento dela uida & ex communitate viri et uxoris , domini , et serui , patris et filii , quarum quaelibet est secundum naturam . | Sic etiam communitas vici est | quid naturale , | Sic etiam communitas vici est | quid naturale , | non solum quia deseruit \\\hline
3.1.4 & por ende | si la comunidat dela casa es ordenada a alcançar lo que es delectable | e para foyr & et quid iniustum . | Si ergo communitas domestica ordinatur ad prosequendum conferens , | et ad fugiendum nociuum : \\\hline
3.1.4 & lo que es enpeçible . | Et la comunidat dela çibdat sobre esto es ordenada a segnir | lo que es iusto e a foyr & et ad fugiendum nociuum : | communitas vero ciuitatis | ultra hoc ordinatur | communitas vero ciuitatis | ultra hoc ordinatur | ad prosequendum iustum , \\\hline
3.1.4 & conuiene | que la comunidat dela casa | e la comunidat dela çibdat sean cosas naturales & et ad fugiendum iniustum , | oportet communitatem domesticam | et ciuilem esse quid naturale . \\\hline
3.1.4 & que la comunidat dela casa | e la comunidat dela çibdat sean cosas naturales | ca si la natura dio al omne palabra natural aquella comunidat & oportet communitatem domesticam | et ciuilem esse quid naturale . | Nam si natura dedit homini sermonem , \\\hline
3.1.4 & e la comunidat dela çibdat sean cosas naturales | ca si la natura dio al omne palabra natural aquella comunidat | que es ordenada a aquellas cosas & et ciuilem esse quid naturale . | Nam si natura dedit homini sermonem , | naturalis est illa communitas | Nam si natura dedit homini sermonem , | naturalis est illa communitas | quae ordinatur ad illa , \\\hline
3.1.4 & non han de ser propreamente | en la comuidat dela casa mas enla comunidat dela çibdat | ca en la çibdat do los çibdadanos han sus possessiones propias & esse in communitate domestica , | sed in communitate ciuili . | In ciuitate enim , | sed in communitate ciuili . | In ciuitate enim , | ubi ciues habent possessiones proprias et distinctas , \\\hline
3.1.4 & por las quales se pueden manteñ en la uida e esto contesçe mayormente segunt que dize el pho | por la comunidat dela çibdat | por que deue contener en ssi todas aquellas cosas & ( secundum Philosophum ) | per communitatem ciuilem , | eo quod ciuitas debeat \\\hline
3.1.5 & e podemos mostrar por tres razones | que sin la comunidat dela çibdat | cosa aprouechosa fue ala uida humanal & Possumus autem triplici via ostendere , | quod praeter communitatem ciuitatis , | utile est humanae vitae \\\hline
3.1.5 & cosa aprouechosa fue ala uida humanal | de establesçer comunidat de regno ¶ | La primera razon se toma de parte del conplimiento dela uida & utile est humanae vitae | statuere communitatem regni . | Prima via sumitur \\\hline
3.1.5 & en el primero libro delas politicas | que la comunidat acabada | que es çibdat se faga de muchsuarios & Nam cum ait Philosophus primo Polit’ | quod communitas perfecta , | quae est ciuitas constans \\\hline
3.1.9 & mas de quanto valie . | Et pues que assi es la comunidat delas possessiones | que ordenaua socrates & plus ualere quam ualeat . | Communitas ergo possessiones | quam ordinabat Socrates \\\hline
3.1.9 & e por ende llama una la la derechura . | Et pue tal que assi es puesta comunidat delas mugers | non se signiria aquel bien & propter quod vocata est Iusta . | Non ergo supposita communitate uxorum | esset bonum illud \\\hline
3.1.9 & commo cuydaua socrat̃s | ca muchͣ comunidat mas aduze de amor | si fuere cierta e conosçida & ut opinabatur Socrates : | nam modica coniunctio plus inducit de amore , | si sit certa et nota , \\\hline
3.1.9 & por que sepan ordenar la çibdat | assi commo conuiene ala comunidat de los çibdadanos | lpho prueua en el primero libro delas politicas & ut sciant sic ciuitatem ordinare , | ut expedit communitati ciuium . | Philosophus 2 Polit’ probat multa mala sequi in ciuitate , \\\hline
3.1.10 & tire la cercidunbre de los fijos e el conosçimiento del parentesco | non es de ella bartal comunidat | ca della se sigue & et notitiam consanguineitatis , | non est commendanda : | quia ex hoc consequitur aliquos \\\hline
3.1.10 & El terçero mal se declara | assi ca conmo de la comunidat sobredichͣ delas mugiets | e de los fijos se sigua iniuria et tuerto de los fijos & Tertium malum sic declaratur . | Nam supposita praedicta communitate , | sequeretur in curia filiorum \\\hline
3.1.10 & e dende se sigue | que puesta tal comunidat commo ordeno soctateᷤ | non se puede auer cuydado conuenible & sequitur quod supposita communitate , | quam ordinauerat Socrates , | non habeatur debita cura , \\\hline
3.1.10 & por ende much era de reprehender la | opimon de socrates dela comunidat | que puso delas mugeres e de los fijos . & et amor libidinosus , | reprehensibilis erat opinio Socratis | de communitate uxorum et filiorum . \\\hline
3.1.10 & e alos prinçipes de ordenar assi la çibdat | por que defendia la comunidat delas fenbras e delas mugeres casadas | e los padres sean çiertos de sus propreos fijos & sic ordinare ciuitatem , | ut prohibita communitate foeminarum et uxorum certificentur parentes de propriis filiis . | Esse res communes , \\\hline
3.1.11 & que la çibdat es es si ordenada | do es tanta comunidat | de los çibdadanos tienen la çibdat & sic ordinatam esse | ubi est tanta communitas ciuium , | reputat eam felicem esse , \\\hline
3.1.11 & e que biuen sin contienda | entre los quales se guarda tan grant comunidat . | mas assi commo dize el philosofo & et absque litigio viuere , | inter quos tanta communitas obseruatur . | Sed ut dicitur secundo Politicorum \\\hline
3.1.11 & que todos los çibdadanos | por la comunidat delas fenbras | e delas mugers creyessen & quod omnes ciues | propter communitatem mulierum et uxorum crederent | se esse consanguinitate coniunctos ; \\\hline
3.1.15 & assi commo si fuessen suyas . | Mas en las mugers e en los fijos deue ser guardada comunidat | non solamente quanto al amor & ac si essent suae . | In uxoribus autem ex filiis debet | reseruari communitas quantum ad amorem : | In uxoribus autem ex filiis debet | reseruari communitas quantum ad amorem : | sed in possessionibus non solum debet \\\hline
3.1.15 & desocͣtes | quanto ala comunidat de los çibdadanos | En essa misma manera podemos saluar el su dicho & Saluauimus igitur dictum Socraticum | quantum ad communitatem ciuium : | sic etiam saluare possumus dictum eius quantum ad unitatem ciuitatis . \\\hline
3.1.15 & Et pues que assi es | assi es poniendo la entençio de socrates dela comunidat delas cosas | e dela vnidat de los çibdadanos & quando ciues se amando et diligendo maxime unirentur . | Sic ergo exposita mente Socratis | de communitate rerum | Sic ergo exposita mente Socratis | de communitate rerum | et de unitate ciuium , \\\hline
3.2.17 & que se dixieren en los consseios . | ca esto fue lo que enssalço la comunidat de Roma | fieldat de buenos consseieros & quae ibi sunt tradita . | Hoc enim fuit , | quod apud Romanam Rempublicam exaltauit fidelitas consiliantium : \\\hline
3.2.17 & e muy alto era conssisto | no secreto dela comunidat de roma | alos quel guardananca era guaruido de grant fialdat . & ait , quod fidum et altum erat | secretum consistorium reipublicae , | silentique salubritate munitum : \\\hline
3.2.19 & tal que sea bueno e uirtuoso | e ame la comunidat | et ꝓmueua los q̃ son eñl su regno e los hõ rre . & quod esset bonus virtuosus | et politiam diligeret , | existentes in regno promoueret et honoraret : \\\hline
3.2.26 & en el quarto libro delas politicas | que non conuiene de apropar las comunidades | delas çibdades alas leyes . & Ideo dicitur 4 Politicorum | quod non oportet | adaptare politias legibus , \\\hline
3.2.26 & delas çibdades alas leyes . | Mas las leyes alas comunidades | de las çibdades las quales leyes conuiene de ser departidas & adaptare politias legibus , | sed leges politiae , | quas leges oportet diuersas esse \\\hline
3.2.26 & de las çibdades las quales leyes conuiene de ser departidas | segunt el departimiento delas comunidades . | Et pues que assi es el que quisiere poner leyes & quas leges oportet diuersas esse | secundum diuersitatem politiarum . | Volens ergo leges ferre , \\\hline
3.2.27 & Por la qual cosa commo el bien comun sea entendido | prinçipalmente de toda la comunidat | assi commo de todo el pueblo o del prinçipe & secundum quas intendimus in bonum illud : | quare cum bonum commune principaliter intendatur a tota communitate | ut a toto populo , vel a principante , \\\hline
3.2.32 & que cosa es la çibdat podemos responder | e dez que es comunidat de çibdadanos | por beuir bien e uirtuosamente & Dici debet | quod est communicatio ciuium propter bene , | et virtuose viuere ; \\\hline

\end{tabular}
