\begin{tabular}{|p{1cm}|p{6.5cm}|p{6.5cm}|}

\hline
3.2.2 & quorum tres sunt boni , et tres sunt mali . Nam regnum aristocratia , et politia sunt principatus boni : \textbf{ tyrannides , oligarchia , et democratia sunt mali . } Docet enim idem ibidem discernere bonum principatum a malo . Nam si in aliquo dominio aut principatu & ca el regno e la aristo carçia que quiere dezer sennorio de buenos e la poliçia que quiere dezer pueblo bien \textbf{ enssenoreante son bueons prinçipados . | La thirama que quiere dezer sennorio malo } e la obligaçia que quiere dezer sennorio duro . Et la democraçia que quiere dez maldat del pueblo enssennoreante son malos prinçipados . \\\hline
3.2.7 & tunc est tyrannus et est pessimus , quia propter suam unitam potentiam potest multa mala efficere . Hanc autem rationem tangit Philosophus quinto Politicorum ubi ait , \textbf{ tyrannidem esse oligarchiam } extremam idest pessimam : quia est maxime nociua subditis . Quarta via sumitur ex eo quod per tale dominium impediuntur & ca por el su poderio muy grande que es ayuntado en vno puede fazer muchs males e esta razon tanne el philosofo en el quinto libro delas politicas \textbf{ do dize que la tirnia es la postrimera obligarçia } que quiere dezer muy mala obligacion por que es muy enpesçedera alos subditos ¶ La quarta razon se toma por que por tal sennorio son enbargados grandes bienes de los çibdadanos . \\\hline
3.2.12 & siue principatus bonorum . Si vero dominentur non quia boni , sed quia diuites , est peruersus \textbf{ et vocatur oligarchia . } Sed si dominatur totus populus et intendat bonum omnium tam insignium quam aliorum , est principatus rectus , & que quiere dezer señorio de buenos . Mas si enssennorear en pocos non por que son buenos mas por que son ricos es llamado obligarçia \textbf{ que quiere dezer señorio tuerto . } Mas quando enssennore a todo el pueblo si entienda al bien comun de todos tan bien de los nobles conmo de los otros es senorio derech e es llamado gouernamiento de pueblo . Mas si el pueblo tiranizare \\\hline

\end{tabular}
