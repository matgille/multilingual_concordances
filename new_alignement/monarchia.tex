\begin{tabular}{|p{1cm}|p{6.5cm}|p{6.5cm}|}

\hline
1.3.3 & brachium periculo se exponit . Sic etiam antiquitus \textbf{ si perspeximus ciuitatem aliquam dominari et tenere monarchiam : } hoc erat , quia ciues pro Republica non dubitabant se morti exponere . & por que todo el cuerpo non ꝑesca . Et en essa misma manera avn si cataremos al tp̃o \textbf{ quando alguna çibdat auie señorio e tenie sennorio sobre las otras esto era } por que los çibdadanos non duda una de se poner ala muerte por el bien comun de todos . \\\hline
3.2.2 & vel intendit bonum proprium . Si intendit bonum commune et subditorum , \textbf{ tunc dicitur Monarchia siue Regnum : } regis autem est intendere commune bonum . Si vero ille unus dominans & Si vnio o aquel entiende bien comun e bien de los subdictos \textbf{ e estonçe es dicho tal sennorio monarch̃ia o e egno } ca al Rey parte nesçe de enteder el bien comun . Et li aquel vno assi \\\hline
3.2.4 & in tantum est rectum et dignum , inquantum tenent locum unius : \textbf{ dominari unum et facere monarchiam , } si debito modo fiat , erit omnino rectior et dignior . & en tanto es derecho e digno en quanto ellos tienen logar de vno \textbf{ el sennorio de vno es meior | e fazer tal monarchia de vno } si se faze en manera conueinble sera en toda manera \\\hline
3.2.4 & multarum manuum , et multorum pedum . Non ergo dici poterit \textbf{ talem unum monarchiam non cognoscere multa ; } quia quantum spectat ad regimen regni , & e en esta manera se fara el prinçipe vn omne de muchs oios e de muchͣs manos e de muchs pies . Et pues que assi es non se puede dezer \textbf{ que vn tal monarchia | o tal prinçipe assi fech̃ de muchos que non conogca } e non sepa muchͣs cosas . por que quanto parte nesçe al \\\hline
3.2.7 & est efficacissimus ad proficiendum : sic tyrannis efficacissima ad nocendum . \textbf{ Monarchia enim quia ibi dominatur unus , } et est ibi virtus unita , ideo vel est optima , & por que es muy vno es muy afincado para aprouechar . Assi la tirania es muy afincada para enpees çer \textbf{ ca el senñorio de vno | enssennorea } e do la uirtud es ayuntada en vno o es muy buencoo es muy malo . \\\hline
3.2.13 & et honores magnos existentes in ipsis . Quinto fiunt insidiae tyrannis ab aliquibus , \textbf{ non ut possideant monarchiam , } sed ut videantur facere actiones aliquas singulares . & e matanlos ¶ Lo quinto son puestos asechos alos tiranos de algunos non \textbf{ por que ayan el su señorio mas por que paresca alos omes } que fazen algunos omes apartadas ca algunos quieren ser en alguna nonbrada o en alguna fama \\\hline
3.2.14 & Una ergo tyrannis potest contrariari alii , et corrumpere ipsam ; \textbf{ ut tyrannis populi contrariatur tyrannidi monarchiae : } et una monarchia tyrannica contrariatur alii . Cum enim aliquis monarcha & e el mal senorio es contrario al malo . Et por ende vna tirama puede ser contraria a otra \textbf{ e corronper la assi cotio la tirania del pueblo | escontrana ala tirama del mal prinçipado } Et vn prinçipado tiranico es contrario a otro prinçipado tiranico e malo ca quando algun \\\hline
3.2.14 & Totus ergo populus efficitur quasi unus tyrannus contra Principem , \textbf{ et tyrannis populi corrumpit tyrannidem monarchiam . } Sic etiam una tyrannis monarchia corrumpit aliam : quia multotiens unus monarcha tyrannus insurgit in alium , & Et por ende todo el pueblo es fech assi commo vn tirano contra el prinçipe e cotra su tirama . \textbf{ la tirama del pueblo corronpe e destruye la tirana del prinçipe bien } assi avn vna tirama de sennorio corronpe a otra por que muchͣs vezes vn prinçipe tirano se leunata contra otro prinçipe tirano \\\hline

\end{tabular}
