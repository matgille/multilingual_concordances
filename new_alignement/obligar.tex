\begin{tabular}{|p{1cm}|p{6.5cm}|p{6.5cm}|}

\hline
3.2.2 & nin entienden el bien comun mas son ricos e apremadores de los otros e entienden a ganançia proprea este prinçiado \textbf{ tal es dich obligartia } que quiere dezir prançipado de ricos . Et pues que assi es dos prinçipados se leuna tan del sennorio de po cos vno derecho & sed sunt diuites , et opprimentes alios intendunt proprium lucrum huiusmodi principatus Oligarchia dicitur , \textbf{ quod idem est } quod principatus diuitum . Consurgit igitur duplex principatus ex dominio paucorum : \\\hline
3.2.7 & que es ayuntado en vno puede fazer muchs males e esta razon tanne el philosofo en el quinto libro delas politicas \textbf{ do dize que la tirnia es la postrimera obligarçia } que quiere dezer muy mala obligacion por que es muy enpesçedera alos subditos ¶ La quarta razon se toma & quia propter suam unitam potentiam potest multa mala efficere . Hanc autem rationem tangit Philosophus quinto Politicorum ubi ait , \textbf{ tyrannidem esse oligarchiam } extremam idest pessimam : quia est maxime nociua subditis . Quarta via sumitur \\\hline
3.2.12 & que quiere dezer señorio de buenos . Mas si enssennorear en pocos non por que son buenos \textbf{ mas por que son ricos es llamado obligarçia } que quiere dezer señorio tuerto . Mas quando enssennore a todo el pueblo si entienda al bien comun de todos tan bien de los nobles conmo de los otros es senorio derech & et vocatur aristocratia siue principatus bonorum . Si vero dominentur non quia boni , \textbf{ sed quia diuites , } est peruersus et vocatur oligarchia . Sed si dominatur totus populus \\\hline
3.2.24 & non ha departimiento de ser assi o en otra manera . mas despues que es puesto a fuerça \textbf{ de obligar alos omes . } Mas la razon por que al derech natural conuinio anneder derecho positiuo es esta por que muchas cosas son derechas naturalmente assi commo natural cosa es al ome de fablar & sed positiuum ex principio antequam sic statutum , nihil differt esse sic vel aliter , postquam autem est editum incipit \textbf{ habere ligandi efficaciam . } Ratio autem , quare iuri naturali oportuit superaddere positiuum , \\\hline
3.2.27 & enssennoreare ¶ Visto que non parte nesçe a cada vno establesçer las leyes de ligo puede paresçer \textbf{ que la ley non ha uirtud de obligar a } ningnon si non fuere obligada e prigo nada . Ca commo la ley sea vn mandamiento del sennor mayor . por el qual somos e reglados e ligados en las nr̃as obras & Viso quod non est cuiuslibet leges condere , de leui potest patere legem \textbf{ non habere vim obligandi , } nisi sit promulgata . Nam cum lex sit quoddam mandatum superioris , \\\hline
3.2.27 & si non viniere o non pudiere veniral conosçimiento de los subditos . \textbf{ Poque la ley aya uirtud e fuerça de obligar } conuiene que sea publicada e pregonada . Mas commo otra sea la ley natural e otra la positiua en vna manera se deue publicar la vna e en otra manera la otra . & nisi peruenerit vel peruenire potuerit ad notitiam subditorum , \textbf{ ad hoc quod lex habeat vim obligandi , } oportet eam promulgatam esse . Sed cum alia sit lex naturalis , alia positiua : \\\hline
3.2.27 & que han de poner con acuçia . por que tales leyes ayan uirtud e fuerça de costrennir \textbf{ e de obligar deuen las publicar } e publicado las guardar las e mantenerlas . Ca segunt dize el philosofo en el quarto libro delas politicas . & quas leges imponant populo cui dominantur . Et postquam excogitauerunt leges imponendas , ut huiusmodi leges vim obligandi habeant , \textbf{ debent eas promulgare , } et promulgatas custodire et obseruaret : quia secundum Philosophum 4 Politicorum circa leges duplex cura esse debet : \\\hline

\end{tabular}
