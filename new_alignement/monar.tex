\begin{tabular}{|p{1cm}|p{6.5cm}|p{6.5cm}|}

\hline
1.3.3 & Sic etiam antiquitus \textbf{ si perspeximus ciuitatem aliquam dominari et tenere monarchiam : } hoc erat , quia ciues pro Republica non dubitabant & Et en essa misma manera avn si cataremos al tp̃o \textbf{ quando alguna çibdat auie señorio e tenie sennorio sobre las otras esto era } por que los çibdadanos non duda una de se poner ala muerte \\\hline
1.3.3 & Dilectatio enim quam habebant Romani \textbf{ ad Rempublicam fecit Romam esse principantem et monarcham . } Hoc ergo modo quoslibet homines decet esse amatiuos , & Ca el amor que auian los romanos al bien comun \textbf{ e publicofizo a Roma ser sennora | e auer sennorio en todo el mundo . } Pues que assi es que esto conuiene a todos los omes de ser amadores \\\hline
3.2.3 & eo quod magis ad unitatem accedat : \textbf{ optima est autem monarchia siue gubernatio unius Regis , } eo quod ibi perfectior unitas reseruetur . & e a vna uoluntad los pocos que los muchos . \textbf{ Mas la monarchia o el gouernamiento de vn Rey es muy bueno } por que yes fallada mas acabada vnidat ¶ \\\hline
3.2.4 & Sed de hoc infra dicetur : \textbf{ ostendetur enim quod sicut monarchia regia est optima ; } ita quia maiori bono maius malum opponitur , & mas desto diremos ayuso mas conplidamente \textbf{ ca mostraremos que assi commo la monarchia | e el prinçipado real es muy bueno } assi por que al mayor bien es contrario el mayor mal \\\hline
3.2.4 & ita quia maiori bono maius malum opponitur , \textbf{ monarchia tyrannica est pessima . } Dominari autem plures dominio recto , & assi por que al mayor bien es contrario el mayor mal \textbf{ por ende el prinçipado thiranico es muy malo | por que es contrario al prinçipado del regno } que es muy bueno . \\\hline
3.2.4 & inquantum tenent locum unius : \textbf{ dominari unum et facere monarchiam , } si debito modo fiat , & en quanto ellos tienen logar de vno \textbf{ el sennorio de vno es meior | e fazer tal monarchia de vno } si se faze en manera \\\hline
3.2.4 & sicut forte faceret unus solus : \textbf{ ideo si fiat monarchia } et dominetur unus Princeps vel unus Rex in toto principatu vel in toto regno , & commo por auentra a se arredraria vno solo . \textbf{ Por ende si fuere la monarchia } e el sennorio do fuere sen nor vn prinçipe \\\hline
3.2.4 & Non ergo dici poterit \textbf{ talem unum monarchiam non cognoscere multa ; } quia quantum spectat & Et pues que assi es non se puede dezer \textbf{ que vn tal monarchia | o tal prinçipe assi fech̃ de muchos que non conogca } e non sepa muchͣs cosas . \\\hline
3.2.6 & decens est tales excessus \textbf{ in ipsa monarchia perfectius reperiri . } Decet enim ipsum regem volentem recte regere & que tases aun ataias \textbf{ e tales condiçions buenas sean falladas en el rey mas conplidamente despues que fuere puesto en el prinçipado | que ante ca conuiene } que el Rey \\\hline
3.2.7 & vel est pessima : \textbf{ nam si monarchia habet intentionem rectam , } tunc est Rex et est optimus principatus : & buencoo es muy malo . \textbf{ ca si el sennor ha la entençion derecha } estonçe es Rey \\\hline
3.2.7 & multa bona efficere : \textbf{ si vero monarchia habet intentionem peruersam , } tunc est tyrannus et est pessimus , & puede fazer muchͣs bueans cosas . \textbf{ Et si por auentura el prinçipe ha la entençion tuerta } estonçe es tirano \\\hline
3.2.13 & Quinto fiunt insidiae tyrannis ab aliquibus , \textbf{ non ut possideant monarchiam , } sed ut videantur & Lo quinto son puestos asechos alos tiranos de algunos non \textbf{ por que ayan el su señorio mas por que paresca alos omes } que fazen algunos omes apartadas \\\hline
3.2.14 & et corrumpere ipsam ; \textbf{ ut tyrannis populi contrariatur tyrannidi monarchiae : } et una monarchia tyrannica contrariatur alii . & Et por ende vna tirama puede ser contraria a otra \textbf{ e corronper la assi cotio la tirania del pueblo | escontrana ala tirama del mal prinçipado } Et vn prinçipado tiranico es contrario a otro prinçipado tiranico e malo \\\hline
3.2.14 & ut tyrannis populi contrariatur tyrannidi monarchiae : \textbf{ et una monarchia tyrannica contrariatur alii . } Cum enim aliquis monarcha & escontrana ala tirama del mal prinçipado \textbf{ Et vn prinçipado tiranico es contrario a otro prinçipado tiranico e malo } ca quando algun \\\hline
3.2.14 & et una monarchia tyrannica contrariatur alii . \textbf{ Cum enim aliquis monarcha } vel aliquis unus Princeps tyrannizet in populum , & escontrana ala tirama del mal prinçipado \textbf{ Et vn prinçipado tiranico es contrario a otro prinçipado tiranico e malo } ca quando algun \\\hline
3.2.14 & quasi unus tyrannus contra Principem , \textbf{ et tyrannis populi corrumpit tyrannidem monarchiam . } Sic etiam una tyrannis monarchia corrumpit aliam : & assi commo vn tirano contra el prinçipe e cotra su tirama . \textbf{ la tirama del pueblo corronpe e destruye la tirana del prinçipe bien } assi avn vna tirama de sennorio corronpe a otra \\\hline
3.2.14 & et tyrannis populi corrumpit tyrannidem monarchiam . \textbf{ Sic etiam una tyrannis monarchia corrumpit aliam : } quia multotiens unus monarcha tyrannus insurgit in alium , & la tirama del pueblo corronpe e destruye la tirana del prinçipe bien \textbf{ assi avn vna tirama de sennorio corronpe a otra } por que muchͣs vezes vn prinçipe tirano se leunata contra otro prinçipe tirano \\\hline
3.2.14 & Sic etiam una tyrannis monarchia corrumpit aliam : \textbf{ quia multotiens unus monarcha tyrannus insurgit in alium , } ut obtineat principatum eius . & assi avn vna tirama de sennorio corronpe a otra \textbf{ por que muchͣs vezes vn prinçipe tirano se leunata contra otro prinçipe tirano } por que gane el su prinçipado . \\\hline

\end{tabular}
